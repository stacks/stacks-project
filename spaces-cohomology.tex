\IfFileExists{stacks-project.cls}{%
\documentclass{stacks-project}
}{%
\documentclass{amsart}
}

% The following AMS packages are automatically loaded with
% the amsart documentclass:
%\usepackage{amsmath}
%\usepackage{amssymb}
%\usepackage{amsthm}

% For dealing with references we use the comment environment
\usepackage{verbatim}
\newenvironment{reference}{\comment}{\endcomment}
%\newenvironment{reference}{}{}
\newenvironment{slogan}{\comment}{\endcomment}
\newenvironment{history}{\comment}{\endcomment}

% For commutative diagrams you can use
% \usepackage{amscd}
\usepackage[all]{xy}

% We use 2cell for 2-commutative diagrams.
\xyoption{2cell}
\UseAllTwocells

% To put source file link in headers.
% Change "template.tex" to "this_filename.tex"
% \usepackage{fancyhdr}
% \pagestyle{fancy}
% \lhead{}
% \chead{}
% \rhead{Source file: \url{template.tex}}
% \lfoot{}
% \cfoot{\thepage}
% \rfoot{}
% \renewcommand{\headrulewidth}{0pt}
% \renewcommand{\footrulewidth}{0pt}
% \renewcommand{\headheight}{12pt}

\usepackage{multicol}

% For cross-file-references
\usepackage{xr-hyper}

% Package for hypertext links:
\usepackage{hyperref}

% For any local file, say "hello.tex" you want to link to please
% use \externaldocument[hello-]{hello}
\externaldocument[introduction-]{introduction}
\externaldocument[conventions-]{conventions}
\externaldocument[sets-]{sets}
\externaldocument[categories-]{categories}
\externaldocument[topology-]{topology}
\externaldocument[sheaves-]{sheaves}
\externaldocument[sites-]{sites}
\externaldocument[stacks-]{stacks}
\externaldocument[fields-]{fields}
\externaldocument[algebra-]{algebra}
\externaldocument[brauer-]{brauer}
\externaldocument[homology-]{homology}
\externaldocument[derived-]{derived}
\externaldocument[simplicial-]{simplicial}
\externaldocument[more-algebra-]{more-algebra}
\externaldocument[smoothing-]{smoothing}
\externaldocument[modules-]{modules}
\externaldocument[sites-modules-]{sites-modules}
\externaldocument[injectives-]{injectives}
\externaldocument[cohomology-]{cohomology}
\externaldocument[sites-cohomology-]{sites-cohomology}
\externaldocument[dga-]{dga}
\externaldocument[dpa-]{dpa}
\externaldocument[hypercovering-]{hypercovering}
\externaldocument[schemes-]{schemes}
\externaldocument[constructions-]{constructions}
\externaldocument[properties-]{properties}
\externaldocument[morphisms-]{morphisms}
\externaldocument[coherent-]{coherent}
\externaldocument[divisors-]{divisors}
\externaldocument[limits-]{limits}
\externaldocument[varieties-]{varieties}
\externaldocument[topologies-]{topologies}
\externaldocument[descent-]{descent}
\externaldocument[perfect-]{perfect}
\externaldocument[more-morphisms-]{more-morphisms}
\externaldocument[flat-]{flat}
\externaldocument[groupoids-]{groupoids}
\externaldocument[more-groupoids-]{more-groupoids}
\externaldocument[etale-]{etale}
\externaldocument[chow-]{chow}
\externaldocument[intersection-]{intersection}
\externaldocument[pic-]{pic}
\externaldocument[adequate-]{adequate}
\externaldocument[dualizing-]{dualizing}
\externaldocument[duality-]{duality}
\externaldocument[discriminant-]{discriminant}
\externaldocument[local-cohomology-]{local-cohomology}
\externaldocument[curves-]{curves}
\externaldocument[resolve-]{resolve}
\externaldocument[models-]{models}
\externaldocument[pione-]{pione}
\externaldocument[etale-cohomology-]{etale-cohomology}
\externaldocument[proetale-]{proetale}
\externaldocument[crystalline-]{crystalline}
\externaldocument[spaces-]{spaces}
\externaldocument[spaces-properties-]{spaces-properties}
\externaldocument[spaces-morphisms-]{spaces-morphisms}
\externaldocument[decent-spaces-]{decent-spaces}
\externaldocument[spaces-cohomology-]{spaces-cohomology}
\externaldocument[spaces-limits-]{spaces-limits}
\externaldocument[spaces-divisors-]{spaces-divisors}
\externaldocument[spaces-over-fields-]{spaces-over-fields}
\externaldocument[spaces-topologies-]{spaces-topologies}
\externaldocument[spaces-descent-]{spaces-descent}
\externaldocument[spaces-perfect-]{spaces-perfect}
\externaldocument[spaces-more-morphisms-]{spaces-more-morphisms}
\externaldocument[spaces-flat-]{spaces-flat}
\externaldocument[spaces-groupoids-]{spaces-groupoids}
\externaldocument[spaces-more-groupoids-]{spaces-more-groupoids}
\externaldocument[bootstrap-]{bootstrap}
\externaldocument[spaces-pushouts-]{spaces-pushouts}
\externaldocument[groupoids-quotients-]{groupoids-quotients}
\externaldocument[spaces-more-cohomology-]{spaces-more-cohomology}
\externaldocument[spaces-simplicial-]{spaces-simplicial}
\externaldocument[formal-spaces-]{formal-spaces}
\externaldocument[restricted-]{restricted}
\externaldocument[spaces-resolve-]{spaces-resolve}
\externaldocument[formal-defos-]{formal-defos}
\externaldocument[defos-]{defos}
\externaldocument[cotangent-]{cotangent}
\externaldocument[examples-defos-]{examples-defos}
\externaldocument[algebraic-]{algebraic}
\externaldocument[examples-stacks-]{examples-stacks}
\externaldocument[stacks-sheaves-]{stacks-sheaves}
\externaldocument[criteria-]{criteria}
\externaldocument[artin-]{artin}
\externaldocument[quot-]{quot}
\externaldocument[stacks-properties-]{stacks-properties}
\externaldocument[stacks-morphisms-]{stacks-morphisms}
\externaldocument[stacks-limits-]{stacks-limits}
\externaldocument[stacks-cohomology-]{stacks-cohomology}
\externaldocument[stacks-perfect-]{stacks-perfect}
\externaldocument[stacks-introduction-]{stacks-introduction}
\externaldocument[stacks-more-morphisms-]{stacks-more-morphisms}
\externaldocument[stacks-geometry-]{stacks-geometry}
\externaldocument[moduli-]{moduli}
\externaldocument[moduli-curves-]{moduli-curves}
\externaldocument[examples-]{examples}
\externaldocument[exercises-]{exercises}
\externaldocument[guide-]{guide}
\externaldocument[desirables-]{desirables}
\externaldocument[coding-]{coding}
\externaldocument[obsolete-]{obsolete}
\externaldocument[fdl-]{fdl}
\externaldocument[index-]{index}

% Theorem environments.
%
\theoremstyle{plain}
\newtheorem{theorem}[subsection]{Theorem}
\newtheorem{proposition}[subsection]{Proposition}
\newtheorem{lemma}[subsection]{Lemma}

\theoremstyle{definition}
\newtheorem{definition}[subsection]{Definition}
\newtheorem{example}[subsection]{Example}
\newtheorem{exercise}[subsection]{Exercise}
\newtheorem{situation}[subsection]{Situation}

\theoremstyle{remark}
\newtheorem{remark}[subsection]{Remark}
\newtheorem{remarks}[subsection]{Remarks}

\numberwithin{equation}{subsection}

% Macros
%
\def\lim{\mathop{\rm lim}\nolimits}
\def\colim{\mathop{\rm colim}\nolimits}
\def\Spec{\mathop{\rm Spec}}
\def\Hom{\mathop{\rm Hom}\nolimits}
\def\Ext{\mathop{\rm Ext}\nolimits}
\def\SheafHom{\mathop{\mathcal{H}\!{\it om}}\nolimits}
\def\SheafExt{\mathop{\mathcal{E}\!{\it xt}}\nolimits}
\def\Sch{\textit{Sch}}
\def\Mor{\mathop{\rm Mor}\nolimits}
\def\Ob{\mathop{\rm Ob}\nolimits}
\def\Sh{\mathop{\textit{Sh}}\nolimits}
\def\NL{\mathop{N\!L}\nolimits}
\def\proetale{{pro\text{-}\acute{e}tale}}
\def\etale{{\acute{e}tale}}
\def\QCoh{\textit{QCoh}}
\def\Ker{\mathop{\rm Ker}}
\def\Im{\mathop{\rm Im}}
\def\Coker{\mathop{\rm Coker}}
\def\Coim{\mathop{\rm Coim}}

%
% Macros for moduli stacks/spaces
%
\def\QCohstack{\mathcal{QC}\!{\it oh}}
\def\Cohstack{\mathcal{C}\!{\it oh}}
\def\Spacesstack{\mathcal{S}\!{\it paces}}
\def\Quotfunctor{{\rm Quot}}
\def\Hilbfunctor{{\rm Hilb}}
\def\Curvesstack{\mathcal{C}\!{\it urves}}
\def\Polarizedstack{\mathcal{P}\!{\it olarized}}
\def\Complexesstack{\mathcal{C}\!{\it omplexes}}
% \Pic is the operator that assigns to X its picard group, usage \Pic(X)
% \Picardstack_{X/B} denotes the Picard stack of X over B
% \Picardfunctor_{X/B} denotes the Picard functor of X over B
\def\Pic{\mathop{\rm Pic}\nolimits}
\def\Picardstack{\mathcal{P}\!{\it ic}}
\def\Picardfunctor{{\rm Pic}}
\def\Deformationcategory{\mathcal{D}\!{\it ef}}


% OK, start here.
%
\begin{document}

\title{Cohomology of Algebraic Spaces}

\maketitle

\phantomsection
\label{section-phantom}

\tableofcontents




\section{Introduction}
\label{section-introduction}

\noindent
In this chapter we write about cohomology of algebraic spaces.
This mean in particular cohomology of quasi-coherent sheaves, i.e.,
we prove analogues of the results in the chapter entitled
``Coherent Cohomology''. Some of the results in this chapter can
be found in \cite{Kn}.





\section{Conventions}
\label{section-conventions}

\noindent
The standing assumption is that all schemes are contained in
a big fppf site $\Sch_{fppf}$. And all rings $A$ considered
have the property that $\Spec(A)$ is (isomorphic) to an
object of this big site.

\medskip\noindent
Let $S$ be a scheme and let $X$ be an algebraic space over $S$.
In this chapter and the following we will write $X \times_S X$
for the product of $X$ with itself (in the category of algebraic
spaces over $S$), instead of $X \times X$.




\section{Derived category of quasi-coherent modules}
\label{section-derived-quasi-coherent}

\noindent
Let $S$ be a scheme. In
Descent, Lemma \ref{descent-lemma-derived-quasi-coherent-small-etale-site}
we proved that the category $D_{QCoh}(\mathcal{O}_S)$ can be defined
in terms of complexes of $\mathcal{O}_S$-modules on the scheme $S$
or by complexes of $\mathcal{O}$-modules on the small \'etale site
of $S$. Hence the following definition is compatible with the definition
in the case of schemes.

\begin{definition}
\label{definition-derived-quasi-coherent}
Let $S$ be a scheme. Let $X$ be an algebraic space over $S$.
The {\it derived category of $\mathcal{O}_X$-modules with
quasi-coherent cohomology sheaves} is denoted
$D_{QCoh}(\mathcal{O}_X)$.
\end{definition}

\noindent
This makes sense by
Properties of Spaces, Lemma
\ref{spaces-properties-lemma-properties-quasi-coherent}
and
Derived Categories, Section \ref{derived-section-triangulated-sub}.








\section{Higher direct images}
\label{section-higher-direct-image}

\noindent
Let $S$ be a scheme. Let $f : X \to Y$ be a quasi-compact and quasi-separated
morphism of representable algebraic spaces $X$ and $Y$ over $S$. Let
$\mathcal{F}$ be a quasi-coherent module on $X$. By
Descent, Lemma \ref{descent-lemma-higher-direct-images-small-etale}
the sheaf $R^if_*\mathcal{F}$ agrees with the
usual higher direct image (computed for the Zariski topologies)
if we think of $X$ and $Y$ as schemes.

\medskip\noindent
More generally, suppose $f : X \to Y$ is a representable, quasi-compact, and
quasi-separated morphism of algebraic spaces over $S$. Let $V$ be a scheme
and let $V \to Y$ be an \'etale surjective morphism. Let $U = V \times_Y X$
and let $f' : U \to V$ be the base change of $f$. Then for any
quasi-coherent $\mathcal{O}_X$-module $\mathcal{F}$ we have
\begin{equation}
\label{equation-representable-higher-direct-image}
R^if'_*(\mathcal{F}|_U) = (R^if_*\mathcal{F})|_V,
\end{equation}
see
Properties of Spaces,
Lemma \ref{spaces-properties-lemma-pushforward-etale-base-change-modules}.
And because $f' : U \to V$ is a quasi-compact and quasi-separated
morphism of schemes, by the remark of the preceding paragraph we may
compute $R^if'_*(\mathcal{F}|_U)$ by thinking of $\mathcal{F}|_U$ as a
quasi-coherent sheaf on the scheme $U$, and $f'$ as a morphism of schemes.
We will frequently use this without further mention.

\medskip\noindent
Next, we prove that higher direct images of quasi-coherent sheaves are
quasi-coherent for any quasi-compact and quasi-separated morphism of
algebraic spaces. In the proof we use a trick; a ``better'' proof would
use a relative Cech complex, as discussed in
Sheaves on Stacks, Sections \ref{stacks-sheaves-section-cech} and
\ref{stacks-sheaves-section-sheaf-cech-complex} ff.

\begin{lemma}
\label{lemma-higher-direct-image}
Let $S$ be a scheme. Let $f : X \to Y$ be a morphism of algebraic spaces
over $S$. If $f$ is quasi-compact and quasi-separated, then $R^if_*$
transforms quasi-coherent $\mathcal{O}_X$-modules into
quasi-coherent $\mathcal{O}_Y$-modules and induces a functor
$Rf_* : D_{QCoh}^+(\mathcal{O}_X) \to D_{QCoh}^+(\mathcal{O}_Y)$.
\end{lemma}

\begin{proof}
Let $V \to Y$ be an \'etale morphism where $V$ is an affine scheme. Set
$U = V \times_Y X$ and denote $f' : U \to V$ the induced morphism.
Let $\mathcal{I}^\bullet$ be a bounded above complex of injective
$\mathcal{O}_X$-modules. By
Properties of Spaces, Lemma
\ref{spaces-properties-lemma-pushforward-etale-base-change-modules}
we have
$$
f'_*(\mathcal{I}^\bullet|_U) = (f_*\mathcal{I}^\bullet)|_V.
$$
The complex $\mathcal{I}^\bullet|_U$ is a bounded below complex
of injective $\mathcal{O}_U$-modules, see
Cohomology on Sites, Lemma \ref{sites-cohomology-lemma-cohomology-of-open}.
Since the property of being a quasi-coherent module is local in the
\'etale topology on $Y$ (see
Properties of Spaces, Lemma
\ref{spaces-properties-lemma-characterize-quasi-coherent})
we may replace $Y$ by $V$, i.e., we may assume $Y$ is an affine scheme.

\medskip\noindent
Assume $Y$ is affine. Since $f$ is quasi-compact we see that $X$
is quasi-compact. Thus we may choose an affine scheme $U$ and a surjective
\'etale morphism $g : U \to X$, see
Properties of Spaces,
Lemma \ref{spaces-properties-lemma-quasi-compact-affine-cover}.
Note that the morphism $g : U \to X$ is representable, separated
and quasi-compact because $X$ is quasi-separated. Hence the lemma
holds for $g$ (either by the discussion above the lemma or by
applying the reduction in the first paragraph of this proof).
It also holds for $f \circ g : U \to Y$ (as this is a morphism
of affine schemes). Moreover, for an injective $\mathcal{O}_U$-module
$\mathcal{I}$ the module $g_*\mathcal{I}$ is injective (see
Homology, Lemma \ref{homology-lemma-adjoint-preserve-injectives})
whence $Rf_* \circ Rg_* = R(g \circ f)_*$, see
Derived Categories, Lemma \ref{derived-lemma-compose-derived-functors}.

\medskip\noindent
In the situation described in the previous paragraph we will show by
induction on $n$ that $IH_n$: for any quasi-coherent sheaf $\mathcal{F}$
on $X$ the sheaves $R^if\mathcal{F}$
are quasi-coherent for $i \leq n$.
The case $n = 0$ follows from
Morphisms of Spaces, Lemma \ref{spaces-morphisms-lemma-pushforward}.
Assume $IH_n$. In the rest of the proof we show that $IH_{n + 1}$ holds.

\medskip\noindent
The hypothesis $IH_n$ implies, via the spectral sequence of
Derived Categories, Lemma \ref{derived-lemma-two-ss-complex-functor},
that $R^if_*\mathcal{G}^\bullet$ is quasi-coherent for $i \leq n$ if
$\mathcal{G}^\bullet$ is a complex of $\mathcal{O}_X$-modules with
$H^j(\mathcal{G}^\bullet) = 0$ for $j < 0$ and $H^j(\mathcal{G}^\bullet)$
is quasi-coherent for all $j$. Suppose $\mathcal{H}$ is
a quasi-coherent $\mathcal{O}_U$-module. Consider the distinguished triangle
$$
g_*\mathcal{H} \to Rg_*\mathcal{H} \to
\tau_{\geq 1}Rg_*\mathcal{H} \to g_*\mathcal{H}[1].
$$
Note that $Rg_*\mathcal{H}$ and
$Rf_*Rg_*\mathcal{H} = R(f \circ g)_*\mathcal{H}$
have quasi-coherent cohomology sheaves (see above). Combined with
the remark above we conclude that $IH_n$ implies that
$R^if_*g_*\mathcal{H}$ is quasi-coherent for $i \leq n + 1$.

\medskip\noindent
Let $\mathcal{F}$ be a quasi-coherent $\mathcal{O}_X$-module. Consider
the exact sequence
$$
0 \to \mathcal{F} \to g_*g^*\mathcal{F} \to \mathcal{G} \to 0
$$
where $\mathcal{G}$ is the cokernel of the first map.
Applying the long exact cohomology sequence we obtain
$$
R^nf_*g_*g^*\mathcal{F} \to
R^nf_*\mathcal{G} \to
R^{n + 1}f_*\mathcal{F} \to
R^{n + 1}f_*g_*g^*\mathcal{F} \to
R^{n + 1}f_*\mathcal{G}
$$
By the above we see that $R^{n + 1}f_*g_*g^*\mathcal{F}$
is quasi-coherent. Thus $R^{n + 1}f_*\mathcal{F}$ has a $2$-step filtration
where the first step is quasi-coherent and the second a subsheaf of
a quasi-coherent sheaf. Applying this to $R^{n + 1}f_*\mathcal{G}$
we find an exact sequence $0 \to \mathcal{A} \to R^{n + 1}f_*\mathcal{G}
\to \mathcal{B}$ wit $\mathcal{A}$, $\mathcal{B}$ quasi-coherent
$\mathcal{O}_Y$-modules. Then the kernel $\mathcal{K}$ of
$R^{n + 1}f_*g_*g^*\mathcal{F} \to R^{n + 1}f_*\mathcal{G}
\to \mathcal{B}$ is quasi-coherent, whereupon we obtain a map
$\mathcal{K} \to \mathcal{A}$ whose kernel $\mathcal{K}'$ is
quasi-coherent too. Hence $R^{n + 1}f_*\mathcal{F}$ sits in an exact
sequence
$$
R^nf_*g_*g^*\mathcal{F} \to
R^nf_*\mathcal{G} \to
R^{n + 1}f_*\mathcal{F} \to \mathcal{K}' \to 0
$$
and we win.
\end{proof}




\section{Colimits and cohomology}
\label{section-colimits}

\noindent
The following lemma in particular applies to diagrams of quasi-coherent
sheaves.

\begin{lemma}
\label{lemma-colimits}
Let $S$ be a scheme. Let $X$ be an algebraic space over $S$.
If $X$ is quasi-compact and quasi-separated, then
$$
\colim_i H^p(X, \mathcal{F}_i)
\longrightarrow
H^p(X, \colim_i \mathcal{F}_i)
$$
for every filtered diagram of abelian sheaves on $X_{\acute{e}tale}$.
\end{lemma}

\begin{proof}
This follows from
Cohomology on Sites, Lemma
\ref{sites-cohomology-lemma-colim-works-over-collection}.
Namely, let $\mathcal{B} \subset \Ob(X_{spaces, \acute{e}tale})$
be the set of quasi-compact and quasi-separated spaces \'etale over $X$.
Note that if $U \in \mathcal{B}$ then, because $U$ is quasi-compact,
the collection of finite coverings $\{U_i \to U\}$ with $U_i \in \mathcal{B}$
is cofinal in the set of coverings of $U$ in $X_{\acute{e}tale}$. By
Morphisms of Spaces, Lemma
\ref{spaces-morphisms-lemma-quasi-compact-quasi-separated-permanence}
the set $\mathcal{B}$ satisfies all the assumptions of
Cohomology on Sites, Lemma
\ref{sites-cohomology-lemma-colim-works-over-collection}.
Since $X \in \mathcal{B}$ we win.
\end{proof}

\begin{lemma}
\label{lemma-colimit-cohomology}
Let $S$ be a scheme. Let $f : X \to Y$ be a quasi-compact and quasi-separated
morphism of algebraic spaces over $S$. Let $\mathcal{F} = \colim \mathcal{F}_i$
be a filtered colimit of abelian sheaves on $X_{\acute{e}tale}$.
Then for any $p \geq 0$ we have
$$
R^pf_*\mathcal{F} = \colim R^pf_*\mathcal{F}_i.
$$
\end{lemma}

\begin{proof}
Recall that $R^pf_*\mathcal{F}$ is the sheaf on $Y_{spaces, \acute{e}tale}$
associated to $V \mapsto H^p(V \times_Y X, \mathcal{F})$, see
Cohomology on Sites, Lemma \ref{sites-cohomology-lemma-higher-direct-images}
and Properties of Spaces, Lemma
\ref{spaces-properties-lemma-functoriality-etale-site}.
Recall that the colimit is the sheaf associated to the presheaf colimit.
Hence we can apply Lemma \ref{lemma-colimits}
to $H^p(V \times_Y X, -)$ where $V$ is affine to conclude (because
when $V$ is affine, then $V \times_Y X$ is quasi-compact and quasi-separated).
Strictly speaking this also uses Properties of Spaces,
Lemma \ref{spaces-properties-lemma-alternative} to see that there exist
enough affine objects.
\end{proof}

\noindent
The following lemma tells us that finitely presented modules behave
as expected in quasi-compact and quasi-separated algebraic spaces.

\begin{lemma}
\label{lemma-finite-presentation-quasi-compact-colimit}
Let $S$ be a scheme. Let $X$ be a quasi-compact and quasi-separated
algebraic space over $S$. Let $I$ be a partially ordered set and
let $(\mathcal{F}_i, \varphi_{ii'})$ be a system over $I$
of quasi-coherent $\mathcal{O}_X$-modules. Let $\mathcal{G}$ be an
$\mathcal{O}_X$-module of finite presentation. Then we have
$$
\colim_i \Hom_X(\mathcal{G}, \mathcal{F}_i)
=
\Hom_X(\mathcal{G}, \colim_i \mathcal{F}_i).
$$
\end{lemma}

\begin{proof}
Choose an affine scheme $U$ and a surjective \'etale morphism
$U \to X$. Set $R = U \times_X U$. Note that $R$ is a quasi-compact
(as $X$ is quasi-separated and $U$ quasi-compact) and separated (as
$U$ is separated) scheme. Hence we have
$$
\colim_i \Hom_U(\mathcal{G}|_U, \mathcal{F}_i|_U)
=
\Hom_U(\mathcal{G}|_U, \colim_i \mathcal{F}_i|_U).
$$
by Modules, Lemma \ref{modules-lemma-finite-presentation-quasi-compact-colimit}
(and the material on restriction to
schemes \'etale over $X$, see
Properties of Spaces, Sections \ref{spaces-properties-section-quasi-coherent}
and \ref{spaces-properties-section-properties-modules}). Similarly for $R$.
Since $\textit{QCoh}(X) = \textit{QCoh}(U, R, s, t, c)$ (see
Properties of Spaces, Proposition
\ref{spaces-properties-proposition-quasi-coherent})
the result follows formally.
\end{proof}




\section{The alternating {\v C}ech complex}
\label{section-alternating-cech}

\noindent
Let $S$ be a scheme. Let $f : U \to X$ be an \'etale morphism of algebraic
spaces over $S$. The functor
$$
j : U_{spaces, \acute{e}tale} \longrightarrow X_{spaces, \acute{e}tale},\quad
V/U \longmapsto V/X
$$
induces an equivalence of $U_{spaces, \acute{e}tale}$ with the localization
$X_{spaces, \acute{e}tale}/U$, see
Properties of Spaces, Section \ref{spaces-properties-section-localize}.
Hence there exist functors
$$
f_! : \textit{Ab}(U_{\acute{e}tale}) \longrightarrow
\textit{Ab}(X_{\acute{e}tale}),\quad
f_! : \textit{Mod}(\mathcal{O}_U) \longrightarrow \textit{Mod}(\mathcal{O}_X),
$$
which are left adjoint to
$$
f^{-1} : \textit{Ab}(X_{\acute{e}tale}) \longrightarrow
\textit{Ab}(U_{\acute{e}tale}),\quad
f^* : \textit{Mod}(\mathcal{O}_X) \longrightarrow \textit{Mod}(\mathcal{O}_U)
$$
see
Modules on Sites, Section \ref{sites-modules-section-localize}.
Warning: This functor, a priori, has
nothing to do with cohomology with compact supports!
We dubbed this functor ``extension by zero'' in the reference above.
Note that the two versions of $f_!$ agree as $f^* = f^{-1}$ for
sheaves of $\mathcal{O}_X$-modules.

\medskip\noindent
As we are going to use this construction below let us recall some of its
properties. Given an abelian sheaf $\mathcal{G}$ on $U_{\acute{e}tale}$
the sheaf $f_!$ is the sheafification of the presheaf
$$
V/X \longmapsto
f_!\mathcal{G}(V) =
\bigoplus\nolimits_{\varphi \in \Mor_X(V, U)}
\mathcal{G}(V \xrightarrow{\varphi} U),
$$
see
Modules on Sites, Lemma \ref{sites-modules-lemma-extension-by-zero}.
Moreover, if $\mathcal{G}$ is an $\mathcal{O}_U$-module, then $f_!\mathcal{G}$
is the sheafification of the exact same presheaf of abelian groups which
is endowed with an $\mathcal{O}_X$-module structure in an obvious way
(see loc.\ cit.). Let $\overline{x} : \Spec(k) \to X$
be a geometric point. Then there is a canonical identification
$$
(f_!\mathcal{G})_{\overline{x}} =
\bigoplus\nolimits_{\overline{u}} \mathcal{G}_{\overline{u}}
$$
where the sum is over all $\overline{u} : \Spec(k) \to U$ such that
$f \circ \overline{u} = \overline{x}$, see
Modules on Sites, Lemma \ref{sites-modules-lemma-stalk-j-shriek}.
In the following we are going to study the sheaf $f_!\underline{\mathbf{Z}}$.
Here $\underline{\mathbf{Z}}$ denotes the constant sheaf on
$X_{\acute{e}tale}$ or $U_{\acute{e}tale}$.

\begin{lemma}
\label{lemma-product-is-tensor-product}
Let $S$ be a scheme. Let $f_i : U_i \to X$ be \'etale morphisms
of algebraic spaces over $S$. Then there are isomorphisms
$$
f_{1, !}\underline{\mathbf{Z}} \otimes_{\mathbf{Z}}
f_{2, !}\underline{\mathbf{Z}}
\longrightarrow
f_{12, !}\underline{\mathbf{Z}}
$$
where $f_{12} : U_1 \times_X U_2 \to X$ is the structure morphism
and
$$
(f_1 \amalg f_2)_! \underline{\mathbf{Z}}
\longrightarrow
f_{1, !}\underline{\mathbf{Z}} \oplus
f_{2, !}\underline{\mathbf{Z}}
$$
\end{lemma}

\begin{proof}
Once we have defined the map it will be an isomorphism by our description
of stalks above. To define the map it suffices to work on the level of
presheaves. Thus we have to define a map
$$
\left(\bigoplus\nolimits_{\varphi_1 \in \Mor_X(V, U_1)} \mathbf{Z}\right)
\otimes_{\mathbf{Z}}
\left(\bigoplus\nolimits_{\varphi_2 \in \Mor_X(V, U_2)} \mathbf{Z}\right)
\longrightarrow
\bigoplus\nolimits_{\varphi \in \Mor_X(V, U_1 \times_X U_2)}
\mathbf{Z}
$$
We map the element $1_{\varphi_1} \otimes 1_{\varphi_2}$ to the element
$1_{\varphi_1 \times \varphi_2}$ with obvious notation. We omit the proof
of the second equality.
\end{proof}

\noindent
Another important feature is the trace map
$$
\text{Tr}_f : f_!\underline{\mathbf{Z}} \longrightarrow \underline{\mathbf{Z}}.
$$
The trace map is adjoint to the
map $\mathbf{Z} \to f^{-1}\underline{\mathbf{Z}}$ (which is an isomorphism).
If $\overline{x}$ is above, then $\text{Tr}_f$ on stalks at $\overline{x}$
is the map
$$
(\text{Tr}_f)_{\overline{x}} :
(f_!\underline{\mathbf{Z}})_{\overline{x}} =
\bigoplus\nolimits_{\overline{u}} \mathbf{Z}
\longrightarrow
\mathbf{Z} = \underline{\mathbf{Z}}_{\overline{x}}
$$
which sums the given integers. This is true because it is adjoint to the map
$1 : \mathbf{Z} \to f^{-1}\underline{\mathbf{Z}}$. In particular, if
$f$ is surjective as well as \'etale then $\text{Tr}_f$ is surjective.

\medskip\noindent
Assume that $f : U \to X$ is a surjective \'etale
morphism of algebraic spaces. Consider the {\it Koszul complex}
associated to the trace map we discussed above
$$
\ldots \to \wedge^3f_!\underline{\mathbf{Z}} \to
\wedge^2f_!\underline{\mathbf{Z}} \to f_!\underline{\mathbf{Z}} \to
\underline{\mathbf{Z}} \to 0
$$
Here the exterior powers are over the sheaf of rings $\underline{\mathbf{Z}}$.
The maps are defined by the rule
$$
e_1 \wedge \ldots \wedge e_n \longmapsto
\sum\nolimits_{i = 1, \ldots, n} (-1)^{i + 1}
\text{Tr}_f(e_i)
e_1 \wedge \ldots \wedge \widehat{e_i} \wedge \ldots \wedge e_n
$$
where $e_1, \ldots, e_n$ are local sections of $f_!\underline{\mathbf{Z}}$.
Let $\overline{x}$ be a geometric point of $X$ and set
$M_{\overline{x}} = (f_!\underline{\mathbf{Z}})_{\overline{x}} =
\bigoplus_{\overline{u}} \mathbf{Z}$. Then the stalk of the complex above at
$\overline{x}$ is the complex
$$
\ldots \to \wedge^3 M_{\overline{x}} \to \wedge^2 M_{\overline{x}}
\to M_{\overline{x}} \to \mathbf{Z} \to 0
$$
which is exact because $M_{\overline{x}} \to \mathbf{Z}$ is surjective, see
More on Algebra, Lemma \ref{more-algebra-lemma-homotopy-koszul-abstract}.
Hence if we let $K^\bullet = K^\bullet(f)$ be the complex with
$K^i = \wedge^{i + 1}f_!\underline{\mathbf{Z}}$, then we obtain a
quasi-isomorphism
\begin{equation}
\label{equation-quasi-isomorphism}
K^\bullet \longrightarrow \underline{\mathbf{Z}}[0]
\end{equation}
We use the complex $K^\bullet$ to define what we call
the alternating {\v C}ech complex associated to $f : U \to X$.

\begin{definition}
\label{definition-alternating-cech-complex}
Let $S$ be a scheme. Let $f : U \to X$ be a surjective \'etale morphism
of algebraic spaces over $S$. Let $\mathcal{F}$ be an object of
$\textit{Ab}(X_{\acute{e}tale})$. The
{\it alternating {\v C}ech complex}\footnote{This may be nonstandard notation}
$\check{\mathcal{C}}^\bullet_{alt}(f, \mathcal{F})$
associated to $\mathcal{F}$ and $f$ is the complex
$$
\Hom(K^0, \mathcal{F}) \to \Hom(K^1, \mathcal{F}) \to
\Hom(K^2, \mathcal{F}) \to \ldots
$$
with Hom groups computed in $\textit{Ab}(X_{\acute{e}tale})$.
\end{definition}

\noindent
The reader may verify that if $U = \coprod U_i$ and $f|_{U_i} : U_i \to X$
is the open immersion of a subspace, then
$\check{\mathcal{C}}_{alt}^\bullet(f, \mathcal{F})$ agrees with the complex
introduced in
Cohomology, Section \ref{cohomology-section-alternating-cech}
for the Zariski covering $X = \bigcup U_i$ and the restriction
of $\mathcal{F}$ to the Zariski site of $X$. What is more important
however, is to relate the cohomology of the alternating
{\v C}ech complex to the cohomology.

\begin{lemma}
\label{lemma-alternating-cech-to-cohomology}
Let $S$ be a scheme. Let $f : U \to X$ be a surjective \'etale morphism
of algebraic spaces over $S$. Let $\mathcal{F}$ be an object of
$\textit{Ab}(X_{\acute{e}tale})$. There exists a canonical map
$$
\check{\mathcal{C}}^\bullet_{alt}(f, \mathcal{F})
\longrightarrow
R\Gamma(X, \mathcal{F})
$$
in $D(\textit{Ab})$. Moreover, there is a spectral sequence with $E_1$-page
$$
E_1^{p, q} =
\text{Ext}_{\textit{Ab}(X_{\acute{e}tale})}^q(K^p, \mathcal{F})
$$
converging to $H^{p + q}(X, \mathcal{F})$ where
$K^p = \wedge^{p + 1}f_!\underline{\mathbf{Z}}$.
\end{lemma}

\begin{proof}
Recall that we have the quasi-isomorphism
$K^\bullet \to \underline{\mathbf{Z}}[0]$, see
(\ref{equation-quasi-isomorphism}).
Choose an injective resolution $\mathcal{F} \to \mathcal{I}^\bullet$
in $\textit{Ab}(X_{\acute{e}tale})$. Consider the double complex
$A^{\bullet, \bullet}$ with terms
$$
A^{p, q} = \Hom(K^p, \mathcal{I}^q)
$$
where the differential $d_1^{p, q} : A^{p, q} \to A^{p + 1, q}$
is the one coming from the differential $K^{p + 1} \to K^p$
and the differential $d_2^{p, q} : A^{p, q} \to A^{p, q + 1}$ is the
one coming from the differential
$\mathcal{I}^q \to \mathcal{I}^{q + 1}$.
Denote $sA^\bullet$ the total complex associated to
the double complex $A^{\bullet, \bullet}$.
We will use the two spectral
sequences $({}'E_r, {}'d_r)$ and $({}''E_r, {}''d_r)$
associated to this double complex, see
Homology, Section \ref{homology-section-double-complex}.

\medskip\noindent
Because $K^\bullet$ is a resolution of $\underline{\mathbf{Z}}$
we see that the complexes
$$
A^{\bullet, q} :
\Hom(K^0, \mathcal{I}^q) \to
\Hom(K^1, \mathcal{I}^q) \to
\Hom(K^2, \mathcal{I}^q) \to \ldots
$$
are acyclic in positive degrees and have $H^0$ equal to
$\Gamma(X, \mathcal{I}^q)$. Hence by
Homology, Lemma \ref{homology-lemma-double-complex-gives-resolution}
and its proof the spectral sequence $({}''E_r, {}''d_r)$ degenerates,
and the natural map
$$
\mathcal{I}^\bullet(X) \longrightarrow sA^\bullet
$$
is a quasi-isomorphism of complexes of abelian groups. In particular
we conclude that $H^n(sA^\bullet) = H^n(X, \mathcal{F})$.

\medskip\noindent
The map $\check{\mathcal{C}}^\bullet_{alt}(f, \mathcal{F}) \to
R\Gamma(X, \mathcal{F})$ of the lemma is the composition of
$\check{\mathcal{C}}^\bullet_{alt}(f, \mathcal{F}) \to SA^\bullet$
with the inverse of the displayed quasi-isomorphism.

\medskip\noindent
Finally, consider the spectral sequence $({}'E_r, {}'d_r)$.
We have
$$
E_1^{p, q} = q\text{th cohomology of }
\Hom(K^p, \mathcal{I}^0) \to
\Hom(K^p, \mathcal{I}^1) \to
\Hom(K^p, \mathcal{I}^2) \to \ldots
$$
This proves the lemma.
\end{proof}

\noindent
It follows from the lemma that it is important to understand the
ext groups $\text{Ext}_{\textit{Ab}(X_{\acute{e}tale})}(K^p, \mathcal{F})$,
i.e., the right derived functors of
$\mathcal{F} \mapsto \Hom(K^p, \mathcal{F})$.

\begin{lemma}
\label{lemma-compute}
Let $S$ be a scheme. Let $f : U \to X$ be a surjective, \'etale, and separated
morphism of algebraic spaces over $S$. For $p \geq 0$ set
$$
W_p = U \times_X \ldots \times_X U \setminus \text{all diagonals}
$$
where the fibre product has $p + 1$ factors.
There is a free action of $S_{p + 1}$ on $W_p$ over $X$ and
$$
\Hom(K^p, \mathcal{F}) = S_{p + 1}\text{-anti-invariant elements of }
\mathcal{F}(W_p)
$$
functorially in $\mathcal{F}$ where
$K^p = \wedge^{p + 1}f_!\underline{\mathbf{Z}}$.
\end{lemma}

\begin{proof}
Because $U \to X$ is separated the diagonal $U \to U \times_X U$ is a
closed immersion. Since $U \to X$ is \'etale the diagonal
$U \to U \times_X U$ is an open immersion, see
Morphisms of Spaces, Lemmas
\ref{spaces-morphisms-lemma-etale-unramified} and
\ref{spaces-morphisms-lemma-diagonal-unramified-morphism}.
Hence $W_p$ is an open and closed subspace of
$U^{p + 1} = U \times_X \ldots \times_X U$. The action of $S_{p + 1}$
on $W_p$ is free as we've thrown out the fixed points of the action.
By
Lemma \ref{lemma-product-is-tensor-product}
we see that
$$
(f_!\underline{\mathbf{Z}})^{\otimes p + 1} =
f^{p + 1}_!\underline{\mathbf{Z}} = (W_p \to X)_!\underline{\mathbf{Z}}
\oplus Rest
$$
where $f^{p + 1} : U^{p + 1} \to X$ is the structure morphism.
Looking at stalks over a geometric point $\overline{x}$ of $X$
we see that
$$
\left(
\bigoplus\nolimits_{\overline{u} \mapsto \overline{x}} \mathbf{Z}
\right)^{\otimes p + 1}
\longrightarrow
(W_p \to X)_!\underline{\mathbf{Z}}_{\overline{x}}
$$
is the quotient whose kernel is generated by all tensors
$1_{\overline{u}_0} \otimes \ldots \otimes 1_{\overline{u}_p}$
where $\overline{u}_i = \overline{u}_j$ for some $i \not = j$.
Thus the quotient map
$$
(f_!\underline{\mathbf{Z}})^{\otimes p + 1}
\longrightarrow
\wedge^{p + 1}f_!\underline{\mathbf{Z}}
$$
factors through $(W_p \to X)_!\underline{\mathbf{Z}}$, i.e., we get
$$
(f_!\underline{\mathbf{Z}})^{\otimes p + 1}
\longrightarrow
(W_p \to X)_!\underline{\mathbf{Z}}
\longrightarrow
\wedge^{p + 1}f_!\underline{\mathbf{Z}}
$$
This already proves that $\Hom(K^p, \mathcal{F})$ is (functorially) a
subgroup of
$$
\Hom((W_p \to X)_!\underline{\mathbf{Z}}, \mathcal{F}) = \mathcal{F}(W_p)
$$
To identify it with the $S_{p + 1}$-anti-invariants we have to prove that
the surjection $(W_p \to X)_!\underline{\mathbf{Z}}
\to \wedge^{p + 1}f_!\underline{\mathbf{Z}}$ is the maximal
$S_{p + 1}$-anti-invariant quotient. In other words, we have to show that
$\wedge^{p + 1}f_!\underline{\mathbf{Z}}$ is the quotient of
$(W_p \to X)_!\underline{\mathbf{Z}}$ by the subsheaf generated by
the local sections $s - \text{sign}(\sigma)\sigma(s)$ where $s$ is
a local section of $(W_p \to X)_!\underline{\mathbf{Z}}$.
This can be checked on the stacks, where it is clear.
\end{proof}

\begin{lemma}
\label{lemma-twist}
Let $S$ be a scheme. Let $W$ be an algebraic space over $S$.
Let $G$ be a finite group acting freely on $W$.
Let $U = W/G$, see
Properties of Spaces, Lemma \ref{spaces-properties-lemma-quotient}.
Let $\chi : G \to \{+1, -1\}$ be a character.
Then there exists a rank 1 locally free sheaf of $\mathbf{Z}$-modules
$\underline{\mathbf{Z}}(\chi)$ on $U_{\acute{e}tale}$ such that for every
abelian sheaf $\mathcal{F}$ on $U_{\acute{e}tale}$ we have
$$
H^0(W, \mathcal{F}|_W)^\chi =
H^0(U, \mathcal{F} \otimes_{\mathbf{Z}} \underline{\mathbf{Z}}(\chi))
$$
\end{lemma}

\begin{proof}
The quotient morphism $q : W \to U$ is a $G$-torsor, i.e., there exists
a surjective \'etale morphism $U' \to U$ such that
$W \times_U U' = \coprod_{g \in G} U'$ as spaces with $G$-action over $U'$.
(Namely, $U' = W$ works.) Hence $q_*\underline{\mathbf{Z}}$ is a finite
locally free $\mathbf{Z}$-module with an action of $G$. For any
geometric point $\overline{u}$ of $U$, then we get $G$-equivariant
isomorphisms
$$
(q_*\underline{\mathbf{Z}})_{\overline{u}}
= \bigoplus\nolimits_{\overline{w} \mapsto \overline{u}} \mathbf{Z}
= \bigoplus\nolimits_{g \in G} \mathbf{Z} = \mathbf{Z}[G]
$$
where the second $=$ uses a geometric point
$\overline{w}_0$ lying over $\overline{u}$ and
maps the summand corresponding to $g \in G$ to the summand
corresponding to $g(\overline{w}_0)$. We have
$$
H^0(W, \mathcal{F}|_W) =
H^0(U, \mathcal{F} \otimes_\mathbf{Z} q_*\underline{\mathbf{Z}})
$$
because
$q_*\mathcal{F}|_W = \mathcal{F} \otimes_\mathbf{Z} q_*\underline{\mathbf{Z}}$
as one can check by restricting to $U'$. Let
$$
\underline{\mathbf{Z}}(\chi) =
(q_*\underline{\mathbf{Z}})^\chi \subset
q_*\underline{\mathbf{Z}}
$$
be the subsheaf of sections that transform according to $\chi$. For
any geometric point $\overline{u}$ of $U$ we have
$$
\underline{\mathbf{Z}}(\chi)_{\overline{u}} =
\mathbf{Z} \cdot \sum\nolimits_g \chi(g) g
\subset
\mathbf{Z}[G] = (q_*\underline{\mathbf{Z}})_{\overline{u}}
$$
It follows that $\underline{\mathbf{Z}}(\chi)$ is locally free of
rank 1 (more precisely, this should be checked after restricting to $U'$).
Note that for any $\mathbf{Z}$-module $M$ the $\chi$-semi-invariants
of $M[G]$ are the elements of the form $m \cdot \sum\nolimits_g \chi(g) g$.
Thus we see that for any abelian sheaf $\mathcal{F}$ on $U$ we have
$$
\left(\mathcal{F} \otimes_\mathbf{Z} q_*\underline{\mathbf{Z}}\right)^\chi
=
\mathcal{F} \otimes_\mathbf{Z} \underline{\mathbf{Z}}(\chi)
$$
because we have equality at all stalks. The result of the lemma follows by
taking global sections.
\end{proof}

\noindent
Now we can put everything together and obtain the following
pleasing result.

\begin{lemma}
\label{lemma-alternating-spectral-sequence}
Let $S$ be a scheme. Let $f : U \to X$ be a surjective, \'etale, and
separated morphism of algebraic spaces over $S$. For $p \geq 0$ set
$$
W_p = U \times_X \ldots \times_X U \setminus \text{all diagonals}
$$
(with $p + 1$ factors) as in Lemma \ref{lemma-compute}.
Let $\chi_p : S_{p + 1} \to \{+1, -1\}$ be the sign character.
Let $U_p = W_p/S_{p + 1}$ and $\underline{\mathbf{Z}}(\chi_p)$ be as in
Lemma \ref{lemma-twist}.
Then the spectral sequence of
Lemma \ref{lemma-alternating-cech-to-cohomology}
has $E_1$-page
$$
E_1^{p, q} =
H^q(U_p, \mathcal{F}|_{U_p} \otimes_\mathbf{Z} \underline{\mathbf{Z}}(\chi_p))
$$
and converges to $H^{p + q}(X, \mathcal{F})$.
\end{lemma}

\begin{proof}
Note that since the action of $S_{p + 1}$ on $W_p$ is over $X$ we do
obtain a morphism $U_p \to X$. Since $W_p \to X$ is \'etale and since
$W_p \to U_p$ is surjective \'etale, it follows
that also $U_p \to X$ is \'etale, see
Morphisms of Spaces, Lemma \ref{spaces-morphisms-lemma-etale-local}.
Therefore an injective object of
$\textit{Ab}(X_{\acute{e}tale})$ restricts to an injective object of
$\textit{Ab}(U_{p, \acute{e}tale})$, see
Cohomology on Sites, Lemma \ref{sites-cohomology-lemma-cohomology-of-open}.
Moreover, the functor
$\mathcal{G} \mapsto
\mathcal{G} \otimes_\mathbf{Z} \underline{\mathbf{Z}}(\chi_p))$
is an auto-equivalence of $\textit{Ab}(U_p)$, whence transforms injective
objects into injective objects and is exact (because
$\underline{\mathbf{Z}}(\chi_p)$ is an invertible
$\underline{\mathbf{Z}}$-module). Thus given an injective resolution
$\mathcal{F} \to \mathcal{I}^\bullet$ in $\textit{Ab}(X_{\acute{e}tale})$
the complex
$$
\Gamma(U_p,
\mathcal{I}^0|_{U_p} \otimes_\mathbf{Z} \underline{\mathbf{Z}}(\chi_p))
\to
\Gamma(U_p,
\mathcal{I}^1|_{U_p} \otimes_\mathbf{Z} \underline{\mathbf{Z}}(\chi_p))
\to
\Gamma(U_p,
\mathcal{I}^2|_{U_p} \otimes_\mathbf{Z} \underline{\mathbf{Z}}(\chi_p))
\to \ldots
$$
computes
$H^*(U_p,
\mathcal{F}|_{U_p} \otimes_\mathbf{Z} \underline{\mathbf{Z}}(\chi_p))$.
On the other hand, by
Lemma \ref{lemma-twist}
it is equal to the complex of $S_{p + 1}$-anti-invariants in
$$
\Gamma(W_p, \mathcal{I}^0) \to
\Gamma(W_p, \mathcal{I}^1) \to
\Gamma(W_p, \mathcal{I}^2) \to \ldots
$$
which by
Lemma \ref{lemma-compute}
is equal to the complex
$$
\Hom(K^p, \mathcal{I}^0) \to
\Hom(K^p, \mathcal{I}^1) \to
\Hom(K^p, \mathcal{I}^2) \to \ldots
$$
which computes
$\text{Ext}^*_{\textit{Ab}(X_{\acute{e}tale})}(K^p, \mathcal{F})$.
Putting everything together we win.
\end{proof}





\section{Higher vanishing for quasi-coherent sheaves}
\label{section-higher-vanishing}

\noindent
In this section we show that given a quasi-compact and
quasi-separated algebraic space $X$ there exists an integer
$n = n(X)$ such that the cohomology of any quasi-coherent
sheaf on $X$ vanishes beyond degree $n$.

\begin{lemma}
\label{lemma-quasi-coherent-twist}
With $S$, $W$, $G$, $U$, $\chi$ as in
Lemma \ref{lemma-twist}.
If $\mathcal{F}$ is a quasi-coherent $\mathcal{O}_U$-module,
then so is $\mathcal{F} \otimes_{\mathbf{Z}} \underline{\mathbf{Z}}(\chi)$.
\end{lemma}

\begin{proof}
The $\mathcal{O}_U$-module structure is clear. To check that
$\mathcal{F} \otimes_{\mathbf{Z}} \underline{\mathbf{Z}}(\chi)$
is quasi-coherent it suffices to check \'etale locally.
Hence the lemma follows as $\underline{\mathbf{Z}}(\chi)$
is finite locally free as a $\underline{\mathbf{Z}}$-module.
\end{proof}

\noindent
The following proposition is interesting even if $X$ is a scheme.
It is the natural generalization of
Cohomology of Schemes, Lemma \ref{coherent-lemma-vanishing-nr-affines}.
Before we state it, observe that given an \'etale morphism
$f : U \to X$ from an affine scheme towards a quasi-separated algebraic
space $X$ the fibres of $f$ are universally bounded, in particular
there exists an integer $d$ such that the fibres of $|U| \to |X|$
all have size at most $d$; this is the implication
$(\eta) \Rightarrow (\delta)$ of
Decent Spaces, Lemma \ref{decent-spaces-lemma-bounded-fibres}.

\begin{proposition}
\label{proposition-vanishing}
Let $S$ be a scheme. Let $X$ be an algebraic space over $S$.
Assume $X$ is quasi-compact and separated.
Let $U$ be an affine scheme, and let
$f : U \to X$ be a surjective \'etale morphism.
Let $d$ be an upper bound for the size of the fibres of
$|U| \to |X|$. Then for any quasi-coherent $\mathcal{O}_X$-module $\mathcal{F}$
we have $H^q(X, \mathcal{F}) = 0$ for $q \geq d$.
\end{proposition}

\begin{proof}
We will use the spectral sequence of
Lemma \ref{lemma-alternating-spectral-sequence}.
The lemma applies since $f$ is separated as $U$ is separated, see
Morphisms of Spaces, Lemma
\ref{spaces-morphisms-lemma-compose-after-separated}.
Since $X$ is separated the scheme $U \times_X \ldots \times_X U$ is a closed
subscheme of
$U \times_{\Spec(\mathbf{Z})} \ldots \times_{\Spec(\mathbf{Z})} U$
hence is affine. Thus $W_p$ is affine. Hence $U_p = W_p/S_{p + 1}$ is an
affine scheme by
Groupoids, Proposition \ref{groupoids-proposition-finite-flat-equivalence}.
The discussion in
Section \ref{section-higher-direct-image}
shows that cohomology of quasi-coherent sheaves on $W_p$ (as an algebraic
space) agrees with the cohomology of the corresponding quasi-coherent
sheaf on the underlying affine scheme, hence vanishes in positive degrees by
Cohomology of Schemes, Lemma
\ref{coherent-lemma-quasi-coherent-affine-cohomology-zero}.
By
Lemma \ref{lemma-quasi-coherent-twist}
the sheaves
$\mathcal{F}|_{U_p} \otimes_\mathbf{Z} \underline{\mathbf{Z}}(\chi_p)$
are quasi-coherent. Hence
$H^q(W_p,
\mathcal{F}|_{U_p} \otimes_\mathbf{Z} \underline{\mathbf{Z}}(\chi_p))$
is zero when $q > 0$. By our definition of the integer $d$ we see that
$W_p = \emptyset$ for $p \geq d$. Hence also
$H^0(W_p,
\mathcal{F}|_{U_p} \otimes_\mathbf{Z} \underline{\mathbf{Z}}(\chi_p))$
is zero when $p \geq d$.
This proves the proposition.
\end{proof}

\noindent
In the following lemma we esthablish that a quasi-compact and
quasi-separated algebraic space has finite cohomological dimension
for quasi-coherent modules. We are explicit about the bound only because
we will use it later to prove a similar result for higher direct
images.

\begin{lemma}
\label{lemma-vanishing-quasi-separated}
Let $S$ be a scheme. Let $X$ be an algebraic space over $S$.
Assume $X$ is quasi-compact and quasi-separated.
Then we can choose
\begin{enumerate}
\item an affine scheme $U$,
\item a surjective \'etale morphism $f : U \to X$,
\item an integer $d$ bounding the degrees of the fibres of $U \to X$,
\item for every $p = 0, 1, \ldots, d$ a surjective \'etale morphism
$V_p \to U_p$ from an affine scheme $V_p$ where $U_p$ is as in
Lemma \ref{lemma-alternating-spectral-sequence}, and
\item an integer $d_p$ bounding the degree of the fibres of $V_p \to U_p$.
\end{enumerate}
Moreover, whenever we have (1) -- (5), then for any quasi-coherent
$\mathcal{O}_X$-module $\mathcal{F}$ we have $H^q(X, \mathcal{F}) = 0$ for
$q \geq \max(d_p + p)$.
\end{lemma}

\begin{proof}
Since $X$ is quasi-compact we can find a surjective \'etale morphism
$U \to X$ with $U$ affine, see
Properties of Spaces, Lemma
\ref{spaces-properties-lemma-quasi-compact-affine-cover}.
By
Decent Spaces, Lemma \ref{decent-spaces-lemma-bounded-fibres}
the fibres of $f$ are universally bounded, hence we can find $d$.
We have $U_p = W_p/S_{p + 1}$ and $W_p \subset U \times_X \ldots \times_X U$
is open and closed. Since $X$ is quasi-separated the schemes
$W_p$ are quasi-compact, hence $U_p$ is quasi-compact.
Since $U$ is separated, the schemes $W_p$ are separated, hence
$U_p$ is separated by (the absolute version of)
Spaces, Lemma \ref{spaces-lemma-quotient-finite-separated}.
By
Properties of Spaces, Lemma
\ref{spaces-properties-lemma-quasi-compact-affine-cover}
we can find the morphisms $V_p \to W_p$.
By
Decent Spaces, Lemma \ref{decent-spaces-lemma-bounded-fibres}
we can find the integers $d_p$.

\medskip\noindent
At this point the proof uses the spectral sequence
$$
E_1^{p, q} =
H^q(U_p, \mathcal{F}|_{U_p} \otimes_\mathbf{Z} \underline{\mathbf{Z}}(\chi_p))
\Rightarrow
H^{p + q}(X, \mathcal{F})
$$
see
Lemma \ref{lemma-alternating-spectral-sequence}.
By definition of the integer $d$ we see that $U_p = 0$ for $p \geq d$.
By Proposition \ref{proposition-vanishing}
and
Lemma \ref{lemma-quasi-coherent-twist}
we see that
$H^q(U_p,
\mathcal{F}|_{U_p} \otimes_\mathbf{Z} \underline{\mathbf{Z}}(\chi_p))$
is zero for $q \geq d_p$ for $p = 0, \ldots, d$.
Whence the lemma.
\end{proof}









\section{Vanishing for higher direct images}
\label{section-vanishing-higher-direct-images}

\noindent
We apply the results of
Section \ref{section-higher-vanishing}
to obtain vanishing of higher direct images of quasi-coherent sheaves
for quasi-compact and quasi-separated morphisms. This is useful because
it allows one to argue by descending induction on the cohomological degree
in certain situations.

\begin{lemma}
\label{lemma-vanishing-higher-direct-images}
Let $S$ be a scheme. Let $f : X \to Y$ be a
morphism of algebraic spaces over $S$.
Assume that
\begin{enumerate}
\item $f$ is quasi-compact and quasi-separated, and
\item $Y$ is quasi-compact.
\end{enumerate}
Then there exists an integer $n(X \to Y)$ such that
for any algebraic space $Y'$, any morphism $Y' \to Y$
and any quasi-coherent sheaf $\mathcal{F}'$ on $X' = Y' \times_Y X$
the higher direct images $R^if'_*\mathcal{F}'$ are zero for
$i \geq n(X \to Y)$.
\end{lemma}

\begin{proof}
Let $V \to Y$ be a surjective \'etale morphism where $V$ is an affine
scheme, see
Properties of Spaces, Lemma
\ref{spaces-properties-lemma-quasi-compact-affine-cover}.
Suppose we prove the result for the base change $f_V : V \times_Y X \to V$.
Then the result holds for $f$ with $n(X \to Y) = n(X_V \to V)$.
Namely, if $Y' \to Y$ and $\mathcal{F}'$ are as in the lemma, then
$R^if'_*\mathcal{F}'|_{V \times_Y Y'}$ is equal to
$R^if'_{V, *}\mathcal{F}'|_{X'_V}$ where
$f'_V : X'_V = V \times_Y Y' \times_Y X \to V \times_Y Y' = Y'_V$, see
Properties of Spaces,
Lemma \ref{spaces-properties-lemma-pushforward-etale-base-change-modules}.
Thus we may assume that $Y$ is an affine scheme.

\medskip\noindent
Moreover, to prove the vanishing for all $Y' \to Y$ and
$\mathcal{F}'$ it suffices to do so when $Y'$ is an affine scheme.
In this case, $R^if'_*\mathcal{F}'$ is quasi-coherent by
Lemma \ref{lemma-higher-direct-image}.
Hence it suffices to prove that $H^i(X', \mathcal{F}') = 0$, because
$H^i(X', \mathcal{F}') = H^0(Y', R^if'_*\mathcal{F}')$ by
Cohomology on Sites, Lemma \ref{sites-cohomology-lemma-apply-Leray}
and the vanishing of higher cohomology of quasi-coherent sheaves
on affine algebraic spaces
(Proposition \ref{proposition-vanishing}).

\medskip\noindent
Choose $U \to X$, $d$, $V_p \to U_p$ and $d_p$ as in
Lemma \ref{lemma-vanishing-quasi-separated}.
For any affine scheme $Y'$ and morphism $Y' \to Y$ denote
$X' = Y' \times_Y X$, $U' = Y' \times_Y U$, $V'_p = Y' \times_Y V_p$.
Then $U' \to X'$, $d' = d$, $V'_p \to U'_p$ and $d'_p = d$
is a collection of choices as in
Lemma \ref{lemma-vanishing-quasi-separated}
for the algebraic space $X'$ (details omitted).
Hence we see that $H^i(X', \mathcal{F}') = 0$ for $i \geq \max(p + d_p)$
and we win.
\end{proof}

\begin{lemma}
\label{lemma-affine-vanishing-higher-direct-images}
Let $S$ be a scheme. Let $f : X \to Y$ be an affine
morphism of algebraic spaces over $S$. Then
$R^if_*\mathcal{F} = 0$ for $i > 0$ and any quasi-coherent
$\mathcal{O}_X$-module $\mathcal{F}$.
\end{lemma}

\begin{proof}
Recall that an affine morphism of algebraic spaces is representable.
Hence this follows from (\ref{equation-representable-higher-direct-image}) and
Cohomology of Schemes, Lemma \ref{coherent-lemma-relative-affine-vanishing}.
\end{proof}








\section{Cohomology and base change, I}
\label{section-cohomology-and-base-change}

\noindent
Let $S$ be a scheme.
Let $f : X \to Y$ be a morphism of algebraic spaces over $S$.
Let $\mathcal{F}$ be a quasi-coherent sheaf on $X$.
Suppose further that $g : Y' \to Y$ is a morphism of algebraic spaces over
$S$. Denote $X' = X_{Y'} = Y' \times_Y X$ the base change of $X$ and denote
$f' : X' \to Y'$ the base change of $f$.
Also write $g' : X' \to X$ the projection,
and set $\mathcal{F}' = (g')^*\mathcal{F}$.
Here is a diagram representing the situation:
\begin{equation}
\label{equation-base-change-diagram}
\vcenter{
\xymatrix{
\mathcal{F}' = (g')^*\mathcal{F} &
X' \ar[r]_{g'} \ar[d]_{f'} &
X \ar[d]^f &
\mathcal{F} \\
Rf'_*\mathcal{F}' &
Y' \ar[r]^g &
Y &
Rf_*\mathcal{F}
}
}
\end{equation}
Here is the basic result for a flat base change.

\begin{lemma}
\label{lemma-flat-base-change-cohomology}
In the situation above, assume that $g$ is flat and that $f$
is quasi-compact and quasi-separated.
Then the base change map for any $i \geq 0$ we have
$$
R^pf'_*\mathcal{F}' = g^*R^pf_*\mathcal{F}
$$
with notation as in (\ref{equation-base-change-diagram}).
\end{lemma}

\begin{proof}
The morphism $g'$ is flat by
Morphisms of Spaces, Lemma \ref{spaces-morphisms-lemma-base-change-flat}.
Note that flatness of $g$ and $g'$ is equivalent to flatness
of the morphisms of small \'etale ringed sites, see
Morphisms of Spaces, Lemma \ref{spaces-morphisms-lemma-flat-morphism-sites}.
Hence we can apply
Cohomology on Sites, Lemma
\ref{sites-cohomology-lemma-base-change-map-flat-case}
to obtain a base change map
$$
g^*R^pf_*\mathcal{F} \longrightarrow R^pf'_*\mathcal{F}'
$$
To prove this map is an isomorphism we can work locally in the \'etale
topology on $Y'$. Thus we may assume that $Y$ and $Y'$ are affine
schemes. Say $Y = \text{Spec}(A)$ and $Y' = \text{Spec}(B)$.
In this case we are really trying to show that the map
$$
H^p(X, \mathcal{F}) \otimes_A B \longrightarrow H^p(X_B, \mathcal{F}_B)
$$
is an isomorphism where $X_B = \text{Spec}(B) \times_{\text{Spec}(A)} X$ and
$\mathcal{F}_B$ is the pullback of $\mathcal{F}$ to $X_B$.

\medskip\noindent
Fix $A \to B$ a flat ring map and let $X$ be a quasi-compact and
quasi-separated algebraic space over $A$. Note that $g' : X_B \to X$
is affine as a base change of $\Spec(B) \to \Spec(A)$. Hence
the higher direct images $R^i(g')_*\mathcal{F}_B$ are zero by
Lemma \ref{lemma-affine-vanishing-higher-direct-images}.
Thus $H^p(X_B, \mathcal{F}_B) = H^p(X, g'_*\mathcal{F}_B)$, see
Cohomology on Sites, Lemma \ref{sites-cohomology-lemma-apply-Leray}.
Moreover, we have
$$
g'_*\mathcal{F}_B = \mathcal{F} \otimes_{\underline{A}} \underline{B}
$$
where $\underline{A}$, $\underline{B}$ denotes the constant sheaf of
rings with value $A$, $B$. Namely, it is clear that there is a map
from right to left. For any affine scheme $U$ \'etale over $X$ we have
\begin{align*}
g'_*\mathcal{F}_B(U) & = \mathcal{F}_B(\Spec(B) \times_{\Spec(A)} U) \\
& =
\Gamma(\Spec(B) \times_{\Spec(A)} U,
(\Spec(B) \times_{\Spec(A)} U \to U)^*\mathcal{F}|_U) \\
& =
B \otimes_A \mathcal{F}(U)
\end{align*}
hence the map is an isomorphism. Write $B = \colim M_i$ as a filtered
colimit of finite free $A$-modules $M_i$ using Lazard's theorem, see
Algebra, Theorem \ref{algebra-theorem-lazard}.
We deduce that
\begin{align*}
H^p(X, g'_*\mathcal{F}_B) &
= H^p(X, \mathcal{F} \otimes_{\underline{A}} \underline{B}) \\
& = H^p(X, \colim_i \mathcal{F} \otimes_{\underline{A}} \underline{M_i}) \\
& = \colim_i H^p(X, \mathcal{F} \otimes_{\underline{A}} \underline{M_i}) \\
& = \colim_i H^p(X, \mathcal{F}) \otimes_A M_i \\
& = H^p(X, \mathcal{F}) \otimes_A \colim_i M_i \\
& = H^p(X, \mathcal{F}) \otimes_A B
\end{align*}
The first equality because
$g'_*\mathcal{F}_B = \mathcal{F} \otimes_{\underline{A}} \underline{B}$
as seen above.
The second because $\otimes$ commutes with colimits.
The third equality because cohomology on $X$ commutes with
colimits (see
Lemma \ref{lemma-colimits}).
The fourth equality because $M_i$ is finite free (i.e., because cohomology
commutes with finite direct sums).
The fith because $\otimes$ commutes with colimits.
The sixth by choice of our system.
\end{proof}

\begin{lemma}
\label{lemma-affine-base-change}
Let $S$ be a scheme. Let $f : X \to Y$ be an affine morphism of algebraic
spaces over $S$. Let $\mathcal{F}$ be a quasi-coherent $\mathcal{O}_X$-module.
In this case $f_*\mathcal{F} \cong Rf_*\mathcal{F}$ is a quasi-coherent
sheaf, and for every diagram (\ref{equation-base-change-diagram})
we have $g^*f_*\mathcal{F} = f'_*(g')^*\mathcal{F}$.
\end{lemma}

\begin{proof}
By the discussion surrounding
(\ref{equation-representable-higher-direct-image})
this reduces to the case of an affine morphism of schemes which
is treated in Cohomology of Schemes, Lemma
\ref{coherent-lemma-affine-base-change}.
\end{proof}


\section{Coherent modules on locally Noetherian algebraic spaces}
\label{section-coherent}

\noindent
This section is the analogue of
Cohomology of Schemes, Section \ref{coherent-section-coherent-sheaves}.
In Modules on Sites, Definition \ref{sites-modules-definition-site-local}
we have defined coherent modules on any ringed topos. We use this notion
to define coherent modules on locally Noetherian algebraic spaces.
Although it is possible to work with coherent modules more generally
we resist the urge to do so.

\begin{definition}
\label{definition-coherent}
Let $S$ be a scheme. Let $X$ be a locally Noetherian algebraic space over $S$.
A quasi-coherent module $\mathcal{F}$ on $X$ is called {\it coherent}
if $\mathcal{F}$ is a coherent $\mathcal{O}_X$-module on the site
$X_{\acute{e}tale}$ in the sense of
Modules on Sites, Definition \ref{sites-modules-definition-site-local}.
\end{definition}

\noindent
Of course this definition is a bit hard to work with. We usually use
the characterization given in the lemma below.

\begin{lemma}
\label{lemma-coherent-Noetherian}
Let $S$ be a scheme.
Let $X$ be a locally Noetherian algebraic space over $S$.
Let $\mathcal{F}$ be an $\mathcal{O}_X$-module.
The following are equivalent
\begin{enumerate}
\item $\mathcal{F}$ is coherent,
\item $\mathcal{F}$ is a quasi-coherent, finite type $\mathcal{O}_X$-module,
\item $\mathcal{F}$ is a finitely presented $\mathcal{O}_X$-module,
\item for any \'etale morphism $\varphi : U \to X$ where $U$ is a scheme
the pullback $\varphi^*\mathcal{F}$ is a coherent module on $U$, and
\item there exists a surjective \'etale morphism $\varphi : U \to X$
where $U$ is a scheme such that the pullback $\varphi^*\mathcal{F}$ is
a coherent module on $U$.
\end{enumerate}
In particular $\mathcal{O}_X$ is coherent, any invertible
$\mathcal{O}_X$-module is coherent, and more generally any
finite locally free $\mathcal{O}_X$-module is coherent.
\end{lemma}

\begin{proof}
To be sure, if $X$ is a locally Noetherian algebraic space and
$U \to X$ is an \'etale morphism, then $U$ is locally Noetherian, see
Properties of Spaces, Section \ref{spaces-properties-section-types-properties}.
The lemma then follows from the points (1) -- (5) made in
Properties of Spaces, Section \ref{spaces-properties-section-properties-modules}
and the corresponding result for coherent modules on locally
Noetherian schemes, see
Cohomology of Schemes, Lemma \ref{coherent-lemma-coherent-Noetherian}.
\end{proof}

\begin{lemma}
\label{lemma-coherent-abelian-Noetherian}
Let $S$ be a scheme. Let $X$ be a locally Noetherian algebraic space over $S$.
The category of coherent $\mathcal{O}_X$-modules is abelian. More precisely,
the kernel and cokernel of a map of coherent $\mathcal{O}_X$-modules are
coherent. Any extension of coherent sheaves is coherent.
\end{lemma}

\begin{proof}
Choose a scheme $U$ and a surjective \'etale morphism $f : U \to X$.
Pullback $f^*$ is an exact functor as it equals a restriction functor, see
Properties of Spaces, Equation
(\ref{spaces-properties-equation-restrict-modules}).
By
Lemma \ref{lemma-coherent-Noetherian} we can check whether an
$\mathcal{O}_X$-module $\mathcal{F}$ is
coherent by checking whether $f^*\mathcal{F}$ is coherent. Hence the
lemma follows from the case of schemes which is
Cohomology of Schemes, Lemma \ref{coherent-lemma-coherent-abelian-Noetherian}.
\end{proof}

\noindent
Coherent modules form a Serre subcategory of the
category of quasi-coherent $\mathcal{O}_X$-modules. This does not hold
for modules on a general ringed topos.

\begin{lemma}
\label{lemma-coherent-Noetherian-quasi-coherent-sub-quotient}
Let $S$ be a scheme.
Let $X$ be a locally Noetherian algebraic space over $S$.
Let $\mathcal{F}$ be a coherent $\mathcal{O}_X$-module.
Any quasi-coherent submodule of $\mathcal{F}$ is coherent.
Any quasi-coherent quotient module of $\mathcal{F}$ is coherent.
\end{lemma}

\begin{proof}
Choose a scheme $U$ and a surjective \'etale morphism $f : U \to X$.
Pullback $f^*$ is an exact functor as it equals a restriction functor, see
Properties of Spaces, Equation
(\ref{spaces-properties-equation-restrict-modules}).
By
Lemma \ref{lemma-coherent-Noetherian} we can check whether an
$\mathcal{O}_X$-module $\mathcal{G}$ is
coherent by checking whether $f^*\mathcal{H}$ is coherent. Hence the
lemma follows from the case of schemes which is
Cohomology of Schemes, Lemma
\ref{coherent-lemma-coherent-Noetherian-quasi-coherent-sub-quotient}.
\end{proof}

\begin{lemma}
\label{lemma-tensor-hom-coherent}
Let $S$ be a scheme.
Let $X$ be a locally Noetherian algebraic space over $S$,.
Let $\mathcal{F}$, $\mathcal{G}$ be coherent $\mathcal{O}_X$-modules.
The $\mathcal{O}_X$-modules $\mathcal{F} \otimes_{\mathcal{O}_X} \mathcal{G}$
and $\SheafHom_{\mathcal{O}_X}(\mathcal{F}, \mathcal{G})$ are
coherent.
\end{lemma}

\begin{proof}
Via Lemma \ref{lemma-coherent-Noetherian} this follows from the result
for schemes, see
Cohomology of Schemes, Lemma \ref{coherent-lemma-tensor-hom-coherent}.
\end{proof}

\begin{lemma}
\label{lemma-local-isomorphism}
Let $S$ be a scheme. Let $X$ be a locally Noetherian algebraic space over $S$.
Let $\mathcal{F}$, $\mathcal{G}$ be coherent $\mathcal{O}_X$-modules.
Let $\varphi : \mathcal{G} \to \mathcal{F}$ be a homomorphism
of $\mathcal{O}_X$-modules. Let $\overline{x}$ be a geometric point of $X$
lying over $x \in |X|$.
\begin{enumerate}
\item If $\mathcal{F}_{\overline{x}} = 0$ then there exists an open
neighbourhood $X' \subset X$ of $x$ such that $\mathcal{F}|_{X'} = 0$.
\item If $\varphi_{\overline{x}} : \mathcal{G}_{\overline{x}} \to
\mathcal{F}_{\overline{x}}$ is injective, then there exists an open
neighbourhood $X' \subset X$ of $x$ such that $\varphi|_{X'}$ is injective.
\item If $\varphi_{\overline{x}} : \mathcal{G}_{\overline{x}} \to
\mathcal{F}_{\overline{x}}$ is surjective, then there exists an open
neighbourhood $X' \subset X$ of $x$ such that $\varphi|_{X'}$ is surjective.
\item If $\varphi_{\overline{x}} : \mathcal{G}_{\overline{x}} \to
\mathcal{F}_{\overline{x}}$ is bijective, then there exists an open
neighbourhood $X' \subset X$ of $x$ such that $\varphi|_{X'}$ is an isomorphism.
\end{enumerate}
\end{lemma}

\begin{proof}
Let $\varphi : U \to X$ be an \'etale morphism where $U$ is a scheme and
let $u \in U$ be a point mapping to $x$. By
Properties of Spaces, Lemmas
\ref{spaces-properties-lemma-stalk-quasi-coherent} and
\ref{spaces-properties-lemma-describe-etale-local-ring}
as well as
More on Algebra, Lemma \ref{more-algebra-lemma-dumb-properties-henselization}
we see that $\varphi_{\overline{x}}$ is injective, surjective, or bijective
if and only if $\varphi_u : \varphi^*\mathcal{F}_u \to \varphi^*\mathcal{G}_u$
has the corresponding property. Thus we can apply the schemes version of
this lemma to see that (after possibly shrinking $U$) the map
$\varphi^*\mathcal{F} \to \varphi^*\mathcal{G}$ is injective, surjective,
or an isomorphism. Let $X' \subset X$ be the open subspace corresponding
to $|\varphi|(|U|) \subset |X|$, see
Properties of Spaces, Lemma \ref{spaces-properties-lemma-open-subspaces}.
Since $\{U \to X'\}$ is a covering for the \'etale topology, we conclude
that $\varphi|_{X'}$ is injective, surjective, or an isomorphism as desired.
Finally, observe that (1) follows from (2) by looking at the map
$\mathcal{F} \to 0$.
\end{proof}

\begin{lemma}
\label{lemma-coherent-support-closed}
Let $S$ be a scheme. Let $X$ be a locally Noetherian algebraic space over $S$.
Let $\mathcal{F}$ be a coherent $\mathcal{O}_X$-module. Let $i : Z \to X$
be the scheme theoretic support of $\mathcal{F}$ and $\mathcal{G}$
the quasi-coherent $\mathcal{O}_Z$-module such that
$i_*\mathcal{G} = \mathcal{F}$, see
Morphisms of Spaces, Definition
\ref{spaces-morphisms-definition-scheme-theoretic-support}.
Then $\mathcal{G}$ is a coherent $\mathcal{O}_Z$-module.
\end{lemma}

\begin{proof}
The statement of the lemma makes sense as a coherent module is in
particular of finite type. Moreover, as $Z \to X$ is a closed immersion
it is locally of finite type and hence $Z$ is locally Noetherian, see
Morphisms of Spaces, Lemmas
\ref{spaces-morphisms-lemma-immersion-locally-finite-type} and
\ref{spaces-morphisms-lemma-locally-finite-type-locally-noetherian}.
Finally, as $\mathcal{G}$ is of finite type it is a coherent
$\mathcal{O}_Z$-module by
Lemma \ref{lemma-coherent-Noetherian}
\end{proof}

\begin{lemma}
\label{lemma-finite-pushforward-coherent}
Let $S$ be a scheme. Let $f : X \to Y$ be a finite morphism of algebraic
spaces over $S$ with $Y$ locally Noetherian. Let $\mathcal{F}$ be a
coherent $\mathcal{O}_X$-module. Assume $f$ is finite and $Y$ locally
Noetherian. Then $R^pf_*\mathcal{F} = 0$ for $p > 0$ and
$f_*\mathcal{F}$ is coherent.
\end{lemma}

\begin{proof}
Choose a scheme $V$ and a surjective \'etale morphism $V \to Y$.
Then $V \times_Y X \to V$ is a finite morphism of locally Noetherian
schemes. By (\ref{equation-representable-higher-direct-image}) we reduce
to the case of schemes which is
Cohomology of Schemes, Lemma \ref{coherent-lemma-finite-pushforward-coherent}.
\end{proof}






\section{Coherent sheaves on Noetherian spaces}
\label{section-coherent-quasi-compact}

\noindent
In this section we mention some properties of coherent sheaves on
Noetherian algebraic spaces.

\begin{lemma}
\label{lemma-acc-coherent}
Let $S$ be a scheme. Let $X$ be a Noetherian algebraic space over $S$.
Let $\mathcal{F}$ be a coherent $\mathcal{O}_X$-module.
The ascending chain condition holds for quasi-coherent submodules
of $\mathcal{F}$. In other words, given any sequence
$$
\mathcal{F}_1 \subset \mathcal{F}_2 \subset \ldots \subset \mathcal{F}
$$
of quasi-coherent submodules, then
$\mathcal{F}_n = \mathcal{F}_{n + 1} = \ldots $ for some $n \geq 0$.
\end{lemma}

\begin{proof}
Choose an affine scheme $U$ and a surjective \'etale morphism $U \to X$
(see Properties of Spaces, Lemma
\ref{spaces-properties-lemma-quasi-compact-affine-cover}).
Then $U$ is a Noetherian scheme (by
Morphisms of Spaces, Lemma
\ref{spaces-morphisms-lemma-locally-finite-type-locally-noetherian}).
If $\mathcal{F}_n|_U = \mathcal{F}_{n + 1}|_U = \ldots$
then $\mathcal{F}_n = \mathcal{F}_{n + 1} = \ldots$.
Hence the result follows from the case of schemes, see
Cohomology of Schemes, Lemma \ref{coherent-lemma-acc-coherent}.
\end{proof}

\begin{lemma}
\label{lemma-power-ideal-kills-sheaf}
Let $S$ be a scheme. Let $X$ be a Noetherian algebraic space over $S$.
Let $\mathcal{F}$ be a coherent sheaf on $X$. Let
$\mathcal{I} \subset \mathcal{O}_X$ be a quasi-coherent sheaf of ideals
corresponding to a closed subspace $Z \subset X$. Then there is some
$n \geq 0$ such that $\mathcal{I}^n\mathcal{F} = 0$ if and only if
$\text{Supp}(\mathcal{F}) \subset Z$ (set theoretically).
\end{lemma}

\begin{proof}
Choose an affine scheme $U$ and a surjective \'etale morphism $U \to X$
(see Properties of Spaces, Lemma
\ref{spaces-properties-lemma-quasi-compact-affine-cover}).
Then $U$ is a Noetherian scheme (by
Morphisms of Spaces, Lemma
\ref{spaces-morphisms-lemma-locally-finite-type-locally-noetherian}).
Note that $\mathcal{I}^n\mathcal{F}|_U = 0$ if and only if
$\mathcal{I}^n\mathcal{F} = 0$ and similarly for the condition on
the support. Hence the result follows from the case of schemes, see
Cohomology of Schemes, Lemma \ref{coherent-lemma-power-ideal-kills-sheaf}.
\end{proof}

\begin{lemma}[Artin-Rees]
\label{lemma-Artin-Rees}
Let $S$ be a scheme. Let $X$ be a Noetherian algebraic space over $S$.
Let $\mathcal{F}$ be a coherent sheaf on $X$. Let
$\mathcal{G} \subset \mathcal{F}$ be a quasi-coherent subsheaf.
Let $\mathcal{I} \subset \mathcal{O}_X$ be a quasi-coherent sheaf of
ideals. Then there exists a $c \geq 0$ such that for all $n \geq c$ we
have
$$
\mathcal{I}^{n - c}(\mathcal{I}^c\mathcal{F} \cap \mathcal{G})
=
\mathcal{I}^n\mathcal{F}
$$
\end{lemma}

\begin{proof}
Choose an affine scheme $U$ and a surjective \'etale morphism $U \to X$
(see Properties of Spaces, Lemma
\ref{spaces-properties-lemma-quasi-compact-affine-cover}).
Then $U$ is a Noetherian scheme (by
Morphisms of Spaces, Lemma
\ref{spaces-morphisms-lemma-locally-finite-type-locally-noetherian}).
The equality of the lemma holds if and only if it holds after
restricting to $U$. Hence the result follows from the case of schemes, see
Cohomology of Schemes, Lemma \ref{coherent-lemma-Artin-Rees}.
\end{proof}

\begin{lemma}
\label{lemma-homs-over-open}
Let $S$ be a scheme. Let $X$ be a Noetherian algebraic space over $S$.
Let $\mathcal{F}$, $\mathcal{G}$ be coherent $\mathcal{O}_X$-modules.
Let $\mathcal{I} \subset \mathcal{O}_X$ be a quasi-coherent sheaf of
ideals. Denote $Z \subset X$ the corresponding closed subspace and
set $U = X \setminus Z$. There is a canonical isomorphism
$$
\colim_n \Hom_{\mathcal{O}_X}(\mathcal{I}^n\mathcal{G}, \mathcal{F})
\longrightarrow
\Hom_{\mathcal{O}_U}(\mathcal{G}|_U, \mathcal{F}|_U).
$$
In particular we have an isomorphism
$$
\colim_n \Hom_{\mathcal{O}_X}(\mathcal{I}^n, \mathcal{F})
\longrightarrow
\Gamma(U, \mathcal{F}).
$$
\end{lemma}

\begin{proof}
Let $W$ be an affine scheme and let $W \to X$ be a surjective \'etale
morphism (see Properties of Spaces, Lemma
\ref{spaces-properties-lemma-quasi-compact-affine-cover}).
Set $R = W \times_X W$. Then $W$ and $R$ are Noetherian schemes, see
Morphisms of Spaces, Lemma
\ref{spaces-morphisms-lemma-locally-finite-type-locally-noetherian}.
Hence the result hold for the restrictions of $\mathcal{F}$, $\mathcal{G}$,
and $\mathcal{I}$, $U$, $Z$ to $W$ and $R$ by
Cohomology of Schemes, Lemma \ref{coherent-lemma-homs-over-open}.
It follows formally that the result holds over $X$.
\end{proof}






\section{Devissage of coherent sheaves}
\label{section-devissage}

\noindent
This section is the analogue of
Cohomology of Schemes, Section \ref{coherent-section-devissage}.

\begin{lemma}
\label{lemma-prepare-filter-support}
Let $S$ be a scheme. Let $X$ be a Noetherian algebraic space over $S$.
Let $\mathcal{F}$ be a coherent sheaf on $X$. Suppose that
$\text{Supp}(\mathcal{F}) = Z \cup Z'$ with $Z$, $Z'$ closed.
Then there exists a short exact sequence of coherent sheaves
$$
0 \to \mathcal{G}' \to \mathcal{F} \to \mathcal{G} \to 0
$$
with $\text{Supp}(\mathcal{G}') \subset Z'$ and
$\text{Supp}(\mathcal{G}) \subset Z$.
\end{lemma}

\begin{proof}
Let $\mathcal{I} \subset \mathcal{O}_X$ be the sheaf of ideals
defining the reduced induced closed subspace structure on $Z$, see
Properties of Spaces, Lemma
\ref{spaces-properties-lemma-reduced-closed-subspace}.
Consider the subsheaves
$\mathcal{G}'_n = \mathcal{I}^n\mathcal{F}$ and the
quotients $\mathcal{G}_n = \mathcal{F}/\mathcal{I}^n\mathcal{F}$.
For each $n$ we have a short exact sequence
$$
0 \to \mathcal{G}'_n \to \mathcal{F} \to \mathcal{G}_n \to 0
$$
For every geometric point $\overline{x}$ of $Z' \setminus Z$ we have
$\mathcal{I}_{\overline{x}} = \mathcal{O}_{X, \overline{x}}$
and hence $\mathcal{G}_{n, \overline{x}} = 0$. Thus we see that
$\text{Supp}(\mathcal{G}_n) \subset Z$. Note that $X \setminus Z'$
is a Noetherian algebraic space. Hence by
Lemma \ref{lemma-power-ideal-kills-sheaf}
there exists an $n$ such that $\mathcal{G}'_n|_{X \setminus Z'} =
\mathcal{I}^n\mathcal{F}|_{X \setminus Z'} = 0$.
For such an $n$ we see that $\text{Supp}(\mathcal{G}'_n) \subset Z'$.
Thus setting $\mathcal{G}' = \mathcal{G}'_n$ and $\mathcal{G} = \mathcal{G}_n$
works.
\end{proof}

\noindent
In the following we will freely use the scheme theoretic support of
finite type modules as defined in Morphisms of Spaces, Definition
\ref{spaces-morphisms-definition-scheme-theoretic-support}.

\begin{lemma}
\label{lemma-prepare-filter-irreducible}
Let $S$ be a scheme. Let $X$ be a Noetherian algebraic space over $S$.
Let $\mathcal{F}$ be a coherent sheaf on $X$. Assume that the scheme
theoretic support of $\mathcal{F}$ is a reduced $Z \subset X$ with
$|Z|$ irreducible. Then there exist an integer $r > 0$, a nonzero
sheaf of ideals $\mathcal{I} \subset \mathcal{O}_Z$, and an injective
map of coherent sheaves
$$
i_*\left(\mathcal{I}^{\oplus r}\right) \to \mathcal{F}
$$
whose cokernel is supported on a proper closed subspace of $Z$.
\end{lemma}

\begin{proof}
By assumption there exists a coherent $\mathcal{O}_Z$-module
$\mathcal{G}$ with support $Z$ and $\mathcal{F} \cong i_*\mathcal{G}$, see
Lemma \ref{lemma-coherent-support-closed}. Hence it suffices to prove the
lemma for the case $Z = X$ and $i = \text{id}$.

\medskip\noindent
By Properties of Spaces, Proposition
\ref{spaces-properties-proposition-locally-quasi-separated-open-dense-scheme}
there exists a dense open subspace $U \subset X$ which is a scheme. Note that
$U$ is a Noetherian integral scheme. After shrinking $U$ we may assume
that $\mathcal{F}|_U \cong \mathcal{O}_U^{\oplus r}$ (for example by
Cohomology of Schemes, Lemma \ref{coherent-lemma-prepare-filter-irreducible}
or by a direct algebra argument). Let $\mathcal{I} \subset \mathcal{O}_X$
be a quasi-coherent sheaf of ideals whose associated closed subspace
is the complement of $U$ in $X$ (see for example
Properties of Spaces, Section \ref{spaces-properties-section-reduced}). 
By Lemma \ref{lemma-homs-over-open} there exists an $n \geq 0$ and a
morphism $\mathcal{I}^n(\mathcal{O}_X^{\oplus r}) \to \mathcal{F}$
which recovers our isomorphism over $U$. Since
$\mathcal{I}^n(\mathcal{O}_X^{\oplus r}) = (\mathcal{I}^n)^{\oplus r}$
we get a map as in the lemma. It is injective: namely, if $\sigma$ is
a nonzero section of $\mathcal{I}^{\oplus r}$ over a scheme $W$ \'etale
over $X$, then because $X$ hence $W$ is reduced the support of $\sigma$
contains a nonempty open of $W$. But the kernel of
$(\mathcal{I}^n)^{\oplus r} \to \mathcal{F}$ is zero
over a dense open, hence $\sigma$ cannot be a section of the kernel.
\end{proof}

\begin{lemma}
\label{lemma-coherent-filter}
Let $S$ be a scheme. Let $X$ be a Noetherian algebraic space over $S$.
Let $\mathcal{F}$ be a coherent sheaf on $X$. There exists a filtration
$$
0 = \mathcal{F}_0 \subset \mathcal{F}_1 \subset
\ldots \subset \mathcal{F}_m = \mathcal{F}
$$
by coherent subsheaves such that for each $j = 1, \ldots, m$
there exists a reduced closed subspace $Z_j \subset X$ with $|Z_j|$
irreducible and a sheaf of ideals $\mathcal{I}_j \subset \mathcal{O}_{Z_j}$
such that
$$
\mathcal{F}_j/\mathcal{F}_{j - 1}
\cong (Z_j \to X)_* \mathcal{I}_j
$$
\end{lemma}

\begin{proof}
Consider the collection
$$
\mathcal{T} =
\left\{
\begin{matrix}
T \subset |X|
\text{ closed such that there exists a coherent sheaf }
\mathcal{F} \\
\text{ with }
\text{Supp}(\mathcal{F}) = T
\text{ for which the lemma is wrong}
\end{matrix}
\right\}
$$
We are trying to show that $\mathcal{T}$ is empty. If not, then
because $|X|$ is Noetherian (Properties of Spaces, Lemma
\ref{spaces-properties-lemma-Noetherian-topology})
we can choose a minimal element $T \in \mathcal{T}$. This means that
there exists a coherent sheaf $\mathcal{F}$ on $X$ whose support is $T$
and for which the lemma does not hold. Clearly $T \not = \emptyset$ since
the only sheaf whose support is empty is the zero sheaf for which the
lemma does hold (with $m = 0$).

\medskip\noindent
If $T$ is not irreducible, then we can write $T = Z_1 \cup Z_2$
with $Z_1, Z_2$ closed and strictly smaller than $T$.
Then we can apply Lemma \ref{lemma-prepare-filter-support}
to get a short exact sequence of coherent sheaves
$$
0 \to
\mathcal{G}_1 \to
\mathcal{F} \to
\mathcal{G}_2 \to 0
$$
with $\text{Supp}(\mathcal{G}_i) \subset Z_i$. By minimality of
$T$ each of $\mathcal{G}_i$ has a filtration as in the statement
of the lemma. By considering the induced filtration on $\mathcal{F}$
we arrive at a contradiction. Hence we conclude
that $T$ is irreducible.

\medskip\noindent
Suppose $T$ is irreducible. Let $\mathcal{J}$ be the sheaf of ideals
defining the reduced induced closed subspace structure on $T$,
see Properties of Spaces, Lemma
\ref{spaces-properties-lemma-reduced-closed-subspace}.
By Lemma \ref{lemma-power-ideal-kills-sheaf} we see there exists
an $n \geq 0$ such that $\mathcal{J}^n\mathcal{F} = 0$. Hence we obtain
a filtration
$$
0 = \mathcal{I}^n\mathcal{F} \subset \mathcal{I}^{n - 1}\mathcal{F}
\subset \ldots \subset \mathcal{I}\mathcal{F} \subset \mathcal{F}
$$
each of whose succesive subquotients is annihilated by $\mathcal{J}$.
Hence if each of these subquotients has a filtration as in the statement
of the lemma then also $\mathcal{F}$ does. In other words we may
assume that $\mathcal{J}$ does annihilate $\mathcal{F}$.

\medskip\noindent
Assume $T$ is irreducible and $\mathcal{J}\mathcal{F} = 0$ where
$\mathcal{J}$ is as above. Then the scheme theoretic support of
$\mathcal{F}$ is $T$, see
Morphisms of Spaces, Lemma \ref{spaces-morphisms-lemma-i-star-equivalence}.
Hence we can apply Lemma \ref{lemma-prepare-filter-irreducible}.
This gives a short exact sequence
$$
0 \to
i_*(\mathcal{I}^{\oplus r}) \to
\mathcal{F} \to
\mathcal{Q} \to 0
$$
where the support of $\mathcal{Q}$ is a proper closed subset of $T$.
Hence we see that $\mathcal{Q}$ has a filtration of the desired type
by minimality of $T$. But then clearly $\mathcal{F}$ does too, which is
our final contradiction.
\end{proof}

\begin{lemma}
\label{lemma-property-initial}
Let $S$ be a scheme. Let $X$ be a Noetherian algebraic space over $S$.
Let $\mathcal{P}$ be a property of coherent sheaves on $X$. Assume
\begin{enumerate}
\item For any short exact sequence of coherent sheaves
$$
0 \to \mathcal{F}_1 \to \mathcal{F} \to \mathcal{F}_2 \to 0
$$
if $\mathcal{F}_i$, $i = 1, 2$ have property $\mathcal{P}$
then so does $\mathcal{F}$.
\item For every reduced closed subspace $Z \subset X$ with $|Z|$ irreducible
and every quasi-coherent sheaf of ideals $\mathcal{I} \subset \mathcal{O}_Z$
we have $\mathcal{P}$ for $i_*\mathcal{I}$.
\end{enumerate}
Then property $\mathcal{P}$ holds for every coherent sheaf on $X$.
\end{lemma}

\begin{proof}
First note that if $\mathcal{F}$ is a coherent sheaf with a filtration
$$
0 = \mathcal{F}_0 \subset \mathcal{F}_1 \subset
\ldots \subset \mathcal{F}_m = \mathcal{F}
$$
by coherent subsheaves such that each of $\mathcal{F}_i/\mathcal{F}_{i - 1}$
has property $\mathcal{P}$, then so does $\mathcal{F}$.
This follows from the property (1) for $\mathcal{P}$.
On the other hand, by Lemma \ref{lemma-coherent-filter}
we can filter any $\mathcal{F}$
with succesive subquotients as in (2).
Hence the lemma follows.
\end{proof}

\noindent
Here is a more useful variant of the lemma above.

\begin{lemma}
\label{lemma-property-higher-rank-cohomological}
Let $S$ be a scheme. Let $X$ be a Noetherian algebraic space over $S$.
Let $\mathcal{P}$ be a property of coherent sheaves on $X$. Assume
\begin{enumerate}
\item For any short exact sequence of coherent sheaves
$$
0 \to \mathcal{F}_1 \to \mathcal{F} \to \mathcal{F}_2 \to 0
$$
if $\mathcal{F}_i$, $i = 1, 2$ have property $\mathcal{P}$
then so does $\mathcal{F}$.
\item If $\mathcal{P}$ holds for a direct sum of coherent sheaves
then it holds for both.
\item For every reduced closed subspace $i : Z \to X$ with
$|Z|$ irreducible there exists a coherent sheaf $\mathcal{G}$ on $Z$
such that
\begin{enumerate}
\item $\text{Supp}(\mathcal{G}) = Z$,
\item for every nonzero quasi-coherent sheaf of ideals
$\mathcal{I} \subset \mathcal{O}_Z$ there exists a quasi-coherent
subsheaf $\mathcal{G}' \subset \mathcal{I}\mathcal{G}$ such that
$\text{Supp}(\mathcal{G}/\mathcal{G}')$ is proper closed in $Z$
and such that $\mathcal{P}$ holds for $i_*\mathcal{G}'$.
\end{enumerate}
\end{enumerate}
Then property $\mathcal{P}$ holds for every coherent sheaf on $X$.
\end{lemma}

\begin{proof}
Consider the collection
$$
\mathcal{T} =
\left\{
\begin{matrix}
T \subset |X|
\text{ closed such that there exists a coherent sheaf }
\mathcal{F} \\
\text{ with }
\text{Supp}(\mathcal{F}) = T
\text{ for which the lemma is wrong}
\end{matrix}
\right\}
$$
We are trying to show that $\mathcal{T}$ is empty. If not, then
because $|X|$ is Noetherian (Properties of Spaces, Lemma
\ref{spaces-properties-lemma-Noetherian-topology})
we can choose a minimal element $T \in \mathcal{T}$. This means that
there exists a coherent sheaf $\mathcal{F}$ on $X$ whose support is $T$
and for which the lemma does not hold. Clearly $T \not = \emptyset$
because the only sheaf with support in $\emptyset$ for which $\mathcal{P}$
does hold (by property (2)).

\medskip\noindent
If $T$ is not irreducible, then we can write $T = Z_1 \cup Z_2$
with $Z_1, Z_2$ closed and strictly smaller than $T$.
Then we can apply Lemma \ref{lemma-prepare-filter-support}
to get a short exact sequence of coherent sheaves
$$
0 \to
\mathcal{G}_1 \to
\mathcal{F} \to
\mathcal{G}_2 \to 0
$$
with $\text{Supp}(\mathcal{G}_i) \subset Z_i$. By minimality of
$T$ each of $\mathcal{G}_i$ has $\mathcal{P}$. Hence $\mathcal{F}$
has property $\mathcal{P}$ by (1), a contradiction.

\medskip\noindent
Suppose $T$ is irreducible. Let $\mathcal{J}$ be the sheaf of ideals
defining the reduced induced closed subspace structure on $T$,
see Properties of Spaces, Lemma
\ref{spaces-properties-lemma-reduced-closed-subspace}.
By Lemma \ref{lemma-power-ideal-kills-sheaf} we see there exists
an $n \geq 0$ such that $\mathcal{J}^n\mathcal{F} = 0$. Hence we obtain
a filtration
$$
0 = \mathcal{I}^n\mathcal{F} \subset \mathcal{I}^{n - 1}\mathcal{F}
\subset \ldots \subset \mathcal{I}\mathcal{F} \subset \mathcal{F}
$$
each of whose succesive subquotients is annihilated by $\mathcal{J}$.
Hence if each of these subquotients has a filtration as in the statement
of the lemma then also $\mathcal{F}$ does. In other words we may
assume that $\mathcal{J}$ does annihilate $\mathcal{F}$.

\medskip\noindent
Assume $T$ is irreducible and $\mathcal{J}\mathcal{F} = 0$ where
$\mathcal{J}$ is as above. Denote $i : Z \to X$ the closed subspace
corresponding to $\mathcal{J}$. Then $\mathcal{F} = i_*\mathcal{H}$
for some coherent $\mathcal{O}_Z$-module $\mathcal{H}$, see
Morphisms of Spaces, Lemma \ref{spaces-morphisms-lemma-i-star-equivalence}
and Lemma \ref{lemma-coherent-support-closed}.
Let $\mathcal{G}$ be the coherent sheaf on $Z$ satisfying
(3)(a) and (3)(b). We apply Lemma \ref{lemma-prepare-filter-irreducible}
to get injective maps
$$
\mathcal{I}_1^{\oplus r_1} \to \mathcal{H}
\quad\text{and}\quad
\mathcal{I}_2^{\oplus r_2} \to \mathcal{G}
$$
where the support of the cokernels are proper closed in $Z$. Hence we find
an nonempty open $V \subset Z$ such that
$$
\mathcal{H}^{\oplus r_2}_V \cong \mathcal{G}^{\oplus r_1}_V
$$
Let $\mathcal{I} \subset \mathcal{O}_Z$ be a quasi-coherent ideal sheaf
cutting out $Z \setminus V$ we obtain
(Lemma \ref{lemma-homs-over-open})
a map
$$
\mathcal{I}^n\mathcal{G}^{\oplus r_1} \longrightarrow \mathcal{H}^{\oplus r_2}
$$
which is an isomorphism over $V$. The kernel is supported on $Z \setminus V$
hence annihilated by some power of $\mathcal{I}$, see
Lemma \ref{lemma-power-ideal-kills-sheaf}. Thus after increasing
$n$ we may assume the displayed map is injective, see
Lemma \ref{lemma-Artin-Rees}. Applying (3)(b) we find
$\mathcal{G}' \subset \mathcal{I}^n\mathcal{G}$ such that
$$
(i_*\mathcal{G}')^{\oplus r_1} \longrightarrow
i_*\mathcal{H}^{\oplus r_2} = \mathcal{F}^{\oplus r_2}
$$
is injective with cokernel supported in a proper closed subset of $Z$
and such that property $\mathcal{P}$ holds for $i_*\mathcal{G}'$.
By (1) property $\mathcal{P}$ holds for $(i_*\mathcal{G}')^{\oplus r_1}$.
By (1) and minimality of $T = |Z|$ property $\mathcal{P}$ holds for
$\mathcal{F}^{\oplus r_2}$. And finally by (2) property $\mathcal{P}$
holds for $\mathcal{F}$ which is the desired contradiction.
\end{proof}








\section{Limits of coherent modules}
\label{section-limits}

\noindent
A colimit of coherent modules (on a locally Noetherian
algebraic space) is typically not coherent. But it is quasi-coherent
as any colimit of quasi-coherent modules on an algebraic space is
quasi-coherent, see Properties of Spaces, Lemma
\ref{spaces-properties-lemma-properties-quasi-coherent}.
Conversely, if the algebraic space is Noetherian, then every quasi-coherent
module is a filtered colimit of coherent modules.

\begin{lemma}
\label{lemma-directed-colimit-coherent}
Let $S$ be a scheme. Let $X$ be a Noetherian algebraic space over $S$.
Every quasi-coherent $\mathcal{O}_X$-module is the filtered colimit
of its coherent submodules.
\end{lemma}

\begin{proof}
Let $\mathcal{F}$ be a quasi-coherent $\mathcal{O}_X$-module.
If $\mathcal{G}, \mathcal{H} \subset \mathcal{F}$ are coherent
$\mathcal{O}_X$-submodules then the image of
$\mathcal{G} \oplus \mathcal{H} \to \mathcal{F}$ is another
coherent $\mathcal{O}_X$-submodule which contains both of them
(see Lemmas \ref{lemma-coherent-abelian-Noetherian} and
\ref{lemma-coherent-Noetherian-quasi-coherent-sub-quotient}).
In this way we see that the system is directed.
Hence it now suffices to show that $\mathcal{F}$ can be written as
a filtered colimit of coherent modules, as then we can take the
images of these modules in $\mathcal{F}$ to conclude there are
enough of them.

\medskip\noindent
Let $U$ be an affine scheme and $U \to X$ a surjective \'etale morphism.
Set $R = U \times_X U$ so that $X = U/R$ as usual. By
Properties of Spaces, Proposition
\ref{spaces-properties-proposition-quasi-coherent}
we see that $\text{Qcoh}(X) = \text{QCoh}(U, R, s, t, c)$. Hence
we reduce to showing the corresponding thing for
$\text{QCoh}(U, R, s, t, c)$. Thus the result follows from
the more general Groupoids, Lemma \ref{groupoids-lemma-colimit-coherent}.
\end{proof}

\begin{lemma}
\label{lemma-direct-colimit-finite-presentation}
Let $S$ be a scheme. Let $f : X \to Y$ be an affine morphism of algebraic
spaces over $S$ with $Y$ Noetherian. Then every quasi-coherent
$\mathcal{O}_X$-module is a filtered colimit of finitely presented
$\mathcal{O}_X$-modules.
\end{lemma}

\begin{proof}
Let $\mathcal{F}$ be a quasi-coherent $\mathcal{O}_X$-module.
Write $f_*\mathcal{F} = \colim \mathcal{H}_i$ with $\mathcal{H}_i$
a coherent $\mathcal{O}_Y$-module, see
Lemma \ref{lemma-directed-colimit-coherent}.
By Lemma \ref{lemma-coherent-Noetherian} the modules $\mathcal{H}_i$
are $\mathcal{O}_Y$-modules of finite presentation. Hence
$f^*\mathcal{H}_i$ is an $\mathcal{O}_X$-module of finite presentation, see
Properties of Spaces, Section
\ref{spaces-properties-section-properties-modules}.
We claim the map
$$
\colim f^*\mathcal{H}_i = f^*f_*\mathcal{F} \to \mathcal{F}
$$
is surjective as $f$ is assumed affine, Namely, choose
a scheme $V$ and a surjective \'etale morphism $V \to Y$. Set
$U = X \times_Y V$. Then $U$ is a scheme, $f' : U \to V$ is affine, and
$U \to X$ is surjective \'etale. By
Properties of Spaces, Lemma
\ref{spaces-properties-lemma-pushforward-etale-base-change-modules}
we see that $f'_*(\mathcal{F}|_U) = f_*\mathcal{F}|_V$ and similarly
for pullbacks. Thus the restriction of $f^*f_*\mathcal{F} \to \mathcal{F}$
to $U$ is the map
$$
f^*f_*\mathcal{F}|_U = (f')^*(f_*\mathcal{F})|_V) =
(f')^*f'_*(\mathcal{F}|_U) \to \mathcal{F}|_U
$$
which is surjective as $f'$ is an affine morphism of schemes.
Hence the claim holds.

\medskip\noindent
We conclude that every quasi-coherent module on $X$ is a quotient of a
filtered colimit of finitely presented modules. In particular, we see that
$\mathcal{F}$ is a cokernel of a map
$$
\colim_{j \in J} \mathcal{G}_j \longrightarrow \colim_{i \in I} \mathcal{H}_i
$$
with $\mathcal{G}_j$ and $\mathcal{H}_i$ finitely presented. Note
that for every $j \in I$ there exist $i \in I$ and a morphism
$\alpha : \mathcal{G}_j \to \mathcal{H}_i$ such that
$$
\xymatrix{
\mathcal{G}_j \ar[r]_\alpha \ar[d] & \mathcal{H}_i \ar[d] \\
\colim_{j \in J} \mathcal{G}_j \ar[r] &
\colim_{i \in I} \mathcal{H}_i
}
$$
commutes, see
Lemma \ref{lemma-finite-presentation-quasi-compact-colimit}.
In this situation $\text{Coker}(\alpha)$ is a finitely presented
$\mathcal{O}_X$-module which comes endowed with a map
$\text{Coker}(\alpha) \to \mathcal{F}$. Consider the set $K$ of
triples $(i, j, \alpha)$ as above. We say that
$(i, j, \alpha) \leq (i', j', \alpha')$ if and only if
$i \leq i'$, $j \leq j'$, and the diagram
$$
\xymatrix{
\mathcal{G}_j \ar[r]_\alpha \ar[d] & \mathcal{H}_i \ar[d] \\
\mathcal{G}_{j'} \ar[r]^{\alpha'} &
\mathcal{H}_{i'}
}
$$
commutes. It follows from the above that $K$ is a directed
partially ordered set,
$$
\mathcal{F} = \colim_{(i, j, \alpha) \in K} \text{Coker}(\alpha),
$$
and we win.
\end{proof}






\section{Vanishing cohomology}
\label{section-vanishing}

\noindent
In this section we show that a quasi-compact and quasi-separated
algebraic space is affine if it has vanishing higher cohomology
for all quasi-coherent sheaves. We do this in a sequence of lemmas
all of which will become obsolete once we prove
Proposition \ref{proposition-vanishing-affine}.

\begin{situation}
\label{situation-vanishing}
Here $S$ is a scheme and $X$ is a quasi-compact and quasi-separated
algebraic space over $S$ with the following property: For every
quasi-coherent $\mathcal{O}_X$-module $\mathcal{F}$ we have
$H^1(X, \mathcal{F}) = 0$. We set $A = \Gamma(X, \mathcal{O}_X)$.
\end{situation}

\noindent
We would like to show that the canonical morphism
$$
p : X \to \Spec(A)
$$
(see Properties of Spaces, Lemma
\ref{spaces-properties-lemma-morphism-to-affine-scheme}) is an isomorphism.
If $M$ is an $A$-module we denote $M \otimes_A \mathcal{O}_X$
the quasi-coherent module $p^*\tilde M$.

\begin{lemma}
\label{lemma-vanishing-compute}
In Situation \ref{situation-vanishing} for an $A$-module $M$ we have
$p_*(M \otimes_A \mathcal{O}_X) = \widetilde{M}$ and
$\Gamma(X, M \otimes_A \mathcal{O}_X) = M$.
\end{lemma}

\begin{proof}
The equality $p_*(M \otimes_A \mathcal{O}_X) = \widetilde{M}$ follows
from the equality $\Gamma(X, M \otimes_A \mathcal{O}_X) = M$ as
$p_*(M \otimes_A \mathcal{O}_X)$ is a quasi-coherent module on
$\Spec(A)$ by Morphisms of Spaces, Lemma
\ref{spaces-morphisms-lemma-pushforward}.
Observe that $\Gamma(X, \bigoplus_{i \in I} \mathcal{O}_X) =
\bigoplus_{i \in I} A$ by Lemma \ref{lemma-colimits}. Hence the
lemma holds for free modules. Choose a short exact sequence
$F_1 \to F_0 \to M$ where $F_0, F_1$ are free $A$-modules. Since
$H^1(X, -)$ is zero the global sections functor is right exact.
Moreover the pullback $p^*$ is right exact as well. Hence we see
that
$$
\Gamma(X, F_1 \otimes_A \mathcal{O}_X) \to
\Gamma(X, F_0 \otimes_A \mathcal{O}_X) \to
\Gamma(X, M \otimes_A \mathcal{O}_X) \to 0
$$
is exact. The result follows.
\end{proof}

\noindent
The following lemma shows that Situation \ref{situation-vanishing}
is preserved by base change of $X \to \Spec(A)$ by $\Spec(A') \to \Spec(A)$.

\begin{lemma}
\label{lemma-vanishing-base-change}
In Situation \ref{situation-vanishing}.
\begin{enumerate}
\item Given an affine morphism $X' \to X$ of algebraic spaces, we have
$H^1(X', \mathcal{F}') = 0$ for every quasi-coherent
$\mathcal{O}_{X'}$-module $\mathcal{F}'$.
\item Given an $A$-algebra $A'$ setting $X' = X \times_{\Spec(A)} \Spec(A')$
the morphism $X' \to X$ is affine and $\Gamma(X', \mathcal{O}_{X'}) = A'$.
\end{enumerate}
\end{lemma}

\begin{proof}
Part (1) follows from Lemma \ref{lemma-affine-vanishing-higher-direct-images}
and the Leray spectral sequence (Cohomology on Sites, Lemma
\ref{sites-cohomology-lemma-Leray}). Let $A \to A'$ be as in (2).
Then $X' \to X$ is affine because affine morphisms are preserved under
base change (Morphisms of Spaces, Lemma
\ref{spaces-morphisms-lemma-base-change-affine}) and the
fact that a morphism of affine schemes is affine. The equality
$\Gamma(X', \mathcal{O}_{X'}) = A'$ follows as
$(X' \to X)_*\mathcal{O}_{X'} = A' \otimes_A \mathcal{O}_X$
by Lemma \ref{lemma-affine-base-change} and thus
$$
\Gamma(X', \mathcal{O}_{X'}) =
\Gamma(X, (X' \to X)_*\mathcal{O}_{X'}) =
\Gamma(X, A' \otimes_A \mathcal{O}_X) = A'
$$
by Lemma \ref{lemma-vanishing-compute}.
\end{proof}

\begin{lemma}
\label{lemma-vanishing-separate-closed}
In Situation \ref{situation-vanishing}. Let $Z_0, Z_1 \subset |X|$
be disjoint closed subsets. Then there exists an $a \in A$ such that
$Z_0 \subset V(a)$ and $Z_1 \subset V(a - 1)$.
\end{lemma}

\begin{proof}
We may and do endow $Z_0$, $Z_1$ with the reduced induced subspace structure
(Properties of Spaces, Definition
\ref{spaces-properties-definition-reduced-induced-space}) and we denote
$i_0 : Z_0 \to X$ and $i_1 : Z_1 \to X$ the corresponding closed immersions.
Since $Z_0 \cap Z_1 = \emptyset$ we see that the canonical map of
quasi-coherent $\mathcal{O}_X$-modules
$$
\mathcal{O}_X
\longrightarrow
i_{0, *}\mathcal{O}_{Z_0} \oplus i_{1, *}\mathcal{O}_{Z_1}
$$
is surjective (look at stalks at geometric points). Since $H^1(X, -)$ is
zero on the kernel of this map the induced map of global sections is
surjective. Thus we can find $a \in A$ which maps to the global section
$(0, 1)$ of the right hand side.
\end{proof}

\begin{lemma}
\label{lemma-vanishing-surjective}
In Situation \ref{situation-vanishing} the morphism $p : X \to \Spec(A)$ is
surjective.
\end{lemma}

\begin{proof}
Let $A \to k$ be a ring homomorphism where $k$ is a field. It suffices to
show that $X_k = \Spec(k) \times_{\Spec(A)} X$ is nonempty. By
Lemma \ref{lemma-vanishing-base-change} we have
$\Gamma(X_k, \mathcal{O}) = k$. Hence $X_k$ is nonempty.
\end{proof}

\begin{lemma}
\label{lemma-vanishing-universally-closed}
In Situation \ref{situation-vanishing} the morphism $p : X \to \Spec(A)$ is
universally closed.
\end{lemma}

\begin{proof}
Let $Z \subset |X|$ be a closed subset. We may and do endow $Z$ with the
reduced induced subspace structure (Properties of Spaces, Definition
\ref{spaces-properties-definition-reduced-induced-space}) and we denote
$i : Z \to X$ the corresponding closed immersions. Then $i$ is affine
(Morphisms of Spaces, Lemma
\ref{spaces-morphisms-lemma-closed-immersion-affine}).
Hence $Z$ is another algebraic space as in Situation \ref{situation-vanishing}
by Lemma \ref{lemma-vanishing-base-change}.
Set $B = \Gamma(Z, \mathcal{O}_Z)$. Since $\mathcal{O}_X \to i_*\mathcal{O}_Z$
is surjective, we see that $A \to B$ is surjective by the vanishing of
$H^1$ of the kernel. Consider the commutative diagram
$$
\xymatrix{
Z \ar[r]_i \ar[d] & X \ar[d] \\
\Spec(B) \ar[r] & \Spec(A)
}
$$
By Lemma \ref{lemma-vanishing-surjective} the map $Z \to \Spec(B)$ is
surjective and by the above $\Spec(B) \to \Spec(A)$ is a closed immersion.
Thus $p$ is closed.

\medskip\noindent
By Lemma \ref{lemma-vanishing-base-change} we see that the base change
of $p$ by $\Spec(A') \to \Spec(A)$ is closed for every ring map $A \to A'$.
Hence $p$ is universally closed by
Morphisms of Spaces, Lemma
\ref{spaces-morphisms-lemma-universally-closed-local}.
\end{proof}

\begin{lemma}
\label{lemma-vanishing-injective}
In Situation \ref{situation-vanishing} the morphism $p : X \to \Spec(A)$ is
universally injective.
\end{lemma}

\begin{proof}
Let $A \to k$ be a ring homomorphism where $k$ is a field. It suffices to
show that $\Spec(k) \times_{\Spec(A)} X$ has at most one point (see
Morphisms of Spaces, Lemma
\ref{spaces-morphisms-lemma-universally-injective-local}).
Thus we may assume that $A$ is a field and we have to show that $|X|$
has at most one point.

\medskip\noindent
Let's think of $X$ as an algebraic space over $\Spec(k)$ and let's
use the notation $X(K)$ to denote $K$-valued points of $X$
for any extension $k \subset K$, see
Morphisms of Spaces, Section \ref{spaces-morphisms-section-points-fields}.
If $k \subset K$ is an algebraically closed field extension
of large transcendence degree, then we see that $X(K) \to |X|$
is surjective, see Morphisms of Spaces, Lemma
\ref{spaces-morphisms-lemma-large-enough}. Hence, after replacing $k$
by $K$, we see that it suffices to prove that $X(k)$ is a singleton
(in the case $A = k)$.

\medskip\noindent
Let $x, x' \in X(k)$. By Decent Spaces, Lemma
\ref{decent-spaces-lemma-algebraic-residue-field-extension-closed-point}
we see that $x$ and $x'$ are closed points of $|X|$. Hence $x$ and $x'$
map to distinct points of $\Spec(k)$ if $x \not = x'$ by
Lemma \ref{lemma-vanishing-separate-closed}. We conclude that
$x = x'$ as desired.
\end{proof}

\begin{lemma}
\label{lemma-vanishing-separated}
In Situation \ref{situation-vanishing} the morphism $p : X \to \Spec(A)$ is
separated.
\end{lemma}

\begin{proof}
We will use the results of
Lemmas \ref{lemma-vanishing-compute},
\ref{lemma-vanishing-base-change}
\ref{lemma-vanishing-surjective},
\ref{lemma-vanishing-universally-closed}, and
\ref{lemma-vanishing-injective}
without further mention.
We will use the valuative criterion of separatedness, see
Morphisms of Spaces, Lemma
\ref{spaces-morphisms-lemma-valuative-criterion-separatedness}.
Let $R$ be a valuation ring over $A$ with fraction field $K$.
Let $\Spec(K) \to X$ be a morphism over $\Spec(A)$. We have
to show that we can extend this to a morphism $\Spec(R) \to X$
in at most one way. We may replace $A$ by $R$ and $X$ by
$\Spec(R) \times_{\Spec(A)} X$. Hence we may assume that $A = R$
is a valuation ring with field of fractions $K$ and that we have a
$K$-point $x$ in $X$.

\medskip\noindent
It is clear that we may replace $X$ by its reduction, see
Properties of Spaces, Lemma
\ref{spaces-properties-lemma-map-into-reduction}.
Since $X \to \Spec(A)$ is a universal homeomorphism we see
that $|X|$ is the closure of $\{x\}$. For every nonzero $f \in A$
the kernel $\mathcal{I}_f$ of $f : \mathcal{O}_X \to \mathcal{O}_X$
is a quasi-coherent sheaf of ideals and any section $\Spec(A) \to X$
of $p$ factors through the closed subscheme defined by $\mathcal{I}_f$.
Hence we may also replace $X$ by the closed subspace cut out by
the quasi-coherent sheaf of ideals $\sum_{f \in A} \mathcal{I}_f$.
In other words, we may assume that any nonzero $f \in A$ is a
nonzerodivisor on $\mathcal{O}_X$.

\medskip\noindent
Let $U$ be an affine scheme and let $U \to X$ be a surjective \'etale
morphism. Note that $U = \Spec(B)$ where $B$ is a reduced $A$-algebra.
Note that $B$ is flat over $A$ by More on Algebra, Lemma
\ref{more-algebra-lemma-valuation-ring-torsion-free-flat}
and the fact $\mathcal{O}_X$ has no nonzero $A$-torsion.
The fibre product $U \times_X \Spec(K) = \Spec(B \otimes_A K) =
\coprod_{i = 1, \ldots, n} \Spec(K_i)$ is a finite disjoint union
of spectra of finite separable field extensions $K_i \supset K$ (for example
because $X$ is a decent space for which the general
Decent Spaces, Lemma \ref{decent-spaces-lemma-UR-finite-above-x} holds).
Choose a finite Galois extension $K \subset K'$ such that each $K_i$
embeds into $K'$ over $K$ and choose a valuation ring $A' \subset K'$
dominating $A$ (see
Algebra, Lemma \ref{algebra-lemma-dominate}).
After replacing $A$ by $A'$ and $X$ by $\Spec(A') \times_{\Spec(A)} X$
we may assume that $K_i = K$ for all $i$ (small detail omitted).

\medskip\noindent
If $X$ is normal then $B$ is a finite product
$B = B_1 \times \ldots \times B_n$ of
normal domains (see Algebra, Lemma
\ref{algebra-lemma-characterize-reduced-ring-normal}). Each of
these has fraction field $K$ by the above. 
One of these rings $B_i$, say $B_1$ has a prime ideal lying over
$\mathfrak m_A$ because $X \to \Spec(A)$ is surjective.
Then $A = B_1$ as $A$ is a valuation ring. Thus we see that
there exists an \'etale morphism $\Spec(A) \to X$!
Of course this implies that $X = \Spec(A)$ (for example by
Morphisms of Spaces, Lemma
\ref{spaces-morphisms-lemma-etale-universally-injective-open}) and we win in
the case that $X$ is normal.

\medskip\noindent
In the general (possibly nonnormal) case we see that $U = \Spec(B)$
has finitely many irreducible components (as all minimal primes of $B$
lie over $(0) \subset A$ by flatness of $A \to B$). Thus we may consider
the normalization $X^\nu \to X$ of $X$, see
Morphisms of Spaces, Lemma \ref{spaces-morphisms-lemma-normalization}.
Note that $X^\nu \to X$ is integral hence affine and universally closed (see
Morphisms of Spaces, Lemma
\ref{spaces-morphisms-lemma-integral-universally-closed}).
Note that $X^\nu \times_X U = U^\nu$, in particular $X^\nu \to \Spec(A)$
is flat (as the integral closure of $B$ in its total quotient ring is
torsion free over $A$ hence flat). Set
$A^\nu = \Gamma(X^\nu, \mathcal{O}_{X^\nu}) = A$ and consider the diagram
$$
\xymatrix{
X^\nu \ar[d] \ar[r] & X \ar[d] \\
\Spec(A^\nu) \ar[r] & \Spec(A)
}
$$
By the lemmas mentioned at the beginning of the proof,
the left vertical arrow is (universally) surjective,
the top vertical arrow is universally closed, and the right vertical
arrow is universally closed. Hence $\Spec(A^\nu) \to \Spec(A)$ is
universally closed. Hence $A \subset A^\nu$ is integral, see
Morphisms, Lemma \ref{morphisms-lemma-integral-universally-closed}. Finally,
$A^\nu$ is a torsion free $A$-algebra with $A^\nu \otimes_A K = K$
(as $\Spec(K)$ maps onto $X_K = X^\nu_K$). Hence $A^\nu = A$.
Observe that $x : \Spec(K) \to X$ lifts to $x^\nu : \Spec(K) \to X^\nu$
and that
$$
U^\nu \times_{X^\nu, x^\nu} \Spec(K) =
X \times_{U, x} \Spec(K) =
\coprod\nolimits_{i = 1, \ldots, n} \Spec(K)
$$
as normalization does not chance the scheme $U$ over its generic points.
Finally, as $X^\nu \to X$ is universally closed any morphism
$\Spec(A) \to X$ extending $x$ lifts to a morphism into $X^\nu$
extending $x^\nu$ (see Decent Spaces, Proposition
\ref{decent-spaces-proposition-characterize-universally-closed}).
Thus we may replace $X$ by $X^\nu$ and assume that $X$
is normal. This case was treated above.
\end{proof}

\begin{proposition}
\label{proposition-vanishing-affine}
Any algebraic space as in Situation \ref{situation-vanishing} is an
affine scheme.
\end{proposition}

\begin{proof}
Choose an affine scheme $U = \Spec(B)$ and a surjective \'etale
morphism $\varphi : U \to X$. Set $R = U \times_X U$. As $p$ is separated
(Lemma \ref{lemma-vanishing-separated}) we see that $R$ is a
closed subscheme of $U \times_{\Spec(A)} U = \Spec(B \otimes_A B)$.
Hence $R = \Spec(C)$ is affine too and the ring map
$$
B \otimes_A B \longrightarrow C
$$
is surjective. Let us denote the two maps $s, t : B \to C$ as usual. Pick
$g_1, \ldots, g_m \in B$ such that $s(g_1), \ldots, s(g_m)$ generate $C$
over $t : B \to C$ (which is possible as $t : B \to C$ is of finite
presentation and the displayed map is surjective). Then $g_1, \ldots, g_m$
give global sections of $\varphi_*\mathcal{O}_U$ and the map
$$
\mathcal{O}_X[z_1, \ldots, z_n] \longrightarrow \varphi_*\mathcal{O}_U,
\quad
z_j \longmapsto g_j
$$
is surjective: you can check this by restricting to $U$.
Namely, $\varphi^*\varphi_*\mathcal{O}_U = t_*\mathcal{O}_R$
(by Lemma \ref{lemma-flat-base-change-cohomology})
hence you get exactly the condition that $s(g_i)$ generate $C$
over $t : B \to C$. By the vanishing of $H^1$ of the kernel we see that
$$
\Gamma(X, \mathcal{O}_X[x_1, \ldots, x_n]) =
A[x_1, \ldots, x_n] \longrightarrow
\Gamma(X, \varphi_*\mathcal{O}_U) = \Gamma(U, \mathcal{O}_U) = B
$$
is surjective. Thus we conclude that $B$ is a finite type $A$-algebra.
Hence $X \to \Spec(A)$ is of finite type and separated.
By Lemma \ref{lemma-vanishing-injective}
and
Morphisms of Spaces, Lemma \ref{spaces-morphisms-lemma-locally-quasi-finite}
it is also locally quasi-finite. Hence $X \to \Spec(A)$ is representable by
Morphisms of Spaces, Lemma
\ref{spaces-morphisms-lemma-locally-quasi-finite-separated-representable}
and $X$ is a scheme. Finally $X$ is affine, hence equal to $\Spec(A)$,
by an application of Cohomology of Schemes, Lemma
\ref{coherent-lemma-quasi-compact-h1-zero-covering}.
\end{proof}





\section{Other chapters}

\begin{multicols}{2}
\begin{enumerate}
\item \hyperref[introduction-section-phantom]{Introduction}
\item \hyperref[conventions-section-phantom]{Conventions}
\item \hyperref[sets-section-phantom]{Set Theory}
\item \hyperref[categories-section-phantom]{Categories}
\item \hyperref[topology-section-phantom]{Topology}
\item \hyperref[sheaves-section-phantom]{Sheaves on Spaces}
\item \hyperref[algebra-section-phantom]{Commutative Algebra}
\item \hyperref[sites-section-phantom]{Sites and Sheaves}
\item \hyperref[homology-section-phantom]{Homological Algebra}
\item \hyperref[derived-section-phantom]{Derived Categories}
\item \hyperref[more-algebra-section-phantom]{More Algebra}
\item \hyperref[simplicial-section-phantom]{Simplicial Methods}
\item \hyperref[modules-section-phantom]{Sheaves of Modules}
\item \hyperref[sites-modules-section-phantom]{Modules on Sites}
\item \hyperref[injectives-section-phantom]{Injectives}
\item \hyperref[cohomology-section-phantom]{Cohomology of Sheaves}
\item \hyperref[sites-cohomology-section-phantom]{Cohomology on Sites}
\item \hyperref[hypercovering-section-phantom]{Hypercoverings}
\item \hyperref[schemes-section-phantom]{Schemes}
\item \hyperref[constructions-section-phantom]{Constructions of Schemes}
\item \hyperref[properties-section-phantom]{Properties of Schemes}
\item \hyperref[morphisms-section-phantom]{Morphisms of Schemes}
\item \hyperref[coherent-section-phantom]{Coherent Cohomology}
\item \hyperref[divisors-section-phantom]{Divisors}
\item \hyperref[limits-section-phantom]{Limits of Schemes}
\item \hyperref[varieties-section-phantom]{Varieties}
\item \hyperref[chow-section-phantom]{Chow Homology}
\item \hyperref[topologies-section-phantom]{Topologies on Schemes}
\item \hyperref[descent-section-phantom]{Descent}
\item \hyperref[more-morphisms-section-phantom]{More on Morphisms}
\item \hyperref[flat-section-phantom]{More on Flatness}
\item \hyperref[groupoids-section-phantom]{Groupoid Schemes}
\item \hyperref[more-groupoids-section-phantom]{More on Groupoid Schemes}
\item \hyperref[etale-section-phantom]{\'Etale Morphisms of Schemes}
\item \hyperref[etale-cohomology-section-phantom]{\'Etale Cohomology}
\item \hyperref[spaces-section-phantom]{Algebraic Spaces}
\item \hyperref[spaces-properties-section-phantom]{Properties of Algebraic Spaces}
\item \hyperref[spaces-morphisms-section-phantom]{Morphisms of Algebraic Spaces}
\item \hyperref[spaces-topologies-section-phantom]{Topologies on Algebraic Spaces}
\item \hyperref[spaces-descent-section-phantom]{Descent and Algebraic Spaces}
\item \hyperref[spaces-more-morphisms-section-phantom]{More on Morphisms of Spaces}
\item \hyperref[quot-section-phantom]{Quot and Hilbert Spaces}
\item \hyperref[stacks-section-phantom]{Stacks}
\item \hyperref[spaces-groupoids-section-phantom]{Groupoids in Algebraic Spaces}
\item \hyperref[spaces-more-groupoids-section-phantom]{More on Groupoids in Spaces}
\item \hyperref[bootstrap-section-phantom]{Bootstrap}
\item \hyperref[examples-stacks-section-phantom]{Examples of Stacks}
\item \hyperref[groupoids-quotients-section-phantom]{Quotients of Groupoids}
\item \hyperref[algebraic-section-phantom]{Algebraic Stacks}
\item \hyperref[criteria-section-phantom]{Criteria for Representability}
\item \hyperref[stacks-properties-section-phantom]{Properties of Algebraic Stacks}
\item \hyperref[stacks-morphisms-section-phantom]{Morphisms of Algebraic Stacks}
\item \hyperref[examples-section-phantom]{Examples}
\item \hyperref[exercises-section-phantom]{Exercises}
\item \hyperref[guide-section-phantom]{Guide to Literature}
\item \hyperref[desirables-section-phantom]{Desirables}
\item \hyperref[coding-section-phantom]{Coding Style}
\item \hyperref[fdl-section-phantom]{GNU Free Documentation License}
\item \hyperref[index-section-phantom]{Auto Generated Index}
\end{enumerate}
\end{multicols}


\bibliography{my}
\bibliographystyle{amsalpha}

\end{document}
