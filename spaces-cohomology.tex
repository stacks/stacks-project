\IfFileExists{stacks-project.cls}{%
\documentclass{stacks-project}
}{%
\documentclass{amsart}
}

% The following AMS packages are automatically loaded with
% the amsart documentclass:
%\usepackage{amsmath}
%\usepackage{amssymb}
%\usepackage{amsthm}

% For dealing with references we use the comment environment
\usepackage{verbatim}
\newenvironment{reference}{\comment}{\endcomment}
%\newenvironment{reference}{}{}
\newenvironment{slogan}{\comment}{\endcomment}
\newenvironment{history}{\comment}{\endcomment}

% For commutative diagrams you can use
% \usepackage{amscd}
\usepackage[all]{xy}

% We use 2cell for 2-commutative diagrams.
\xyoption{2cell}
\UseAllTwocells

% To put source file link in headers.
% Change "template.tex" to "this_filename.tex"
% \usepackage{fancyhdr}
% \pagestyle{fancy}
% \lhead{}
% \chead{}
% \rhead{Source file: \url{template.tex}}
% \lfoot{}
% \cfoot{\thepage}
% \rfoot{}
% \renewcommand{\headrulewidth}{0pt}
% \renewcommand{\footrulewidth}{0pt}
% \renewcommand{\headheight}{12pt}

\usepackage{multicol}

% For cross-file-references
\usepackage{xr-hyper}

% Package for hypertext links:
\usepackage{hyperref}

% For any local file, say "hello.tex" you want to link to please
% use \externaldocument[hello-]{hello}
\externaldocument[introduction-]{introduction}
\externaldocument[conventions-]{conventions}
\externaldocument[sets-]{sets}
\externaldocument[categories-]{categories}
\externaldocument[topology-]{topology}
\externaldocument[sheaves-]{sheaves}
\externaldocument[sites-]{sites}
\externaldocument[stacks-]{stacks}
\externaldocument[fields-]{fields}
\externaldocument[algebra-]{algebra}
\externaldocument[brauer-]{brauer}
\externaldocument[homology-]{homology}
\externaldocument[derived-]{derived}
\externaldocument[simplicial-]{simplicial}
\externaldocument[more-algebra-]{more-algebra}
\externaldocument[smoothing-]{smoothing}
\externaldocument[modules-]{modules}
\externaldocument[sites-modules-]{sites-modules}
\externaldocument[injectives-]{injectives}
\externaldocument[cohomology-]{cohomology}
\externaldocument[sites-cohomology-]{sites-cohomology}
\externaldocument[dga-]{dga}
\externaldocument[dpa-]{dpa}
\externaldocument[hypercovering-]{hypercovering}
\externaldocument[schemes-]{schemes}
\externaldocument[constructions-]{constructions}
\externaldocument[properties-]{properties}
\externaldocument[morphisms-]{morphisms}
\externaldocument[coherent-]{coherent}
\externaldocument[divisors-]{divisors}
\externaldocument[limits-]{limits}
\externaldocument[varieties-]{varieties}
\externaldocument[topologies-]{topologies}
\externaldocument[descent-]{descent}
\externaldocument[perfect-]{perfect}
\externaldocument[more-morphisms-]{more-morphisms}
\externaldocument[flat-]{flat}
\externaldocument[groupoids-]{groupoids}
\externaldocument[more-groupoids-]{more-groupoids}
\externaldocument[etale-]{etale}
\externaldocument[chow-]{chow}
\externaldocument[intersection-]{intersection}
\externaldocument[pic-]{pic}
\externaldocument[adequate-]{adequate}
\externaldocument[dualizing-]{dualizing}
\externaldocument[duality-]{duality}
\externaldocument[discriminant-]{discriminant}
\externaldocument[local-cohomology-]{local-cohomology}
\externaldocument[curves-]{curves}
\externaldocument[resolve-]{resolve}
\externaldocument[models-]{models}
\externaldocument[pione-]{pione}
\externaldocument[etale-cohomology-]{etale-cohomology}
\externaldocument[proetale-]{proetale}
\externaldocument[crystalline-]{crystalline}
\externaldocument[spaces-]{spaces}
\externaldocument[spaces-properties-]{spaces-properties}
\externaldocument[spaces-morphisms-]{spaces-morphisms}
\externaldocument[decent-spaces-]{decent-spaces}
\externaldocument[spaces-cohomology-]{spaces-cohomology}
\externaldocument[spaces-limits-]{spaces-limits}
\externaldocument[spaces-divisors-]{spaces-divisors}
\externaldocument[spaces-over-fields-]{spaces-over-fields}
\externaldocument[spaces-topologies-]{spaces-topologies}
\externaldocument[spaces-descent-]{spaces-descent}
\externaldocument[spaces-perfect-]{spaces-perfect}
\externaldocument[spaces-more-morphisms-]{spaces-more-morphisms}
\externaldocument[spaces-flat-]{spaces-flat}
\externaldocument[spaces-groupoids-]{spaces-groupoids}
\externaldocument[spaces-more-groupoids-]{spaces-more-groupoids}
\externaldocument[bootstrap-]{bootstrap}
\externaldocument[spaces-pushouts-]{spaces-pushouts}
\externaldocument[groupoids-quotients-]{groupoids-quotients}
\externaldocument[spaces-more-cohomology-]{spaces-more-cohomology}
\externaldocument[spaces-simplicial-]{spaces-simplicial}
\externaldocument[formal-spaces-]{formal-spaces}
\externaldocument[restricted-]{restricted}
\externaldocument[spaces-resolve-]{spaces-resolve}
\externaldocument[formal-defos-]{formal-defos}
\externaldocument[defos-]{defos}
\externaldocument[cotangent-]{cotangent}
\externaldocument[examples-defos-]{examples-defos}
\externaldocument[algebraic-]{algebraic}
\externaldocument[examples-stacks-]{examples-stacks}
\externaldocument[stacks-sheaves-]{stacks-sheaves}
\externaldocument[criteria-]{criteria}
\externaldocument[artin-]{artin}
\externaldocument[quot-]{quot}
\externaldocument[stacks-properties-]{stacks-properties}
\externaldocument[stacks-morphisms-]{stacks-morphisms}
\externaldocument[stacks-limits-]{stacks-limits}
\externaldocument[stacks-cohomology-]{stacks-cohomology}
\externaldocument[stacks-perfect-]{stacks-perfect}
\externaldocument[stacks-introduction-]{stacks-introduction}
\externaldocument[stacks-more-morphisms-]{stacks-more-morphisms}
\externaldocument[stacks-geometry-]{stacks-geometry}
\externaldocument[moduli-]{moduli}
\externaldocument[moduli-curves-]{moduli-curves}
\externaldocument[examples-]{examples}
\externaldocument[exercises-]{exercises}
\externaldocument[guide-]{guide}
\externaldocument[desirables-]{desirables}
\externaldocument[coding-]{coding}
\externaldocument[obsolete-]{obsolete}
\externaldocument[fdl-]{fdl}
\externaldocument[index-]{index}

% Theorem environments.
%
\theoremstyle{plain}
\newtheorem{theorem}[subsection]{Theorem}
\newtheorem{proposition}[subsection]{Proposition}
\newtheorem{lemma}[subsection]{Lemma}

\theoremstyle{definition}
\newtheorem{definition}[subsection]{Definition}
\newtheorem{example}[subsection]{Example}
\newtheorem{exercise}[subsection]{Exercise}
\newtheorem{situation}[subsection]{Situation}

\theoremstyle{remark}
\newtheorem{remark}[subsection]{Remark}
\newtheorem{remarks}[subsection]{Remarks}

\numberwithin{equation}{subsection}

% Macros
%
\def\lim{\mathop{\rm lim}\nolimits}
\def\colim{\mathop{\rm colim}\nolimits}
\def\Spec{\mathop{\rm Spec}}
\def\Hom{\mathop{\rm Hom}\nolimits}
\def\Ext{\mathop{\rm Ext}\nolimits}
\def\SheafHom{\mathop{\mathcal{H}\!{\it om}}\nolimits}
\def\SheafExt{\mathop{\mathcal{E}\!{\it xt}}\nolimits}
\def\Sch{\textit{Sch}}
\def\Mor{\mathop{\rm Mor}\nolimits}
\def\Ob{\mathop{\rm Ob}\nolimits}
\def\Sh{\mathop{\textit{Sh}}\nolimits}
\def\NL{\mathop{N\!L}\nolimits}
\def\proetale{{pro\text{-}\acute{e}tale}}
\def\etale{{\acute{e}tale}}
\def\QCoh{\textit{QCoh}}
\def\Ker{\mathop{\rm Ker}}
\def\Im{\mathop{\rm Im}}
\def\Coker{\mathop{\rm Coker}}
\def\Coim{\mathop{\rm Coim}}

%
% Macros for moduli stacks/spaces
%
\def\QCohstack{\mathcal{QC}\!{\it oh}}
\def\Cohstack{\mathcal{C}\!{\it oh}}
\def\Spacesstack{\mathcal{S}\!{\it paces}}
\def\Quotfunctor{{\rm Quot}}
\def\Hilbfunctor{{\rm Hilb}}
\def\Curvesstack{\mathcal{C}\!{\it urves}}
\def\Polarizedstack{\mathcal{P}\!{\it olarized}}
\def\Complexesstack{\mathcal{C}\!{\it omplexes}}
% \Pic is the operator that assigns to X its picard group, usage \Pic(X)
% \Picardstack_{X/B} denotes the Picard stack of X over B
% \Picardfunctor_{X/B} denotes the Picard functor of X over B
\def\Pic{\mathop{\rm Pic}\nolimits}
\def\Picardstack{\mathcal{P}\!{\it ic}}
\def\Picardfunctor{{\rm Pic}}
\def\Deformationcategory{\mathcal{D}\!{\it ef}}


% OK, start here.
%
\begin{document}

\title{Cohomology of Algebraic Spaces}

\maketitle

\phantomsection
\label{section-phantom}

\tableofcontents




\section{Introduction}
\label{section-introduction}

\noindent
In this chapter we write about cohomology of algebraic spaces.
Although we prove some results on cohomology of abelian sheaves,
we focus mainly on cohomology of quasi-coherent sheaves, i.e.,
we prove analogues of the results in the chapter ``Cohomology of Schemes''.
Some of the results in this chapter can be found in \cite{Kn}.

\medskip\noindent
An important missing ingredient in this chapter is the
{\it induction principle}, i.e., the analogue for quasi-compact
and quasi-separated algebraic spaces of
Cohomology of Schemes, Lemma \ref{coherent-lemma-induction-principle}.
This is formulated precisely and proved in detail in
Derived Categories of Spaces, Section \ref{spaces-perfect-section-induction}.
Instead of the induction principle, in this chapter we use the
alternating {\v C}ech complex, see
Section \ref{section-alternating-cech}.
It is designed to prove vanishing statements such as
Proposition \ref{proposition-vanishing},
but in some cases the induction principle is a more powerful
and perhaps more ``standard'' tool. We encourage the reader
to take a look at the induction principle
after reading some of the material in this section.



\section{Conventions}
\label{section-conventions}

\noindent
The standing assumption is that all schemes are contained in
a big fppf site $\Sch_{fppf}$. And all rings $A$ considered
have the property that $\Spec(A)$ is (isomorphic) to an
object of this big site.

\medskip\noindent
Let $S$ be a scheme and let $X$ be an algebraic space over $S$.
In this chapter and the following we will write $X \times_S X$
for the product of $X$ with itself (in the category of algebraic
spaces over $S$), instead of $X \times X$.








\section{Higher direct images}
\label{section-higher-direct-image}

\noindent
Let $S$ be a scheme. Let $X$ be a representable algebraic space over $S$.
Let $\mathcal{F}$ be a quasi-coherent module on $X$ (see
Properties of Spaces, Section \ref{spaces-properties-section-quasi-coherent}).
By
Descent, Proposition \ref{descent-proposition-same-cohomology-quasi-coherent}
the cohomology groups $H^i(X, \mathcal{F})$ agree with the usual
cohomology group computed in the Zariski topology of the corresponding
quasi-coherent module on the scheme representing $X$.

\medskip\noindent
More generally, let $f : X \to Y$ be a quasi-compact and quasi-separated
morphism of representable algebraic spaces $X$ and $Y$. Let
$\mathcal{F}$ be a quasi-coherent module on $X$. By
Descent, Lemma \ref{descent-lemma-higher-direct-images-small-etale}
the sheaf $R^if_*\mathcal{F}$ agrees with the
usual higher direct image computed for the Zariski topology
of the quasi-coherent module on the scheme representing $X$
mapping to the scheme representing $Y$.

\medskip\noindent
More generally still, suppose $f : X \to Y$ is a
representable, quasi-compact, and
quasi-separated morphism of algebraic spaces over $S$. Let $V$ be a scheme
and let $V \to Y$ be an \'etale surjective morphism. Let $U = V \times_Y X$
and let $f' : U \to V$ be the base change of $f$. Then for any
quasi-coherent $\mathcal{O}_X$-module $\mathcal{F}$ we have
\begin{equation}
\label{equation-representable-higher-direct-image}
R^if'_*(\mathcal{F}|_U) = (R^if_*\mathcal{F})|_V,
\end{equation}
see
Properties of Spaces,
Lemma \ref{spaces-properties-lemma-pushforward-etale-base-change-modules}.
And because $f' : U \to V$ is a quasi-compact and quasi-separated
morphism of schemes, by the remark of the preceding paragraph we may
compute $R^if'_*(\mathcal{F}|_U)$ by thinking of $\mathcal{F}|_U$ as a
quasi-coherent sheaf on the scheme $U$, and $f'$ as a morphism of schemes.
We will frequently use this without further mention.

\medskip\noindent
Next, we prove that higher direct images of quasi-coherent sheaves are
quasi-coherent for any quasi-compact and quasi-separated morphism of
algebraic spaces. In the proof we use a trick; a ``better'' proof would
use a relative {\v C}ech complex, as discussed in
Sheaves on Stacks, Sections \ref{stacks-sheaves-section-cech} and
\ref{stacks-sheaves-section-sheaf-cech-complex} ff.

\begin{lemma}
\label{lemma-higher-direct-image}
Let $S$ be a scheme. Let $f : X \to Y$ be a morphism of algebraic spaces
over $S$. If $f$ is quasi-compact and quasi-separated, then $R^if_*$
transforms quasi-coherent $\mathcal{O}_X$-modules into
quasi-coherent $\mathcal{O}_Y$-modules.
\end{lemma}

\begin{proof}
Let $V \to Y$ be an \'etale morphism where $V$ is an affine scheme. Set
$U = V \times_Y X$ and denote $f' : U \to V$ the induced morphism.
Let $\mathcal{F}$ be a quasi-coherent $\mathcal{O}_X$-module. By
Properties of Spaces, Lemma
\ref{spaces-properties-lemma-pushforward-etale-base-change-modules}
we have
$R^if'_*(\mathcal{F}|_U) = (R^if_*\mathcal{F})|_V$.
Since the property of being a quasi-coherent module is local in the
\'etale topology on $Y$ (see
Properties of Spaces, Lemma
\ref{spaces-properties-lemma-characterize-quasi-coherent})
we may replace $Y$ by $V$, i.e., we may assume $Y$ is an affine scheme.

\medskip\noindent
Assume $Y$ is affine. Since $f$ is quasi-compact we see that $X$
is quasi-compact. Thus we may choose an affine scheme $U$ and a surjective
\'etale morphism $g : U \to X$, see
Properties of Spaces,
Lemma \ref{spaces-properties-lemma-quasi-compact-affine-cover}.
Picture
$$
\xymatrix{
U \ar[r]_g \ar[rd]_{f \circ g} & X \ar[d]^f \\
& Y
}
$$
The morphism $g : U \to X$ is representable, separated
and quasi-compact because $X$ is quasi-separated. Hence the lemma
holds for $g$ (by the discussion above the lemma).
It also holds for $f \circ g : U \to Y$ (as this is a morphism
of affine schemes).

\medskip\noindent
In the situation described in the previous paragraph we will show by
induction on $n$ that $IH_n$: for any quasi-coherent sheaf $\mathcal{F}$
on $X$ the sheaves $R^if\mathcal{F}$
are quasi-coherent for $i \leq n$.
The case $n = 0$ follows from
Morphisms of Spaces, Lemma \ref{spaces-morphisms-lemma-pushforward}.
Assume $IH_n$. In the rest of the proof we show that $IH_{n + 1}$ holds.

\medskip\noindent
Let $\mathcal{H}$ be a quasi-coherent $\mathcal{O}_U$-module.
Consider the Leray spectral sequence
$$
E_2^{p, q} = R^pf_* R^qg_* \mathcal{H}
\Rightarrow
R^{p + q}(f \circ g)_*\mathcal{H}
$$
Cohomology on Sites, Lemma \ref{sites-cohomology-lemma-relative-Leray}.
As $R^qg_*\mathcal{H}$ is quasi-coherent by $IH_n$ all the sheaves
$R^pf_*R^qg_*\mathcal{H}$ are quasi-coherent for $p \leq n$.
The sheaves $R^{p + q}(f \circ g)_*\mathcal{H}$ are all
quasi-coherent (in fact zero for $p + q > 0$ but we do not need this).
Looking in degrees $\leq n + 1$ the only module which we do not
yet know is quasi-coherent is $E_2^{n + 1, 0} = R^{n + 1}f_*g_*\mathcal{H}$.
Moreover, the differentials
$d_r^{n + 1, 0} : E_r^{n + 1, 0} \to E_r^{n + 1 + r, 1 - r}$
are zero as the target is zero. Using that $\QCoh(\mathcal{O}_X)$
is a weak Serre subcategory of $\textit{Mod}(\mathcal{O}_X)$
(Properties of Spaces, Lemma
\ref{spaces-properties-lemma-properties-quasi-coherent}) it
follows that $R^{n + 1}f_*g_*\mathcal{H}$
is quasi-coherent (details omitted).

\medskip\noindent
Let $\mathcal{F}$ be a quasi-coherent $\mathcal{O}_X$-module.
Set $\mathcal{H} = g^*\mathcal{F}$. The adjunction mapping
$\mathcal{F} \to g_*g^*\mathcal{F} = g_*\mathcal{H}$ is injective
as $U \to X$ is surjective \'etale. Consider the exact sequence
$$
0 \to \mathcal{F} \to g_*\mathcal{H} \to \mathcal{G} \to 0
$$
where $\mathcal{G}$ is the cokernel of the first map and in particular
quasi-coherent. Applying the long exact cohomology sequence we obtain
$$
R^nf_*g_*\mathcal{H} \to
R^nf_*\mathcal{G} \to
R^{n + 1}f_*\mathcal{F} \to
R^{n + 1}f_*g_*\mathcal{H} \to
R^{n + 1}f_*\mathcal{G}
$$
The cokernel of the first arrow is quasi-coherent and
we have seen above that $R^{n + 1}f_*g_*\mathcal{H}$ is quasi-coherent.
Thus $R^{n + 1}f_*\mathcal{F}$ has a $2$-step filtration
where the first step is quasi-coherent and the second a submodule of
a quasi-coherent sheaf. Since $\mathcal{F}$ is an arbitrary quasi-coherent
$\mathcal{O}_X$-module, this result also holds for $\mathcal{G}$.
Thus we can choose an exact sequence
$0 \to \mathcal{A} \to R^{n + 1}f_*\mathcal{G} \to \mathcal{B}$
with $\mathcal{A}$, $\mathcal{B}$ quasi-coherent $\mathcal{O}_Y$-modules.
Then the kernel $\mathcal{K}$ of
$R^{n + 1}f_*g_*\mathcal{H} \to R^{n + 1}f_*\mathcal{G}
\to \mathcal{B}$ is quasi-coherent, whereupon we obtain a map
$\mathcal{K} \to \mathcal{A}$ whose kernel $\mathcal{K}'$ is
quasi-coherent too. Hence $R^{n + 1}f_*\mathcal{F}$ sits in an exact
sequence
$$
R^nf_*g_*\mathcal{H} \to
R^nf_*\mathcal{G} \to
R^{n + 1}f_*\mathcal{F} \to \mathcal{K}' \to 0
$$
with all modules quasi-coherent except for possibly $R^{n + 1}f_*\mathcal{F}$.
We conclude that $R^{n + 1}f_*\mathcal{F}$ is quasi-coherent, i.e.,
$IH_{n + 1}$ holds as desired.
\end{proof}

\begin{lemma}
\label{lemma-quasi-coherence-higher-direct-images-application}
Let $S$ be a scheme. Let $f : X \to Y$ be a quasi-separated and quasi-compact
morphism of algebraic spaces over $S$. For any quasi-coherent
$\mathcal{O}_X$-module $\mathcal{F}$ and any affine object $V$ of
$Y_\etale$ we have
$$
H^q(V \times_Y X, \mathcal{F}) = H^0(V, R^qf_*\mathcal{F})
$$
for all $q \in \mathbf{Z}$.
\end{lemma}

\begin{proof}
Since formation of $Rf_*$ commutes with \'etale localization
(Properties of Spaces, Lemma
\ref{spaces-properties-lemma-pushforward-etale-base-change-modules})
we may replace $Y$ by $V$ and assume $Y = V$ is affine.
Consider the Leray spectral sequence
$E_2^{p, q} = H^p(Y, R^qf_*\mathcal{F})$
converging to $H^{p + q}(X, \mathcal{F})$, see
Cohomology on Sites, Lemma \ref{sites-cohomology-lemma-Leray}.
By Lemma \ref{lemma-higher-direct-image}
we see that the sheaves $R^qf_*\mathcal{F}$ are quasi-coherent. By
Cohomology of Schemes, Lemma
\ref{coherent-lemma-quasi-coherent-affine-cohomology-zero}
we see that $E_2^{p, q} = 0$ when $p > 0$.
Hence the spectral sequence degenerates at $E_2$ and we win.
\end{proof}




\section{Finite morphisms}
\label{section-finite-morphisms}

\noindent
Here are some results which hold for all abelian sheaves
(in particular also quasi-coherent modules).
We {\bf warn} the reader that these lemmas do not hold for
finite morphisms of schemes and the Zariski topology.

\begin{lemma}
\label{lemma-finite-higher-direct-image-zero}
Let $S$ be a scheme. Let $f : X \to Y$ be an integral (for example finite)
morphism of algebraic spaces. Then
$f_* : \textit{Ab}(X_\etale) \to \textit{Ab}(Y_\etale)$
is an exact functor and $R^pf_* = 0$ for $p > 0$.
\end{lemma}

\begin{proof}
By Properties of Spaces, Lemma
\ref{spaces-properties-lemma-pushforward-etale-base-change}
we may compute the higher direct images on an \'etale cover of $Y$.
Hence we may assume $Y$ is a scheme. This implies that
$X$ is a scheme (Morphisms of Spaces, Lemma
\ref{spaces-morphisms-lemma-integral-local}).
In this case we may apply
\'Etale Cohomology, Lemma \ref{etale-cohomology-lemma-what-integral}.
For the finite case the reader may wish to consult the less technical
\'Etale Cohomology, Proposition
\ref{etale-cohomology-proposition-finite-higher-direct-image-zero}.
\end{proof}

\begin{lemma}
\label{lemma-stalk-push-finite}
Let $S$ be a scheme. Let $f : X \to Y$ be a finite morphism of algebraic
spaces over $S$. Let $\overline{y}$ be a geometric point of $Y$ with
lifts $\overline{x}_1, \ldots, \overline{x}_n$ in $X$. Then
$$
(f_*\mathcal{F})_{\overline{y}} =
\bigoplus\nolimits_{i = 1, \ldots, n}
\mathcal{F}_{\overline{x}_i}
$$
for any sheaf $\mathcal{F}$ on $X_\etale$.
\end{lemma}

\begin{proof}
Choose an \'etale neighbourhood $(V, \overline{v})$ of $\overline{y}$.
Then the stalk $(f_*\mathcal{F})_{\overline{y}}$
is the stalk of $f_*\mathcal{F}|_V$ at $\overline{v}$.
By Properties of Spaces,
Lemma \ref{spaces-properties-lemma-pushforward-etale-base-change}
we may replace $Y$ by $V$ and $X$ by $X \times_Y V$.
Then $Z \to X$ is a finite morphism of schemes and the result is
\'Etale Cohomology, Proposition
\ref{etale-cohomology-proposition-finite-higher-direct-image-zero}.
\end{proof}

\begin{lemma}
\label{lemma-finite-rings}
Let $S$ be a scheme. Let $\pi : X \to Y$ be a finite morphism of algebraic
spaces over $S$. Let $\mathcal{A}$ be a sheaf of rings on $X_\etale$.
Let $\mathcal{B}$ be a sheaf of rings on $Y_\etale$.
Let $\varphi : \mathcal{B} \to \pi_*\mathcal{A}$
be a homomorphism of sheaves of rings so that we obtain a
morphism of ringed topoi
$$
f = (\pi, \varphi) :
(\Sh(X_\etale), \mathcal{A})
\longrightarrow
(\Sh(Y_\etale), \mathcal{B}).
$$
For a sheaf of $\mathcal{A}$-modules $\mathcal{F}$ and a
sheaf of $\mathcal{B}$-modules $\mathcal{G}$ the canonical map
$$
\mathcal{G} \otimes_\mathcal{B} f_*\mathcal{F}
\longrightarrow
f_*(f^*\mathcal{G} \otimes_\mathcal{A} \mathcal{F}).
$$
is an isomorphism.
\end{lemma}

\begin{proof}
The map is the map adjoint to the map
$$
f^*\mathcal{G} \otimes_\mathcal{A}
f^* f_*\mathcal{F} =
f^*(\mathcal{G} \otimes_\mathcal{B} f_*\mathcal{F})
\longrightarrow
f^*\mathcal{G} \otimes_\mathcal{A} \mathcal{F}
$$
coming from $\text{id} : f^*\mathcal{G} \to f^*\mathcal{G}$
and the adjunction map $f^* f_*\mathcal{F} \to \mathcal{F}$.
To see this map is an isomorphism, we may check on stalks
(Properties of Spaces, Theorem
\ref{spaces-properties-theorem-exactness-stalks}).
Let $\overline{y}$ be a geometric point of $Y$ and
let $\overline{x}_1, \ldots, \overline{x}_n$ be the geometric
points of $X$ lying over $\overline{y}$.
Working out what our maps does on stalks, we see that we
have to show
$$
\mathcal{G}_{\overline{y}}
\otimes_{\mathcal{B}_{\overline{y}}}
\left(
\bigoplus\nolimits_{i = 1, \ldots, n} \mathcal{F}_{\overline{x}_i}
\right) =
\bigoplus\nolimits_{i = 1, \ldots, n}
(\mathcal{G}_{\overline{y}}
\otimes_{\mathcal{B}_{\overline{x}}}
\mathcal{A}_{\overline{x}_i}) \otimes_{\mathcal{A}_{\overline{x}_i}}
\mathcal{F}_{\overline{x}_i}
$$
which holds true. Here we have used that
taking tensor products commutes with taking stalks, the
behaviour of stalks under pullback
Properties of Spaces, Lemma \ref{spaces-properties-lemma-stalk-pullback}, and
the behaviour of stalks under pushforward along a closed immersion
Lemma \ref{lemma-stalk-push-finite}.
\end{proof}

\noindent
We end this section with an insanely general projection formula
for finite morphisms.

\begin{lemma}
\label{lemma-projection-formula-finite}
With $S$, $X$, $Y$, $\pi$, $\mathcal{A}$, $\mathcal{B}$, $\varphi$, and $f$
as in Lemma \ref{lemma-finite-rings} we have
$$
K \otimes_\mathcal{B}^\mathbf{L} Rf_*M =
Rf_*(Lf^*K \otimes_\mathcal{A}^\mathbf{L} M)
$$
in $D(\mathcal{B})$ for any $K \in D(\mathcal{B})$ and
$M \in D(\mathcal{A})$.
\end{lemma}

\begin{proof}
Since $f_*$ is exact (Lemma \ref{lemma-finite-higher-direct-image-zero})
the functor $Rf_*$ is computed by applying $f_*$ to any representative complex.
Choose a complex $\mathcal{K}^\bullet$ of $\mathcal{B}$-modules
representing $K$ which is K-flat and such that $f^*\mathcal{K}^\bullet$
is K-flat, see Cohomology on Sites, Lemma
\ref{sites-cohomology-lemma-pullback-K-flat}.
Choose any complex $\mathcal{M}^\bullet$ of $\mathcal{A}$-modules
representing $M$. Then
we have to show
$$
\text{Tot}(\mathcal{K}^\bullet \otimes_\mathcal{B} f_*\mathcal{M}^\bullet)
=
f_*\text{Tot}(f^*\mathcal{K}^\bullet \otimes_\mathcal{A} \mathcal{M}^\bullet)
$$
because by our choices these complexes represent the right and left hand
side of the formula in the lemma.
Since $f_*$ commutes with direct sums
(for example by the description of the stalks in
Lemma \ref{lemma-stalk-push-finite}),
this reduces to the equalities
$$
\mathcal{K}^n \otimes_\mathcal{B} f_*\mathcal{M}^m
=
f_*(f^*\mathcal{K}^n \otimes_\mathcal{A} \mathcal{M}^m)
$$
which are true by Lemma \ref{lemma-finite-rings}.
\end{proof}









\section{Colimits and cohomology}
\label{section-colimits}

\noindent
The following lemma in particular applies to diagrams of quasi-coherent
sheaves.

\begin{lemma}
\label{lemma-colimits}
Let $S$ be a scheme. Let $X$ be an algebraic space over $S$.
If $X$ is quasi-compact and quasi-separated, then
$$
\colim_i H^p(X, \mathcal{F}_i)
\longrightarrow
H^p(X, \colim_i \mathcal{F}_i)
$$
is an isomorphism
for every filtered diagram of abelian sheaves on $X_\etale$.
\end{lemma}

\begin{proof}
This follows from
Cohomology on Sites, Lemma
\ref{sites-cohomology-lemma-colim-works-over-collection}.
Namely, let $\mathcal{B} \subset \Ob(X_{spaces, \etale})$
be the set of quasi-compact and quasi-separated spaces \'etale over $X$.
Note that if $U \in \mathcal{B}$ then, because $U$ is quasi-compact,
the collection of finite coverings $\{U_i \to U\}$ with $U_i \in \mathcal{B}$
is cofinal in the set of coverings of $U$ in $X_{spaces, \etale}$. By
Morphisms of Spaces, Lemma
\ref{spaces-morphisms-lemma-quasi-compact-quasi-separated-permanence}
the set $\mathcal{B}$ satisfies all the assumptions of
Cohomology on Sites, Lemma
\ref{sites-cohomology-lemma-colim-works-over-collection}.
Since $X \in \mathcal{B}$ we win.
\end{proof}

\begin{lemma}
\label{lemma-colimit-cohomology}
\begin{slogan}
Higher direct images of qcqs morphisms commute with filtered colimits
of sheaves.
\end{slogan}
Let $S$ be a scheme. Let $f : X \to Y$ be a quasi-compact and quasi-separated
morphism of algebraic spaces over $S$. Let $\mathcal{F} = \colim \mathcal{F}_i$
be a filtered colimit of abelian sheaves on $X_\etale$.
Then for any $p \geq 0$ we have
$$
R^pf_*\mathcal{F} = \colim R^pf_*\mathcal{F}_i.
$$
\end{lemma}

\begin{proof}
Recall that $R^pf_*\mathcal{F}$ is the sheaf on $Y_{spaces, \etale}$
associated to $V \mapsto H^p(V \times_Y X, \mathcal{F})$, see
Cohomology on Sites, Lemma \ref{sites-cohomology-lemma-higher-direct-images}
and Properties of Spaces, Lemma
\ref{spaces-properties-lemma-functoriality-etale-site}.
Recall that the colimit is the sheaf associated to the presheaf colimit.
Hence we can apply Lemma \ref{lemma-colimits}
to $H^p(V \times_Y X, -)$ where $V$ is affine to conclude (because
when $V$ is affine, then $V \times_Y X$ is quasi-compact and quasi-separated).
Strictly speaking this also uses Properties of Spaces,
Lemma \ref{spaces-properties-lemma-alternative} to see that there exist
enough affine objects.
\end{proof}

\noindent
The following lemma tells us that finitely presented modules behave
as expected in quasi-compact and quasi-separated algebraic spaces.

\begin{lemma}
\label{lemma-finite-presentation-quasi-compact-colimit}
Let $S$ be a scheme. Let $X$ be a quasi-compact and quasi-separated
algebraic space over $S$. Let $I$ be a directed set and
let $(\mathcal{F}_i, \varphi_{ii'})$ be a system over $I$
of quasi-coherent $\mathcal{O}_X$-modules. Let $\mathcal{G}$ be an
$\mathcal{O}_X$-module of finite presentation. Then we have
$$
\colim_i \Hom_X(\mathcal{G}, \mathcal{F}_i)
=
\Hom_X(\mathcal{G}, \colim_i \mathcal{F}_i).
$$
\end{lemma}

\begin{proof}
Choose an affine scheme $U$ and a surjective \'etale morphism
$U \to X$. Set $R = U \times_X U$. Note that $R$ is a quasi-compact
(as $X$ is quasi-separated and $U$ quasi-compact) and separated (as
$U$ is separated) scheme. Hence we have
$$
\colim_i \Hom_U(\mathcal{G}|_U, \mathcal{F}_i|_U)
=
\Hom_U(\mathcal{G}|_U, \colim_i \mathcal{F}_i|_U).
$$
by Modules, Lemma \ref{modules-lemma-finite-presentation-quasi-compact-colimit}
(and the material on restriction to
schemes \'etale over $X$, see
Properties of Spaces, Sections \ref{spaces-properties-section-quasi-coherent}
and \ref{spaces-properties-section-properties-modules}). Similarly for $R$.
Since $\QCoh(\mathcal{O}_X) = \QCoh(U, R, s, t, c)$ (see
Properties of Spaces, Proposition
\ref{spaces-properties-proposition-quasi-coherent})
the result follows formally.
\end{proof}




\section{The alternating {\v C}ech complex}
\label{section-alternating-cech}

\noindent
Let $S$ be a scheme. Let $f : U \to X$ be an \'etale morphism of algebraic
spaces over $S$. The functor
$$
j : U_{spaces, \etale} \longrightarrow X_{spaces, \etale},\quad
V/U \longmapsto V/X
$$
induces an equivalence of $U_{spaces, \etale}$ with the localization
$X_{spaces, \etale}/U$, see
Properties of Spaces, Section \ref{spaces-properties-section-localize}.
Hence there exist functors
$$
f_! : \textit{Ab}(U_\etale) \longrightarrow
\textit{Ab}(X_\etale),\quad
f_! : \textit{Mod}(\mathcal{O}_U) \longrightarrow \textit{Mod}(\mathcal{O}_X),
$$
which are left adjoint to
$$
f^{-1} : \textit{Ab}(X_\etale) \longrightarrow
\textit{Ab}(U_\etale),\quad
f^* : \textit{Mod}(\mathcal{O}_X) \longrightarrow \textit{Mod}(\mathcal{O}_U)
$$
see
Modules on Sites, Section \ref{sites-modules-section-localize}.
Warning: This functor, a priori, has
nothing to do with cohomology with compact supports!
We dubbed this functor ``extension by zero'' in the reference above.
Note that the two versions of $f_!$ agree as $f^* = f^{-1}$ for
sheaves of $\mathcal{O}_X$-modules.

\medskip\noindent
As we are going to use this construction below let us recall some of its
properties. Given an abelian sheaf $\mathcal{G}$ on $U_\etale$
the sheaf $f_!$ is the sheafification of the presheaf
$$
V/X \longmapsto
f_!\mathcal{G}(V) =
\bigoplus\nolimits_{\varphi \in \Mor_X(V, U)}
\mathcal{G}(V \xrightarrow{\varphi} U),
$$
see
Modules on Sites, Lemma \ref{sites-modules-lemma-extension-by-zero}.
Moreover, if $\mathcal{G}$ is an $\mathcal{O}_U$-module, then $f_!\mathcal{G}$
is the sheafification of the exact same presheaf of abelian groups which
is endowed with an $\mathcal{O}_X$-module structure in an obvious way
(see loc.\ cit.). Let $\overline{x} : \Spec(k) \to X$
be a geometric point. Then there is a canonical identification
$$
(f_!\mathcal{G})_{\overline{x}} =
\bigoplus\nolimits_{\overline{u}} \mathcal{G}_{\overline{u}}
$$
where the sum is over all $\overline{u} : \Spec(k) \to U$ such that
$f \circ \overline{u} = \overline{x}$, see
Modules on Sites, Lemma \ref{sites-modules-lemma-stalk-j-shriek}
and
Properties of Spaces, Lemma
\ref{spaces-properties-lemma-points-small-etale-site}.
In the following we are going to study the sheaf $f_!\underline{\mathbf{Z}}$.
Here $\underline{\mathbf{Z}}$ denotes the constant sheaf on
$X_\etale$ or $U_\etale$.

\begin{lemma}
\label{lemma-product-is-tensor-product}
Let $S$ be a scheme. Let $f_i : U_i \to X$ be \'etale morphisms
of algebraic spaces over $S$. Then there are isomorphisms
$$
f_{1, !}\underline{\mathbf{Z}} \otimes_{\mathbf{Z}}
f_{2, !}\underline{\mathbf{Z}}
\longrightarrow
f_{12, !}\underline{\mathbf{Z}}
$$
where $f_{12} : U_1 \times_X U_2 \to X$ is the structure morphism
and
$$
(f_1 \amalg f_2)_! \underline{\mathbf{Z}}
\longrightarrow
f_{1, !}\underline{\mathbf{Z}} \oplus
f_{2, !}\underline{\mathbf{Z}}
$$
\end{lemma}

\begin{proof}
Once we have defined the map it will be an isomorphism by our description
of stalks above. To define the map it suffices to work on the level of
presheaves. Thus we have to define a map
$$
\left(\bigoplus\nolimits_{\varphi_1 \in \Mor_X(V, U_1)} \mathbf{Z}\right)
\otimes_{\mathbf{Z}}
\left(\bigoplus\nolimits_{\varphi_2 \in \Mor_X(V, U_2)} \mathbf{Z}\right)
\longrightarrow
\bigoplus\nolimits_{\varphi \in \Mor_X(V, U_1 \times_X U_2)}
\mathbf{Z}
$$
We map the element $1_{\varphi_1} \otimes 1_{\varphi_2}$ to the element
$1_{\varphi_1 \times \varphi_2}$ with obvious notation. We omit the proof
of the second equality.
\end{proof}

\noindent
Another important feature is the trace map
$$
\text{Tr}_f : f_!\underline{\mathbf{Z}} \longrightarrow \underline{\mathbf{Z}}.
$$
The trace map is adjoint to the
map $\mathbf{Z} \to f^{-1}\underline{\mathbf{Z}}$ (which is an isomorphism).
If $\overline{x}$ is above, then $\text{Tr}_f$ on stalks at $\overline{x}$
is the map
$$
(\text{Tr}_f)_{\overline{x}} :
(f_!\underline{\mathbf{Z}})_{\overline{x}} =
\bigoplus\nolimits_{\overline{u}} \mathbf{Z}
\longrightarrow
\mathbf{Z} = \underline{\mathbf{Z}}_{\overline{x}}
$$
which sums the given integers. This is true because it is adjoint to the map
$1 : \mathbf{Z} \to f^{-1}\underline{\mathbf{Z}}$. In particular, if
$f$ is surjective as well as \'etale then $\text{Tr}_f$ is surjective.

\medskip\noindent
Assume that $f : U \to X$ is a surjective \'etale
morphism of algebraic spaces. Consider the {\it Koszul complex}
associated to the trace map we discussed above
$$
\ldots \to \wedge^3f_!\underline{\mathbf{Z}} \to
\wedge^2f_!\underline{\mathbf{Z}} \to f_!\underline{\mathbf{Z}} \to
\underline{\mathbf{Z}} \to 0
$$
Here the exterior powers are over the sheaf of rings $\underline{\mathbf{Z}}$.
The maps are defined by the rule
$$
e_1 \wedge \ldots \wedge e_n \longmapsto
\sum\nolimits_{i = 1, \ldots, n} (-1)^{i + 1}
\text{Tr}_f(e_i)
e_1 \wedge \ldots \wedge \widehat{e_i} \wedge \ldots \wedge e_n
$$
where $e_1, \ldots, e_n$ are local sections of $f_!\underline{\mathbf{Z}}$.
Let $\overline{x}$ be a geometric point of $X$ and set
$M_{\overline{x}} = (f_!\underline{\mathbf{Z}})_{\overline{x}} =
\bigoplus_{\overline{u}} \mathbf{Z}$. Then the stalk of the complex above at
$\overline{x}$ is the complex
$$
\ldots \to \wedge^3 M_{\overline{x}} \to \wedge^2 M_{\overline{x}}
\to M_{\overline{x}} \to \mathbf{Z} \to 0
$$
which is exact because $M_{\overline{x}} \to \mathbf{Z}$ is surjective, see
More on Algebra, Lemma \ref{more-algebra-lemma-homotopy-koszul-abstract}.
Hence if we let $K^\bullet = K^\bullet(f)$ be the complex with
$K^i = \wedge^{i + 1}f_!\underline{\mathbf{Z}}$, then we obtain a
quasi-isomorphism
\begin{equation}
\label{equation-quasi-isomorphism}
K^\bullet \longrightarrow \underline{\mathbf{Z}}[0]
\end{equation}
We use the complex $K^\bullet$ to define what we call
the alternating {\v C}ech complex associated to $f : U \to X$.

\begin{definition}
\label{definition-alternating-cech-complex}
Let $S$ be a scheme. Let $f : U \to X$ be a surjective \'etale morphism
of algebraic spaces over $S$. Let $\mathcal{F}$ be an object of
$\textit{Ab}(X_\etale)$. The
{\it alternating {\v C}ech complex}\footnote{This may be nonstandard notation}
$\check{\mathcal{C}}^\bullet_{alt}(f, \mathcal{F})$
associated to $\mathcal{F}$ and $f$ is the complex
$$
\Hom(K^0, \mathcal{F}) \to \Hom(K^1, \mathcal{F}) \to
\Hom(K^2, \mathcal{F}) \to \ldots
$$
with Hom groups computed in $\textit{Ab}(X_\etale)$.
\end{definition}

\noindent
The reader may verify that if $U = \coprod U_i$ and $f|_{U_i} : U_i \to X$
is the open immersion of a subspace, then
$\check{\mathcal{C}}_{alt}^\bullet(f, \mathcal{F})$ agrees with the complex
introduced in
Cohomology, Section \ref{cohomology-section-alternating-cech}
for the Zariski covering $X = \bigcup U_i$ and the restriction
of $\mathcal{F}$ to the Zariski site of $X$. What is more important
however, is to relate the cohomology of the alternating
{\v C}ech complex to the cohomology.

\begin{lemma}
\label{lemma-alternating-cech-to-cohomology}
Let $S$ be a scheme. Let $f : U \to X$ be a surjective \'etale morphism
of algebraic spaces over $S$. Let $\mathcal{F}$ be an object of
$\textit{Ab}(X_\etale)$. There exists a canonical map
$$
\check{\mathcal{C}}^\bullet_{alt}(f, \mathcal{F})
\longrightarrow
R\Gamma(X, \mathcal{F})
$$
in $D(\textit{Ab})$. Moreover, there is a spectral sequence with $E_1$-page
$$
E_1^{p, q} =
\Ext_{\textit{Ab}(X_\etale)}^q(K^p, \mathcal{F})
$$
converging to $H^{p + q}(X, \mathcal{F})$ where
$K^p = \wedge^{p + 1}f_!\underline{\mathbf{Z}}$.
\end{lemma}

\begin{proof}
Recall that we have the quasi-isomorphism
$K^\bullet \to \underline{\mathbf{Z}}[0]$, see
(\ref{equation-quasi-isomorphism}).
Choose an injective resolution $\mathcal{F} \to \mathcal{I}^\bullet$
in $\textit{Ab}(X_\etale)$. Consider the double complex
$A^{\bullet, \bullet}$ with terms
$$
A^{p, q} = \Hom(K^p, \mathcal{I}^q)
$$
where the differential $d_1^{p, q} : A^{p, q} \to A^{p + 1, q}$
is the one coming from the differential $K^{p + 1} \to K^p$
and the differential $d_2^{p, q} : A^{p, q} \to A^{p, q + 1}$ is the
one coming from the differential
$\mathcal{I}^q \to \mathcal{I}^{q + 1}$.
Denote $sA^\bullet$ the total complex associated to
the double complex $A^{\bullet, \bullet}$.
We will use the two spectral
sequences $({}'E_r, {}'d_r)$ and $({}''E_r, {}''d_r)$
associated to this double complex, see
Homology, Section \ref{homology-section-double-complex}.

\medskip\noindent
Because $K^\bullet$ is a resolution of $\underline{\mathbf{Z}}$
we see that the complexes
$$
A^{\bullet, q} :
\Hom(K^0, \mathcal{I}^q) \to
\Hom(K^1, \mathcal{I}^q) \to
\Hom(K^2, \mathcal{I}^q) \to \ldots
$$
are acyclic in positive degrees and have $H^0$ equal to
$\Gamma(X, \mathcal{I}^q)$. Hence by
Homology, Lemma \ref{homology-lemma-double-complex-gives-resolution}
and its proof the spectral sequence $({}''E_r, {}''d_r)$ degenerates,
and the natural map
$$
\mathcal{I}^\bullet(X) \longrightarrow sA^\bullet
$$
is a quasi-isomorphism of complexes of abelian groups. In particular
we conclude that $H^n(sA^\bullet) = H^n(X, \mathcal{F})$.

\medskip\noindent
The map $\check{\mathcal{C}}^\bullet_{alt}(f, \mathcal{F}) \to
R\Gamma(X, \mathcal{F})$ of the lemma is the composition of
$\check{\mathcal{C}}^\bullet_{alt}(f, \mathcal{F}) \to SA^\bullet$
with the inverse of the displayed quasi-isomorphism.

\medskip\noindent
Finally, consider the spectral sequence $({}'E_r, {}'d_r)$.
We have
$$
E_1^{p, q} = q\text{th cohomology of }
\Hom(K^p, \mathcal{I}^0) \to
\Hom(K^p, \mathcal{I}^1) \to
\Hom(K^p, \mathcal{I}^2) \to \ldots
$$
This proves the lemma.
\end{proof}

\noindent
It follows from the lemma that it is important to understand the
ext groups $\Ext_{\textit{Ab}(X_\etale)}(K^p, \mathcal{F})$,
i.e., the right derived functors of
$\mathcal{F} \mapsto \Hom(K^p, \mathcal{F})$.

\begin{lemma}
\label{lemma-compute}
Let $S$ be a scheme. Let $f : U \to X$ be a surjective, \'etale, and separated
morphism of algebraic spaces over $S$. For $p \geq 0$ set
$$
W_p = U \times_X \ldots \times_X U \setminus \text{all diagonals}
$$
where the fibre product has $p + 1$ factors.
There is a free action of $S_{p + 1}$ on $W_p$ over $X$ and
$$
\Hom(K^p, \mathcal{F}) = S_{p + 1}\text{-anti-invariant elements of }
\mathcal{F}(W_p)
$$
functorially in $\mathcal{F}$ where
$K^p = \wedge^{p + 1}f_!\underline{\mathbf{Z}}$.
\end{lemma}

\begin{proof}
Because $U \to X$ is separated the diagonal $U \to U \times_X U$ is a
closed immersion. Since $U \to X$ is \'etale the diagonal
$U \to U \times_X U$ is an open immersion, see
Morphisms of Spaces, Lemmas
\ref{spaces-morphisms-lemma-etale-unramified} and
\ref{spaces-morphisms-lemma-diagonal-unramified-morphism}.
Hence $W_p$ is an open and closed subspace of
$U^{p + 1} = U \times_X \ldots \times_X U$. The action of $S_{p + 1}$
on $W_p$ is free as we've thrown out the fixed points of the action.
By
Lemma \ref{lemma-product-is-tensor-product}
we see that
$$
(f_!\underline{\mathbf{Z}})^{\otimes p + 1} =
f^{p + 1}_!\underline{\mathbf{Z}} = (W_p \to X)_!\underline{\mathbf{Z}}
\oplus Rest
$$
where $f^{p + 1} : U^{p + 1} \to X$ is the structure morphism.
Looking at stalks over a geometric point $\overline{x}$ of $X$
we see that
$$
\left(
\bigoplus\nolimits_{\overline{u} \mapsto \overline{x}} \mathbf{Z}
\right)^{\otimes p + 1}
\longrightarrow
(W_p \to X)_!\underline{\mathbf{Z}}_{\overline{x}}
$$
is the quotient whose kernel is generated by all tensors
$1_{\overline{u}_0} \otimes \ldots \otimes 1_{\overline{u}_p}$
where $\overline{u}_i = \overline{u}_j$ for some $i \not = j$.
Thus the quotient map
$$
(f_!\underline{\mathbf{Z}})^{\otimes p + 1}
\longrightarrow
\wedge^{p + 1}f_!\underline{\mathbf{Z}}
$$
factors through $(W_p \to X)_!\underline{\mathbf{Z}}$, i.e., we get
$$
(f_!\underline{\mathbf{Z}})^{\otimes p + 1}
\longrightarrow
(W_p \to X)_!\underline{\mathbf{Z}}
\longrightarrow
\wedge^{p + 1}f_!\underline{\mathbf{Z}}
$$
This already proves that $\Hom(K^p, \mathcal{F})$ is (functorially) a
subgroup of
$$
\Hom((W_p \to X)_!\underline{\mathbf{Z}}, \mathcal{F}) = \mathcal{F}(W_p)
$$
To identify it with the $S_{p + 1}$-anti-invariants we have to prove that
the surjection $(W_p \to X)_!\underline{\mathbf{Z}}
\to \wedge^{p + 1}f_!\underline{\mathbf{Z}}$ is the maximal
$S_{p + 1}$-anti-invariant quotient. In other words, we have to show that
$\wedge^{p + 1}f_!\underline{\mathbf{Z}}$ is the quotient of
$(W_p \to X)_!\underline{\mathbf{Z}}$ by the subsheaf generated by
the local sections $s - \text{sign}(\sigma)\sigma(s)$ where $s$ is
a local section of $(W_p \to X)_!\underline{\mathbf{Z}}$.
This can be checked on the stalks, where it is clear.
\end{proof}

\begin{lemma}
\label{lemma-twist}
Let $S$ be a scheme. Let $W$ be an algebraic space over $S$.
Let $G$ be a finite group acting freely on $W$.
Let $U = W/G$, see
Properties of Spaces, Lemma \ref{spaces-properties-lemma-quotient}.
Let $\chi : G \to \{+1, -1\}$ be a character.
Then there exists a rank 1 locally free sheaf of $\mathbf{Z}$-modules
$\underline{\mathbf{Z}}(\chi)$ on $U_\etale$ such that for every
abelian sheaf $\mathcal{F}$ on $U_\etale$ we have
$$
H^0(W, \mathcal{F}|_W)^\chi =
H^0(U, \mathcal{F} \otimes_{\mathbf{Z}} \underline{\mathbf{Z}}(\chi))
$$
\end{lemma}

\begin{proof}
The quotient morphism $q : W \to U$ is a $G$-torsor, i.e., there exists
a surjective \'etale morphism $U' \to U$ such that
$W \times_U U' = \coprod_{g \in G} U'$ as spaces with $G$-action over $U'$.
(Namely, $U' = W$ works.) Hence $q_*\underline{\mathbf{Z}}$ is a finite
locally free $\mathbf{Z}$-module with an action of $G$. For any
geometric point $\overline{u}$ of $U$, then we get $G$-equivariant
isomorphisms
$$
(q_*\underline{\mathbf{Z}})_{\overline{u}}
= \bigoplus\nolimits_{\overline{w} \mapsto \overline{u}} \mathbf{Z}
= \bigoplus\nolimits_{g \in G} \mathbf{Z} = \mathbf{Z}[G]
$$
where the second $=$ uses a geometric point
$\overline{w}_0$ lying over $\overline{u}$ and
maps the summand corresponding to $g \in G$ to the summand
corresponding to $g(\overline{w}_0)$. We have
$$
H^0(W, \mathcal{F}|_W) =
H^0(U, \mathcal{F} \otimes_\mathbf{Z} q_*\underline{\mathbf{Z}})
$$
because
$q_*\mathcal{F}|_W = \mathcal{F} \otimes_\mathbf{Z} q_*\underline{\mathbf{Z}}$
as one can check by restricting to $U'$. Let
$$
\underline{\mathbf{Z}}(\chi) =
(q_*\underline{\mathbf{Z}})^\chi \subset
q_*\underline{\mathbf{Z}}
$$
be the subsheaf of sections that transform according to $\chi$. For
any geometric point $\overline{u}$ of $U$ we have
$$
\underline{\mathbf{Z}}(\chi)_{\overline{u}} =
\mathbf{Z} \cdot \sum\nolimits_g \chi(g) g
\subset
\mathbf{Z}[G] = (q_*\underline{\mathbf{Z}})_{\overline{u}}
$$
It follows that $\underline{\mathbf{Z}}(\chi)$ is locally free of
rank 1 (more precisely, this should be checked after restricting to $U'$).
Note that for any $\mathbf{Z}$-module $M$ the $\chi$-semi-invariants
of $M[G]$ are the elements of the form $m \cdot \sum\nolimits_g \chi(g) g$.
Thus we see that for any abelian sheaf $\mathcal{F}$ on $U$ we have
$$
\left(\mathcal{F} \otimes_\mathbf{Z} q_*\underline{\mathbf{Z}}\right)^\chi
=
\mathcal{F} \otimes_\mathbf{Z} \underline{\mathbf{Z}}(\chi)
$$
because we have equality at all stalks. The result of the lemma follows by
taking global sections.
\end{proof}

\noindent
Now we can put everything together and obtain the following
pleasing result.

\begin{lemma}
\label{lemma-alternating-spectral-sequence}
Let $S$ be a scheme. Let $f : U \to X$ be a surjective, \'etale, and
separated morphism of algebraic spaces over $S$. For $p \geq 0$ set
$$
W_p = U \times_X \ldots \times_X U \setminus \text{all diagonals}
$$
(with $p + 1$ factors) as in Lemma \ref{lemma-compute}.
Let $\chi_p : S_{p + 1} \to \{+1, -1\}$ be the sign character.
Let $U_p = W_p/S_{p + 1}$ and $\underline{\mathbf{Z}}(\chi_p)$ be as in
Lemma \ref{lemma-twist}.
Then the spectral sequence of
Lemma \ref{lemma-alternating-cech-to-cohomology}
has $E_1$-page
$$
E_1^{p, q} =
H^q(U_p, \mathcal{F}|_{U_p} \otimes_\mathbf{Z} \underline{\mathbf{Z}}(\chi_p))
$$
and converges to $H^{p + q}(X, \mathcal{F})$.
\end{lemma}

\begin{proof}
Note that since the action of $S_{p + 1}$ on $W_p$ is over $X$ we do
obtain a morphism $U_p \to X$. Since $W_p \to X$ is \'etale and since
$W_p \to U_p$ is surjective \'etale, it follows
that also $U_p \to X$ is \'etale, see
Morphisms of Spaces, Lemma \ref{spaces-morphisms-lemma-etale-local}.
Therefore an injective object of
$\textit{Ab}(X_\etale)$ restricts to an injective object of
$\textit{Ab}(U_{p, \etale})$, see
Cohomology on Sites, Lemma \ref{sites-cohomology-lemma-cohomology-of-open}.
Moreover, the functor
$\mathcal{G} \mapsto
\mathcal{G} \otimes_\mathbf{Z} \underline{\mathbf{Z}}(\chi_p))$
is an auto-equivalence of $\textit{Ab}(U_p)$, whence transforms injective
objects into injective objects and is exact (because
$\underline{\mathbf{Z}}(\chi_p)$ is an invertible
$\underline{\mathbf{Z}}$-module). Thus given an injective resolution
$\mathcal{F} \to \mathcal{I}^\bullet$ in $\textit{Ab}(X_\etale)$
the complex
$$
\Gamma(U_p,
\mathcal{I}^0|_{U_p} \otimes_\mathbf{Z} \underline{\mathbf{Z}}(\chi_p))
\to
\Gamma(U_p,
\mathcal{I}^1|_{U_p} \otimes_\mathbf{Z} \underline{\mathbf{Z}}(\chi_p))
\to
\Gamma(U_p,
\mathcal{I}^2|_{U_p} \otimes_\mathbf{Z} \underline{\mathbf{Z}}(\chi_p))
\to \ldots
$$
computes
$H^*(U_p,
\mathcal{F}|_{U_p} \otimes_\mathbf{Z} \underline{\mathbf{Z}}(\chi_p))$.
On the other hand, by
Lemma \ref{lemma-twist}
it is equal to the complex of $S_{p + 1}$-anti-invariants in
$$
\Gamma(W_p, \mathcal{I}^0) \to
\Gamma(W_p, \mathcal{I}^1) \to
\Gamma(W_p, \mathcal{I}^2) \to \ldots
$$
which by
Lemma \ref{lemma-compute}
is equal to the complex
$$
\Hom(K^p, \mathcal{I}^0) \to
\Hom(K^p, \mathcal{I}^1) \to
\Hom(K^p, \mathcal{I}^2) \to \ldots
$$
which computes
$\Ext^*_{\textit{Ab}(X_\etale)}(K^p, \mathcal{F})$.
Putting everything together we win.
\end{proof}





\section{Higher vanishing for quasi-coherent sheaves}
\label{section-higher-vanishing}

\noindent
In this section we show that given a quasi-compact and
quasi-separated algebraic space $X$ there exists an integer
$n = n(X)$ such that the cohomology of any quasi-coherent
sheaf on $X$ vanishes beyond degree $n$.

\begin{lemma}
\label{lemma-quasi-coherent-twist}
With $S$, $W$, $G$, $U$, $\chi$ as in
Lemma \ref{lemma-twist}.
If $\mathcal{F}$ is a quasi-coherent $\mathcal{O}_U$-module,
then so is $\mathcal{F} \otimes_{\mathbf{Z}} \underline{\mathbf{Z}}(\chi)$.
\end{lemma}

\begin{proof}
The $\mathcal{O}_U$-module structure is clear. To check that
$\mathcal{F} \otimes_{\mathbf{Z}} \underline{\mathbf{Z}}(\chi)$
is quasi-coherent it suffices to check \'etale locally.
Hence the lemma follows as $\underline{\mathbf{Z}}(\chi)$
is finite locally free as a $\underline{\mathbf{Z}}$-module.
\end{proof}

\noindent
The following proposition is interesting even if $X$ is a scheme.
It is the natural generalization of
Cohomology of Schemes, Lemma \ref{coherent-lemma-vanishing-nr-affines}.
Before we state it, observe that given an \'etale morphism
$f : U \to X$ from an affine scheme towards a quasi-separated algebraic
space $X$ the fibres of $f$ are universally bounded, in particular
there exists an integer $d$ such that the fibres of $|U| \to |X|$
all have size at most $d$; this is the implication
$(\eta) \Rightarrow (\delta)$ of
Decent Spaces, Lemma \ref{decent-spaces-lemma-bounded-fibres}.

\begin{proposition}
\label{proposition-vanishing}
Let $S$ be a scheme. Let $X$ be an algebraic space over $S$.
Assume $X$ is quasi-compact and separated.
Let $U$ be an affine scheme, and let
$f : U \to X$ be a surjective \'etale morphism.
Let $d$ be an upper bound for the size of the fibres of
$|U| \to |X|$. Then for any quasi-coherent $\mathcal{O}_X$-module $\mathcal{F}$
we have $H^q(X, \mathcal{F}) = 0$ for $q \geq d$.
\end{proposition}

\begin{proof}
We will use the spectral sequence of
Lemma \ref{lemma-alternating-spectral-sequence}.
The lemma applies since $f$ is separated as $U$ is separated, see
Morphisms of Spaces, Lemma
\ref{spaces-morphisms-lemma-compose-after-separated}.
Since $X$ is separated the scheme $U \times_X \ldots \times_X U$ is a closed
subscheme of
$U \times_{\Spec(\mathbf{Z})} \ldots \times_{\Spec(\mathbf{Z})} U$
hence is affine. Thus $W_p$ is affine. Hence $U_p = W_p/S_{p + 1}$ is an
affine scheme by
Groupoids, Proposition \ref{groupoids-proposition-finite-flat-equivalence}.
The discussion in
Section \ref{section-higher-direct-image}
shows that cohomology of quasi-coherent sheaves on $W_p$ (as an algebraic
space) agrees with the cohomology of the corresponding quasi-coherent
sheaf on the underlying affine scheme, hence vanishes in positive degrees by
Cohomology of Schemes, Lemma
\ref{coherent-lemma-quasi-coherent-affine-cohomology-zero}.
By
Lemma \ref{lemma-quasi-coherent-twist}
the sheaves
$\mathcal{F}|_{U_p} \otimes_\mathbf{Z} \underline{\mathbf{Z}}(\chi_p)$
are quasi-coherent. Hence
$H^q(W_p,
\mathcal{F}|_{U_p} \otimes_\mathbf{Z} \underline{\mathbf{Z}}(\chi_p))$
is zero when $q > 0$. By our definition of the integer $d$ we see that
$W_p = \emptyset$ for $p \geq d$. Hence also
$H^0(W_p,
\mathcal{F}|_{U_p} \otimes_\mathbf{Z} \underline{\mathbf{Z}}(\chi_p))$
is zero when $p \geq d$.
This proves the proposition.
\end{proof}

\noindent
In the following lemma we establish that a quasi-compact and
quasi-separated algebraic space has finite cohomological dimension
for quasi-coherent modules. We are explicit about the bound only because
we will use it later to prove a similar result for higher direct
images.

\begin{lemma}
\label{lemma-vanishing-quasi-separated}
Let $S$ be a scheme. Let $X$ be an algebraic space over $S$.
Assume $X$ is quasi-compact and quasi-separated.
Then we can choose
\begin{enumerate}
\item an affine scheme $U$,
\item a surjective \'etale morphism $f : U \to X$,
\item an integer $d$ bounding the degrees of the fibres of $U \to X$,
\item for every $p = 0, 1, \ldots, d$ a surjective \'etale morphism
$V_p \to U_p$ from an affine scheme $V_p$ where $U_p$ is as in
Lemma \ref{lemma-alternating-spectral-sequence}, and
\item an integer $d_p$ bounding the degree of the fibres of $V_p \to U_p$.
\end{enumerate}
Moreover, whenever we have (1) -- (5), then for any quasi-coherent
$\mathcal{O}_X$-module $\mathcal{F}$ we have $H^q(X, \mathcal{F}) = 0$ for
$q \geq \max(d_p + p)$.
\end{lemma}

\begin{proof}
Since $X$ is quasi-compact we can find a surjective \'etale morphism
$U \to X$ with $U$ affine, see
Properties of Spaces, Lemma
\ref{spaces-properties-lemma-quasi-compact-affine-cover}.
By
Decent Spaces, Lemma \ref{decent-spaces-lemma-bounded-fibres}
the fibres of $f$ are universally bounded, hence we can find $d$.
We have $U_p = W_p/S_{p + 1}$ and $W_p \subset U \times_X \ldots \times_X U$
is open and closed. Since $X$ is quasi-separated the schemes
$W_p$ are quasi-compact, hence $U_p$ is quasi-compact.
Since $U$ is separated, the schemes $W_p$ are separated, hence
$U_p$ is separated by (the absolute version of)
Spaces, Lemma \ref{spaces-lemma-quotient-finite-separated}.
By
Properties of Spaces, Lemma
\ref{spaces-properties-lemma-quasi-compact-affine-cover}
we can find the morphisms $V_p \to W_p$.
By
Decent Spaces, Lemma \ref{decent-spaces-lemma-bounded-fibres}
we can find the integers $d_p$.

\medskip\noindent
At this point the proof uses the spectral sequence
$$
E_1^{p, q} =
H^q(U_p, \mathcal{F}|_{U_p} \otimes_\mathbf{Z} \underline{\mathbf{Z}}(\chi_p))
\Rightarrow
H^{p + q}(X, \mathcal{F})
$$
see
Lemma \ref{lemma-alternating-spectral-sequence}.
By definition of the integer $d$ we see that $U_p = 0$ for $p \geq d$.
By Proposition \ref{proposition-vanishing}
and
Lemma \ref{lemma-quasi-coherent-twist}
we see that
$H^q(U_p,
\mathcal{F}|_{U_p} \otimes_\mathbf{Z} \underline{\mathbf{Z}}(\chi_p))$
is zero for $q \geq d_p$ for $p = 0, \ldots, d$.
Whence the lemma.
\end{proof}









\section{Vanishing for higher direct images}
\label{section-vanishing-higher-direct-images}

\noindent
We apply the results of
Section \ref{section-higher-vanishing}
to obtain vanishing of higher direct images of quasi-coherent sheaves
for quasi-compact and quasi-separated morphisms. This is useful because
it allows one to argue by descending induction on the cohomological degree
in certain situations.

\begin{lemma}
\label{lemma-vanishing-higher-direct-images}
Let $S$ be a scheme. Let $f : X \to Y$ be a
morphism of algebraic spaces over $S$.
Assume that
\begin{enumerate}
\item $f$ is quasi-compact and quasi-separated, and
\item $Y$ is quasi-compact.
\end{enumerate}
Then there exists an integer $n(X \to Y)$ such that
for any algebraic space $Y'$, any morphism $Y' \to Y$
and any quasi-coherent sheaf $\mathcal{F}'$ on $X' = Y' \times_Y X$
the higher direct images $R^if'_*\mathcal{F}'$ are zero for
$i \geq n(X \to Y)$.
\end{lemma}

\begin{proof}
Let $V \to Y$ be a surjective \'etale morphism where $V$ is an affine
scheme, see
Properties of Spaces, Lemma
\ref{spaces-properties-lemma-quasi-compact-affine-cover}.
Suppose we prove the result for the base change $f_V : V \times_Y X \to V$.
Then the result holds for $f$ with $n(X \to Y) = n(X_V \to V)$.
Namely, if $Y' \to Y$ and $\mathcal{F}'$ are as in the lemma, then
$R^if'_*\mathcal{F}'|_{V \times_Y Y'}$ is equal to
$R^if'_{V, *}\mathcal{F}'|_{X'_V}$ where
$f'_V : X'_V = V \times_Y Y' \times_Y X \to V \times_Y Y' = Y'_V$, see
Properties of Spaces,
Lemma \ref{spaces-properties-lemma-pushforward-etale-base-change-modules}.
Thus we may assume that $Y$ is an affine scheme.

\medskip\noindent
Moreover, to prove the vanishing for all $Y' \to Y$ and
$\mathcal{F}'$ it suffices to do so when $Y'$ is an affine scheme.
In this case, $R^if'_*\mathcal{F}'$ is quasi-coherent by
Lemma \ref{lemma-higher-direct-image}.
Hence it suffices to prove that $H^i(X', \mathcal{F}') = 0$, because
$H^i(X', \mathcal{F}') = H^0(Y', R^if'_*\mathcal{F}')$ by
Cohomology on Sites, Lemma \ref{sites-cohomology-lemma-apply-Leray}
and the vanishing of higher cohomology of quasi-coherent sheaves
on affine algebraic spaces
(Proposition \ref{proposition-vanishing}).

\medskip\noindent
Choose $U \to X$, $d$, $V_p \to U_p$ and $d_p$ as in
Lemma \ref{lemma-vanishing-quasi-separated}.
For any affine scheme $Y'$ and morphism $Y' \to Y$ denote
$X' = Y' \times_Y X$, $U' = Y' \times_Y U$, $V'_p = Y' \times_Y V_p$.
Then $U' \to X'$, $d' = d$, $V'_p \to U'_p$ and $d'_p = d$
is a collection of choices as in
Lemma \ref{lemma-vanishing-quasi-separated}
for the algebraic space $X'$ (details omitted).
Hence we see that $H^i(X', \mathcal{F}') = 0$ for $i \geq \max(p + d_p)$
and we win.
\end{proof}

\begin{lemma}
\label{lemma-affine-vanishing-higher-direct-images}
Let $S$ be a scheme. Let $f : X \to Y$ be an affine
morphism of algebraic spaces over $S$. Then
$R^if_*\mathcal{F} = 0$ for $i > 0$ and any quasi-coherent
$\mathcal{O}_X$-module $\mathcal{F}$.
\end{lemma}

\begin{proof}
Recall that an affine morphism of algebraic spaces is representable.
Hence this follows from (\ref{equation-representable-higher-direct-image}) and
Cohomology of Schemes, Lemma \ref{coherent-lemma-relative-affine-vanishing}.
\end{proof}

\begin{lemma}
\label{lemma-relative-affine-cohomology}
Let $S$ be a scheme. Let $f : X \to Y$ be an affine
morphism of algebraic spaces over $S$.
Let $\mathcal{F}$ be a quasi-coherent $\mathcal{O}_X$-module.
Then $H^i(X, \mathcal{F}) = H^i(S, f_*\mathcal{F})$ for all $i \geq 0$.
\end{lemma}

\begin{proof}
Follows from Lemma \ref{lemma-affine-vanishing-higher-direct-images}
and the Leray spectral sequence. See
Cohomology on Sites, Lemma \ref{sites-cohomology-lemma-apply-Leray}.
\end{proof}






\section{Cohomology with support in a closed subspace}
\label{section-cohomology-support}

\noindent
This section is the analogue of Cohomology, Section
\ref{cohomology-section-cohomology-support}
and \'Etale Cohomology, Section
\ref{etale-cohomology-section-cohomology-support}
for abelian sheaves on algebraic spaces.

\medskip\noindent
Let $S$ be a scheme.
Let $X$ be an algebraic space over $S$ and let $Z \subset X$ be a closed
subspace. Let $\mathcal{F}$ be an abelian sheaf on $X_\etale$. We let
$$
\Gamma_Z(X, \mathcal{F}) =
\{s \in \mathcal{F}(X) \mid \text{Supp}(s) \subset Z\}
$$
be the sections with support in $Z$
(Properties of Spaces, Definition \ref{spaces-properties-definition-support}).
This is a left exact functor which is not exact in general.
Hence we obtain a derived functor
$$
R\Gamma_Z(X, -) : D(X_\etale) \longrightarrow D(\textit{Ab})
$$
and cohomology groups with support in $Z$ defined by
$H^q_Z(X, \mathcal{F}) = R^q\Gamma_Z(X, \mathcal{F})$.

\medskip\noindent
Let $\mathcal{I}$ be an injective abelian sheaf on $X_\etale$. Let
$U \subset X$ be the open subspace which is the complement of $Z$.
Then the restriction map $\mathcal{I}(X) \to \mathcal{I}(U)$ is surjective
(Cohomology on Sites, Lemma
\ref{sites-cohomology-lemma-restriction-along-monomorphism-surjective})
with kernel $\Gamma_Z(X, \mathcal{I})$. It immediately follows that
for $K \in D(X_\etale)$ there is a distinguished triangle
$$
R\Gamma_Z(X, K) \to R\Gamma(X, K) \to R\Gamma(U, K) \to R\Gamma_Z(X, K)[1]
$$
in $D(\textit{Ab})$. As a consequence we obtain a long exact cohomology
sequence
$$
\ldots \to H^i_Z(X, K) \to H^i(X, K) \to H^i(U, K) \to
H^{i + 1}_Z(X, K) \to \ldots
$$
for any $K$ in $D(X_\etale)$.

\medskip\noindent
For an abelian sheaf $\mathcal{F}$ on $X_\etale$ we can consider
the {\it subsheaf of sections with support in $Z$}, denoted
$\mathcal{H}_Z(\mathcal{F})$, defined by the rule
$$
\mathcal{H}_Z(\mathcal{F})(U) =
\{s \in \mathcal{F}(U) \mid \text{Supp}(s) \subset U \times_X Z\}
$$
Here we use the support of a section from
Properties of Spaces, Definition \ref{spaces-properties-definition-support}.
Using the equivalence of Morphisms of Spaces, Lemma
\ref{spaces-morphisms-lemma-closed-immersion-push-pull}
we may view $\mathcal{H}_Z(\mathcal{F})$ as an abelian sheaf on
$Z_\etale$. Thus we obtain a functor
$$
\textit{Ab}(X_\etale) \longrightarrow \textit{Ab}(Z_\etale),\quad
\mathcal{F} \longmapsto \mathcal{H}_Z(\mathcal{F})
$$
which is left exact, but in general not exact.

\begin{lemma}
\label{lemma-sections-with-support-acyclic}
Let $S$ be a scheme.
Let $i : Z \to X$ be a closed immersion of algebraic spaces over $S$.
Let $\mathcal{I}$ be an injective abelian sheaf on $X_\etale$.
Then $\mathcal{H}_Z(\mathcal{I})$ is an injective abelian sheaf
on $Z_\etale$.
\end{lemma}

\begin{proof}
Observe that for any abelian sheaf $\mathcal{G}$ on $Z_\etale$
we have
$$
\Hom_Z(\mathcal{G}, \mathcal{H}_Z(\mathcal{F})) =
\Hom_X(i_*\mathcal{G}, \mathcal{F})
$$
because after all any section of $i_*\mathcal{G}$ has support in $Z$.
Since $i_*$ is exact (Lemma \ref{lemma-finite-higher-direct-image-zero})
and as $\mathcal{I}$ is injective on $X_\etale$ we conclude that
$\mathcal{H}_Z(\mathcal{I})$ is injective on $Z_\etale$.
\end{proof}

\noindent
Denote
$$
R\mathcal{H}_Z : D(X_\etale) \longrightarrow D(Z_\etale)
$$
the derived functor. We set
$\mathcal{H}^q_Z(\mathcal{F}) = R^q\mathcal{H}_Z(\mathcal{F})$ so that
$\mathcal{H}^0_Z(\mathcal{F}) = \mathcal{H}_Z(\mathcal{F})$.
By the lemma above we have a Grothendieck spectral sequence
$$
E_2^{p, q} = H^p(Z, \mathcal{H}^q_Z(\mathcal{F}))
\Rightarrow H^{p + q}_Z(X, \mathcal{F})
$$

\begin{lemma}
\label{lemma-cohomology-with-support-sheaf-on-support}
Let $S$ be a scheme. Let $i : Z \to X$ be a closed immersion of
algebraic spaces over $S$. Let $\mathcal{G}$ be an injective abelian
sheaf on $Z_\etale$. Then $\mathcal{H}^p_Z(i_*\mathcal{G}) = 0$ for $p > 0$.
\end{lemma}

\begin{proof}
This is true because the functor $i_*$ is exact
(Lemma \ref{lemma-finite-higher-direct-image-zero}) and transforms
injective abelian sheaves into injective abelian sheaves
(Cohomology on Sites, Lemma
\ref{sites-cohomology-lemma-pushforward-injective-flat}).
\end{proof}

\begin{lemma}
\label{lemma-etale-localization-sheaf-with-support}
Let $S$ be a scheme. Let $f : X \to Y$ be an \'etale morphism of
algebraic spaces over $S$. Let $Z \subset Y$ be a closed subspace
such that $f^{-1}(Z) \to Z$ is an isomorphism of algebraic spaces.
Let $\mathcal{F}$ be an abelian sheaf on $X$. Then
$$
\mathcal{H}^q_Z(\mathcal{F}) = \mathcal{H}^q_{f^{-1}(Z)}(f^{-1}\mathcal{F})
$$
as abelian sheaves on $Z = f^{-1}(Z)$ and we
have $H^q_Z(Y, \mathcal{F}) = H^q_{f^{-1}(Z)}(X, f^{-1}\mathcal{F})$.
\end{lemma}

\begin{proof}
Because $f$ is \'etale an injective resolution of $\mathcal{F}$
pulls back to an injective resolution of $f^{-1}\mathcal{F}$.
Hence it suffices to check the equality for $\mathcal{H}_Z(-)$
which follows from the definitions. The proof for cohomology with
supports is the same. Some details omitted.
\end{proof}

\noindent
Let $S$ be a scheme and let $X$ be an algebraic space over $S$.
Let $T \subset |X|$ be a closed subset.
We denote $D_T(X_\etale)$ the
strictly full saturated triangulated subcategory of $D(X_\etale)$
consisting of objects whose cohomology sheaves are supported on $T$.

\begin{lemma}
\label{lemma-complexes-with-support-on-closed}
Let $S$ be a scheme.
Let $i : Z \to X$ be a closed immersion of algebraic spaces over $S$.
The map $Ri_* = i_* : D(Z_\etale) \to D(X_\etale)$
induces an equivalence $D(Z_\etale) \to D_{|Z|}(X_\etale)$ with quasi-inverse
$$
i^{-1}|_{D_Z(X_\etale)} = R\mathcal{H}_Z|_{D_{|Z|}(X_\etale)}
$$
\end{lemma}

\begin{proof}
Recall that $i^{-1}$ and $i_*$ is an adjoint pair of
exact functors such that $i^{-1}i_*$ is isomorphic to the identify
functor on abelian sheaves. See
Properties of Spaces, Lemma
\ref{spaces-properties-lemma-stalk-pullback} and
Morphisms of Spaces, Lemma
\ref{spaces-morphisms-lemma-closed-immersion-push-pull}.
Thus $i_* : D(Z_\etale) \to D_Z(X_\etale)$ is fully faithful and
$i^{-1}$ determines
a left inverse. On the other hand, suppose that $K$ is an object of
$D_Z(X_\etale)$ and consider the adjunction map
$K \to i_*i^{-1}K$.
Using exactness of $i_*$ and $i^{-1}$
this induces the adjunction maps
$H^n(K) \to i_*i^{-1}H^n(K)$ on cohomology sheaves.
Since these cohomology
sheaves are supported on $Z$ we see these adjunction maps are isomorphisms
and we conclude that $D(Z_\etale) \to D_Z(X_\etale)$ is an equivalence.

\medskip\noindent
To finish the proof we have to show that $R\mathcal{H}_Z(K) = i^{-1}K$
if $K$ is an object of $D_Z(X_\etale)$. To do this we can use that
$K = i_*i^{-1}K$
as we've just proved this is the case. Then we
can choose a K-injective representative $\mathcal{I}^\bullet$ for
$i^{-1}K$.
Since $i_*$ is the right adjoint to the exact functor
$i^{-1}$, the
complex $i_*\mathcal{I}^\bullet$ is K-injective
(Derived Categories, Lemma \ref{derived-lemma-adjoint-preserve-K-injectives}).
We see that $R\mathcal{H}_Z(K)$ is computed by
$\mathcal{H}_Z(i_*\mathcal{I}^\bullet) = \mathcal{I}^\bullet$
as desired.
\end{proof}






\section{Vanishing above the dimension}
\label{section-vanishing-above-dimension}

\noindent
Let $S$ be a scheme. Let $X$ be a quasi-compact and quasi-separated
algebraic space over $S$. In this case $|X|$ is a spectral space, see
Properties of Spaces, Lemma
\ref{spaces-properties-lemma-quasi-compact-quasi-separated-spectral}.
Moreover, the dimension of $X$ (as defined in
Properties of Spaces, Definition \ref{spaces-properties-definition-dimension})
is equal to the Krull dimension of $|X|$, see
Decent Spaces, Lemma \ref{decent-spaces-lemma-dimension-decent-space}.
We will show that for quasi-coherent sheaves on $X$ we have vanishing of
cohomology above the dimension. This result is already interesting for
quasi-separated algebraic spaces of finite type over a field.

\begin{lemma}
\label{lemma-vanishing-above-dimension}
Let $S$ be a scheme. Let $X$ be a quasi-compact and quasi-separated
algebraic space over $S$. Assume $\dim(X) \leq d$ for some integer $d$.
Let $\mathcal{F}$ be a quasi-coherent sheaf $\mathcal{F}$ on $X$.
\begin{enumerate}
\item $H^q(X, \mathcal{F}) = 0$ for $q > d$,
\item $H^d(X, \mathcal{F}) \to H^d(U, \mathcal{F})$ is surjective
for any quasi-compact open $U \subset X$,
\item $H^q_Z(X, \mathcal{F}) = 0$ for $q > d$ for any closed subspace
$Z \subset X$ whose complement is quasi-compact.
\end{enumerate}
\end{lemma}

\begin{proof}
By Properties of Spaces, Lemma
\ref{spaces-properties-lemma-dimension-decent-invariant-under-etale}
every algebraic space $Y$ \'etale over $X$ has dimension $\leq d$.
If $Y$ is quasi-separated, the dimension of $Y$ is equal to the
Krull dimension of $|Y|$ by 
Decent Spaces, Lemma \ref{decent-spaces-lemma-dimension-decent-space}.
Also, if $Y$ is a scheme, then \'etale cohomology of $\mathcal{F}$
over $Y$, resp.\ \'etale cohomology of $\mathcal{F}$ with support in a
closed subscheme, agrees with usual cohomology of $\mathcal{F}$,
resp.\ usual cohomology with support in the closed subscheme.
See
Descent, Proposition \ref{descent-proposition-same-cohomology-quasi-coherent}
and
\'Etale Cohomology, Lemma
\ref{etale-cohomology-lemma-cohomology-with-support-quasi-coherent}.
We will use these facts without further mention.

\medskip\noindent
By Decent Spaces, Lemma
\ref{decent-spaces-lemma-filter-quasi-compact-quasi-separated}
there exist an integer $n$ and open subspaces
$$
\emptyset = U_{n + 1} \subset
U_n \subset U_{n - 1} \subset \ldots \subset U_1 = X
$$
with the following property: setting $T_p = U_p \setminus U_{p + 1}$
(with reduced induced subspace structure) there exists a quasi-compact
separated scheme $V_p$ and a surjective \'etale morphism $f_p : V_p \to U_p$
such that $f_p^{-1}(T_p) \to T_p$ is an isomorphism.

\medskip\noindent
As $U_n = V_n$ is a scheme, our initial remarks imply the cohomology of
$\mathcal{F}$ over $U_n$ vanishes in degrees $> d$ by
Cohomology, Proposition
\ref{cohomology-proposition-cohomological-dimension-spectral}.
Suppose we have shown, by induction, that
$H^q(U_{p + 1}, \mathcal{F}|_{U_{p + 1}}) = 0$ for $q > d$.
It suffices to show $H_{T_p}^q(U_p, \mathcal{F})$ for
$q > d$ is zero in order to conclude the vanishing of cohomology
of $\mathcal{F}$ over $U_p$ in degrees $> d$.
However, we have
$$
H^q_{T_p}(U_p, \mathcal{F}) = H^q_{f_p^{-1}(T_p)}(V_p, \mathcal{F})
$$
by Lemma \ref{lemma-etale-localization-sheaf-with-support}
and as $V_p$ is a scheme we obtain the desired vanishing from
Cohomology, Proposition
\ref{cohomology-proposition-cohomological-dimension-spectral}.
In this way we conclude that (1) is true.

\medskip\noindent
To prove (2) let $U \subset X$ be a quasi-compact open subspace.
Consider the open subspace $U' = U \cup U_n$. Let $Z = U' \setminus U$.
Then $g : U_n \to U'$ is an \'etale morphism such that
$g^{-1}(Z) \to Z$ is an isomorphism. Hence by
Lemma \ref{lemma-etale-localization-sheaf-with-support}
we have $H^q_Z(U', \mathcal{F}) = H^q_Z(U_n, \mathcal{F})$
which vanishes in degree $> d$ because $U_n$ is a scheme
and we can apply
Cohomology, Proposition
\ref{cohomology-proposition-cohomological-dimension-spectral}.
We conclude that $H^d(U', \mathcal{F}) \to H^d(U, \mathcal{F})$
is surjective. Assume, by induction, that we have reduced
our problem to the case where $U$ contains $U_{p + 1}$.
Then we set $U' = U \cup U_p$, set $Z = U' \setminus U$, and
we argue using the morphism $f_p : V_p \to U'$ which is \'etale
and has the property that $f_p^{-1}(Z) \to Z$ is an isomorphism.
In other words, we again see that
$$
H^q_Z(U', \mathcal{F}) = H^q_{f_p^{-1}(Z)}(V_p, \mathcal{F})
$$
and we again see this vanishes in degrees $> d$.
We conclude that $H^d(U', \mathcal{F}) \to H^d(U, \mathcal{F})$
is surjective. Eventually we reach the stage where $U_1 = X \subset U$
which finishes the proof.

\medskip\noindent
A formal argument shows that (2) implies (3).
\end{proof}








\section{Cohomology and base change, I}
\label{section-cohomology-and-base-change}

\noindent
Let $S$ be a scheme.
Let $f : X \to Y$ be a morphism of algebraic spaces over $S$.
Let $\mathcal{F}$ be a quasi-coherent sheaf on $X$.
Suppose further that $g : Y' \to Y$ is a morphism of algebraic spaces over
$S$. Denote $X' = X_{Y'} = Y' \times_Y X$ the base change of $X$ and denote
$f' : X' \to Y'$ the base change of $f$.
Also write $g' : X' \to X$ the projection,
and set $\mathcal{F}' = (g')^*\mathcal{F}$.
Here is a diagram representing the situation:
\begin{equation}
\label{equation-base-change-diagram}
\vcenter{
\xymatrix{
\mathcal{F}' = (g')^*\mathcal{F} &
X' \ar[r]_{g'} \ar[d]_{f'} &
X \ar[d]^f &
\mathcal{F} \\
Rf'_*\mathcal{F}' &
Y' \ar[r]^g &
Y &
Rf_*\mathcal{F}
}
}
\end{equation}
Here is the simplest case of the base change property we have in mind.

\begin{lemma}
\label{lemma-affine-base-change}
Let $S$ be a scheme. Let $f : X \to Y$ be an affine morphism of algebraic
spaces over $S$. Let $\mathcal{F}$ be a quasi-coherent $\mathcal{O}_X$-module.
In this case $f_*\mathcal{F} \cong Rf_*\mathcal{F}$ is a quasi-coherent
sheaf, and for every diagram (\ref{equation-base-change-diagram})
we have
$$
g^*f_*\mathcal{F} = f'_*(g')^*\mathcal{F}.
$$
\end{lemma}

\begin{proof}
By the discussion surrounding
(\ref{equation-representable-higher-direct-image})
this reduces to the case of an affine morphism of schemes which
is treated in Cohomology of Schemes, Lemma
\ref{coherent-lemma-affine-base-change}.
\end{proof}

\begin{lemma}[Flat base change]
\label{lemma-flat-base-change-cohomology}
Let $S$ be a scheme. Consider a cartesian diagram of algebraic spaces
$$
\xymatrix{
X' \ar[d]_{f'} \ar[r]_{g'} & X \ar[d]^f \\
Y' \ar[r]^g & Y
}
$$
over $S$.
Let $\mathcal{F}$ be a quasi-coherent $\mathcal{O}_X$-module
with pullback $\mathcal{F}' = (g')^*\mathcal{F}$.
Assume that $g$ is flat and that $f$ is quasi-compact and quasi-separated.
For any $i \geq 0$
\begin{enumerate}
\item the base change map of
Cohomology on Sites, Lemma
\ref{sites-cohomology-lemma-base-change-map-flat-case}
is an isomorphism
$$
g^*R^if_*\mathcal{F} \longrightarrow R^if'_*\mathcal{F}',
$$
\item if $Y = \Spec(A)$ and $Y' = \Spec(B)$, then
$H^i(X, \mathcal{F}) \otimes_A B = H^i(X', \mathcal{F}')$.
\end{enumerate}
\end{lemma}

\begin{proof}
The morphism $g'$ is flat by
Morphisms of Spaces, Lemma \ref{spaces-morphisms-lemma-base-change-flat}.
Note that flatness of $g$ and $g'$ is equivalent to flatness
of the morphisms of small \'etale ringed sites, see
Morphisms of Spaces, Lemma \ref{spaces-morphisms-lemma-flat-morphism-sites}.
Hence we can apply
Cohomology on Sites, Lemma
\ref{sites-cohomology-lemma-base-change-map-flat-case}
to obtain a base change map
$$
g^*R^pf_*\mathcal{F} \longrightarrow R^pf'_*\mathcal{F}'
$$
To prove this map is an isomorphism we can work locally in the \'etale
topology on $Y'$. Thus we may assume that $Y$ and $Y'$ are affine
schemes. Say $Y = \Spec(A)$ and $Y' = \Spec(B)$.
In this case we are really trying to show that the map
$$
H^p(X, \mathcal{F}) \otimes_A B \longrightarrow H^p(X_B, \mathcal{F}_B)
$$
is an isomorphism where $X_B = \Spec(B) \times_{\Spec(A)} X$ and
$\mathcal{F}_B$ is the pullback of $\mathcal{F}$ to $X_B$.
In other words, it suffices to prove (2).

\medskip\noindent
Fix $A \to B$ a flat ring map and let $X$ be a quasi-compact and
quasi-separated algebraic space over $A$. Note that $g' : X_B \to X$
is affine as a base change of $\Spec(B) \to \Spec(A)$. Hence
the higher direct images $R^i(g')_*\mathcal{F}_B$ are zero by
Lemma \ref{lemma-affine-vanishing-higher-direct-images}.
Thus $H^p(X_B, \mathcal{F}_B) = H^p(X, g'_*\mathcal{F}_B)$, see
Cohomology on Sites, Lemma \ref{sites-cohomology-lemma-apply-Leray}.
Moreover, we have
$$
g'_*\mathcal{F}_B = \mathcal{F} \otimes_{\underline{A}} \underline{B}
$$
where $\underline{A}$, $\underline{B}$ denotes the constant sheaf of
rings with value $A$, $B$. Namely, it is clear that there is a map
from right to left. For any affine scheme $U$ \'etale over $X$ we have
\begin{align*}
g'_*\mathcal{F}_B(U) & = \mathcal{F}_B(\Spec(B) \times_{\Spec(A)} U) \\
& =
\Gamma(\Spec(B) \times_{\Spec(A)} U,
(\Spec(B) \times_{\Spec(A)} U \to U)^*\mathcal{F}|_U) \\
& =
B \otimes_A \mathcal{F}(U)
\end{align*}
hence the map is an isomorphism. Write $B = \colim M_i$ as a filtered
colimit of finite free $A$-modules $M_i$ using Lazard's theorem, see
Algebra, Theorem \ref{algebra-theorem-lazard}.
We deduce that
\begin{align*}
H^p(X, g'_*\mathcal{F}_B) &
= H^p(X, \mathcal{F} \otimes_{\underline{A}} \underline{B}) \\
& = H^p(X, \colim_i \mathcal{F} \otimes_{\underline{A}} \underline{M_i}) \\
& = \colim_i H^p(X, \mathcal{F} \otimes_{\underline{A}} \underline{M_i}) \\
& = \colim_i H^p(X, \mathcal{F}) \otimes_A M_i \\
& = H^p(X, \mathcal{F}) \otimes_A \colim_i M_i \\
& = H^p(X, \mathcal{F}) \otimes_A B
\end{align*}
The first equality because
$g'_*\mathcal{F}_B = \mathcal{F} \otimes_{\underline{A}} \underline{B}$
as seen above.
The second because $\otimes$ commutes with colimits.
The third equality because cohomology on $X$ commutes with
colimits (see
Lemma \ref{lemma-colimits}).
The fourth equality because $M_i$ is finite free (i.e., because cohomology
commutes with finite direct sums).
The fifth because $\otimes$ commutes with colimits.
The sixth by choice of our system.
\end{proof}


\section{Coherent modules on locally Noetherian algebraic spaces}
\label{section-coherent}

\noindent
This section is the analogue of
Cohomology of Schemes, Section \ref{coherent-section-coherent-sheaves}.
In Modules on Sites, Definition \ref{sites-modules-definition-site-local}
we have defined coherent modules on any ringed topos. We use this notion
to define coherent modules on locally Noetherian algebraic spaces.
Although it is possible to work with coherent modules more generally
we resist the urge to do so.

\begin{definition}
\label{definition-coherent}
Let $S$ be a scheme. Let $X$ be a locally Noetherian algebraic space over $S$.
A quasi-coherent module $\mathcal{F}$ on $X$ is called {\it coherent}
if $\mathcal{F}$ is a coherent $\mathcal{O}_X$-module on the site
$X_\etale$ in the sense of
Modules on Sites, Definition \ref{sites-modules-definition-site-local}.
\end{definition}

\noindent
Of course this definition is a bit hard to work with. We usually use
the characterization given in the lemma below.

\begin{lemma}
\label{lemma-coherent-Noetherian}
Let $S$ be a scheme.
Let $X$ be a locally Noetherian algebraic space over $S$.
Let $\mathcal{F}$ be an $\mathcal{O}_X$-module.
The following are equivalent
\begin{enumerate}
\item $\mathcal{F}$ is coherent,
\item $\mathcal{F}$ is a quasi-coherent, finite type $\mathcal{O}_X$-module,
\item $\mathcal{F}$ is a finitely presented $\mathcal{O}_X$-module,
\item for any \'etale morphism $\varphi : U \to X$ where $U$ is a scheme
the pullback $\varphi^*\mathcal{F}$ is a coherent module on $U$, and
\item there exists a surjective \'etale morphism $\varphi : U \to X$
where $U$ is a scheme such that the pullback $\varphi^*\mathcal{F}$ is
a coherent module on $U$.
\end{enumerate}
In particular $\mathcal{O}_X$ is coherent, any invertible
$\mathcal{O}_X$-module is coherent, and more generally any
finite locally free $\mathcal{O}_X$-module is coherent.
\end{lemma}

\begin{proof}
To be sure, if $X$ is a locally Noetherian algebraic space and
$U \to X$ is an \'etale morphism, then $U$ is locally Noetherian, see
Properties of Spaces, Section \ref{spaces-properties-section-types-properties}.
The lemma then follows from the points (1) -- (5) made in
Properties of Spaces, Section \ref{spaces-properties-section-properties-modules}
and the corresponding result for coherent modules on locally
Noetherian schemes, see
Cohomology of Schemes, Lemma \ref{coherent-lemma-coherent-Noetherian}.
\end{proof}

\begin{lemma}
\label{lemma-coherent-abelian-Noetherian}
Let $S$ be a scheme. Let $X$ be a locally Noetherian algebraic space over $S$.
The category of coherent $\mathcal{O}_X$-modules is abelian. More precisely,
the kernel and cokernel of a map of coherent $\mathcal{O}_X$-modules are
coherent. Any extension of coherent sheaves is coherent.
\end{lemma}

\begin{proof}
Choose a scheme $U$ and a surjective \'etale morphism $f : U \to X$.
Pullback $f^*$ is an exact functor as it equals a restriction functor, see
Properties of Spaces, Equation
(\ref{spaces-properties-equation-restrict-modules}).
By
Lemma \ref{lemma-coherent-Noetherian} we can check whether an
$\mathcal{O}_X$-module $\mathcal{F}$ is
coherent by checking whether $f^*\mathcal{F}$ is coherent. Hence the
lemma follows from the case of schemes which is
Cohomology of Schemes, Lemma \ref{coherent-lemma-coherent-abelian-Noetherian}.
\end{proof}

\noindent
Coherent modules form a Serre subcategory of the
category of quasi-coherent $\mathcal{O}_X$-modules. This does not hold
for modules on a general ringed topos.

\begin{lemma}
\label{lemma-coherent-Noetherian-quasi-coherent-sub-quotient}
Let $S$ be a scheme.
Let $X$ be a locally Noetherian algebraic space over $S$.
Let $\mathcal{F}$ be a coherent $\mathcal{O}_X$-module.
Any quasi-coherent submodule of $\mathcal{F}$ is coherent.
Any quasi-coherent quotient module of $\mathcal{F}$ is coherent.
\end{lemma}

\begin{proof}
Choose a scheme $U$ and a surjective \'etale morphism $f : U \to X$.
Pullback $f^*$ is an exact functor as it equals a restriction functor, see
Properties of Spaces, Equation
(\ref{spaces-properties-equation-restrict-modules}).
By
Lemma \ref{lemma-coherent-Noetherian} we can check whether an
$\mathcal{O}_X$-module $\mathcal{G}$ is
coherent by checking whether $f^*\mathcal{H}$ is coherent. Hence the
lemma follows from the case of schemes which is
Cohomology of Schemes, Lemma
\ref{coherent-lemma-coherent-Noetherian-quasi-coherent-sub-quotient}.
\end{proof}

\begin{lemma}
\label{lemma-tensor-hom-coherent}
Let $S$ be a scheme.
Let $X$ be a locally Noetherian algebraic space over $S$,.
Let $\mathcal{F}$, $\mathcal{G}$ be coherent $\mathcal{O}_X$-modules.
The $\mathcal{O}_X$-modules $\mathcal{F} \otimes_{\mathcal{O}_X} \mathcal{G}$
and $\SheafHom_{\mathcal{O}_X}(\mathcal{F}, \mathcal{G})$ are
coherent.
\end{lemma}

\begin{proof}
Via Lemma \ref{lemma-coherent-Noetherian} this follows from the result
for schemes, see
Cohomology of Schemes, Lemma \ref{coherent-lemma-tensor-hom-coherent}.
\end{proof}

\begin{lemma}
\label{lemma-local-isomorphism}
Let $S$ be a scheme. Let $X$ be a locally Noetherian algebraic space over $S$.
Let $\mathcal{F}$, $\mathcal{G}$ be coherent $\mathcal{O}_X$-modules.
Let $\varphi : \mathcal{G} \to \mathcal{F}$ be a homomorphism
of $\mathcal{O}_X$-modules. Let $\overline{x}$ be a geometric point of $X$
lying over $x \in |X|$.
\begin{enumerate}
\item If $\mathcal{F}_{\overline{x}} = 0$ then there exists an open
neighbourhood $X' \subset X$ of $x$ such that $\mathcal{F}|_{X'} = 0$.
\item If $\varphi_{\overline{x}} : \mathcal{G}_{\overline{x}} \to
\mathcal{F}_{\overline{x}}$ is injective, then there exists an open
neighbourhood $X' \subset X$ of $x$ such that $\varphi|_{X'}$ is injective.
\item If $\varphi_{\overline{x}} : \mathcal{G}_{\overline{x}} \to
\mathcal{F}_{\overline{x}}$ is surjective, then there exists an open
neighbourhood $X' \subset X$ of $x$ such that $\varphi|_{X'}$ is surjective.
\item If $\varphi_{\overline{x}} : \mathcal{G}_{\overline{x}} \to
\mathcal{F}_{\overline{x}}$ is bijective, then there exists an open
neighbourhood $X' \subset X$ of $x$ such that $\varphi|_{X'}$ is an isomorphism.
\end{enumerate}
\end{lemma}

\begin{proof}
Let $\varphi : U \to X$ be an \'etale morphism where $U$ is a scheme and
let $u \in U$ be a point mapping to $x$. By
Properties of Spaces, Lemmas
\ref{spaces-properties-lemma-stalk-quasi-coherent} and
\ref{spaces-properties-lemma-describe-etale-local-ring}
as well as
More on Algebra, Lemma \ref{more-algebra-lemma-dumb-properties-henselization}
we see that $\varphi_{\overline{x}}$ is injective, surjective, or bijective
if and only if $\varphi_u : \varphi^*\mathcal{F}_u \to \varphi^*\mathcal{G}_u$
has the corresponding property. Thus we can apply the schemes version of
this lemma to see that (after possibly shrinking $U$) the map
$\varphi^*\mathcal{F} \to \varphi^*\mathcal{G}$ is injective, surjective,
or an isomorphism. Let $X' \subset X$ be the open subspace corresponding
to $|\varphi|(|U|) \subset |X|$, see
Properties of Spaces, Lemma \ref{spaces-properties-lemma-open-subspaces}.
Since $\{U \to X'\}$ is a covering for the \'etale topology, we conclude
that $\varphi|_{X'}$ is injective, surjective, or an isomorphism as desired.
Finally, observe that (1) follows from (2) by looking at the map
$\mathcal{F} \to 0$.
\end{proof}

\begin{lemma}
\label{lemma-coherent-support-closed}
Let $S$ be a scheme. Let $X$ be a locally Noetherian algebraic space over $S$.
Let $\mathcal{F}$ be a coherent $\mathcal{O}_X$-module. Let $i : Z \to X$
be the scheme theoretic support of $\mathcal{F}$ and $\mathcal{G}$
the quasi-coherent $\mathcal{O}_Z$-module such that
$i_*\mathcal{G} = \mathcal{F}$, see
Morphisms of Spaces, Definition
\ref{spaces-morphisms-definition-scheme-theoretic-support}.
Then $\mathcal{G}$ is a coherent $\mathcal{O}_Z$-module.
\end{lemma}

\begin{proof}
The statement of the lemma makes sense as a coherent module is in
particular of finite type. Moreover, as $Z \to X$ is a closed immersion
it is locally of finite type and hence $Z$ is locally Noetherian, see
Morphisms of Spaces, Lemmas
\ref{spaces-morphisms-lemma-immersion-locally-finite-type} and
\ref{spaces-morphisms-lemma-locally-finite-type-locally-noetherian}.
Finally, as $\mathcal{G}$ is of finite type it is a coherent
$\mathcal{O}_Z$-module by
Lemma \ref{lemma-coherent-Noetherian}
\end{proof}

\begin{lemma}
\label{lemma-i-star-equivalence}
Let $S$ be a scheme. Let $i : Z \to X$ be a closed immersion of locally
Noetherian algebraic spaces over $S$.
Let $\mathcal{I} \subset \mathcal{O}_X$ be the quasi-coherent sheaf of ideals
cutting out $Z$. The functor $i_*$ induces an equivalence between the
category of coherent $\mathcal{O}_X$-modules annihilated by $\mathcal{I}$
and the category of coherent $\mathcal{O}_Z$-modules.
\end{lemma}

\begin{proof}
The functor is fully faithful by
Morphisms of Spaces, Lemma \ref{spaces-morphisms-lemma-i-star-equivalence}.
Let $\mathcal{F}$ be a coherent $\mathcal{O}_X$-module
annihilated by $\mathcal{I}$. By
Morphisms of Spaces, Lemma \ref{spaces-morphisms-lemma-i-star-equivalence}
we can write $\mathcal{F} = i_*\mathcal{G}$ for some quasi-coherent
sheaf $\mathcal{G}$ on $Z$. To check that $\mathcal{G}$ is coherent
we can work \'etale locally (Lemma \ref{lemma-coherent-Noetherian}).
Choosing an \'etale covering by a scheme we conclude that
$\mathcal{G}$ is coherent by the case of schemes
(Cohomology of Schemes, Lemma \ref{coherent-lemma-i-star-equivalence}).
Hence the functor is fully faithful and the proof is done.
\end{proof}

\begin{lemma}
\label{lemma-finite-pushforward-coherent}
Let $S$ be a scheme. Let $f : X \to Y$ be a finite morphism of algebraic
spaces over $S$ with $Y$ locally Noetherian. Let $\mathcal{F}$ be a
coherent $\mathcal{O}_X$-module. Assume $f$ is finite and $Y$ locally
Noetherian. Then $R^pf_*\mathcal{F} = 0$ for $p > 0$ and
$f_*\mathcal{F}$ is coherent.
\end{lemma}

\begin{proof}
Choose a scheme $V$ and a surjective \'etale morphism $V \to Y$.
Then $V \times_Y X \to V$ is a finite morphism of locally Noetherian
schemes. By (\ref{equation-representable-higher-direct-image}) we reduce
to the case of schemes which is
Cohomology of Schemes, Lemma \ref{coherent-lemma-finite-pushforward-coherent}.
\end{proof}






\section{Coherent sheaves on Noetherian spaces}
\label{section-coherent-quasi-compact}

\noindent
In this section we mention some properties of coherent sheaves on
Noetherian algebraic spaces.

\begin{lemma}
\label{lemma-acc-coherent}
Let $S$ be a scheme. Let $X$ be a Noetherian algebraic space over $S$.
Let $\mathcal{F}$ be a coherent $\mathcal{O}_X$-module.
The ascending chain condition holds for quasi-coherent submodules
of $\mathcal{F}$. In other words, given any sequence
$$
\mathcal{F}_1 \subset \mathcal{F}_2 \subset \ldots \subset \mathcal{F}
$$
of quasi-coherent submodules, then
$\mathcal{F}_n = \mathcal{F}_{n + 1} = \ldots $ for some $n \geq 0$.
\end{lemma}

\begin{proof}
Choose an affine scheme $U$ and a surjective \'etale morphism $U \to X$
(see Properties of Spaces, Lemma
\ref{spaces-properties-lemma-quasi-compact-affine-cover}).
Then $U$ is a Noetherian scheme (by
Morphisms of Spaces, Lemma
\ref{spaces-morphisms-lemma-locally-finite-type-locally-noetherian}).
If $\mathcal{F}_n|_U = \mathcal{F}_{n + 1}|_U = \ldots$
then $\mathcal{F}_n = \mathcal{F}_{n + 1} = \ldots$.
Hence the result follows from the case of schemes, see
Cohomology of Schemes, Lemma \ref{coherent-lemma-acc-coherent}.
\end{proof}

\begin{lemma}
\label{lemma-power-ideal-kills-sheaf}
Let $S$ be a scheme. Let $X$ be a Noetherian algebraic space over $S$.
Let $\mathcal{F}$ be a coherent sheaf on $X$. Let
$\mathcal{I} \subset \mathcal{O}_X$ be a quasi-coherent sheaf of ideals
corresponding to a closed subspace $Z \subset X$. Then there is some
$n \geq 0$ such that $\mathcal{I}^n\mathcal{F} = 0$ if and only if
$\text{Supp}(\mathcal{F}) \subset Z$ (set theoretically).
\end{lemma}

\begin{proof}
Choose an affine scheme $U$ and a surjective \'etale morphism $U \to X$
(see Properties of Spaces, Lemma
\ref{spaces-properties-lemma-quasi-compact-affine-cover}).
Then $U$ is a Noetherian scheme (by
Morphisms of Spaces, Lemma
\ref{spaces-morphisms-lemma-locally-finite-type-locally-noetherian}).
Note that $\mathcal{I}^n\mathcal{F}|_U = 0$ if and only if
$\mathcal{I}^n\mathcal{F} = 0$ and similarly for the condition on
the support. Hence the result follows from the case of schemes, see
Cohomology of Schemes, Lemma \ref{coherent-lemma-power-ideal-kills-sheaf}.
\end{proof}

\begin{lemma}[Artin-Rees]
\label{lemma-Artin-Rees}
Let $S$ be a scheme. Let $X$ be a Noetherian algebraic space over $S$.
Let $\mathcal{F}$ be a coherent sheaf on $X$. Let
$\mathcal{G} \subset \mathcal{F}$ be a quasi-coherent subsheaf.
Let $\mathcal{I} \subset \mathcal{O}_X$ be a quasi-coherent sheaf of
ideals. Then there exists a $c \geq 0$ such that for all $n \geq c$ we
have
$$
\mathcal{I}^{n - c}(\mathcal{I}^c\mathcal{F} \cap \mathcal{G})
=
\mathcal{I}^n\mathcal{F}
$$
\end{lemma}

\begin{proof}
Choose an affine scheme $U$ and a surjective \'etale morphism $U \to X$
(see Properties of Spaces, Lemma
\ref{spaces-properties-lemma-quasi-compact-affine-cover}).
Then $U$ is a Noetherian scheme (by
Morphisms of Spaces, Lemma
\ref{spaces-morphisms-lemma-locally-finite-type-locally-noetherian}).
The equality of the lemma holds if and only if it holds after
restricting to $U$. Hence the result follows from the case of schemes, see
Cohomology of Schemes, Lemma \ref{coherent-lemma-Artin-Rees}.
\end{proof}

\begin{lemma}
\label{lemma-homs-over-open}
Let $S$ be a scheme. Let $X$ be a Noetherian algebraic space over $S$.
Let $\mathcal{F}$ be a quasi-coherent $\mathcal{O}_X$-module.
Let $\mathcal{G}$ be a coherent $\mathcal{O}_X$-module.
Let $\mathcal{I} \subset \mathcal{O}_X$ be a quasi-coherent sheaf of
ideals. Denote $Z \subset X$ the corresponding closed subspace and
set $U = X \setminus Z$. There is a canonical isomorphism
$$
\colim_n \Hom_{\mathcal{O}_X}(\mathcal{I}^n\mathcal{G}, \mathcal{F})
\longrightarrow
\Hom_{\mathcal{O}_U}(\mathcal{G}|_U, \mathcal{F}|_U).
$$
In particular we have an isomorphism
$$
\colim_n \Hom_{\mathcal{O}_X}(\mathcal{I}^n, \mathcal{F})
\longrightarrow
\Gamma(U, \mathcal{F}).
$$
\end{lemma}

\begin{proof}
Let $W$ be an affine scheme and let $W \to X$ be a surjective \'etale
morphism (see Properties of Spaces, Lemma
\ref{spaces-properties-lemma-quasi-compact-affine-cover}).
Set $R = W \times_X W$. Then $W$ and $R$ are Noetherian schemes, see
Morphisms of Spaces, Lemma
\ref{spaces-morphisms-lemma-locally-finite-type-locally-noetherian}.
Hence the result hold for the restrictions of $\mathcal{F}$, $\mathcal{G}$,
and $\mathcal{I}$, $U$, $Z$ to $W$ and $R$ by
Cohomology of Schemes, Lemma \ref{coherent-lemma-homs-over-open}.
It follows formally that the result holds over $X$.
\end{proof}






\section{Devissage of coherent sheaves}
\label{section-devissage}

\noindent
This section is the analogue of
Cohomology of Schemes, Section \ref{coherent-section-devissage}.

\begin{lemma}
\label{lemma-prepare-filter-support}
Let $S$ be a scheme. Let $X$ be a Noetherian algebraic space over $S$.
Let $\mathcal{F}$ be a coherent sheaf on $X$. Suppose that
$\text{Supp}(\mathcal{F}) = Z \cup Z'$ with $Z$, $Z'$ closed.
Then there exists a short exact sequence of coherent sheaves
$$
0 \to \mathcal{G}' \to \mathcal{F} \to \mathcal{G} \to 0
$$
with $\text{Supp}(\mathcal{G}') \subset Z'$ and
$\text{Supp}(\mathcal{G}) \subset Z$.
\end{lemma}

\begin{proof}
Let $\mathcal{I} \subset \mathcal{O}_X$ be the sheaf of ideals
defining the reduced induced closed subspace structure on $Z$, see
Properties of Spaces, Lemma
\ref{spaces-properties-lemma-reduced-closed-subspace}.
Consider the subsheaves
$\mathcal{G}'_n = \mathcal{I}^n\mathcal{F}$ and the
quotients $\mathcal{G}_n = \mathcal{F}/\mathcal{I}^n\mathcal{F}$.
For each $n$ we have a short exact sequence
$$
0 \to \mathcal{G}'_n \to \mathcal{F} \to \mathcal{G}_n \to 0
$$
For every geometric point $\overline{x}$ of $Z' \setminus Z$ we have
$\mathcal{I}_{\overline{x}} = \mathcal{O}_{X, \overline{x}}$
and hence $\mathcal{G}_{n, \overline{x}} = 0$. Thus we see that
$\text{Supp}(\mathcal{G}_n) \subset Z$. Note that $X \setminus Z'$
is a Noetherian algebraic space. Hence by
Lemma \ref{lemma-power-ideal-kills-sheaf}
there exists an $n$ such that $\mathcal{G}'_n|_{X \setminus Z'} =
\mathcal{I}^n\mathcal{F}|_{X \setminus Z'} = 0$.
For such an $n$ we see that $\text{Supp}(\mathcal{G}'_n) \subset Z'$.
Thus setting $\mathcal{G}' = \mathcal{G}'_n$ and $\mathcal{G} = \mathcal{G}_n$
works.
\end{proof}

\noindent
In the following we will freely use the scheme theoretic support of
finite type modules as defined in Morphisms of Spaces, Definition
\ref{spaces-morphisms-definition-scheme-theoretic-support}.

\begin{lemma}
\label{lemma-prepare-filter-irreducible}
Let $S$ be a scheme. Let $X$ be a Noetherian algebraic space over $S$.
Let $\mathcal{F}$ be a coherent sheaf on $X$. Assume that the scheme
theoretic support of $\mathcal{F}$ is a reduced $Z \subset X$ with
$|Z|$ irreducible. Then there exist an integer $r > 0$, a nonzero
sheaf of ideals $\mathcal{I} \subset \mathcal{O}_Z$, and an injective
map of coherent sheaves
$$
i_*\left(\mathcal{I}^{\oplus r}\right) \to \mathcal{F}
$$
whose cokernel is supported on a proper closed subspace of $Z$.
\end{lemma}

\begin{proof}
By assumption there exists a coherent $\mathcal{O}_Z$-module
$\mathcal{G}$ with support $Z$ and $\mathcal{F} \cong i_*\mathcal{G}$, see
Lemma \ref{lemma-coherent-support-closed}. Hence it suffices to prove the
lemma for the case $Z = X$ and $i = \text{id}$.

\medskip\noindent
By Properties of Spaces, Proposition
\ref{spaces-properties-proposition-locally-quasi-separated-open-dense-scheme}
there exists a dense open subspace $U \subset X$ which is a scheme. Note that
$U$ is a Noetherian integral scheme. After shrinking $U$ we may assume
that $\mathcal{F}|_U \cong \mathcal{O}_U^{\oplus r}$ (for example by
Cohomology of Schemes, Lemma \ref{coherent-lemma-prepare-filter-irreducible}
or by a direct algebra argument). Let $\mathcal{I} \subset \mathcal{O}_X$
be a quasi-coherent sheaf of ideals whose associated closed subspace
is the complement of $U$ in $X$ (see for example
Properties of Spaces, Section \ref{spaces-properties-section-reduced}). 
By Lemma \ref{lemma-homs-over-open} there exists an $n \geq 0$ and a
morphism $\mathcal{I}^n(\mathcal{O}_X^{\oplus r}) \to \mathcal{F}$
which recovers our isomorphism over $U$. Since
$\mathcal{I}^n(\mathcal{O}_X^{\oplus r}) = (\mathcal{I}^n)^{\oplus r}$
we get a map as in the lemma. It is injective: namely, if $\sigma$ is
a nonzero section of $\mathcal{I}^{\oplus r}$ over a scheme $W$ \'etale
over $X$, then because $X$ hence $W$ is reduced the support of $\sigma$
contains a nonempty open of $W$. But the kernel of
$(\mathcal{I}^n)^{\oplus r} \to \mathcal{F}$ is zero
over a dense open, hence $\sigma$ cannot be a section of the kernel.
\end{proof}

\begin{lemma}
\label{lemma-coherent-filter}
Let $S$ be a scheme. Let $X$ be a Noetherian algebraic space over $S$.
Let $\mathcal{F}$ be a coherent sheaf on $X$. There exists a filtration
$$
0 = \mathcal{F}_0 \subset \mathcal{F}_1 \subset
\ldots \subset \mathcal{F}_m = \mathcal{F}
$$
by coherent subsheaves such that for each $j = 1, \ldots, m$
there exists a reduced closed subspace $Z_j \subset X$ with $|Z_j|$
irreducible and a sheaf of ideals $\mathcal{I}_j \subset \mathcal{O}_{Z_j}$
such that
$$
\mathcal{F}_j/\mathcal{F}_{j - 1}
\cong (Z_j \to X)_* \mathcal{I}_j
$$
\end{lemma}

\begin{proof}
Consider the collection
$$
\mathcal{T} =
\left\{
\begin{matrix}
T \subset |X|
\text{ closed such that there exists a coherent sheaf }
\mathcal{F} \\
\text{ with }
\text{Supp}(\mathcal{F}) = T
\text{ for which the lemma is wrong}
\end{matrix}
\right\}
$$
We are trying to show that $\mathcal{T}$ is empty. If not, then
because $|X|$ is Noetherian (Properties of Spaces, Lemma
\ref{spaces-properties-lemma-Noetherian-topology})
we can choose a minimal element $T \in \mathcal{T}$. This means that
there exists a coherent sheaf $\mathcal{F}$ on $X$ whose support is $T$
and for which the lemma does not hold. Clearly $T \not = \emptyset$ since
the only sheaf whose support is empty is the zero sheaf for which the
lemma does hold (with $m = 0$).

\medskip\noindent
If $T$ is not irreducible, then we can write $T = Z_1 \cup Z_2$
with $Z_1, Z_2$ closed and strictly smaller than $T$.
Then we can apply Lemma \ref{lemma-prepare-filter-support}
to get a short exact sequence of coherent sheaves
$$
0 \to
\mathcal{G}_1 \to
\mathcal{F} \to
\mathcal{G}_2 \to 0
$$
with $\text{Supp}(\mathcal{G}_i) \subset Z_i$. By minimality of
$T$ each of $\mathcal{G}_i$ has a filtration as in the statement
of the lemma. By considering the induced filtration on $\mathcal{F}$
we arrive at a contradiction. Hence we conclude
that $T$ is irreducible.

\medskip\noindent
Suppose $T$ is irreducible. Let $\mathcal{J}$ be the sheaf of ideals
defining the reduced induced closed subspace structure on $T$,
see Properties of Spaces, Lemma
\ref{spaces-properties-lemma-reduced-closed-subspace}.
By Lemma \ref{lemma-power-ideal-kills-sheaf} we see there exists
an $n \geq 0$ such that $\mathcal{J}^n\mathcal{F} = 0$. Hence we obtain
a filtration
$$
0 = \mathcal{I}^n\mathcal{F} \subset \mathcal{I}^{n - 1}\mathcal{F}
\subset \ldots \subset \mathcal{I}\mathcal{F} \subset \mathcal{F}
$$
each of whose successive subquotients is annihilated by $\mathcal{J}$.
Hence if each of these subquotients has a filtration as in the statement
of the lemma then also $\mathcal{F}$ does. In other words we may
assume that $\mathcal{J}$ does annihilate $\mathcal{F}$.

\medskip\noindent
Assume $T$ is irreducible and $\mathcal{J}\mathcal{F} = 0$ where
$\mathcal{J}$ is as above. Then the scheme theoretic support of
$\mathcal{F}$ is $T$, see
Morphisms of Spaces, Lemma \ref{spaces-morphisms-lemma-i-star-equivalence}.
Hence we can apply Lemma \ref{lemma-prepare-filter-irreducible}.
This gives a short exact sequence
$$
0 \to
i_*(\mathcal{I}^{\oplus r}) \to
\mathcal{F} \to
\mathcal{Q} \to 0
$$
where the support of $\mathcal{Q}$ is a proper closed subset of $T$.
Hence we see that $\mathcal{Q}$ has a filtration of the desired type
by minimality of $T$. But then clearly $\mathcal{F}$ does too, which is
our final contradiction.
\end{proof}

\begin{lemma}
\label{lemma-property-initial}
Let $S$ be a scheme. Let $X$ be a Noetherian algebraic space over $S$.
Let $\mathcal{P}$ be a property of coherent sheaves on $X$. Assume
\begin{enumerate}
\item For any short exact sequence of coherent sheaves
$$
0 \to \mathcal{F}_1 \to \mathcal{F} \to \mathcal{F}_2 \to 0
$$
if $\mathcal{F}_i$, $i = 1, 2$ have property $\mathcal{P}$
then so does $\mathcal{F}$.
\item For every reduced closed subspace $Z \subset X$ with $|Z|$ irreducible
and every quasi-coherent sheaf of ideals $\mathcal{I} \subset \mathcal{O}_Z$
we have $\mathcal{P}$ for $i_*\mathcal{I}$.
\end{enumerate}
Then property $\mathcal{P}$ holds for every coherent sheaf on $X$.
\end{lemma}

\begin{proof}
First note that if $\mathcal{F}$ is a coherent sheaf with a filtration
$$
0 = \mathcal{F}_0 \subset \mathcal{F}_1 \subset
\ldots \subset \mathcal{F}_m = \mathcal{F}
$$
by coherent subsheaves such that each of $\mathcal{F}_i/\mathcal{F}_{i - 1}$
has property $\mathcal{P}$, then so does $\mathcal{F}$.
This follows from the property (1) for $\mathcal{P}$.
On the other hand, by Lemma \ref{lemma-coherent-filter}
we can filter any $\mathcal{F}$
with successive subquotients as in (2).
Hence the lemma follows.
\end{proof}

\noindent
Here is a more useful variant of the lemma above.

\begin{lemma}
\label{lemma-property-higher-rank-cohomological}
Let $S$ be a scheme. Let $X$ be a Noetherian algebraic space over $S$.
Let $\mathcal{P}$ be a property of coherent sheaves on $X$. Assume
\begin{enumerate}
\item For any short exact sequence of coherent sheaves
$$
0 \to \mathcal{F}_1 \to \mathcal{F} \to \mathcal{F}_2 \to 0
$$
if $\mathcal{F}_i$, $i = 1, 2$ have property $\mathcal{P}$
then so does $\mathcal{F}$.
\item If $\mathcal{P}$ holds for $\mathcal{F}^{\oplus r}$ for
some $r \geq 1$, then it holds for $\mathcal{F}$.
\item For every reduced closed subspace $i : Z \to X$ with
$|Z|$ irreducible there exists a coherent sheaf $\mathcal{G}$ on $Z$
such that
\begin{enumerate}
\item $\text{Supp}(\mathcal{G}) = Z$,
\item for every nonzero quasi-coherent sheaf of ideals
$\mathcal{I} \subset \mathcal{O}_Z$ there exists a quasi-coherent
subsheaf $\mathcal{G}' \subset \mathcal{I}\mathcal{G}$ such that
$\text{Supp}(\mathcal{G}/\mathcal{G}')$ is proper closed in $Z$
and such that $\mathcal{P}$ holds for $i_*\mathcal{G}'$.
\end{enumerate}
\end{enumerate}
Then property $\mathcal{P}$ holds for every coherent sheaf on $X$.
\end{lemma}

\begin{proof}
Consider the collection
$$
\mathcal{T} =
\left\{
\begin{matrix}
T \subset |X|
\text{ nonempty closed such that there exists a coherent sheaf } \\
\mathcal{F}
\text{ with }
\text{Supp}(\mathcal{F}) = T
\text{ for which the lemma is wrong}
\end{matrix}
\right\}
$$
We are trying to show that $\mathcal{T}$ is empty. If not, then
because $|X|$ is Noetherian (Properties of Spaces, Lemma
\ref{spaces-properties-lemma-Noetherian-topology})
we can choose a minimal element $T \in \mathcal{T}$. This means that
there exists a coherent sheaf $\mathcal{F}$ on $X$ whose support is $T$
and for which the lemma does not hold.

\medskip\noindent
If $T$ is not irreducible, then we can write $T = Z_1 \cup Z_2$
with $Z_1, Z_2$ closed and strictly smaller than $T$.
Then we can apply Lemma \ref{lemma-prepare-filter-support}
to get a short exact sequence of coherent sheaves
$$
0 \to
\mathcal{G}_1 \to
\mathcal{F} \to
\mathcal{G}_2 \to 0
$$
with $\text{Supp}(\mathcal{G}_i) \subset Z_i$. By minimality of
$T$ each of $\mathcal{G}_i$ has $\mathcal{P}$. Hence $\mathcal{F}$
has property $\mathcal{P}$ by (1), a contradiction.

\medskip\noindent
Suppose $T$ is irreducible. Let $\mathcal{J}$ be the sheaf of ideals
defining the reduced induced closed subspace structure on $T$,
see Properties of Spaces, Lemma
\ref{spaces-properties-lemma-reduced-closed-subspace}.
By Lemma \ref{lemma-power-ideal-kills-sheaf} we see there exists
an $n \geq 0$ such that $\mathcal{J}^n\mathcal{F} = 0$. Hence we obtain
a filtration
$$
0 = \mathcal{J}^n\mathcal{F} \subset \mathcal{J}^{n - 1}\mathcal{F}
\subset \ldots \subset \mathcal{J}\mathcal{F} \subset \mathcal{F}
$$
each of whose successive subquotients is annihilated by $\mathcal{J}$.
Hence if each of these subquotients has a filtration as in the statement
of the lemma then also $\mathcal{F}$ does by (1). In other words we may
assume that $\mathcal{J}$ does annihilate $\mathcal{F}$.

\medskip\noindent
Assume $T$ is irreducible and $\mathcal{J}\mathcal{F} = 0$ where
$\mathcal{J}$ is as above. Denote $i : Z \to X$ the closed subspace
corresponding to $\mathcal{J}$. Then $\mathcal{F} = i_*\mathcal{H}$
for some coherent $\mathcal{O}_Z$-module $\mathcal{H}$, see
Morphisms of Spaces, Lemma \ref{spaces-morphisms-lemma-i-star-equivalence}
and Lemma \ref{lemma-coherent-support-closed}.
Let $\mathcal{G}$ be the coherent sheaf on $Z$ satisfying
(3)(a) and (3)(b). We apply Lemma \ref{lemma-prepare-filter-irreducible}
to get injective maps
$$
\mathcal{I}_1^{\oplus r_1} \to \mathcal{H}
\quad\text{and}\quad
\mathcal{I}_2^{\oplus r_2} \to \mathcal{G}
$$
where the support of the cokernels are proper closed in $Z$. Hence we find
an nonempty open $V \subset Z$ such that
$$
\mathcal{H}^{\oplus r_2}_V \cong \mathcal{G}^{\oplus r_1}_V
$$
Let $\mathcal{I} \subset \mathcal{O}_Z$ be a quasi-coherent ideal sheaf
cutting out $Z \setminus V$ we obtain
(Lemma \ref{lemma-homs-over-open})
a map
$$
\mathcal{I}^n\mathcal{G}^{\oplus r_1} \longrightarrow \mathcal{H}^{\oplus r_2}
$$
which is an isomorphism over $V$. The kernel is supported on $Z \setminus V$
hence annihilated by some power of $\mathcal{I}$, see
Lemma \ref{lemma-power-ideal-kills-sheaf}. Thus after increasing
$n$ we may assume the displayed map is injective, see
Lemma \ref{lemma-Artin-Rees}. Applying (3)(b) we find
$\mathcal{G}' \subset \mathcal{I}^n\mathcal{G}$ such that
$$
(i_*\mathcal{G}')^{\oplus r_1} \longrightarrow
i_*\mathcal{H}^{\oplus r_2} = \mathcal{F}^{\oplus r_2}
$$
is injective with cokernel supported in a proper closed subset of $Z$
and such that property $\mathcal{P}$ holds for $i_*\mathcal{G}'$.
By (1) property $\mathcal{P}$ holds for $(i_*\mathcal{G}')^{\oplus r_1}$.
By (1) and minimality of $T = |Z|$ property $\mathcal{P}$ holds for
$\mathcal{F}^{\oplus r_2}$. And finally by (2) property $\mathcal{P}$
holds for $\mathcal{F}$ which is the desired contradiction.
\end{proof}

\begin{lemma}
\label{lemma-property-higher-rank-cohomological-variant}
Let $S$ be a scheme. Let $X$ be a Noetherian algebraic space over $S$.
Let $\mathcal{P}$ be a property of coherent sheaves on $X$. Assume
\begin{enumerate}
\item For any short exact sequence of coherent sheaves on $X$
if two out of three have property $\mathcal{P}$ so does the third.
\item If $\mathcal{P}$ holds  for $\mathcal{F}^{\oplus r}$ for
some $r \geq 1$, then it holds for $\mathcal{F}$.
\item For every reduced closed subspace $i : Z \to X$ with
$|Z|$ irreducible there exists a coherent sheaf $\mathcal{G}$ on $X$
whose scheme theoretic support is $Z$ such that $\mathcal{P}$ holds for
$\mathcal{G}$.
\end{enumerate}
Then property $\mathcal{P}$ holds for every coherent sheaf on $X$.
\end{lemma}

\begin{proof}
We will show that conditions (1) and (2) of
Lemma \ref{lemma-property-initial} hold. This is clear for condition (1).
To show that (2) holds, let
$$
\mathcal{T} =
\left\{
\begin{matrix}
i : Z \to X \text{ reduced closed subspace with }|Z|\text{ irreducible such}\\
\text{ that }i_*\mathcal{I}\text{ does not have }\mathcal{P}
\text{ for some quasi-coherent }\mathcal{I} \subset \mathcal{O}_Z
\end{matrix}
\right\}
$$
If $\mathcal{T}$ is nonempty, then since $X$ is Noetherian, we can
find an $i : Z \to X$ which is minimal in $\mathcal{T}$. We will show
that this leads to a contradiction.

\medskip\noindent
Let $\mathcal{G}$ be the sheaf whose scheme theoretic support is $Z$ whose
existence is assumed in assumption (3). Let
$\varphi : i_*\mathcal{I}^{\oplus r} \to \mathcal{G}$ be as in
Lemma \ref{lemma-prepare-filter-irreducible}. Let
$$
0 = \mathcal{F}_0 \subset \mathcal{F}_1 \subset
\ldots \subset \mathcal{F}_m = \Coker(\varphi)
$$
be a filtration as in Lemma \ref{lemma-coherent-filter}. By minimality
of $Z$ and assumption (1) we see that $\Coker(\varphi)$ has
property $\mathcal{P}$. As $\varphi$ is injective we conclude using
assumption (1) once more that $i_*\mathcal{I}^{\oplus r}$ has property
$\mathcal{P}$. Using assumption (2) we conclude that $i_*\mathcal{I}$
has property $\mathcal{P}$.

\medskip\noindent
Finally, if $\mathcal{J} \subset \mathcal{O}_Z$ is a second quasi-coherent
sheaf of ideals, set $\mathcal{K} = \mathcal{I} \cap \mathcal{J}$
and consider the short exact sequences
$$
0 \to \mathcal{K} \to \mathcal{I} \to \mathcal{I}/\mathcal{K} \to 0
\quad
\text{and}
\quad
0 \to \mathcal{K} \to \mathcal{J} \to \mathcal{J}/\mathcal{K} \to 0
$$
Arguing as above, using the minimality of $Z$, we see that
$i_*\mathcal{I}/\mathcal{K}$ and $i_*\mathcal{J}/\mathcal{K}$
satisfy $\mathcal{P}$. Hence by assumption (1) we conclude that
$i_*\mathcal{K}$ and then $i_*\mathcal{J}$ satisfy $\mathcal{P}$.
In other words, $Z$ is not an element of $\mathcal{T}$ which is
the desired contradiction.
\end{proof}






\section{Limits of coherent modules}
\label{section-limits}

\noindent
A colimit of coherent modules (on a locally Noetherian
algebraic space) is typically not coherent. But it is quasi-coherent
as any colimit of quasi-coherent modules on an algebraic space is
quasi-coherent, see Properties of Spaces, Lemma
\ref{spaces-properties-lemma-properties-quasi-coherent}.
Conversely, if the algebraic space is Noetherian, then every quasi-coherent
module is a filtered colimit of coherent modules.

\begin{lemma}
\label{lemma-directed-colimit-coherent}
Let $S$ be a scheme. Let $X$ be a Noetherian algebraic space over $S$.
Every quasi-coherent $\mathcal{O}_X$-module is the filtered colimit
of its coherent submodules.
\end{lemma}

\begin{proof}
Let $\mathcal{F}$ be a quasi-coherent $\mathcal{O}_X$-module.
If $\mathcal{G}, \mathcal{H} \subset \mathcal{F}$ are coherent
$\mathcal{O}_X$-submodules then the image of
$\mathcal{G} \oplus \mathcal{H} \to \mathcal{F}$ is another
coherent $\mathcal{O}_X$-submodule which contains both of them
(see Lemmas \ref{lemma-coherent-abelian-Noetherian} and
\ref{lemma-coherent-Noetherian-quasi-coherent-sub-quotient}).
In this way we see that the system is directed.
Hence it now suffices to show that $\mathcal{F}$ can be written as
a filtered colimit of coherent modules, as then we can take the
images of these modules in $\mathcal{F}$ to conclude there are
enough of them.

\medskip\noindent
Let $U$ be an affine scheme and $U \to X$ a surjective \'etale morphism.
Set $R = U \times_X U$ so that $X = U/R$ as usual. By
Properties of Spaces, Proposition
\ref{spaces-properties-proposition-quasi-coherent}
we see that $\QCoh(\mathcal{O}_X) = \QCoh(U, R, s, t, c)$.
Hence we reduce to showing the corresponding thing for
$\QCoh(U, R, s, t, c)$. Thus the result follows from
the more general Groupoids, Lemma \ref{groupoids-lemma-colimit-coherent}.
\end{proof}

\begin{lemma}
\label{lemma-direct-colimit-finite-presentation}
Let $S$ be a scheme. Let $f : X \to Y$ be an affine morphism of algebraic
spaces over $S$ with $Y$ Noetherian. Then every quasi-coherent
$\mathcal{O}_X$-module is a filtered colimit of finitely presented
$\mathcal{O}_X$-modules.
\end{lemma}

\begin{proof}
Let $\mathcal{F}$ be a quasi-coherent $\mathcal{O}_X$-module.
Write $f_*\mathcal{F} = \colim \mathcal{H}_i$ with $\mathcal{H}_i$
a coherent $\mathcal{O}_Y$-module, see
Lemma \ref{lemma-directed-colimit-coherent}.
By Lemma \ref{lemma-coherent-Noetherian} the modules $\mathcal{H}_i$
are $\mathcal{O}_Y$-modules of finite presentation. Hence
$f^*\mathcal{H}_i$ is an $\mathcal{O}_X$-module of finite presentation, see
Properties of Spaces, Section
\ref{spaces-properties-section-properties-modules}.
We claim the map
$$
\colim f^*\mathcal{H}_i = f^*f_*\mathcal{F} \to \mathcal{F}
$$
is surjective as $f$ is assumed affine, Namely, choose
a scheme $V$ and a surjective \'etale morphism $V \to Y$. Set
$U = X \times_Y V$. Then $U$ is a scheme, $f' : U \to V$ is affine, and
$U \to X$ is surjective \'etale. By
Properties of Spaces, Lemma
\ref{spaces-properties-lemma-pushforward-etale-base-change-modules}
we see that $f'_*(\mathcal{F}|_U) = f_*\mathcal{F}|_V$ and similarly
for pullbacks. Thus the restriction of $f^*f_*\mathcal{F} \to \mathcal{F}$
to $U$ is the map
$$
f^*f_*\mathcal{F}|_U = (f')^*(f_*\mathcal{F})|_V) =
(f')^*f'_*(\mathcal{F}|_U) \to \mathcal{F}|_U
$$
which is surjective as $f'$ is an affine morphism of schemes.
Hence the claim holds.

\medskip\noindent
We conclude that every quasi-coherent module on $X$ is a quotient of a
filtered colimit of finitely presented modules. In particular, we see that
$\mathcal{F}$ is a cokernel of a map
$$
\colim_{j \in J} \mathcal{G}_j \longrightarrow \colim_{i \in I} \mathcal{H}_i
$$
with $\mathcal{G}_j$ and $\mathcal{H}_i$ finitely presented. Note
that for every $j \in I$ there exist $i \in I$ and a morphism
$\alpha : \mathcal{G}_j \to \mathcal{H}_i$ such that
$$
\xymatrix{
\mathcal{G}_j \ar[r]_\alpha \ar[d] & \mathcal{H}_i \ar[d] \\
\colim_{j \in J} \mathcal{G}_j \ar[r] &
\colim_{i \in I} \mathcal{H}_i
}
$$
commutes, see
Lemma \ref{lemma-finite-presentation-quasi-compact-colimit}.
In this situation $\Coker(\alpha)$ is a finitely presented
$\mathcal{O}_X$-module which comes endowed with a map
$\Coker(\alpha) \to \mathcal{F}$. Consider the set $K$ of
triples $(i, j, \alpha)$ as above. We say that
$(i, j, \alpha) \leq (i', j', \alpha')$ if and only if
$i \leq i'$, $j \leq j'$, and the diagram
$$
\xymatrix{
\mathcal{G}_j \ar[r]_\alpha \ar[d] & \mathcal{H}_i \ar[d] \\
\mathcal{G}_{j'} \ar[r]^{\alpha'} &
\mathcal{H}_{i'}
}
$$
commutes. It follows from the above that $K$ is
a directed partially ordered set,
$$
\mathcal{F} = \colim_{(i, j, \alpha) \in K} \Coker(\alpha),
$$
and we win.
\end{proof}






\section{Vanishing of cohomology}
\label{section-vanishing}

\noindent
In this section we show that a quasi-compact and quasi-separated
algebraic space is affine if it has vanishing higher cohomology
for all quasi-coherent sheaves. We do this in a sequence of lemmas
all of which will become obsolete once we prove
Proposition \ref{proposition-vanishing-affine}.

\begin{situation}
\label{situation-vanishing}
Here $S$ is a scheme and $X$ is a quasi-compact and quasi-separated
algebraic space over $S$ with the following property: For every
quasi-coherent $\mathcal{O}_X$-module $\mathcal{F}$ we have
$H^1(X, \mathcal{F}) = 0$. We set $A = \Gamma(X, \mathcal{O}_X)$.
\end{situation}

\noindent
We would like to show that the canonical morphism
$$
p : X \longrightarrow \Spec(A)
$$
(see Properties of Spaces, Lemma
\ref{spaces-properties-lemma-morphism-to-affine-scheme}) is an isomorphism.
If $M$ is an $A$-module we denote $M \otimes_A \mathcal{O}_X$
the quasi-coherent module $p^*\tilde M$.

\begin{lemma}
\label{lemma-vanishing-compute}
In Situation \ref{situation-vanishing} for an $A$-module $M$ we have
$p_*(M \otimes_A \mathcal{O}_X) = \widetilde{M}$ and
$\Gamma(X, M \otimes_A \mathcal{O}_X) = M$.
\end{lemma}

\begin{proof}
The equality $p_*(M \otimes_A \mathcal{O}_X) = \widetilde{M}$ follows
from the equality $\Gamma(X, M \otimes_A \mathcal{O}_X) = M$ as
$p_*(M \otimes_A \mathcal{O}_X)$ is a quasi-coherent module on
$\Spec(A)$ by Morphisms of Spaces, Lemma
\ref{spaces-morphisms-lemma-pushforward}.
Observe that $\Gamma(X, \bigoplus_{i \in I} \mathcal{O}_X) =
\bigoplus_{i \in I} A$ by Lemma \ref{lemma-colimits}. Hence the
lemma holds for free modules. Choose a short exact sequence
$F_1 \to F_0 \to M$ where $F_0, F_1$ are free $A$-modules. Since
$H^1(X, -)$ is zero the global sections functor is right exact.
Moreover the pullback $p^*$ is right exact as well. Hence we see
that
$$
\Gamma(X, F_1 \otimes_A \mathcal{O}_X) \to
\Gamma(X, F_0 \otimes_A \mathcal{O}_X) \to
\Gamma(X, M \otimes_A \mathcal{O}_X) \to 0
$$
is exact. The result follows.
\end{proof}

\noindent
The following lemma shows that Situation \ref{situation-vanishing}
is preserved by base change of $X \to \Spec(A)$ by $\Spec(A') \to \Spec(A)$.

\begin{lemma}
\label{lemma-vanishing-base-change}
In Situation \ref{situation-vanishing}.
\begin{enumerate}
\item Given an affine morphism $X' \to X$ of algebraic spaces, we have
$H^1(X', \mathcal{F}') = 0$ for every quasi-coherent
$\mathcal{O}_{X'}$-module $\mathcal{F}'$.
\item Given an $A$-algebra $A'$ setting $X' = X \times_{\Spec(A)} \Spec(A')$
the morphism $X' \to X$ is affine and $\Gamma(X', \mathcal{O}_{X'}) = A'$.
\end{enumerate}
\end{lemma}

\begin{proof}
Part (1) follows from Lemma \ref{lemma-affine-vanishing-higher-direct-images}
and the Leray spectral sequence (Cohomology on Sites, Lemma
\ref{sites-cohomology-lemma-Leray}). Let $A \to A'$ be as in (2).
Then $X' \to X$ is affine because affine morphisms are preserved under
base change (Morphisms of Spaces, Lemma
\ref{spaces-morphisms-lemma-base-change-affine}) and the
fact that a morphism of affine schemes is affine. The equality
$\Gamma(X', \mathcal{O}_{X'}) = A'$ follows as
$(X' \to X)_*\mathcal{O}_{X'} = A' \otimes_A \mathcal{O}_X$
by Lemma \ref{lemma-affine-base-change} and thus
$$
\Gamma(X', \mathcal{O}_{X'}) =
\Gamma(X, (X' \to X)_*\mathcal{O}_{X'}) =
\Gamma(X, A' \otimes_A \mathcal{O}_X) = A'
$$
by Lemma \ref{lemma-vanishing-compute}.
\end{proof}

\begin{lemma}
\label{lemma-vanishing-separate-closed}
In Situation \ref{situation-vanishing}. Let $Z_0, Z_1 \subset |X|$
be disjoint closed subsets. Then there exists an $a \in A$ such that
$Z_0 \subset V(a)$ and $Z_1 \subset V(a - 1)$.
\end{lemma}

\begin{proof}
We may and do endow $Z_0$, $Z_1$ with the reduced induced subspace structure
(Properties of Spaces, Definition
\ref{spaces-properties-definition-reduced-induced-space}) and we denote
$i_0 : Z_0 \to X$ and $i_1 : Z_1 \to X$ the corresponding closed immersions.
Since $Z_0 \cap Z_1 = \emptyset$ we see that the canonical map of
quasi-coherent $\mathcal{O}_X$-modules
$$
\mathcal{O}_X
\longrightarrow
i_{0, *}\mathcal{O}_{Z_0} \oplus i_{1, *}\mathcal{O}_{Z_1}
$$
is surjective (look at stalks at geometric points). Since $H^1(X, -)$ is
zero on the kernel of this map the induced map of global sections is
surjective. Thus we can find $a \in A$ which maps to the global section
$(0, 1)$ of the right hand side.
\end{proof}

\begin{lemma}
\label{lemma-vanishing-injective}
In Situation \ref{situation-vanishing} the morphism $p : X \to \Spec(A)$ is
universally injective.
\end{lemma}

\begin{proof}
Let $A \to k$ be a ring homomorphism where $k$ is a field. It suffices to
show that $\Spec(k) \times_{\Spec(A)} X$ has at most one point (see
Morphisms of Spaces, Lemma
\ref{spaces-morphisms-lemma-universally-injective-local}).
Using Lemma \ref{lemma-vanishing-base-change} we may assume that $A$
is a field and we have to show that $|X|$ has at most one point.

\medskip\noindent
Let's think of $X$ as an algebraic space over $\Spec(k)$ and let's
use the notation $X(K)$ to denote $K$-valued points of $X$
for any extension $k \subset K$, see
Morphisms of Spaces, Section \ref{spaces-morphisms-section-points-fields}.
If $k \subset K$ is an algebraically closed field extension
of large transcendence degree, then we see that $X(K) \to |X|$
is surjective, see Morphisms of Spaces, Lemma
\ref{spaces-morphisms-lemma-large-enough}. Hence, after replacing $k$
by $K$, we see that it suffices to prove that $X(k)$ is a singleton
(in the case $A = k)$.

\medskip\noindent
Let $x, x' \in X(k)$. By Decent Spaces, Lemma
\ref{decent-spaces-lemma-algebraic-residue-field-extension-closed-point}
we see that $x$ and $x'$ are closed points of $|X|$. Hence $x$ and $x'$
map to distinct points of $\Spec(k)$ if $x \not = x'$ by
Lemma \ref{lemma-vanishing-separate-closed}. We conclude that
$x = x'$ as desired.
\end{proof}

\begin{lemma}
\label{lemma-vanishing-separated}
In Situation \ref{situation-vanishing} the morphism $p : X \to \Spec(A)$ is
separated.
\end{lemma}

\begin{proof}
By Decent Spaces, Lemma
\ref{decent-spaces-lemma-there-is-a-scheme-integral-over}
we can find a scheme $Y$ and a surjective integral morphism
$Y \to X$. Since an integral morphism is affine, we can apply
Lemma \ref{lemma-vanishing-base-change}
to see that $H^1(Y, \mathcal{G}) = 0$ for every
quasi-coherent $\mathcal{O}_Y$-module $\mathcal{G}$.
Since $Y \to X$ is quasi-compact and $X$ is quasi-compact,
we see that $Y$ is quasi-compact.
Since $Y$ is a scheme, we may apply
Cohomology of Schemes, Lemma
\ref{coherent-lemma-quasi-compact-h1-zero-covering}
to see that $Y$ is affine. Hence $Y$ is separated.
Note that an integral morphism is affine and universally closed, see
Morphisms of Spaces, Lemma
\ref{spaces-morphisms-lemma-integral-universally-closed}.
By Morphisms of Spaces, Lemma
\ref{spaces-morphisms-lemma-image-universally-closed-separated}
we see that $X$ is a separated algebraic space.
\end{proof}

\begin{proposition}
\label{proposition-vanishing-affine}
\begin{slogan}
Serre's criterion for affineness in the setting of algebraic spaces.
\end{slogan}
A quasi-compact and quasi-separated algebraic space is affine
if and only if all higher cohomology groups of quasi-coherent sheaves
vanish. More precisely, any algebraic space as in
Situation \ref{situation-vanishing} is an affine scheme.
\end{proposition}

\begin{proof}
Choose an affine scheme $U = \Spec(B)$ and a surjective \'etale
morphism $\varphi : U \to X$. Set $R = U \times_X U$. As $p$ is separated
(Lemma \ref{lemma-vanishing-separated}) we see that $R$ is a
closed subscheme of $U \times_{\Spec(A)} U = \Spec(B \otimes_A B)$.
Hence $R = \Spec(C)$ is affine too and the ring map
$$
B \otimes_A B \longrightarrow C
$$
is surjective. Let us denote the two maps $s, t : B \to C$ as usual. Pick
$g_1, \ldots, g_m \in B$ such that $s(g_1), \ldots, s(g_m)$ generate $C$
over $t : B \to C$ (which is possible as $t : B \to C$ is of finite
presentation and the displayed map is surjective). Then $g_1, \ldots, g_m$
give global sections of $\varphi_*\mathcal{O}_U$ and the map
$$
\mathcal{O}_X[z_1, \ldots, z_n] \longrightarrow \varphi_*\mathcal{O}_U,
\quad
z_j \longmapsto g_j
$$
is surjective: you can check this by restricting to $U$.
Namely, $\varphi^*\varphi_*\mathcal{O}_U = t_*\mathcal{O}_R$
(by Lemma \ref{lemma-flat-base-change-cohomology})
hence you get exactly the condition that $s(g_i)$ generate $C$
over $t : B \to C$. By the vanishing of $H^1$ of the kernel we see that
$$
\Gamma(X, \mathcal{O}_X[x_1, \ldots, x_n]) =
A[x_1, \ldots, x_n] \longrightarrow
\Gamma(X, \varphi_*\mathcal{O}_U) = \Gamma(U, \mathcal{O}_U) = B
$$
is surjective. Thus we conclude that $B$ is a finite type $A$-algebra.
Hence $X \to \Spec(A)$ is of finite type and separated.
By Lemma \ref{lemma-vanishing-injective}
and
Morphisms of Spaces, Lemma \ref{spaces-morphisms-lemma-locally-quasi-finite}
it is also locally quasi-finite. Hence $X \to \Spec(A)$ is representable by
Morphisms of Spaces, Lemma
\ref{spaces-morphisms-lemma-locally-quasi-finite-separated-representable}
and $X$ is a scheme. Finally $X$ is affine, hence equal to $\Spec(A)$,
by an application of Cohomology of Schemes, Lemma
\ref{coherent-lemma-quasi-compact-h1-zero-covering}.
\end{proof}

\begin{lemma}
\label{lemma-Noetherian-h1-zero}
Let $S$ be a scheme. Let $X$ be a Noetherian algebraic space over $S$.
Assume that for every coherent $\mathcal{O}_X$-module
$\mathcal{F}$ we have $H^1(X, \mathcal{F}) = 0$.
Then $X$ is an affine scheme.
\end{lemma}

\begin{proof}
The assumption implies that $H^1(X, \mathcal{F}) = 0$ for every quasi-coherent
$\mathcal{O}_X$-module $\mathcal{F}$ by
Lemmas \ref{lemma-directed-colimit-coherent} and \ref{lemma-colimits}.
Then $X$ is affine by
Proposition \ref{proposition-vanishing-affine}.
\end{proof}

\begin{lemma}
\label{lemma-Noetherian-h1-zero-invertible}
Let $S$ be a scheme. Let $X$ be a Noetherian algebraic space over $S$.
Let $\mathcal{L}$ be an invertible $\mathcal{O}_X$-module.
Assume that for every coherent $\mathcal{O}_X$-module
$\mathcal{F}$ there exists an $n \geq 1$ such that
$H^1(X, \mathcal{F} \otimes_{\mathcal{O}_X} \mathcal{L}^{\otimes n}) = 0$.
Then $X$ is a scheme and $\mathcal{L}$ is ample on $X$.
\end{lemma}

\begin{proof}
Let $s \in H^0(X, \mathcal{L}^{\otimes d})$ be a global section.
Let $U \subset X$ be the open subspace over which $s$ is a generator
of $\mathcal{L}^{\otimes d}$. In particular we have
$\mathcal{L}^{\otimes d}|_U \cong \mathcal{O}_U$.
We claim that $U$ is affine.

\medskip\noindent
Proof of the claim. We will show that $H^1(U, \mathcal{F}) = 0$
for every quasi-coherent $\mathcal{O}_U$-module $\mathcal{F}$.
This will prove the claim by Proposition \ref{proposition-vanishing-affine}.
Denote $j : U \to X$ the inclusion morphism.
Since \'etale locally the morphism $j$ is affine
(by Morphisms, Lemma \ref{morphisms-lemma-affine-s-open})
we see that $j$ is affine (Morphisms of Spaces, Lemma
\ref{spaces-morphisms-lemma-affine-local}).
Hence we have
$$
H^1(U, \mathcal{F}) = H^1(X, j_*\mathcal{F})
$$
by Lemma \ref{lemma-affine-vanishing-higher-direct-images}
(and Cohomology on Sites, Lemma \ref{sites-cohomology-lemma-apply-Leray}).
Write $j_*\mathcal{F} = \colim \mathcal{F}_i$ as a filtered colimit
of coherent $\mathcal{O}_X$-modules, see
Lemma \ref{lemma-directed-colimit-coherent}. Then
$$
H^1(X, j_*\mathcal{F}) = \colim H^1(X, \mathcal{F}_i)
$$
by Lemma \ref{lemma-colimits}.
Thus it suffices to show that $H^1(X, \mathcal{F}_i)$ maps
to zero in $H^1(U, j^*\mathcal{F}_i)$. By assumption there exists
an $n \geq 1$ such that
$$
H^1(X,
\mathcal{F}_i \otimes_{\mathcal{O}_X}
(\mathcal{O}_X \oplus \mathcal{L} \oplus \ldots
\oplus \mathcal{L}^{\otimes d - 1})
\otimes_{\mathcal{O}_X} \mathcal{L}^{\otimes n}) = 0
$$
Hence there exists an $a \geq 0$ such that
$H^1(X, \mathcal{F}_i \otimes_{\mathcal{O}_X} \mathcal{L}^{\otimes ad}) = 0$.
On the other hand, the map
$$
s^a : \mathcal{F}_i \longrightarrow
\mathcal{F}_i \otimes_{\mathcal{O}_X} \mathcal{L}^{\otimes ad}
$$
is an isomorphism after restriction to $U$. Contemplating the
commutative diagram
$$
\xymatrix{
H^1(X, \mathcal{F}_i) \ar[r] \ar[d]_{s^a} & H^1(U, j^*\mathcal{F}_i)
\ar[d]^{\cong} \\
H^1(X, \mathcal{F}_i \otimes_{\mathcal{O}_X} \mathcal{L}^{\otimes ad}) \ar[r] &
H^1(U,
j^*(\mathcal{F}_i \otimes_{\mathcal{O}_X} \mathcal{L}^{\otimes ad}))
}
$$
we conclude that the map  $H^1(X, \mathcal{F}_i) \to H^1(U, j^*\mathcal{F}_i)$
is zero and the claim holds.

\medskip\noindent
Let $x \in |X|$ be a closed point. By Decent Spaces, Lemma
\ref{decent-spaces-lemma-decent-space-closed-point}
we can represent $x$ by a closed immersion $i : \Spec(k) \to X$
(this also uses that a quasi-separated algebraic space is
decent, see Decent Spaces, Section
\ref{decent-spaces-section-reasonable-decent}).
Thus $\mathcal{O}_X \to i_*\mathcal{O}_{\Spec(k)}$ is surjective.
Let $\mathcal{I} \subset \mathcal{O}_X$ be the kernel and choose
$d \geq 1$ such that
$H^1(X, \mathcal{I} \otimes_{\mathcal{O}_X} \mathcal{L}^{\otimes d}) = 0$.
Then
$$
H^0(X, \mathcal{L}^{\otimes d}) \to
H^0(X,
i_*\mathcal{O}_{\Spec(k)} \otimes_{\mathcal{O}_X} \mathcal{L}^{\otimes d}) =
H^0(\Spec(k), i^*\mathcal{L}^{\otimes d}) \cong k
$$
is surjective by the long exact cohomology sequence. Hence
there exists an $s \in H^0(X, \mathcal{L}^{\otimes d})$
such that $x \in U$ where $U$ is the open subspace corresponding to $s$
as above. Thus $x$ is in the schematic locus
(see Properties of Spaces, Lemma \ref{spaces-properties-lemma-subscheme})
of $X$ by our claim.

\medskip\noindent
To conclude that $X$ is a scheme, it suffices to show that
any open subset of $|X|$ which contains all the closed points
is equal to $|X|$. This follows from the fact that $|X|$
is a Noetherian topological space, see
Properties of Spaces, Lemma \ref{spaces-properties-lemma-Noetherian-sober}.
Finally, if $X$ is a scheme, then we can apply
Cohomology of Schemes, Lemma
\ref{coherent-lemma-quasi-compact-h1-zero-invertible}
to conclude that $\mathcal{L}$ is ample.
\end{proof}









\section{Finite morphisms and affines}
\label{section-finite-affine}

\noindent
This section is the analogue of
Cohomology of Schemes, Section \ref{coherent-section-finite-affine}.

\begin{lemma}
\label{lemma-image-affine-finite-morphism-affine-Noetherian}
Let $S$ be a scheme. Let $f : Y \to X$ be a morphism of algebraic
spaces over $S$. Assume
\begin{enumerate}
\item $f$ finite,
\item $f$ surjective,
\item $Y$ affine, and
\item $X$ Noetherian.
\end{enumerate}
Then $X$ is affine.
\end{lemma}

\begin{proof}
We will prove that under the assumptions of the lemma for any coherent
$\mathcal{O}_X$-module $\mathcal{F}$ we have $H^1(X, \mathcal{F}) = 0$.
This implies that $H^1(X, \mathcal{F}) = 0$ for every quasi-coherent
$\mathcal{O}_X$-module $\mathcal{F}$ by
Lemmas \ref{lemma-directed-colimit-coherent} and \ref{lemma-colimits}.
Then it follows that $X$ is affine from
Proposition \ref{proposition-vanishing-affine}.

\medskip\noindent
Let $\mathcal{P}$ be the property of coherent sheaves
$\mathcal{F}$ on $X$ defined by the rule
$$
\mathcal{P}(\mathcal{F}) \Leftrightarrow H^1(X, \mathcal{F}) = 0.
$$
We are going to apply Lemma \ref{lemma-property-higher-rank-cohomological}.
Thus we have to verify (1), (2) and (3) of that lemma for $\mathcal{P}$.
Property (1) follows from the long exact cohomology sequence associated
to a short exact sequence of sheaves. Property (2) follows since
$H^1(X, -)$ is an additive functor. To see (3) let $i : Z \to X$ be
a reduced closed subspace with $|Z|$ irreducible. Let $W = Z \times_X Y$
and denote $i' : W \to Y$ the corresponding closed immersion.
Denote $f' : W \to Z$ the other projection which is a finite
morphism of algebraic spaces. Since $W$ is a closed subscheme of $Y$, it
is affine. We claim that
$\mathcal{G} = f_*i'_*\mathcal{O}_W = i_*f'_*\mathcal{O}_W$
satisfies properties (3)(a) and (3)(b) of
Lemma \ref{lemma-property-higher-rank-cohomological}
which will finish the proof. Property (3)(a) is clear as $W \to Z$ is
surjective (because $f$ is surjective). To see (3)(b) let
$\mathcal{I}$ be a nonzero quasi-coherent sheaf of ideals on $Z$.
We simply take $\mathcal{G}' = \mathcal{I} \mathcal{G}$.
Namely, we have
$$
\mathcal{I} \mathcal{G} = f'_*(\mathcal{I}')
$$
where $\mathcal{I}' = \Im((f')^*\mathcal{I} \to \mathcal{O}_W)$.
This is true because $f'$ is a (representable) affine morphism of
algebraic spaces and hence the result can be checked on an \'etale
covering of $Z$ by a scheme in which case the result is
Cohomology of Schemes, Lemma
\ref{coherent-lemma-affine-morphism-projection-ideal}.
Finally, $f'$ is affine, hence $R^1f'_*\mathcal{I}' = 0$ by
Lemma \ref{lemma-affine-vanishing-higher-direct-images}.
As $W$ is affine we have $H^1(W, \mathcal{I}') = 0$ hence the Leray
spectral sequence (in the form
Cohomology on Sites, Lemma \ref{sites-cohomology-lemma-apply-Leray})
implies that $H^1(Z, f'_*\mathcal{I}') = 0$.
Since $i : Z \to X$ is affine we conclude that
$R^1i_*f'_*\mathcal{I}' = 0$ hence $H^1(X, i_*f'_*\mathcal{I}') = 0$
by Leray again and we win.
\end{proof}








\section{A weak version of Chow's lemma}
\label{section-weak-chow}

\noindent
In this section we quickly prove the following lemma in order
to help us prove the basic results on cohomology of coherent
modules on proper algebraic spaces.

\begin{lemma}
\label{lemma-weak-chow}
Let $A$ be a ring. Let $X$ be an algebraic space over $\Spec(A)$
whose structure morphism $X \to \Spec(A)$ is separated of finite type.
Then there exists a proper surjective morphism $X' \to X$
where $X'$ is a scheme which is H-quasi-projective over $\Spec(A)$.
\end{lemma}

\begin{proof}
Let $W$ be an affine scheme and let $f : W \to X$ be a surjective
\'etale morphism. There exists an integer $d$ such that all geometric
fibres of f have $\leq d$ points (because $X$ is a separated algebraic
hence reasonable, see
Decent Spaces, Lemma \ref{decent-spaces-lemma-bounded-fibres}).
Picking $d$ minimal we get a nonempty open $U \subset X$ such that
$f^{-1}(U) \to U$ is finite \'etale of degree $d$, see
Decent Spaces, Lemma
\ref{decent-spaces-lemma-quasi-compact-reasonable-stratification}.
Let
$$
V \subset W \times_X W \times_X \ldots \times_X W
$$
($d$ factors in the fibre product) be the complement of all the diagonals.
Because $W \to X$ is separated the diagonal $W \to W \times_X W$ is a
closed immersion. Since $W \to X$ is \'etale the diagonal
$W \to W \times_X W$ is an open immersion, see
Morphisms of Spaces, Lemmas
\ref{spaces-morphisms-lemma-etale-unramified} and
\ref{spaces-morphisms-lemma-diagonal-unramified-morphism}.
Hence the diagonals are open and closed subschemes
of the quasi-compact scheme $W \times_X \ldots \times_X W$.
In particular we conclude $V$ is a quasi-compact scheme.
Choose an open immersion $W \subset Y$ with $Y$ H-projective over
$A$ (this is possible as $W$ is affine and of finite type over $A$;
for example we can use
Morphisms, Lemmas
\ref{morphisms-lemma-quasi-affine-finite-type-over-S} and
\ref{morphisms-lemma-H-quasi-projective-open-H-projective}).
Let
$$
Z \subset Y \times_A Y \times_A \ldots \times_A Y
$$
be the scheme theoretic image of the composition
$V \to W \times_X \ldots \times_X W \to Y \times_A \ldots \times_A Y$.
Observe that this morphism is quasi-compact since $V$ is quasi-compact
and $Y \times_A \ldots \times_A Y$ is separated.
Note that $V \to Z$ is an open immersion as
$V \to Y \times_A \ldots \times_A Y$ is an immersion, see
Morphisms, Lemma \ref{morphisms-lemma-quasi-compact-immersion}.
The projection morphisms give $d$ morphisms $g_i : Z \to Y$.
These morphisms $g_i$ are projective as $Y$ is projective over $A$, see
material in Morphisms, Section \ref{morphisms-section-projective}.
We set
$$
X' = \bigcup g_i^{-1}(W) \subset Z
$$
There is a morphism $X' \to X$ whose restriction to $g_i^{-1}(W)$ is
the composition $g_i^{-1}(W) \to W \to X$.
Namely, these morphisms agree over $V$ hence agree over
$g_i^{-1}(W) \cap g_j^{-1}(W)$ by
Morphisms of Spaces, Lemma \ref{spaces-morphisms-lemma-equality-of-morphisms}.
Claim: the morphism $X' \to X$ is proper.

\medskip\noindent
If the claim holds, then the lemma follows by induction on $d$.
Namely, by construction $X'$ is H-quasi-projective over $\Spec(A)$.
The image of $X' \to X$ contains the open $U$ as $V$ surjects onto $U$.
Denote $T$ the reduced induced algebraic space structure on $X \setminus U$.
Then $T \times_X W$ is a closed subscheme of $W$, hence affine.
Moreover, the morphism $T \times_X W \to T$ is \'etale and every geometric
fibre has $< d$ points. By induction hypothesis there exists a proper
surjective morphism $T' \to T$ where $T'$ is a scheme H-quasi-projective
over $\Spec(A)$. Since $T$ is a closed subspace of $X$ we see that
$T' \to X$ is a proper morphism. Thus the lemma follows by taking the
proper surjective morphism $X' \amalg T' \to X$.

\medskip\noindent
Proof of the claim. By construction the morphism $X' \to X$ is separated
and of finite type. We will check conditions (1) -- (4) of
Morphisms of Spaces, Lemma
\ref{spaces-morphisms-lemma-refined-valuative-criterion-universally-closed}
for the morphisms $V \to X'$ and $X' \to X$.
Conditions (1) and (2) we have seen above.
Condition (3) holds as $X' \to X$ is separated (as a morphism whose
source is a separated algebraic space). Thus it suffices to check
liftability to $X'$ for diagrams
$$
\xymatrix{
\Spec(K) \ar[r] \ar[d] & V \ar[d] \\
\Spec(R) \ar[r] & X
}
$$
where $R$ is a valuation ring with fraction field $K$.
Note that the top horizontal map is given by $d$ pairwise distinct
$K$-valued points $w_1, \ldots, w_d$ of $W$. In fact, this
is a complete set of inverse images of the point $x \in X(K)$
coming from the diagram. Since $W \to X$ is surjective,
we can, after possibly replacing $R$ by an extension of valuation rings,
lift the morphism $\Spec(R) \to X$ to a morphism $w : \Spec(R) \to W$, see
Morphisms of Spaces, Lemma
\ref{spaces-morphisms-lemma-lift-valuation-ring-through-flat-morphism}.
Since $w_1, \ldots, w_d$ is a complete collection of inverse images of
$x$ we see that $w|_{\Spec(K)}$ is equal to one of them, say $w_i$.
Thus we see that we get a commutative diagram
$$
\xymatrix{
\Spec(K) \ar[r] \ar[d] &  Z \ar[d]_{g_i}\\
\Spec(R) \ar[r]^w & Y
}
$$
By the valuative criterion of properness for the projective
morphism $g_i$ we can lift $w$ to $z : \Spec(R) \to Z$, see
Morphisms, Lemma \ref{morphisms-lemma-locally-projective-proper}
and
Schemes, Proposition \ref{schemes-proposition-characterize-universally-closed}.
The image of $z$ is in $g_i^{-1}(W) \subset X'$ and the proof is complete.
\end{proof}







\section{Noetherian valuative criterion}
\label{section-Noetherian-valuative-criterion}

\noindent
We prove a version of the valuative criterion for properness
using discrete valuation rings. More precise (and therefore more
technical) versions can be found in
Limits of Spaces, Section
\ref{spaces-limits-section-Noetherian-valuative-criterion}.

\begin{lemma}
\label{lemma-check-separated-dvr}
Let $S$ be a scheme. Let $f : X \to Y$ be a morphism of algebraic
spaces over $S$. Assume
\begin{enumerate}
\item $Y$ is locally Noetherian,
\item $f$ is locally of finite type and quasi-separated,
\item for every commutative diagram
$$
\xymatrix{
\Spec(K) \ar[r] \ar[d] & X \ar[d] \\
\Spec(A) \ar[r] \ar@{-->}[ru] & Y
}
$$
where $A$ is a discrete valuation ring and $K$ its fraction field,
there is at most one dotted arrow making the diagram commute.
\end{enumerate}
Then $f$ is separated.
\end{lemma}

\begin{proof}
To prove $f$ is separated, we may work \'etale locally on $Y$
(Morphisms of Spaces, Lemma \ref{spaces-morphisms-lemma-separated-local}).
Choose an affine scheme $U$ and an \'etale morphism $U \to X \times_Y X$.
Set $V = X \times_{\Delta, X \times_Y X} U$ which is quasi-compact because
$f$ is quasi-separated. Consider a commutative diagram
$$
\xymatrix{
\Spec(K) \ar[r] \ar[d] & V \ar[d] \\
\Spec(A) \ar[r] \ar@{-->}[ru] & U
}
$$
We can interpret the composition $\Spec(A) \to U \to X \times_Y X$
as a pair of morphisms $a, b : \Spec(A) \to X$ agreeing as morphisms
into $Y$ and equal when restricted to $\Spec(K)$. Hence our assumption
(3) guarantees $a = b$ and we find the dotted arrow in the diagram.
By Limits, Lemma \ref{limits-lemma-Noetherian-dvr-valuative-proper}
we conclude that $V \to U$ is proper. In other words, $\Delta$ is proper.
Since $\Delta$ is a monomorphism, we find that $\Delta$ is a
closed immersion (\'Etale Morphisms, Lemma
\ref{etale-lemma-characterize-closed-immersion}) as desired.
\end{proof}

\begin{lemma}
\label{lemma-check-proper-dvr}
Let $S$ be a scheme. Let $f : X \to Y$ be a morphism of algebraic
spaces over $S$. Assume
\begin{enumerate}
\item $Y$ is locally Noetherian,
\item $f$ is of finite type and quasi-separated,
\item for every commutative diagram
$$
\xymatrix{
\Spec(K) \ar[r] \ar[d] & X \ar[d] \\
\Spec(A) \ar[r] \ar@{-->}[ru] & Y
}
$$
where $A$ is a discrete valuation ring and $K$ its fraction field,
there is a unique dotted arrow making the diagram commute.
\end{enumerate}
Then $f$ is proper.
\end{lemma}

\begin{proof}
It suffices to prove $f$ is universally closed because $f$ is separated by
Lemma \ref{lemma-check-separated-dvr}.
To do this we may work \'etale locally on $Y$
(Morphisms of Spaces, Lemma
\ref{spaces-morphisms-lemma-universally-closed-local}).
Hence we may assume $Y = \Spec(A)$ is a Noetherian affine scheme.
Choose $X' \to X$ as in the weak form of Chow's lemma
(Lemma \ref{lemma-weak-chow}). We claim that $X' \to \Spec(A)$
is universally closed. The claim implies the lemma by
Morphisms of Spaces, Lemma \ref{spaces-morphisms-lemma-image-proper-is-proper}.
To prove this, according to
Limits, Lemma \ref{limits-lemma-check-universally-closed-Noetherian}
it suffices to prove that in every solid commutative diagram
$$
\xymatrix{
\Spec(K) \ar[r] \ar[d] & X' \ar[r] & X \ar[d] \\
\Spec(A) \ar[rr] \ar@{-->}[ru]^a \ar@{-->}[rru]_b & & Y
}
$$
where $A$ is a dvr with fraction field $K$ we can find the
dotted arrow $a$. By assumption we can find the dotted arrow $b$.
Then the morphism $X' \times_{X, b} \Spec(A) \to \Spec(A)$
is a proper morphism of schemes and by the valuative criterion
for morphisms of schemes we can lift $b$ to the desired morphism $a$.
\end{proof}

\begin{remark}[Variant for complete discrete valuation rings]
\label{remark-variant}
In Lemmas \ref{lemma-check-separated-dvr} and \ref{lemma-check-proper-dvr}
it suffices to consider complete discrete valuation rings.
To be precise in Lemma \ref{lemma-check-separated-dvr} we can replace
condition (3) by the following condition: Given any commutative diagram
$$
\xymatrix{
\Spec(K) \ar[r] \ar[d] & X \ar[d] \\
\Spec(A) \ar[r] \ar@{-->}[ru] & Y
}
$$
where $A$ is a complete discrete valuation ring with fraction field $K$
there exists at most one dotted arrow making the diagram commute. Namely, given
any diagram as in Lemma \ref{lemma-check-separated-dvr} (3)
the completion $A^\wedge$ is a discrete valuation ring
(More on Algebra, Lemma \ref{more-algebra-lemma-completion-dvr})
and the uniqueness of the arrow $\Spec(A^\wedge) \to X$
implies the uniqueness of the arrow $\Spec(A) \to X$
for example by Properties of Spaces, Proposition
\ref{spaces-properties-proposition-sheaf-fpqc}.
Similarly in Lemma \ref{lemma-check-proper-dvr}
we can replace condition (3) by the following condition:
Given any commutative diagram
$$
\xymatrix{
\Spec(K) \ar[r] \ar[d] & X \ar[d] \\
\Spec(A) \ar[r] & Y
}
$$
where $A$ is a complete discrete valuation ring with fraction field $K$
there exists an extension $A \subset A'$ of complete discrete valuation rings
inducing a fraction field extension $K \subset K'$ such that there exists a
unique arrow $\Spec(A') \to X$ making the diagram
$$
\xymatrix{
\Spec(K') \ar[r] \ar[d] & \Spec(K) \ar[r] & X \ar[d] \\
\Spec(A') \ar[r] \ar[rru] & \Spec(A) \ar[r] & Y
}
$$
commute. Namely, given any diagram as in Lemma \ref{lemma-check-proper-dvr}
part (3) the existence of any commutative diagram
$$
\xymatrix{
\Spec(L) \ar[r] \ar[d] & \Spec(K) \ar[r] & X \ar[d] \\
\Spec(B) \ar[r] \ar[rru] & \Spec(A) \ar[r] & Y
}
$$
for {\it any} extension $A \subset B$ of discrete valuation rings
will imply there exists an arrow $\Spec(A) \to X$ fitting into
the diagram. This was shown in
Morphisms of Spaces, Lemma \ref{spaces-morphisms-lemma-push-down-solution}.
In fact, it follows from these considerations that it suffices to look
for dotted arrows in diagrams for any class of discrete valuation rings
such that, given any discrete valuation ring, there is an extension of it
that is in the class. For example, we could take complete discrete valuation
rings with algebraically closed residue field.
\end{remark}







\section{Higher direct images of coherent sheaves}
\label{section-proper-pushforward}

\noindent
In this section we prove the fundamental fact that the higher
direct images of a coherent sheaf under a proper morphism
are coherent. First we prove a helper lemma.

\begin{lemma}
\label{lemma-kill-by-twisting}
Let $S$ be a scheme. Consider a commutative diagram
$$
\xymatrix{
X \ar[r]_i \ar[rd]_f & \mathbf{P}^n_Y \ar[d] \\
& Y
}
$$
of algebraic spaces over $S$. Assume $i$ is a closed immersion
and $Y$ Noetherian. Set $\mathcal{L} = i^*\mathcal{O}_{\mathbf{P}^n_Y}(1)$.
Let $\mathcal{F}$ be a coherent module on $X$.
Then there exists an integer $d_0$ such that for all $d \geq d_0$ we have
$R^pf_*(\mathcal{F} \otimes_{\mathcal{O}_X} \mathcal{L}^{\otimes d}) = 0$
for all $p > 0$.
\end{lemma}

\begin{proof}
Checking whether $R^pf_*(\mathcal{F} \otimes \mathcal{L}^{\otimes d})$
is zero can be done \'etale locally on $Y$, see
Equation (\ref{equation-representable-higher-direct-image}).
Hence we may assume $Y$ is the spectrum of a Noetherian ring. In this case
$X$ is a scheme and the result follows from
Cohomology of Schemes, Lemma \ref{coherent-lemma-kill-by-twisting}.
\end{proof}

\begin{lemma}
\label{lemma-proper-pushforward-coherent}
Let $S$ be a scheme. Let $f : X \to Y$ be a proper morphism
of algebraic spaces over $S$ with $Y$ locally Noetherian.
Let $\mathcal{F}$ be a coherent $\mathcal{O}_X$-module.
Then $R^if_*\mathcal{F}$ is a coherent $\mathcal{O}_Y$-module
for all $i \geq 0$.
\end{lemma}

\begin{proof}
We first remark that $X$ is a locally Noetherian algebraic space
by Morphisms of Spaces, Lemma
\ref{spaces-morphisms-lemma-locally-finite-type-locally-noetherian}.
Hence the statement of the lemma makes sense. Moreover, computing
$R^if_*\mathcal{F}$ commutes with \'etale localization on $Y$
(Properties of Spaces, Lemma
\ref{spaces-properties-lemma-pushforward-etale-base-change-modules})
and checking whether $R^if_*\mathcal{F}$ coherent can be done
\'etale locally on $Y$ (Lemma \ref{lemma-coherent-Noetherian}).
Hence we may assume that $Y = \Spec(A)$ is a Noetherian affine scheme.

\medskip\noindent
Assume $Y = \Spec(A)$ is an affine scheme. Note that $f$ is locally
of finite presentation
(Morphisms of Spaces, Lemma
\ref{spaces-morphisms-lemma-noetherian-finite-type-finite-presentation}).
Thus it is of finite presentation, hence $X$ is Noetherian
(Morphisms of Spaces, Lemma
\ref{spaces-morphisms-lemma-finite-presentation-noetherian}).
Thus Lemma \ref{lemma-property-higher-rank-cohomological-variant}
applies to the category of coherent modules of $X$.
For a coherent sheaf $\mathcal{F}$ on $X$ we say $\mathcal{P}$ holds
if and only if $R^if_*\mathcal{F}$ is a coherent module on $\Spec(A)$.
We will show that conditions (1), (2), and (3) of
Lemma \ref{lemma-property-higher-rank-cohomological-variant} hold
for this property thereby finishing the proof of the lemma.

\medskip\noindent
Verification of condition (1). Let
$$
0 \to \mathcal{F}_1 \to \mathcal{F}_2 \to \mathcal{F}_3 \to 0
$$
be a short exact sequence of coherent sheaves on $X$.
Consider the long exact sequence of higher direct images
$$
R^{p - 1}f_*\mathcal{F}_3 \to
R^pf_*\mathcal{F}_1 \to
R^pf_*\mathcal{F}_2 \to
R^pf_*\mathcal{F}_3 \to
R^{p + 1}f_*\mathcal{F}_1
$$
Then it is clear that if 2-out-of-3 of the sheaves $\mathcal{F}_i$
have property $\mathcal{P}$, then the higher direct images of the
third are sandwiched in this exact complex between two coherent
sheaves. Hence these higher direct images are also coherent by
Lemmas \ref{lemma-coherent-abelian-Noetherian} and
\ref{lemma-coherent-Noetherian-quasi-coherent-sub-quotient}.
Hence property $\mathcal{P}$ holds for the third as well.

\medskip\noindent
Verification of condition (2). This follows immediately from the fact
that $R^if_*(\mathcal{F}_1 \oplus \mathcal{F}_2) =
R^if_*\mathcal{F}_1 \oplus R^if_*\mathcal{F}_2$ and that a summand
of a coherent module is coherent (see lemmas cited above).

\medskip\noindent
Verification of condition (3). Let $i : Z \to X$ be a closed immersion
with $Z$ reduced and $|Z|$ irreducible. Set $g = f \circ i : Z \to \Spec(A)$.
Let $\mathcal{G}$ be a coherent module on $Z$ whose scheme theoretic support
is equal to $Z$ such that $R^pg_*\mathcal{G}$ is coherent for all $p$.
Then $\mathcal{F} = i_*\mathcal{G}$ is a coherent module on
$X$ whose support scheme theoretic support is $Z$ such that
$R^pf_*\mathcal{F} = R^pg_*\mathcal{G}$. To see this use
the Leray spectral sequence
(Cohomology on Sites, Lemma \ref{sites-cohomology-lemma-relative-Leray})
and the fact that $R^qi_*\mathcal{G} = 0$ for $q > 0$ by
Lemma \ref{lemma-affine-vanishing-higher-direct-images}
and the fact that a closed immersion is affine.
(Morphisms of Spaces, Lemma
\ref{spaces-morphisms-lemma-closed-immersion-affine}).
Thus we reduce to finding a coherent sheaf $\mathcal{G}$ on $Z$
with support equal to $Z$ such that $R^pg_*\mathcal{G}$ is coherent
for all $p$.

\medskip\noindent
We apply Lemma \ref{lemma-weak-chow} to the morphism $Z \to \Spec(A)$.
Thus we get a diagram
$$
\xymatrix{
Z \ar[rd]_g & Z' \ar[d]^-{g'} \ar[l]^\pi \ar[r]_i & \mathbf{P}^n_A \ar[dl] \\
& \Spec(A) &
}
$$
with $\pi : Z' \to Z$ proper surjective and $i$ an immersion.
Since $Z \to \Spec(A)$ is proper we conclude that $g'$ is proper
(Morphisms of Spaces, Lemma \ref{spaces-morphisms-lemma-composition-proper}).
Hence $i$ is a closed immersion
(Morphisms of Spaces, Lemmas
\ref{spaces-morphisms-lemma-universally-closed-permanence} and
\ref{spaces-morphisms-lemma-immersion-when-closed}).
It follows that the morphism
$i' = (i, \pi) : \mathbf{P}^n_A \times_{\Spec(A)} Z' = \mathbf{P}^n_Z$ is
a closed immersion
(Morphisms of Spaces, Lemma \ref{spaces-morphisms-lemma-semi-diagonal}).
Set
$$
\mathcal{L} =
i^*\mathcal{O}_{\mathbf{P}^n_A}(1) =
(i')^*\mathcal{O}_{\mathbf{P}^n_Z}(1)
$$
We may apply Lemma \ref{lemma-kill-by-twisting}
to $\mathcal{L}$ and $\pi$ as well as $\mathcal{L}$ and $g'$. 
Hence for all $d \gg 0$ we have
$R^p\pi_*\mathcal{L}^{\otimes d} = 0$ for all $p > 0$ and
$R^p(g')_*\mathcal{L}^{\otimes d} = 0$ for all $p > 0$.
Set $\mathcal{G} = \pi_*\mathcal{L}^{\otimes d}$.
By the Leray spectral sequence
(Cohomology on Sites, Lemma \ref{sites-cohomology-lemma-relative-Leray})
we have
$$
E_2^{p, q} = R^pg_* R^q\pi_*\mathcal{L}^{\otimes d}
\Rightarrow
R^{p + q}(g')_*\mathcal{L}^{\otimes d}
$$
and by choice of $d$ the only nonzero terms in $E_2^{p, q}$ are
those with $q = 0$ and the only nonzero terms of
$R^{p + q}(g')_*\mathcal{L}^{\otimes d}$ are those with $p = q = 0$.
This implies that $R^pg_*\mathcal{G} = 0$ for $p > 0$ and
that $g_*\mathcal{G} = (g')_*\mathcal{L}^{\otimes n}$.
Applying
Cohomology of Schemes, Lemma
\ref{coherent-lemma-locally-projective-pushforward}
we see that $g_*\mathcal{G} = (g')_*\mathcal{L}^{\otimes d}$ is
coherent.

\medskip\noindent
We still have to check that the support of $\mathcal{G}$ is $Z$.
This follows from the fact that $\mathcal{L}^{\otimes d}$ has
lots of global sections. We spell it out here.
Note that $\mathcal{L}^{\otimes d}$ is globally generated for all $d \geq 0$
because the same is true for $\mathcal{O}_{\mathbf{P}^n}(d)$.
Pick a point $z \in Z'$ mapping to the generic point $\xi$ of $Z$
which we can do as $\pi$ is surjective.
(Observe that $Z$ does indeed have a generic point as $|Z|$ is irreducible
and $Z$ is Noetherian, hence quasi-separated, hence $|Z|$ is a sober
topological space by
Properties of Spaces, Lemma
\ref{spaces-properties-lemma-quasi-separated-sober}.)
Pick $s \in \Gamma(Z', \mathcal{L}^{\otimes d})$ which does not vanish at $z$.
Since $\Gamma(Z, \mathcal{G}) = \Gamma(Z', \mathcal{L}^{\otimes d})$
we may think of $s$ as a global section of $\mathcal{G}$.
Choose a geometric point $\overline{z}$ of $Z'$ lying over $z$
and denote $\overline{\xi} = g' \circ \overline{z}$
the corresponding geometric point of $Z$. The adjunction map
$$
(g')^*\mathcal{G} =
(g')^*g'_*\mathcal{L}^{\otimes d} \longrightarrow \mathcal{L}^{\otimes d}
$$
induces a map of stalks
$\mathcal{G}_{\overline{\xi}} \to \mathcal{L}_{\overline{z}}$, see
Properties of Spaces, Lemma
\ref{spaces-properties-lemma-stalk-pullback-quasi-coherent}.
Moreover the adjunction map sends the pullback of $s$ (viewed as a section
of $\mathcal{G}$) to $s$ (viewed as a section of $\mathcal{L}^{\otimes d}$).
Thus the image of $s$ in the vector space which is the source of the arrow
$$
\mathcal{G}_{\overline{\xi}} \otimes \kappa(\overline{\xi})
\longrightarrow
\mathcal{L}^{\otimes d}_{\overline{z}} \otimes \kappa(\overline{z})
$$
isn't zero since by choice of $s$ the image in the target of the arrow
is nonzero. Hence $\xi$ is in the support of $\mathcal{G}$
(Morphisms of Spaces, Lemma \ref{spaces-morphisms-lemma-support-finite-type}).
Since $|Z|$ is irreducible and $Z$ is reduced we conclude that
the scheme theoretic support of $\mathcal{G}$ is all of $Z$ as desired.
\end{proof}

\begin{lemma}
\label{lemma-proper-over-affine-cohomology-finite}
Let $A$ be a Noetherian ring.
Let $f : X \to \Spec(A)$ be a proper morphism of algebraic spaces.
Let $\mathcal{F}$ be a coherent $\mathcal{O}_X$-module.
Then $H^i(X, \mathcal{F})$ is finite $A$-module for all $i \geq 0$.
\end{lemma}

\begin{proof}
This is just the affine case of Lemma \ref{lemma-proper-pushforward-coherent}.
Namely, by Lemma \ref{lemma-higher-direct-image} we know that
$R^if_*\mathcal{F}$ is a quasi-coherent sheaf. Hence it is the quasi-coherent
sheaf associated to the $A$-module
$\Gamma(\Spec(A), R^if_*\mathcal{F}) = H^i(X, \mathcal{F})$.
The equality holds by
Cohomology on Sites, Lemma \ref{sites-cohomology-lemma-apply-Leray}
and vanishing of higher cohomology groups of quasi-coherent modules
on affine schemes (Cohomology of Schemes, Lemma
\ref{coherent-lemma-quasi-coherent-affine-cohomology-zero}).
By Lemma \ref{lemma-coherent-Noetherian} we see $R^if_*\mathcal{F}$ is
a coherent sheaf if and only if $H^i(X, \mathcal{F})$
is an $A$-module of finite type. Hence
Lemma \ref{lemma-proper-pushforward-coherent} gives us the conclusion.
\end{proof}

\begin{lemma}
\label{lemma-graded-finiteness}
Let $A$ be a Noetherian ring.
Let $B$ be a finitely generated graded $A$-algebra.
Let $f : X \to \Spec(A)$ be a proper morphism of algebraic spaces.
Set $\mathcal{B} = f^*\widetilde B$.
Let $\mathcal{F}$ be a quasi-coherent
graded $\mathcal{B}$-module of finite type.
For every $p \geq 0$ the graded $B$-module $H^p(X, \mathcal{F})$
is a finite $B$-module.
\end{lemma}

\begin{proof}
To prove this we consider the fibre product diagram
$$
\xymatrix{
X' = \Spec(B) \times_{\Spec(A)} X
\ar[r]_-\pi \ar[d]_{f'} &
X \ar[d]^f \\
\Spec(B) \ar[r] &
\Spec(A)
}
$$
Note that $f'$ is a proper morphism, see
Morphisms of Spaces, Lemma \ref{spaces-morphisms-lemma-base-change-proper}.
Also, $B$ is a finitely generated $A$-algebra, and hence
Noetherian (Algebra, Lemma \ref{algebra-lemma-Noetherian-permanence}).
This implies that $X'$ is a Noetherian algebraic space
(Morphisms of Spaces, Lemma
\ref{spaces-morphisms-lemma-finite-presentation-noetherian}).
Note that $X'$ is the relative spectrum of the quasi-coherent
$\mathcal{O}_X$-algebra $\mathcal{B}$ by
Morphisms of Spaces, Lemma
\ref{spaces-morphisms-lemma-affine-equivalence-algebras}.
Since $\mathcal{F}$ is a quasi-coherent $\mathcal{B}$-module
we see that there is a unique quasi-coherent
$\mathcal{O}_{X'}$-module $\mathcal{F}'$ such that
$\pi_*\mathcal{F}' = \mathcal{F}$, see
Morphisms of Spaces, Lemma
\ref{spaces-morphisms-lemma-affine-equivalence-modules}.
Since $\mathcal{F}$ is finite type as a $\mathcal{B}$-module we
conclude that $\mathcal{F}'$ is a finite type
$\mathcal{O}_{X'}$-module (details omitted). In other words,
$\mathcal{F}'$ is a coherent $\mathcal{O}_{X'}$-module
(Lemma \ref{lemma-coherent-Noetherian}).
Since the morphism $\pi : X' \to X$ is affine we have
$$
H^p(X, \mathcal{F}) = H^p(X', \mathcal{F}')
$$
by
Lemma \ref{lemma-affine-vanishing-higher-direct-images}
and
Cohomology on Sites, Lemma \ref{sites-cohomology-lemma-apply-Leray}.
Thus the lemma follows from
Lemma \ref{lemma-proper-over-affine-cohomology-finite}.
\end{proof}




\section{The theorem on formal functions}
\label{section-theorem-formal-functions}

\noindent
This section is the analogue of
Cohomology of Schemes, Section \ref{coherent-section-theorem-formal-functions}.
We encourage the reader to read that section first.

\begin{situation}
\label{situation-formal-functions}
Here $A$ is a Noetherian ring and $I \subset A$ is an ideal.
Also, $f : X \to \Spec(A)$ is a proper morphism of algebraic spaces
and $\mathcal{F}$ is a coherent sheaf on $X$.
\end{situation}

\noindent
In this situation we denote $I^n\mathcal{F}$ the quasi-coherent
submodule of $\mathcal{F}$ generated as an $\mathcal{O}_X$-module
by products of local sections of $\mathcal{F}$ and elements of $I^n$.
In other words, it is the image of the map
$f^*\widetilde{I} \otimes_{\mathcal{O}_X} \mathcal{F} \to \mathcal{F}$.

\begin{lemma}
\label{lemma-cohomology-powers-ideal-times-F}
In Situation \ref{situation-formal-functions}.
Set $B = \bigoplus_{n \geq 0} I^n$.
Then for every $p \geq 0$ the graded $B$-module
$\bigoplus_{n \geq 0} H^p(X, I^n\mathcal{F})$ is
a finite $B$-module.
\end{lemma}

\begin{proof}
Let $\mathcal{B} = \bigoplus I^n\mathcal{O}_X = f^*\widetilde{B}$.
Then $\bigoplus I^n\mathcal{F}$ is a finite type
graded $\mathcal{B}$-module. Hence the result follows
from Lemma \ref{lemma-graded-finiteness}.
\end{proof}

\begin{lemma}
\label{lemma-cohomology-powers-ideal-application}
In Situation \ref{situation-formal-functions}.
For every $p \geq 0$ there exists an integer $c \geq 0$ such that
\begin{enumerate}
\item the multiplication map
$I^{n - c} \otimes H^p(X, I^c\mathcal{F}) \to H^p(X, I^n\mathcal{F})$
is surjective for all $n \geq c$, and
\item the image of $H^p(X, I^{n + m}\mathcal{F}) \to H^p(X, I^n\mathcal{F})$
is contained in the submodule $I^{m - c} H^p(X, I^n\mathcal{F})$
for all $n \geq 0$, $m \geq c$.
\end{enumerate}
\end{lemma}

\begin{proof}
By Lemma \ref{lemma-cohomology-powers-ideal-times-F}
we can find $d_1, \ldots, d_t \geq 0$, and
$x_i \in H^p(X, I^{d_i}\mathcal{F})$ such that
$\bigoplus_{n \geq 0} H^p(X, I^n\mathcal{F})$ is generated
by $x_1, \ldots, x_t$ over $B = \bigoplus_{n \geq 0} I^n$.
Take $c = \max\{d_i\}$. It is clear that (1) holds.
For (2) let $b = \max(0, n - c)$.
Consider the commutative diagram of $A$-modules
$$
\xymatrix{
I^{n + m - c - b} \otimes I^b \otimes
H^p(X, I^c\mathcal{F}) \ar[r] \ar[d] &
I^{n + m - c} \otimes H^p(X, I^c\mathcal{F}) \ar[r] &
H^p(X, I^{n + m}\mathcal{F}) \ar[d] \\
I^{n + m - c - b} \otimes H^p(X, I^n\mathcal{F}) \ar[rr] & &
H^p(X, I^n\mathcal{F})
}
$$
By part (1) of the lemma the composition of the horizontal arrows
is surjective if $n + m \geq c$. On the other hand, it is clear
that $n + m - c - b \geq m - c$. Hence part (2).
\end{proof}

\begin{lemma}
\label{lemma-ML-cohomology-powers-ideal}
In Situation \ref{situation-formal-functions}.
Fix $p \geq 0$.
\begin{enumerate}
\item There exists a $c_1 \geq 0$ such that for all $n \geq c_1$
we have
$$
\Ker(
H^p(X, \mathcal{F}) \to H^p(X, \mathcal{F}/I^n\mathcal{F})
)
\subset
I^{n - c_1}H^p(X, \mathcal{F}).
$$
\item The inverse system
$$
\left(H^p(X, \mathcal{F}/I^n\mathcal{F})\right)_{n \in \mathbf{N}}
$$
satisfies the Mittag-Leffler condition (see
Homology, Definition \ref{homology-definition-Mittag-Leffler}).
\item In fact for any $p$ and $n$ there exists a $c_2(n) \geq n$
such that
$$
\Im(H^p(X, \mathcal{F}/I^k\mathcal{F})
\to H^p(X, \mathcal{F}/I^n\mathcal{F}))
=
\Im(H^p(X, \mathcal{F})
\to H^p(X, \mathcal{F}/I^n\mathcal{F}))
$$
for all $k \geq c_2(n)$.
\end{enumerate}
\end{lemma}

\begin{proof}
Let $c_1 = \max\{c_p, c_{p + 1}\}$, where $c_p, c_{p +1}$ are the integers
found in Lemma \ref{lemma-cohomology-powers-ideal-application} for
$H^p$ and $H^{p + 1}$. We will use this constant in the proofs of
(1), (2) and (3).

\medskip\noindent
Let us prove part (1). Consider the short exact sequence
$$
0 \to I^n\mathcal{F} \to \mathcal{F} \to \mathcal{F}/I^n\mathcal{F} \to 0
$$
From the long exact cohomology sequence we see that
$$
\Ker(
H^p(X, \mathcal{F}) \to H^p(X, \mathcal{F}/I^n\mathcal{F})
)
=
\Im(
H^p(X, I^n\mathcal{F}) \to H^p(X, \mathcal{F})
)
$$
Hence by our choice of $c_1$ we see that this is contained in
$I^{n - c_1}H^p(X, \mathcal{F})$ for $n \geq c_1$.

\medskip\noindent
Note that part (3) implies part (2) by definition of the Mittag-Leffler
condition.

\medskip\noindent
Let us prove part (3).
Fix an $n$ throughout the rest of the proof.
Consider the commutative diagram
$$
\xymatrix{
0 \ar[r] &
I^n\mathcal{F} \ar[r] &
\mathcal{F} \ar[r] &
\mathcal{F}/I^n\mathcal{F} \ar[r] &
0 \\
0 \ar[r] &
I^{n + m}\mathcal{F} \ar[r] \ar[u] &
\mathcal{F} \ar[r] \ar[u] &
\mathcal{F}/I^{n + m}\mathcal{F} \ar[r] \ar[u] &
0
}
$$
This gives rise to the following commutative diagram
$$
\xymatrix{
H^p(X, I^n\mathcal{F}) \ar[r] &
H^p(X, \mathcal{F}) \ar[r] &
H^p(X, \mathcal{F}/I^n\mathcal{F}) \ar[r]_\delta &
H^{p + 1}(X, I^n\mathcal{F}) \\
H^p(X, I^{n + m}\mathcal{F}) \ar[r] \ar[u] &
H^p(X, \mathcal{F}) \ar[r] \ar[u]^1 &
H^p(X, \mathcal{F}/I^{n + m}\mathcal{F}) \ar[r] \ar[u] &
H^{p + 1}(X, I^{n + m}\mathcal{F}) \ar[u]^a
}
$$
If $m \geq c_1$ we see that the image of $a$ is
contained in $I^{m - c_1} H^{p + 1}(X, I^n\mathcal{F})$.
By the Artin-Rees lemma (see Algebra, Lemma \ref{algebra-lemma-map-AR})
there exists an integer $c_3(n)$ such that
$$
I^N H^{p + 1}(X, I^n\mathcal{F}) \cap \Im(\delta)
\subset
\delta\left(I^{N - c_3(n)}H^p(X, \mathcal{F}/I^n\mathcal{F})\right)
$$
for all $N \geq c_3(n)$. As $H^p(X, \mathcal{F}/I^n\mathcal{F})$
is annihilated by $I^n$, we see that if $m \geq c_3(n) + c_1 + n$,
then
$$
\Im(H^p(X, \mathcal{F}/I^{n + m}\mathcal{F})
\to H^p(X, \mathcal{F}/I^n\mathcal{F}))
=
\Im(H^p(X, \mathcal{F})
\to H^p(X, \mathcal{F}/I^n\mathcal{F}))
$$
In other words, part (3) holds with $c_2(n) = c_3(n) + c_1 + n$.
\end{proof}

\begin{theorem}[Theorem on formal functions]
\label{theorem-formal-functions}
In Situation \ref{situation-formal-functions}. Fix $p \geq 0$.
The system of maps
$$
H^p(X, \mathcal{F})/I^nH^p(X, \mathcal{F})
\longrightarrow
H^p(X, \mathcal{F}/I^n\mathcal{F})
$$
define an isomorphism of limits
$$
H^p(X, \mathcal{F})^\wedge
\longrightarrow
\lim_n H^p(X, \mathcal{F}/I^n\mathcal{F})
$$
where the left hand side is the completion of the $A$-module
$H^p(X, \mathcal{F})$ with respect to the ideal $I$, see
Algebra, Section \ref{algebra-section-completion}.
Moreover, this is in fact a homeomorphism for the limit topologies.
\end{theorem}

\begin{proof}
In fact, this follows immediately from
Lemma \ref{lemma-ML-cohomology-powers-ideal}. We spell out the details.
Set $M = H^p(X, \mathcal{F})$ and $M_n = H^p(X, \mathcal{F}/I^n\mathcal{F})$.
Denote $N_n = \Im(M \to M_n)$.
By the description of the limit in Homology, Section
\ref{homology-section-inverse-systems} we have
$$
\lim_n M_n
=
\{(x_n) \in \prod M_n \mid \varphi_i(x_n) = x_{n - 1}, \ n = 2, 3, \ldots\}
$$
Pick an element $x = (x_n) \in \lim_n M_n$.
By Lemma \ref{lemma-ML-cohomology-powers-ideal} part (3)
we have $x_n \in N_n$ for all $n$ since by
definition $x_n$ is the image of some $x_{n + m} \in M_{n + m}$ for
all $m$. By Lemma \ref{lemma-ML-cohomology-powers-ideal} part (1)
we see that there exists a factorization
$$
M \to N_n \to M/I^{n - c_1}M
$$
of the reduction map. Denote $y_n \in M/I^{n - c_1}M$ the image of $x_n$
for $n \geq c_1$. Since for $n' \geq n$ the composition
$M \to M_{n'} \to M_n$ is the given map $M \to M_n$ we see that
$y_{n'}$ maps to $y_n$ under the canonical map
$M/I^{n' - c_1}M \to M/I^{n - c_1}M$. Hence $y = (y_{n + c_1})$
defines an element of $\lim_n M/I^nM$.
We omit the verification that $y$ maps to $x$ under the
map
$$
M^\wedge = \lim_n M/I^nM \longrightarrow \lim_n M_n
$$
of the lemma. We also omit the verification on topologies.
\end{proof}

\begin{lemma}
\label{lemma-spell-out-theorem-formal-functions}
Let $A$ be a ring. Let $I \subset A$ be an ideal. Assume $A$ is
Noetherian and complete with respect to $I$.
Let $f : X \to \Spec(A)$ be a proper morphism of algebraic spaces.
Let $\mathcal{F}$ be a coherent sheaf on $X$. Then
$$
H^p(X, \mathcal{F}) = \lim_n H^p(X, \mathcal{F}/I^n\mathcal{F})
$$
for all $p \geq 0$.
\end{lemma}

\begin{proof}
This is a reformulation of the theorem on formal functions
(Theorem \ref{theorem-formal-functions}) in the
case of a complete Noetherian base ring. Namely, in this case the
$A$-module $H^p(X, \mathcal{F})$ is finite
(Lemma \ref{lemma-proper-over-affine-cohomology-finite}) hence
$I$-adically complete (Algebra, Lemma \ref{algebra-lemma-completion-tensor})
and we see that completion on the left hand side is not necessary.
\end{proof}

\begin{lemma}
\label{lemma-formal-functions-stalk}
Let $S$ be a scheme. Let $f : X \to Y$ be a morphism of algebraic spaces
over $S$ and let $\mathcal{F}$ be a quasi-coherent sheaf on $Y$. Assume
\begin{enumerate}
\item $Y$ locally Noetherian,
\item $f$ proper, and
\item $\mathcal{F}$ coherent.
\end{enumerate}
Let $\overline{y}$ be a geometric point of $Y$.
Consider the ``infinitesimal neighbourhoods''
$$
\xymatrix{
X_n =
\Spec(\mathcal{O}_{Y, \overline{y}}/\mathfrak m_{\overline{y}}^n) \times_Y X
\ar[r]_-{i_n} \ar[d]_{f_n} &
X \ar[d]^f \\
\Spec(\mathcal{O}_{Y, \overline{y}}/\mathfrak m_{\overline{y}}^n)
\ar[r]^-{c_n} & Y
}
$$
of the fibre $X_1 = X_{\overline{y}}$ and set
$\mathcal{F}_n = i_n^*\mathcal{F}$. Then we have
$$
\left(R^pf_*\mathcal{F}\right)_{\overline{y}}^\wedge
\cong
\lim_n H^p(X_n, \mathcal{F}_n)
$$
as $\mathcal{O}_{Y, \overline{y}}^\wedge$-modules.
\end{lemma}

\begin{proof}
This is just a reformulation of a special case of the theorem
on formal functions, Theorem \ref{theorem-formal-functions}.
Let us spell it out. Note that $\mathcal{O}_{Y, \overline{y}}$
is a Noetherian local ring, see
Properties of Spaces, Lemma
\ref{spaces-properties-lemma-Noetherian-local-ring-Noetherian}.
Consider the canonical morphism
$c : \Spec(\mathcal{O}_{Y, \overline{y}}) \to Y$.
This is a flat morphism as it identifies local rings.
Denote $f' : X' \to \Spec(\mathcal{O}_{Y, \overline{y}})$
the base change of $f$ to this local ring. We see that
$c^*R^pf_*\mathcal{F} = R^pf'_*\mathcal{F}'$ by
Lemma \ref{lemma-flat-base-change-cohomology}.
Moreover, we have canonical identifications $X_n = X'_n$
for all $n \geq 1$.

\medskip\noindent
Hence we may assume that $Y = \Spec(A)$ is the spectrum of
a strictly henselian Noetherian local ring $A$ with maximal ideal
$\mathfrak m$ and that $\overline{y} \to Y$ is equal to
$\Spec(A/\mathfrak m) \to Y$. It follows that
$$
\left(R^pf_*\mathcal{F}\right)_{\overline{y}} =
\Gamma(Y, R^pf_*\mathcal{F}) =
H^p(X, \mathcal{F})
$$
because $(Y, \overline{y})$ is an initial object in the category
of \'etale neighbourhoods of $\overline{y}$.
The morphisms $c_n$ are each closed immersions.
Hence their base changes $i_n$ are closed immersions as well.
Note that $i_{n, *}\mathcal{F}_n = i_{n, *}i_n^*\mathcal{F}
= \mathcal{F}/\mathfrak m^n\mathcal{F}$. By the Leray spectral sequence
for $i_n$, and Lemma \ref{lemma-finite-pushforward-coherent} we see that
$$
H^p(X_n, \mathcal{F}_n) =
H^p(X, i_{n, *}\mathcal{F}) =
H^p(X, \mathcal{F}/\mathfrak m^n\mathcal{F})
$$
Hence we may indeed apply the theorem on formal functions to compute
the limit in the statement of the lemma and we win.
\end{proof}

\noindent
Here is a lemma which we will generalize later to fibres of
dimension $ > 0$, namely the next lemma.

\begin{lemma}
\label{lemma-higher-direct-images-zero-finite-fibre}
Let $S$ be a scheme. Let $f : X \to Y$ be a morphism of algebraic spaces
over $S$. Let $\overline{y}$ be a geometric point of $Y$.
Assume
\begin{enumerate}
\item $Y$ locally Noetherian,
\item $f$ is proper, and
\item $X_{\overline{y}}$ has discrete underlying topological space.
\end{enumerate}
Then for any coherent sheaf $\mathcal{F}$ on $X$ we have
$(R^pf_*\mathcal{F})_{\overline{y}} = 0$ for all $p > 0$.
\end{lemma}

\begin{proof}
Let $\kappa(\overline{y})$ be the residue field of the local
ring of $\mathcal{O}_{Y, \overline{y}}$. As in
Lemma \ref{lemma-formal-functions-stalk}
we set $X_{\overline{y}} = X_1 = \Spec(\kappa(\overline{y})) \times_Y X$.
By Morphisms of Spaces, Lemma
\ref{spaces-morphisms-lemma-quasi-finite-at-point}
the morphism $f : X \to Y$ is quasi-finite at each of
the points of the fibre of $X \to Y$ over $\overline{y}$.
It follows that $X_{\overline{y}} \to \overline{y}$ is separated and
quasi-finite. Hence $X_{\overline{y}}$ is a scheme
by Morphisms of Spaces, Proposition
\ref{spaces-morphisms-proposition-locally-quasi-finite-separated-over-scheme}.
Since it is quasi-compact its underlying topological space is a finite
discrete space. Then it is an affine scheme by
Schemes, Lemma \ref{schemes-lemma-scheme-finite-discrete-affine}.
By Lemma \ref{lemma-image-affine-finite-morphism-affine-Noetherian}
it follows that the algebraic spaces $X_n$ are affine schemes as well.
Moreover, the underlying topological of each $X_n$ is the same
as that of $X_1$. Hence it follows that $H^p(X_n, \mathcal{F}_n) = 0$
for all $p > 0$. Hence we see that
$(R^pf_*\mathcal{F})_{\overline{y}}^\wedge = 0$
by Lemma \ref{lemma-formal-functions-stalk}.
Note that $R^pf_*\mathcal{F}$ is coherent by
Lemma \ref{lemma-proper-pushforward-coherent} and
hence $R^pf_*\mathcal{F}_{\overline{y}}$ is a finite
$\mathcal{O}_{Y, \overline{y}}$-module.
By Algebra, Lemma \ref{algebra-lemma-completion-tensor}
this implies that $(R^pf_*\mathcal{F})_{\overline{y}} = 0$.
\end{proof}

\begin{lemma}
\label{lemma-higher-direct-images-zero-above-dimension-fibre}
\begin{slogan}
For proper maps, stalks of higher direct images are trivial in degrees
larger than the dimension of the fibre.
\end{slogan}
Let $S$ be a scheme.
Let $f : X \to Y$ be a morphism of algebraic spaces over $S$.
Let $\overline{y}$ be a geometric point of $Y$.
Assume
\begin{enumerate}
\item $Y$ locally Noetherian,
\item $f$ is proper, and
\item $\dim(X_{\overline{y}}) = d$.
\end{enumerate}
Then for any coherent sheaf $\mathcal{F}$ on $X$ we have
$(R^pf_*\mathcal{F})_{\overline{y}} = 0$ for all $p > d$.
\end{lemma}

\begin{proof}
Let $\kappa(\overline{y})$ be the residue field of the local
ring of $\mathcal{O}_{Y, \overline{y}}$. As in
Lemma \ref{lemma-formal-functions-stalk}
we set $X_{\overline{y}} = X_1 = \Spec(\kappa(\overline{y})) \times_Y X$.
Moreover, the underlying topological space of each infinitesimal
neighbourhood $X_n$ is the same as that of $X_{\overline{y}}$.
Hence $H^p(X_n, \mathcal{F}_n) = 0$ for all $p > d$ by
Lemma \ref{lemma-vanishing-above-dimension}.
Hence we see that $(R^pf_*\mathcal{F})_{\overline{y}}^\wedge = 0$
by Lemma \ref{lemma-formal-functions-stalk} for $p > d$.
Note that $R^pf_*\mathcal{F}$ is coherent by
Lemma \ref{lemma-proper-pushforward-coherent} and
hence $R^pf_*\mathcal{F}_{\overline{y}}$ is a finite
$\mathcal{O}_{Y, \overline{y}}$-module.
By Algebra, Lemma \ref{algebra-lemma-completion-tensor}
this implies that $(R^pf_*\mathcal{F})_{\overline{y}} = 0$.
\end{proof}






\section{Applications of the theorem on formal functions}
\label{section-applications-formal-functions}

\noindent
We will add more here as needed.

\begin{lemma}
\label{lemma-characterize-finite}
(For a more general version see
More on Morphisms of Spaces, Lemma
\ref{spaces-more-morphisms-lemma-characterize-finite}).
Let $S$ be a scheme.
Let $f : X \to Y$ be a morphism of algebraic spaces over $S$.
Assume $Y$ is locally Noetherian.
The following are equivalent
\begin{enumerate}
\item $f$ is finite, and
\item $f$ is proper and $|X_k|$ is a discrete space
for every morphism $\Spec(k) \to Y$ where $k$ is a field.
\end{enumerate}
\end{lemma}

\begin{proof}
A finite morphism is proper according to
Morphisms of Spaces, Lemma \ref{spaces-morphisms-lemma-finite-proper}.
A finite morphism is quasi-finite according to
Morphisms of Spaces, Lemma \ref{spaces-morphisms-lemma-finite-quasi-finite}.
A quasi-finite morphism has discrete fibres $X_k$, see
Morphisms of Spaces, Lemma \ref{spaces-morphisms-lemma-locally-quasi-finite}.
Hence a finite morphism is proper and has discrete fibres $X_k$.

\medskip\noindent
Assume $f$ is proper with discrete fibres $X_k$.
We want to show $f$ is finite.
In fact it suffices to prove $f$ is affine.
Namely, if $f$ is affine, then it follows that
$f$ is integral by
Morphisms of Spaces, Lemma
\ref{spaces-morphisms-lemma-integral-universally-closed}
whereupon it follows from
Morphisms of Spaces, Lemma \ref{spaces-morphisms-lemma-finite-integral}
that $f$ is finite.

\medskip\noindent
To show that $f$ is affine we may assume that $Y$ is affine, and our
goal is to show that $X$ is affine too.
Since $f$ is proper we see that $X$ is separated and quasi-compact.
We will show that for any coherent
$\mathcal{O}_X$-module $\mathcal{F}$ we have $H^1(X, \mathcal{F}) = 0$.
This implies that $H^1(X, \mathcal{F}) = 0$ for every quasi-coherent
$\mathcal{O}_X$-module $\mathcal{F}$ by
Lemmas \ref{lemma-directed-colimit-coherent} and \ref{lemma-colimits}.
Then it follows that $X$ is affine from
Proposition \ref{proposition-vanishing-affine}. By
Lemma \ref{lemma-higher-direct-images-zero-finite-fibre}
we conclude that the stalks of $R^1f_*\mathcal{F}$
are zero for all geometric points of $Y$.
In other words, $R^1f_*\mathcal{F} = 0$. Hence we see from
the Leray Spectral Sequence for $f$ that
$H^1(X , \mathcal{F}) = H^1(Y, f_*\mathcal{F})$.
Since $Y$ is affine, and $f_*\mathcal{F}$ is quasi-coherent
(Morphisms of Spaces, Lemma \ref{spaces-morphisms-lemma-pushforward})
we conclude $H^1(Y, f_*\mathcal{F}) = 0$
from
Cohomology of Schemes, Lemma
\ref{coherent-lemma-quasi-coherent-affine-cohomology-zero}.
Hence $H^1(X, \mathcal{F}) = 0$ as desired.
\end{proof}

\noindent
As a consequence we have the following useful result.

\begin{lemma}
\label{lemma-proper-finite-fibre-finite-in-neighbourhood}
(For a more general version see
More on Morphisms of Spaces, Lemma
\ref{spaces-more-morphisms-lemma-proper-finite-fibre-finite-in-neighbourhood}).
Let $S$ be a scheme. Let $f : X \to Y$ be a morphism of algebraic spaces
over $S$. Let $\overline{y}$ be a geometric point of $Y$.
Assume
\begin{enumerate}
\item $Y$ is locally Noetherian,
\item $f$ is proper, and
\item $|X_{\overline{y}}|$ is finite.
\end{enumerate}
Then there exists an open neighbourhood $V \subset Y$ of $\overline{y}$
such that $f|_{f^{-1}(V)} : f^{-1}(V) \to V$ is finite.
\end{lemma}

\begin{proof}
The morphism $f$ is quasi-finite at all the geometric points of $X$
lying over $\overline{y}$ by
Morphisms of Spaces, Lemma \ref{spaces-morphisms-lemma-quasi-finite-at-point}.
By Morphisms of Spaces, Lemma
\ref{spaces-morphisms-lemma-locally-finite-type-quasi-finite-part} the
set of points at which $f$ is quasi-finite is an open subspace $U \subset X$.
Let $Z = X \setminus U$. Then $\overline{y} \not \in f(Z)$. Since $f$
is proper the set $f(Z) \subset Y$ is closed. Choose any open neighbourhood
$V \subset Y$ of $\overline{y}$ with $Z \cap V = \emptyset$. Then
$f^{-1}(V) \to V$ is locally quasi-finite and proper.
Hence $f^{-1}(V) \to V$ has discrete fibres $X_k$
(Morphisms of Spaces, Lemma
\ref{spaces-morphisms-lemma-locally-quasi-finite})
which are quasi-compact hence finite.
Thus $f^{-1}(V) \to V$
is finite by Lemma \ref{lemma-characterize-finite}.
\end{proof}




\section{Other chapters}

\begin{multicols}{2}
\begin{enumerate}
\item \hyperref[introduction-section-phantom]{Introduction}
\item \hyperref[conventions-section-phantom]{Conventions}
\item \hyperref[sets-section-phantom]{Set Theory}
\item \hyperref[categories-section-phantom]{Categories}
\item \hyperref[topology-section-phantom]{Topology}
\item \hyperref[sheaves-section-phantom]{Sheaves on Spaces}
\item \hyperref[algebra-section-phantom]{Commutative Algebra}
\item \hyperref[sites-section-phantom]{Sites and Sheaves}
\item \hyperref[homology-section-phantom]{Homological Algebra}
\item \hyperref[derived-section-phantom]{Derived Categories}
\item \hyperref[more-algebra-section-phantom]{More Algebra}
\item \hyperref[simplicial-section-phantom]{Simplicial Methods}
\item \hyperref[modules-section-phantom]{Sheaves of Modules}
\item \hyperref[sites-modules-section-phantom]{Modules on Sites}
\item \hyperref[injectives-section-phantom]{Injectives}
\item \hyperref[cohomology-section-phantom]{Cohomology of Sheaves}
\item \hyperref[sites-cohomology-section-phantom]{Cohomology on Sites}
\item \hyperref[hypercovering-section-phantom]{Hypercoverings}
\item \hyperref[schemes-section-phantom]{Schemes}
\item \hyperref[constructions-section-phantom]{Constructions of Schemes}
\item \hyperref[properties-section-phantom]{Properties of Schemes}
\item \hyperref[morphisms-section-phantom]{Morphisms of Schemes}
\item \hyperref[coherent-section-phantom]{Coherent Cohomology}
\item \hyperref[divisors-section-phantom]{Divisors}
\item \hyperref[limits-section-phantom]{Limits of Schemes}
\item \hyperref[varieties-section-phantom]{Varieties}
\item \hyperref[chow-section-phantom]{Chow Homology}
\item \hyperref[topologies-section-phantom]{Topologies on Schemes}
\item \hyperref[descent-section-phantom]{Descent}
\item \hyperref[more-morphisms-section-phantom]{More on Morphisms}
\item \hyperref[flat-section-phantom]{More on Flatness}
\item \hyperref[groupoids-section-phantom]{Groupoid Schemes}
\item \hyperref[more-groupoids-section-phantom]{More on Groupoid Schemes}
\item \hyperref[etale-section-phantom]{\'Etale Morphisms of Schemes}
\item \hyperref[etale-cohomology-section-phantom]{\'Etale Cohomology}
\item \hyperref[spaces-section-phantom]{Algebraic Spaces}
\item \hyperref[spaces-properties-section-phantom]{Properties of Algebraic Spaces}
\item \hyperref[spaces-morphisms-section-phantom]{Morphisms of Algebraic Spaces}
\item \hyperref[spaces-topologies-section-phantom]{Topologies on Algebraic Spaces}
\item \hyperref[spaces-descent-section-phantom]{Descent and Algebraic Spaces}
\item \hyperref[spaces-more-morphisms-section-phantom]{More on Morphisms of Spaces}
\item \hyperref[quot-section-phantom]{Quot and Hilbert Spaces}
\item \hyperref[stacks-section-phantom]{Stacks}
\item \hyperref[spaces-groupoids-section-phantom]{Groupoids in Algebraic Spaces}
\item \hyperref[spaces-more-groupoids-section-phantom]{More on Groupoids in Spaces}
\item \hyperref[bootstrap-section-phantom]{Bootstrap}
\item \hyperref[examples-stacks-section-phantom]{Examples of Stacks}
\item \hyperref[groupoids-quotients-section-phantom]{Quotients of Groupoids}
\item \hyperref[algebraic-section-phantom]{Algebraic Stacks}
\item \hyperref[criteria-section-phantom]{Criteria for Representability}
\item \hyperref[stacks-properties-section-phantom]{Properties of Algebraic Stacks}
\item \hyperref[stacks-morphisms-section-phantom]{Morphisms of Algebraic Stacks}
\item \hyperref[examples-section-phantom]{Examples}
\item \hyperref[exercises-section-phantom]{Exercises}
\item \hyperref[guide-section-phantom]{Guide to Literature}
\item \hyperref[desirables-section-phantom]{Desirables}
\item \hyperref[coding-section-phantom]{Coding Style}
\item \hyperref[fdl-section-phantom]{GNU Free Documentation License}
\item \hyperref[index-section-phantom]{Auto Generated Index}
\end{enumerate}
\end{multicols}


\bibliography{my}
\bibliographystyle{amsalpha}

\end{document}
