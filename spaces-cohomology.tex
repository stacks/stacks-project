\IfFileExists{stacks-project.cls}{%
\documentclass{stacks-project}
}{%
\documentclass{amsart}
}

% The following AMS packages are automatically loaded with
% the amsart documentclass:
%\usepackage{amsmath}
%\usepackage{amssymb}
%\usepackage{amsthm}

% For dealing with references we use the comment environment
\usepackage{verbatim}
\newenvironment{reference}{\comment}{\endcomment}
%\newenvironment{reference}{}{}
\newenvironment{slogan}{\comment}{\endcomment}
\newenvironment{history}{\comment}{\endcomment}

% For commutative diagrams you can use
% \usepackage{amscd}
\usepackage[all]{xy}

% We use 2cell for 2-commutative diagrams.
\xyoption{2cell}
\UseAllTwocells

% To put source file link in headers.
% Change "template.tex" to "this_filename.tex"
% \usepackage{fancyhdr}
% \pagestyle{fancy}
% \lhead{}
% \chead{}
% \rhead{Source file: \url{template.tex}}
% \lfoot{}
% \cfoot{\thepage}
% \rfoot{}
% \renewcommand{\headrulewidth}{0pt}
% \renewcommand{\footrulewidth}{0pt}
% \renewcommand{\headheight}{12pt}

\usepackage{multicol}

% For cross-file-references
\usepackage{xr-hyper}

% Package for hypertext links:
\usepackage{hyperref}

% For any local file, say "hello.tex" you want to link to please
% use \externaldocument[hello-]{hello}
\externaldocument[introduction-]{introduction}
\externaldocument[conventions-]{conventions}
\externaldocument[sets-]{sets}
\externaldocument[categories-]{categories}
\externaldocument[topology-]{topology}
\externaldocument[sheaves-]{sheaves}
\externaldocument[sites-]{sites}
\externaldocument[stacks-]{stacks}
\externaldocument[fields-]{fields}
\externaldocument[algebra-]{algebra}
\externaldocument[brauer-]{brauer}
\externaldocument[homology-]{homology}
\externaldocument[derived-]{derived}
\externaldocument[simplicial-]{simplicial}
\externaldocument[more-algebra-]{more-algebra}
\externaldocument[smoothing-]{smoothing}
\externaldocument[modules-]{modules}
\externaldocument[sites-modules-]{sites-modules}
\externaldocument[injectives-]{injectives}
\externaldocument[cohomology-]{cohomology}
\externaldocument[sites-cohomology-]{sites-cohomology}
\externaldocument[dga-]{dga}
\externaldocument[dpa-]{dpa}
\externaldocument[hypercovering-]{hypercovering}
\externaldocument[schemes-]{schemes}
\externaldocument[constructions-]{constructions}
\externaldocument[properties-]{properties}
\externaldocument[morphisms-]{morphisms}
\externaldocument[coherent-]{coherent}
\externaldocument[divisors-]{divisors}
\externaldocument[limits-]{limits}
\externaldocument[varieties-]{varieties}
\externaldocument[topologies-]{topologies}
\externaldocument[descent-]{descent}
\externaldocument[perfect-]{perfect}
\externaldocument[more-morphisms-]{more-morphisms}
\externaldocument[flat-]{flat}
\externaldocument[groupoids-]{groupoids}
\externaldocument[more-groupoids-]{more-groupoids}
\externaldocument[etale-]{etale}
\externaldocument[chow-]{chow}
\externaldocument[intersection-]{intersection}
\externaldocument[pic-]{pic}
\externaldocument[adequate-]{adequate}
\externaldocument[dualizing-]{dualizing}
\externaldocument[duality-]{duality}
\externaldocument[discriminant-]{discriminant}
\externaldocument[local-cohomology-]{local-cohomology}
\externaldocument[curves-]{curves}
\externaldocument[resolve-]{resolve}
\externaldocument[models-]{models}
\externaldocument[pione-]{pione}
\externaldocument[etale-cohomology-]{etale-cohomology}
\externaldocument[proetale-]{proetale}
\externaldocument[crystalline-]{crystalline}
\externaldocument[spaces-]{spaces}
\externaldocument[spaces-properties-]{spaces-properties}
\externaldocument[spaces-morphisms-]{spaces-morphisms}
\externaldocument[decent-spaces-]{decent-spaces}
\externaldocument[spaces-cohomology-]{spaces-cohomology}
\externaldocument[spaces-limits-]{spaces-limits}
\externaldocument[spaces-divisors-]{spaces-divisors}
\externaldocument[spaces-over-fields-]{spaces-over-fields}
\externaldocument[spaces-topologies-]{spaces-topologies}
\externaldocument[spaces-descent-]{spaces-descent}
\externaldocument[spaces-perfect-]{spaces-perfect}
\externaldocument[spaces-more-morphisms-]{spaces-more-morphisms}
\externaldocument[spaces-flat-]{spaces-flat}
\externaldocument[spaces-groupoids-]{spaces-groupoids}
\externaldocument[spaces-more-groupoids-]{spaces-more-groupoids}
\externaldocument[bootstrap-]{bootstrap}
\externaldocument[spaces-pushouts-]{spaces-pushouts}
\externaldocument[groupoids-quotients-]{groupoids-quotients}
\externaldocument[spaces-more-cohomology-]{spaces-more-cohomology}
\externaldocument[spaces-simplicial-]{spaces-simplicial}
\externaldocument[formal-spaces-]{formal-spaces}
\externaldocument[restricted-]{restricted}
\externaldocument[spaces-resolve-]{spaces-resolve}
\externaldocument[formal-defos-]{formal-defos}
\externaldocument[defos-]{defos}
\externaldocument[cotangent-]{cotangent}
\externaldocument[examples-defos-]{examples-defos}
\externaldocument[algebraic-]{algebraic}
\externaldocument[examples-stacks-]{examples-stacks}
\externaldocument[stacks-sheaves-]{stacks-sheaves}
\externaldocument[criteria-]{criteria}
\externaldocument[artin-]{artin}
\externaldocument[quot-]{quot}
\externaldocument[stacks-properties-]{stacks-properties}
\externaldocument[stacks-morphisms-]{stacks-morphisms}
\externaldocument[stacks-limits-]{stacks-limits}
\externaldocument[stacks-cohomology-]{stacks-cohomology}
\externaldocument[stacks-perfect-]{stacks-perfect}
\externaldocument[stacks-introduction-]{stacks-introduction}
\externaldocument[stacks-more-morphisms-]{stacks-more-morphisms}
\externaldocument[stacks-geometry-]{stacks-geometry}
\externaldocument[moduli-]{moduli}
\externaldocument[moduli-curves-]{moduli-curves}
\externaldocument[examples-]{examples}
\externaldocument[exercises-]{exercises}
\externaldocument[guide-]{guide}
\externaldocument[desirables-]{desirables}
\externaldocument[coding-]{coding}
\externaldocument[obsolete-]{obsolete}
\externaldocument[fdl-]{fdl}
\externaldocument[index-]{index}

% Theorem environments.
%
\theoremstyle{plain}
\newtheorem{theorem}[subsection]{Theorem}
\newtheorem{proposition}[subsection]{Proposition}
\newtheorem{lemma}[subsection]{Lemma}

\theoremstyle{definition}
\newtheorem{definition}[subsection]{Definition}
\newtheorem{example}[subsection]{Example}
\newtheorem{exercise}[subsection]{Exercise}
\newtheorem{situation}[subsection]{Situation}

\theoremstyle{remark}
\newtheorem{remark}[subsection]{Remark}
\newtheorem{remarks}[subsection]{Remarks}

\numberwithin{equation}{subsection}

% Macros
%
\def\lim{\mathop{\rm lim}\nolimits}
\def\colim{\mathop{\rm colim}\nolimits}
\def\Spec{\mathop{\rm Spec}}
\def\Hom{\mathop{\rm Hom}\nolimits}
\def\Ext{\mathop{\rm Ext}\nolimits}
\def\SheafHom{\mathop{\mathcal{H}\!{\it om}}\nolimits}
\def\SheafExt{\mathop{\mathcal{E}\!{\it xt}}\nolimits}
\def\Sch{\textit{Sch}}
\def\Mor{\mathop{\rm Mor}\nolimits}
\def\Ob{\mathop{\rm Ob}\nolimits}
\def\Sh{\mathop{\textit{Sh}}\nolimits}
\def\NL{\mathop{N\!L}\nolimits}
\def\proetale{{pro\text{-}\acute{e}tale}}
\def\etale{{\acute{e}tale}}
\def\QCoh{\textit{QCoh}}
\def\Ker{\mathop{\rm Ker}}
\def\Im{\mathop{\rm Im}}
\def\Coker{\mathop{\rm Coker}}
\def\Coim{\mathop{\rm Coim}}

%
% Macros for moduli stacks/spaces
%
\def\QCohstack{\mathcal{QC}\!{\it oh}}
\def\Cohstack{\mathcal{C}\!{\it oh}}
\def\Spacesstack{\mathcal{S}\!{\it paces}}
\def\Quotfunctor{{\rm Quot}}
\def\Hilbfunctor{{\rm Hilb}}
\def\Curvesstack{\mathcal{C}\!{\it urves}}
\def\Polarizedstack{\mathcal{P}\!{\it olarized}}
\def\Complexesstack{\mathcal{C}\!{\it omplexes}}
% \Pic is the operator that assigns to X its picard group, usage \Pic(X)
% \Picardstack_{X/B} denotes the Picard stack of X over B
% \Picardfunctor_{X/B} denotes the Picard functor of X over B
\def\Pic{\mathop{\rm Pic}\nolimits}
\def\Picardstack{\mathcal{P}\!{\it ic}}
\def\Picardfunctor{{\rm Pic}}
\def\Deformationcategory{\mathcal{D}\!{\it ef}}


% OK, start here.
%
\begin{document}

\title{Cohomology of Algebraic Spaces}

\maketitle

\phantomsection
\label{section-phantom}

\tableofcontents




\section{Introduction}
\label{section-introduction}

\noindent
In this chapter we write about cohomology of algebraic spaces.
This mean in particular cohomology of quasi-coherent sheaves, i.e.,
we prove analogues of the results in the chapter entitled
``Coherent Cohomology''. Some of the results in this chapter can
be found in \cite{knutson}.



\section{Derived category of quasi-coherent modules}
\label{section-derived-quasi-coherent}

\noindent
Let $S$ be a scheme. In
Descent, Lemma \ref{descent-lemma-derived-quasi-coherent-small-etale-site}
we proved that the category $D_{QCoh}(\mathcal{O}_S)$ can be defined
in terms of complexes of $\mathcal{O}_S$-modules on the scheme $S$
or by complexes of $\mathcal{O}$-modules on the small \'etale site
of $S$. Hence the following definition is compatible with the definition
in the case of schemes.

\begin{definition}
\label{definition-derived-quasi-coherent}
Let $S$ be a scheme. Let $X$ be an algebraic space over $S$.
The {\it derived category of $\mathcal{O}_X$-modules with
quasi-coherent cohomology sheaves} is denoted
$D_{QCoh}(\mathcal{O}_X)$.
\end{definition}

\noindent
This makes sense by
Properties of Spaces, Lemma
\ref{spaces-properties-lemma-properties-quasi-coherent}
and
Derived Categories, Section \ref{derived-section-triangulated-sub}.








\section{Higher direct images}
\label{section-higher-direct-image}

\noindent
Let $S$ be a scheme. Let $f : X \to Y$ be a quasi-compact and quasi-separated
morphism of representable algebraic spaces $X$ and $Y$ over $S$. Let
$\mathcal{F}$ be a quasi-coherent module on $X$. By
Descent, Lemma \ref{descent-lemma-higher-direct-images-small-etale}
the sheaf $R^if_*\mathcal{F}$ agrees with the
usual higher direct image (computed for the Zariski topologies)
if we think of $X$ and $Y$ as schemes.

\medskip\noindent
More generally, suppose $f : X \to Y$ is a representable, quasi-compact, and
quasi-separated morphism of algebraic spaces over $S$. Let $V$ be a scheme
and let $V \to Y$ be an \'etale surjective morphism. Let $U = V \times_Y X$
and let $f' : U \to V$ be the base change of $f$. Then for any
quasi-coherent $\mathcal{O}_X$-module $\mathcal{F}$ we have
\begin{equation}
\label{equation-representable-higher-direct-image}
R^if'_*(\mathcal{F}|_U) = (R^if_*\mathcal{F})|_V,
\end{equation}
see
Properties of Spaces,
Lemma \ref{spaces-properties-lemma-pushforward-etale-base-change-modules}.
And because $f' : U \to V$ is a quasi-compact and quasi-separated
morphism of schemes, by the remark of the preceding paragraph we may
compute $R^if'_*(\mathcal{F}|_U)$ by thinking of $\mathcal{F}|_U$ as a
quasi-coherent sheaf on the scheme $U$, and $f'$ as a morphism of schemes.
We will frequently use this without further mention.

\medskip\noindent
Next, we prove that higher direct images of quasi-coherent sheaves are
quasi-coherent for any quasi-compact and quasi-separated morphism of
algebraic spaces. In the proof we use a trick; a ``better'' proof would
use a relative Cech complex, as discussed in
Sheaves on Stacks, Sections \ref{stacks-sheaves-section-cech} and
\ref{stacks-sheaves-section-sheaf-cech-complex} ff.

\begin{lemma}
\label{lemma-higher-direct-image}
Let $S$ be a scheme. Let $f : X \to Y$ be a morphism of algebraic spaces
over $S$. If $f$ is quasi-compact and quasi-separated, then $R^if_*$
transforms quasi-coherent $\mathcal{O}_X$-modules into
quasi-coherent $\mathcal{O}_Y$-modules and induces a functor
$Rf_* : D_{QCoh}^+(\mathcal{O}_X) \to D_{QCoh}^+(\mathcal{O}_Y)$.
\end{lemma}

\begin{proof}
Let $V \to Y$ be an \'etale morphism where $V$ is an affine scheme. Set
$U = V \times_Y X$ and denote $f' : U \to V$ the induced morphism.
Let $\mathcal{I}^\bullet$ be a bounded above complex of injective
$\mathcal{O}_X$-modules. By
Properties of Spaces, Lemma
\ref{spaces-properties-lemma-pushforward-etale-base-change-modules}
we have
$$
f'_*(\mathcal{I}^\bullet|_U) = (f_*\mathcal{I}^\bullet)|_V.
$$
The complex $\mathcal{I}^\bullet|_U$ is a bounded below complex
of injective $\mathcal{O}_U$-modules, see
Cohomology on Sites, Lemma \ref{sites-cohomology-lemma-cohomology-of-open}.
Since the property of being a quasi-coherent module is local in the
\'etale topology on $Y$ (see
Properties of Spaces, Lemma
\ref{spaces-properties-lemma-characterize-quasi-coherent})
we may replace $Y$ by $V$, i.e., we may assume $Y$ is an affine scheme.

\medskip\noindent
Assume $Y$ is affine. Since $f$ is quasi-compact we see that $X$
is quasi-compact. Thus we may choose an affine scheme $U$ and a surjective
\'etale morphism $g : U \to X$, see
Properties of Spaces,
Lemma \ref{spaces-properties-lemma-quasi-compact-affine-cover}.
Note that the morphism $g : U \to X$ is representable, separated
and quasi-compact because $X$ is quasi-separated. Hence the lemma
holds for $g$ (either by the discussion above the lemma or by
applying the reduction in the first paragraph of this proof).
It also holds for $f \circ g : U \to Y$ (as this is a morphism
of affine schemes). Moreover, for an injective $\mathcal{O}_U$-module
$\mathcal{I}$ the module $g_*\mathcal{I}$ is injective (see
Homology, Lemma \ref{homology-lemma-adjoint-preserve-injectives})
whence $Rf_* \circ Rg_* = R(g \circ f)_*$, see
Derived Categories, Lemma \ref{derived-lemma-compose-derived-functors}.

\medskip\noindent
In the situation described in the previous paragraph we will show by
induction on $n$ that $IH_n$: for any quasi-coherent sheaf $\mathcal{F}$
on $X$ the sheaves $R^if\mathcal{F}$
are quasi-coherent for $i \leq n$.
The case $n = 0$ follows from
Morphisms of Spaces, Lemma \ref{spaces-morphisms-lemma-pushforward}.
Assume $IH_n$. In the rest of the proof we show that $IH_{n + 1}$ holds.

\medskip\noindent
The hypothesis $IH_n$ implies, via the spectral sequence of
Derived Categories, Lemma \ref{derived-lemma-two-ss-complex-functor},
that $R^if_*\mathcal{G}^\bullet$ is quasi-coherent for $i \leq n$ if
$\mathcal{G}^\bullet$ is a complex of $\mathcal{O}_X$-modules with
$H^j(\mathcal{G}^\bullet) = 0$ for $j < 0$ and $H^j(\mathcal{G}^\bullet)$
is quasi-coherent for all $j$. Suppose $\mathcal{H}$ is
a quasi-coherent $\mathcal{O}_U$-module. Consider the distinguished triangle
$$
g_*\mathcal{H} \to Rg_*\mathcal{H} \to
\tau_{\geq 1}Rg_*\mathcal{H} \to g_*\mathcal{H}[1].
$$
Note that $Rg_*\mathcal{H}$ and
$Rf_*Rg_*\mathcal{H} = R(f \circ g)_*\mathcal{H}$
have quasi-coherent cohomology sheaves (see above). Combined with
the remark above we conclude that $IH_n$ implies that
$R^if_*g_*\mathcal{H}$ is quasi-coherent for $i \leq n + 1$.

\medskip\noindent
Let $\mathcal{F}$ be a quasi-coherent $\mathcal{O}_X$-module. Consider
the exact sequence
$$
0 \to \mathcal{F} \to g_*g^*\mathcal{F} \to \mathcal{G} \to 0
$$
where $\mathcal{G}$ is the cokernel of the first map.
Applying the long exact cohomology sequence we obtain
$$
R^nf_*g_*g^*\mathcal{F} \to
R^nf_*\mathcal{G} \to
R^{n + 1}f_*\mathcal{F} \to
R^{n + 1}f_*g_*g^*\mathcal{F} \to
R^{n + 1}f_*\mathcal{G}
$$
By the above we see that $R^{n + 1}f_*g_*g^*\mathcal{F}$
is quasi-coherent. Thus $R^{n + 1}f_*\mathcal{F}$ has a $2$-step filtration
where the first step is quasi-coherent and the second a subsheaf of
a quasi-coherent sheaf. Applying this to $R^{n + 1}f_*\mathcal{G}$
we find an exact sequence $0 \to \mathcal{A} \to R^{n + 1}f_*\mathcal{G}
\to \mathcal{B}$ wit $\mathcal{A}$, $\mathcal{B}$ quasi-coherent
$\mathcal{O}_Y$-modules. Then the kernel $\mathcal{K}$ of
$R^{n + 1}f_*g_*g^*\mathcal{F} \to R^{n + 1}f_*\mathcal{G}
\to \mathcal{B}$ is quasi-coherent, whereupon we obtain a map
$\mathcal{K} \to \mathcal{A}$ whose kernel $\mathcal{K}'$ is
quasi-coherent too. Hence $R^{n + 1}f_*\mathcal{F}$ sits in an exact
sequence
$$
R^nf_*g_*g^*\mathcal{F} \to
R^nf_*\mathcal{G} \to
R^{n + 1}f_*\mathcal{F} \to \mathcal{K}' \to 0
$$
and we win.
\end{proof}





\section{Other chapters}

\begin{multicols}{2}
\begin{enumerate}
\item \hyperref[introduction-section-phantom]{Introduction}
\item \hyperref[conventions-section-phantom]{Conventions}
\item \hyperref[sets-section-phantom]{Set Theory}
\item \hyperref[categories-section-phantom]{Categories}
\item \hyperref[topology-section-phantom]{Topology}
\item \hyperref[sheaves-section-phantom]{Sheaves on Spaces}
\item \hyperref[algebra-section-phantom]{Commutative Algebra}
\item \hyperref[sites-section-phantom]{Sites and Sheaves}
\item \hyperref[homology-section-phantom]{Homological Algebra}
\item \hyperref[derived-section-phantom]{Derived Categories}
\item \hyperref[more-algebra-section-phantom]{More Algebra}
\item \hyperref[simplicial-section-phantom]{Simplicial Methods}
\item \hyperref[modules-section-phantom]{Sheaves of Modules}
\item \hyperref[sites-modules-section-phantom]{Modules on Sites}
\item \hyperref[injectives-section-phantom]{Injectives}
\item \hyperref[cohomology-section-phantom]{Cohomology of Sheaves}
\item \hyperref[sites-cohomology-section-phantom]{Cohomology on Sites}
\item \hyperref[hypercovering-section-phantom]{Hypercoverings}
\item \hyperref[schemes-section-phantom]{Schemes}
\item \hyperref[constructions-section-phantom]{Constructions of Schemes}
\item \hyperref[properties-section-phantom]{Properties of Schemes}
\item \hyperref[morphisms-section-phantom]{Morphisms of Schemes}
\item \hyperref[coherent-section-phantom]{Coherent Cohomology}
\item \hyperref[divisors-section-phantom]{Divisors}
\item \hyperref[limits-section-phantom]{Limits of Schemes}
\item \hyperref[varieties-section-phantom]{Varieties}
\item \hyperref[chow-section-phantom]{Chow Homology}
\item \hyperref[topologies-section-phantom]{Topologies on Schemes}
\item \hyperref[descent-section-phantom]{Descent}
\item \hyperref[more-morphisms-section-phantom]{More on Morphisms}
\item \hyperref[flat-section-phantom]{More on Flatness}
\item \hyperref[groupoids-section-phantom]{Groupoid Schemes}
\item \hyperref[more-groupoids-section-phantom]{More on Groupoid Schemes}
\item \hyperref[etale-section-phantom]{\'Etale Morphisms of Schemes}
\item \hyperref[etale-cohomology-section-phantom]{\'Etale Cohomology}
\item \hyperref[spaces-section-phantom]{Algebraic Spaces}
\item \hyperref[spaces-properties-section-phantom]{Properties of Algebraic Spaces}
\item \hyperref[spaces-morphisms-section-phantom]{Morphisms of Algebraic Spaces}
\item \hyperref[spaces-topologies-section-phantom]{Topologies on Algebraic Spaces}
\item \hyperref[spaces-descent-section-phantom]{Descent and Algebraic Spaces}
\item \hyperref[spaces-more-morphisms-section-phantom]{More on Morphisms of Spaces}
\item \hyperref[quot-section-phantom]{Quot and Hilbert Spaces}
\item \hyperref[stacks-section-phantom]{Stacks}
\item \hyperref[spaces-groupoids-section-phantom]{Groupoids in Algebraic Spaces}
\item \hyperref[spaces-more-groupoids-section-phantom]{More on Groupoids in Spaces}
\item \hyperref[bootstrap-section-phantom]{Bootstrap}
\item \hyperref[examples-stacks-section-phantom]{Examples of Stacks}
\item \hyperref[groupoids-quotients-section-phantom]{Quotients of Groupoids}
\item \hyperref[algebraic-section-phantom]{Algebraic Stacks}
\item \hyperref[criteria-section-phantom]{Criteria for Representability}
\item \hyperref[stacks-properties-section-phantom]{Properties of Algebraic Stacks}
\item \hyperref[stacks-morphisms-section-phantom]{Morphisms of Algebraic Stacks}
\item \hyperref[examples-section-phantom]{Examples}
\item \hyperref[exercises-section-phantom]{Exercises}
\item \hyperref[guide-section-phantom]{Guide to Literature}
\item \hyperref[desirables-section-phantom]{Desirables}
\item \hyperref[coding-section-phantom]{Coding Style}
\item \hyperref[fdl-section-phantom]{GNU Free Documentation License}
\item \hyperref[index-section-phantom]{Auto Generated Index}
\end{enumerate}
\end{multicols}


\bibliography{my}
\bibliographystyle{amsalpha}

\end{document}
