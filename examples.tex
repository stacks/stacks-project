\IfFileExists{stacks-project.cls}{%
\documentclass{stacks-project}
}{%
\documentclass{amsart}
}

% The following AMS packages are automatically loaded with
% the amsart documentclass:
%\usepackage{amsmath}
%\usepackage{amssymb}
%\usepackage{amsthm}

% For dealing with references we use the comment environment
\usepackage{verbatim}
\newenvironment{reference}{\comment}{\endcomment}
%\newenvironment{reference}{}{}
\newenvironment{slogan}{\comment}{\endcomment}
\newenvironment{history}{\comment}{\endcomment}

% For commutative diagrams you can use
% \usepackage{amscd}
\usepackage[all]{xy}

% We use 2cell for 2-commutative diagrams.
\xyoption{2cell}
\UseAllTwocells

% To put source file link in headers.
% Change "template.tex" to "this_filename.tex"
% \usepackage{fancyhdr}
% \pagestyle{fancy}
% \lhead{}
% \chead{}
% \rhead{Source file: \url{template.tex}}
% \lfoot{}
% \cfoot{\thepage}
% \rfoot{}
% \renewcommand{\headrulewidth}{0pt}
% \renewcommand{\footrulewidth}{0pt}
% \renewcommand{\headheight}{12pt}

\usepackage{multicol}

% For cross-file-references
\usepackage{xr-hyper}

% Package for hypertext links:
\usepackage{hyperref}

% For any local file, say "hello.tex" you want to link to please
% use \externaldocument[hello-]{hello}
\externaldocument[introduction-]{introduction}
\externaldocument[conventions-]{conventions}
\externaldocument[sets-]{sets}
\externaldocument[categories-]{categories}
\externaldocument[topology-]{topology}
\externaldocument[sheaves-]{sheaves}
\externaldocument[sites-]{sites}
\externaldocument[stacks-]{stacks}
\externaldocument[fields-]{fields}
\externaldocument[algebra-]{algebra}
\externaldocument[brauer-]{brauer}
\externaldocument[homology-]{homology}
\externaldocument[derived-]{derived}
\externaldocument[simplicial-]{simplicial}
\externaldocument[more-algebra-]{more-algebra}
\externaldocument[smoothing-]{smoothing}
\externaldocument[modules-]{modules}
\externaldocument[sites-modules-]{sites-modules}
\externaldocument[injectives-]{injectives}
\externaldocument[cohomology-]{cohomology}
\externaldocument[sites-cohomology-]{sites-cohomology}
\externaldocument[dga-]{dga}
\externaldocument[dpa-]{dpa}
\externaldocument[hypercovering-]{hypercovering}
\externaldocument[schemes-]{schemes}
\externaldocument[constructions-]{constructions}
\externaldocument[properties-]{properties}
\externaldocument[morphisms-]{morphisms}
\externaldocument[coherent-]{coherent}
\externaldocument[divisors-]{divisors}
\externaldocument[limits-]{limits}
\externaldocument[varieties-]{varieties}
\externaldocument[topologies-]{topologies}
\externaldocument[descent-]{descent}
\externaldocument[perfect-]{perfect}
\externaldocument[more-morphisms-]{more-morphisms}
\externaldocument[flat-]{flat}
\externaldocument[groupoids-]{groupoids}
\externaldocument[more-groupoids-]{more-groupoids}
\externaldocument[etale-]{etale}
\externaldocument[chow-]{chow}
\externaldocument[intersection-]{intersection}
\externaldocument[pic-]{pic}
\externaldocument[adequate-]{adequate}
\externaldocument[dualizing-]{dualizing}
\externaldocument[duality-]{duality}
\externaldocument[discriminant-]{discriminant}
\externaldocument[local-cohomology-]{local-cohomology}
\externaldocument[curves-]{curves}
\externaldocument[resolve-]{resolve}
\externaldocument[models-]{models}
\externaldocument[pione-]{pione}
\externaldocument[etale-cohomology-]{etale-cohomology}
\externaldocument[proetale-]{proetale}
\externaldocument[crystalline-]{crystalline}
\externaldocument[spaces-]{spaces}
\externaldocument[spaces-properties-]{spaces-properties}
\externaldocument[spaces-morphisms-]{spaces-morphisms}
\externaldocument[decent-spaces-]{decent-spaces}
\externaldocument[spaces-cohomology-]{spaces-cohomology}
\externaldocument[spaces-limits-]{spaces-limits}
\externaldocument[spaces-divisors-]{spaces-divisors}
\externaldocument[spaces-over-fields-]{spaces-over-fields}
\externaldocument[spaces-topologies-]{spaces-topologies}
\externaldocument[spaces-descent-]{spaces-descent}
\externaldocument[spaces-perfect-]{spaces-perfect}
\externaldocument[spaces-more-morphisms-]{spaces-more-morphisms}
\externaldocument[spaces-flat-]{spaces-flat}
\externaldocument[spaces-groupoids-]{spaces-groupoids}
\externaldocument[spaces-more-groupoids-]{spaces-more-groupoids}
\externaldocument[bootstrap-]{bootstrap}
\externaldocument[spaces-pushouts-]{spaces-pushouts}
\externaldocument[groupoids-quotients-]{groupoids-quotients}
\externaldocument[spaces-more-cohomology-]{spaces-more-cohomology}
\externaldocument[spaces-simplicial-]{spaces-simplicial}
\externaldocument[formal-spaces-]{formal-spaces}
\externaldocument[restricted-]{restricted}
\externaldocument[spaces-resolve-]{spaces-resolve}
\externaldocument[formal-defos-]{formal-defos}
\externaldocument[defos-]{defos}
\externaldocument[cotangent-]{cotangent}
\externaldocument[examples-defos-]{examples-defos}
\externaldocument[algebraic-]{algebraic}
\externaldocument[examples-stacks-]{examples-stacks}
\externaldocument[stacks-sheaves-]{stacks-sheaves}
\externaldocument[criteria-]{criteria}
\externaldocument[artin-]{artin}
\externaldocument[quot-]{quot}
\externaldocument[stacks-properties-]{stacks-properties}
\externaldocument[stacks-morphisms-]{stacks-morphisms}
\externaldocument[stacks-limits-]{stacks-limits}
\externaldocument[stacks-cohomology-]{stacks-cohomology}
\externaldocument[stacks-perfect-]{stacks-perfect}
\externaldocument[stacks-introduction-]{stacks-introduction}
\externaldocument[stacks-more-morphisms-]{stacks-more-morphisms}
\externaldocument[stacks-geometry-]{stacks-geometry}
\externaldocument[moduli-]{moduli}
\externaldocument[moduli-curves-]{moduli-curves}
\externaldocument[examples-]{examples}
\externaldocument[exercises-]{exercises}
\externaldocument[guide-]{guide}
\externaldocument[desirables-]{desirables}
\externaldocument[coding-]{coding}
\externaldocument[obsolete-]{obsolete}
\externaldocument[fdl-]{fdl}
\externaldocument[index-]{index}

% Theorem environments.
%
\theoremstyle{plain}
\newtheorem{theorem}[subsection]{Theorem}
\newtheorem{proposition}[subsection]{Proposition}
\newtheorem{lemma}[subsection]{Lemma}

\theoremstyle{definition}
\newtheorem{definition}[subsection]{Definition}
\newtheorem{example}[subsection]{Example}
\newtheorem{exercise}[subsection]{Exercise}
\newtheorem{situation}[subsection]{Situation}

\theoremstyle{remark}
\newtheorem{remark}[subsection]{Remark}
\newtheorem{remarks}[subsection]{Remarks}

\numberwithin{equation}{subsection}

% Macros
%
\def\lim{\mathop{\rm lim}\nolimits}
\def\colim{\mathop{\rm colim}\nolimits}
\def\Spec{\mathop{\rm Spec}}
\def\Hom{\mathop{\rm Hom}\nolimits}
\def\Ext{\mathop{\rm Ext}\nolimits}
\def\SheafHom{\mathop{\mathcal{H}\!{\it om}}\nolimits}
\def\SheafExt{\mathop{\mathcal{E}\!{\it xt}}\nolimits}
\def\Sch{\textit{Sch}}
\def\Mor{\mathop{\rm Mor}\nolimits}
\def\Ob{\mathop{\rm Ob}\nolimits}
\def\Sh{\mathop{\textit{Sh}}\nolimits}
\def\NL{\mathop{N\!L}\nolimits}
\def\proetale{{pro\text{-}\acute{e}tale}}
\def\etale{{\acute{e}tale}}
\def\QCoh{\textit{QCoh}}
\def\Ker{\mathop{\rm Ker}}
\def\Im{\mathop{\rm Im}}
\def\Coker{\mathop{\rm Coker}}
\def\Coim{\mathop{\rm Coim}}

%
% Macros for moduli stacks/spaces
%
\def\QCohstack{\mathcal{QC}\!{\it oh}}
\def\Cohstack{\mathcal{C}\!{\it oh}}
\def\Spacesstack{\mathcal{S}\!{\it paces}}
\def\Quotfunctor{{\rm Quot}}
\def\Hilbfunctor{{\rm Hilb}}
\def\Curvesstack{\mathcal{C}\!{\it urves}}
\def\Polarizedstack{\mathcal{P}\!{\it olarized}}
\def\Complexesstack{\mathcal{C}\!{\it omplexes}}
% \Pic is the operator that assigns to X its picard group, usage \Pic(X)
% \Picardstack_{X/B} denotes the Picard stack of X over B
% \Picardfunctor_{X/B} denotes the Picard functor of X over B
\def\Pic{\mathop{\rm Pic}\nolimits}
\def\Picardstack{\mathcal{P}\!{\it ic}}
\def\Picardfunctor{{\rm Pic}}
\def\Deformationcategory{\mathcal{D}\!{\it ef}}


% OK, start here.
%
\begin{document}

\title{Examples}


\maketitle

\phantomsection
\label{section-phantom}

\tableofcontents

\section{Introduction}
\label{section-introduction}

\noindent
This chapters will contain examples which illuminate the theory.


\section{Noncomplete completion}
\label{section-noncomplete-completion}

\noindent
This example is taken from an upublished note of
Bart de Smit and Hendrik Lenstra.
Let $R = k[x_1, x_2, x_3, \ldots]$.
Let $\mathfrak m = (x_1, x_2, x_3, \ldots)$.
Let $R^\wedge = \lim R/\mathfrak m^n$ be the completion of $R$ with
respect to $\mathfrak m$. Note that an element $f$ of $R^\wedge$
can be thought of as an expression
$$
f = \sum a_I x^I
$$
in multinomial notation such that for each $d \geq 0$ there
are only finitely many nonzero $a_I$ for $|I| = d$.
To show that $R^\wedge$ is not complete it suffices to
show that $K_2 = \text{Ker}(R^\wedge \to R/\mathfrak m^2)$
is not equal to $\mathfrak m^2R^\wedge = (\mathfrak m^\wedge)^2$, see
Algebra, Lemma \ref{algebra-lemma-hathat}.
Note that an element of $(\mathfrak m^\wedge)^2$ is a finite sum
\begin{equation}
\label{equation-sum}
\sum\nolimits_{i = 1, \ldots, t} f_i g_i
\end{equation}
with $f_i, g_i \in R^\wedge$ having vanishing constant terms.
To get an example we are going to choose an $z \in K_2$
of the form
$$
z = z_1 + z_2 + z_3 + \ldots
$$
with the following properties
\begin{enumerate}
\item there exist sequences of integers
$1 < d_1 < d_2 < d_3 < \ldots $ and
$0 < n_1 < n_2 < n_3 < \ldots$ such that
$z_i \in k[x_{n_i}, x_{n_i + 1}, \ldots, x_{n_{i + 1} - 1}]$
homogeneous of degree $d_i$,
\item in the ring $k[[x_{n_i}, x_{n_i + 1}, \ldots, x_{n_{i + 1} - 1}]]$
the element $z_i$ cannot be written as a sum (\ref{equation-sum})
with $t \leq i$.
\end{enumerate}
Clearly this implies that $z$ is not in $(\mathfrak m^\wedge)^2$
because the image of the relation (\ref{equation-sum}) in the
ring $k[[x_{n_i}, x_{n_i + 1}, \ldots, x_{n_{i + 1} - 1}]]$
for $i$ large enough would produce a contradiction. Hence it suffices
to prove that for all $t > 0$ there exists a $d \gg 0$ and an integer
$n$ such that we can find an homogeneous element
$z \in k[x_1, \ldots, x_n]$ of degree $d$ which cannot be written as
a sum (\ref{equation-sum}) for the given $t$ in $k[[x_1, \ldots, x_n]]$.
Take $n > 2t$ and any $d > 1$ prime to the characteristic of $p$ and
set $z = \sum_{i = 1, \ldots, n} x_i^d$. Then the vanishing locus
of the ideal
$$
(\frac{\partial z}{\partial x_1}, \ldots, \frac{\partial z}{\partial x_n})
=
(dx_1^{d - 1}, \ldots, dx_n^{d - 1})
$$
consists of one point. On the other hand,
$$
\frac{\partial ( \sum\nolimits_{i = 1, \ldots, t} f_i g_i ) }{\partial x_j}
\in (f_1, \ldots, f_t, g_1, \ldots, g_t)
$$
by the Leibniz rule and hence the vanishing locus of these derivatives
contains at least
$$
V(f_1, \ldots, f_t, g_1, \ldots, g_t) \subset
\text{Spec}(k[[x_1, \ldots, x_n]]).
$$
Hence this is a contradition as the dimension of
$V(f_1, \ldots, f_t, g_1, \ldots, g_t)$ is at least $n - 2t \geq 1$.

\begin{lemma}
\label{lemma-noncomplete-completion}
There exists a local ring $R$ and a maximal ideal $\mathfrak m$ such that
the completion of $R$ with respect to $\mathfrak m$ is not a complete
local ring.
\end{lemma}

\begin{proof}
This follows from the above as the completion of
the localization $R_{\mathfrak m}$ is equal to the completion of $R$
considered above.
\end{proof}





\section{Noncomplete quotient}
\label{section-noncomplete-quotient}

\noindent
Let $k$ be a field. Let
$$
R = k[t, z_1, z_2, z_3, \ldots, w_1, w_2, w_3, \ldots, x]/
(z_it - x^iw_i, z_i w_j)
$$
Note that in particular $z_iz_jt = 0$ in this ring. Any element $f$ of $R$
can be uniquely written as a finite sum
$$
f = \sum\nolimits_{i = 0, \ldots, d} f_i x^i
$$
where each $f_i \in k[t, z_i, w_j]$ has no terms involving the products
$z_it$ or $z_iw_j$. Moreover, if $f$ is written in this way, then
$f \in (x^n)$ if and only if $f_i = 0$ for $i < n$.
So $x$ is a nonzero divisor and $\bigcap (x^n) = 0$.
Let $R^\wedge$ be the completion of $R$ with respect to the ideal $(x)$.
Note that $R^\wedge$ is $(x)$-adically complete, see
Algebra, Lemma \ref{algebra-lemma-hathat-finitely-generated}.
By the above we see that an element of $R^\wedge$ can be uniquely written
as an infinite sum
$$
f = \sum\nolimits_{i = 0}^\infty f_i x^i
$$
where each $f_i \in k[t, z_i, w_j]$ has no terms involving the products
$z_it$ or $z_iw_j$. Consider the element
$$
f = \sum\nolimits_{i = 1}^\infty x^{i - 1} w_i =
xw_1 + x^2w_2 + x^3w_3 + \ldots
$$
i.e., we have $f_n = w_n$. Note that $f \in (t , x^n)$ for every $n$
because $x^mw_m \in (t)$ for all $m$.
We claim that $f \not \in (t)$. To prove this assume that
$tg = f$ where $g = \sum g_lx^l$ in canonical form as above.
Since $tz_iz_j = 0$ we may as well assume that none of the $g_l$ have
terms involving the products $z_iz_j$. Examining the process to
get $tg$ in canonical form we see the following:
Given any term $c m$ of $g_l$ where $c \in k$ and $m$ is a
monomial in $t, z_i, w_j$ and we make the following replacement
\begin{enumerate}
\item if the monomial $m$ does not involve any $z_i$, then $ctm$ is
a term of $f_l$, and
\item if the monomial $m$ does involve a $z_i$ then it is eqal to
$m = z_i$ and we see that $cw_i$ is term of $f_{l + i}$.
\end{enumerate}
Since $g_0$ is a polynomial only finitely many of the variables $z_i$
occur in it. Pick $n$ such that $z_n$ does not occur in $g_0$.
Then the rules above show that $w_n$ does not occur in $f_n$ which is
a contradiction. It follows that $R^\wedge/(t)$ is not complete, see
Algebra, Lemma \ref{algebra-lemma-quotient-complete}.

\begin{lemma}
\label{lemma-noncomplete-quotient}
There exists a ring $R$ complete with respect to a principal ideal
$I$ and a principal ideal $J$ such that $R/J$ is not $I$-adically
complete.
\end{lemma}

\begin{proof}
See discussion above.
\end{proof}



\section{Completion is not exact}
\label{section-completion-not-exact}

\noindent
A quick example is the following. Suppose that $R = k[t]$. Let
$P = K = \bigoplus_{n \in \mathbf{N}} R$ and
$M = \bigoplus_{n \in \mathbf{N}} R/(t^n)$. Then there is a short exact
sequence $0 \to K \to P \to M \to 0$ where the first map is given by
multiplication by $t^n$ on the $n$th summand. We claim that
$0 \to K^\wedge \to P^\wedge \to M^\wedge \to 0$ is not exact in the middle.
Namely, $\xi = (t^2, t^3, t^4, \ldots) \in P^\wedge$ maps to zero in
$M^\wedge$ but is not in the image of $K^\wedge \to P^\wedge$, because
it would be the image of $(t, t, t, \ldots)$ which is not an element of
$K^\wedge$.

\medskip\noindent
A ``smaller'' example is the following. In the situation of
Lemma \ref{lemma-noncomplete-quotient}
the short exact sequence $0 \to J \to R \to R/J \to 0$ does not remain
exact after completion. Namely, if $f \in J$ is a generator, then
$f : R \to J$ is surjective, hence $R \to J^\wedge$ is surjective, hence
the image of $J^\wedge \to R$ is $(f) = J$ but the fact that
$R/J$ is noncomplete means that the kernel of the surjection
$R \to (R/J)^\wedge$ is strictly bigger than $J$, see
Algebra, Lemmas \ref{algebra-lemma-completion-generalities} and
\ref{algebra-lemma-quotient-complete}.
By the same token the sequence
$R \to R \to R/(f) \to 0$ does not remain exact on completion.

\begin{lemma}
\label{lemma-completion-not-exact}
Completion is not an exact functor in general; it is not even
right exact in general. This holds even when $I$ is finitely
generated on the category of finitely presented modules.
\end{lemma}

\begin{proof}
See discussion above.
\end{proof}


\section{A Noetherian ring of infinite dimension}
\label{section-Noetherian-infinite-dimension}

\noindent
A Noetherian local ring has finite dimension as we saw in
Algebra, Proprosition \ref{algebra-proposition-dimension}.
But there exist Noetherian rings of infinite dimension.
See \cite[Appendix, Example 1]{Nagata}.

\medskip\noindent
Namely, let $k$ be a field, and consider the ring
$$
R = k[x_1, x_2, x_3, \ldots ].
$$
Let $\mathfrak p_i = (x_{2^{i - 1}}, x_{2^{i - 1} + 1}, \ldots, x_{2^i - 1})$
for $i = 1, 2, \ldots$ which are prime ideals of $R$. Let $S$
be the multiplicative subset
$$
S = \bigcap\nolimits_{i \geq 1} (R \setminus \mathfrak p_i).
$$
Consider the ring $A = S^{-1}R$.
We claim that
\begin{enumerate}
\item The maximal ideals of the ring $A$ are the ideals
$\mathfrak m_i = \mathfrak p_iA$.
\item We have $A_{\mathfrak m_i} = R_{\mathfrak p_i}$ which is
a Noetherian local ring of dimension $2^i$.
\item The ring $A$ is Noetherian.
\end{enumerate}
Hence it is clear that this is the example we are looking for.
Details omitted.




\section{Local rings with nonreduced completion}
\label{section-local-completion-nonreduced}

\noindent
In Algebra, Example \ref{algebra-example-bad-dvr-char-p} we gave an example
of a characteristic $p$ Noetherian local domain $R$ of dimension $1$
whose completion is nonreduced. In this section we present the example
of \cite[Proposition 3.1]{Ferrand-Raynaud} which gives a similar
ring in characteristic zero.

\medskip\noindent
Let $\mathbf{C}\{x\}$ be the ring of convergent power series over
the field $\mathbf{C}$ of complex numbers. The ring of all power series
$\mathbf{C}[[x]]$ is its completion. Let $K = \mathbf{C}\{x\}[1/x] = f.f.(B)$
be the field of convergent Laurent series. The $K$-module
$\Omega_{K/\mathbf{C}}$ of algebraic differentials
of $K$ over $\mathbf{C}$ is an infinite dimensional $K$-vector space
(proof omitted). We may choose $f_n \in x\mathbf{C}\{x\}$,
$n \geq 1$ such that
$
\text{d}x, \text{d}f_1, \text{d}f_2, \ldots
$
are part of a basis of $\Omega_{K/\mathbf{C}}$. Thus we can
find a $\mathbf{C}$-derivation
$$
D : \mathbf{C}\{x\} \longrightarrow \mathbf{C}((x))
$$
such that $D(x) = 0$ and $D(f_i) = x^{-n}$. Let
$$
A = \{f \in \mathbf{C}\{x\} \mid D(f) \in \mathbf{C}[[x]]\}
$$
We claim that
\begin{enumerate}
\item $\mathbf{C}\{x\}$ is integral over $A$,
\item $A$ is a local domain,
\item $\dim(A) = 1$,
\item the maximal ideal of $A$ is generated by $x$ and $xf_1$,
\item $A$ is Noetherian, and
\item the completion of $A$ is equal to the ring of dual numbers
over $\mathbf{C}[[x]]$.
\end{enumerate}
Since the dual numbers are nonreduced the ring $A$ gives the example.

\medskip\noindent
Note that if $0 \not = f \in x\mathbf{C}\{x\}$ then
we may write $D(f) = h/f^n$ for some $n \geq 0$ and $h \in \mathbf{C}[[x]]$.
Hence $D(f^{n + 1}/(n + 1)) \in \mathbf{C}[[x]]$
and $D(f^{n + 2}/(n + 2)) \in \mathbf{C}[[x]]$. Thus we
see $f^{n + 1}, f^{n + 2} \in A$!
In particular we see (1) holds. We also conclude that
the fraction field of $A$ is equal to the fraction field of
$\mathbf{C}\{x\}$. It also follows immediately that
$A \cap x\mathbf{C}\{x\}$ is the set of nonunits of $A$, hence
$A$ is a local domain of dimension $1$. If we can show (4)
then it will follow that $A$ is Noetherian (proof omitted).
Suppose that $f \in A \cap x\mathbf{C}\{x\}$. Write
$D(f) = h$, $h \in \mathbf{C}[[x]]$. Write $h = c + xh'$
with $c \in \mathbf{C}$, $h' \in \mathbf{C}[[x]]$. Then
$D(f - cxf_1) = c + xh' - c = xh'$. On the other hand
$f - cxf_1 = xg$ with $g \in \mathbf{C}\{x\}$, but by the
computation above we have $D(g) = h' \in \mathbf{C}[[x]]$
and hence $g \in A$. Thus $f = cxf_1 + xg \in (x, xf_1)$ as desired.

\medskip\noindent
Finally, why is the completion of $A$ nonreduced? Denote $\hat A$ the
completion of $A$. Of course this maps surjectively to the completion
$\mathbf{C}[[x]]$ of $\mathbf{C}\{x\}$ because $x \in A$. Denote
this map $\psi : \hat A \to \mathbf{C}[[x]]$.
Above we saw that $\mathfrak m_A = (x, xf_1)$
and hence $D(\mathfrak m_A^n) \subset (x^{n - 1})$ by an easy
computation. Thus $D : A \to \mathbf{C}[[x]]$ is continuous and
gives rise to a continuous derivation $\hat D : \hat A \to \mathbf{C}[[x]]$
over $\psi$. Hence we get a ring map
$$
\psi + \epsilon \hat D :
\hat A
\longrightarrow
\mathbf{C}[[x]][\epsilon].
$$
Since $\hat A$ is a one dimensional Noetherian complete local ring, if we
can show this arrow is surjective then it will follow that $\hat A$
is nonreduced. Actually the map is an isomorphism but we omit the
verification of this. The subring $\mathbf{C}[x]_{(x)} \subset A$
gives rise to a map $i : \mathbf{C}[[x]] \to \hat A$ on completions such
that $i \circ \psi = \text{id}$ and such that $D \circ i = 0$
(as $D(x) = 0$ by construction). Consider the elements $x^nf_n \in A$.
We have
$$
(\psi + \epsilon D)(x^nf_n) = x^n f_n + \epsilon
$$
for all $n \geq 1$. Surjectivity easily follows from these remarks.




\section{A non catenary Noetherian local ring}
\label{section-non-catenary-Noetherian-local}

\noindent
Even though there is a succesful dimension theory of Noetherian local rings
there are non-catenary Noetherian local rings. An example may be found in
\cite[Appendix, Example 2]{Nagata}. In fact, we will present this example
in the simplest case. Namely, we will construct a local Noetherian domain $A$
of dimension $2$ which is not universally catenary. (Note that $A$ is
automatically catenary, see
Exercise \ref{exercises-exercise-Noetherian-local-domain-dim-2-catenary}.)
The existence of a Noetherian local ring which is not universally
catenary implies the existence of a Noetherian local ring which
is not catenary -- and we spell this out at the end of this section
in the particular example at hand.

\medskip\noindent
Let $k$ be a field, and consider the formal power series ring
$k[[x]]$ in one variable over $k$. Let
$$
z = \sum\nolimits_{i = 1}^\infty a_i x^i
$$
be a formal power series. We assume $z$ as an element of the Laurent
series field $k((x)) = f.f.(k[[x]])$ is transcendental over $k(x)$.
Put
$$
z_j
=
x^{-j}(z - \sum\nolimits_{i = 1, \ldots, j - 1} a_i x^i)
=
\sum\nolimits_{i = j}^\infty a_i x^{i - j}
\in k[[x]].
$$
Note that $Z = z_1$.
Let $R$ be the subring of $k[[x]]$ generated by $x$, $z$ and all of the
$z_j$, in other words
$$
R = k[x, z_1, z_2, z_3, \ldots ] \subset k[[x]].
$$
Consider the ideals $\mathfrak m = (x)$ and
$\mathfrak n = (x - 1, z_1, z_2, \ldots)$ of $R$.

\medskip\noindent
We have $x(z_{j + 1} + a_j) = z_j$. Hence $R/\mathfrak m = k$
and $\mathfrak m$ is a maximal ideal. Moreover, any element of $R$
not in $\mathfrak m$ maps to a unit in $k[[x]]$ and hence
$R_{\mathfrak m} \subset k[[x]]$. In fact it is easy to deduce
that $R_{\mathfrak m}$ is a discrete valuation ring and residue
field $k$.

\medskip\noindent
We claim that
$$
R/(x - 1) =
k[x, z_1, z_2, z_3, \ldots ]/(x - 1)
\cong
k[z].
$$
Namely, the relation above implies that
$(x - 1)(z_{j + 1} + a_j) = -z_{j + 1} - a_j + z_j$, and hence
we may express the class of $z_{j + 1}$ in terms of $z_j$ in
the quotient $R/(x - 1)$. Since the fraction field of $R$
has transcendence degree $2$ over $k$ by construction we see that $z$ is
transcendental over $k$ in $R/(x - 1)$, whence the desired isomorphism.
Hence $\mathfrak n = (x - 1, z)$ and is a maximal ideal. In fact the
map
$$
k[x, x^{-1}, z]_{(x - 1, z)} \longrightarrow R_{\mathfrak n}
$$
is an isomorphism (since $x^{-1}$ is invertible in $R_{\mathfrak n}$
and since $z_{j + 1} = x^{-1}z_j - a_j = \ldots = f_j(x, x^{-1}, z)$).
This shows that $R_{\mathfrak n}$ is a regular local ring
of dimension $2$ and residue field $k$.

\medskip\noindent
Let $S$ be the multiplicative subset
$$
S =
(R \setminus \mathfrak m) \cap (R \setminus \mathfrak n) =
R \setminus (\mathfrak m \cup \mathfrak n)
$$
and set $B = S^{-1}R$. We claim that
\begin{enumerate}
\item The ring $B$ is a $k$-algebra.
\item The maximal ideals of the ring $B$ are the two ideals
$\mathfrak mB$ and $\mathfrak nB$.
\item The residue fields at these maximal ideals is $k$.
\item We have $B_{\mathfrak mB} = R_{\mathfrak m}$
and $B_{\mathfrak nB} = R_{\mathfrak n}$
which are Noetherian regular local rings of dimensions $1$ and $2$.
\item The ring $B$ is Noetherian.
\end{enumerate}
We omit the details of the verifications.

\medskip\noindent
Whenever given a $k$-algebra $B$ with the properties listed above we
get an example as follows. Take $A = k + \text{rad}(B) \subset B$,
in our case $\text{rad}(B) = \mathfrak mB + \mathfrak nB$.
It is easy to see that $B$ is finite over $A$ and hence $A$ is
Noetherian by Eakin's theorem (see \cite{Eakin}, or
\cite[Appendix A1]{Nagata}, or insert future reference here).
Also $A$ is a local domain with the same fraction field as $B$ and
residue field $k$. Since the dimension of $B$ is $2$ we see that $A$
has dimension $2$ as well, by
Algebra, Lemma \ref{algebra-lemma-integral-sub-dim-equal}.

\medskip\noindent
If $A$ were universally catenary then the dimension formula,
Algebra, Lemma \ref{algebra-lemma-dimension-formula}
would give $\dim(B_{\mathfrak mB}) = 2$ contradiction.

\medskip\noindent
Note that $B$ is generated by one element over $A$.
Hence $B = A[x]/\mathfrak p$ for some prime
$\mathfrak p$ of $A[x]$. Let $\mathfrak m' \subset A[x]$ be
the maximal ideal corresponding to $\mathfrak mB$. Then on
the one hand $\dim(A[x]_{\mathfrak m'}) = 3$ and on the
other hand
$$
(0)
\subset \mathfrak pA[x]_{\mathfrak m'}
\subset \mathfrak m'A[x]_{\mathfrak m'}
$$
is a maximal chain of primes. Hence $A[x]_{\mathfrak m'}$ is
an example of a non catenary Noetherian local ring.




\section{Non-quasi-affine variety with quasi-affine normalization}
\label{section-nonquasi-affine}

\noindent
The existence of an example of this kind is mentioned in
\cite[II Remark 6.6.13]{EGA}. They refer to the fifth volume of
EGA for such an example, but the fifth volume did not appear.

\medskip\noindent
Let $k$ be a field.
Let $Y = \mathbf{A}^2_k \setminus \{(0, 0)\}$.
We are going to construct a finite surjective birational morphism
$\pi : Y \longrightarrow X$
with $X$ a variety over $k$ such that $X$ is not quasi-affine.
Namely, consider the following curves in $Y$:
$$
\begin{matrix}
C_1 & : & x = 0 \\
C_2 & : & y = 0
\end{matrix}
$$
Note that $C_1 \cap C_2 = \emptyset$. We choose the isomorphism
$\varphi : C_1 \to C_2$, $(0, y) \mapsto (y^{-1}, 0)$.
We claim there is a unique morphism $\pi : Y \to X$ as above
such that
$$
\xymatrix{
C_1
\ar@<1ex>[rr]^{\text{id}} \ar@<-1ex>[rr]_{\varphi}
& &
Y \ar[r]^\pi & X
}
$$
is a coequalizer diagram in the category of varieties (and even in
the category of schemes). Accepting this for the moment let us
show that such an $X$ cannot be quasi-affine. Namely, it is clear
that we would get
$$
\Gamma(X, \mathcal{O}_X) =
\{ f \in k[x, y] \mid f(0, y) = f(y^{-1}, 0)\} =
k \oplus (xy) \subset k[x, y].
$$
In particular these functions do not separate the points $(1, 0)$
and $(-1, 0)$ whose images in $X$ (we will see below) are distinct
(if the characteristic of $k$ is not $2$).

\medskip\noindent
To show that $X$ exists consider the Zariski open
$D(x + y) \subset Y$ of $Y$. This is the spectrum
of the ring
$k[x, y, 1/(x + y)]$
and the curves $C_1$, $C_2$ are completely contained in
$D(x + y)$. Moreover the morphism
$$
C_1 \coprod C_2
\longrightarrow
D(x + y) \cap Y = \text{Spec}(k[x, y, 1/(x + y)])
$$
is a closed immersion. It follows from
Algebra, Lemma \ref{algebra-lemma-fibre-product-finite-type}
that the ring
$$
A =
\{f \in k[x, y, 1/(x + y)] \mid f(0, y) = f(y^{-1}, 0)\}
$$
is of finite type over $k$. On the other hand we have the open
$D(xy) \subset Y$ of $Y$ which is disjoint from the curves $C_1$
and $C_2$. It is the spectrum of the ring
$$
B = k[x, y, 1/xy].
$$
Note that we have $A_{xy} \cong B_{x + y}$ (since $A$ clearly contains
the elements $xyP(x, y)$ any polynomial $P$ and the element $xy/(x + y)$).
The scheme $X$ is obtained by glueing the affine schemes
$\text{Spec}(A)$ and $\text{Spec}(B)$ using the isomorphism
$A_{xy} \cong B_{x + y}$ and hence is clearly of finite type over
$k$. To see that it is separated one has to show that the
ring map $A \otimes_k B \to B_{x + y}$ is surjective. To see
this use that $A \otimes_k B$ contains the element
$xy/(x + y) \otimes 1/xy$ which maps to $1/(x + y)$.
The morphism $X \to Y$ is given by the natural maps
$D(x + y) \to \text{Spec}(A)$ and $D(xy) \to \text{Spec}(B)$.
Since these are both finite we deduce that $X \to Y$ is finite
as desired. We omit the verification that $X$ is indeed the
coequalizer of the displayed diagram above, however, see
(insert future reference for push outs in the category of schemes
here). Note that the morphism $\pi : Y \to X$ does
map the points  $(1, 0)$
and $(-1, 0)$ to distinct points in $X$ because the
function $(x + y^3)/(x + y)^2 \in A$ has value
$1/1$, resp.\ $-1/(-1)^2 = -1$ which are always distinct
(unless the characteristic is $2$ -- please find your own points
for characteristic $2$). We summarize this discussion in the
form of a lemma.

\begin{lemma}
\label{lemma-quasi-affine-normalization-not-quasi-affine}
Let $k$ be a field.
There exists a variety $X$ whose normalization is quasi-affine but
which is itself not quasi-affine.
\end{lemma}

\begin{proof}
See discussion above and (insert future reference on normalization here).
\end{proof}



\section{A finite flat module which is not projective}
\label{section-finite-flat-not-projective}

\noindent
This is a copy of
Algebra, Remark \ref{algebra-remark-warning}.
It is not true that a finite $R$-module which is
$R$-flat is automatically projective. A counter
example is where $R = \mathcal{C}^\infty(\mathbf{R})$
is the ring of infinitely differentiable functions on
$\mathbf{R}$, and $M = R_{\mathfrak m} = R/I$ where
$\mathfrak m = \{f \in R \mid f(0) = 0\}$ and
$I = \{f \in R \mid \exists \epsilon, \epsilon > 0 :
f(x) = 0\ \forall x, |x| < \epsilon\}$.

\medskip\noindent
The morphism $\text{Spec}(R/I) \to \text{Spec}(R)$ is also
an example of a flat closed immersion which is not open.

\begin{lemma}
\label{lemma-finite-flat-non-projective}
Strange flat modules.
\begin{enumerate}
\item There exists a ring $R$ and a finite flat $R$-module $M$ which is
not projective.
\item There exists a closed immersion which is flat but not open.
\end{enumerate}
\end{lemma}

\begin{proof}
See discussion above.
\end{proof}




\section{Zero dimensional local ring with nonzero flat ideal}
\label{section-zero-dimensional-flat-ideal}

\noindent
In \cite{Lazard} there is an example of a zero dimensional local ring with a
nonzero flat ideal. Here is the construction. Let $k$ be a field.
Let $X_i, Y_i$, $i \geq 1$ be variables. Take
$R = k[X_i, Y_i]/(X_i - Y_i X_{i + 1}, Y_i^2)$. Denote $x_i$, resp.\ $y_i$
the image of $X_i$, resp.\ $Y_i$ in this ring. Note that
$$
x_i = y_i x_{i + 1} = y_i y_{i +1} x_{i + 2} =
y_i y_{i + 1} y_{i + 2} x_{i + 3} = \ldots
$$
in this ring. The ring $R$ has only
one prime ideal, namely $\mathfrak m = (x_i, y_i)$. We claim that
the ideal $I = (x_i)$ is flat as an $R$-module.

\medskip\noindent
Note that the annihilator of $x_i$ in $R$ is the ideal
$(x_1, x_2, x_3, \ldots, y_i, y_{i + 1}, y_{i + 2}, \ldots)$.
Consider the $R$-module $M$ generated by elements $e_i$, $i \geq 1$ and
relations $e_i = y_i e_{i + 1}$. Then $M$ is flat as it is the
colimit $\text{colim}_i\ R$ of copies of $R$ with transition maps
$$
R \xrightarrow{y_1} R \xrightarrow{y_2} R \xrightarrow{y_3} \ldots
$$
Note that the annilator of $e_i$ in $M$ is the ideal
$(x_1, x_2, x_3, \ldots, y_i, y_{i + 1}, y_{i + 2}, \ldots)$.
Since every element of $M$, resp.\ $I$ can be written as
$f e_i$, resp.\ $h x_i$ for some $f, h \in R$ we see that the
map $M \to I$, $e_i \to x_i$ is an isomorphism and $I$ is flat.

\begin{lemma}
\label{lemma-zero-dimensional-flat-ideal}
There exists a local ring $R$ with a unique prime ideal
and a nonzero ideal $I \subset R$ which is a flat $R$-module
\end{lemma}

\begin{proof}
See discussion above.
\end{proof}




\section{Finite type, not finitely presented, flat at prime}
\label{section-ft-not-fp-flat-at-prime}

\noindent
Let $k$ be a field. Consider the local ring $A_0 = k[x, y]_{(x, y)}$.
Denote $\mathfrak p_{0, n} = (y + x^n + x^{n + 1})$. This is a prime ideal.
Set
$$
A = A_0[z_1, z_2, z_3, \ldots]/(z_n z_m, z_n(y + x^n + x^{2n + 1}))
$$
Note that $A \to A_0$ is a surjection whose kernel is an ideal of
square zero. Hence $A$ is also a local ring and the prime ideals of $A$
are in one-to-one correspondence with the prime ideals of $A_0$.
Denote $\mathfrak p_n$ the prime ideal of $A$ corresponding to
$\mathfrak p_{0, n}$. Observe that $\mathfrak p_n$ is the annihilator
of $z_n$ in $A$. Let
$$
C = A[z]/(xz^2 + z + y)[\frac{1}{2zx + 1}].
$$
Note that $A \to C$ is an \'etale ring map, see
Algebra, Example \ref{algebra-example-make-standard-smooth}.
Let $\mathfrak q \subset C$ be the maximal ideal generated by
$x$, $y$, $z$ and all $z_n$. As $A \to C$ is flat we see that the
annihilator of $z_n$ in $C$ is $\mathfrak p_nC$. We compute
\begin{align*}
C/\mathfrak p_n C
& =
A_0/(y + x^n + x^{2n + 1}) \\
& =
k[x]_{(x)}[z]/(xz^2 + z - x^n - x^{2n + 1}) \\
& =
k[x]_{(x)}[z]/(z - x^n) \times k[x]_{(x)}[z]/(xz + x^{n + 1} + 1) \\
& =
k[x]_{(x)} \times k(x)
\end{align*}
because $(z - x^n)(xz + x^{n + 1} + 1) = xz^2 + z - x^n - x^{2n + 1}$.
Hence we see that $\mathfrak p_nC = \mathfrak r_n \cap \mathfrak q_n$
with $\mathfrak r_n = \mathfrak p_nC + (z - x^n)C$ and
$\mathfrak q_n = \mathfrak p_nC + (xz + x^{n + 1} + 1)C$.
Since $\mathfrak q_n + \mathfrak r_n = C$ we also get
$\mathfrak p_nC = \mathfrak r_n \mathfrak q_n$,
It follows that $\mathfrak q_n$ is the annihilator of $\xi_n = (z - x^n)z_n$.
Observe that on the one hand $\mathfrak r_n \subset \mathfrak q$, and
on the other hand $\mathfrak q_n + \mathfrak q = C$. This follows for example
because $\mathfrak q_n$ is a maximal ideal of $C$ distinct from $\mathfrak q$.
Similarly we have $\mathfrak q_n + \mathfrak q_m = C$.
At this point we let
$$
B = \text{Im}(C \longrightarrow C_{\mathfrak q})
$$
We observe that the elements $\xi_n$ map to zero in $B$ as $xz + x^{n + 1} + 1$
is not in $\mathfrak q$. Denote $\mathfrak q' \subset B$ the image of
$\mathfrak q$. By construction $B$ is a finite type $A$-algebra, with
$B_{\mathfrak q'} \cong C_{\mathfrak q}$. In particular we see that
$B_{\mathfrak q'}$ is flat over $A$.

\medskip\noindent
We claim there does not exist an element $g' \in B$, $g' \not \in \mathfrak q'$
such that $B_{g'}$ is of finite presentation over $A$. We sketch a proof of
this claim. Choose an element $g \in C$ which maps to $g' \in B$.
Consider the map $C_g \to B_{g'}$. By
Algebra, Lemma \ref{algebra-lemma-finite-presentation-independent}
we see that $B_g$ is finitely presented over $A$ if and only if the kernel
of $C_g \to B_{g'}$ is finitely generated. But the element $g \in C$ is
not contained in $\mathfrak q$, hence maps to a nonzero element of
$A_0[z]/(xz^2 + z + y)$. Hence $g$ can only be contained in finitely
many of the prime ideals $\mathfrak q_n$, because the primes
$(y + x^n + x^{2n + 1}, xz + x^{n + 1} + 1)$ are an infinite collection
of codimension 1 points of the 2-dimensional irreducible Noetherian space
$\text{Spec}(k[x, y, z]/(xz^2 + z + y))$. The map
$$
\bigoplus\nolimits_{g \not \in \mathfrak q_n} C/\mathfrak q_n
\longrightarrow
C_g, \quad
(c_n) \longrightarrow \sum c_n \xi_n
$$
is injective and its image is the kernel of $C_g \to B_{g'}$. We omit the
proof of this statement. (Hint: Write $A = A_0 \oplus I$ as an $A_0$-module
where $I$ is the kernel of $A \to A_0$. Similarly, write $C = C_0 \oplus IC$.
Write
$IC = \bigoplus Cz_n \cong \bigoplus (C/\mathfrak r_n \oplus C/\mathfrak q_n)$
and study the effect of multiplication by $g$ on the summands.)
This concludes the sketch of the proof of the claim.
This also proves that $B_{g'}$ is not flat over $A$ for any $g'$ as above.
Namely, if it were flat, then the annihilator of the image of $z_n$ in
$B_{g'}$ would be $\mathfrak p_nB_{g'}$, and would not contain $z - x^n$.

\medskip\noindent
As a consequence we can answer (negatively) a question posed in
\cite[Part I, Remarques (3.4.7) (\romannumeral 5)]{GruRay}.
Here is a precise statement.

\begin{lemma}
\label{lemma-example-raynaud-gruson}
There exists a local ring $A$, a finite type ring map $A \to B$ and a prime
$\mathfrak q$ lying over $\mathfrak m_A$ such that $B_{\mathfrak q}$ is flat
over $A$, and for any element $g \in B$, $g \not \in \mathfrak q$
the ring $B_g$ is neither finitely presented over $A$ nor flat over $A$.
\end{lemma}

\begin{proof}
See discussion above.
\end{proof}


\section{Finite type, flat and not of finite presentation}
\label{section-finite-type-flat-not-finite-presentation}

\noindent
In this section we give some examples of ring maps and morphisms
which are of finite type and flat but not of finite presentation.

\medskip\noindent
Let $R$ be a ring which has an ideal $I$ such that $R/I$ is a finite
flat module but not projective, see
Section \ref{section-finite-flat-not-projective}
for an explicit example. Note that this means that $I$ is not
finitely generated, see
Algebra, Lemma \ref{algebra-lemma-finitely-generated-pure-ideal}.
Note that $I = I^2$, see
Algebra, Lemma \ref{algebra-lemma-pure}.
The base ring in our examples will be
$R$ and correspondingly the base scheme $S = \text{Spec}(R)$.

\medskip\noindent
Consider the ring map $R \to R \oplus R/I\epsilon$ where
$\epsilon^2 = 0$ by convention. This is a finite, flat
ring map which is not of finite presentation. All the fibre rings
are complete intersections and geometrically irreducible.

\medskip\noindent
Let $A = R[x, y]/(xy, ay; a \in I)$. Note that as an $R$-module
we have $A = \bigoplus_{i \geq 0} Ry^i \oplus \bigoplus_{j > 0} R/Ix^j$.
Hence $R \to A$ is a flat finite type ring map which is not of finite
presentation. Each fibre ring
is isomorphic to either $\kappa(\mathfrak p)[x, y]/(xy)$
or $\kappa(\mathfrak p)[x]$.

\medskip\noindent
We can turn the previous example into a projective morphism by
taking $B = R[X_0, X_1, X_2]/(X_1X_2, aX_2; a \in I)$.
In this case $X = \text{Proj}(B) \to S$ is a proper flat morphism
which is not of finite presentation such that for each $s \in S$
the fibre $X_s$ is isomorphic either to $\mathbf{P}^1_s$ or to
the closed subscheme of $\mathbf{P}^2_s$ defined by the vanishing of
$X_1X_2$ (this is a projective nodal curve of arithmetic genus $0$).

\medskip\noindent
Let $M = R \oplus R \oplus R/I$. Set $B = \text{Sym}_R(M)$ the 
symmetric algebra on $M$. Set $X = \text{Proj}(B)$.
Then $X \to S$ is a proper flat morphism, not of finite presentation
such that for $s \in S$ the geometric fibre is isomorphic to either
$\mathbf{P}^1_s$ or $\mathbf{P}^2_s$. In particular these fibres are
smooth and geometrically irreducible.

\begin{lemma}
\label{lemma-finite-type-flat-not-finitely-presented}
There exist examples of
\begin{enumerate}
\item a flat finite type ring map with geometrically irreducible
complete intersection fibre rings which is not of finite presentation,
\item a flat finite type ring map with geometrically connected,
geometrically reduced, dimension 1, complete intersection fibre rings
which is not of finite presentation,
\item a proper flat morphism of schemes $X \to S$ each of whose fibres
is isomorphic to either $\mathbf{P}^1_s$ or to the vanishing locus of
$X_1X_2$ in $\mathbf{P}^2_s$ which is not of finite presentation, and
\item a proper flat morphism of schemes $X \to S$ each of whose
fibres is isomorphic to either $\mathbf{P}^1_s$ or $\mathbf{P}^2_s$
which is not of finite presentation.
\end{enumerate}
\end{lemma}

\begin{proof}
See discussion above.
\end{proof}



\section{Topology of a finite type ring map}
\label{section-topology-finite-type}

\noindent
Let $A \to B$ be a local map of local domains.
If $A$ is Noetherian, $A \to B$ is essentially of finite type,
and $A/\mathfrak m_A \subset B/\mathfrak m_B$ is finite then there
exists a prime $\mathfrak q \subset B$, $\mathfrak q \not = \mathfrak m_B$
such that $A \to B/\mathfrak q$ is the localization of a quasi-finite ring
map. See
More on Morphisms, Lemma
\ref{more-morphisms-lemma-quasi-finite-quasi-section-meeting-nearby-open}.

\medskip\noindent
In this section we give an example that shows this result is false $A$
is no longer Noetherian. Namely, let $k$ be a field and set
$$
A = \{a_0 + a_1 x + a_2 x^2 + \ldots \mid a_0 \in k, a_i \in k((y))
\text{ for }i\geq 1\}
$$
and
$$
C = \{a_0 + a_1 x + a_2 x^2 + \ldots \mid a_0 \in k[y], a_i \in k((y))
\text{ for }i\geq 1\}.
$$
The inclusion $A \to C$ is of finite type as $C$ is generated by $y$
over $A$. We claim that $A$ is a local ring with maximal ideal
$\mathfrak m = \{a_1 x + a_2 x^2 + \ldots \in A\}$ and no prime
ideals besides $(0)$ and $\mathfrak m$. Namely, an element
$f = a_0 + a_1 x + a_2 x^2 + \ldots$ of $A$ is invertible as soon as
$a_0 \not = 0$. If $\mathfrak q \subset A$ is a nonzero prime ideal,
and $f = a_i x^i + \ldots \in \mathfrak q$, then using properties of
power series one sees that for any $g \in k((y))$ the element
$g^{i + 1} x^{i + 1} \in \mathfrak q$, i.e., $gx \in \mathfrak q$.
This proves that $\mathfrak q = \mathfrak m$.

\medskip\noindent
As to the spectrum of the ring $C$, arguing in the same way as above
we see that any nonzero prime ideal contains the prime
$\mathfrak p = \{a_1 x + a_2 x^2 + \ldots \in C\}$ which lies
over $\mathfrak m$. Thus the only prime of $C$ which lies over
$(0)$ is $(0)$. Set $\mathfrak m_C = yC + \mathfrak p$ and
$B = C_{\mathfrak m_C}$. Then $A \to B$ is the desired example.

\begin{lemma}
\label{lemma-topology-finite-type}
There exists a local homomorphism $A \to B$ of local domains which is
essentially of finite type and such that $A/\mathfrak m_A \to B/\mathfrak m_B$
is finite such that for every prime
$\mathfrak q \not = \mathfrak m_B$ of $B$ the ring map
$A \to B/\mathfrak q$ is not the localization of a quasi-finite ring map.
\end{lemma}

\begin{proof}
See the discussion above.
\end{proof}



\section{Pure not universally pure}
\label{section-pure-not-universally}

\noindent
Let $k$ be a field. Let
$$
R = k[[x, xy, xy^2, \ldots]] \subset k[[x, y]].
$$
In other words, a power series $f \in k[[x, y]]$ is in $R$ if and only
if $f(0, y)$ is a constant. In particular $R[1/x] = k[[x, y]][1/x]$
and $R/xR$ is a local ring with a maximal ideal whose square
is zero. Denote $R[y] \subset k[[x, y]]$ the set of power series
$f \in k[[x, y]]$ such that $f(0, y)$ is a polynomial in $y$. Then
$R \to R[y]$ is a finite type but not finitely presented ring map
which induces an isomorphism after inverting $x$. Also there is a
surjection $R[y]/xR[y] \to k[y]$ whose kernel has square zero. Consider
the finitely presented ring map $R \to S = R[t]/(xt - xy)$.
Again $R[1/x] \to S[1/x]$ is an isomorphism and in this case
$S/xS \cong (R/xR)[t]/(xy)$ maps onto $k[t]$ with nilpotent kernel.
There is a surjection $S \to R[y]$, $t \longmapsto y$ which induces
an isomorphism on inverting $x$ and a surjection with nilpotent kernel
modulo $x$. Hence the kernel of $S \to R[y]$ is locally nilpotent.
In particular $S \to R[y]$ is a universal homeomorphism.

\medskip\noindent
First we claim that $S$ is an $S$-module which is relatively pure
over $R$. Since on inverting $x$ we obtain an isomorphism we only need
to check this at the maximal ideal $\mathfrak m \subset R$. Since
$R$ is complete with respect to its maximal ideal it is henselian
hence we need only check that every prime $\mathfrak p \subset R$,
$\mathfrak p \not = \mathfrak m$, the unique prime $\mathfrak q$
of $S$ lying over $\mathfrak p$ satisfies
$\mathfrak mS + \mathfrak q \not = S$. Since $\mathfrak p \not = \mathfrak m$
it corresponds to a unique prime ideal of $k[[x, y]][1/x]$. Hence
either $\mathfrak p = (0)$ or $\mathfrak p = (f)$ for some
irreducible element $f \in k[[x, y]]$ which is not associated to $x$
(here we use that $k[[x, y]]$ is a UFD -- insert future reference here).
In the first case $\mathfrak q = (0)$ and the result is clear. In the
second case we may multiply $f$ by a unit so that $f \in R[y]$
(Weierstrass preparation; details omitted). Then it is easy to see that
$R[y]/fR[y] \cong k[[x, y]]/(f)$ hence $f$ defines a prime ideal
of $R[y]$ and $\mathfrak mR[y] + fR[y] \not = R[y]$.
Since $S \to R[y]$ is a universal homeomorphism we deduce the
desired result for $S$ also.

\medskip\noindent
Second we claim that $S$ is not universally relatively pure over $R$.
Namely, to see this it sufffices to find a valuation ring
$\mathcal{O}$ and a local ring map $R \to \mathcal{O}$ such that
$\text{Spec}(R[y] \otimes_R \mathcal{O}) \to \text{Spec}(\mathcal{O})$
does not hit the closed point of $\text{Spec}(\mathcal{O})$.
Equivalently, we have to find $\varphi : R \to \mathcal{O}$ such that
$\varphi(x) \not = 0$ and $v(\varphi(x)) > v(\varphi(xy))$ where $v$
is the valuation of $\mathcal{O}$.
(Because this means that the valuation of $y$ is negative.)
To do this consider the ring map
$$
R
\longrightarrow
\{a_0 + a_1 x + a_2 x^2 + \ldots \mid a_0 \in k[y^{-1}], a_i \in k((y))\}
$$
defined in the obvious way. We can find a valuation ring $\mathcal{O}$
dominating the localization of the right hand side at the maximal
ideal $(y^{-1}, x)$ and we win.

\begin{lemma}
\label{lemma-pure-not-universally-pure}
There exists a morphism of affine schemes of finite presentation
$X \to S$ and an $\mathcal{O}_X$-module $\mathcal{F}$ of finite presentation
such that $\mathcal{F}$ is pure relative to $S$, but not universally
pure relative to $S$.
\end{lemma}

\begin{proof}
See discussion above.
\end{proof}



\section{A formally smooth non-flat ring map}
\label{section-formally-smooth-nonflat}

\noindent
Let $k$ be a field of characteristic zero.
Consider the $k$-algebra $k[\mathbf{Q}]$.
This is the $k$-algebra with basis
$x_\alpha, \alpha \in \mathbf{Q}$ and multiplication determined by
$x_\alpha x_\beta = x_{\alpha + \beta}$. Consider the $k$-algebra
homomorphism
$$
k[\mathbf{Q}] \longrightarrow k,\quad
x_\alpha \longmapsto 1.
$$
We claim this is formally smooth. Since it is surjective it is
formally unramified. Let $A' \to A$ be a surjection whose kernel
has square zero. Let $\varphi : k[\mathbf{Q}] \to A$ be a ring map.
As $k$ is of characteristic zero, it is formally smooth over
$\mathbf{Q}$, see
Algebra, Proposition
\ref{algebra-proposition-characterize-separable-field-extensions}.
Hence there exists a lift $k \to A'$ of the restriction
of $\varphi$ to $k$. Pick a lift $y_n \in A'$ of $\varphi(x_{1/n})$
for $n \in \mathbf{N}$. Note that $y_1$ is a unit in $A'$. Hence we can write
$(y_n)^n = y_1(1 + z_n)$ for some $z_n \in I$. Thus we can replace $y_n$
by $y_n(1 - (1/n)z)n)$ and we get that $y_n^n = y_1$. Having done this
we get a $k$-algebra map
$$
\varphi' : k[\mathbf{Q}] \longrightarrow A', \quad
x_{p/q} \longmapsto y_q^p.
$$
In this way we see that the map $k[\mathbf{Q}] \to k$ is
formally smooth. Finally, this ring map is not flat, for example as the
nonzero divisor $x_2 - 1$ is mapped to zero.

\begin{lemma}
\label{lemma-formally-smooth-nonflat}
There exists a formally smooth ring map which is not flat.
\end{lemma}

\begin{proof}
See discussion above.
\end{proof}


\section{Ideals generated by sets of idempotents and localization}
\label{section-ideal-locally-idempotents}

\noindent
Let $R$ be a ring. Consider the ring
$$
B(R) = R[x_n; n \in \mathbf{Z}]/(x_n(x_n - 1), x_nx_m; n \not = m)
$$
It is easy to show that every prime $\mathfrak q \subset B(R)$
is either of the form
$$
\mathfrak q = \mathfrak pB(R) + (x_n; n \in \mathbf{Z})
$$
or of the form
$$
\mathfrak q =
\mathfrak pB(R) + (x_n - 1) + (x_m; n \not = m, m \in \mathbf{Z}).
$$
Hence we see that
$$
\text{Spec}(B(R)) =
\text{Spec}(R) \amalg \coprod\nolimits_{n \in \mathbf{Z}} \text{Spec}(R)
$$
where the topology is not just the disjoint union topology. It has the
following properties: Each of the copies indexed by $n \in \mathbf{Z}$
is an open subscheme, namely it is the standard open $D(x_n)$.
The "central" copy of $\text{Spec}(R)$ is in the closure of the union
of any infinitely many of the other copies of $\text{Spec}(R)$.
Note that this last copy of $\text{Spec}(R)$ is cut out by the ideal
$(x_n, n \in \mathbf{Z})$ which is generated by the idempotents $x_n$.
Hence we see that if $\text{Spec}(R)$ is connected,
then the decomposition above is exactly the decomposition of
$\text{Spec}(B(R))$ into connected components.

\medskip\noindent
Next, let $A = \mathbf{C}[x, y]/((y - x^2 + 1)(y + x^2 - 1))$.
The spectrum of $A$ consists of two irreducible components
$C_1 = \text{Spec}(A_1)$, $C_2 = \text{Spec}(A_2)$
with $A_1 = \mathbf{C}[x, y]/(y - x^2 + 1)$ and
$A_2 = \mathbf{C}[x, y]/(y + x^2 - 1)$. Note that these are
parametrized by $(x, y) = (t, t^2 - 1)$ and $(x, y) = (t, -t^2 + 1)$
which meet in $P = (-1, 0)$ and $Q = (1, 0)$. We can make a twisted
version of $B(A)$ where we glue $B(A_1)$ to $B(A_2)$ in the following
way: Above $P$ we let $x_n \in B(A_1) \otimes \kappa(P)$
correspond to $x_n \in B(A_2) \otimes \kappa(P)$, but above $Q$
we let $x_n \in B(A_1) \otimes \kappa(P)$
correspond to $x_{n + 1} \in B(A_2) \otimes \kappa(P)$.
Let $B^{twist}(A)$ denote the resulting $A$-algebra.
Details omitted. By construction
$B^{twist}(A)$ is Zariski locally over $A$ isomorphic to the untwisted
version. Namely, this happens over both the principal open
$\text{Spec}(A) \setminus \{P\}$
and the principal open $\text{Spec}(A) \setminus \{Q\}$.
However, our choice of glueing produces enough "monodromy" such that
$\text{Spec}(B^{twist}(A))$ is connected (details omitted).
Finally, there is a central copy of
$\text{Spec}(A) \to \text{Spec}(B^{twist}(A))$
which gives a closed subscheme whose ideal is Zariski locally
on $B^{twist}(A)$ cut out by ideals generated by idempotents, but
not globally (as $B^{twist}(A)$ has no nontrivial idempotents).

\begin{lemma}
\label{lemma-not-generated-idempotents}
There exists an affine scheme $X = \text{Spec}(A)$ and a
closed subscheme $T \subset X$ such that $T$ is Zariski locally
on $X$ cut out by ideals generated by idempotents, but
$T$ is not cut out by an ideal generated by idempotents.
\end{lemma}

\begin{proof}
See above.
\end{proof}





\section{Non flasque quasi-coherent sheaf associated to injective module}
\label{section-nonflasque}

\noindent
For more examples of this type see \cite[Expos\'e II, Appendix I]{SGA6}
where Illusie explains some examples due to Verdier.

\medskip\noindent
Consider the affine scheme $X = \text{Spec}(A)$
where
$$
A = k[f, g, x, y, \{a_n, b_n\}_{n \geq1}]/
(fy - gx, \{a_nf^n + b_ng^n\}_{n \geq 1})
$$
is the ring from
Properties, Example \ref{properties-example-does-not-work-in-general}.
Set $I = (f, g) \subset A$.
Consider the quasi-compact open $U = D(f) \cup D(g)$ of $X$.
We have seen in loc.\ cit.\ that there is a section
$s \in \mathcal{O}_X(U)$ which does not come from an $A$-module
map $I^n \to A$ for any $n \geq 0$.

\medskip\noindent
Let $\alpha : A \to J$ be the embedding of $A$ into an injective $A$-module.
Let $Q = J/\alpha(A)$ and denote $\beta : J \to Q$ the quotient map.
We claim that the map
$$
\Gamma(X, \widetilde{J})
\longrightarrow
\Gamma(U, \widetilde{J})
$$
is not surjective. Namely, we claim that $\alpha(s)$ is not in the image.
To see this, we argue by contradiction. So assume that $x \in J$ is an
element which restricts to $\alpha(s)$ over $U$. Then $\beta(x) \in Q$
is an element which restricts to $0$ over $U$. Hence we know that
$I^n\beta(x) = 0$ for some $n$, see
Properties,
Lemma \ref{properties-lemma-sections-over-quasi-compact-open-in-affine}.
This implies that we get a morphism
$\varphi : I^n \to A$, $h \mapsto \alpha^{-1}(hx)$. It is easy to see that
this morphism $\varphi$ gives rise to the section $s$ via the map of
Properties,
Lemma \ref{properties-lemma-sections-over-quasi-compact-open-in-affine}
which is a contradiction.

\begin{lemma}
\label{lemma-nonflasque}
There exists an affine scheme $X = \text{Spec}(A)$ and an injective
$A$-module $J$ such that $\widetilde{J}$ is not a flasque sheaf on $X$.
Even the restriction $\Gamma(X, \widetilde{J}) \to \Gamma(U, \widetilde{J})$
with $U$ quasi-compact open need not be surjective.
\end{lemma}

\begin{proof}
See above.
\end{proof}









\section{A torsor which is not an fppf torsor}
\label{section-torsor-not-fppf}

\noindent
In
Groupoids, Remark \ref{groupoids-remark-fun-with-torsors}
we raise the question whether any $G$-torsor is a $G$-torsor for the
fppf topology. In this section we show that this is not always the case.

\medskip\noindent
Let $k$ be a field. All schemes and stacks are over $k$ in what follows.
Let $G \to \text{Spec}(k)$ be the group scheme
$$
G = (\mu_{2, k})^\infty =
\mu_{2, k} \times_k \mu_{2, k} \times_k \mu_{2, k} \times_k \ldots =
\text{lim}_n\ (\mu_{2, k})^n
$$
where $\mu_{2, k}$ is the group scheme of second roots of unity over
$\text{Spec}(k)$, see
Groupoids, Example \ref{groupoids-example-roots-of-unity}.
As an inverse limit of affine schemes we see that $G$ is an affine group
scheme. In fact it is the spectrum of the ring
$k[t_1, t_2, t_3, \ldots]/(t_i^2 - 1)$. The multiplication map
$m : G \times_k G \to G$ is on the algebra level given by
$t_i \mapsto t_i \otimes t_i$.

\medskip\noindent
We claim that any $G$-torsor over $k$ is of the form
$$
P = \text{Spec}(k[x_1, x_2, x_3, \ldots]/(x_i^2 - a_i))
$$
for certain $a_i \in k^*$ and with $G$-action $G \times_k P \to P$
given by $x_i \to t_i \otimes x_i$ on the algebra level.
We omit the proof.
Actually for the example we only need that $P$ is a $G$-torsor
which is clear since over $k' = k(\sqrt{a_1}, \sqrt{a_2}, \ldots)$
the scheme $P$ becomes isomorphic to $G$ in a $G$-equivariant manner.
Note that $P$ is trivial if and only if $k' = k$ since if
$P$ has a $k$-rational point then all of the $a_i$ are squares.

\medskip\noindent
We claim that $P$ is an fppf torsor if and only if the field extension
$k \subset k' = k(\sqrt{a_1}, \sqrt{a_2}, \ldots)$ is finite. 
If $k'$ is finite over $k$, then
$\{\text{Spec}(k') \to \text{Spec}(k)\}$
is an fppf covering which trivializes $P$ and we see that $P$ is indeed
an fppf torsor. Conversely, suppose that $P$ is a $G$-torsor for the
fppf topology. This means that there exists an fppf covering
$\{S_i \to \text{Spec}(k)\}$ such that each $P_{S_i}$ is trivial.
Pick an $i$ such that $S_i$ is not empty. Let $s \in S_i$ be a closed
point. By
Varieties, Lemma \ref{varieties-lemma-locally-finite-type-Jacobson}
the field extension $k \subset \kappa(s)$ is finite, and by construction
$P_{\kappa(s)}$ has a $\kappa(s)$-rational point. Thus we see that
$k \subset k' \subset \kappa(s)$ and $k'$ is finite over $k$.

\medskip\noindent
To get an explicit example take $k = \mathbf{Q}$ and $a_i = i$
for example (or $a_i$ is the $i$th prime if you like).










\section{Stack with quasi-compact flat covering which is not algebraic}
\label{section-not-algebraic-stack}

\noindent
In this section we briefly describe an example due to Brian Conrad.
You can find the example online at
\href{http://mathoverflow.net/questions/15082/fpqc-covers-of-stacks/15269#15269}
{this location}. Our example is slightly different.

\medskip\noindent
Let $k$ be an algebraically closed field.
All schemes and stacks are over $k$ in what follows.
Let $G \to \text{Spec}(k)$ be an affine group scheme.
In
Examples of Stacks,
Proposition \ref{examples-stacks-proposition-equal-quotient-stacks}
we have seen that 
$\mathcal{X} = [\text{Spec}(k)/G]$ is a stack in groupoids over
$(\textit{Sch}/\text{Spec}(k))_{fppf}$ which can be described
as follows. A $1$-morphism $T \to \mathcal{X}$
corresponds by definition to an fppf $G_T$-torsor $P$ over $T$.
The diagonal $1$-morphism
$$
\Delta :
\mathcal{X}
\longrightarrow
\mathcal{X} \times_{\text{Spec}(k)} \mathcal{X}
$$
is representable and affine. The reason for this is that given any pair
of $G_T$-torsors $P_1, P_2$ in the fppf topology over a scheme $S/k$ the
scheme $\text{Isom}(P_1, P_2)$ is affine over $T$.
The trivial $G$-torsor over $\text{Spec}(k)$ defines a $1$-morphism
$$
f : \text{Spec}(k) \longrightarrow \mathcal{X}.
$$
We claim that this is a surjective $1$-morphism. The reason is simply
that by definition for any $1$-morphism $T \to \mathcal{X}$ there exists
a fppf covering $\{T_i \to T\}$ such that $P_{T_i}$ is isomorphic
to the trivial $G_{T_i}$-torsor. Hence the compositions
$T_i \to T \to \mathcal{X}$ factor through $f$. Thus it is clear that
the projection $T \times_{\mathcal{X}} \text{Spec}(k) \to \mathcal{X}$
is surjective (which is how we define the property that $f$ is surjective, see
Algebraic Stacks,
Definition \ref{algebraic-definition-relative-representable-property}).
In a similar way you show that $f$ is quasi-compact and flat (details omitted).
We also record here the observation that
$$
\text{Spec}(k) \times_{\mathcal{X}} \text{Spec}(k) \cong G
$$
as schemes over $k$.

\medskip\noindent
Suppose there exists a surjective smooth morphism
$p : U \to \mathcal{X}$ where $U$ is a scheme.
Consider the fibre product
$$
\xymatrix{
W \ar[d] \ar[r] & U \ar[d] \\
\text{Spec}(k) \ar[r] & \mathcal{X}
}
$$
Then we see that $W$ is a nonempty smooth scheme over $k$ which
hence has a $k$-point. This means that we can factor $f$ through $U$.
Hence we obtain
$$
G \cong
\text{Spec}(k) \times_{\mathcal{X}} \text{Spec}(k) \cong
(\text{Spec}(k) \times_k \text{Spec}(k))
\times_{(U \times_k U)}
(U \times_{\mathcal{X}} U)
$$
and since the projections $U \times_{\mathcal{X}} U \to U$ were
assumed smooth we conclude that
$U \times_{\mathcal{X}} U \to U \times_k U$ is
locally of finite type, see
Morphisms, Lemma \ref{morphisms-lemma-permanence-finite-type}.
It follows that in this case $G$ is locally of finite type over $k$.
Alltogether we have proved the following lemma (which can be
significantly generalized).

\begin{lemma}
\label{lemma-BG-algebraic}
Let $k$ be a field. Let $G$ be an affine group scheme over $k$.
If the stack $[\text{Spec}(k)/G]$ has a smooth covering by a
scheme, then $G$ is of finite type over $k$.
\end{lemma}

\begin{proof}
See discussion above.
\end{proof}

\noindent
To get an explicit example as in the title of this section, take for example
$G = (\mu_{2,k})^\infty$ the group scheme of
Section \ref{section-torsor-not-fppf}, which is not locally of finite type
over $k$. By the discussion above we see that
$\mathcal{X} = [\text{Spec}(k)/G]$ has properties (1) and (2) of
Algebraic Stacks, Definition \ref{algebraic-definition-algebraic-stack},
but not property (3). Hence $\mathcal{X}$ is not an algebraic stack.
On the other hand, there does exists a scheme $U$ an a surjective,
flat, quasi-compact morphism $U \to \mathcal{X}$, namely the morphism
$f : \text{Spec}(k) \to \mathcal{X}$ we studied above.






\section{Other chapters}

\begin{multicols}{2}
\begin{enumerate}
\item \hyperref[introduction-section-phantom]{Introduction}
\item \hyperref[conventions-section-phantom]{Conventions}
\item \hyperref[sets-section-phantom]{Set Theory}
\item \hyperref[categories-section-phantom]{Categories}
\item \hyperref[topology-section-phantom]{Topology}
\item \hyperref[sheaves-section-phantom]{Sheaves on Spaces}
\item \hyperref[algebra-section-phantom]{Commutative Algebra}
\item \hyperref[sites-section-phantom]{Sites and Sheaves}
\item \hyperref[homology-section-phantom]{Homological Algebra}
\item \hyperref[derived-section-phantom]{Derived Categories}
\item \hyperref[more-algebra-section-phantom]{More Algebra}
\item \hyperref[simplicial-section-phantom]{Simplicial Methods}
\item \hyperref[modules-section-phantom]{Sheaves of Modules}
\item \hyperref[sites-modules-section-phantom]{Modules on Sites}
\item \hyperref[injectives-section-phantom]{Injectives}
\item \hyperref[cohomology-section-phantom]{Cohomology of Sheaves}
\item \hyperref[sites-cohomology-section-phantom]{Cohomology on Sites}
\item \hyperref[hypercovering-section-phantom]{Hypercoverings}
\item \hyperref[schemes-section-phantom]{Schemes}
\item \hyperref[constructions-section-phantom]{Constructions of Schemes}
\item \hyperref[properties-section-phantom]{Properties of Schemes}
\item \hyperref[morphisms-section-phantom]{Morphisms of Schemes}
\item \hyperref[coherent-section-phantom]{Coherent Cohomology}
\item \hyperref[divisors-section-phantom]{Divisors}
\item \hyperref[limits-section-phantom]{Limits of Schemes}
\item \hyperref[varieties-section-phantom]{Varieties}
\item \hyperref[chow-section-phantom]{Chow Homology}
\item \hyperref[topologies-section-phantom]{Topologies on Schemes}
\item \hyperref[descent-section-phantom]{Descent}
\item \hyperref[more-morphisms-section-phantom]{More on Morphisms}
\item \hyperref[flat-section-phantom]{More on Flatness}
\item \hyperref[groupoids-section-phantom]{Groupoid Schemes}
\item \hyperref[more-groupoids-section-phantom]{More on Groupoid Schemes}
\item \hyperref[etale-section-phantom]{\'Etale Morphisms of Schemes}
\item \hyperref[etale-cohomology-section-phantom]{\'Etale Cohomology}
\item \hyperref[spaces-section-phantom]{Algebraic Spaces}
\item \hyperref[spaces-properties-section-phantom]{Properties of Algebraic Spaces}
\item \hyperref[spaces-morphisms-section-phantom]{Morphisms of Algebraic Spaces}
\item \hyperref[spaces-topologies-section-phantom]{Topologies on Algebraic Spaces}
\item \hyperref[spaces-descent-section-phantom]{Descent and Algebraic Spaces}
\item \hyperref[spaces-more-morphisms-section-phantom]{More on Morphisms of Spaces}
\item \hyperref[quot-section-phantom]{Quot and Hilbert Spaces}
\item \hyperref[stacks-section-phantom]{Stacks}
\item \hyperref[spaces-groupoids-section-phantom]{Groupoids in Algebraic Spaces}
\item \hyperref[spaces-more-groupoids-section-phantom]{More on Groupoids in Spaces}
\item \hyperref[bootstrap-section-phantom]{Bootstrap}
\item \hyperref[examples-stacks-section-phantom]{Examples of Stacks}
\item \hyperref[groupoids-quotients-section-phantom]{Quotients of Groupoids}
\item \hyperref[algebraic-section-phantom]{Algebraic Stacks}
\item \hyperref[criteria-section-phantom]{Criteria for Representability}
\item \hyperref[stacks-properties-section-phantom]{Properties of Algebraic Stacks}
\item \hyperref[stacks-morphisms-section-phantom]{Morphisms of Algebraic Stacks}
\item \hyperref[examples-section-phantom]{Examples}
\item \hyperref[exercises-section-phantom]{Exercises}
\item \hyperref[guide-section-phantom]{Guide to Literature}
\item \hyperref[desirables-section-phantom]{Desirables}
\item \hyperref[coding-section-phantom]{Coding Style}
\item \hyperref[fdl-section-phantom]{GNU Free Documentation License}
\item \hyperref[index-section-phantom]{Auto Generated Index}
\end{enumerate}
\end{multicols}


\bibliography{my}
\bibliographystyle{amsalpha}

\end{document}
