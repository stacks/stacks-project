\IfFileExists{stacks-project.cls}{%
\documentclass{stacks-project}
}{%
\documentclass{amsart}
}

% The following AMS packages are automatically loaded with
% the amsart documentclass:
%\usepackage{amsmath}
%\usepackage{amssymb}
%\usepackage{amsthm}

% For dealing with references we use the comment environment
\usepackage{verbatim}
\newenvironment{reference}{\comment}{\endcomment}
%\newenvironment{reference}{}{}
\newenvironment{slogan}{\comment}{\endcomment}
\newenvironment{history}{\comment}{\endcomment}

% For commutative diagrams you can use
% \usepackage{amscd}
\usepackage[all]{xy}

% We use 2cell for 2-commutative diagrams.
\xyoption{2cell}
\UseAllTwocells

% To put source file link in headers.
% Change "template.tex" to "this_filename.tex"
% \usepackage{fancyhdr}
% \pagestyle{fancy}
% \lhead{}
% \chead{}
% \rhead{Source file: \url{template.tex}}
% \lfoot{}
% \cfoot{\thepage}
% \rfoot{}
% \renewcommand{\headrulewidth}{0pt}
% \renewcommand{\footrulewidth}{0pt}
% \renewcommand{\headheight}{12pt}

\usepackage{multicol}

% For cross-file-references
\usepackage{xr-hyper}

% Package for hypertext links:
\usepackage{hyperref}

% For any local file, say "hello.tex" you want to link to please
% use \externaldocument[hello-]{hello}
\externaldocument[introduction-]{introduction}
\externaldocument[conventions-]{conventions}
\externaldocument[sets-]{sets}
\externaldocument[categories-]{categories}
\externaldocument[topology-]{topology}
\externaldocument[sheaves-]{sheaves}
\externaldocument[sites-]{sites}
\externaldocument[stacks-]{stacks}
\externaldocument[fields-]{fields}
\externaldocument[algebra-]{algebra}
\externaldocument[brauer-]{brauer}
\externaldocument[homology-]{homology}
\externaldocument[derived-]{derived}
\externaldocument[simplicial-]{simplicial}
\externaldocument[more-algebra-]{more-algebra}
\externaldocument[smoothing-]{smoothing}
\externaldocument[modules-]{modules}
\externaldocument[sites-modules-]{sites-modules}
\externaldocument[injectives-]{injectives}
\externaldocument[cohomology-]{cohomology}
\externaldocument[sites-cohomology-]{sites-cohomology}
\externaldocument[dga-]{dga}
\externaldocument[dpa-]{dpa}
\externaldocument[hypercovering-]{hypercovering}
\externaldocument[schemes-]{schemes}
\externaldocument[constructions-]{constructions}
\externaldocument[properties-]{properties}
\externaldocument[morphisms-]{morphisms}
\externaldocument[coherent-]{coherent}
\externaldocument[divisors-]{divisors}
\externaldocument[limits-]{limits}
\externaldocument[varieties-]{varieties}
\externaldocument[topologies-]{topologies}
\externaldocument[descent-]{descent}
\externaldocument[perfect-]{perfect}
\externaldocument[more-morphisms-]{more-morphisms}
\externaldocument[flat-]{flat}
\externaldocument[groupoids-]{groupoids}
\externaldocument[more-groupoids-]{more-groupoids}
\externaldocument[etale-]{etale}
\externaldocument[chow-]{chow}
\externaldocument[intersection-]{intersection}
\externaldocument[pic-]{pic}
\externaldocument[adequate-]{adequate}
\externaldocument[dualizing-]{dualizing}
\externaldocument[duality-]{duality}
\externaldocument[discriminant-]{discriminant}
\externaldocument[local-cohomology-]{local-cohomology}
\externaldocument[curves-]{curves}
\externaldocument[resolve-]{resolve}
\externaldocument[models-]{models}
\externaldocument[pione-]{pione}
\externaldocument[etale-cohomology-]{etale-cohomology}
\externaldocument[proetale-]{proetale}
\externaldocument[crystalline-]{crystalline}
\externaldocument[spaces-]{spaces}
\externaldocument[spaces-properties-]{spaces-properties}
\externaldocument[spaces-morphisms-]{spaces-morphisms}
\externaldocument[decent-spaces-]{decent-spaces}
\externaldocument[spaces-cohomology-]{spaces-cohomology}
\externaldocument[spaces-limits-]{spaces-limits}
\externaldocument[spaces-divisors-]{spaces-divisors}
\externaldocument[spaces-over-fields-]{spaces-over-fields}
\externaldocument[spaces-topologies-]{spaces-topologies}
\externaldocument[spaces-descent-]{spaces-descent}
\externaldocument[spaces-perfect-]{spaces-perfect}
\externaldocument[spaces-more-morphisms-]{spaces-more-morphisms}
\externaldocument[spaces-flat-]{spaces-flat}
\externaldocument[spaces-groupoids-]{spaces-groupoids}
\externaldocument[spaces-more-groupoids-]{spaces-more-groupoids}
\externaldocument[bootstrap-]{bootstrap}
\externaldocument[spaces-pushouts-]{spaces-pushouts}
\externaldocument[groupoids-quotients-]{groupoids-quotients}
\externaldocument[spaces-more-cohomology-]{spaces-more-cohomology}
\externaldocument[spaces-simplicial-]{spaces-simplicial}
\externaldocument[formal-spaces-]{formal-spaces}
\externaldocument[restricted-]{restricted}
\externaldocument[spaces-resolve-]{spaces-resolve}
\externaldocument[formal-defos-]{formal-defos}
\externaldocument[defos-]{defos}
\externaldocument[cotangent-]{cotangent}
\externaldocument[examples-defos-]{examples-defos}
\externaldocument[algebraic-]{algebraic}
\externaldocument[examples-stacks-]{examples-stacks}
\externaldocument[stacks-sheaves-]{stacks-sheaves}
\externaldocument[criteria-]{criteria}
\externaldocument[artin-]{artin}
\externaldocument[quot-]{quot}
\externaldocument[stacks-properties-]{stacks-properties}
\externaldocument[stacks-morphisms-]{stacks-morphisms}
\externaldocument[stacks-limits-]{stacks-limits}
\externaldocument[stacks-cohomology-]{stacks-cohomology}
\externaldocument[stacks-perfect-]{stacks-perfect}
\externaldocument[stacks-introduction-]{stacks-introduction}
\externaldocument[stacks-more-morphisms-]{stacks-more-morphisms}
\externaldocument[stacks-geometry-]{stacks-geometry}
\externaldocument[moduli-]{moduli}
\externaldocument[moduli-curves-]{moduli-curves}
\externaldocument[examples-]{examples}
\externaldocument[exercises-]{exercises}
\externaldocument[guide-]{guide}
\externaldocument[desirables-]{desirables}
\externaldocument[coding-]{coding}
\externaldocument[obsolete-]{obsolete}
\externaldocument[fdl-]{fdl}
\externaldocument[index-]{index}

% Theorem environments.
%
\theoremstyle{plain}
\newtheorem{theorem}[subsection]{Theorem}
\newtheorem{proposition}[subsection]{Proposition}
\newtheorem{lemma}[subsection]{Lemma}

\theoremstyle{definition}
\newtheorem{definition}[subsection]{Definition}
\newtheorem{example}[subsection]{Example}
\newtheorem{exercise}[subsection]{Exercise}
\newtheorem{situation}[subsection]{Situation}

\theoremstyle{remark}
\newtheorem{remark}[subsection]{Remark}
\newtheorem{remarks}[subsection]{Remarks}

\numberwithin{equation}{subsection}

% Macros
%
\def\lim{\mathop{\rm lim}\nolimits}
\def\colim{\mathop{\rm colim}\nolimits}
\def\Spec{\mathop{\rm Spec}}
\def\Hom{\mathop{\rm Hom}\nolimits}
\def\Ext{\mathop{\rm Ext}\nolimits}
\def\SheafHom{\mathop{\mathcal{H}\!{\it om}}\nolimits}
\def\SheafExt{\mathop{\mathcal{E}\!{\it xt}}\nolimits}
\def\Sch{\textit{Sch}}
\def\Mor{\mathop{\rm Mor}\nolimits}
\def\Ob{\mathop{\rm Ob}\nolimits}
\def\Sh{\mathop{\textit{Sh}}\nolimits}
\def\NL{\mathop{N\!L}\nolimits}
\def\proetale{{pro\text{-}\acute{e}tale}}
\def\etale{{\acute{e}tale}}
\def\QCoh{\textit{QCoh}}
\def\Ker{\mathop{\rm Ker}}
\def\Im{\mathop{\rm Im}}
\def\Coker{\mathop{\rm Coker}}
\def\Coim{\mathop{\rm Coim}}

%
% Macros for moduli stacks/spaces
%
\def\QCohstack{\mathcal{QC}\!{\it oh}}
\def\Cohstack{\mathcal{C}\!{\it oh}}
\def\Spacesstack{\mathcal{S}\!{\it paces}}
\def\Quotfunctor{{\rm Quot}}
\def\Hilbfunctor{{\rm Hilb}}
\def\Curvesstack{\mathcal{C}\!{\it urves}}
\def\Polarizedstack{\mathcal{P}\!{\it olarized}}
\def\Complexesstack{\mathcal{C}\!{\it omplexes}}
% \Pic is the operator that assigns to X its picard group, usage \Pic(X)
% \Picardstack_{X/B} denotes the Picard stack of X over B
% \Picardfunctor_{X/B} denotes the Picard functor of X over B
\def\Pic{\mathop{\rm Pic}\nolimits}
\def\Picardstack{\mathcal{P}\!{\it ic}}
\def\Picardfunctor{{\rm Pic}}
\def\Deformationcategory{\mathcal{D}\!{\it ef}}


% OK, start here.
%
\begin{document}

\title{Examples}


\maketitle

\phantomsection
\label{section-phantom}

\tableofcontents

\section{Introduction}
\label{section-introduction}

\noindent
This chapters will contain examples which illuminate the theory.




\section{A Noetherian ring of infinite dimension}
\label{section-Noetherian-infinite-dimension}

\noindent
A Noetherian local ring has finite dimension as we saw in
Algebra, Proprosition \ref{algebra-proposition-dimension}.
But there exist Noetherian rings of infinite dimension.
See \cite[Appendix, Example 1]{Nagata}.

\medskip\noindent
Namely, let $k$ be a field, and consider the ring
$$
R = k[x_1, x_2, x_3, \ldots ].
$$
Let $\mathfrak p_i = (x_{2^{i - 1}}, x_{2^{i - 1} + 1}, \ldots, x_{2^i - 1})$
for $i = 1, 2, \ldots$ which are prime ideals of $R$. Let $S$
be the multiplicative subset
$$
S = \bigcap\nolimits_{i \geq 1} (R \setminus \mathfrak p_i).
$$
Consider the ring $A = S^{-1}R$.
We claim that
\begin{enumerate}
\item The maximal ideals of the ring $A$ are the ideals
$\mathfrak m_i = \mathfrak p_iA$.
\item We have $A_{\mathfrak m_i} = R_{\mathfrak p_i}$ which is
a Noetherian local ring of dimension $2^i$.
\item The ring $A$ is Noetherian.
\end{enumerate}
Hence it is clear that this is the example we are looking for.
Details omitted.




\section{Local rings with nonreduced completion}
\label{section-local-completion-nonreduced}

\noindent
In Algebra, Example \ref{algebra-example-bad-dvr-char-p} we gave an example
of a characteristic $p$ Noetherian local domain $R$ of dimension $1$
whose completion is nonreduced. In this section we present the example
of \cite[Proposition 3.1]{Ferrand-Raynaud} which gives a similar
ring in characteristic zero.

\medskip\noindent
Let $\mathbf{C}\{x\}$ be the ring of convergent power series over
the field $\mathbf{C}$ of complex numbers. The ring of all power series
$\mathbf{C}[[x]]$ is its completion. Let $K = \mathbf{C}\{x\}[1/x] = f.f.(B)$
be the field of convergent Laurent series. The $K$-module
$\Omega_{K/\mathbf{C}}$ of algebraic differentials
of $K$ over $\mathbf{C}$ is an infinite dimensional $K$-vector space
(proof omitted). We may choose $f_n \in x\mathbf{C}\{x\}$,
$n \geq 1$ such that
$
\text{d}x, \text{d}f_1, \text{d}f_2, \ldots
$
are part of a basis of $\Omega_{K/\mathbf{C}}$. Thus we can
find a $\mathbf{C}$-derivation
$$
D : \mathbf{C}\{x\} \longrightarrow \mathbf{C}((x))
$$
such that $D(x) = 0$ and $D(f_i) = x^{-n}$. Let
$$
A = \{f \in \mathbf{C}\{x\} \mid D(f) \in \mathbf{C}[[x]]\}
$$
We claim that
\begin{enumerate}
\item $\mathbf{C}\{x\}$ is integral over $A$,
\item $A$ is a local domain,
\item $\dim(A) = 1$,
\item the maximal ideal of $A$ is generated by $x$ and $xf_1$,
\item $A$ is Noetherian, and
\item the completion of $A$ is equal to the ring of dual numbers
over $\mathbf{C}[[x]]$.
\end{enumerate}
Since the dual numbers are nonreduced the ring $A$ gives the example.

\medskip\noindent
Note that if $0 \not = f \in x\mathbf{C}\{x\}$ then
we may write $D(f) = h/f^n$ for some $n \geq 0$ and $h \in \mathbf{C}[[x]]$.
Hence $D(f^{n + 1}/(n + 1)) \in \mathbf{C}[[x]]$
and $D(f^{n + 2}/(n + 2)) \in \mathbf{C}[[x]]$. Thus we
see $f^{n + 1}, f^{n + 2} \in A$!
In particular we see (1) holds. We also conclude that
the fraction field of $A$ is equal to the fraction field of
$\mathbf{C}\{x\}$. It also follows immediately that
$A \cap x\mathbf{C}\{x\}$ is the set of nonunits of $A$, hence
$A$ is a local domain of dimension $1$. If we can show (4)
then it will follow that $A$ is Noetherian (proof omitted).
Suppose that $f \in A \cap x\mathbf{C}\{x\}$. Write
$D(f) = h$, $h \in \mathbf{C}[[x]]$. Write $h = c + xh'$
with $c \in \mathbf{C}$, $h' \in \mathbf{C}[[x]]$. Then
$D(f - cxf_1) = c + xh' - c = xh'$. On the other hand
$f - cxf_1 = xg$ with $g \in \mathbf{C}\{x\}$, but by the
computation above we have $D(g) = h' \in \mathbf{C}[[x]]$
and hence $g \in A$. Thus $f = cxf_1 + xg \in (x, xf_1)$ as desired.

\medskip\noindent
Finally, why is the completion of $A$ nonreduced? Denote $\hat A$ the
completion of $A$. Of course this maps surjectively to the completion
$\mathbf{C}[[x]]$ of $\mathbf{C}\{x\}$ because $x \in A$. Denote
this map $\psi : \hat A \to \mathbf{C}[[x]]$.
Above we saw that $\mathfrak m_A = (x, xf_1)$
and hence $D(\mathfrak m_A^n) \subset (x^{n - 1})$ by an easy
computation. Thus $D : A \to \mathbf{C}[[x]]$ is continuous and
gives rise to a continuous derivation $\hat D : \hat A \to \mathbf{C}[[x]]$
over $\psi$. Hence we get a ring map
$$
\psi + \epsilon \hat D :
\hat A
\longrightarrow
\mathbf{C}[[x]][\epsilon].
$$
Since $\hat A$ is a one dimensional Noetherian complete local ring, if we
can show this arrow is surjective then it will follow that $\hat A$
is nonreduced. Actually the map is an isomorphism but we omit the
verification of this. The subring $\mathbf{C}[x]_{(x)} \subset A$
gives rise to a map $i : \mathbf{C}[[x]] \to \hat A$ on completions such
that $i \circ \psi = \text{id}$ and such that $D \circ i = 0$
(as $D(x) = 0$ by construction). Consider the elements $x^nf_n \in A$.
We have
$$
(\psi + \epsilon D)(x^nf_n) = x^n f_n + \epsilon
$$
for all $n \geq 1$. Surjectivity easily follows from these remarks.




\section{A non catenary Noetherian local ring}
\label{section-non-catenary-Noetherian-local}

\noindent
Even though there is a succesful dimension theory of Noetherian local rings
there are non-catenary Noetherian local rings. An example may be found in
\cite[Appendix, Example 2]{Nagata}. In fact, we will present this example
in the simplest case. Namely, we will construct a local Noetherian domain $A$
of dimension $2$ which is not universally catenary. (Note that $A$ is
automatically catenary, see
Exercise \ref{exercises-exercise-Noetherian-local-domain-dim-2-catenary}.)
The existence of a Noetherian local ring which is not universally
catenary implies the existence of a Noetherian local ring which
is not catenary -- and we spell this out at the end of this section
in the particular example at hand.

\medskip\noindent
Let $k$ be a field, and consider the formal power series ring
$k[[x]]$ in one variable over $k$. Let
$$
z = \sum\nolimits_{i = 1}^\infty a_i x^i
$$
be a formal power series. We assume $z$ as an element of the Laurent
series field $k((x)) = f.f.(k[[x]])$ is transcendental over $k(x)$.
Put
$$
z_j
=
x^{-j}(z - \sum\nolimits_{i = 1, \ldots, j - 1} a_i x^i)
=
\sum\nolimits_{i = j}^\infty a_i x^{i - j}
\in k[[x]].
$$
Note that $Z = z_1$.
Let $R$ be the subring of $k[[x]]$ generated by $x$, $z$ and all of the
$z_j$, in other words
$$
R = k[x, z_1, z_2, z_3, \ldots ] \subset k[[x]].
$$
Consider the ideals $\mathfrak m = (x)$ and
$\mathfrak n = (x - 1, z_1, z_2, \ldots)$ of $R$.

\medskip\noindent
We have $x(z_{j + 1} + a_j) = z_j$. Hence $R/\mathfrak m = k$
and $\mathfrak m$ is a maximal ideal. Moreover, any element of $R$
not in $\mathfrak m$ maps to a unit in $k[[x]]$ and hence
$R_{\mathfrak m} \subset k[[x]]$. In fact it is easy to deduce
that $R_{\mathfrak m}$ is a discrete valuation ring and residue
field $k$.

\medskip\noindent
We claim that
$$
R/(x - 1) =
k[x, z_1, z_2, z_3, \ldots ]/(x - 1)
\cong
k[z].
$$
Namely, the relation above implies that
$(x - 1)(z_{j + 1} + a_j) = -z_{j + 1} - a_j + z_j$, and hence
we may express the class of $z_{j + 1}$ in terms of $z_j$ in
the quotient $R/(x - 1)$. Since the fraction field of $R$
has transcendence degree $2$ over $k$ by construction we see that $z$ is
transcendental over $k$ in $R/(x - 1)$, whence the desired isomorphism.
Hence $\mathfrak n = (x - 1, z)$ and is a maximal ideal. In fact the
map
$$
k[x, x^{-1}, z]_{(x - 1, z)} \longrightarrow R_{\mathfrak n}
$$
is an isomorphism (since $x^{-1}$ is invertible in $R_{\mathfrak n}$
and since $z_{j + 1} = x^{-1}z_j - a_j = \ldots = f_j(x, x^{-1}, z)$).
This shows that $R_{\mathfrak n}$ is a regular local ring
of dimension $2$ and residue field $k$.

\medskip\noindent
Let $S$ be the multiplicative subset
$$
S =
(R \setminus \mathfrak m) \cap (R \setminus \mathfrak n) =
R \setminus (\mathfrak m \cup \mathfrak n)
$$
and set $B = S^{-1}R$. We claim that
\begin{enumerate}
\item The ring $B$ is a $k$-algebra.
\item The maximal ideals of the ring $B$ are the two ideals
$\mathfrak mB$ and $\mathfrak nB$.
\item The residue fields at these maximal ideals is $k$.
\item We have $B_{\mathfrak mB} = R_{\mathfrak m}$
and $B_{\mathfrak nB} = R_{\mathfrak n}$
which are Noetherian regular local rings of dimensions $1$ and $2$.
\item The ring $B$ is Noetherian.
\end{enumerate}
We omit the details of the verifications.

\medskip\noindent
Whenever given a $k$-algebra $B$ with the properties listed above we
get an example as follows. Take $A = k + \text{rad}(B) \subset B$,
in our case $\text{rad}(B) = \mathfrak mB + \mathfrak nB$.
It is easy to see that $B$ is finite over $A$ and hence $A$ is
Noetherian by Eakin's theorem (see \cite{Eakin}, or
\cite[Appendix A1]{Nagata}, or insert future reference here).
Also $A$ is a local domain with the same fraction field as $B$ and
residue field $k$. Since the dimension of $B$ is $2$ we see that $A$
has dimension $2$ as well, by
Algebra, Lemma \ref{algebra-lemma-integral-sub-dim-equal}.

\medskip\noindent
If $A$ were universally catenary then the dimension formula,
Algebra, Lemma \ref{algebra-lemma-dimension-formula}
would give $\dim(B_{\mathfrak mB}) = 2$ contradiction.

\medskip\noindent
Note that $B$ is generated by one element over $A$.
Hence $B = A[x]/\mathfrak p$ for some prime
$\mathfrak p$ of $A[x]$. Let $\mathfrak m' \subset A[x]$ be
the maximal ideal corresponding to $\mathfrak mB$. Then on
the one hand $\dim(A[x]_{\mathfrak m'}) = 3$ and on the
other hand
$$
(0)
\subset \mathfrak pA[x]_{\mathfrak m'}
\subset \mathfrak m'A[x]_{\mathfrak m'}
$$
is a maximal chain of primes. Hence $A[x]_{\mathfrak m'}$ is
an example of a non catenary Noetherian local ring.




\section{Non-quasi-affine variety with quasi-affine normalization}
\label{section-nonquasi-affine}

\noindent
The existence of an example of this kind is mentioned in
\cite[II Remark 6.6.13]{EGA}. They refer to the fifth volume of
EGA for such an example, but the fifth volume did not appear.

\medskip\noindent
Let $k$ be a field.
Let $Y = \mathbf{A}^2_k \setminus \{(0, 0)\}$.
We are going to construct a finite surjective birational morphism
$\pi : Y \longrightarrow X$
with $X$ a variety over $k$ such that $X$ is not quasi-affine.
Namely, consider the following curves in $Y$:
$$
\begin{matrix}
C_1 & : & x = 0 \\
C_2 & : & y = 0
\end{matrix}
$$
Note that $C_1 \cap C_2 = \emptyset$. We choose the isomorphism
$\varphi : C_1 \to C_2$, $(0, y) \mapsto (y^{-1}, 0)$.
We claim there is a unique morphism $\pi : Y \to X$ as above
such that
$$
\xymatrix{
C_1
\ar@<1ex>[rr]^{\text{id}} \ar@<-1ex>[rr]_{\varphi}
& &
Y \ar[r]^\pi & X
}
$$
is a coequalizer diagram in the category of varieties (and even in
the category of schemes). Accepting this for the moment let us
show that such an $X$ cannot be quasi-affine. Namely, it is clear
that we would get
$$
\Gamma(X, \mathcal{O}_X) =
\{ f \in k[x, y] \mid f(0, y) = f(y^{-1}, 0)\} =
k \oplus (xy) \subset k[x, y].
$$
In particular these functions do not separate the points $(1, 0)$
and $(-1, 0)$ whose images in $X$ (we will see below) are distinct
(if the characteristic of $k$ is not $2$).

\medskip\noindent
To show that $X$ exists consider the Zariski open
$D(x + y) \subset Y$ of $Y$. This is the spectrum
of the ring
$k[x, y, 1/(x + y)]$
and the curves $C_1$, $C_2$ are completely contained in
$D(x + y)$. Moreover the morphism
$$
C_1 \coprod C_2
\longrightarrow
D(x + y) \cap Y = \text{Spec}(k[x, y, 1/(x + y)])
$$
is a closed immersion. It follows from
Algebra, Lemma \ref{algebra-lemma-fibre-product-finite-type}
that the ring
$$
A =
\{f \in k[x, y, 1/(x + y)] \mid f(0, y) = f(y^{-1}, 0)\}
$$
is of finite type over $k$. On the other hand we have the open
$D(xy) \subset Y$ of $Y$ which is disjoint from the curves $C_1$
and $C_2$. It is the spectrum of the ring
$$
B = k[x, y, 1/xy].
$$
Note that we have $A_{xy} \cong B_{x + y}$ (since $A$ clearly contains
the elements $xyP(x, y)$ any polynomial $P$ and the element $xy/(x + y)$).
The scheme $X$ is obtained by glueing the affine schemes
$\text{Spec}(A)$ and $\text{Spec}(B)$ using the isomorphism
$A_{xy} \cong B_{x + y}$ and hence is clearly of finite type over
$k$. To see that it is separated one has to show that the
ring map $A \otimes_k B \to B_{x + y}$ is surjective. To see
this use that $A \otimes_k B$ contains the element
$xy/(x + y) \otimes 1/xy$ which maps to $1/(x + y)$.
The morphism $X \to Y$ is given by the natural maps
$D(x + y) \to \text{Spec}(A)$ and $D(xy) \to \text{Spec}(B)$.
Since these are both finite we deduce that $X \to Y$ is finite
as desired. We omit the verification that $X$ is indeed the
coequalizer of the displayed diagram above, however, see
(insert future reference for push outs in the category of schemes
here). Note that the morphism $\pi : Y \to X$ does
map the points  $(1, 0)$
and $(-1, 0)$ to distinct points in $X$ because the
function $(x + y^3)/(x + y)^2 \in A$ has value
$1/1$, resp.\ $-1/(-1)^2 = -1$ which are always distinct
(unless the characteristic is $2$ -- please find your own points
for characteristic $2$). We summarize this discussion in the
form of a lemma.

\begin{lemma}
\label{lemma-quasi-affine-normalization-not-quasi-affine}
Let $k$ be a field.
There exists a variety $X$ whose normalization is quasi-affine but
which is itself not quasi-affine.
\end{lemma}

\begin{proof}
See discussion above and (insert future reference on normalization here).
\end{proof}



\section{Non flasque quasi-coherent sheaf associated to injective module}
\label{section-nonflasque}

\noindent
For more examples of this type see \cite[Expos\'e II, Appendix I]{SGA6}
where Illusie explains some examples due to Verdier.

\medskip\noindent
Consider the affine scheme $X = \text{Spec}(A)$
where
$$
A = k[f, g, x, y, \{a_n, b_n\}_{n \geq1}]/
(fy - gx, \{a_nf^n + b_ng^n\}_{n \geq 1})
$$
is the ring from
Properties, Example \ref{properties-example-does-not-work-in-general}.
Set $I = (f, g) \subset A$.
Consider the quasi-compact open $U = D(f) \cup D(g)$ of $X$.
We have seen in loc.\ cit.\ that there is a section
$s \in \mathcal{O}_X(U)$ which does not come from an $A$-module
map $I^n \to A$ for any $n \geq 0$.

\medskip\noindent
Let $\alpha : A \to J$ be the embedding of $A$ into an injective $A$-module.
Let $Q = J/\alpha(A)$ and denote $\beta : J \to Q$ the quotient map.
We claim that the map
$$
\Gamma(X, \widetilde{J})
\longrightarrow
\Gamma(U, \widetilde{J})
$$
is not surjective. Namely, we claim that $\alpha(s)$ is not in the image.
To see this, we argue by contradiction. So assume that $x \in J$ is an
element which restricts to $\alpha(s)$ over $U$. Then $\beta(x) \in Q$
is an element which restricts to $0$ over $U$. Hence we know that
$I^n\beta(x) = 0$ for some $n$, see
Properties,
Lemma \ref{properties-lemma-sections-over-quasi-compact-open-in-affine}.
This implies that we get a morphism
$\varphi : I^n \to A$, $h \mapsto \alpha^{-1}(hx)$. It is easy to see that
this morphism $\varphi$ gives rise to the section $s$ via the map of
Properties,
Lemma \ref{properties-lemma-sections-over-quasi-compact-open-in-affine}
which is a contradiction.

\begin{lemma}
\label{lemma-nonflasque}
There exists an affine scheme $X = \text{Spec}(A)$ and an injective
$A$-module $J$ such that $\widetilde{J}$ is not a flasque sheaf on $X$.
Even the restriction $\Gamma(X, \widetilde{J}) \to \Gamma(U, \widetilde{J})$
with $U$ quasi-compact open need not be surjective.
\end{lemma}

\begin{proof}
See above.
\end{proof}



\section{Examples of stacks in groupoids}
\label{section-examples-stacks}

\noindent
Here are some algebraic geometric examples of stacks in groupoids.
In these examples we
do not worry about whether the stacks mentioned are algebraic stacks.
We simply think of them as stacks over $(\textit{Sch}/S)_{fppf}$.
(In all the examples we suppose given a big fppf site
$\textit{Sch}_{fppf}$, and that all schemes considered are contained
in this site. In particular the base scheme $S$ is contained
in this site.)

\medskip\noindent
In the following example we use $G$-torsors; these are defined in
Groupoids, Definition \ref{groupoids-definition-principal-homogeneous-space}.

\begin{example}
\label{example-X-mod-G-groupscheme}
Let $S$ be a scheme.
Let $G \to S$ be a group scheme over $S$.
Assume that $G \to S$ is affine.
Let $X \to S$ be a scheme over $S$.
Let $a : G \times_S X \to X$
be an action of $G$ on $X$ over $S$.
The {\it quotient stack} $[X/G]$ is defined as follows.
\begin{enumerate}
\item An object of $[X/G]$ consists of a triple
$(T \to S, P \to T, \varphi : P \to X)$ where
\begin{enumerate}
\item $T \to S$ is a scheme over $S$,
\item $P \to T$ is a $G_T$-torsor in fppf topology\footnote{We can also
consider here instead all $G_T$-torsors, i.e., those pseudo torsors which
become trivial over the members of an fpqc covering of $T$. This would lead
to a second stack in groupoids $[X/G]'$ over $(\textit{Sch}/S)_{fppf}$.
In general the two stacks obtained do not agree, see
Section \ref{section-torsor-not-fppf}.
But if $G \to S$ is flat and of finite presentation,
then they do (insert future reference here). If $G$ is not
of finite type over $S$, then neither $[X/G]'$ nor $[X/G]$ will be
an algebraic stack in general, see
Section \ref{section-not-algebraic-stack}.
The upshot is that in many interesting cases the difference is irrelevant.}
over $T$, and
\item $\varphi : P \to X$ is a $G$-equivariant morphism of schemes.
\end{enumerate}
\item A morphism
$(h, f) : (T/S, P/T, \varphi) \to (T'/S, P'/T', \varphi')$
is given by a morphism of schemes $f : T \to T'$ and a $G$-equivariant
morphism $h : P \to P'$ over $f$ which induces an isomorphism
$P \cong T \times_{T'} P'$, and such that $\varphi = \varphi' \circ h$.
\item the functor $[X/G] \to (\textit{Sch}/S)_{fppf}$ is the forgetful
functor $(T/S, P/T, \varphi) \mapsto T/S$.
\end{enumerate}
It is not so hard to verify that this is a fibred category. (Omitted.)
This is a stack in groupoids for the fppf topology. We omit the proof.
The hardest part is to show descent for objects. To do this one shows that
a torsor for an affine group scheme is affine over its base (insert future
reference here), and we have descent for affine schemes over schemes (see
Descent, Lemma \ref{descent-lemma-affine}).
\end{example}

\noindent
The following example is almost, but not quite the same.
The difference is that in the example above $G$ is an affine group scheme
over $S$, and in the example below $G$ is an abstract group.
The key to the examples is that in both cases all
$\underline{G}$-torsors on $(\textit{Sch}/S)_{fppf}$ are representable.

\begin{example}
\label{example-X-mod-G-group}
Let $S$ be a scheme.
Let $G$ be a group.
Let $X \to S$ be a scheme over $S$.
Let $a : G \times X \to X$ be an action of $G$ on $X$ over $S$.
The stack $[X/G]$ is defined as follows.
\begin{enumerate}
\item An object of $[X/G]$ consists of a triple
$(T \to S, P \to T, \varphi : P \to X)$ where
\begin{enumerate}
\item $T \to S$ is a scheme over $S$,
\item $P \to T$ is a $G_T$-torsor over $T$ in the fppf topology, and
\item $\varphi : P \to X$ is a $G$-equivariant morphism of schemes.
\end{enumerate}
\item A morphism
$(h, f) : (T/S, P/T, \varphi) \to (T'/S, P'/T', \varphi')$
is given by a morphism of schemes $f : T \to T'$ and a $G$-equivariant
morphism $h : P \to P'$ over $f$ which induces an isomorphism
$P \cong T \times_{T'} P'$, and such that $\varphi = \varphi' \circ h$.
\item the functor $[X/G] \to (\textit{Sch}/S)_{fppf}$ is the forgetful
functor $(T/S, P/T, \varphi) \mapsto T/S$.
\end{enumerate}
It is not so hard to verify that this is a fibred category. (Omitted.)
This is a stack in groupoids for the fppf topology. We omit the proof.
The hardest part is to show descent for objects.
To do this, note that a torsor for an abstract group
is a separated and etale scheme over its base, and by
More on Morphisms, Lemma
\ref{more-morphisms-lemma-separated-locally-quasi-finite-morphisms-fppf-descend}
we have descent for separated and locally quasi-finite schemes over schemes.
\end{example}

\begin{example}
\label{example-picard-stack}
Let $f : X \to S$ be a morphism of schemes.
The {\it Picard stack} $\textit{Pic}_{X/S}$ is defined as follows
\begin{enumerate}
\item An object is a pair $(T/S, \mathcal{L})$, where $T/S$ is a
scheme over $S$ and $\mathcal{L}$ is in invertible sheaf on
the base change $X_T$,
\item a morphism $(f, \alpha) : (T/S, \mathcal{L}) \to (T'/S, \mathcal{L}')$
is given by a morphism of schemes $f : T \to T'$ over $S$ and an
isomorphism $\alpha : f^*\mathcal{L}' \to \mathcal{L}$, and
\item the structure functor $\textit{Pic}_{X/S} \to (\textit{Sch}/S)_{fppf}$
is given by the forgetful functor $(T/S, \mathcal{L}) \mapsto T/S$.
\end{enumerate}
It is not hard to verify that this is a fibred category (omitted).
In fact it is a stack in groupoids for the fppf topology.
As usual, the hardest part is to show descent for objects.
But, if $\{T_i \to T\}$ is an fppf covering for $T$, then the
pullbacks $\{X_{T_i} \to X_T\}$ form an fppf covering as well.
Hence given invertible sheaves $\mathcal{L}_i$ on the schemes
$X_{T_i}$ and isomorphisms between their pullbacks over the schemes
$$
X_{T_i \times_T T_j} = X_{T_i} \times_{X_T} X_{T_i}
$$
satisfying a suitable cocycle condition we can descend the
$\mathcal{L}_i$ to an invertible sheaf $\mathcal{L}$ over $X_T$ using
Descent, Proposition \ref{descent-proposition-fpqc-descent-quasi-coherent}. 
Details omitted.
\end{example}


\section{Examples of inertia stacks}
\label{section-examples-inertia}

\noindent
Here are some examples of inertia stacks. In these examples we
do not worry about whether the stacks mentioned are algebraic stacks.
We simply think of them as stacks over $(\textit{Sch}/S)_{fppf}$.

\begin{example}
\label{example-inertia-stack-of-X-mod-G}
Let $S$ be a scheme. Let $G$ be a commutative group.
Let $X \to S$ be a scheme over $S$.
Let $a : G \times X \to X$ be an action of $G$ on $X$.
For $g \in G$ we denote $g : X \to X$ the corresponding automorphism.
In this case the inertia stack of $[X/G]$
(see Example \ref{example-X-mod-G-group})
is given by
$$	
I_{[X/G]} = \coprod\nolimits_{g\in G} [X^g/G],
$$	
where, given an element $g$ of $G$, the symbol $X^g$ denotes the
scheme $X^g = \{x \in X \mid g(x) = x\}$. In a formula
$X^g$ is really the fibre
product
$$
X^g =  X \times_{(1, 1), X \times_S X, (g, 1)} X.
$$ 
Indeed, for any $S$-scheme $T$, a
$T$-point on the inertia stack of $[X/G]$ consists of a
triple $(P/T, \phi, \alpha)$ consisting of a $G$-torsor
$P\to T$ together with a $G$-equivariant isomorphism
$\phi : P \to X$, together
with an automorphism $\alpha$ of $P\to T$ over $T$ such that
$\phi \circ \alpha = \phi$.
Since $G$ is a sheaf of \emph{commutative} groups,
$\alpha$ is, locally in the fppf topology over $T$,
given by multiplication by some element $g$ of $G$.
The condition that $\phi \circ \alpha = \phi$ means that $\phi$
factors through the inclusion of $X^g$
in $X$, i.e., $\phi$ is obtained by composing that inclusion with a
morphism $P \to X^\gamma$.
The above discussion allows us to define a morphism of fibred categories
$I_{[X/G]} \to \coprod_{g\in G} [X^g/G]$ given on $T$-points by the discussion
above. We omit showing that this is an equivalence.
\end{example}

\begin{example}
\label{example-inertia-stack-of-picard}
Let $X\to S$ be a morphism of schemes.
Assume that for any $T \to S$ the base change $f_T : X_T \to T$
has the property that the map $\mathcal{O}_T \to f_{T, *}\mathcal{O}_{X_T}$
is an isomorphism. (This implies that $f$ is
{\it cohomologically flat in dimension $0$} (insert future reference here)
but is stronger.) Consider the Picard stack $\textit{Pic}_{X/S}$, see
Example \ref{example-picard-stack}.
The points of its inertia stack over an
$S$-scheme $T$ consist of pairs $(\mathcal{L}, \alpha)$
where $\mathcal{L}$ is a line bundle
on $X_T$ and $\alpha$ is an automorphism of that line bundle.
I.e., we can think of $\alpha$ as an element of
$H^0(X_T, \mathcal{O}_{X_T})^\times = H^0(T, \mathcal{O}_T^*)$
by our condition. Note that $H^0(T,\mathcal{O}_T^*) = \mathbf{G}_{m,S}(T)$,
see Groupoids, Example \ref{groupoids-example-multiplicative-group}.
Hence the inertia stack of $\textit{Pic}_{X/S}$ is
$$
I_{\textit{Pic}_{X/S}} = \mathbf{G}_{m,S} \times_S \textit{Pic}_{X/S}.
$$
as a stack over $(\textit{Sch}/S)_{fppf}$.
\end{example}













\section{Other chapters}

\begin{multicols}{2}
\begin{enumerate}
\item \hyperref[introduction-section-phantom]{Introduction}
\item \hyperref[conventions-section-phantom]{Conventions}
\item \hyperref[sets-section-phantom]{Set Theory}
\item \hyperref[categories-section-phantom]{Categories}
\item \hyperref[topology-section-phantom]{Topology}
\item \hyperref[sheaves-section-phantom]{Sheaves on Spaces}
\item \hyperref[algebra-section-phantom]{Commutative Algebra}
\item \hyperref[sites-section-phantom]{Sites and Sheaves}
\item \hyperref[homology-section-phantom]{Homological Algebra}
\item \hyperref[derived-section-phantom]{Derived Categories}
\item \hyperref[more-algebra-section-phantom]{More Algebra}
\item \hyperref[simplicial-section-phantom]{Simplicial Methods}
\item \hyperref[modules-section-phantom]{Sheaves of Modules}
\item \hyperref[sites-modules-section-phantom]{Modules on Sites}
\item \hyperref[injectives-section-phantom]{Injectives}
\item \hyperref[cohomology-section-phantom]{Cohomology of Sheaves}
\item \hyperref[sites-cohomology-section-phantom]{Cohomology on Sites}
\item \hyperref[hypercovering-section-phantom]{Hypercoverings}
\item \hyperref[schemes-section-phantom]{Schemes}
\item \hyperref[constructions-section-phantom]{Constructions of Schemes}
\item \hyperref[properties-section-phantom]{Properties of Schemes}
\item \hyperref[morphisms-section-phantom]{Morphisms of Schemes}
\item \hyperref[coherent-section-phantom]{Coherent Cohomology}
\item \hyperref[divisors-section-phantom]{Divisors}
\item \hyperref[limits-section-phantom]{Limits of Schemes}
\item \hyperref[varieties-section-phantom]{Varieties}
\item \hyperref[chow-section-phantom]{Chow Homology}
\item \hyperref[topologies-section-phantom]{Topologies on Schemes}
\item \hyperref[descent-section-phantom]{Descent}
\item \hyperref[more-morphisms-section-phantom]{More on Morphisms}
\item \hyperref[flat-section-phantom]{More on Flatness}
\item \hyperref[groupoids-section-phantom]{Groupoid Schemes}
\item \hyperref[more-groupoids-section-phantom]{More on Groupoid Schemes}
\item \hyperref[etale-section-phantom]{\'Etale Morphisms of Schemes}
\item \hyperref[etale-cohomology-section-phantom]{\'Etale Cohomology}
\item \hyperref[spaces-section-phantom]{Algebraic Spaces}
\item \hyperref[spaces-properties-section-phantom]{Properties of Algebraic Spaces}
\item \hyperref[spaces-morphisms-section-phantom]{Morphisms of Algebraic Spaces}
\item \hyperref[spaces-topologies-section-phantom]{Topologies on Algebraic Spaces}
\item \hyperref[spaces-descent-section-phantom]{Descent and Algebraic Spaces}
\item \hyperref[spaces-more-morphisms-section-phantom]{More on Morphisms of Spaces}
\item \hyperref[quot-section-phantom]{Quot and Hilbert Spaces}
\item \hyperref[stacks-section-phantom]{Stacks}
\item \hyperref[spaces-groupoids-section-phantom]{Groupoids in Algebraic Spaces}
\item \hyperref[spaces-more-groupoids-section-phantom]{More on Groupoids in Spaces}
\item \hyperref[bootstrap-section-phantom]{Bootstrap}
\item \hyperref[examples-stacks-section-phantom]{Examples of Stacks}
\item \hyperref[groupoids-quotients-section-phantom]{Quotients of Groupoids}
\item \hyperref[algebraic-section-phantom]{Algebraic Stacks}
\item \hyperref[criteria-section-phantom]{Criteria for Representability}
\item \hyperref[stacks-properties-section-phantom]{Properties of Algebraic Stacks}
\item \hyperref[stacks-morphisms-section-phantom]{Morphisms of Algebraic Stacks}
\item \hyperref[examples-section-phantom]{Examples}
\item \hyperref[exercises-section-phantom]{Exercises}
\item \hyperref[guide-section-phantom]{Guide to Literature}
\item \hyperref[desirables-section-phantom]{Desirables}
\item \hyperref[coding-section-phantom]{Coding Style}
\item \hyperref[fdl-section-phantom]{GNU Free Documentation License}
\item \hyperref[index-section-phantom]{Auto Generated Index}
\end{enumerate}
\end{multicols}


\bibliography{my}
\bibliographystyle{amsalpha}

\end{document}
