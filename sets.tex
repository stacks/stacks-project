\IfFileExists{stacks-project.cls}{%
\documentclass{stacks-project}
}{%
\documentclass{amsart}
}

% The following AMS packages are automatically loaded with
% the amsart documentclass:
%\usepackage{amsmath}
%\usepackage{amssymb}
%\usepackage{amsthm}

% For dealing with references we use the comment environment
\usepackage{verbatim}
\newenvironment{reference}{\comment}{\endcomment}
%\newenvironment{reference}{}{}
\newenvironment{slogan}{\comment}{\endcomment}
\newenvironment{history}{\comment}{\endcomment}

% For commutative diagrams you can use
% \usepackage{amscd}
\usepackage[all]{xy}

% We use 2cell for 2-commutative diagrams.
\xyoption{2cell}
\UseAllTwocells

% To put source file link in headers.
% Change "template.tex" to "this_filename.tex"
% \usepackage{fancyhdr}
% \pagestyle{fancy}
% \lhead{}
% \chead{}
% \rhead{Source file: \url{template.tex}}
% \lfoot{}
% \cfoot{\thepage}
% \rfoot{}
% \renewcommand{\headrulewidth}{0pt}
% \renewcommand{\footrulewidth}{0pt}
% \renewcommand{\headheight}{12pt}

\usepackage{multicol}

% For cross-file-references
\usepackage{xr-hyper}

% Package for hypertext links:
\usepackage{hyperref}

% For any local file, say "hello.tex" you want to link to please
% use \externaldocument[hello-]{hello}
\externaldocument[introduction-]{introduction}
\externaldocument[conventions-]{conventions}
\externaldocument[sets-]{sets}
\externaldocument[categories-]{categories}
\externaldocument[topology-]{topology}
\externaldocument[sheaves-]{sheaves}
\externaldocument[sites-]{sites}
\externaldocument[stacks-]{stacks}
\externaldocument[fields-]{fields}
\externaldocument[algebra-]{algebra}
\externaldocument[brauer-]{brauer}
\externaldocument[homology-]{homology}
\externaldocument[derived-]{derived}
\externaldocument[simplicial-]{simplicial}
\externaldocument[more-algebra-]{more-algebra}
\externaldocument[smoothing-]{smoothing}
\externaldocument[modules-]{modules}
\externaldocument[sites-modules-]{sites-modules}
\externaldocument[injectives-]{injectives}
\externaldocument[cohomology-]{cohomology}
\externaldocument[sites-cohomology-]{sites-cohomology}
\externaldocument[dga-]{dga}
\externaldocument[dpa-]{dpa}
\externaldocument[hypercovering-]{hypercovering}
\externaldocument[schemes-]{schemes}
\externaldocument[constructions-]{constructions}
\externaldocument[properties-]{properties}
\externaldocument[morphisms-]{morphisms}
\externaldocument[coherent-]{coherent}
\externaldocument[divisors-]{divisors}
\externaldocument[limits-]{limits}
\externaldocument[varieties-]{varieties}
\externaldocument[topologies-]{topologies}
\externaldocument[descent-]{descent}
\externaldocument[perfect-]{perfect}
\externaldocument[more-morphisms-]{more-morphisms}
\externaldocument[flat-]{flat}
\externaldocument[groupoids-]{groupoids}
\externaldocument[more-groupoids-]{more-groupoids}
\externaldocument[etale-]{etale}
\externaldocument[chow-]{chow}
\externaldocument[intersection-]{intersection}
\externaldocument[pic-]{pic}
\externaldocument[adequate-]{adequate}
\externaldocument[dualizing-]{dualizing}
\externaldocument[duality-]{duality}
\externaldocument[discriminant-]{discriminant}
\externaldocument[local-cohomology-]{local-cohomology}
\externaldocument[curves-]{curves}
\externaldocument[resolve-]{resolve}
\externaldocument[models-]{models}
\externaldocument[pione-]{pione}
\externaldocument[etale-cohomology-]{etale-cohomology}
\externaldocument[proetale-]{proetale}
\externaldocument[crystalline-]{crystalline}
\externaldocument[spaces-]{spaces}
\externaldocument[spaces-properties-]{spaces-properties}
\externaldocument[spaces-morphisms-]{spaces-morphisms}
\externaldocument[decent-spaces-]{decent-spaces}
\externaldocument[spaces-cohomology-]{spaces-cohomology}
\externaldocument[spaces-limits-]{spaces-limits}
\externaldocument[spaces-divisors-]{spaces-divisors}
\externaldocument[spaces-over-fields-]{spaces-over-fields}
\externaldocument[spaces-topologies-]{spaces-topologies}
\externaldocument[spaces-descent-]{spaces-descent}
\externaldocument[spaces-perfect-]{spaces-perfect}
\externaldocument[spaces-more-morphisms-]{spaces-more-morphisms}
\externaldocument[spaces-flat-]{spaces-flat}
\externaldocument[spaces-groupoids-]{spaces-groupoids}
\externaldocument[spaces-more-groupoids-]{spaces-more-groupoids}
\externaldocument[bootstrap-]{bootstrap}
\externaldocument[spaces-pushouts-]{spaces-pushouts}
\externaldocument[groupoids-quotients-]{groupoids-quotients}
\externaldocument[spaces-more-cohomology-]{spaces-more-cohomology}
\externaldocument[spaces-simplicial-]{spaces-simplicial}
\externaldocument[formal-spaces-]{formal-spaces}
\externaldocument[restricted-]{restricted}
\externaldocument[spaces-resolve-]{spaces-resolve}
\externaldocument[formal-defos-]{formal-defos}
\externaldocument[defos-]{defos}
\externaldocument[cotangent-]{cotangent}
\externaldocument[examples-defos-]{examples-defos}
\externaldocument[algebraic-]{algebraic}
\externaldocument[examples-stacks-]{examples-stacks}
\externaldocument[stacks-sheaves-]{stacks-sheaves}
\externaldocument[criteria-]{criteria}
\externaldocument[artin-]{artin}
\externaldocument[quot-]{quot}
\externaldocument[stacks-properties-]{stacks-properties}
\externaldocument[stacks-morphisms-]{stacks-morphisms}
\externaldocument[stacks-limits-]{stacks-limits}
\externaldocument[stacks-cohomology-]{stacks-cohomology}
\externaldocument[stacks-perfect-]{stacks-perfect}
\externaldocument[stacks-introduction-]{stacks-introduction}
\externaldocument[stacks-more-morphisms-]{stacks-more-morphisms}
\externaldocument[stacks-geometry-]{stacks-geometry}
\externaldocument[moduli-]{moduli}
\externaldocument[moduli-curves-]{moduli-curves}
\externaldocument[examples-]{examples}
\externaldocument[exercises-]{exercises}
\externaldocument[guide-]{guide}
\externaldocument[desirables-]{desirables}
\externaldocument[coding-]{coding}
\externaldocument[obsolete-]{obsolete}
\externaldocument[fdl-]{fdl}
\externaldocument[index-]{index}

% Theorem environments.
%
\theoremstyle{plain}
\newtheorem{theorem}[subsection]{Theorem}
\newtheorem{proposition}[subsection]{Proposition}
\newtheorem{lemma}[subsection]{Lemma}

\theoremstyle{definition}
\newtheorem{definition}[subsection]{Definition}
\newtheorem{example}[subsection]{Example}
\newtheorem{exercise}[subsection]{Exercise}
\newtheorem{situation}[subsection]{Situation}

\theoremstyle{remark}
\newtheorem{remark}[subsection]{Remark}
\newtheorem{remarks}[subsection]{Remarks}

\numberwithin{equation}{subsection}

% Macros
%
\def\lim{\mathop{\rm lim}\nolimits}
\def\colim{\mathop{\rm colim}\nolimits}
\def\Spec{\mathop{\rm Spec}}
\def\Hom{\mathop{\rm Hom}\nolimits}
\def\Ext{\mathop{\rm Ext}\nolimits}
\def\SheafHom{\mathop{\mathcal{H}\!{\it om}}\nolimits}
\def\SheafExt{\mathop{\mathcal{E}\!{\it xt}}\nolimits}
\def\Sch{\textit{Sch}}
\def\Mor{\mathop{\rm Mor}\nolimits}
\def\Ob{\mathop{\rm Ob}\nolimits}
\def\Sh{\mathop{\textit{Sh}}\nolimits}
\def\NL{\mathop{N\!L}\nolimits}
\def\proetale{{pro\text{-}\acute{e}tale}}
\def\etale{{\acute{e}tale}}
\def\QCoh{\textit{QCoh}}
\def\Ker{\mathop{\rm Ker}}
\def\Im{\mathop{\rm Im}}
\def\Coker{\mathop{\rm Coker}}
\def\Coim{\mathop{\rm Coim}}

%
% Macros for moduli stacks/spaces
%
\def\QCohstack{\mathcal{QC}\!{\it oh}}
\def\Cohstack{\mathcal{C}\!{\it oh}}
\def\Spacesstack{\mathcal{S}\!{\it paces}}
\def\Quotfunctor{{\rm Quot}}
\def\Hilbfunctor{{\rm Hilb}}
\def\Curvesstack{\mathcal{C}\!{\it urves}}
\def\Polarizedstack{\mathcal{P}\!{\it olarized}}
\def\Complexesstack{\mathcal{C}\!{\it omplexes}}
% \Pic is the operator that assigns to X its picard group, usage \Pic(X)
% \Picardstack_{X/B} denotes the Picard stack of X over B
% \Picardfunctor_{X/B} denotes the Picard functor of X over B
\def\Pic{\mathop{\rm Pic}\nolimits}
\def\Picardstack{\mathcal{P}\!{\it ic}}
\def\Picardfunctor{{\rm Pic}}
\def\Deformationcategory{\mathcal{D}\!{\it ef}}


% OK, start here.
%
\begin{document}

\title{Set theory}

%\begin{abstract}
%\end{abstract}

\maketitle

\tableofcontents

\section{Introduction}
\label{section-introduction}

\noindent
We need some set theory every now and then. We use Zermelo-Fraenkel set theory
with the axiom of choice (ZFC) as described in \cite{Kunen} and \cite{Jech}.
Since we are talking about potentially large objects
(categories and categories of categories) we should be careful.

\section{Everything is a set}
\label{section-sets-everything}

\noindent
Most mathematiciens think of set theory as providing the basic
foundations for mathematics. So how does this really work?
For example, how do we translate the sentence
``$X$ is a scheme'' into set theory? Well, we just unravel the
definitions: A scheme is a locally ringed space such that every
point has an open neighbourhood which is an affine scheme. 
A locally ringed space is a ringed space such that every stalk
of the structure sheaf is a local ring. A ringed space is
a pair $(X, \mathcal{O}_X)$ consisting of a topological space
$X$ and a sheaf of rings $\mathcal{O}_X$ on it. A topological
space is a pair $(X, \tau)$ consisiting of a set
$X$ and a set of subsets $\tau \subset \mathcal{P}(X)$ 
satisfying the axions of a topology. And so on and
so forth.

\medskip\noindent
So how, given a set $S$ would we recognize whether it is a scheme?
The first thing we look for is whether the set $S$ is an ordered pair.
This is defined (see \cite{Jech}, page 7) as saying that $S$
has the form $(a,b) := \{\{a\},\{a,b\}\}$ for some sets $a, b$. If this is
the case, then we would take a look to see whether $a$ is an
ordered pair $(c,d)$. If so we would check whether 
$d \subset \mathcal{P}(c)$, and if so whether $d$ forms the collection
of sets for a topology on the set $c$. And so on and so forth.

\medskip\noindent
So even though it would take a considerable amount of work to write
a complete formula $\phi_{scheme}(x)$ with one free variable $x$ in set theory 
that expresses the notion ``$x$ is a scheme'', it is possible to do so.
The same thing should be true for any mathematical object.

\section{Classes}
\label{section-classes}

\noindent
Informally we use the notion of a {\it class}. Given a formula
$\phi(x,p_1,\ldots,p_n)$ we call
$$
C = \{x : \phi(x,p_1,\ldots,p_n)\}
$$
a {\it class}. A class is easier to manipulate than the formula
that defines it but it is not strictly speaking a mathematical
object. For example, if $R$ is a ring then we may
consider the class of all $R$-modules (since after all we
may translate the sentence ``$M$ is an $R$-module''
into a formula in set theory which then defines a class).
A {\it proper class} is a class which is not a set.

\medskip\noindent
In this way we may consider the category of $R$-modules
which is a ``big'' category, in other words it has a
proper class of objects. Similarly we may consider
the ``big'' category of schemes, the ``big'' category
of rings, etc.

\section{The hierarchy of sets}
\label{section-sets-hierarchy}

\noindent
A set $T$ is {\it transitive} if $x\in T$ implies $x\subset T$.
A set $\alpha$ is an {\it ordinal} if it is transitive and well-ordered
by $\in$. In this case we define $\alpha + 1 = \alpha \cup \{\alpha\}$,
which is another ordinal called the {\it successor} of $\alpha$.
An ordinal $\alpha$ is called a {\it successor ordinal} if 
there exists an ordinal $\beta$ such that $\alpha = \beta + 1$.
If $\alpha$ is not a successor ordinal, then $\alpha$ is called
a {\it limit ordinal} and we have
$$
\alpha 
=
\bigcup\nolimits_{\gamma \in \alpha} \gamma
$$
We define, by transfinite induction, $V_0 = \emptyset$,
$V_{\alpha + 1} = P(V_\alpha)$ (power set),
and for a limit ordinal $\alpha$,
$$
V_\alpha = \bigcup\nolimits_{\beta < \alpha} V_\beta.
$$
Note that each $V_\alpha$ is a transitive set.

\begin{lemma}
\label{axiom-regularity}
(See \cite{Jech}, Lemma 6.3.)
Every set is an element of $V_\alpha$ for some ordinal $\alpha$.
\end{lemma}

\noindent
In \cite[Chapter III]{Kunen} it is explained that this lemma is
equivalent to the axiom of foundation.

\section{Cardinality}
\label{section-cardinals}

\noindent
The {\it cardinality} of a set $A$ is the least ordinal $\alpha$
such that there exists a bijection between $A$ and $\alpha$.
We say an ordinal $\alpha$ is a {\it cardinal} if and only
if it occurs as the cardinality of some set, i.e.,
$\alpha = |\alpha|$. We use the greek letters $\kappa$, $\lambda$
for cardinals. The first infinite ordinal is denoted $\omega$; it is
also the first inifinite cardinal, and in this context it is
denoted $\aleph_0$.

\section{Reflection principle}
\label{section-reflection-principle}

\noindent
Some of this material is in the chapter of \cite{Kunen} called
``Easy consistency proofs''.

\medskip\noindent
Let $\phi(x_1,\ldots,x_n)$ be a formula of set theory.
Let us use the convention that this notation implies that
all the free variables in $\phi$ occur among $x_1, \ldots, x_n$.
Let $M$ be a set.
The formula $\phi^M(x_1, \ldots, x_n)$ is the
formula obtained from $\phi(x_1, \ldots, x_n)$ by replacing all the
$\forall x$ and $\exists x$ by $\forall x\in M$ and $\exists x\in M$.
So the formula
$\phi(x_1,x_2) = \exists x (x\in x_1 \wedge x\in x_2)$
is turned  into
$\phi^M(x_1,x_2) = \exists x \in M (x\in x_1 \wedge x\in x_2)$.
The formula $\phi^M$ is called the {\it relativization of $\phi$
to $M$}.

\begin{theorem}
\label{theorem-reflection-principle}
See \cite[Theorem 12.14]{Jech} or \cite[Theorem 7.4]{Kunen}.
Suppose given $\phi_1(x_1, \ldots, x_n), \ldots, \phi_m(x_1, \ldots, x_n)$
a {\bf finite} collection of
formulas of set theory. Let $M_0$ be a set.
There exists a set $M$ such that
$M_0 \subset M$ and
$\forall x_1, \ldots, x_n \in M$, we have
$$
\forall i = 1, \ldots, m,\  
\phi_i^{M}(x_1,\ldots,x_n)
\Leftrightarrow
\forall i = 1, \ldots, m,\  
\phi_i(x_1,\ldots,x_n).
$$
In fact we may take $M = V_\alpha$ for some limit ordinal $\alpha$.
\end{theorem}

\noindent
We view this theorem as saying the following: Given any
$x_1, \ldots, x_n \in M$ the formulas hold with the bound variables ranging
through all sets if and only if they hold for the bound variables ranging
through elements of $V_\alpha$. This theorem is a meta-theorem, since
it deals with the formulas of set theory directly.
It actually says that given the finite list of formulas
$\phi_1, \ldots, \phi_m$ with at most free variables $x_1, \ldots, x_n$
the sentence
$$
\begin{matrix}
\forall M_0\ \exists M,\ M_0 \subset M\ \forall x_1, \ldots, x_n \in M \\
\phi_1(x_1,\ldots,x_n) \wedge \ldots \wedge \phi_m(x_1,\ldots,x_n)
\leftrightarrow
\phi_1^M(x_1,\ldots,x_n) \wedge \ldots \wedge \phi_m^M(x_1,\ldots,x_n)
\end{matrix}
$$
is provable in ZFC. In other words, whenever we actually write down
a finite list of formulas $\phi_i$ we get a theorem.

\section{Constructing categories of schemes}
\label{section-fibre-product-schemes}

\noindent
We will discuss how to apply this to produce, given an initial
set of schemes, a ``small'' category of schemes closed under
a list of natural operations. Before we do so we introduce the
size of a scheme. Given a scheme $S$ we define
$$
\text{size}(S) = \max(\aleph_0, \kappa_1, \kappa_2)
$$
where we define the cardinal numbers $\kappa_1$ and $\kappa_2$ as follows
\begin{enumerate}
\item We let $\kappa_1$ be the cardinality of the set of affine opens of $S$.
\item We let $\kappa_2$ be the supremum of all the cardinalities of
all $\Gamma(U, \mathcal{O}_S)$ for all $U \subset S$ open.
\end{enumerate}

\begin{lemma}
\label{lemma-bounded-size}
For every cardinal $\kappa$ there exists a set $A$ such
that every element of $A$ is a scheme, and such that for every
scheme $S$ with $\text{Size}(S) \leq \kappa$ there is
an element $X \in A$ such that $X \cong S$ (isomorphism
of schemes).
\end{lemma}

\begin{proof}
Omitted. Hint: think about how any scheme is isomorphic to a scheme
obtained by glueing affines.
\end{proof}

\noindent
We denote $Bound$ the function which to each
cardinal $\kappa$ associates $Bound(\kappa) = \kappa^{\aleph_0}$.
We could make this function grow much more rapidly, e.g., we could
set $Bound(\kappa) = \kappa^\kappa$, and the result below would still hold.
For any ordinal $\alpha$ we denote $\textit{Sch}_\alpha$ the full
subcategory of category of schemes whose objects are elements of
$V_\alpha$. Here is the result we are going to prove.

\begin{lemma}
\label{lemma-construct-category}
With notations $\text{size}$, $f$ and $\textit{Sch}_\alpha$ as above.
Let $S_0$ be a set of schemes. There exists a limit ordinal
$\alpha$ with the following properties:
\begin{enumerate}
\item We have $S_0 \subset V_\alpha$, in other words
$S_0 \subset \text{Ob}(\textit{Sch}_\alpha)$.
\label{inclusion}
\item For any $S \in \text{Ob}(\textit{Sch}_\alpha)$ and any
scheme $T$ with $\text{size}(T) \leq Bound(\text{size}(S))$
there exists a scheme $S' \in \text{Ob}(\textit{Sch}_\alpha)$
such that $T \cong S'$.
\label{bounded}
\item For any countable diagram\footnote{Both the set of objects and
the morphism sets are countable.} category $\mathcal{I}$ and
any functor $F : \mathcal{I} \to \textit{Sch}_\alpha$ the limit
$\text{lim}_\mathcal{I} F$ exists in $\textit{Sch}_\alpha$ if and
only if it exists in $\textit{Sch}$ and moreover in this case
the natural morphism between them is an isomorphism.
\label{limit}
\item For any countable diagram category $\mathcal{I}$ and
any functor $F : \mathcal{I} \to \textit{Sch}_\alpha$ the colimit
$\text{colim}_\mathcal{I} F$ exists in $\textit{Sch}_\alpha$ if and
only if it exists in $\textit{Sch}$ and moreover in this case
the natural morphism between them is an isomorphism.
\label{colimit}
\end{enumerate}
\end{lemma}

\begin{proof}
This is really a direct application of the reflection theorem and
likely completely trivial to anybody willing to read more.
We indicate the steps. First we define our formulas. Each
time we indicate the free variables.
\begin{enumerate}
\item The formula $\phi_1(X)$ is the formula $\phi_{scheme}(X)$
discussed in Section \ref{section-sets-everything}.
\item The formula $\phi_2(A, X, Y)$ is the formula
expressing that $A = \text{Mor}_{\textit{Sch}}(X, Y)$.
In words, it expresses the condition that $X$ and $Y$ are schemes
and that $A$ is the set of morphisms of schemes from $X$ to $Y$.
\item We set $\phi_3(X, Y) = \exists A, \phi_2(A, X, Y)$.
\item The formula $\phi_4(A, X)$ is the formula expressing
that $X$ is a scheme and that $A$ is a set of schemes such
that for any scheme $Y$ with
$\text{size}(Y) \leq Bound(\text{size}(X))$ there
exists a $Z \in A$ with $Z \cong Y$.
\item We set $\phi_5(X) = \exists A, \phi_4(A, X)$.
\item The formula $\phi_6(A, f, B)$ expresses that
$A$, $B$ are set, $f : A \to B$ is a set map, and
$f(A) = B$.
\item We set $\phi_7(A, f) = \exists B, \phi_6(A, f, B)$.
\item The formula $\phi_6(A, \mathcal{I}, B)$
is the formula expressing that $\mathcal{I}$
is a countable category, $A$ is a set whose elements are schemes,
$B$ is the set whose elements are exactly the functors
$F$ from $\mathcal{I}$ to the category of schemes
such that $F(i) \in A$ for all $i \in \text{Ob}(\mathcal{I})$.
\item We set $\phi_7(A, \mathcal{I}) =
\exists B, \phi_6(A, \mathcal{I}, B)$.
\item The formula $\phi_6(\mathcal{I}, F, X, t)$ is the formula
expressing that $\mathcal{I}$ is a countable category, $F$ is 
a functor from $\mathcal{I}$ to the category of schemes,
$X$ is a scheme, and $t : \underline{X} \to F$
is a transformation of functors from the constant
functor $\underline{X} : \mathcal{I} \to \textit{Sch}$,
$i \mapsto X$ to $F$ which induces an isomorphism
$X \to \text{lim}_{\mathcal{I}} F$.
\item We set
$\phi_7(\mathcal{I}, F) = \exists X, \exists t, \phi_6(\mathcal{I}, F, X, t)$.
\item The formula $\phi_8(\mathcal{I}, F, X, t)$ is the formula
expressing that $\mathcal{I}$ is a countable category, $F$ is 
a functor from $\mathcal{I}$ to the category of schemes,
$X$ is a scheme, and $t : F \to \underline{X}$
is a transformation of functors from $F$ to the constant
functor $\underline{X} : \mathcal{I} \to \textit{Sch}$,
$i \mapsto X$ which induces an isomorphism
$\text{colim}_{\mathcal{I}} F \to X$.
\item We set
$\phi_9(\mathcal{I}, F) = \exists X, \exists t, \phi_6(\mathcal{I}, F, X, t)$.
\end{enumerate}
We let $CountCat$ be a set all of whose elements are categories
such that every countable category is isomorphic to an element
of $CountCat$. (Such a set exists; proof omitted.)
By Theorem \ref{theorem-reflection-principle} there exists a
limit ordinal $\alpha$
such that $S_0 \cup CountCat \subset V_\alpha$ and such that
for all $A$, $X$, $Y$, $\mathcal{I}$, $F$, $t$ in $V_\alpha$ we
have
\begin{equation}
\label{equation-equivalence}
\phi_i(A, X, Y, \mathcal{I}, F, t)
\leftrightarrow
\phi_i^{V_\alpha}(A, X, Y, \mathcal{I}, F, t),\ \  i = 1, \ldots, 9.
\end{equation}
What does this mean? For example for $\phi_1$ it means that
any $X \in V_\alpha$ is a scheme if and only if $\phi_1^{V_\alpha}(X)$
holds. This is just an example which is not useful because we have
no other way of interpreting the formula $\phi_1^{V_\alpha}(X)$.
It is only when you combine two or more of the formulas
that the result is meaningful.

\medskip\noindent
For example, consider $\phi_2$ and $\phi_3$.
The equivalence (\ref{equation-equivalence}) above
for $\phi_2$ means that given $A, X, Y \in V_\alpha$ we see that
$X$ and $Y$ are schemes, and
$A$ is the set of morphisms of schemes from $X$ to $Y$ if and only
if $\phi_2^{V_\alpha}(A, X, Y)$ holds. On the other hand,
suppose that $X, Y \in V_\alpha$ are schemes. Then
$\phi_2(X, Y)$ holds, since it just expresses the fact that
the collection of morphisms of schemes from $X$ to $Y$ is a set.
Hence from (\ref{equation-equivalence}) we conclude that
$\phi_3^{V_\alpha}(X, Y) =
\exists A \in V_\alpha, \phi_2^{V_\alpha}(A, X, Y)$ holds.
Because of what we said above about $\phi_2^{V_\alpha}(A, X, Y)$
we conclude that $\text{Mor}_{\textit{Sch}}(X, Y)$ is an
element of $V_\alpha$. Since $V_\alpha$ is a transitive set,
see \cite[III Lemma 2.3]{Kunen}, we see that also 
all morphisms between $X$ and $Y$ are elements of $V_\alpha$.
Allthough this is pleasing it is not strictly speaking necessary for the
following, and we could have omitted $\phi_1$, $\phi_2$, and
$\phi_3$ from the list of formulas.

\medskip\noindent
Next, consider $\phi_4$ and $\phi_5$. Then (\ref{equation-equivalence})
for $\phi_4$ means that 
for all $A, X \in V_\alpha$ we have that $X$ is a scheme
and $A$ is a set of schemes containing at least one
element out of each isomorphism class of schemes of 
size at most $Bound(\text{size}(X))$ if and only
if $\phi_4^{V_\alpha}(A, X)$ holds.  Suppose that
$X\in V_\alpha$ is a scheme. Then $\phi_5(X)$ holds,
since it just expresses the fact that
the collection of isomorphism classes of schemes
of bounded size is a set, see Lemma \ref{lemma-bounded-size}.
Hence from (\ref{equation-equivalence}) we conclude that
$\phi_5^{V_\alpha}(X) =
\exists A \in V_\alpha, \phi_4^{V_\alpha}(A, X)$ holds.
Thus we conclude that for any scheme $X$, $X \in V_\alpha$
there is a set $A \in V_\alpha$ whose elements are schemes
and containing at least one
element out of each isomorphism class of schemes of 
size at most $Bound(\text{size}(X))$. Again since
$V_\alpha$ is a transitive set we conclude that
property (\ref{bounded}) of the lemma holds.

\medskip\noindent
The rest of the proof of the lemma simply repeats this argument
over and over again. We suggest skipping the rest of the proof.

\medskip\noindent
We consider $\phi_6$ and $\phi_7$. Then (\ref{equation-equivalence})
for $\phi_6$ means that for all $A, f, B \in V_\alpha$ we have that
$f$ is a set map from $A$ to $B$ and $f(A) = B$
if and only if $\phi_6^{V_\alpha}(A, f, B)$ holds.
Note that $\phi_7(A, f)$ is true whenever $A$ is a set
and $f$ is a map of sets with source $A$. Thus we see that
for $A, f \in V_\alpha$ also the image of $f$ is an element
of $V_\alpha$.










\end{proof}







\medskip\noindent
Consider the following statement: ``Let $X$ be a scheme.''
As discussed in Section \ref{section-sets-everything}
there is a set theoretic formula expressing this.
Say it is $\phi_{scheme}(X)$.
This has one free variable, namely $X$, and some bound variables.
Similarly, let us denote
$\varphi_{mor}(f, X, Y)$ the set theoretic formula
which expresses that $f$ is a morphism of schemes
from $X$ to $Y$. We also can encode the statement
``$A$ is the set of morphisms between the schemes $X$ and $Y$''.
Formally, the corresponding set theoretic formula would look
like
$$
\phi_{set-mor}(A, X, Y) = 
\phi_{scheme}(X) \wedge \phi_{scheme}(Y) \wedge
(\forall f (f \in A \leftrightarrow \phi_{mor}(f, X, Y)))
$$
Clearly, this formula has free variables $X$, $Y$, $A$ and bound variable
$f$ and other bound variables occuring in $\phi_{scheme}$ and
$\phi_{mor}$. Let $\phi_{mor-set-exists}(X, Y)$
be the settheoretic formula expressing the mathematical sentence
``Let $X$, $Y$ be schemes. There exists a set $A$ such that
$\phi_{set-mor}(A, X, Y)$''. In a formula
$$
\phi_{mor-set-exists}(X, Y) = \exists A \phi_{set-mor}(A, X, Y).
$$
Consider the following sentence: ``The scheme $Z$ together
with the morphisms of schemes $p : Z \to X$, and $q : Z \to Y$ is a
fibre product of the morphisms of schemes $f : X \to S$ and
$g : Y \to S$.'' Let $\phi_{prod}(Z, X, Y, S, p, q, f, g)$ denote
a formula of set theory which is a translation of the above.
The free variables of $\phi_{prod}$ are $Z, X, Y, S, p, q, f, g$
as indicated.
Consider the following sentence: ``For every pair of morphisms of
schemes $f : X \to S$ and $g : Y \to S$ with the same target scheme $S$
there exists a fibre product $(Z, p, q)$ of $f$ and $g$.''
The free variables in this formula are $X$, $Y$, $S$.
The corresponding set theory formula is by the above
equal to
$$
\phi_{fib-prods-exist}(X, Y, S)
=
\forall f\ \forall g\ \exists Z\ \exists p\ \exists q\ 
\phi_{prod}(Z, X, Y, S, p, q, f, g)
$$
In addition to the above, let us consider the sentence:
``$Y$ is a scheme, $F$ is a set of schemes, for all schemes $X$,
for all morphisms of schemes $g : X \to Y$ of finite type, there exists a
$T \in F$ such that $X \cong T$.'' Denote this formula
$\phi_{set-ft}(F, Y)$. Moreover, set
$\phi_{exist-set-ft}(Y) = \exists F \phi_{set-ft}(F, Y)$.

\medskip\noindent
For an ordinal $\alpha$ we denote $\textit{Sch}_\alpha$ the subcategory
of the ``big'' category $\textit{Sch}$ of all schemes, whose
objects and morphisms are elements of $V_\alpha$.
Suppose given an initial set of schemes $\mathcal{C}_0$.
Apply the reflection theorem with $M_0 = \mathcal{C}_0$
and the sentences
$\phi_{scheme}(X)$,
$\phi_{mor}(f, X, Y)$,
$\phi_{set-mor}(A, X, Y)$,
$\phi_{mor-set-exists}(X, Y)$,
$\phi_{prod}(Z, X, Y, S, p, q, f, g)$,
$\phi_{fib-prods-exist}(X, Y, S)$,
$\phi_{set-ft}(F, Y)$, and
$\phi_{exist-set-ft}(Y)$. The theorem implies
there exists a limit ordinal $\alpha$ such that
whenever we choose $X, Y, A, Z, S, p, q, f, g$ in
$V_\alpha$ then the assertions $\phi_{scheme}^{V_\alpha}(X)$, etc
hold if and only if the assertions $\phi_{scheme}(X)$, etc hold.
We work out formula by formula the significance.
\begin{enumerate}
\item For the formula $\phi_{scheme}$ this means that for
all $X \in V_\alpha$ to check that $X$ is a scheme,
it is the same to check that $\phi_{scheme}^{V_\alpha}(X)$ holds.
\item For the formula $\phi_{mor}$ it similarly states that
given $f, X, Y \in V_\alpha$ to see $f$ is a morphism
between $X$ and $Y$ we check whether
$\phi_{mor}^{V_\alpha}(f, X, Y)$ holds.
\item For the formula $\phi_{set-mor}$ it implies that given $X$, $Y$ and
$A$ in $V_\alpha$ we have that $A$ is the set of all morphisms of schemes
from $X$ to $Y$ if and only if $A$ is the set of morphisms of schemes
$f : X \to Y$ with $f \in V_\alpha$!
\item The formula $\phi_{mor-set-exists}(X, Y)$ is true, no matter
what $X$ and $Y$ we choose.
Hence for all $X, Y$ in $\textit{Sch}_\alpha$ the
formula $\phi_{mor-set-exists}^{V_\alpha}(X, Y)$ is true.
Thus given $X, Y$ in $\textit{Sch}_\alpha$
there is a set $A$ in $V_\alpha$ whose elements are the morphisms
of schemes between $X$ and $Y$ in $V_\alpha$.
\item Combining (3) and (4) we see, given $X, Y$ in $\textit{Sch}_\alpha$,
the set $\text{Mor}_{\textit{Sch}}(X, Y) \in V_\alpha$
and all its elements are elements of $V_\alpha$ as well.
In particular the embedding
$\textit{Sch}_\alpha \subset \textit{Sch}$ is fully faithful.
\item Let $Z, X, Y, S, p, q, f, g \in V_\alpha$.
The equivalence of $\phi_{prod}(Z, X, Y, S, p, q, f, g)$
and $\phi_{prod}^{V_\alpha}(Z, X, Y, S, p, q, f, g)$ means that
$Z, p ,q$ form a fibre product of $f : X \to S$ and $g : Y \to S$
in $\textit{Sch}$ if and only if
$\phi_{prod}^{V_\alpha}(Z, X, Y, S, p, q, f, g)$ holds.%
\footnote{At this point in the discussion we do not know
what, if anything, $\phi^{V_\alpha}_{prod}$ signifies in terms of the
category $\textit{Sch}_\alpha$. It may very well depend on
the precise shape of the formula chosen.}
\item The formula $\phi_{fib-prods-exist}(X, Y, S)$ is true
for any triple of schemes $X, Y, S$, see Schemes,
Lemma \ref{schemes-lemma-fibre-products}. Hence we see that if
we choose $X, Y, S$ in $\textit{Sch}_\alpha$ then we have
$$
\forall f\in V_\alpha\ \forall g\in V_\alpha\ 
\exists Z\in V_\alpha\ \exists p\in V_\alpha\ \exists q\in V_\alpha\ 
\phi_{prod}^{V_\alpha}(Z, X, Y, S, p, q, f, g)
$$
\item Let $X$, $Y$, $S$ be objects of $\textit{Sch}_\alpha$.
Combining (5), (6) and (7) we see that no matter which morphisms
of schemes $f : X \to S$ and $g : Y \to S$ we choose there exist
$Z, p, q \in V_\alpha$ which are a fibre product in the
category $\textit{Sch}$ of all schemes. We conclude that
$\textit{Sch}_\alpha$ has fibre products and the embedding
$\textit{Sch}_\alpha \subset \textit{Sch}$ commutes with
taking fibre products.
\item Let $F, Y \in V_\alpha$. The equivalence of
$\phi_{set-ft}(F, Y)$ and $\phi^{V_\alpha}_{set-ft}(F, Y)$
means that ``$F$ is a set of schemes such that for any finite
type morphism of schemes $g : X \to Y$ there is at least one
$T \in F$ such that $T \cong X$'' is equivalent to
$\phi^{V_\alpha}_{set-ft}(F, Y)$.
\item For every scheme $Y$ the formula $\phi_{exist-set-ft}(Y)$
is true, see Morphisms, Lemma \ref{morphisms-lemma-set-finite-type-over}.
Thus we see that if $Y \in V_\alpha$, then
$\exists F \in V_\alpha\ \phi^{V_\alpha}_{set-ft}(F, Y)$ holds.
\item Combining (9), (10) we conclude that given any
$Y$ object of $\textit{Sch}_\alpha$ there exists a set of schemes
$F \in V_\alpha$ such that for any finite type scheme
$X \to Y$ over $Y$, then $X$ is isomorphic to an element
of $F$. In particular, since $V_\alpha$ is a transitive set,
we also see that $X$ is isomorphic to an object of $\textit{Sch}_\alpha$.%
\footnote{This use of the transitivity of $V_\alpha$ can easily be
avoided by adding a suitable sentence to the list considered.}
\end{enumerate}
Note that (6), (7), and (8) are proved in exactly the same way
as (9), (10), and (11).
Summarizing we see the following.

\begin{lemma}
\label{lemma-create-category-schemes}
Given an initial set of schemes $\mathcal{C}_0$ we can find a limit ordinal
$\alpha$ such that
\begin{enumerate}
\item we have $\mathcal{C}_0 \subset \text{Ob}(\textit{Sch}_\alpha)$,
\item the embedding $\textit{Sch}_\alpha \subset \textit{Sch}$ is
fully faithful,
\item fibre products exist in $\textit{Sch}_\alpha$ and satisfy
the universal property of fibre products in the class of {\it all} schemes, and
\item given any finite type morphism of schemes $g : X \to S$
whose target is in $\textit{Sch}_\alpha$ there exists an
object $Y$ of $\textit{Sch}_\alpha$ which is isomorphic to $X$.
\end{enumerate}
\end{lemma}

\begin{proof}
See discussion above.
\end{proof}

\noindent
It would not be hard to construct a ``small'' category of schemes
having the properties listed in the lemma directly. The appeal of
using the reflection principle is that it garantees you can find
$\alpha$ no matter what the constraints are. For example
suppose we have are given a ``big'' category which has fibre products.
Clearly exactly the same reasoning applies, and we can always
find arbitrarily large ``small'' subcategories having fibre
products.

\begin{remark}
\label{remark-more-schemes}
It would be easy to show that we can choose $\alpha$ with the following
list of additional properties:
\begin{enumerate}
\item Finite coproducts in $\textit{Sch}_\alpha$
exist and are the same as in $\textit{Sch}$.
\item Given $X$ in $\textit{Sch}_\alpha$ all locally closed subschemes
of $X$ are in $\textit{Sch}_\alpha$ (the collection of locally closed
subschemes of $X$ forms a set, see
Schemes, Section \ref{schemes-section-immersions}).
\item Given a cardinal $\kappa$ any affine scheme $\text{Spec}(R)$ 
with $|R| \leq \kappa$ is isomorphic to an object of $V_\alpha$.
\item Etc.
\end{enumerate}
However, this does not seem particularly useful.
\end{remark}

\section{Injective modules}
\label{section-injective-modules}

\noindent
Let $R$ be a ring and let $M$ be an $R$-module.
Consider the sentence: ``The $R$-module $M$ is injective.''
This can be formulated symbolically as follows
\begin{eqnarray*}
\text{Let }M \text{ be an $R$-module},\\
\forall\ X,Y\ \text{$R$-modules},\\
\forall\ j : X\to Y\ \text{injective $R$-module map},\\
\forall\ x : X\to M\ \text{$R$-module map},\\
\exists\ y : Y\to M\ \text{$R$-module map}\\
y \circ j = x
\end{eqnarray*}
Note that this formula does have a free variable (namely $M$).
Let $\mathcal{A}_0$ denote an initial set of $R$-modules.
(For example we could choose $\mathcal{A}_0$ to be the set
consisting of all cokernels of maps of free modules -- this
clearly is a set.)
The reflection principle says that we can find a set
of $R$-modules $\mathcal{A}$ (which we think of as a full subcategory
of the category of $R$-modules) containing the given set $\mathcal{A}_0$
such that $M \in \mathcal{A}$ is an injective
object in the category $\mathcal{A}$ if and only if
it is injective as an $R$-module. 

\section{Categories of modules}
\label{section-categories-modules}

\noindent
We continue with the discussion above.
Consider the assertion: ``Given a category $\mathcal{I}$
and a functor $F$ from $\mathcal{I}$ to $R$-modules,
denoted $i \mapsto M_i = F(i)$. Both the limit $\lim_{i\in \mathcal{I}} M_i$
and the colimit $\text{colim}_{i \in \mathcal{I}} M_i$ exist.'' 
This we can turn into a set theoretic formula with free
variables for $I$ and $F$. Next pick some ordinal $\alpha$.
Using the reflection principle
we can find a category of $R$-modules
in which all limits and colimits whose index category 
have cardinality at most that of $\alpha$ exist and
are equal to the corresponding limits in the ``big''
category of all $R$-modules.

\section{Sets with group action}
\label{section-sets-with-group-action}

\noindent
Let $G$ be a group and let $\mathcal{S}$ be a 
set of $G$-sets. For any ordinal $\alpha$ we denote
$G\textit{-Sets}_\alpha$ the subcategory of $G$-sets
whose objects and morphisms are in $V_\alpha$.
We claim there exists a limit ordinal $\alpha$ such that
\begin{enumerate}
\item $\mathcal{S} \subset \text{Ob}(G\textit{-Sets}_\alpha)$,
\item the embedding $G\textit{-Sets}_\alpha \subset G\textit{-Sets}$
is fully faithful,
\item fibred product exist in $G\textit{-Sets}_\alpha$
and are the same as in $G\textit{-Sets}$,
\item given $U$ in $G\textit{-Sets}_\alpha$, and any monomorphism
$V \to U$ in $G\textit{-Sets}$ there exists a
$V' \in \text{Ob}(G\textit{-Sets})$ isomorphic to $V$,
\item given $U$ in $G\textit{-Sets}_\alpha$, and any epimorphism
$U \to V$ in $G\textit{-Sets}$ there exists a
$V' \in \text{Ob}(G\textit{-Sets})$ isomorphic to $V$, and
\item the left $G$-set corresponding to $G$ is an object of
$G\textit{-Sets}_\alpha$.
\end{enumerate}
We are going to produce $\alpha$ by applying the reflection
principle, with a suitable finite list of set theoretic formulas
for each property. Properties (1) and (6) are clear.
We have already discussed how to formulate set
theoretic formulas that produce $\alpha$ having property (2), (3)
see Section \ref{section-fibre-product-schemes}. In order to 
get property (4) consider the sentence: ``Let $U$ be a $G$-set
and let $Q$ be a set. The elements of $Q$ are epimorphisms
$f : U \to V_f$, and for every epimorphism $U \to V$
there exists some $f \in Q$ such that $V \cong V_f$.''
Write this as $\phi_{set-epis}(U, Q)$. Consider also
the formula $\phi_{set-epis-exists}(U) = \exists Q, \phi_{set-epis}(U, Q)$.
We add these two to the list of formulas when applying the
reflection theorem. Exactly as in Section \ref{section-fibre-product-schemes}
one can argue that in this case, given $U$ in $G\textit{-Sets}_\alpha$
and $Q \in V_\alpha$ we have that $Q$ is an exhaustive list of
isomorphism types of targets of epimorphisms if and only if
$\phi_{set-epis}^{V_\alpha}(U, Q)$. Moreover, the truth of
$\phi_{set-epis-exists}(U)$ for any $U$ implies that for
any $U$ in $G\textit{-Sets}_\alpha$ there exists a 
$Q \in V_\alpha$ such that $\phi_{set-epis}^{V_\alpha}(U, Q)$ holds.
Hence for any $U$ in $G\textit{-Sets}_\alpha$ there exists a
$Q \in V_\alpha$ which is an exhaustive list of
isomorphism types of targets of epimorphisms in $G\textit{-sets}$.
Thus the reflection theorem will assure that
the formula above (together with the others) produces an $\alpha$
with property (5). Similarly for property (4).

\section{Coverings of a site}
\label{section-coverings-site}

\noindent
Suppose that $\mathcal{C}$ is a category and
that $\text{Cov}(\mathcal{C})$ is a proper class of coverings
satisfying properties (1), (2) and (3) of Sites,
Definition \ref{sites-definition-site}.
Moreover, let $\text{Cov}_0 \subset \text{Cov}(\mathcal{C})$
be a set contained in $\text{Cov}(\mathcal{C})$.
In this section we show how to replace $\text{Cov}(\mathcal{C})$
by $\text{Cov}(\mathcal{C}) \cap V_\alpha$ to get a site
with a set of coverings.

\medskip\noindent
We recall the following notion, see Sites, Definition
\ref{sites-definition-combinatorial-tautological}.
Two families of morphisms $\{\varphi_i : U_i \to U\}_{i\in I}$, and
$\{\psi_j : W_j \to U\}_{j\in J}$ with the same target of $\mathcal{C}$ are
called {\it combinatorially equivalent} if there exist maps
$\alpha : I \to J$ and $\beta : J\to I$ such that
$\varphi_i = \psi_{\alpha(i)}$ and $\psi_j = \varphi_{\beta(j)}$.
This defines an equivalence relation on families of morphisms
having a fixed target.
We will use the reflection principle to show the following.

\begin{lemma}
\label{lemma-coverings-site}
In the situation above. There is a limit ordinal $\alpha$ such that
\begin{enumerate}
\item we have $\text{Cov}_0 \subset \text{Cov}(\mathcal{C}) \cap V_\alpha$,
\item the set of coverings
$\text{Cov}(\mathcal{C}) \cap V_\alpha$ satisfies
(1), (2) and (3) of Sites, Definition \ref{sites-definition-site}
as well, and
\item every covering in $\text{Cov}(\mathcal{C})$
is combinatorially equivalent
to a covering in $\text{Cov}(\mathcal{C}) \cap V_\alpha$.
\end{enumerate}
\end{lemma}

\begin{proof}
To prove this, we first consider the set $\mathcal{S}$ of all
families of morphisms with fixed target. In other words, an element of
$\mathcal{S}$ is a subset $T$ of the set $\text{Arrows}(\mathcal{C})$
such that all elements of $T$ have the same target. Next, we define
$\mathcal{S}_\tau \subset \mathcal{S}$ to be the subset of those
$T$ which are combinatorially equivalent to some covering
$\xi = \{\varphi_i : U_i \to U\}_{i\in I} \in \text{Cov}(\mathcal{C})$.
In other words, $T \in \mathcal{S}_\tau$ if and only if $T$
is the image of the map $supp_\xi : I \to \text{Arrows}(\mathcal{C})$,
$i\mapsto \varphi_i$ for some covering
$\xi = \{\varphi_i : U_i \to U\}_{i\in I}$
of $\text{Cov}(\mathcal{C})$. Thus $\mathcal{S}_\tau$
is a set. Consider the set theoretical formula: ``For all $T \in \mathcal{S}$
then $T\in \mathcal{S}_\tau$ if and only if there exists
a covering $\xi$ in $\text{Cov}(\mathcal{C})$ such that
$T$ is the image of $supp_\xi$.'' By construction of $\mathcal{S}_\tau$
this formula holds. As a first application of the
reflection principle we find an limit ordinal $\alpha_0$ such that
this formula holds in $V_{\alpha_0}$. At this point already every
covering in $\text{Cov}(\mathcal{C})$ is combinatorially equivalent
to a covering in $\text{Cov}(\mathcal{C}) \cap V_{\alpha_0}$.

\medskip\noindent
The second and final step is to rewrite properties (1), (2) and (3) of
Sites, Definition \ref{sites-definition-site} as set theoretic formulas.
This is similar to the arguments in the sections above and is
left to the reader. Apply the reflection
principle Theorem \ref{theorem-reflection-principle} using these formulas
with the initial set $T$ of the statement of the theorem being equal to
$V_{\alpha_0}$.
\end{proof}


\section{Other chapters}

\begin{multicols}{2}
\begin{enumerate}
\item \hyperref[introduction-section-phantom]{Introduction}
\item \hyperref[conventions-section-phantom]{Conventions}
\item \hyperref[sets-section-phantom]{Set Theory}
\item \hyperref[categories-section-phantom]{Categories}
\item \hyperref[topology-section-phantom]{Topology}
\item \hyperref[sheaves-section-phantom]{Sheaves on Spaces}
\item \hyperref[algebra-section-phantom]{Commutative Algebra}
\item \hyperref[sites-section-phantom]{Sites and Sheaves}
\item \hyperref[homology-section-phantom]{Homological Algebra}
\item \hyperref[derived-section-phantom]{Derived Categories}
\item \hyperref[more-algebra-section-phantom]{More Algebra}
\item \hyperref[simplicial-section-phantom]{Simplicial Methods}
\item \hyperref[modules-section-phantom]{Sheaves of Modules}
\item \hyperref[sites-modules-section-phantom]{Modules on Sites}
\item \hyperref[injectives-section-phantom]{Injectives}
\item \hyperref[cohomology-section-phantom]{Cohomology of Sheaves}
\item \hyperref[sites-cohomology-section-phantom]{Cohomology on Sites}
\item \hyperref[hypercovering-section-phantom]{Hypercoverings}
\item \hyperref[schemes-section-phantom]{Schemes}
\item \hyperref[constructions-section-phantom]{Constructions of Schemes}
\item \hyperref[properties-section-phantom]{Properties of Schemes}
\item \hyperref[morphisms-section-phantom]{Morphisms of Schemes}
\item \hyperref[coherent-section-phantom]{Coherent Cohomology}
\item \hyperref[divisors-section-phantom]{Divisors}
\item \hyperref[limits-section-phantom]{Limits of Schemes}
\item \hyperref[varieties-section-phantom]{Varieties}
\item \hyperref[chow-section-phantom]{Chow Homology}
\item \hyperref[topologies-section-phantom]{Topologies on Schemes}
\item \hyperref[descent-section-phantom]{Descent}
\item \hyperref[more-morphisms-section-phantom]{More on Morphisms}
\item \hyperref[flat-section-phantom]{More on Flatness}
\item \hyperref[groupoids-section-phantom]{Groupoid Schemes}
\item \hyperref[more-groupoids-section-phantom]{More on Groupoid Schemes}
\item \hyperref[etale-section-phantom]{\'Etale Morphisms of Schemes}
\item \hyperref[etale-cohomology-section-phantom]{\'Etale Cohomology}
\item \hyperref[spaces-section-phantom]{Algebraic Spaces}
\item \hyperref[spaces-properties-section-phantom]{Properties of Algebraic Spaces}
\item \hyperref[spaces-morphisms-section-phantom]{Morphisms of Algebraic Spaces}
\item \hyperref[spaces-topologies-section-phantom]{Topologies on Algebraic Spaces}
\item \hyperref[spaces-descent-section-phantom]{Descent and Algebraic Spaces}
\item \hyperref[spaces-more-morphisms-section-phantom]{More on Morphisms of Spaces}
\item \hyperref[quot-section-phantom]{Quot and Hilbert Spaces}
\item \hyperref[stacks-section-phantom]{Stacks}
\item \hyperref[spaces-groupoids-section-phantom]{Groupoids in Algebraic Spaces}
\item \hyperref[spaces-more-groupoids-section-phantom]{More on Groupoids in Spaces}
\item \hyperref[bootstrap-section-phantom]{Bootstrap}
\item \hyperref[examples-stacks-section-phantom]{Examples of Stacks}
\item \hyperref[groupoids-quotients-section-phantom]{Quotients of Groupoids}
\item \hyperref[algebraic-section-phantom]{Algebraic Stacks}
\item \hyperref[criteria-section-phantom]{Criteria for Representability}
\item \hyperref[stacks-properties-section-phantom]{Properties of Algebraic Stacks}
\item \hyperref[stacks-morphisms-section-phantom]{Morphisms of Algebraic Stacks}
\item \hyperref[examples-section-phantom]{Examples}
\item \hyperref[exercises-section-phantom]{Exercises}
\item \hyperref[guide-section-phantom]{Guide to Literature}
\item \hyperref[desirables-section-phantom]{Desirables}
\item \hyperref[coding-section-phantom]{Coding Style}
\item \hyperref[fdl-section-phantom]{GNU Free Documentation License}
\item \hyperref[index-section-phantom]{Auto Generated Index}
\end{enumerate}
\end{multicols}


\bibliography{my}
\bibliographystyle{alpha}

\end{document}
