\IfFileExists{stacks-project.cls}{%
\documentclass{stacks-project}
}{%
\documentclass{amsart}
}

% The following AMS packages are automatically loaded with
% the amsart documentclass:
%\usepackage{amsmath}
%\usepackage{amssymb}
%\usepackage{amsthm}

% For dealing with references we use the comment environment
\usepackage{verbatim}
\newenvironment{reference}{\comment}{\endcomment}
%\newenvironment{reference}{}{}
\newenvironment{slogan}{\comment}{\endcomment}
\newenvironment{history}{\comment}{\endcomment}

% For commutative diagrams you can use
% \usepackage{amscd}
\usepackage[all]{xy}

% We use 2cell for 2-commutative diagrams.
\xyoption{2cell}
\UseAllTwocells

% To put source file link in headers.
% Change "template.tex" to "this_filename.tex"
% \usepackage{fancyhdr}
% \pagestyle{fancy}
% \lhead{}
% \chead{}
% \rhead{Source file: \url{template.tex}}
% \lfoot{}
% \cfoot{\thepage}
% \rfoot{}
% \renewcommand{\headrulewidth}{0pt}
% \renewcommand{\footrulewidth}{0pt}
% \renewcommand{\headheight}{12pt}

\usepackage{multicol}

% For cross-file-references
\usepackage{xr-hyper}

% Package for hypertext links:
\usepackage{hyperref}

% For any local file, say "hello.tex" you want to link to please
% use \externaldocument[hello-]{hello}
\externaldocument[introduction-]{introduction}
\externaldocument[conventions-]{conventions}
\externaldocument[sets-]{sets}
\externaldocument[categories-]{categories}
\externaldocument[topology-]{topology}
\externaldocument[sheaves-]{sheaves}
\externaldocument[sites-]{sites}
\externaldocument[stacks-]{stacks}
\externaldocument[fields-]{fields}
\externaldocument[algebra-]{algebra}
\externaldocument[brauer-]{brauer}
\externaldocument[homology-]{homology}
\externaldocument[derived-]{derived}
\externaldocument[simplicial-]{simplicial}
\externaldocument[more-algebra-]{more-algebra}
\externaldocument[smoothing-]{smoothing}
\externaldocument[modules-]{modules}
\externaldocument[sites-modules-]{sites-modules}
\externaldocument[injectives-]{injectives}
\externaldocument[cohomology-]{cohomology}
\externaldocument[sites-cohomology-]{sites-cohomology}
\externaldocument[dga-]{dga}
\externaldocument[dpa-]{dpa}
\externaldocument[hypercovering-]{hypercovering}
\externaldocument[schemes-]{schemes}
\externaldocument[constructions-]{constructions}
\externaldocument[properties-]{properties}
\externaldocument[morphisms-]{morphisms}
\externaldocument[coherent-]{coherent}
\externaldocument[divisors-]{divisors}
\externaldocument[limits-]{limits}
\externaldocument[varieties-]{varieties}
\externaldocument[topologies-]{topologies}
\externaldocument[descent-]{descent}
\externaldocument[perfect-]{perfect}
\externaldocument[more-morphisms-]{more-morphisms}
\externaldocument[flat-]{flat}
\externaldocument[groupoids-]{groupoids}
\externaldocument[more-groupoids-]{more-groupoids}
\externaldocument[etale-]{etale}
\externaldocument[chow-]{chow}
\externaldocument[intersection-]{intersection}
\externaldocument[pic-]{pic}
\externaldocument[adequate-]{adequate}
\externaldocument[dualizing-]{dualizing}
\externaldocument[duality-]{duality}
\externaldocument[discriminant-]{discriminant}
\externaldocument[local-cohomology-]{local-cohomology}
\externaldocument[curves-]{curves}
\externaldocument[resolve-]{resolve}
\externaldocument[models-]{models}
\externaldocument[pione-]{pione}
\externaldocument[etale-cohomology-]{etale-cohomology}
\externaldocument[proetale-]{proetale}
\externaldocument[crystalline-]{crystalline}
\externaldocument[spaces-]{spaces}
\externaldocument[spaces-properties-]{spaces-properties}
\externaldocument[spaces-morphisms-]{spaces-morphisms}
\externaldocument[decent-spaces-]{decent-spaces}
\externaldocument[spaces-cohomology-]{spaces-cohomology}
\externaldocument[spaces-limits-]{spaces-limits}
\externaldocument[spaces-divisors-]{spaces-divisors}
\externaldocument[spaces-over-fields-]{spaces-over-fields}
\externaldocument[spaces-topologies-]{spaces-topologies}
\externaldocument[spaces-descent-]{spaces-descent}
\externaldocument[spaces-perfect-]{spaces-perfect}
\externaldocument[spaces-more-morphisms-]{spaces-more-morphisms}
\externaldocument[spaces-flat-]{spaces-flat}
\externaldocument[spaces-groupoids-]{spaces-groupoids}
\externaldocument[spaces-more-groupoids-]{spaces-more-groupoids}
\externaldocument[bootstrap-]{bootstrap}
\externaldocument[spaces-pushouts-]{spaces-pushouts}
\externaldocument[groupoids-quotients-]{groupoids-quotients}
\externaldocument[spaces-more-cohomology-]{spaces-more-cohomology}
\externaldocument[spaces-simplicial-]{spaces-simplicial}
\externaldocument[spaces-duality-]{spaces-duality}
\externaldocument[formal-spaces-]{formal-spaces}
\externaldocument[restricted-]{restricted}
\externaldocument[spaces-resolve-]{spaces-resolve}
\externaldocument[formal-defos-]{formal-defos}
\externaldocument[defos-]{defos}
\externaldocument[cotangent-]{cotangent}
\externaldocument[examples-defos-]{examples-defos}
\externaldocument[algebraic-]{algebraic}
\externaldocument[examples-stacks-]{examples-stacks}
\externaldocument[stacks-sheaves-]{stacks-sheaves}
\externaldocument[criteria-]{criteria}
\externaldocument[artin-]{artin}
\externaldocument[quot-]{quot}
\externaldocument[stacks-properties-]{stacks-properties}
\externaldocument[stacks-morphisms-]{stacks-morphisms}
\externaldocument[stacks-limits-]{stacks-limits}
\externaldocument[stacks-cohomology-]{stacks-cohomology}
\externaldocument[stacks-perfect-]{stacks-perfect}
\externaldocument[stacks-introduction-]{stacks-introduction}
\externaldocument[stacks-more-morphisms-]{stacks-more-morphisms}
\externaldocument[stacks-geometry-]{stacks-geometry}
\externaldocument[moduli-]{moduli}
\externaldocument[moduli-curves-]{moduli-curves}
\externaldocument[examples-]{examples}
\externaldocument[exercises-]{exercises}
\externaldocument[guide-]{guide}
\externaldocument[desirables-]{desirables}
\externaldocument[coding-]{coding}
\externaldocument[obsolete-]{obsolete}
\externaldocument[fdl-]{fdl}
\externaldocument[index-]{index}

% Theorem environments.
%
\theoremstyle{plain}
\newtheorem{theorem}[subsection]{Theorem}
\newtheorem{proposition}[subsection]{Proposition}
\newtheorem{lemma}[subsection]{Lemma}

\theoremstyle{definition}
\newtheorem{definition}[subsection]{Definition}
\newtheorem{example}[subsection]{Example}
\newtheorem{exercise}[subsection]{Exercise}
\newtheorem{situation}[subsection]{Situation}

\theoremstyle{remark}
\newtheorem{remark}[subsection]{Remark}
\newtheorem{remarks}[subsection]{Remarks}

\numberwithin{equation}{subsection}

% Macros
%
\def\lim{\mathop{\mathrm{lim}}\nolimits}
\def\colim{\mathop{\mathrm{colim}}\nolimits}
\def\Spec{\mathop{\mathrm{Spec}}}
\def\Hom{\mathop{\mathrm{Hom}}\nolimits}
\def\Ext{\mathop{\mathrm{Ext}}\nolimits}
\def\SheafHom{\mathop{\mathcal{H}\!\mathit{om}}\nolimits}
\def\SheafExt{\mathop{\mathcal{E}\!\mathit{xt}}\nolimits}
\def\Sch{\mathit{Sch}}
\def\Mor{\operatorname{Mor}\nolimits}
\def\Ob{\mathop{\mathrm{Ob}}\nolimits}
\def\Sh{\mathop{\mathit{Sh}}\nolimits}
\def\NL{\mathop{N\!L}\nolimits}
\def\proetale{{pro\text{-}\acute{e}tale}}
\def\etale{{\acute{e}tale}}
\def\QCoh{\mathit{QCoh}}
\def\Ker{\mathop{\mathrm{Ker}}}
\def\Im{\mathop{\mathrm{Im}}}
\def\Coker{\mathop{\mathrm{Coker}}}
\def\Coim{\mathop{\mathrm{Coim}}}

%
% Macros for moduli stacks/spaces
%
\def\QCohstack{\mathcal{QC}\!\mathit{oh}}
\def\Cohstack{\mathcal{C}\!\mathit{oh}}
\def\Spacesstack{\mathcal{S}\!\mathit{paces}}
\def\Quotfunctor{\mathrm{Quot}}
\def\Hilbfunctor{\mathrm{Hilb}}
\def\Curvesstack{\mathcal{C}\!\mathit{urves}}
\def\Polarizedstack{\mathcal{P}\!\mathit{olarized}}
\def\Complexesstack{\mathcal{C}\!\mathit{omplexes}}
% \Pic is the operator that assigns to X its picard group, usage \Pic(X)
% \Picardstack_{X/B} denotes the Picard stack of X over B
% \Picardfunctor_{X/B} denotes the Picard functor of X over B
\def\Pic{\mathop{\mathrm{Pic}}\nolimits}
\def\Picardstack{\mathcal{P}\!\mathit{ic}}
\def\Picardfunctor{\mathrm{Pic}}
\def\Deformationcategory{\mathcal{D}\!\mathit{ef}}


% OK, start here.
%
\begin{document}

\title{Set Theory}


\maketitle

\phantomsection
\label{section-phantom}

\tableofcontents

\section{Introduction}
\label{section-introduction}

\noindent
We need some set theory every now and then. We use Zermelo-Fraenkel set theory
with the axiom of choice (ZFC) as described in \cite{Kunen} and \cite{Jech}.

\section{Everything is a set}
\label{section-sets-everything}

\noindent
Most mathematicians think of set theory as providing the basic
foundations for mathematics. So how does this really work?
For example, how do we translate the sentence
``$X$ is a scheme'' into set theory? Well, we just unravel the
definitions: A scheme is a locally ringed space such that every
point has an open neighbourhood which is an affine scheme.
A locally ringed space is a ringed space such that every stalk
of the structure sheaf is a local ring. A ringed space is
a pair $(X, \mathcal{O}_X)$ consisting of a topological space
$X$ and a sheaf of rings $\mathcal{O}_X$ on it. A topological
space is a pair $(X, \tau)$ consisting of a set
$X$ and a set of subsets $\tau \subset \mathcal{P}(X)$
satisfying the axioms of a topology. And so on and
so forth.

\medskip\noindent
So how, given a set $S$ would we recognize whether it is a scheme?
The first thing we look for is whether the set $S$ is an ordered pair.
This is defined (see \cite{Jech}, page 7) as saying that $S$
has the form $(a, b) := \{\{a\}, \{a, b\}\}$ for some sets $a, b$. If this is
the case, then we would take a look to see whether $a$ is an
ordered pair $(c, d)$. If so we would check whether
$d \subset \mathcal{P}(c)$, and if so whether $d$ forms the collection
of sets for a topology on the set $c$. And so on and so forth.

\medskip\noindent
So even though it would take a considerable amount of work to write
a complete formula $\phi_{scheme}(x)$ with one free variable $x$ in set theory
that expresses the notion ``$x$ is a scheme'', it is possible to do so.
The same thing should be true for any mathematical object.

\section{Classes}
\label{section-classes}

\noindent
Informally we use the notion of a {\it class}. Given a formula
$\phi(x, p_1, \ldots, p_n)$, we call
$$
C = \{x : \phi(x, p_1, \ldots, p_n)\}
$$
a {\it class}. A class is easier to manipulate than the formula
that defines it, but it is not strictly speaking a mathematical
object. For example, if $R$ is a ring, then we may
consider the class of all $R$-modules (since after all we
may translate the sentence ``$M$ is an $R$-module''
into a formula in set theory, which then defines a class).
A {\it proper class} is a class which is not a set.

\medskip\noindent
In this way we may consider the category of $R$-modules,
which is a ``big'' category---in other words, it has a
proper class of objects. Similarly, we may consider
the ``big'' category of schemes, the ``big'' category
of rings, etc.



\section{Ordinals}
\label{section-ordinals}

\noindent
A set $T$ is {\it transitive} if $x\in T$ implies $x\subset T$.
A set $\alpha$ is an {\it ordinal} if it is transitive and well-ordered
by $\in$. In this case, we define $\alpha + 1 = \alpha \cup \{\alpha\}$,
which is another ordinal called the {\it successor} of $\alpha$.
An ordinal $\alpha$ is called a {\it successor ordinal} if
there exists an ordinal $\beta$ such that $\alpha = \beta + 1$.
The smallest ordinal is $\emptyset$ which is also denoted $0$.
If $\alpha$ is not $0$, and not a successor ordinal, then $\alpha$ is called
a {\it limit ordinal} and we have
$$
\alpha
=
\bigcup\nolimits_{\gamma \in \alpha} \gamma.
$$
The first limit ordinal is $\omega$ and it is also the first
infinite ordinal. The first uncountable ordinal $\omega_1$
is the set of all countable ordinals.
The collection of all ordinals is a proper class.
It is well-ordered by $\in$ in the following sense: any nonempty set
(or even class) of ordinals has a least element.
Given a set $A$ of ordinals, we define the {\it supremum}
of $A$ to be $\sup_{\alpha \in A} \alpha =
\bigcup_{\alpha \in A} \alpha$. It is the least ordinal bigger
or equal to all $\alpha \in A$.
Given any well-ordered set $(S, <)$, there is a unique ordinal
$\alpha$ such that $(S, <) \cong (\alpha, \in)$; this is
called the {\it order type} of the well-ordered set.

\section{The hierarchy of sets}
\label{section-sets-hierarchy}

\noindent
We define, by transfinite induction, $V_0 = \emptyset$,
$V_{\alpha + 1} = P(V_\alpha)$ (power set),
and for a limit ordinal $\alpha$,
$$
V_\alpha = \bigcup\nolimits_{\beta < \alpha} V_\beta.
$$
Note that each $V_\alpha$ is a transitive set.

\begin{lemma}
\label{lemma-axiom-regularity}
Every set is an element of $V_\alpha$ for some ordinal $\alpha$.
\end{lemma}

\begin{proof}
See \cite[Lemma 6.3]{Jech}.
\end{proof}

\noindent
In \cite[Chapter III]{Kunen} it is explained that this lemma is
equivalent to the axiom of foundation. The {\it rank} of
a set $S$ is the least ordinal $\alpha$ such that $S \in V_{\alpha + 1}$.
By a {\it partial universe} we shall mean a suitably large set of the form
$V_\alpha$ which will be clear from the context.

\section{Cardinality}
\label{section-cardinals}

\noindent
The {\it cardinality} of a set $A$ is the least ordinal $\alpha$
such that there exists a bijection between $A$ and $\alpha$.
We sometimes use the notation $\alpha = |A|$ to indicate this.
We say an ordinal $\alpha$ is a {\it cardinal} if and only
if it occurs as the cardinality of some set $A$---in other words, if
$\alpha = |A|$. We use the greek letters $\kappa$, $\lambda$
for cardinals. The first infinite cardinal is $\omega$, and in this
context it is denoted $\aleph_0$. A set is {\it countable} if its cardinality
is $\leq \aleph_0$. If $\alpha$ is an ordinal, then we denote
$\alpha^+$ the least cardinal $> \alpha$. You can use this to
define $\aleph_1 = \aleph_0^+$, $\aleph_2 = \aleph_1^+$, etc, and
in fact you can define $\aleph_\alpha$ for any ordinal $\alpha$ by
transfinite induction. We note the equality $\aleph_1 = \omega_1$.

\medskip\noindent
The {\it addition} of cardinals $\kappa, \lambda$
is denoted $\kappa \oplus \lambda$; it is the cardinality of
$\kappa \amalg \lambda$. The {\it multiplication} of cardinals
$\kappa, \lambda$ is denoted $\kappa \otimes \lambda$; it is the
cardinality of $\kappa \times \lambda$. If
$\kappa$ and $\lambda$ are infinite cardinals, then
$\kappa \oplus \lambda = \kappa \otimes \lambda = \max(\kappa, \lambda)$.
The {\it exponentiation}
of cardinals $\kappa, \lambda$ is denoted $\kappa^\lambda$; it is
the cardinality of the set of (set) maps from $\lambda$ to $\kappa$.
Given any set $K$ of cardinals, the {\it supremum} of $K$
is $\sup_{\kappa \in K} \kappa = \bigcup_{\kappa \in K} \kappa$,
which is also a cardinal.

\section{Cofinality}
\label{section-cofinality}

\noindent
A {\it cofinal subset} $S$ of a well-ordered set $T$ is a subset
$S \subset T$ such that $\forall t \in T \exists s\in S (t \leq s)$.
Note that a subset of a well-ordered set is a well-ordered set
(with induced ordering). Given an ordinal $\alpha$, the {\it cofinality}
$\text{cf}(\alpha)$ of $\alpha$ is the least ordinal $\beta$
which occurs as the order type of some cofinal subset of $\alpha$.
The cofinality of an ordinal is always a cardinal (this is clear from
the definition). Hence alternatively we can define the cofinality of
$\alpha$ as the least cardinality of a cofinal subset of $\alpha$.

\begin{lemma}
\label{lemma-map-from-set-lifts}
Suppose that $T = \colim_{\alpha < \beta} T_\alpha$
is a colimit of sets indexed by ordinals less than a given ordinal $\beta$.
Suppose that $\varphi : S \to T$ is a map of sets.
Then $\varphi$ lifts to a map into $T_\alpha$ for some $\alpha < \beta$
provided that $\beta$ is not a limit of ordinals indexed by $S$,
in other words, if $\beta$ is an ordinal with $\text{cf}(\beta) > |S|$.
\end{lemma}

\begin{proof}
For each element $s \in S$ pick a $\alpha_s < \beta$ and an element
$t_s \in T_{\alpha_s}$ which maps to $\varphi(s)$ in $T$.
By assumption $\alpha = \sup_{s \in S} \alpha_s$ is strictly smaller
than $\beta$. Hence the map $\varphi_\alpha : S \to T_\alpha$
which assigns to $s$ the image of $t_s$ in $T_\alpha$ is a solution.
\end{proof}

\noindent
The following is essentially Grothendieck's argument for the existence
of ordinals with arbitrarily large cofinality which he used to prove
the existence of enough injectives in certain abelian categories, see
\cite{Tohoku}.

\begin{proposition}
\label{proposition-exist-ordinals-large-cofinality}
Let $\kappa$ be a cardinal. Then there exists an ordinal
whose cofinality is bigger than $\kappa$.
\end{proposition}

\begin{proof}
If $\kappa$ is finite, then $\omega = \text{cf}(\omega)$ works.
Let us thus assume that $\kappa$ is infinite.
Consider the smallest ordinal $\alpha$ whose cardinality is strictly greater
than $\kappa$. We claim that $\text{cf}(\alpha) > \kappa$.
Note that $\alpha$ is a limit ordinal, since if $\alpha = \beta + 1$, then
$|\alpha| = |\beta|$ (because $\alpha$ and $\beta$ are infinite) and
this contradicts the minimality of $\alpha$. (Of course $\alpha$ is also
a cardinal, but we do not need this.) To get a contradiction
suppose $S \subset \alpha$ is a cofinal
subset with $|S| \leq \kappa$. For $\beta \in S$, i.e., $\beta < \alpha$,
we have $|\beta| \leq \kappa$ by minimality of $\alpha$. As $\alpha$ is
a limit ordinal and $S$ cofinal in $\alpha$ we obtain
$\alpha = \bigcup_{\beta \in S} \beta$. Hence
$|\alpha| \leq |S| \otimes \kappa \leq \kappa \otimes \kappa \leq \kappa$
which is a contradiction with our choice of $\alpha$.
\end{proof}



\section{Reflection principle}
\label{section-reflection-principle}

\noindent
Some of this material is in the chapter of \cite{Kunen} called
``Easy consistency proofs''.

\medskip\noindent
Let $\phi(x_1, \ldots, x_n)$ be a formula of set theory.
Let us use the convention that this notation implies that
all the free variables in $\phi$ occur among $x_1, \ldots, x_n$.
Let $M$ be a set.
The formula $\phi^M(x_1, \ldots, x_n)$ is the
formula obtained from $\phi(x_1, \ldots, x_n)$ by replacing all the
$\forall x$ and $\exists x$ by $\forall x\in M$ and $\exists x\in M$,
respectively. So the formula
$\phi(x_1, x_2) = \exists x (x\in x_1 \wedge x\in x_2)$
is turned  into
$\phi^M(x_1, x_2) = \exists x \in M (x\in x_1 \wedge x\in x_2)$.
The formula $\phi^M$ is called the {\it relativization of $\phi$
to $M$}.

\begin{theorem}
\label{theorem-reflection-principle}
Suppose given $\phi_1(x_1, \ldots, x_n), \ldots, \phi_m(x_1, \ldots, x_n)$
a {\bf finite} collection of
formulas of set theory. Let $M_0$ be a set.
There exists a set $M$ such that
$M_0 \subset M$ and
$\forall x_1, \ldots, x_n \in M$, we have
$$
\forall i = 1, \ldots, m, \ \phi_i^{M}(x_1, \ldots, x_n)
\Leftrightarrow
\forall i = 1, \ldots, m, \ \phi_i(x_1, \ldots, x_n).
$$
In fact we may take $M = V_\alpha$ for some limit ordinal $\alpha$.
\end{theorem}

\begin{proof}
See \cite[Theorem 12.14]{Jech} or \cite[Theorem 7.4]{Kunen}.
\end{proof}

\noindent
We view this theorem as saying the following: Given any
$x_1, \ldots, x_n \in M$ the formulas hold with the bound variables ranging
through all sets if and only if they hold for the bound variables ranging
through elements of $V_\alpha$. This theorem is a meta-theorem because
it deals with the formulas of set theory directly.
It actually says that given the finite list of formulas
$\phi_1, \ldots, \phi_m$ with at most free variables $x_1, \ldots, x_n$
the sentence
$$
\begin{matrix}
\forall M_0\ \exists M, \ M_0 \subset M\ \forall x_1, \ldots, x_n \in M \\
\phi_1(x_1, \ldots, x_n) \wedge \ldots \wedge \phi_m(x_1, \ldots, x_n)
\leftrightarrow
\phi_1^M(x_1, \ldots, x_n) \wedge \ldots \wedge \phi_m^M(x_1, \ldots, x_n)
\end{matrix}
$$
is provable in ZFC. In other words, whenever we actually write down
a finite list of formulas $\phi_i$, we get a theorem.

\medskip\noindent
It is somewhat hard to use this theorem in ``ordinary mathematics''
since the meaning of the formulas $\phi_i^M(x_1, \ldots, x_n)$
is not so clear! Instead, we will use the idea of the proof of the
reflection principle to prove the existence results we need directly.

\section{Constructing categories of schemes}
\label{section-categories-schemes}

\noindent
We will discuss how to apply this to produce, given an initial
set of schemes, a ``small'' category of schemes closed under
a list of natural operations. Before we do so, we introduce the
size of a scheme. Given a scheme $S$ we define
$$
\text{size}(S) = \max(\aleph_0, \kappa_1, \kappa_2),
$$
where we define the cardinal numbers $\kappa_1$ and $\kappa_2$ as follows:
\begin{enumerate}
\item We let $\kappa_1$ be the cardinality of the set of affine opens of $S$.
\item We let $\kappa_2$ be the supremum of all the cardinalities of
all $\Gamma(U, \mathcal{O}_S)$ for all $U \subset S$ affine open.
\end{enumerate}

\begin{lemma}
\label{lemma-bounded-size}
For every cardinal $\kappa$, there exists a set $A$ such
that every element of $A$ is a scheme and such that for every
scheme $S$ with $\text{size}(S) \leq \kappa$, there is
an element $X \in A$ such that $X \cong S$ (isomorphism
of schemes).
\end{lemma}

\begin{proof}
Omitted. Hint: think about how any scheme is isomorphic to a scheme
obtained by glueing affines.
\end{proof}

\noindent
We denote $Bound$ the function which to each
cardinal $\kappa$ associates
\begin{equation}
\label{equation-bound}
Bound(\kappa) = \max\{\kappa^{\aleph_0}, \kappa^+\}.
\end{equation}
We could make this function grow much more rapidly, e.g., we could
set $Bound(\kappa) = \kappa^\kappa$, and the result below would still hold.
For any ordinal $\alpha$, we denote $\Sch_\alpha$ the full
subcategory of category of schemes whose objects are elements of
$V_\alpha$. Here is the result we are going to prove.

\begin{lemma}
\label{lemma-construct-category}
With notations $\text{size}$, $Bound$ and $\Sch_\alpha$ as above.
Let $S_0$ be a set of schemes. There exists a limit ordinal
$\alpha$ with the following properties:
\begin{enumerate}
\item
\label{item-inclusion}
We have $S_0 \subset V_\alpha$; in other words,
$S_0 \subset \Ob(\Sch_\alpha)$.
\item
\label{item-bounded}
For any $S \in \Ob(\Sch_\alpha)$ and any
scheme $T$ with $\text{size}(T) \leq Bound(\text{size}(S))$,
there exists a scheme $S' \in \Ob(\Sch_\alpha)$
such that $T \cong S'$.
\item
\label{item-limit}
For any countable\footnote{Both the set of objects and
the morphism sets are countable. In fact you can prove the lemma with
$\aleph_0$ replaced by any cardinal whatsoever in (3) and (4).} diagram
category $\mathcal{I}$ and
any functor $F : \mathcal{I} \to \Sch_\alpha$, the limit
$\lim_\mathcal{I} F$ exists in $\Sch_\alpha$ if and
only if it exists in $\Sch$ and moreover, in this case,
the natural morphism between them is an isomorphism.
\item
\label{item-colimit}
For any countable diagram category $\mathcal{I}$ and
any functor $F : \mathcal{I} \to \Sch_\alpha$, the colimit
$\colim_\mathcal{I} F$ exists in $\Sch_\alpha$ if and
only if it exists in $\Sch$ and moreover, in this case,
the natural morphism between them is an isomorphism.
\end{enumerate}
\end{lemma}

\begin{proof}
We define, by transfinite induction, a function $f$ which associates
to every ordinal an ordinal as follows. Let $f(0) = 0$.
Given $f(\alpha)$, we define $f(\alpha + 1)$ to be the least
ordinal $\beta$ such that the following hold:
\begin{enumerate}
\item We have $\alpha + 1 \leq \beta$ and $f(\alpha) \leq \beta$.
\item For any $S \in \Ob(\Sch_{f(\alpha)})$ and any
scheme $T$ with $\text{size}(T) \leq Bound(\text{size}(S))$,
there exists a scheme $S' \in \Ob(\Sch_\beta)$
such that $T \cong S'$.
\item For any countable diagram category $\mathcal{I}$ and
any functor $F : \mathcal{I} \to \Sch_{f(\alpha)}$, if
the limit $\lim_\mathcal{I} F$ or the colimit
$\colim_\mathcal{I} F$ exists in $\Sch$,
then it is isomorphic to a scheme in $\Sch_\beta$.
\end{enumerate}
To see $\beta$ exists, we argue as follows. Since
$\Ob(\Sch_{f(\alpha)})$ is a set, we see that
$\kappa =
\sup_{S \in \Ob(\Sch_{f(\alpha)})} Bound(\text{size}(S))$
exists and is a cardinal.
Let $A$ be a set of schemes obtained starting with $\kappa$
as in Lemma \ref{lemma-bounded-size}.
There is a set $CountCat$ of countable
categories such that any countable category is isomorphic to
an element of $CountCat$. Hence in (3) above we may assume
that $\mathcal{I}$ is an element in $CountCat$. This means that
the pairs $(\mathcal{I}, F)$ in (3) range over a set.
Thus, there exists a set $B$ whose elements are schemes
such that for every $(\mathcal{I}, F)$ as in (3), if the
limit or colimit exists, then it is isomorphic to an element in $B$.
Hence, if we pick any $\beta$ such that $A \cup B \subset V_\beta$
and $\beta > \max\{\alpha + 1, f(\alpha)\}$, then (1)--(3) hold.
Since every nonempty collection of ordinals has a least element,
we see that $f(\alpha + 1)$ is well defined. Finally, if $\alpha$
is a limit ordinal, then we set
$f(\alpha) = \sup_{\alpha' < \alpha} f(\alpha')$.

\medskip\noindent
Pick $\beta_0$ such that $S_0 \subset V_{\beta_0}$.
By construction $f(\beta) \geq \beta$ and we see that
also $S_0 \subset V_{f(\beta_0)}$. Moreover, as $f$ is
nondecreasing, we see $S_0 \subset V_{f(\beta)}$ is true for any
$\beta \geq \beta_0$.
Next, choose any ordinal $\beta_1 > \beta_0$ with cofinality
$\text{cf}(\beta_1) > \omega = \aleph_0$. This is possible
since the cofinality of ordinals gets arbitrarily large, see
Proposition \ref{proposition-exist-ordinals-large-cofinality}.
We claim that
$\alpha = f(\beta_1)$ is a solution to the problem posed in the lemma.

\medskip\noindent
The first property of the lemma holds by our choice
of $\beta_1 > \beta_0$ above.

\medskip\noindent
Since $\beta_1$ is a limit ordinal (as its cofinality is infinite),
we get $f(\beta_1) = \sup_{\beta < \beta_1} f(\beta)$.
Hence $\{f(\beta) \mid \beta < \beta_1\} \subset f(\beta_1)$ is a
cofinal subset. Hence we see that
$$
V_\alpha = V_{f(\beta_1)} = \bigcup\nolimits_{\beta < \beta_1} V_{f(\beta)}.
$$
Now, let $S \in \Ob(\Sch_\alpha)$. We define
$\beta(S)$ to be the least ordinal $\beta$ such that
$S \in \Ob(\Sch_{f(\beta)})$. By the above we see
that always $\beta(S) < \beta_1$. Since
$\Ob(\Sch_{f(\beta + 1)}) \subset
\Ob(\Sch_\alpha)$, we
see by construction of $f$ above that the second property of the lemma
is satisfied.

\medskip\noindent
Suppose that $\{S_1, S_2, \ldots\} \subset \Ob(\Sch_\alpha)$
is a countable collection. Consider the function
$\omega \to \beta_1$, $n \mapsto \beta(S_n)$. Since the cofinality
of $\beta_1$ is $> \omega$, the image of this function cannot be a
cofinal subset. Hence there exists a $\beta < \beta_1$ such
that $\{S_1, S_2, \ldots\} \subset \Ob(\Sch_{f(\beta)})$.
It follows that any functor $F : \mathcal{I} \to \Sch_\alpha$
factors through one of the subcategories $\Sch_{f(\beta)}$.
Thus, if there exists a scheme $X$ that is the colimit or limit
of the diagram $F$, then, by construction of $f$, we see
$X$ is isomorphic to an object
of $\Sch_{f(\beta + 1)}$ which is a subcategory of
$\Sch_\alpha$. This proves the last two assertions of
the lemma.
\end{proof}

\begin{remark}
\label{remark-how-to-use-reflection}
The lemma above can also be proved using the reflection principle.
However, one has to be careful. Namely, suppose the sentence
$\phi_{scheme}(X)$ expresses the property ``$X$ is a scheme'', then
what does the formula $\phi_{scheme}^{V_\alpha}(X)$ mean?
It is true that the reflection principle says we can find $\alpha$ such that
for all $X \in V_\alpha$ we have
$\phi_{scheme}(X) \leftrightarrow \phi_{scheme}^{V_\alpha}(X)$
but this is entirely useless. It is only by combining two such
statements that something interesting happens. For example suppose
$\phi_{red}(X, Y)$ expresses the property ``$X$, $Y$ are schemes,
and $Y$ is the reduction of $X$'' (see
Schemes, Definition \ref{schemes-definition-reduced-induced-scheme}).
Suppose we apply the reflection principle to the pair of
formulas $\phi_1(X, Y) = \phi_{red}(X, Y)$,
$\phi_2(X) = \exists Y, \phi_1(X, Y)$. Then it is easy to see that
any $\alpha$ produced by the reflection principle has the property that
given $X \in \Ob(\Sch_\alpha)$ the reduction of
$X$ is also an object of $\Sch_\alpha$ (left as an exercise).
\end{remark}

\begin{lemma}
\label{lemma-bound-affine}
Let $S$ be an affine scheme.
Let $R = \Gamma(S, \mathcal{O}_S)$.
Then the size of $S$ is equal to $\max\{ \aleph_0, |R|\}$.
\end{lemma}

\begin{proof}
There are at most $\max\{|R|, \aleph_0\}$ affine opens of
$\Spec(R)$. This is clear since any affine open
$U \subset \Spec(R)$ is a finite union of principal
opens $D(f_1) \cup \ldots \cup D(f_n)$ and hence the number
of affine opens is at most $\sup_n |R|^n = \max\{|R|, \aleph_0\}$,
see \cite[Ch. I, 10.13]{Kunen}. On the other hand, we see that
$\Gamma(U, \mathcal{O}) \subset R_{f_1} \times \ldots \times R_{f_n}$
and hence $|\Gamma(U, \mathcal{O})| \leq
\max\{\aleph_0, |R_{f_1}|, \ldots, |R_{f_n}|\}$. Thus
it suffices to prove that $|R_f| \leq \max\{\aleph_0, |R|\}$
which is omitted.
\end{proof}

\begin{lemma}
\label{lemma-bound-size}
Let $S$ be a scheme. Let $S = \bigcup_{i \in I} S_i$ be
an open covering. Then
$\text{size}(S) \leq \max\{|I|, \sup_i\{\text{size}(S_i)\}\}$.
\end{lemma}

\begin{proof}
Let $U \subset S$ be any affine open. Since $U$ is quasi-compact
there exist finitely many elements $i_1, \ldots, i_n \in I$
and affine opens $U_i \subset U \cap S_i$ such that
$U = U_1 \cup U_2 \cup \ldots \cup U_n$. Thus
$$
|\Gamma(U, \mathcal{O}_U)|
\leq
|\Gamma(U_1, \mathcal{O})|
\otimes
\ldots
\otimes
|\Gamma(U_n, \mathcal{O})|
\leq \sup\nolimits_i\{\text{size}(S_i)\}
$$
Moreover, it shows that the set of affine opens of $S$ has
cardinality less than or equal to the cardinality of the set
$$
\coprod_{n \in \omega}
\coprod_{i_1, \ldots, i_n \in I}
\{\text{affine opens of }S_{i_1}\}
\times
\ldots
\times
\{\text{affine opens of }S_{i_n}\}.
$$
Each of the sets inside the disjoint union has cardinality at most
$\sup_i\{\text{size}(S_i)\}$. The index set has cardinality at most
$\max\{|I|, \aleph_0\}$, see \cite[Ch. I, 10.13]{Kunen}.
Hence by \cite[Lemma 5.8]{Jech} the cardinality
of the coproduct is at most $\max\{\aleph_0, |I|\}
\otimes \sup_i\{\text{size}(S_i)\}$. The lemma follows.
\end{proof}

\begin{lemma}
\label{lemma-bound-size-fibre-product}
Let $f : X \to S$, $g : Y \to S$ be morphisms of schemes.
Then we have
$\text{size}(X \times_S Y) \leq \max\{\text{size}(X), \text{size}(Y)\}$.
\end{lemma}

\begin{proof}
Let $S = \bigcup_{k \in K} S_k$ be an affine open covering.
Let $X = \bigcup_{i \in I} U_i$, $Y = \bigcup_{j \in J} V_j$
be affine open coverings with $I$, $J$ of cardinality
$\leq \text{size}(X), \text{size}(Y)$.
For each $i \in I$ there exists a finite set $K_i$ of $k \in K$
such that $f(U_i) \subset \bigcup_{k \in K_i} S_k$.
For each $j \in J$ there exists a finite set $K_j$ of $k \in K$
such that $g(V_j) \subset \bigcup_{k \in K_j} S_k$.
Hence $f(X), g(Y)$ are contained in
$S' = \bigcup_{k \in K'} S_k$ with
$K' = \bigcup_{i \in I} K_i \cup \bigcup_{j \in J} K_j$.
Note that the cardinality of $K'$
is at most $\max\{\aleph_0, |I|, |J|\}$. Applying
Lemma \ref{lemma-bound-size}
we see that it suffices to prove that
$\text{size}(f^{-1}(S_k) \times_{S_k} g^{-1}(S_k))
\leq \max\{\text{size}(X), \text{size}(Y))\}$ for $k \in K'$.
In other words, we may assume that $S$ is affine.

\medskip\noindent
Assume $S$ affine.
Let $X = \bigcup_{i \in I} U_i$, $Y = \bigcup_{j \in J} V_j$
be affine open coverings with $I$, $J$ of cardinality
$\leq \text{size}(X), \text{size}(Y)$.
Again by
Lemma \ref{lemma-bound-size}
it suffices to prove the lemma for the products
$U_i \times_S V_j$. By
Lemma \ref{lemma-bound-affine}
we see that it suffices to show that
$$
|A \otimes_C B| \leq \max\{\aleph_0, |A|, |B|\}.
$$
We omit the proof of this inequality.
\end{proof}

\begin{lemma}
\label{lemma-bound-finite-type}
Let $S$ be a scheme.
Let $f : X \to S$ be locally of finite type with $X$ quasi-compact.
Then $\text{size}(X) \leq \text{size}(S)$.
\end{lemma}

\begin{proof}
We can find a finite affine open covering $X = \bigcup_{i = 1, \ldots n} U_i$
such that each $U_i$ maps into an affine open $S_i$ of $S$. Thus by
Lemma \ref{lemma-bound-size}
we reduce to the case where both $S$ and $X$ are affine. In this case by
Lemma \ref{lemma-bound-affine}
we see that it suffices to show
$$
|A[x_1, \ldots, x_n]| \leq \max\{\aleph_0, |A|\}.
$$
We omit the proof of this inequality.
\end{proof}

\noindent
In
Algebra, Lemma \ref{algebra-lemma-epimorphism-cardinality}
we will show that if $A \to B$ is an epimorphism of rings, then
$|B| \leq \max(|A|, \aleph_0)$.
The analogue for schemes is the following lemma.

\begin{lemma}
\label{lemma-bound-monomorphism}
Let $f : X \to Y$ be a monomorphism of schemes.
If at least one of the following properties
holds, then $\text{size}(X) \leq \text{size}(Y)$:
\begin{enumerate}
\item $f$ is quasi-compact,
\item $f$ is locally of finite presentation,
\item add more here as needed.
\end{enumerate}
But the bound does not hold for monomorphisms
which are locally of finite type.
\end{lemma}

\begin{proof}
Let $Y = \bigcup_{j \in J} V_j$ be an affine open covering of $Y$
with $|J| \leq \text{size}(Y)$. By Lemma \ref{lemma-bound-size}
it suffices to bound the size of the inverse image of $V_j$ in $X$.
Hence we reduce to the case that $Y$ is affine, say $Y = \Spec(B)$.
For any affine open $\Spec(A) \subset X$ we have
$|A| \leq \max(|B|, \aleph_0) = \text{size}(Y)$, see remark above
and Lemma \ref{lemma-bound-affine}. Thus it suffices to show
that $X$ has at most $\text{size}(Y)$ affine opens. This is clear
if $X$ is quasi-compact, whence case (1) holds.
In case (2) the number of isomorphism classes of $B$-algebras $A$
that can occur is bounded by $\text{size}(B)$, because each
$A$ is of finite type over $B$, hence isomorphic to an algebra
$B[x_1, \ldots, x_n]/(f_1, \ldots, f_m)$
for some $n, m$, and $f_j \in B[x_1, \ldots, x_n]$. However, as
$X \to Y$ is a monomorphism, there is a unique morphism
$\Spec(A) \to X$ over $Y = \Spec(B)$ if there is one,
hence the number of affine
opens of $X$ is bounded by the number of these isomorphism classes.

\medskip\noindent
To prove the final statement of the lemma consider the ring
$B = \prod_{n \in \mathbf{N}} \mathbf{F}_2$ and set $Y = \Spec(B)$.
For every ultrafilter $\mathcal{U}$ on $\mathbf{N}$ we obtain a maximal
ideal $\mathfrak m_\mathcal{U}$ with residue field $\mathbf{F}_2$;
the map $B \to \mathbf{F}_2$ sends the element $(x_n)$ to
$\lim_\mathcal{U} x_n$. Details omitted.
The morphism of schemes $X = \coprod_\mathcal{U} \Spec(\mathbf{F}_2) \to Y$
is a monomorphism as all the points are distinct. However the cardinality
of the set of affine open subschemes of $X$ is equal to the cardinality
of the set of ultrafilters on $\mathbf{N}$ which is
$2^{2^{\aleph_0}}$. We conclude as $|B| = 2^{\aleph_0} < 2^{2^{\aleph_0}}$.
\end{proof}

\begin{lemma}
\label{lemma-what-is-in-it}
Let $\alpha$ be an ordinal as in Lemma \ref{lemma-construct-category} above.
The category $\Sch_\alpha$ satisfies the following
properties:
\begin{enumerate}
\item If $X, Y, S \in \Ob(\Sch_\alpha)$, then
for any morphisms $f : X \to S$, $g : Y \to S$ the fibre
product $X \times_S Y$ in $\Sch_\alpha$ exists
and is a fibre product in the category of schemes.
\item Given any at most countable collection $S_1, S_2, \ldots$
of elements of $\Ob(\Sch_\alpha)$, the coproduct
$\coprod_i S_i$ exists in $\Ob(\Sch_\alpha)$ and
is a coproduct in the category of schemes.
\item For any $S \in \Ob(\Sch_\alpha)$ and
any open immersion $U \to S$, there exists a
$V \in \Ob(\Sch_\alpha)$ with $V \cong U$.
\item For any $S \in \Ob(\Sch_\alpha)$ and
any closed immersion $T \to S$, there exists an
$S' \in \Ob(\Sch_\alpha)$ with $S' \cong T$.
\item For any $S \in \Ob(\Sch_\alpha)$ and
any finite type morphism $T \to S$, there exists an
$S' \in \Ob(\Sch_\alpha)$ with $S' \cong T$.
\item Suppose $S$ is a scheme which has an open covering
$S = \bigcup_{i \in I} S_i$ such that there exists
a $T \in \Ob(\Sch_\alpha)$ with
(a) $\text{size}(S_i) \leq \text{size}(T)^{\aleph_0}$ for all
$i \in I$, and (b) $|I| \leq \text{size}(T)^{\aleph_0}$.
Then $S$ is isomorphic to an object of $\Sch_\alpha$.
\item For any $S \in \Ob(\Sch_\alpha)$ and
any morphism $f : T \to S$ locally of finite type such
that $T$ can be covered by at most
$\text{size}(S)^{\aleph_0}$ open affines, there exists an
$S' \in \Ob(\Sch_\alpha)$ with $S' \cong T$.
For example this holds if $T$ can be covered by at most
$|\mathbf{R}| = 2^{\aleph_0} = \aleph_0^{\aleph_0}$ open affines.
\item For any $S \in \Ob(\Sch_\alpha)$ and
any monomorphism $T \to S$ which is either locally of finite presentation
or quasi-compact, there exists an
$S' \in \Ob(\Sch_\alpha)$ with $S' \cong T$.
\item Suppose that $T \in \Ob(\Sch_\alpha)$ is
affine. Write $R = \Gamma(T, \mathcal{O}_T)$.
Then any of the following schemes is isomorphic to a scheme
in $\Sch_\alpha$:
\begin{enumerate}
\item For any ideal $I \subset R$ with completion
$R^* = \lim_n R/I^n$, the scheme $\Spec(R^*)$.
\item For any finite type $R$-algebra $R'$, the
scheme $\Spec(R')$.
\item For any localization $S^{-1}R$, the scheme $\Spec(S^{-1}R)$.
\item For any prime $\mathfrak p \subset R$, the scheme
$\Spec(\overline{\kappa(\mathfrak p)})$.
\item For any subring $R' \subset R$, the scheme
$\Spec(R')$.
\item Any scheme of finite type over a ring of cardinality at most
$|R|^{\aleph_0}$.
\item And so on.
\end{enumerate}
\end{enumerate}
\end{lemma}

\begin{proof}
Statements (1) and (2) follow directly from the definitions.
Statement (3) follows as the size of an open subscheme $U$ of $S$ is
clearly smaller than or equal to the size of $S$.
Statement (4) follows from (5).
Statement (5) follows from (7).
Statement (6) follows as the size of $S$ is
$\leq \max\{|I|, \sup_i \text{size}(S_i)\} \leq \text{size}(T)^{\aleph_0}$
by Lemma \ref{lemma-bound-size}. Statement (7) follows from (6).
Namely, for any affine open $V \subset T$ we have
$\text{size}(V) \leq \text{size}(S)$ by
Lemma \ref{lemma-bound-finite-type}.
Thus, we see that (6) applies in the situation of (7).
Part (8) follows from
Lemma \ref{lemma-bound-monomorphism}.

\medskip\noindent
Statement (9) is translated, via Lemma \ref{lemma-bound-affine},
into an upper bound on the cardinality of the rings
$R^*$, $S^{-1}R$, $\overline{\kappa(\mathfrak p)}$, $R'$, etc.
Perhaps the most interesting one is the ring $R^*$. As a
set, it is the image of a surjective map $R^{\mathbf{N}} \to R^*$.
Since $|R^{\mathbf{N}}| = |R|^{\aleph_0}$, we see that
it works by our choice of $Bound(\kappa)$ being at least $\kappa^{\aleph_0}$.
Phew! (The cardinality of the algebraic closure of a field
is the same as the cardinality of the field, or it is $\aleph_0$.)
\end{proof}

\begin{remark}
\label{remark-what-is-not-in-it}
Let $R$ be a ring. Suppose we consider the ring
$\prod_{\mathfrak p \in \Spec(R)} \kappa(\mathfrak p)$.
The cardinality of this ring is bounded by $|R|^{2^{|R|}}$,
but is not bounded by $|R|^{\aleph_0}$ in general.
For example if $R = \mathbf{C}[x]$ it is not bounded by
$|R|^{\aleph_0}$ and if $R = \prod_{n \in \mathbf{N}} \mathbf{F}_2$
it is not bounded by $|R|^{|R|}$.
Thus the ``And so on'' of Lemma \ref{lemma-what-is-in-it} above
should be taken with a grain of salt. Of course, if it ever becomes
necessary to consider these rings in arguments pertaining to
fppf/\'etale cohomology, then we can change the function
$Bound$ above into the function $\kappa \mapsto \kappa^{2^\kappa}$.
\end{remark}

\noindent
In the following lemma we use the notion of an fpqc covering which
is introduced in Topologies, Section \ref{topologies-section-fpqc}.

\begin{lemma}
\label{lemma-bound-by-covering}
Let $f : X \to Y$ be a morphism of schemes. Assume there exists an
fpqc covering $\{g_j : Y_j \to Y\}_{j \in J}$ such that $g_j$ factors
through $f$. Then $\text{size}(Y) \leq \text{size}(X)$.
\end{lemma}

\begin{proof}
Let $V \subset Y$ be an affine open. By definition there exist
$n \geq 0$ and $a : \{1, \ldots, n\} \to J$ and affine opens
$V_i \subset Y_{a(i)}$ such that
$V = g_{a(1)}(V_1) \cup \ldots \cup g_{a(n)}(V_n)$.
Denote $h_j : Y_j \to X$ a morphism such that $f \circ h_j = g_j$.
Then $h_{a(1)}(V_1) \cup \ldots \cup h_{a(n)}(V_n)$ is
a quasi-compact subset of $f^{-1}(V)$. Hence we can find a
quasi-compact open $W \subset f^{-1}(V)$ which contains
$h_{a(i)}(V_i)$ for $i = 1, \ldots, n$.
In particular $V = f(W)$.

\medskip\noindent
On the one hand this shows that the cardinality of the set of
affine opens of $Y$ is at most the cardinality of the set $S$ of
quasi-compact opens of $X$. Since every quasi-compact open of
$X$ is a finite union of affines, we see that the cardinality
of this set is at most $\sup |S|^n = \max(\aleph_0, |S|)$.
On the other hand, we have
$\mathcal{O}_Y(V) \subset \prod_{i = 1, \ldots, n} \mathcal{O}_{Y_{a(i)}}(V_i)$
because $\{V_i \to V\}$ is an fpqc covering. Hence
$\mathcal{O}_Y(V) \subset \mathcal{O}_X(W)$ because $V_i \to V$
factors through $W$. Again since $W$ has a finite covering by
affine opens of $X$ we conclude that $|\mathcal{O}_Y(V)|$
is bounded by the size of $X$. The lemma now follows
from the definition of the size of a scheme.
\end{proof}

\noindent
In the following lemma we use the notion of an fppf covering which
is introduced in Topologies, Section \ref{topologies-section-fppf}.

\begin{lemma}
\label{lemma-bound-fppf-covering}
Let $\{f_i : X_i \to X\}_{i \in I}$ be an fppf covering of a scheme.
There exists an fppf covering $\{W_j \to X\}_{j \in J}$
which is a refinement of $\{X_i \to X\}_{i \in I}$ such that
$\text{size}(\coprod W_j) \leq \text{size}(X)$.
\end{lemma}

\begin{proof}
Choose an affine open covering $X = \bigcup_{a \in A} U_a$ with
$|A| \leq \text{size}(X)$. For each $a$ we can choose
a finite subset $I_a \subset I$ and for $i \in I_a$ a quasi-compact open
$W_{a, i} \subset X_i$ such that $U_a = \bigcup_{i \in I_a} f_i(W_{a, i})$.
Then $\text{size}(W_{a, i}) \leq \text{size}(X)$ by
Lemma \ref{lemma-bound-finite-type}.
We conclude that
$\text{size}(\coprod_a \coprod_{i \in I_a} W_{i, a}) \leq \text{size}(X)$
by Lemma \ref{lemma-bound-size}.
\end{proof}





\section{Sets with group action}
\label{section-sets-with-group-action}

\noindent
Let $G$ be a group. We denote $G\textit{-Sets}$ the ``big'' category
of $G$-sets. For any ordinal $\alpha$, we denote
$G\textit{-Sets}_\alpha$ the full subcategory of $G\textit{-Sets}$
whose objects are in $V_\alpha$. As a notion for size of a $G$-set
we take $\text{size}(S) = \max\{\aleph_0, |G|, |S|\}$ (where $|G|$ and
$|S|$ are the cardinality of the underlying sets). As above we use the function
$Bound(\kappa) = \kappa^{\aleph_0}$.

\begin{lemma}
\label{lemma-sets-with-group-action}
With notations $G$, $G\textit{-Sets}_\alpha$, $\text{size}$,
and $Bound$ as above. Let $S_0$ be a set of $G$-sets.
There exists a limit ordinal $\alpha$ with the following properties:
\begin{enumerate}
\item We have $S_0 \cup \{{}_GG\} \subset \Ob(G\textit{-Sets}_\alpha)$.
\item For any $S \in \Ob(G\textit{-Sets}_\alpha)$ and any
$G$-set $T$ with $\text{size}(T) \leq Bound(\text{size}(S))$,
there exists an $S' \in \Ob(G\textit{-Sets}_\alpha)$
that is isomorphic to $T$.
\item For any countable diagram category $\mathcal{I}$ and
any functor $F : \mathcal{I} \to G\textit{-Sets}_\alpha$, the
limit $\lim_\mathcal{I} F$ and colimit
$\colim_\mathcal{I} F$ exist in $G\textit{-Sets}_\alpha$
and are the same as in $G\textit{-Sets}$.
\end{enumerate}
\end{lemma}

\begin{proof}
Omitted. Similar to but easier than the proof of
Lemma \ref{lemma-construct-category} above.
\end{proof}

\begin{lemma}
\label{lemma-what-is-in-it-G-sets}
Let $\alpha$ be an ordinal as in Lemma \ref{lemma-sets-with-group-action}
above. The category $G\textit{-Sets}_\alpha$ satisfies the following
properties:
\begin{enumerate}
\item The $G$-set ${}_GG$ is an object of $G\textit{-Sets}_\alpha$.
\item (Co)Products, fibre products, and pushouts
exist in $G\textit{-Sets}_\alpha$
and are the same as their counterparts in $G\textit{-Sets}$.
\item Given an object $U$ of $G\textit{-Sets}_\alpha$,
any $G$-stable subset $O \subset U$  is isomorphic to an object
of $G\textit{-Sets}_\alpha$.
\end{enumerate}
\end{lemma}

\begin{proof}
Omitted.
\end{proof}

\section{Coverings of a site}
\label{section-coverings-site}

\noindent
Suppose that $\mathcal{C}$ is a category (as in
Categories, Definition \ref{categories-definition-category}) and
that $\text{Cov}(\mathcal{C})$ is a proper class of coverings
satisfying properties (1), (2), and (3) of Sites,
Definition \ref{sites-definition-site}. We list them here:
\begin{enumerate}
\item If $V \to U$ is an isomorphism, then $\{V \to U\} \in
\text{Cov}(\mathcal{C})$.
\item If $\{U_i \to U\}_{i\in I} \in \text{Cov}(\mathcal{C})$ and for each
$i$ we have $\{V_{ij} \to U_i\}_{j\in J_i} \in \text{Cov}(\mathcal{C})$, then
$\{V_{ij} \to U\}_{i \in I, j\in J_i} \in \text{Cov}(\mathcal{C})$.
\item If $\{U_i \to U\}_{i\in I}\in \text{Cov}(\mathcal{C})$
and $V \to U$ is a morphism of $\mathcal{C}$, then $U_i \times_U V$
exists for all $i$ and
$\{U_i \times_U V \to V \}_{i\in I} \in \text{Cov}(\mathcal{C})$.
\end{enumerate}
For an ordinal $\alpha$, we set
$\text{Cov}(\mathcal{C})_\alpha = \text{Cov}(\mathcal{C}) \cap V_\alpha$.
Given an ordinal $\alpha$ and a cardinal $\kappa$, we set
$\text{Cov}(\mathcal{C})_{\kappa, \alpha}$ equal to the set
of elements
$\mathcal{U} =
\{\varphi_i : U_i \to U\}_{i\in I} \in \text{Cov}(\mathcal{C})_\alpha$
such that $|I| \leq \kappa$.

\medskip\noindent
We recall the following notion, see Sites, Definition
\ref{sites-definition-combinatorial-tautological}.
Two families of morphisms, $\{\varphi_i : U_i \to U\}_{i\in I}$ and
$\{\psi_j : W_j \to U\}_{j\in J}$, with the same target of $\mathcal{C}$ are
called {\it combinatorially equivalent} if there exist maps
$\alpha : I \to J$ and $\beta : J\to I$ such that
$\varphi_i = \psi_{\alpha(i)}$ and $\psi_j = \varphi_{\beta(j)}$.
This defines an equivalence relation on families of morphisms
having a fixed target.

\begin{lemma}
\label{lemma-coverings-site}
With notations as above.
Let $\text{Cov}_0 \subset \text{Cov}(\mathcal{C})$
be a set contained in $\text{Cov}(\mathcal{C})$.
There exist a cardinal $\kappa$ and a limit ordinal $\alpha$
with the following properties:
\begin{enumerate}
\item We have $\text{Cov}_0 \subset \text{Cov}(\mathcal{C})_{\kappa, \alpha}$.
\item The set of coverings
$\text{Cov}(\mathcal{C})_{\kappa, \alpha}$ satisfies
(1), (2), and (3) of Sites, Definition \ref{sites-definition-site} (see above).
In other words $(\mathcal{C}, \text{Cov}(\mathcal{C})_{\kappa, \alpha})$
is a site.
\item Every covering in $\text{Cov}(\mathcal{C})$
is combinatorially equivalent
to a covering in $\text{Cov}(\mathcal{C})_{\kappa, \alpha}$.
\end{enumerate}
\end{lemma}

\begin{proof}
To prove this, we first consider the set $\mathcal{S}$ of all
sets of morphisms of $\mathcal{C}$ with fixed target.
In other words, an element of $\mathcal{S}$ is a subset $T$
of $\text{Arrows}(\mathcal{C})$ such that all
elements of $T$ have the same target. Given a family
$\mathcal{U} = \{\varphi_i : U_i \to U\}_{i\in I}$ of morphisms with fixed
target, we define
$Supp(\mathcal{U}) = \{ \varphi \in \text{Arrows}(\mathcal{C})
\mid \exists i\in I, \varphi = \varphi_i\}$.
Note that two families $\mathcal{U} =  \{\varphi_i : U_i \to U\}_{i\in I}$
and $\mathcal{V} = \{V_j \to V\}_{j \in J}$ are combinatorially
equivalent if and only if $Supp(\mathcal{U}) = Supp(\mathcal{V})$.
Next, we define
$\mathcal{S}_\tau \subset \mathcal{S}$ to be the subset
$\mathcal{S}_\tau = \{ T \in \mathcal{S} \mid
\exists\ \mathcal{U} \in \text{Cov}(\mathcal{C}) \ T = Supp(\mathcal{U})\}$.
For every element $T \in \mathcal{S}_\tau$, set
$\beta(T)$ to equal the least ordinal $\beta$ such that
there exists a $\mathcal{U} \in \text{Cov}(\mathcal{C})_\beta$
such that $T = \text{Supp}(\mathcal{U})$. Finally, set
$\beta_0 = \sup_{T \in S_\tau} \beta(T)$.
At this point it follows that every $\mathcal{U} \in \text{Cov}(\mathcal{C})$
is combinatorially equivalent to some element
of $\text{Cov}(\mathcal{C})_{\beta_0}$.

\medskip\noindent
Let $\kappa$ be the maximum of $\aleph_0$,
the cardinality $|\text{Arrows}(\mathcal{C})|$,
$$
\sup\nolimits_{\{U_i \to U\}_{i\in I} \in \text{Cov}(\mathcal{C})_{\beta_0}}
|I|,
\quad\text{and}\quad
\sup\nolimits_{\{U_i \to U\}_{i\in I} \in \text{Cov}_0} |I|.
$$
Since $\kappa$ is an infinite cardinal, we have
$\kappa \otimes \kappa = \kappa$. Note that obviously
$\text{Cov}(\mathcal{C})_{\beta_0} =
\text{Cov}(\mathcal{C})_{\kappa, \beta_0}$.

\medskip\noindent
We define, by transfinite induction, a function $f$ which associates
to every ordinal an ordinal as follows. Let $f(0) = 0$.
Given $f(\alpha)$, we define $f(\alpha + 1)$ to be the least
ordinal $\beta$ such that the following hold:
\begin{enumerate}
\item We have $\alpha + 1 \leq \beta$ and $f(\alpha) \leq \beta$.
\item If $\{U_i \to U\}_{i\in I}
\in \text{Cov}(\mathcal{C})_{\kappa, f(\alpha)}$
and for each $i$ we have
$\{W_{ij} \to U_i\}_{j\in J_i}
\in \text{Cov}(\mathcal{C})_{\kappa, f(\alpha)}$,
then
$\{W_{ij} \to U\}_{i \in I, j\in J_i}
\in \text{Cov}(\mathcal{C})_{\kappa, \beta}$.
\item If $\{U_i \to U\}_{i\in I}
\in \text{Cov}(\mathcal{C})_{\kappa, \alpha}$
and $W \to U$ is a morphism of $\mathcal{C}$, then
$\{U_i \times_U W \to W \}_{i\in I}
\in \text{Cov}(\mathcal{C})_{\kappa, \beta}$.
\end{enumerate}
To see $\beta$ exists we note that clearly the collection of all
coverings $\{W_{ij} \to U\}$ and $\{U_i \times_U W \to W \}$ that occur in
(2) and (3) form a set. Hence there is some ordinal $\beta$ such that
$V_\beta$ contains all of these coverings. Moreover, the index set
of the covering $\{W_{ij} \to U\}$ has cardinality
$\sum_{i \in I} |J_i| \leq \kappa \otimes \kappa = \kappa$, and
hence these coverings are contained in
$\text{Cov}(\mathcal{C})_{\kappa, \beta}$.
Since every nonempty collection of ordinals has a least element
we see that $f(\alpha + 1)$ is well defined. Finally, if $\alpha$
is a limit ordinal, then we set
$f(\alpha) = \sup_{\alpha' < \alpha} f(\alpha')$.

\medskip\noindent
Pick an ordinal $\beta_1$ such that
$\text{Arrows}(\mathcal{C}) \subset V_{\beta_1}$,
$\text{Cov}_0 \subset V_{\beta_0}$,
and $\beta_1 \geq \beta_0$.
By construction $f(\beta_1) \geq \beta_1$ and we see that
the same properties hold for $V_{f(\beta_1)}$. Moreover, as $f$ is
nondecreasing this remains true for any $\beta \geq \beta_1$.
Next, choose any ordinal $\beta_2 > \beta_1$ with
cofinality $\text{cf}(\beta_2) > \kappa$. This is possible
since the cofinality of ordinals gets arbitrarily large, see
Proposition \ref{proposition-exist-ordinals-large-cofinality}.
We claim that the pair $\kappa$,
$\alpha = f(\beta_2)$ is a solution to the problem posed in the lemma.

\medskip\noindent
The first and third property of the lemma holds by our choices
of $\kappa$, $\beta_2 > \beta_1 > \beta_0$ above.

\medskip\noindent
Since $\beta_2$ is a limit ordinal (as its cofinality is infinite)
we get $f(\beta_2) = \sup_{\beta < \beta_2} f(\beta)$.
Hence $\{f(\beta) \mid \beta < \beta_2\} \subset f(\beta_2)$ is a
cofinal subset. Hence we see that
$$
V_\alpha = V_{f(\beta_2)} = \bigcup\nolimits_{\beta < \beta_2} V_{f(\beta)}.
$$
Now, let $\mathcal{U} \in \text{Cov}_{\kappa, \alpha}$.
We define $\beta(\mathcal{U})$ to be the least ordinal $\beta$ such that
$\mathcal{U} \in \text{Cov}_{\kappa, f(\beta)}$. By the above we see
that always $\beta(\mathcal{U}) < \beta_2$.

\medskip\noindent
We have to show properties (1), (2), and (3) defining a site
hold for the pair $(\mathcal{C}, \text{Cov}_{\kappa, \alpha})$.
The first holds because by our choice of $\beta_2$
all arrows of $\mathcal{C}$ are contained in $V_{f(\beta_2)}$.
For the third, we use that given a covering
$\mathcal{U} = \{U_i \to U\}_{i \in I}
\in \text{Cov}(\mathcal{C})_{\kappa, \alpha}$
we have $\beta(\mathcal{U}) < \beta_2$ and hence
any base change of $\mathcal{U}$ is by construction of
$f$ contained in $\text{Cov}(\mathcal{C})_{\kappa, f(\beta + 1)}$
and hence in $\text{Cov}(\mathcal{C})_{\kappa, \alpha}$.

\medskip\noindent
Finally, for the second condition, suppose that $\{U_i \to U\}_{i\in I}
\in \text{Cov}(\mathcal{C})_{\kappa, f(\alpha)}$
and for each $i$ we have
$\mathcal{W}_i = \{W_{ij} \to U_i\}_{j\in J_i}
\in \text{Cov}(\mathcal{C})_{\kappa, f(\alpha)}$.
Consider the function
$I \to \beta_2$, $i \mapsto \beta(\mathcal{W}_i)$. Since the cofinality
of $\beta_2$ is $> \kappa \geq |I|$ the image of this function cannot be a
cofinal subset. Hence there exists a $\beta < \beta_1$ such
that $\mathcal{W}_i \in \text{Cov}_{\kappa, f(\beta)}$ for all $i \in I$.
It follows that the covering $\{W_{ij} \to U\}_{i\in I, j \in J_i}$
is an element of $\text{Cov}(\mathcal{C})_{\kappa, f(\beta + 1)}
\subset \text{Cov}(\mathcal{C})_{\kappa, \alpha}$ as desired.
\end{proof}

\begin{remark}
\label{remark-better}
It is likely the case that, for some limit ordinal $\alpha$,
the set of coverings $\text{Cov}(\mathcal{C})_\alpha$ satisfies
the conditions of the lemma. This is after all what an application
of the reflection principle would appear to give (modulo caveats as
described at the end of Section \ref{section-reflection-principle}
and in Remark \ref{remark-how-to-use-reflection}).
\end{remark}








\section{Abelian categories and injectives}
\label{section-abelian-categories-injectives}

\noindent
The following lemma applies to the category of modules over a sheaf
of rings on a site.

\begin{lemma}
\label{lemma-abelian-injectives}
Suppose given a big category $\mathcal{A}$ (see
Categories, Remark \ref{categories-remark-big-categories}).
Assume $\mathcal{A}$ is abelian and has enough injectives.
See Homology, Definitions \ref{homology-definition-abelian-category}
and \ref{homology-definition-enough-injectives}.
Then for any given set of objects $\{A_s\}_{s\in S}$
of $\mathcal{A}$, there is an abelian subcategory
$\mathcal{A}' \subset \mathcal{A}$
with the following properties:
\begin{enumerate}
\item $\Ob(\mathcal{A}')$ is a set,
\item $\Ob(\mathcal{A}')$ contains $A_s$ for each $s \in S$,
\item $\mathcal{A}'$ has enough injectives, and
\item an object of $\mathcal{A}'$ is injective if and only if it
is an injective object of $\mathcal{A}$.
\end{enumerate}
\end{lemma}

\begin{proof}
Omitted.
\end{proof}









\begin{multicols}{2}[\section{Other chapters}]
\noindent
Preliminaries
\begin{enumerate}
\item \hyperref[introduction-section-phantom]{Introduction}
\item \hyperref[conventions-section-phantom]{Conventions}
\item \hyperref[sets-section-phantom]{Set Theory}
\item \hyperref[categories-section-phantom]{Categories}
\item \hyperref[topology-section-phantom]{Topology}
\item \hyperref[sheaves-section-phantom]{Sheaves on Spaces}
\item \hyperref[sites-section-phantom]{Sites and Sheaves}
\item \hyperref[stacks-section-phantom]{Stacks}
\item \hyperref[fields-section-phantom]{Fields}
\item \hyperref[algebra-section-phantom]{Commutative Algebra}
\item \hyperref[brauer-section-phantom]{Brauer Groups}
\item \hyperref[homology-section-phantom]{Homological Algebra}
\item \hyperref[derived-section-phantom]{Derived Categories}
\item \hyperref[simplicial-section-phantom]{Simplicial Methods}
\item \hyperref[more-algebra-section-phantom]{More on Algebra}
\item \hyperref[smoothing-section-phantom]{Smoothing Ring Maps}
\item \hyperref[modules-section-phantom]{Sheaves of Modules}
\item \hyperref[sites-modules-section-phantom]{Modules on Sites}
\item \hyperref[injectives-section-phantom]{Injectives}
\item \hyperref[cohomology-section-phantom]{Cohomology of Sheaves}
\item \hyperref[sites-cohomology-section-phantom]{Cohomology on Sites}
\item \hyperref[dga-section-phantom]{Differential Graded Algebra}
\item \hyperref[dpa-section-phantom]{Divided Power Algebra}
\item \hyperref[hypercovering-section-phantom]{Hypercoverings}
\end{enumerate}
Schemes
\begin{enumerate}
\setcounter{enumi}{24}
\item \hyperref[schemes-section-phantom]{Schemes}
\item \hyperref[constructions-section-phantom]{Constructions of Schemes}
\item \hyperref[properties-section-phantom]{Properties of Schemes}
\item \hyperref[morphisms-section-phantom]{Morphisms of Schemes}
\item \hyperref[coherent-section-phantom]{Cohomology of Schemes}
\item \hyperref[divisors-section-phantom]{Divisors}
\item \hyperref[limits-section-phantom]{Limits of Schemes}
\item \hyperref[varieties-section-phantom]{Varieties}
\item \hyperref[topologies-section-phantom]{Topologies on Schemes}
\item \hyperref[descent-section-phantom]{Descent}
\item \hyperref[perfect-section-phantom]{Derived Categories of Schemes}
\item \hyperref[more-morphisms-section-phantom]{More on Morphisms}
\item \hyperref[flat-section-phantom]{More on Flatness}
\item \hyperref[groupoids-section-phantom]{Groupoid Schemes}
\item \hyperref[more-groupoids-section-phantom]{More on Groupoid Schemes}
\item \hyperref[etale-section-phantom]{\'Etale Morphisms of Schemes}
\end{enumerate}
Topics in Scheme Theory
\begin{enumerate}
\setcounter{enumi}{40}
\item \hyperref[chow-section-phantom]{Chow Homology}
\item \hyperref[intersection-section-phantom]{Intersection Theory}
\item \hyperref[weil-section-phantom]{Weil Cohomology Theories}
\item \hyperref[pic-section-phantom]{Picard Schemes of Curves}
\item \hyperref[adequate-section-phantom]{Adequate Modules}
\item \hyperref[dualizing-section-phantom]{Dualizing Complexes}
\item \hyperref[duality-section-phantom]{Duality for Schemes}
\item \hyperref[discriminant-section-phantom]{Discriminants and Differents}
\item \hyperref[local-cohomology-section-phantom]{Local Cohomology}
\item \hyperref[algebraization-section-phantom]{Algebraic and Formal Geometry}
\item \hyperref[curves-section-phantom]{Algebraic Curves}
\item \hyperref[resolve-section-phantom]{Resolution of Surfaces}
\item \hyperref[models-section-phantom]{Semistable Reduction}
\item \hyperref[pione-section-phantom]{Fundamental Groups of Schemes}
\item \hyperref[etale-cohomology-section-phantom]{\'Etale Cohomology}
\item \hyperref[crystalline-section-phantom]{Crystalline Cohomology}
\item \hyperref[proetale-section-phantom]{Pro-\'etale Cohomology}
\item \hyperref[more-etale-section-phantom]{More \'Etale Cohomology}
\item \hyperref[trace-section-phantom]{The Trace Formula}
\end{enumerate}
Algebraic Spaces
\begin{enumerate}
\setcounter{enumi}{59}
\item \hyperref[spaces-section-phantom]{Algebraic Spaces}
\item \hyperref[spaces-properties-section-phantom]{Properties of Algebraic Spaces}
\item \hyperref[spaces-morphisms-section-phantom]{Morphisms of Algebraic Spaces}
\item \hyperref[decent-spaces-section-phantom]{Decent Algebraic Spaces}
\item \hyperref[spaces-cohomology-section-phantom]{Cohomology of Algebraic Spaces}
\item \hyperref[spaces-limits-section-phantom]{Limits of Algebraic Spaces}
\item \hyperref[spaces-divisors-section-phantom]{Divisors on Algebraic Spaces}
\item \hyperref[spaces-over-fields-section-phantom]{Algebraic Spaces over Fields}
\item \hyperref[spaces-topologies-section-phantom]{Topologies on Algebraic Spaces}
\item \hyperref[spaces-descent-section-phantom]{Descent and Algebraic Spaces}
\item \hyperref[spaces-perfect-section-phantom]{Derived Categories of Spaces}
\item \hyperref[spaces-more-morphisms-section-phantom]{More on Morphisms of Spaces}
\item \hyperref[spaces-flat-section-phantom]{Flatness on Algebraic Spaces}
\item \hyperref[spaces-groupoids-section-phantom]{Groupoids in Algebraic Spaces}
\item \hyperref[spaces-more-groupoids-section-phantom]{More on Groupoids in Spaces}
\item \hyperref[bootstrap-section-phantom]{Bootstrap}
\item \hyperref[spaces-pushouts-section-phantom]{Pushouts of Algebraic Spaces}
\end{enumerate}
Topics in Geometry
\begin{enumerate}
\setcounter{enumi}{76}
\item \hyperref[spaces-chow-section-phantom]{Chow Groups of Spaces}
\item \hyperref[groupoids-quotients-section-phantom]{Quotients of Groupoids}
\item \hyperref[spaces-more-cohomology-section-phantom]{More on Cohomology of Spaces}
\item \hyperref[spaces-simplicial-section-phantom]{Simplicial Spaces}
\item \hyperref[spaces-duality-section-phantom]{Duality for Spaces}
\item \hyperref[formal-spaces-section-phantom]{Formal Algebraic Spaces}
\item \hyperref[restricted-section-phantom]{Restricted Power Series}
\item \hyperref[spaces-resolve-section-phantom]{Resolution of Surfaces Revisited}
\end{enumerate}
Deformation Theory
\begin{enumerate}
\setcounter{enumi}{84}
\item \hyperref[formal-defos-section-phantom]{Formal Deformation Theory}
\item \hyperref[defos-section-phantom]{Deformation Theory}
\item \hyperref[cotangent-section-phantom]{The Cotangent Complex}
\item \hyperref[examples-defos-section-phantom]{Deformation Problems}
\end{enumerate}
Algebraic Stacks
\begin{enumerate}
\setcounter{enumi}{88}
\item \hyperref[algebraic-section-phantom]{Algebraic Stacks}
\item \hyperref[examples-stacks-section-phantom]{Examples of Stacks}
\item \hyperref[stacks-sheaves-section-phantom]{Sheaves on Algebraic Stacks}
\item \hyperref[criteria-section-phantom]{Criteria for Representability}
\item \hyperref[artin-section-phantom]{Artin's Axioms}
\item \hyperref[quot-section-phantom]{Quot and Hilbert Spaces}
\item \hyperref[stacks-properties-section-phantom]{Properties of Algebraic Stacks}
\item \hyperref[stacks-morphisms-section-phantom]{Morphisms of Algebraic Stacks}
\item \hyperref[stacks-limits-section-phantom]{Limits of Algebraic Stacks}
\item \hyperref[stacks-cohomology-section-phantom]{Cohomology of Algebraic Stacks}
\item \hyperref[stacks-perfect-section-phantom]{Derived Categories of Stacks}
\item \hyperref[stacks-introduction-section-phantom]{Introducing Algebraic Stacks}
\item \hyperref[stacks-more-morphisms-section-phantom]{More on Morphisms of Stacks}
\item \hyperref[stacks-geometry-section-phantom]{The Geometry of Stacks}
\end{enumerate}
Topics in Moduli Theory
\begin{enumerate}
\setcounter{enumi}{102}
\item \hyperref[moduli-section-phantom]{Moduli Stacks}
\item \hyperref[moduli-curves-section-phantom]{Moduli of Curves}
\end{enumerate}
Miscellany
\begin{enumerate}
\setcounter{enumi}{104}
\item \hyperref[examples-section-phantom]{Examples}
\item \hyperref[exercises-section-phantom]{Exercises}
\item \hyperref[guide-section-phantom]{Guide to Literature}
\item \hyperref[desirables-section-phantom]{Desirables}
\item \hyperref[coding-section-phantom]{Coding Style}
\item \hyperref[obsolete-section-phantom]{Obsolete}
\item \hyperref[fdl-section-phantom]{GNU Free Documentation License}
\item \hyperref[index-section-phantom]{Auto Generated Index}
\end{enumerate}
\end{multicols}


\bibliography{my}
\bibliographystyle{amsalpha}

\end{document}
