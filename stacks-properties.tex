\IfFileExists{stacks-project.cls}{%
\documentclass{stacks-project}
}{%
\documentclass{amsart}
}

% The following AMS packages are automatically loaded with
% the amsart documentclass:
%\usepackage{amsmath}
%\usepackage{amssymb}
%\usepackage{amsthm}

% For dealing with references we use the comment environment
\usepackage{verbatim}
\newenvironment{reference}{\comment}{\endcomment}
%\newenvironment{reference}{}{}
\newenvironment{slogan}{\comment}{\endcomment}
\newenvironment{history}{\comment}{\endcomment}

% For commutative diagrams you can use
% \usepackage{amscd}
\usepackage[all]{xy}

% We use 2cell for 2-commutative diagrams.
\xyoption{2cell}
\UseAllTwocells

% To put source file link in headers.
% Change "template.tex" to "this_filename.tex"
% \usepackage{fancyhdr}
% \pagestyle{fancy}
% \lhead{}
% \chead{}
% \rhead{Source file: \url{template.tex}}
% \lfoot{}
% \cfoot{\thepage}
% \rfoot{}
% \renewcommand{\headrulewidth}{0pt}
% \renewcommand{\footrulewidth}{0pt}
% \renewcommand{\headheight}{12pt}

\usepackage{multicol}

% For cross-file-references
\usepackage{xr-hyper}

% Package for hypertext links:
\usepackage{hyperref}

% For any local file, say "hello.tex" you want to link to please
% use \externaldocument[hello-]{hello}
\externaldocument[introduction-]{introduction}
\externaldocument[conventions-]{conventions}
\externaldocument[sets-]{sets}
\externaldocument[categories-]{categories}
\externaldocument[topology-]{topology}
\externaldocument[sheaves-]{sheaves}
\externaldocument[sites-]{sites}
\externaldocument[stacks-]{stacks}
\externaldocument[fields-]{fields}
\externaldocument[algebra-]{algebra}
\externaldocument[brauer-]{brauer}
\externaldocument[homology-]{homology}
\externaldocument[derived-]{derived}
\externaldocument[simplicial-]{simplicial}
\externaldocument[more-algebra-]{more-algebra}
\externaldocument[smoothing-]{smoothing}
\externaldocument[modules-]{modules}
\externaldocument[sites-modules-]{sites-modules}
\externaldocument[injectives-]{injectives}
\externaldocument[cohomology-]{cohomology}
\externaldocument[sites-cohomology-]{sites-cohomology}
\externaldocument[dga-]{dga}
\externaldocument[dpa-]{dpa}
\externaldocument[hypercovering-]{hypercovering}
\externaldocument[schemes-]{schemes}
\externaldocument[constructions-]{constructions}
\externaldocument[properties-]{properties}
\externaldocument[morphisms-]{morphisms}
\externaldocument[coherent-]{coherent}
\externaldocument[divisors-]{divisors}
\externaldocument[limits-]{limits}
\externaldocument[varieties-]{varieties}
\externaldocument[topologies-]{topologies}
\externaldocument[descent-]{descent}
\externaldocument[perfect-]{perfect}
\externaldocument[more-morphisms-]{more-morphisms}
\externaldocument[flat-]{flat}
\externaldocument[groupoids-]{groupoids}
\externaldocument[more-groupoids-]{more-groupoids}
\externaldocument[etale-]{etale}
\externaldocument[chow-]{chow}
\externaldocument[intersection-]{intersection}
\externaldocument[pic-]{pic}
\externaldocument[adequate-]{adequate}
\externaldocument[dualizing-]{dualizing}
\externaldocument[duality-]{duality}
\externaldocument[discriminant-]{discriminant}
\externaldocument[local-cohomology-]{local-cohomology}
\externaldocument[curves-]{curves}
\externaldocument[resolve-]{resolve}
\externaldocument[models-]{models}
\externaldocument[pione-]{pione}
\externaldocument[etale-cohomology-]{etale-cohomology}
\externaldocument[proetale-]{proetale}
\externaldocument[crystalline-]{crystalline}
\externaldocument[spaces-]{spaces}
\externaldocument[spaces-properties-]{spaces-properties}
\externaldocument[spaces-morphisms-]{spaces-morphisms}
\externaldocument[decent-spaces-]{decent-spaces}
\externaldocument[spaces-cohomology-]{spaces-cohomology}
\externaldocument[spaces-limits-]{spaces-limits}
\externaldocument[spaces-divisors-]{spaces-divisors}
\externaldocument[spaces-over-fields-]{spaces-over-fields}
\externaldocument[spaces-topologies-]{spaces-topologies}
\externaldocument[spaces-descent-]{spaces-descent}
\externaldocument[spaces-perfect-]{spaces-perfect}
\externaldocument[spaces-more-morphisms-]{spaces-more-morphisms}
\externaldocument[spaces-flat-]{spaces-flat}
\externaldocument[spaces-groupoids-]{spaces-groupoids}
\externaldocument[spaces-more-groupoids-]{spaces-more-groupoids}
\externaldocument[bootstrap-]{bootstrap}
\externaldocument[spaces-pushouts-]{spaces-pushouts}
\externaldocument[groupoids-quotients-]{groupoids-quotients}
\externaldocument[spaces-more-cohomology-]{spaces-more-cohomology}
\externaldocument[spaces-simplicial-]{spaces-simplicial}
\externaldocument[formal-spaces-]{formal-spaces}
\externaldocument[restricted-]{restricted}
\externaldocument[spaces-resolve-]{spaces-resolve}
\externaldocument[formal-defos-]{formal-defos}
\externaldocument[defos-]{defos}
\externaldocument[cotangent-]{cotangent}
\externaldocument[examples-defos-]{examples-defos}
\externaldocument[algebraic-]{algebraic}
\externaldocument[examples-stacks-]{examples-stacks}
\externaldocument[stacks-sheaves-]{stacks-sheaves}
\externaldocument[criteria-]{criteria}
\externaldocument[artin-]{artin}
\externaldocument[quot-]{quot}
\externaldocument[stacks-properties-]{stacks-properties}
\externaldocument[stacks-morphisms-]{stacks-morphisms}
\externaldocument[stacks-limits-]{stacks-limits}
\externaldocument[stacks-cohomology-]{stacks-cohomology}
\externaldocument[stacks-perfect-]{stacks-perfect}
\externaldocument[stacks-introduction-]{stacks-introduction}
\externaldocument[stacks-more-morphisms-]{stacks-more-morphisms}
\externaldocument[stacks-geometry-]{stacks-geometry}
\externaldocument[moduli-]{moduli}
\externaldocument[moduli-curves-]{moduli-curves}
\externaldocument[examples-]{examples}
\externaldocument[exercises-]{exercises}
\externaldocument[guide-]{guide}
\externaldocument[desirables-]{desirables}
\externaldocument[coding-]{coding}
\externaldocument[obsolete-]{obsolete}
\externaldocument[fdl-]{fdl}
\externaldocument[index-]{index}

% Theorem environments.
%
\theoremstyle{plain}
\newtheorem{theorem}[subsection]{Theorem}
\newtheorem{proposition}[subsection]{Proposition}
\newtheorem{lemma}[subsection]{Lemma}

\theoremstyle{definition}
\newtheorem{definition}[subsection]{Definition}
\newtheorem{example}[subsection]{Example}
\newtheorem{exercise}[subsection]{Exercise}
\newtheorem{situation}[subsection]{Situation}

\theoremstyle{remark}
\newtheorem{remark}[subsection]{Remark}
\newtheorem{remarks}[subsection]{Remarks}

\numberwithin{equation}{subsection}

% Macros
%
\def\lim{\mathop{\rm lim}\nolimits}
\def\colim{\mathop{\rm colim}\nolimits}
\def\Spec{\mathop{\rm Spec}}
\def\Hom{\mathop{\rm Hom}\nolimits}
\def\Ext{\mathop{\rm Ext}\nolimits}
\def\SheafHom{\mathop{\mathcal{H}\!{\it om}}\nolimits}
\def\SheafExt{\mathop{\mathcal{E}\!{\it xt}}\nolimits}
\def\Sch{\textit{Sch}}
\def\Mor{\mathop{\rm Mor}\nolimits}
\def\Ob{\mathop{\rm Ob}\nolimits}
\def\Sh{\mathop{\textit{Sh}}\nolimits}
\def\NL{\mathop{N\!L}\nolimits}
\def\proetale{{pro\text{-}\acute{e}tale}}
\def\etale{{\acute{e}tale}}
\def\QCoh{\textit{QCoh}}
\def\Ker{\mathop{\rm Ker}}
\def\Im{\mathop{\rm Im}}
\def\Coker{\mathop{\rm Coker}}
\def\Coim{\mathop{\rm Coim}}

%
% Macros for moduli stacks/spaces
%
\def\QCohstack{\mathcal{QC}\!{\it oh}}
\def\Cohstack{\mathcal{C}\!{\it oh}}
\def\Spacesstack{\mathcal{S}\!{\it paces}}
\def\Quotfunctor{{\rm Quot}}
\def\Hilbfunctor{{\rm Hilb}}
\def\Curvesstack{\mathcal{C}\!{\it urves}}
\def\Polarizedstack{\mathcal{P}\!{\it olarized}}
\def\Complexesstack{\mathcal{C}\!{\it omplexes}}
% \Pic is the operator that assigns to X its picard group, usage \Pic(X)
% \Picardstack_{X/B} denotes the Picard stack of X over B
% \Picardfunctor_{X/B} denotes the Picard functor of X over B
\def\Pic{\mathop{\rm Pic}\nolimits}
\def\Picardstack{\mathcal{P}\!{\it ic}}
\def\Picardfunctor{{\rm Pic}}
\def\Deformationcategory{\mathcal{D}\!{\it ef}}


% OK, start here.
%
\begin{document}

\title{Properties of Algebraic Stacks}


\maketitle

\phantomsection
\label{section-phantom}

\tableofcontents

\section{Introduction}
\label{section-introduction}

\noindent
Please see
Algebraic Stacks, Section \ref{algebraic-section-introduction}
for a brief introduction to algebraic stacks, and please read
some of that chapter for our foundations of algebraic stacks.
The intent is that in that chapter we are careful to distinguish
between schemes, algebraic spaces, algebraic stacks, and starting
with this chapter we employ the customary abuse of language where
all of these concepts are used interchangeably.

\medskip\noindent
The goal of this chapter is to introduce some basic notions and
properties of algebraic stacks. A fundamental
reference for the case of quasi-separated algebraic stacks with representable
diagonal is \cite{LM-B}.



\section{Conventions and abuse of language}
\label{section-conventions}

\noindent
We choose a big fppf site $\textit{Sch}_{fppf}$.
All schemes are contained in $\textit{Sch}_{fppf}$.
And all rings $A$ considered have the property that
$\text{Spec}(A)$ is (isomorphic) to an object of this big site.

\medskip\noindent
We also fix a base scheme $S$, by the conventions above an element
of $\textit{Sch}_{fppf}$. The reader who is only interested
in the absolute case can take $S = \text{Spec}(\mathbf{Z})$.

\medskip\noindent
Here are our conventions regarding algebraic stacks:
\begin{enumerate}
\item When we say {\it algebraic stack} we will mean an algebraic
stacks over $S$, i.e., a category fibred in groupoids
$p : \mathcal{X} \to (\textit{Sch}/S)_{fppf}$
which satisfies the conditions of
Algebraic Stacks, Definition \ref{algebraic-definition-algebraic-stack}.
\item We will say $f : \mathcal{X} \to \mathcal{Y}$ is a
{\it morphism of algebraic stacks} to indicate a $1$-morphism
of algebraic stacks over $S$, i.e., a $1$-morphism of categories fibred
in groupoids over $(\textit{Sch}/S)_{fppf}$, see
Algebraic Stacks,
Definition \ref{algebraic-definition-morphism-algebraic-stacks}.
\item A {\it $2$-morphism} $\alpha : f \to g$ will
indicate a $2$-morphism in the $2$-category of algebraic stacks over
$S$, see
Algebraic Stacks,
Definition \ref{algebraic-definition-morphism-algebraic-stacks}.
\item Given morphisms $\mathcal{X} \to \mathcal{Z}$
and $\mathcal{Y} \to \mathcal{Z}$ of algebraic stacks
we abusively call the $2$-fibre product
$\mathcal{X} \times_{\mathcal{Z}} \mathcal{Y}$ the {\it fibre product}.
\item We will write $\mathcal{X} \times_S \mathcal{Y}$ for the
product of the algebraic stacks $\mathcal{X}$, $\mathcal{Y}$.
\item We will often abuse notation and say two algebraic stacks
$\mathcal{X}$ and $\mathcal{Y}$ are {\it isomorphic} if they are
equivalent in this $2$-category.
\end{enumerate}

\medskip\noindent
Here are our conventions regarding algebraic spaces.
\begin{enumerate}
\item If we say $X$ is an {\it algebraic space} then we mean that
$X$ is an algebraic space over $S$, i.e., $X$ is a presheaf on
$(\textit{Sch}/S)_{fppf}$ which satisfies the conditions of
Spaces, Definition \ref{spaces-definition-algebraic-space}.
\item A {\it morphism of algebraic spaces} $f :X \to Y$ is a morphism
of algebraic spaces over $S$ as defined in
Spaces, Definition \ref{spaces-definition-morphism-algebraic-spaces}.
\item We will {\bf not} distinguish between an algebraic space $X$
and the algebraic stack $\mathcal{S}_X \to (\textit{Sch}/S)_{fppf}$
it gives rise to, see
Algebraic Stacks, Lemma \ref{algebraic-lemma-representable-algebraic}.
\item In particular, a {\it morphism} $f : X \to \mathcal{Y}$ from $X$
to an algebraic stack $\mathcal{Y}$ means a morphism
$f : \mathcal{S}_X \to \mathcal{Y}$ of algebraic stacks.
Similarly for morphisms $\mathcal{Y} \to X$.
\item Moreover, given an algebraic stack $\mathcal{X}$ we say
{\it $\mathcal{X}$ is an algebraic space} to indicate that $\mathcal{X}$
is representable by an algebraic space, see
Algebraic Stacks,
Definition \ref{algebraic-definition-representable-by-algebraic-space}.
\item We will use the following notational convention: If we
indicate an algebraic stack by a roman capital
(such as $X, Y, Z, A, B, \ldots$) then it will be the case that
its inertia stack is trivial, and hence it is an algebraic space, see
Algebraic Stacks,
Proposition \ref{algebraic-proposition-algebraic-stack-no-automorphisms}.
\end{enumerate}

\medskip\noindent
Here are our conventions regarding schemes.
\begin{enumerate}
\item If we say $X$ is a {\it scheme} then we mean that
$X$ is a scheme over $S$, i.e., $X$ is an object of $(\textit{Sch}/S)_{fppf}$.
\item By a {\it morphism of schemes} we mean a morphism of schemes over $S$.
\item We will {\bf not} distinguish between a scheme $X$ and the
algebraic stack $\mathcal{S}_X \to (\textit{Sch}/S)_{fppf}$ it gives rise
to, see
Algebraic Stacks, Lemma \ref{algebraic-lemma-representable-algebraic}.
\item In particular, a {\it morphism} $f : X \to \mathcal{Y}$ from
a scheme $X$ to an algebraic stack $\mathcal{Y}$ means a morphism
$f : \mathcal{S}_X \to \mathcal{Y}$ of algebraic stacks.
Similarly for morphisms $\mathcal{Y} \to X$.
\item Moreover, given an algebraic stack $\mathcal{X}$ we say
{\it $\mathcal{X}$ is a scheme} to indicate that $\mathcal{X}$
is representable, see
Algebraic Stacks, Section \ref{algebraic-section-representable}.
\end{enumerate}

\medskip\noindent
Here are our conventions regarding morphism of algebraic stacks:
\begin{enumerate}
\item A morphism $f : \mathcal{X} \to \mathcal{Y}$
of algebraic stacks is {\it representable}, or
{\it representable by schemes} if for every scheme
$T$ and morphism $T \to \mathcal{Y}$ the fibre product
$T \times_{\mathcal{Y}} \mathcal{X}$ is a scheme.
See
Algebraic Stacks, Section \ref{algebraic-section-representable-morphism}.
\item A morphism $f : \mathcal{X} \to \mathcal{Y}$
of algebraic stacks is {\it representable by algebraic spaces}
if for every scheme $T$ and morphism $T \to \mathcal{Y}$ the fibre product
$T \times_{\mathcal{Y}} \mathcal{X}$ is an algebraic space.
See Algebraic Stacks,
Definition \ref{algebraic-definition-representable-by-algebraic-spaces}.
In this case $Z \times_{\mathcal{Y}} \mathcal{X}$ is an algebraic
space whenever $Z \to \mathcal{Y}$ is a morphism whose source is
an algebraic space, see
Algebraic Stacks,
Lemma \ref{algebraic-lemma-base-change-by-space-representable-by-space}.
\end{enumerate}
Note that every morphism $X \to \mathcal{Y}$ from an algebraic space
to an algebraic stack is representable by algebraic spaces, see
Algebraic Stacks, Lemma \ref{algebraic-lemma-representable-diagonal}.
We will use this basic result without further mention.




\section{Properties of morphisms representable by algebraic spaces}
\label{section-properties-morphisms}

\noindent
We will study properties of (arbitrary) morphisms of algebraic stacks in its
own chapter. For morphisms representable by algebraic spaces we know what
it means to be surjective, smooth, or \'etale, etc. This applies in particular
to morphisms $X \to \mathcal{Y}$ from algebraic spaces to algebraic stacks.
In this section, we recall how this works, we list the properties to which
this applies, and we prove a few easy lemmas.

\medskip\noindent
Our first lemma says a morphism is representable by algebraic spaces
if it is so after a base change by a flat,
locally finitely presented, surjective morphism.

\begin{lemma}
\label{lemma-check-representable-covering}
Let $f : \mathcal{X} \to \mathcal{Y}$ be a morphism of algebraic stacks.
Let $W$ be an algebraic space and let $W \to \mathcal{Y}$ be surjective,
locally of finite presentation, and flat. The following are equivalent
\begin{enumerate}
\item $f$ is representable by algebraic spaces, and
\item $W \times_{\mathcal{Y}} \mathcal{X}$ is an algebraic space.
\end{enumerate}
\end{lemma}

\begin{proof}
The implication (1) $\Rightarrow$ (2) is
Algebraic Stacks,
Lemma \ref{algebraic-lemma-base-change-by-space-representable-by-space}.
Conversely, let $W \to \mathcal{Y}$ be as in (2). To prove (1) it
suffices to show that $f$ is faithful on fibre categories, see
Algebraic Stacks,
Lemma \ref{algebraic-lemma-characterize-representable-by-algebraic-spaces}.
Assumption (2) implies in particular that
$W \times_{\mathcal{Y}} \mathcal{X} \to W$ is faithful.
Hence the faithfulness of $f$ follows from
Stacks, Lemma \ref{stacks-lemma-faithful-descent}.
\end{proof}

\medskip\noindent
Let $P$ be a property of morphisms of algebraic spaces which is
fppf local on the target and preserved by arbitrary base change.
Let $f : \mathcal{X} \to \mathcal{Y}$ be a morphism of algebraic stacks
representable by algebraic spaces. Then we say
{\it $f$ has property $P$} if and only if for every scheme $T$
and morphism $T \to \mathcal{Y}$ the morphism of algebraic spaces
$T \times_{\mathcal{Y}} \mathcal{X} \to T$ has property $P$, see
Algebraic Stacks,
Definition \ref{algebraic-definition-relative-representable-property}.

\medskip\noindent
It turns out that if $f : \mathcal{X} \to \mathcal{Y}$ is representable
by algebraic spaces and has property $P$, then for any morphism of algebraic
stacks $\mathcal{Y}' \to \mathcal{Y}$ the base change
$\mathcal{Y}' \times_{\mathcal{Y}} \mathcal{X} \to \mathcal{Y}'$
has property $P$, see
Algebraic Stacks,
Lemmas \ref{algebraic-lemma-base-change-representable-by-spaces} and
\ref{algebraic-lemma-base-change-representable-transformations-property}.
If the property $P$ is preserved under compositions, then this holds
also in the setting of morphisms of algebraic stacks representable by
algebraic spaces, see
Algebraic Stacks,
Lemmas \ref{algebraic-lemma-composition-representable-by-spaces} and
\ref{algebraic-lemma-composition-representable-transformations-property}.
Moreover, in this case products
$\mathcal{X}_1 \times \mathcal{X}_2 \to \mathcal{Y}_1 \times \mathcal{Y}_2$
of morphisms representable by algebraic spaces having property $\mathcal{P}$
have property $\mathcal{P}$, see
Algebraic Stacks, Lemma
\ref{algebraic-lemma-product-representable-transformations-property}.

\medskip\noindent
Finally, if we have two properties $P, P'$ of morphisms of algebraic spaces
which are fppf local on the target and preserved by arbitrary base change
and if $P(f) \Rightarrow P'(f)$ for every morphism $f$, then the same
implication holds for the corresponding property of morphisms of algebraic
stacks representable by algebraic spaces, see
Algebraic Stacks, Lemma
\ref{algebraic-lemma-representable-transformations-property-implication}.
We will use this {\it without further mention} in the following and in the
following chapters.

\medskip\noindent
The discussion above applies to each of the following properties of
morphisms of algebraic spaces
\begin{enumerate}
\item quasi-compact, see
Morphisms of Spaces,
Lemma \ref{spaces-morphisms-lemma-base-change-quasi-compact}
and
Descent for Spaces,
Lemma \ref{spaces-descent-lemma-descending-property-quasi-compact},
\item quasi-separated, see
Morphisms of Spaces,
Lemma \ref{spaces-morphisms-lemma-base-change-separated}
and
Descent for Spaces,
Lemma \ref{spaces-descent-lemma-descending-property-quasi-separated},
\item universally closed, see
Morphisms of Spaces,
Lemma \ref{spaces-morphisms-lemma-base-change-universally-closed}
and
Descent for Spaces,
Lemma \ref{spaces-descent-lemma-descending-property-universally-closed},
\item universally open, see
Morphisms of Spaces,
Lemma \ref{spaces-morphisms-lemma-base-change-universally-open}
and
Descent for Spaces,
Lemma \ref{spaces-descent-lemma-descending-property-universally-open},
\item surjective, see
Morphisms of Spaces,
Lemma \ref{spaces-morphisms-lemma-base-change-surjective}
and
Descent for Spaces,
Lemma \ref{spaces-descent-lemma-descending-property-surjective},
\item universally injective, see
Morphisms of Spaces,
Lemma \ref{spaces-morphisms-lemma-base-change-universally-injective}
and
Descent for Spaces,
Lemma \ref{spaces-descent-lemma-descending-property-universally-injective},
\item locally of finite type, see
Morphisms of Spaces,
Lemma \ref{spaces-morphisms-lemma-base-change-finite-type}
and
Descent for Spaces,
Lemma \ref{spaces-descent-lemma-descending-property-locally-finite-type},
\item locally of finite presentation, see
Morphisms of Spaces,
Lemma \ref{spaces-morphisms-lemma-base-change-finite-presentation}
and
Descent for Spaces, Lemma
\ref{spaces-descent-lemma-descending-property-locally-finite-presentation},
\item finite type, see
Morphisms of Spaces,
Lemma \ref{spaces-morphisms-lemma-base-change-finite-type}
and
Descent for Spaces,
Lemma \ref{spaces-descent-lemma-descending-property-finite-type},
\item finite presentation, see
Morphisms of Spaces,
Lemma \ref{spaces-morphisms-lemma-base-change-finite-presentation}
and
Descent for Spaces, Lemma
\ref{spaces-descent-lemma-descending-property-finite-presentation},
\item flat, see
Morphisms of Spaces,
Lemma \ref{spaces-morphisms-lemma-base-change-flat}
and
Descent for Spaces,
Lemma \ref{spaces-descent-lemma-descending-property-flat},
\item open immersion, see
Morphisms of Spaces,
Section \ref{spaces-morphisms-section-immersions}
and
Descent for Spaces,
Lemma \ref{spaces-descent-lemma-descending-property-open-immersion},
\item isomorphism, see
Descent for Spaces,
Lemma \ref{spaces-descent-lemma-descending-property-isomorphism},
\item affine, see
Morphisms of Spaces,
Lemma \ref{spaces-morphisms-lemma-base-change-affine}
and
Descent for Spaces,
Lemma \ref{spaces-descent-lemma-descending-property-affine},
\item closed immersion, see
Morphisms of Spaces, Section \ref{spaces-morphisms-section-immersions}
and
Descent for Spaces,
Lemma \ref{spaces-descent-lemma-descending-property-closed-immersion},
\item separated, see
Morphisms of Spaces,
Lemma \ref{spaces-morphisms-lemma-base-change-separated}
and
Descent for Spaces,
Lemma \ref{spaces-descent-lemma-descending-property-separated},
\item proper, see
Morphisms of Spaces,
Lemma \ref{spaces-morphisms-lemma-base-change-proper}
and
Descent for Spaces,
Lemma \ref{spaces-descent-lemma-descending-property-proper},
\item quasi-affine, see
Morphisms of Spaces,
Lemma \ref{spaces-morphisms-lemma-base-change-quasi-affine}
and
Descent for Spaces,
Lemma \ref{spaces-descent-lemma-descending-property-quasi-affine},
\item integral, see
Morphisms of Spaces,
Lemma \ref{spaces-morphisms-lemma-base-change-integral}
and
Descent for Spaces,
Lemma \ref{spaces-descent-lemma-descending-property-integral},
\item finite, see
Morphisms of Spaces,
Lemma \ref{spaces-morphisms-lemma-base-change-integral}
and
Descent for Spaces,
Lemma \ref{spaces-descent-lemma-descending-property-finite},
\item (locally) quasi-finite, see
Morphisms of Spaces,
Lemma \ref{spaces-morphisms-lemma-base-change-quasi-finite}
and
Descent for Spaces,
Lemma \ref{spaces-descent-lemma-descending-property-quasi-finite},
\item syntomic, see
Morphisms of Spaces,
Lemma \ref{spaces-morphisms-lemma-base-change-syntomic}
and
Descent for Spaces,
Lemma \ref{spaces-descent-lemma-descending-property-syntomic},
\item smooth, see
Morphisms of Spaces,
Lemma \ref{spaces-morphisms-lemma-base-change-smooth}
and
Descent for Spaces,
Lemma \ref{spaces-descent-lemma-descending-property-smooth},
\item unramified, see
Morphisms of Spaces,
Lemma \ref{spaces-morphisms-lemma-base-change-unramified}
and
Descent for Spaces,
Lemma \ref{spaces-descent-lemma-descending-property-unramified},
\item \'etale, see
Morphisms of Spaces,
Lemma \ref{spaces-morphisms-lemma-base-change-etale}
and
Descent for Spaces,
Lemma \ref{spaces-descent-lemma-descending-property-etale},
\item finite locally free, see
Morphisms of Spaces,
Lemma \ref{spaces-morphisms-lemma-base-change-finite-locally-free}
and
Descent for Spaces,
Lemma \ref{spaces-descent-lemma-descending-property-finite-locally-free},
\item monomorphism, see
Morphisms of Spaces,
Lemma \ref{spaces-morphisms-lemma-base-change-monomorphism}
and
Descent for Spaces,
Lemma \ref{spaces-descent-lemma-descending-property-monomorphism},
\item immersion, see
Morphisms of Spaces, Section \ref{spaces-morphisms-section-immersions}
and
Descent for Spaces,
Lemma \ref{spaces-descent-lemma-descending-fppf-property-immersion},
\item locally separated, see
Morphisms of Spaces,
Lemma \ref{spaces-morphisms-lemma-base-change-separated}
and
Descent for Spaces,
Lemma \ref{spaces-descent-lemma-descending-fppf-property-locally-separated},
\end{enumerate}

\begin{lemma}
\label{lemma-property-spaces-too}
Let $P$ be a property of morphisms of algebraic spaces as above.
Let $f : \mathcal{X} \to \mathcal{Y}$ be a morphism of algebraic stacks
representable by algebraic spaces. The following are equivalent:
\begin{enumerate}
\item $f$ has $P$,
\item for every algebraic space $Z$ and morphism $Z \to \mathcal{Y}$
the morphism $Z \times_{\mathcal{Y}} \mathcal{X} \to Z$ has $P$.
\end{enumerate}
\end{lemma}

\begin{proof}
The implication (2) $\Rightarrow$ (1) is immediate. Assume (1).
Let $Z \to \mathcal{Y}$ be as in (2). Choose a scheme $U$ and a
surjective \'etale morphism $U \to Z$. By assumption the morphism
$U \times_{\mathcal{Y}} \mathcal{X} \to U$ has $P$. But the diagram
$$
\xymatrix{
U \times_{\mathcal{Y}} \mathcal{X} \ar[d] \ar[r] &
Z \times_{\mathcal{Y}} \mathcal{X} \ar[d] \\
U \ar[r] & Z
}
$$
is cartesian, hence the right vertical arrow has $P$ as
$\{U \to Z\}$ is an fppf covering.
\end{proof}

\noindent
The following lemma tells us it suffices to check $P$
after a base change by a surjective, flat, locally finitely presented
morphism.

\begin{lemma}
\label{lemma-check-property-covering}
Let $P$ be a property of morphisms of algebraic spaces as above.
Let $f : \mathcal{X} \to \mathcal{Y}$ be a morphism of algebraic stacks
representable by algebraic spaces.
Let $W$ be an algebraic space and let $W \to \mathcal{Y}$ be surjective,
locally of finite presentation, and flat.
Set $V = W \times_{\mathcal{Y}} \mathcal{X}$. Then
$$
(f\text{ has }P) \Leftrightarrow (\text{the projection }V \to W\text{ has }P).
$$
\end{lemma}

\begin{proof}
The implication from left to right follows from
Lemma \ref{lemma-property-spaces-too}.
Assume $V \to W$ has $P$. Let $T$ be a scheme, and let
$T \to \mathcal{Y}$ be a morphism. Consider the commutative diagram
$$
\xymatrix{
T \times_{\mathcal{Y}} \mathcal{X} \ar[d] &
T \times_{\mathcal{Y}} W \ar[d] \ar[l] \ar[r] &
W \ar[d] \\
T & T \times_{\mathcal{Y}} V \ar[l] \ar[r] & V
}
$$
of algebraic spaces. The squares are cartesian.
The bottom left morphism is a surjective, flat morphism which is locally of
finite presentation, hence $\{T \times_{\mathcal{Y}} V \to T\}$ is an
fppf covering. Hence the fact that the right vertical arrow has property
$P$ implies that the left vertical arrow has property $P$.
\end{proof}

\begin{lemma}
\label{lemma-representable-in-terms-presentations}
Let $g : \mathcal{X}' \to \mathcal{X}$ be a morphism of algebraic stacks
which is representable by algebraic spaces. Let $[U/R] \to \mathcal{X}$
be a presentation. Set $U' = U \times_{\mathcal{X}} \mathcal{X}'$,
and $R' = R \times_{\mathcal{X}} \mathcal{X}'$.
Then there exists a groupoid in algebraic spaces of the form
$(U', R', s', t', c')$, a presentation $[U'/R'] \to \mathcal{X}'$,
and the diagram
$$
\xymatrix{
[U'/R'] \ar[d]_{[\text{pr}]} \ar[r] & \mathcal{X}' \ar[d]^g \\
[U/R] \ar[r] & \mathcal{X}
}
$$
is $2$-commutative where the morphism $[\text{pr}]$ comes from a
morphism of groupoids
$\text{pr} : (U', R', s', t', c') \to (U, R, s, t, c)$.
\end{lemma}

\begin{proof}
Since $U \to \mathcal{Y}$ is surjective and smooth, see
Algebraic Stacks,
Lemma \ref{algebraic-lemma-smooth-quotient-smooth-presentation}
the base change $U' \to \mathcal{X}'$ is also surjective and smooth.
Hence, by
Algebraic Stacks,
Lemma \ref{algebraic-lemma-stack-presentation}
it suffices to show that $R' = U' \times_{\mathcal{X}'} U'$ in order to
get a smooth groupoid $(U', R', s', t', c')$ and a presentation
$[U'/R'] \to \mathcal{X}'$.
Using that $R = V \times_{\mathcal{Y}} V$ (see
Groupoids in Spaces,
Lemma \ref{spaces-groupoids-lemma-quotient-stack-2-cartesian})
this follows from
$$
R' =
U \times_{\mathcal{X}} U \times_{\mathcal{X}} \mathcal{X}' =
(U \times_{\mathcal{X}} \mathcal{X}')
\times_{\mathcal{X}'}
(U \times_{\mathcal{X}} \mathcal{X}')
$$
see
Categories, Lemmas \ref{categories-lemma-associativity-2-fibre-product} and
\ref{categories-lemma-2-fibre-product-erase-factor}.
Clearly the projection morphisms $U' \to U$ and $R' \to R$ give the
desired morphism of groupoids
$\text{pr} : (U', R', s', t', c') \to (U, R, s, t, c)$.
Hence the morphism $[\text{pr}]$ of quotient stacks by
Groupoids in Spaces,
Lemma \ref{spaces-groupoids-lemma-quotient-stack-functorial}.

\medskip\noindent
We still have to show that the diagram $2$-commutes.
It is clear that the diagram
$$
\xymatrix{
U' \ar[d]_{\text{pr}_U} \ar[r]_{f'} & \mathcal{X}' \ar[d]^g \\
U \ar[r]^f & \mathcal{X}
}
$$
$2$-commutes where $\text{pr}_U : U' \to U$ is the projection.
There is a canonical $2$-arrow
$\tau : f \circ t \to f \circ s$ in $\text{Mor}(R, \mathcal{X})$
coming from $R = U \times_{\mathcal{X}} U$, $t = \text{pr}_0$, and
$s = \text{pr}_1$. Using the isomorphism
$R' \to U' \times_{\mathcal{X}'} U'$ we get similarly an isomorphism
$\tau' : f' \circ t' \to f' \circ s'$. Note that
$g \circ f' \circ t' = f \circ t \circ \text{pr}_R$ and
$g \circ f' \circ s' = f \circ s \circ \text{pr}_R$, where
$\text{pr}_R : R' \to R$ is the projection. Thus it makes sense to ask
if
\begin{equation}
\label{equation-verify}
\tau \star \text{id}_{\text{pr}_R} = \text{id}_g \star \tau'.
\end{equation}
Now we make two claims: (1) if Equation (\ref{equation-verify}) holds,
then the diagram $2$-commutes, and (2) Equation (\ref{equation-verify}) holds.
We omit the proof of both claims. Hints: part (1) follows from the
construction of $f = f_{can}$ and $f' = f'_{can}$ in
Algebraic Stacks, Lemma \ref{algebraic-lemma-map-space-into-stack}.
Part (2) follows by carefully working through the definitions.
\end{proof}

\begin{remark}
\label{remark-representable-over}
Let $\mathcal{Y}$ be an algebraic stack. Consider the following
$2$-category:
\begin{enumerate}
\item An object is a morphism $f : \mathcal{X} \to \mathcal{Y}$
which is representable by algebraic spaces,
\item a $1$-morphism
$(g, \beta) :
(f_1 : \mathcal{X}_1 \to \mathcal{Y})
\to
(f_2 : \mathcal{X}_2 \to \mathcal{Y})$
consists of a morphism $g : \mathcal{X}_1 \to \mathcal{X}_2$ and a
$2$-morphism $\beta : f_1 \to f_2 \circ g$, and
\item a $2$-morphism between
$(g, \beta), (g', \beta') :
(f_1 : \mathcal{X}_1 \to \mathcal{Y})
\to
(f_2 : \mathcal{X}_2 \to \mathcal{Y})$
is a $2$-morphism $\alpha : g \to g'$ such that
$(\text{id}_{f_2} \star \alpha) \circ \beta = \beta'$.
\end{enumerate}
Let us denote this $2$-category $\textit{Spaces}/\mathcal{Y}$ by
analogy with the notation of
Topologies on Spaces, Section \ref{spaces-topologies-section-procedure}.
Now we claim that in this $2$-category the morphism categories
$$
\text{Mor}_{\textit{Spaces}/\mathcal{Y}}(
(f_1 : \mathcal{X}_1 \to \mathcal{Y}),
(f_2 : \mathcal{X}_2 \to \mathcal{Y}))
$$
are all setoids. Namely, a $2$-morphism $\alpha$ is a rule which to each
object $x_1$ of $\mathcal{X}_1$ assigns an isomorphism
$\alpha_{x_1} : g(x_1) \longrightarrow g'(x_1)$
in the relevant fibre category of $\mathcal{X}_2$ such that the diagram
$$
\xymatrix{
& f_2(x_1) \ar[ld]_{\beta_{x_1}} \ar[rd]^{\beta'_{x_1}} \\
f_2(g(x_1)) \ar[rr]^{f_2(\alpha_{x_1})} & &
f_2(g'(x_1))
}
$$
commutes. But since $f_2$ is faithful (see
Algebraic Stacks,
Lemma \ref{algebraic-lemma-characterize-representable-by-algebraic-spaces})
this means that if $\alpha_{x_1}$ exists, then it is unique! In other words the
$2$-category $\textit{Spaces}/\mathcal{Y}$ is very close to
being a category. Namely, if we replace $1$-morphisms by isomorphism
classes of $1$-morphisms we obtain a category. We will often
perform this replacement without further mention.
\end{remark}




\section{Points of algebraic stacks}
\label{section-points}

\noindent
Let $\mathcal{X}$ be an algebraic stack. Let $K, L$ be two fields
and let $p : \text{Spec}(K) \to \mathcal{X}$ and
$q : \text{Spec}(L) \to \mathcal{X}$ be morphisms.
We say that $p$ and $q$ are {\it equivalent} if there exists a
field $\Omega$ and a $2$-commutative diagram
$$
\xymatrix{
\text{Spec}(\Omega) \ar[r] \ar[d] &
\text{Spec}(L) \ar[d]^q \\
\text{Spec}(K) \ar[r]^p &
\mathcal{X}.
}
$$

\begin{lemma}
\label{lemma-equivalence}
The notion above does indeed define an equivalence relation on
morphisms from spectra of fields into the algebraic stack $\mathcal{X}$.
\end{lemma}

\begin{proof}
It is clear that the relation is reflexive and symmetric.
Hence we have to prove that it is transitive. This comes down
to the following: Given a diagram
$$
\xymatrix{
\text{Spec}(\Omega) \ar[r]_b \ar[d]_a &
\text{Spec}(L) \ar[d]^q & \text{Spec}(\Omega') \ar[l]^{b'} \ar[d]^{a'} \\
\text{Spec}(K) \ar[r]^p &
\mathcal{X} &
\text{Spec}(K') \ar[l]_{p'}
}
$$
with both squares $2$-commutative we have to show that $p$ is equivalent to
$p'$. By the $2$-Yoneda lemma (see
Algebraic Stacks, Section \ref{algebraic-section-2-yoneda})
the morphisms $p$, $p'$, and $q$ are given by objects
$x$, $x'$, and $y$ in the fibre categories of $\mathcal{X}$ over
$\text{Spec}(K)$, $\text{Spec}(K')$, and $\text{Spec}(L)$. The
$2$-commutativity of the squares means that there are isomorphisms
$\alpha : a^*x \to b^*y$ and $\alpha' : (a')^*x' \to (b')^*y$
in the fibre categories
of $\mathcal{X}$ over $\text{Spec}(\Omega)$ and $\text{Spec}(\Omega')$.
Choose any field $\Omega''$ and embeddings
$\Omega \to \Omega''$ and $\Omega' \to \Omega''$ agreeing on $L$.
Then we can extend the diagram above to
$$
\xymatrix{
& \text{Spec}(\Omega'') \ar[ld]_c \ar[d]^{q'} \ar[rd]^{c'} \\
\text{Spec}(\Omega) \ar[r]_b \ar[d]_a &
\text{Spec}(L) \ar[d]^q & \text{Spec}(\Omega') \ar[l]^{b'} \ar[d]^{a'} \\
\text{Spec}(K) \ar[r]^p &
\mathcal{X} &
\text{Spec}(K') \ar[l]_{p'}
}
$$
with commutative triangles and
$$
(q')^*(\alpha')^{-1} \circ (q')^*\alpha :
(a \circ c)^*x
\longrightarrow
(a' \circ c')^*x'
$$
is an isomorphism in the fibre category over $\text{Spec}(\Omega'')$.
Hence $p$ is equivalent to $p'$ as desired.
\end{proof}

\begin{definition}
\label{definition-points}
Let $\mathcal{X}$ be an algebraic stack.
A {\it point} of $\mathcal{X}$ is an equivalence class of morphisms
from spectra of fields into $\mathcal{X}$.
The set of points of $\mathcal{X}$ is denoted $|\mathcal{X}|$.
\end{definition}

\noindent
This agrees with our definition of points of algebraic spaces, see
Properties of Spaces, Definition \ref{spaces-properties-definition-points}.
Moreover, for a scheme we recover the usual notion of points, see
Properties of Spaces, Lemma \ref{spaces-properties-lemma-scheme-points}.
If $f : \mathcal{X} \to \mathcal{Y}$ is a morphism of algebraic stacks
then there is an induced map $|f| : |\mathcal{X}| \to |\mathcal{Y}|$ which
maps a representative $x : \text{Spec}(K) \to \mathcal{X}$ to the
representative $f \circ x : \text{Spec}(K) \to \mathcal{Y}$. This is
well defined: namely $2$-isomorphic $1$-morphisms remain $2$-isomorphic after
pre- or post-composing by a $1$-morphism because you can horizontally
pre- or post-compose by the identity of the given $1$-morphism. This holds
in any (strict) $(2,1)$-category. If
$$
\xymatrix{
\mathcal{X} \ar[d] \ar[r] & \mathcal{Y} \ar[d] \\
\mathcal{W} \ar[r] & \mathcal{Z}
}
$$
is a $2$-commutative diagram of algebraic stacks, then the diagram
of sets
$$
\xymatrix{
|\mathcal{X}| \ar[d] \ar[r] & |\mathcal{Y}| \ar[d] \\
|\mathcal{W}| \ar[r] & |\mathcal{Z}|
}
$$
is commutative.
In particular, if $\mathcal{X} \to \mathcal{Y}$ is an equivalence
then $|\mathcal{X}| \to |\mathcal{Y}|$ is a bijection.

\begin{lemma}
\label{lemma-points-cartesian}
Let
$$
\xymatrix{
\mathcal{Z} \times_{\mathcal{Y}} \mathcal{X} \ar[r] \ar[d] &
\mathcal{X} \ar[d] \\
\mathcal{Z} \ar[r] & \mathcal{Y}
}
$$
be a fibre product of algebraic stacks. Then the map of sets
of points
$$
|\mathcal{Z} \times_{\mathcal{Y}} \mathcal{X}|
\longrightarrow
|\mathcal{Z}| \times_{|\mathcal{Y}|} |\mathcal{X}|
$$
is surjective.
\end{lemma}

\begin{proof}
Namely, suppose given fields $K$, $L$ and morphisms
$\text{Spec}(K) \to \mathcal{X}$, $\text{Spec}(L) \to \mathcal{Z}$,
then the assumption that they agree as elements of $|\mathcal{Y}|$ means that
there is a common extension $K \subset M$ and $L \subset M$
such that
$\text{Spec}(M) \to \text{Spec}(K) \to \mathcal{X} \to \mathcal{Y}$ and
$\text{Spec}(M) \to \text{Spec}(L) \to \mathcal{Z} \to \mathcal{Y}$
are $2$-isomorphic. And this is exactly the condition which says you get a
morphism $\text{Spec}(M) \to \mathcal{Z} \times_{\mathcal{Y}} \mathcal{X}$.
\end{proof}

\begin{lemma}
\label{lemma-characterize-surjective}
Let $f : \mathcal{X} \to \mathcal{Y}$ be a morphism of algebraic stacks
which is representable by algebraic spaces. The following are equivalent:
\begin{enumerate}
\item $|f| : |\mathcal{X}| \to |\mathcal{Y}|$ is surjective, and
\item $f$ is surjective.
\end{enumerate}
\end{lemma}

\begin{proof}
Assume (1). Let $T \to \mathcal{Y}$ be a morphism whose source is a scheme.
To prove (2) we have to show that the morphism of algebraic spaces
$T \times_{\mathcal{Y}} \mathcal{X} \to T$ is surjective. By
Morphisms of Spaces, Definition \ref{spaces-morphisms-definition-surjective}
this means we have to show that
$|T \times_{\mathcal{Y}} \mathcal{X}| \to |T|$ is surjective.
Applying
Lemma \ref{lemma-points-cartesian}
we see that this follows from (1).

\medskip\noindent
Conversely, assume (2). Let $y : \text{Spec}(K) \to \mathcal{Y}$ be a
morphism from the spectrum of a field into $\mathcal{Y}$. By assumption the
morphism
$\text{Spec}(K) \times_{y, \mathcal{Y}} \mathcal{X} \to \text{Spec}(K)$
of algebraic spaces is surjective. By
Morphisms of Spaces, Definition \ref{spaces-morphisms-definition-surjective}
this means there exists a field extension
$K \subset K'$ and a morphism
$\text{Spec}(K') \to \text{Spec}(K) \times_{y, \mathcal{Y}} \mathcal{X}$
such that the left square of the diagram
$$
\xymatrix{
\text{Spec}(K') \ar[r] \ar[d] &
\text{Spec}(K) \times_{y, \mathcal{Y}} \mathcal{X} \ar[d] \ar[r] &
\mathcal{X} \ar[d]
\\
\text{Spec}(K) \ar@{=}[r] &
\text{Spec}(K) \ar[r]^-y &
\mathcal{Y}
}
$$
is commutative. This shows that $|X| \to |\mathcal{Y}|$ is surjective.
\end{proof}

\noindent
Here is a lemma explaining how to compute the set of points in terms
of a presentation.

\begin{lemma}
\label{lemma-points-presentation}
Let $\mathcal{X}$ be an algebraic stack.
Let $\mathcal{X} = [U/R]$ be a presentation of $\mathcal{X}$, see
Algebraic Stacks, Definition \ref{algebraic-definition-presentation}.
Then the image of $|R| \to |U| \times |U|$ is an equivalence relation
and $|\mathcal{X}|$ is the quotient of $|U|$ by this equivalence relation.
\end{lemma}

\begin{proof}
The assumption means that we have a smooth groupoid $(U, R, s, t, c)$
in algebraic spaces, and an equivalence $f : [U/R] \to \mathcal{X}$.
We may assume $\mathcal{X} = [U/R]$.
The induced morphism $p : U \to \mathcal{X}$ is smooth and surjective, see
Algebraic Stacks,
Lemma \ref{algebraic-lemma-smooth-quotient-smooth-presentation}.
Hence $|U| \to |\mathcal{X}|$ is surjective by
Lemma \ref{lemma-characterize-surjective}.
Note that $R = U \times_{\mathcal{X}} U$, see
Groupoids in Spaces,
Lemma \ref{spaces-groupoids-lemma-quotient-stack-2-cartesian}.
Hence
Lemma \ref{lemma-points-cartesian}
implies the map
$$
|R| \longrightarrow |U| \times_{|\mathcal{X}|} |U|
$$
is surjective. Hence the image of $|R| \to |U| \times |U|$ is
exactly the set of pairs $(u_1, u_2) \in |U| \times |U|$
such that $u_1$ and $u_2$ have the same image in $|\mathcal{X}|$.
Combining these two statements we get the result of the lemma.
\end{proof}

\begin{remark}
\label{remark-more-general-presentation}
The result of
Lemma \ref{lemma-points-presentation}
can be generalized as follows.
Let $\mathcal{X}$ be an algebraic stack.
Let $U$ be an algebraic space and let $f : U \to \mathcal{X}$ be a surjective
morphism (which makes sense by
Section \ref{section-properties-morphisms}).
Let $R = U \times_{\mathcal{X}} U$, let $(U, R, s, t, c)$ be the groupoid
in algebraic spaces, and let $f_{can} : [U/R] \to \mathcal{X}$ be the
canonical morphism as constructed in
Algebraic Stacks, Lemma \ref{algebraic-lemma-map-space-into-stack}.
Then the image of $|R| \to |U| \times |U|$ is an equivalence relation
and $|\mathcal{X}| = |U|/|R|$. The proof of
Lemma \ref{lemma-points-presentation}
works without change. (Of course in general $[U/R]$ is not an algebraic
stack, and in general $f_{can}$ is not an isomorphism.)
\end{remark}

\begin{lemma}
\label{lemma-topology-points}
There exists a unique topology on the sets of points
of algebraic stacks with the following properties:
\begin{enumerate}
\item for every morphism of algebraic stacks $\mathcal{X} \to \mathcal{Y}$
the map $|\mathcal{X}| \to |\mathcal{Y}|$ is continuous, and
\item for every morphism $U \to \mathcal{X}$ which is flat and locally
of finite presentation with $U$ an algebraic space
the map of topological spaces $|U| \to |\mathcal{X}|$ is continuous and open.
\end{enumerate}
\end{lemma}

\begin{proof}
Choose a morphism $p : U \to \mathcal{X}$ which is
surjective, flat, and locally of finite presentation
with $U$ an algebraic space. Such exist by the definition of an algebraic
stack, as a smooth morphism is flat and locally of finite presentation
(see
Morphisms of Spaces,
Lemmas \ref{spaces-morphisms-lemma-smooth-locally-finite-presentation} and
\ref{spaces-morphisms-lemma-smooth-flat}).
We define a topology on $|\mathcal{X}|$ by the rule:
$W \subset |\mathcal{X}|$ is open if and only if $|p|^{-1}(W)$ is open
in $|U|$. To show that this is independent of the choice of $p$, let
$p' : U' \to \mathcal{X}$ be another morphism which is surjective, flat,
locally of finite presentation from an algebraic space to
$\mathcal{X}$. Set $U'' = U \times_{\mathcal{X}} U'$
so that we have a $2$-commutative diagram
$$
\xymatrix{
U'' \ar[r] \ar[d] & U' \ar[d] \\
U \ar[r] & \mathcal{X}
}
$$
As $U \to \mathcal{X}$ and $U' \to \mathcal{X}$ are surjective, flat,
locally of finite presentation we see that $U'' \to U'$ and $U'' \to U$
are surjective, flat and locally of finite presentation, see
Lemma \ref{lemma-property-spaces-too}.
Hence the maps $|U''| \to |U'|$ and $|U''| \to |U|$ are continuous, open
and surjective, see
Morphisms of Spaces,
Definition \ref{spaces-morphisms-definition-surjective} and
Lemma \ref{spaces-morphisms-lemma-fppf-open}.
This clearly implies that our definition is independent of the choice
of $p : U \to \mathcal{X}$.

\medskip\noindent
Let $f : \mathcal{X} \to \mathcal{Y}$ be a morphism of algebraic stacks.
By
Algebraic Stacks, Lemma \ref{algebraic-lemma-lift-morphism-presentations}
we can find a $2$-commutative diagram
$$
\xymatrix{
U \ar[d]_x \ar[r]_a & V \ar[d]^y \\
\mathcal{X} \ar[r]^f & \mathcal{Y}
}
$$
with surjective smooth vertical arrows.
Consider the associated commutative diagram
$$
\xymatrix{
|U| \ar[d]_{|x|} \ar[r]_{|a|} & |V| \ar[d]^{|y|} \\
|\mathcal{X}| \ar[r]^{|f|} & |\mathcal{Y}|
}
$$
of sets. If $W \subset |\mathcal{Y}|$ is open, then by the definition
above this means exactly that $|y|^{-1}(W)$ is open in $|V|$. Since
$|a|$ is continuous we conclude that
$|a|^{-1}|y|^{-1}(W) = |x|^{-1}|f|^{-1}(W)$ is open in $|W|$ which means
by definition that $|f|^{-1}(W)$ is open in $|\mathcal{X}|$.
Thus $|f|$ is continuous.

\medskip\noindent
Finally, we have to show that if $U$ is an algebraic space, and
$U \to \mathcal{X}$ is flat and locally of finite presentation, then
$|U| \to |\mathcal{X}|$ is open. Let $V \to \mathcal{X}$ be surjective,
flat, and locally of finite presentation with $V$ an algebraic space.
Consider the commutative diagram
$$
\xymatrix{
|U \times_{\mathcal{X}} V| \ar[r]_e \ar[rd]_f &
|U| \times_{|\mathcal{X}|} |V| \ar[d]_c \ar[r]_d &
|V| \ar[d]^b \\
& |U| \ar[r]^a & |\mathcal{X}|
}
$$
Now the morphism $U \times_{\mathcal{X}} V \to U$ is surjective, i.e,
$f : |U \times_{\mathcal{X}} V| \to |U|$ is surjective.
The left top horizontal arrow is surjective, see
Lemma \ref{lemma-points-cartesian}.
The morphism $U \times_{\mathcal{X}} V \to V$ is flat and locally of finite
presentation, hence $d \circ e : |U \times_{\mathcal{X}} V| \to |V|$ is open,
see
Morphisms of Spaces, Lemma \ref{spaces-morphisms-lemma-fppf-open}.
Pick $W \subset |U|$ open. The properties above imply that
$b^{-1}(a(W)) = (d \circ e)(f^{-1}(W))$ is open, which by construction means
that $a(W)$ is open as desired.
\end{proof}

\begin{definition}
\label{definition-topological-space}
Let $\mathcal{X}$ be an algebraic stack.
The underlying {\it topological space} of $\mathcal{X}$ is the set of points
$|\mathcal{X}|$ endowed with the topology constructed in
Lemma \ref{lemma-topology-points}.
\end{definition}

\noindent
This definition does not conflict with the already existing topology
on $|\mathcal{X}|$ if $\mathcal{X}$ is an algebraic space.

\begin{lemma}
\label{lemma-space-locally-quasi-compact}
Let $\mathcal{X}$ be an algebraic stack.
Every point of $|\mathcal{X}|$ has a fundamental system of
quasi-compact open neighbourhoods.
In particular $|\mathcal{X}|$ is locally quasi-compact in the sense of
Topology, Definition \ref{topology-definition-locally-quasi-compact}.
\end{lemma}

\begin{proof}
This follows formally from the fact that there exists a scheme
$U$ and a surjective, open, continuous map $U \to |\mathcal{X}|$
of topological spaces. Namely, if $U \to \mathcal{X}$ is surjective and
smooth, then
Lemma \ref{lemma-topology-points}
garantees that $|U| \to |\mathcal{X}|$ is continuous, surjective, and open.
\end{proof}











\section{Surjective morphisms}
\label{section-surjective}

\noindent
Let $f : \mathcal{X} \to \mathcal{Y}$ be a morphism of algebraic stacks
which is representable by algebraic spaces. In
Section \ref{section-properties-morphisms}
we have already defined what it means for $f$ to be surjective. In
Lemma \ref{lemma-characterize-surjective}
we have seen that this is equivalent to requiring
$|f| : |\mathcal{X}| \to |\mathcal{Y}|$ to be surjective.
This clears the way for the following definition.

\begin{definition}
\label{definition-surjective}
Let $f : \mathcal{X} \to \mathcal{Y}$ be a morphism of algebraic stacks.
We say $f$ is {\it surjective} if the map
$|f| : |\mathcal{X}| \to |\mathcal{Y}|$ of associated topological spaces
is surjective.
\end{definition}

\noindent
Here are some lemmas.

\begin{lemma}
\label{lemma-composition-surjective}
The composition of surjective morphisms is surjective.
\end{lemma}

\begin{proof}
Omitted.
\end{proof}

\begin{lemma}
\label{lemma-base-change-surjective}
The base change of a surjective morphism is surjective.
\end{lemma}

\begin{proof}
Omitted. Hint: Use
Lemma \ref{lemma-points-cartesian}.
\end{proof}















\section{Quasi-compact algebraic stacks}
\label{section-quasi-compact}

\noindent
The following definition is equivalent with the definition for algebraic
spaces by
Properties of Spaces, Lemma \ref{spaces-properties-lemma-quasi-compact-space}.

\begin{definition}
\label{definition-quasi-compact}
Let $\mathcal{X}$ be an algebraic stack.
We say $\mathcal{X}$ is {\it quasi-compact}
if and only if $|\mathcal{X}|$ is quasi-compact.
\end{definition}

\begin{lemma}
\label{lemma-quasi-compact-stack}
Let $\mathcal{X}$ be an algebraic stack.
The following are equivalent:
\begin{enumerate}
\item $\mathcal{X}$ is quasi-compact,
\item there exists a surjective smooth morphism $U \to \mathcal{X}$
with $U$ a quasi-compact scheme,
\item there exists a surjective smooth morphism $U \to \mathcal{X}$
with $U$ a quasi-compact algebraic space, and
\item there exists a surjective morphism $\mathcal{U} \to \mathcal{X}$
of algebraic stacks such that $\mathcal{U}$ is quasi-compact.
\end{enumerate}
\end{lemma}

\begin{proof}
We will use
Lemma \ref{lemma-characterize-surjective}.
Suppose $\mathcal{U}$ and $\mathcal{U} \to \mathcal{X}$ are as in (4).
Then since $|\mathcal{U}| \to |\mathcal{X}|$ is surjective and
continuous we conclude that $|\mathcal{X}|$ is quasi-compact.
Thus (4) implies (1). The implications (2) $\Rightarrow$ (3) $\Rightarrow$ (4)
are immediate. Assume (1), i.e., $\mathcal{X}$ is quasi-compact, i.e., that
$|\mathcal{X}|$ is quasi-compact. Choose a scheme $U$ and a surjective
smooth morphism $U \to \mathcal{X}$. Then since $|U| \to |\mathcal{X}|$
is open we see that there exists a quasi-compact open $U' \subset U$
such that $|U'| \to |X|$ is surjective (and still smooth).
Hence (2) holds.
\end{proof}

\begin{lemma}
\label{lemma-finite-disjoint-quasi-compact}
A finite disjoint union of quasi-compact algebraic stacks is
a quasi-compact algebraic stack.
\end{lemma}

\begin{proof}
This is clear from the corresponding topological fact.
\end{proof}


\section{Properties of algebraic stacks defined by properties of schemes}
\label{section-types-properties}

\noindent
Any smooth local property of schemes gives rise to a corresponding
property of algebraic stacks via the following lemma. Note that a
property of schemes which is smooth local is also \'etale local
as any \'etale covering is also a smooth covering. Hence for a smooth
local property $P$ of schemes we know what it means to say that
an algebraic space has $P$, see
Properties of Spaces, Section \ref{section-types-properties}.

\begin{lemma}
\label{lemma-type-property}
Let $\mathcal{P}$ be a property of schemes which is local in the smooth
topology, see
Descent, Definition \ref{descent-definition-property-local}.
Let $\mathcal{X}$ be an algebraic stack. The following are equivalent
\begin{enumerate}
\item for some scheme $U$ and some surjective smooth morphism
$U \to \mathcal{X}$ the scheme $U$ has property $\mathcal{P}$,
\item for every scheme $U$ and every smooth morphism $U \to \mathcal{X}$
the scheme $U$ has property $\mathcal{P}$,
\item for some algebraic space $U$ and some surjective smooth morphism
$U \to \mathcal{X}$ the algebraic space $U$ has property $\mathcal{P}$, and
\item for every algebraic space $U$ and every smooth morphism
$U \to \mathcal{X}$ the algebraic space $U$ has property $\mathcal{P}$.
\end{enumerate}
If $\mathcal{X}$ is a scheme this is equivalent to $\mathcal{P}(U)$.
If $\mathcal{X}$ is an algebraic space this is equivalent to
$X$ having property $\mathcal{P}$.
\end{lemma}

\begin{proof}
Let $U \to \mathcal{X}$ surjective and smooth with $U$ an algebraic space.
Let $V \to \mathcal{X}$ be a smooth morphism with $V$ an algebraic space.
Choose schemes $U'$ and $V'$ and surjective \'etale morphisms
$U' \to U$ and $V' \to V$. Finally, choose a scheme $W$ and a
surjective \'etale morphism $W \to V' \times_{\mathcal{X}} U'$.
Then $W \to V'$ and $W \to U'$ are smooth morphisms of schemes
as compositions of \'etale and smooth morphisms of algebraic spaces, see
Morphisms of Spaces, Lemmas \ref{spaces-morphisms-lemma-etale-smooth} and
\ref{spaces-morphisms-lemma-composition-smooth}.
Moreover, $W \to V'$ is surjective as $U' \to \mathcal{X}$ is surjective.
Hence, we have
$$
\mathcal{P}(U) \Leftrightarrow
\mathcal{P}(U') \Rightarrow
\mathcal{P}(W) \Rightarrow
\mathcal{P}(V') \Leftrightarrow \mathcal{P}(V)
$$
where the equivalences are by definition of property $\mathcal{P}$ for
algebraic spaces, and the two implications come from
Descent, Definition \ref{descent-definition-property-local}.
This proves (3) $\Rightarrow$ (4).

\medskip\noindent
The implications (2) $\Rightarrow$ (1), (1) $\Rightarrow$ (3),
and (4) $\Rightarrow$ (2) are immediate.
\end{proof}

\begin{definition}
\label{definition-type-property}
Let $\mathcal{X}$ be an algebraic stack.
Let $\mathcal{P}$ be a property of schemes which is 
local in the smooth topology.
We say $\mathcal{X}$ {\it has property $\mathcal{P}$}
if any of the equivalent conditions of
Lemma \ref{lemma-type-property}
hold.
\end{definition}

\begin{remark}
\label{remark-list-properties-local-smooth-topology}
Here is a list of properties which are local for the smooth topology
(keep in mind that the fpqc, fppf, and syntomic topologies are
stronger than the smooth topology):
\begin{enumerate}
\item locally Noetherian, see
Descent, Lemma \ref{descent-lemma-Noetherian-local-fppf},
\item Jacobson, see
Descent, Lemma \ref{descent-lemma-Jacobson-local-fppf},
\item locally Noetherian and $(S_k)$, see
Descent, Lemma \ref{descent-lemma-Sk-local-syntomic},
\item Cohen-Macaulay, see
Descent, Lemma \ref{descent-lemma-CM-local-syntomic},
\item reduced, see
Descent, Lemma \ref{descent-lemma-reduced-local-smooth},
\item normal, see
Descent, Lemma \ref{descent-lemma-normal-local-smooth},
\item locally Noetherian and $(R_k)$, see
Descent, Lemma \ref{descent-lemma-Rk-local-smooth},
\item regular, see
Descent, Lemma \ref{descent-lemma-regular-local-smooth},
\item Nagata, see
Descent, Lemma \ref{descent-lemma-Nagata-local-smooth}.
\end{enumerate}
\end{remark}

\noindent
Any smooth local property of germs of schemes gives rise to a corresponding
property of algebraic stacks. Note that a property of germs which is smooth
local is also \'etale local. Hence for a smooth local property of germs of
schemes $P$ we know what it means to say that an algebraic space $X$ has
property $P$ at $x \in |X|$, see
Properties of Spaces, Section \ref{section-types-properties}.

\begin{lemma}
\label{lemma-local-source-target-at-point}
Let $\mathcal{X}$ be an algebraic stack.
Let $x \in |\mathcal{X}|$ be a point of $\mathcal{X}$.
Let $\mathcal{P}$ be a property of germs of schemes which is smooth local, see
Descent, Definition \ref{descent-definition-local-at-point}.
The following are equivalent
\begin{enumerate}
\item for any smooth morphism $U \to \mathcal{X}$ with $U$ a scheme
and $u \in U$ with $a(u) = x$ we have $\mathcal{P}(U, u)$,
\item for some smooth morphism $U \to \mathcal{X}$ with $U$ a scheme
and some $u \in U$ with $a(u) = x$ we have $\mathcal{P}(U, u)$,
\item for any smooth morphism $U \to \mathcal{X}$ with $U$ an algebraic space
and $u \in |U|$ with $a(u) = x$ the algebraic space $U$ has property
$\mathcal{P}$ at $u$, and
\item for some smooth morphism $U \to \mathcal{X}$ with $U$ a an
algebraic space and some $u \in |U|$ with $a(u) = x$ the algebraic space
$U$ has property $\mathcal{P}$ at $u$.
\end{enumerate}
If $\mathcal{X}$ is representable, then this is equivalent to
$\mathcal{P}(\mathcal{X}, x)$. If $\mathcal{X}$ is an algebraic space then
this is equivalent to $\mathcal{X}$ having property $\mathcal{P}$ at $x$.
\end{lemma}

\begin{proof}
Let $a : U \to \mathcal{X}$ and $u \in |U|$ as in (3). Let
$b : V \to \mathcal{X}$ be another smooth morphism with $V$ an algebraic
space and $v \in |V|$ with $b(v) = x$ also.
Choose a scheme $U'$, an \'etale morphism $U' \to U$ and $u' \in U'$
mapping to $u$. Choose a scheme $V'$, an \'etale morphism $V' \to V$
and $v' \in V'$ mapping to $v$. By
Lemma \ref{lemma-points-cartesian}
there exists a point $\overline{w} \in |V' \times_{\mathcal{X}} U'|$
mapping to $u'$ and $v'$. Choose a scheme $W$ and a surjective \'etale
morphism $W \to V' \times_{\mathcal{X}} U'$. We may choose a
$w \in |W|$ mapping to $\overline{w}$ (see
Properties of Spaces,
Lemma \ref{spaces-properties-lemma-characterize-surjective}).
Then $W \to V'$ and $W \to U'$ are smooth morphisms of schemes
as compositions of \'etale and smooth morphisms of algebraic spaces, see
Morphisms of Spaces, Lemmas \ref{spaces-morphisms-lemma-etale-smooth} and
\ref{spaces-morphisms-lemma-composition-smooth}.
Hence
$$
\mathcal{P}(U, u)
\Leftrightarrow
\mathcal{P}(U', u')
\Leftrightarrow
\mathcal{P}(W, w)
\Leftrightarrow
\mathcal{P}(V', v')
\Leftrightarrow
\mathcal{P}(V, v)
$$
The outer two equivalences by
Properties of Spaces,
Definition \ref{spaces-properties-definition-property-at-point}
and the other two by what it means to be a smooth local property
of germs of schemes. This proves (4) $\Rightarrow$ (3).

\medskip\noindent
The implications (1) $\Rightarrow$ (2), (2) $\Rightarrow$ (4),
and (3) $\Rightarrow$ (1) are immediate.
\end{proof}

\begin{definition}
\label{definition-property-at-point}
Let $\mathcal{P}$ be a property of germs of schemes which is 
smooth local. Let $\mathcal{X}$ be an algebraic stack.
Let $x \in |\mathcal{X}|$.
We say $\mathcal{X}$ {\it has property $\mathcal{P}$ at $x$}
if any of the equivalent conditions of
Lemma \ref{lemma-local-source-target-at-point}
holds.
\end{definition}










\section{Monomorphisms of algebraic stacks}
\label{section-monomorphisms}

\noindent
We define a monomorphism of algebraic stacks in the following way.
We will see in
Lemma \ref{lemma-monomorphism}
that this is compatible with the corresponding $2$-category theoretic notion.

\begin{definition}
\label{definition-monomorphism}
Let $f : \mathcal{X} \to \mathcal{Y}$ be a morphism of algebraic stacks.
We say $f$ is a {\it monomorphism}
if it is representable by algebraic spaces and a monomorphism in the sense of
Section \ref{section-properties-morphisms}.
\end{definition}

\noindent
First some basic lemmas.

\begin{lemma}
\label{lemma-base-change-monomorphism}
Let $\mathcal{X} \to \mathcal{Y}$ be a morphism of algebraic stacks.
Let $\mathcal{Z} \to \mathcal{Y}$ be a monomorphism
Then $\mathcal{Z} \times_{\mathcal{Y}} \mathcal{X} \to \mathcal{X}$
is a monomorphism.
\end{lemma}

\begin{proof}
This follows from the general discussion in
Section \ref{section-properties-morphisms}.
\end{proof}

\begin{lemma}
\label{lemma-composition-monomorphism}
Compositions of monomorphisms of algebraic stacks are monomorphisms.
\end{lemma}

\begin{proof}
This follows from the general discussion in
Section \ref{section-properties-morphisms}
and
Morphisms of Spaces,
Lemma \ref{spaces-morphisms-lemma-composition-monomorphism}.
\end{proof}

\begin{lemma}
\label{lemma-monomorphism}
Let $f : \mathcal{X} \to \mathcal{Y}$ be a morphism of algebraic stacks.
The following are equivalent:
\begin{enumerate}
\item $f$ is a monomorphism,
\item $f$ is fully faithful,
\item the diagonal
$\Delta_f : \mathcal{X} \to \mathcal{X} \times_{\mathcal{Y}} \mathcal{X}$
is an equivalence, and
\item there exists an algebraic space $W$ and a surjective, flat morphism
$W \to \mathcal{Y}$ which is locally of finite presentation such that
$V = \mathcal{X} \times_{\mathcal{Y}} W$ is an algebraic space, and the
morphism $V \to W$ is a monomorphism of algebraic spaces.
\end{enumerate}
\end{lemma}

\begin{proof}
The equivalence of (1) and (4) follows from the general discussion in
Section \ref{section-properties-morphisms}
and in particular
Lemmas \ref{lemma-check-representable-covering} and
\ref{lemma-check-property-covering}.

\medskip\noindent
The equivalence of (2) and (3) is
Categories, Lemma \ref{categories-lemma-fully-faithful-diagonal-equivalence}.

\medskip\noindent
Assume the equivalent conditions (2) and (3). Then $f$ is representable
by algebraic spaces according to
Algebraic Stacks,
Lemma \ref{algebraic-lemma-characterize-representable-by-algebraic-spaces}.
Moreover, the $2$-Yoneda lemma combined with the fully faithfullness
implies that for every scheme $T$ the functor
$$
\text{Mor}(T, \mathcal{X})
\longrightarrow
\text{Mor}(T, \mathcal{Y})
$$
is fully faithful. Hence given a morphism $y : T \to \mathcal{Y}$ there exists
up to unique $2$-isomorphism at most one morphism $x : T \to \mathcal{X}$
such that $y \cong f \circ x$. In particular, given a morphism of schemes
$h : T' \to T$ there exists at most one lift
$\tilde h : T' \to T \times_{\mathcal{Y}} \mathcal{X}$ of $h$.
Thus $T \times_{\mathcal{Y}} \mathcal{X} \to T$ is a monomorphism of
algebraic spaces, which proves that (1) holds.

\medskip\noindent
Finally, assume that (1) holds. Then for any scheme $T$ and morphism
$y : T \to \mathcal{Y}$ the fibre product $T \times_{\mathcal{Y}} \mathcal{X}$
is an algebraic space, and $T \times_{\mathcal{Y}} \mathcal{X} \to T$
is a monomorphism. Hence there exists up to unique isomorphism
exactly one pair $(x, \alpha)$ where $x : T \to \mathcal{X}$ is a morphism
and $\alpha : f \circ x \to y$ is a $2$-morphism. Applying
the $2$-Yoneda lemma this says exactly that $f$ is fully faithful, i.e.,
that (2) holds.
\end{proof}

\begin{lemma}
\label{lemma-monomorphism-injective-points}
A monomorphism of algebraic stacks induces an injective map of
sets of points.
\end{lemma}

\begin{proof}
Let $f : \mathcal{X} \to \mathcal{Y}$ be a monomorphism of algebraic stacks.
Suppose that $x_i : \text{Spec}(K_i) \to \mathcal{X}$ be morphisms such that
$f \circ x_1$ and $f \circ x_2$ define the same element of $|\mathcal{Y}|$.
Applying the definition we find a common extension $\Omega$ with corresponding
morphisms $c_i : \text{Spec}(\Omega) \to \text{Spec}(K_i)$ and a
$2$-isomorphism $\beta : f \circ x_1 \circ c_1 \to f \circ x_1 \circ c_2$.
As $f$ is fully faithful, see
Lemma \ref{lemma-monomorphism},
we can lift $\beta$ to an isomorphism
$\alpha : f \circ x_1 \circ c_1 \to f \circ x_1 \circ c_2$.
Hence $x_1$ and $x_2$ define the same point of $|\mathcal{X}|$
as desired.
\end{proof}








\section{Immersions of algebraic stacks}
\label{section-immersions}

\noindent
Immersions of algebraic stacks are defined as follows.

\begin{definition}
\label{definition-immersion}
Immersions.
\begin{enumerate}
\item A morphism of algebraic stacks is called an {\it open immersion}
if it is representable, and an open immersion
in the sense of
Section \ref{section-properties-morphisms}.
\item A morphism of algebraic stacks is called a {\it closed immersion}
if it is representable, and a closed immersion
in the sense of
Section \ref{section-properties-morphisms}.
\item A morphism of algebraic stacks is called an {\it immersion}
if it is representable, and an immersion
in the sense of
Section \ref{section-properties-morphisms}.
\end{enumerate}
\end{definition}

\noindent
This is not the most convenient way to think about immersions for us.
For us it is a little bit more convenient to think of an immersion
as a morphism of algebraic stacks which is representable by algebraic
spaces and is an immersion in the sense of
Section \ref{section-properties-morphisms}.
Similarly for closed and open immersions.
Since this is clearly equivalent to the notion just defined we shall
use this characterization without further mention.
We prove a few simple lemmas about this notion.

\begin{lemma}
\label{lemma-base-change-immersion}
Let $\mathcal{X} \to \mathcal{Y}$ be a morphism of algebraic stacks.
Let $\mathcal{Z} \to \mathcal{Y}$ be a
(closed, resp.\ open) immersion.
Then $\mathcal{Z} \times_{\mathcal{Y}} \mathcal{X} \to \mathcal{X}$
is a (closed, resp.\ open) immersion.
\end{lemma}

\begin{proof}
This follows from the general discussion in
Section \ref{section-properties-morphisms}.
\end{proof}

\begin{lemma}
\label{lemma-composition-immersion}
Compositions of immersions of algebraic stacks are immersions.
Similarly for closed immersions and open immersions.
\end{lemma}

\begin{proof}
This follows from the general discussion in
Section \ref{section-properties-morphisms}
and
Spaces, Lemma \ref{spaces-lemma-composition-immersions}.
\end{proof}

\begin{lemma}
\label{lemma-check-immersion-covering}
Let $f : \mathcal{X} \to \mathcal{Y}$ be a morphism of algebraic stacks.
let $W$ be an algebraic space and let $W \to \mathcal{Y}$ be a surjective,
flat morphism which is locally of finite presentation. The following
are equivalent:
\begin{enumerate}
\item $f$ is an (open, resp.\ closed) immersion, and
\item $V = W \times_{\mathcal{Y}} \mathcal{X}$ is an algebraic space, and
$V \to W$ is an (open, resp.\ closed) immersion.
\end{enumerate}
\end{lemma}

\begin{proof}
This follows from the general discussion in
Section \ref{section-properties-morphisms}
and in particular
Lemmas \ref{lemma-check-representable-covering} and
\ref{lemma-check-property-covering}.
\end{proof}

\begin{lemma}
\label{lemma-immersion-monomorphism}
An immersion is a monomorphism.
\end{lemma}

\begin{proof}
See
Morphisms of Spaces,
Lemma \ref{spaces-morphisms-lemma-immersions-monomorphisms}.
\end{proof}

\noindent
The following two lemmas explain how to think about
immersions in terms of presentations.

\begin{lemma}
\label{lemma-immersion-into-presentation}
Let $(U, R, s, t, c)$ be a smooth groupoid in algebraic spaces.
Let $i : \mathcal{Z} \to [U/R]$ be an immersion.
Then there exists an $R$-invariant locally closed subspace
$Z \subset U$ and a presentation $[Z/R_Z] \to \mathcal{Z}$
where $R_Z$ is the restriction of $R$ to $Z$ such that
$$
\xymatrix{
[Z/R_Z] \ar[dr] \ar[rr] & & \mathcal{Z} \ar[ld]^i \\
& [U/R]
}
$$
is $2$-commutative. If $i$ is a closed (resp.\ open) immersion
then $Z$ is a closed (resp.\ open) subspace of $U$.
\end{lemma}

\begin{proof}
By
Lemma \ref{lemma-representable-in-terms-presentations}
we get a commutative diagram
$$
\xymatrix{
[U'/R'] \ar[dr] \ar[rr] & & \mathcal{Z} \ar[ld] \\
& [U/R]
}
$$
where $U' = \mathcal{Z} \times_{[U/R]} U$ and
$R' = \mathcal{Z} \times_{[U/R]} R$.
Since $\mathcal{Z} \to [U/R]$ is an immersion we see that
$U' \to U$ is an immersion of algebraic spaces. Let $Z \subset U$
be the locally closed subspace such that $U' \to U$ factors through
$Z$ and induces an isomorphism $U' \to Z$.
It is clear from the construction of $R'$ that
$R' = U' \times_{U, t} R = R \times_{s, U} U'$.
This implies that $Z \cong U'$ is $R$-invariant and that the image of
$R' \to R$ identifies $R'$ with the restriction
$R_Z = s^{-1}(Z) = t^{-1}(Z)$ of $R$ to $Z$. Hence the lemma holds.
\end{proof}

\begin{lemma}
\label{lemma-immersion-presentation}
Let $(U, R, s, t, c)$ be a smooth groupoid in algebraic spaces.
Let $\mathcal{X} = [U/R]$ be the associated algebraic stack, see
Algebraic Stacks,
Theorem \ref{algebraic-theorem-smooth-groupoid-gives-algebraic-stack}.
Let $Z \subset U$ be an $R$-invariant locally closed subspace. Then
$$
[Z/R_Z] \longrightarrow [U/R]
$$
is an immersion of algebraic stacks, where $R_Z$ is the restriction
of $R$ to $Z$. If $Z \subset U$ is open (resp.\ closed) then the morphism
is an open (resp.\ closed) immersion of algebraic stacks.
\end{lemma}

\begin{proof}
Recall that by
Groupoids in Spaces,
Definition \ref{spaces-groupoids-definition-invariant-open}
(see also discussion following the definition)
we have $R_Z = s^{-1}(Z) = t^{-1}(Z)$ as locally closed subspaces
of $R$. Hence the two morphisms $R_Z \to Z$ are smooth as base changes
of $s$ and $t$. Hence
$(Z, R_Z, s|_{R_Z}, t|_{R_Z}, c|_{R_Z \times_{s, Z, t} R_Z})$ is
a smooth groupoid in algebraic spaces, and we see that
$[Z/R_Z]$ is an algebraic stack, see
Algebraic Stacks,
Theorem \ref{algebraic-theorem-smooth-groupoid-gives-algebraic-stack}.
The assumptions of
Groupoids in Spaces,
Lemma \ref{spaces-groupoids-lemma-criterion-fibre-product}
are all satisfied and it follows that we have a $2$-fibre square
$$
\xymatrix{
Z \ar[d] \ar[r] & [Z/R_Z] \ar[d] \\
U \ar[r] & [U/R]
}
$$
It follows from this and
Lemma \ref{lemma-check-representable-covering}
that $[Z/R_Z] \to [U/R]$ is representable by algebraic spaces,
whereupon it follows from
Lemma \ref{lemma-check-property-covering}
that the right vertical arrow is an immersion (resp.\ closed immersion,
resp.\ open immersion) if and only if the left vertical arrow is.
\end{proof}

\noindent
We can define open, closed, and locally closed substacks as follows.

\begin{definition}
\label{definition-substacks}
Let $\mathcal{X}$ be an algebraic stack.
\begin{enumerate}
\item An {\it open substack} of $\mathcal{X}$ is a strictly full subcategory
$\mathcal{X}' \subset \mathcal{X}$ such that $\mathcal{X}'$ is an algebraic
stack and $\mathcal{X}' \to \mathcal{X}$ is an open immersion.
\item A {\it closed substack} of $\mathcal{X}$ is a strictly full subcategory
$\mathcal{X}' \subset \mathcal{X}$ such that $\mathcal{X}'$ is an algebraic
stack and $\mathcal{X}' \to \mathcal{X}$ is a closed immersion.
\item A {\it locally closed substack} of $\mathcal{X}$ is a strictly full
subcategory $\mathcal{X}' \subset \mathcal{X}$ such that $\mathcal{X}'$
is an algebraic stack and $\mathcal{X}' \to \mathcal{X}$ is an immersion.
\end{enumerate}
\end{definition}

\noindent
This definition should be used with caution. Namely, if
$f : \mathcal{X} \to \mathcal{Y}$ is an equivalence of algebraic stacks
and $\mathcal{X}' \subset \mathcal{X}$ is an open substack, then it
is not necessarily the case that the subcategory $f(\mathcal{X}')$
is an open substack of $\mathcal{Y}$. The problem is that it may not
be a {\it strictly} full subcategory; but this is also the only problem.
Here is a formal statement.

\begin{lemma}
\label{lemma-substack-image}
For any immersion $i : \mathcal{Z} \to \mathcal{X}$ there exists a
unique locally closed substack $\mathcal{X}' \subset \mathcal{X}$
such that $i$ factors as the composition of
an equivalence $i' : \mathcal{Z} \to \mathcal{X}'$
followed by the inclusion morphism $\mathcal{X}' \to \mathcal{X}$.
If $i$ is a closed (resp.\ open) immersion, then $\mathcal{X}'$
is a closed (resp.\ open) substack of $\mathcal{X}$.
\end{lemma}

\begin{proof}
Omitted.
\end{proof}

\begin{lemma}
\label{lemma-substacks-presentation}
Let $[U/R] \to \mathcal{X}$ be a presentation of an algebraic stack.
There is a canonical bijection
$$
R\text{-invariant locally closed subspaces }Z\text{ of }U
\leftrightarrow
\text{locally closed substacks }\mathcal{Z}\text{ of }\mathcal{X}
$$
where if $Z$ corresponds to $\mathcal{Z}$, then
$[Z/R_Z] \to \mathcal{Z}$ is a presentation fitting into a $2$-commutative
diagram with the given presentation of $\mathcal{X}$.
Similarly for closed substacks and open substacks.
\end{lemma}

\begin{proof}
Omitted. Hints: Use
Lemma \ref{lemma-immersion-into-presentation}
to go from right to left and
Lemma \ref{lemma-immersion-presentation}
from left to right.
\end{proof}

\begin{lemma}
\label{lemma-open-substacks}
Let $\mathcal{X}$ be an algebraic stack. The rule
$\mathcal{U} \mapsto |\mathcal{U}|$ defines an inclusion preserving
bijection between open substacks of $\mathcal{X}$ and open subsets
of $|\mathcal{X}|$.
\end{lemma}

\begin{proof}
Choose a presentation $[U/R] \to \mathcal{X}$, see
Algebraic Stacks, Lemma \ref{algebraic-lemma-stack-presentation}.
By
Lemma \ref{lemma-substacks-presentation}
we see that open substacks correspond to $R$-invariant open subschemes
of $U$. On the other hand
Lemmas \ref{lemma-points-presentation} and \ref{lemma-topology-points}
guarantee these correspond bijectively to open subsets of $|\mathcal{X}|$.
\end{proof}

\begin{lemma}
\label{lemma-open-image-substack}
Let $\mathcal X$ be an algebraic stack. Let $U$ be an algebraic space and
$U \to \mathcal X$ a surjective smooth morphism. For an open immersion
$V \hookrightarrow U$, there exists an algebraic stack $\mathcal Y$, an
open immersion $\mathcal Y \to \mathcal X$, and a surjective smooth
morphism $V \to \mathcal Y$. 
\end{lemma}

\begin{proof}
We define a category fibred in groupoids $\mathcal Y$ by letting the fiber
category $\mathcal{Y}_T$ over an object $T$ of $(\textit{Sch}/S)_{fppf}$ be 
the full subcategory of $\mathcal{X}_T$ consisting of all
$y \in \text{Ob}(\mathcal{X}_T)$ such that the projection morphism
$V \times_{\mathcal X, y} T \to T$ surjective. Now for any morphism
$x : T \to \mathcal X$, the $2$-fibred product
$T \times_{x, \mathcal X} \mathcal Y$ has fiber category over $T'$ consisting
of triples $(f : T' \to T, y \in \mathcal{X}_{T'}, f^*x \simeq y)$ such that
$V \times_{\mathcal X, y} T' \to T'$ is surjective.
Note that $T \times_{x, \mathcal X} \mathcal Y$ is fibered in setoids since
$\mathcal Y \to \mathcal X$ is faithful (see
Stacks, Lemma \ref{stacks-lemma-2-fibre-product-gives-stack-in-setoids}).
Now the isomorphism $f^*x \simeq y$ gives the diagram
$$
\xymatrix{
V \times_{\mathcal X, y} T' \ar[d] \ar[r] &
V \times_{\mathcal X,x} T \ar[r] \ar[d] &
V \ar[d] \\
T' \ar[r]^f &
T \ar[r]^x &
\mathcal X
}
$$
where both squares are cartesian. The morphism
$V \times_{\mathcal X, x} T \to T$ is smooth by base change, and hence open.
Let $T_0 \subset T$ be its image. From the cartesian squares we deduce that 
$V \times_{\mathcal X,y} T' \to T'$ is surjective if and only if $f$ lands
in $T_0$.  Therefore $T \times_{x, \mathcal X} \mathcal Y$ is representable by
$T_0$, so the inclusion $\mathcal Y \to \mathcal X$ is an open immersion.
By
Algebraic Stacks, Lemma \ref{algebraic-lemma-open-fibred-category-is-algebraic}
we conclude that $\mathcal{Y}$ is an algebraic stack.
Lastly if we denote the morphism $V \to \mathcal X$ by $g$, we have
$V \times_{\mathcal X} V \to V$ is surjective (the diagonal gives a
section). Hence $g$ is in the image of $\mathcal{Y}_V \to \mathcal{X}_V$, i.e.,
we obtain a morphism $g' : V \to \mathcal{Y}$ fitting into the commutative
diagram
$$
\xymatrix{
V \ar[r] \ar[d]^{g'} & U \ar[d] \\
\mathcal{Y} \ar[r] & \mathcal{X}
}
$$
Since $V \times_{g, \mathcal X} \mathcal Y \to V$ is a
monomorphism, it is in fact an isomorphism since $(1, g')$ defines a section.
Therefore $g' : V \to \mathcal Y$ is a smooth morphism, as it is the
base change of the smooth morphism $g : V \to \mathcal{X}$.
It is surjective by our construction of $\mathcal{Y}$ which finishes
the proof of the lemma.
\end{proof}

\begin{lemma} 
\label{lemma-union-open-substacks} 
Let $\mathcal X$ be an algebraic stack and $\mathcal{X}_i \subset \mathcal X$
a collection of open substacks indexed by $i \in I$. Then there exists an
open substack, which we denote
$\bigcup_{i\in I} \mathcal{X}_i \subset \mathcal X$, such that 
the $\mathcal{X}_i$ are open substacks covering it.
\end{lemma}

\begin{proof}
We define a fibred subcategory
$\mathcal{X}' = \bigcup_{i \in I} \mathcal{X}_i$ 
by letting the fiber category over an object $T$ of $(\textit{Sch}/S)_{fppf}$
be the full subcategory of $\mathcal{X}_T$ consisting of all
$x \in \text{Ob}(\mathcal{X}_T)$ such that the morphism
$\bigsqcup_{i \in I} (\mathcal{X}_i \times_{\mathcal X} T) \to T$
is surjective. Let $x_i \in \text{Ob}((\mathcal{X}_i)_T)$.
Then $(x_i, 1)$ gives a section of 
$\mathcal{X}_i \times_{\mathcal X} T \to T$, so we have an isomorphism. Thus
$\mathcal{X}_i \subset \mathcal{X}'$ is a full subcategory. 
Now let $x \in \text{Ob}(\mathcal{X}_T)$. Then
$\mathcal{X}_i \times_{\mathcal X} T$ is representable
by an open subscheme $T_i \subset T$. The $2$-fibred product 
$\mathcal{X}' \times_{\mathcal X} T$ has fiber over $T'$ consisting
of $(y \in \mathcal{X}_{T'}, f : T' \to T, f^*x \simeq y)$ such that 
$\bigsqcup (\mathcal{X}_i \times_{\mathcal X, y} T') \to T'$ is surjective. 
The isomorphism $f^*x \simeq y$ induces an isomorphism
$\mathcal{X}_i \times_{\mathcal X,y} T' \simeq T_i \times_T T'$.
Then the $T_i \times_T T'$ cover $T'$ if and only if $f$ lands in
$\bigcup T_i$. Therefore we have a diagram
$$
\xymatrix{
T_i \ar[r] \ar[d] &
\bigcup T_i \ar[r] \ar[d] &
T \ar[d] \\ 
\mathcal{X}_i \ar[r] &
\mathcal{X}' \ar[r] &
\mathcal{X}
}
$$
with both squares cartesian. By
Algebraic Stacks, Lemma \ref{algebraic-lemma-open-fibred-category-is-algebraic}
we conclude that $\mathcal{X'} \subset \mathcal{X}$ is algebraic and an
open substack. It is also clear from the cartesian squares above that the
morphism $\bigsqcup_{i \in I} \mathcal{X}_i \to \mathcal{X}'$ which
finishes the proof of the lemma.
\end{proof}

\begin{lemma} 
\label{lemma-quasicompact-finite-subcover} 
Let $\mathcal X$ be an algebraic stack and $\mathcal X' \subset \mathcal X$
a quasi-compact open substack. Suppose that we have a collection of open
substacks $\mathcal{X}_i \subset \mathcal X$  indexed by $i \in I$ such
that $\mathcal{X}' \subset \bigcup_{i \in I} \mathcal{X}_i$,
where we define the union as in Lemma \ref{lemma-union-open-substacks}. 
Then there exists a finite subset $I' \subset I$ such that
$\mathcal{X}' \subset \bigcup_{i \in I'} \mathcal{X}_i$. 
\end{lemma}

\begin{proof}
Since $\mathcal X$ is algebraic, there exists a scheme $U$ with a surjective
smooth morphism $U \to \mathcal X$. Let $U_i \subset U$ be the open subscheme
representing $\mathcal{X}_i \times_{\mathcal X} U$ and $U' \subset U$ the
open subscheme representing $\mathcal{X}' \times_{\mathcal X} U$. By
hypothesis, $U'\subset \bigcup_{i\in I} U_i$. From the proof of
Lemma \ref{lemma-quasi-compact-stack},
there is a quasi-compact open $V \subset U'$ such that $V \to \mathcal{X}'$
is a surjective smooth morphism. Therefore there exists a finite subset
$I' \subset I$ such that $V \subset \bigcup_{i \in I'}
U_i$. We claim that $\mathcal{X}' \subset \bigcup_{i \in I'} \mathcal{X}_i$. 
Take $x \in \text{Ob}(\mathcal{X}'_T)$ for
$T \in \text{Ob}((\textit{Sch}/S)_{fppf})$. 
Since $\mathcal{X}' \to \mathcal{X}$ is a monomorphism, we have cartesian
squares
$$
\xymatrix{
V \times_{\mathcal{X}} T \ar[r] \ar[d] &
T \ar[d]^x \ar@{=}[r] &
T \ar[d]^x \\
V \ar[r] &
\mathcal{X}' \ar[r] &
\mathcal X
}
$$
By base change, $V \times_{\mathcal X} T \to T$ is surjective. Therefore 
$\bigcup_{i \in I'} U_i \times_{\mathcal X} T \to T$ is also surjective. 
Let $T_i \subset T$ be the open subscheme representing
$\mathcal{X}_i \times_{\mathcal X} T$. 
By a formal argument, we have a Cartesian square
$$
\xymatrix{
U_i \times_{\mathcal{X}_i} T_i \ar[r] \ar[d] &
U \times_{\mathcal X} T \ar[d] \\
T_i \ar[r] & T
}
$$
where the vertical arrows are surjective by base change. Since 
$U_i \times_{\mathcal{X}_i} T_i \simeq U_i \times_{\mathcal X} T$,
we find that $\bigcup_{i \in I'} T_i = T$. Hence
$x$ is an object of $(\bigcup_{i\in I'} \mathcal{X}_i)_T$ by
definition of the union. Observe that the inclusion
$\mathcal{X}' \subset \bigcup_{i \in I'} \mathcal{X}_i$
is automatically an open substack.
\end{proof}

\begin{lemma}
\label{lemma-zariski-open-cover-stack-is-space}
Let $\mathcal X$ be an algebraic stack.
Let $\mathcal{X}_i$, $i \in I$ be a set of open substacks of $\mathcal{X}$.
Assume
\begin{enumerate}
\item $\mathcal{X} = \bigcup_{i \in I} \mathcal{X}_i$, and
\item each $\mathcal{X}_i$ is an algebraic space.
\end{enumerate}
Then $\mathcal{X}$ is an algebraic space.
\end{lemma} 

\begin{proof}
Apply
Stacks, Lemma \ref{stacks-lemma-stack-in-setoids-descent}
to the morphism $\coprod_{i \in I} \mathcal{X}_i \to \mathcal{X}$
and the morphism $\text{id} : \mathcal{X} \to \mathcal{X}$ to
see that $\mathcal{X}$ is a stack in setoids.
Hence $\mathcal{X}$ is an algebraic space, see
Algebraic Stacks,
Proposition \ref{algebraic-proposition-algebraic-stack-no-automorphisms}.
\end{proof}

\begin{lemma}
\label{lemma-zariski-open-cover-stack-is-scheme}
Let $\mathcal X$ be an algebraic stack.
Let $\mathcal{X}_i$, $i \in I$ be a set of open substacks of $\mathcal{X}$.
Assume
\begin{enumerate}
\item $\mathcal{X} = \bigcup_{i \in I} \mathcal{X}_i$, and
\item each $\mathcal{X}_i$ is a scheme
\end{enumerate}
Then $\mathcal{X}$ is a scheme.
\end{lemma} 

\begin{proof}
By
Lemma \ref{lemma-zariski-open-cover-stack-is-space}
we see that $\mathcal{X}$ is an algebraic space. Since any algebraic
space has a largest open subspace which is a scheme, see
Properties of Spaces, Lemma \ref{spaces-properties-lemma-subscheme}
we see that $\mathcal{X}$ is a scheme.
\end{proof}












\section{Reduced algebraic stacks}
\label{section-reduced}

\noindent
We have already defined reduced algebraic stacks in
Section \ref{section-types-properties}.

\begin{lemma}
\label{lemma-reduced-closed-substack}
Let $\mathcal{X}$ be an algebraic stack.
Let $T \subset |\mathcal{X}|$ be a closed subset.
There exists a unique closed substack $\mathcal{Z} \subset \mathcal{X}$
with the following properties:
(a) we have $|\mathcal{Z}| = T$, and (b) $\mathcal{Z}$ is reduced.
\end{lemma}

\begin{proof}
Let $U \to \mathcal{X}$ be a surjective smooth morphism, where $U$ is an
algebraic space. Set $R = U \times_{\mathcal{X}} U$, so that there is a
presentation $[U/R] \to \mathcal{X}$, see
Algebraic Stacks, Lemma \ref{algebraic-lemma-stack-presentation}.
As usual we denote $s, t : R \to U$ the two smooth projection morphisms.
By Lemma \ref{lemma-points-presentation}
we see that $T$ corresponds to a closed subset $T' \subset |U|$ such
that $|s|^{-1}(T') = |t|^{-1}(T')$.
Let $Z \subset U$ be the reduced induced algebraic space
structure on $T'$, see
Properties of Spaces,
Definition \ref{spaces-properties-definition-reduced-induced-space}.
The fibre products
$Z \times_{U, t} R$ and $R \times_{s, U} Z$ are closed subspaces
of $R$
(Spaces, Lemma \ref{spaces-lemma-base-change-immersions}).
The projections $Z \times_{U, t} R \to Z$ and
$R \times_{s, U} Z \to Z$ are smooth by
Morphisms of Spaces, Lemma \ref{spaces-morphisms-lemma-base-change-smooth}.
Thus as $Z$ is reduced, it follows that
$Z \times_{U, t} R$ and $R \times_{s, U} Z$ are reduced, see
Remark \ref{remark-list-properties-local-smooth-topology}.
Since
$$
|Z \times_{U, t} R| = |t|^{-1}(T') = |s|^{-1}(T') = R \times_{s, U} Z
$$
we conclude from the uniqueness in
Properties of Spaces,
Lemma \ref{spaces-properties-lemma-reduced-closed-subspace}
that $Z \times_{U, t} R = R \times_{s, U} Z$.
Hence $Z$ is an $R$-invariant closed subspace of $U$.
By the corresondence of
Lemma \ref{lemma-substacks-presentation}
(and its proof)
we obtain a closed substack $\mathcal{Z} \subset \mathcal{Z}$
with a presentation $[Z/R_Z] \to \mathcal{Z}$.
Then $|\mathcal{Z}| = |Z|/|R_Z| = |T'|/\sim$ is the given
closed subset $T$. We omit the proof of unicity.
\end{proof}

\begin{lemma}
\label{lemma-reduced-stack-determined-by-points}
Let $\mathcal{X}$ be an algebraic stack.
If $\mathcal{X}' \subset \mathcal{X}$
is a closed substack, $\mathcal{X}$ is reduced and
$|\mathcal{X}'| = |\mathcal{X}|$, then $\mathcal{X}' = \mathcal{X}$.
\end{lemma}

\begin{proof}
Choose a presentation $[U/R] \to \mathcal{X}$ with $U$ a scheme.
As $\mathcal{X}$ is reduced, we see that $U$ is reduced (by definition
of reduced algebraic stacks). By
Lemma \ref{lemma-substacks-presentation}
$\mathcal{X}'$ corresponds to an $R$-invariant closed subscheme $Z \subset U$.
But now $|Z| \subset |U|$ is the inverse image of $|\mathcal{X}'|$, and
hence $|Z| = |U|$. Hence $Z$ is a closed subscheme of $U$ whose underlying
sets of points agree. By
Schemes, Lemma \ref{schemes-lemma-map-into-reduction}
the map $\text{id}_U : U \to U$ factors through $Z \to U$, and hence
$Z = U$, i.e., $\mathcal{X}' = \mathcal{X}$.
\end{proof}

\begin{lemma}
\label{lemma-map-into-reduction}
Let $\mathcal{X}$, $\mathcal{Y}$ be algebraic stacks.
Let $\mathcal{Z} \subset \mathcal{X}$ be a closed substack
Assume $\mathcal{Y}$ is reduced.
A morphism $f : \mathcal{Y} \to \mathcal{X}$ factors through
$\mathcal{Z}$ if and only if
$f(|\mathcal{Y}|) \subset |\mathcal{Z}|$.
\end{lemma}

\begin{proof}
Assume $f(|\mathcal{Y}|) \subset |\mathcal{Z}|$. Consider
$\mathcal{Y} \times_{\mathcal{X}} \mathcal{Z} \to \mathcal{Y}$.
There is an equivalence
$\mathcal{Y} \times_{\mathcal{X}} \mathcal{Z} \to \mathcal{Y}'$
where $\mathcal{Y}'$ is a closed substack of $\mathcal{Y}$, see
Lemmas \ref{lemma-base-change-immersion} and
\ref{lemma-substack-image}.
Using
Lemmas \ref{lemma-points-cartesian},
\ref{lemma-monomorphism-injective-points}, and
\ref{lemma-immersion-monomorphism}
we see that $|\mathcal{Y}'| = |\mathcal{Y}|$. Hence we have
reduced the lemma to
Lemma \ref{lemma-reduced-stack-determined-by-points}.
\end{proof}

\begin{definition}
\label{definition-reduced-induced-stack}
Let $\mathcal{X}$ be an algebraic stack.
Let $Z \subset |\mathcal{X}|$ be a closed subset.
An {\it algebraic stack structure on $Z$} is given by a closed substack
$\mathcal{Z}$ of $\mathcal{X}$ with $|\mathcal{Z}|$ equal to $Z$.
The {\it reduced induced algebraic stack structure}
on $Z$ is the one constructed in
Lemma \ref{lemma-reduced-closed-substack}.
The {\it reduction $\mathcal{X}_{red}$ of $\mathcal{X}$}
is the reduced induced algebraic stack structure on $|\mathcal{X}|$.
\end{definition}

\noindent
In fact we can use this to define the reduced induced algebraic
stack structure on a locally closed subset.

\begin{remark}
\label{remark-stack-structure-locally-closed-subset}
Let $X$ be an algebraic stack.
Let $T \subset |\mathcal{X}|$ be a locally closed subset.
Let $\partial T$ be the boundary of $T$ in
the topological space $|\mathcal{X}|$. In a formula
$$
\partial T = \overline{T} \setminus T.
$$
Let $\mathcal{U} \subset \mathcal{X}$ be the open substack of $X$ with
$|\mathcal{U}| = |\mathcal{X}| \setminus \partial T$, see
Lemma \ref{lemma-open-substacks}.
Let $\mathcal{Z}$ be the reduced closed substack of $\mathcal{U}$ with
$|\mathcal{Z}| = T$ obtained by taking the reduced induced
closed subspace structure, see
Definition \ref{definition-reduced-induced-stack}.
By construction $\mathcal{Z} \to \mathcal{U}$ is a closed immersion of
algebraic stacks and $\mathcal{U} \to \mathcal{X}$ is an open immersion,
hence $\mathcal{Z} \to \mathcal{X}$ is an immersion of algebraic stacks by
Lemma \ref{lemma-composition-immersion}.
Note that $\mathcal{Z}$ is a reduced algebraic stack and that
$|\mathcal{Z}| = T$ as subsets of $|X|$. We sometimes say
$\mathcal{Z}$ is the {\it reduced induced substack structure} on $T$.
\end{remark}








\section{Other chapters}

\begin{multicols}{2}
\begin{enumerate}
\item \hyperref[introduction-section-phantom]{Introduction}
\item \hyperref[conventions-section-phantom]{Conventions}
\item \hyperref[sets-section-phantom]{Set Theory}
\item \hyperref[categories-section-phantom]{Categories}
\item \hyperref[topology-section-phantom]{Topology}
\item \hyperref[sheaves-section-phantom]{Sheaves on Spaces}
\item \hyperref[algebra-section-phantom]{Commutative Algebra}
\item \hyperref[sites-section-phantom]{Sites and Sheaves}
\item \hyperref[homology-section-phantom]{Homological Algebra}
\item \hyperref[derived-section-phantom]{Derived Categories}
\item \hyperref[more-algebra-section-phantom]{More Algebra}
\item \hyperref[simplicial-section-phantom]{Simplicial Methods}
\item \hyperref[modules-section-phantom]{Sheaves of Modules}
\item \hyperref[sites-modules-section-phantom]{Modules on Sites}
\item \hyperref[injectives-section-phantom]{Injectives}
\item \hyperref[cohomology-section-phantom]{Cohomology of Sheaves}
\item \hyperref[sites-cohomology-section-phantom]{Cohomology on Sites}
\item \hyperref[hypercovering-section-phantom]{Hypercoverings}
\item \hyperref[schemes-section-phantom]{Schemes}
\item \hyperref[constructions-section-phantom]{Constructions of Schemes}
\item \hyperref[properties-section-phantom]{Properties of Schemes}
\item \hyperref[morphisms-section-phantom]{Morphisms of Schemes}
\item \hyperref[coherent-section-phantom]{Coherent Cohomology}
\item \hyperref[divisors-section-phantom]{Divisors}
\item \hyperref[limits-section-phantom]{Limits of Schemes}
\item \hyperref[varieties-section-phantom]{Varieties}
\item \hyperref[chow-section-phantom]{Chow Homology}
\item \hyperref[topologies-section-phantom]{Topologies on Schemes}
\item \hyperref[descent-section-phantom]{Descent}
\item \hyperref[more-morphisms-section-phantom]{More on Morphisms}
\item \hyperref[flat-section-phantom]{More on Flatness}
\item \hyperref[groupoids-section-phantom]{Groupoid Schemes}
\item \hyperref[more-groupoids-section-phantom]{More on Groupoid Schemes}
\item \hyperref[etale-section-phantom]{\'Etale Morphisms of Schemes}
\item \hyperref[etale-cohomology-section-phantom]{\'Etale Cohomology}
\item \hyperref[spaces-section-phantom]{Algebraic Spaces}
\item \hyperref[spaces-properties-section-phantom]{Properties of Algebraic Spaces}
\item \hyperref[spaces-morphisms-section-phantom]{Morphisms of Algebraic Spaces}
\item \hyperref[spaces-topologies-section-phantom]{Topologies on Algebraic Spaces}
\item \hyperref[spaces-descent-section-phantom]{Descent and Algebraic Spaces}
\item \hyperref[spaces-more-morphisms-section-phantom]{More on Morphisms of Spaces}
\item \hyperref[quot-section-phantom]{Quot and Hilbert Spaces}
\item \hyperref[stacks-section-phantom]{Stacks}
\item \hyperref[spaces-groupoids-section-phantom]{Groupoids in Algebraic Spaces}
\item \hyperref[spaces-more-groupoids-section-phantom]{More on Groupoids in Spaces}
\item \hyperref[bootstrap-section-phantom]{Bootstrap}
\item \hyperref[examples-stacks-section-phantom]{Examples of Stacks}
\item \hyperref[groupoids-quotients-section-phantom]{Quotients of Groupoids}
\item \hyperref[algebraic-section-phantom]{Algebraic Stacks}
\item \hyperref[criteria-section-phantom]{Criteria for Representability}
\item \hyperref[stacks-properties-section-phantom]{Properties of Algebraic Stacks}
\item \hyperref[stacks-morphisms-section-phantom]{Morphisms of Algebraic Stacks}
\item \hyperref[examples-section-phantom]{Examples}
\item \hyperref[exercises-section-phantom]{Exercises}
\item \hyperref[guide-section-phantom]{Guide to Literature}
\item \hyperref[desirables-section-phantom]{Desirables}
\item \hyperref[coding-section-phantom]{Coding Style}
\item \hyperref[fdl-section-phantom]{GNU Free Documentation License}
\item \hyperref[index-section-phantom]{Auto Generated Index}
\end{enumerate}
\end{multicols}


\bibliography{my}
\bibliographystyle{amsalpha}

\end{document}
