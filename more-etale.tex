\IfFileExists{stacks-project.cls}{%
\documentclass{stacks-project}
}{%
\documentclass{amsart}
}

% The following AMS packages are automatically loaded with
% the amsart documentclass:
%\usepackage{amsmath}
%\usepackage{amssymb}
%\usepackage{amsthm}

% For dealing with references we use the comment environment
\usepackage{verbatim}
\newenvironment{reference}{\comment}{\endcomment}
%\newenvironment{reference}{}{}
\newenvironment{slogan}{\comment}{\endcomment}
\newenvironment{history}{\comment}{\endcomment}

% For commutative diagrams you can use
% \usepackage{amscd}
\usepackage[all]{xy}

% We use 2cell for 2-commutative diagrams.
\xyoption{2cell}
\UseAllTwocells

% To put source file link in headers.
% Change "template.tex" to "this_filename.tex"
% \usepackage{fancyhdr}
% \pagestyle{fancy}
% \lhead{}
% \chead{}
% \rhead{Source file: \url{template.tex}}
% \lfoot{}
% \cfoot{\thepage}
% \rfoot{}
% \renewcommand{\headrulewidth}{0pt}
% \renewcommand{\footrulewidth}{0pt}
% \renewcommand{\headheight}{12pt}

\usepackage{multicol}

% For cross-file-references
\usepackage{xr-hyper}

% Package for hypertext links:
\usepackage{hyperref}

% For any local file, say "hello.tex" you want to link to please
% use \externaldocument[hello-]{hello}
\externaldocument[introduction-]{introduction}
\externaldocument[conventions-]{conventions}
\externaldocument[sets-]{sets}
\externaldocument[categories-]{categories}
\externaldocument[topology-]{topology}
\externaldocument[sheaves-]{sheaves}
\externaldocument[sites-]{sites}
\externaldocument[stacks-]{stacks}
\externaldocument[fields-]{fields}
\externaldocument[algebra-]{algebra}
\externaldocument[brauer-]{brauer}
\externaldocument[homology-]{homology}
\externaldocument[derived-]{derived}
\externaldocument[simplicial-]{simplicial}
\externaldocument[more-algebra-]{more-algebra}
\externaldocument[smoothing-]{smoothing}
\externaldocument[modules-]{modules}
\externaldocument[sites-modules-]{sites-modules}
\externaldocument[injectives-]{injectives}
\externaldocument[cohomology-]{cohomology}
\externaldocument[sites-cohomology-]{sites-cohomology}
\externaldocument[dga-]{dga}
\externaldocument[dpa-]{dpa}
\externaldocument[hypercovering-]{hypercovering}
\externaldocument[schemes-]{schemes}
\externaldocument[constructions-]{constructions}
\externaldocument[properties-]{properties}
\externaldocument[morphisms-]{morphisms}
\externaldocument[coherent-]{coherent}
\externaldocument[divisors-]{divisors}
\externaldocument[limits-]{limits}
\externaldocument[varieties-]{varieties}
\externaldocument[topologies-]{topologies}
\externaldocument[descent-]{descent}
\externaldocument[perfect-]{perfect}
\externaldocument[more-morphisms-]{more-morphisms}
\externaldocument[flat-]{flat}
\externaldocument[groupoids-]{groupoids}
\externaldocument[more-groupoids-]{more-groupoids}
\externaldocument[etale-]{etale}
\externaldocument[chow-]{chow}
\externaldocument[intersection-]{intersection}
\externaldocument[pic-]{pic}
\externaldocument[adequate-]{adequate}
\externaldocument[dualizing-]{dualizing}
\externaldocument[duality-]{duality}
\externaldocument[discriminant-]{discriminant}
\externaldocument[local-cohomology-]{local-cohomology}
\externaldocument[curves-]{curves}
\externaldocument[resolve-]{resolve}
\externaldocument[models-]{models}
\externaldocument[pione-]{pione}
\externaldocument[etale-cohomology-]{etale-cohomology}
\externaldocument[proetale-]{proetale}
\externaldocument[crystalline-]{crystalline}
\externaldocument[spaces-]{spaces}
\externaldocument[spaces-properties-]{spaces-properties}
\externaldocument[spaces-morphisms-]{spaces-morphisms}
\externaldocument[decent-spaces-]{decent-spaces}
\externaldocument[spaces-cohomology-]{spaces-cohomology}
\externaldocument[spaces-limits-]{spaces-limits}
\externaldocument[spaces-divisors-]{spaces-divisors}
\externaldocument[spaces-over-fields-]{spaces-over-fields}
\externaldocument[spaces-topologies-]{spaces-topologies}
\externaldocument[spaces-descent-]{spaces-descent}
\externaldocument[spaces-perfect-]{spaces-perfect}
\externaldocument[spaces-more-morphisms-]{spaces-more-morphisms}
\externaldocument[spaces-flat-]{spaces-flat}
\externaldocument[spaces-groupoids-]{spaces-groupoids}
\externaldocument[spaces-more-groupoids-]{spaces-more-groupoids}
\externaldocument[bootstrap-]{bootstrap}
\externaldocument[spaces-pushouts-]{spaces-pushouts}
\externaldocument[groupoids-quotients-]{groupoids-quotients}
\externaldocument[spaces-more-cohomology-]{spaces-more-cohomology}
\externaldocument[spaces-simplicial-]{spaces-simplicial}
\externaldocument[formal-spaces-]{formal-spaces}
\externaldocument[restricted-]{restricted}
\externaldocument[spaces-resolve-]{spaces-resolve}
\externaldocument[formal-defos-]{formal-defos}
\externaldocument[defos-]{defos}
\externaldocument[cotangent-]{cotangent}
\externaldocument[examples-defos-]{examples-defos}
\externaldocument[algebraic-]{algebraic}
\externaldocument[examples-stacks-]{examples-stacks}
\externaldocument[stacks-sheaves-]{stacks-sheaves}
\externaldocument[criteria-]{criteria}
\externaldocument[artin-]{artin}
\externaldocument[quot-]{quot}
\externaldocument[stacks-properties-]{stacks-properties}
\externaldocument[stacks-morphisms-]{stacks-morphisms}
\externaldocument[stacks-limits-]{stacks-limits}
\externaldocument[stacks-cohomology-]{stacks-cohomology}
\externaldocument[stacks-perfect-]{stacks-perfect}
\externaldocument[stacks-introduction-]{stacks-introduction}
\externaldocument[stacks-more-morphisms-]{stacks-more-morphisms}
\externaldocument[stacks-geometry-]{stacks-geometry}
\externaldocument[moduli-]{moduli}
\externaldocument[moduli-curves-]{moduli-curves}
\externaldocument[examples-]{examples}
\externaldocument[exercises-]{exercises}
\externaldocument[guide-]{guide}
\externaldocument[desirables-]{desirables}
\externaldocument[coding-]{coding}
\externaldocument[obsolete-]{obsolete}
\externaldocument[fdl-]{fdl}
\externaldocument[index-]{index}

% Theorem environments.
%
\theoremstyle{plain}
\newtheorem{theorem}[subsection]{Theorem}
\newtheorem{proposition}[subsection]{Proposition}
\newtheorem{lemma}[subsection]{Lemma}

\theoremstyle{definition}
\newtheorem{definition}[subsection]{Definition}
\newtheorem{example}[subsection]{Example}
\newtheorem{exercise}[subsection]{Exercise}
\newtheorem{situation}[subsection]{Situation}

\theoremstyle{remark}
\newtheorem{remark}[subsection]{Remark}
\newtheorem{remarks}[subsection]{Remarks}

\numberwithin{equation}{subsection}

% Macros
%
\def\lim{\mathop{\rm lim}\nolimits}
\def\colim{\mathop{\rm colim}\nolimits}
\def\Spec{\mathop{\rm Spec}}
\def\Hom{\mathop{\rm Hom}\nolimits}
\def\Ext{\mathop{\rm Ext}\nolimits}
\def\SheafHom{\mathop{\mathcal{H}\!{\it om}}\nolimits}
\def\SheafExt{\mathop{\mathcal{E}\!{\it xt}}\nolimits}
\def\Sch{\textit{Sch}}
\def\Mor{\mathop{\rm Mor}\nolimits}
\def\Ob{\mathop{\rm Ob}\nolimits}
\def\Sh{\mathop{\textit{Sh}}\nolimits}
\def\NL{\mathop{N\!L}\nolimits}
\def\proetale{{pro\text{-}\acute{e}tale}}
\def\etale{{\acute{e}tale}}
\def\QCoh{\textit{QCoh}}
\def\Ker{\mathop{\rm Ker}}
\def\Im{\mathop{\rm Im}}
\def\Coker{\mathop{\rm Coker}}
\def\Coim{\mathop{\rm Coim}}

%
% Macros for moduli stacks/spaces
%
\def\QCohstack{\mathcal{QC}\!{\it oh}}
\def\Cohstack{\mathcal{C}\!{\it oh}}
\def\Spacesstack{\mathcal{S}\!{\it paces}}
\def\Quotfunctor{{\rm Quot}}
\def\Hilbfunctor{{\rm Hilb}}
\def\Curvesstack{\mathcal{C}\!{\it urves}}
\def\Polarizedstack{\mathcal{P}\!{\it olarized}}
\def\Complexesstack{\mathcal{C}\!{\it omplexes}}
% \Pic is the operator that assigns to X its picard group, usage \Pic(X)
% \Picardstack_{X/B} denotes the Picard stack of X over B
% \Picardfunctor_{X/B} denotes the Picard functor of X over B
\def\Pic{\mathop{\rm Pic}\nolimits}
\def\Picardstack{\mathcal{P}\!{\it ic}}
\def\Picardfunctor{{\rm Pic}}
\def\Deformationcategory{\mathcal{D}\!{\it ef}}


% OK, start here.
%
\begin{document}

\title{More \'Etale Cohomology}


\maketitle

\phantomsection
\label{section-phantom}

\tableofcontents


\section{Introduction}
\label{section-introduction}

\noindent
This chapter is the second in a series of chapter on the \'etale cohomology
of schemes. To read the first chapter, please visit
\'Etale Cohomology, Section \ref{etale-cohomology-section-introduction}.

\medskip\noindent
The split with the previous chapter is roughly speaking that anything
concerning ``shriek functors'' (cohomology with compact support and
its right adjoint) and anything using this material goes into this chapter.






\section{Growing sections}
\label{section-growing}

\noindent
In this section we discuss results of the following type.

\begin{lemma}
\label{lemma-section-support-in-locally-closed-pre}
Let $X$ be a scheme. Let $\mathcal{F}$ be an abelian sheaf on $X_\etale$.
Let $\varphi : U' \to U$ be a morphism of $X_\etale$. Let $Z' \subset U'$ be a
closed subscheme such that $Z' \to U' \to U$ is a closed immersion
with image $Z \subset U$. Then there is a canonical bijection
$$
\{s \in \mathcal{F}(U) \mid \text{Supp}(s) \subset Z\} =
\{s' \in \mathcal{F}(U') \mid \text{Supp}(s') \subset Z'\}
$$
which is given by restriction if $\varphi^{-1}(Z) = Z'$.
\end{lemma}

\begin{proof}
Consider the closed subscheme $Z'' = \varphi^{-1}(Z)$ of $U'$.
Then $Z' \subset Z''$ is closed because $Z'$ is closed in $U'$.
On the other hand, $Z' \to Z''$ is an \'etale morphism
(as a morphism between schemes \'etale over $Z$) and hence
open. Thus $Z'' = Z' \amalg T$ for some closed subset $T$.
The open covering $U' = (U' \setminus T) \cup (U' \setminus Z')$
shows that
$$
\{s' \in \mathcal{F}(U') \mid \text{Supp}(s') \subset Z'\} =
\{s' \in \mathcal{F}(U' \setminus T) \mid \text{Supp}(s') \subset Z'\}
$$
and the \'etale covering $\{U' \setminus T \to U, U \setminus Z \to U\}$
shows that
$$
\{s \in \mathcal{F}(U) \mid \text{Supp}(s) \subset Z\} =
\{s' \in \mathcal{F}(U' \setminus T) \mid \text{Supp}(s') \subset Z'\}
$$
This finishes the proof.
\end{proof}

\begin{lemma}
\label{lemma-section-support-in-locally-closed}
Let $X$ be a scheme. Let $Z \subset X$ be a locally closed subscheme.
Let $\mathcal{F}$ be an abelian sheaf on $X_\etale$. Given
$U, U' \subset X$ open containing $Z$ as a closed subscheme,
there is a canonical bijection
$$
\{s \in \mathcal{F}(U) \mid \text{Supp}(s) \subset Z\} =
\{s \in \mathcal{F}(U') \mid \text{Supp}(s) \subset Z\}
$$
which is given by restriction if $U' \subset U$.
\end{lemma}

\begin{proof}
Since $Z$ is a closed subscheme of $U \cap U'$, it suffices to
prove the lemma when $U' \subset U$. Then it is a special case
of Lemma \ref{lemma-section-support-in-locally-closed-pre}.
\end{proof}

\noindent
Let us introduce a bit of nonstandard notation which will stand us
in good stead later. Namely, in the situation of
Lemma \ref{lemma-section-support-in-locally-closed} above, let us denote
$$
H_Z(\mathcal{F}) = \{s \in \mathcal{F}(U) \mid \text{Supp}(s) \subset Z\}
$$
where $U \subset X$ is any choice of open subscheme containing $Z$ as a closed
subscheme. The reader who is troubled by the lack of precision this entails
may choose $U = X \setminus \partial Z$ where
$\partial Z = \overline{Z}\setminus Z$ is the ``boundary'' of $Z$ in $X$.
However, in many of the arguments below the flexibility of choosing
different opens will play a role. Here are some properties of this
construction:
\begin{enumerate}
\item
\label{item-inclusion}
If $Z \subset Z'$ are locally closed subschemes of $X$ and $Z$ is
closed in $Z'$, then there is a natural injective map
$$
H_Z(\mathcal{F}) \to H_{Z'}(\mathcal{F}).
$$
\item
\label{item-pullback}
If $f : Y \to X$ is a morphism of schemes and $Z \subset X$ is a locally
closed subscheme, then there is a natural
pullback map $f^* : H_Z(\mathcal{F}) \to H_{f^{-1}Z}(f^{-1}\mathcal{F})$.
\end{enumerate}
It will be convenient to extend our notation to the following situation:
suppose that we have $W \in X_\etale$ and a locally closed subscheme
$Z \subset W$. Then we will denote
$$
H_Z(\mathcal{F}) =
\{s \in \mathcal{F}(U) \mid \text{Supp}(s) \subset Z\} =
H_Z(\mathcal{F}|_{W_\etale})
$$
where $U \subset W$ is any choice of open subscheme containing $Z$
as a closed subscheme, exactly as above\footnote{In fact,
Lemma \ref{lemma-section-support-in-locally-closed-pre}
shows, given $Z$ over $X$ which is isomorphic to a locally closed
subscheme of some object $W$ of $X_\etale$, that
the choice of $W$ is irrelevant.}.







\section{Sections with compact support}
\label{section-compact-support}

\noindent
A reference for this section is \cite[Exposee XVII, Section 6]{SGA4}.
Let $f : X \to Y$ be a morphism of schemes which is separated and
locally of finite type. In this section we define a functor
$f_! : \textit{Ab}(X_\etale) \to \textit{Ab}(Y_\etale)$
by taking $f_!\mathcal{F} \subset f_*\mathcal{F}$
to be the subsheaf of sections which have proper support relative to $Y$
(suitably defined).

\medskip\noindent
Warning: The functor $f_!$ is the zeroth cohomology sheaf of a functor
$Rf_!$ on the derived category (insert future reference), but
$Rf_!$ is not the derived functor of $f_!$.

\begin{lemma}
\label{lemma-f-shriek-separated}
Let $f : X \to Y$ be a morphism of schemes which is locally of finite type.
Let $\mathcal{F}$ be an abelian sheaf on $X_\etale$. The rule
$$
Y_\etale \longrightarrow \textit{Ab},\quad
V \longmapsto \{s \in f_*\mathcal{F}(V) = \mathcal{F}(X_V) \mid
\text{Supp}(s) \subset X_V \text{ is proper over }V\}
$$
is an abelian subsheaf of $f_*\mathcal{F}$.
\end{lemma}

\noindent
Warning: This sheaf isn't the ``correct one'' if $f$ is not separated.

\begin{proof}
Recall that the support of a section is closed
(\'Etale Cohomology, Lemma \ref{etale-cohomology-lemma-support-section-closed})
hence the material in
Cohomology of Schemes, Section \ref{coherent-section-proper-over-base}
applies. By the lemma above and
Cohomology of Schemes, Lemma \ref{coherent-lemma-union-closed-proper-over-base}
we find that our subset of $f_*\mathcal{F}(V)$ is a subgroup.
By Cohomology of Schemes, Lemma
\ref{coherent-lemma-base-change-closed-proper-over-base}
we see that our rule defines a sub presheaf.
Finally, suppose that we have $s \in f_*\mathcal{F}(V)$
and an \'etale covering $\{V_i \to V\}$ such that
$s|_{V_i}$ has support proper over $V_i$.
Observe that the support of $s|_{V_i}$ is the inverse
image of the support of $s|_V$ (use the characterization
of the support in terms of stalks and
\'Etale Cohomology, Lemma \ref{etale-cohomology-lemma-stalk-pullback}).
Whence the support of $s$ is proper over $V$ by
Descent, Lemma \ref{descent-lemma-descending-property-proper-over-base}.
This proves that our rule satisfies the sheaf condition.
\end{proof}

\begin{lemma}
\label{lemma-separated-etale-shriek}
Let $j : U \to X$ be a separated \'etale morphism. Let $\mathcal{F}$
be an abelian sheaf on $U_\etale$. The image of the injective map
$j_!\mathcal{F} \to j_*\mathcal{F}$ of
\'Etale Cohomology, Lemma
\ref{etale-cohomology-lemma-shriek-into-star-separated-etale}
is the subsheaf of Lemma \ref{lemma-f-shriek-separated}.
\end{lemma}

\noindent
An alternative would be to move this lemma later and prove this
using the descrition of the stalks of both sheaves.

\begin{proof}
The construction of $j_!\mathcal{F} \to j_*\mathcal{F}$ in the proof of
\'Etale Cohomology, Lemma
\ref{etale-cohomology-lemma-shriek-into-star-separated-etale}
is via the construction of a map
$j_{p!}\mathcal{F} \to j_*\mathcal{F}$ of presheaves
whose image is clearly contained in the subsheaf of
Lemma \ref{lemma-f-shriek-separated}.
Hence since $j_!\mathcal{F}$ is the sheafification of
$j_{p!}\mathcal{F}$ we conclude the image of
$j_!\mathcal{F} \to j_*\mathcal{F}$ is contained in
this subsheaf. Conversely, let $s \in j_*\mathcal{F}(V)$
have support $Z$ proper over $V$. Then $Z \to V$ is
finite with closed image $Z' \subset V$, see
More on Morphisms, Lemma \ref{more-morphisms-lemma-characterize-finite}.
The restriction of $s$ to $V \setminus Z'$ is zero and the zero section is
contained in the image of $j_!\mathcal{F} \to j_*\mathcal{F}$.
On the other hand, if $v \in Z'$, then we can find
an \'etale neighbourhood
$(V', v') \to (V, v)$ such that we have a decomposition
$U_{V'} = W \amalg U'_1 \amalg \ldots \amalg U'_n$
into open and closed subschemes with $U'_i \to V'$ an isomorphism
and with $T_{V'} \subset U'_1 \amalg \ldots \amalg U'_n$, see
\'Etale Morphisms, Lemma \ref{etale-lemma-etale-etale-local-technical}.
Inverting the isomorphisms $U'_i \to V'$
we obtain $n$ morphisms $\varphi'_i : V' \to U$
and sections $s'_i$ over $V'$ by pulling back $s$.
Then the section $\sum (\varphi'_i, s'_i)$ of
$j_{p!}\mathcal{F}$ over $V'$, see formula for $j_{p!}\mathcal{F}(V')$
in proof of \'Etale Cohomology, Lemma
\ref{etale-cohomology-lemma-shriek-into-star-separated-etale},
maps to the restriction of $s$ to $V'$ by construction.
We conclude that $s$ is \'etale locally in the image
of $j_!\mathcal{F} \to j_*\mathcal{F}$ and the proof is complete.
\end{proof}

\begin{definition}
\label{definition-f-shriek-separated}
Let $f : X \to Y$ be a morphism of schemes which is separated (!) and
locally of finite type. Let $\mathcal{F}$ be an abelian sheaf on
$X_\etale$. The subsheaf $f_!\mathcal{F} \subset f_*\mathcal{F}$
constructed in Lemma \ref{lemma-f-shriek-separated} is called the
{\it direct image with compact support}.
\end{definition}

\noindent
By Lemma \ref{lemma-separated-etale-shriek} this does not conflict with
\'Etale Cohomology, Definition \ref{etale-cohomology-definition-extension-zero}
as we have agreement when both definitions apply. Here is a sanity check.

\begin{lemma}
\label{lemma-proper-f-shriek}
Let $f : X \to Y$ be a proper morphism of schemes.
Then $f_! = f_*$.
\end{lemma}

\begin{proof}
Immediate from the construction of $f_!$.
\end{proof}

\noindent
A very useful observation is the following.

\begin{remark}[Covariance with respect to open embeddings]
\label{remark-covariance-f-shriek-separated}
Let $f : X \to Y$ be morphism of schemes which is separated and
locally of finite type. Let $\mathcal{F}$ be an abelian sheaf on $X_\etale$.
Let $X' \subset X$ be an open subscheme. Denote $f' : X' \to Y$
the restriction of $f$. There is a canonical injective map
$$
f'_!(\mathcal{F}|_{X'}) \longrightarrow f_!\mathcal{F}
$$
Namely, let $V \in Y_\etale$ and consider a section
$s' \in f'_*(\mathcal{F}|_{X'})(V) = \mathcal{F}(X' \times_Y V)$
with support $Z'$ proper over $V$. Then $Z'$ is closed in $X \times_Y V$
as well, see Cohomology of Schemes, Lemma
\ref{coherent-lemma-functoriality-closed-proper-over-base}.
Thus there is a unique section
$s \in \mathcal{F}(X \times_Y V) = f_*\mathcal{F}(V)$
whose restriction to $X' \times_Y V$ is $s'$ and whose restriction
to $X \times_Y V \setminus Z'$ is zero, see
Lemma \ref{lemma-section-support-in-locally-closed}. This construction is
compatible with restriction maps and hence induces the desired map of
sheaves $f'_!(\mathcal{F}|_{X'}) \to f_!\mathcal{F}$ which is clearly
injective. By construction we obtain a commutative diagram
$$
\xymatrix{
f'_!(\mathcal{F}|_{X'}) \ar[r] \ar[d] &
f_!\mathcal{F} \ar[d] \\
f'_*(\mathcal{F}|_{X'}) &
f_*\mathcal{F} \ar[l]
}
$$
functorial in $\mathcal{F}$. It is clear that for $X'' \subset X'$ open
with $f'' = f|_{X''} : X'' \to Y$ the composition of the canonical maps
$f''_!\mathcal{F}|_{X''} \to f'_!\mathcal{F}|_{X'} \to f_!\mathcal{F}$
just constructed is the canonical map
$f''_!\mathcal{F}|_{X''} \to f_!\mathcal{F}$.
\end{remark}

\begin{lemma}
\label{lemma-compactify-f-shriek-separated}
Let $Y$ be a scheme. Let $j : X \to \overline{X}$ be an open
immersion of schemes over $Y$ with $\overline{X}$ proper over $Y$.
Denote $f : X \to Y$ and $\overline{f} : \overline{X} \to Y$
the structure morphisms. For $\mathcal{F} \in \textit{Ab}(X_\etale)$
there is a canonical isomorphism (see proof)
$$
f_!\mathcal{F} \longrightarrow \overline{f}_!j_!\mathcal{F}
$$
As we have $\overline{f}_! = \overline{f}_*$ by
Lemma \ref{lemma-proper-f-shriek} we obtain
$\overline{f}_* \circ j_! = f_!$ as functors
$\textit{Ab}(X_\etale) \to \textit{Ab}(Y_\etale)$.
\end{lemma}

\begin{proof}
We have $(j_!\mathcal{F})|_X = \mathcal{F}$, see
\'Etale Cohomology, Lemma \ref{etale-cohomology-lemma-jshriek-open}.
Thus the displayed arrow is the injective map
$f_!(\mathcal{G}|_X) \to \overline{f}_!\mathcal{G}$
of Remark \ref{remark-covariance-f-shriek-separated}
for $\mathcal{G} = j_!\mathcal{F}$. The explicit nature
of this map implies that it now suffices to show: if $V \in Y_\etale$ and
$s \in \overline{f}_!\mathcal{G}(V) = \overline{f}_*\mathcal{G}(V) =
\mathcal{G}(\overline{X}_V)$
is a section, then the support of $s$ is contained in the open
$X_V \subset \overline{X}_V$. This is immediate from the fact
that the stalks of $\mathcal{G}$ are zero at geometric
points of $\overline{X} \setminus X$.
\end{proof}

\noindent
We want to relate the stalks of $f_!\mathcal{F}$ to sections with
compact support on fibres. In order to state this, we need a definition.

\begin{definition}
\label{definition-compact-support}
Let $X$ be a separated scheme locally of finite type over a field $k$.
Let $\mathcal{F}$ be an abelian sheaf on $X_\etale$. We let
$H^0_c(X, \mathcal{F}) \subset H^0(X, \mathcal{F})$ be the
set of sections whose support is proper over $k$.
\end{definition}

\noindent
Warning: This definition isn't the ``correct one'' if $X$ isn't
separated over $k$.

\begin{lemma}
\label{lemma-proper-compact-support}
Let $X$ be a proper scheme over a field $k$. Then
$H^0_c(X, \mathcal{F}) = H^0(X, \mathcal{F})$.
\end{lemma}

\begin{proof}
Immediate from the construction of $H^0_c$.
\end{proof}

\begin{remark}[Open embeddings and compactly supported sections]
\label{remark-covariance-compact-support}
Let $X$ be a separated scheme locally of finite type over a field $k$.
Let $\mathcal{F}$ be an abelian sheaf on $X_\etale$.
Exactly as in Remark \ref{remark-covariance-f-shriek-separated}
there are injective maps
$$
H^0_c(X', \mathcal{F}|_{X'}) \longrightarrow H^0_c(X, \mathcal{F})
$$
which turn $H^0_c$ into a ``cosheaf'' on the Zariski site of $X$.
\end{remark}

\begin{lemma}
\label{lemma-compactify-compact-support}
Let $k$ be a field. Let $j : X \to \overline{X}$ be an open
immersion of schemes over $k$ with $\overline{X}$ proper over $k$.
For $\mathcal{F} \in \textit{Ab}(X_\etale)$
there is a canonical isomorphism (see proof)
$$
H^0_c(X, \mathcal{F}) \longrightarrow
H^0_c(\overline{X}, j_!\mathcal{F}) =
H^0(\overline{X}, j_!\mathcal{F})
$$
where we have the equality on the right by
Lemma \ref{lemma-proper-compact-support}.
\end{lemma}

\begin{proof}
We have $(j_!\mathcal{F})|_X = \mathcal{F}$, see
\'Etale Cohomology, Lemma \ref{etale-cohomology-lemma-jshriek-open}.
Thus the displayed arrow is the injective map
$H^0_c(X, \mathcal{G}|_X) \to H^0_c(\overline{X}, \mathcal{G})$
of Remark \ref{remark-covariance-compact-support}
for $\mathcal{G} = j_!\mathcal{F}$. The explicit nature
of this map implies that it now suffices to show: if
$s \in H^0(\overline{X}, \mathcal{G})$ is a section, then the support of
$s$ is contained in the open $X$. This is immediate from the fact
that the stalks of $\mathcal{G}$ are zero at geometric
points of $\overline{X} \setminus X$.
\end{proof}

\begin{lemma}
\label{lemma-stalk-f-shriek-separated}
Let $f : X \to Y$ be a morphism of schemes which is separated and
locally of finite type. Let $\mathcal{F}$ be an abelian sheaf on
$X_\etale$. Then there is a canonical isomorphism
$$
(f_!\mathcal{F})_{\overline{y}}
\longrightarrow
H^0_c(X_{\overline{y}}, \mathcal{F}|_{X_{\overline{y}}})
$$
for any geometric point $\overline{y} : \Spec(k) \to Y$.
\end{lemma}

\begin{proof}
Recall that $(f_*\mathcal{F})_{\overline{y}} = \colim f_*\mathcal{F}(V)$
where the colimit is over the \'etale neighbourhoods $(V, \overline{v})$
of $\overline{y}$. If $s \in f_*\mathcal{F}(V) = \mathcal{F}(X_V)$,
then we can pullback $s$ to a section of $\mathcal{F}$ over
$(X_V)_{\overline{v}} = X_{\overline{y}}$. Thus we obtain a canonical map
$$
c_{\overline{y}} :
(f_*\mathcal{F})_{\overline{y}}
\longrightarrow 
H^0(X_{\overline{y}}, \mathcal{F}|_{X_{\overline{y}}})
$$
We claim that this map induces a bijection between the subgroups
$(f_!\mathcal{F})_{\overline{y}}$ and
$H^0_c(X_{\overline{y}}, \mathcal{F}|_{X_{\overline{y}}})$.
The claim implies the lemma, but is a little bit more precise
in that it describes the identification of the lemma as given
by pullbacks of sections of $\mathcal{F}$ to the geometric fibre of $f$.

\medskip\noindent
Observe that any element
$s \in (f_!\mathcal{F})_{\overline{y}} \subset (f_*\mathcal{F})_{\overline{y}}$
is mapped by $c_{\overline{y}}$ to an element of
$H^0_c(X_{\overline{y}}, \mathcal{F}|_{X_{\overline{y}}}) \subset
H^0(X_{\overline{y}}, \mathcal{F}|_{X_{\overline{y}}})$.
This is true because taking the support of a section
commutes with pullback and because properness is preserved by
base change. This at least produces the map in the statement of the lemma.
To prove that it is an isomorphism we may work Zariski
locally on $Y$ and hence we may and do assume $Y$ is affine.

\medskip\noindent
An observation that we will use below
is that given an open subscheme $X' \subset X$
and if $f' = f|_{X'}$, then we obtain a commutative diagram
$$
\xymatrix{
(f'_!(\mathcal{F}|_{X'}))_{\overline{y}} \ar[r] \ar[d] &
H^0_c(X'_{\overline{y}}, \mathcal{F}|_{X'_{\overline{y}}}) \ar[d] \\
(f_!\mathcal{F})_{\overline{y}} \ar[r] &
H^0_c(X_{\overline{y}}, \mathcal{F}|_{X_{\overline{y}}})
}
$$
where the horizontal arrows are the maps constructed above and
the vertical arrows are given in
Remarks \ref{remark-covariance-f-shriek-separated} and
\ref{remark-covariance-compact-support}.
The reason is that given an \'etale neighbourhood $(V, \overline{v})$
of $\overline{y}$ and a section $s \in f_*\mathcal{F}(V) = \mathcal{F}(X_V)$
whose support $Z$ happens to be contained in $X'_V$ and is proper over $V$,
so that $s$ gives rise to an element of both
$(f'_!(\mathcal{F}|_{X'}))_{\overline{y}}$ and
$(f_!\mathcal{F})_{\overline{y}}$ which correspond via
the vertical arrow of the diagram, then these elements are mapped via the
horizontal arrows to the pullback $s|_{X_{\overline{y}}}$ of $s$ to
$X_{\overline{y}}$ whose support $Z_{\overline{y}}$ is contained in
$X'_{\overline{y}}$ and hence this restriction gives rise to
a compatible pair of elements of
$H^0_c(X'_{\overline{y}}, \mathcal{F}|_{X'_{\overline{y}}})$
and
$H^0_c(X_{\overline{y}}, \mathcal{F}|_{X_{\overline{y}}})$.

\medskip\noindent
Suppose $s \in (f_!\mathcal{F})_{\overline{y}}$ maps to zero in
$H^0_c(X_{\overline{y}}, \mathcal{F}|_{X_{\overline{y}}})$.
Say $s$ corresponds to $s \in f_*\mathcal{F}(V) = \mathcal{F}(X_V)$
with support $Z$ proper over $V$. We may assume that $V$ is affine
and hence $Z$ is quasi-compact. Then we may choose a quasi-compact open
$X' \subset X$ containing the image of $Z$. Then $Z$ is contained in
$X'_V$ and hence $s$ is the image of an element
$s' \in f'_!(\mathcal{F}|_{X'})(V)$ where $f' = f|_{X'}$ as in
the previous paragraph. Then $s'$ maps to zero in
$H^0_c(X'_{\overline{y}}, \mathcal{F}|_{X'_{\overline{y}}})$.
Hence in order to prove injectivity, we may replace $X$ by
$X'$, i.e., we may assume $X$ is quasi-compact. We will prove
this case below.

\medskip\noindent
Suppose that
$t \in H^0_c(X_{\overline{y}}, \mathcal{F}|_{X_{\overline{y}}})$.
Then the support of $t$ is contained in a quasi-compact
open subscheme $W \subset X_{\overline{y}}$.
Hence we can find a quasi-compact open subscheme
$X' \subset X$ such that $X'_{\overline{y}}$ contains $W$.
Then it is clear that $t$ is contained in the image
of the injective map
$H^0_c(X'_{\overline{y}}, \mathcal{F}|_{X'_{\overline{y}}}) \to
H^0_c(X_{\overline{y}}, \mathcal{F}|_{X_{\overline{y}}})$.
Hence in order to show surjectivity, we may replace $X$
by $X'$, i.e., we may assume $X$ is quasi-compact.
We will prove this case below.

\medskip\noindent
In this last paragraph of the proof we prove the lemma in case
$X$ is quasi-compact and $Y$ is affine. By More on Flatness, Theorem
\ref{flat-theorem-nagata} there exists a compactification
$j : X \to \overline{X}$ over $Y$. Set $\mathcal{G} = j_!\mathcal{F}$
so that $\mathcal{F} = \mathcal{G}|_X$ by
\'Etale Cohomology, Lemma \ref{etale-cohomology-lemma-jshriek-open}.
By the disussion above we get a commutative diagram
$$
\xymatrix{
(f_!\mathcal{F})_{\overline{y}} \ar[r] \ar[d] &
H^0_c(X_{\overline{y}}, \mathcal{F}|_{X_{\overline{y}}}) \ar[d] \\
(\overline{f}_!\mathcal{G})_{\overline{y}} \ar[r] &
H^0_c(\overline{X}_{\overline{y}}, \mathcal{G}|_{\overline{X}_{\overline{y}}})
}
$$
By Lemmas \ref{lemma-compactify-f-shriek-separated} and
\ref{lemma-compactify-compact-support} the vertical maps
are isomorphisms. This reduces us to the case of the proper
morphism $\overline{X} \to Y$. For a proper morphism our map
is an isomorphism by
Lemmas \ref{lemma-proper-f-shriek} and \ref{lemma-proper-compact-support}
and proper base change for pushforwards, see
\'Etale Cohomology, Lemma
\ref{etale-cohomology-lemma-proper-pushforward-stalk}.
\end{proof}

\begin{lemma}
\label{lemma-base-change-f-shriek-separated}
Consider a cartesian square
$$
\xymatrix{
X' \ar[r]_{g'} \ar[d]_{f'} & X \ar[d]^f \\
Y' \ar[r]^g & Y
}
$$
of schemes with $f$ separated and locally of finite type.
For any abelian sheaf $\mathcal{F}$ on $X_\etale$ we have
$f'_!(g')^{-1}\mathcal{F} = g^{-1}f_!\mathcal{F}$.
\end{lemma}

\begin{proof}
In great generality there is a pullback map
$g^{-1}f_*\mathcal{F} \to f'_*(g')^{-1}\mathcal{F}$, see
Sites, Section \ref{sites-section-pullback}.
We claim that this map sends $g^{-1}f_!\mathcal{F}$
into the subsheaf $f'_!(g')^{-1}\mathcal{F}$
and induces the isomorphism in the lemma.

\medskip\noindent
Choose a geometric point $\overline{y}': \Spec(k) \to Y'$ and denote
$\overline{y} = g \circ \overline{y}'$ the image in $Y$. There is a
commutative diagram
$$
\xymatrix{
(f_*\mathcal{F})_{\overline{y}} \ar[r] \ar[d] &
H^0(X_{\overline{y}}, \mathcal{F}|_{X_{\overline{y}}}) \ar[d] \\
(f'_*(g')^{-1}\mathcal{F})_{\overline{y}'} \ar[r] &
H^0(X'_{\overline{y}'}, (g')^{-1}\mathcal{F}|_{X'_{\overline{y}'}})
}
$$
where the horizontal maps were used in the proof of
Lemma \ref{lemma-stalk-f-shriek-separated}
and the vertical maps are the pullback maps above.
The diagram commutes because each of the four maps
in question is given by pulling back local sections along
a morphism of schemes and the underlying diagram of morphisms
of schemes commutes. Since the diagram in the statement of the lemma
is cartesian we have $X'_{\overline{y}'} = X_{\overline{y}}$.
Hence by Lemma \ref{lemma-stalk-f-shriek-separated}
and its proof we obtain a commutative diagram
$$
\xymatrix{
(f_*\mathcal{F})_{\overline{y}} \ar[rrr] \ar[ddd] & & &
H^0(X_{\overline{y}}, \mathcal{F}|_{X_{\overline{y}}}) \ar[ddd] \\
& (f_!\mathcal{F})_{\overline{y}} \ar[r] \ar@{..>}[d] \ar[lu] &
H^0_c(X_{\overline{y}}, \mathcal{F}|_{X_{\overline{y}}}) \ar[d] \ar[ru] \\
& (f'_!(g')^{-1}\mathcal{F})_{\overline{y}'} \ar[r] \ar[ld] &
H^0_c(X'_{\overline{y}'}, (g')^{-1}\mathcal{F}|_{X'_{\overline{y}'}}) \ar[rd]\\
(f'_*(g')^{-1}\mathcal{F})_{\overline{y}'} \ar[rrr] & & &
H^0(X'_{\overline{y}'}, (g')^{-1}\mathcal{F}|_{X'_{\overline{y}'}})
}
$$
where the horizontal arrows of the inner square are isomorphisms
and the two right vertical arrows are equalities. Also, the
se, sw, ne, nw arrows are injective. It follows that there is a unique
bijective dotted arrow fitting into the diagram. We conclude that
$g^{-1}f_!\mathcal{F} \subset g^{-1}f_*\mathcal{F} \to f'_*(g')^{-1}\mathcal{F}$
is mapped into the subsheaf
$f'_!(g')^{-1}\mathcal{F} \subset f'_*(g')^{-1}\mathcal{F}$
because this is true on stalks, see
\'Etale Cohomology, Theorem \ref{etale-cohomology-theorem-exactness-stalks}.
The same theorem then implies that the induced map is an isomorphism
and the proof is complete.
\end{proof}

\begin{lemma}
\label{lemma-f-shriek-composition}
Let $f : X \to Y$ and $g : Y \to Z$ be composable morphisms of schemes which
are separated and locally of finite type. Let $\mathcal{F}$ be an abelian
sheaf on $X_\etale$. Then $g_!f_!\mathcal{F} = (g \circ f)_!\mathcal{F}$
as subsheaves of $(g \circ f)_*\mathcal{F}$.
\end{lemma}

\begin{proof}
We strongly urge the reader to prove this for themselves.
Let $W \in Z_\etale$ and
$s \in (g \circ f)_*\mathcal{F}(W) = \mathcal{F}(X_W)$.
Denote $T \subset X_W$ the support of $s$; this is a closed
subset. Observe that $s$ is a section of $(g \circ f)_!\mathcal{F}$
if and only if $T$ is proper over $W$. We have
$f_!\mathcal{F} \subset f_*\mathcal{F}$ and hence
$g_!f_!\mathcal{F} \subset g_!f_*\mathcal{F} \subset g_*f_*\mathcal{F}$.
On the other hand, $s$ is a section of $g_!f_!\mathcal{F}$ if and only
if (a) $T$ is proper over $Y_W$ and (b) the support $T'$ of $s$
viewed as section of $f_!\mathcal{F}$ is proper over $W$.
If (a) holds, then the image of $T$ in $Y_W$ is closed and since
$f_!\mathcal{F} \subset f_*\mathcal{F}$ we see that
$T' \subset Y_W$ is the image of $T$ (details omitted; look at stalks).

\medskip\noindent
The conclusion is that we have to show a closed subset $T \subset X_W$
is proper over $W$ if and only if $T$ is proper over $Y_W$
and the image of $T$ in $Y_W$ is proper over $W$. Let us endow $T$
with the reduced induced closed subscheme structure.
If $T$ is proper over $W$, then $T \to Y_W$ is proper by
Morphisms, Lemma \ref{morphisms-lemma-image-proper-scheme-closed}
and the image of $T$ in $Y_W$ is proper over $W$ by
Cohomology of Schemes, Lemma
\ref{coherent-lemma-functoriality-closed-proper-over-base}.
Conversely, if $T$ is proper over $Y_W$
and the image of $T$ in $Y_W$ is proper over $W$,
then the morphism $T \to W$ is proper as a composition
of proper morphisms (here we endow the closed image of $T$
in $Y_W$ with its reduced induced scheme structure to turn the
question into one about morphisms of schemes), see
Morphisms, Lemma \ref{morphisms-lemma-composition-proper}.
\end{proof}

\begin{remark}
\label{remark-f-shriek-base-change-composition}
The isomorphisms between functors
constructed above satisfy the following two properties:
\begin{enumerate}
\item Let $f : X \to Y$, $g : Y \to Z$, and $h : Z \to T$ be composable
morphisms of schemes which are separated and locally of finite type.
Then the diagram
$$
\xymatrix{
(h \circ g \circ f)_! \ar[r] \ar[d] &
(h \circ g)_! \circ f_! \ar[d] \\
h_! \circ (g \circ f)_! \ar[r] &
h_! \circ g_! \circ f_!
}
$$
commutes where the arrows are those of Lemma \ref{lemma-f-shriek-composition}.
\item Suppose that we have a diagram of schemes
$$
\xymatrix{
X' \ar[d]_{f'} \ar[r]_c & X \ar[d]^f \\
Y' \ar[d]_{g'} \ar[r]_b & Y \ar[d]^g \\
Z' \ar[r]^a & Z
}
$$
with both squares cartesian and $f$ and $g$ separated and
locally of finite type. Then the diagram
$$
\xymatrix{
a^{-1} \circ (g \circ f)_! \ar[d] \ar[rr] & &
(g' \circ f')_! \circ c^{-1} \ar[d] \\
a^{-1} \circ g_! \circ f_! \ar[r] &
g'_! \circ b^{-1} \circ f_! \ar[r] &
g'_! \circ f'_! \circ c^{-1}
}
$$
commutes where the horizontal arrows are those of
Lemma \ref{lemma-base-change-f-shriek-separated}
the arrows are those of Lemma \ref{lemma-f-shriek-composition}.
\end{enumerate}
Part (1) holds true because we have a similar commutative
diagram for pushforwards. Part (2) holds by the very general
compatibility of base change maps for pushforwards
(Sites, Remark \ref{sites-remark-compose-base-change})
and the fact that the isomorphisms in
Lemmas \ref{lemma-base-change-f-shriek-separated} and
\ref{lemma-f-shriek-composition}
are constructed using the corresponding maps fo pushforwards.
\end{remark}

\begin{lemma}
\label{lemma-colim-f-shriek-separated}
Let $f : X \to Y$ be morphism of schemes which is separated and
locally of finite type. Let $X = \bigcup_{i \in I} X_i$ be an
open covering such that for all $i, j \in I$ there exists a $k$
with $X_i \cup X_j \subset X_k$. Denote $f_i : X_i \to Y$
the restriction of $f$. Then
$$
f_!\mathcal{F} = \colim_{i \in I} f_{i, !}(\mathcal{F}|_{X_i})
$$
functorially in $\mathcal{F} \in \textit{Ab}(X_\etale)$
where the transition maps are the ones constructed in
Remark \ref{remark-covariance-f-shriek-separated}.
\end{lemma}

\begin{proof}
It suffices to show that the canonical map from
right to left is a bijection when evaluated on a quasi-compact
object $V$ of $Y_\etale$.
Observe that the colimit on the right hand side is directed
and has injective transition maps.
Thus we can use
Sites, Lemma \ref{sites-lemma-directed-colimits-sections}
to evaluate the colimit. Hence, the statement comes down
to the observation that a closed subset $Z \subset X_V$ proper over $V$
is quasi-compact and hence is contained in $X_{i, V}$ for some $i$.
\end{proof}

\begin{lemma}
\label{lemma-f-shriek-separated-direct-sums}
Let $f : X \to Y$ be a morphism of schemes which is separated and
locally of finite type. Then functor $f_!$ commutes with direct sums.
\end{lemma}

\begin{proof}
Let $\mathcal{F} = \bigoplus \mathcal{F}_i$. To show that the map
$\bigoplus f_!\mathcal{F}_i \to f_!\mathcal{F}$ is an isomorphism,
it suffices to show that these sheaves have the same sections over
a quasi-compact object $V$ of $Y_\etale$. Replacing $Y$ by $V$
it suffices to show
$H^0(Y, f_!\mathcal{F}) \subset H^0(X, \mathcal{F})$
is equal to
$\bigoplus H^0(Y, f_!\mathcal{F}_i)
\subset \bigoplus H^0(X, \mathcal{F}_i)
\subset H^0(X, \bigoplus \mathcal{F}_i)$.
In this case, by writing $X$ as the union of its quasi-compact opens
and using Lemma \ref{lemma-colim-f-shriek-separated}
we reduce to the case where $X$ is quasi-compact as well.
Then $H^0(X, \mathcal{F}) = \bigoplus H^0(X, \mathcal{F}_i)$
by \'Etale Cohomology, Theorem \ref{etale-cohomology-theorem-colimit}.
Looking at supports of sections the reader easily concludes.
\end{proof}

\begin{lemma}
\label{lemma-lqf-f-shriek-separated-colimits}
Let $f : X \to Y$ be a morphism of schemes which is separated and
locally quasi-finite. Then
\begin{enumerate}
\item for $\mathcal{F}$ in $\textit{Ab}(X_\etale)$ and a geometric
point $\overline{y} : \Spec(k) \to Y$ we have
$$
(f_!\mathcal{F})_{\overline{y}} =
\bigoplus\nolimits_{f(\overline{x}) = \overline{y}} \mathcal{F}_{\overline{x}}
$$
functorially in $\mathcal{F}$, and
\item the functor $f_!$ is exact.
\end{enumerate}
\end{lemma}

\begin{proof}
The functor $f_!$ is left exact by construction. Right exactness may
be checked on stalks
(\'Etale Cohomology, Theorem \ref{etale-cohomology-theorem-exactness-stalks}).
Thus it suffices to prove part (1).

\medskip\noindent
Let $\overline{y} : \Spec(k) \to Y$ be a geometric point.
The scheme $X_{\overline{y}}$ has a discrete underlying
topological space
(Morphisms, Lemma \ref{morphisms-lemma-locally-quasi-finite-fibres})
and all the residue fields at the points are equal to $k$
(as finite extensions of $k$). Hence
$\{\overline{x} : \Spec(k) \to X : f(\overline{x}) = \overline{y}\}$
is equal to the set of points of $X_{\overline{y}}$.
Thus the computation of the stalk follows from the more general
Lemma \ref{lemma-stalk-f-shriek-separated}.
\end{proof}









\section{Sections with finite support}
\label{section-finite-support}

\noindent
In this section we extend the construction of
Section \ref{section-compact-support} to not necessarily
separated locally quasi-finite morphisms.

\medskip\noindent
Let $f : X \to Y$ be a locally quasi-finite morphism of schemes.
Let $\mathcal{F}$ be an abelian sheaf on $X_\etale$. Given $V$ in
$Y_\etale$ denote $X_V = X \times_Y V$ the base change. We are going
to consider the group of finite formal sums
\begin{equation}
\label{equation-formal-sum}
s = \sum\nolimits_{i = 1, \ldots, n} (Z_i, s_i)
\end{equation}
where $Z_i \subset X_V$ is a locally closed subscheme such that the
morphism $Z_i \to V$ is finite\footnote{Since $f$ is locally quasi-finite,
the morphism $Z_i \to V$ is finite if and only if it is proper.}
and where $s_i \in H_{Z_i}(\mathcal{F})$. Here, as in
Section \ref{section-growing}, we set
$$
H_{Z_i}(\mathcal{F}) =
\{s \in \mathcal{F}(U_i) \mid \text{Supp}(s) \subset Z_i\}
$$
where $U_i \subset X_V$ is an open subscheme containing $Z$ as a
closed subscheme. We are going to consider these formal sums modulo the
following relations
\begin{enumerate}
\item
\label{item-sum}
$(Z, s) + (Z, s') = (Z, s + s')$,
\item
\label{item-sub}
$(Z, s) = (Z', s)$ if $Z \subset Z'$.
\end{enumerate}
Observe that the second relation makes sense: since $Z \to V$ is finite
and $Z' \to V$ is separated, the inclusion $Z \to Z'$ is closed and we
can use the map discussed in (\ref{item-inclusion}).

\medskip\noindent
Let us denote $f_{p!}\mathcal{F}(V)$ the quotient of the abelian
group of formal sums (\ref{equation-formal-sum}) by these relations.
The first relation tells us that $f_{p!}\mathcal{F}(V)$ is a quotient
of the direct sum of the abelian groups $H_Z(\mathcal{F})$
over all locally closed subschemes $Z \subset X_V$ finite over $V$.
The second relation tells us that we are really taking the colimit
\begin{equation}
\label{equation-colimit-definition}
f_{p!}\mathcal{F}(V) = \colim_Z H_Z(\mathcal{F})
\end{equation}
This formula will be a convenient abstract way to think about
our construction.

\medskip\noindent
Next, we observe that there is a natural way to turn this construction
into a presheaf $f_{p!}\mathcal{F}$ of abelian groups on $Y_\etale$.
Namely, given $V' \to V$ in $Y_\etale$ we obtain the base change morphism
$X_{V'} \to X_V$. If $Z \subset X_V$ is a locally closed subscheme
finite over $V$, then the scheme theoretic inverse image $Z' \subset X_{V'}$
is finite over $V'$. Moreover, if $U \subset X_V$ is an open such
that $Z$ is closed in $U$, then the inverse image $U' \subset X_{V'}$
is an open such that $Z'$ is closed in $U'$. Hence the restriction
mapping $\mathcal{F}(U) \to \mathcal{F}(U')$ of $\mathcal{F}$
sends $H_Z(\mathcal{F})$ into $H_{Z'}(\mathcal{F})$; this is a special
case of the functoriality discussed in (\ref{item-pullback}) above.
Clearly, these maps are compatible with inclusions
$Z_1 \subset Z_2$ of such locally closed subschemes of $X_V$ and
we obtain a map
$$
f_{p!}\mathcal{F}(V) = \colim_Z H_Z(\mathcal{F})
\longrightarrow
\colim_{Z'} H_{Z'}(\mathcal{F}) =
f_{p!}\mathcal{F}(V')
$$
These maps indeed turn $f_{p!}\mathcal{F}$ into a presheaf of abelian
groups on $Y_\etale$. We omit the details.

\medskip\noindent
A final observation is that the construction of $f_{p!}\mathcal{F}$
is functorial in $\mathcal{F}$ in $\textit{Ab}(X_\etale)$.
We conclude that given a locally quasi-finite morphism $f : X \to Y$
we have constructed a functor
$$
f_{p!} :
\textit{Ab}(X_\etale)
\longrightarrow
\textit{PAb}(Y_\etale)
$$
from the category of abelian sheaves on $X_\etale$ to the category
of abelian presheaves on $Y_\etale$. Before we define $f_!$ as the
sheafification of this functor, let us check that it agrees with
the construction in Section \ref{section-compact-support}
and with the construction in
\'Etale Cohomology, Section \ref{etale-cohomology-section-extension-by-zero}
when both apply.

\begin{lemma}
\label{lemma-finite-support-f-shriek-separated}
Let $f : X \to Y$ be a separated and locally quasi-finite morphism
of schemes. Functorially in $\mathcal{F} \in \textit{Ab}(X_\etale)$
there is a canonical isomorphism(!)
$$
f_{p!}\mathcal{F} \longrightarrow f_!\mathcal{F}
$$
of abelian presheaves which identifies the sheaf
$f_!\mathcal{F}$ of Definition \ref{definition-f-shriek-separated}
with the presheaf $f_{p!}\mathcal{F}$ constructed above.
\end{lemma}

\begin{proof}
Let $V$ be an object of $Y_\etale$. If $Z \subset X_V$ is locally closed
and finite over $V$, then, since $f$ is separated, we see that
the morphism $Z \to X_V$ is a closed immersion. Moreover, if
$Z_i$, $i = 1, \ldots, n$ are closed subschemes of $X_V$ finite
over $V$, then $Z_1 \cup \ldots \cup Z_n$ (scheme theoretic union)
is a closed subscheme finite over $V$. Hence in this case the colimit
(\ref{equation-colimit-definition}) defining $f_{p!}\mathcal{F}(V)$
is directed and we find that $f_{!p}\mathcal{F}(V)$ is simply equal
to the set of sections of $\mathcal{F}(X_V)$ whose support is finite over $V$.
Since any closed subset of $X_V$ which is proper over $V$ is
actually finite over $V$ (as $f$ is locally quasi-finite)
we conclude that this is equal to $f_!\mathcal{F}(V)$
by its very definition.
\end{proof}

\begin{lemma}
\label{lemma-finite-support-stalk}
Let $f : X \to Y$ be a morphism of schemes which is locally quasi-finite.
Let $\overline{y} : \Spec(k) \to Y$ be a geometric point.
Functorially in $\mathcal{F}$ in $\textit{Ab}(X_\etale)$ we have
$$
(f_{p!}\mathcal{F})_{\overline{y}} =
\bigoplus\nolimits_{f(\overline{x}) = \overline{y}} \mathcal{F}_{\overline{x}}
$$
\end{lemma}

\begin{proof}
Recall that the stalk at $\overline{y}$ of a presheaf is defined by the
usual colimit over \'etale neighbourhoods $(V, \overline{v})$
of $\overline{y}$, see \'Etale Cohomology, Definition
\ref{etale-cohomology-definition-stalk}. Accordingly
suppose $s = \sum_{i = 1, \ldots, n} (Z_i, s_i)$ as in
(\ref{equation-formal-sum}) is an element of $f_{p!}\mathcal{F}(V)$
where $(V, \overline{v})$ is an \'etale neighbourhood of $\overline{y}$.
Then since
$$
X_{\overline{y}} = (X_V)_{\overline{v}} \supset Z_{i, \overline{v}}
$$
and since $s_i$ is a section of $\mathcal{F}$ on an open neighbourhood
of $Z_i$ in $X_V$ we can send $s$ to
$$
\sum\nolimits_{i = 1, \ldots, n} 
\sum\nolimits_{\overline{x} \in Z_{i, \overline{v}}}
\left(\text{class of }s_i\text{ in }\mathcal{F}_{\overline{x}}\right)
\quad\in\quad
\bigoplus\nolimits_{f(\overline{x}) = \overline{y}} \mathcal{F}_{\overline{x}}
$$
We omit the verification that this is compatible with restriction
maps and that the relations (\ref{item-sum}) $(Z, s) + (Z, s') - (Z, s + s')$
and (\ref{item-sub}) $(Z, s) - (Z', s)$ if $Z \subset Z'$ are sent to zero.
Thus we obtain a map
$$
(f_{p!}\mathcal{F})_{\overline{y}}
\longrightarrow
\bigoplus\nolimits_{f(\overline{x}) = \overline{y}} \mathcal{F}_{\overline{x}}
$$

\medskip\noindent
Let us prove this arrow is surjective. For this it suffices to pick
an $\overline{x}$ with $f(\overline{x}) = \overline{y}$ and prove that
an element $s$ in the summand $\mathcal{F}_{\overline{x}}$ is in the
image. Let $s$ correspond to the element $s \in \mathcal{F}(U)$
where $(U, \overline{u})$ is an \'etale neighbourhood of $\overline{x}$.
Since $f$ is locally quasi-finite, the morphism $U \to Y$
is locally quasi-finite too. By More on Morphisms, Lemma
\ref{more-morphisms-lemma-etale-makes-quasi-finite-finite-multiple-points-var}
we can find an \'etale neighbourhood $(V, \overline{v})$ of
$\overline{y}$, an open subscheme
$$
W \subset U \times_Y V,
$$
and a geometric point $\overline{w}$ mapping to $\overline{u}$ and
$\overline{v}$ such that $W \to V$ is finite and $\overline{w}$ is the
only geometric point of $W$ mapping to $\overline{v}$. (We omit the translation
between the language of geometric points we are currently using and the
language of points and residue field extensions used in the
statement of the lemma.) Observe that $W \to X_V = X \times_Y V$
is \'etale. Choose an affine open neighbourhood $W' \subset X_V$
of the image $\overline{w}'$ of $\overline{w}$. Since $\overline{w}$
is the only point of $W$ over $\overline{v}$ and since $W \to V$
is closed, after replacing $V$ by an open neighbourhood of $\overline{v}$,
we may assume $W \to X_V$ maps into $W'$. Then $W \to W'$ is finite and
\'etale and there is a unique geometric point $\overline{w}$ of $W$
lying over $\overline{w}'$. It follows that $W \to W'$ is an open immersion
over an open neighbourhood of $\overline{w}'$ in $W'$, see
\'Etale Morphisms, Lemma \ref{etale-lemma-finite-etale-one-point}.
Shrinking $V$ and $W'$ we may assume $W \to W'$ is an isomorphism.
Thus $s$ may be viewed as a section $s'$ of $\mathcal{F}$ over
the open subscheme $W' \subset X_V$ which is finite over $V$.
Hence by definition $(W', s')$ defines an element of $j_{p!}\mathcal{F}(V)$
which maps to $s$ as desired.

\medskip\noindent
Let us prove the arrow is injective. To do this, let
$s = \sum_{i = 1, \ldots, n} (Z_i, s_i)$ as in (\ref{equation-formal-sum})
be an element of $f_{p!}\mathcal{F}(V)$ where $(V, \overline{v})$ is an
\'etale neighbourhood of $\overline{y}$. Assume $s$ maps to zero
under the map constructed above. First, after replacing
$(V, \overline{v})$ by an \'etale neighbourhood of itself,
we may assume there exist decompositions
$Z_i = Z_{i, 1} \amalg \ldots \amalg Z_{i, m_i}$ into open and closed
subschemes such that each $Z_{i, j}$ has exactly one geometric point
over $\overline{v}$. Say under the obvious direct sum decomposition
$$
H_{Z_i}(\mathcal{F}) = \bigoplus H_{Z_{i, j}}(\mathcal{F})
$$
the element $s_i$ corresponds to $\sum s_{i, j}$. We may use relations
(\ref{item-sum}) and (\ref{item-sub}) to replace $s$ by
$\sum_{i = 1, \ldots, n} \sum_{j = 1, \ldots, m_i} (Z_{i, j}, s_{i, j})$.
In other words, we may assume $Z_i$ has a unique geometric point
lying over $\overline{v}$. Let $\overline{x}_1, \ldots, \overline{x}_m$
be the geometric points of $X$ over $\overline{y}$ corresponding to
the geometric points of our $Z_i$ over $\overline{v}$; note that for
one $j \in \{1, \ldots, m\}$ there may be multiple indices $i$ for which
$\overline{x}_j$ corresponds to a point of $Z_i$.
By More on Morphisms, Lemma
\ref{more-morphisms-lemma-etale-makes-quasi-finite-finite-multiple-points-var}
applied to both $X_V \to V$
after replacing $(V, \overline{v})$ by an \'etale neighbourhood of itself
we may assume there exist open subschemes
$$
W_j \subset X \times_Y V,\quad j = 1, \ldots, m
$$
and a geometric point $\overline{w}_j$ of $W_j$ mapping to $\overline{x}_j$ and
$\overline{v}$ such that $W_j \to V$ is finite and $\overline{w}_j$ is the
only geometric point of $W_j$ mapping to $\overline{v}$.
After shrinking $V$ we may assume $Z_i \subset W_j$ for some $j$
and we have the map $H_{Z_i}(\mathcal{F}) \to H_{W_j}(\mathcal{F})$.
Thus by the relation (\ref{item-sub})
we see that our element is equivalent to an element of the form
$$
\sum\nolimits_{j = 1, \ldots, m} (W_j, t_j)
$$
for some $t_j \in H_{W_j}(\mathcal{F})$. Clearly, this element is mapped
simply to the class of $t_j$ in the summand $\mathcal{F}_{\overline{x}_j}$.
Since $s$ maps to zero, we find that $t_j$ maps to zero in
$\mathcal{F}_{\overline{x}_j}$. This implies that $t_j$ restricts
to zero on an open neighbourhood of $\overline{w}_j$ in $W_j$, see
\'Etale Cohomology, Lemma \ref{etale-cohomology-lemma-zero-over-image}.
Shrinking $V$ once more we obtain $t_j = 0$ for all $j$ as desired.
\end{proof}

\begin{lemma}
\label{lemma-finite-support-etale-shriek}
Let $f = j : U \to X$ be an \'etale of schemes. Denote $j_{p!}$
the construction of \'Etale Cohomology, Equation
(\ref{etale-cohomology-equation-j-p-shriek})
and denote $f_{p!}$ the construction above. Functorially in
$\mathcal{F} \in \textit{Ab}(X_\etale)$ there is a canonical map
$$
j_{p!}\mathcal{F} \longrightarrow f_{p!}\mathcal{F}
$$
of abelian presheaves which identifies the sheaf
$j_!\mathcal{F} = (j_{p!}\mathcal{F})^\#$ of \'Etale Cohomology,
Definition \ref{etale-cohomology-definition-extension-zero}
with $(f_{p!}\mathcal{F})^\#$.
\end{lemma}

\begin{proof}
Please read the proof of \'Etale Cohomology, Lemma
\ref{etale-cohomology-lemma-shriek-into-star-separated-etale}
before reading the proof of this lemma.
Let $V$ be an object of $X_\etale$. Recall that
$$
j_{p!}\mathcal{F}(V) =
\bigoplus\nolimits_{\varphi : V \to U} \mathcal{F}(V \xrightarrow{\varphi} U)
$$
Given $\varphi$ we obtain an open subscheme
$Z_\varphi \subset U_V = U \times_X V$, namely,
the image of the graph of $\varphi$. Via $\varphi$
we obtain an isomorphism $V \to Z_\varphi$ over $U$
and we can think of an element
$$
s_\varphi \in \mathcal{F}(V \xrightarrow{\varphi} U) =
\mathcal{F}(Z_\varphi) = H_{Z_\varphi}(\mathcal{F})
$$
as a section of $\mathcal{F}$ over $Z_{\varphi}$. Since
$Z_\varphi \subset U_V$ is open, we actually have
$H_{Z_\varphi}(\mathcal{F}) = \mathcal{F}(Z_\varphi)$
and we can think of $s_\varphi$ as an element of $H_{Z_\varphi}(\mathcal{F})$.
Having said this, our map $j_{p!}\mathcal{F} \to f_{p!}\mathcal{F}$
is defined by the rule
$$
\sum\nolimits_{i = 1, \ldots, n} s_{\varphi_i}
\longmapsto
\sum\nolimits_{i = 1, \ldots, n} (Z_{\varphi_i}, s_{\varphi_i})
$$
with right hand side a sum as in (\ref{equation-formal-sum}).
We omit the verification that this is compatible with restriction
mappings and functorial in $\mathcal{F}$.

\medskip\noindent
To finish the proof, we claim that given a geometric point
$\overline{y} : \Spec(k) \to Y$ there is a commutative diagram
$$
\xymatrix{
(j_{p!}\mathcal{F})_{\overline{y}} \ar[r] \ar[d] &
\bigoplus_{j(\overline{x}) = \overline{y}} \mathcal{F}_{\overline{x}}
\ar@{=}[d] \\
(f_{p!}\mathcal{F})_{\overline{y}} \ar[r] &
\bigoplus_{f(\overline{x}) = \overline{y}} \mathcal{F}_{\overline{x}}
}
$$
where the top horizontal arrow is constructed in the proof of
\'Etale Cohomology, Proposition
\ref{etale-cohomology-proposition-describe-jshriek},
the bottom horizontal arrow is constructed in the proof of
Lemma \ref{lemma-finite-support-stalk},
the right vertical arrow is the obvious equality, and
the left veritical arrow is the map defined in the previous
paragraph on stalks. The claim follows in a straightforward manner
from the explicit description of all of the arrows involved
here and in the references given.
Since the horizontal arrows are isomorphisms
we conclude so is the left vertical arrow. Hence we find that
our map induces an isomorphism on sheafifications by
\'Etale Cohomology, Theorem \ref{etale-cohomology-theorem-exactness-stalks}.
\end{proof}

\begin{definition}
\label{definition-f-shriek-lqf}
Let $f : X \to Y$ be a locally quasi-finite morphism of schemes.
We define the {\it direct image with compact support} to be the
functor
$$
f_! : \textit{Ab}(X_\etale) \longrightarrow \textit{Ab}(Y_\etale)
$$
defined by the formula $f_!\mathcal{F} = (f_{p!}\mathcal{F})^\#$,
i.e., $f_!\mathcal{F}$ is the sheafification of the presheaf
$f_{p!}\mathcal{F}$ constructed above.
\end{definition}

\noindent
By Lemma \ref{lemma-finite-support-f-shriek-separated}
this does not conflict with Definition \ref{definition-f-shriek-separated}
(when both definitions apply) and by
Lemma \ref{lemma-finite-support-etale-shriek}
this does not conflict with
\'Etale Cohomology, Definition \ref{etale-cohomology-definition-extension-zero}
(when both definitions apply).

\begin{lemma}
\label{lemma-lqf-f-shriek-stalk}
Let $f : X \to Y$ be a locally quasi-finite morphism of schemes. Then
\begin{enumerate}
\item for $\mathcal{F}$ in $\textit{Ab}(X_\etale)$ and a geometric
point $\overline{y} : \Spec(k) \to Y$ we have
$$
(f_!\mathcal{F})_{\overline{y}} =
\bigoplus\nolimits_{f(\overline{x}) = \overline{y}} \mathcal{F}_{\overline{x}}
$$
functorially in $\mathcal{F}$, and
\item the functor $f_! : \textit{Ab}(X_\etale) \to \textit{Ab}(Y_\etale)$
is exact and commutes with direct sums.
\end{enumerate}
\end{lemma}

\begin{proof}
The formula for the stalks is immediate (and in fact equivalent) to
Lemma \ref{lemma-finite-support-stalk}.
The exactness of the functor follows immediately from this
and the fact that exactness may be checked on stalks, see
\'Etale Cohomology, Theorem \ref{etale-cohomology-theorem-exactness-stalks}.
\end{proof}

\begin{remark}[Covariance with respect to open embeddings]
\label{remark-covariance-lqf-f-shriek}
Let $f : X \to Y$ be locally quasi-finite morphism of schemes. Let
$\mathcal{F}$ be an abelian sheaf on $X_\etale$.
Let $X' \subset X$ be an open subscheme and denote $f' : X' \to Y$
the restriction of $f$.
We claim there is a canonical map
$$
f'_!(\mathcal{F}|_{X'}) \longrightarrow f_!\mathcal{F}
$$
Namely, this map will be the sheafification of a canonical map
$$
f'_{p!}(\mathcal{F}|_{X'}) \to f_{p!}\mathcal{F}
$$
constructed as follows. Let $V \in Y_\etale$ and consider a section
$s' = \sum_{i = 1, \ldots, n} (Z'_i, s'_i)$ as in
(\ref{equation-formal-sum}) defining an element of
$f'_{p!}(\mathcal{F}|_{X'})(V)$.
Then $Z'_i \subset X'_V$ may also be viewed as a locally closed subscheme
of $X_V$ and we have $H_{Z'_i}(\mathcal{F}|_{X'}) = H_{Z'_i}(\mathcal{F})$.
We will map $s'$ to the exact same sum
$s = \sum_{i = 1, \ldots, n} (Z'_i, s'_i)$
but now viewed as an element of $f_{p!}\mathcal{F}(V)$.
We omit the verification that this construction is compatible with
restriction mappings and functorial in $\mathcal{F}$.
This construction has the following properties:
\begin{enumerate}
\item The maps $f'_{p!}\mathcal{F}' \to f_{p!}\mathcal{F}$ and
$f'_!\mathcal{F}' \to f_!\mathcal{F}$ are compatible with
the description of stalks given in Lemmas
\ref{lemma-finite-support-stalk} and \ref{lemma-lqf-f-shriek-stalk}.
\item If $f$ is separated, then the map
$f'_{p!}\mathcal{F}' \to f_{p!}\mathcal{F}$ is the same as the map
constructed in Remark \ref{remark-covariance-f-shriek-separated}
via the isomorphism in Lemma \ref{lemma-finite-support-f-shriek-separated}.
\item If $X'' \subset X'$ is another open, then the composition of
$f''_{p!}(\mathcal{F}|_{X''}) \to f'_{p!}(\mathcal{F}|_{X'}) \to
f_{p!}\mathcal{F}$ is the map
$f''_{p!}(\mathcal{F}|_{X''}) \to f_{p!}\mathcal{F}$ for the
inclusion $X'' \subset X$. Sheafifying we conclude
the same holds true for
$f''_!(\mathcal{F}|_{X''}) \to f'_!(\mathcal{F}|_{X'}) \to f_!\mathcal{F}$.
\item The map $f'_!\mathcal{F}' \to f_!\mathcal{F}$ is injective
because we can check this on stalks.
\end{enumerate}
All of these statements are easily proven by representing elements
as finite sums as above and considering what happens to these elements.
\end{remark}

\begin{lemma}
\label{lemma-lqf-colimit-f-shriek}
Let $f : X \to Y$ be a locally quasi-finite morphism of schemes.
Let $X = \bigcup_{i \in I} X_i$ be an open covering. Then there
exists an exact complex
$$
\ldots \to
\bigoplus\nolimits_{i_0, i_1, i_2} f_{i_0i_1i_2, !}
\mathcal{F}|_{X_{i_0i_1i_2}} \to
\bigoplus\nolimits_{i_0, i_1} f_{i_0i_1, !} \mathcal{F}|_{X_{i_0i_1}} \to
\bigoplus\nolimits_{i_0} f_{i_0, !} \mathcal{F}|_{X_{i_0}}
\to f_!\mathcal{F} \to 0
$$
functorial in $\mathcal{F} \in \textit{Ab}(X_\etale)$, see
proof for details.
\end{lemma}

\begin{proof}
Here as usual we set $X_{i_0 \ldots i_p} = X_{i_0} \cap \ldots \cap X_{i_p}$
and we denote $f_{i_0 \ldots i_p}$ the restriction of $f$ to
$X_{i_0 \ldots i_p}$. The maps in the complex are the maps
constructed in Remark \ref{remark-covariance-lqf-f-shriek}
with sign rules as in the {\v C}ech complex.
Exactness follows easily from the description of stalks in
Lemma \ref{lemma-lqf-f-shriek-stalk}. Details omitted.
\end{proof}

\begin{remark}[Alternative construction]
\label{remark-alternative-lqf-f-shriek}
Lemma \ref{lemma-lqf-colimit-f-shriek}
gives an alternative construction of the functor $f_!$
for locally quasi-finite morphisms $f$.
Namely, given a locally quasi-finite morphism $f : X \to Y$ of schemes
we can choose an open covering $X = \bigcup_{i \in I} X_i$
such that each $f_i : X_i \to Y$ is separated. For example choose
an affine open covering of $X$. Then we
can define $f_!\mathcal{F}$ as the cokernel of the penultimate map
of the complex of the lemma, i.e.,
$$
f_!\mathcal{F} = \Coker\left(
\bigoplus\nolimits_{i_0, i_1} f_{i_0i_1, !} \mathcal{F}|_{X_{i_0i_1}} \to
\bigoplus\nolimits_{i_0} f_{i_0, !} \mathcal{F}|_{X_{i_0}}
\right)
$$
where we can use the construction of $f_{i_0, !}$ and
$f_{i_0i_1, !}$ in Section \ref{section-compact-support}
because the morphisms $f_{i_0}$ and $f_{i_0 i_1}$ are separated.
One can then compute the stalks of $f_!$ (using the separated
case, namely Lemma \ref{lemma-lqf-f-shriek-separated-colimits})
and obtain the result of Lemma \ref{lemma-lqf-f-shriek-stalk}.
Having done so all the other results of this section can be
deduced from this as well.
\end{remark}

\begin{remark}
\label{remark-construct-map-presheaves-downstairs}
Let $g : Y' \to Y$ be a morphism of schemes.
For an abelian presheaf $\mathcal{G}'$ on $Y'_\etale$ let us denote
$g_*\mathcal{G}'$ the presheaf $V \mapsto \mathcal{G}'(Y' \times_Y V)$.
If $\alpha : \mathcal{G} \to g_*\mathcal{G}'$ is a map of abelian presheaves
on $Y_\etale$, then there is a unique map
$\alpha^\# : \mathcal{G}^\# \to g_*((\mathcal{G}')^\#)$
of abelian sheaves on $Y_\etale$ such that the diagram
$$
\xymatrix{
\mathcal{G} \ar[d] \ar[r]_\alpha & g_*\mathcal{G}' \ar[d] \\
\mathcal{G}^\# \ar[r]^-{\alpha^\#} & g_*((\mathcal{G}')^\#)
}
$$
is commutative where the vertical maps come from the canonical maps
$\mathcal{G} \to \mathcal{G}^\#$ and $\mathcal{G}' \to (\mathcal{G}')^\#$. If
$\alpha' : g^{-1}\mathcal{G}^\# \to (\mathcal{G}')^\#$
is the map adjoint to $\alpha^\#$, then for a geometric point
$\overline{y}' : \Spec(k) \to Y'$ with image
$\overline{y} = g \circ \overline{y}'$ in $Y$, the map
$$
\alpha'_{\overline{y}'} :
\mathcal{G}_{\overline{y}} =
(\mathcal{G}^\#)_{\overline{y}} =
(g^{-1}\mathcal{G}^\#)_{\overline{y}'}
\longrightarrow
(\mathcal{G}')^\#_{\overline{y}'} =
\mathcal{G}'_{\overline{y}'}
$$
is given by mapping the class in the stalk of a section $s$ of $\mathcal{G}$
over an \'etale neighbourhood $(V, \overline{v})$ to the class of the section
$\alpha(s)$ in $g_*\mathcal{G}'(V) = \mathcal{G}'(Y' \times_Y V)$
over the \'etale neighbourhood $(Y' \times_Y V, (\overline{y}', \overline{v}))$
in the stalk of $\mathcal{G}'$ at $\overline{y}'$.
\end{remark}

\begin{lemma}
\label{lemma-lqf-base-change-f-shriek}
Consider a cartesian square
$$
\xymatrix{
X' \ar[r]_{g'} \ar[d]_{f'} & X \ar[d]^f \\
Y' \ar[r]^g & Y
}
$$
of schemes with $f$ locally quasi-finite. For any abelian sheaf $\mathcal{F}$
on $X_\etale$ we have $f'_!(g')^{-1}\mathcal{F} = g^{-1}f_!\mathcal{F}$.
\end{lemma}

\begin{proof}
With conventions as in Remark \ref{remark-construct-map-presheaves-downstairs}
we will explicitly construct a map
$$
c : f_{p!}\mathcal{F} \longrightarrow g_*f'_{p!}(g')^{-1}\mathcal{F}
$$
of abelian presheaves on $Y_\etale$. By the discussion in
Remark \ref{remark-construct-map-presheaves-downstairs}
this will determine a canonical map
$g^{-1}f_!\mathcal{F} \to f'_!(g')^{-1}\mathcal{F}$. Finally, we
will show this map induces isomorphisms on stalks and conclude by
\'Etale Cohomology, Theorem \ref{etale-cohomology-theorem-exactness-stalks}.

\medskip\noindent
Construction of the map $c$. Let $V \in Y_\etale$ and consider a section
$s = \sum_{i = 1, \ldots, n} (Z_i, s_i)$ as in
(\ref{equation-formal-sum}) defining an element of $f_{p!}\mathcal{F}(V)$.
The value of $g_*f'_{p!}(g')^{-1}\mathcal{F}$ at $V$ is
$f'_{p!}(g')^{-1}\mathcal{F}(V')$ where $V' = V \times_Y Y'$.
Denote $Z'_i \subset X'_{V'}$ the base change of $Z_i$ to $V'$.
By (\ref{item-pullback}) there is a pullback map
$H_{Z_i}(\mathcal{F}) \to H_{Z'_i}((g')^{-1}\mathcal{F})$.
Denoting $s'_i \in H_{Z'_i}((g')^{-1}\mathcal{F})$ the image of $s_i$
under pullback, we set $c(s) = \sum_{i = 1, \ldots, n} (Z'_i, s'_i)$ as in
(\ref{equation-formal-sum}) defining an element of
$f'_{p!}(g')^{-1}\mathcal{F}(V')$. We omit the verification
that this construction is compatible the relations
(\ref{item-sum}) and (\ref{item-sub}) and compatible
with restriction mappings. The construction is clearly
functorial in $\mathcal{F}$.

\medskip\noindent
Let $\overline{y}' : \Spec(k) \to Y'$ be a geometric point with image
$\overline{y} = g \circ \overline{y}'$ in $Y$. Observe that
$X'_{\overline{y}'} = X_{\overline{y}}$ by transitivity of
fibre products. Hence $g'$ produces a bijection
$\{f'(\overline{x}') = \overline{y}'\} \to \{f(\overline{x}) = \overline{y}\}$
and if $\overline{x}'$ maps to $\overline{x}$, then
$((g')^{-1}\mathcal{F})_{\overline{x}'} = \mathcal{F}_{\overline{x}}$
by \'Etale Cohomology, Lemma \ref{etale-cohomology-lemma-stalk-pullback}.
Now we claim that the diagram
$$
\xymatrix{
(g^{-1}f_!\mathcal{F})_{\overline{y}'} \ar@{=}[r] \ar[d] &
(f_!\mathcal{F})_{\overline{y}} \ar[r] \ar[ld] &
\bigoplus\nolimits_{f(\overline{x}) = \overline{y}}
\mathcal{F}_{\overline{x}} \ar[d]
\\
(f'_!(g')^{-1}\mathcal{F})_{\overline{y}'} \ar[rr] & &
\bigoplus\nolimits_{f'(\overline{x}') = \overline{y}'}
(g')^{-1}\mathcal{F}_{\overline{x}'}
}
$$
commutes where the horizontal arrows are given in the proof of
Lemma \ref{lemma-finite-support-stalk} and where the right vertical
arrow is an equality by what we just said above. The southwest arrow is
described in Remark \ref{remark-construct-map-presheaves-downstairs}
as the pullback map, i.e.,
simply given by our construction $c$ above. Then the simple
description of the image of a sum $\sum (Z_i, z_i)$ in the
stalk at $\overline{x}$ given in the proof of
Lemma \ref{lemma-finite-support-stalk} immediately shows the
diagram commutes. This finishes the proof of the lemma.
\end{proof}

\begin{lemma}
\label{lemma-lqf-separated-shriek-composition}
Let $f' : X \to Y'$ and $g : Y' \to Y$ be composable morphisms of schemes
with $f'$ and $f = g \circ f'$ locally quasi-finite and $g$ separated and
locally of finite type. Then there is a canonical isomorphism of functors
$g_! \circ f'_! = f_!$. This isomorphism is compatible with
\begin{enumerate}
\item[(a)] covariance with respect to open embeddings as in
Remarks \ref{remark-covariance-f-shriek-separated} and
\ref{remark-covariance-lqf-f-shriek},
\item[(b)] the base change isomorphisms of
Lemmas \ref{lemma-lqf-base-change-f-shriek}
and \ref{lemma-base-change-f-shriek-separated}, and
\item[(c)] equal to the isomorphism of Lemma \ref{lemma-f-shriek-composition}
via the identifications of Lemma \ref{lemma-finite-support-f-shriek-separated}
in case $f'$ is separated.
\end{enumerate}
\end{lemma}

\begin{proof}
Let $\mathcal{F}$ be an abelian sheaf on $X_\etale$. With conventions as in
Remark \ref{remark-construct-map-presheaves-downstairs} we will explicitly
construct a map
$$
c : f_{p!}\mathcal{F} \longrightarrow g_*f'_{p!}\mathcal{F}
$$
of abelian presheaves on $Y_\etale$. By the discussion in
Remark \ref{remark-construct-map-presheaves-downstairs}
this will determine a canonical map
$c^\# : f_!\mathcal{F} \to g_*f'_!\mathcal{F}$.
We will show that $c^\#$ has image contained in the subsheaf
$g_!f'_!\mathcal{F}$, thereby obtaining a map
$c' : f_!\mathcal{F} \to g_!f'_!\mathcal{F}$. Next, we will prove
(a), (b), and (c) that. Finally, part (b)
will allow us to show that $c'$ is an isomorphism.

\medskip\noindent
Construction of the map $c$. Let $V \in Y_\etale$ and
let $s = \sum (Z_i, s_i)$ be a sum as in (\ref{equation-formal-sum})
defining an element of $f_{p!}\mathcal{F}(V)$.
Recall that $Z_i \subset X_V = X \times_Y V$
is a locally closed subscheme finite over $V$.
Setting $V' = Y' \times_Y V$ we get $X_{V'} = X \times_{Y'} V' = X_V$.
Hence $Z_i \subset X_{V'}$ is locally closed and
$Z_i$ is finite over $V'$ because $g$ is separated
(Morphisms, Lemma \ref{morphisms-lemma-finite-permanence}).
Hence we may set $c(s) = \sum (Z_i, s_i)$ but now viewed
as an element of $f'_{p!}\mathcal{F}(V') = (g_*f'_{p!}\mathcal{F})(V)$.
The construction is clearly compatible with relations
(\ref{item-sum}) and (\ref{item-sub})
and compatible with restriction mappings and hence we obtain the map $c$.

\medskip\noindent
Observe that in the discussion above our section $c(s) = \sum (Z_i, s_i)$ of
$f'_!\mathcal{F}$ over $V'$ restricts to zero on
$V' \setminus \Im(\coprod Z_i \to V')$. Since $\Im(\coprod Z_i \to V')$
is proper over $V$ (for example by Morphisms, Lemma
\ref{morphisms-lemma-scheme-theoretic-image-is-proper})
we conclude that $c(s)$ defines a section of
$g_!f'_!\mathcal{F} \subset g_*f'_!\mathcal{F}$ over $V$.
Since every local section of $f_!\mathcal{F}$ locally comes from a
local section of $f_{p!}\mathcal{F}$ we conclude that the image
of $c^\#$ is contained in $g_!f'_!\mathcal{F}$.
Thus we obtain an induced map $c' : f_!\mathcal{F} \to g_!f'_!\mathcal{F}$
factoring $c^\#$ as predicted in the first paragraph of the proof.

\medskip\noindent
Proof of (a). Let $Y'_1 \subset Y'$ be an open subscheme
and set $X_1 = (f')^{-1}(W')$. We obtain a diagram
$$
\xymatrix{
X_1 \ar[d]_{f'_1} \ar[r]_a \ar@/_2em/[dd]_{f_1} &
X \ar[d]^{f'} \ar@/^2em/[dd]^f \\
Y'_1 \ar[d]_{g_1} \ar[r]_{b'} &
Y' \ar[d]^g \\
Y \ar@{=}[r] &
Y
}
$$
where the horizontal arrows are open immersions. Then our claim is that
the diagram
$$
\xymatrix{
f_{1, !}\mathcal{F}|_{X_1} \ar[r]_{c'_1} \ar[dd] &
g_{1, !}f'_{1, !}\mathcal{F}|_{X_1} \ar@{=}[d] \\
& g_{1, !}(f'_!\mathcal{F})|_{Y'_1} \ar[d] \\
f_!\mathcal{F} \ar[r]^{c'} &
g_!f'_!\mathcal{F} \ar[r] & g_*f'_!\mathcal{F}
}
$$
commutes where the left vertical arrow is
Remark \ref{remark-covariance-lqf-f-shriek} and
the right vertical arrow is Remark \ref{remark-covariance-f-shriek-separated}.
The equality sign in the diagram comes about because $f'_1$
is the restriction of $f'$ to $Y'_1$ and our construction
of $f'_!$ is local on the base.
Finally, to prove the commutativity we choose an object $V$ of
$Y_\etale$ and a formal sum $s_1 = \sum (Z_{1, i}, s_{1, i})$ as in
(\ref{equation-formal-sum}) defining an element of
$f_{1, p!}\mathcal{F}|_{X_1}(V)$. Recall this means
$Z_{1, i} \subset X_1 \times_Y V$ is locally closed finite over $V$
and $s_{1, i} \in H_{Z_{1, i}}(\mathcal{F})$.
Then we chase this section
across the maps involved, but we only need to show we
end up with the same element of
$g_*f'_!\mathcal{F}(V) = f'_!\mathcal{F}(Y' \times_Y V)$.
Going around both sides of the diagram the reader immediately
sees we end up with the element $\sum (Z_{1, i}, s_{1, i})$
where now $Z_{1, i}$ is viewed as a locally closed subscheme
of $X \times_{Y'} (Y' \times_Y V) = X \times_Y V$ finite over
$Y' \times_Y V$.

\medskip\noindent
Proof of (b). Let $b : Y_1 \to Y$ be a morphism of schemes. Let us form the
commutative diagram
$$
\xymatrix{
X_1 \ar[d]_{f'_1} \ar[r]_a \ar@/_2em/[dd]_{f_1} &
X \ar[d]^{f'} \ar@/^2em/[dd]^f \\
Y'_1 \ar[d]_{g_1} \ar[r]_{b'} &
Y' \ar[d]^g \\
Y_1 \ar[r]^b &
Y
}
$$
with cartesian squares. We claim that our construction is compatible
with the base change maps of Lemmas \ref{lemma-lqf-base-change-f-shriek}
and \ref{lemma-base-change-f-shriek-separated}, i.e.,
that the top rectangle of the diagram
$$
\xymatrix{
b^{-1}f_!\mathcal{F} \ar[rr] \ar[d]_{b^{-1}c'} & &
f_{1, !}a^{-1}\mathcal{F} \ar[d]^{c_1'} \\
b^{-1}g_!f'_!\mathcal{F} \ar[r] \ar[d] &
g_{1, !}(b')^{-1}f'_!\mathcal{F} \ar[r] \ar[d] &
g_{1, !}f'_{1, !}a^{-1}\mathcal{F} \ar[d] \\
b^{-1}g_*f'_!\mathcal{F} \ar[r] &
g_{1, *}(b')^{-1}f'_!\mathcal{F} \ar[r] &
g_{1, *}f'_{1, !}a^{-1}\mathcal{F}
}
$$
commutes. The verification of this is completely routine and we
urge the reader to skip it. Since the arrows going from the middle
row down to the bottom row are injective, it suffices to show that
the outer diagram commutes.
To show this it suffices to take a local section of
$b^{-1}f_!\mathcal{F}$ and show we end up with the same local
section of $g_{1, *}f'_{1, !}a^{-1}\mathcal{F}$
going around either way. However, in fact it suffices to check
this for local sections which are of the the pullback by $b$ of
a section $s = \sum (Z_i, s_i)$ of $f_{p!}\mathcal{F}(V)$
as above (since such pullbacks generate the abelian sheaf
$b^{-1}f_!\mathcal{F}$). Denote $V_1$, $V'_1$, and $Z_{1, i}$
the base change of $V$, $V' = Y' \times_Y V$, $Z_i$ by $Y_1 \to Y$.
Recall that $Z_i$ is a locally closed subscheme of $X_V = X_{V'}$
and hence $Z_{1, i}$ is a locally closed subscheme
of $(X_1)_{V_1} = (X_1)_{V'_1}$. Then $b^{-1}c'$ sends the pullback
of $s$ to the pullback of the local section $c(s) \sum (Z_i, s_i)$ viewed
as an element of $f'_{p!}\mathcal{F}(V') = (g_*f'_{p!}\mathcal{F})(V)$.
The composition of the bottom two base change maps
simply maps this to $\sum (Z_{i, 1}, s_{1, i})$ viewed as an
element of $f'_{1, p!}a^{-1}\mathcal{F}(V'_1) =
g_{1, *}f'_{1, p!}a^{-1}\mathcal{F}(V_1)$.
On the other hand, the base change map at the top of the diagram
sends the pullback of $s$ to $\sum (Z_{1, i}, s_{1, i})$ viewed
as an element of $f_{1, !}a^{-1}\mathcal{F}(V_1)$.
Then finally $c'_1$ by its very construction does indeed
map this to $\sum (Z_{i, 1}, s_{1, i})$ viewed as an
element of $f'_{1, p!}a^{-1}\mathcal{F}(V'_1) =
g_{1, *}f'_{1, p!}a^{-1}\mathcal{F}(V_1)$ and the commutativity
has been verified.

\medskip\noindent
Proof of (c). This follows from comparing the definitions
for both maps; we omit the details.

\medskip\noindent
To finish the proof it suffices to show that the pullback of
$c'$ via any geometric point $\overline{y} : \Spec(k) \to Y$
is an isomorphism. Namely, pulling back by $\overline{y}$
is the same thing as taking stalks and $\overline{y}$
(\'Etale Cohomology, Remark \ref{etale-cohomology-remark-stalk-pullback})
and hence we can invoke
\'Etale Cohomology, Theorem \ref{etale-cohomology-theorem-exactness-stalks}.
By the compatibility (b) just shown, we 
conclude that we may assume $Y$ is the spectrum of $k$
and we have to show that $c'$ is an isomorphism.
To do this it suffices to show that the induced map
$$
\bigoplus\nolimits_{x \in X} \mathcal{F}_x = H^0(Y, f_!\mathcal{F})
\longrightarrow
H^0(Y, g_!f'_!\mathcal{F}) = H^0_c(Y', f'_!\mathcal{F})
$$
is an isomorphism. The equalities hold by
Lemmas \ref{lemma-lqf-f-shriek-stalk} and
\ref{lemma-stalk-f-shriek-separated}.
Recall that $X$ is a disjoint union of
spectra of Artinian local rings with residue field $k$, see
Varieties, Lemma \ref{varieties-lemma-algebraic-scheme-dim-0}.
Since the left and right hand side commute with direct
sums (details omitted) we may assume that $\mathcal{F}$ is a skyscraper
sheaf $x_*A$ supported at some $x \in X$.
Then $f'_!\mathcal{F}$ is the skyscraper sheaf at the
image $y'$ of $x$ in $Y$ by Lemma \ref{lemma-lqf-f-shriek-stalk}.
In this case it is obvious that our construction
produces the identity map $A \to H^0_c(Y', y'_*A) = A$
as desired.
\end{proof}

\begin{lemma}
\label{lemma-lqf-shriek-composition}
Let $f : X \to Y$ and $g : Y \to Z$ be composable locally quasi-finite
morphisms of schemes. Then there is a canonical isomorphism of functors
$$
(g \circ f)_! \longrightarrow g_! \circ f_!
$$
These isomorphisms satisfy the following properties:
\begin{enumerate}
\item If $f$ and $g$ are separated, then the isomorphism agrees
with Lemma \ref{lemma-f-shriek-composition}.
\item If $g$ is separated, then the isomorphism agrees with
Lemma \ref{lemma-lqf-separated-shriek-composition}.
\item For a geometric point $\overline{z} : \Spec(k) \to Z$ the diagram
$$
\xymatrix{
((g \circ f)_!\mathcal{F})_{\overline{z}} \ar[d] \ar[rr] & &
\bigoplus\nolimits_{g(f(\overline{x})) = \overline{z}}
\mathcal{F}_{\overline{x}} \ar@{=}[d] \\
(g_!f_!\mathcal{F})_{\overline{z}} \ar[r] &
\bigoplus\nolimits_{g(\overline{y}) = \overline{z}}
(f_!\mathcal{F})_{\overline{y}} \ar[r] &
\bigoplus\nolimits_{g(f(\overline{x})) = \overline{z}}
\mathcal{F}_{\overline{x}}
}
$$
is commutative where the horizontal arrows are given by
Lemma \ref{lemma-lqf-f-shriek-stalk}.
\item Let $h : Z \to T$ be a third locally quasi-finite
morphism of schemes. Then the diagram
$$
\xymatrix{
(h \circ g \circ f)_! \ar[r] \ar[d] &
(h \circ g)_! \circ f_! \ar[d] \\
h_! \circ (g \circ f)_! \ar[r] &
h_! \circ g_! \circ f_!
}
$$
commutes.
\item Suppose that we have a diagram of schemes
$$
\xymatrix{
X' \ar[d]_{f'} \ar[r]_c & X \ar[d]^f \\
Y' \ar[d]_{g'} \ar[r]_b & Y \ar[d]^g \\
Z' \ar[r]^a & Z
}
$$
with both squares cartesian and $f$ and $g$
locally quasi-finite. Then the diagram
$$
\xymatrix{
a^{-1} \circ (g \circ f)_! \ar[d] \ar[rr] & &
(g' \circ f')_! \circ c^{-1} \ar[d] \\
a^{-1} \circ g_! \circ f_! \ar[r] &
g'_! \circ b^{-1} \circ f_! \ar[r] &
g'_! \circ f'_! \circ c^{-1}
}
$$
commutes where the horizontal arrows are those of
Lemma \ref{lemma-lqf-base-change-f-shriek}.
\end{enumerate}
\end{lemma}

\begin{proof}
If $f$ and $g$ are separated, then this is a special case of
Lemma \ref{lemma-f-shriek-composition}.
If $g$ is separated, then this is a special case of
Lemma \ref{lemma-lqf-separated-shriek-composition}
which moreover agrees with the case where $f$ and $g$ are separated.

\medskip\noindent
Construction in the general case. Choose an open covering $Y = \bigcup Y_i$
such that the restriction $g_i : Y_i \to Z$ of $g$ is separated.
Set $X_i = f^{-1}(Y_i)$ and denote $f_i : X_i \to Y_i$ the restriction
of $f$. Also denote $h = g \circ f$ and $h_i : X_i \to Z$ the restriction
of $h$. Consider the following diagram
$$
\xymatrix{
\bigoplus\nolimits_{i_0, i_1}
h_{i_0i_1, !}\mathcal{F}|_{X_{i_0i_1}} \ar[r] \ar[d] &
\bigoplus\nolimits_{i_0} h_{i_0, !}\mathcal{F}|_{X_{i_0}} \ar[r] \ar[d] &
h_!\mathcal{F} \ar[r] \ar@{..>}[dd] &
0 \\
\bigoplus\nolimits_{i_0, i_1}
g_{i_0i_1, !} f_{i_0i_1, !}\mathcal{F}|_{X_{i_0i_1}} \ar[r] \ar[d] &
\bigoplus\nolimits_{i_0}
g_{i_0, !} f_{i_0, !}\mathcal{F}|_{X_{i_0}} \ar[d] \\
\bigoplus\nolimits_{i_0, i_1}
g_{i_0i_1, !} (f_!\mathcal{F})|_{Y_{i_0i_1}} \ar[r] &
\bigoplus\nolimits_{i_0}
g_{i_0, !} (f_!\mathcal{F})|_{Y_{i_0}} \ar[r] &
g_!f_!\mathcal{F} \ar[r] &
0
}
$$
By Lemma \ref{lemma-lqf-colimit-f-shriek} the top and bottom row
in the diagram are exact. By Lemma \ref{lemma-lqf-separated-shriek-composition}
the top left square commutes. The vertical arrows in the
lower left square come about because
$(f_!\mathcal{F})|_{Y_{i_0i_1}} = f_{i_0i_1, !}\mathcal{F}|_{X_{i_0i_1}}$ and
$(f_!\mathcal{F})|_{Y_{i_0}} = f_{i_0, !}\mathcal{F}|_{X_{i_0}}$
as the construction of $f_!$ is local on the base. Moreover, these
equalities are (of course) compatible with the identifications
$((f_!\mathcal{F})|_{Y_{i_0}})|_{Y_{i_0i_1}} =
(f_!\mathcal{F})|_{Y_{i_0i_1}}$ and
$(f_{i_0, !}\mathcal{F}|_{X_{i_0}})|_{Y_{i_0i_1}} =
f_{i_0i_1, !}\mathcal{F}|_{X_{i_0i_1}}$
which are used (together with the covariance for open embeddings
for $Y_{i_0i_1} \subset Y_{i_0}$)
to define the horizontal maps of the lower left square.
Thus this square commutes as well.
In this way we conclude there is a unique
dotted arrow as indicated in the diagram and
moreover this arrow is an isomorphism.

\medskip\noindent
Proof of properties (1) -- (5). Fix the open covering $Y = \bigcup Y_i$.
Observe that if $Y \to Z$ happens to be separated, then we get a dotted
arrow fitting into the huge diagram above by using the map of
Lemma \ref{lemma-lqf-separated-shriek-composition} (by the very properties of that lemma).
This proves (2) and hence also (1) by the compatibility of the
maps of Lemma \ref{lemma-lqf-separated-shriek-composition}
and Lemma \ref{lemma-f-shriek-composition}.
Next, for any scheme $Z'$ over $Z$, we obtain the compatibility in (5)
for the map $(g' \circ f')_! \to g'_! \circ f'_!$
constructed using the open covering $Y' = \bigcup b^{-1}(Y_i)$.
This is clear from the corresponding compatibility of the maps
constructed in Lemma \ref{lemma-lqf-separated-shriek-composition}.
In particular, we can consider a geometric point
$\overline{z} : \Spec(k) \to Z$. Since
$X_{\overline{z}} \to Y_{\overline{z}} \to \Spec(k)$
are separated maps, we find that the base change of
$(g \circ f)_!\mathcal{F} \to g_! f_! \mathcal{F}$
by $\overline{z}$ is equal to the map of
Lemma \ref{lemma-f-shriek-composition}.
The reader then immediately sees that we obtain property (3).
Of course, property (3) guarantees that our transformation of functors
$(g \circ f)_! \to g_! \circ f_!$ constructed using the open covering
$Y = \bigcup Y_i$ doesn't depend on the choice of this open covering.
Finally, property (4) follows by looking at what happens on stalks
using the already proven property (3).
\end{proof}











\section{Upper shriek for locally quasi-finite morphisms}
\label{section-duality-locally-quasi-finite}

\noindent
For a locally quasi-finite morphism $f : X \to Y$ of schemes, the
functor $f_! : \textit{Ab}(X_\etale) \to \textit{Ab}(Y_\etale)$ commutes
with direct sums and is exact, see Lemma \ref{lemma-lqf-f-shriek-stalk}.
This suggests that it has a right adjoint which we will denote $f^!$.

\medskip\noindent
Warning: This functor is the non-derived version!

\begin{lemma}
\label{lemma-lqf-f-upper-shriek}
Let $f : X \to Y$ be a locally quasi-finite morphism of schemes. The functor
$f_! : \textit{Ab}(X_\etale) \to \textit{Ab}(Y_\etale)$
has a right adjoint
$f^! : \textit{Ab}(Y_\etale) \to \textit{Ab}(X_\etale)$.
Moreover, we have $f^!(\overline{y}_*A) =
\prod_{f(\overline{x}) = \overline{y}} \overline{x}_*A$.
\end{lemma}

\begin{proof}
Let $E \subset \Ob(\textit{Ab}(Y_\etale))$ be the class consisting of
products of skyscraper sheaves. We claim that
\begin{enumerate}
\item every $\mathcal{G}$ in $\textit{Ab}(Y_\etale)$ is a subsheaf
of an element of $E$, and
\item for every $\mathcal{G} \in E$ there exists an object
$\mathcal{H}$ of $\textit{Ab}(X_\etale)$ such that
$\Hom(f_!\mathcal{F}, \mathcal{G}) = \Hom(\mathcal{F}, \mathcal{H})$
functorially in $\mathcal{F}$.
\end{enumerate}
Once the claim has been verified, the dual of
Homology, Lemma \ref{homology-lemma-partially-defined-adjoint}
produces the adjoint functor $f^!$.

\medskip\noindent
Part (1) is true because we can map $\mathcal{G}$ to the sheaf
$\prod \overline{y}_*\mathcal{G}_{\overline{y}}$ where the
product is over all geometric points of $Y$. This is an injection by
\'Etale Cohomology, Theorem \ref{etale-cohomology-theorem-exactness-stalks}.
(This is the first step in the Godement resolution when
done in the setting of abelian sheaves on topological spaces.)

\medskip\noindent
Part (2) and the final statement of the lemma can be seen as follows.
Suppose that
$\mathcal{G} = \prod \overline{y}_*A_{\overline{y}}$
for some abelian groups $A_{\overline{y}}$. Then
$$
\Hom(f_!\mathcal{F}, \mathcal{G}) =
\prod \Hom(f_!\mathcal{F}, \overline{y}_*A_{\overline{y}})
$$
Thus it suffices to find abelian sheaves $\mathcal{H}_{\overline{y}}$
on $X_\etale$ representing the functors
$\mathcal{F} \mapsto \Hom(f_!\mathcal{F}, \overline{y}_*A_{\overline{y}})$
and to take $\mathcal{H} = \prod \mathcal{H}_{\overline{y}}$.
This reduces us to the case $\mathcal{H} = \overline{y}_*A$
for some fixed geometric point $\overline{y} : \Spec(k) \to Y$
and some fixed abelian group $A$. We claim that in this case
$\mathcal{H} = \prod_{f(\overline{x}) = \overline{y}} \overline{x}_*A$ works.
This will finish the proof of the lemma.
Namely, we have
$$
\Hom(f_!\mathcal{F}, \overline{y}_*A) =
\Hom_{\textit{Ab}}((f_!\mathcal{F})_{\overline{y}}, A) =
\Hom_{\textit{Ab}}(\bigoplus\nolimits_{f(\overline{x}) = \overline{y}}
\mathcal{F}_{\overline{x}}, A)
$$
by the description of stalks in
Lemma \ref{lemma-lqf-f-shriek-stalk}
on the one hand and on the other hand we have
$$
\Hom(\mathcal{F}, \mathcal{H}) =
\prod\nolimits_{f(\overline{x}) = \overline{y}}
\Hom(\mathcal{F}, \overline{x}_*A) =
\prod\nolimits_{f(\overline{x}) = \overline{y}}
\Hom_{\textit{Ab}}(\mathcal{F}_{\overline{x}}, A)
$$
We leave it to the reader to identify these as functors of $\mathcal{F}$.
\end{proof}

\begin{lemma}
\label{lemma-etale-upper-shriek}
Let $j : U \to X$ be an \'etale morphism. Then $j^! = j^{-1}$.
\end{lemma}

\begin{proof}
This is true because $j_!$ as defined in
Section \ref{section-finite-support}
agrees with $j_!$ as defined in \'Etale Cohomology, Section
\ref{etale-cohomology-section-extension-by-zero}, see
Lemma \ref{lemma-finite-support-etale-shriek}. Finally, in
\'Etale Cohomology, Section \ref{etale-cohomology-section-extension-by-zero}
the functor $j_!$ is defined
as the left adjoint of $j^{-1}$ and hence we conclude by
uniqueness of adjoint functors.
\end{proof}

\begin{lemma}
\label{lemma-upper-shriek-restriction}
Let $f : X \to Y$ and $g : Y \to Z$ be separated and locally quasi-finite
morphisms. There is a canonical isomorphism $(g \circ f)^! \to f^! \circ g^!$.
Given a third locally quasi-finite morphism $h : Z \to T$
the diagram
$$
\xymatrix{
(h \circ g \circ f)^! \ar[r] \ar[d] &
f^! \circ (h \circ g)^! \ar[d] \\
(g \circ f)^! \circ h^! \ar[r] & f^! \circ g^! \circ h^!
}
$$
commutes.
\end{lemma}

\begin{proof}
By uniqueness of adjoint functors, this immediately translates
into the corresponding (dual) statement for the functors $f_!$.
See Lemma \ref{lemma-lqf-shriek-composition}.
\end{proof}

\begin{lemma}
\label{lemma-upper-shriek-restriction-etale}
Let $j : U \to X$ and $j' : V \to U$ be \'etale morphisms.
The isomorphism $(j \circ j')^{-1} = (j')^{-1} \circ j^{-1}$
and the isomorphism $(j \circ j')^! = (j')^! \circ j^!$ of
Lemma \ref{lemma-upper-shriek-restriction}
agree via the isomorphism of Lemma \ref{lemma-etale-upper-shriek}.
\end{lemma}

\begin{proof}
Omitted.
\end{proof}

\begin{lemma}
\label{lemma-lqf-base-change-upper-shriek}
Consider a cartesian square
$$
\xymatrix{
X' \ar[r]_{g'} \ar[d]_{f'} & X \ar[d]^f \\
Y' \ar[r]^g & Y
}
$$
of schemes with $f$ locally quasi-finite. For any abelian sheaf $\mathcal{F}$
on $Y'_\etale$ we have $(g')_*(f')^!\mathcal{F} = f^!g_*\mathcal{F}$.
\end{lemma}

\begin{proof}
By uniqueness of adjoint functors, this follows from
the corresponding (dual) statement for the functors $f_!$.
See Lemma \ref{lemma-lqf-base-change-f-shriek}.
\end{proof}

\begin{remark}
\label{remark-pointed-sets}
The material in this section can be generalized to sheaves of pointed sets.
Namely, for a site $\mathcal{C}$ denote $\Sh^*(\mathcal{C})$ the category of
sheaves of pointed sets. The constructions in this and the preceding section
apply, mutatis mutandis, to sheaves of pointed sets. Thus given a locally
quasi-finite morphism $f : X \to Y$ of schemes we obtain
an adjoint pair of functors
$$
f_! : \Sh^*(X_\etale) \longrightarrow \Sh^*(Y_\etale)
\quad\text{and}\quad
f^! : \Sh^*(Y_\etale) \longrightarrow \Sh^*(X_\etale)
$$
such that for every geometric point $\overline{y}$ of $Y$ there are
isomorphisms
$$
(f_!\mathcal{F})_{\overline{y}} =
\coprod\nolimits_{f(\overline{x}) = \overline{y}}
\mathcal{F}_{\overline{x}}
$$
(coproduct taken in the category of pointed sets) functorial in
$\mathcal{F} \in \Sh^*(X_\etale)$ and isomorphisms
$$
f^!(\overline{y}_*S) =
\prod\nolimits_{f(\overline{x}) = \overline{y}}
\overline{x}_*S
$$
functorial in the pointed set $S$. If
$F : \textit{Ab}(X_\etale) \to \Sh^*(X_\etale)$ and
$F : \textit{Ab}(Y_\etale) \to \Sh^*(Y_\etale)$
denote the forgetful functors, compatibility between the constructions
will guarantee the existence of canonical maps
$$
f_!F(\mathcal{F}) \longrightarrow F(f_!\mathcal{F})
$$
functorial in $\mathcal{F} \in \textit{Ab}(X_\etale)$ and
$$
F(f^!\mathcal{G}) \longrightarrow f^!F(\mathcal{G})
$$
functorial in $\mathcal{G} \in \textit{Ab}(Y_\etale)$
which produce the obvious maps on stalks, resp.\ skyscraper sheaves.
In fact, the transformation $F \circ f^! \to f^! \circ F$ is an isomorphism
(because $f^!$ commutes with products).
\end{remark}







\section{Derived upper shriek for locally quasi-finite morphisms}
\label{section-derived-duality-locally-quasi-finite}

\noindent
We can take the derived versions of the functors in
Section \ref{section-duality-locally-quasi-finite}
and obtain the following.

\begin{lemma}
\label{lemma-lqf-shriek-derived}
Let $f : X \to Y$ be a locally quasi-finite morphism of schemes. The functors
$f_!$ and $f^!$ of Definition \ref{definition-f-shriek-lqf} and
Lemma \ref{lemma-lqf-f-upper-shriek}
induce adjoint functors $f_! : D(X_\etale) \to D(Y_\etale)$
and $Rf^! : D(Y_\etale) \to D(X_\etale)$ on derived categories.
\end{lemma}

\noindent
In the separated case the functor $f_!$ is defined in
Section \ref{section-compact-support}.

\begin{proof}
This follows immediately from
Derived Categories, Lemma \ref{derived-lemma-derived-adjoint-functors},
the fact that $f_!$ is exact (Lemma \ref{lemma-lqf-f-shriek-stalk})
and hence $Lf_! = f_!$
and the fact that we have enough K-injective complexes of abelian sheaves
on $Y_\etale$ so that $Rf^!$ is defined.
\end{proof}

\begin{lemma}
\label{lemma-shriek-lqf-and-proper}
Consider a commutative diagram of schemes
$$
\xymatrix{
X \ar[r]_f \ar[rd]_g & Y \ar[d]^h \\
& Z
}
$$
with $f$ and $g$ locally quasi-finite and $h$ proper. For any torsion ring
$\Lambda$ and $K$ in $D(X_\etale, \Lambda)$ there is a canonical
isomorphism $g_!K \to Rh_*(f_!K)$ in $D(Z_\etale, \Lambda)$.
\end{lemma}

\begin{proof}
Represent $K$ by a complex $\mathcal{K}^\bullet$ of sheaves of
$\Lambda$-modules on $X_\etale$. Choose a quasi-isomorphism
$f_!\mathcal{K}^\bullet \to \mathcal{I}^\bullet$ into a K-injective
complex $\mathcal{I}^\bullet$ of sheaves of $\Lambda$-modules on $Y_\etale$.
Consider the map
$$
g_!\mathcal{K}^\bullet =
h_!f_!\mathcal{K}^\bullet =
h_*f_!\mathcal{K}^\bullet
\longrightarrow
h_*\mathcal{I}^\bullet
$$
where the equalities are
Lemmas \ref{lemma-lqf-separated-shriek-composition}
(or the easier Lemma \ref{lemma-f-shriek-composition} if $f$ is separated)
and \ref{lemma-proper-f-shriek}. This map of complexes determines the map
$g_!K \to Rh_*(f_!K)$ of the statement of the lemma.
To check the map is an isomorphism we may work locally on $Z$.
Hence we may assume that the dimension of fibres of $h$ is bounded,
see \'Etale Cohomology, Lemma
\ref{etale-cohomology-lemma-morphism-finite-type-bounded-dimension}.
Then we see that $Rh_*$ has finite cohomological dimension, see
\'Etale Cohomology, Lemma
\ref{etale-cohomology-lemma-cohomological-dimension-proper}.
Hence by Derived Categories, Lemma \ref{derived-lemma-unbounded-right-derived},
if we show that $R^qh_*(f_!\mathcal{F}) = 0$ for $q > 0$
and any sheaf $\mathcal{F}$ of $\Lambda$-modules on $X_\etale$, then
$h_*f_!\mathcal{K}^\bullet \to h_*\mathcal{I}^\bullet$ is a quasi-isomorphism.

\medskip\noindent
Observe that $\mathcal{G} = f_!\mathcal{F}$ is a sheaf of $\Lambda$-modules
on $Y$ whose stalks are nonzero only at points $y \in Y$ such that
$\kappa(y)/\kappa(h(y))$ is a finite extension. This follows from the
description of stalks of $f_!\mathcal{F}$ in
Lemma \ref{lemma-lqf-f-shriek-stalk}
and the fact that both $f$ and $g$ are locally quasi-finite.
Hence by the proper base change theorem (\'Etale cohomology, Lemma
\ref{etale-cohomology-lemma-proper-base-change-stalk})
it suffices to show that $H^q(Y_{\overline{z}}, \mathcal{H}) = 0$
where $\mathcal{H}$ is a sheaf on the proper scheme $Y_{\overline{z}}$
over $\kappa(\overline{z})$ whose support is contained in the set
of closed points. Thus the required vanishing by \'Etale Cohomology, Lemma
\ref{etale-cohomology-lemma-supported-in-closed-points}.
\end{proof}












\section{Derived lower shriek via compactifications}
\label{section-derived-lower-shriek-compactification}

\noindent
Let $f : X \to Y$ be a finite type separated morphism of schemes
with $Y$ quasi-compact and quasi-separated. Choose a compactification
$j : X \to \overline{X}$ over $Y$, see
More on Flatness, Theorem \ref{flat-theorem-nagata}.
Given a torsion ring $\Lambda$ we define
$$
Rf_! = R\overline{f}_* \circ j_! :
D(X_\etale, \Lambda)
\longrightarrow
D(Y_\etale, \Lambda)
$$
Here is the obligatory lemma.

\begin{lemma}
\label{lemma-shriek-well-defined}
Let $f : X \to Y$ be a finite type separated morphism of quasi-compact
and quasi-separated schemes. The functor $Rf_!$ is, up to canonical
isomorphism, independent of the choice of the compactification.
\end{lemma}

\begin{proof}
Consider the category of compactifications of $X$ over $Y$, which is
cofiltered according to More on Flatness, Theorem \ref{flat-theorem-nagata} and
Lemmas \ref{flat-lemma-compactifications-cofiltered} and
\ref{flat-lemma-compactifyable}.
To every choice of a compactification
$$
j : X \to \overline{X},\quad \overline{f} : \overline{X} \to Y
$$
the construction above associates the functor $R\overline{f}_* \circ j_! :
D(X_\etale, \Lambda) \to D(Y_\etale, \Lambda)$.
Let's be a little more explicit. Given a complex $\mathcal{K}^\bullet$
of sheaves of $\Lambda$-modules on $X_\etale$, we choose a quasi-isomorphism
$j_!\mathcal{K}^\bullet \to \mathcal{I}^\bullet$ into a K-injective
complex of sheaves of $\Lambda$-modules on $\overline{X}_\etale$.
Then our functor sends $\mathcal{K}^\bullet$ to
$\overline{f}_*\mathcal{I}^\bullet$.

\medskip\noindent
Suppose given a morphism $g : \overline{X}_1 \to \overline{X}_2$
between compactifications $j_i : X \to \overline{X}_i$ over $Y$.
Then we get an isomorphism
$$
R\overline{f}_{2, *} \circ j_{2, !} =
R\overline{f}_{2, *} \circ Rg_* \circ j_{1, !} =
R\overline{f}_{1, *} \circ j_{1, !}
$$
using Lemma \ref{lemma-shriek-lqf-and-proper} in the first equality.

\medskip\noindent
To finish the proof, since the category of compactifications of $X$ over $Y$
is cofiltered, it suffices to show compositions of morphisms of
compactifications of $X$ over $Y$ are turned into compositions of
isomorphisms of functors\footnote{Namely, if $\alpha, \beta : F \to G$
are morphisms of functors and $\gamma : G \to H$ is an isomorphism
of functors such that $\gamma \circ \alpha = \gamma \circ \beta$, then
we conclude $\alpha = \beta$.}. To do this, suppose that
$j_3 : X \to \overline{X}_3$
is a third compactification and that $h : \overline{X}_2 \to \overline{X}_3$
is a morphism of compactifications. Then we have to show that the
composition
$$
R\overline{f}_{3, *} \circ j_{3, !} =
R\overline{f}_{3, *} \circ Rh_* \circ j_{2, !} =
R\overline{f}_{2, *} \circ j_{2, !} =
R\overline{f}_{2, *} \circ Rg_* \circ j_{1, !} =
R\overline{f}_{1, *} \circ j_{1, !}
$$
is equal to the isomorphism of functors constructed using simply
$j_3$, $g \circ h$, and $j_1$. A calculation shows that it suffices to
prove that the composition of the maps
$$
j_{3, !} \to Rh_* \circ j_{2, !} \to Rh_* \circ Rg_* \circ j_{1, !}
$$
of Lemma \ref{lemma-shriek-lqf-and-proper} agrees with the corresponding
map $j_{3, !} \to R(h \circ g)_* \circ j_{1, !}$
via the identification $R(h \circ g)_* = Rh_* \circ Rg_*$.
In the proof of Lemma \ref{lemma-shriek-lqf-and-proper}
we saw that these maps come from the corresponding maps
$j_{3, !} \to h_! \circ j_{2, !} \to h_! \circ g_! \circ j_{1, !}$ and
$j_{3, !} \to (h \circ g)_! \circ j_{1, !} \to h_! \circ g_! \circ j_{1, !}$
on (complexes of) abelian sheaves with arrows as in
Lemma \ref{lemma-f-shriek-composition}.
(Namely, the proof shows for example that $Rh_*j_{2, !}\mathcal{F}^\bullet$
is represented by
$h_*j_{2, !}\mathcal{F}^\bullet = h_!j_{2, !}\mathcal{F}^\bullet$
and the map $j_{3, !}\mathcal{F}^\bullet \to Rh_*j_{2, !}\mathcal{F}^\bullet$
is represented by the map of complexes
$j_{3, !}\mathcal{F}^\bullet \to h_!j_{2, !}\mathcal{F}^\bullet$
given by the construction in Lemma \ref{lemma-f-shriek-composition}.)
Then finally we have the equality because these maps agree
by the first property of Remark \ref{remark-f-shriek-base-change-composition}.
\end{proof}
































\section{Other chapters}

\begin{multicols}{2}
\begin{enumerate}
\item \hyperref[introduction-section-phantom]{Introduction}
\item \hyperref[conventions-section-phantom]{Conventions}
\item \hyperref[sets-section-phantom]{Set Theory}
\item \hyperref[categories-section-phantom]{Categories}
\item \hyperref[topology-section-phantom]{Topology}
\item \hyperref[sheaves-section-phantom]{Sheaves on Spaces}
\item \hyperref[algebra-section-phantom]{Commutative Algebra}
\item \hyperref[sites-section-phantom]{Sites and Sheaves}
\item \hyperref[homology-section-phantom]{Homological Algebra}
\item \hyperref[derived-section-phantom]{Derived Categories}
\item \hyperref[more-algebra-section-phantom]{More Algebra}
\item \hyperref[simplicial-section-phantom]{Simplicial Methods}
\item \hyperref[modules-section-phantom]{Sheaves of Modules}
\item \hyperref[sites-modules-section-phantom]{Modules on Sites}
\item \hyperref[injectives-section-phantom]{Injectives}
\item \hyperref[cohomology-section-phantom]{Cohomology of Sheaves}
\item \hyperref[sites-cohomology-section-phantom]{Cohomology on Sites}
\item \hyperref[hypercovering-section-phantom]{Hypercoverings}
\item \hyperref[schemes-section-phantom]{Schemes}
\item \hyperref[constructions-section-phantom]{Constructions of Schemes}
\item \hyperref[properties-section-phantom]{Properties of Schemes}
\item \hyperref[morphisms-section-phantom]{Morphisms of Schemes}
\item \hyperref[coherent-section-phantom]{Coherent Cohomology}
\item \hyperref[divisors-section-phantom]{Divisors}
\item \hyperref[limits-section-phantom]{Limits of Schemes}
\item \hyperref[varieties-section-phantom]{Varieties}
\item \hyperref[chow-section-phantom]{Chow Homology}
\item \hyperref[topologies-section-phantom]{Topologies on Schemes}
\item \hyperref[descent-section-phantom]{Descent}
\item \hyperref[more-morphisms-section-phantom]{More on Morphisms}
\item \hyperref[flat-section-phantom]{More on Flatness}
\item \hyperref[groupoids-section-phantom]{Groupoid Schemes}
\item \hyperref[more-groupoids-section-phantom]{More on Groupoid Schemes}
\item \hyperref[etale-section-phantom]{\'Etale Morphisms of Schemes}
\item \hyperref[etale-cohomology-section-phantom]{\'Etale Cohomology}
\item \hyperref[spaces-section-phantom]{Algebraic Spaces}
\item \hyperref[spaces-properties-section-phantom]{Properties of Algebraic Spaces}
\item \hyperref[spaces-morphisms-section-phantom]{Morphisms of Algebraic Spaces}
\item \hyperref[spaces-topologies-section-phantom]{Topologies on Algebraic Spaces}
\item \hyperref[spaces-descent-section-phantom]{Descent and Algebraic Spaces}
\item \hyperref[spaces-more-morphisms-section-phantom]{More on Morphisms of Spaces}
\item \hyperref[quot-section-phantom]{Quot and Hilbert Spaces}
\item \hyperref[stacks-section-phantom]{Stacks}
\item \hyperref[spaces-groupoids-section-phantom]{Groupoids in Algebraic Spaces}
\item \hyperref[spaces-more-groupoids-section-phantom]{More on Groupoids in Spaces}
\item \hyperref[bootstrap-section-phantom]{Bootstrap}
\item \hyperref[examples-stacks-section-phantom]{Examples of Stacks}
\item \hyperref[groupoids-quotients-section-phantom]{Quotients of Groupoids}
\item \hyperref[algebraic-section-phantom]{Algebraic Stacks}
\item \hyperref[criteria-section-phantom]{Criteria for Representability}
\item \hyperref[stacks-properties-section-phantom]{Properties of Algebraic Stacks}
\item \hyperref[stacks-morphisms-section-phantom]{Morphisms of Algebraic Stacks}
\item \hyperref[examples-section-phantom]{Examples}
\item \hyperref[exercises-section-phantom]{Exercises}
\item \hyperref[guide-section-phantom]{Guide to Literature}
\item \hyperref[desirables-section-phantom]{Desirables}
\item \hyperref[coding-section-phantom]{Coding Style}
\item \hyperref[fdl-section-phantom]{GNU Free Documentation License}
\item \hyperref[index-section-phantom]{Auto Generated Index}
\end{enumerate}
\end{multicols}


\bibliography{my}
\bibliographystyle{amsalpha}

\end{document}
