\documentclass{amsart}

% The following AMS packages are automatically loaded with
% the amsart documentclass:
%\usepackage{amsmath}
%\usepackage{amssymb}
%\usepackage{amsthm}

% For commutative diagrams you can use
% \usepackage{amscd}
\usepackage[all]{xy}

% We use 2cell for 2-commutative diagrams.
\xyoption{2cell}
\UseAllTwocells

% To put source file link in headers.
% Change "template.tex" to "this_filename.tex"
% \usepackage{fancyhdr}
% \pagestyle{fancy}
% \lhead{}
% \chead{}
% \rhead{Source file: \url{template.tex}}
% \lfoot{}
% \cfoot{\thepage}
% \rfoot{}
% \renewcommand{\headrulewidth}{0pt}
% \renewcommand{\footrulewidth}{0pt}
% \renewcommand{\headheight}{12pt}

\usepackage{multicol}

% For cross-file-references
\usepackage{xr-hyper}

% Package for hypertext links:
\usepackage{hyperref}

% For any local file, say "hello.tex" you want to link to please
% use \externaldocument[hello-]{hello}
\externaldocument[introduction-]{introduction}
\externaldocument[conventions-]{conventions}
\externaldocument[sets-]{sets}
\externaldocument[categories-]{categories}
\externaldocument[topology-]{topology}
\externaldocument[sheaves-]{sheaves}
\externaldocument[algebra-]{algebra}
\externaldocument[sites-]{sites}
\externaldocument[homology-]{homology}
\externaldocument[simplicial-]{simplicial}
\externaldocument[modules-]{modules}
\externaldocument[injectives-]{injectives}
\externaldocument[cohomology-]{cohomology}
\externaldocument[schemes-]{schemes}
\externaldocument[etale-]{etale}
\externaldocument[spaces-]{spaces}
\externaldocument[stacks-]{stacks}
\externaldocument[stacks-groupoids-]{stacks-groupoids}
\externaldocument[algebraic-]{algebraic}
\externaldocument[flat-]{flat}
\externaldocument[desirables-]{desirables}
\externaldocument[hypercovering-]{hypercovering}
\externaldocument[coding-]{coding}
\externaldocument[fdl-]{fdl}

% Theorem environments.
%
\theoremstyle{plain}
\newtheorem{theorem}[subsection]{Theorem}
\newtheorem{proposition}[subsection]{Proposition}
\newtheorem{lemma}[subsection]{Lemma}

\theoremstyle{definition}
\newtheorem{definition}[subsection]{Definition}
\newtheorem{example}[subsection]{Example}
\newtheorem{exercise}[subsection]{Exercise}
\newtheorem{situation}[subsection]{Situation}

\theoremstyle{remark}
\newtheorem{remark}[subsection]{Remark}
\newtheorem{remarks}[subsection]{Remarks}

\numberwithin{equation}{subsection}
