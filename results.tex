\IfFileExists{stacks-project.cls}{%
\documentclass{stacks-project}
}{%
\documentclass{amsart}
}

% The following AMS packages are automatically loaded with
% the amsart documentclass:
%\usepackage{amsmath}
%\usepackage{amssymb}
%\usepackage{amsthm}

% For dealing with references we use the comment environment
\usepackage{verbatim}
\newenvironment{reference}{\comment}{\endcomment}
%\newenvironment{reference}{}{}
\newenvironment{slogan}{\comment}{\endcomment}
\newenvironment{history}{\comment}{\endcomment}

% For commutative diagrams you can use
% \usepackage{amscd}
\usepackage[all]{xy}

% We use 2cell for 2-commutative diagrams.
\xyoption{2cell}
\UseAllTwocells

% To put source file link in headers.
% Change "template.tex" to "this_filename.tex"
% \usepackage{fancyhdr}
% \pagestyle{fancy}
% \lhead{}
% \chead{}
% \rhead{Source file: \url{template.tex}}
% \lfoot{}
% \cfoot{\thepage}
% \rfoot{}
% \renewcommand{\headrulewidth}{0pt}
% \renewcommand{\footrulewidth}{0pt}
% \renewcommand{\headheight}{12pt}

\usepackage{multicol}

% For cross-file-references
\usepackage{xr-hyper}

% Package for hypertext links:
\usepackage{hyperref}

% For any local file, say "hello.tex" you want to link to please
% use \externaldocument[hello-]{hello}
\externaldocument[introduction-]{introduction}
\externaldocument[conventions-]{conventions}
\externaldocument[sets-]{sets}
\externaldocument[categories-]{categories}
\externaldocument[topology-]{topology}
\externaldocument[sheaves-]{sheaves}
\externaldocument[sites-]{sites}
\externaldocument[stacks-]{stacks}
\externaldocument[fields-]{fields}
\externaldocument[algebra-]{algebra}
\externaldocument[brauer-]{brauer}
\externaldocument[homology-]{homology}
\externaldocument[derived-]{derived}
\externaldocument[simplicial-]{simplicial}
\externaldocument[more-algebra-]{more-algebra}
\externaldocument[smoothing-]{smoothing}
\externaldocument[modules-]{modules}
\externaldocument[sites-modules-]{sites-modules}
\externaldocument[injectives-]{injectives}
\externaldocument[cohomology-]{cohomology}
\externaldocument[sites-cohomology-]{sites-cohomology}
\externaldocument[dga-]{dga}
\externaldocument[dpa-]{dpa}
\externaldocument[hypercovering-]{hypercovering}
\externaldocument[schemes-]{schemes}
\externaldocument[constructions-]{constructions}
\externaldocument[properties-]{properties}
\externaldocument[morphisms-]{morphisms}
\externaldocument[coherent-]{coherent}
\externaldocument[divisors-]{divisors}
\externaldocument[limits-]{limits}
\externaldocument[varieties-]{varieties}
\externaldocument[topologies-]{topologies}
\externaldocument[descent-]{descent}
\externaldocument[perfect-]{perfect}
\externaldocument[more-morphisms-]{more-morphisms}
\externaldocument[flat-]{flat}
\externaldocument[groupoids-]{groupoids}
\externaldocument[more-groupoids-]{more-groupoids}
\externaldocument[etale-]{etale}
\externaldocument[chow-]{chow}
\externaldocument[intersection-]{intersection}
\externaldocument[pic-]{pic}
\externaldocument[adequate-]{adequate}
\externaldocument[dualizing-]{dualizing}
\externaldocument[duality-]{duality}
\externaldocument[discriminant-]{discriminant}
\externaldocument[local-cohomology-]{local-cohomology}
\externaldocument[curves-]{curves}
\externaldocument[resolve-]{resolve}
\externaldocument[models-]{models}
\externaldocument[pione-]{pione}
\externaldocument[etale-cohomology-]{etale-cohomology}
\externaldocument[proetale-]{proetale}
\externaldocument[crystalline-]{crystalline}
\externaldocument[spaces-]{spaces}
\externaldocument[spaces-properties-]{spaces-properties}
\externaldocument[spaces-morphisms-]{spaces-morphisms}
\externaldocument[decent-spaces-]{decent-spaces}
\externaldocument[spaces-cohomology-]{spaces-cohomology}
\externaldocument[spaces-limits-]{spaces-limits}
\externaldocument[spaces-divisors-]{spaces-divisors}
\externaldocument[spaces-over-fields-]{spaces-over-fields}
\externaldocument[spaces-topologies-]{spaces-topologies}
\externaldocument[spaces-descent-]{spaces-descent}
\externaldocument[spaces-perfect-]{spaces-perfect}
\externaldocument[spaces-more-morphisms-]{spaces-more-morphisms}
\externaldocument[spaces-flat-]{spaces-flat}
\externaldocument[spaces-groupoids-]{spaces-groupoids}
\externaldocument[spaces-more-groupoids-]{spaces-more-groupoids}
\externaldocument[bootstrap-]{bootstrap}
\externaldocument[spaces-pushouts-]{spaces-pushouts}
\externaldocument[groupoids-quotients-]{groupoids-quotients}
\externaldocument[spaces-more-cohomology-]{spaces-more-cohomology}
\externaldocument[spaces-simplicial-]{spaces-simplicial}
\externaldocument[formal-spaces-]{formal-spaces}
\externaldocument[restricted-]{restricted}
\externaldocument[spaces-resolve-]{spaces-resolve}
\externaldocument[formal-defos-]{formal-defos}
\externaldocument[defos-]{defos}
\externaldocument[cotangent-]{cotangent}
\externaldocument[examples-defos-]{examples-defos}
\externaldocument[algebraic-]{algebraic}
\externaldocument[examples-stacks-]{examples-stacks}
\externaldocument[stacks-sheaves-]{stacks-sheaves}
\externaldocument[criteria-]{criteria}
\externaldocument[artin-]{artin}
\externaldocument[quot-]{quot}
\externaldocument[stacks-properties-]{stacks-properties}
\externaldocument[stacks-morphisms-]{stacks-morphisms}
\externaldocument[stacks-limits-]{stacks-limits}
\externaldocument[stacks-cohomology-]{stacks-cohomology}
\externaldocument[stacks-perfect-]{stacks-perfect}
\externaldocument[stacks-introduction-]{stacks-introduction}
\externaldocument[stacks-more-morphisms-]{stacks-more-morphisms}
\externaldocument[stacks-geometry-]{stacks-geometry}
\externaldocument[moduli-]{moduli}
\externaldocument[moduli-curves-]{moduli-curves}
\externaldocument[examples-]{examples}
\externaldocument[exercises-]{exercises}
\externaldocument[guide-]{guide}
\externaldocument[desirables-]{desirables}
\externaldocument[coding-]{coding}
\externaldocument[obsolete-]{obsolete}
\externaldocument[fdl-]{fdl}
\externaldocument[index-]{index}

% Theorem environments.
%
\theoremstyle{plain}
\newtheorem{theorem}[subsection]{Theorem}
\newtheorem{proposition}[subsection]{Proposition}
\newtheorem{lemma}[subsection]{Lemma}

\theoremstyle{definition}
\newtheorem{definition}[subsection]{Definition}
\newtheorem{example}[subsection]{Example}
\newtheorem{exercise}[subsection]{Exercise}
\newtheorem{situation}[subsection]{Situation}

\theoremstyle{remark}
\newtheorem{remark}[subsection]{Remark}
\newtheorem{remarks}[subsection]{Remarks}

\numberwithin{equation}{subsection}

% Macros
%
\def\lim{\mathop{\rm lim}\nolimits}
\def\colim{\mathop{\rm colim}\nolimits}
\def\Spec{\mathop{\rm Spec}}
\def\Hom{\mathop{\rm Hom}\nolimits}
\def\Ext{\mathop{\rm Ext}\nolimits}
\def\SheafHom{\mathop{\mathcal{H}\!{\it om}}\nolimits}
\def\SheafExt{\mathop{\mathcal{E}\!{\it xt}}\nolimits}
\def\Sch{\textit{Sch}}
\def\Mor{\mathop{\rm Mor}\nolimits}
\def\Ob{\mathop{\rm Ob}\nolimits}
\def\Sh{\mathop{\textit{Sh}}\nolimits}
\def\NL{\mathop{N\!L}\nolimits}
\def\proetale{{pro\text{-}\acute{e}tale}}
\def\etale{{\acute{e}tale}}
\def\QCoh{\textit{QCoh}}
\def\Ker{\mathop{\rm Ker}}
\def\Im{\mathop{\rm Im}}
\def\Coker{\mathop{\rm Coker}}
\def\Coim{\mathop{\rm Coim}}

%
% Macros for moduli stacks/spaces
%
\def\QCohstack{\mathcal{QC}\!{\it oh}}
\def\Cohstack{\mathcal{C}\!{\it oh}}
\def\Spacesstack{\mathcal{S}\!{\it paces}}
\def\Quotfunctor{{\rm Quot}}
\def\Hilbfunctor{{\rm Hilb}}
\def\Curvesstack{\mathcal{C}\!{\it urves}}
\def\Polarizedstack{\mathcal{P}\!{\it olarized}}
\def\Complexesstack{\mathcal{C}\!{\it omplexes}}
% \Pic is the operator that assigns to X its picard group, usage \Pic(X)
% \Picardstack_{X/B} denotes the Picard stack of X over B
% \Picardfunctor_{X/B} denotes the Picard functor of X over B
\def\Pic{\mathop{\rm Pic}\nolimits}
\def\Picardstack{\mathcal{P}\!{\it ic}}
\def\Picardfunctor{{\rm Pic}}
\def\Deformationcategory{\mathcal{D}\!{\it ef}}


% OK, start here.
%
\begin{document}

\title{Results on Schemes}


\maketitle

\tableofcontents

\section{Introduction}
\label{section-introduction}

\noindent
In this chapter we start proving some basic theorems of algebraic geometry.
A basic reference is \cite{EGA}.















\section{Absolute Noetherian Approximation}
\label{section-approximation}

\noindent
See Categories, Section \ref{categories-section-posets-limits}
for our conventions regarding directed systems.

\begin{lemma}
\label{lemma-directed-inverse-system-affine schemes-has-limit}
Let $I$ be a directed partially ordered set.
Let $(S_i, f_{ii'})$ be an inverse system of
schemes over $I$.  If all the schemes $S_i$
are affine, then the limit $S = \text{lim}_i\ S_i$ exists
in the category of schemes.
In fact $S$ is affine and $S = \text{Spec}(\text{colim}_i\ R_i)$
with $R_i = \Gamma(S_i, \mathcal{O})$.
\end{lemma}

\begin{proof}
Just define $S = \text{Spec}(\text{colim}_i\ R_i)$.
It follows from Schemes, Lemma \ref{schemes-lemma-morphism-into-affine}
that $S$ is the limit even in the category of locally ringed spaces.
\end{proof}

\begin{lemma}
\label{lemma-directed-inverse-system-has-limit}
Let $I$ be a directed partially ordered set.
Let $(S_i, f_{ii'})$ be an inverse system of
schemes over $I$. If all the morphisms $f_{ii'} : S_i \to S_{i'}$
are affine, then the limit $\text{lim}_i\ S_i$ exists
in the category of schemes.
Moreover, each of the morphisms $S \to S_i$ is affine.
\end{lemma}

\begin{proof}
Choose $i_0 \in I$. (Note that $I$ is nonempty as the limit is directed.)
For convenience write $S_0 = S_{i_0}$ and $i_0 = 0$.
For every $i \geq 0$ consider the quasi-coherent sheaf of
$\mathcal{O}_{S_0}$-algebras $\mathcal{A}_i = f_{i0,*}\mathcal{O}_{S_i}$.
Recall that $S_i = \underline{\text{Spec}}_{S_0}(\mathcal{A})$,
see Morphisms, Lemma \ref{lemma-affine-equivalence-algebras}.
Set $\mathcal{A} = \text{colim}_{i \geq 0}\ \mathcal{A}_i$.
This is a quasi-coherent sheaf of $\mathcal{O}_{S_0}$-algebras,
see Schemes, Section \ref{schemes-section-quasi-coherent}.
Set $S = \underline{\text{Spec}}_{S_0}(\mathcal{A})$.
By the lemma referenced above we get for $i \geq 0$ affine morphisms
$f_i : S \to S_i$ compatible with the transition morphisms.
By Lemma \ref{lemma-directed-inverse-system-affine schemes-has-limit} above
we see that for any affine open $U_0 \subset S_0$ the
inverse image $U = f_0^{-1}(U_0) \subset S$ is the limit of the
system of opens $U_i = f_i^{-1}(U_0)$, $i \geq 0$ in the
category of schemes.

\medskip\noindent
Let $T$ be a scheme. Let $g_i : T \to S_i$ be a compatible system
of morphisms. To show that $S = \text{lim}_i\ S_i$ we have
to prove there is a unique morphism $g : T \to S$ with
$g_i = f_i \circ g$ for all $i \in I$.
For every $t \in T$ there exists an affine open
$U_0 \subset S_0$ containing $g_0(t)$. Let $V \subset g_0^{-1}(U_0)$
be an affine open neighbourhood containing $t$.
By the remarks above we obtain a unique morphism
$g_V : V \to U = f_0^{-1}(U_0)$ such that $f_i \circ g_V = g_i|_{U_i}$
for all $i$. The open sets $V \subset T$ so constructed form
a basis for the topology of $T$. Hence the morphisms $g_V$ glue to a morphism
$g : T \to S$ because of the uniqueness property. This gives the
desired morphism $g : T \to S$.
\end{proof}

\begin{lemma}
\label{lemma-limit-quasi-affine}
Let $I$ be a directed partially ordered set.
Let $(S_i, f_{ii'})$ be an inverse system of
schemes over $I$. Assume
\begin{enumerate}
\item all the morphisms $f_{ii'} : S_i \to S_{i'}$ are affine,
\item all the schemes $S_i$ are quasi-compact and quasi-separated, and
\item the limit $S = \text{lim}_i\ S_i$ is quasi-affine.
\end{enumerate}
Then for some $i_0 \in I$ the schemes $S_i$ for $i \geq i_0$
are quasi-affine.
\end{lemma}

\begin{proof}
Choose an index $i_0 \in I$. For convenience we write $S_0 = S_{i_0}$ and
$i_0 = 0$. In the proof of Lemma \ref{} we constructed
$S$ as the relative spectrum of
$\text{colim}_{i \geq 0}\ f_{i0,*}\mathcal{O}_{S_i}$.
Each morphism $f_i : S \to S_i$ is affine, in particular $f_0 : S \to S_0$
is affine.
For any $s \in S$ we may choose an affine open
$U_0 \subset S_0$ containing $f_0(s)$. Since $S$ is quasi-affine
we may choose an element $a_s \in \Gamma(S, \mathcal{O}_S)$ such
that $s \in D(a_s) \subset f_0^{-1}(U_0)$, and such that
$D(a_s)$ is affine.
Since $S_0$ is quasi-compact and quasi-separated we see that
$\Gamma(S, \mathcal{O}_S) = 
\Gamma(S_0, \text{colim}_i\ f_{i0,*}\mathcal{O}_{S_i}) =
\text{colim}_i\ \Gamma(S_0, f_{i0,*}\mathcal{O}_{S_i}) =
\text{colim}_i\ \Gamma(S_i, \mathcal{O}_{S_i})$,
see Modules, Lemma \ref{}.
Hence there exists an $i \geq 0$ such that $a_s$
comes from an element $a_{s, i} \in \Gamma(S_i, \mathcal{O}_{S_i})$.
For any $j \geq i$, $j \in I$ we denote $a_{s, j}$
the image of $a_{s, i}$ in the global sections of the
structure sheaf of $S_j$.
Consider the opens $D(a_{s, j}) \subset S_j$
and $U_j = f_{j0}^{-1}(U_0)$. By construction we have that
$D(a_s) \subset f_0^{-1}(U_0)$.
This means that the limit of the schemes
$D(a_{s,j}) \setminus U_j$ is empty.






\end{proof}



\begin{lemma}
\label{lemma-approximate}
Let $S$ be a quasi-compact and quasi-separate scheme.
There exist a directed partially ordered set $I$
and an inverse system of schemes $(S_i, f_{ii'})$ over $I$
such that
\begin{enumerate}
\item the transition morphisms $f_{ii'}$ are affine
\item each $S_i$ is of finite type over $\mathbf{Z}$, and
\item $S = \text{lim}_i\ S_i$.
\end{enumerate}
\end{lemma}

\begin{proof}
Choose an affine open covering $S = \bigcup_{j = 1, \ldots, m} U_j$
with $m$ minimal. We will prove the lemma by induction on $m$.
The lemma is obvious when $m = 1$ since any ring is the
directed colimit of its finitely generated $\mathbf{Z}$-subalgebras.

\medskip\noindent
Hence we may assume that
Write $U_i = \text{Spec}(R_i)$.


For each $1 \leq i < j \leq n$ choose a finite collection
of opens $W_{ijk} \subset U_i \cap U_j$, $k = 1, \ldots, n_{i,j}$
such that $U_i \cap U_j = \bigcup W_{ijk}$ and such that
$W_{ijk}$ is a standard affine open of both $U_i$ and $U_j$.
Choose $f_{ijk} \in R_i$ such that $W_{ijk} = D(f_{ijk})$ as opens
of $U_i = \text{Spec}(R_i)$.
Choose $g_{ijk} \in R_j$ such that $W_{ijk} = D(g_{ijk})$ as opens
of $U_j = \text{Spec}(R_j)$.





\end{proof}











\section{Other chapters}

\begin{multicols}{2}
\begin{enumerate}
\item \hyperref[introduction-section-phantom]{Introduction}
\item \hyperref[conventions-section-phantom]{Conventions}
\item \hyperref[sets-section-phantom]{Set Theory}
\item \hyperref[categories-section-phantom]{Categories}
\item \hyperref[topology-section-phantom]{Topology}
\item \hyperref[sheaves-section-phantom]{Sheaves on Spaces}
\item \hyperref[algebra-section-phantom]{Commutative Algebra}
\item \hyperref[sites-section-phantom]{Sites and Sheaves}
\item \hyperref[homology-section-phantom]{Homological Algebra}
\item \hyperref[derived-section-phantom]{Derived Categories}
\item \hyperref[more-algebra-section-phantom]{More Algebra}
\item \hyperref[simplicial-section-phantom]{Simplicial Methods}
\item \hyperref[modules-section-phantom]{Sheaves of Modules}
\item \hyperref[sites-modules-section-phantom]{Modules on Sites}
\item \hyperref[injectives-section-phantom]{Injectives}
\item \hyperref[cohomology-section-phantom]{Cohomology of Sheaves}
\item \hyperref[sites-cohomology-section-phantom]{Cohomology on Sites}
\item \hyperref[hypercovering-section-phantom]{Hypercoverings}
\item \hyperref[schemes-section-phantom]{Schemes}
\item \hyperref[constructions-section-phantom]{Constructions of Schemes}
\item \hyperref[properties-section-phantom]{Properties of Schemes}
\item \hyperref[morphisms-section-phantom]{Morphisms of Schemes}
\item \hyperref[coherent-section-phantom]{Coherent Cohomology}
\item \hyperref[divisors-section-phantom]{Divisors}
\item \hyperref[limits-section-phantom]{Limits of Schemes}
\item \hyperref[varieties-section-phantom]{Varieties}
\item \hyperref[chow-section-phantom]{Chow Homology}
\item \hyperref[topologies-section-phantom]{Topologies on Schemes}
\item \hyperref[descent-section-phantom]{Descent}
\item \hyperref[more-morphisms-section-phantom]{More on Morphisms}
\item \hyperref[flat-section-phantom]{More on Flatness}
\item \hyperref[groupoids-section-phantom]{Groupoid Schemes}
\item \hyperref[more-groupoids-section-phantom]{More on Groupoid Schemes}
\item \hyperref[etale-section-phantom]{\'Etale Morphisms of Schemes}
\item \hyperref[etale-cohomology-section-phantom]{\'Etale Cohomology}
\item \hyperref[spaces-section-phantom]{Algebraic Spaces}
\item \hyperref[spaces-properties-section-phantom]{Properties of Algebraic Spaces}
\item \hyperref[spaces-morphisms-section-phantom]{Morphisms of Algebraic Spaces}
\item \hyperref[spaces-topologies-section-phantom]{Topologies on Algebraic Spaces}
\item \hyperref[spaces-descent-section-phantom]{Descent and Algebraic Spaces}
\item \hyperref[spaces-more-morphisms-section-phantom]{More on Morphisms of Spaces}
\item \hyperref[quot-section-phantom]{Quot and Hilbert Spaces}
\item \hyperref[stacks-section-phantom]{Stacks}
\item \hyperref[spaces-groupoids-section-phantom]{Groupoids in Algebraic Spaces}
\item \hyperref[spaces-more-groupoids-section-phantom]{More on Groupoids in Spaces}
\item \hyperref[bootstrap-section-phantom]{Bootstrap}
\item \hyperref[examples-stacks-section-phantom]{Examples of Stacks}
\item \hyperref[groupoids-quotients-section-phantom]{Quotients of Groupoids}
\item \hyperref[algebraic-section-phantom]{Algebraic Stacks}
\item \hyperref[criteria-section-phantom]{Criteria for Representability}
\item \hyperref[stacks-properties-section-phantom]{Properties of Algebraic Stacks}
\item \hyperref[stacks-morphisms-section-phantom]{Morphisms of Algebraic Stacks}
\item \hyperref[examples-section-phantom]{Examples}
\item \hyperref[exercises-section-phantom]{Exercises}
\item \hyperref[guide-section-phantom]{Guide to Literature}
\item \hyperref[desirables-section-phantom]{Desirables}
\item \hyperref[coding-section-phantom]{Coding Style}
\item \hyperref[fdl-section-phantom]{GNU Free Documentation License}
\item \hyperref[index-section-phantom]{Auto Generated Index}
\end{enumerate}
\end{multicols}


\bibliography{my}
\bibliographystyle{alpha}

\end{document}
