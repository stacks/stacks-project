\IfFileExists{stacks-project.cls}{%
\documentclass{stacks-project}
}{%
\documentclass{amsart}
}

% The following AMS packages are automatically loaded with
% the amsart documentclass:
%\usepackage{amsmath}
%\usepackage{amssymb}
%\usepackage{amsthm}

% For dealing with references we use the comment environment
\usepackage{verbatim}
\newenvironment{reference}{\comment}{\endcomment}
%\newenvironment{reference}{}{}
\newenvironment{slogan}{\comment}{\endcomment}
\newenvironment{history}{\comment}{\endcomment}

% For commutative diagrams you can use
% \usepackage{amscd}
\usepackage[all]{xy}

% We use 2cell for 2-commutative diagrams.
\xyoption{2cell}
\UseAllTwocells

% To put source file link in headers.
% Change "template.tex" to "this_filename.tex"
% \usepackage{fancyhdr}
% \pagestyle{fancy}
% \lhead{}
% \chead{}
% \rhead{Source file: \url{template.tex}}
% \lfoot{}
% \cfoot{\thepage}
% \rfoot{}
% \renewcommand{\headrulewidth}{0pt}
% \renewcommand{\footrulewidth}{0pt}
% \renewcommand{\headheight}{12pt}

\usepackage{multicol}

% For cross-file-references
\usepackage{xr-hyper}

% Package for hypertext links:
\usepackage{hyperref}

% For any local file, say "hello.tex" you want to link to please
% use \externaldocument[hello-]{hello}
\externaldocument[introduction-]{introduction}
\externaldocument[conventions-]{conventions}
\externaldocument[sets-]{sets}
\externaldocument[categories-]{categories}
\externaldocument[topology-]{topology}
\externaldocument[sheaves-]{sheaves}
\externaldocument[sites-]{sites}
\externaldocument[stacks-]{stacks}
\externaldocument[fields-]{fields}
\externaldocument[algebra-]{algebra}
\externaldocument[brauer-]{brauer}
\externaldocument[homology-]{homology}
\externaldocument[derived-]{derived}
\externaldocument[simplicial-]{simplicial}
\externaldocument[more-algebra-]{more-algebra}
\externaldocument[smoothing-]{smoothing}
\externaldocument[modules-]{modules}
\externaldocument[sites-modules-]{sites-modules}
\externaldocument[injectives-]{injectives}
\externaldocument[cohomology-]{cohomology}
\externaldocument[sites-cohomology-]{sites-cohomology}
\externaldocument[dga-]{dga}
\externaldocument[dpa-]{dpa}
\externaldocument[hypercovering-]{hypercovering}
\externaldocument[schemes-]{schemes}
\externaldocument[constructions-]{constructions}
\externaldocument[properties-]{properties}
\externaldocument[morphisms-]{morphisms}
\externaldocument[coherent-]{coherent}
\externaldocument[divisors-]{divisors}
\externaldocument[limits-]{limits}
\externaldocument[varieties-]{varieties}
\externaldocument[topologies-]{topologies}
\externaldocument[descent-]{descent}
\externaldocument[perfect-]{perfect}
\externaldocument[more-morphisms-]{more-morphisms}
\externaldocument[flat-]{flat}
\externaldocument[groupoids-]{groupoids}
\externaldocument[more-groupoids-]{more-groupoids}
\externaldocument[etale-]{etale}
\externaldocument[chow-]{chow}
\externaldocument[intersection-]{intersection}
\externaldocument[pic-]{pic}
\externaldocument[adequate-]{adequate}
\externaldocument[dualizing-]{dualizing}
\externaldocument[duality-]{duality}
\externaldocument[discriminant-]{discriminant}
\externaldocument[local-cohomology-]{local-cohomology}
\externaldocument[curves-]{curves}
\externaldocument[resolve-]{resolve}
\externaldocument[models-]{models}
\externaldocument[pione-]{pione}
\externaldocument[etale-cohomology-]{etale-cohomology}
\externaldocument[proetale-]{proetale}
\externaldocument[crystalline-]{crystalline}
\externaldocument[spaces-]{spaces}
\externaldocument[spaces-properties-]{spaces-properties}
\externaldocument[spaces-morphisms-]{spaces-morphisms}
\externaldocument[decent-spaces-]{decent-spaces}
\externaldocument[spaces-cohomology-]{spaces-cohomology}
\externaldocument[spaces-limits-]{spaces-limits}
\externaldocument[spaces-divisors-]{spaces-divisors}
\externaldocument[spaces-over-fields-]{spaces-over-fields}
\externaldocument[spaces-topologies-]{spaces-topologies}
\externaldocument[spaces-descent-]{spaces-descent}
\externaldocument[spaces-perfect-]{spaces-perfect}
\externaldocument[spaces-more-morphisms-]{spaces-more-morphisms}
\externaldocument[spaces-flat-]{spaces-flat}
\externaldocument[spaces-groupoids-]{spaces-groupoids}
\externaldocument[spaces-more-groupoids-]{spaces-more-groupoids}
\externaldocument[bootstrap-]{bootstrap}
\externaldocument[spaces-pushouts-]{spaces-pushouts}
\externaldocument[groupoids-quotients-]{groupoids-quotients}
\externaldocument[spaces-more-cohomology-]{spaces-more-cohomology}
\externaldocument[spaces-simplicial-]{spaces-simplicial}
\externaldocument[spaces-duality-]{spaces-duality}
\externaldocument[formal-spaces-]{formal-spaces}
\externaldocument[restricted-]{restricted}
\externaldocument[spaces-resolve-]{spaces-resolve}
\externaldocument[formal-defos-]{formal-defos}
\externaldocument[defos-]{defos}
\externaldocument[cotangent-]{cotangent}
\externaldocument[examples-defos-]{examples-defos}
\externaldocument[algebraic-]{algebraic}
\externaldocument[examples-stacks-]{examples-stacks}
\externaldocument[stacks-sheaves-]{stacks-sheaves}
\externaldocument[criteria-]{criteria}
\externaldocument[artin-]{artin}
\externaldocument[quot-]{quot}
\externaldocument[stacks-properties-]{stacks-properties}
\externaldocument[stacks-morphisms-]{stacks-morphisms}
\externaldocument[stacks-limits-]{stacks-limits}
\externaldocument[stacks-cohomology-]{stacks-cohomology}
\externaldocument[stacks-perfect-]{stacks-perfect}
\externaldocument[stacks-introduction-]{stacks-introduction}
\externaldocument[stacks-more-morphisms-]{stacks-more-morphisms}
\externaldocument[stacks-geometry-]{stacks-geometry}
\externaldocument[moduli-]{moduli}
\externaldocument[moduli-curves-]{moduli-curves}
\externaldocument[examples-]{examples}
\externaldocument[exercises-]{exercises}
\externaldocument[guide-]{guide}
\externaldocument[desirables-]{desirables}
\externaldocument[coding-]{coding}
\externaldocument[obsolete-]{obsolete}
\externaldocument[fdl-]{fdl}
\externaldocument[index-]{index}

% Theorem environments.
%
\theoremstyle{plain}
\newtheorem{theorem}[subsection]{Theorem}
\newtheorem{proposition}[subsection]{Proposition}
\newtheorem{lemma}[subsection]{Lemma}

\theoremstyle{definition}
\newtheorem{definition}[subsection]{Definition}
\newtheorem{example}[subsection]{Example}
\newtheorem{exercise}[subsection]{Exercise}
\newtheorem{situation}[subsection]{Situation}

\theoremstyle{remark}
\newtheorem{remark}[subsection]{Remark}
\newtheorem{remarks}[subsection]{Remarks}

\numberwithin{equation}{subsection}

% Macros
%
\def\lim{\mathop{\mathrm{lim}}\nolimits}
\def\colim{\mathop{\mathrm{colim}}\nolimits}
\def\Spec{\mathop{\mathrm{Spec}}}
\def\Hom{\mathop{\mathrm{Hom}}\nolimits}
\def\Ext{\mathop{\mathrm{Ext}}\nolimits}
\def\SheafHom{\mathop{\mathcal{H}\!\mathit{om}}\nolimits}
\def\SheafExt{\mathop{\mathcal{E}\!\mathit{xt}}\nolimits}
\def\Sch{\mathit{Sch}}
\def\Mor{\operatorname{Mor}\nolimits}
\def\Ob{\mathop{\mathrm{Ob}}\nolimits}
\def\Sh{\mathop{\mathit{Sh}}\nolimits}
\def\NL{\mathop{N\!L}\nolimits}
\def\proetale{{pro\text{-}\acute{e}tale}}
\def\etale{{\acute{e}tale}}
\def\QCoh{\mathit{QCoh}}
\def\Ker{\mathop{\mathrm{Ker}}}
\def\Im{\mathop{\mathrm{Im}}}
\def\Coker{\mathop{\mathrm{Coker}}}
\def\Coim{\mathop{\mathrm{Coim}}}

%
% Macros for moduli stacks/spaces
%
\def\QCohstack{\mathcal{QC}\!\mathit{oh}}
\def\Cohstack{\mathcal{C}\!\mathit{oh}}
\def\Spacesstack{\mathcal{S}\!\mathit{paces}}
\def\Quotfunctor{\mathrm{Quot}}
\def\Hilbfunctor{\mathrm{Hilb}}
\def\Curvesstack{\mathcal{C}\!\mathit{urves}}
\def\Polarizedstack{\mathcal{P}\!\mathit{olarized}}
\def\Complexesstack{\mathcal{C}\!\mathit{omplexes}}
% \Pic is the operator that assigns to X its picard group, usage \Pic(X)
% \Picardstack_{X/B} denotes the Picard stack of X over B
% \Picardfunctor_{X/B} denotes the Picard functor of X over B
\def\Pic{\mathop{\mathrm{Pic}}\nolimits}
\def\Picardstack{\mathcal{P}\!\mathit{ic}}
\def\Picardfunctor{\mathrm{Pic}}
\def\Deformationcategory{\mathcal{D}\!\mathit{ef}}


% OK, start here.
%
\begin{document}

\title{Flatness on Algebraic Spaces}

\maketitle

\phantomsection
\label{section-phantom}

\tableofcontents



\section{Introduction}
\label{section-introduction}

\noindent
In this chapter, we discuss some advanced results on flat modules and
flat morphisms in the setting of algebraic spaces. We strongly encourage
the reader to take a look at the corresponding
chapter in the setting of schemes first, see
More on Flatness, Section \ref{flat-section-introduction}.
A reference is the paper \cite{GruRay} by Raynaud and Gruson.







\section{Impurities}
\label{section-impure}

\noindent
The section is the analogue of
More on Flatness, Section \ref{flat-section-impure}.

\begin{situation}
\label{situation-pre-pure}
Let $S$ be a scheme. Let $f : X \to Y$ be a
finite type, decent\footnote{Quasi-separated morphisms are decent, see
Decent Spaces, Lemma
\ref{decent-spaces-lemma-properties-trivial-implications}.
For any morphism $\Spec(k) \to Y$ where $k$ is a field,
the algebraic space $X_k$ is of finite presentation over $k$
because it is of finite type over $k$ and quasi-separated by
Decent Spaces, Lemma
\ref{decent-spaces-lemma-locally-Noetherian-decent-quasi-separated}.}
morphism of algebraic spaces over $S$. Also, $\mathcal{F}$ is a finite type
quasi-coherent $\mathcal{O}_X$-module. Finally $y \in |Y|$ is a point of $Y$.
\end{situation}

\noindent
In this situation consider a scheme $T$, a morphism $g : T \to Y$,
a point $t \in T$ with $g(t) = y$, a specialization $t' \leadsto t$ in
$T$, and a point $\xi \in |X_T|$ lying over $t'$. Here $X_T = T \times_Y X$.
Picture
\begin{equation}
\label{equation-impurity}
\vcenter{
\xymatrix{
\xi \ar@{|->}[d] & \\
t' \ar@{~>}[r] & t \ar@{|->}[r] \ar[r] & y
}
}
\quad\quad
\vcenter{
\xymatrix{
X_T \ar[d]_{f_T} \ar[r] & X \ar[d]^f \\
T \ar[r]^g & Y
}
}
\end{equation}
Moreover, denote $\mathcal{F}_T$ the pullback of $\mathcal{F}$ to $X_T$.

\begin{definition}
\label{definition-impurity}
In
Situation \ref{situation-pre-pure}
we say a diagram (\ref{equation-impurity}) defines an
{\it impurity of $\mathcal{F}$ above $y$}
if $\xi \in \text{Ass}_{X_T/T}(\mathcal{F}_T)$ and
$t \not \in f_T(\overline{\{\xi\}})$. We will indicate
this by saying ``let $(g : T \to Y, t' \leadsto t, \xi)$ be
an impurity of $\mathcal{F}$ above $y$''.
\end{definition}

\noindent
Another way to say this is: $(g : T \to Y, t' \leadsto t, \xi)$ is
an impurity of $\mathcal{F}$ above $y$ if there exists no specialization
$\xi \leadsto \theta$ in the topological space $|X_T|$ with
$f_T(\theta) = t$. Specializations in non-decent algebraic spaces
do not behave well. If the morphism $f$ is decent, then $X_T$
is a decent algebraic space for all morphisms $g : T \to Y$ as above, see
Decent Spaces, Definition \ref{decent-spaces-definition-relative-conditions}.

\begin{lemma}
\label{lemma-impure-limit}
In Situation \ref{situation-pre-pure}.
Let $(g : T \to S, t' \leadsto t, \xi)$ be an impurity of
$\mathcal{F}$ above $y$. Assume $T = \lim_{i \in I} T_i$ is a directed limit
of affine schemes over $Y$. Then for some $i$ the triple
$(T_i \to Y, t'_i \leadsto t_i, \xi_i)$ is an impurity of
$\mathcal{F}$ above $y$.
\end{lemma}

\begin{proof}
The notation in the statement means this: Let $p_i : T \to T_i$
be the projection morphisms, let $t_i = p_i(t)$ and $t'_i = p_i(t')$.
Finally $\xi_i \in |X_{T_i}|$ is the image of $\xi$. By
Divisors on Spaces, Lemma
\ref{spaces-divisors-lemma-base-change-relative-assassin}
we have $\xi_i \in \text{Ass}_{X_{T_i}/T_i}(\mathcal{F}_{T_i})$.
Thus the only point is to show that
$t_i \not \in f_{T_i}(\overline{\{\xi_i\}})$ for some $i$.

\medskip\noindent
Let $Z_i \subset X_{T_i}$ be the reduced induced scheme structure
on $\overline{\{\xi_i\}} \subset |X_{T_i}|$
and let $Z \subset X_T$ be the reduced induced scheme structure on
$\overline{\{\xi\}} \subset |X_T|$.
Then $Z = \lim Z_i$ by
Limits of Spaces, Lemma \ref{spaces-limits-lemma-inverse-limit-irreducibles}
(the lemma applies because each $X_{T_i}$ is decent).
Choose a field $k$ and a morphism $\Spec(k) \to T$ whose image is $t$.
Then
$$
\emptyset =
Z \times_T \Spec(k) = (\lim Z_i) \times_{(\lim T_i)} \Spec(k)
= \lim Z_i \times_{T_i} \Spec(k)
$$
because limits commute with fibred products (limits commute with limits).
Each $Z_i \times_{T_i} \Spec(k)$ is quasi-compact because $X_{T_i} \to T_i$
is of finite type and hence $Z_i \to T_i$ is of finite type.
Hence $Z_i \times_{T_i} \Spec(k)$ is empty for some $i$ by
Limits of Spaces, Lemma \ref{spaces-limits-lemma-limit-nonempty}.
Since the image of the composition $\Spec(k) \to T \to T_i$ is $t_i$
we obtain what we want.
\end{proof}

\noindent
Impurities go up along flat base change.

\begin{lemma}
\label{lemma-flat-ascent-impurity}
In Situation \ref{situation-pre-pure}.
Let $(Y_1, y_1) \to (Y, y)$ be a morphism of pointed
algebraic spaces over $S$. Assume $Y_1 \to Y$ is flat at $y_1$.
If $(T \to Y, t' \leadsto t, \xi)$ is an impurity of
$\mathcal{F}$ above $y$, then there exists an impurity
$(T_1 \to Y_1, t_1' \leadsto t_1, \xi_1)$ of the pullback
$\mathcal{F}_1$ of $\mathcal{F}$ to $X_1 = Y_1 \times_Y X$
over $y_1$ such that $T_1$ is \'etale over $Y_1 \times_Y T$.
\end{lemma}

\begin{proof}
Choose an \'etale morphism $T_1 \to Y_1 \times_Y T$ where $T_1$
is a scheme and let $t_1 \in T_1$ be a point mapping to $y_1$ and $t$.
It is possible to find a pair $(T_1, t_1)$ like this by
Properties of Spaces, Lemma \ref{spaces-properties-lemma-points-cartesian}.
The morphism of schemes $T_1 \to T$ is flat at $t_1$
(use Morphisms of Spaces, Lemma \ref{spaces-morphisms-lemma-base-change-flat}
and the definition of flat morphisms of algebraic spaces)
there exists a specialization $t'_1 \leadsto t_1$ lying over
$t' \leadsto t$, see
Morphisms, Lemma \ref{morphisms-lemma-generalizations-lift-flat}.
Choose a point $\xi_1 \in |X_{T_1}|$ mapping to $t'_1$
and $\xi$ with $\xi_1 \in \text{Ass}_{X_{T_1}/T_1}(\mathcal{F}_{T_1})$.
point of $\Spec(\kappa(t'_1) \otimes_{\kappa(t')} \kappa(\xi))$.
This is possible by
Divisors on Spaces, Lemma
\ref{spaces-divisors-lemma-base-change-relative-assassin}.
As the closure $Z_1$ of $\{\xi_1\}$ in $|X_{T_1}|$ maps into the
closure of $\{\xi\}$ in $|X_T|$ we conclude that
the image of $Z_1$ in $|T_1|$ cannot contain $t_1$.
Hence $(T_1 \to Y_1, t'_1 \leadsto t_1, \xi_1)$
is an impurity of $\mathcal{F}_1$ above $Y_1$.
\end{proof}

\begin{lemma}
\label{lemma-pure-along-X-y}
In Situation \ref{situation-pre-pure}. Let $\overline{y}$ be a geometric
point lying over $y$. Let $\mathcal{O} = \mathcal{O}_{Y, \overline{y}}$
be the \'etale local ring of $Y$ at $\overline{y}$. Denote
$Y^{sh} = \Spec(\mathcal{O})$, $X^{sh} = X \times_Y Y^{sh}$, and
$\mathcal{F}^{sh}$ the pullback of $\mathcal{F}$ to $X^{sh}$.
The following are equivalent
\begin{enumerate}
\item there exists an impurity
$(Y^{sh} \to Y, y' \leadsto \overline{y}, \xi)$
of $\mathcal{F}$ above $y$,
\item every point of $\text{Ass}_{X^{sh}/Y^{sh}}(\mathcal{F}^{sh})$
specializes to a point of the closed fibre $X_{\overline{y}}$,
\item there exists an impurity $(T \to Y, t' \leadsto t, \xi)$
of $\mathcal{F}$ above $y$ such that $(T, t) \to (Y, y)$ is an
\'etale neighbourhood, and
\item there exists an impurity $(T \to Y, t' \leadsto t, \xi)$
of $\mathcal{F}$ above $y$ such that $T \to Y$ is quasi-finite at $t$.
\end{enumerate}
\end{lemma}

\begin{proof}
That parts (1) and (2) are equivalent is immediate from the definition.

\medskip\noindent
Recall that $\mathcal{O} = \mathcal{O}_{Y, \overline{y}}$
is the filtered colimit of $\mathcal{O}(V)$ over the category
of \'etale neighbourhoods $(V, \overline{v}) \to (Y, \overline{y})$
(Properties of Spaces, Lemma \ref{spaces-properties-lemma-cofinal-etale}).
Moreover, it suffices to consider affine \'etale neighbourhoods $V$.
Hence $Y^{sh} = \Spec(\mathcal{O}) = \lim \Spec(\mathcal{O}(V)) = \lim V$.
Thus we see that (1) implies (3) by Lemma \ref{lemma-impure-limit}.

\medskip\noindent
Since an \'etale morphism is locally quasi-finite
(Morphisms of Spaces, Lemma
\ref{spaces-morphisms-lemma-etale-locally-quasi-finite})
we see that (3) implies (4).

\medskip\noindent
Finally, assume (4). After replacing $T$ by an open neighbourhood of $t$
we may assume $T \to Y$ is locally quasi-finite.
By Lemma \ref{lemma-flat-ascent-impurity}
we find an impurity
$(T_1 \to Y^{sh}, t_1' \leadsto t_1, \xi_1)$
with $T_1 \to T \times_Y Y^{sh}$
\'etale. Since an \'etale morphism is locally quasi-finite
and using Morphisms of Spaces, Lemma
\ref{spaces-morphisms-lemma-base-change-quasi-finite} and
Morphisms, Lemma \ref{morphisms-lemma-composition-quasi-finite}
we see that $T_1 \to Y^{sh}$ is locally quasi-finite.
As $\mathcal{O}$ is strictly henselian, we can apply More on Morphisms, Lemma
\ref{more-morphisms-lemma-etale-makes-quasi-finite-finite-at-point}
to see that after replacing $T_1$ by an open and closed neighbourhood
of $t_1$ we may assume that $T_1 \to Y^{sh} = \Spec(\mathcal{O})$
is finite. Let $\theta \in |X^{sh}|$ be the image of
$\xi_1$ and let $y' \in \Spec(\mathcal{O})$ be the image
of $t_1'$. By Divisors on Spaces, Lemma
\ref{spaces-divisors-lemma-base-change-relative-assassin}
we see that $\theta \in \text{Ass}_{X^{sh}/Y^{sh}}(\mathcal{F}^{sh})$.
Since $\pi : X_{T_1} \to X^{sh}$
is finite, it induces a closed map $|X_{T_1}| \to |X^{sh}|$.
Hence the image of $\overline{\{\xi_1\}}$ is $\overline{\{\theta\}}$.
It follows that $(Y^{sh} \to Y, y' \leadsto \overline{y}, \theta)$
is an impurity of $\mathcal{F}$ above $y$ and the proof is complete.
\end{proof}








\section{Relatively pure modules}
\label{section-pure}

\noindent
This section is the analogue of
More on Flatness, Section \ref{flat-section-pure}.

\begin{definition}
\label{definition-pure}
In Situation \ref{situation-pre-pure}.
\begin{enumerate}
\item We say $\mathcal{F}$ is {\it pure above $y$} if {\bf none} of the
equivalent conditions of Lemma \ref{lemma-pure-along-X-y} hold.
\item We say $\mathcal{F}$ is {\it universally pure above $y$}
if there does not exist any impurity of $\mathcal{F}$ above $y$.
\item We say that $X$ is {\it pure above $y$} if $\mathcal{O}_X$
is pure above $y$.
\item We say $\mathcal{F}$ is {\it universally $Y$-pure}, or
{\it universally pure relative to $Y$} if $\mathcal{F}$ is universally
pure above $y$ for every $y \in |Y|$.
\item We say $\mathcal{F}$ is {\it $Y$-pure}, or
{\it pure relative to $Y$} if $\mathcal{F}$ is pure above $y$
for every $y \in |Y|$.
\item We say that $X$ is {\it $Y$-pure} or {\it pure relative to $Y$}
if $\mathcal{O}_X$ is pure relative to $Y$.
\end{enumerate}
\end{definition}

\noindent
The obligatory lemmas follow.

\begin{lemma}
\label{lemma-base-change-universally}
In Situation \ref{situation-pre-pure}.
\begin{enumerate}
\item $\mathcal{F}$ is universally pure above $y$, and
\item for every morphism $(Y', y') \to (Y, y)$ of pointed algebraic spaces
the pullback $\mathcal{F}_{Y'}$ is pure above $y'$.
\end{enumerate}
In particular, $\mathcal{F}$ is universally pure relative to $Y$ if and
only if every base change $\mathcal{F}_{Y'}$ of $\mathcal{F}$ is
pure relative to $Y'$.
\end{lemma}

\begin{proof}
This is formal.
\end{proof}

\begin{lemma}
\label{lemma-quasi-finite-base-change}
In Situation \ref{situation-pre-pure}.
Let $(Y', y') \to (Y, y)$ be a morphism of pointed algebraic spaces.
If $Y' \to Y$ is quasi-finite at $y'$ and $\mathcal{F}$ is pure above $y$,
then $\mathcal{F}_{Y'}$ is pure above $y'$.
\end{lemma}

\begin{proof}
It $(T \to Y', t' \leadsto t, \xi)$ is an impurity of
$\mathcal{F}_{Y'}$ above $y'$ with $T \to Y'$ quasi-finite at $t$,
then $(T \to Y, t' \to t, \xi)$ is an impurity of $\mathcal{F}$
above $y$ with $T \to Y$ quasi-finite at $t$, see
Morphisms of Spaces, Lemma
\ref{spaces-morphisms-lemma-composition-quasi-finite}.
Hence the lemma follows immediately from the definition of purity.
\end{proof}

\noindent
Purity satisfies flat descent.

\begin{lemma}
\label{lemma-flat-descend-pure}
In Situation \ref{situation-pre-pure}.
Let $(Y_1, y_1) \to (Y, y)$ be a morphism of pointed algebraic spaces.
Assume $Y_1 \to Y$ is flat at $y_1$.
\begin{enumerate}
\item If $\mathcal{F}_{Y_1}$ is pure above $y_1$,
then $\mathcal{F}$ is pure above $y$.
\item If $\mathcal{F}_{Y_1}$ is universally pure above $y_1$,
then $\mathcal{F}$ is universally pure above $y$.
\end{enumerate}
\end{lemma}

\begin{proof}
This is true because impurities go up along a flat base change, see
Lemma \ref{lemma-flat-ascent-impurity}. For example
part (1) follows because by any impurity $(T \to Y, t' \leadsto t, \xi)$
of $\mathcal{F}$ above $y$ with $T \to Y$ quasi-finite at $t$
by the lemma leads to an impurity
$(T_1 \to Y_1, t_1' \leadsto t_1, \xi_1)$ of the pullback
$\mathcal{F}_1$ of $\mathcal{F}$ to $X_1 = Y_1 \times_Y X$
over $y_1$ such that $T_1$ is \'etale over $Y_1 \times_Y T$.
Hence $T_1 \to Y_1$ is quasi-finite at $t_1$ because
\'etale morphisms are locally quasi-finite and compositions
of locally quasi-finite morphisms are locally quasi-finite
(Morphisms of Spaces, Lemmas
\ref{spaces-morphisms-lemma-etale-locally-quasi-finite} and
\ref{spaces-morphisms-lemma-composition-quasi-finite}).
Similarly for part (2).
\end{proof}

\begin{lemma}
\label{lemma-supported-on-closed}
In Situation \ref{situation-pre-pure}. Let $i : Z \to X$ be a closed immersion
and assume that $\mathcal{F} = i_*\mathcal{G}$ for some
finite type, quasi-coherent sheaf $\mathcal{G}$ on $Z$.
Then $\mathcal{G}$ is (universally) pure above $y$
if and only if $\mathcal{F}$ is (universally) pure above $y$.
\end{lemma}

\begin{proof}
This follows from Divisors on Spaces, Lemma
\ref{spaces-divisors-lemma-relative-weak-assassin-finite}.
\end{proof}

\begin{lemma}
\label{lemma-proper-pure}
In Situation \ref{situation-pre-pure}.
\begin{enumerate}
\item If the support of $\mathcal{F}$ is proper over $Y$, then
$\mathcal{F}$ is universally pure relative to $Y$.
\item If $f$ is proper, then
$\mathcal{F}$ is universally pure relative to $Y$.
\item If $f$ is proper, then $X$ is universally pure relative to $Y$.
\end{enumerate}
\end{lemma}

\begin{proof}
First we reduce (1) to (2). Namely, let $Z \subset X$ be the
scheme theoretic support of $\mathcal{F}$
(Morphisms of Spaces, Definition
\ref{spaces-morphisms-definition-scheme-theoretic-support}). Let $i : Z \to X$
be the corresponding closed immersion and write
$\mathcal{F} = i_*\mathcal{G}$ for some finite type quasi-coherent
$\mathcal{O}_Z$-module $\mathcal{G}$.
In case (1) $Z \to Y$ is proper by assumption.
Thus by Lemma \ref{lemma-supported-on-closed} case (1) reduces to case (2).

\medskip\noindent
Assume $f$ is proper.
Let $(g : T \to Y, t' \leadsto t, \xi)$ be an impurity of $\mathcal{F}$
above $y$. Since $f$ is proper, it is universally closed. Hence
$f_T : X_T \to T$ is closed. Since $f_T(\xi) = t'$ this implies that
$t \in f(\overline{\{\xi\}})$ which is a contradiction.
\end{proof}







\section{Flat finite type modules}
\label{section-finite-type-flat}

\noindent
Please compare with
More on Flatness, Sections
\ref{flat-section-finite-type-flat-I},
\ref{flat-section-finite-type-flat-II}, and
\ref{flat-section-finite-type-flat-III}.
Most of these results have immediate consequences
of algebraic spaces by \'etale localization.

\begin{lemma}
\label{lemma-existence-complete}
Let $S$ be a scheme.
Let $X \to Y$ be a finite type morphism of algebraic spaces over $S$.
Let $\mathcal{F}$ be a finite type quasi-coherent $\mathcal{O}_X$-module.
Let $y \in |Y|$ be a point. There exists an \'etale morphism
$(Y', y') \to (Y, y)$ with $Y'$ an affine scheme and \'etale morphisms
$h_i : W_i \to X_{Y'}$, $i = 1, \ldots, n$ such that for each
$i$ there exists a complete d\'evissage of $\mathcal{F}_i/W_i/Y'$ over $y'$,
where $\mathcal{F}_i$ is the pullback of $\mathcal{F}$ to $W_i$
and such that $|(X_{Y'})_{y'}| \subset \bigcup h_i(W_i)$.
\end{lemma}

\begin{proof}
The question is \'etale local on $Y$ hence we may assume $Y$
is an affine scheme. Then $X$ is quasi-compact, hence we can
choose an affine scheme $X'$ and a surjective \'etale morphism
$X' \to X$. Then we may apply
More on Flatness, Lemma \ref{flat-lemma-existence-complete}
to $X' \to Y$, $(X' \to Y)^*\mathcal{F}$, and $y$ to
get what we want.
\end{proof}

\begin{lemma}
\label{lemma-open-in-fibre-where-flat}
Let $S$ be a scheme.
Let $f : X \to Y$ be a morphism of algebraic spaces over $S$
which is locally of finite type.
Let $\mathcal{F}$ be a quasi-coherent $\mathcal{O}_X$-module of finite type.
Let $y \in |Y|$ and $F = f^{-1}(\{y\}) \subset |X|$. Then the set
$$
\{x \in F \mid \mathcal{F} \text{ flat over }Y\text{ at }x\}
$$
is open in $F$.
\end{lemma}

\begin{proof}
Choose a scheme $V$, a point $v \in V$, and an \'etale morphism $V \to Y$
mapping $v$ to $y$. Choose a scheme $U$ and a surjective \'etale
morphism $U \to V \times_Y X$. Then $|U_v| \to F$ is an open continuous
map of topological spaces as $|U| \to |X|$ is continuous and open.
Hence the result follows from the case of schemes which is
More on Flatness, Lemma \ref{flat-lemma-open-in-fibre-where-flat}.
\end{proof}

\begin{lemma}
\label{lemma-bourbaki-finite-type-general-base-at-point}
Let $S$ be a scheme.
Let $f : X \to Y$ be a morphism of algebraic spaces over $S$ which is
locally of finite type. Let $x \in |X|$ with image $y \in |Y|$.
Let $\mathcal{F}$ be a finite type quasi-coherent sheaf on $X$.
Let $\mathcal{G}$ be a quasi-coherent sheaf on $Y$.
If $\mathcal{F}$ is flat at $x$ over $Y$, then
$$
x \in \text{WeakAss}_X(\mathcal{F} \otimes_{\mathcal{O}_X} f^*\mathcal{G})
\Leftrightarrow
y \in \text{WeakAss}_Y(\mathcal{G})
\text{ and }
x \in \text{Ass}_{X/Y}(\mathcal{F}).
$$
\end{lemma}

\begin{proof}
Choose a commutative diagram
$$
\xymatrix{
U \ar[d] \ar[r]_g & V \ar[d] \\
X \ar[r]^f & Y
}
$$
where $U$ and $V$ are schemes and the vertical arrows are surjective \'etale.
Choose $u \in U$ mapping to $x$. Let $\mathcal{E} = \mathcal{F}|_U$
and $\mathcal{H} = \mathcal{G}|_V$.
Let $v \in V$ be the image of $u$. Then
$x \in \text{WeakAss}_X(\mathcal{F} \otimes_{\mathcal{O}_X} f^*\mathcal{G})$
if and only if
$u \in \text{WeakAss}_X(\mathcal{E} \otimes_{\mathcal{O}_X} g^*\mathcal{H})$
by Divisors on Spaces, Definition
\ref{spaces-divisors-definition-weakly-associated}.
Similarly, $y \in \text{WeakAss}_Y(\mathcal{G})$ if and only if
$v \in \text{WeakAss}_V(\mathcal{H})$.
Finally, we have $x \in \text{Ass}_{X/Y}(\mathcal{F})$ if and only if
$u \in \text{Ass}_{U_v}(\mathcal{E}|_{U_v})$ by
Divisors on Spaces, Definition
\ref{spaces-divisors-definition-relative-weak-assassin}.
Observe that flatness of $\mathcal{F}$ at $x$ is
equivalent to flatness of $\mathcal{E}$ at $u$, see
Morphisms of Spaces, Definition \ref{spaces-morphisms-definition-flat-module}.
The equivalence for $g : U \to V$, $\mathcal{E}$, $\mathcal{H}$, $u$, and $v$
is More on Flatness, Lemma
\ref{flat-lemma-bourbaki-finite-type-general-base-at-point}.
\end{proof}

\begin{lemma}
\label{lemma-bourbaki-finite-type-general-base}
Let $S$ be a scheme. Let $f : X \to Y$ be a morphism of algebraic spaces
over $S$ which is locally of finite type.
Let $\mathcal{F}$ be a finite type quasi-coherent sheaf on $X$
which is flat over $Y$. Let $\mathcal{G}$ be a quasi-coherent sheaf on $Y$.
Then we have
$$
\text{WeakAss}_X(\mathcal{F} \otimes_{\mathcal{O}_X} f^*\mathcal{G}) =
\text{Ass}_{X/Y}(\mathcal{F}) \cap
|f|^{-1}(\text{WeakAss}_Y(\mathcal{G}))
$$
\end{lemma}

\begin{proof}
Immediate consequence of
Lemma \ref{lemma-bourbaki-finite-type-general-base-at-point}.
\end{proof}

\begin{theorem}
\label{theorem-finite-type-flat}
Let $S$ be a scheme.
Let $f : X \to Y$ be a morphism of algebraic spaces over $S$.
Let $\mathcal{F}$ be a quasi-coherent $\mathcal{O}_X$-module.
Assume
\begin{enumerate}
\item $X \to Y$ is locally of finite presentation,
\item $\mathcal{F}$ is an $\mathcal{O}_X$-module of finite type, and
\item the set of weakly associated points of $Y$ is locally finite in $Y$.
\end{enumerate}
Then $U = \{x \in |X| : \mathcal{F}\text{ flat at }x\text{ over }Y\}$
is open in $X$ and $\mathcal{F}|_U$ is an $\mathcal{O}_U$-module
of finite presentation and flat over $Y$.
\end{theorem}

\begin{proof}
Condition (3) means that if $V \to Y$ is a surjective \'etale morphism
where $V$ is a scheme, then the weakly associated points of $V$
are locally finite on the scheme $V$. (Recall that the weakly associated
points of $V$ are exactly the inverse image of the weakly associated
points of $Y$ by Divisors on Spaces, Definition
\ref{spaces-divisors-definition-weakly-associated}.)
Having said this the question
is \'etale local on $X$ and $Y$, hence we may assume $X$ and $Y$
are schemes. Thus the result follows from
More on Flatness, Theorem \ref{flat-theorem-finite-type-flat}.
\end{proof}

\begin{lemma}
\label{lemma-finite-type-flat-along-fibre-free-variant}
Let $S$ be a scheme. Let $f : X \to Y$ be a morphism of algebraic spaces
over $S$. Let $\mathcal{F}$ be a quasi-coherent sheaf on $X$.
Let $y \in |Y|$. Set $F = f^{-1}(\{y\}) \subset |X|$. Assume that
\begin{enumerate}
\item $f$ is of finite type,
\item $\mathcal{F}$ is of finite type, and
\item $\mathcal{F}$ is flat over $Y$ at all $x \in F$.
\end{enumerate}
Then there exists an \'etale morphism $(Y', y') \to (Y, y)$
where $Y'$ is a scheme and a commutative diagram of algebraic spaces
$$
\xymatrix{
X \ar[d] & X' \ar[l]^g \ar[d] \\
Y & \Spec(\mathcal{O}_{Y', y'}) \ar[l]
}
$$
such that $X' \to X \times_Y \Spec(\mathcal{O}_{Y', y'})$
is \'etale, $|X'_{y'}| \to F$ is surjective, $X'$ is affine,
and $\Gamma(X', g^*\mathcal{F})$ is a free $\mathcal{O}_{Y', y'}$-module.
\end{lemma}

\begin{proof}
Choose an \'etale morphism $(Y', y') \to (Y, y)$ where $Y'$ is an
affine scheme. Then $X \times_Y Y'$ is quasi-compact.
Choose an affine scheme $X'$ and a surjective \'etale morphism
$X' \to X \times_Y Y'$. Picture
$$
\xymatrix{
X \ar[d] & X' \ar[l]^g \ar[d] \\
Y & Y' \ar[l]
}
$$
Then $\mathcal{F}' = g^*\mathcal{F}$ is flat over $Y'$ at all
points of $X'_{y'}$, see Morphisms of Spaces, Lemma
\ref{spaces-morphisms-lemma-base-change-module-flat}.
Hence we can apply the lemma in the case of schemes
(More on Flatness, Lemma
\ref{flat-lemma-finite-type-flat-along-fibre-free-variant})
to the morphism
$X' \to Y'$, the quasi-coherent sheaf $g^*\mathcal{F}$, and the point $y'$.
This gives an \'etale morphism $(Y'', y'') \to (Y', y')$ and a commutative
diagram
$$
\xymatrix{
X \ar[d] & X' \ar[l]^g \ar[d] & X'' \ar[l]^{g'} \ar[d] \\
Y & Y' \ar[l] & \Spec(\mathcal{O}_{Y'', y''}) \ar[l]
}
$$
To get what we want we take $(Y'', y'') \to (Y, y)$
and $g \circ g' : X'' \to X$.
\end{proof}

\begin{theorem}
\label{theorem-check-flatness-at-associated-points}
Let $S$ be a scheme.
Let $f : X \to Y$ be a morphism of algebraic spaces over $S$
which is locally of finite type.
Let $\mathcal{F}$ be a quasi-coherent $\mathcal{O}_X$-module of finite type.
Let $x \in |X|$ with image $y \in |Y|$.
Set $F = f^{-1}(\{y\}) \subset |X|$.
Consider the conditions
\begin{enumerate}
\item $\mathcal{F}$ is flat at $x$ over $Y$, and
\item for every $x' \in F \cap \text{Ass}_{X/Y}(\mathcal{F})$ which
specializes to $x$ we have that $\mathcal{F}$ is flat at $x'$ over $Y$.
\end{enumerate}
Then we always have (2) $\Rightarrow$ (1). If $X$ and $Y$ are
decent, then (1) $\Rightarrow$ (2).
\end{theorem}

\begin{proof}
Assume (2).
Choose a scheme $V$ and a surjective \'etale morphism $V \to Y$.
Choose a scheme $U$ and a surjective \'etale morphism $U \to V \times_Y X$.
Choose a point $u \in U$ mapping to $x$. Let $v \in V$ be the image of $u$.
We will deduce the result from the corresponding result for
$\mathcal{F}|_U = (U \to X)^*\mathcal{F}$ and the point $u$.
$U_v$. This works because $\text{Ass}_{U/V}(\mathcal{F}|_U) \cap |U_v|$
is equal to $\text{Ass}_{U_v}(\mathcal{F}|_{U_v})$ and equal to the inverse
image of $F \cap \text{Ass}_{X/Y}(\mathcal{F})$.
Since the map $|U_v| \to F$ is continuous we see that
specializations in $|U_v|$ map to specializations in $F$,
hence condition (2) is inherited by $U \to V$,
$\mathcal{F}|_U$, and the point $u$.
Thus More on Flatness, Theorem
\ref{flat-theorem-check-flatness-at-associated-points} applies
and we conclude that (1) holds.

\medskip\noindent
If $Y$ is decent, then we can represent
$y$ by a quasi-compact monomorphism $\Spec(k) \to Y$
(by definition of decent spaces, see
Decent Spaces, Definition \ref{decent-spaces-definition-very-reasonable}).
Then $F = |X_k|$, see
Decent Spaces, Lemma \ref{decent-spaces-lemma-topology-fibre}.
If in addition $X$ is decent (or more generally if $f$ is decent, see
Decent Spaces, Definition \ref{decent-spaces-definition-relative-conditions}
and Decent Spaces, Lemma
\ref{decent-spaces-lemma-property-for-morphism-out-of-property}),
then $X_y$ is a decent space too. Furthermore, specializations in
$F$ can be lifted to specializations
in $U_v \to X_y$, see
Decent Spaces, Lemma \ref{decent-spaces-lemma-decent-specialization}.
Having said this it is clear that the reverse implication
holds, because it holds in the case of schemes.
\end{proof}

\begin{lemma}
\label{lemma-check-along-closed-fibre}
Let $S$ be a local scheme with closed point $s$.
Let $f : X \to S$ be a morphism from an algebraic space $X$ to $S$
which is locally of finite type.
Let $\mathcal{F}$ be a finite type quasi-coherent $\mathcal{O}_X$-module.
Assume that
\begin{enumerate}
\item every point of $\text{Ass}_{X/S}(\mathcal{F})$ specializes
to a point of the closed fibre $X_s$\footnote{For example this holds if
$f$ is finite type and $\mathcal{F}$ is pure along $X_s$, or
if $f$ is proper.},
\item $\mathcal{F}$ is flat over $S$ at every point of $X_s$.
\end{enumerate}
Then $\mathcal{F}$ is flat over $S$.
\end{lemma}

\begin{proof}
This is immediate from the fact that it suffices to check for
flatness at points of the relative assassin of $\mathcal{F}$
over $S$ by
Theorem \ref{theorem-check-flatness-at-associated-points}.
\end{proof}




\section{Flat finitely presented modules}
\label{section-finitely-presented-flat}

\noindent
This is the analogue of More on Flatness, Section
\ref{flat-section-finitely-presented-flat}.

\begin{proposition}
\label{proposition-finite-presentation-flat-at-point}
Let $S$ be a scheme.
Let $f : X \to Y$ be a morphism of algebraic spaces over $S$.
Let $\mathcal{F}$ be a quasi-coherent sheaf on $X$.
Let $x \in |X|$ with image $y \in |Y|$.
Assume that
\begin{enumerate}
\item $f$ is locally of finite presentation,
\item $\mathcal{F}$ is of finite presentation, and
\item $\mathcal{F}$ is flat at $x$ over $Y$.
\end{enumerate}
Then there exists a commutative diagram of pointed schemes
$$
\xymatrix{
(X, x) \ar[d] & (X', x') \ar[l]^g \ar[d] \\
(Y, y) & (Y', y') \ar[l]
}
$$
whose horizontal arrows are \'etale such that $X'$, $Y'$
are affine and such that
$\Gamma(X', g^*\mathcal{F})$ is a projective
$\Gamma(Y', \mathcal{O}_{Y'})$-module.
\end{proposition}

\begin{proof}
As formulated this proposition immediately reduces
to the case of schemes, which is
More on Flatness, Proposition
\ref{flat-proposition-finite-presentation-flat-at-point}.
\end{proof}

\begin{lemma}
\label{lemma-finite-presentation-flat-along-fibre}
Let $S$ be a scheme.
Let $f : X \to Y$ be a morphism of algebraic spaces over $S$.
Let $\mathcal{F}$ be a quasi-coherent sheaf on $X$.
Let $y \in |Y|$. Set $F = f^{-1}(\{y\}) \subset |X|$. Assume that
\begin{enumerate}
\item $f$ is of finite presentation,
\item $\mathcal{F}$ is of finite presentation, and
\item $\mathcal{F}$ is flat over $Y$ at all $x \in F$.
\end{enumerate}
Then there exists a commutative diagram of algebraic spaces
$$
\xymatrix{
X \ar[d] & X' \ar[l]^g \ar[d] \\
Y & Y' \ar[l]_h
}
$$
such that $h$ and $g$ are \'etale, there is a point
$y' \in |Y'|$ mapping to $y$, we have $F \subset g(|X'|)$,
the algebraic spaces $X'$, $Y'$ are affine, and
$\Gamma(X', g^*\mathcal{F})$ is a projective
$\Gamma(Y', \mathcal{O}_{Y'})$-module.
\end{lemma}

\begin{proof}
As formulated this lemma immediately reduces
to the case of schemes, which is
More on Flatness, Lemma
\ref{flat-lemma-finite-presentation-flat-along-fibre}.
\end{proof}




\section{A criterion for purity}
\label{section-criterion-purity}

\noindent
This section is the analogue of More on Flatness, Section
\ref{flat-section-criterion-purity}.

\begin{lemma}
\label{lemma-associated-point-specializes}
Let $S$ be a scheme. Let $X$ be a decent algebraic space
locally of finite type over $S$.
Let $\mathcal{F}$ be a finite type, quasi-coherent $\mathcal{O}_X$-module.
Let $s \in S$ such that $\mathcal{F}$ is flat over $S$ at all points of $X_s$.
Let $x' \in \text{Ass}_{X/S}(\mathcal{F})$. If
the closure of $\{x'\}$ in $|X|$ meets $|X_s|$, then the closure
meets $\text{Ass}_{X/S}(\mathcal{F}) \cap |X_s|$.
\end{lemma}

\begin{proof}
Observe that $|X_s| \subset |X|$ is the set of points of $|X|$
lying over $s \in S$, see
Decent Spaces, Lemma \ref{decent-spaces-lemma-topology-fibre}.
Let $t \in |X_s|$ be a specialization of $x'$ in $|X|$.
Choose an affine scheme $U$ and a point $u \in U$ and
an \'etale morphism $\varphi : U \to X$ mapping $u$ to $t$.
By Decent Spaces, Lemma \ref{decent-spaces-lemma-decent-specialization}
we can choose a specialization $u' \leadsto u$
with $u'$ mapping to $x'$. Set $g = f \circ \varphi$.
Observe that $s' = g(u') = f(x')$ specializes to $s$.
By our definition of $\text{Ass}_{X/S}(\mathcal{F})$
we have $u' \in \text{Ass}_{U/S}(\varphi^*\mathcal{F})$.
By the schemes version of this lemma
(More on Flatness, Lemma \ref{flat-lemma-associated-point-specializes})
we see that there is a specialization $u' \leadsto u$ with
$u \in \text{Ass}_{U_s}(\varphi^*\mathcal{F}_s) =
\text{Ass}_{U/S}(\varphi^*\mathcal{F}) \cap U_s$.
Hence $x = \varphi(u) \in \text{Ass}_{X/S}(\mathcal{F})$
lies over $s$ and the lemma is proved.
\end{proof}

\begin{lemma}
\label{lemma-contains-relative-ass-after-base-change}
Let $Y$ be an algebraic space over a scheme $S$. Let $g : X' \to X$ be a
morphism of algebraic spaces over $Y$ with $X$ locally of finite type over $Y$.
Let $\mathcal{F}$ be a quasi-coherent $\mathcal{O}_X$-module.
If $\text{Ass}_{X/Y}(\mathcal{F}) \subset g(|X'|)$, then for any morphism
$Z \to Y$ we have $\text{Ass}_{X_Z/Z}(\mathcal{F}_Z) \subset g_Z(|X'_Z|)$.
\end{lemma}

\begin{proof}
By Properties of Spaces, Lemma \ref{spaces-properties-lemma-points-cartesian}
the map $|X'_Z| \to |X_Z| \times_{|X|} |X'|$ is surjective as
$X'_Z$ is equal to $X_Z \times_X X'$.
By Divisors on Spaces, Lemma
\ref{spaces-divisors-lemma-base-change-relative-assassin}
the map $|X_Z| \to |X|$ sends $\text{Ass}_{X_Z/Z}(\mathcal{F}_Z)$
into $\text{Ass}_{X/Y}(\mathcal{F})$. The lemma follows.
\end{proof}

\begin{lemma}
\label{lemma-pure-on-top}
Let $Y$ be an algebraic space over a scheme $S$. Let $g : X' \to X$ be an
\'etale morphism of algebraic spaces over $Y$. Assume the structure
morphisms $X' \to Y$ and $X \to Y$ are decent and of finite type.
Let $\mathcal{F}$ be a finite type, quasi-coherent $\mathcal{O}_X$-module.
Let $y \in |Y|$. Set $F = f^{-1}(\{y\}) \subset |X|$.
\begin{enumerate}
\item If $\text{Ass}_{X/Y}(\mathcal{F}) \subset g(|X'|)$
and $g^*\mathcal{F}$ is (universally) pure above $y$, then
$\mathcal{F}$ is (universally) pure above $y$.
\item If $\mathcal{F}$ is pure above $y$, $g(|X'|)$ contains $F$, and
$Y$ is affine local with closed point $y$, then
$\text{Ass}_{X/Y}(\mathcal{F}) \subset g(|X'|)$.
\item If $\mathcal{F}$ is pure above $y$, $\mathcal{F}$ is flat
at all points of $F$, $g(|X'|)$ contains
$\text{Ass}_{X/Y}(\mathcal{F}) \cap F$, and $Y$ is affine local
with closed point $y$, then
$\text{Ass}_{X/Y}(\mathcal{F}) \subset g(|X'|)$.
\item Add more here.
\end{enumerate}
\end{lemma}

\begin{proof}
The assumptions on $X \to Y$ and $X' \to Y$ guarantee that
we may apply the material in Sections \ref{section-impure} and
\ref{section-pure}
to these morphisms and the sheaves $\mathcal{F}$ and $g^*\mathcal{F}$.
Since $g$ is \'etale we see that
$\text{Ass}_{X'/Y}(g^*\mathcal{F})$
is the inverse image of $\text{Ass}_{X/Y}(\mathcal{F})$
and the same remains true after base change.

\medskip\noindent
Proof of (1). Assume $\text{Ass}_{X/Y}(\mathcal{F}) \subset g(|X'|)$.
Suppose that $(T \to Y, t' \leadsto t, \xi)$
is an impurity of $\mathcal{F}$ above $y$. Since
$\text{Ass}_{X_T/T}(\mathcal{F}_T) \subset g_T(|X'_T|)$ by
Lemma \ref{lemma-contains-relative-ass-after-base-change}
we can choose
a point $\xi' \in |X'_T|$ mapping to $\xi$. By the above we see
that $(T \to Y, t' \leadsto t, \xi')$ is an impurity of
$g^*\mathcal{F}$ above $y'$. This implies (1) is true.

\medskip\noindent
Proof of (2). This follows from the fact that $g(|X'|)$ is open
in $|X|$ and the fact that by purity every point of
$\text{Ass}_{X/Y}(\mathcal{F})$ specializes to a point of $F$.

\medskip\noindent
Proof of (3). This follows from the fact that $g(|X'|)$ is open
in $|X|$ and the fact that by purity combined with
Lemma \ref{lemma-associated-point-specializes} every point of
$\text{Ass}_{X/Y}(\mathcal{F})$ specializes to a point of
$\text{Ass}_{X/Y}(\mathcal{F}) \cap F$.
\end{proof}

\begin{lemma}
\label{lemma-finite-type-flat-pure-along-fibre-is-universal}
Let $S$ be a scheme. Let $f : X \to Y$ be a morphism of algebraic
spaces over $S$.
Let $\mathcal{F}$ be a quasi-coherent $\mathcal{O}_X$-module.
Let $y \in |Y|$.
Assume
\begin{enumerate}
\item $f$ is decent and of finite type,
\item $\mathcal{F}$ is of finite type,
\item $\mathcal{F}$ is flat over $Y$ at all points lying over $y$, and
\item $\mathcal{F}$ is pure above $y$.
\end{enumerate}
Then $\mathcal{F}$ is universally pure above $y$.
\end{lemma}

\begin{proof}
Consider the morphism $\Spec(\mathcal{O}_{Y, \overline{y}}) \to Y$.
This is a flat morphism from the spectrum of a strictly henselian
local ring which maps the closed point to $y$.
By Lemma \ref{lemma-flat-descend-pure} we reduce to the case
described in the next paragraph.

\medskip\noindent
Assume $Y$ is the spectrum of a strictly henselian local ring $R$
with closed point $y$.
By Lemma \ref{lemma-finite-type-flat-along-fibre-free-variant}
there exists an \'etale morphism $g : X' \to X$ with
$g(|X'|) \supset |X_y|$, with $X'$ affine, and with
$\Gamma(X', g^*\mathcal{F})$ a free $R$-module.
Then $g^*\mathcal{F}$ is universally pure relative to $Y$, see
More on Flatness, Lemma \ref{flat-lemma-affine-locally-projective-pure}.
Hence it suffices to prove that
$g(|X'|)$ contains $\text{Ass}_{X/Y}(\mathcal{F})$
by Lemma \ref{lemma-pure-on-top} part (1).
This in turn follows from
Lemma \ref{lemma-pure-on-top} part (2).
\end{proof}

\begin{lemma}
\label{lemma-finite-type-flat-pure-is-universal}
Let $S$ be a scheme.
Let $f : X \to Y$ be a decent, finite type morphism of algebraic
spaces over $S$.
Let $\mathcal{F}$ be a finite type quasi-coherent $\mathcal{O}_X$-module.
Assume $\mathcal{F}$ is flat over $Y$. In this case
$\mathcal{F}$ is pure relative to $Y$ if and only if $\mathcal{F}$
is universally pure relative to $Y$.
\end{lemma}

\begin{proof}
Immediate consequence of
Lemma \ref{lemma-finite-type-flat-pure-along-fibre-is-universal}
and the definitions.
\end{proof}

\begin{lemma}
\label{lemma-universally-separating}
Let $Y$ be an algebraic space over a scheme $S$.
Let $g : X' \to X$ be a flat morphism of algebraic spaces over $Y$
with $X$ locally of finite type over $Y$.
Let $\mathcal{F}$ be a finite type quasi-coherent $\mathcal{O}_X$-module
which is flat over $Y$. If $\text{Ass}_{X/Y}(\mathcal{F}) \subset g(|X'|)$
then the canonical map
$$
\mathcal{F} \longrightarrow g_*g^*\mathcal{F}
$$
is injective, and remains injective after any base change.
\end{lemma}

\begin{proof}
The final assertion means that $\mathcal{F}_Z \to (g_Z)_*g_Z^*\mathcal{F}_Z$
is injective for any morphism $Z \to Y$. Since the assumption on
the relative assassin is preserved by base change
(Lemma \ref{lemma-contains-relative-ass-after-base-change})
it suffices to prove the injectivity of the displayed arrow.

\medskip\noindent
Let $\mathcal{K} = \Ker(\mathcal{F} \to g_*g^*\mathcal{F})$.
Our goal is to prove that $\mathcal{K} = 0$.
In order to do this it suffices to prove that
$\text{WeakAss}_X(\mathcal{K}) = \emptyset$, see
Divisors on Spaces, Lemma \ref{spaces-divisors-lemma-weakly-ass-zero}.
We have
$\text{WeakAss}_X(\mathcal{K}) \subset \text{WeakAss}_X(\mathcal{F})$, see
Divisors on Spaces, Lemma \ref{spaces-divisors-lemma-ses-weakly-ass}.
As $\mathcal{F}$ is flat we see from
Lemma \ref{lemma-bourbaki-finite-type-general-base}
that $\text{WeakAss}_X(\mathcal{F}) \subset \text{Ass}_{X/Y}(\mathcal{F})$.
By assumption any point $x$ of $\text{Ass}_{X/Y}(\mathcal{F})$
is the image of some $x' \in |X'|$. Since $g$ is flat the
local ring map
$\mathcal{O}_{X, \overline{x}} \to \mathcal{O}_{X', \overline{x}'}$
is faithfully flat, hence the map
$$
\mathcal{F}_{\overline{x}}
\longrightarrow
(g^*\mathcal{F})_{\overline{x}'} =
\mathcal{F}_{\overline{x}} \otimes_{\mathcal{O}_{X, \overline{x}}}
\mathcal{O}_{X', \overline{x}'}
$$
is injective (see
Algebra, Lemma \ref{algebra-lemma-faithfully-flat-universally-injective}).
Since the displayed arrow factors through
$\mathcal{F}_{\overline{x}} \to (g_*g^*\mathcal{F})_{\overline{x}}$,
we conclude that
$\mathcal{K}_{\overline{x}} = 0$. Hence $x$ cannot be a weakly associated
point of $\mathcal{K}$ and we win.
\end{proof}









\section{Flattening functors}
\label{section-F-zero}

\noindent
This section is the analogue of
More on Flatness, Section \ref{flat-section-flattening-functors}.
We urge the reader to skip this section on a first reading.

\begin{situation}
\label{situation-iso}
Let $S$ be a scheme.
Let $f : X \to B$ be a morphism of algebraic spaces over $S$.
Let $u : \mathcal{F} \to \mathcal{G}$ be a homomorphism of
quasi-coherent $\mathcal{O}_X$-modules. For any scheme $T$ over
$B$ we will denote $u_T : \mathcal{F}_T \to \mathcal{G}_T$ the
base change of $u$ to $T$, in other words, $u_T$ is the pullback
of $u$ via the projection morphism $X_T = X \times_B T \to X$.
In this situation we can consider the functor
\begin{equation}
\label{equation-iso}
F_{iso} : (\Sch/B)^{opp} \longrightarrow \textit{Sets}, \quad
T \longrightarrow \left\{
\begin{matrix}
\{*\} & \text{if }u_T\text{ is an isomorphism}, \\
\emptyset & \text{else.}
\end{matrix}
\right.
\end{equation}
There are variants $F_{inj}$, $F_{surj}$, $F_{zero}$ where we ask that
$u_T$ is injective, surjective, or zero.
\end{situation}

\noindent
In Situation \ref{situation-iso} we sometimes think of the functors
$F_{iso}$, $F_{inj}$, $F_{surj}$, and $F_{zero}$ as functors
$(\Sch/S)^{opp} \to \textit{Sets}$ endowed with a morphism
$F_{iso} \to B$, $F_{inj} \to B$, $F_{surj} \to B$, and $F_{zero} \to B$.
Namely, if $T$ is a scheme over $S$, then an element $h \in F_{iso}(T)$
is a morphism $h : T \to B$
such that the base change of $u$ via $h$ is an isomorphism.
In particular, when we say
that $F_{iso}$ is an algebraic space, we mean that the corresponding
functor $(\Sch/S)^{opp} \to \textit{Sets}$ is an algebraic space.

\begin{lemma}
\label{lemma-iso-sheaf}
In Situation \ref{situation-iso}.
Each of the functors $F_{iso}$, $F_{inj}$, $F_{surj}$, $F_{zero}$
satisfies the sheaf property for the fpqc topology.
\end{lemma}

\begin{proof}
Let $\{T_i \to T\}_{i \in I}$ be an fpqc covering of schemes over $B$.
Set $X_i = X_{T_i} = X \times_S T_i$ and $u_i = u_{T_i}$.
Note that $\{X_i \to X_T\}_{i \in I}$ is an fpqc covering of $X_T$, see
Topologies on Spaces, Lemma \ref{spaces-topologies-lemma-fpqc}.
In particular, for every $x \in |X_T|$ there exists an $i \in I$
and an $x_i \in |X_i|$ mapping to $x$. Since
$\mathcal{O}_{X_T, \overline{x}} \to \mathcal{O}_{X_i, \overline{x_i}}$
is flat, hence faithfully flat (see
Morphisms of Spaces, Section \ref{spaces-morphisms-section-flat}).
we conclude that $(u_i)_{x_i}$ is injective, surjective, bijective, or zero
if and only if $(u_T)_x$ is injective, surjective, bijective, or zero.
The lemma follows.
\end{proof}

\begin{lemma}
\label{lemma-iso-go-up}
In Situation \ref{situation-iso} let $X' \to X$ be a flat morphism
of algebraic spaces. Denote $u' : \mathcal{F}' \to \mathcal{G}'$
the pullback of $u$ to $X'$. Denote $F'_{iso}$, $F'_{inj}$, $F'_{surj}$,
$F'_{zero}$ the functors on $\Sch/B$ associated to $u'$.
\begin{enumerate}
\item If $\mathcal{G}$ is of finite type and the image of $|X'| \to |X|$
contains the support of $\mathcal{G}$, then $F_{surj} = F'_{surj}$
and $F_{zero} = F'_{zero}$.
\item If $\mathcal{F}$ is of finite type and the image of $|X'| \to |X|$
contains the support of $\mathcal{F}$, then $F_{inj} = F'_{inj}$
and $F_{zero} = F'_{zero}$.
\item If $\mathcal{F}$ and $\mathcal{G}$ are of finite type and the image of
$|X'| \to |X|$ contains the supports of $\mathcal{F}$ and $\mathcal{G}$,
then $F_{iso} = F'_{iso}$.
\end{enumerate}
\end{lemma}

\begin{proof}
let $v : \mathcal{H} \to \mathcal{E}$ be a map of quasi-coherent
modules on an algebraic space $Y$ and let $\varphi : Y' \to Y$ be a
surjective flat morphism of algebraic spaces, then $v$ is
an isomorphism, injective, surjective, or zero if and only if $\varphi^*v$ is
an isomorphism, injective, surjective, or zero. Namely,
for every $y \in |Y|$ there exists a $y' \in |Y'|$ and the map
of local rings
$\mathcal{O}_{Y, \overline{y}} \to \mathcal{O}_{Y', \overline{y'}}$
is faithfully flat (see
Morphisms of Spaces, Section \ref{spaces-morphisms-section-flat}).
Of course, to check for injectivity or being zero it suffices to look
at the points in the support of $\mathcal{H}$, and to check for
surjectivity it suffices to look at points in the support of $\mathcal{E}$.
Moreover, under the finite type assumptions as in the statement of
the lemma, taking the supports commutes with base change, see
Morphisms of Spaces, Lemma \ref{spaces-morphisms-lemma-support-finite-type}.
Thus the lemma is clear.
\end{proof}

\noindent
Recall that we've defined the scheme theoretic support of a finite
type quasi-coherent module in Morphisms of Spaces, Definition
\ref{spaces-morphisms-definition-scheme-theoretic-support}.

\begin{lemma}
\label{lemma-iso-limits}
In Situation \ref{situation-iso}.
\begin{enumerate}
\item If $\mathcal{G}$ is of finite type and the scheme theoretic support
of $\mathcal{G}$ is quasi-compact over $B$, then $F_{surj}$ is limit
preserving.
\item If $\mathcal{F}$ of finite type and the scheme theoretic support
of $\mathcal{F}$ is quasi-compact over $B$, then
$F_{zero}$ is limit preserving.
\item If $\mathcal{F}$ is of finite type,
$\mathcal{G}$ is of finite presentation, and the
scheme theoretic supports of $\mathcal{F}$ and $\mathcal{G}$ are
quasi-compact over $B$, then $F_{iso}$ is limit preserving.
\end{enumerate}
\end{lemma}

\begin{proof}
Proof of (1). Let $i : Z \to X$ be the scheme theoretic support of
$\mathcal{G}$ and think of $\mathcal{G}$ as a finite type quasi-coherent
module on $Z$. We may replace $X$ by $Z$ and $u$ by the map
$i^*\mathcal{F} \to \mathcal{G}$ (details omitted). Hence we may assume
$f$ is quasi-compact and $\mathcal{G}$ of finite type.
Let $T = \lim_{i \in I} T_i$ be a directed limit of affine $B$-schemes
and assume that $u_T$ is surjective.
Set $X_i = X_{T_i} = X \times_S T_i$ and
$u_i = u_{T_i} : \mathcal{F}_i = \mathcal{F}_{T_i}
\to \mathcal{G}_i = \mathcal{G}_{T_i}$.
To prove (1) we have to show that $u_i$ is surjective for some $i$.
Pick $0 \in I$ and replace $I$ by $\{i \mid i \geq 0\}$.
Since $f$ is quasi-compact we see $X_0$ is quasi-compact.
Hence we may choose a surjective \'etale morphism $\varphi_0 : W_0 \to X_0$
where $W_0$ is an affine scheme. Set $W = W_0 \times_{T_0} T$
and $W_i = W_0 \times_{T_0} T_i$ for $i \geq 0$.  These
are affine schemes endowed
with a surjective \'etale morphisms $\varphi : W \to X_T$ and
$\varphi_i : W_i \to X_i$. Note that $W = \lim W_i$.
Hence $\varphi^*u_T$ is surjective and it suffices to prove that
$\varphi_i^*u_i$ is surjective for some $i$. Thus we have reduced
the problem to the affine case which is
Algebra, Lemma \ref{algebra-lemma-module-map-property-in-colimit} part (2).

\medskip\noindent
Proof of (2). Assume $\mathcal{F}$ is of finite type with scheme theoretic
support $Z \subset B$ quasi-compact over $B$. Let $T = \lim_{i \in I} T_i$
be a directed limit of affine $B$-schemes and assume that $u_T$ is zero.
Set $X_i = T_i \times_B X$ and denote $u_i : \mathcal{F}_i \to \mathcal{G}_i$
the pullback. Choose $0 \in I$ and replace $I$ by
$\{i \mid i \geq 0\}$. Set $Z_0 = Z \times_X X_0$. By
Morphisms of Spaces, Lemma \ref{spaces-morphisms-lemma-support-finite-type}
the support of $\mathcal{F}_i$ is $|Z_0|$. Since $|Z_0|$ is quasi-compact
we can find an affine scheme $W_0$ and an \'etale morphism $W_0 \to X_0$
such that $|Z_0| \subset \Im(|W_0| \to |X_0|)$.
Set $W = W_0 \times_{T_0} T$ and $W_i = W_0 \times_{T_0} T_i$ for $i \geq 0$.
These are affine schemes endowed
with \'etale morphisms $\varphi : W \to X_T$ and
$\varphi_i : W_i \to X_i$. Note that $W = \lim W_i$
and that the support of $\mathcal{F}_T$ and $\mathcal{F}_i$
is contained in the image of $|W| \to |X_T|$ and $|W_i| \to |X_i|$.
Now $\varphi^*u_T$ is injective and it suffices to prove that
$\varphi_i^*u_i$ is injective for some $i$.
Thus we have reduced the problem to the affine case which is
Algebra, Lemma \ref{algebra-lemma-module-map-property-in-colimit} part (1).

\medskip\noindent
Proof of (3). This can be proven in exactly the same manner as in the
previous two paragraphs using
Algebra, Lemma \ref{algebra-lemma-module-map-property-in-colimit} part (3).
We can also deduce it from (1) and (2) as follows.
Let $T = \lim_{i \in I} T_i$ be a directed limit of affine $B$-schemes
and assume that $u_T$ is an isomorphism. By part (1) there exists
an $0 \in I$ such that $u_{T_0}$ is surjective. Set
$\mathcal{K} = \Ker(u_{T_0})$ and consider the map of quasi-coherent
modules $v : \mathcal{K} \to \mathcal{F}_{T_0}$. For $i \geq 0$ the base
change $v_{T_i}$ is zero if and only if $u_i$ is an isomorphism. Moreover,
$v_T$ is zero. Since $\mathcal{G}_{T_0}$
is of finite presentation, $\mathcal{F}_{T_0}$ is of finite type, and
$u_{T_0}$ is surjective we conclude that $\mathcal{K}$ is of finite type
(Modules on Sites, Lemma
\ref{sites-modules-lemma-kernel-surjection-finite-onto-finite-presentation}).
It is clear that the support of $\mathcal{K}$ is contained in the
support of $\mathcal{F}_{T_0}$ which is quasi-compact over $T_0$.
Hence we can apply part (2) to see that $v_{T_i}$ is zero for some $i$.
\end{proof}

\begin{lemma}
\label{lemma-relate-zero-iso}
In Situation \ref{situation-iso} suppose given an exact sequence
$$
\mathcal{F} \xrightarrow{u} \mathcal{G} \xrightarrow{v} \mathcal{H} \to 0
$$
Then we have $F_{v, iso} = F_{u, zero}$ with obvious notation.
\end{lemma}

\begin{proof}
Since pullback is right exact we see that
$\mathcal{F}_T \to \mathcal{G}_T \to \mathcal{H}_T \to 0$
is exact for every scheme $T$ over $B$. Hence $u_T$ is
surjective if and only if $v_T$ is an isomorphism.
\end{proof}

\begin{lemma}
\label{lemma-relate-zero-affine}
In Situation \ref{situation-iso} suppose given an affine morphism
$i : Z \to X$ and a quasi-coherent $\mathcal{O}_Z$-module $\mathcal{H}$
such that $\mathcal{G} = i_*\mathcal{H}$. Let
$v : i^*\mathcal{F} \to \mathcal{H}$ be the map adjoint to $u$.
Then
\begin{enumerate}
\item $F_{v, zero} = F_{u, zero}$, and
\item if $i$ is a closed immersion, then $F_{v, surj} = F_{u, surj}$.
\end{enumerate}
\end{lemma}

\begin{proof}
Let $T$ be a scheme over $B$. Denote $i_T : Z_T \to X_T$
the base change of $i$ and $\mathcal{H}_T$ the pullback of $\mathcal{H}$
to $Z_T$. Observe that $(i^*\mathcal{F})_T = i_T^*\mathcal{F}_T$
and $i_{T, *}\mathcal{H}_T = (i_*\mathcal{H})_T$.
The first statement follows from commutativity of pullbacks
and the second from Cohomology of Spaces, Lemma
\ref{spaces-cohomology-lemma-affine-base-change}.
Hence we see that $u_T$ and $v_T$ are adjoint maps as well.
Thus $u_T = 0$ if and only if $v_T = 0$. This proves (1).
In case (2) we see that $u_T$ is surjective if and only if
$v_T$ is surjective because $u_T$ factors as
$$
\mathcal{F}_T \to
i_{T, *}i_T^*\mathcal{F}_T \xrightarrow{i_{T, *}v_T} i_{T, *}\mathcal{H}_T
$$
and the fact that $i_{T, *}$ is an exact functor
fully faithfully embedding the category of quasi-coherent modules on
$Z_T$ into the category of quasi-coherent $\mathcal{O}_{X_T}$-modules.
See Morphisms of Spaces, Lemma \ref{spaces-morphisms-lemma-i-star-equivalence}.
\end{proof}

\begin{lemma}
\label{lemma-relate-zero-affine-push}
In Situation \ref{situation-iso} suppose given an affine morphism
$g : X \to X'$. Set $u' = f_*u : f_*\mathcal{F} \to f_*\mathcal{G}$.
Then $F_{u, iso} = F_{u', iso}$, $F_{u, inj} = F_{u', inj}$,
$F_{u, surj} = F_{u', surj}$, and $F_{u, zero} = F_{u', zero}$.
\end{lemma}

\begin{proof}
By Cohomology of Spaces, Lemma
\ref{spaces-cohomology-lemma-affine-base-change}
we have $g_{T, *}u_T = u'_T$.
Moreover, $g_{T, *} : \QCoh(\mathcal{O}_{X_T}) \to \QCoh(\mathcal{O}_X)$
is a faithful, exact functor reflecting isomorphisms, injective maps,
and surjective maps.
\end{proof}

\begin{situation}
\label{situation-flat}
Let $S$ be a scheme.
Let $f : X \to Y$ be a morphism of algebraic spaces over $S$.
Let $\mathcal{F}$ be a quasi-coherent $\mathcal{O}_X$-module.
For any scheme $T$ over $Y$ we will denote $\mathcal{F}_T$ the
base change of $\mathcal{F}$ to $T$, in other words, $\mathcal{F}_T$
is the pullback of $\mathcal{F}$ via the projection morphism
$X_T = X \times_Y T \to X$. Since the base change of a flat module
is flat we obtain a functor
\begin{equation}
\label{equation-flat}
F_{flat} : (\Sch/Y)^{opp} \longrightarrow \textit{Sets}, \quad
T \longrightarrow \left\{
\begin{matrix}
\{*\} & \text{if } \mathcal{F}_T \text{ is flat over }T, \\
\emptyset & \text{else.}
\end{matrix}
\right.
\end{equation}
\end{situation}

\noindent
In Situation \ref{situation-flat} we sometimes think of $F_{flat}$ as a functor
$(\Sch/S)^{opp} \to \textit{Sets}$ endowed with a morphism
$F_{flat} \to Y$. Namely, if $T$ is a scheme over $S$, then
an element $h \in F_{flat}(T)$ is a morphism $h : T \to Y$
such that the base change of $\mathcal{F}$ via $h$ is flat over $T$.
In particular, when we say
that $F_{flat}$ is an algebraic space, we mean that the corresponding
functor $(\Sch/S)^{opp} \to \textit{Sets}$ is an algebraic space.

\begin{lemma}
\label{lemma-flat}
In Situation \ref{situation-flat}.
\begin{enumerate}
\item The functor $F_{flat}$ satisfies the sheaf property for the fpqc topology.
\item If $f$ is quasi-compact and locally of finite presentation
and $\mathcal{F}$ is of finite presentation, then the functor
$F_{flat}$ is limit preserving.
\end{enumerate}
\end{lemma}

\begin{proof}
Part (1) follows from the following statement: If $T' \to T$ is a surjective
flat morphism of algebraic spaces over $Y$, then
$\mathcal{F}_{T'}$ is flat over $T'$ if and only if
$\mathcal{F}_T$ is flat over $T$, see
Morphisms of Spaces, Lemma \ref{spaces-morphisms-lemma-base-change-module-flat}.
Part (2) follows from
Limits of Spaces, Lemma \ref{spaces-limits-lemma-descend-flat}
if $f$ is also quasi-separated (i.e., $f$ is of finite presentation).
For the general case, first reduce to the case where the
base is affine and then cover $X$ by finitely many affines
to reduce to the quasi-separated case. Details omitted.
\end{proof}










\section{Making a map zero}
\label{section-zero-map}

\noindent
This section has no analogue in the corresponding chapter on schemes.

\begin{situation}
\label{situation-somewhat-closed}
Let $S = \Spec(R)$ be an affine scheme. Let $X$ be an algebraic space over
$S$. Let $u : \mathcal{F} \to \mathcal{G}$ be a map of quasi-coherent
$\mathcal{O}_X$-modules. Assume $\mathcal{G}$ flat over $S$.
\end{situation}

\begin{lemma}
\label{lemma-F-zero-somewhat-closed}
In Situation \ref{situation-somewhat-closed}. Let $T \to S$
be a quasi-compact morphism of schemes such that the base change $u_T$ is
zero. Then exists a closed subscheme $Z \subset S$ such that
(a) $T \to S$ factors through $Z$ and (b) the base change $u_Z$ is zero.
If $\mathcal{F}$ is a finite type $\mathcal{O}_X$-module and
the scheme theoretic support of $\mathcal{F}$ is quasi-compact,
then we can take $Z \to S$ of finite presentation.
\end{lemma}

\begin{proof}
Let $U \to X$ be a surjective \'etale morphism of algebraic spaces
where $U = \coprod U_i$ is a disjoint union of affine schemes (see
Properties of Spaces, Lemma
\ref{spaces-properties-lemma-cover-by-union-affines}).
By Lemma \ref{lemma-iso-go-up} we see that we may
replace $X$ by $U$. In other words, we may assume that $X = \coprod X_i$
is a disjoint union of affine schemes $X_i$. Suppose that we can prove
the lemma for $u_i = u|_{X_i}$. Then we find a closed subscheme
$Z_i \subset S$ such that $T \to S$ factors through $Z_i$ and
$u_{i, Z_i}$ is zero. If
$Z_i = \Spec(R/I_i) \subset \Spec(R) = S$, then taking
$Z = \Spec(R/\sum I_i)$ works. Thus we may assume that
$X = \Spec(A)$ is affine.

\medskip\noindent
Choose a finite affine open covering $T = T_1 \cup \ldots \cup T_m$.
It is clear that we may replace $T$ by $\coprod_{j = 1, \ldots, m} T_j$.
Hence we may assume $T$ is affine. Say $T = \Spec(R')$.
Let $u : M \to N$ be the homomorphisms of $A$-modules
corresponding to $u : \mathcal{F} \to \mathcal{G}$.
Then $N$ is a flat $R$-module as $\mathcal{G}$ is flat over $S$.
The assumption of the lemma means that the composition
$$
M \otimes_R R' \to N \otimes_R R'
$$
is zero. Let $z \in M$. By Lazard's theorem
(Algebra, Theorem \ref{algebra-theorem-lazard}) and the fact
that $\otimes$ commutes with colimits we can find free $R$-module
$F_z$, an element $\tilde z \in F_z$, and a map $F_z \to N$ such that
$u(z)$ is the image of $\tilde z$ and $\tilde z$ maps to zero in
$F_z \otimes_R R'$. Choose a basis $\{e_{z, \alpha}\}$ of $F_z$ and write
$\tilde z = \sum f_{z, \alpha} e_{z, \alpha}$ with $f_{z, \alpha} \in R$.
Let $I \subset R$ be the ideal generated by the elements $f_{z, \alpha}$
with $z$ ranging over all elements of $M$.
By construction $I$ maps to zero in $R'$ and the elements $\tilde z$
map to zero in $F_z/IF_z$ whence in $N/IN$. Thus $Z = \Spec(R/I)$
is a solution to the problem in this case.

\medskip\noindent
Assume $\mathcal{F}$ is of finite type with quasi-compact scheme
theoretic support. Write $Z = \Spec(R/I)$.
Write $I = \bigcup I_\lambda$ as a filtered union of finitely generated
ideals. Set $Z_\lambda = \Spec(R/I_\lambda)$, so $Z = \colim Z_\lambda$.
Since $u_Z$ is zero, we see that $u_{Z_\lambda}$ is zero
for some $\lambda$ by Lemma \ref{lemma-iso-limits}.
This finishes the proof of the lemma.
\end{proof}

\begin{lemma}
\label{lemma-F-zero-module-map}
Let $A$ be a ring. Let $u : M \to N$ be a map of $A$-modules.
If $N$ is projective as an $A$-module, then there exists an ideal
$I \subset A$ such that for any ring map $\varphi : A \to B$
the following are equivalent
\begin{enumerate}
\item $u \otimes 1 : M \otimes_A B \to N \otimes_A B$ is zero, and
\item $\varphi(I) = 0$.
\end{enumerate}
\end{lemma}

\begin{proof}
As $N$ is projective we can find a projective $A$-module $C$
such that $F = N \oplus C$ is a free $R$-module.
By replacing $u$ by $u \oplus 1 : F = M \oplus C \to N \oplus C$
we see that we may assume $N$ is free. In this case let $I$ be
the ideal of $A$ generated by coefficients of all the elements of
$\Im(u)$ with respect to some (fixed) basis of $N$.
\end{proof}

\begin{lemma}
\label{lemma-F-zero-somewhat-closed-points}
In Situation \ref{situation-somewhat-closed}.
Let $T \subset S$ be a subset. Let $s \in S$ be in the closure of $T$.
For $t \in T$, let $u_t$ be the pullback of $u$ to $X_t$
and let $u_s$ be the pullback of $u$ to $X_s$.
If $X$ is locally of finite presentation over $S$,
$\mathcal{G}$ is of finite presentation\footnote{It would
suffice if $X$ is locally of finite type over $S$
and $\mathcal{G}$ is finitely presented relative to $S$,
but this notion hasn't yet been defined in the setting
of algebraic spaces. The definition for schemes is
given in More on Morphisms, Section
\ref{more-morphisms-section-finite-type-finite-presentation}.}, and
$u_t = 0$ for all $t \in T$, then $u_s = 0$.
\end{lemma}

\begin{proof}
To check whether $u_s$ is zero, is \'etale local on the fibre $X_s$.
Hence we may pick a point $x \in |X_s| \subset |X|$ and check
in an \'etale neighbourhood. Choose
$$
\xymatrix{
(X, x) \ar[d] & (X', x') \ar[l]^g \ar[d] \\
(S, s) & (S', s') \ar[l]
}
$$
as in Proposition \ref{proposition-finite-presentation-flat-at-point}.
Let $T' \subset S'$ be the inverse image of $T$. Observe that
$s'$ is in the closure of $T'$ because $S' \to S$ is open.
Hence we reduce to the algebra problem described in the
next paragraph.

\medskip\noindent
We have an $R$-module map $u : M \to N$ such that $N$ is projective
as an $R$-module and such that
$u_t : M \otimes_R \kappa(t) \to N \otimes_R \kappa(t)$
is zero for each $t \in T$. Problem: show that $u_s = 0$.
Let $I \subset R$ be the ideal defined in Lemma \ref{lemma-F-zero-module-map}.
Then $I$ maps to zero in $\kappa(t)$ for all $t \in T$.
Hence $T \subset V(I)$. Since $s$ is in the closure of $T$
we see that $s \in V(I)$. Hence $u_s = 0$.
\end{proof}

\noindent
It would be interesting to find a ``simple'' direct proof of either
Lemma \ref{lemma-F-zero-closed-pure} or
Lemma \ref{lemma-F-zero-closed-proper}
using arguments like those used in
Lemmas \ref{lemma-F-zero-somewhat-closed} and
\ref{lemma-F-zero-somewhat-closed-points}.
A ``classical'' proof of this lemma when $f : X \to B$ is a projective
morphism and $B$ a Noetherian scheme would be: (a) choose a relatively ample
invertible sheaf $\mathcal{O}_X(1)$, (b) set
$u_n : f_*\mathcal{F}(n) \to f_*\mathcal{G}(n)$,
(c) observe that $f_*\mathcal{G}(n)$ is a finite locally free sheaf
for all $n \gg 0$, and (d) $F_{zero}$ is represented by the vanishing
locus of $u_n$ for some $n \gg 0$.

\begin{lemma}
\label{lemma-F-zero-closed-pure}
In Situation \ref{situation-iso}. Assume
\begin{enumerate}
\item $f$ is of finite presentation, and
\item $\mathcal{G}$ is of finite presentation,
flat over $B$, and pure relative to $B$.
\end{enumerate}
Then $F_{zero}$ is an algebraic space and $F_{zero} \to B$
is a closed immersion. If $\mathcal{F}$ is of finite type, then
$F_{zero} \to B$ is of finite presentation.
\end{lemma}

\begin{proof}
By Lemma \ref{lemma-finite-type-flat-pure-is-universal}
the module $\mathcal{G}$ is universally pure relative to $B$.
In order to prove that $F_{zero}$ is an algebraic space,
it suffices to show that $F_{zero} \to B$ is representable, see
Spaces, Lemma \ref{spaces-lemma-representable-over-space}.
Let $B' \to B$ be a morphism where $B'$ is a scheme and let
$u' : \mathcal{F}' \to \mathcal{G}'$ be the pullback of $u$ to $X' = X_{B'}$.
Then the associated functor $F'_{zero}$ equals $F_{zero} \times_B B'$.
This reduces us to the case that $B$ is a scheme.

\medskip\noindent
Assume $B$ is a scheme. We will show that $F_{zero}$ is representable
by a closed subscheme of $B$. By Lemma \ref{lemma-iso-sheaf} and
Descent, Lemmas \ref{descent-lemma-closed-immersion} and
\ref{descent-lemma-descent-data-sheaves}
the question is local for the \'etale topology on $B$. Let $b \in B$.
We first replace $B$ by an affine neighbourhood of $b$.
Choose a diagram
$$
\xymatrix{
X \ar[d] & X' \ar[l]^g \ar[d] \\
B & B' \ar[l]
}
$$
and $b' \in B'$ mapping to $b \in B$
as in Lemma \ref{lemma-finite-presentation-flat-along-fibre}.
As we are working \'etale locally, we may replace
$B$ by $B'$ and assume that we have a diagram
$$
\xymatrix{
X \ar[rd] & & X' \ar[ll]^g \ar[ld] \\
& B
}
$$
with $B$ and $X'$ affine such that $\Gamma(X', g^*\mathcal{G})$
is a projective $\Gamma(B, \mathcal{O}_B)$-module and
$g(|X'|) \supset |X_b|$. Let $U \subset X$ be the open subspace
with $|U| = g(|X'|)$. By
Divisors on Spaces, Lemma
\ref{spaces-divisors-lemma-relative-assassin-constructible} the set
$$
E = \{t \in B : \text{Ass}_{X_t}(\mathcal{G}_t) \subset |U_t|\} =
\{t \in B : \text{Ass}_{X/B}(\mathcal{G}) \cap |X_t| \subset |U_t|\}
$$
is constructible in $B$. By Lemma \ref{lemma-pure-on-top} part (2)
we see that $E$ contains $\Spec(\mathcal{O}_{B, b})$. By
Morphisms, Lemma \ref{morphisms-lemma-constructible-containing-open}
we see that $E$ contains an open neighbourhood of $b$. Hence after
replacing $B$ by a smaller affine neighbourhood of $b$ we may assume that
$\text{Ass}_{X/B}(\mathcal{G}) \subset g(|X'|)$.

\medskip\noindent
From Lemma \ref{lemma-universally-separating}
it follows that $u : \mathcal{F} \to \mathcal{G}$ is injective if and only if
$g^*u : g^*\mathcal{F} \to g^*\mathcal{G}$ is injective, and the same remains
true after any base change. Hence we have reduced to the case where,
in addition to the assumptions in the theorem, $X \to B$ is a morphism of
affine schemes and $\Gamma(X, \mathcal{G})$ is a projective
$\Gamma(B, \mathcal{O}_B)$-module. This case follows immediately from
Lemma \ref{lemma-F-zero-module-map}.

\medskip\noindent
We still have to show that $F_{zero} \to B$ is of finite presentation if
$\mathcal{F}$ is of finite type. This follows from
Lemma \ref{lemma-iso-limits} combined with
Limits of Spaces, Proposition
\ref{spaces-limits-proposition-characterize-locally-finite-presentation}.
\end{proof}

\begin{lemma}
\label{lemma-F-zero-closed-proper}
In Situation \ref{situation-iso}. Assume
\begin{enumerate}
\item $f$ is locally of finite presentation,
\item $\mathcal{G}$ is an $\mathcal{O}_X$-module of finite presentation
flat over $B$,
\item the support of $\mathcal{G}$ is proper over $B$.
\end{enumerate}
Then the functor $F_{zero}$ is an algebraic space and $F_{zero} \to B$
is a closed immersion. If $\mathcal{F}$ is of finite type, then
$F_{zero} \to B$ is of finite presentation.
\end{lemma}

\begin{proof}
If $f$ is of finite presentation, then this follows immediately from
Lemmas \ref{lemma-F-zero-closed-pure} and \ref{lemma-proper-pure}.
This is the only case of interest and we urge the reader to skip
the rest of the proof, which deals with the possibility (allowed
by the assumptions in this lemma)
that $f$ is not quasi-separated or quasi-compact.

\medskip\noindent
Let $i : Z \to X$ be the closed subspace cut out by the zeroth
fitting ideal of $\mathcal{G}$
(Divisors on Spaces, Section
\ref{spaces-divisors-section-fitting-ideals}).
Then $Z \to B$ is proper by assumption (see
Derived Categories of Spaces, Section
\ref{spaces-perfect-section-proper-over-base}).
On the other hand $i$ is of finite presentation
(Divisors on Spaces, Lemma
\ref{spaces-divisors-lemma-fitting-ideal-of-finitely-presented} and
Morphisms of Spaces, Lemma
\ref{spaces-morphisms-lemma-closed-immersion-finite-presentation}).
There exists a quasi-coherent $\mathcal{O}_Z$-module
$\mathcal{H}$ of finite type with $i_*\mathcal{H} = \mathcal{G}$
(Divisors on Spaces, Lemma
\ref{spaces-divisors-lemma-on-subscheme-cut-out-by-Fit-0}).
In fact $\mathcal{H}$ is of finite presentation as an $\mathcal{O}_Z$-module
by Algebra, Lemma \ref{algebra-lemma-finitely-presented-over-subring}
(details omitted).
Then $F_{zero}$ is the same as the functor $F_{zero}$
for the map $i^*\mathcal{F} \to \mathcal{H}$ adjoint to $u$, see
Lemma \ref{lemma-relate-zero-affine}.
The sheaf $\mathcal{H}$ is flat relative to $B$ because
the same is true for $\mathcal{G}$ (check on stalks; details omitted).
Moreover, note that if $\mathcal{F}$ is of finite type,
then $i^*\mathcal{F}$ is of finite type.
Hence we have reduced the lemma to the case
discussed in the first paragraph of the proof.
\end{proof}








\section{Flattening a map}
\label{section-flattening-map}

\noindent
This section is the analogue of
More on Flatness, Section \ref{flat-section-flattening-map}.
In particular the following result is a variant of
More on Flatness, Theorem \ref{flat-theorem-flattening-map}.

\begin{theorem}
\label{theorem-flattening-map}
In Situation \ref{situation-iso} assume
\begin{enumerate}
\item $f$ is of finite presentation,
\item $\mathcal{F}$ is of finite presentation, flat over $B$, and
pure relative to $B$, and
\item $u$ is surjective.
\end{enumerate}
Then $F_{iso}$ is representable by a closed immersion $Z \to B$.
Moreover $Z \to S$ is of finite presentation if $\mathcal{G}$ is
of finite presentation.
\end{theorem}

\begin{proof}
Let $\mathcal{K} = \Ker(u)$ and denote $v : \mathcal{K} \to \mathcal{F}$
the inclusion. By Lemma \ref{lemma-relate-zero-iso} we see that
$F_{u, iso} = F_{v, zero}$. By Lemma \ref{lemma-F-zero-closed-pure}
applied to $v$ we see that $F_{u, iso} = F_{v, zero}$ is representable
by a closed subspace of $B$. Note that $\mathcal{K}$ is of finite type
if $\mathcal{G}$ is of finite presentation, see
Modules on Sites, Lemma
\ref{sites-modules-lemma-kernel-surjection-finite-onto-finite-presentation}.
Hence we also obtain the final statement of the lemma.
\end{proof}

\begin{lemma}
\label{lemma-F-iso-closed}
In Situation \ref{situation-iso}. Assume
\begin{enumerate}
\item $f$ is locally of finite presentation,
\item $\mathcal{F}$ is locally of finite presentation and flat over $B$,
\item the support of $\mathcal{F}$ is proper over $B$, and
\item $u$ is surjective.
\end{enumerate}
Then the functor $F_{iso}$ is an algebraic space and $F_{iso} \to B$
is a closed immersion. If $\mathcal{G}$ is of finite presentation, then
$F_{iso} \to B$ is of finite presentation.
\end{lemma}

\begin{proof}
Let $\mathcal{K} = \Ker(u)$ and denote $v : \mathcal{K} \to \mathcal{F}$
the inclusion. By Lemma \ref{lemma-relate-zero-iso} we see that
$F_{u, iso} = F_{v, zero}$. By Lemma \ref{lemma-F-zero-closed-proper}
applied to $v$ we see that $F_{u, iso} = F_{v, zero}$ is representable
by a closed subspace of $B$. Note that $\mathcal{K}$ is of finite type
if $\mathcal{G}$ is of finite presentation, see
Modules on Sites, Lemma
\ref{sites-modules-lemma-kernel-surjection-finite-onto-finite-presentation}.
Hence we also obtain the final statement of the lemma.
\end{proof}

\noindent
We will use the following (easy) result when discussing the Quot functor.

\begin{lemma}
\label{lemma-F-surj-open}
In Situation \ref{situation-iso}. Assume
\begin{enumerate}
\item $f$ is locally of finite presentation,
\item $\mathcal{G}$ is of finite type,
\item the support of $\mathcal{G}$ is proper over $B$.
\end{enumerate}
Then $F_{surj}$ is an algebraic space and $F_{surj} \to B$
is an open immersion.
\end{lemma}

\begin{proof}
Consider $\Coker(u)$. Observe that
$\Coker(u_T) = \Coker(u)_T$ for any $T/B$.
Note that formation of the support of a finite type
quasi-coherent module commutes with pullback
(Morphisms of Spaces, Lemma \ref{spaces-morphisms-lemma-support-covering}).
Hence $F_{surj}$ is representable by the open subspace of $B$
corresponding to the open set
$$
|B| \setminus |f|(\text{Supp}(\Coker(u)))
$$
see Properties of Spaces, Lemma \ref{spaces-properties-lemma-open-subspaces}.
This is an open because $|f|$ is closed on $\text{Supp}(\mathcal{G})$
and $\text{Supp}(\Coker(u))$ is a closed subset of
$\text{Supp}(\mathcal{G})$.
\end{proof}






\section{Flattening in the local case}
\label{section-flattening-local}

\noindent
This section is the analogue of More on Flatness, Section
\ref{flat-section-flattening-local}.

\begin{lemma}
\label{lemma-freebie}
Let $S$ be the spectrum of a henselian local ring with closed point $s$.
Let $X \to S$ be a morphism of algebraic spaces which is
locally of finite type.
Let $\mathcal{F}$ be a finite type quasi-coherent $\mathcal{O}_X$-module.
Let $E \subset |X_s|$ be a subset. There exists a closed subscheme
$Z \subset S$ with the following property: for any morphism of pointed
schemes $(T, t) \to (S, s)$ the following are equivalent
\begin{enumerate}
\item $\mathcal{F}_T$ is flat over $T$ at all points of
$|X_t|$ which map to a point of $E \subset |X_s|$, and
\item $\Spec(\mathcal{O}_{T, t}) \to S$ factors through $Z$.
\end{enumerate}
Moreover, if $X \to S$ is locally of finite presentation,
$\mathcal{F}$ is of finite presentation, and $E \subset |X_s|$ is
closed and quasi-compact, then $Z \to S$ is of finite presentation.
\end{lemma}

\begin{proof}
Choose a scheme $U$ and an \'etale morphism $\varphi : U \to X$.
Let $E' \subset |U_s|$ be the inverse image of $E$. If
$E' \to E$ is surjective, then condition (1) is equivalent to:
$(\varphi^*\mathcal{F})_T$ is flat over $T$ at all points of
$|U_t|$ which map to a point of $E' \subset |U_t|$.
Choosing $\varphi$ to be surjective, we reduced to the case of schemes which is
More on Flatness, Lemma \ref{flat-lemma-freebie}.
If $E$ is closed and quasi-compact, then we may choose $U$ to be
affine such that $E' \to E$ is surjective. Then $E'$ is closed
and quasi-compact and the final statement follows from the
final statement of
More on Flatness, Lemma \ref{flat-lemma-freebie}.
\end{proof}









\section{Universal flattening}
\label{section-flattening-final}

\noindent
This section is the analogue of
More on Flatness, Section \ref{flat-section-flattening-final}.
Our main aim is to prove
Lemma \ref{lemma-when-universal-flattening}.
However, we do not see a way to deduce this result
from the corresponding result for schemes directly.
Hence we have to redevelop some of the material here.
But first a definition.

\begin{definition}
\label{definition-flattening}
Let $S$ be a scheme.
Let $X \to Y$ be a morphism of algebraic spaces over $S$.
Let $\mathcal{F}$ be a quasi-coherent $\mathcal{O}_X$-module.
We say that the {\it universal flattening of $\mathcal{F}$ exists}
if the functor $F_{flat}$ defined in Situation \ref{situation-flat}
is an algebraic space.
We say that the {\it universal flattening of $X$ exists}
if the universal flattening of $\mathcal{O}_X$ exists.
\end{definition}

\noindent
This is a bit unsatisfactory, because here the definition of
universal flattening does not agree with the one used in the
case of schemes, as we don't know whether every monomorphism of
algebraic spaces is representable (More on Morphisms of Spaces,
Section \ref{spaces-more-morphisms-section-monomorphisms}).
Hopefully no confusion will ever result from this.

\begin{lemma}
\label{lemma-pre-flat-dimension-n}
Let $S$ be a scheme.
Let $f : X \to Y$ be a morphism of algebraic spaces which is
locally of finite type.
Let $\mathcal{F}$ be a quasi-coherent $\mathcal{O}_X$-module of finite
type. Let $n \geq 0$. The following are equivalent
\begin{enumerate}
\item for some commutative diagram
$$
\xymatrix{
U \ar[d]_\varphi \ar[r] & V \ar[d] \\
X \ar[r] & Y
}
$$
with surjective, \'etale vertical arrows where $U$ and $V$ are
schemes, the sheaf $\varphi^*\mathcal{F}$ is flat over $V$
in dimensions $\geq n$ (More on Flatness, Definition
\ref{flat-definition-flat-dimension-n}),
\item for every commutative diagram
$$
\xymatrix{
U \ar[d]_\varphi \ar[r] & V \ar[d] \\
X \ar[r] & Y
}
$$
with \'etale vertical arrows where $U$ and $V$ are schemes,
the sheaf $\varphi^*\mathcal{F}$ is flat over $V$ in dimensions $\geq n$, and
\item for $x \in |X|$ such that $\mathcal{F}$ is not flat at $x$
over $Y$ the transcendence degree of $x/f(x)$ is $< n$ (Morphisms of Spaces,
Definition \ref{spaces-morphisms-definition-dimension-fibre}).
\end{enumerate}
If this is true, then it remains true after any base change $Y' \to Y$.
\end{lemma}

\begin{proof}
Suppose that we have a diagram as in (1). Then the equivalence of
the conditions in More on Flatness, Lemma \ref{flat-lemma-pre-flat-dimension-n}
shows that (1) and (3) are equivalent. But condition (3) is inherited
by $\varphi^*\mathcal{F}$ for any $U \to V$ as in (2).
Whence we see that (3) implies (2) by the result for schemes again.
The result for schemes also implies the statement on base change.
\end{proof}

\begin{definition}
\label{definition-flat-dimension-n}
Let $S$ be a scheme. Let $f : X \to Y$ be a morphism of algebraic spaces
over $S$ which is locally of finite type.
Let $\mathcal{F}$ be a quasi-coherent $\mathcal{O}_X$-module of finite type.
Let $n \geq 0$.
We say {\it $\mathcal{F}$ is flat over $Y$ in dimensions $\geq n$}
if the equivalent conditions of Lemma \ref{lemma-pre-flat-dimension-n}
are satisfied.
\end{definition}

\begin{situation}
\label{situation-flat-dimension-n}
Let $S$ be a scheme. Let $f : X \to Y$ be a morphism of algebraic spaces
over $S$ which is locally of finite type. Let $\mathcal{F}$ be a quasi-coherent
$\mathcal{O}_X$-module of finite type. For any scheme $T$ over $Y$ we will
denote $\mathcal{F}_T$ the base change of $\mathcal{F}$ to $T$, in other words,
$\mathcal{F}_T$ is the pullback of $\mathcal{F}$ via the projection morphism
$X_T = X \times_Y T \to X$. Note that $f_T : X_T \to T$ is of finite type
and that $\mathcal{F}_T$ is an $\mathcal{O}_{X_T}$-module of finite type
(Morphisms of Spaces, Lemma
\ref{spaces-morphisms-lemma-base-change-finite-type}
and
Modules on Sites, Lemma \ref{sites-modules-lemma-local-pullback}).
Let $n \geq 0$. By Definition \ref{definition-flat-dimension-n} and
Lemma \ref{lemma-pre-flat-dimension-n} we obtain a functor
\begin{equation}
\label{equation-flat-dimension-n}
F_n : (\Sch/Y)^{opp} \longrightarrow \textit{Sets}, \quad
T \longrightarrow \left\{
\begin{matrix}
\{*\} & \text{if }\mathcal{F}_T\text{ is flat over }T\text{ in }\dim \geq n, \\
\emptyset & \text{else.}
\end{matrix}
\right.
\end{equation}
\end{situation}

\noindent
In Situation \ref{situation-flat-dimension-n} we sometimes think of
$F_n$ as a functor $(\Sch/S)^{opp} \to \textit{Sets}$ endowed with a morphism
$F_n \to Y$. Namely, if $T$ is a scheme over $S$, then
an element $h \in F_n(T)$ is a morphism $h : T \to Y$
such that the base change of $\mathcal{F}$ via $h$ is flat over $T$
in $\dim \geq n$. In particular, when we say
that $F_n$ is an algebraic space, we mean that the corresponding
functor $(\Sch/S)^{opp} \to \textit{Sets}$ is an algebraic space.

\begin{lemma}
\label{lemma-flat-dimension-n}
In Situation \ref{situation-flat-dimension-n}.
\begin{enumerate}
\item The functor $F_n$ satisfies the sheaf property for the fpqc topology.
\item If $f$ is quasi-compact and locally of finite presentation
and $\mathcal{F}$ is of finite presentation, then the functor $F_n$ is
limit preserving.
\end{enumerate}
\end{lemma}

\begin{proof}
Proof of (1). Suppose that $\{T_i \to T\}$ is an fpqc covering of
a scheme $T$ over $Y$. We have to show that if $F_n(T_i)$ is nonempty
for all $i$, then $F_n(T)$ is nonempty.
Choose a diagram as in part (1) of Lemma \ref{lemma-pre-flat-dimension-n}.
Denote $F'_n$ the corresponding functor for
$\varphi^*\mathcal{F}$ and the morphism $U \to V$.
By More on Flatness, Lemma \ref{flat-lemma-flat-dimension-n}
we have the sheaf property for $F'_n$.
Thus we get the sheaf property for $F_n$ because
for $T \to Y$ we have $F_n(T) = F'_n(V \times_Y T)$
by Lemma \ref{lemma-pre-flat-dimension-n}
and because $\{V \times_Y T_i \to V \times_Y T\}$
is an fpqc covering.

\medskip\noindent
Proof of (2). Suppose that $T = \lim_{i \in I} T_i$ is a filtered limit
of affine schemes $T_i$ over $Y$ and assume that $F_n(T)$ is nonempty.
We have to show that $F_n(T_i)$ is nonempty for some $i$.
Choose a diagram as in
part (1) of Lemma \ref{lemma-pre-flat-dimension-n}.
Fix $i \in I$ and choose an affine open $W_i \subset V \times_Y T_i$
mapping surjectively onto $T_i$. For $i' \geq i$ let $W_{i'}$
be the inverse image of $W_i$ in $V \times_Y T_{i'}$ and
let $W \subset V \times_Y T$ be the inverse image of $W_i$.
Then $W = \lim_{i' \geq i} W_i$ is a filtered limit of affine
schemes over $V$. By
Lemma \ref{lemma-pre-flat-dimension-n} again
it suffices to show that $F'_n(W_{i'})$ is nonempty for
some $i' \geq i$. But we know that $F'_n(W)$ is nonempty
because of our assumption that $F_n(T) = F'_n(V \times_Y T)$
is nonempty. Thus we can apply
More on Flatness, Lemma \ref{flat-lemma-flat-dimension-n}
to conclude.
\end{proof}

\begin{lemma}
\label{lemma-localize-flat-dimension-n}
In Situation \ref{situation-flat-dimension-n}.
Let $h : X' \to X$ be an \'etale morphism.
Set $\mathcal{F}' = h^*\mathcal{F}$ and $f' = f \circ h$.
Let $F_n'$ be (\ref{equation-flat-dimension-n})
associated to $(f' : X' \to Y, \mathcal{F}')$.
Then $F_n$ is a subfunctor of $F_n'$ and if
$h(X') \supset \text{Ass}_{X/Y}(\mathcal{F})$, then $F_n = F'_n$.
\end{lemma}

\begin{proof}
Choose $U \to X$, $V \to Y$, $U \to V$ as in part (1) of
Lemma \ref{lemma-pre-flat-dimension-n}. Choose a surjective
\'etale morphism $U' \to U \times_X X'$ where $U'$ is a scheme.
Then we have the lemma for the two functors
$F_{U, n}$ and $F_{U', n}$ determined by $U' \to U$ and $\mathcal{F}|_U$
over $V$, see
More on Flatness, Lemma \ref{flat-lemma-localize-flat-dimension-n}.
On the other hand, Lemma \ref{lemma-pre-flat-dimension-n}
tells us that given $T \to Y$ we have
$F_n(T) = F_{U, n}(V \times_Y T)$
and
$F'_n(T) = F_{U', n}(V \times_Y T)$.
This proves the lemma.
\end{proof}

\begin{theorem}
\label{theorem-flat-dimension-n-representable}
In Situation \ref{situation-flat-dimension-n}.
Assume moreover that $f$ is of finite presentation, that
$\mathcal{F}$ is an $\mathcal{O}_X$-module of finite presentation,
and that $\mathcal{F}$ is pure relative to $Y$.
Then $F_n$ is an algebraic space and
$F_n \to Y$ is a monomorphism of finite presentation.
\end{theorem}

\begin{proof}
The functor $F_n$ is a sheaf for the fppf topology by
Lemma \ref{lemma-flat-dimension-n}.
Since $F_n \to Y$ is a monomorphism of sheaves on
$(\Sch/S)_{fppf}$ we see that $\Delta : F_n \to F_n \times F_n$
is the pullback of the diagonal $\Delta_Y : Y \to Y \times_S Y$.
Hence the representability of $\Delta_Y$ implies the same
thing for $F_n$. Therefore it suffices to prove that
there exists a scheme $W$ over $S$ and a surjective \'etale morphism
$W \to F_n$.

\medskip\noindent
To construct $W \to F_n$ choose an \'etale covering $\{Y_i \to Y\}$
with $Y_i$ a scheme. Let $X_i = X \times_Y Y_i$ and let
$\mathcal{F}_i$ be the pullback of $\mathcal{F}$ to $X_i$.
Then $\mathcal{F}_i$ is pure relative to $Y_i$ either by definition
or by Lemma \ref{lemma-quasi-finite-base-change}.
The other assumptions of the theorem are preserved as well.
Finally, the restriction of $F_n$ to $Y_i$ is
the functor $F_n$ corresponding to $X_i \to Y_i$ and $\mathcal{F}_i$.
Hence it suffices to show: Given $\mathcal{F}$ and $f : X \to Y$
as in the statement of the theorem where $Y$ is a scheme, the
functor $F_n$ is representable by a scheme $Z_n$ and
$Z_n \to Y$ is a monomorphism of finite presentation.

\medskip\noindent
Observe that a monomorphism of finite presentation is
separated and quasi-finite (Morphisms, Lemma
\ref{morphisms-lemma-monomorphism-loc-finite-type-loc-quasi-finite}).
Hence combining
Descent, Lemma \ref{descent-lemma-descent-data-sheaves},
More on Morphisms, Lemma
\ref{more-morphisms-lemma-separated-locally-quasi-finite-morphisms-fppf-descend}
, and
Descent, Lemmas \ref{descent-lemma-descending-property-monomorphism} and
\ref{descent-lemma-descending-property-finite-presentation}
we see that the question is local for the \'etale topology on $Y$.

\medskip\noindent
In particular the situation is local for the Zariski topology on $Y$
and we may assume that $Y$ is affine. In this case the dimension of the
fibres of $f$ is bounded above, hence we see that $F_n$ is representable
for $n$ large enough. Thus we may use descending induction on $n$.
Suppose that we know $F_{n + 1}$ is representable by a monomorphism
$Z_{n + 1} \to Y$ of finite presentation. Consider the base change
$X_{n + 1} = Z_{n + 1} \times_Y X$ and the pullback $\mathcal{F}_{n + 1}$
of $\mathcal{F}$ to $X_{n + 1}$. The morphism $Z_{n + 1} \to Y$ is
quasi-finite as it is a monomorphism of finite presentation, hence
Lemma \ref{lemma-quasi-finite-base-change}
implies that $\mathcal{F}_{n + 1}$ is pure relative to $Z_{n + 1}$.
Since $F_n$ is a subfunctor of $F_{n + 1}$ we conclude that in order
to prove the result for $F_n$ it suffices to prove the result for the
corresponding functor for the situation
$\mathcal{F}_{n + 1}/X_{n + 1}/Z_{n + 1}$.
In this way we reduce to proving the result for $F_n$ in case
$Y_{n + 1} = Y$, i.e., we may assume that $\mathcal{F}$ is flat
in dimensions $\geq n + 1$ over $Y$.

\medskip\noindent
Fix $n$ and assume $\mathcal{F}$ is flat in dimensions $\geq n + 1$
over the affine scheme $Y$.
To finish the proof we have to show that $F_n$ is representable
by a monomorphism $Z_n \to S$ of finite presentation.
Since the question is local in the \'etale topology on $Y$ it suffices to
show that for every $y \in Y$ there exists an \'etale neighbourhood
$(Y', y') \to (Y, y)$ such that the result holds after base change to $Y'$.
Thus by
Lemma \ref{lemma-existence-complete}
we may assume there exist \'etale morphisms $h_j : W_j \to X$,
$j = 1, \ldots, m$ such that for each $j$ there exists a complete
d\'evissage of $\mathcal{F}_j/W_j/Y$ over $y$,
where $\mathcal{F}_j$ is the pullback of $\mathcal{F}$ to $W_j$
and such that $|X_y| \subset \bigcup h_j(W_j)$. Since
$h_j$ is \'etale, by
Lemma \ref{lemma-pre-flat-dimension-n}
the sheaves $\mathcal{F}_j$ are still flat over in
dimensions $\geq n + 1$ over $Y$.
Set $W = \bigcup h_j(W_j)$, which is a quasi-compact open of $X$.
As $\mathcal{F}$ is pure along $X_y$ we see that
$$
E = \{t \in |Y| : \text{Ass}_{X_t}(\mathcal{F}_t) \subset W \}.
$$
contains all generalizations of $y$. By
Divisors on Spaces,
Lemma \ref{spaces-divisors-lemma-relative-assassin-constructible}
$E$ is a constructible subset of $Y$. We have seen that
$\Spec(\mathcal{O}_{Y, y}) \subset E$. By
Morphisms, Lemma \ref{morphisms-lemma-constructible-containing-open}
we see that $E$ contains an open neighbourhood of $y$.
Hence after shrinking $Y$ we may assume that $E = Y$.
It follows from
Lemma \ref{lemma-localize-flat-dimension-n}
that it suffices to prove the lemma for the functor $F_n$ associated to
$X = \coprod W_j$ and $\mathcal{F} = \coprod \mathcal{F}_j$.
If $F_{j, n}$ denotes the functor for $W_j \to Y$ and the sheaf
$\mathcal{F}_j$ we see that $F_n = \prod F_{j, n}$. Hence it suffices
to prove each $F_{j, n}$ is representable by some monomorphism
$Z_{j, n} \to Y$ of finite presentation, since then
$$
Z_n = Z_{1, n} \times_Y \ldots \times_Y Z_{m, n}
$$
Thus we have reduced the theorem to the special case handled in
More on Flatness, Lemma \ref{flat-lemma-flat-dimension-n-representable}.
\end{proof}

\noindent
Thus we finally obtain the desired result.

\begin{lemma}
\label{lemma-when-universal-flattening}
Let $S$ be a scheme.
Let $f : X \to Y$ be a morphism of algebraic spaces over $S$.
Let $\mathcal{F}$ be a quasi-coherent $\mathcal{O}_X$-module.
\begin{enumerate}
\item If $f$ is of finite presentation, $\mathcal{F}$ is an
$\mathcal{O}_X$-module of finite presentation, and $\mathcal{F}$ is
pure relative to $Y$, then there exists a universal flattening
$Y' \to Y$ of $\mathcal{F}$. Moreover $Y' \to Y$ is a monomorphism
of finite presentation.
\item If $f$ is of finite presentation and $X$ is pure relative to $Y$,
then there exists a universal flattening $Y' \to Y$ of $X$.
Moreover $Y' \to Y$ is a monomorphism of finite presentation.
\item If $f$ is proper and of finite presentation and $\mathcal{F}$ is an
$\mathcal{O}_X$-module of finite presentation, then there exists a
universal flattening $Y' \to Y$ of $\mathcal{F}$. Moreover $Y' \to Y$ is
a monomorphism of finite presentation.
\item If $f$ is proper and of finite presentation
then there exists a universal flattening $Y' \to Y$ of $X$.
\end{enumerate}
\end{lemma}

\begin{proof}
These statements follow immediately from
Theorem \ref{theorem-flat-dimension-n-representable}
applied to $F_0 = F_{flat}$
and the fact that if $f$ is proper then $\mathcal{F}$ is automatically
pure over the base, see
Lemma \ref{lemma-proper-pure}.
\end{proof}






\section{Grothendieck's Existence Theorem}
\label{section-existence}

\noindent
This section is the analogue of
More on Flatness, Section \ref{flat-section-existence}
and continues the discussion in
More on Morphisms of Spaces, Section
\ref{spaces-more-morphisms-section-existence-proper}.
We will work in the following situation.

\begin{situation}
\label{situation-existence}
Here we have an inverse system of rings $(A_n)$ with surjective transition
maps whose kernels are locally nilpotent. Set $A = \lim A_n$. We have an
algebraic space $X$ separated and of finite presentation over $A$.
We set $X_n = X \times_{\Spec(A)} \Spec(A_n)$ and we
view it as a closed subspace of $X$. We assume further given a system
$(\mathcal{F}_n, \varphi_n)$ where $\mathcal{F}_n$ is a finitely presented
$\mathcal{O}_{X_n}$-module, flat over $A_n$, with support proper over $A_n$,
and
$$
\varphi_n :
\mathcal{F}_n \otimes_{\mathcal{O}_{X_n}} \mathcal{O}_{X_{n - 1}}
\longrightarrow
\mathcal{F}_{n - 1}
$$
is an isomorphism (notation using the equivalence of
Morphisms of Spaces, Lemma \ref{spaces-morphisms-lemma-i-star-equivalence}).
\end{situation}

\noindent
Our goal is to see if we can find a quasi-coherent sheaf $\mathcal{F}$
on $X$ such that
$\mathcal{F}_n = \mathcal{F} \otimes_{\mathcal{O}_X} \mathcal{O}_{X_n}$
for all $n$.

\begin{lemma}
\label{lemma-compute-what-it-should-be}
In Situation \ref{situation-existence} consider
$$
K = R\lim_{D_\QCoh(\mathcal{O}_X)}(\mathcal{F}_n) =
DQ_X(R\lim_{D(\mathcal{O}_X)}\mathcal{F}_n)
$$
Then $K$ is in $D^b_{\QCoh}(\mathcal{O}_X)$ and in fact
$K$ has nonzero cohomology sheaves only in degrees $\geq 0$.
\end{lemma}

\begin{proof}
Special case of
Derived Categories of Spaces, Example
\ref{spaces-perfect-example-inverse-limit-quasi-coherent}.
\end{proof}

\begin{lemma}
\label{lemma-compute-against-perfect}
In Situation \ref{situation-existence} let $K$ be as in
Lemma \ref{lemma-compute-what-it-should-be}. For any perfect
object $E$ of $D(\mathcal{O}_X)$ we have
\begin{enumerate}
\item $M = R\Gamma(X, K \otimes^\mathbf{L} E)$ is a perfect object of $D(A)$
and there is a canonical isomorphism
$R\Gamma(X_n, \mathcal{F}_n \otimes^\mathbf{L} E|_{X_n}) =
M \otimes_A^\mathbf{L} A_n$
in $D(A_n)$,
\item $N = R\Hom_X(E, K)$ is a perfect object of $D(A)$
and there is a canonical isomorphism
$R\Hom_{X_n}(E|_{X_n}, \mathcal{F}_n) = N \otimes_A^\mathbf{L} A_n$
in $D(A_n)$.
\end{enumerate}
In both statements $E|_{X_n}$ denotes the derived pullback
of $E$ to $X_n$.
\end{lemma}

\begin{proof}
Proof of (2). Write $E_n = E|_{X_n}$ and
$N_n = R\Hom_{X_n}(E_n, \mathcal{F}_n)$.
Recall that $R\Hom_{X_n}(-, -)$ is equal to
$R\Gamma(X_n, R\SheafHom(-, -))$, see
Cohomology on Sites, Section \ref{sites-cohomology-section-global-RHom}.
Hence by Derived Categories of Spaces, Lemma
\ref{spaces-perfect-lemma-base-change-RHom-perfect}
we see that $N_n$ is a perfect object of $D(A_n)$
whose formation commutes with base change. Thus the maps
$N_n \otimes_{A_n}^\mathbf{L} A_{n - 1} \to N_{n - 1}$
coming from $\varphi_n$ are isomorphisms.
By More on Algebra, Lemma \ref{more-algebra-lemma-Rlim-perfect-gives-perfect}
we find that $R\lim N_n$ is perfect and
that its base change back to $A_n$ recovers $N_n$.
On the other hand, the exact functor
$R\Hom_X(E, -) : D_\QCoh(\mathcal{O}_X) \to D(A)$
of triangulated categories commutes with products
and hence with derived limits, whence
$$
R\Hom_X(E, K) =
R\lim R\Hom_X(E, \mathcal{F}_n) =
R\lim R\Hom_X(E_n, \mathcal{F}_n) =
R\lim N_n
$$
This proves (2). To see that (1) holds, translate it into (2)
using Cohomology on Sites, Lemma
\ref{sites-cohomology-lemma-dual-perfect-complex}.
\end{proof}

\begin{lemma}
\label{lemma-relative-pseudo-coherence}
In Situation \ref{situation-existence} let $K$ be as in
Lemma \ref{lemma-compute-what-it-should-be}. Then $K$
is pseudo-coherent relative to $A$.
\end{lemma}

\begin{proof}
Combinging Lemma \ref{lemma-compute-against-perfect} and
Derived Categories of Spaces, Lemma \ref{spaces-perfect-lemma-perfect-enough}
we see that $R\Gamma(X, K \otimes^\mathbf{L} E)$
is pseudo-coherent in $D(A)$ for all pseudo-coherent
$E$ in $D(\mathcal{O}_X)$. Thus the lemma follows from
More on Morphisms of Spaces, Lemma
\ref{spaces-more-morphisms-lemma-characterize-pseudo-coherent}.
\end{proof}

\begin{lemma}
\label{lemma-compute-over-affine}
In Situation \ref{situation-existence} let $K$ be as in
Lemma \ref{lemma-compute-what-it-should-be}. For any
\'etale morphism $U \to X$ with $U$ quasi-compact and quasi-separated we have
$$
R\Gamma(U, K) \otimes_A^\mathbf{L} A_n =
R\Gamma(U_n, \mathcal{F}_n)
$$
in $D(A_n)$ where $U_n = U \times_X X_n$.
\end{lemma}

\begin{proof}
Fix $n$. By Derived Categories of Spaces, Lemma
\ref{spaces-perfect-lemma-computing-sections-as-colim}
there exists a system of perfect complexes $E_m$
on $X$ such that
$R\Gamma(U, K) = \text{hocolim} R\Gamma(X, K \otimes^\mathbf{L} E_m)$.
In fact, this formula holds not just for $K$ but for every object of
$D_\QCoh(\mathcal{O}_X)$.
Applying this to $\mathcal{F}_n$
we obtain
\begin{align*}
R\Gamma(U_n, \mathcal{F}_n)
& =
R\Gamma(U, \mathcal{F}_n) \\
& =
\text{hocolim}_m R\Gamma(X, \mathcal{F}_n \otimes^\mathbf{L} E_m) \\
& =
\text{hocolim}_m R\Gamma(X_n, \mathcal{F}_n \otimes^\mathbf{L} E_m|_{X_n})
\end{align*}
Using Lemma \ref{lemma-compute-against-perfect}
and the fact that $- \otimes_A^\mathbf{L} A_n$
commutes with homotopy colimits we obtain the result.
\end{proof}

\begin{lemma}
\label{lemma-finitely-presented}
In Situation \ref{situation-existence} let $K$ be as in
Lemma \ref{lemma-compute-what-it-should-be}.
Denote $X_0 \subset |X|$ the closed subset
consisting of points lying over the closed subset
$\Spec(A_1) = \Spec(A_2) = \ldots$ of $\Spec(A)$.
There exists an open subspace $W \subset X$ containing $X_0$
such that
\begin{enumerate}
\item $H^i(K)|_W$ is zero unless $i = 0$,
\item $\mathcal{F} = H^0(K)|_W$ is of finite presentation, and
\item $\mathcal{F}_n = \mathcal{F} \otimes_{\mathcal{O}_X} \mathcal{O}_{X_n}$.
\end{enumerate}
\end{lemma}

\begin{proof}
Fix $n \geq 1$. By construction there is a canonical map
$K \to \mathcal{F}_n$ in $D_\QCoh(\mathcal{O}_X)$
and hence a canonical map $H^0(K) \to \mathcal{F}_n$
of quasi-coherent sheaves. This explains the meaning of part (3).

\medskip\noindent
Let $x \in X_0$ be a point. We will find an open neighbourhood $W$
of $x$ such that (1), (2), and (3) are true. Since $X_0$ is quasi-compact
this will prove the lemma. Let $U \to X$ be an \'etale morphism
with $U$ affine and $u \in U$ a point mapping to $x$. Since $|U| \to |X|$
is open it suffices to find an open neighbourhood of $u$ in $U$
where (1), (2), and (3) are true. Say $U = \Spec(B)$.
Choose a surjection $P \to B$ with $P$ smooth over $A$.
By Lemma \ref{lemma-relative-pseudo-coherence}
and the definition of relative pseudo-coherence
there exists a bounded above complex $F^\bullet$
of finite free $P$-modules representing
$Ri_*K$ where $i : U \to \Spec(P)$ is the closed
immersion induced by the presentation.
Let $M_n$ be the $B$-module corresponding to $\mathcal{F}_n|_U$.
By Lemma \ref{lemma-compute-over-affine}
$$
H^i(F^\bullet \otimes_A A_n) =
\left\{
\begin{matrix}
0 & \text{if} & i \not = 0 \\
M_n & \text{if} & i = 0
\end{matrix}
\right.
$$
Let $i$ be the maximal index such that $F^i$ is nonzero.
If $i \leq 0$, then (1), (2), and (3) are true.
If not, then $i > 0$ and we see that the rank of the map
$$
F^{i - 1} \to F^i
$$
in the point $u$ is maximal. Hence in an open neighbourhood
of $u$ inside $\Spec(P)$ the rank is maximal. Thus after replacing
$P$ by a principal localization we may assume that the displayed
map is surjective. Since $F^i$ is finite free we may choose
a splitting $F^{i - 1} = F' \oplus F^i$. Then we may
replace $F^\bullet$ by the complex
$$
\ldots \to F^{i - 2} \to F' \to 0 \to \ldots
$$
and we win by induction on $i$.
\end{proof}

\begin{lemma}
\label{lemma-proper-support}
In Situation \ref{situation-existence} let $K$ be as in
Lemma \ref{lemma-compute-what-it-should-be}. Let $W \subset X$
be as in Lemma \ref{lemma-finitely-presented}.
Set $\mathcal{F} = H^0(K)|_W$. Then, after possibly shrinking the open $W$,
the support of $\mathcal{F}$ is proper over $A$.
\end{lemma}

\begin{proof}
Fix $n \geq 1$. Let $I_n = \Ker(A \to A_n)$.
By More on Algebra, Lemma \ref{more-algebra-lemma-limit-henselian}
the pair $(A, I_n)$ is henselian.
Let $Z \subset W$ be the scheme theoretic support of $\mathcal{F}$.
This is a closed subspace as $\mathcal{F}$ is of finite presentation.
By part (3) of Lemma \ref{lemma-finitely-presented}
we see that $Z \times_{\Spec(A)} \Spec(A_n)$
is equal to the support of $\mathcal{F}_n$ and hence
proper over $\Spec(A/I)$.
By More on Morphisms of Spaces, Lemma
\ref{spaces-more-morphisms-lemma-split-off-proper-part-henselian}
we can write $Z = Z_1 \amalg Z_2$ with $Z_1, Z_2$ open and
closed in $Z$, with $Z_1$ proper
over $A$, and with $Z_1 \times_{\Spec(A)} \Spec(A/I_n)$
equal to the support of $\mathcal{F}_n$.
In other words, $|Z_2|$ does not meet $X_0$.
Hence after replacing $W$ by $W \setminus Z_2$ we obtain the lemma.
\end{proof}

\begin{theorem}[Grothendieck Existence Theorem]
\label{theorem-existence}
In Situation \ref{situation-existence}
there exists a finitely presented $\mathcal{O}_X$-module
$\mathcal{F}$, flat over $A$, with support proper over $A$,
such that
$\mathcal{F}_n = \mathcal{F} \otimes_{\mathcal{O}_X} \mathcal{O}_{X_n}$
for all $n$ compatibly with the maps $\varphi_n$.
\end{theorem}

\begin{proof}
Apply Lemmas \ref{lemma-compute-what-it-should-be},
\ref{lemma-compute-against-perfect},
\ref{lemma-relative-pseudo-coherence},
\ref{lemma-compute-over-affine},
\ref{lemma-finitely-presented}, and
\ref{lemma-proper-support}
to get an open subspace $W \subset X$ containing all points
lying over $\Spec(A_n)$
and a finitely presented $\mathcal{O}_W$-module $\mathcal{F}$
whose support is proper over $A$ with
$\mathcal{F}_n = \mathcal{F} \otimes_{\mathcal{O}_W} \mathcal{O}_{X_n}$
for all $n \geq 1$. (This makes sense as $X_n \subset W$.)
By Lemma \ref{lemma-proper-pure} we see that $\mathcal{F}$
is universally pure relative to $\Spec(A)$.
By Theorem \ref{theorem-flat-dimension-n-representable}
(for explanation, see Lemma \ref{lemma-when-universal-flattening})
there exists a universal flattening $S' \to \Spec(A)$
of $\mathcal{F}$ and moreover the morphism $S' \to \Spec(A)$
is a monomorphism of finite presentation.
In particular $S'$ is a scheme (this follows from the proof
of the theorem but it also follows a postoriori by
Morphisms of Spaces, Proposition
\ref{spaces-morphisms-proposition-locally-quasi-finite-separated-over-scheme}).
Since the base change of $\mathcal{F}$ to $\Spec(A_n)$
is $\mathcal{F}_n$ we find that $\Spec(A_n) \to \Spec(A)$
factors (uniquely) through $S'$ for each $n$.
By More on Flatness, Lemma \ref{flat-lemma-monomorphism-isomorphism}
we see that $S' = \Spec(A)$.
This means that $\mathcal{F}$ is flat over $A$.
Finally, since the scheme theoretic support $Z$ of $\mathcal{F}$
is proper over $\Spec(A)$, the morphism $Z \to X$ is closed.
Hence the pushforward $(W \to X)_*\mathcal{F}$ is supported
on $W$ and has all the desired properties.
\end{proof}







\section{Grothendieck's Existence Theorem, bis}
\label{section-existence-derived}

\noindent
In this section we prove an analogue for Grothendieck's existence theorem
in the derived category, following the method used in
Section \ref{section-existence} for quasi-coherent modules.
This section is the analogue of
More on Flatness, Section \ref{flat-section-existence-derived}
for algebraic spaces. The classical case (for algebraic spaces)
is discussed in More on Morphisms of Spaces, Section
\ref{spaces-more-morphisms-section-existence-proper}.
We will work in the following situation.

\begin{situation}
\label{situation-existence-derived}
Here we have an inverse system of rings $(A_n)$ with surjective transition
maps whose kernels are locally nilpotent. Set $A = \lim A_n$. We have an
algebraic space $X$ proper, flat, and of finite presentation over $A$.
We set $X_n = X \times_{\Spec(A)} \Spec(A_n)$ and we
view it as a closed subspace of $X$. We assume further given a system
$(K_n, \varphi_n)$ where $K_n$ is a pseudo-coherent object of
$D(\mathcal{O}_{X_n})$ and
$$
\varphi_n : K_n \longrightarrow K_{n - 1}
$$
is a map in $D(\mathcal{O}_{X_n})$ which induces an isomorphism
$K_n \otimes_{\mathcal{O}_{X_n}}^\mathbf{L} \mathcal{O}_{X_{n - 1}}
\to K_{n - 1}$ in $D(\mathcal{O}_{X_{n - 1}})$.
\end{situation}

\noindent
More precisely, we should write
$\varphi_n : K_n \to Ri_{n - 1, *}K_{n - 1}$
where $i_{n - 1} : X_{n - 1} \to X_n$ is the inclusion morphism
and in this notation the condition is that the adjoint
map $Li_{n - 1}^*K_n \to K_{n - 1}$ is an isomorphism.
Our goal is to find a pseudo-coherent $K \in D(\mathcal{O}_X)$
such that $K_n = K \otimes_{\mathcal{O}_X}^\mathbf{L} \mathcal{O}_{X_n}$
for all $n$ (with the same abuse of notation).

\begin{lemma}
\label{lemma-compute-what-it-should-be-derived}
In Situation \ref{situation-existence-derived} consider
$$
K = R\lim_{D_\QCoh(\mathcal{O}_X)}(K_n) =
DQ_X(R\lim_{D(\mathcal{O}_X)} K_n)
$$
Then $K$ is in $D^-_{\QCoh}(\mathcal{O}_X)$.
\end{lemma}

\begin{proof}
The functor $DQ_X$ exists because $X$ is quasi-compact and
quasi-separated, see Derived Categories of Spaces, Lemma
\ref{spaces-perfect-lemma-better-coherator}.
Since $DQ_X$ is a right adjoint it commutes with products
and therefore with derived limits. Hence the equality
in the statement of the lemma.

\medskip\noindent
By Derived Categories of Spaces,
Lemma \ref{spaces-perfect-lemma-boundedness-better-coherator}
the functor $DQ_X$ has bounded cohomological dimension.
Hence it suffices to show that $R\lim K_n \in D^-(\mathcal{O}_X)$.
To see this, let $U \to X$ be \'etale with $U$ affine.
Then there is a canonical exact sequence
$$
0 \to
R^1\lim H^{m - 1}(U, K_n) \to H^m(U, R\lim K_n) \to
\lim H^m(U, K_n) \to 0
$$
by Cohomology on Sites, Lemma
\ref{sites-cohomology-lemma-RGamma-commutes-with-Rlim}.
Since $U$ is affine and $K_n$ is pseudo-coherent (and hence has
quasi-coherent cohomology sheaves by
Derived Categories of Spaces, Lemma \ref{spaces-perfect-lemma-pseudo-coherent})
we see that $H^m(U, K_n) = H^m(K_n)(U)$ by
Derived Categories of Schemes, Lemma \ref{perfect-lemma-affine-compare-bounded}.
Thus we conclude that it suffices to show that $K_n$
is bounded above independent of $n$.

\medskip\noindent
Since $K_n$ is pseudo-coherent we have $K_n \in D^-(\mathcal{O}_{X_n})$.
Suppose that $a_n$ is maximal such that $H^{a_n}(K_n)$ is nonzero.
Of course $a_1 \leq a_2 \leq a_3 \leq \ldots$.
Note that $H^{a_n}(K_n)$ is an
$\mathcal{O}_{X_n}$-module of finite presentation
(Cohomology on Sites, Lemma
\ref{sites-cohomology-lemma-finite-cohomology}).
We have $H^{a_n}(K_{n - 1}) =
H^{a_n}(K_n) \otimes_{\mathcal{O}_{X_n}} \mathcal{O}_{X_{n - 1}}$.
Since $X_{n - 1} \to X_n$ is a thickening, it follows from
Nakayama's lemma (Algebra, Lemma \ref{algebra-lemma-NAK}) that if
$H^{a_n}(K_n) \otimes_{\mathcal{O}_{X_n}} \mathcal{O}_{X_{n - 1}}$
is zero, then $H^{a_n}(K_n)$ is zero too (argue by checking on stalks
for example; small detail omitted).
Thus $a_{n - 1} = a_n$ for all $n$ and we conclude.
\end{proof}

\begin{lemma}
\label{lemma-compute-against-perfect-derived}
In Situation \ref{situation-existence-derived} let $K$ be as in
Lemma \ref{lemma-compute-what-it-should-be-derived}. For any perfect
object $E$ of $D(\mathcal{O}_X)$ the cohomology
$$
M = R\Gamma(X, K \otimes^\mathbf{L} E)
$$
is a pseudo-coherent object of $D(A)$ and there is a canonical isomorphism
$$
R\Gamma(X_n, K_n \otimes^\mathbf{L} E|_{X_n}) = M \otimes_A^\mathbf{L} A_n
$$
in $D(A_n)$. Here $E|_{X_n}$ denotes the derived pullback of $E$ to $X_n$.
\end{lemma}

\begin{proof}
Write $E_n = E|_{X_n}$ and
$M_n = R\Gamma(X_n, K_n \otimes^\mathbf{L} E|_{X_n})$.
By Derived Categories of Spaces, Lemma
\ref{spaces-perfect-lemma-flat-proper-pseudo-coherent-direct-image-general}
we see that $M_n$ is a pseudo-coherent object of $D(A_n)$
whose formation commutes with base change. Thus the maps
$M_n \otimes_{A_n}^\mathbf{L} A_{n - 1} \to M_{n - 1}$
coming from $\varphi_n$ are isomorphisms. By
More on Algebra, Lemma
\ref{more-algebra-lemma-Rlim-pseudo-coherent-gives-pseudo-coherent}
we find that $R\lim M_n$ is pseudo-coherent and
that its base change back to $A_n$ recovers $M_n$.
On the other hand, the exact functor
$R\Gamma(X, -) : D_\QCoh(\mathcal{O}_X) \to D(A)$
of triangulated categories commutes with products
and hence with derived limits, whence
$$
R\Gamma(X, E \otimes^\mathbf{L} K) =
R\lim R\Gamma(X,  E \otimes^\mathbf{L} K_n) =
R\lim R\Gamma(X_n, E_n \otimes^\mathbf{L} K_n) =
R\lim M_n
$$
as desired.
\end{proof}

\begin{lemma}
\label{lemma-relative-pseudo-coherence-derived}
In Situation \ref{situation-existence-derived} let $K$ be as in
Lemma \ref{lemma-compute-what-it-should-be-derived}. Then $K$
is pseudo-coherent on $X$.
\end{lemma}

\begin{proof}
Combinging Lemma \ref{lemma-compute-against-perfect-derived} and
Derived Categories of Spaces, Lemma
\ref{spaces-perfect-lemma-perfect-enough}
we see that $R\Gamma(X, K \otimes^\mathbf{L} E)$
is pseudo-coherent in $D(A)$ for all pseudo-coherent
$E$ in $D(\mathcal{O}_X)$. Thus it follows from
More on Morphisms of Spaces, Lemma
\ref{spaces-more-morphisms-lemma-characterize-pseudo-coherent}
that $K$ is pseudo-coherent relative to $A$.
Since $X$ is of flat and of finite presentation
over $A$, this is the same as being pseudo-coherent on $X$, see
More on Morphisms of Spaces, Lemma
\ref{spaces-more-morphisms-lemma-relative-pseudo-coherent-is-moot}.
\end{proof}

\begin{lemma}
\label{lemma-compute-over-affine-derived}
In Situation \ref{situation-existence-derived} let $K$ be as in
Lemma \ref{lemma-compute-what-it-should-be-derived}. For any
\'etale morphism $U \to X$ with $U$ quasi-compact and quasi-separated we have
$$
R\Gamma(U, K) \otimes_A^\mathbf{L} A_n =
R\Gamma(U_n, K_n)
$$
in $D(A_n)$ where $U_n = U \times_X X_n$.
\end{lemma}

\begin{proof}
Fix $n$. By Derived Categories of Spaces, Lemma
\ref{spaces-perfect-lemma-computing-sections-as-colim}
there exists a system of perfect complexes $E_m$
on $X$ such that
$R\Gamma(U, K) = \text{hocolim} R\Gamma(X, K \otimes^\mathbf{L} E_m)$.
In fact, this formula holds not just for $K$ but for every object of
$D_\QCoh(\mathcal{O}_X)$.
Applying this to $K_n$
we obtain
\begin{align*}
R\Gamma(U_n, K_n)
& =
R\Gamma(U, K_n) \\
& =
\text{hocolim}_m R\Gamma(X, K_n \otimes^\mathbf{L} E_m) \\
& =
\text{hocolim}_m R\Gamma(X_n, K_n \otimes^\mathbf{L} E_m|_{X_n})
\end{align*}
Using Lemma \ref{lemma-compute-against-perfect-derived}
and the fact that $- \otimes_A^\mathbf{L} A_n$
commutes with homotopy colimits we obtain the result.
\end{proof}

\begin{theorem}[Derived Grothendieck Existence Theorem]
\label{theorem-existence-derived}
In Situation \ref{situation-existence-derived}
there exists a pseudo-coherent $K$ in $D(\mathcal{O}_X)$
such that $K_n = K \otimes_{\mathcal{O}_X}^\mathbf{L} \mathcal{O}_{X_n}$
for all $n$ compatibly with the maps $\varphi_n$.
\end{theorem}

\begin{proof}
Apply Lemmas \ref{lemma-compute-what-it-should-be-derived},
\ref{lemma-compute-against-perfect-derived},
\ref{lemma-relative-pseudo-coherence-derived}
to get a pseudo-coherent object $K$ of $D(\mathcal{O}_X)$.
Choosing affine $U$ in Lemma
\ref{lemma-compute-over-affine-derived}
it follows immediately that $K$ restricts to $K_n$ over $X_n$.
\end{proof}

\begin{remark}
\label{remark-correct-generality}
The result in this section can be generalized. It is probably correct
if we only assume $X \to \Spec(A)$ to be separated, of finite presentation,
and $K_n$ pseudo-coherent relative to $A_n$ supported on a closed
subset of $X_n$ proper over $A_n$. The outcome will be a $K$ which
is pseudo-coherent relative to $A$ supported on a closed subset
proper over $A$. If we ever need this, we will
formulate a precise statement and prove it here.
\end{remark}










\begin{multicols}{2}[\section{Other chapters}]
\noindent
Preliminaries
\begin{enumerate}
\item \hyperref[introduction-section-phantom]{Introduction}
\item \hyperref[conventions-section-phantom]{Conventions}
\item \hyperref[sets-section-phantom]{Set Theory}
\item \hyperref[categories-section-phantom]{Categories}
\item \hyperref[topology-section-phantom]{Topology}
\item \hyperref[sheaves-section-phantom]{Sheaves on Spaces}
\item \hyperref[sites-section-phantom]{Sites and Sheaves}
\item \hyperref[stacks-section-phantom]{Stacks}
\item \hyperref[fields-section-phantom]{Fields}
\item \hyperref[algebra-section-phantom]{Commutative Algebra}
\item \hyperref[brauer-section-phantom]{Brauer Groups}
\item \hyperref[homology-section-phantom]{Homological Algebra}
\item \hyperref[derived-section-phantom]{Derived Categories}
\item \hyperref[simplicial-section-phantom]{Simplicial Methods}
\item \hyperref[more-algebra-section-phantom]{More on Algebra}
\item \hyperref[smoothing-section-phantom]{Smoothing Ring Maps}
\item \hyperref[modules-section-phantom]{Sheaves of Modules}
\item \hyperref[sites-modules-section-phantom]{Modules on Sites}
\item \hyperref[injectives-section-phantom]{Injectives}
\item \hyperref[cohomology-section-phantom]{Cohomology of Sheaves}
\item \hyperref[sites-cohomology-section-phantom]{Cohomology on Sites}
\item \hyperref[dga-section-phantom]{Differential Graded Algebra}
\item \hyperref[dpa-section-phantom]{Divided Power Algebra}
\item \hyperref[hypercovering-section-phantom]{Hypercoverings}
\end{enumerate}
Schemes
\begin{enumerate}
\setcounter{enumi}{24}
\item \hyperref[schemes-section-phantom]{Schemes}
\item \hyperref[constructions-section-phantom]{Constructions of Schemes}
\item \hyperref[properties-section-phantom]{Properties of Schemes}
\item \hyperref[morphisms-section-phantom]{Morphisms of Schemes}
\item \hyperref[coherent-section-phantom]{Cohomology of Schemes}
\item \hyperref[divisors-section-phantom]{Divisors}
\item \hyperref[limits-section-phantom]{Limits of Schemes}
\item \hyperref[varieties-section-phantom]{Varieties}
\item \hyperref[topologies-section-phantom]{Topologies on Schemes}
\item \hyperref[descent-section-phantom]{Descent}
\item \hyperref[perfect-section-phantom]{Derived Categories of Schemes}
\item \hyperref[more-morphisms-section-phantom]{More on Morphisms}
\item \hyperref[flat-section-phantom]{More on Flatness}
\item \hyperref[groupoids-section-phantom]{Groupoid Schemes}
\item \hyperref[more-groupoids-section-phantom]{More on Groupoid Schemes}
\item \hyperref[etale-section-phantom]{\'Etale Morphisms of Schemes}
\end{enumerate}
Topics in Scheme Theory
\begin{enumerate}
\setcounter{enumi}{40}
\item \hyperref[chow-section-phantom]{Chow Homology}
\item \hyperref[intersection-section-phantom]{Intersection Theory}
\item \hyperref[weil-section-phantom]{Weil Cohomology Theories}
\item \hyperref[pic-section-phantom]{Picard Schemes of Curves}
\item \hyperref[adequate-section-phantom]{Adequate Modules}
\item \hyperref[dualizing-section-phantom]{Dualizing Complexes}
\item \hyperref[duality-section-phantom]{Duality for Schemes}
\item \hyperref[discriminant-section-phantom]{Discriminants and Differents}
\item \hyperref[local-cohomology-section-phantom]{Local Cohomology}
\item \hyperref[algebraization-section-phantom]{Algebraic and Formal Geometry}
\item \hyperref[curves-section-phantom]{Algebraic Curves}
\item \hyperref[resolve-section-phantom]{Resolution of Surfaces}
\item \hyperref[models-section-phantom]{Semistable Reduction}
\item \hyperref[pione-section-phantom]{Fundamental Groups of Schemes}
\item \hyperref[etale-cohomology-section-phantom]{\'Etale Cohomology}
\item \hyperref[crystalline-section-phantom]{Crystalline Cohomology}
\item \hyperref[proetale-section-phantom]{Pro-\'etale Cohomology}
\item \hyperref[more-etale-section-phantom]{More \'Etale Cohomology}
\item \hyperref[trace-section-phantom]{The Trace Formula}
\end{enumerate}
Algebraic Spaces
\begin{enumerate}
\setcounter{enumi}{59}
\item \hyperref[spaces-section-phantom]{Algebraic Spaces}
\item \hyperref[spaces-properties-section-phantom]{Properties of Algebraic Spaces}
\item \hyperref[spaces-morphisms-section-phantom]{Morphisms of Algebraic Spaces}
\item \hyperref[decent-spaces-section-phantom]{Decent Algebraic Spaces}
\item \hyperref[spaces-cohomology-section-phantom]{Cohomology of Algebraic Spaces}
\item \hyperref[spaces-limits-section-phantom]{Limits of Algebraic Spaces}
\item \hyperref[spaces-divisors-section-phantom]{Divisors on Algebraic Spaces}
\item \hyperref[spaces-over-fields-section-phantom]{Algebraic Spaces over Fields}
\item \hyperref[spaces-topologies-section-phantom]{Topologies on Algebraic Spaces}
\item \hyperref[spaces-descent-section-phantom]{Descent and Algebraic Spaces}
\item \hyperref[spaces-perfect-section-phantom]{Derived Categories of Spaces}
\item \hyperref[spaces-more-morphisms-section-phantom]{More on Morphisms of Spaces}
\item \hyperref[spaces-flat-section-phantom]{Flatness on Algebraic Spaces}
\item \hyperref[spaces-groupoids-section-phantom]{Groupoids in Algebraic Spaces}
\item \hyperref[spaces-more-groupoids-section-phantom]{More on Groupoids in Spaces}
\item \hyperref[bootstrap-section-phantom]{Bootstrap}
\item \hyperref[spaces-pushouts-section-phantom]{Pushouts of Algebraic Spaces}
\end{enumerate}
Topics in Geometry
\begin{enumerate}
\setcounter{enumi}{76}
\item \hyperref[spaces-chow-section-phantom]{Chow Groups of Spaces}
\item \hyperref[groupoids-quotients-section-phantom]{Quotients of Groupoids}
\item \hyperref[spaces-more-cohomology-section-phantom]{More on Cohomology of Spaces}
\item \hyperref[spaces-simplicial-section-phantom]{Simplicial Spaces}
\item \hyperref[spaces-duality-section-phantom]{Duality for Spaces}
\item \hyperref[formal-spaces-section-phantom]{Formal Algebraic Spaces}
\item \hyperref[restricted-section-phantom]{Restricted Power Series}
\item \hyperref[spaces-resolve-section-phantom]{Resolution of Surfaces Revisited}
\end{enumerate}
Deformation Theory
\begin{enumerate}
\setcounter{enumi}{84}
\item \hyperref[formal-defos-section-phantom]{Formal Deformation Theory}
\item \hyperref[defos-section-phantom]{Deformation Theory}
\item \hyperref[cotangent-section-phantom]{The Cotangent Complex}
\item \hyperref[examples-defos-section-phantom]{Deformation Problems}
\end{enumerate}
Algebraic Stacks
\begin{enumerate}
\setcounter{enumi}{88}
\item \hyperref[algebraic-section-phantom]{Algebraic Stacks}
\item \hyperref[examples-stacks-section-phantom]{Examples of Stacks}
\item \hyperref[stacks-sheaves-section-phantom]{Sheaves on Algebraic Stacks}
\item \hyperref[criteria-section-phantom]{Criteria for Representability}
\item \hyperref[artin-section-phantom]{Artin's Axioms}
\item \hyperref[quot-section-phantom]{Quot and Hilbert Spaces}
\item \hyperref[stacks-properties-section-phantom]{Properties of Algebraic Stacks}
\item \hyperref[stacks-morphisms-section-phantom]{Morphisms of Algebraic Stacks}
\item \hyperref[stacks-limits-section-phantom]{Limits of Algebraic Stacks}
\item \hyperref[stacks-cohomology-section-phantom]{Cohomology of Algebraic Stacks}
\item \hyperref[stacks-perfect-section-phantom]{Derived Categories of Stacks}
\item \hyperref[stacks-introduction-section-phantom]{Introducing Algebraic Stacks}
\item \hyperref[stacks-more-morphisms-section-phantom]{More on Morphisms of Stacks}
\item \hyperref[stacks-geometry-section-phantom]{The Geometry of Stacks}
\end{enumerate}
Topics in Moduli Theory
\begin{enumerate}
\setcounter{enumi}{102}
\item \hyperref[moduli-section-phantom]{Moduli Stacks}
\item \hyperref[moduli-curves-section-phantom]{Moduli of Curves}
\end{enumerate}
Miscellany
\begin{enumerate}
\setcounter{enumi}{104}
\item \hyperref[examples-section-phantom]{Examples}
\item \hyperref[exercises-section-phantom]{Exercises}
\item \hyperref[guide-section-phantom]{Guide to Literature}
\item \hyperref[desirables-section-phantom]{Desirables}
\item \hyperref[coding-section-phantom]{Coding Style}
\item \hyperref[obsolete-section-phantom]{Obsolete}
\item \hyperref[fdl-section-phantom]{GNU Free Documentation License}
\item \hyperref[index-section-phantom]{Auto Generated Index}
\end{enumerate}
\end{multicols}



\bibliography{my}
\bibliographystyle{amsalpha}

\end{document}
