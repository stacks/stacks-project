\documentclass{amsart}

% The following AMS packages are automatically loaded with amsart 
% documentclass:
%\usepackage{amsmath}
%\usepackage{amssymb}
%\usepackage{amsthm}


% For commutative diagrams you can use
% \usepackage{amscd}
% but Jason prefers xypic
% \usepackage[all]{xy}

% To put source file link in headers.
% Change "introduction.tex" to "this_filename.tex"
\usepackage{fancyhdr}
\pagestyle{fancy}
\lhead{}
\chead{}
\rhead{Source file: \url{src/introduction.tex}}
\lfoot{}
\cfoot{\thepage}
\rfoot{}
\renewcommand{\headrulewidth}{0pt}
\renewcommand{\footrulewidth}{0pt}
\renewcommand{\headheight}{12pt}

% For cross-file-references
\usepackage{xr-hyper}

% Package for hypertext links:
\usepackage[colorlinks=true]{hyperref}
% For any local file, say "hello.tex" you want to refer to please use
% \externaldocument[hello-]{hello}
\externaldocument[conventions-]{conventions}

% The macro \autoref uses the macros \figurename, etc.
% We list the default values and we change some of them
% to start with a captial.
% Figure	\figurename
% Table		\tablename
% Part		\partname
% Appendix	\appendixname
% Equation	\equationname
% item		\Itemname
% \renewcommand{\Itemname}{Item}
\renewcommand{\Itemautorefname}{Item}
% chapter	\Chaptername
% \renewcommand{\Chaptername}{Chapter}
% \renewcommand{\Chapterautorefname}{Chapter}
% section	\sectionname
\renewcommand{\sectionname}{Section}
\renewcommand{\sectionautorefname}{Section}
% subsection	\subsectionname
\renewcommand{\subsectionname}{Subsection}
\renewcommand{\subsectionautorefname}{Subsection}
% subsubsection	\subsubsectionname
\renewcommand{\subsubsectionname}{Subsubsection}
\renewcommand{\subsubsectionautorefname}{Subsubsection}
% paragraph	\paragraphname
\renewcommand{\paragraphname}{Paragraph}
\renewcommand{\paragraphautorefname}{Paragraph}
% footnote	\Hfootnotename
% \renewcommand{\Hfootnotename}{Footnote}
\renewcommand{\Hfootnoteautorefname}{Footnote}
% Equation	\AMSname
% Theorem	\theoremname


% Theorem environments.
%
\newtheorem{theorem}{Theorem}[subsection]
\newtheorem{proposition}[theorem]{Proposition}
\newtheorem{lemma}[theorem]{Lemma}

\theoremstyle{definition}
\newtheorem{definition}[theorem]{Definition}
\newtheorem{example}[theorem]{Example}
\newtheorem{exercise}[theorem]{Exercise}
\newtheorem{situation}[theorem]{Situation}

\theoremstyle{remark}
\newtheorem{remark}[theorem]{Remark}
\newtheorem{remarks}[theorem]{Remarks}

\numberwithin{equation}{subsection}


% OK, start here.
%
\begin{document}

\title{Introduction to the Algebraic Stacks Documents}

%\begin{abstract}
%\end{abstract}

\maketitle
\thispagestyle{fancy}

\begin{verbatim}
Copyright (c) 2005 Johan de Jong and the algebraic_geometry mailing list.
Permission is granted to copy, distribute and/or modify this document
under the terms of the GNU Free Documentation License, Version 1.2
or any later version published by the Free Software Foundation;
with no Invariant Sections, no Front-Cover Texts, and no Back-Cover Texts.
A copy of the license is included in the section entitled "GNU
Free Documentation License".
\end{verbatim}

\tableofcontents

\section{Overview}
\label{section-overview}

\noindent
Besides the book by Laumon and Moret-Bailley, see \cite{LM-B}, and the work
(in progress) by Fulton et al, we think there is a place for an open source
textbook on algebraic stacks.

\smallskip\noindent
To read about this project and its goals please visit the
\href{http://www.math.columbia.edu/mailman/listinfo/algebraic_geometry}%
{mailing list}. This is also the place to submit patches, etc.

\smallskip\noindent
To continue reading, 
\begin{enumerate}

\item visit the next section: Conventions,
\autoref{conventions-section-comments}, or 

\item go back to the
table of contents: \url{index.html#contents}.

\end{enumerate}

\bibliography{my}
\bibliographystyle{alpha}

\end{document}
