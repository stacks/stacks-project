\IfFileExists{stacks-project.cls}{%
\documentclass{stacks-project}
}{%
\documentclass{amsart}
}

% The following AMS packages are automatically loaded with
% the amsart documentclass:
%\usepackage{amsmath}
%\usepackage{amssymb}
%\usepackage{amsthm}

% For dealing with references we use the comment environment
\usepackage{verbatim}
\newenvironment{reference}{\comment}{\endcomment}
%\newenvironment{reference}{}{}
\newenvironment{slogan}{\comment}{\endcomment}
\newenvironment{history}{\comment}{\endcomment}

% For commutative diagrams you can use
% \usepackage{amscd}
\usepackage[all]{xy}

% We use 2cell for 2-commutative diagrams.
\xyoption{2cell}
\UseAllTwocells

% To put source file link in headers.
% Change "template.tex" to "this_filename.tex"
% \usepackage{fancyhdr}
% \pagestyle{fancy}
% \lhead{}
% \chead{}
% \rhead{Source file: \url{template.tex}}
% \lfoot{}
% \cfoot{\thepage}
% \rfoot{}
% \renewcommand{\headrulewidth}{0pt}
% \renewcommand{\footrulewidth}{0pt}
% \renewcommand{\headheight}{12pt}

\usepackage{multicol}

% For cross-file-references
\usepackage{xr-hyper}

% Package for hypertext links:
\usepackage{hyperref}

% For any local file, say "hello.tex" you want to link to please
% use \externaldocument[hello-]{hello}
\externaldocument[introduction-]{introduction}
\externaldocument[conventions-]{conventions}
\externaldocument[sets-]{sets}
\externaldocument[categories-]{categories}
\externaldocument[topology-]{topology}
\externaldocument[sheaves-]{sheaves}
\externaldocument[sites-]{sites}
\externaldocument[stacks-]{stacks}
\externaldocument[fields-]{fields}
\externaldocument[algebra-]{algebra}
\externaldocument[brauer-]{brauer}
\externaldocument[homology-]{homology}
\externaldocument[derived-]{derived}
\externaldocument[simplicial-]{simplicial}
\externaldocument[more-algebra-]{more-algebra}
\externaldocument[smoothing-]{smoothing}
\externaldocument[modules-]{modules}
\externaldocument[sites-modules-]{sites-modules}
\externaldocument[injectives-]{injectives}
\externaldocument[cohomology-]{cohomology}
\externaldocument[sites-cohomology-]{sites-cohomology}
\externaldocument[dga-]{dga}
\externaldocument[dpa-]{dpa}
\externaldocument[hypercovering-]{hypercovering}
\externaldocument[schemes-]{schemes}
\externaldocument[constructions-]{constructions}
\externaldocument[properties-]{properties}
\externaldocument[morphisms-]{morphisms}
\externaldocument[coherent-]{coherent}
\externaldocument[divisors-]{divisors}
\externaldocument[limits-]{limits}
\externaldocument[varieties-]{varieties}
\externaldocument[topologies-]{topologies}
\externaldocument[descent-]{descent}
\externaldocument[perfect-]{perfect}
\externaldocument[more-morphisms-]{more-morphisms}
\externaldocument[flat-]{flat}
\externaldocument[groupoids-]{groupoids}
\externaldocument[more-groupoids-]{more-groupoids}
\externaldocument[etale-]{etale}
\externaldocument[chow-]{chow}
\externaldocument[intersection-]{intersection}
\externaldocument[pic-]{pic}
\externaldocument[adequate-]{adequate}
\externaldocument[dualizing-]{dualizing}
\externaldocument[duality-]{duality}
\externaldocument[discriminant-]{discriminant}
\externaldocument[local-cohomology-]{local-cohomology}
\externaldocument[curves-]{curves}
\externaldocument[resolve-]{resolve}
\externaldocument[models-]{models}
\externaldocument[pione-]{pione}
\externaldocument[etale-cohomology-]{etale-cohomology}
\externaldocument[proetale-]{proetale}
\externaldocument[crystalline-]{crystalline}
\externaldocument[spaces-]{spaces}
\externaldocument[spaces-properties-]{spaces-properties}
\externaldocument[spaces-morphisms-]{spaces-morphisms}
\externaldocument[decent-spaces-]{decent-spaces}
\externaldocument[spaces-cohomology-]{spaces-cohomology}
\externaldocument[spaces-limits-]{spaces-limits}
\externaldocument[spaces-divisors-]{spaces-divisors}
\externaldocument[spaces-over-fields-]{spaces-over-fields}
\externaldocument[spaces-topologies-]{spaces-topologies}
\externaldocument[spaces-descent-]{spaces-descent}
\externaldocument[spaces-perfect-]{spaces-perfect}
\externaldocument[spaces-more-morphisms-]{spaces-more-morphisms}
\externaldocument[spaces-flat-]{spaces-flat}
\externaldocument[spaces-groupoids-]{spaces-groupoids}
\externaldocument[spaces-more-groupoids-]{spaces-more-groupoids}
\externaldocument[bootstrap-]{bootstrap}
\externaldocument[spaces-pushouts-]{spaces-pushouts}
\externaldocument[groupoids-quotients-]{groupoids-quotients}
\externaldocument[spaces-more-cohomology-]{spaces-more-cohomology}
\externaldocument[spaces-simplicial-]{spaces-simplicial}
\externaldocument[formal-spaces-]{formal-spaces}
\externaldocument[restricted-]{restricted}
\externaldocument[spaces-resolve-]{spaces-resolve}
\externaldocument[formal-defos-]{formal-defos}
\externaldocument[defos-]{defos}
\externaldocument[cotangent-]{cotangent}
\externaldocument[examples-defos-]{examples-defos}
\externaldocument[algebraic-]{algebraic}
\externaldocument[examples-stacks-]{examples-stacks}
\externaldocument[stacks-sheaves-]{stacks-sheaves}
\externaldocument[criteria-]{criteria}
\externaldocument[artin-]{artin}
\externaldocument[quot-]{quot}
\externaldocument[stacks-properties-]{stacks-properties}
\externaldocument[stacks-morphisms-]{stacks-morphisms}
\externaldocument[stacks-limits-]{stacks-limits}
\externaldocument[stacks-cohomology-]{stacks-cohomology}
\externaldocument[stacks-perfect-]{stacks-perfect}
\externaldocument[stacks-introduction-]{stacks-introduction}
\externaldocument[stacks-more-morphisms-]{stacks-more-morphisms}
\externaldocument[stacks-geometry-]{stacks-geometry}
\externaldocument[moduli-]{moduli}
\externaldocument[moduli-curves-]{moduli-curves}
\externaldocument[examples-]{examples}
\externaldocument[exercises-]{exercises}
\externaldocument[guide-]{guide}
\externaldocument[desirables-]{desirables}
\externaldocument[coding-]{coding}
\externaldocument[obsolete-]{obsolete}
\externaldocument[fdl-]{fdl}
\externaldocument[index-]{index}

% Theorem environments.
%
\theoremstyle{plain}
\newtheorem{theorem}[subsection]{Theorem}
\newtheorem{proposition}[subsection]{Proposition}
\newtheorem{lemma}[subsection]{Lemma}

\theoremstyle{definition}
\newtheorem{definition}[subsection]{Definition}
\newtheorem{example}[subsection]{Example}
\newtheorem{exercise}[subsection]{Exercise}
\newtheorem{situation}[subsection]{Situation}

\theoremstyle{remark}
\newtheorem{remark}[subsection]{Remark}
\newtheorem{remarks}[subsection]{Remarks}

\numberwithin{equation}{subsection}

% Macros
%
\def\lim{\mathop{\rm lim}\nolimits}
\def\colim{\mathop{\rm colim}\nolimits}
\def\Spec{\mathop{\rm Spec}}
\def\Hom{\mathop{\rm Hom}\nolimits}
\def\Ext{\mathop{\rm Ext}\nolimits}
\def\SheafHom{\mathop{\mathcal{H}\!{\it om}}\nolimits}
\def\SheafExt{\mathop{\mathcal{E}\!{\it xt}}\nolimits}
\def\Sch{\textit{Sch}}
\def\Mor{\mathop{\rm Mor}\nolimits}
\def\Ob{\mathop{\rm Ob}\nolimits}
\def\Sh{\mathop{\textit{Sh}}\nolimits}
\def\NL{\mathop{N\!L}\nolimits}
\def\proetale{{pro\text{-}\acute{e}tale}}
\def\etale{{\acute{e}tale}}
\def\QCoh{\textit{QCoh}}
\def\Ker{\mathop{\rm Ker}}
\def\Im{\mathop{\rm Im}}
\def\Coker{\mathop{\rm Coker}}
\def\Coim{\mathop{\rm Coim}}

%
% Macros for moduli stacks/spaces
%
\def\QCohstack{\mathcal{QC}\!{\it oh}}
\def\Cohstack{\mathcal{C}\!{\it oh}}
\def\Spacesstack{\mathcal{S}\!{\it paces}}
\def\Quotfunctor{{\rm Quot}}
\def\Hilbfunctor{{\rm Hilb}}
\def\Curvesstack{\mathcal{C}\!{\it urves}}
\def\Polarizedstack{\mathcal{P}\!{\it olarized}}
\def\Complexesstack{\mathcal{C}\!{\it omplexes}}
% \Pic is the operator that assigns to X its picard group, usage \Pic(X)
% \Picardstack_{X/B} denotes the Picard stack of X over B
% \Picardfunctor_{X/B} denotes the Picard functor of X over B
\def\Pic{\mathop{\rm Pic}\nolimits}
\def\Picardstack{\mathcal{P}\!{\it ic}}
\def\Picardfunctor{{\rm Pic}}
\def\Deformationcategory{\mathcal{D}\!{\it ef}}


% OK, start here.
%
\begin{document}

\title{More on groupoid schemes}


\maketitle

\phantomsection
\label{section-phantom}

\tableofcontents

\section{Introduction}
\label{section-introduction}

\noindent
This chapter is devoted to advanced topics on groupoid schemes.
Even though the results are stated in terms of groupoid schemes, the
reader should keep in mind the $2$-cartesian diagram
\begin{equation}
\label{equation-quotient-stack}
\vcenter{
\xymatrix{
R \ar[r] \ar[d] & U \ar[d] \\
U \ar[r] & [U/R]
}
}
\end{equation}
where $[U/R]$ is the quotient stack, see
Groupoids in Spaces, Remark \ref{spaces-groupoids-remark-fundamental-square}.
Many of the results are motivated by thinking about this diagram.
See for example the beautiful paper \cite{K-M} by Keel and Mori.





\section{Notation}
\label{section-notation}

\noindent
We continue to abide by the conventions and notation introduced in
Groupoids, Section \ref{groupoids-section-notation}.






\section{Useful diagrams}
\label{section-diagrams}

\noindent
We briefly restate the results of
Groupoids, Lemmas \ref{groupoids-lemma-diagram} and
\ref{groupoids-lemma-diagram-pull}
for easy reference in this chapter.
Let $S$ be a scheme.
Let $(U, R, s, t, c)$ be a groupoid scheme over $S$.
In the commutative diagram
\begin{equation}
\label{equation-diagram}
\vcenter{
\xymatrix{
& U & \\
R \ar[d]_s \ar[ru]^t &
R \times_{s, U, t} R
\ar[l]^-{\text{pr}_0} \ar[d]^{\text{pr}_1} \ar[r]_-c &
R \ar[d]^s \ar[lu]_t \\
U & R \ar[l]_t \ar[r]^s & U
}
}
\end{equation}
the two lower squares are fibre product squares.
Moreover, the triangle on top (which is really a square)
is also cartesian.

\medskip\noindent
The diagram
\begin{equation}
\label{equation-pull}
\vcenter{
\xymatrix{
R \times_{t, U, t} R
\ar@<1ex>[r]^-{\text{pr}_1} \ar@<-1ex>[r]_-{\text{pr}_0}
\ar[d]_{\text{pr}_0 \times c \circ (i, 1)} &
R \ar[r]^t \ar[d]^{\text{id}_R} &
U \ar[d]^{\text{id}_U} \\
R \times_{s, U, t} R
\ar@<1ex>[r]^-c \ar@<-1ex>[r]_-{\text{pr}_0} \ar[d]_{\text{pr}_1} &
R \ar[r]^t \ar[d]^s &
U \\
R \ar@<1ex>[r]^s \ar@<-1ex>[r]_t &
U
}
}
\end{equation}
is commutative. The two top rows are isomorphic via the vertical maps given.
The two lower left squares are cartesian.










\section{Properties of groupoids}
\label{section-technical-lemma}

\noindent
Let $(U, R, s, t, c)$ be a groupoid scheme.
The idea behind the results in this section is that $s: R \to U$
is a base changes of the morphism $U \to [U/R]$ (see
Diagram (\ref{equation-quotient-stack}).
Hence the local properties of $s : R \to U$ should reflect local
properties of the morphism $U \to [U/R]$.
This doesn't work, because $[U/R]$ is not always an algebraic stack, and
hence we cannot speak of geometric or algebraic properties of
$U \to [U/R]$.
But it turns out that we can make some of it work without even
referring to the quotient stack at all.

\medskip\noindent
Here is a first example of such a result. The open $W \subset U'$ found
in the lemma is roughly speaking the locus where the morphism
$U' \to [U/R]$ has property $\mathcal{P}$.

\begin{lemma}
\label{lemma-local-source}
Let $S$ be a scheme.
Let $(U, R, s, t, c, e, i)$ be a groupoid over $S$.
Let $g : U' \to U$ be a morphism of schemes.
Denote $h$ the composition
$$
\xymatrix{
h : U' \times_{g, U, t} R \ar[r]_-{\text{pr}_1} & R \ar[r]_s & U.
}
$$
Let $\mathcal{P}, \mathcal{Q}, \mathcal{R}$ be properties of morphisms
of schemes. Assume
\begin{enumerate}
\item $\mathcal{R} \Rightarrow \mathcal{Q}$,
\item $\mathcal{Q}$ is preserved under base change and composition,
\item for any morphism $f : X \to Y$ which has $\mathcal{Q}$ there exists a
largest open $W(\mathcal{P}, f) \subset X$ such that $f|_{W(\mathcal{P}, f)}$
has $\mathcal{P}$, and
\item for any morphism $f : X \to Y$ which has $\mathcal{Q}$,
and any morphism $Y' \to Y$ which has $\mathcal{R}$ we have
$Y' \times_Y W(\mathcal{P}, f) = W(\mathcal{P}, f')$, where
$f' : X_{Y'} \to Y'$ is the base change of $f$.
\end{enumerate}
If $s,t$ have $\mathcal{R}$ and $g$ has $\mathcal{Q}$, then
there exists an open subscheme $W \subset U'$ such that
$W \times_{g, U, t} R = W(\mathcal{P}, h)$.
\end{lemma}

\begin{proof}
Note that the following diagram is commutative
$$
\xymatrix{
U' \times_{g, U, t} R \times_{t, U, t} R
\ar[rr]_-{\text{pr}_{12}}
\ar@<1ex>[d]^-{\text{pr}_{02}} \ar@<-1ex>[d]_-{\text{pr}_{01}} & &
R \times_{t, U, t} R
\ar@<1ex>[d]^-{\text{pr}_1} \ar@<-1ex>[d]_-{\text{pr}_0}
\\
U' \times_{g, U, t} R \ar[rr]^{\text{pr}_1} & & R
}
$$
with both squares cartesian (this uses that the two maps
$t \circ \text{pr}_i : R \times_{t, U, t} R \to U$ are equal).
Combining this with the properties of diagram (\ref{equation-pull})
we get a commutative diagram
$$
\xymatrix{
U' \times_{g, U, t} R \times_{t, U, t} R
\ar[rr]_-{c \circ (i, 1)}
\ar@<1ex>[d]^-{\text{pr}_{02}} \ar@<-1ex>[d]_-{\text{pr}_{01}} & &
R
\ar@<1ex>[d]^-{s} \ar@<-1ex>[d]_-{t}
\\
U' \times_{g, U, t} R \ar[rr]^h & & U
}
$$
where both squares are cartesian.

\medskip\noindent
Assume $s,t$ have $\mathcal{R}$ and $g$ has $\mathcal{Q}$.
Then $h$ has $\mathcal{Q}$ as a composition of $s$ (which has
$\mathcal{R}$ hence $\mathcal{Q}$) and a base change of $g$ (which
has $\mathcal{Q}$). Thus $W(\mathcal{P}, h) \subset U' \times_{g, U, t} R$
exists. By our assumptions we have
$\text{pr}_{01}^{-1}(W(\mathcal{P}, h)) =
\text{pr}_{02}^{-1}(W(\mathcal{P}, h))$
since both are the largest open on which $c \circ (i, 1)$ has $\mathcal{P}$.
Note that the projection $U' \times_{g, U, t} R \to U'$ has a section, namely
$\sigma : U' \to U' \times_{g, U, t} R$, $u' \mapsto (u', e(g(u')))$.
Also via the isomorphism
$$
(U' \times_{g, U, t} R) \times_{U'} (U' \times_{g, U, t} R)
=
U' \times_{g, U, t} R \times_{t, U, t} R
$$
the two projections of the left hand side
to $U' \times_{g, U, t} R$ agree with the morphisms $\text{pr}_{01}$
and $\text{pr}_{02}$ on the right hand side. Since
$\text{pr}_{01}^{-1}(W(\mathcal{P}, h)) =
\text{pr}_{02}^{-1}(W(\mathcal{P}, h))$
we conclude that $W(\mathcal{P}, h)$ is the inverse image of a subset of $U$,
which is necessarily the open set
$W = \sigma^{-1}(W(\mathcal{P}, h))$.
\end{proof}

\begin{remark}
\label{remark-local-source-warning}
Warning:
Lemma \ref{lemma-local-source}
should be used with care.
For example, it applies to $\mathcal{P}=$``flat'', $\mathcal{Q}=$``empty'',
and $\mathcal{R}=$``flat and locally of finite presentation''. But given a
morphism of schemes $f : X \to Y$ the largest open $W \subset X$ such that
$f|_W$ is flat is {\it not} the set of points where $f$ is flat!
\end{remark}

\begin{remark}
\label{remark-local-source-apply}
Notwithstanding the warning in
Remark \ref{remark-local-source-warning}
there are some cases where
Lemma \ref{lemma-local-source}
can be used without causing too much ambiguity.
We give a list. In each case we omit the verification of
assumptions (1) and (2) and we give references which imply
(3) and (4). Here is the list:
\begin{enumerate}
\item $\mathcal{Q} = \mathcal{R} =$``locally of finite type'', and
$\mathcal{P} =$``relative dimension $\leq d$''.
See
Morphisms, Definition \ref{morphisms-definition-relative-dimension-d}
and
Morphisms, Lemmas \ref{morphisms-lemma-openness-bounded-dimension-fibres} and
\ref{morphisms-lemma-dimension-fibre-after-base-change}.
\item $\mathcal{Q} = \mathcal{R} =$``locally of finite type'', and
$\mathcal{P} =$``locally quasi-finite''.
This is the case $d = 0$ of the previous item, see
Morphisms, Lemma \ref{morphisms-lemma-locally-quasi-finite-rel-dimension-0}.
\item $\mathcal{Q} = \mathcal{R} =$``locally of finite type'', and
$\mathcal{P} =$``unramified''.
See
Morphisms, Lemmas \ref{morphisms-lemma-unramified-characterize} and
\ref{morphisms-lemma-set-points-where-fibres-unramified}.
\end{enumerate}
What is interesting about the cases listed above is that we do not
need to assume that $s, t$ are flat to get a conclusion about the locus
where the morphism $h$ has property $\mathcal{P}$. We continue the
list:
\begin{enumerate}
\item[(4)] $\mathcal{Q} =$``locally of finite presentation'',
$\mathcal{R} =$``flat and locally of finite presentation'', and
$\mathcal{P} =$``flat''. See
More on Morphisms, Lemmas \ref{more-morphisms-theorem-openess-flatness} and
Lemma \ref{more-morphisms-lemma-flat-locus-base-change}.
\item[(5)] $\mathcal{Q} =$``locally of finite presentation'',
$\mathcal{R} =$``flat and locally of finite presentation'', and
$\mathcal{P}=$``Cohen-Macaulay''. See
More on Morphisms, Definition \ref{more-morphisms-definition-CM}
and
More on Morphisms, Lemmas \ref{more-morphisms-lemma-base-change-CM} and
\ref{more-morphisms-lemma-flat-finite-presentation-CM-open}.
\item[(6)] $\mathcal{Q} =$``locally of finite presentation'',
$\mathcal{R} =$``flat and locally of finite presentation'', and
$\mathcal{P}=$``syntomic'' use
Morphisms, Lemma \ref{morphisms-lemma-set-points-where-fibres-lci}
(the locus is automically open).
\item[(7)] $\mathcal{Q} =$``locally of finite presentation'',
$\mathcal{R} =$``flat and locally of finite presentation'', and
$\mathcal{P}=$``smooth''. See
Morphisms, Lemma \ref{morphisms-lemma-set-points-where-fibres-smooth}
(the locus is automically open).
\item[(8)] $\mathcal{Q} =$``locally of finite presentation'',
$\mathcal{R} =$``flat and locally of finite presentation'', and
$\mathcal{P}=$``etale''. See
Morphisms, Lemma \ref{morphisms-lemma-set-points-where-fibres-etale}
(the locus is automically open).
\end{enumerate}
\end{remark}

\noindent
Here is the second result. The $R$-invariant open $W \subset U$ should be
thought of as the inverse image of the largest open of $[U/R]$ over which
the morphism $U \to [U/R]$ has property $\mathcal{P}$.

\begin{lemma}
\label{lemma-property-invariant}
Let $S$ be a scheme.
Let $(U, R, s, t, c)$ be a groupoid over $S$.
Let $\tau \in \{Zariski,\linebreak[0] fppf, \linebreak[0] etale,\linebreak[0]
smooth,\linebreak[0] syntomic\}$\footnote{The fact that $fpqc$ is missing
is not a typo.}. Let $\mathcal{P}$ be a property of morphisms of schemes
which is $\tau$-local on the target
(Descent, Definition \ref{descent-definition-property-morphisms-local}).
Assume $\{s : R \to U\}$ and $\{t : R \to U\}$ are coverings for the
$\tau$-topology. Let $W \subset U$ be the maximal open subscheme such that
$s|_{s^{-1}(W)} : s^{-1}(W) \to W$ has property $\mathcal{P}$.
Then $W$ is $R$-invariant, see
Groupoids, Definition \ref{groupoids-definition-invariant-open}.
\end{lemma}

\begin{proof}
Because $\mathcal{P}$ is $\tau$-local on the base, we see that in particular
$\mathcal{P}$ is Zariski local on the base. Hence $W$ is simply the union
of all Zariski open subschemes of $U$ over which the morphism $s$ has property
$\mathcal{P}$, i.e., the assertion of the lemma makes sense.
Next, consider
diagram (\ref{equation-diagram}).
Let $W_1 \subset R$ be the maximal open subscheme over which the morphism
$\text{pr}_1 : R \times_{s, U, t} R \to R$ has property $\mathcal{P}$.
By assumption the morphisms $t$, $s$ are open immersions
(resp.\ flat of finite presentation, etale, smooth, or syntomic).
In particular $s, t$ are open, see
Morphisms, Lemma \ref{morphisms-lemma-fppf-open}.
Hence $W' = s(W_1)$ and $W'' = t(W_1)$ are open subschemes of $U$.
Moreover, $\{s|_{W_1} : W_1 \to W'\}$ and $\{t|_{W_1} : W_1 \to W''\}$
are $\tau$-coverings by our assumption that
$\{s : R \to U\}$ and $\{t : R \to U\}$ are $\tau$-coverings, see
Topologies, Definitions
\ref{topologies-definition-zariski-covering},
\ref{topologies-definition-fppf-covering},
\ref{topologies-definition-etale-covering},
\ref{topologies-definition-smooth-covering}, and
\ref{topologies-definition-syntomic-covering}.
Since the two lower squares of the
diagram (\ref{equation-diagram})
are cartesian we now conclude that
\begin{enumerate}
\item $s : R \to U$ has property $\mathcal{P}$ over $W'$,
\item $t : R \to U$ has property $\mathcal{P}$ over $W''$, and also
\item $\text{pr}_1 : R \times_{s, U, t} R \to R$ has property
$\mathcal{P}$ over $t^{-1}(W)$, and
\item $\text{pr}_1 : R \times_{s, U, t} R \to R$ has property
$\mathcal{P}$ over $s^{-1}(W)$.
\end{enumerate}
Clarification: The first two statements come from descending the property
through the coverings mentioned above, the second two by going up along the
coverings $\{s|_{s^{-1}(W)} : s^{-1}(W) \to W\}$ and
$\{t|_{t^{-1}(W)} : t^{-1}(W) \to W\}$.
All in all we conclude that
$W', W'' \subset W$ and $t^{-1}(W), s^{-1}(W) \subset W_1$. In other words
$W_1 = s^{-1}(W) = t^{-1}(W)$ as desired.
\end{proof}



\section{Comparing fibres}
\label{section-fibres}

\noindent
Let $(U, R, s, t, c, e, i)$ be a groupoid scheme over $S$.
Diagram (\ref{equation-diagram})
gives us a way to compare the fibres of the map $s : R \to U$ in a groupoid.
For a point $u \in U$ we will denote $F_u = s^{-1}(u)$ the scheme
theoretic fibre of $s : R \to U$ over $u$. For example the diagram
implies that if $u, u' \in U$ are points such
that $s(r) = u$ and $t(r) = u'$, then
$(F_u)_{\kappa(r)} \cong (F_{u'})_{\kappa(r)}$.
This is a special case of the more general and more precise
Lemma \ref{lemma-two-fibres}
below. To see this take $r' = i(r)$.

\medskip\noindent
A pair $(X, x)$ consisting of a scheme $X$ and a point $x \in X$ is sometimes
called the {\it germ of $X$ at $x$}.
A {\it morphism of germs} $f : (X, x) \to (S, s)$
is a morphism $f : U \to S$ defined on an open neighbourhood
of $x$ with $f(x) = s$. Two such
$f$, $f'$ are said to give the same morphism of germs
if and only if $f$ and $f'$ agree in some open neighbourhood of $x$.
Let $\tau \in \{Zariski, etale, smooth, syntomic, fppf\}$.
We temporarily introduce the following concept: We say that two morphisms
of germs $f : (X, x) \to (S, s)$ and $f' : (X', x') \to (S', s')$
are {\it isomorphic locally on the base in the $\tau$-topology},
if there exists a pointed scheme $(S'', s'')$ and morphisms of germs
$g : (S'', s'') \to (S, s)$, and $g' : (S'', s'') \to (S', s')$
such that
\begin{enumerate}
\item $g$ and $g'$ are an open immersion (resp.\ etale, smooth, syntomic,
flat and locally of finite presentation) at $s''$,
\item there exists an isomorphism
$$
(S'' \times_{g, S, f} X, \tilde x)
\cong
(S'' \times_{g', S', f'} X', \tilde  x')
$$
of germs over the germ $(S'', s'')$ for some choice of points
$\tilde x$ and $\tilde x'$ lying over $(s'', x)$ and $(s'', x')$.
\end{enumerate}
Finally, we simply say that the maps of germs
$f : (X, x) \to (S, s)$ and $f' : (X', x') \to (S', s')$
are {\it flat locally on the base isomorphic} if there exist
$S'', s'', g, g'$ as above but with (1) replaced by
the condition that $g$ and $g'$ are flat at $s''$ (this is
much weaker than any of the $\tau$ conditions above
as a flat morphism need not be open).

\begin{lemma}
\label{lemma-two-fibres}
Let $S$ be a scheme.
Let $(U, R, s, t, c)$ be a groupoid over $S$.
Let $r, r' \in R$ with $t(r) = t(r')$ in $U$.
Set $u = s(r)$, $u' = s(r')$.
Denote $F_u = s^{-1}(u)$ and $F_{u'} = s^{-1}(u')$ the scheme
theoretic fibres.
\begin{enumerate}
\item There exists a common field extension
$\kappa(u) \subset k$, $\kappa(u') \subset k$ and
an isomorphism $(F_u)_k \cong (F_{u'})_k$.
\item We may choose the isomorphism of (1) such that a point
lying over $r$ maps to a point lying over $r'$.
\item If the morphisms $s$, $t$ are flat then the morphisms of germs
$s : (R, r) \to (U, u)$ and $s : (R, r') \to (U, u')$ are flat
locally on the base isomorphic.
\item If the morphisms $s$, $t$ are etale
(resp.\ smooth, syntomic, or flat and locally of finite presentation)
then the morphisms of germs $s : (R, r) \to (U, u)$ and
$s : (R, r') \to (U, u')$ are locally on the base isomorphic
in the etale (resp.\ smooth, syntomic, or fppf) topology.
\end{enumerate}
\end{lemma}

\begin{proof}
We repeatedly use the properties and the existence of
diagram (\ref{equation-diagram}).
By the properties of the diagram (and
Schemes, Lemma \ref{schemes-lemma-points-fibre-product})
there exists a point $\xi$ of $R \times_{s, U, t} R$
with $\text{pr}_0(\xi) = r$ and $c(\xi) = r'$.
Let $\tilde r = \text{pr}_1(\xi) \in R$.

\medskip\noindent
Proof of (1). Set $k = \kappa(\tilde r)$. Since $t(\tilde r) = u$
and $s(\tilde r) = u'$ we see that $k$ is a common extension
of both $\kappa(u)$ and $\kappa(u')$ and in fact that
both $(F_u)_k$ and $(F_{u'})_k$ are isomorphic to the fibre of
$\text{pr}_1 : R \times_{s, U, t} R \to R$ over $\tilde r$.
Hence (1) is proved.

\medskip\noindent
Part (2) follows since the point $\xi$ maps to $r$, resp.\ $r'$.

\medskip\noindent
Part (3) is clear from the above (using the point $\xi$ for
$\tilde u$ and $\tilde u'$) and the definitions.

\medskip\noindent
If $s$ and $t$ are flat and of finite presentation, then
they are open morphisms (Morphisms, Lemma \ref{morphisms-lemma-fppf-open}).
Hence the image of some affine open neighbourhood $V''$ of $\tilde r$ will
cover an open neighbourhood $V$ of $u$, resp.\ $V'$ of $u'$.
These can be used to show that properties (1) and (2) of the
definition of ``locally on the base isomorphic in the
$\tau$-topology''.
\end{proof}







\section{Cohen-Macaulay presentations}
\label{section-CM}

\noindent
Given any groupoid $(U, R, s, t, c)$ with $s, t$ flat and
locally of finite presentation there exists an ``equivalent''
groupoid $(U', R', s', t', c')$ such that $s'$ and $t'$ are
Cohen-Macaulay morphisms (and locally of finite presentation). See
More on Morphisms, Section \ref{more-morphisms-section-CM}
for more information on Cohen-Macaulay morphisms.
Here ``equivalent'' can be taken to mean that the quotient stacks
$[U/R]$ and $[U'/R']$ are equivalent stacks, see
Groupoid Spaces, Section \ref{spaces-groupoids-section-stacks}
and Section \ref{spaces-groupoids-section-quotient-stack-restrict}.

\begin{lemma}
\label{lemma-make-CM}
Let $S$ be a scheme.
Let $(U, R, s, t, c)$ be a groupoid over $S$.
Assume $s$ and $t$ are flat and locally of finite presentation.
Then there exists an open $U' \subset U$ such that
\begin{enumerate}
\item $t^{-1}(U') \subset R$ is the largest open subscheme of
$R$ on which the morphism $s$ is Cohen-Macaulay,
\item $s^{-1}(U') \subset R$ is the largest open subscheme of
$R$ on which the morphism $t$ is Cohen-Macaulay,
\item the morphism $t|_{s^{-1}(U')} : s^{-1}(U') \to U$ is
surjective,
\item the morphism $s|_{t^{-1}(U')} : t^{-1}(U') \to U$ is
surjective, and
\item the restriction $R' = s^{-1}(U') \cap t^{-1}(U')$
of $R$ to $U'$ defines a groupoid $(U', R', s', t', c')$ which has the property
that the morphisms $s'$ and $t'$ are Cohen-Macaulay and locally of
finite presentation.
\end{enumerate}
\end{lemma}

\begin{proof}
Apply
Lemma \ref{lemma-local-source}
with
$g = \text{id}$ and
$\mathcal{Q} =$``locally of finite presentation'',
$\mathcal{R} =$``flat and locally of finite presentation'', and
$\mathcal{P}=$``Cohen-Macaulay'', see
Remark \ref{remark-local-source-apply}.
This gives us an open $U' \subset U$ such that
Let $t^{-1}(U') \subset R$ is the largest open subscheme of $R$
on which the morphism $s$ is Cohen-Macaulay.
This proves (1).
Let $i : R \to R$ be the inverse of the groupoid.
Since $i$ is an isomorphism, and $s \circ i = t$ and $t \circ i = s$
we see that $s^{-1}(U')$ is also the largest open of $R$ on which $t$ is
Cohen-Macaulay. This proves (2).
By 
More on Morphisms,
Lemma \ref{more-morphisms-lemma-flat-finite-presentation-CM-open}
the open subset $t^{-1}(U')$ is dense in every fibre of $s : R \to U$.
This proves (3). Same argument for (4).
Part (5) is a formal consequence of (1) and (2) and the discussion
of restrictions in
Groupoids, Section \ref{groupoids-section-restrict-groupoid}.
\end{proof}











\section{Slicing groupoids}
\label{section-slicing}


\begin{lemma}
\label{lemma-slice}
Let $S$ be a scheme.
Let $(U, R, s, t, c, e, i)$ be a groupoid scheme over $S$.
Let $G \to U$ be the stabilizer group scheme.
Assume $s$ and $t$ are Cohen-Macaulay and locally of finite presentation.
Let $u \in U$.
Set $r = e(u)$ and $F_u = s^{-1}(u)$. Set
$$
d_1 = \dim(G_u),\quad d_2 = \dim_r(F_u).
$$
If $d_2 > d_1$, then there exist an affine scheme $U'$
and a morphism $g : U' \to U$ such that
\begin{enumerate}
\item $g$ is an immersion and locally of finite presentation,
\item $u \in U'$,
\item the composition
$$
h = s \circ \text{pr}_1 : U' \times_{g, U, t} R \longrightarrow U
$$
is Cohen-Macaulay at $(u, r)$, and
\item letting $R'$ be the restriction of $R$ to $U'$ and setting
$F'_u = (s')^{-1}(u)$ we have $\dim_r(F'_u) = d_2 - 1$.
\end{enumerate}
WARNING: The formulation of this lemma is currently wrong.
If $u$ is a point of finite type over $S$ then it is probably
OK. The error in the proof is that the second displayed equation
is wrong.
\end{lemma}

\begin{proof}
Let $\text{Spec}(A) \subset U$ be an affine neighbourhood of $u$.
Let $\text{Spec}(B) \subset R$ be an affine neighbourhood of $r$
which maps via $j$ into the open
$\text{Spec}(A) \times_S \text{Spec}(A) \subset U \times_S U$.
Let $\mathfrak p \subset A$ be the prime ideal corresponding to $u$.
Let $\mathfrak q \subset B$ be the prime ideal corresponding to $r$.
Pictures:
$$
\vcenter{
\xymatrix{
B & A \ar[l]^s \\
A \ar[u]^t
}
}
\quad\text{and}\quad
\vcenter{
\xymatrix{
B_{\mathfrak q} & A_{\mathfrak p} \ar[l]^s \\
A_{\mathfrak p} \ar[u]^t
}
}
$$
Note that $\kappa(\mathfrak p)$ is identified with $\kappa(\mathfrak q)$
via both arrows as $r = e(u)$. The ring maps $s, t : A \to B$ are
locally of finite type and flat. By assumption the ring
$$
\mathcal{O}_{F_u, r} = B_{\mathfrak q}/s(\mathfrak p)B_{\mathfrak q}
$$
is Cohen-Macaulay of dimension $d_2$ (equality of dimension by
Morphisms, Lemma \ref{morphisms-lemma-dimension-fibre-at-a-point}).
By assumption the ring
$$
\mathcal{O}_{G_u, r}
=
B_{\mathfrak q}/(s(\mathfrak p)B_{\mathfrak q} + t(\mathfrak p)B_{\mathfrak q})
$$
has dimension $d_1$ (equality of dimension by
Groupoids, Lemma \ref{groupoids-lemma-group-scheme-finite-type-field}).
We claim this implies we can find
an element $f \in \mathfrak p$ such that
$$
\dim(B_{\mathfrak q}/(s(\mathfrak p)B_{\mathfrak q} + fB_{\mathfrak q}) < d_2
$$
Namely, suppose $\mathfrak n_j \supset s(\mathfrak p)B_{\mathfrak q}$,
$j = 1, \ldots, m$ correspond to the minimal primes of the local ring
$B_{\mathfrak q}/s(\mathfrak p)B_{\mathfrak q}$.
There are finitely many as this ring is Noetherian (since it is essentially
of finite type over a field -- but also because a Cohen-Macaulay ring is
Noetherian). By the Cohen-Macaulay condition we have
$\dim(B_{\mathfrak q}/\mathfrak n_j) = d_2$, for example by
Algebra, Lemma \ref{algebra-lemma-CM-dim-formula}.
Then the conditions $d_1 < d_2$ and
$\dim(B_{\mathfrak q}/(\mathfrak n_j + t(\mathfrak p)B_{\mathfrak q})) = d_1$
implies that
$\mathfrak p \not \subset t^{-1}(\mathfrak n_i)$. Since there are only
finitely many minimal primes in the Noetherian ring
$B_{\mathfrak q}/s(\mathfrak p)B_{\mathfrak q}$ we get our desired $f$
by prime avoidence, see
Algebra, Lemma \ref{algebra-lemma-silly}.

\medskip\noindent
Set $A' = A/fA$ and $U' = \text{Spec}(A')$. Then it is clear that
$U' \to U$ is an immersion, locally of finite presentation
and that $u \in U'$. Moreover, the morphism
$$
U' \times_{g, U, t} R \longrightarrow U
$$
factors through $\text{Spec}(A)$ and corresponds to the ring map
$$
\xymatrix{
B/t(f)B \ar@{=}[r] & A/(f) \otimes_{A, t} B & A \ar[l]_-s
}
$$
Now, we see $t(f)$ is not a zero divisor on
$B_{\mathfrak q}/s(\mathfrak p)B_{\mathfrak q}$ as this is a
Cohen-Macaulay ring of positive dimension and $f$ is not contained
in any minimal prime, see for example
Algebra, Lemma \ref{algebra-lemma-reformulate-CM}.
Hence by
Algebra, Lemma \ref{algebra-lemma-grothendieck-general}
we conclude that $s : A_{\mathfrak p} \to B_{\mathfrak q}/t(f)B_{\mathfrak q}$
is flat with fibre ring
$B_{\mathfrak q}/(s(\mathfrak p)B_{\mathfrak q} + t(f)B_{\mathfrak q})$
which is Cohen-Macaulay of $1$ lower dimension by
Algebra, Lemma \ref{algebra-lemma-reformulate-CM}
again. This implies the final assertion of the lemma and we win.
\end{proof}










\section{Etale localization of groupoids}
\label{section-etale-localize}

\noindent
Lemma \ref{lemma-quasi-finite-over-base-j-proper}
will be used to prove results on algebraic spaces
separated and quasi-finite over a scheme (insert future reference here).

\begin{lemma}
\label{lemma-quasi-finite-over-base}
Let $S$ be a scheme.
Let $(U, R, s, t, c)$ be a groupoid scheme over $S$.
Let $p \in S$ be a point, and let $u \in U$ be a point lying over $p$.
Assume that
\begin{enumerate}
\item $U \to S$ is locally of finite type,
\item $U \to S$ is quasi-finite at $u$,
\item $U \to S$ is separated,
\item $R \to S$ is separated,
\item $s$, $t$ are flat and locally of finite presentation, and
\item $s^{-1}(\{u\})$ is finite.
\end{enumerate}
Then there exists an etale neighbourhood $(S', p') \to (S, p)$ with
$\kappa(p) = \kappa(p')$ and a base change diagram
$$
\xymatrix{
R' \coprod W'
\ar@{=}[r] &
S' \times_S R
\ar[r] \ar@<2ex>[d]^{s'} \ar@<-2ex>[d]_{t'} &
R \ar@<1ex>[d]^s \ar@<-1ex>[d]_t \\
U' \coprod W
\ar@{=}[r] &
S' \times_S U
\ar[r] \ar[d] &
U \ar[d] \\
 &
S' \ar[r] &
S
}
$$
where the equal signs are decompositions into open and closed
subschemes such that
\begin{enumerate}
\item[(a)] there exists a point $u'$ of $U'$ mapping to $u$ in $U$,
\item[(b)] the fibre $(U')_{p'}$ consists of the points of
$(S' \times_S U)_{p'}$ equivalent to $u'$,
\item[(c)] the fibre $(R')_{p'} = (s')^{-1}(\{u'\})$,
\item[(d)] the schemes $U'$, $R'$ are finite over $S'$,
\item[(e)] we have $s'(R') \subset U'$, $t'(R') \subset U'$,
\item[(f)] we have
$c'(R' \times_{t', U', s'} R') \subset R'$
where $c'$ is the base change of $c$, and
\item[(g)] the morphisms $s', t', c'$ determine a groupoid structure
by taking the system
$(U', R', s'|_{R'}, t'|_{R'}, c'|_{R' \times_{t', U', s'} R'})$.
\end{enumerate}
\end{lemma}

\begin{proof}
Let us denote $f : U \to S$ the structure morphism of $U$.
Note that $s^{-1}(\{u\}) = \{r_1, \ldots, r_n\}$
is finite implies that $s$ is quasi-finite
at each of these finitely many inverse images, see
Morphisms, Lemma \ref{morphisms-lemma-finite-fibre}.
Hence we see that $f \circ s : R \to S$ is quasi-finite at each $r_i$
(Morphisms, Lemma \ref{morphisms-lemma-composition-quasi-finite}).
Since $t$ is flat and locally of finite presentation,
the morphism of fibres $t_p : R_p \to U_p$ is flat and locally of
finite presentation (Morphisms,
Lemmas \ref{morphisms-lemma-base-change-flat} and
\ref{morphisms-lemma-base-change-finite-presentation}),
hence open (Morphisms,
Lemma \ref{morphisms-lemma-fppf-open}). Thus the fact that $r_i$ is isolated
in $R_p$ implies that $u_i = t(r_i)$ is isolated in $U_p$.
In other words, we see that $f$ is quasi-finite at $u_1, \ldots, u_n$.
Note that $t^{-1}(\{u\}) = \{r_1, \ldots, r_n\}$ and that
$s(\{r_1, \ldots, r_n\}) = \{u_1, \ldots, u_n\}$ since $R$ is a
groupoid. We remark that the cardinality of the set
$\{r_1, \ldots, r_n\}$ is in general greater than the cardinality
of the set $\{u_1, \ldots, u_n\}$. Note that $u \in \{u_1, \ldots, u_n\}$.

\medskip\noindent
We may apply
More on Morphisms,
Lemma \ref{more-morphisms-lemma-etale-splits-off-quasi-finite-part-technical}
to the morphisms
$U \to S$, and $R \to S$ to get an etale neighbourhood
$(S', p') \to (S, p)$ which induces a trivial residue field extension
such that $S' \times_S U$ and $S' \times_S R$ decompose as
$$
S' \times_S U = U' \coprod W, \quad
S' \times_S R = R' \coprod W'
$$
with $U' \to S'$ finite and
$U'_{p'}$ mapping bijectively to $\{u_1, \ldots, u_n\}$, and
$R' \to S'$ finite and
$R'_{p'}$ mapping bijectively to $\{r_1, \ldots, r_n\}$.
Moreover, no point of $W_{p'}$ (resp.\ $W'_{p'}$) maps to
any of the points $u_i$ (resp.\ $r_i$). At this point (a), (b), (c), and (d)
of the lemma are satisfied.

\medskip\noindent
Consider the set $(s'|_{R'})^{-1}(W)$. This is open and closed in $R'$
and does not contain any points of $R'$ lying over $p'$. Since
$R' \to S'$ is closed, we may assume, after shrinking $S'$, that
it is empty. In other words $s'$ maps $R'$ into $U'$.
Similarly, we can arrange it so that $t'$ maps $R'$ into $U'$.
In the same manner, consider the set
$(c'|_{R' \times_{s', U', t'} R'})^{-1}(W')$.
It is open and closed in the scheme $R' \times_{s', U', t'} R'$
which is finite over $S'$, does not contain any points lying
over $p'$, and hence after shrinking $S'$ we may assume
that it is empty. We may repeat the argument also with the identity
$e : U \to R$ and the inverse $i : R \to R$ so that we may assume
(after shrinking $S'$ some more) that $(e'|_{U'})^{-1}(W') = \emptyset$
and $(i'|_{R'})^{-1}(W') = \emptyset$.

\medskip\noindent
At this point we see that we may consider the structure
$$
(U', R', s'|_{R'}, t'|_{R'}, c'|_{R' \times_{t', U', s'} R'},
e'|_{U'}, i'|_{R'}).
$$
The axioms of a groupoid scheme over $S'$ hold
because they hold for the groupoid scheme
$(S' \times_S U, S' \times_S R, s', t', c', e', i')$.
\end{proof}

\begin{lemma}
\label{lemma-quasi-finite-over-base-j-proper}
Let $S$ be a scheme.
Let $(U, R, s, t, c)$ be a groupoid scheme over $S$.
Let $p \in S$ be a point, and let $u \in U$ be a point lying over $p$.
Assume assumptions (1) -- (6) of
Lemma \ref{lemma-quasi-finite-over-base}
hold as well as
\begin{enumerate}
\item[(7)] $j : R \to U \times_S U$ is universally closed\footnote{In view of
the other conditions this is equivalent to requiring $j$ to be proper.}.
\end{enumerate}
Then we can choose $(S', p') \to (S, p)$ and decompositions
$S' \times_S U = U' \amalg W$ and $S' \times_S R = R' \amalg W'$
and $u' \in U'$ such that (a) -- (g) of
Lemma \ref{lemma-quasi-finite-over-base}
hold as well as
\begin{enumerate}
\item[(h)] $R'$ is the restriction of $S' \times_S R$ to $U'$.
\end{enumerate}
\end{lemma}

\begin{proof}
We apply Lemma \ref{lemma-quasi-finite-over-base} for the
groupoid $(U, R, s, t, c)$ over the scheme $S$ with points $p$ and $u$.
Hence we get an etale neighbourhood
$(S', p') \to (S, p)$ and disjoint union decompositions
$$
S' \times_S U = U' \amalg W, \quad
S' \times_S R = R' \amalg W'
$$
and $u' \in U'$ satisfying conclusions (a), (b), (c), (d), (e), (f), and (g).
We may shrink $S'$ to a smaller neighbourhood of $p'$ without
affecting the conclusions (a) -- (g). We will show that for a suitable
shrinking conclusion (h) holds as well.
Let us denote $j'$ the base change of $j$ to $S'$.
By conclusion (e) it is clear that
$$
j'^{-1}(U' \times_{S'} U') = R' \amalg Rest
$$
for some open and closed $Rest$ piece. Since $U' \to S'$ is finite
by conclusion (d) we see that $U' \times_{S'} U'$ is finite over $S'$.
Since $j$ is universally closed, also $j'$ is universally closed, and
hence $j'|_{Rest}$ is universally closed too. By conclusions
(b) and (c) we see that the fibre of
$$
(U' \times_{S'} U' \to S') \circ j'|_{Rest} :
Rest
\longrightarrow
S'
$$
over $p'$ is empty. Hence, since $Rest \to S'$ is closed as a composition
of closed morphisms, after shrinking $S'$, we may assume that
$Rest = \emptyset$. And this is exactly the condition that $R'$ is
the restriction of $S' \times_S R$ to the open subscheme
$U' \subset S' \times_S U$, see
Groupoids, Lemma \ref{groupoids-lemma-restrict-groupoid-relation}
and its proof.
\end{proof}

\noindent
The following lemma is \cite[Proposition 4.2]{K-M}.
The formulation is minimal in the sense that we only state the assumptions
used in the proof; we know of no applications but the case where
the morphisms $s$, $t$ are also flat (and of finite presentation).
And clearly if $s$, $t$ are flat, then so are $s'$ and $t'$, whence
also $s'|_P$, $t'|_P$.
Note that assumptions (3) and (4) are implied by the assumption
that the fibre of $s$ over $u$ is a finite set, see
Morphisms, Lemma \ref{morphisms-lemma-finite-fibre}.

\begin{lemma}
\label{lemma-quasi-finite-groupoid}
Let $S$ be a scheme.
Let $(U, R, s, t, c)$ be a groupoid scheme over $S$.
Let $u \in U$ be a point. Assume that
\begin{enumerate}
\item $s, t : R \to U$ are separated,
\item $s$, $t$ are locally of finite type, and
\item there are finitely many points $r_1, \ldots, r_n \in R$ with
$s(r_i) = t(r_i) = u$, and
\item $s$ is quasi-finite at each $r_i$.
\end{enumerate}
Then\footnote{The proof of this is currently incomplete and the formulation
of this lemma may have to be changed to assuming $s,t$ are flat and locally
of finite presentation. We are working on it.}
there exists an etale neighbourhood $(U', u') \to (U, u)$
with $\kappa(u) = \kappa(u')$
such that the restriction $R' = R|_{U'}$ of $R$ to $U'$ decomposes
into open and closed subschemes
$$
R' = R|_{U'} = P \coprod W
$$
with
\begin{enumerate}
\item $(U', P, s'|_P, t'|_P, c'|_P)$ is a groupoid scheme over $S$,
\item the morphisms $s'|_P : P \to U'$ and $t'|_P : P \to U'$ are
finite,
\item any point $r' \in R'$ with $s'(r') = t'(r') = u'$ is contained
in $P$.
\end{enumerate}
\end{lemma}

\begin{proof}
Since $(U, R, s, t, c)$ is a groupoid scheme, we see that also
$t$ is quasi-finite at each $r_i$. By
More on Morphisms,
Lemma \ref{more-morphisms-lemma-etale-splits-off-quasi-finite-part-technical}
we can find an etale neighbourhood $(U', u') \to (U, u)$ with
$\kappa(u) = \kappa(u')$ such that we have disjoint decompositions
$$
R \times_{s, U} U' = V_1 \coprod W_1,
\quad\text{and}\quad
U' \times_{U, t} R = V_2 \coprod W_2
$$
with the following properties
\begin{enumerate}
\item $V_1 \to U'$ is finite,
\item $V_2 \to U'$ is finite,
\item $(V_1)_{u'} = \{r_1, \ldots, r_n\}$ as sets (via the identification
$(R \times_{s, U} U')_{u'} = R_u$),
\item $(V_2)_{u'} = \{r_1, \ldots, r_n\}$ as sets (via the identification
$(U' \times_{U, t} R)_{u'} = R_u$).
\end{enumerate}
By definition of the restriction
(Groupoids, Definition \ref{groupoids-definition-restrict-groupoid})
we have
$$
R|_{U'} = (V_1 \coprod W_1) \times_R (V_1 \coprod W_1).
$$
Hence if we take $P = V_1 \times_R V_2$, then $P$ is open and closed
in the restriction and the morphisms $s'|_P, t'|_P : P \to U'$ are finite.
FIXME: Actually there seems to be no reason for this to hold.

\medskip\noindent
Since $R'$ is a groupoid we have $(s')^{-1}(\{u'\}) = (t')^{-1}(\{u'\})$
and because $\kappa(u) = \kappa(u')$ this set maps bijectively to
$s^{-1}(\{u\}) = t^{-1}(\{u\})$. Hence the set
$\{p' \in P' \mid s'(p') = t'(p') = u'\}$
maps bijectively to $\{r_1, \ldots, r_n\}$.
By our choice of $V_1$ and $V_2$ we see that the subset
$P \cap (s')^{-1}(\{u'\})$ maps bijectively
to $\{r_1, \ldots, r_n\}$. Thus we conclude that (3) holds.

\medskip\noindent
Finally, we have to show that (1) holds, i.e., that
$(U', P, s'|_P, t'|_P, c'|_P)$ is a groupoid.
Consider the morphism $c' : R' \times_{s', U', t'} R' \to R'$.
We want to show that, after possibly shrinking $U'$, that
$c'(P \times_{s'|_P, U', t'|_P} P) \subset P$. To see this
consider $T = (c')^{-1}(W) \cap P \times_{s'|_P, U', t'|_P} P$.
This is an open and closed subscheme of $P \times_{s'|_P, U', t'|_P} P$.
By the discussion in the preceding paragraph we have
$$
(s'|_P)^{-1}(\{u'\}) = \{r_1, \ldots, r_n\} = (t'|_P)^{-1}(\{u'\})
$$
and hence $c'$ maps this set into itself! Hence, if
$p_0, p_1, p_2: R' \times_{t',U',s'} R' \rightarrow U'$ are
the three projections, then $T \cap p_i^{-1}(\{u'\}) = 0$ for $i = 0, 1, 2$.
Since the restrictions of $p_i$ to $P \times_{s'|_P, U', t'|_P} P$ are finite
(because $s'|_P$ and $t'|_P$ are finite), we see that $p_i(T) \subset U'$
is a closed subscheme which does not contain the point $u'$.
Hence we may replace $U'$ by $U' \setminus p_0(T) \cup p_1(T) \cup p_2(T)$
to get to the case where $T = \emptyset$. At this point
we see that $c'(P \times_{s'|_P, U', t'|_P} P) \subset P$
as desired. We omit the verifications that, possibly after shrinking $S'$,
we may assume that also $e'(U') \subset P$ and $i'(P) \subset P$.
Hence we get a groupoid scheme $(U', P, s'|_P, t'|_P, c'|_P, e'|_P, i'|_P)$
and the lemma follows.
\end{proof}
























\section{Other chapters}

\begin{multicols}{2}
\begin{enumerate}
\item \hyperref[introduction-section-phantom]{Introduction}
\item \hyperref[conventions-section-phantom]{Conventions}
\item \hyperref[sets-section-phantom]{Set Theory}
\item \hyperref[categories-section-phantom]{Categories}
\item \hyperref[topology-section-phantom]{Topology}
\item \hyperref[sheaves-section-phantom]{Sheaves on Spaces}
\item \hyperref[algebra-section-phantom]{Commutative Algebra}
\item \hyperref[sites-section-phantom]{Sites and Sheaves}
\item \hyperref[homology-section-phantom]{Homological Algebra}
\item \hyperref[derived-section-phantom]{Derived Categories}
\item \hyperref[more-algebra-section-phantom]{More Algebra}
\item \hyperref[simplicial-section-phantom]{Simplicial Methods}
\item \hyperref[modules-section-phantom]{Sheaves of Modules}
\item \hyperref[sites-modules-section-phantom]{Modules on Sites}
\item \hyperref[injectives-section-phantom]{Injectives}
\item \hyperref[cohomology-section-phantom]{Cohomology of Sheaves}
\item \hyperref[sites-cohomology-section-phantom]{Cohomology on Sites}
\item \hyperref[hypercovering-section-phantom]{Hypercoverings}
\item \hyperref[schemes-section-phantom]{Schemes}
\item \hyperref[constructions-section-phantom]{Constructions of Schemes}
\item \hyperref[properties-section-phantom]{Properties of Schemes}
\item \hyperref[morphisms-section-phantom]{Morphisms of Schemes}
\item \hyperref[coherent-section-phantom]{Coherent Cohomology}
\item \hyperref[divisors-section-phantom]{Divisors}
\item \hyperref[limits-section-phantom]{Limits of Schemes}
\item \hyperref[varieties-section-phantom]{Varieties}
\item \hyperref[chow-section-phantom]{Chow Homology}
\item \hyperref[topologies-section-phantom]{Topologies on Schemes}
\item \hyperref[descent-section-phantom]{Descent}
\item \hyperref[more-morphisms-section-phantom]{More on Morphisms}
\item \hyperref[flat-section-phantom]{More on Flatness}
\item \hyperref[groupoids-section-phantom]{Groupoid Schemes}
\item \hyperref[more-groupoids-section-phantom]{More on Groupoid Schemes}
\item \hyperref[etale-section-phantom]{\'Etale Morphisms of Schemes}
\item \hyperref[etale-cohomology-section-phantom]{\'Etale Cohomology}
\item \hyperref[spaces-section-phantom]{Algebraic Spaces}
\item \hyperref[spaces-properties-section-phantom]{Properties of Algebraic Spaces}
\item \hyperref[spaces-morphisms-section-phantom]{Morphisms of Algebraic Spaces}
\item \hyperref[spaces-topologies-section-phantom]{Topologies on Algebraic Spaces}
\item \hyperref[spaces-descent-section-phantom]{Descent and Algebraic Spaces}
\item \hyperref[spaces-more-morphisms-section-phantom]{More on Morphisms of Spaces}
\item \hyperref[quot-section-phantom]{Quot and Hilbert Spaces}
\item \hyperref[stacks-section-phantom]{Stacks}
\item \hyperref[spaces-groupoids-section-phantom]{Groupoids in Algebraic Spaces}
\item \hyperref[spaces-more-groupoids-section-phantom]{More on Groupoids in Spaces}
\item \hyperref[bootstrap-section-phantom]{Bootstrap}
\item \hyperref[examples-stacks-section-phantom]{Examples of Stacks}
\item \hyperref[groupoids-quotients-section-phantom]{Quotients of Groupoids}
\item \hyperref[algebraic-section-phantom]{Algebraic Stacks}
\item \hyperref[criteria-section-phantom]{Criteria for Representability}
\item \hyperref[stacks-properties-section-phantom]{Properties of Algebraic Stacks}
\item \hyperref[stacks-morphisms-section-phantom]{Morphisms of Algebraic Stacks}
\item \hyperref[examples-section-phantom]{Examples}
\item \hyperref[exercises-section-phantom]{Exercises}
\item \hyperref[guide-section-phantom]{Guide to Literature}
\item \hyperref[desirables-section-phantom]{Desirables}
\item \hyperref[coding-section-phantom]{Coding Style}
\item \hyperref[fdl-section-phantom]{GNU Free Documentation License}
\item \hyperref[index-section-phantom]{Auto Generated Index}
\end{enumerate}
\end{multicols}


\bibliography{my}
\bibliographystyle{amsalpha}

\end{document}
