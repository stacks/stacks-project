\IfFileExists{stacks-project.cls}{%
\documentclass{stacks-project}
}{%
\documentclass{amsart}
}

% The following AMS packages are automatically loaded with
% the amsart documentclass:
%\usepackage{amsmath}
%\usepackage{amssymb}
%\usepackage{amsthm}

% For dealing with references we use the comment environment
\usepackage{verbatim}
\newenvironment{reference}{\comment}{\endcomment}
%\newenvironment{reference}{}{}
\newenvironment{slogan}{\comment}{\endcomment}
\newenvironment{history}{\comment}{\endcomment}

% For commutative diagrams you can use
% \usepackage{amscd}
\usepackage[all]{xy}

% We use 2cell for 2-commutative diagrams.
\xyoption{2cell}
\UseAllTwocells

% To put source file link in headers.
% Change "template.tex" to "this_filename.tex"
% \usepackage{fancyhdr}
% \pagestyle{fancy}
% \lhead{}
% \chead{}
% \rhead{Source file: \url{template.tex}}
% \lfoot{}
% \cfoot{\thepage}
% \rfoot{}
% \renewcommand{\headrulewidth}{0pt}
% \renewcommand{\footrulewidth}{0pt}
% \renewcommand{\headheight}{12pt}

\usepackage{multicol}

% For cross-file-references
\usepackage{xr-hyper}

% Package for hypertext links:
\usepackage{hyperref}

% For any local file, say "hello.tex" you want to link to please
% use \externaldocument[hello-]{hello}
\externaldocument[introduction-]{introduction}
\externaldocument[conventions-]{conventions}
\externaldocument[sets-]{sets}
\externaldocument[categories-]{categories}
\externaldocument[topology-]{topology}
\externaldocument[sheaves-]{sheaves}
\externaldocument[sites-]{sites}
\externaldocument[stacks-]{stacks}
\externaldocument[fields-]{fields}
\externaldocument[algebra-]{algebra}
\externaldocument[brauer-]{brauer}
\externaldocument[homology-]{homology}
\externaldocument[derived-]{derived}
\externaldocument[simplicial-]{simplicial}
\externaldocument[more-algebra-]{more-algebra}
\externaldocument[smoothing-]{smoothing}
\externaldocument[modules-]{modules}
\externaldocument[sites-modules-]{sites-modules}
\externaldocument[injectives-]{injectives}
\externaldocument[cohomology-]{cohomology}
\externaldocument[sites-cohomology-]{sites-cohomology}
\externaldocument[dga-]{dga}
\externaldocument[dpa-]{dpa}
\externaldocument[hypercovering-]{hypercovering}
\externaldocument[schemes-]{schemes}
\externaldocument[constructions-]{constructions}
\externaldocument[properties-]{properties}
\externaldocument[morphisms-]{morphisms}
\externaldocument[coherent-]{coherent}
\externaldocument[divisors-]{divisors}
\externaldocument[limits-]{limits}
\externaldocument[varieties-]{varieties}
\externaldocument[topologies-]{topologies}
\externaldocument[descent-]{descent}
\externaldocument[perfect-]{perfect}
\externaldocument[more-morphisms-]{more-morphisms}
\externaldocument[flat-]{flat}
\externaldocument[groupoids-]{groupoids}
\externaldocument[more-groupoids-]{more-groupoids}
\externaldocument[etale-]{etale}
\externaldocument[chow-]{chow}
\externaldocument[intersection-]{intersection}
\externaldocument[pic-]{pic}
\externaldocument[adequate-]{adequate}
\externaldocument[dualizing-]{dualizing}
\externaldocument[duality-]{duality}
\externaldocument[discriminant-]{discriminant}
\externaldocument[local-cohomology-]{local-cohomology}
\externaldocument[curves-]{curves}
\externaldocument[resolve-]{resolve}
\externaldocument[models-]{models}
\externaldocument[pione-]{pione}
\externaldocument[etale-cohomology-]{etale-cohomology}
\externaldocument[proetale-]{proetale}
\externaldocument[crystalline-]{crystalline}
\externaldocument[spaces-]{spaces}
\externaldocument[spaces-properties-]{spaces-properties}
\externaldocument[spaces-morphisms-]{spaces-morphisms}
\externaldocument[decent-spaces-]{decent-spaces}
\externaldocument[spaces-cohomology-]{spaces-cohomology}
\externaldocument[spaces-limits-]{spaces-limits}
\externaldocument[spaces-divisors-]{spaces-divisors}
\externaldocument[spaces-over-fields-]{spaces-over-fields}
\externaldocument[spaces-topologies-]{spaces-topologies}
\externaldocument[spaces-descent-]{spaces-descent}
\externaldocument[spaces-perfect-]{spaces-perfect}
\externaldocument[spaces-more-morphisms-]{spaces-more-morphisms}
\externaldocument[spaces-flat-]{spaces-flat}
\externaldocument[spaces-groupoids-]{spaces-groupoids}
\externaldocument[spaces-more-groupoids-]{spaces-more-groupoids}
\externaldocument[bootstrap-]{bootstrap}
\externaldocument[spaces-pushouts-]{spaces-pushouts}
\externaldocument[groupoids-quotients-]{groupoids-quotients}
\externaldocument[spaces-more-cohomology-]{spaces-more-cohomology}
\externaldocument[spaces-simplicial-]{spaces-simplicial}
\externaldocument[formal-spaces-]{formal-spaces}
\externaldocument[restricted-]{restricted}
\externaldocument[spaces-resolve-]{spaces-resolve}
\externaldocument[formal-defos-]{formal-defos}
\externaldocument[defos-]{defos}
\externaldocument[cotangent-]{cotangent}
\externaldocument[examples-defos-]{examples-defos}
\externaldocument[algebraic-]{algebraic}
\externaldocument[examples-stacks-]{examples-stacks}
\externaldocument[stacks-sheaves-]{stacks-sheaves}
\externaldocument[criteria-]{criteria}
\externaldocument[artin-]{artin}
\externaldocument[quot-]{quot}
\externaldocument[stacks-properties-]{stacks-properties}
\externaldocument[stacks-morphisms-]{stacks-morphisms}
\externaldocument[stacks-limits-]{stacks-limits}
\externaldocument[stacks-cohomology-]{stacks-cohomology}
\externaldocument[stacks-perfect-]{stacks-perfect}
\externaldocument[stacks-introduction-]{stacks-introduction}
\externaldocument[stacks-more-morphisms-]{stacks-more-morphisms}
\externaldocument[stacks-geometry-]{stacks-geometry}
\externaldocument[moduli-]{moduli}
\externaldocument[moduli-curves-]{moduli-curves}
\externaldocument[examples-]{examples}
\externaldocument[exercises-]{exercises}
\externaldocument[guide-]{guide}
\externaldocument[desirables-]{desirables}
\externaldocument[coding-]{coding}
\externaldocument[obsolete-]{obsolete}
\externaldocument[fdl-]{fdl}
\externaldocument[index-]{index}

% Theorem environments.
%
\theoremstyle{plain}
\newtheorem{theorem}[subsection]{Theorem}
\newtheorem{proposition}[subsection]{Proposition}
\newtheorem{lemma}[subsection]{Lemma}

\theoremstyle{definition}
\newtheorem{definition}[subsection]{Definition}
\newtheorem{example}[subsection]{Example}
\newtheorem{exercise}[subsection]{Exercise}
\newtheorem{situation}[subsection]{Situation}

\theoremstyle{remark}
\newtheorem{remark}[subsection]{Remark}
\newtheorem{remarks}[subsection]{Remarks}

\numberwithin{equation}{subsection}

% Macros
%
\def\lim{\mathop{\rm lim}\nolimits}
\def\colim{\mathop{\rm colim}\nolimits}
\def\Spec{\mathop{\rm Spec}}
\def\Hom{\mathop{\rm Hom}\nolimits}
\def\Ext{\mathop{\rm Ext}\nolimits}
\def\SheafHom{\mathop{\mathcal{H}\!{\it om}}\nolimits}
\def\SheafExt{\mathop{\mathcal{E}\!{\it xt}}\nolimits}
\def\Sch{\textit{Sch}}
\def\Mor{\mathop{\rm Mor}\nolimits}
\def\Ob{\mathop{\rm Ob}\nolimits}
\def\Sh{\mathop{\textit{Sh}}\nolimits}
\def\NL{\mathop{N\!L}\nolimits}
\def\proetale{{pro\text{-}\acute{e}tale}}
\def\etale{{\acute{e}tale}}
\def\QCoh{\textit{QCoh}}
\def\Ker{\mathop{\rm Ker}}
\def\Im{\mathop{\rm Im}}
\def\Coker{\mathop{\rm Coker}}
\def\Coim{\mathop{\rm Coim}}

%
% Macros for moduli stacks/spaces
%
\def\QCohstack{\mathcal{QC}\!{\it oh}}
\def\Cohstack{\mathcal{C}\!{\it oh}}
\def\Spacesstack{\mathcal{S}\!{\it paces}}
\def\Quotfunctor{{\rm Quot}}
\def\Hilbfunctor{{\rm Hilb}}
\def\Curvesstack{\mathcal{C}\!{\it urves}}
\def\Polarizedstack{\mathcal{P}\!{\it olarized}}
\def\Complexesstack{\mathcal{C}\!{\it omplexes}}
% \Pic is the operator that assigns to X its picard group, usage \Pic(X)
% \Picardstack_{X/B} denotes the Picard stack of X over B
% \Picardfunctor_{X/B} denotes the Picard functor of X over B
\def\Pic{\mathop{\rm Pic}\nolimits}
\def\Picardstack{\mathcal{P}\!{\it ic}}
\def\Picardfunctor{{\rm Pic}}
\def\Deformationcategory{\mathcal{D}\!{\it ef}}


% OK, start here.
%
\begin{document}

\title{Algebraic Spaces}


\maketitle

\tableofcontents

\section{Introduction}
\label{section-introduction}

\noindent
This is where we define algebraic spaces and make some very elementary
observations. Algebraic spaces were first introduced by Mike Artin,
see \cite{ArtinI} and \cite{ArtinII}.





\section{General remarks}
\label{section-general}

\noindent
We work in a suitable big fppf site $\textit{Sch}_{fppf}$
as in Topologies, Definition \ref{topologies-definition-big-fppf-site}.
We will record elsewhere what changes if you change the big
fppf site (insert future reference here).

\medskip\noindent
We will always work relative to a base $S$ contained in $\textit{Sch}_{fppf}$.
And we will then work with the big fppf site $(\textit{Sch}/S)_{fppf}$,
see Topologies, Definition \ref{topologies-definition-big-small-fppf}.
The absolute case can be recovered by taking
$S = \text{Spec}(\mathbf{Z})$.

\medskip\noindent
Note that any fpqc covering is a family of universally effective
epimorphisms, see
Descent, Lemma \ref{fpqc-descent-lemma-fpqc-universal-effective-epimorphisms}.
Hence the topology on $\textit{Sch}_{fppf}$
is weaker than the canonical topology and all representable presheaves
are sheaves.







\section{Representable morphisms of presheaves}
\label{section-representable}

\noindent
Let $S$ be a scheme contained in $\textit{Sch}_{fppf}$.
Let $F, G : (\textit{Sch}/S)_{fppf}^{opp} \to \textit{Sets}$.
Let $a : F \to G$ be a representable transformation of functors, see
Categories,
Definition \ref{categories-definition-representable-map-presheaves}.
This means that for every
$U \in \text{Ob}((\textit{Sch}/S)_{fppf})$ and
any $\xi \in G(U)$ the fiber product $h_U \times_{\xi, G} F$ is representable.
Choose a representing object $V_\xi$ and an isomorphism
$h_{V_\xi} \to h_U \times_G F$.
By the Yoneda Lemma \ref{categories-lemma-yoneda}, the projection
$h_{V_\xi} \to h_U \times_G F \to h_U$ comes from a unique morphism
of schemes $a_\xi : V_\xi \to U$. Suggestively we could represent this by the
diagram
$$
\xymatrix{
V_\xi \ar@{~>}[r] \ar[d]_{a_\xi} & h_{V_\xi} \ar[d] \ar[r] & F \ar[d]^a \\
U \ar@{~>}[r] & h_U \ar[r]^\xi & G 
}
$$
where the squiggly arrows represent the Yoneda embedding.

\begin{definition}
\label{definition-relative-representable-property}
With $S$, and $a : F \to G$ representable as above.
Let $P$ be a property of morphisms of schemes preserved under any base change,
see Schemes, Definition \ref{schemes-definition-preserved-by-base-change}.
We say that {\it $a$ has property $P$} if for every
$U \in \text{Ob}((\textit{Sch}/S)_{fppf})$ and
any $\xi \in G(U)$ the resulting morphism of schemes
$V_\xi \to U$ has property $P$.
\end{definition}

\noindent
It is important to note that we will only use this definition for
properties of morphisms that are stable under base change. This is
not because the definition doens't make sense otherwise; rather it
is because we may want to give a different definition which is
better suited to the definition we have in mind.

\begin{lemma}
\label{lemma-representable-diagonal}
Let $S$ be a scheme contained in $\textit{Sch}_{fppf}$.
Let $F$ be a presheaf of sets on $(\textit{Sch}/S)_{fppf}$.
The following are equivalent:
\begin{enumerate}
\item The diagonal $F \to F \times F$ is representable.
\item For every scheme $U$ over $S$,
$U/S \in \text{Ob}((\textit{Sch}/S)_{fppf})$
and any $\xi \in F(U)$ the map $\xi : h_U \to F$ is representable.
\end{enumerate}
\end{lemma}

\begin{proof}
This is completely formal, see
Categories, Lemma \ref{categories-lemma-representable-diagonal}.
It depends only on the fact that the category $(\textit{Sch}/S)_{fppf}$
has products of pairs of objects and fibre products, see
Topologies, Lemma \ref{topologies-lemma-fibre-products-fppf}.
\end{proof}










\section{Algebraic spaces}
\label{section-algebraic-spaces}


\begin{definition}
\label{definition-algebraic-space}
Let $S$ be a scheme contained in $\textit{Sch}_{fppf}$.
An {\it algebraic space over $S$} is a presheaf
$$
F : (\textit{Sch}/S)^{opp}_{fppf} \longrightarrow \textit{Sets}
$$
with the following properties
\begin{enumerate}
\item The presheaf $F$ is a sheaf.
\item The diagonal morphism $F  \to F \times F$ is representable.
\item There exists a scheme $U \in \text{Ob}(\textit{Sch}_{fppf})$
and a map $h_U \to F$ which is surjective, and etale.
\end{enumerate}
\end{definition}

\noindent
There are two differences with the ``usual'' definition, for example the
definition in Knutson's book \cite{Kn}.

\medskip\noindent
The first is that we require $F$ to be a sheaf in the fppf topology.
Our reason for doing this is that many natural examples
of algebraic spaces satsify the sheaf condition for the fppf topology
(and even the fpqc topology). It somehow seems to us that the fppf topology
is the natural topology to work with. In the end the resulting category
of algebraic spaces ends up being ``the same''. Namely, allthough the actual
sheaves $F$ being considered will actually be different
in the end the category of algebraic spaces defined using sheaves
in the etale topology is equivalent the the category we define here.
This will be clear later when we introduce presentations (insert
future reference here).

\medskip\noindent
The second is that we only require the diagonal map for $F$ to be
representable, whereas in \cite{Kn} it is required that it also
be quasi-compact. If $F = h_U$ for some scheme $U$ over $S$
this corresponds to the condition that $S$ be quasi-separated.
Our point of view is to try to prove a certain
number of the results that follow only assuming that the diagonal
of $F$ be representable, and simply add an addition hypothesis wherever
this is necessary. In any case it has the pleasing consquence that
the following lemma is true.

\begin{lemma}
\label{lemma-scheme-is-space}
A scheme is an algebraic space. More precisely,
given $T \in \text{Ob}((\textit{Sch}/S)_{fppf})$ the
functor $h_T$ is an algebraic space.
\end{lemma}

\begin{proof}
The functor $h_T$ is a sheaf by our remarks in Section \ref{section-general}.
The diagonal $h_T \to h_T \times h_T = h_{T \times T}$ is
representable because $(\textit{Sch}/S)_{fppf}$ has fibre products.
The identity map $h_T \to h_T$ is surjective etale.
\end{proof}




\section{Other chapters}

\begin{multicols}{2}
\begin{enumerate}
\item \hyperref[introduction-section-phantom]{Introduction}
\item \hyperref[conventions-section-phantom]{Conventions}
\item \hyperref[sets-section-phantom]{Set Theory}
\item \hyperref[categories-section-phantom]{Categories}
\item \hyperref[topology-section-phantom]{Topology}
\item \hyperref[sheaves-section-phantom]{Sheaves on Spaces}
\item \hyperref[algebra-section-phantom]{Commutative Algebra}
\item \hyperref[sites-section-phantom]{Sites and Sheaves}
\item \hyperref[homology-section-phantom]{Homological Algebra}
\item \hyperref[derived-section-phantom]{Derived Categories}
\item \hyperref[more-algebra-section-phantom]{More Algebra}
\item \hyperref[simplicial-section-phantom]{Simplicial Methods}
\item \hyperref[modules-section-phantom]{Sheaves of Modules}
\item \hyperref[sites-modules-section-phantom]{Modules on Sites}
\item \hyperref[injectives-section-phantom]{Injectives}
\item \hyperref[cohomology-section-phantom]{Cohomology of Sheaves}
\item \hyperref[sites-cohomology-section-phantom]{Cohomology on Sites}
\item \hyperref[hypercovering-section-phantom]{Hypercoverings}
\item \hyperref[schemes-section-phantom]{Schemes}
\item \hyperref[constructions-section-phantom]{Constructions of Schemes}
\item \hyperref[properties-section-phantom]{Properties of Schemes}
\item \hyperref[morphisms-section-phantom]{Morphisms of Schemes}
\item \hyperref[coherent-section-phantom]{Coherent Cohomology}
\item \hyperref[divisors-section-phantom]{Divisors}
\item \hyperref[limits-section-phantom]{Limits of Schemes}
\item \hyperref[varieties-section-phantom]{Varieties}
\item \hyperref[chow-section-phantom]{Chow Homology}
\item \hyperref[topologies-section-phantom]{Topologies on Schemes}
\item \hyperref[descent-section-phantom]{Descent}
\item \hyperref[more-morphisms-section-phantom]{More on Morphisms}
\item \hyperref[flat-section-phantom]{More on Flatness}
\item \hyperref[groupoids-section-phantom]{Groupoid Schemes}
\item \hyperref[more-groupoids-section-phantom]{More on Groupoid Schemes}
\item \hyperref[etale-section-phantom]{\'Etale Morphisms of Schemes}
\item \hyperref[etale-cohomology-section-phantom]{\'Etale Cohomology}
\item \hyperref[spaces-section-phantom]{Algebraic Spaces}
\item \hyperref[spaces-properties-section-phantom]{Properties of Algebraic Spaces}
\item \hyperref[spaces-morphisms-section-phantom]{Morphisms of Algebraic Spaces}
\item \hyperref[spaces-topologies-section-phantom]{Topologies on Algebraic Spaces}
\item \hyperref[spaces-descent-section-phantom]{Descent and Algebraic Spaces}
\item \hyperref[spaces-more-morphisms-section-phantom]{More on Morphisms of Spaces}
\item \hyperref[quot-section-phantom]{Quot and Hilbert Spaces}
\item \hyperref[stacks-section-phantom]{Stacks}
\item \hyperref[spaces-groupoids-section-phantom]{Groupoids in Algebraic Spaces}
\item \hyperref[spaces-more-groupoids-section-phantom]{More on Groupoids in Spaces}
\item \hyperref[bootstrap-section-phantom]{Bootstrap}
\item \hyperref[examples-stacks-section-phantom]{Examples of Stacks}
\item \hyperref[groupoids-quotients-section-phantom]{Quotients of Groupoids}
\item \hyperref[algebraic-section-phantom]{Algebraic Stacks}
\item \hyperref[criteria-section-phantom]{Criteria for Representability}
\item \hyperref[stacks-properties-section-phantom]{Properties of Algebraic Stacks}
\item \hyperref[stacks-morphisms-section-phantom]{Morphisms of Algebraic Stacks}
\item \hyperref[examples-section-phantom]{Examples}
\item \hyperref[exercises-section-phantom]{Exercises}
\item \hyperref[guide-section-phantom]{Guide to Literature}
\item \hyperref[desirables-section-phantom]{Desirables}
\item \hyperref[coding-section-phantom]{Coding Style}
\item \hyperref[fdl-section-phantom]{GNU Free Documentation License}
\item \hyperref[index-section-phantom]{Auto Generated Index}
\end{enumerate}
\end{multicols}


\bibliography{my}
\bibliographystyle{alpha}

\end{document}
