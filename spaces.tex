\documentclass{amsart}

% The following AMS packages are automatically loaded with amsart 
% documentclass:
%\usepackage{amsmath}
%\usepackage{amssymb}
%\usepackage{amsthm}

% For commutative diagrams you can use
% \usepackage{amscd}
% but Jason prefers xypic
\usepackage[all]{xy}

% To put source file link in headers.
% Change "template.tex" to "this_filename.tex"
\usepackage{fancyhdr}
\pagestyle{fancy}
\lhead{}
\chead{}
\rhead{Source file: \url{src/algebraic.tex}}
\lfoot{}
\cfoot{\thepage}
\rfoot{}
\renewcommand{\headrulewidth}{0pt}
\renewcommand{\footrulewidth}{0pt}
\renewcommand{\headheight}{12pt}

% For cross-file-references
\usepackage{xr-hyper}

% Package for hypertext links:
\usepackage[colorlinks=true]{hyperref}
% For any local file, say "hello.tex" you want to refer to please use
% \externaldocument[hello-]{hello}
\externaldocument[conventions-]{conventions}
\externaldocument[categories-]{categories}
\externaldocument[hypercovering-]{hypercovering}
\externaldocument[schemes-]{schemes}
\externaldocument[desirables-]{desirables}
\externaldocument[fdl-]{fdl}

% The macro \autoref uses the macros \figurename, etc.
% We list the default values and we change some of them
% to start with a captial.
% Figure	\figurename
% Table		\tablename
% Part		\partname
% Appendix	\appendixname
% Equation	\equationname
% item		\Itemname
% \renewcommand{\Itemname}{Item}
\renewcommand{\Itemautorefname}{Item}
% chapter	\Chaptername
% \renewcommand{\Chaptername}{Chapter}
% \renewcommand{\Chapterautorefname}{Chapter}
% section	\sectionname
\renewcommand{\sectionname}{Section}
\renewcommand{\sectionautorefname}{Section}
% subsection	\subsectionname
\renewcommand{\subsectionname}{Subsection}
\renewcommand{\subsectionautorefname}{Subsection}
% subsubsection	\subsubsectionname
\renewcommand{\subsubsectionname}{Subsubsection}
\renewcommand{\subsubsectionautorefname}{Subsubsection}
% paragraph	\paragraphname
\renewcommand{\paragraphname}{Paragraph}
\renewcommand{\paragraphautorefname}{Paragraph}
% footnote	\Hfootnotename
% \renewcommand{\Hfootnotename}{Footnote}
\renewcommand{\Hfootnoteautorefname}{Footnote}
% Equation	\AMSname
% Theorem	\theoremname


% Theorem environments.
%
\newtheorem{theorem}{Theorem}[subsection]
\newtheorem{proposition}[theorem]{Proposition}
\newtheorem{lemma}[theorem]{Lemma}

\theoremstyle{definition}
\newtheorem{definition}[theorem]{Definition}
\newtheorem{example}[theorem]{Example}
\newtheorem{exercise}[theorem]{Exercise}
\newtheorem{situation}[theorem]{Situation}

\theoremstyle{remark}
\newtheorem{remark}[theorem]{Remark}
\newtheorem{remarks}[theorem]{Remarks}

\numberwithin{equation}{subsection}


% OK, start here.
%
\begin{document}

\title{Algebraic spaces}

%\begin{abstract}
%\end{abstract}

\maketitle
\thispagestyle{fancy}

\tableofcontents

\section{Introduction}
\label{section-introduction}

\noindent
This is where we define algebraic spaces and make some very elementary
observations. Algebraic spaces were first introduced by Mike Artin,
see \cite{ArtinI} and \cite{ArtinII}.

\section{Definitions}
\label{section-definitions}

\subsection{Algebraic spaces}
\label{subsection-algebraic-spaces}

\noindent
FIXME.

\begin{definition}
An algebraic space is a stack $\mathcal{S}$ over $\text{Aff}$ such that
\begin{enumerate}
\item every fibre category is setlike, see Categories,
\autoref{categories-subsection-fibred-in-sets}, 
\item the diagonal morphism
$\Delta : \mathcal{S} \to \mathcal{S}\times\mathcal{S}$
is representable by schemes, see Schemes,
\autoref{schemes-subsection-definition-representable-by-schemes} and
\item there exists a stack $\mathcal{X}$ representable by a scheme, see
Schemes, \autoref{schemes-subsection-stack-representable-by-scheme}
and an \'etale surjective morphism $\mathcal{X} \to \mathcal{S}$,
see Schemes,
\autoref{schemes-definition-property-morphism-representable-by-schemes}.
\end{enumerate}
\end{definition}

\begin{remark}
\label{remark-definition-correct}
If you try to define some kind of more general algebraic space by requiring
only that the diagonal is representable by algebraic spaces, and that there is
a surjective etale morphism of an algebraic space onto $\mathcal{S}$, then 
you actually end up with the same notion.
(FIXME: internal references, proofs.)
\end{remark}

\subsection{Morphisms representable by algebraic spaces}
\label{subsection-morphism-representable-by-algebraic-spaces}

\noindent
Here is the formal definition. Please also see the informal discussion below.

\begin{definition}
\label{definition-representable-by-algebraic-spaces}
Let $f : \mathcal{X} \to \mathcal{Y}$ be a morphism of categories
fibred in groupoids over $\text{Aff}$. We say $f$ is representable by
algebraic spaces if for every stack $\mathcal{S}$ representable by a scheme
(see Schemes, Definition \ref{schemes-definition-representable-by-scheme}),
and every morphism $\mathcal{U} \to \mathcal{Y}$, the 2-fibre product
$\mathcal{S}\times_\mathcal{Y}\mathcal{X}$ is an algebraic space.
\end{definition}

\noindent
Informal discussion. Suppose that, with the notation of the definition,
$S$ represents $\mathcal{S}$. Suppose that $W$ is a scheme and that
$\text{Aff}/W \to \mathcal{S}\times_\mathcal{Y}\mathcal{X}$ is 
etale and surjective. According to
Schemes, Lemma \ref{schemes-lemma-morphism-stacks-representable-by-schemes}
we get a morphism of schemes $g : W \to S$ and a 2-commutative diagram
of stacks
$$
\xymatrix{
\text{Aff}/W \ar[d]^g \ar[r] &
\mathcal{S}\times_\mathcal{X}\mathcal{Y} \ar[d] \ar[r] &
\mathcal{Y} \ar[d] \\
\text{Aff}/S &
\mathcal{S} \ar[l]^j \ar[r] & \mathcal{X}
}
$$

\begin{definition}
\label{definition-property-morphism-representable-by-algebraic-spaces}
Let $P$ be a property of morphisms of schemes, that is etale local
on the source and such that if the morphism $f : X \to Y$ has property $P$,
then so does every base change of $f$. (FIXME: introduce base change.)
We say that a morphism of stacks $\mathcal{X}
\to \mathcal{Y}$ representable by algebraic spaces has property
$P$ if for every diagram as above the morphism of schemes
$g : W \to S$ has property $P$.
\end{definition}

\noindent
FIXME. Explain rationale behind this definition: what else could it be?

\section{Other chapters}

\begin{multicols}{2}
\begin{enumerate}
\item \hyperref[introduction-section-phantom]{Introduction}
\item \hyperref[conventions-section-phantom]{Conventions}
\item \hyperref[sets-section-phantom]{Set Theory}
\item \hyperref[categories-section-phantom]{Categories}
\item \hyperref[topology-section-phantom]{Topology}
\item \hyperref[sheaves-section-phantom]{Sheaves on Spaces}
\item \hyperref[algebra-section-phantom]{Commutative Algebra}
\item \hyperref[sites-section-phantom]{Sites and Sheaves}
\item \hyperref[homology-section-phantom]{Homological Algebra}
\item \hyperref[derived-section-phantom]{Derived Categories}
\item \hyperref[more-algebra-section-phantom]{More Algebra}
\item \hyperref[simplicial-section-phantom]{Simplicial Methods}
\item \hyperref[modules-section-phantom]{Sheaves of Modules}
\item \hyperref[sites-modules-section-phantom]{Modules on Sites}
\item \hyperref[injectives-section-phantom]{Injectives}
\item \hyperref[cohomology-section-phantom]{Cohomology of Sheaves}
\item \hyperref[sites-cohomology-section-phantom]{Cohomology on Sites}
\item \hyperref[hypercovering-section-phantom]{Hypercoverings}
\item \hyperref[schemes-section-phantom]{Schemes}
\item \hyperref[constructions-section-phantom]{Constructions of Schemes}
\item \hyperref[properties-section-phantom]{Properties of Schemes}
\item \hyperref[morphisms-section-phantom]{Morphisms of Schemes}
\item \hyperref[coherent-section-phantom]{Coherent Cohomology}
\item \hyperref[divisors-section-phantom]{Divisors}
\item \hyperref[limits-section-phantom]{Limits of Schemes}
\item \hyperref[varieties-section-phantom]{Varieties}
\item \hyperref[chow-section-phantom]{Chow Homology}
\item \hyperref[topologies-section-phantom]{Topologies on Schemes}
\item \hyperref[descent-section-phantom]{Descent}
\item \hyperref[more-morphisms-section-phantom]{More on Morphisms}
\item \hyperref[flat-section-phantom]{More on Flatness}
\item \hyperref[groupoids-section-phantom]{Groupoid Schemes}
\item \hyperref[more-groupoids-section-phantom]{More on Groupoid Schemes}
\item \hyperref[etale-section-phantom]{\'Etale Morphisms of Schemes}
\item \hyperref[etale-cohomology-section-phantom]{\'Etale Cohomology}
\item \hyperref[spaces-section-phantom]{Algebraic Spaces}
\item \hyperref[spaces-properties-section-phantom]{Properties of Algebraic Spaces}
\item \hyperref[spaces-morphisms-section-phantom]{Morphisms of Algebraic Spaces}
\item \hyperref[spaces-topologies-section-phantom]{Topologies on Algebraic Spaces}
\item \hyperref[spaces-descent-section-phantom]{Descent and Algebraic Spaces}
\item \hyperref[spaces-more-morphisms-section-phantom]{More on Morphisms of Spaces}
\item \hyperref[quot-section-phantom]{Quot and Hilbert Spaces}
\item \hyperref[stacks-section-phantom]{Stacks}
\item \hyperref[spaces-groupoids-section-phantom]{Groupoids in Algebraic Spaces}
\item \hyperref[spaces-more-groupoids-section-phantom]{More on Groupoids in Spaces}
\item \hyperref[bootstrap-section-phantom]{Bootstrap}
\item \hyperref[examples-stacks-section-phantom]{Examples of Stacks}
\item \hyperref[groupoids-quotients-section-phantom]{Quotients of Groupoids}
\item \hyperref[algebraic-section-phantom]{Algebraic Stacks}
\item \hyperref[criteria-section-phantom]{Criteria for Representability}
\item \hyperref[stacks-properties-section-phantom]{Properties of Algebraic Stacks}
\item \hyperref[stacks-morphisms-section-phantom]{Morphisms of Algebraic Stacks}
\item \hyperref[examples-section-phantom]{Examples}
\item \hyperref[exercises-section-phantom]{Exercises}
\item \hyperref[guide-section-phantom]{Guide to Literature}
\item \hyperref[desirables-section-phantom]{Desirables}
\item \hyperref[coding-section-phantom]{Coding Style}
\item \hyperref[fdl-section-phantom]{GNU Free Documentation License}
\item \hyperref[index-section-phantom]{Auto Generated Index}
\end{enumerate}
\end{multicols}


\bibliography{my}
\bibliographystyle{alpha}

\end{document}
