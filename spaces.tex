\IfFileExists{stacks-project.cls}{%
\documentclass{stacks-project}
}{%
\documentclass{amsart}
}

% The following AMS packages are automatically loaded with
% the amsart documentclass:
%\usepackage{amsmath}
%\usepackage{amssymb}
%\usepackage{amsthm}

% For dealing with references we use the comment environment
\usepackage{verbatim}
\newenvironment{reference}{\comment}{\endcomment}
%\newenvironment{reference}{}{}
\newenvironment{slogan}{\comment}{\endcomment}
\newenvironment{history}{\comment}{\endcomment}

% For commutative diagrams you can use
% \usepackage{amscd}
\usepackage[all]{xy}

% We use 2cell for 2-commutative diagrams.
\xyoption{2cell}
\UseAllTwocells

% To put source file link in headers.
% Change "template.tex" to "this_filename.tex"
% \usepackage{fancyhdr}
% \pagestyle{fancy}
% \lhead{}
% \chead{}
% \rhead{Source file: \url{template.tex}}
% \lfoot{}
% \cfoot{\thepage}
% \rfoot{}
% \renewcommand{\headrulewidth}{0pt}
% \renewcommand{\footrulewidth}{0pt}
% \renewcommand{\headheight}{12pt}

\usepackage{multicol}

% For cross-file-references
\usepackage{xr-hyper}

% Package for hypertext links:
\usepackage{hyperref}

% For any local file, say "hello.tex" you want to link to please
% use \externaldocument[hello-]{hello}
\externaldocument[introduction-]{introduction}
\externaldocument[conventions-]{conventions}
\externaldocument[sets-]{sets}
\externaldocument[categories-]{categories}
\externaldocument[topology-]{topology}
\externaldocument[sheaves-]{sheaves}
\externaldocument[sites-]{sites}
\externaldocument[stacks-]{stacks}
\externaldocument[fields-]{fields}
\externaldocument[algebra-]{algebra}
\externaldocument[brauer-]{brauer}
\externaldocument[homology-]{homology}
\externaldocument[derived-]{derived}
\externaldocument[simplicial-]{simplicial}
\externaldocument[more-algebra-]{more-algebra}
\externaldocument[smoothing-]{smoothing}
\externaldocument[modules-]{modules}
\externaldocument[sites-modules-]{sites-modules}
\externaldocument[injectives-]{injectives}
\externaldocument[cohomology-]{cohomology}
\externaldocument[sites-cohomology-]{sites-cohomology}
\externaldocument[dga-]{dga}
\externaldocument[dpa-]{dpa}
\externaldocument[hypercovering-]{hypercovering}
\externaldocument[schemes-]{schemes}
\externaldocument[constructions-]{constructions}
\externaldocument[properties-]{properties}
\externaldocument[morphisms-]{morphisms}
\externaldocument[coherent-]{coherent}
\externaldocument[divisors-]{divisors}
\externaldocument[limits-]{limits}
\externaldocument[varieties-]{varieties}
\externaldocument[topologies-]{topologies}
\externaldocument[descent-]{descent}
\externaldocument[perfect-]{perfect}
\externaldocument[more-morphisms-]{more-morphisms}
\externaldocument[flat-]{flat}
\externaldocument[groupoids-]{groupoids}
\externaldocument[more-groupoids-]{more-groupoids}
\externaldocument[etale-]{etale}
\externaldocument[chow-]{chow}
\externaldocument[intersection-]{intersection}
\externaldocument[pic-]{pic}
\externaldocument[adequate-]{adequate}
\externaldocument[dualizing-]{dualizing}
\externaldocument[duality-]{duality}
\externaldocument[discriminant-]{discriminant}
\externaldocument[local-cohomology-]{local-cohomology}
\externaldocument[curves-]{curves}
\externaldocument[resolve-]{resolve}
\externaldocument[models-]{models}
\externaldocument[pione-]{pione}
\externaldocument[etale-cohomology-]{etale-cohomology}
\externaldocument[proetale-]{proetale}
\externaldocument[crystalline-]{crystalline}
\externaldocument[spaces-]{spaces}
\externaldocument[spaces-properties-]{spaces-properties}
\externaldocument[spaces-morphisms-]{spaces-morphisms}
\externaldocument[decent-spaces-]{decent-spaces}
\externaldocument[spaces-cohomology-]{spaces-cohomology}
\externaldocument[spaces-limits-]{spaces-limits}
\externaldocument[spaces-divisors-]{spaces-divisors}
\externaldocument[spaces-over-fields-]{spaces-over-fields}
\externaldocument[spaces-topologies-]{spaces-topologies}
\externaldocument[spaces-descent-]{spaces-descent}
\externaldocument[spaces-perfect-]{spaces-perfect}
\externaldocument[spaces-more-morphisms-]{spaces-more-morphisms}
\externaldocument[spaces-flat-]{spaces-flat}
\externaldocument[spaces-groupoids-]{spaces-groupoids}
\externaldocument[spaces-more-groupoids-]{spaces-more-groupoids}
\externaldocument[bootstrap-]{bootstrap}
\externaldocument[spaces-pushouts-]{spaces-pushouts}
\externaldocument[groupoids-quotients-]{groupoids-quotients}
\externaldocument[spaces-more-cohomology-]{spaces-more-cohomology}
\externaldocument[spaces-simplicial-]{spaces-simplicial}
\externaldocument[formal-spaces-]{formal-spaces}
\externaldocument[restricted-]{restricted}
\externaldocument[spaces-resolve-]{spaces-resolve}
\externaldocument[formal-defos-]{formal-defos}
\externaldocument[defos-]{defos}
\externaldocument[cotangent-]{cotangent}
\externaldocument[examples-defos-]{examples-defos}
\externaldocument[algebraic-]{algebraic}
\externaldocument[examples-stacks-]{examples-stacks}
\externaldocument[stacks-sheaves-]{stacks-sheaves}
\externaldocument[criteria-]{criteria}
\externaldocument[artin-]{artin}
\externaldocument[quot-]{quot}
\externaldocument[stacks-properties-]{stacks-properties}
\externaldocument[stacks-morphisms-]{stacks-morphisms}
\externaldocument[stacks-limits-]{stacks-limits}
\externaldocument[stacks-cohomology-]{stacks-cohomology}
\externaldocument[stacks-perfect-]{stacks-perfect}
\externaldocument[stacks-introduction-]{stacks-introduction}
\externaldocument[stacks-more-morphisms-]{stacks-more-morphisms}
\externaldocument[stacks-geometry-]{stacks-geometry}
\externaldocument[moduli-]{moduli}
\externaldocument[moduli-curves-]{moduli-curves}
\externaldocument[examples-]{examples}
\externaldocument[exercises-]{exercises}
\externaldocument[guide-]{guide}
\externaldocument[desirables-]{desirables}
\externaldocument[coding-]{coding}
\externaldocument[obsolete-]{obsolete}
\externaldocument[fdl-]{fdl}
\externaldocument[index-]{index}

% Theorem environments.
%
\theoremstyle{plain}
\newtheorem{theorem}[subsection]{Theorem}
\newtheorem{proposition}[subsection]{Proposition}
\newtheorem{lemma}[subsection]{Lemma}

\theoremstyle{definition}
\newtheorem{definition}[subsection]{Definition}
\newtheorem{example}[subsection]{Example}
\newtheorem{exercise}[subsection]{Exercise}
\newtheorem{situation}[subsection]{Situation}

\theoremstyle{remark}
\newtheorem{remark}[subsection]{Remark}
\newtheorem{remarks}[subsection]{Remarks}

\numberwithin{equation}{subsection}

% Macros
%
\def\lim{\mathop{\rm lim}\nolimits}
\def\colim{\mathop{\rm colim}\nolimits}
\def\Spec{\mathop{\rm Spec}}
\def\Hom{\mathop{\rm Hom}\nolimits}
\def\Ext{\mathop{\rm Ext}\nolimits}
\def\SheafHom{\mathop{\mathcal{H}\!{\it om}}\nolimits}
\def\SheafExt{\mathop{\mathcal{E}\!{\it xt}}\nolimits}
\def\Sch{\textit{Sch}}
\def\Mor{\mathop{\rm Mor}\nolimits}
\def\Ob{\mathop{\rm Ob}\nolimits}
\def\Sh{\mathop{\textit{Sh}}\nolimits}
\def\NL{\mathop{N\!L}\nolimits}
\def\proetale{{pro\text{-}\acute{e}tale}}
\def\etale{{\acute{e}tale}}
\def\QCoh{\textit{QCoh}}
\def\Ker{\mathop{\rm Ker}}
\def\Im{\mathop{\rm Im}}
\def\Coker{\mathop{\rm Coker}}
\def\Coim{\mathop{\rm Coim}}

%
% Macros for moduli stacks/spaces
%
\def\QCohstack{\mathcal{QC}\!{\it oh}}
\def\Cohstack{\mathcal{C}\!{\it oh}}
\def\Spacesstack{\mathcal{S}\!{\it paces}}
\def\Quotfunctor{{\rm Quot}}
\def\Hilbfunctor{{\rm Hilb}}
\def\Curvesstack{\mathcal{C}\!{\it urves}}
\def\Polarizedstack{\mathcal{P}\!{\it olarized}}
\def\Complexesstack{\mathcal{C}\!{\it omplexes}}
% \Pic is the operator that assigns to X its picard group, usage \Pic(X)
% \Picardstack_{X/B} denotes the Picard stack of X over B
% \Picardfunctor_{X/B} denotes the Picard functor of X over B
\def\Pic{\mathop{\rm Pic}\nolimits}
\def\Picardstack{\mathcal{P}\!{\it ic}}
\def\Picardfunctor{{\rm Pic}}
\def\Deformationcategory{\mathcal{D}\!{\it ef}}


% OK, start here.
%
\begin{document}

\title{Algebraic Spaces}


\maketitle

\phantomsection
\label{section-phantom}

\tableofcontents

\section{Introduction}
\label{section-introduction}

\noindent
This is where we define algebraic spaces and make some very elementary
observations. Algebraic spaces were first introduced by Mike Artin,
see \cite{ArtinI} and \cite{ArtinII}.





\section{General remarks}
\label{section-general}

\noindent
We work in a suitable big fppf site $\textit{Sch}_{fppf}$
as in Topologies, Definition \ref{topologies-definition-big-fppf-site}.
So, if not explicitly stated otherwise all schemes will be objects
of $\textit{Sch}_{fppf}$.
We will record elsewhere what changes if you change the big
fppf site (insert future reference here).

\medskip\noindent
We will always work relative to a base $S$ contained in $\textit{Sch}_{fppf}$.
And we will then work with the big fppf site $(\textit{Sch}/S)_{fppf}$,
see Topologies, Definition \ref{topologies-definition-big-small-fppf}.
The absolute case can be recovered by taking
$S = \text{Spec}(\mathbf{Z})$.

\medskip\noindent
If $U, T$ are schemes over $S$, then we denote
$U(T)$ for the set of $T$-valued points {\it over} $S$.
In a formula: $U(T) = \text{Mor}_S(T, U)$.

\medskip\noindent
Note that any fpqc covering is a family of universally effective
epimorphisms, see
Descent, Lemma \ref{fpqc-descent-lemma-fpqc-universal-effective-epimorphisms}.
Hence the topology on $\textit{Sch}_{fppf}$
is weaker than the canonical topology and all representable presheaves
are sheaves.







\section{Representable morphisms of presheaves}
\label{section-representable}

\noindent
Let $S$ be a scheme contained in $\textit{Sch}_{fppf}$.
Let $F, G : (\textit{Sch}/S)_{fppf}^{opp} \to \textit{Sets}$.
Let $a : F \to G$ be a representable transformation of functors, see
Categories,
Definition \ref{categories-definition-representable-map-presheaves}.
This means that for every
$U \in \text{Ob}((\textit{Sch}/S)_{fppf})$ and
any $\xi \in G(U)$ the fiber product $h_U \times_{\xi, G} F$ is representable.
Choose a representing object $V_\xi$ and an isomorphism
$h_{V_\xi} \to h_U \times_G F$.
By the Yoneda lemma, see Categories, Lemma \ref{categories-lemma-yoneda},
the projection $h_{V_\xi} \to h_U \times_G F \to h_U$ comes from a unique
morphism of schemes $a_\xi : V_\xi \to U$.
Suggestively we could represent this by the diagram
$$
\xymatrix{
V_\xi \ar@{~>}[r] \ar[d]_{a_\xi} & h_{V_\xi} \ar[d] \ar[r] & F \ar[d]^a \\
U \ar@{~>}[r] & h_U \ar[r]^\xi & G
}
$$
where the squiggly arrows represent the Yoneda embedding.
Here are some lemmas about this notion that work in great generality.

\begin{lemma}
\label{lemma-morphism-schemes-gives-representable-transformation}
Let $S$, $X$, $Y$ be objects of $\textit{Sch}_{fppf}$.
Let $f : X \to Y$ be a morphism of schemes.
Then
$$
h_f : h_X \longrightarrow h_Y
$$
is a representable transformation of functors.
\end{lemma}

\begin{proof}
This is entirely formal and relies only on the fact that
the category $(\textit{Sch}/S)_{fppf}$ has fibre products.
\end{proof}

\begin{lemma}
\label{lemma-composition-representable-transformations}
Let $S$ be a scheme contained in $\textit{Sch}_{fppf}$.
Let $F, G, H : (\textit{Sch}/S)_{fppf}^{opp} \to \textit{Sets}$.
Let $a : F \to G$, $b : G \to H$ be representable transformations of functors.
Then
$$
b \circ a : F \longrightarrow H
$$
is a representable transformation of functors.
\end{lemma}

\begin{proof}
This is entirely formal and works in any category.
\end{proof}

\begin{lemma}
\label{lemma-base-change-representable-transformations}
Let $S$ be a scheme contained in $\textit{Sch}_{fppf}$.
Let $F, G, H : (\textit{Sch}/S)_{fppf}^{opp} \to \textit{Sets}$.
Let $a : F \to G$ be a representable transformations of functors.
Let $b : H \to G$ be any transformation of functors.
Consider the fibre product diagram
$$
\xymatrix{
H \times_{b, G, a} F \ar[r]_-{b'} \ar[d]_{a'} & F \ar[d]^a \\
H \ar[r]^b & G
}
$$
Then the base change $a'$ is a representable transformation of functors.
\end{lemma}

\begin{proof}
This is entirely formal and works in any category.
\end{proof}

\begin{lemma}
\label{lemma-product-representable-transformations}
Let $S$ be a scheme contained in $\textit{Sch}_{fppf}$.
Let $F_i, G_i : (\textit{Sch}/S)_{fppf}^{opp} \to \textit{Sets}$, $i = 1, 2$.
Let $a_i : F_i \to G_i$, $i = 1, 2$
be representable transformations of functors.
Then
$$
a_1 \times a_2 : F_1 \times F_2 \longrightarrow G_1 \times G_2
$$
is a representable transformation of functors.
\end{lemma}

\begin{proof}
Write $a_1 \times a_2$ as the composition
$F_1 \times F_2 \to G_1 \times F_2 \to G_1 \times G_2$.
The first arrow is the base change of $a_1$ by the map
$G_1 \times F_2 \to G_1$, and the second arrow
is the base change of $a_2$ by the map
$G_1 \times G_2 \to G_2$. Hence this lemma is a formal
consequence of Lemmas \ref{lemma-composition-representable-transformations}
and \ref{lemma-base-change-representable-transformations}.
\end{proof}














\section{Lists of useful properties of morphisms of schemes}
\label{section-lists}

\noindent
For ease of reference we list in the following remarks the
properties of morphisms which possess some of the properties
required of them in later results.

\begin{remark}
\label{remark-list-properties-stable-base-change}
Here is a list of properties/types of morphisms
which are {\it stable under arbitrary base change}:
\begin{enumerate}
\item closed, open, and locally closed immersions, see
Schemes, Lemma \ref{schemes-lemma-base-change-immersion},
\item quasi-compact, see
Schemes, Lemma \ref{schemes-lemma-quasi-compact-preserved-base-change},
\item universally closed, see
Schemes, Definition \ref{schemes-definition-universally-closed},
\item (quasi-)separated, see
Schemes, Lemma \ref{schemes-lemma-separated-permanence},
\item surjective, see
Morphisms, Lemma \ref{morphisms-lemma-base-change-surjective},
\item radicial (or universally injective), see
Morphisms, Lemma \ref{morphisms-lemma-radicial-universally-injective},
\item affine, see
Morphisms, Lemma \ref{morphisms-lemma-base-change-affine},
\item quasi-affine, see
Morphisms, Lemma \ref{morphisms-lemma-base-change-quasi-affine},
\item (locally) of finite type, see
Morphisms, Lemma \ref{morphisms-lemma-base-change-finite-type},
\item (locally) quasi-finite, see
Morphisms, Lemma \ref{morphisms-lemma-base-change-quasi-finite},
\item (locally) of finite presentation, see
Morphisms, Lemma \ref{morphisms-lemma-base-change-finite-presentation},
\item locally of finite type of relative dimension $d$, see
Morphisms, Lemma \ref{morphisms-lemma-base-change-relative-dimension-d},
\item universally open, see
Morphisms, Definition \ref{morphisms-definition-open},
\item flat, see
Morphisms, Lemma \ref{morphisms-lemma-base-change-flat},
\item syntomic, see
Morphisms, Lemma \ref{morphisms-lemma-base-change-syntomic},
\item smooth, see
Morphisms, Lemma \ref{morphisms-lemma-base-change-smooth},
\item unramified, see
Morphisms, Lemma \ref{morphisms-lemma-base-change-unramified},
\item etale, see
Morphisms, Lemma \ref{morphisms-lemma-base-change-etale},
\item proper, see
Morphisms, Lemma \ref{morphisms-lemma-base-change-proper},
\item H-projective, see
Morphisms, Lemma \ref{morphisms-lemma-H-projective-base-change},
\item (locally) projective, see
Morphisms, Lemma \ref{morphisms-lemma-base-change-projective},
\item finite or integral, see
Morphisms, Lemma \ref{morphisms-lemma-base-change-finite},
\item finite locally free, see
Morphisms, Lemma \ref{morphisms-lemma-base-change-finite-locally-free}.
\end{enumerate}
Add more as needed.
\end{remark}

\begin{remark}
\label{remark-list-properties-stable-composition}
Of the properties of morphisms which are stable under base change
(as listed in
Remark \ref{remark-list-properties-stable-base-change})
the following are also {\it stable under compositions}:
\begin{enumerate}
\item closed, open and locally closed immersions, see
Schemes, Lemma \ref{schemes-lemma-composition-immersion},
\item quasi-compact, see
Schemes, Lemma \ref{schemes-lemma-composition-quasi-compact},
\item universally closed, see
Schemes, Definition \ref{morphisms-lemma-composition-proper},
\item (quasi-)separated, see
Schemes, Lemma \ref{schemes-lemma-separated-permanence},
\item surjective, see
Morphisms, Lemma \ref{morphisms-lemma-composition-surjective},
\item radicial (or universally injective), see
Morphisms, Lemma \ref{morphisms-lemma-composition-radicial},
\item affine, see
Morphisms, Lemma \ref{morphisms-lemma-composition-affine},
\item quasi-affine, see
Morphisms, Lemma \ref{morphisms-lemma-composition-quasi-affine},
\item (locally) of finite type, see
Morphisms, Lemma \ref{morphisms-lemma-composition-finite-type},
\item (locally) quasi-finite, see
Morphisms, Lemma \ref{morphisms-lemma-composition-quasi-finite},
\item (locally) of finite presentation, see
Morphisms, Lemma \ref{morphisms-lemma-composition-finite-presentation},
\item universally open, see
Morphisms, Definition \ref{morphisms-lemma-composition-open},
\item flat, see
Morphisms, Lemma \ref{morphisms-lemma-composition-flat},
\item syntomic, see
Morphisms, Lemma \ref{morphisms-lemma-composition-syntomic},
\item smooth, see
Morphisms, Lemma \ref{morphisms-lemma-composition-smooth},
\item unramified, see
Morphisms, Lemma \ref{morphisms-lemma-composition-unramified},
\item etale, see
Morphisms, Lemma \ref{morphisms-lemma-composition-etale},
\item proper, see
Morphisms, Lemma \ref{morphisms-lemma-composition-proper},
\item H-projective, see
Morphisms, Lemma \ref{morphisms-lemma-H-projective-composition},
\item finite or integral, see
Morphisms, Lemma \ref{morphisms-lemma-composition-finite},
\item finite locally free, see
Morphisms, Lemma \ref{morphisms-lemma-composition-finite-locally-free}.
\end{enumerate}
Add more as needed.
\end{remark}

\begin{remark}
\label{remark-list-properties-fpqc-local-base}
Of the properties mentioned which are stable under base change
(as listed in Remark \ref{remark-list-properties-stable-base-change})
the following are also {\it fpqc local on the base}
(and a fortiori fppf local on the base):
\begin{enumerate}
\item for immersions we have this for
\begin{enumerate}
\item closed immersions, see
Descent, Lemma \ref{fpqc-descent-lemma-descending-property-closed-immersion},
\item open immersions see
Descent, Lemma \ref{fpqc-descent-lemma-descending-property-open-immersion}, and
\item quasi-compact immersions, see
Descent,
Lemma \ref{fpqc-descent-lemma-descending-property-quasi-compact-immersion},
\end{enumerate}
\item quasi-compact, see
Descent, Lemma \ref{fpqc-descent-lemma-descending-property-quasi-compact},
\item universally closed, see
Descent,
Definition \ref{fpqc-descent-lemma-descending-property-universally-closed},
\item (quasi-)separated, see
Descent, Lemmas
\ref{fpqc-descent-lemma-descending-property-quasi-separated}, and
\ref{fpqc-descent-lemma-descending-property-separated},
\item surjective, see
Descent, Lemma \ref{fpqc-descent-lemma-descending-property-surjective},
\item radicial (or universally injective), see
Descent, Lemma \ref{fpqc-descent-lemma-descending-property-radicial},
\item affine, see
Descent, Lemma \ref{fpqc-descent-lemma-descending-property-affine},
\item quasi-affine, see
Descent, Lemma \ref{fpqc-descent-lemma-descending-property-quasi-affine},
\item (locally) of finite type, see
Descent,
Lemmas \ref{fpqc-descent-lemma-descending-property-locally-finite-type}, and
\ref{fpqc-descent-lemma-descending-property-finite-type},
\item (locally) quasi-finite, see
Descent, Lemma \ref{fpqc-descent-lemma-descending-property-quasi-finite},
\item (locally) of finite presentation, see
Descent, Lemmas
\ref{fpqc-descent-lemma-descending-property-locally-finite-presentation}, and
\ref{fpqc-descent-lemma-descending-property-finite-presentation},
\item locally of finite type of relative dimension $d$, see
Descent,
Lemma \ref{fpqc-descent-lemma-descending-property-relative-dimension-d},
\item universally open, see
Descent, Lemma \ref{fpqc-descent-lemma-descending-property-universally-open},
\item flat, see
Descent, Lemma \ref{fpqc-descent-lemma-descending-property-flat},
\item syntomic, see
Descent, Lemma \ref{fpqc-descent-lemma-descending-property-syntomic},
\item smooth, see
Descent, Lemma \ref{fpqc-descent-lemma-descending-property-smooth},
\item unramified, see
Descent, Lemma \ref{fpqc-descent-lemma-descending-property-unramified},
\item etale, see
Descent, Lemma \ref{fpqc-descent-lemma-descending-property-etale},
\item proper, see
Descent, Lemma \ref{fpqc-descent-lemma-descending-property-proper},
\item finite or integral, see
Descent, Lemma \ref{fpqc-descent-lemma-descending-property-finite},
\item finite locally free, see
Descent,
Lemma \ref{fpqc-descent-lemma-descending-property-finite-locally-free}.
\end{enumerate}
Add more here as needed.
\end{remark}








\section{Properties of representable morphisms of presheaves}
\label{section-representable-properties}

\noindent
Here is the definition that makes this work.

\begin{definition}
\label{definition-relative-representable-property}
With $S$, and $a : F \to G$ representable as above.
Let $P$ be a property of morphisms of schemes which
\begin{enumerate}
\item is preserved under any base change,
see Schemes, Definition \ref{schemes-definition-preserved-by-base-change},
and
\item is fppf local on the base, see
Descent, Definition \ref{fpqc-descent-definition-property-local}.
\end{enumerate}
In this case we say that $a$ has {\it property $P$} if for every
$U \in \text{Ob}((\textit{Sch}/S)_{fppf})$ and
any $\xi \in G(U)$ the resulting morphism of schemes
$V_\xi \to U$ has property $P$.
\end{definition}

\noindent
It is important to note that we will only use this definition for
properties of morphisms that are stable under base change, and
local in the fppf topology on the base. This is
not because the definition doesn't make sense otherwise; rather it
is because we may want to give a different definition which is
better suited to the property we have in mind.

\medskip\noindent
Here is a sanity check.

\begin{lemma}
\label{lemma-morphism-schemes-gives-representable-transformation-property}
Let $S$, $X$, $Y$ be objects of $\textit{Sch}_{fppf}$.
Let $f : X \to Y$ be a morphism of schemes.
Let $\mathcal{P}$ be as in
Definition \ref{definition-relative-representable-property}.
Then $h_X \longrightarrow h_Y$ has propery $\mathcal{P}$ if
and only if $f$ has property $\mathcal{P}$.
\end{lemma}

\begin{proof}
Note that the lemma makes sense by
Lemma \ref{lemma-morphism-schemes-gives-representable-transformation}.
Proof omitted.
\end{proof}

\begin{lemma}
\label{lemma-composition-representable-transformations-property}
Let $S$ be a scheme contained in $\textit{Sch}_{fppf}$.
Let $F, G, H : (\textit{Sch}/S)_{fppf}^{opp} \to \textit{Sets}$.
Let $\mathcal{P}$ be a property as in
Definition \ref{definition-relative-representable-property}
which is stable under composition.
Let $a : F \to G$, $b : G \to H$ be representable transformations of functors.
If $a$ and $b$ have property $\mathcal{P}$ so does
$b \circ a : F \longrightarrow H$.
\end{lemma}

\begin{proof}
Note that the lemma makes sense by
Lemma \ref{lemma-composition-representable-transformations}.
Proof omitted.
\end{proof}

\begin{lemma}
\label{lemma-base-change-representable-transformations-property}
Let $S$ be a scheme contained in $\textit{Sch}_{fppf}$.
Let $F, G, H : (\textit{Sch}/S)_{fppf}^{opp} \to \textit{Sets}$.
Let $\mathcal{P}$ be a property as in
Definition \ref{definition-relative-representable-property}.
Let $a : F \to G$ be a representable transformations of functors.
Let $b : H \to G$ be any transformation of functors.
Consider the fibre product diagram
$$
\xymatrix{
H \times_{b, G, a} F \ar[r]_-{b'} \ar[d]_{a'} & F \ar[d]^a \\
H \ar[r]^b & G
}
$$
If $a$ has property $\mathcal{P}$ then also the base change $a'$
has property $\mathcal{P}$.
\end{lemma}

\begin{proof}
Note that the lemma makes sense by
Lemma \ref{lemma-base-change-representable-transformations}.
Proof omitted.
\end{proof}

\begin{lemma}
\label{lemma-product-representable-transformations-property}
Let $S$ be a scheme contained in $\textit{Sch}_{fppf}$.
Let $F_i, G_i : (\textit{Sch}/S)_{fppf}^{opp} \to \textit{Sets}$,
$i = 1, 2$.
Let $a_i : F_i \to G_i$, $i = 1, 2$ be representable transformations
of functors.
Let $\mathcal{P}$ be a property as in
Definition \ref{definition-relative-representable-property}
which is stable under composition.
If $a_1$ and $a_2$ have property $\mathcal{P}$ so does
$a_1 \times a_2 : F_1 \times F_2 \longrightarrow G_1 \times G_2$.
\end{lemma}

\begin{proof}
Note that the lemma makes sense by
Lemma \ref{lemma-product-representable-transformations}.
Proof omitted.
\end{proof}

\noindent
Here is a characterization of those functors for which the
diagonal is representable.

\begin{lemma}
\label{lemma-representable-diagonal}
Let $S$ be a scheme contained in $\textit{Sch}_{fppf}$.
Let $F$ be a presheaf of sets on $(\textit{Sch}/S)_{fppf}$.
The following are equivalent:
\begin{enumerate}
\item The diagonal $F \to F \times F$ is representable.
\item For every scheme $U$ over $S$,
$U/S \in \text{Ob}((\textit{Sch}/S)_{fppf})$
and any $\xi \in F(U)$ the map $\xi : h_U \to F$ is representable.
\end{enumerate}
\end{lemma}

\begin{proof}
This is completely formal, see
Categories, Lemma \ref{categories-lemma-representable-diagonal}.
It depends only on the fact that the category $(\textit{Sch}/S)_{fppf}$
has products of pairs of objects and fibre products, see
Topologies, Lemma \ref{topologies-lemma-fibre-products-fppf}.
\end{proof}

\noindent
In the situation of the lemma, for any morphism
$\xi : h_U \to F$ as in the lemma, it makes sense
to say that $\xi$ has property $P$, for any property
as in Definition \ref{definition-relative-representable-property}.
In particular this holds for $P = $ ``surjective'' and $P = $ ``etale'', see
Remark \ref{remark-list-properties-fpqc-local-base}
above. We will use these in the definition
of algebraic spaces below. 


















\section{Algebraic spaces}
\label{section-algebraic-spaces}

\noindent
Here is the definition.

\begin{definition}
\label{definition-algebraic-space}
Let $S$ be a scheme contained in $\textit{Sch}_{fppf}$.
An {\it algebraic space over $S$} is a presheaf
$$
F : (\textit{Sch}/S)^{opp}_{fppf} \longrightarrow \textit{Sets}
$$
with the following properties
\begin{enumerate}
\item The presheaf $F$ is a sheaf.
\item The diagonal morphism $F  \to F \times F$ is representable.
\item There exists a scheme $U \in \text{Ob}(\textit{Sch}_{fppf})$
and a map $h_U \to F$ which is surjective, and etale.
\end{enumerate}
\end{definition}

\noindent
There are two differences with the ``usual'' definition, for example the
definition in Knutson's book \cite{Kn}.

\medskip\noindent
The first is that we require $F$ to be a sheaf in the fppf topology.
One reason for doing this is that many natural examples
of algebraic spaces satisfy the sheaf condition for the fppf coverings
(and even for fpqc coverings). Also, one of the reasons that algebraic
spaces have been so useful is via Mike Artin's results on algebraic spaces.
Built into his method is a condition which garantees the result is
locally of finite presentation over $S$.
Combined it somehow seems to us that the fppf topology
is the natural topology to work with. In the end the resulting category
of algebraic spaces ends up being ``the same''. Namely, allthough the actual
sheaves $F$ being considered may be different, in the end the
category of algebraic spaces defined using sheaves in the etale topology
is equivalent the the category we define here. This will be clear later
when we introduce presentations (insert future reference here).

\medskip\noindent
The second is that we only require the diagonal map for $F$ to be
representable, whereas in \cite{Kn} it is required that it also
be quasi-compact. If $F = h_U$ for some scheme $U$ over $S$
this corresponds to the condition that $S$ be quasi-separated.
Our point of view is to try to prove a certain
number of the results that follow only assuming that the diagonal
of $F$ be representable, and simply add an addition hypothesis wherever
this is necessary. In any case it has the pleasing consquence that
the following lemma is true.

\begin{lemma}
\label{lemma-scheme-is-space}
A scheme is an algebraic space. More precisely,
given a scheme $T \in \text{Ob}((\textit{Sch}/S)_{fppf})$
the representable functor $h_T$ is an algebraic space.
\end{lemma}

\begin{proof}
The functor $h_T$ is a sheaf by our remarks in Section \ref{section-general}.
The diagonal $h_T \to h_T \times h_T = h_{T \times T}$ is
representable because $(\textit{Sch}/S)_{fppf}$ has fibre products.
The identity map $h_T \to h_T$ is surjective etale.
\end{proof}

\begin{definition}
\label{definition-morphism-algebraic-spaces}
Let $F$, $F'$ be algebraic spaces over $S$.
A {\it morphism $f : F \to F'$ of algebraic spaces over $S$}
is a transformation of functors from $F$ to $F'$.
\end{definition}

\noindent
The category of algebraic spaces over $S$ contains the category
$(\textit{Sch}/S)_{fppf}$ as a full subcategory via the
Yoneda embedding $T/S \mapsto h_T$. From now on we no longer distinghuish
between a scheme $T/S$ and the algebraic space it represents.
Thus when we say ``Let $f : T \to F$ be a morphism from the scheme
$T$ to the algebraic space $F$'', we mean that
$T \in \text{Ob}((\textit{Sch}/S)_{fppf})$, that $F$ is an
algebraic space over $S$, and that $f : h_T \to F$ is a morphism
of algebraic spaces over $S$.








\section{Glueing algebraic spaces}
\label{section-glueing-algebraic-spaces}

\noindent
In this section we really start abusing notation and not
distinguish between schemes and the spaces they represent.

\begin{lemma}
\label{lemma-representable-sheaf-coproduct-sheaves}
Let $S \in \text{Ob}(\textit{Sch}_{fppf})$.
Let $U \in \text{Ob}(\textit{Sch}/S)_{fppf}$.
Given a set $I$ and sheaves $F_i$ on $\text{Ob}(\textit{Sch}/S)_{fppf}$,
if $U \cong \coprod_{i\in I} F_i$
as sheaves, then each $F_i$ is representable by an open and closed
subscheme $U_i$ and $U \cong \coprod U_i$ as schemes.
\end{lemma}

\begin{proof}
By assumption this means there exists an fppf covering
$\{U_j \to U\}_{j \in J}$ such that each $U_j \to U$
factors through $F_{i(j)}$ for some $i(j) \in I$.
Denote $V_j = \text{Im}(U_j \to U)$.
This is an open of $U$ by
Morphisms, Lemma \ref{morphisms-lemma-fppf-open}, and
$\{U_j \to V_j\}$ is an fppf covering. Hence it follows that
$V_j \to U$ factors through $F_{i(j)}$ since $F_{i(j)}$ is
a subsheaf. It follows from $F_i \cap F_{i'} = \empty$, $i \not = i'$
that $V_j \cap V_{j'} = \emptyset$
unless $i(j) = i(j')$. Hence we can take
$U_i = \bigcup_{j,\ i(j) = i} V_j$ and everything is clear.
\end{proof}

\begin{lemma}
\label{lemma-algebraic-space-coproduct-sheaves}
Let $S \in \text{Ob}(\textit{Sch}_{fppf})$.
Let $F$ be an algebraic space over $S$.
Given a set $I$ and sheaves $F_i$ on
$\text{Ob}(\textit{Sch}/S)_{fppf}$,
if $F \cong \coprod_{i\in I} F_i$ as sheaves,
then each $F_i$ is an algebraic space over $S$.
\end{lemma}

\begin{proof}
It follows directly from the representability of
$F \to F \times F$ that each diagonal morphism
$F_i \to F_i \times F_i$ is representable.
Choose a scheme $U$ in $(\textit{Sch}/S)_{fppf}$ and a surjective
etale morphism $U \to \coprod F_i$ (this exist by hypothesis).
By considering the inverse image of $F_i$ we get a decomposition
of $U$ (as a sheaf) into a coproduct of sheaves.
By Lemma \ref{lemma-representable-sheaf-coproduct-sheaves}
we get correspondingly $U \cong \coprod U_i$.
Then it follows easily that $U_i \to F_i$ is surjective
and etale (from the corresponding property of $U \to F$).
\end{proof}

\noindent
The condition on the size of $I$ in the
following lemma may be ignored by those not worried about
set theoretic questions.

\begin{lemma}
\label{lemma-coproduct-algebraic-spaces}
Let $S \in \text{Ob}(\textit{Sch}_{fppf})$.
Suppose given a set $I$ and algebraic spaces $F_i$, $i \in I$.
Then $F = \coprod_{i \in I} F_i$ is an algebraic space
provided $I$ is not too large: for example given
surjective etale morphisms $U_i \to F_i$ such that
$\coprod U_i$ is isomorphic to an object of $(\textit{Sch}/S)_{fppf}$,
then $F$ is an algebraic space.
\end{lemma}

\begin{proof}
By construction $F$ is a sheaf. We omit the verification that the
diagonal morphism of $F$ is representable. Finally, if $U$ is an
object of $(\textit{Sch}/S)_{fppf}$ isomorphic to $\coprod_{i \in I} U_i$
then it is straightforward to verify that the resulting map
$U \to \coprod F_i$ is surjective and etale.
\end{proof}

\noindent
Here is the analogue of Schemes, Lemma \ref{schemes-lemma-glue-functors}.

\begin{lemma}
\label{lemma-glueing-algebraic-spaces}
Let $S \in \text{Ob}(\textit{Sch}_{fppf})$.
Let $F$ be a presheaf of sets on $(\textit{Sch}/S)_{fppf}$.
Assume
\begin{enumerate}
\item $F$ is a sheaf,
\item there exists an index set $I$
and subfunctors $F_i \subset F$ such that
\begin{enumerate}
\item $\coprod F_i$ is an algebraic space\footnote{
This basically just means each $F_i$ is an algebraic space, see
Lemmas \ref{lemma-algebraic-space-coproduct-sheaves}
and \ref{lemma-coproduct-algebraic-spaces}.},
\item each $F_i \to F$ is a representable,
\item each $F_i \to F$ is an open immersion (see
Definition \ref{definition-relative-representable-property} and
Remark \ref{remark-list-properties-fpqc-local-base}),
and
\item the map of sheaves $\coprod F_i \to F$ is surjective.
\end{enumerate}
\end{enumerate}
Then $F$ is an algebraic space.
\end{lemma}

\begin{proof}
Let $T$, $T'$ be objects of $(\textit{Sch}/S)_{fppf}$.
Let $T \to F$, $T' \to F$ morphisms.
The assumptions imply that there exists an open covering
$T = \bigcup V_i$ such that $V_i = T \times_F F_i$.
Note that this in particular implies that
$\coprod F_i \to F$ is surjective in the Zariski topology!
Also write similarly $T' = \bigcup V'_i$ with $V'_i = T' \times_F F_i$.

\medskip\noindent
To show that the diagonal $F \to F \times F$ is representable
we have to show that $G = T \times_F T'$ is representable.
Consider the subfunctors $G_i = G \times_F F_i$.
Note that $G_i = V_i \times_{F_i} V'_i$, and hence is representable
as $F_i$ is an algebraic space.
By the above the $G_i$ form a Zariski covering of $F$.
Hence by Schemes, Lemma \ref{schemes-lemma-glue-functors}
we see $G$ is representable.

\medskip\noindent
Choose a scheme $U \in \text{Ob}(\textit{Sch}/S)_{fppf}$
and a surjective
etale morphism $U \to \coprod F_i$ (this exist by hypothesis).
We may write $U = \coprod U_i$ with $U_i$ the inverse image of $F_i$,
see Lemma \ref{lemma-representable-sheaf-coproduct-sheaves}.
We claim that $U \to F$ is surjective and
etale. Surjectivity follows as $\coprod F_i \to F$ is surjective.
Consider the fibre product $U \times_F T$ where $T \to F$ is as
above. We have to show that $U \times_F T \to T$ is etale.
Since $U \times_F T = \coprod U_i \times_F T$ it suffices to show
each $U_i \times_F T \to T$ is etale. Since
$U_i \times_F T = U_i \times_{F_i} V_i$ this follows from the
fact that $U_i \to F_i$ is etale and $V_i \to T$ is an open immersion
(and Morphisms, Lemmas \ref{morphisms-lemma-open-immersion-etale}
and \ref{morphisms-lemma-composition-etale}).
\end{proof}














\section{Presentations of algebraic spaces}
\label{section-presentations}

\noindent
Given an algebraic space we can find a ``presentation'' of it.


\begin{lemma}
\label{lemma-space-presentation}
Let $F$ be an algebraic space over $S$. Let $f : U \to F$ be a
surjective etale morphism from a scheme to $F$. Set $R = U \times_F U$.
Then
\begin{enumerate}
\item $j : R \to U \times_S U$ defines an equivalence relation on
$U$ over $S$ (see
Groupoids, Definition \ref{groupoids-definition-equivalence-relation}).
\item the morphisms $s, t : R \to U$ are etale, and
\item the diagram
$$
\xymatrix{
R \ar@<1ex>[r] \ar@<-1ex>[r] &
U \ar[r] &
F
}
$$
is a coequalizer diagram in $\textit{Sh}((\textit{Sch}/S)_{fppf})$.
\end{enumerate}
\end{lemma}

\begin{proof}
Let $T/S$ be an object of $(\textit{Sch}/S)_{fppf}$.
Then $R(T) = \{(a, b) \in U(T) \times U(T) \mid f \circ a = f \circ b\}$
which is clearly defines an equivalence relation on $U(T)$.
The morphisms $s, t : R \to U$ are etale because the morphism
$U \to F$ is etale.

\medskip\noindent
To prove (3) we first show that
$U \to F$ is a surjection of sheaves, see
Sites, Definition \ref{sites-definition-sheaves-injective-surjective}.
Let $\xi \in F(T)$ with $T$ as above. Let $V = T \times_{\xi, F, f}U$.
By assumption $V$ is a scheme and $V \to T$ is surjective etale.
Hence $\{V \to T\}$ is a covering for the fppf topology.
Since $\xi|_V$ factors through $U$ by construction we
conclude $U \to F$ is surjective. To conclude we
have to show that given any two morphisms
$a, b : T \to U$ such that $f \circ a = f \circ b$ there is a
morphism $c : T \to R$ such that $a = \text{pr}_0 \circ c$
and $b = \text{pr}_1 \circ b$. This is clear from the definition
of $R$.
\end{proof}

\noindent
This lemma suggests the following definitions.

\begin{definition}
\label{definition-etale-equivalence-relation}
Let $S$ be a scheme. Let $U$ be a scheme over $S$.
An {\it etale equivalence relation} on $U$ over $S$
is an equivalence relation $j : R \to U \times_S U$
such that $s, t : R \to U$ are etale morphisms of schemes.
\end{definition}

\begin{definition}
\label{definition-presentation}
Let $F$ be an algebraic space over $S$.
A {\it presentation} of $F$ is given by a scheme
$U$ over $S$ and an etale equivalence relation $R$ on $U$ over $S$, and
a surjective etale morphism $U \to F$ such that $R = U \times_F U$.
\end{definition}

\noindent
Equivalently we could ask for the existence of an isomorphism
$$
U/R \cong F
$$
where the quotient $U/R$ is as defined in
Groupoids, Section \ref{groupoids-section-representable-quotients}.
To construct algebraic spaces we will study the converse question, namely,
for which pairs $(U, R)$ the quotient sheaf $U/R$ is an algebraic space.
See Theorem \ref{theorem-presentation} for the end result. 





























\section{Algebraic spaces and equivalence relations}
\label{section-spaces-from-equivalence-relations}

\noindent
Suppose given a scheme $U$ over $S$
and an etale equivalence relation $R$ on $U$ over $S$.
We would like to show this defines an algebraic space.
The problem is to show that the quotient sheaf $U/R$
(see Groupoids, Definition \ref{groupoids-definition-quotient-sheaf})
has all the properties required
of it in Definition \ref{definition-algebraic-space}.

\begin{lemma}
\label{lemma-pullback-etale-equivalence-relation}
Let $S$ be a scheme. Let $U$ be a scheme over $S$.
Let $j = (s, t) : R \to U \times_S U$
be an etale equivalence relation on $U$ over $S$.
Let $U' \to U$ be an etale morphism.
Let $R'$ be the restriction of $R$ to $U'$, see
Groupoids, Definition \ref{groupoids-definition-restrict-relation}.
Then $j' : R' \to U' \times_S U'$ is an etale equivalence
relation also.
\end{lemma}

\begin{proof}
It is clear from the description of $s', t'$ in
Groupoids, Lemma \ref{groupoids-lemma-restrict-groupoid}
that $s' , t' : R' \to U'$ are etale
as compositions of base changes of etale morphisms
(see Morphisms, Lemma \ref{morphisms-lemma-base-change-etale}
and \ref{morphisms-lemma-composition-etale}).
\end{proof}

\begin{lemma}
\label{lemma-finding-opens}
Let $S$ be a scheme.
Let $U$ be a scheme over $S$.
Let $j = (s, t) : R \to U \times_S U$ be a pre-relation.
Let $g : U' \to U$ be a morphism.
Assume
\begin{enumerate}
\item $j$ is an equivalence relation,
\item $s, t : R \to U$ are surjective, flat and
locally of finite presentation,
\item $g$ is flat and locally of finite presentation.
\end{enumerate}
Let $R' = R|_{U'}$ be the restriction of $R$ to $U$. Then
$R'/U' \to R/U$ is representable, and is an open immersion.
\end{lemma}

\begin{proof}
By Groupoids, Lemma \ref{groupoids-lemma-restrict-relation}
the morphism $j' = (t', s') : R' \to U' \times_S U'$
defines an equivalence relation. Since $g$ is flat and locally of
finite presentation we see that $g$ is universally open as well
(Morphisms, Lemma \ref{morphisms-lemma-fppf-open}).
For the same reason $s, t$ are universally open as well.
Let $W^1 = g(U') \subset U$, and let $W = t(s^{-1}(W^1))$.
Then $W^1$ and $W$ are open in $U$. Moreover, as $j$ is an
equivalence relation we have $t(s^{-1}(W)) = W$.

\medskip\noindent
By Groupoids, Lemma \ref{groupoids-lemma-quotient-groupoid-restrict}
the map of sheaves $F' = U'/R' \to F = U/R$ is injective.
Let $a : T \to F$ be a morphism from a scheme into $U/R$.
We have to show that $T \times_F F'$ is representable
by an open subscheme of $T$.

\medskip\noindent
The morphism $a$ is given by the following data:
an fppf covering $\{\varphi_j : T_j \to T\}_{j \in J}$ of $T$ and
morphsms $a_j : T_j \to U$ such that the maps
$$
a_j \times a_{j'} :
T_j \times_T T_{j'}
\longrightarrow
U \times_S U
$$
factor through $j : R \to U \times_S U$ via some (unique) maps
$r_{jj'} : T_j \times_T T_{j'} \to R$. The system
$(a_j)$ corresponds to $a$ in the sense that the diagrams
$$
\xymatrix{
T_j \ar[r]_{a_j} \ar[d] & U \ar[d] \\
T \ar[r]^a & F
}
$$
commute.

\medskip\noindent
Consider the open subsets $W_j = a_j^{-1}(W) \subset T_j$.
Since $t(s^{-1}(W)) = W$ we see that
$$
W_j \times_T T_{j'} =
r_{jj'}^{-1}(t^{-1}(W)) = r_{jj'}^{-1}(s^{-1}(W)) =
T_j \times_T W_{j'}.
$$
By Morphisms, Lemma \ref{morphisms-lemma-fpqc-quotient-topology}
and Descent, Lemma \ref{fpqc-descent-lemma-equiv-fibre-product}
applied to $\coprod T_j \to T$ this means there exists an open
$W_T \subset T$ such that $\varphi_j^{-1}(W_T) = W_j$ for all $j \in J$.
We claim that $W_T \to T$ represents $T \times_F F' \to T$.

\medskip\noindent
First, let us show that $W_T \to T \to F$ is an element of
$F'(W_T)$. Since $\{W_j \to W_T\}_{j \in J}$ is an
fppf covering of $W_T$, it is enough to show that
each $W_j \to U \to F$ is an element of $F'(W_j)$ (as $F'$ is a sheaf
for the fppf topology). Consider the commutative diagram
$$
\xymatrix{
W'_j \ar[rr] \ar[dd] \ar[rd] & & U' \ar[d]^g \\
& s^{-1}(W^1) \ar[r]_s \ar[d]^t & W^1 \ar[d] \\
W_j \ar[r]^{a_j|_{W_j}} & W \ar[r] & F
}
$$
where $W'_j = W_j \times_{W} s^{-1}(W^1) \times_{W^1} U'$.
Since $t$ and $g$ are surjective, flat and locally of finite
presentation, so is $W'_j \to W_j$. Hence the restriction of
the element $W_j \to U \to F$ to $W'_j$ is an element of $F'$
as desired.

\medskip\noindent
Suppose that $f : T' \to T$ is a morphism of schemes
such that $a|_{T'} \in F'(T')$. We have to show that
$f$ factors through the open $W_T$. Since
$\{T' \times_T T_j \to T\}$ is an fppf covering of $T'$
it is enough to show each $T' \times_T T_j \to T$
factors through $W_T$. Hence we may assume $f$ factors
as $\varphi_j \circ f_j : T' \to T_j \to T$ for some $j$.
In this case the condition $a|_{T'} \in F'(T')$ means that there exists
some fppf covering $\{\psi_i : T'_i \to T'\}_{i \in I}$ and some
morphisms $b_i : T'_i \to U'$ such that
$$
\xymatrix{
T'_i \ar[r]_{b_i} \ar[d]_{f_j \circ \psi_i} & U' \ar[r]_g & U \ar[d] \\
T_j \ar[r]^{a_j} & U \ar[r] & F
}
$$
is commutative. This commutativity means that there exists a
morphism $r'_i : T'_i \to R$ such that
$t \circ r'_i = a_j \circ f_j \circ \psi_i$, and
$s \circ r'_i = g \circ b_i$. This implies that
$\text{Im}(f_j \circ \psi_i) \subset W_j$ and we win.
\end{proof}

\noindent
The following lemma is not completely trivial although it looks
like it should be trivial.

\begin{lemma}
\label{lemma-when-it-works-it-works}
Let $S$ be a scheme. Let $U$ be a scheme over $S$.
Let $j = (s, t) : R \to U \times_S U$
be an etale equivalence relation on $U$ over $S$.
If the quotient $U/R$ is an algebraic space, then
$U \to U/R$ is etale and surjective. Hence
$(U, R, U \to U/R)$ is a presentation of the algebraic
space $U/R$.
\end{lemma}

\begin{proof}
Denote $c : U \to U/R$ the morphism in question.
Let $T$ be a scheme and let $a : T \to U/R$ be a morphism.
We have to show that the morphism (of schemes)
$\pi : T \times_{a, R/U, c} U \to T$ is etale and surjective.
The morphism $a$ corresponds to an fppf covering
$\{\varphi_i : T_i \to T\}$ and morphisms $a_i : T_i \to U$ such
that
$a_i \times a_{i'} : T_i \times_T T_{i'} \to U\times_S U$
factors through $R$, and such that $c \circ a_i = \varphi_i \circ a$.
Hence
$$
T_i \times_{\varphi_i, T} T \times_{a, R/U, c} U =
T_i \times_{c \circ a_i, R/U, c} U =
T_i \times_{a_i, U} U \times_{c, R/U, c} U = T_i \times_{a_i, U, t} R.
$$
Since $t$ is etale and surjective we conclude that
the base change of $\pi$ to $T_i$ is surjective and etale.
Since the property of being surjective and etale is local
on the base in the fpqc topology (see
Remark \ref{remark-list-properties-fpqc-local-base})
we win.
\end{proof}

\begin{lemma}
\label{lemma-presentation-quasi-compact}
Let $S$ be a scheme.
Let $U$ be a scheme over $S$.
Let $j = (s, t) : R \to U \times_S U$
be an etale equivalence relation on $U$ over $S$.
Assume that $U$ is affine. Then the quotient $F = U/R$
is an algebraic space, and $U \to F$ is etale and surjective.
\end{lemma}

\begin{proof}
Since $j : R \to U \times_S U$ is a monomorphism we see that $j$ is separated
(see Schemes, Lemma \ref{schemes-lemma-monomorphism-separated}).
Since $U$ is affine we see that $U \times_S U$
(which comes equipped with a monomorphism into the affine scheme
$U \times U$) is separated. Hence we see that $R$ is separated.
In particular the morphisms $s, t$ are separated as well as etale.

\medskip\noindent
Since the compostition $R \to U \times_S U \to U$ is
locally of finite type we conclude that
$j$ is locally of finite type (see
Morphisms, Lemma \ref{morphisms-lemma-permanence-finite-type}).
As $j$ is also a monomorphism it has finite fibres and
we see that $j$ is locally quasi-finite by
Morphisms, Lemma \ref{morphisms-lemma-finite-fibre}.
Alltogether we see that $j$ is separated and locally quasi-finite.

\medskip\noindent
Our first step is to show that the quotient map
$c : U \to F$ is representable.
Consider a scheme $T$ and a morphism $a : T \to F$.
We have to show that the sheaf $G = T \times_{a, F, c} U$
is representable.
As seen in the proofs of Lemmas \ref{lemma-finding-opens} and
\ref{lemma-when-it-works-it-works} there exists an fppf covering
$\{\varphi_i : T_i \to T\}_{i \in I}$ and morphisms $a_i : T_i \to U$
such that $a_i \times a_{i'} : T_i \times_T T_{i'} \to U\times_S U$
factors through $R$, and such that $c \circ a_i = \varphi_i \circ a$.
As in the proof of Lemma \ref{lemma-when-it-works-it-works} we see that
\begin{eqnarray*}
T_i \times_{\varphi_i, T} G & = &
T_i \times_{\varphi_i, T} T \times_{a, R/U, c} U \\
& = & T_i \times_{c \circ a_i, R/U, c} U \\
& = & T_i \times_{a_i, U} U \times_{c, R/U, c} U \\
& = & T_i \times_{a_i, U, t} R
\end{eqnarray*}
Since $t$ is separated and etale, and in particular
separated and locally quasi-finite (by Morphisms, Lemmas
\ref{morphisms-lemma-unramified-quasi-finite} and
\ref{morphisms-lemma-flat-unramified-etale})
we see that the restriction
of $G$ to each $T_i$ is representable by a morphism of schemes
$X_i \to T_i$ which is separated and locally quasi-finite. By
Descent, Lemma \ref{fpqc-descent-lemma-descent-data-sheaves}
we obtain a descent datum $(X_i, \varphi_{ii'})$ relative
to the fppf-covering $\{T_i \to T\}$. Since each
$X_i \to T_i$ is separated and locally quasi-finite we see by
More on Morphisms, Lemma
\ref{more-morphisms-lemma-separated-locally-quasi-finite-morphisms-fppf-descend}
that this descent datum is effective.
Hence by
Descent, Lemma \ref{fpqc-descent-lemma-descent-data-sheaves} (2)
we conclude that $G$ is representable as desired.

\medskip\noindent
The second step of the proof is to show that $U \to F$ is surjective and
etale. This is clear from the above since in the first step above we
saw that $G = T \times_{a, F, c} U$ is a scheme over $T$ which base changes
to schemes $X_i \to T_i$ which are surjective and etale. Thus $G \to T$
is surjective and etale (see
Remark \ref{remark-list-properties-fpqc-local-base}).
Alternatively one can reread the proof of
Lemma \ref{lemma-when-it-works-it-works} in the current
situation.

\medskip\noindent
The third and final step is to show that the diagonal map $F \to F \times F$
is representable. We first observe that the diagram
$$
\xymatrix{
R \ar[r] \ar[d]_j & F \ar[d]^\Delta \\
U \times_S U \ar[r] & F \times F
}
$$
is a fibre product square. By
Lemma \ref{lemma-product-representable-transformations} the morphism
$U \times_S U \to F \times F$ is representable (note that
$h_U \times h_U = h_{U \times_S U}$). Moreover, by
Lemma \ref{lemma-product-representable-transformations-property}
the morphism $U \times_S U \to F \times F$ is surjective
and etale (note also that etale and surjective occur in the lists of
Remarks \ref{remark-list-properties-fpqc-local-base}
and \ref{remark-list-properties-stable-composition}).
It follows either from
Lemma \ref{lemma-base-change-representable-transformations}
and the diagram above, or by writing $R \to F$ as $R \to U \to F$ and
Lemmas
\ref{lemma-morphism-schemes-gives-representable-transformation} and
\ref{lemma-composition-representable-transformations} that
$R \to F$ is representable as well. Let $T$ be a scheme and let
$a : T \to F \times F$ be a morphism. We have to show that
$G = T \times_{a, F \times F, \Delta} F$ is representable.
By what was said above the morphism (of schemes)
$$
T' = (U \times_S U) \times_{F \times F, a} T \longrightarrow T
$$
is surjective and etale. Hence $\{T' \to T\}$ is an etale
covering of $T$. Note also that
$$
T' \times_T G = T' \times_{U \times_S U, j} R
$$
as can be seen contemplating the following cube
$$
\xymatrix{
& R \ar[rr] \ar[dd] & & F \ar[dd] \\
T' \times_T G \ar[rr] \ar[dd] \ar[ru] & & G \ar[dd] \ar[ru] & \\
& U \times_S U \ar[rr] & & F \times F \\
T' \ar[rr] \ar[ru] & & T \ar[ru]
}
$$
Hence we see that the restriction of $G$ to $T'$ is representable
by a scheme $X$, and moreover that the morphism $X \to T'$ is
a base change of the morphism $j$. Hence $X \to T'$ is
separated and locally quasi-finite (see second paragraph of the proof).
By Descent, Lemma \ref{fpqc-descent-lemma-descent-data-sheaves}
we obtain a descent datum $(X, \varphi)$ relative
to the fppf-covering $\{T' \to T\}$. Since
$X \to T$ is separated and locally quasi-finite we see by
More on Morphisms, Lemma
\ref{more-morphisms-lemma-separated-locally-quasi-finite-morphisms-fppf-descend}
that this descent datum is effective.
Hence by
Descent, Lemma \ref{fpqc-descent-lemma-descent-data-sheaves} (2)
we conclude that $G$ is representable as desired.
\end{proof}

\begin{theorem}
\label{theorem-presentation}
Let $S$ be a scheme. Let $U$ be a scheme over $S$.
Let $j = (s, t) : R \to U \times_S U$
be an etale equivalence relation on $U$ over $S$.
Then the quotient $U/R$ is an algebraic space,
and $U \to U/R$ is etale and surjective, in other words
$(U, R, U \to U/R)$ is a presentation of $U/R$.
\end{theorem}

\begin{proof}
By Lemma \ref{lemma-when-it-works-it-works}
it suffice to just prove that $U/R$ is an algebraic space.
Let $U' \to U$ be a surjective, etale morphism.
Then $\{U' \to U\}$ is in particular an fppf covering.
Let $R'$ be the restriction of $R$ to $U'$, see
Groupoids, Definition \ref{groupoids-definition-restrict-relation}.
According to
Groupoids, Lemma \ref{groupoids-lemma-quotient-groupoid-restrict}
we see that $U/R \cong U'/R'$.
By Lemma \ref{lemma-pullback-etale-equivalence-relation} $R'$ is an
etale equivalence relation on $U'$. Thus we may replace $U$ by $U'$.

\medskip\noindent
We apply the previous remark to $U' = \coprod U_i$, where
$U = \bigcup U_i$ is an affine open covering of $S$. Hence we
may and do assume that $U = \coprod U_i$ where
each $U_i$ is an affine scheme.

\medskip\noindent
Consider the restriction $R_i$ of $R$ to $U_i$.
By Lemma \ref{lemma-pullback-etale-equivalence-relation}
this is an etale equivalence relation.
Set $F_i = U_i/R_i$ and $F = U/R$.
It is clear that $\coprod F_i \to F$ is surjective.
By Lemma \ref{lemma-finding-opens} each $F_i \to F$
is representable, and an open immersion.
By Lemma \ref{lemma-presentation-quasi-compact}
applied to $(U_i, R_i)$ we see that $F_i$ is an algebraic space.
Then by Lemma \ref{lemma-when-it-works-it-works} we see that
$U_i \to F_i$ is etale and surjective.
From Lemma \ref{lemma-coproduct-algebraic-spaces}
it follows that $\coprod F_i$ is an algebraic space.
Finally, we have verified all
hypotheses of Lemma \ref{lemma-glueing-algebraic-spaces}
and it follows that $F = U/R$ is an algebraic space.
\end{proof}



















\section{Other chapters}

\begin{multicols}{2}
\begin{enumerate}
\item \hyperref[introduction-section-phantom]{Introduction}
\item \hyperref[conventions-section-phantom]{Conventions}
\item \hyperref[sets-section-phantom]{Set Theory}
\item \hyperref[categories-section-phantom]{Categories}
\item \hyperref[topology-section-phantom]{Topology}
\item \hyperref[sheaves-section-phantom]{Sheaves on Spaces}
\item \hyperref[algebra-section-phantom]{Commutative Algebra}
\item \hyperref[sites-section-phantom]{Sites and Sheaves}
\item \hyperref[homology-section-phantom]{Homological Algebra}
\item \hyperref[derived-section-phantom]{Derived Categories}
\item \hyperref[more-algebra-section-phantom]{More Algebra}
\item \hyperref[simplicial-section-phantom]{Simplicial Methods}
\item \hyperref[modules-section-phantom]{Sheaves of Modules}
\item \hyperref[sites-modules-section-phantom]{Modules on Sites}
\item \hyperref[injectives-section-phantom]{Injectives}
\item \hyperref[cohomology-section-phantom]{Cohomology of Sheaves}
\item \hyperref[sites-cohomology-section-phantom]{Cohomology on Sites}
\item \hyperref[hypercovering-section-phantom]{Hypercoverings}
\item \hyperref[schemes-section-phantom]{Schemes}
\item \hyperref[constructions-section-phantom]{Constructions of Schemes}
\item \hyperref[properties-section-phantom]{Properties of Schemes}
\item \hyperref[morphisms-section-phantom]{Morphisms of Schemes}
\item \hyperref[coherent-section-phantom]{Coherent Cohomology}
\item \hyperref[divisors-section-phantom]{Divisors}
\item \hyperref[limits-section-phantom]{Limits of Schemes}
\item \hyperref[varieties-section-phantom]{Varieties}
\item \hyperref[chow-section-phantom]{Chow Homology}
\item \hyperref[topologies-section-phantom]{Topologies on Schemes}
\item \hyperref[descent-section-phantom]{Descent}
\item \hyperref[more-morphisms-section-phantom]{More on Morphisms}
\item \hyperref[flat-section-phantom]{More on Flatness}
\item \hyperref[groupoids-section-phantom]{Groupoid Schemes}
\item \hyperref[more-groupoids-section-phantom]{More on Groupoid Schemes}
\item \hyperref[etale-section-phantom]{\'Etale Morphisms of Schemes}
\item \hyperref[etale-cohomology-section-phantom]{\'Etale Cohomology}
\item \hyperref[spaces-section-phantom]{Algebraic Spaces}
\item \hyperref[spaces-properties-section-phantom]{Properties of Algebraic Spaces}
\item \hyperref[spaces-morphisms-section-phantom]{Morphisms of Algebraic Spaces}
\item \hyperref[spaces-topologies-section-phantom]{Topologies on Algebraic Spaces}
\item \hyperref[spaces-descent-section-phantom]{Descent and Algebraic Spaces}
\item \hyperref[spaces-more-morphisms-section-phantom]{More on Morphisms of Spaces}
\item \hyperref[quot-section-phantom]{Quot and Hilbert Spaces}
\item \hyperref[stacks-section-phantom]{Stacks}
\item \hyperref[spaces-groupoids-section-phantom]{Groupoids in Algebraic Spaces}
\item \hyperref[spaces-more-groupoids-section-phantom]{More on Groupoids in Spaces}
\item \hyperref[bootstrap-section-phantom]{Bootstrap}
\item \hyperref[examples-stacks-section-phantom]{Examples of Stacks}
\item \hyperref[groupoids-quotients-section-phantom]{Quotients of Groupoids}
\item \hyperref[algebraic-section-phantom]{Algebraic Stacks}
\item \hyperref[criteria-section-phantom]{Criteria for Representability}
\item \hyperref[stacks-properties-section-phantom]{Properties of Algebraic Stacks}
\item \hyperref[stacks-morphisms-section-phantom]{Morphisms of Algebraic Stacks}
\item \hyperref[examples-section-phantom]{Examples}
\item \hyperref[exercises-section-phantom]{Exercises}
\item \hyperref[guide-section-phantom]{Guide to Literature}
\item \hyperref[desirables-section-phantom]{Desirables}
\item \hyperref[coding-section-phantom]{Coding Style}
\item \hyperref[fdl-section-phantom]{GNU Free Documentation License}
\item \hyperref[index-section-phantom]{Auto Generated Index}
\end{enumerate}
\end{multicols}


\bibliography{my}
\bibliographystyle{alpha}

\end{document}
