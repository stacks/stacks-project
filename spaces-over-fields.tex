\IfFileExists{stacks-project.cls}{%
\documentclass{stacks-project}
}{%
\documentclass{amsart}
}

% The following AMS packages are automatically loaded with
% the amsart documentclass:
%\usepackage{amsmath}
%\usepackage{amssymb}
%\usepackage{amsthm}

% For dealing with references we use the comment environment
\usepackage{verbatim}
\newenvironment{reference}{\comment}{\endcomment}
%\newenvironment{reference}{}{}
\newenvironment{slogan}{\comment}{\endcomment}
\newenvironment{history}{\comment}{\endcomment}

% For commutative diagrams you can use
% \usepackage{amscd}
\usepackage[all]{xy}

% We use 2cell for 2-commutative diagrams.
\xyoption{2cell}
\UseAllTwocells

% To put source file link in headers.
% Change "template.tex" to "this_filename.tex"
% \usepackage{fancyhdr}
% \pagestyle{fancy}
% \lhead{}
% \chead{}
% \rhead{Source file: \url{template.tex}}
% \lfoot{}
% \cfoot{\thepage}
% \rfoot{}
% \renewcommand{\headrulewidth}{0pt}
% \renewcommand{\footrulewidth}{0pt}
% \renewcommand{\headheight}{12pt}

\usepackage{multicol}

% For cross-file-references
\usepackage{xr-hyper}

% Package for hypertext links:
\usepackage{hyperref}

% For any local file, say "hello.tex" you want to link to please
% use \externaldocument[hello-]{hello}
\externaldocument[introduction-]{introduction}
\externaldocument[conventions-]{conventions}
\externaldocument[sets-]{sets}
\externaldocument[categories-]{categories}
\externaldocument[topology-]{topology}
\externaldocument[sheaves-]{sheaves}
\externaldocument[sites-]{sites}
\externaldocument[stacks-]{stacks}
\externaldocument[fields-]{fields}
\externaldocument[algebra-]{algebra}
\externaldocument[brauer-]{brauer}
\externaldocument[homology-]{homology}
\externaldocument[derived-]{derived}
\externaldocument[simplicial-]{simplicial}
\externaldocument[more-algebra-]{more-algebra}
\externaldocument[smoothing-]{smoothing}
\externaldocument[modules-]{modules}
\externaldocument[sites-modules-]{sites-modules}
\externaldocument[injectives-]{injectives}
\externaldocument[cohomology-]{cohomology}
\externaldocument[sites-cohomology-]{sites-cohomology}
\externaldocument[dga-]{dga}
\externaldocument[dpa-]{dpa}
\externaldocument[hypercovering-]{hypercovering}
\externaldocument[schemes-]{schemes}
\externaldocument[constructions-]{constructions}
\externaldocument[properties-]{properties}
\externaldocument[morphisms-]{morphisms}
\externaldocument[coherent-]{coherent}
\externaldocument[divisors-]{divisors}
\externaldocument[limits-]{limits}
\externaldocument[varieties-]{varieties}
\externaldocument[topologies-]{topologies}
\externaldocument[descent-]{descent}
\externaldocument[perfect-]{perfect}
\externaldocument[more-morphisms-]{more-morphisms}
\externaldocument[flat-]{flat}
\externaldocument[groupoids-]{groupoids}
\externaldocument[more-groupoids-]{more-groupoids}
\externaldocument[etale-]{etale}
\externaldocument[chow-]{chow}
\externaldocument[intersection-]{intersection}
\externaldocument[pic-]{pic}
\externaldocument[adequate-]{adequate}
\externaldocument[dualizing-]{dualizing}
\externaldocument[duality-]{duality}
\externaldocument[discriminant-]{discriminant}
\externaldocument[local-cohomology-]{local-cohomology}
\externaldocument[curves-]{curves}
\externaldocument[resolve-]{resolve}
\externaldocument[models-]{models}
\externaldocument[pione-]{pione}
\externaldocument[etale-cohomology-]{etale-cohomology}
\externaldocument[proetale-]{proetale}
\externaldocument[crystalline-]{crystalline}
\externaldocument[spaces-]{spaces}
\externaldocument[spaces-properties-]{spaces-properties}
\externaldocument[spaces-morphisms-]{spaces-morphisms}
\externaldocument[decent-spaces-]{decent-spaces}
\externaldocument[spaces-cohomology-]{spaces-cohomology}
\externaldocument[spaces-limits-]{spaces-limits}
\externaldocument[spaces-divisors-]{spaces-divisors}
\externaldocument[spaces-over-fields-]{spaces-over-fields}
\externaldocument[spaces-topologies-]{spaces-topologies}
\externaldocument[spaces-descent-]{spaces-descent}
\externaldocument[spaces-perfect-]{spaces-perfect}
\externaldocument[spaces-more-morphisms-]{spaces-more-morphisms}
\externaldocument[spaces-flat-]{spaces-flat}
\externaldocument[spaces-groupoids-]{spaces-groupoids}
\externaldocument[spaces-more-groupoids-]{spaces-more-groupoids}
\externaldocument[bootstrap-]{bootstrap}
\externaldocument[spaces-pushouts-]{spaces-pushouts}
\externaldocument[groupoids-quotients-]{groupoids-quotients}
\externaldocument[spaces-more-cohomology-]{spaces-more-cohomology}
\externaldocument[spaces-simplicial-]{spaces-simplicial}
\externaldocument[spaces-duality-]{spaces-duality}
\externaldocument[formal-spaces-]{formal-spaces}
\externaldocument[restricted-]{restricted}
\externaldocument[spaces-resolve-]{spaces-resolve}
\externaldocument[formal-defos-]{formal-defos}
\externaldocument[defos-]{defos}
\externaldocument[cotangent-]{cotangent}
\externaldocument[examples-defos-]{examples-defos}
\externaldocument[algebraic-]{algebraic}
\externaldocument[examples-stacks-]{examples-stacks}
\externaldocument[stacks-sheaves-]{stacks-sheaves}
\externaldocument[criteria-]{criteria}
\externaldocument[artin-]{artin}
\externaldocument[quot-]{quot}
\externaldocument[stacks-properties-]{stacks-properties}
\externaldocument[stacks-morphisms-]{stacks-morphisms}
\externaldocument[stacks-limits-]{stacks-limits}
\externaldocument[stacks-cohomology-]{stacks-cohomology}
\externaldocument[stacks-perfect-]{stacks-perfect}
\externaldocument[stacks-introduction-]{stacks-introduction}
\externaldocument[stacks-more-morphisms-]{stacks-more-morphisms}
\externaldocument[stacks-geometry-]{stacks-geometry}
\externaldocument[moduli-]{moduli}
\externaldocument[moduli-curves-]{moduli-curves}
\externaldocument[examples-]{examples}
\externaldocument[exercises-]{exercises}
\externaldocument[guide-]{guide}
\externaldocument[desirables-]{desirables}
\externaldocument[coding-]{coding}
\externaldocument[obsolete-]{obsolete}
\externaldocument[fdl-]{fdl}
\externaldocument[index-]{index}

% Theorem environments.
%
\theoremstyle{plain}
\newtheorem{theorem}[subsection]{Theorem}
\newtheorem{proposition}[subsection]{Proposition}
\newtheorem{lemma}[subsection]{Lemma}

\theoremstyle{definition}
\newtheorem{definition}[subsection]{Definition}
\newtheorem{example}[subsection]{Example}
\newtheorem{exercise}[subsection]{Exercise}
\newtheorem{situation}[subsection]{Situation}

\theoremstyle{remark}
\newtheorem{remark}[subsection]{Remark}
\newtheorem{remarks}[subsection]{Remarks}

\numberwithin{equation}{subsection}

% Macros
%
\def\lim{\mathop{\mathrm{lim}}\nolimits}
\def\colim{\mathop{\mathrm{colim}}\nolimits}
\def\Spec{\mathop{\mathrm{Spec}}}
\def\Hom{\mathop{\mathrm{Hom}}\nolimits}
\def\Ext{\mathop{\mathrm{Ext}}\nolimits}
\def\SheafHom{\mathop{\mathcal{H}\!\mathit{om}}\nolimits}
\def\SheafExt{\mathop{\mathcal{E}\!\mathit{xt}}\nolimits}
\def\Sch{\mathit{Sch}}
\def\Mor{\operatorname{Mor}\nolimits}
\def\Ob{\mathop{\mathrm{Ob}}\nolimits}
\def\Sh{\mathop{\mathit{Sh}}\nolimits}
\def\NL{\mathop{N\!L}\nolimits}
\def\proetale{{pro\text{-}\acute{e}tale}}
\def\etale{{\acute{e}tale}}
\def\QCoh{\mathit{QCoh}}
\def\Ker{\mathop{\mathrm{Ker}}}
\def\Im{\mathop{\mathrm{Im}}}
\def\Coker{\mathop{\mathrm{Coker}}}
\def\Coim{\mathop{\mathrm{Coim}}}

%
% Macros for moduli stacks/spaces
%
\def\QCohstack{\mathcal{QC}\!\mathit{oh}}
\def\Cohstack{\mathcal{C}\!\mathit{oh}}
\def\Spacesstack{\mathcal{S}\!\mathit{paces}}
\def\Quotfunctor{\mathrm{Quot}}
\def\Hilbfunctor{\mathrm{Hilb}}
\def\Curvesstack{\mathcal{C}\!\mathit{urves}}
\def\Polarizedstack{\mathcal{P}\!\mathit{olarized}}
\def\Complexesstack{\mathcal{C}\!\mathit{omplexes}}
% \Pic is the operator that assigns to X its picard group, usage \Pic(X)
% \Picardstack_{X/B} denotes the Picard stack of X over B
% \Picardfunctor_{X/B} denotes the Picard functor of X over B
\def\Pic{\mathop{\mathrm{Pic}}\nolimits}
\def\Picardstack{\mathcal{P}\!\mathit{ic}}
\def\Picardfunctor{\mathrm{Pic}}
\def\Deformationcategory{\mathcal{D}\!\mathit{ef}}


% OK, start here.
%
\begin{document}

\title{Algebraic Spaces over Fields}


\maketitle

\phantomsection
\label{section-phantom}

\tableofcontents

\section{Introduction}
\label{section-introduction}

\noindent
This chapter is the analogue of the chapter on varieties in the setting
of algebraic spaces. A reference for algebraic spaces is
\cite{Kn}.


\section{Conventions}
\label{section-conventions}

\noindent
The standing assumption is that all schemes are contained in
a big fppf site $\Sch_{fppf}$. And all rings $A$ considered
have the property that $\Spec(A)$ is (isomorphic) to an
object of this big site.

\medskip\noindent
Let $S$ be a scheme and let $X$ be an algebraic space over $S$.
In this chapter and the following we will write $X \times_S X$
for the product of $X$ with itself (in the category of algebraic
spaces over $S$), instead of $X \times X$.



\section{Generically finite morphisms}
\label{section-generically-finite}

\noindent
This section continues the discussion in
Decent Spaces, Section \ref{section-generically-finite}
and the analogue for morphisms of algebraic spaces of
Varieties, Section \ref{varieties-section-generically-finite}.

\begin{lemma}
\label{lemma-quasi-finite-in-codim-1}
Let $S$ be a scheme. Let $f : X \to Y$ be a morphism of algebraic spaces
over $S$. Assume $f$ is locally of finite type and $Y$ is locally Noetherian.
Let $y \in |Y|$ be a point of codimension $\leq 1$ on $Y$.
Let $X^0 \subset |X|$ be the set of points of codimension $0$ on $X$.
Assume in addition one of the following conditions is satisfied
\begin{enumerate}
\item for every $x \in X^0$ the transcendence degree of $x/f(x)$ is $0$,
\item for every $x \in X^0$ with $f(x) \leadsto y$
the transcendence degree of $x/f(x)$ is $0$,
\item $f$ is quasi-finite at every $x \in X^0$,
\item $f$ is quasi-finite at a dense set of points of $|X|$,
\item add more here.
\end{enumerate}
Then $f$ is quasi-finite at every point of $X$ lying over $y$.
\end{lemma}

\begin{proof}
We want to reduce the proof to the case of schemes. To do this we
choose a commutative diagram
$$
\xymatrix{
U \ar[r] \ar[d]_g & X \ar[d]^f \\
V \ar[r] & Y
}
$$
where $U$, $V$ are schemes and where the horizontal arrows are \'etale
and surjective. Pick $v \in V$ mapping to $y$. Observe that
$V$ is locally Noetherian and that $\dim(\mathcal{O}_{V, v}) \leq 1$
(see Properties of Spaces, Definitions
\ref{spaces-properties-definition-dimension-local-ring} and
Remark \ref{spaces-properties-remark-list-properties-local-etale-topology}).
The fibre $U_v$ of $U \to V$ over $v$ surjects onto
$f^{-1}(\{y\}) \subset |X|$. The inverse image of $X^0$ in $U$
is exactly the set of
generic points of irreducible components of $U$ (Properties of Spaces, Lemma
\ref{spaces-properties-lemma-codimension-0-points}).
If $\eta \in U$ is such a point with image $x \in X^0$, then
the transcendence degree of $x / f(x)$ is the transcendence
degree of $\kappa(\eta)$ over $\kappa(g(\eta))$
(Morphisms of Spaces, Definition
\ref{spaces-morphisms-definition-dimension-fibre}).
Observe that $U \to V$ is quasi-finite at $u \in U$ if and only if
$f$ is quasi-finite at the image of $u$ in $X$.

\medskip\noindent
Case (1). Here case (1) of
Varieties, Lemma \ref{varieties-lemma-quasi-finite-in-codim-1} applies
and we conclude that $U \to V$ is quasi-finite at all points of $U_v$.
Hence $f$ is quasi-finite at every point lying over $y$.

\medskip\noindent
Case (2). Let $u \in U$ be a generic point of an irreducible component
whose image in $V$ specializes to $v$. Then the image $x \in X^0$ of
$u$ has the property that $f(x) \leadsto y$. Hence we see that
case (2) of
Varieties, Lemma \ref{varieties-lemma-quasi-finite-in-codim-1} applies
and we conclude as before.

\medskip\noindent
Case (3) follows from case (3) of
Varieties, Lemma \ref{varieties-lemma-quasi-finite-in-codim-1}.

\medskip\noindent
In case (4), since $|U| \to |X|$ is open, we see that
the set of points where $U \to V$ is quasi-finite is dense as well.
Hence case (4) of
Varieties, Lemma \ref{varieties-lemma-quasi-finite-in-codim-1} applies.
\end{proof}

\begin{lemma}
\label{lemma-finite-in-codim-1}
Let $S$ be a scheme. Let $f : X \to Y$ be a morphism of algebraic spaces
over $S$. Assume $f$ is proper and $Y$ is locally Noetherian.
Let $y \in Y$ be a point of codimension $\leq 1$ in $Y$.
Let $X^0 \subset |X|$ be the set of points of codimension $0$ on $X$.
Assume in addition one of the
following conditions is satisfied
\begin{enumerate}
\item for every $x \in X^0$ the transcendence degree of $x/f(x)$ is $0$,
\item for every $x \in X^0$ with $f(x) \leadsto y$ the transcendence degree
of $x/f(x)$ is $0$,
\item $f$ is quasi-finite at every $x \in X^0$,
\item $f$ is quasi-finite at a dense set of points of $|X|$,
\item add more here.
\end{enumerate}
Then there exists an open subspace $Y' \subset Y$ containing $y$ such that
$Y' \times_Y X \to Y'$ is finite.
\end{lemma}

\begin{proof}
By Lemma \ref{lemma-quasi-finite-in-codim-1} the morphism $f$ is
quasi-finite at every point lying over $y$. Let $\overline{y} : \Spec(k) \to Y$
be a geometric point lying over $y$. Then $|X_{\overline{y}}|$ is a
discrete space (Decent Spaces, Lemma
\ref{decent-spaces-lemma-conditions-on-fibre-and-qf}).
Since $X_{\overline{y}}$ is quasi-compact as $f$ is proper we conclude
that $|X_{\overline{y}}|$ is finite.
Thus we can apply Cohomology of Spaces, Lemma
\ref{spaces-cohomology-lemma-proper-finite-fibre-finite-in-neighbourhood}
to conclude.
\end{proof}

\begin{lemma}
\label{lemma-modification-normal-iso-over-codimension-1}
Let $S$ be a scheme. Let $X$ be a Noetherian algebraic space over $S$.
Let $f : Y \to X$ be a birational proper morphism of algebraic spaces
with $Y$ reduced.
Let $U \subset X$ be the maximal open over which $f$ is an isomorphism.
Then $U$ contains
\begin{enumerate}
\item every point of codimension $0$ in $X$,
\item every $x \in |X|$ of codimension $1$ on $X$ such that the local ring of
$X$ at $x$ is normal (Properties of Spaces, Remark
\ref{spaces-properties-remark-list-properties-local-ring-local-etale-topology}),
and
\item every $x \in |X|$ such that the fibre of $|Y| \to |X|$ over $x$ is
finite and such that the local ring of $X$ at $x$ is normal.
\end{enumerate}
\end{lemma}

\begin{proof}
Part (1) follows from Decent Spaces, Lemma
\ref{decent-spaces-lemma-birational-isomorphism-over-dense-open}
(and the fact that the Noetherian algebraic spaces $X$ and $Y$
are quasi-separated and hence decent).
Part (2) follows from part (3) and Lemma \ref{lemma-finite-in-codim-1}
(and the fact that finite morphisms have finite fibres).
Let $x \in |X|$ be as in (3). By
Cohomology of Spaces, Lemma
\ref{spaces-cohomology-lemma-proper-finite-fibre-finite-in-neighbourhood}
(which applies by Decent Spaces, Lemma
\ref{decent-spaces-lemma-conditions-on-fibre-and-qf})
we may assume $f$ is finite. Choose an affine scheme $X'$ and
an \'etale morphism $X' \to X$ and a point $x' \in X$ mapping to $x$.
It suffices to show there exists an open neighbourhood $U'$ of $x' \in X'$
such that $Y \times_X X' \to X'$ is an isomorphism over $U'$
(namely, then $U$ contains the image of $U'$ in $X$, see Spaces, Lemma
\ref{spaces-lemma-descent-representable-transformations-property}).
Then $Y \times_X X' \to X$ is a finite birational
(Decent Spaces, Lemma \ref{decent-spaces-lemma-birational-etale-localization})
morphism. Since a finite morphism is affine we reduce to
the case of a finite birational morphism of Noetherian affine schemes
$Y \to X$ and $x \in X$ such that $\mathcal{O}_{X, x}$ is a
normal domain. This is treated in Varieties, Lemma
\ref{varieties-lemma-modification-normal-iso-over-codimension-1}.
\end{proof}






\section{Integral algebraic spaces}
\label{section-integral-spaces}

\noindent
We have not yet defined the notion of an integral algebraic space. The
problem is that being integral is not an \'etale local property of schemes.
We could use the property, that $X$ is reduced and $|X|$ is irreducible,
given in Properties, Lemma \ref{properties-lemma-characterize-integral}
to define integral algebraic spaces. In this case the algebraic
space described in Spaces, Example \ref{spaces-example-infinite-product}
would be integral which does not seem right.
To avoid this type of pathology we will in addition assume that $X$ is a
decent algebraic space, although perhaps a weaker alternative exists.

\begin{definition}
\label{definition-integral-algebraic-space}
Let $S$ be a scheme. We say an algebraic space $X$ over $S$ is
{\it integral} if it is reduced, decent, and $|X|$ is irreducible.
\end{definition}

\noindent
In this case the irreducible topological space $|X|$ is sober
(Decent Spaces, Proposition \ref{decent-spaces-proposition-reasonable-sober}).
Hence it has a unique generic point $x$.
Then $x$ is contained in the schematic locus of $X$
(Decent Spaces, Theorem \ref{decent-spaces-theorem-decent-open-dense-scheme})
and we can look at its residue field as a substitute for
the function field of $X$ (not yet defined; insert future reference here).
In Decent Spaces, Lemma
\ref{decent-spaces-lemma-finitely-many-irreducible-components}
we characterized decent algebraic spaces with finitely many
irreducible components. Applying that lemma we see that an
algebraic space $X$ is integral if it is
reduced, has an irreducible dense open subscheme $X'$
with generic point $x'$ and the morphism $x' \to X$ is quasi-compact.

\begin{lemma}
\label{lemma-integral-sections}
Let $S$ be a scheme. Let $X$ be an integral algebraic space over $S$.
Then $\Gamma(X, \mathcal{O}_X)$ is a domain.
\end{lemma}

\begin{proof}
Set $R = \Gamma(X, \mathcal{O}_X)$. If $f, g \in R$ are nonzero and
$fg = 0$ then $X = V(f) \cup V(g)$ where $V(f)$ denotes the closed subspace
of $X$ cut out by $f$. Since $X$ is irreducible, we see that either
$V(f) = X$ or $V(g) = X$. Then either $f = 0$ or $g = 0$ by
Properties of Spaces, Lemma \ref{spaces-properties-lemma-reduced-space}.
\end{proof}

\noindent
The following lemma characterizes dominant morphisms of finite degree
between integral algebraic spaces.

\begin{lemma}
\label{lemma-finite-degree}
Let $S$ be a scheme. Let $X$, $Y$ be integral algebraic spaces over $S$
Let $x \in |X|$ and $y \in |Y|$ be the generic points. Let $f : X \to Y$
be locally of finite type. Assume $f$ is dominant
(Morphisms of Spaces, Definition \ref{spaces-morphisms-definition-dominant}).
The following are equivalent:
\begin{enumerate}
\item the transcendence degree of $x/y$ is $0$,
\item the extension $\kappa(x) \supset \kappa(y)$ (see proof) is finite,
\item there exist nonempty affine opens $U \subset X$ and $V \subset Y$
such that $f(U) \subset V$ and $f|_U : U \to V$ is finite,
\item $f$ is quasi-finite at $x$, and
\item $x$ is the only point of $|X|$ mapping to $y$.
\end{enumerate}
If $f$ is separated or if $f$ is quasi-compact, then these are
also equivalent to
\begin{enumerate}
\item[(6)] there exists a nonempty affine open $V \subset Y$ such
that $f^{-1}(V) \to V$ is finite.
\end{enumerate}
\end{lemma}

\begin{proof}
By elementary topology, we see that $f(x) = y$ as $f$ is dominant.
Let $Y' \subset Y$ be the schematic locus of $Y$ and let
$X' \subset f^{-1}(Y')$ be the schematic locus of $f^{-1}(Y')$.
By the discussion above, using
Decent Spaces, Proposition \ref{decent-spaces-proposition-reasonable-sober} and
Theorem \ref{decent-spaces-theorem-decent-open-dense-scheme},
we see that $x \in |X'|$ and $y \in |Y'|$.
Then $f|_{X'} : X' \to Y'$ is a morphism of integral schemes
which is locally of finite type. Thus we see that (1), (2), (3)
are equivalent by Morphisms, Lemma \ref{morphisms-lemma-finite-degree}.

\medskip\noindent
Condition (4) implies condition (1) by
Morphisms of Spaces, Lemma \ref{spaces-morphisms-lemma-compare-tr-deg}
applied to $X \to Y \to Y$.
On the other hand, condition (3) implies condition (4) as
a finite morphism is quasi-finite and as $x \in U$ because $x$
is the generic point. Thus (1) -- (4) are equivalent.

\medskip\noindent
Assume the equivalent conditions (1) -- (4). Suppose that
$x' \mapsto y$. Then $x \leadsto x'$ is a specialization in the
fibre of $|X| \to |Y|$ over $y$. If $x' \not = x$, then $f$ is not
quasi-finite at $x$ by Decent Spaces, Lemma
\ref{decent-spaces-lemma-conditions-on-point-in-fibre-and-qf}.
Hence $x = x'$ and (5) holds. Conversely, if (5) holds, then
(5) holds for the morphism of schemes $X' \to Y'$ (see above)
and we can use
Morphisms, Lemma \ref{morphisms-lemma-finite-degree}
to see that (1) holds.

\medskip\noindent
Observe that (6) implies the equivalent conditions (1) -- (5)
without any further assumptions on $f$. To finish the proof
we have to show the equivalent conditions (1) -- (5) imply (6).
This follows from Decent Spaces, Lemma
\ref{decent-spaces-lemma-finite-over-dense-open}.
\end{proof}

\begin{definition}
\label{definition-degree}
Let $S$ be a scheme.
Let $X$ and $Y$ be integral algebraic spaces over $S$.
Let $f : X \to Y$ be locally of finite type and dominant.
Assume any of the equivalent conditions (1) -- (5) of
Lemma \ref{lemma-finite-degree}. Let $x \in |X|$ and $y \in |Y|$
be the generic points. Then the positive integer
$$
\text{deg}(X/Y) = [\kappa(x) : \kappa(y)]
$$
is called the {\it degree of $X$ over $Y$}.
\end{definition}

\noindent
Here is a lemma about normal integral algebraic spaces.

\begin{lemma}
\label{lemma-normal-integral-cover-by-affines}
Let $S$ be a scheme. Let $X$ be a normal integral algebraic space over $S$.
For every $x \in |X|$ there exists a normal integral affine scheme $U$
and an \'etale morphism $U \to X$ such that $x$ is in the image.
\end{lemma}

\begin{proof}
Choose an affine scheme $U$ and an \'etale morphism $U \to X$ such that
$x$ is in the image. Let $u_i$, $i \in I$ be the generic points of irreducible
components of $U$. Then each $u_i$ maps to the generic point of $X$
(Decent Spaces, Lemma \ref{decent-spaces-lemma-decent-generic-points}). By 
our definition of a decent space
(Decent Spaces, Definition \ref{decent-spaces-definition-very-reasonable}),
we see that $I$ is finite. Hence $U = \Spec(A)$ where $A$ is a normal ring
with finitely many minimal primes.
Thus $A = \prod_{i \in I} A_i$ is a product of normal domains by
Algebra, Lemma \ref{algebra-lemma-characterize-reduced-ring-normal}.
Then $U = \coprod U_i$ with $U_i = \Spec(A_i)$ and $x$ is in the image of
$U_i \to X$ for some $i$. This proves the lemma.
\end{proof}

\begin{lemma}
\label{lemma-normal-integral-sections}
Let $S$ be a scheme. Let $X$ be a normal integral algebraic space over $S$.
Then $\Gamma(X, \mathcal{O}_X)$ is a normal domain.
\end{lemma}

\begin{proof}
Set $R = \Gamma(X, \mathcal{O}_X)$. Then $R$ is a domain by
Lemma \ref{lemma-integral-sections}.
Let $f = a/b$ be an element of the fraction field of $R$
which is integral over $R$.
For any $U \to X$ \'etale with $U$ a scheme there is at most one
$f_U \in \Gamma(U, \mathcal{O}_U)$ with $b|_U f_U = a|_U$.
Namely, $U$ is reduced and the generic points of $U$ map to
the generic point of $X$ which implies that $b|_U$ is a
nonzerodivisor.
For every $x \in |X|$ we choose $U \to X$ as in
Lemma \ref{lemma-normal-integral-cover-by-affines}.
Then there is a unique $f_U \in \Gamma(U, \mathcal{O}_U)$
with $b|_U f_U = a|_U$ because
$\Gamma(U, \mathcal{O}_U)$ is a normal domain by
Properties, Lemma \ref{properties-lemma-normal-integral-sections}.
By the uniqueness mentioned above these $f_U$
glue and define a global section $f$ of the structure
sheaf, i.e., of $R$.
\end{proof}






\section{Modifications and alterations}
\label{section-modifications-alterations}

\noindent
Using our notion of an integral algebraic space we can define a modification
as follows.

\begin{definition}
\label{definition-modification}
Let $S$ be a scheme. Let $X$ be an integral algebraic space over $S$. A
{\it modification of $X$} is a birational proper morphism
$f : X' \to X$ of algebraic spaces over $S$ with $X'$ integral.
\end{definition}

\noindent
For birational morphisms of algebraic spaces, see
Decent Spaces, Definition \ref{decent-spaces-definition-birational}.

\begin{lemma}
\label{lemma-modification-iso-over-open}
Let $f : X' \to X$ be a modification as in
Definition \ref{definition-modification}.
There exists a nonempty open $U \subset X$ such that $f^{-1}(U) \to U$
is an isomorphism.
\end{lemma}

\begin{proof}
By
Lemma \ref{lemma-finite-degree} there exists a nonempty $U \subset X$ such
that $f^{-1}(U) \to U$ is finite. By generic flatness
(Morphisms of Spaces, Proposition
\ref{spaces-morphisms-proposition-generic-flatness-reduced})
we may assume $f^{-1}(U) \to U$ is flat and of finite presentation.
So $f^{-1}(U) \to U$ is finite locally free
(Morphisms of Spaces, Lemma \ref{spaces-morphisms-lemma-finite-flat}).
Since $f$ is birational, the degree of $X'$ over $X$ is $1$.
Hence $f^{-1}(U) \to U$ is finite locally free of degree $1$,
in other words it is an isomorphism.
\end{proof}

\begin{definition}
\label{definition-alteration}
Let $S$ be a scheme. Let $X$ be an integral algebraic space over $S$.
An {\it alteration of $X$} is a proper dominant morphism $f : Y \to X$
of algebraic spaces over $S$ with $Y$ integral such that $f^{-1}(U) \to U$
is finite for some nonempty open $U \subset X$.
\end{definition}

\noindent
If $f : Y \to X$ is a dominant and proper morphism between integral
algebraic spaces, then it is an alteration as soon as the induced
extension of residue fields in generic points is finite. Here is the
precise statement.

\begin{lemma}
\label{lemma-alteration-generically-finite}
Let $S$ be a scheme. Let $f : X \to Y$ be a proper dominant morphism of
integral algebraic spaces over $S$. Then $f$ is an alteration
if and only if any of the equivalent conditions (1) -- (6) of
Lemma \ref{lemma-finite-degree} hold.
\end{lemma}

\begin{proof}
Immediate consequence of the lemma referenced in the statement.
\end{proof}





\section{Schematic locus}
\label{section-schematic}

\noindent
We have already proven a number of results on the schematic locus
of an algebraic space. Here is a list of references:
\begin{enumerate}
\item Properties of Spaces, Sections
\ref{spaces-properties-section-schematic} and
\ref{spaces-properties-section-getting-a-scheme},
\item Decent Spaces, Section \ref{decent-spaces-section-schematic},
\item Properties of Spaces, Lemma
\ref{spaces-properties-lemma-point-like-spaces}
$\Leftarrow$
Decent Spaces, Lemma \ref{decent-spaces-lemma-decent-point-like-spaces}
$\Leftarrow$
Decent Spaces, Lemma \ref{decent-spaces-lemma-when-field},
\item Limits of Spaces, Section \ref{spaces-limits-section-affine}, and
\item Limits of Spaces, Section \ref{spaces-limits-section-representable}.
\end{enumerate}
There are some cases where certain types of morphisms of algebraic spaces
are automatically representable, for example
separated, locally quasi-finite morphisms (Morphisms of Spaces, Lemma
\ref{spaces-morphisms-lemma-locally-quasi-finite-separated-representable}),
and flat monomorphisms (More on Morphisms of Spaces, Lemma
\ref{spaces-more-morphisms-lemma-flat-case})
In Section \ref{section-schematic-and-field-extension}
we will study what happens with the schematic
locus under extension of base field.

\begin{lemma}
\label{lemma-locally-finite-type-dim-zero}
Let $S$ be a scheme. Let $X$ be an algebraic space over $S$.
In each of the following cases $X$ is a scheme:
\begin{enumerate}
\item $X$ is quasi-compact and quasi-separated and $\dim(X) = 0$,
\item $X$ is locally of finite type over a field $k$ and $\dim(X) = 0$,
\item $X$ is Noetherian and $\dim(X) = 0$, and
\item add more here.
\end{enumerate}
\end{lemma}

\begin{proof}
Cases (2) and (3) follow immediately from case (1) but we will give a separate
proofs of (2) and (3) as these proofs use significantly less theory.

\medskip\noindent
Proof of (3). Let $U$ be an affine scheme and let $U \to X$ be an
\'etale morphism. Set $R = U \times_X U$. The two projection
morphisms $s, t : R \to U$ are \'etale morphisms of schemes. By
Properties of Spaces, Definition \ref{spaces-properties-definition-dimension}
we see that $\dim(U) = 0$ and $\dim(R) = 0$.
Since $R$ is a locally Noetherian scheme of dimension $0$,
we see that $R$ is a disjoint union of spectra of
Artinian local rings
(Properties, Lemma \ref{properties-lemma-locally-Noetherian-dimension-0}).
Since we assumed that $X$ is Noetherian (so quasi-separated) we
conclude that $R$ is quasi-compact. Hence $R$ is an affine scheme
(use Schemes, Lemma \ref{schemes-lemma-disjoint-union-affines}).
The \'etale morphisms $s, t : R \to U$ induce finite residue field
extensions. Hence $s$ and $t$ are finite by
Algebra, Lemma
\ref{algebra-lemma-essentially-of-finite-type-into-artinian-local}
(small detail omitted). 
Thus
Groupoids, Proposition \ref{groupoids-proposition-finite-flat-equivalence}
shows that $X = U/R$ is an affine scheme.

\medskip\noindent
Proof of (2) -- almost identical to the proof of (4).
Let $U$ be an affine scheme and let $U \to X$ be an \'etale morphism.
Set $R = U \times_X U$. The two projection morphisms
$s, t : R \to U$ are \'etale morphisms of schemes. By
Properties of Spaces, Definition \ref{spaces-properties-definition-dimension}
we see that $\dim(U) = 0$ and similarly $\dim(R) = 0$.
On the other hand, the morphism $U \to \Spec(k)$ is locally of finite
type as the composition of the \'etale morphism $U \to X$ and
$X \to \Spec(k)$, see
Morphisms of Spaces,
Lemmas \ref{spaces-morphisms-lemma-composition-finite-type} and
\ref{spaces-morphisms-lemma-etale-locally-finite-type}.
Similarly, $R \to \Spec(k)$ is locally of finite type.
Hence by
Varieties, Lemma \ref{varieties-lemma-algebraic-scheme-dim-0}
we see that $U$ and $R$ are disjoint unions of spectra of
local Artinian $k$-algebras finite over $k$. The same thing
is therefore true of $U \times_{\Spec(k)} U$. As
$$
R = U \times_X U \longrightarrow U \times_{\Spec(k)} U
$$
is a monomorphism, we see that $R$ is a finite(!) union of spectra of
finite $k$-algebras. It follows that $R$ is affine, see
Schemes, Lemma \ref{schemes-lemma-disjoint-union-affines}.
Applying
Varieties, Lemma \ref{varieties-lemma-algebraic-scheme-dim-0}
once more we see that $R$ is finite over $k$. Hence $s, t$
are finite, see
Morphisms, Lemma \ref{morphisms-lemma-finite-permanence}.
Thus
Groupoids, Proposition \ref{groupoids-proposition-finite-flat-equivalence}
shows that the open subspace $U/R$ of $X$ is an affine scheme. Since the
schematic locus of $X$ is an open subspace (see
Properties of Spaces, Lemma \ref{spaces-properties-lemma-subscheme}),
and since $U \to X$ was an arbitrary \'etale morphism from an affine scheme
we conclude that $X$ is a scheme.

\medskip\noindent
Proof of (1). By Cohomology of Spaces, Lemma
\ref{spaces-cohomology-lemma-vanishing-above-dimension}
we have vanishing of higher cohomology groups for all
quasi-coherent sheaves $\mathcal{F}$ on $X$. Hence $X$
is affine (in particular a scheme) by
Cohomology of Spaces, Proposition
\ref{spaces-cohomology-proposition-vanishing-affine}.
\end{proof}

\noindent
The following lemma tells us that a quasi-separated algebraic space
is a scheme away from codimension $1$.

\begin{lemma}
\label{lemma-generic-point-in-schematic-locus}
Let $S$ be a scheme. Let $X$ be a quasi-separated algebraic space over $S$.
Let $x \in |X|$. The following are equivalent
\begin{enumerate}
\item $x$ is a point of codimension $0$ on $X$,
\item the local ring of $X$ at $x$ has dimension $0$, and
\item $x$ is a generic point of an irreducible component of $|X|$.
\end{enumerate}
If true, then there exists an open subspace of $X$
containing $x$ which is a scheme.
\end{lemma}

\begin{proof}
The equivalence of (1), (2), and (3) follows from
Decent Spaces, Lemma \ref{decent-spaces-lemma-decent-generic-points}
and the fact that a quasi-separated algebraic space is decent
(Decent Spaces, Section \ref{decent-spaces-section-reasonable-decent}).
However in the next paragraph we will give a more elementary proof of the
equivalence.

\medskip\noindent
Note that (1) and (2) are equivalent by definition
(Properties of Spaces, Definition
\ref{spaces-properties-definition-dimension-local-ring}).
To prove the equivalence of (1) and (3) we may assume $X$ is quasi-compact.
Choose
$$
\emptyset = U_{n + 1} \subset
U_n \subset U_{n - 1} \subset \ldots \subset U_1 = X
$$
and $f_i : V_i \to U_i$ as in Decent Spaces, Lemma
\ref{decent-spaces-lemma-filter-quasi-compact-quasi-separated}.
Say $x \in U_i$, $x \not \in U_{i + 1}$. Then $x = f_i(y)$ for
a unique $y \in V_i$. If (1) holds, then $y$ is a generic point of
an irreducible component of $V_i$ (Properties of Spaces, Lemma
\ref{spaces-properties-lemma-codimension-0-points}).
Since $f_i^{-1}(U_{i + 1})$ is a quasi-compact open of $V_i$
not containing $y$, there is an open neighbourhood $W \subset V_i$
of $y$ disjoint from $f_i^{-1}(V_i)$
(see
Properties, Lemma \ref{properties-lemma-generic-point-in-constructible}
or more simply Algebra, Lemma
\ref{algebra-lemma-standard-open-containing-maximal-point}).
Then $f_i|_W : W \to X$ is an isomorphism onto its image and hence
$x = f_i(y)$ is a generic point of $|X|$. Conversely, assume (3) holds.
Then $f_i$ maps $\overline{\{y\}}$ onto the irreducible component
$\overline{\{x\}}$ of $|U_i|$. Since $|f_i|$ is bijective over
$\overline{\{x\}}$, it follows that $\overline{\{y\}}$
is an irreducible component of $U_i$. Thus $x$ is a point of
codimension $0$.

\medskip\noindent
The final statement of the lemma is
Properties of Spaces, Proposition
\ref{spaces-properties-proposition-locally-quasi-separated-open-dense-scheme}.
\end{proof}

\noindent
The following lemma says that a separated locally Noetherian algebraic
space is a scheme in codimension $1$, i.e., away from codimension $2$.

\begin{lemma}
\label{lemma-codim-1-point-in-schematic-locus}
\begin{slogan}
Separated algebraic spaces are schemes in codimension 1.
\end{slogan}
Let $S$ be a scheme. Let $X$ be an algebraic space over $S$.
Let $x \in |X|$. If $X$ is separated, locally Noetherian, and
the dimension of the local ring of $X$ at $x$ is $\leq 1$
(Properties of Spaces, Definition
\ref{spaces-properties-definition-dimension-local-ring}),
then there exists an open subspace of $X$ containing $x$ which is a scheme.
\end{lemma}

\begin{proof}
(Please see the remark below for a different approach avoiding the material on
finite groupoids.) We can replace $X$ by an quasi-compact neighbourhood of
$x$, hence we may assume $X$ is quasi-compact, separated, and Noetherian.
There exists a scheme $U$ and a finite surjective morphism $U \to X$,
see Limits of Spaces, Proposition
\ref{spaces-limits-proposition-there-is-a-scheme-finite-over}.
Let $R = U \times_X U$. Then $j : R \to U \times_S U$ is an equivalence
relation and we obtain a groupoid scheme $(U, R, s, t, c)$ over $S$
with $s, t$ finite and $U$ Noetherian and separated.
Let $\{u_1, \ldots, u_n\} \subset U$ be the set of points mapping to $x$. 
Then $\dim(\mathcal{O}_{U, u_i}) \leq 1$ by
Decent Spaces, Lemma
\ref{decent-spaces-lemma-dimension-local-ring-quasi-finite}.

\medskip\noindent
By More on Groupoids, Lemma
\ref{more-groupoids-lemma-find-affine-codimension-1}
there exists an $R$-invariant affine open $W \subset U$ containing
the orbit $\{u_1, \ldots, u_n\}$. Since $U \to X$ is finite surjective
the continuous map $|U| \to |X|$ is closed surjective, hence
submersive by Topology, Lemma
\ref{topology-lemma-closed-morphism-quotient-topology}.
Thus $f(W)$ is open and there is an open subspace $X' \subset X$
with $f : W \to X'$ a surjective finite morphism.
Then $X'$ is an affine scheme by
Cohomology of Spaces, Lemma
\ref{spaces-cohomology-lemma-image-affine-finite-morphism-affine-Noetherian}
and the proof is finished.
\end{proof}

\begin{remark}
\label{remark-alternate-proof-scheme-codim-1}
Here is a sketch of a proof of
Lemma \ref{lemma-codim-1-point-in-schematic-locus}
which avoids using
More on Groupoids, Lemma
\ref{more-groupoids-lemma-find-affine-codimension-1}.

\medskip\noindent
Step 1. We may assume $X$ is a reduced Noetherian separated algebraic space
(for example by Cohomology of Spaces, Lemma
\ref{spaces-cohomology-lemma-image-affine-finite-morphism-affine-Noetherian}
or by
Limits of Spaces, Lemma \ref{spaces-limits-lemma-reduction-scheme})
and we may choose a finite surjective morphism
$Y \to X$ where $Y$ is a Noetherian scheme (by
Limits of Spaces, Proposition
\ref{spaces-limits-proposition-there-is-a-scheme-finite-over}).

\medskip\noindent
Step 2. After replacing $X$ by an open neighbourhood of $x$, there
exists a birational finite morphism $X' \to X$ and a closed subscheme
$Y' \subset X' \times_X Y$ such that $Y' \to X'$ is surjective
finite locally free. Namely, because $X$ is reduced there is a dense
open subspace $U \subset X$ over which $Y$ is flat (Morphisms of Spaces,
Proposition \ref{spaces-morphisms-proposition-generic-flatness-reduced}).
Then we can choose a $U$-admissible blow up $b : \tilde X \to X$ such
that the strict transform $\tilde Y$ of $Y$ is flat over $\tilde X$, see
More on Morphisms of Spaces, Lemma
\ref{spaces-more-morphisms-lemma-flat-after-blowing-up}.
(An alternative is to use Hilbert schemes if one wants to avoid using
the result on blow ups).
Then we let $X' \subset \tilde X$ be the scheme theoretic
closure of $b^{-1}(U)$ and $Y' = X' \times_{\tilde X} \tilde Y$.
Since $x$ is a codimension $1$ point, we see that $X' \to X$ is finite over a
neighbourhood of $x$ (Lemma \ref{lemma-finite-in-codim-1}).

\medskip\noindent
Step 3. After shrinking $X$ to a smaller neighbourhood of $x$ we get that
$X'$ is a scheme. This holds because $Y'$ is a scheme and $Y' \to X'$
being finite locally free and because every finite set of codimension $1$
points of $Y'$ is contained in an affine open. Use
Properties of Spaces, Proposition
\ref{spaces-properties-proposition-finite-flat-equivalence-global}
and
Varieties, Proposition
\ref{varieties-proposition-finite-set-of-points-of-codim-1-in-affine}.

\medskip\noindent
Step 4. There exists an affine open $W' \subset X'$ containing all points
lying over $x$ which is the inverse image of an open subspace of $X$.
To prove this let $Z \subset X$ be the closure of the set of points
where $X' \to X$ is not an isomorphism. We may assume $x \in Z$ otherwise
we are already done. Then $x$ is a generic point of an irreducible
component of $Z$ and after shrinking $X$ we may assume $Z$ is an affine scheme
(Lemma \ref{lemma-generic-point-in-schematic-locus}).
Then the inverse image $Z' \subset X'$ is an affine scheme as well.
Say $x_1, \ldots, x_n \in Z'$ are the points mapping to $x$.
Then we can find an affine open $W'$ in $X'$ whose intersection with
$Z'$ is the inverse image of a principal open of $Z$ containing $x$.
Namely, we first pick an affine open $W' \subset X'$ containing
$x_1, \ldots, x_n$ using Varieties, Proposition
\ref{varieties-proposition-finite-set-of-points-of-codim-1-in-affine}.
Then we pick a principal open $D(f) \subset Z$ containing $x$
whose inverse image $D(f|_{Z'})$ is contained in $W' \cap Z'$.
Then we pick $f' \in \Gamma(W', \mathcal{O}_{W'})$ restricting
to $f|_{Z'}$ and we replace $W'$ by $D(f') \subset W'$.
Since $X' \to X$ is an isomorphism away from $Z' \to Z$ the choice
of $W'$ guarantees that the image $W \subset X$ of $W'$ is open
with inverse image $W'$ in $X'$.

\medskip\noindent
Step 5. Then $W' \to W$ is a finite surjective morphism and $W$ is a scheme by
Cohomology of Spaces, Lemma
\ref{spaces-cohomology-lemma-image-affine-finite-morphism-affine-Noetherian}
and the proof is complete.
\end{remark}





\section{Schematic locus and field extension}
\label{section-schematic-and-field-extension}

\noindent
It can happen that a nonrepresentable algebraic space over a field $k$
becomes representable (i.e., a scheme) after base change to an extension
of $k$. See Spaces, Example \ref{spaces-example-non-representable-descent}.
In this section we address this issue.

\begin{lemma}
\label{lemma-scheme-after-purely-inseparable-base-change}
Let $k$ be a field. Let $X$ be an algebraic space over $k$.
If there exists a purely inseparable field extension $k \subset k'$
such that $X_{k'}$ is a scheme, then $X$ is a scheme.
\end{lemma}

\begin{proof}
The morphism $X_{k'} \to X$ is integral, surjective, and
universally injective. Hence this lemma follows from
Limits of Spaces, Lemma
\ref{spaces-limits-lemma-integral-universally-bijective-scheme}.
\end{proof}

\begin{lemma}
\label{lemma-when-scheme-after-base-change}
Let $k$ be a field with algebraic closure $\overline{k}$.
Let $X$ be a quasi-separated algebraic space over $k$.
\begin{enumerate}
\item If there exists a field extension $k \subset K$ such that
$X_K$ is a scheme, then $X_{\overline{k}}$ is a scheme.
\item If $X$ is quasi-compact and there exists a field extension
$k \subset K$ such that $X_K$ is a scheme, then $X_{k'}$
is a scheme for some finite separable extension $k'$ of $k$.
\end{enumerate}
\end{lemma}

\begin{proof}
Since every algebraic space is the union of its quasi-compact open
subspaces, we see that the first part of the lemma follows from
the second part (some details omitted). Thus we assume $X$ is quasi-compact
and we assume given an extension $k \subset K$ with $K_K$ representable.
Write $K = \bigcup A$ as the colimit of finitely generated $k$-subalgebras
$A$. By Limits of Spaces, Lemma \ref{spaces-limits-lemma-limit-is-scheme}
we see that $X_A$ is a scheme for some $A$. Choose a maximal ideal
$\mathfrak m \subset A$. By the Hilbert Nullstellensatz
(Algebra, Theorem \ref{algebra-theorem-nullstellensatz})
the residue field $k' = A/\mathfrak m$ is a finite extension of $k$.
Thus we see that $X_{k'}$ is a scheme. If $k' \supset k$ is not
separable, let $k' \supset k'' \supset k$ be the subextension
found in Fields, Lemma \ref{fields-lemma-separable-first}.
Since $k'/k''$ is purely inseparable, by
Lemma \ref{lemma-scheme-after-purely-inseparable-base-change}
the algebraic space $X_{k''}$ is a scheme. Since $k''|k$ is separable
the proof is complete.
\end{proof}

\begin{lemma}
\label{lemma-base-change-by-Galois}
Let $k \subset k'$ be a finite Galois extension with Galois group $G$.
Let $X$ be an algebraic space over $k$. Then $G$ acts freely on the
algebraic space $X_{k'}$ and $X = X_{k'}/G$ in the sense of
Properties of Spaces, Lemma \ref{spaces-properties-lemma-quotient}.
\end{lemma}

\begin{proof}
Omitted. Hints: First show that $\Spec(k) = \Spec(k')/G$.
Then use compatinility of taking quotients with base change.
\end{proof}

\begin{lemma}
\label{lemma-when-quotient-scheme-at-point}
Let $S$ be a scheme. Let $X$ be an algebraic space over $S$ and
let $G$ be a finite group acting freely on $X$. Set $Y = X/G$ as
in Properties of Spaces, Lemma \ref{spaces-properties-lemma-quotient}.
For $y \in |Y|$ the following are equivalent
\begin{enumerate}
\item $y$ is in the schematic locus of $Y$, and
\item there exists an affine open $U \subset X$
containing the preimage of $y$.
\end{enumerate}
\end{lemma}

\begin{proof}
It follows from the construction of $Y = X/G$ in
Properties of Spaces, Lemma \ref{spaces-properties-lemma-quotient}
that the morphism $X \to Y$ is surjective and \'etale.
Of course we have $X \times_Y X = X \times G$ hence the morphism
$X \to Y$ is even finite \'etale. It is also surjective.
Thus the lemma follows from
Decent Spaces, Lemma \ref{decent-spaces-lemma-when-quotient-scheme-at-point}.
\end{proof}

\begin{lemma}
\label{lemma-scheme-after-purely-transcendental-base-change}
Let $k$ be a field. Let $X$ be a quasi-separated
algebraic space over $k$. If there exists a purely transcendental
field extension $k \subset K$ such that $X_K$ is a scheme, then
$X$ is a scheme.
\end{lemma}

\begin{proof}
Since every algebraic space is the union of its quasi-compact open
subspaces, we may assume $X$ is quasi-compact (some details omitted).
Recall (Fields, Definition \ref{fields-definition-transcendence})
that the assumption on the extension $K/k$ signifies that
$K$ is the fraction field of a polynomial ring (in possibly infinitely
many variables) over $k$. Thus $K = \bigcup A$ is the union of subalgebras
each of which is a localization of a finite polynomial algebra over $k$.
By Limits of Spaces, Lemma \ref{spaces-limits-lemma-limit-is-scheme}
we see that $X_A$ is a scheme for some $A$. Write
$$
A = k[x_1, \ldots, x_n][1/f]
$$
for some nonzero $f \in k[x_1, \ldots, x_n]$.

\medskip\noindent
If $k$ is infinite then we can finish the proof as follows: choose
$a_1, \ldots, a_n \in k$ with $f(a_1, \ldots, a_n) \not = 0$.
Then $(a_1, \ldots, a_n)$ define an $k$-algebra map $A \to k$
mapping $x_i$ to $a_i$ and $1/f$ to $1/f(a_1, \ldots, a_n)$.
Thus the base change $X_A \times_{\Spec(A)} \Spec(k) \cong X$ is a
scheme as desired.

\medskip\noindent
In this paragraph we finish the proof in case $k$ is finite. In this
case we write $X = \lim X_i$ with $X_i$ of finite presentation over $k$
and with affine transition morphisms
(Limits of Spaces, Lemma \ref{spaces-limits-lemma-relative-approximation}).
Using Limits of Spaces, Lemma \ref{spaces-limits-lemma-limit-is-scheme}
we see that $X_{i, A}$ is a scheme for some $i$. Thus we may assume
$X \to \Spec(k)$ is of finite presentation. Let $x \in |X|$ be a closed
point. We may represent $x$ by a closed immersion
$\Spec(\kappa) \to X$
(Decent Spaces, Lemma \ref{decent-spaces-lemma-decent-space-closed-point}).
Then $\Spec(\kappa) \to \Spec(k)$ is of finite type, hence $\kappa$
is a finite extension of $k$ (by the Hilbert Nullstellensatz, see
Algebra, Theorem \ref{algebra-theorem-nullstellensatz};
some details omitted). Say $[\kappa : k] = d$. Choose an integer
$n \gg 0$ prime to $d$ and let $k \subset k'$ be the extension
of degree $n$. Then $k'/k$ is Galois with $G = \text{Aut}(k'/k)$
cyclic of order $n$. If $n$ is large enough there will be $k$-algebra
homomorphism $A \to k'$ by the same reason as above.
Then $X_{k'}$ is a scheme and $X = X_{k'}/G$
(Lemma \ref{lemma-base-change-by-Galois}).
On the other hand, since $n$ and $d$ are relatively prime we see that
$$
\Spec(\kappa) \times_{X} X_{k'} =
\Spec(\kappa) \times_{\Spec(k)} \Spec(k') =
\Spec(\kappa \otimes_k k')
$$
is the spectrum of a field. In other words, the fibre of $X_{k'} \to X$
over $x$ consists of a single point. Thus by
Lemma \ref{lemma-when-quotient-scheme-at-point}
we see that $x$ is in the schematic locus of $X$ as desired.
\end{proof}

\begin{remark}
\label{remark-when-does-the-argument-work}
Let $k$ be finite field. Let $K \supset k$ be a geometrically
irreducible field extension. Then $K$ is the limit of geometrically
irreducible finite type $k$-algebras $A$. Given $A$ the estimates
of Lang and Weil \cite{LW}, show that for $n \gg 0$ there exists
an $k$-algebra homomorphism $A \to k'$ with $k'/k$ of degree $n$.
Analyzing the argument given in the proof of
Lemma \ref{lemma-scheme-after-purely-transcendental-base-change}
we see that if $X$ is a quasi-separated algebraic space over $k$
and $X_K$ is a scheme, then $X$ is a scheme. If we ever need this
result we will precisely formulate it and prove it here.
\end{remark}

\begin{lemma}
\label{lemma-scheme-over-algebraic-closure-enough-affines}
Let $k$ be a field with algebraic closure $\overline{k}$. Let $X$
be an algebraic space over $k$ such that
\begin{enumerate}
\item $X$ is decent and locally of finite type over $k$,
\item $X_{\overline{k}}$ is a scheme, and
\item any finite set of $\overline{k}$-rational points of $X_{\overline{k}}$
are contained in an affine.
\end{enumerate}
Then $X$ is a scheme.
\end{lemma}

\begin{proof}
If $k \subset K$ is an extension, then the base change $X_K$ is
decent (Decent Spaces, Lemma
\ref{decent-spaces-lemma-representable-named-properties})
and locally of finite type
over $K$ (Morphisms of Spaces, Lemma
\ref{spaces-morphisms-lemma-base-change-finite-type}).
By Lemma \ref{lemma-scheme-after-purely-inseparable-base-change}
it suffices to prove that $X$ becomes a scheme after base change to
the perfection of $k$, hence we may assume $k$ is a perfect field
(this step isn't strictly necessary, but makes the other arguments
easier to think about).
By covering $X$ by quasi-compact opens we see that it suffices to prove
the lemma in case $X$ is quasi-compact (small detail omitted).
In this case $|X|$ is a sober topological space
(Decent Spaces, Proposition
\ref{decent-spaces-proposition-reasonable-sober}).
Hence it suffices to show that every closed point in $|X|$
is contained in the schematic locus of $X$
(use Properties of Spaces, Lemma \ref{spaces-properties-lemma-subscheme} and
Topology, Lemma \ref{topology-lemma-quasi-compact-closed-point}).

\medskip\noindent
Let $x \in |X|$ be a closed point. By Decent Spaces, Lemma
\ref{decent-spaces-lemma-decent-space-closed-point}
we can find a closed immersion $\Spec(l) \to X$ representing $x$.
Then $\Spec(l) \to \Spec(k)$ is of finite type (Morphisms of Spaces,
Lemma \ref{spaces-morphisms-lemma-composition-finite-type}) and we
conclude that $l$ is a finite extension of $k$
by the Hilbert Nullstellensatz (Algebra, Theorem
\ref{algebra-theorem-nullstellensatz}). It is separable because
$k$ is perfect. Thus the scheme
$$
\Spec(l) \times_X X_{\overline{k}} =
\Spec(l) \times_{\Spec(k)} \Spec(\overline{k}) =
\Spec(l \otimes_k \overline{k})
$$
is the disjoint union of a finite number of $\overline{k}$-rational points.
By assumption (3) we can find an affine open $W \subset X_{\overline{k}}$
containing these points.

\medskip\noindent
By Lemma \ref{lemma-when-scheme-after-base-change} we see that $X_{k'}$
is a scheme for some finite extension $k'/k$. After enlarging
$k'$ we may assume that there exists an affine open $U' \subset X_{k'}$
whose base change to $\overline{k}$ recovers $W$
(use that $X_{\overline{k}}$ is the limit of the schemes $X_{k''}$
for $k' \subset k'' \subset \overline{k}$ finite and use
Limits, Lemmas \ref{limits-lemma-descend-opens} and
\ref{limits-lemma-limit-affine}). We may assume
that $k'/k$ is a Galois extension (take the normal closure
Fields, Lemma \ref{fields-lemma-normal-closure} and use
that $k$ is perfect). Set $G = \text{Gal}(k'/k)$.
By construction the $G$-invariant closed subscheme
$\Spec(l) \times_X X_{k'}$ is contained in $U'$.
Thus $x$ is in the schematic locus by
Lemmas \ref{lemma-base-change-by-Galois} and
\ref{lemma-when-quotient-scheme-at-point}.
\end{proof}

\noindent
The following two lemmas should go somewhere else.
Please compare the next lemma to
Decent Spaces, Lemma \ref{decent-spaces-lemma-conditions-on-space-over-field}.

\begin{lemma}
\label{lemma-locally-quasi-finite-over-field}
Let $k$ be a field. Let $X$ be an algebraic space over $k$.
The following are equivalent
\begin{enumerate}
\item $X$ is locally quasi-finite over $k$,
\item $X$ is locally of finite type over $k$ and has dimension $0$,
\item $X$ is a scheme and is locally quasi-finite over $k$,
\item $X$ is a scheme and is locally of finite type over $k$ and has
dimension $0$, and
\item $X$ is a disjoint union of spectra of Artinian local $k$-algebras
$A$ over $k$ with $\dim_k(A) < \infty$.
\end{enumerate}
\end{lemma}

\begin{proof}
Because we are over a field relative dimension of $X/k$ is the same as
the dimension of $X$. Hence by
Morphisms of Spaces,
Lemma \ref{spaces-morphisms-lemma-locally-quasi-finite-rel-dimension-0}
we see that (1) and (2) are equivalent. Hence it follows from
Lemma \ref{lemma-locally-finite-type-dim-zero}
(and trivial implications) that (1) -- (4) are equivalent.
Finally,
Varieties, Lemma \ref{varieties-lemma-algebraic-scheme-dim-0}
shows that (1) -- (4) are equivalent with (5).
\end{proof}

\begin{lemma}
\label{lemma-mono-towards-locally-quasi-finite-over-field}
Let $k$ be a field. Let $f : X \to Y$ be a monomorphism of algebraic spaces
over $k$. If $Y$ is locally quasi-finite over $k$ so is $X$.
\end{lemma}

\begin{proof}
Assume $Y$ is locally quasi-finite over $k$. By
Lemma \ref{lemma-locally-quasi-finite-over-field}
we see that $Y = \coprod \Spec(A_i)$ where each $A_i$ is an
Artinian local ring finite over $k$. By
Decent Spaces, Lemma
\ref{decent-spaces-lemma-monomorphism-toward-disjoint-union-dim-0-rings}
we see that $X$ is a scheme. Consider $X_i = f^{-1}(\Spec(A_i))$.
Then $X_i$ has either one or zero points. If $X_i$ has zero points there
is nothing to prove. If $X_i$ has one point, then
$X_i = \Spec(B_i)$ with $B_i$ a zero dimensional local ring
and $A_i \to B_i$ is an epimorphism of rings. In particular
$A_i/\mathfrak m_{A_i} = B_i/\mathfrak m_{A_i}B_i$ and we see that
$A_i \to B_i$ is surjective by Nakayama's lemma,
Algebra, Lemma \ref{algebra-lemma-NAK}
(because $\mathfrak m_{A_i}$ is a nilpotent ideal!).
Thus $B_i$ is a finite local $k$-algebra, and we conclude by
Lemma \ref{lemma-locally-quasi-finite-over-field}
that $X \to \Spec(k)$ is locally quasi-finite.
\end{proof}




\section{Geometrically connected algebraic spaces}
\label{section-geometrically-connected}

\noindent
If $X$ is a connected algebraic space over a field, then it can happen that
$X$ becomes disconnected after extending the ground field. This does not
happen for geometrically connected schemes.

\begin{definition}
\label{definition-geometrically-connected}
Let $X$ be an algebraic space over the field $k$. We say $X$ is
{\it geometrically connected} over $k$ if the base change $X_{k'}$
is connected for every field extension $k'$ of $k$.
\end{definition}

\noindent
By convention a connected topological space is nonempty; hence a fortiori
geometrically connected algebraic spaces are nonempty.

\begin{lemma}
\label{lemma-geometrically-connected-check-after-extension}
Let $X$ be an algebraic space over the field $k$.
Let $k \subset k'$ be a field extension.
Then $X$ is geometrically connected over $k$ if and only if
$X_{k'}$ is geometrically connected over $k'$.
\end{lemma}

\begin{proof}
If $X$ is geometrically connected over $k$, then it is clear that
$X_{k'}$ is geometrically connected over $k'$. For the converse, note
that for any field extension $k \subset k''$ there exists a common
field extension $k' \subset k'''$ and $k'' \subset k'''$. As the
morphism $X_{k'''} \to X_{k''}$ is surjective (as a base change of
a surjective morphism between spectra of fields) we see that the
connectedness of $X_{k'''}$ implies the connectedness of $X_{k''}$.
Thus if $X_{k'}$ is geometrically connected over $k'$ then
$X$ is geometrically connected over $k$.
\end{proof}

\begin{lemma}
\label{lemma-bijection-connected-components}
Let $k$ be a field. Let $X$, $Y$ be algebraic spaces over $k$.
Assume $X$ is geometrically connected over $k$.
Then the projection morphism
$$
p : X \times_k Y \longrightarrow Y
$$
induces a bijection between connected components.
\end{lemma}

\begin{proof}
Let $y \in |Y|$ be represented by a morphism $\Spec(K) \to Y$ be a morphism
where $K$ is a field. The fibre of $|X \times_k Y| \to |Y|$ over $y$
is the image of $|Y_K| \to |X \times_k Y|$ by
Properties of Spaces, Lemma \ref{spaces-properties-lemma-points-cartesian}.
Thus these fibres are connected by our assumption that $Y$ is
geometrically connected. By
Morphisms of Spaces, Lemma
\ref{spaces-morphisms-lemma-space-over-field-universally-open}
the map $|p|$ is open.
Thus we may apply Topology,
Lemma \ref{topology-lemma-connected-fibres-connected-components}
to conclude.
\end{proof}

\begin{lemma}
\label{lemma-separably-closed-field-connected-components}
Let $k \subset k'$ be an extension of fields. Let $X$ be an algebraic space
over $k$. Assume $k$ separably algebraically closed. Then the morphism
$X_{k'} \to X$ induces a bijection of connected components. In particular,
$X$ is geometrically connected over $k$ if and only if $X$ is connected.
\end{lemma}

\begin{proof}
Since $k$ is separably algebraically closed we see that
$k'$ is geometrically connected over $k$, see
Algebra,
Lemma \ref{algebra-lemma-separably-closed-connected-implies-geometric}.
Hence $Z = \Spec(k')$ is geometrically connected over $k$ by
Varieties, Lemma \ref{varieties-lemma-affine-geometrically-connected}.
Since $X_{k'} = Z \times_k X$ the result is a special case of
Lemma \ref{lemma-bijection-connected-components}.
\end{proof}

\begin{lemma}
\label{lemma-characterize-geometrically-connected}
Let $k$ be a field. Let $X$ be an algebraic space over $k$.
Let $\overline{k}$ be a separable algebraic closure of $k$.
Then $X$ is geometrically connected if and only if the base change
$X_{\overline{k}}$ is connected.
\end{lemma}

\begin{proof}
Assume $X_{\overline{k}}$ is connected. Let $k \subset k'$ be a field
extension. There exists a field extension $\overline{k} \subset \overline{k}'$
such that $k'$ embeds into $\overline{k}'$ as an extension of $k$.
By Lemma \ref{lemma-separably-closed-field-connected-components}
we see that $X_{\overline{k}'}$ is connected.
Since $X_{\overline{k}'} \to X_{k'}$ is surjective we conclude
that $X_{k'}$ is connected as desired.
\end{proof}

\noindent
Let $k$ be a field. Let $k \subset \overline{k}$ be a (possibly infinite)
Galois extension. For example $\overline{k}$ could be the
separable algebraic closure of $k$.
For any $\sigma \in \text{Gal}(\overline{k}/k)$ we get a corresponding
automorphism
$
\Spec(\sigma) :
\Spec(\overline{k})
\longrightarrow
\Spec(\overline{k})
$.
Note that
$\Spec(\sigma) \circ \Spec(\tau) = \Spec(\tau \circ \sigma)$.
Hence we get an action
$$
\text{Gal}(\overline{k}/k)^{opp} \times \Spec(\overline{k})
\longrightarrow
\Spec(\overline{k})
$$
of the opposite group on the scheme $\Spec(\overline{k})$.
Let $X$ be an algebraic space over $k$. Since
$X_{\overline{k}} =
\Spec(\overline{k}) \times_{\Spec(k)} X$
by definition we see that the action above induces a canonical action
\begin{equation}
\label{equation-galois-action-base-change-kbar}
\text{Gal}(\overline{k}/k)^{opp} \times X_{\overline{k}}
\longrightarrow
X_{\overline{k}}.
\end{equation}

\begin{lemma}
\label{lemma-Galois-action-quasi-compact-open}
Let $k$ be a field. Let $X$ be an algebraic space over $k$.
Let $\overline{k}$ be a (possibly infinite) Galois extension of $k$.
Let $V \subset X_{\overline{k}}$ be a quasi-compact open.
Then
\begin{enumerate}
\item there exists a finite subextension $k \subset k' \subset \overline{k}$
and a quasi-compact open $V' \subset X_{k'}$ such that
$V = (V')_{\overline{k}}$,
\item there exists an open subgroup $H \subset \text{Gal}(\overline{k}/k)$
such that $\sigma(V) = V$ for all $\sigma \in H$.
\end{enumerate}
\end{lemma}

\begin{proof}
Choose a scheme $U$ and a surjective \'etale morphism $U \to X$.
Choose a quasi-compact open $W \subset U_{\overline{k}}$ whose
image in $X_{\overline{k}}$ is $V$. This is possible because
$|U_{\overline{k}}| \to |X_{\overline{k}}|$ is continuous and because
$|U_{\overline{k}}|$ has a basis of quasi-compact opens. We can apply
Varieties, Lemma
\ref{varieties-lemma-Galois-action-quasi-compact-open}
to $W \subset U_{\overline{k}}$ to obtain the lemma.
\end{proof}

\begin{lemma}
\label{lemma-closed-fixed-by-Galois}
Let $k$ be a field. Let $k \subset \overline{k}$ be a (possibly infinite)
Galois extension. Let $X$ be an algebraic space over $k$. Let
$\overline{T} \subset |X_{\overline{k}}|$ have the following properties
\begin{enumerate}
\item $\overline{T}$ is a closed subset of $|X_{\overline{k}}|$,
\item for every $\sigma \in \text{Gal}(\overline{k}/k)$
we have $\sigma(\overline{T}) = \overline{T}$.
\end{enumerate}
Then there exists a closed subset $T \subset |X|$ whose inverse image
in $|X_{k'}|$ is $\overline{T}$.
\end{lemma}

\begin{proof}
Let $T \subset |X|$ be the image of $\overline{T}$.
Since $|X_{\overline{k}}| \to |X|$ is surjective, the statement means
that $T$ is closed and that its inverse image is $\overline{T}$.
Choose a scheme $U$ and a surjective \'etale morphism $U \to X$.
By the case of schemes
(see Varieties, Lemma \ref{varieties-lemma-closed-fixed-by-Galois})
there exists a closed subset $T' \subset |U|$ whose inverse image
in $|U_{\overline{k}}|$ is the inverse image of $\overline{T}$.
Since $|U_{\overline{k}}| \to |X_{\overline{k}}|$ is surjective,
we see that $T'$ is the inverse image of $T$ via $|U| \to |X|$.
By our construction of the topology on $|X|$ this means that $T$ is
closed. In the same manner one sees that $\overline{T}$ is the inverse
image of $T$.
\end{proof}

\begin{lemma}
\label{lemma-characterize-geometrically-disconnected}
Let $k$ be a field. Let $X$ be an algebraic space over $k$.
The following are equivalent
\begin{enumerate}
\item $X$ is geometrically connected,
\item for every finite separable field extension $k \subset k'$
the scheme $X_{k'}$ is connected.
\end{enumerate}
\end{lemma}

\begin{proof}
This proof is identical to the proof of
Varieties, Lemma \ref{varieties-lemma-characterize-geometrically-disconnected}
except that
we replace
Varieties, Lemma \ref{varieties-lemma-characterize-geometrically-connected}
by Lemma \ref{lemma-characterize-geometrically-connected},
we replace
Varieties, Lemma \ref{varieties-lemma-Galois-action-quasi-compact-open}
by Lemma \ref{lemma-Galois-action-quasi-compact-open}, and
we replace
Varieties, Lemma \ref{varieties-lemma-closed-fixed-by-Galois}
by Lemma \ref{lemma-closed-fixed-by-Galois}.
We urge the reader to read that proof in stead of this one.

\medskip\noindent
It follows immediately from the definition that (1) implies (2).
Assume that $X$ is not geometrically connected.
Let $k \subset \overline{k}$ be a separable algebraic
closure of $k$. By
Lemma \ref{lemma-characterize-geometrically-connected}
it follows that $X_{\overline{k}}$ is disconnected.
Say $X_{\overline{k}} = \overline{U} \amalg \overline{V}$
with $\overline{U}$ and $\overline{V}$ open, closed, and nonempty
algebraic subspaces of $X_{\overline{k}}$.

\medskip\noindent
Suppose that $W \subset X$ is any quasi-compact open subspace.
Then $W_{\overline{k}} \cap \overline{U}$ and
$W_{\overline{k}} \cap \overline{V}$ are open and closed subspaces of
$W_{\overline{k}}$. In particular $W_{\overline{k}} \cap \overline{U}$ and
$W_{\overline{k}} \cap \overline{V}$ are quasi-compact, and by
Lemma \ref{lemma-Galois-action-quasi-compact-open}
both $W_{\overline{k}} \cap \overline{U}$ and
$W_{\overline{k}} \cap \overline{V}$
are defined over a finite subextension and invariant under an
open subgroup of $\text{Gal}(\overline{k}/k)$.
We will use this without further mention in the following.

\medskip\noindent
Pick $W_0 \subset X$ quasi-compact open subspace such that both
$W_{0, \overline{k}} \cap \overline{U}$ and
$W_{0, \overline{k}} \cap \overline{V}$ are nonempty.
Choose a finite subextension $k \subset k' \subset \overline{k}$
and a decomposition $W_{0, k'} = U_0' \amalg V_0'$ into open and closed
subsets such that
$W_{0, \overline{k}} \cap \overline{U} = (U'_0)_{\overline{k}}$ and
$W_{0, \overline{k}} \cap \overline{V} = (V'_0)_{\overline{k}}$.
Let $H = \text{Gal}(\overline{k}/k') \subset \text{Gal}(\overline{k}/k)$.
In particular
$\sigma(W_{0, \overline{k}} \cap \overline{U}) =
W_{0, \overline{k}} \cap \overline{U}$ and similarly for
$\overline{V}$.

\medskip\noindent
Having chosen $W_0$, $k'$ as above, for every quasi-compact open subspace
$W \subset X$ we set
$$
U_W =
\bigcap\nolimits_{\sigma \in H} \sigma(W_{\overline{k}} \cap \overline{U}),
\quad
V_W =
\bigcup\nolimits_{\sigma \in H} \sigma(W_{\overline{k}} \cap \overline{V}).
$$
Now, since $W_{\overline{k}} \cap \overline{U}$ and
$W_{\overline{k}} \cap \overline{V}$ are fixed by an open subgroup of
$\text{Gal}(\overline{k}/k)$ we see that the union and intersection
above are finite. Hence $U_W$ and $V_W$ are both open and closed subspaces.
Also, by construction $W_{\bar k} = U_W \amalg V_W$.

\medskip\noindent
We claim that if $W \subset W' \subset X$ are quasi-compact
open subspaces, then $W_{\overline{k}} \cap U_{W'} = U_W$ and
$W_{\overline{k}} \cap V_{W'} = V_W$. Verification omitted.
Hence we see that upon defining $U = \bigcup_{W \subset X} U_W$
and $V = \bigcup_{W \subset X} V_W$ we obtain
$X_{\overline{k}} = U \amalg V$ is a disjoint union of open
and closed subsets.
It is clear that $V$ is nonempty as it is constructed by taking
unions (locally). On the other hand, $U$ is nonempty since it contains
$W_0 \cap \overline{U}$ by construction. Finally, $U, V \subset X_{\bar k}$
are closed and $H$-invariant by construction. Hence by
Lemma \ref{lemma-closed-fixed-by-Galois}
we have $U = (U')_{\bar k}$, and $V = (V')_{\bar k}$ for some
closed $U', V' \subset X_{k'}$. Clearly $X_{k'} = U' \amalg V'$
and we see that $X_{k'}$ is disconnected as desired.
\end{proof}









\section{Spaces smooth over fields}
\label{section-smooth}



\begin{lemma}
\label{lemma-smooth-regular}
Let $k$ be a field.
Let $X$ be an algebraic space smooth over $k$.
Then $X$ is a regular algebraic space.
\end{lemma}

\begin{proof}
Choose a scheme $U$ and a surjective \'etale morphism $U \to X$.
The morphism $U \to \Spec(k)$ is smooth as a composition of
an \'etale (hence smooth) morphism and a smooth morphism (see
Morphisms of Spaces, Lemmas \ref{spaces-morphisms-lemma-etale-smooth}
and \ref{spaces-morphisms-lemma-composition-smooth}).
Hence $U$ is regular by
Varieties, Lemma \ref{varieties-lemma-smooth-regular}.
By
Properties of Spaces, Definition
\ref{spaces-properties-definition-type-property}
this means that $X$ is regular.
\end{proof}

\begin{lemma}
\label{lemma-smooth-separable-closed-points-dense}
Let $k$ be a field. Let $X$ be an algebraic space smooth over $\Spec(k)$.
The set of $x \in |X|$ which are image of morphisms $\Spec(k') \to X$
with $k' \supset k$ finite separable is dense in $|X|$.
\end{lemma}

\begin{proof}
Choose a scheme $U$ and a surjective \'etale morphism $U \to X$.
The morphism $U \to \Spec(k)$ is smooth as a composition of
an \'etale (hence smooth) morphism and a smooth morphism (see
Morphisms of Spaces, Lemmas \ref{spaces-morphisms-lemma-etale-smooth}
and \ref{spaces-morphisms-lemma-composition-smooth}).
Hence we can apply Varieties, Lemma
\ref{varieties-lemma-smooth-separable-closed-points-dense} to see that
the closed points of $U$ whose residue fields are finite separable over
$k$ are dense. This implies the lemma by our definition of the
topology on $|X|$.
\end{proof}








\begin{multicols}{2}[\section{Other chapters}]
\noindent
Preliminaries
\begin{enumerate}
\item \hyperref[introduction-section-phantom]{Introduction}
\item \hyperref[conventions-section-phantom]{Conventions}
\item \hyperref[sets-section-phantom]{Set Theory}
\item \hyperref[categories-section-phantom]{Categories}
\item \hyperref[topology-section-phantom]{Topology}
\item \hyperref[sheaves-section-phantom]{Sheaves on Spaces}
\item \hyperref[sites-section-phantom]{Sites and Sheaves}
\item \hyperref[stacks-section-phantom]{Stacks}
\item \hyperref[fields-section-phantom]{Fields}
\item \hyperref[algebra-section-phantom]{Commutative Algebra}
\item \hyperref[brauer-section-phantom]{Brauer Groups}
\item \hyperref[homology-section-phantom]{Homological Algebra}
\item \hyperref[derived-section-phantom]{Derived Categories}
\item \hyperref[simplicial-section-phantom]{Simplicial Methods}
\item \hyperref[more-algebra-section-phantom]{More on Algebra}
\item \hyperref[smoothing-section-phantom]{Smoothing Ring Maps}
\item \hyperref[modules-section-phantom]{Sheaves of Modules}
\item \hyperref[sites-modules-section-phantom]{Modules on Sites}
\item \hyperref[injectives-section-phantom]{Injectives}
\item \hyperref[cohomology-section-phantom]{Cohomology of Sheaves}
\item \hyperref[sites-cohomology-section-phantom]{Cohomology on Sites}
\item \hyperref[dga-section-phantom]{Differential Graded Algebra}
\item \hyperref[dpa-section-phantom]{Divided Power Algebra}
\item \hyperref[hypercovering-section-phantom]{Hypercoverings}
\end{enumerate}
Schemes
\begin{enumerate}
\setcounter{enumi}{24}
\item \hyperref[schemes-section-phantom]{Schemes}
\item \hyperref[constructions-section-phantom]{Constructions of Schemes}
\item \hyperref[properties-section-phantom]{Properties of Schemes}
\item \hyperref[morphisms-section-phantom]{Morphisms of Schemes}
\item \hyperref[coherent-section-phantom]{Cohomology of Schemes}
\item \hyperref[divisors-section-phantom]{Divisors}
\item \hyperref[limits-section-phantom]{Limits of Schemes}
\item \hyperref[varieties-section-phantom]{Varieties}
\item \hyperref[topologies-section-phantom]{Topologies on Schemes}
\item \hyperref[descent-section-phantom]{Descent}
\item \hyperref[perfect-section-phantom]{Derived Categories of Schemes}
\item \hyperref[more-morphisms-section-phantom]{More on Morphisms}
\item \hyperref[flat-section-phantom]{More on Flatness}
\item \hyperref[groupoids-section-phantom]{Groupoid Schemes}
\item \hyperref[more-groupoids-section-phantom]{More on Groupoid Schemes}
\item \hyperref[etale-section-phantom]{\'Etale Morphisms of Schemes}
\end{enumerate}
Topics in Scheme Theory
\begin{enumerate}
\setcounter{enumi}{40}
\item \hyperref[chow-section-phantom]{Chow Homology}
\item \hyperref[intersection-section-phantom]{Intersection Theory}
\item \hyperref[weil-section-phantom]{Weil Cohomology Theories}
\item \hyperref[pic-section-phantom]{Picard Schemes of Curves}
\item \hyperref[adequate-section-phantom]{Adequate Modules}
\item \hyperref[dualizing-section-phantom]{Dualizing Complexes}
\item \hyperref[duality-section-phantom]{Duality for Schemes}
\item \hyperref[discriminant-section-phantom]{Discriminants and Differents}
\item \hyperref[local-cohomology-section-phantom]{Local Cohomology}
\item \hyperref[algebraization-section-phantom]{Algebraic and Formal Geometry}
\item \hyperref[curves-section-phantom]{Algebraic Curves}
\item \hyperref[resolve-section-phantom]{Resolution of Surfaces}
\item \hyperref[models-section-phantom]{Semistable Reduction}
\item \hyperref[pione-section-phantom]{Fundamental Groups of Schemes}
\item \hyperref[etale-cohomology-section-phantom]{\'Etale Cohomology}
\item \hyperref[crystalline-section-phantom]{Crystalline Cohomology}
\item \hyperref[proetale-section-phantom]{Pro-\'etale Cohomology}
\item \hyperref[more-etale-section-phantom]{More \'Etale Cohomology}
\item \hyperref[trace-section-phantom]{The Trace Formula}
\end{enumerate}
Algebraic Spaces
\begin{enumerate}
\setcounter{enumi}{59}
\item \hyperref[spaces-section-phantom]{Algebraic Spaces}
\item \hyperref[spaces-properties-section-phantom]{Properties of Algebraic Spaces}
\item \hyperref[spaces-morphisms-section-phantom]{Morphisms of Algebraic Spaces}
\item \hyperref[decent-spaces-section-phantom]{Decent Algebraic Spaces}
\item \hyperref[spaces-cohomology-section-phantom]{Cohomology of Algebraic Spaces}
\item \hyperref[spaces-limits-section-phantom]{Limits of Algebraic Spaces}
\item \hyperref[spaces-divisors-section-phantom]{Divisors on Algebraic Spaces}
\item \hyperref[spaces-over-fields-section-phantom]{Algebraic Spaces over Fields}
\item \hyperref[spaces-topologies-section-phantom]{Topologies on Algebraic Spaces}
\item \hyperref[spaces-descent-section-phantom]{Descent and Algebraic Spaces}
\item \hyperref[spaces-perfect-section-phantom]{Derived Categories of Spaces}
\item \hyperref[spaces-more-morphisms-section-phantom]{More on Morphisms of Spaces}
\item \hyperref[spaces-flat-section-phantom]{Flatness on Algebraic Spaces}
\item \hyperref[spaces-groupoids-section-phantom]{Groupoids in Algebraic Spaces}
\item \hyperref[spaces-more-groupoids-section-phantom]{More on Groupoids in Spaces}
\item \hyperref[bootstrap-section-phantom]{Bootstrap}
\item \hyperref[spaces-pushouts-section-phantom]{Pushouts of Algebraic Spaces}
\end{enumerate}
Topics in Geometry
\begin{enumerate}
\setcounter{enumi}{76}
\item \hyperref[spaces-chow-section-phantom]{Chow Groups of Spaces}
\item \hyperref[groupoids-quotients-section-phantom]{Quotients of Groupoids}
\item \hyperref[spaces-more-cohomology-section-phantom]{More on Cohomology of Spaces}
\item \hyperref[spaces-simplicial-section-phantom]{Simplicial Spaces}
\item \hyperref[spaces-duality-section-phantom]{Duality for Spaces}
\item \hyperref[formal-spaces-section-phantom]{Formal Algebraic Spaces}
\item \hyperref[restricted-section-phantom]{Restricted Power Series}
\item \hyperref[spaces-resolve-section-phantom]{Resolution of Surfaces Revisited}
\end{enumerate}
Deformation Theory
\begin{enumerate}
\setcounter{enumi}{84}
\item \hyperref[formal-defos-section-phantom]{Formal Deformation Theory}
\item \hyperref[defos-section-phantom]{Deformation Theory}
\item \hyperref[cotangent-section-phantom]{The Cotangent Complex}
\item \hyperref[examples-defos-section-phantom]{Deformation Problems}
\end{enumerate}
Algebraic Stacks
\begin{enumerate}
\setcounter{enumi}{88}
\item \hyperref[algebraic-section-phantom]{Algebraic Stacks}
\item \hyperref[examples-stacks-section-phantom]{Examples of Stacks}
\item \hyperref[stacks-sheaves-section-phantom]{Sheaves on Algebraic Stacks}
\item \hyperref[criteria-section-phantom]{Criteria for Representability}
\item \hyperref[artin-section-phantom]{Artin's Axioms}
\item \hyperref[quot-section-phantom]{Quot and Hilbert Spaces}
\item \hyperref[stacks-properties-section-phantom]{Properties of Algebraic Stacks}
\item \hyperref[stacks-morphisms-section-phantom]{Morphisms of Algebraic Stacks}
\item \hyperref[stacks-limits-section-phantom]{Limits of Algebraic Stacks}
\item \hyperref[stacks-cohomology-section-phantom]{Cohomology of Algebraic Stacks}
\item \hyperref[stacks-perfect-section-phantom]{Derived Categories of Stacks}
\item \hyperref[stacks-introduction-section-phantom]{Introducing Algebraic Stacks}
\item \hyperref[stacks-more-morphisms-section-phantom]{More on Morphisms of Stacks}
\item \hyperref[stacks-geometry-section-phantom]{The Geometry of Stacks}
\end{enumerate}
Topics in Moduli Theory
\begin{enumerate}
\setcounter{enumi}{102}
\item \hyperref[moduli-section-phantom]{Moduli Stacks}
\item \hyperref[moduli-curves-section-phantom]{Moduli of Curves}
\end{enumerate}
Miscellany
\begin{enumerate}
\setcounter{enumi}{104}
\item \hyperref[examples-section-phantom]{Examples}
\item \hyperref[exercises-section-phantom]{Exercises}
\item \hyperref[guide-section-phantom]{Guide to Literature}
\item \hyperref[desirables-section-phantom]{Desirables}
\item \hyperref[coding-section-phantom]{Coding Style}
\item \hyperref[obsolete-section-phantom]{Obsolete}
\item \hyperref[fdl-section-phantom]{GNU Free Documentation License}
\item \hyperref[index-section-phantom]{Auto Generated Index}
\end{enumerate}
\end{multicols}


\bibliography{my}
\bibliographystyle{amsalpha}

\end{document}
