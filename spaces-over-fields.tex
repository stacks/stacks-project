\IfFileExists{stacks-project.cls}{%
\documentclass{stacks-project}
}{%
\documentclass{amsart}
}

% The following AMS packages are automatically loaded with
% the amsart documentclass:
%\usepackage{amsmath}
%\usepackage{amssymb}
%\usepackage{amsthm}

% For dealing with references we use the comment environment
\usepackage{verbatim}
\newenvironment{reference}{\comment}{\endcomment}
%\newenvironment{reference}{}{}
\newenvironment{slogan}{\comment}{\endcomment}
\newenvironment{history}{\comment}{\endcomment}

% For commutative diagrams you can use
% \usepackage{amscd}
\usepackage[all]{xy}

% We use 2cell for 2-commutative diagrams.
\xyoption{2cell}
\UseAllTwocells

% To put source file link in headers.
% Change "template.tex" to "this_filename.tex"
% \usepackage{fancyhdr}
% \pagestyle{fancy}
% \lhead{}
% \chead{}
% \rhead{Source file: \url{template.tex}}
% \lfoot{}
% \cfoot{\thepage}
% \rfoot{}
% \renewcommand{\headrulewidth}{0pt}
% \renewcommand{\footrulewidth}{0pt}
% \renewcommand{\headheight}{12pt}

\usepackage{multicol}

% For cross-file-references
\usepackage{xr-hyper}

% Package for hypertext links:
\usepackage{hyperref}

% For any local file, say "hello.tex" you want to link to please
% use \externaldocument[hello-]{hello}
\externaldocument[introduction-]{introduction}
\externaldocument[conventions-]{conventions}
\externaldocument[sets-]{sets}
\externaldocument[categories-]{categories}
\externaldocument[topology-]{topology}
\externaldocument[sheaves-]{sheaves}
\externaldocument[sites-]{sites}
\externaldocument[stacks-]{stacks}
\externaldocument[fields-]{fields}
\externaldocument[algebra-]{algebra}
\externaldocument[brauer-]{brauer}
\externaldocument[homology-]{homology}
\externaldocument[derived-]{derived}
\externaldocument[simplicial-]{simplicial}
\externaldocument[more-algebra-]{more-algebra}
\externaldocument[smoothing-]{smoothing}
\externaldocument[modules-]{modules}
\externaldocument[sites-modules-]{sites-modules}
\externaldocument[injectives-]{injectives}
\externaldocument[cohomology-]{cohomology}
\externaldocument[sites-cohomology-]{sites-cohomology}
\externaldocument[dga-]{dga}
\externaldocument[dpa-]{dpa}
\externaldocument[hypercovering-]{hypercovering}
\externaldocument[schemes-]{schemes}
\externaldocument[constructions-]{constructions}
\externaldocument[properties-]{properties}
\externaldocument[morphisms-]{morphisms}
\externaldocument[coherent-]{coherent}
\externaldocument[divisors-]{divisors}
\externaldocument[limits-]{limits}
\externaldocument[varieties-]{varieties}
\externaldocument[topologies-]{topologies}
\externaldocument[descent-]{descent}
\externaldocument[perfect-]{perfect}
\externaldocument[more-morphisms-]{more-morphisms}
\externaldocument[flat-]{flat}
\externaldocument[groupoids-]{groupoids}
\externaldocument[more-groupoids-]{more-groupoids}
\externaldocument[etale-]{etale}
\externaldocument[chow-]{chow}
\externaldocument[intersection-]{intersection}
\externaldocument[pic-]{pic}
\externaldocument[adequate-]{adequate}
\externaldocument[dualizing-]{dualizing}
\externaldocument[duality-]{duality}
\externaldocument[discriminant-]{discriminant}
\externaldocument[local-cohomology-]{local-cohomology}
\externaldocument[curves-]{curves}
\externaldocument[resolve-]{resolve}
\externaldocument[models-]{models}
\externaldocument[pione-]{pione}
\externaldocument[etale-cohomology-]{etale-cohomology}
\externaldocument[proetale-]{proetale}
\externaldocument[crystalline-]{crystalline}
\externaldocument[spaces-]{spaces}
\externaldocument[spaces-properties-]{spaces-properties}
\externaldocument[spaces-morphisms-]{spaces-morphisms}
\externaldocument[decent-spaces-]{decent-spaces}
\externaldocument[spaces-cohomology-]{spaces-cohomology}
\externaldocument[spaces-limits-]{spaces-limits}
\externaldocument[spaces-divisors-]{spaces-divisors}
\externaldocument[spaces-over-fields-]{spaces-over-fields}
\externaldocument[spaces-topologies-]{spaces-topologies}
\externaldocument[spaces-descent-]{spaces-descent}
\externaldocument[spaces-perfect-]{spaces-perfect}
\externaldocument[spaces-more-morphisms-]{spaces-more-morphisms}
\externaldocument[spaces-flat-]{spaces-flat}
\externaldocument[spaces-groupoids-]{spaces-groupoids}
\externaldocument[spaces-more-groupoids-]{spaces-more-groupoids}
\externaldocument[bootstrap-]{bootstrap}
\externaldocument[spaces-pushouts-]{spaces-pushouts}
\externaldocument[groupoids-quotients-]{groupoids-quotients}
\externaldocument[spaces-more-cohomology-]{spaces-more-cohomology}
\externaldocument[spaces-simplicial-]{spaces-simplicial}
\externaldocument[formal-spaces-]{formal-spaces}
\externaldocument[restricted-]{restricted}
\externaldocument[spaces-resolve-]{spaces-resolve}
\externaldocument[formal-defos-]{formal-defos}
\externaldocument[defos-]{defos}
\externaldocument[cotangent-]{cotangent}
\externaldocument[examples-defos-]{examples-defos}
\externaldocument[algebraic-]{algebraic}
\externaldocument[examples-stacks-]{examples-stacks}
\externaldocument[stacks-sheaves-]{stacks-sheaves}
\externaldocument[criteria-]{criteria}
\externaldocument[artin-]{artin}
\externaldocument[quot-]{quot}
\externaldocument[stacks-properties-]{stacks-properties}
\externaldocument[stacks-morphisms-]{stacks-morphisms}
\externaldocument[stacks-limits-]{stacks-limits}
\externaldocument[stacks-cohomology-]{stacks-cohomology}
\externaldocument[stacks-perfect-]{stacks-perfect}
\externaldocument[stacks-introduction-]{stacks-introduction}
\externaldocument[stacks-more-morphisms-]{stacks-more-morphisms}
\externaldocument[stacks-geometry-]{stacks-geometry}
\externaldocument[moduli-]{moduli}
\externaldocument[moduli-curves-]{moduli-curves}
\externaldocument[examples-]{examples}
\externaldocument[exercises-]{exercises}
\externaldocument[guide-]{guide}
\externaldocument[desirables-]{desirables}
\externaldocument[coding-]{coding}
\externaldocument[obsolete-]{obsolete}
\externaldocument[fdl-]{fdl}
\externaldocument[index-]{index}

% Theorem environments.
%
\theoremstyle{plain}
\newtheorem{theorem}[subsection]{Theorem}
\newtheorem{proposition}[subsection]{Proposition}
\newtheorem{lemma}[subsection]{Lemma}

\theoremstyle{definition}
\newtheorem{definition}[subsection]{Definition}
\newtheorem{example}[subsection]{Example}
\newtheorem{exercise}[subsection]{Exercise}
\newtheorem{situation}[subsection]{Situation}

\theoremstyle{remark}
\newtheorem{remark}[subsection]{Remark}
\newtheorem{remarks}[subsection]{Remarks}

\numberwithin{equation}{subsection}

% Macros
%
\def\lim{\mathop{\rm lim}\nolimits}
\def\colim{\mathop{\rm colim}\nolimits}
\def\Spec{\mathop{\rm Spec}}
\def\Hom{\mathop{\rm Hom}\nolimits}
\def\Ext{\mathop{\rm Ext}\nolimits}
\def\SheafHom{\mathop{\mathcal{H}\!{\it om}}\nolimits}
\def\SheafExt{\mathop{\mathcal{E}\!{\it xt}}\nolimits}
\def\Sch{\textit{Sch}}
\def\Mor{\mathop{\rm Mor}\nolimits}
\def\Ob{\mathop{\rm Ob}\nolimits}
\def\Sh{\mathop{\textit{Sh}}\nolimits}
\def\NL{\mathop{N\!L}\nolimits}
\def\proetale{{pro\text{-}\acute{e}tale}}
\def\etale{{\acute{e}tale}}
\def\QCoh{\textit{QCoh}}
\def\Ker{\mathop{\rm Ker}}
\def\Im{\mathop{\rm Im}}
\def\Coker{\mathop{\rm Coker}}
\def\Coim{\mathop{\rm Coim}}

%
% Macros for moduli stacks/spaces
%
\def\QCohstack{\mathcal{QC}\!{\it oh}}
\def\Cohstack{\mathcal{C}\!{\it oh}}
\def\Spacesstack{\mathcal{S}\!{\it paces}}
\def\Quotfunctor{{\rm Quot}}
\def\Hilbfunctor{{\rm Hilb}}
\def\Curvesstack{\mathcal{C}\!{\it urves}}
\def\Polarizedstack{\mathcal{P}\!{\it olarized}}
\def\Complexesstack{\mathcal{C}\!{\it omplexes}}
% \Pic is the operator that assigns to X its picard group, usage \Pic(X)
% \Picardstack_{X/B} denotes the Picard stack of X over B
% \Picardfunctor_{X/B} denotes the Picard functor of X over B
\def\Pic{\mathop{\rm Pic}\nolimits}
\def\Picardstack{\mathcal{P}\!{\it ic}}
\def\Picardfunctor{{\rm Pic}}
\def\Deformationcategory{\mathcal{D}\!{\it ef}}


% OK, start here.
%
\begin{document}

\title{Algebraic Spaces over Fields}


\maketitle

\phantomsection
\label{section-phantom}

\tableofcontents

\section{Introduction}
\label{section-introduction}

\noindent
This chapter is the analogue of the chapter on varieties in the setting
of algebraic spaces. A reference for algebraic spaces is
\cite{Kn}.


\section{Conventions}
\label{section-conventions}

\noindent
The standing assumption is that all schemes are contained in
a big fppf site $\Sch_{fppf}$. And all rings $A$ considered
have the property that $\Spec(A)$ is (isomorphic) to an
object of this big site.

\medskip\noindent
Let $S$ be a scheme and let $X$ be an algebraic space over $S$.
In this chapter and the following we will write $X \times_S X$
for the product of $X$ with itself (in the category of algebraic
spaces over $S$), instead of $X \times X$.




\section{Geometric components}
\label{section-geometric-components}


\begin{lemma}
\label{lemma-minimal-primes-tensor-strictly-henselian}
Let $k$ be an algebraically closed field. Let $A$, $B$ be strictly
henselian local $k$-algebras with residue field equal to $k$.
Let $C$ be the strict henselization of $A \otimes_k B$ at the maximal
ideal $\mathfrak m_A \otimes_k B + A \otimes_k \mathfrak m_B$.
Then the minimal primes of $C$ correspond $1$-to-$1$ to pairs of
minimal primes of $A$ and $B$.
\end{lemma}

\begin{proof}
First note that a minimal prime $\mathfrak r$ of $C$ maps to a minimal
prime $\mathfrak p$ in $A$ and to a minimal prime $\mathfrak q$ of $B$
because the ring maps $A \to C$ and $B \to C$ are flat (by going down for
flat ring map
Algebra, Lemma \ref{algebra-lemma-flat-going-down}).
Hence it suffices to show that the strict henselization of
$(A/\mathfrak p \otimes_k B/\mathfrak q)_{
\mathfrak m_A \otimes_k B + A \otimes_k \mathfrak m_B}$
has a unique minimal prime ideal. By
Algebra, Lemma \ref{algebra-lemma-quotient-strict-henselization}
the rings $A/\mathfrak p$, $B/\mathfrak q$ are strictly henselian.
Hence we may assume that $A$ and $B$ are strictly henselian
local domains and our goal is to show that $C$ has a unique minimal prime. By
Properties of Spaces,
Lemma \ref{spaces-properties-lemma-geometrically-unibranch}.
we see that the integral closure $A'$ of $A$ in its fraction field
is a normal local domain with residue field $k$ and similarly for the
integral closure $B'$ of $B$ into its fraction field. By
Algebra, Lemma \ref{algebra-lemma-geometrically-normal-tensor-normal}
we see that $A' \otimes_k B'$ is a normal ring. Hence its localization
$$
R = (A' \otimes_k B')_{
\mathfrak m_{A'} \otimes_k B' + A' \otimes_k \mathfrak m_{B'}}
$$
is a normal local domain. Note that $A \otimes_k B \to A' \otimes_k B'$
is integral (hence gong up holds --
Algebra, Lemma \ref{algebra-lemma-integral-going-up})
and that $\mathfrak m_{A'} \otimes_k B' + A' \otimes_k \mathfrak m_{B'}$
is the unique maximal ideal of $A' \otimes_k B'$
lying over $\mathfrak m_A \otimes_k B + A \otimes_k \mathfrak m_B$.
Hence we see that
$$
R = (A' \otimes_k B')_{
\mathfrak m_A \otimes_k B + A \otimes_k \mathfrak m_B}
$$
by
Algebra, Lemma \ref{algebra-lemma-unique-prime-over-localize-below}.
It follows that
$$
(A \otimes_k B)_{
\mathfrak m_A \otimes_k B + A \otimes_k \mathfrak m_B}
\longrightarrow
R
$$
is integral. We conclude that $R$ is the integral closure of
$(A \otimes_k B)_{
\mathfrak m_A \otimes_k B + A \otimes_k \mathfrak m_B}$
in its fraction field, and by
Properties of Spaces,
Lemma \ref{spaces-properties-lemma-geometrically-unibranch}
once again we conclude that $C$ has a unique prime ideal.
\end{proof}





\section{Generically finite morphisms}
\label{section-generically-finite}

\noindent
This section discusses for morphisms of algebraic spaces
the material discussed in Morphisms, Section
\ref{morphisms-section-generically-finite}
and
Varieties, Section \ref{varieties-section-generically-finite}
for morphisms of schemes.

\begin{lemma}
\label{lemma-generically-finite}
Let $S$ be a scheme. Let $f : X \to Y$ be a morphism of
algebraic spaces over $S$. Assume that
$f$ is quasi-separated of finite type and $Y$ is decent.
Let $y \in |Y|$ be a generic point of an irreducible component of $|Y|$.
The following are equivalent:
\begin{enumerate}
\item the set $f^{-1}(\{y\})$ is finite,
\item $X \to Y$ is quasi-finite at all points of $|X|$ over $y$,
\item there exists an open subspace $Y' \subset Y$ with $y \in |Y'|$
such that $Y' \times_Y X \to Y'$ is finite.
\end{enumerate}
\end{lemma}

\begin{proof}
Since $Y$ is decent and $f$ is quasi-separated, we see that
$X$ is decent too; to see this use
Decent Spaces, Lemmas
\ref{decent-spaces-lemma-properties-trivial-implications} and
\ref{decent-spaces-lemma-property-over-property}.
Hence Decent Spaces, Lemma \ref{decent-spaces-lemma-conditions-on-fibre-and-qf}
applies and we see that (1) implies (2). On the other hand,
we see that (2) implies (1) by Morphisms of Spaces, Lemma
\ref{spaces-morphisms-lemma-quasi-finite-at-a-finite-number-of-points}.
The same lemma also shows that (3) implies (1).

\medskip\noindent
Assume the equivalent conditions of (1) and (2). Choose an affine scheme $V$
and an \'etale morphism $V \to Y$ mapping a point $v \in V$ to $y$. Then $v$
is a generic point of an irreducible component of $V$ by Decent Spaces, Lemma
\ref{decent-spaces-lemma-decent-generic-points}. Choose an affine scheme $U$
and a surjective \'etale morphism $U \to V \times_Y X$. Then $U \to V$ is of
finite type. The morphism $U \to V$ is quasi-finite at every point lying over
$v$ by (2). It follows that the fibre of $U \to V$ over $v$ is finite
(Morphisms, Lemma
\ref{morphisms-lemma-quasi-finite-at-a-finite-number-of-points}). By
Morphisms, Lemma \ref{morphisms-lemma-generically-finite}
after shrinking $V$ we may assume that $U \to V$ is finite.
Let
$$
R = U \times_{V \times_Y X} U
$$
Since $f$ is quasi-separated, we see that $V \times_Y X$ is quasi-separated
and hence $R$ is a quasi-compact scheme. Moreover the morphisms
$R \to V$ is quasi-finite as the composition of an \'etale morphism
$R \to U$ and a finite morphism $U \to V$. Hence we may apply
Morphisms, Lemma \ref{morphisms-lemma-generically-finite}
once more and after shrinking $V$ we may assume that $R \to V$ is
finite as well. This of course implies that the two projections
$R \to V$ are finite \'etale. It follows that
$V/R = V \times_Y X$ is an affine scheme, see
Groupoids, Proposition \ref{groupoids-proposition-finite-flat-equivalence}.
By Morphisms, Lemma \ref{morphisms-lemma-image-proper-is-proper}
we conclude that $V \times_Y X \to V$ is proper and by
Morphisms, Lemma \ref{morphisms-lemma-finite-proper}
we conclude that $V \times_Y X \to V$ is finite.
Finally, we let $Y' \subset Y$ be the open subspace of $Y$
corresponding to the image of $|V| \to |Y|$.
By Morphisms of Spaces, Lemma \ref{spaces-morphisms-lemma-integral-local}
we conclude that $Y' \times_Y X \to Y'$ is finite as the base
change to $V$ is finite and as $V \to Y'$ is a surjective \'etale
morphism.
\end{proof}

\begin{lemma}
\label{lemma-generically-finite-reprise}
Let $S$ be a scheme. Let $f : X \to Y$ be a morphism of
algebraic spaces over $S$. Assume that $f$ is quasi-separated
and locally of finite type and $Y$ is quasi-separated.
Let $y \in |Y|$ be a generic point of an irreducible component of $|Y|$.
The following are equivalent:
\begin{enumerate}
\item the set $f^{-1}(\{y\})$ is finite,
\item there exist open subspaces $X' \subset X$ and $Y' \subset Y$
with $f(X') \subset Y'$, $y \in |Y'|$, and $f^{-1}(\{y\}) \subset |X'|$
such that $f|_{X'} : X' \to Y'$ is finite.
\end{enumerate}
\end{lemma}

\begin{proof}
This is just an application of Lemma \ref{lemma-generically-finite}.
we may first replace $Y$ by a quasi-compact open subspace containing $y$.
If (1) holds, then we can find a quasi-compact open subspace $X' \subset X$
containing $f^{-1}(\{y\})$. Since $Y$ is quasi-separated, the morphism
$f|_{X'} : X' \to Y$ is quasi-compact and quasi-separated
(Morphisms of Spaces, Lemma
\ref{spaces-morphisms-lemma-quasi-compact-quasi-separated-permanence}).
Applying Lemma \ref{lemma-generically-finite}
to $f|_{X'} : X' \to Y$ we see that (2) holds.
We omit the proof that (2) implies (1).
\end{proof}

\begin{lemma}
\label{lemma-quasi-finite-in-codim-1}
Let $S$ be a scheme. Let $f : X \to Y$ be a morphism of algebraic spaces
over $S$. Assume $f$ is locally of finite type, $Y$ is locally Noetherian,
and $X$ is a decent algebraic space. Let $y \in Y$ be a point such that
the dimension of the local ring at $y$ is $\leq 1$.
Assume in addition one of the following conditions is satisfied
\begin{enumerate}
\item for every generic point $x$ of an irreducible component
of $|X|$ the transcendence degree of $x/f(x)$ is $0$,
\item for every generic point $x$ of an irreducible component
of $|X|$ such that $f(x) \leadsto y$ the transcendence degree
of $x/f(x)$ is $0$,
\item $f$ is quasi-finite at every generic point of $|X|$,
\item $f$ is quasi-finite at a dense set of points of $|X|$,
\item add more here.
\end{enumerate}
Then $f$ is quasi-finite at every point of $X$ lying over $y$.
\end{lemma}

\begin{proof}
Observe that $X$ is locally Noetherian
(Morphisms of Spaces, Lemma
\ref{spaces-morphisms-lemma-locally-finite-type-locally-noetherian})
and hence $|X|$ is locally Noetherian
(Properties of Spaces, Lemma
\ref{spaces-properties-lemma-Noetherian-topology}).
Since $X$ is decent $|X|$ is also a sober topological space
(Decent Spaces, Proposition \ref{decent-spaces-proposition-reasonable-sober}).
The set of points at which morphism is quasi-finite is open
(Morphisms of Spaces, Lemma
\ref{spaces-morphisms-lemma-base-change-quasi-finite-locus}).
A dense open of a sober locally Noetherian topological space
contains all generic point of irreducible components, hence (4) implies (3).
Condition (3) implies condition (1) for example by
Morphisms of Spaces, Lemma
\ref{spaces-morphisms-lemma-compare-tr-deg}
applied to $X \to Y \to Y$.
Condition (1) implies condition (2).
Thus it suffices to prove the lemma in case (2) holds.

\medskip\noindent
We want to reduce the proof to the case of schemes. To do this we
choose a commutative diagram
$$
\xymatrix{
U \ar[r] \ar[d]_g & X \ar[d]^f \\
V \ar[r] & Y
}
$$
where $U$, $V$ are schemes and where the horizontal arrows are \'etale.
Say $v \in V$ maps to $y$. Let $u \in U$ be a generic point of an
irreducible component of $U$. Then $\dim(\mathcal{O}_{U, u}) = 0$
which implies that $x = f(u)$ is a generic point of an irreducible
component of $|X|$ by Decent Spaces, Lemma
\ref{decent-spaces-lemma-decent-generic-points}.
Moreover, if $g(u) \leadsto v$, then of course $f(x) \leadsto y$.
Thus we see that $\kappa(u)/\kappa(g(u))$ is a field extension
of transcendence degree $0$. In other words, assumption (2)
of Varieties, Lemma \ref{varieties-lemma-quasi-finite-in-codim-1}
is satisfied for $g : U \to V$ and $v \in V$.
We conclude that $g$ is quasi-finite at all points of $U$ lying
over $v$ as desired.
\end{proof}

\begin{lemma}
\label{lemma-finite-in-codim-1}
Let $S$ be a scheme. Let $f : X \to Y$ be a morphism of algebraic spaces
over $S$. Assume $f$ is proper, $Y$ is locally Noetherian, and $X$ is a
decent algebraic space. Let $y \in Y$ be a point such that the dimension
of the local ring at $y$ is $\leq 1$. Assume in addition one of the
following conditions is satisfied
\begin{enumerate}
\item for every generic point $x$ of an irreducible component
of $|X|$ the transcendence degree of $x/f(x)$ is $0$,
\item for every generic point $x$ of an irreducible component
of $|X|$ such that $f(x) \leadsto y$ the transcendence degree
of $x/f(x)$ is $0$,
\item $f$ is quasi-finite at every generic point of $|X|$,
\item $f$ is quasi-finite at a dense set of points of $|X|$,
\item add more here.
\end{enumerate}
Then there exists an open subspace $Y' \subset Y$ containing $y$ such that
$Y' \times_Y X \to Y'$ is finite.
\end{lemma}

\begin{proof}
By Lemma \ref{lemma-quasi-finite-in-codim-1} the morphism $f$ is
quasi-finite at every point lying over $y$. Let $\Spec(k) \to Y$
be any morphism from the spectrum of a field
in the equivalence class of $y$. Then $|X_k|$ is a
discrete space (Decent Spaces, Lemma
\ref{decent-spaces-lemma-conditions-on-fibre-and-qf}).
Since $X_k$ is quasi-compact as $f$ is proper we conclude
that $|X_k|$ is finite.
Thus we can apply Cohomology of Spaces, Lemma
\ref{spaces-cohomology-lemma-proper-finite-fibre-finite-in-neighbourhood}
to conclude.
\end{proof}




\section{Integral algebraic spaces}
\label{section-integral-spaces}

\noindent
We have not yet defined the notion of an integral algebraic space. The
problem is that begin integral is not an \'etale local property of schemes.
We could use the property, that $X$ is reduced and $|X|$ is irreducible,
given in Properties, Lemma \ref{properties-lemma-characterize-integral}
to define integral algebraic spaces. In this case the algebraic
space described in Spaces, Example \ref{spaces-example-infinite-product}
would be integral which does not seem right.
To avoid this type of patholopgy we will in addition assume that $X$ is a
decent algebraic space, although perhaps a weaker alternative exists.

\begin{definition}
\label{definition-integral-algebraic-space}
Let $S$ be a scheme. We say an algebraic space $X$ over $S$ is
{\it integral} if it is reduced, decent, and $|X|$ is irreducible.
\end{definition}

\noindent
In this case the irreducible topological space $|X|$ is sober
(Decent Spaces, Proposition \ref{decent-spaces-proposition-reasonable-sober}).
Hence it has a unique generic point $x$.
Then $x$ is contained in the schematic locus of $X$
(Decent Spaces, Theorem \ref{decent-spaces-theorem-decent-open-dense-scheme})
and we can look at its residue field as a substitute for
the function field of $X$ (not yet defined; insert future reference here).

\medskip\noindent
The following lemma characterizes dominant morphisms of finite degree
between integral algebraic spaces.

\begin{lemma}
\label{lemma-finite-degree}
Let $S$ be a scheme. Let $X$, $Y$ be integral algebraic spaces over $S$
Let $x \in |X|$ and $y \in |Y|$ be the generic points. Let $f : X \to Y$
be locally of finite type. Assume $f$ is dominant
(Morphisms of Spaces, Definition \ref{spaces-morphisms-definition-dominant}).
The following are equivalent:
\begin{enumerate}
\item the transcendence degree of $x/y$ is $0$,
\item the extension $\kappa(x) \supset \kappa(y)$ (see proof) is finite,
\item there exist nonempty affine opens $U \subset X$ and $V \subset Y$
such that $f(U) \subset V$ and $f|_U : U \to V$ is finite,
\item $f$ is quasi-finite at $x$, and
\item $x$ is the only point of $|X|$ mapping to $y$.
\end{enumerate}
If $f$ is separated, or if $f$ is quasi-compact, then these are
also equivalent to
\begin{enumerate}
\item[(6)] there exists a nonempty affine open $V \subset Y$ such
that $f^{-1}(V) \to V$ is finite.
\end{enumerate}
\end{lemma}

\begin{proof}
By elementary topology, we see that $f(x) = y$ as $f$ is dominant.
Let $Y' \subset Y$ be the schematic locus of $Y$ and let
$X' \subset f^{-1}(Y')$ be the schematic locus of $X$.
By the discussion above, using
Decent Spaces, Proposition \ref{decent-spaces-proposition-reasonable-sober}
and
Decent Spaces, Theorem \ref{decent-spaces-theorem-decent-open-dense-scheme},
we see that $x \in |X'|$ and $y \in |Y|$.
Then $f|_{X'} : X' \to Y'$ is a morphism of integral schemes
which is locally of finite type. Thus we see that (1), (2), (3)
are equivalent by Morphisms, Lemma \ref{morphisms-lemma-finite-degree}.

\medskip\noindent
Condition (4) implies condition (1) by
Morphisms of Spaces, Lemma \ref{spaces-morphisms-lemma-compare-tr-deg}
applied that $X \to Y \to Y$.
On the other hand, condition (3) implies condition (4) as
a finite morphism is quasi-finite and as $x \in U$ because $x$
is the generic point. Thus (1) -- (4) are equivalent.

\medskip\noindent
Assume the equivalent conditions (1) -- (4). Suppose that
$x' \mapsto y$. Then $x \leadsto x'$ is a specialization in the
fibre of $|X| \to |Y|$ over $y$. If $x' \not = x$, then $f$ is not
quasi-finite at $x$ by Decent Spaces, Lemma
\ref{decent-spaces-lemma-conditions-on-point-in-fibre-and-qf}.
Hence $x = x'$ and (5) holds. Conversely, if (5) holds, then
(5) holds for the morphism of schemes $X' \to Y'$ (see above)
and we can use
Morphisms, Lemma \ref{morphisms-lemma-finite-degree}
to see that (1) holds.

\medskip\noindent
Observe that (6) implies the equivalent conditions (1) -- (5)
without any further assumptions on $f$. Assume (1) -- (5) hold.
To prove (6) we may shrink $Y$ and assume that $Y$ is an affine scheme
and that there exists an affine open $U \subset X$ such that $U \to Y$
is finite.

\medskip\noindent
Assume $f$ is quasi-compact. Then $Z = X \setminus U$ is a quasi-compact
closed subspace of $X$ such that $y \not \in f(Z)$. Then there exists
an open neighbourhood of $y$ which is disjoint from $f(Z)$
(details omitted; hint: use
a variant of Morphisms, Lemma \ref{morphisms-lemma-quasi-compact-dominant}).
After shrinking $Y$ we obtain $X = U$.

\medskip\noindent
Assume $f$ separated. Then $U \to X$ has closed image by
Morphisms of Spaces, Lemma
\ref{spaces-morphisms-lemma-universally-closed-permanence}.
Since $|X|$ is irreducible we get $U = X$.
\end{proof}

\begin{definition}
\label{definition-degree}
Let $S$ be a scheme.
Let $X$ and $Y$ be integral algebraic spaces over $S$.
Let $f : X \to Y$ be locally of finite type and dominant.
Assume any of the equivalent conditions (1) -- (5) of
Lemma \ref{lemma-finite-degree}. Let $x \in |X|$ and $y \in |Y|$
be the generic points. Then the positive integer
$$
\text{deg}(X/Y) = [\kappa(x) : \kappa(y)]
$$
is called the {\it degree of $X$ over $Y$}.
\end{definition}






\section{Modifications and alterations}
\label{section-modifications-alterations}

\noindent
Using our notion of an integral algebraic space we can define a modification
as follows.

\begin{definition}
\label{definition-modification}
Let $S$ be a scheme. Let $X$ be an integral algebraic space. A
{\it modification of $X$} is a birational proper morphism
$f : X' \to X$ of algebraic spaces over $S$ with $X'$ integral.
\end{definition}

\noindent
For birational morphisms of algebraic spaces, see
Decent Spaces, Definition \ref{decent-spaces-definition-birational}.

\begin{lemma}
\label{lemma-modification-iso-over-open}
Let $f : X' \to X$ be a modification as in
Definition \ref{definition-modification}.
There exists a nonempty open $U \subset X$ such that $f^{-1}(U) \to U$
is an isomorphism.
\end{lemma}

\begin{proof}
By
Lemma \ref{lemma-finite-degree} there exists a nonempty $U \subset X$ such
that $f^{-1}(U) \to U$ is finite. By generic flatness
(Morphisms of Spaces, Proposition
\ref{spaces-morphisms-proposition-generic-flatness-reduced})
we may assume $f^{-1}(U) \to U$ is flat and of finite presentation.
So $f^{-1}(U) \to U$ is finite locally free
(Morphisms of Spaces, Lemma \ref{spaces-morphisms-lemma-finite-flat}).
Since $f$ is birational, the degree of $X'$ over $X$ is $1$.
Hence $f^{-1}(U) \to U$ is finite locally free of degree $1$,
in other words it is an isomorphism.
\end{proof}

\begin{definition}
\label{definition-alteration}
Let $S$ be a scheme. Let $X$ be an integral algebraic space over $S$.
An {\it alteration of $X$} is a proper dominant morphism $f : Y \to X$
of algebraic spaces over $S$ with $Y$ integral such that $f^{-1}(U) \to U$
is finite for some nonempty open $U \subset X$.
\end{definition}

\noindent
If $f : Y \to X$ is a dominant and proper morphism between integral
algebraic spaces, then it is an alteration as soon as the induced
extension of residue fields in generic points is finite. Here is the
precise statement.

\begin{lemma}
\label{lemma-alteration-generically-finite}
Let $S$ be a scheme. Let $f : X \to Y$ be a proper dominant morphism of
integral algebraic spaces over $S$. Then $f$ is an alteration
if and only if any of the equivalent conditions (1) -- (6) of
Lemma \ref{lemma-finite-degree} hold.
\end{lemma}

\begin{proof}
Immediate consequence of the lemma referenced in the statement.
\end{proof}





\section{Schematic locus}
\label{section-schematic}

\noindent
We have already proven a number of results on the schematic locus
of an algebraic space in Properties of Spaces, Sections
\ref{spaces-properties-section-schematic} and
\ref{spaces-properties-section-getting-a-scheme}
and
Decent Spaces, Section \ref{decent-spaces-section-schematic}.

\begin{lemma}
\label{lemma-locally-finite-type-dim-zero}
Let $X$ be an algebraic space over some base scheme $S$.
In each of the following cases $X$ is a scheme:
\begin{enumerate}
\item $X$ is quasi-compact and quasi-separated and $\dim(X) = 0$,
\item $X$ is locally of finite type over a field $k$ and $\dim(X) = 0$,
\item $X$ is Noetherian and $\dim(X) = 0$, and
\item add more here.
\end{enumerate}
\end{lemma}

\begin{proof}
Cases (2) and (3) follow immediately from case (1) but we will give a separate
proofs of (2) and (3) as these proofs use significantly less theory.

\medskip\noindent
Proof of (3). Let $U$ be an affine scheme and let $U \to X$ be an
\'etale morphism. Set $R = U \times_X U$. The two projection
morphisms $s, t : R \to U$ are \'etale morphisms of schemes. By
Properties of Spaces, Definition \ref{spaces-properties-definition-dimension}
we see that $\dim(U) = 0$ and $\dim(R) = 0$.
Since $R$ is a locally Noetherian scheme of dimension $0$,
we see that $R$ is a disjoint union of spectra of
Artinian local rings
(Properties, Lemma \ref{properties-lemma-locally-Noetherian-dimension-0}).
Since we assumed that $X$ is Noetherian (so quasi-separated) we
conclude that $R$ is quasi-compact. Hence $R$ is an affine scheme
(use Schemes, Lemma \ref{schemes-lemma-disjoint-union-affines}).
The \'etale morphisms $s, t : R \to U$ induce finite residue field
extensions. Hence $s$ and $t$ are finite by
Algebra, Lemma
\ref{algebra-lemma-essentially-of-finite-type-into-artinian-local}
(small detail omitted). 
Thus
Groupoids, Proposition \ref{groupoids-proposition-finite-flat-equivalence}
shows that $X = U/R$ is an affine scheme.

\medskip\noindent
Proof of (2) -- almost identical to the proof of (4).
Let $U$ be an affine scheme and let $U \to X$ be an \'etale morphism.
Set $R = U \times_X U$. The two projection morphisms
$s, t : R \to U$ are \'etale morphisms of schemes. By
Properties of Spaces, Definition \ref{spaces-properties-definition-dimension}
we see that $\dim(U) = 0$ and similarly $\dim(R) = 0$.
On the other hand, the morphism $U \to \Spec(k)$ is locally of finite
type as the composition of the \'etale morphism $U \to X$ and
$X \to \Spec(k)$, see
Morphisms of Spaces,
Lemmas \ref{spaces-morphisms-lemma-composition-finite-type} and
\ref{spaces-morphisms-lemma-etale-locally-finite-type}.
Similarly, $R \to \Spec(k)$ is locally of finite type.
Hence by
Varieties, Lemma \ref{varieties-lemma-algebraic-scheme-dim-0}
we see that $U$ and $R$ are disjoint unions of spectra of
local Artinian $k$-algebras finite over $k$. The same thing
is therefore true of $U \times_{\Spec(k)} U$. As
$$
R = U \times_X U \longrightarrow U \times_{\Spec(k)} U
$$
is a monomorphism, we see that $R$ is a finite(!) union of spectra of
finite $k$-algebras. It follows that $R$ is affine, see
Schemes, Lemma \ref{schemes-lemma-disjoint-union-affines}.
Applying
Varieties, Lemma \ref{varieties-lemma-algebraic-scheme-dim-0}
once more we see that $R$ is finite over $k$. Hence $s, t$
are finite, see
Morphisms, Lemma \ref{morphisms-lemma-finite-permanence}.
Thus
Groupoids, Proposition \ref{groupoids-proposition-finite-flat-equivalence}
shows that the open subspace $U/R$ of $X$ is an affine scheme. Since the
schematic locus of $X$ is an open subspace (see
Properties of Spaces, Lemma \ref{spaces-properties-lemma-subscheme}),
and since $U \to X$ was an arbitrary \'etale morphism from an affine scheme
we conclude that $X$ is a scheme.

\medskip\noindent
Proof of (1). By Cohomology of Spaces, Lemma
\ref{spaces-cohomology-lemma-vanishing-above-dimension}
we have vanishing of higher cohomology groups for all
quasi-coherent sheaves $\mathcal{F}$ on $X$. Hence $X$
is affine (in particular a scheme) by
Cohomology of Spaces, Proposition
\ref{spaces-cohomology-proposition-vanishing-affine}.
\end{proof}

\noindent
Please compare the following lemma to
Decent Spaces, Lemma \ref{decent-spaces-lemma-conditions-on-space-over-field}.

\begin{lemma}
\label{lemma-locally-quasi-finite-over-field}
Let $k$ be a field. Let $X$ be an algebraic space over $k$.
The following are equivalent
\begin{enumerate}
\item $X$ is locally quasi-finite over $k$,
\item $X$ is locally of finite type over $k$ and has dimension $0$,
\item $X$ is a scheme and is locally quasi-finite over $k$,
\item $X$ is a scheme and is locally of finite type over $k$ and has
dimension $0$, and
\item $X$ is a disjoint union of spectra of Artinian local $k$-algebras
$A$ over $k$ with $\dim_k(A) < \infty$.
\end{enumerate}
\end{lemma}

\begin{proof}
Because we are over a field relative dimension of $X/k$ is the same as
the dimension of $X$. Hence by
Morphisms of Spaces,
Lemma \ref{spaces-morphisms-lemma-locally-quasi-finite-rel-dimension-0}
we see that (1) and (2) are equivalent. Hence it follows from
Lemma \ref{lemma-locally-finite-type-dim-zero}
(and trivial implications) that (1) -- (4) are equivalent.
Finally,
Varieties, Lemma \ref{varieties-lemma-algebraic-scheme-dim-0}
shows that (1) -- (4) are equivalent with (5).
\end{proof}

\begin{lemma}
\label{lemma-mono-towards-locally-quasi-finite-over-field}
Let $k$ be a field. Let $f : X \to Y$ be a monomorphism of algebraic spaces
over $k$. If $Y$ is locally quasi-finite over $k$ so is $X$.
\end{lemma}

\begin{proof}
Assume $Y$ is locally quasi-finite over $k$. By
Lemma \ref{lemma-locally-quasi-finite-over-field}
we see that $Y = \coprod \Spec(A_i)$ where each $A_i$ is an
Artinian local ring finite over $k$. By
Decent Spaces, Lemma
\ref{decent-spaces-lemma-monomorphism-toward-disjoint-union-dim-0-rings}
we see that $X$ is a scheme. Consider $X_i = f^{-1}(\Spec(A_i))$.
Then $X_i$ has either one or zero points. If $X_i$ has zero points there
is nothing to prove. If $X_i$ has one point, then
$X_i = \Spec(B_i)$ with $B_i$ a zero dimensional local ring
and $A_i \to B_i$ is an epimorphism of rings. In particular
$A_i/\mathfrak m_{A_i} = B_i/\mathfrak m_{A_i}B_i$ and we see that
$A_i \to B_i$ is surjective by Nakayama's lemma,
Algebra, Lemma \ref{algebra-lemma-NAK}
(because $\mathfrak m_{A_i}$ is a nilpotent ideal!).
Thus $B_i$ is a finite local $k$-algebra, and we conclude by
Lemma \ref{lemma-locally-quasi-finite-over-field}
that $X \to \Spec(k)$ is locally quasi-finite.
\end{proof}

\noindent
The following lemma tells us that a quasi-separated algebraic space
is a scheme away from codimension $1$.

\begin{lemma}
\label{lemma-generic-point-in-schematic-locus}
Let $S$ be a scheme. Let $X$ be an algebraic space over $S$.
Let $x \in |X|$. If $X$ is quasi-separated and $x$ is a generic
point of an irreducible component of $|X|$, then there exists
an open subspace of $X$ containing $x$ which is a scheme.
\end{lemma}

\begin{proof}
We can replace $X$ by an quasi-compact neighbourhood of $x$, hence
we may assume $X$ is quasi-compact and quasi-separated.
Choose a stratification
$$
\emptyset = U_{n + 1} \subset
U_n \subset U_{n - 1} \subset \ldots \subset U_1 = X
$$
and $f_p : V_p \to U_p$ and $T_p = U_p \setminus U_{p + 1}$
as in Decent Spaces, Lemma
\ref{decent-spaces-lemma-filter-quasi-compact-quasi-separated}.
Then $x \in T_p$ for a unique $p$. Let $v \in f_p^{-1}(T_p)$
be the corresponding point. Note that $v$ is a generic point of
an irreducible component of $V_p$ by
Decent Spaces, Lemma \ref{decent-spaces-lemma-decent-generic-points}.
Since $U_{p + 1}$ is quasi-compact and $f_p : V_p \to U_p$ is a
quasi-compact morphism (Morphisms of Spaces, Lemma
\ref{spaces-morphisms-lemma-quasi-compact-quasi-separated-permanence}),
we see that $f_p^{-1}(T_p) = V_p \setminus f_p^{-1}(U_{p + 1})$
is a constructible closed subset of $V_p$. Hence an open neighbourhood
$W$ of $v \in V_p$ is contained in $f_p^{-1}(T_p)$, see
Properties, Lemma \ref{properties-lemma-generic-point-in-constructible}.
Then $f_p(W) \subset X$ is an open neighbourhood of $x$
and $f_p|_W : W \to f_p(W)$ is an \'etale morphism which induces
an isomorphism on the reductions (by our choice of the stratification).
It follows that $W \to f_p(W)$ is an isomorphism (Morphisms of Spaces, Lemma
\ref{spaces-morphisms-lemma-etale-universally-injective-open}).
This concludes the proof.
\end{proof}

\noindent
The following lemma says that a separated locally Noetherian algebraic
space is a scheme in codimension $1$, i.e., away from codimension $2$.

\begin{lemma}
\label{lemma-codim-1-point-in-schematic-locus}
Let $S$ be a scheme. Let $X$ be an algebraic space over $S$.
Let $x \in |X|$. If $X$ is separated, locally Noetherian, and
the dimension of the local ring of $X$ at $x$ is $\leq 1$
(Properties of Spaces, Definition
\ref{spaces-properties-definition-dimension-local-ring}),
then there exists an open subspace of $X$ containing $x$ which is a scheme.
\end{lemma}

\begin{proof}
(Please see the remark below for a different approach avoiding the material on
finite groupoids.) We can replace $X$ by an quasi-compact neighbourhood of
$x$, hence we may assume $X$ is quasi-compact, separated, and Noetherian.
There exists a scheme $U$ and a finite surjective morphism $U \to X$,
see Limits of Spaces, Proposition
\ref{spaces-limits-proposition-there-is-a-scheme-finite-over}.
Let $R = U \times_X U$. Then $j : R \to U \times_S U$ is an equivalence
relation and we obtain a groupoid scheme $(U, R, s, t, c)$ over $S$
with $s, t$ finite and $U$ Noetherian and separated.
Let $\{u_1, \ldots, u_n\} \subset U$ be the set of points mapping to $x$. 
Then $\dim(\mathcal{O}_{U, u_i}) \leq 1$ by
Decent Spaces, Lemma
\ref{decent-spaces-lemma-dimension-local-ring-quasi-finite}.

\medskip\noindent
By More on Groupoids, Lemma
\ref{more-groupoids-lemma-find-affine-codimension-1}
there exists an $R$-invariant affine open $W \subset U$ containing
the orbit $\{u_1, \ldots, u_n\}$. Since $U \to X$ is finite surjective
the continuous map $|U| \to |X|$ is closed surjective, hence
submersive by Topology, Lemma
\ref{topology-lemma-closed-morphism-quotient-topology}.
Thus $f(W)$ is open and there is an open subspace $X' \subset X$
with $f : W \to X'$ a surjective finite morphism.
Then $X'$ is an affine scheme by
Cohomology of Spaces, Lemma
\ref{spaces-cohomology-lemma-image-affine-finite-morphism-affine-Noetherian}
and the proof is finished.
\end{proof}

\begin{remark}
\label{remark-alternate-proof-scheme-codim-1}
Here is a sketch of a proof of
Lemma \ref{lemma-codim-1-point-in-schematic-locus}
which avoids using
More on Groupoids, Lemma
\ref{more-groupoids-lemma-find-affine-codimension-1}.

\medskip\noindent
Step 1. We may assume $X$ is a reduced Noetherian separated algebraic space
(for example by Cohomology of Spaces, Lemma
\ref{spaces-cohomology-lemma-image-affine-finite-morphism-affine-Noetherian}
or by
Limits of Spaces, Lemma \ref{spaces-limits-lemma-reduction-scheme})
and we may choose a finite surjective morphism
$Y \to X$ where $Y$ is a Noetherian scheme (by
Limits of Spaces, Proposition
\ref{spaces-limits-proposition-there-is-a-scheme-finite-over}).

\medskip\noindent
Step 2. After replacing $X$ by an open neighbourhood of $x$, there
exists a birational finite morphism $X' \to X$ and a closed subscheme
$Y' \subset X' \times_X Y$ such that $Y' \to X'$ is surjective
finite locally free. Namely, because $X$ is reduced there is a dense
open subspace $U \subset X$ over which $Y$ is flat (Morphisms of Spaces,
Proposition \ref{spaces-morphisms-proposition-generic-flatness-reduced}).
Then we can choose a $U$-admissible blow up $b : \tilde X \to X$ such
that the strict transform $\tilde Y$ of $Y$ is flat over $\tilde X$, see
More on Morphisms of Spaces, Lemma
\ref{spaces-more-morphisms-lemma-flat-after-blowing-up}.
(An alternative is to use Hilbert schemes if one wants to avoid using
the result on blow ups).
Then we let $X' \subset \tilde X$ be the scheme theoretic
closure of $b^{-1}(U)$ and $Y' = X' \times_{\tilde X} \tilde Y$.
Since $x$ is a codimension $1$ point, we see that $X' \to X$ is finite over a
neighbourhood of $x$ (Lemma \ref{lemma-finite-in-codim-1}).

\medskip\noindent
Step 3. After shrinking $X$ to a smaller neighbourhood of $x$ we get that
$X'$ is a scheme. This holds because $Y'$ is a scheme and $Y' \to X'$
being finite locally free and because every finite set of codimension $1$
points of $Y'$ is contained in an affine open. Use
Properties of Spaces, Proposition
\ref{spaces-properties-proposition-finite-flat-equivalence-global}
and
Varieties, Proposition
\ref{varieties-proposition-finite-set-of-points-of-codim-1-in-affine}.

\medskip\noindent
Step 4. There exists an affine open $W' \subset X'$ containing all points
lying over $x$ which is the inverse image of an open subspace of $X$.
To prove this let $Z \subset X$ be the closure of the set of points
where $X' \to X$ is not an isomorphism. We may assume $x \in Z$ otherwise
we are already done. Then $x$ is a generic point of an irreducible
component of $Z$ and after shrinking $X$ we may assume $Z$ is an affine scheme
(Lemma \ref{lemma-generic-point-in-schematic-locus}).
Then the inverse image $Z' \subset X'$ is an affine scheme as well.
Say $x_1, \ldots, x_n \in Z'$ are the points mapping to $x$.
Then we can find an affine open $W'$ in $X'$ whose intersection with
$Z'$ is the inverse image of a principal open of $Z$ containing $x$.
Namely, we first pick an affine open $W' \subset X'$ containing
$x_1, \ldots, x_n$ using Varieties, Proposition
\ref{varieties-proposition-finite-set-of-points-of-codim-1-in-affine}.
Then we pick a principal open $D(f) \subset Z$ containing $x$
whose inverse image $D(f|_{Z'})$ is contained in $W' \cap Z'$.
Then we pick $f' \in \Gamma(W', \mathcal{O}_{W'})$ restricting
to $f|_{Z'}$ and we replace $W'$ by $D(f') \subset W'$.
Since $X' \to X$ is an isomorphism away from $Z' \to Z$ the choice
of $W'$ guarantees that the image $W \subset X$ of $W'$ is open
with inverse image $W'$ in $X'$.

\medskip\noindent
Step 5. Then $W' \to W$ is a finite surjective morphism and $W$ is a scheme by
Cohomology of Spaces, Lemma
\ref{spaces-cohomology-lemma-image-affine-finite-morphism-affine-Noetherian}
and the proof is complete.
\end{remark}








\section{Geometrically connected algebraic spaces}
\label{section-geometrically-connected}

\noindent
If $X$ is a connected algebraic space over a field, then it can happen that
$X$ becomes disconnected after extending the ground field. This does not
happen for geometrically connected schemes.

\begin{definition}
\label{definition-geometrically-connected}
Let $X$ be an algebraic space over the field $k$. We say $X$ is
{\it geometrically connected} over $k$ if the base change $X_{k'}$
is connected for every field extension $k'$ of $k$.
\end{definition}

\noindent
By convention a connected topological space is nonempty; hence a fortiori
geometrically connected algebraic spaces are nonempty.

\begin{lemma}
\label{lemma-geometrically-connected-check-after-extension}
Let $X$ be an algebraic space over the field $k$.
Let $k \subset k'$ be a field extension.
Then $X$ is geometrically connected over $k$ if and only if
$X_{k'}$ is geometrically connected over $k'$.
\end{lemma}

\begin{proof}
If $X$ is geometrically connected over $k$, then it is clear that
$X_{k'}$ is geometrically connected over $k'$. For the converse, note
that for any field extension $k \subset k''$ there exists a common
field extension $k' \subset k'''$ and $k'' \subset k'''$. As the
morphism $X_{k'''} \to X_{k''}$ is surjective (as a base change of
a surjective morphism between spectra of fields) we see that the
connectedness of $X_{k'''}$ implies the connectedness of $X_{k''}$.
Thus if $X_{k'}$ is geometrically connected over $k'$ then
$X$ is geometrically connected over $k$.
\end{proof}

\begin{lemma}
\label{lemma-bijection-connected-components}
Let $k$ be a field. Let $X$, $Y$ be algebraic spaces over $k$.
Assume $X$ is geometrically connected over $k$.
Then the projection morphism
$$
p : X \times_k Y \longrightarrow Y
$$
induces a bijection between connected components.
\end{lemma}

\begin{proof}
Let $y \in |Y|$ be represented by a morphism $\Spec(K) \to Y$ be a morphism
where $K$ is a field. The fibre of $|X \times_k Y| \to |Y|$ over $y$
is the image of $|Y_K| \to |X \times_k Y|$ by
Properties of Spaces, Lemma \ref{spaces-properties-lemma-points-cartesian}.
Thus these fibres are connected by our assumption that $Y$ is
geometrically connected. By
Morphisms of Spaces, Lemma
\ref{spaces-morphisms-lemma-space-over-field-universally-open}
the map $|p|$ is open.
Thus we may apply Topology,
Lemma \ref{topology-lemma-connected-fibres-connected-components}
to conclude.
\end{proof}

\begin{lemma}
\label{lemma-separably-closed-field-connected-components}
Let $k \subset k'$ be an extension of fields. Let $X$ be an algebraic space
over $k$. Assume $k$ separably algebraically closed. Then the morphism
$X_{k'} \to X$ induces a bijection of connected components. In particular,
$X$ is geometrically connected over $k$ if and only if $X$ is connected.
\end{lemma}

\begin{proof}
Since $k$ is separably algebraically closed we see that
$k'$ is geometrically connected over $k$, see
Algebra,
Lemma \ref{algebra-lemma-separably-closed-connected-implies-geometric}.
Hence $Z = \Spec(k')$ is geometrically connected over $k$ by
Varieties, Lemma \ref{varieties-lemma-affine-geometrically-connected}.
Since $X_{k'} = Z \times_k X$ the result is a special case of
Lemma \ref{lemma-bijection-connected-components}.
\end{proof}

\begin{lemma}
\label{lemma-characterize-geometrically-connected}
Let $k$ be a field. Let $X$ be an algebraic space over $k$.
Let $\overline{k}$ be a separable algebraic closure of $k$.
Then $X$ is geometrically connected if and only if the base change
$X_{\overline{k}}$ is connected.
\end{lemma}

\begin{proof}
Assume $X_{\overline{k}}$ is connected. Let $k \subset k'$ be a field
extension. There exists a field extension $\overline{k} \subset \overline{k}'$
such that $k'$ embeds into $\overline{k}'$ as an extension of $k$.
By Lemma \ref{lemma-separably-closed-field-connected-components}
we see that $X_{\overline{k}'}$ is connected.
Since $X_{\overline{k}'} \to X_{k'}$ is surjective we conclude
that $X_{k'}$ is connected as desired.
\end{proof}

\noindent
Let $k$ be a field. Let $k \subset \overline{k}$ be a (possibly infinite)
Galois extension. For example $\overline{k}$ could be the
separable algebraic closure of $k$.
For any $\sigma \in \text{Gal}(\overline{k}/k)$ we get a corresponding
automorphism
$
\Spec(\sigma) :
\Spec(\overline{k})
\longrightarrow
\Spec(\overline{k})
$.
Note that
$\Spec(\sigma) \circ \Spec(\tau) = \Spec(\tau \circ \sigma)$.
Hence we get an action
$$
\text{Gal}(\overline{k}/k)^{opp} \times \Spec(\overline{k})
\longrightarrow
\Spec(\overline{k})
$$
of the opposite group on the scheme $\Spec(\overline{k})$.
Let $X$ be an algebraic space over $k$. Since
$X_{\overline{k}} =
\Spec(\overline{k}) \times_{\Spec(k)} X$
by definition we see that the action above induces a canonical action
\begin{equation}
\label{equation-galois-action-base-change-kbar}
\text{Gal}(\overline{k}/k)^{opp} \times X_{\overline{k}}
\longrightarrow
X_{\overline{k}}.
\end{equation}

\begin{lemma}
\label{lemma-Galois-action-quasi-compact-open}
Let $k$ be a field. Let $X$ be an algebraic space over $k$.
Let $\overline{k}$ be a (possibly infinite) Galois extension of $k$.
Let $V \subset X_{\overline{k}}$ be a quasi-compact open.
Then
\begin{enumerate}
\item there exists a finite subextension $k \subset k' \subset \overline{k}$
and a quasi-compact open $V' \subset X_{k'}$ such that
$V = (V')_{\overline{k}}$,
\item there exists an open subgroup $H \subset \text{Gal}(\overline{k}/k)$
such that $\sigma(V) = V$ for all $\sigma \in H$.
\end{enumerate}
\end{lemma}

\begin{proof}
Choose a scheme $U$ and a surjective \'etale morphism $U \to X$.
Choose a quasi-compact open $W \subset U_{\overline{k}}$ whose
image in $X_{\overline{k}}$ is $V$. This is possible because
$|U_{\overline{k}}| \to |X_{\overline{k}}|$ is continuous and because
$|U_{\overline{k}}|$ has a basis of quasi-compact opens. We can apply
Varieties, Lemma
\ref{varieties-lemma-Galois-action-quasi-compact-open}
to $W \subset U_{\overline{k}}$ to obtain the lemma.
\end{proof}

\begin{lemma}
\label{lemma-closed-fixed-by-Galois}
Let $k$ be a field. Let $k \subset \overline{k}$ be a (possibly infinite)
Galois extension. Let $X$ be an algebraic space over $k$. Let
$\overline{T} \subset |X_{\overline{k}}|$ have the following properties
\begin{enumerate}
\item $\overline{T}$ is a closed subset of $|X_{\overline{k}}|$,
\item for every $\sigma \in \text{Gal}(\overline{k}/k)$
we have $\sigma(\overline{T}) = \overline{T}$.
\end{enumerate}
Then there exists a closed subset $T \subset |X|$ whose inverse image
in $|X_{k'}|$ is $\overline{T}$.
\end{lemma}

\begin{proof}
Let $T \subset |X|$ be the image of $\overline{T}$.
Since $|X_{\overline{k}}| \to |X|$ is surjective, the statement means
that $T$ is closed and that its inverse image is $\overline{T}$.
Choose a scheme $U$ and a surjective \'etale morphism $U \to X$.
By the case of schemes
(see Varieties, Lemma \ref{varieties-lemma-closed-fixed-by-Galois})
there exists a closed subset $T' \subset |U|$ whose inverse image
in $|U_{\overline{k}}|$ is the inverse image of $\overline{T}$.
Since $|U_{\overline{k}}| \to |X_{\overline{k}}|$ is surjective,
we see that $T'$ is the inverse image of $T$ via $|U| \to |X|$.
By our construction of the topology on $|X|$ this means that $T$ is
closed. In the same manner one sees that $\overline{T}$ is the inverse
image of $T$.
\end{proof}

\begin{lemma}
\label{lemma-characterize-geometrically-disconnected}
Let $k$ be a field. Let $X$ be an algebraic space over $k$.
The following are equivalent
\begin{enumerate}
\item $X$ is geometrically connected,
\item for every finite separable field extension $k \subset k'$
the scheme $X_{k'}$ is connected.
\end{enumerate}
\end{lemma}

\begin{proof}
This proof is identical to the proof of
Varieties, Lemma \ref{varieties-lemma-characterize-geometrically-disconnected}
except that
we replace
Varieties, Lemma \ref{varieties-lemma-characterize-geometrically-connected}
by Lemma \ref{lemma-characterize-geometrically-connected},
we replace
Varieties, Lemma \ref{varieties-lemma-Galois-action-quasi-compact-open}
by Lemma \ref{lemma-Galois-action-quasi-compact-open}, and
we replace
Varieties, Lemma \ref{varieties-lemma-closed-fixed-by-Galois}
by Lemma \ref{lemma-closed-fixed-by-Galois}.
We urge the reader to read that proof in stead of this one.

\medskip\noindent
It follows immediately from the definition that (1) implies (2).
Assume that $X$ is not geometrically connected.
Let $k \subset \overline{k}$ be a separable algebraic
closure of $k$. By
Lemma \ref{lemma-characterize-geometrically-connected}
it follows that $X_{\overline{k}}$ is disconnected.
Say $X_{\overline{k}} = \overline{U} \amalg \overline{V}$
with $\overline{U}$ and $\overline{V}$ open, closed, and nonempty
algebraic subspaces of $X_{\overline{k}}$.

\medskip\noindent
Suppose that $W \subset X$ is any quasi-compact open subspace.
Then $W_{\overline{k}} \cap \overline{U}$ and
$W_{\overline{k}} \cap \overline{V}$ are open and closed subspaces of
$W_{\overline{k}}$. In particular $W_{\overline{k}} \cap \overline{U}$ and
$W_{\overline{k}} \cap \overline{V}$ are quasi-compact, and by
Lemma \ref{lemma-Galois-action-quasi-compact-open}
both $W_{\overline{k}} \cap \overline{U}$ and
$W_{\overline{k}} \cap \overline{V}$
are defined over a finite subextension and invariant under an
open subgroup of $\text{Gal}(\overline{k}/k)$.
We will use this without further mention in the following.

\medskip\noindent
Pick $W_0 \subset X$ quasi-compact open subspace such that both
$W_{0, \overline{k}} \cap \overline{U}$ and
$W_{0, \overline{k}} \cap \overline{V}$ are nonempty.
Choose a finite subextension $k \subset k' \subset \overline{k}$
and a decomposition $W_{0, k'} = U_0' \amalg V_0'$ into open and closed
subsets such that
$W_{0, \overline{k}} \cap \overline{U} = (U'_0)_{\overline{k}}$ and
$W_{0, \overline{k}} \cap \overline{V} = (V'_0)_{\overline{k}}$.
Let $H = \text{Gal}(\overline{k}/k') \subset \text{Gal}(\overline{k}/k)$.
In particular
$\sigma(W_{0, \overline{k}} \cap \overline{U}) =
W_{0, \overline{k}} \cap \overline{U}$ and similarly for
$\overline{V}$.

\medskip\noindent
Having chosen $W_0$, $k'$ as above, for every quasi-compact open subspace
$W \subset X$ we set
$$
U_W =
\bigcap\nolimits_{\sigma \in H} \sigma(W_{\overline{k}} \cap \overline{U}),
\quad
V_W =
\bigcup\nolimits_{\sigma \in H} \sigma(W_{\overline{k}} \cap \overline{V}).
$$
Now, since $W_{\overline{k}} \cap \overline{U}$ and
$W_{\overline{k}} \cap \overline{V}$ are fixed by an open subgroup of
$\text{Gal}(\overline{k}/k)$ we see that the union and intersection
above are finite. Hence $U_W$ and $V_W$ are both open and closed subspaces.
Also, by construction $W_{\bar k} = U_W \amalg V_W$.

\medskip\noindent
We claim that if $W \subset W' \subset X$ are quasi-compact
open subspaces, then $W_{\overline{k}} \cap U_{W'} = U_W$ and
$W_{\overline{k}} \cap V_{W'} = V_W$. Verification omitted.
Hence we see that upon defining $U = \bigcup_{W \subset X} U_W$
and $V = \bigcup_{W \subset X} V_W$ we obtain
$X_{\overline{k}} = U \amalg V$ is a disjoint union of open
and closed subsets.
It is clear that $V$ is nonempty as it is constructed by taking
unions (locally). On the other hand, $U$ is nonempty since it contains
$W_0 \cap \overline{U}$ by construction. Finally, $U, V \subset X_{\bar k}$
are closed and $H$-invariant by construction. Hence by
Lemma \ref{lemma-closed-fixed-by-Galois}
we have $U = (U')_{\bar k}$, and $V = (V')_{\bar k}$ for some
closed $U', V' \subset X_{k'}$. Clearly $X_{k'} = U' \amalg V'$
and we see that $X_{k'}$ is disconnected as desired.
\end{proof}









\section{Spaces smooth over fields}
\label{section-smooth}



\begin{lemma}
\label{lemma-smooth-regular}
Let $k$ be a field.
Let $X$ be an algebraic space smooth over $k$.
Then $X$ is a regular algebraic space.
\end{lemma}

\begin{proof}
Choose a scheme $U$ and a surjective \'etale morphism $U \to X$.
The morphism $U \to \Spec(k)$ is smooth as a composition of
an \'etale (hence smooth) morphism and a smooth morphism (see
Morphisms of Spaces, Lemmas \ref{spaces-morphisms-lemma-etale-smooth}
and \ref{spaces-morphisms-lemma-composition-smooth}).
Hence $U$ is regular by
Varieties, Lemma \ref{varieties-lemma-smooth-regular}.
By
Properties of Spaces, Definition
\ref{spaces-properties-definition-type-property}
this means that $X$ is regular.
\end{proof}

\begin{lemma}
\label{lemma-smooth-separable-closed-points-dense}
Let $k$ be a field. Let $X$ be an algebraic space smooth over $\Spec(k)$.
The set of $x \in |X|$ which are image of morphisms $\Spec(k') \to X$
with $k' \supset k$ finite separable is dense in $|X|$.
\end{lemma}

\begin{proof}
Choose a scheme $U$ and a surjective \'etale morphism $U \to X$.
The morphism $U \to \Spec(k)$ is smooth as a composition of
an \'etale (hence smooth) morphism and a smooth morphism (see
Morphisms of Spaces, Lemmas \ref{spaces-morphisms-lemma-etale-smooth}
and \ref{spaces-morphisms-lemma-composition-smooth}).
Hence we can apply Varieties, Lemma
\ref{varieties-lemma-smooth-separable-closed-points-dense} to see that
the closed points of $U$ whose residue fields are finite separable over
$k$ are dense. This implies the lemma by our definition of the
topology on $|X|$.
\end{proof}








\section{Other chapters}

\begin{multicols}{2}
\begin{enumerate}
\item \hyperref[introduction-section-phantom]{Introduction}
\item \hyperref[conventions-section-phantom]{Conventions}
\item \hyperref[sets-section-phantom]{Set Theory}
\item \hyperref[categories-section-phantom]{Categories}
\item \hyperref[topology-section-phantom]{Topology}
\item \hyperref[sheaves-section-phantom]{Sheaves on Spaces}
\item \hyperref[algebra-section-phantom]{Commutative Algebra}
\item \hyperref[sites-section-phantom]{Sites and Sheaves}
\item \hyperref[homology-section-phantom]{Homological Algebra}
\item \hyperref[derived-section-phantom]{Derived Categories}
\item \hyperref[more-algebra-section-phantom]{More Algebra}
\item \hyperref[simplicial-section-phantom]{Simplicial Methods}
\item \hyperref[modules-section-phantom]{Sheaves of Modules}
\item \hyperref[sites-modules-section-phantom]{Modules on Sites}
\item \hyperref[injectives-section-phantom]{Injectives}
\item \hyperref[cohomology-section-phantom]{Cohomology of Sheaves}
\item \hyperref[sites-cohomology-section-phantom]{Cohomology on Sites}
\item \hyperref[hypercovering-section-phantom]{Hypercoverings}
\item \hyperref[schemes-section-phantom]{Schemes}
\item \hyperref[constructions-section-phantom]{Constructions of Schemes}
\item \hyperref[properties-section-phantom]{Properties of Schemes}
\item \hyperref[morphisms-section-phantom]{Morphisms of Schemes}
\item \hyperref[coherent-section-phantom]{Coherent Cohomology}
\item \hyperref[divisors-section-phantom]{Divisors}
\item \hyperref[limits-section-phantom]{Limits of Schemes}
\item \hyperref[varieties-section-phantom]{Varieties}
\item \hyperref[chow-section-phantom]{Chow Homology}
\item \hyperref[topologies-section-phantom]{Topologies on Schemes}
\item \hyperref[descent-section-phantom]{Descent}
\item \hyperref[more-morphisms-section-phantom]{More on Morphisms}
\item \hyperref[flat-section-phantom]{More on Flatness}
\item \hyperref[groupoids-section-phantom]{Groupoid Schemes}
\item \hyperref[more-groupoids-section-phantom]{More on Groupoid Schemes}
\item \hyperref[etale-section-phantom]{\'Etale Morphisms of Schemes}
\item \hyperref[etale-cohomology-section-phantom]{\'Etale Cohomology}
\item \hyperref[spaces-section-phantom]{Algebraic Spaces}
\item \hyperref[spaces-properties-section-phantom]{Properties of Algebraic Spaces}
\item \hyperref[spaces-morphisms-section-phantom]{Morphisms of Algebraic Spaces}
\item \hyperref[spaces-topologies-section-phantom]{Topologies on Algebraic Spaces}
\item \hyperref[spaces-descent-section-phantom]{Descent and Algebraic Spaces}
\item \hyperref[spaces-more-morphisms-section-phantom]{More on Morphisms of Spaces}
\item \hyperref[quot-section-phantom]{Quot and Hilbert Spaces}
\item \hyperref[stacks-section-phantom]{Stacks}
\item \hyperref[spaces-groupoids-section-phantom]{Groupoids in Algebraic Spaces}
\item \hyperref[spaces-more-groupoids-section-phantom]{More on Groupoids in Spaces}
\item \hyperref[bootstrap-section-phantom]{Bootstrap}
\item \hyperref[examples-stacks-section-phantom]{Examples of Stacks}
\item \hyperref[groupoids-quotients-section-phantom]{Quotients of Groupoids}
\item \hyperref[algebraic-section-phantom]{Algebraic Stacks}
\item \hyperref[criteria-section-phantom]{Criteria for Representability}
\item \hyperref[stacks-properties-section-phantom]{Properties of Algebraic Stacks}
\item \hyperref[stacks-morphisms-section-phantom]{Morphisms of Algebraic Stacks}
\item \hyperref[examples-section-phantom]{Examples}
\item \hyperref[exercises-section-phantom]{Exercises}
\item \hyperref[guide-section-phantom]{Guide to Literature}
\item \hyperref[desirables-section-phantom]{Desirables}
\item \hyperref[coding-section-phantom]{Coding Style}
\item \hyperref[fdl-section-phantom]{GNU Free Documentation License}
\item \hyperref[index-section-phantom]{Auto Generated Index}
\end{enumerate}
\end{multicols}


\bibliography{my}
\bibliographystyle{amsalpha}

\end{document}
