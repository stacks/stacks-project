\IfFileExists{stacks-project.cls}{%
\documentclass{stacks-project}
}{%
\documentclass{amsart}
}

% The following AMS packages are automatically loaded with
% the amsart documentclass:
%\usepackage{amsmath}
%\usepackage{amssymb}
%\usepackage{amsthm}

% For dealing with references we use the comment environment
\usepackage{verbatim}
\newenvironment{reference}{\comment}{\endcomment}
%\newenvironment{reference}{}{}
\newenvironment{slogan}{\comment}{\endcomment}
\newenvironment{history}{\comment}{\endcomment}

% For commutative diagrams you can use
% \usepackage{amscd}
\usepackage[all]{xy}

% We use 2cell for 2-commutative diagrams.
\xyoption{2cell}
\UseAllTwocells

% To put source file link in headers.
% Change "template.tex" to "this_filename.tex"
% \usepackage{fancyhdr}
% \pagestyle{fancy}
% \lhead{}
% \chead{}
% \rhead{Source file: \url{template.tex}}
% \lfoot{}
% \cfoot{\thepage}
% \rfoot{}
% \renewcommand{\headrulewidth}{0pt}
% \renewcommand{\footrulewidth}{0pt}
% \renewcommand{\headheight}{12pt}

\usepackage{multicol}

% For cross-file-references
\usepackage{xr-hyper}

% Package for hypertext links:
\usepackage{hyperref}

% For any local file, say "hello.tex" you want to link to please
% use \externaldocument[hello-]{hello}
\externaldocument[introduction-]{introduction}
\externaldocument[conventions-]{conventions}
\externaldocument[sets-]{sets}
\externaldocument[categories-]{categories}
\externaldocument[topology-]{topology}
\externaldocument[sheaves-]{sheaves}
\externaldocument[sites-]{sites}
\externaldocument[stacks-]{stacks}
\externaldocument[fields-]{fields}
\externaldocument[algebra-]{algebra}
\externaldocument[brauer-]{brauer}
\externaldocument[homology-]{homology}
\externaldocument[derived-]{derived}
\externaldocument[simplicial-]{simplicial}
\externaldocument[more-algebra-]{more-algebra}
\externaldocument[smoothing-]{smoothing}
\externaldocument[modules-]{modules}
\externaldocument[sites-modules-]{sites-modules}
\externaldocument[injectives-]{injectives}
\externaldocument[cohomology-]{cohomology}
\externaldocument[sites-cohomology-]{sites-cohomology}
\externaldocument[dga-]{dga}
\externaldocument[dpa-]{dpa}
\externaldocument[hypercovering-]{hypercovering}
\externaldocument[schemes-]{schemes}
\externaldocument[constructions-]{constructions}
\externaldocument[properties-]{properties}
\externaldocument[morphisms-]{morphisms}
\externaldocument[coherent-]{coherent}
\externaldocument[divisors-]{divisors}
\externaldocument[limits-]{limits}
\externaldocument[varieties-]{varieties}
\externaldocument[topologies-]{topologies}
\externaldocument[descent-]{descent}
\externaldocument[perfect-]{perfect}
\externaldocument[more-morphisms-]{more-morphisms}
\externaldocument[flat-]{flat}
\externaldocument[groupoids-]{groupoids}
\externaldocument[more-groupoids-]{more-groupoids}
\externaldocument[etale-]{etale}
\externaldocument[chow-]{chow}
\externaldocument[intersection-]{intersection}
\externaldocument[pic-]{pic}
\externaldocument[adequate-]{adequate}
\externaldocument[dualizing-]{dualizing}
\externaldocument[duality-]{duality}
\externaldocument[discriminant-]{discriminant}
\externaldocument[local-cohomology-]{local-cohomology}
\externaldocument[curves-]{curves}
\externaldocument[resolve-]{resolve}
\externaldocument[models-]{models}
\externaldocument[pione-]{pione}
\externaldocument[etale-cohomology-]{etale-cohomology}
\externaldocument[proetale-]{proetale}
\externaldocument[crystalline-]{crystalline}
\externaldocument[spaces-]{spaces}
\externaldocument[spaces-properties-]{spaces-properties}
\externaldocument[spaces-morphisms-]{spaces-morphisms}
\externaldocument[decent-spaces-]{decent-spaces}
\externaldocument[spaces-cohomology-]{spaces-cohomology}
\externaldocument[spaces-limits-]{spaces-limits}
\externaldocument[spaces-divisors-]{spaces-divisors}
\externaldocument[spaces-over-fields-]{spaces-over-fields}
\externaldocument[spaces-topologies-]{spaces-topologies}
\externaldocument[spaces-descent-]{spaces-descent}
\externaldocument[spaces-perfect-]{spaces-perfect}
\externaldocument[spaces-more-morphisms-]{spaces-more-morphisms}
\externaldocument[spaces-flat-]{spaces-flat}
\externaldocument[spaces-groupoids-]{spaces-groupoids}
\externaldocument[spaces-more-groupoids-]{spaces-more-groupoids}
\externaldocument[bootstrap-]{bootstrap}
\externaldocument[spaces-pushouts-]{spaces-pushouts}
\externaldocument[groupoids-quotients-]{groupoids-quotients}
\externaldocument[spaces-more-cohomology-]{spaces-more-cohomology}
\externaldocument[spaces-simplicial-]{spaces-simplicial}
\externaldocument[formal-spaces-]{formal-spaces}
\externaldocument[restricted-]{restricted}
\externaldocument[spaces-resolve-]{spaces-resolve}
\externaldocument[formal-defos-]{formal-defos}
\externaldocument[defos-]{defos}
\externaldocument[cotangent-]{cotangent}
\externaldocument[examples-defos-]{examples-defos}
\externaldocument[algebraic-]{algebraic}
\externaldocument[examples-stacks-]{examples-stacks}
\externaldocument[stacks-sheaves-]{stacks-sheaves}
\externaldocument[criteria-]{criteria}
\externaldocument[artin-]{artin}
\externaldocument[quot-]{quot}
\externaldocument[stacks-properties-]{stacks-properties}
\externaldocument[stacks-morphisms-]{stacks-morphisms}
\externaldocument[stacks-limits-]{stacks-limits}
\externaldocument[stacks-cohomology-]{stacks-cohomology}
\externaldocument[stacks-perfect-]{stacks-perfect}
\externaldocument[stacks-introduction-]{stacks-introduction}
\externaldocument[stacks-more-morphisms-]{stacks-more-morphisms}
\externaldocument[stacks-geometry-]{stacks-geometry}
\externaldocument[moduli-]{moduli}
\externaldocument[moduli-curves-]{moduli-curves}
\externaldocument[examples-]{examples}
\externaldocument[exercises-]{exercises}
\externaldocument[guide-]{guide}
\externaldocument[desirables-]{desirables}
\externaldocument[coding-]{coding}
\externaldocument[obsolete-]{obsolete}
\externaldocument[fdl-]{fdl}
\externaldocument[index-]{index}

% Theorem environments.
%
\theoremstyle{plain}
\newtheorem{theorem}[subsection]{Theorem}
\newtheorem{proposition}[subsection]{Proposition}
\newtheorem{lemma}[subsection]{Lemma}

\theoremstyle{definition}
\newtheorem{definition}[subsection]{Definition}
\newtheorem{example}[subsection]{Example}
\newtheorem{exercise}[subsection]{Exercise}
\newtheorem{situation}[subsection]{Situation}

\theoremstyle{remark}
\newtheorem{remark}[subsection]{Remark}
\newtheorem{remarks}[subsection]{Remarks}

\numberwithin{equation}{subsection}

% Macros
%
\def\lim{\mathop{\rm lim}\nolimits}
\def\colim{\mathop{\rm colim}\nolimits}
\def\Spec{\mathop{\rm Spec}}
\def\Hom{\mathop{\rm Hom}\nolimits}
\def\Ext{\mathop{\rm Ext}\nolimits}
\def\SheafHom{\mathop{\mathcal{H}\!{\it om}}\nolimits}
\def\SheafExt{\mathop{\mathcal{E}\!{\it xt}}\nolimits}
\def\Sch{\textit{Sch}}
\def\Mor{\mathop{\rm Mor}\nolimits}
\def\Ob{\mathop{\rm Ob}\nolimits}
\def\Sh{\mathop{\textit{Sh}}\nolimits}
\def\NL{\mathop{N\!L}\nolimits}
\def\proetale{{pro\text{-}\acute{e}tale}}
\def\etale{{\acute{e}tale}}
\def\QCoh{\textit{QCoh}}
\def\Ker{\mathop{\rm Ker}}
\def\Im{\mathop{\rm Im}}
\def\Coker{\mathop{\rm Coker}}
\def\Coim{\mathop{\rm Coim}}

%
% Macros for moduli stacks/spaces
%
\def\QCohstack{\mathcal{QC}\!{\it oh}}
\def\Cohstack{\mathcal{C}\!{\it oh}}
\def\Spacesstack{\mathcal{S}\!{\it paces}}
\def\Quotfunctor{{\rm Quot}}
\def\Hilbfunctor{{\rm Hilb}}
\def\Curvesstack{\mathcal{C}\!{\it urves}}
\def\Polarizedstack{\mathcal{P}\!{\it olarized}}
\def\Complexesstack{\mathcal{C}\!{\it omplexes}}
% \Pic is the operator that assigns to X its picard group, usage \Pic(X)
% \Picardstack_{X/B} denotes the Picard stack of X over B
% \Picardfunctor_{X/B} denotes the Picard functor of X over B
\def\Pic{\mathop{\rm Pic}\nolimits}
\def\Picardstack{\mathcal{P}\!{\it ic}}
\def\Picardfunctor{{\rm Pic}}
\def\Deformationcategory{\mathcal{D}\!{\it ef}}


% OK, start here.
%
\begin{document}

\title{Limits of Algebraic Spaces}


\maketitle

\phantomsection
\label{section-phantom}

\tableofcontents

\section{Introduction}
\label{section-introduction}

\noindent
In this chapter we put material related to limits of algebraic spaces.
A first topic is the characterization of algebraic spaces $F$ locally
of finite presentation over the base $S$ as limit preserving functors.
We continue with a study of limits of inverse systems over directed
partially ordered sets with affine transition maps. We discuss absolute
Noetherian approximation for quasi-compact and quasi-separated algebraic
spaces following \cite{CLO}. Another approach is due to David Rydh (see
\cite{rydh_approx}) whose results also cover absolute Noetherian
approximation for certain algebraic stacks.


\section{Conventions}
\label{section-conventions}

\noindent
The standing assumption is that all schemes are contained in
a big fppf site $\Sch_{fppf}$. And all rings $A$ considered
have the property that $\Spec(A)$ is (isomorphic) to an
object of this big site.

\medskip\noindent
Let $S$ be a scheme and let $X$ be an algebraic space over $S$.
In this chapter and the following we will write $X \times_S X$
for the product of $X$ with itself (in the category of algebraic
spaces over $S$), instead of $X \times X$.











\section{Morphisms of finite presentation}
\label{section-finite-presentation}

\noindent
In this section we generalize
Limits, Proposition
\ref{limits-proposition-characterize-locally-finite-presentation}
to morphisms of algebraic spaces.
The motivation for the following definition comes from
the proposition just cited.

\begin{definition}
\label{definition-locally-finite-presentation}
Let $S$ be a scheme.
\begin{enumerate}
\item A functor $F : (\Sch/S)_{fppf}^{opp} \to \textit{Sets}$
is said to be {\it locally of finite presentation} or {\it limit preserving} if
for every affine scheme $T$ over $S$ which is a limit $T = \lim T_i$
of a directed inverse system of affine schemes $T_i$ over $S$, we have
$$
F(T) = \colim F(T_i).
$$
We sometimes say that $F$ is {\it locally of finite presentation over $S$}.
\item Let $F, G : (\Sch/S)_{fppf}^{opp} \to \textit{Sets}$.
A transformation of functors $a : F \to G$
is {\it locally of finite presentation} if for every scheme $T$ over $S$
and every $y \in G(T)$ the functor
$$
F_y : (\Sch/T)_{fppf}^{opp} \longrightarrow \textit{Sets}, \quad
T'/T \longmapsto \{x \in F(T') \mid a(x) = y|_{T'}\}
$$
is locally of finite presentation over $T$\footnote{The characterization (2) in
Lemma \ref{lemma-characterize-relative-limit-preserving}
may be easier to parse.}. We sometimes say that
$F$ is {\it relatively limit preserving} over $G$.
\end{enumerate}
\end{definition}

\noindent
The functor $F_y$ is in some sense the fiber of
$a : F \to G$ over $y$, except that it is a presheaf on the big fppf
site of $T$. A formula for this functor is:
\begin{equation}
\label{equation-fibre-map-functors}
F_y =
F|_{(\Sch/T)_{fppf}}
{\times}_{G|_{(\Sch/T)_{fppf}}}
*
\end{equation}
Here $*$ is the final object in the category of (pre)sheaves
on $(\Sch/T)_{fppf}$ (see
Sites, Example \ref{sites-example-singleton-sheaf})
and the map $* \to G|_{(\Sch/T)_{fppf}}$ is given by $y$.
Note that if $j : (\Sch/T)_{fppf} \to (\Sch/S)_{fppf}$
is the localization functor, then the formula above becomes
$F_y = j^{-1}F \times_{j^{-1}G} *$ and $j_!F_y$ is just the fiber product
$F \times_{G, y} T$. (See
Sites, Section \ref{sites-section-localize},
for information on localization, and especially
Sites, Remark \ref{sites-remark-localize-presheaves}
for information on $j_!$ for presheaves.)

\medskip\noindent
At this point we temporarily have two definitions of what it means
for a morphism $X \to Y$ of algebraic spaces over $S$ to be locally of finite
presentation. Namely, one by
Morphisms of Spaces,
Definition \ref{spaces-morphisms-definition-locally-finite-presentation}
and one using that $X \to Y$ is a transformation of functors so that
Definition \ref{definition-locally-finite-presentation}
applies. We will show in
Proposition \ref{proposition-characterize-locally-finite-presentation}
that these two definitions agree.

\begin{lemma}
\label{lemma-characterize-relative-limit-preserving}
Let $S$ be a scheme. Let $a : F \to G$ be a transformation of functors
$(\Sch/S)_{fppf}^{opp} \to \textit{Sets}$.
The following are equivalent
\begin{enumerate}
\item $F$ is relatively limit preserving over $G$, and
\item for every every affine scheme $T$ over $S$ which is a
limit $T = \lim T_i$ of a directed inverse system of affine
schemes $T_i$ over $S$ the diagram of sets
$$
\xymatrix{
\colim_i F(T_i) \ar[r] \ar[d]_a & F(T) \ar[d]^a \\
\colim_i G(T_i) \ar[r] & G(T)
}
$$
is a fibre product diagram.
\end{enumerate}
\end{lemma}

\begin{proof}
Assume (1). Consider $T = \lim_{i \in I} T_i$ as in (2). Let
$(y, x_T)$ be an element of the fibre product
$\colim_i G(T_i) \times_{G(T)} F(T)$.
Then $y$ comes from $y_i \in G(T_i)$ for some $i$.
Consider the functor $F_{y_i}$ on $(\Sch/T_i)_{fppf}$ as in
Definition \ref{definition-locally-finite-presentation}.
We see that $x_T \in F_{y_i}(T)$. Moreover $T = \lim_{i' \geq i} T_{i'}$
is a directed system of affine schemes over $T_i$. Hence (1) implies
that $x_T$ the image of a unique element $x$ of
$\colim_{i' \geq i} F_{y_i}(T_{i'})$. Thus $x$ is the unique
element of $\colim F(T_i)$ which maps to the pair $(y, x_T)$.
This proves that (2) holds.

\medskip\noindent
Assume (2). Let $T$ be a scheme and $y_T \in G(T)$. We have to show that
$F_{y_T}$ is limit preserving. Let $T' = \lim_{i \in I} T'_i$ be an
affine scheme over $T$ which is the directed limit of affine scheme $T'_i$
over $T$. Let $x_{T'} \in F_{y_T}$. Pick $i \in I$ which is possible as
$I$ is a directed partially ordered set. Denote $y_i \in F(T'_i)$ the
image of $y_{T'}$. Then we see that $(y_i, x_{T'})$ is an
element of the fibre product
$\colim_i G(T'_i) \times_{G(T')} F(T')$.
Hence by (2) we get a unique element $x$ of $\colim_i F(T'_i)$
mapping to $(y_i, x_{T'})$. It is clear that $x$ defines an element
of $\colim_i F_y(T'_i)$ mapping to $x_{T'}$ and we win.
\end{proof}

\begin{lemma}
\label{lemma-composition-locally-finite-presentation}
Let $S$ be a scheme contained in $\Sch_{fppf}$.
Let $F, G, H : (\Sch/S)_{fppf}^{opp} \to \textit{Sets}$.
Let $a : F \to G$, $b : G \to H$ be transformations of functors.
If $a$ and $b$ are locally of finite presentation, then
$$
b \circ a : F \longrightarrow H
$$
is locally of finite presentation.
\end{lemma}

\begin{proof}
Let $T = \lim_{i \in I} T_i$ as in characterization (2) of
Lemma \ref{lemma-characterize-relative-limit-preserving}.
Consider the diagram
$$
\xymatrix{
\colim_i F(T_i) \ar[r] \ar[d]_a & F(T) \ar[d]^a \\
\colim_i G(T_i) \ar[r] \ar[d]_b & G(T) \ar[d]^b \\
\colim_i H(T_i) \ar[r] & H(T)
}
$$
By assumption the two squares are fibre product squares. Hence the
outer rectangle is a fibre product diagram too which proves the lemma.
\end{proof}

\begin{lemma}
\label{lemma-base-change-locally-finite-presentation}
Let $S$ be a scheme contained in $\Sch_{fppf}$.
Let $F, G, H : (\Sch/S)_{fppf}^{opp} \to \textit{Sets}$.
Let $a : F \to G$, $b : H \to G$ be transformations of functors.
Consider the fibre product diagram
$$
\xymatrix{
H \times_{b, G, a} F \ar[r]_-{b'} \ar[d]_{a'} & F \ar[d]^a \\
H \ar[r]^b & G
}
$$
If $a$ is locally of finite presentation, then the base change $a'$ is
locally of finite presentation.
\end{lemma}

\begin{proof}
Omitted. Hint: This is formal.
\end{proof}

\begin{lemma}
\label{lemma-limit-fppf-topology}
Let $T$ be an affine scheme which is written as a limit
$T = \lim_{i \in I} T_i$ of a directed inverse system of affine schemes.
\begin{enumerate}
\item Let $\mathcal{V} = \{V_j \to T\}_{j = 1, \ldots, m}$ be a standard fppf
covering of $T$, see
Topologies, Definition \ref{topologies-definition-standard-fppf}.
Then there exists an index $i$ and a standard fppf covering
$\mathcal{V}_i = \{V_{i, j} \to T_i\}_{j = 1, \ldots, m}$
whose base change $T \times_{T_i} \mathcal{V}_i$ to $T$
is isomorphic to $\mathcal{V}$.
\item Let $\mathcal{V}_i$, $\mathcal{V}'_i$ be a pair of standard
fppf coverings of $T_i$. If
$f : T \times_{T_i} \mathcal{V} \to T \times_{T_i} \mathcal{V}'_i$ is
a morphism of coverings of $T$, then there exists an index
$i' \geq i$ and a morphism
$f_{i'} : T_{i'} \times_{T_i} \mathcal{V} \to
T_{i'} \times_{T_i} \mathcal{V}'_i$
whose base change to $T$ is $f$.
\item If
$f, g : \mathcal{V} \to \mathcal{V}'_i$
are morphisms of standard fppf coverings of $T_i$ whose
base changes $f_T, g_T$ to $T$ are equal then there exists an
index $i' \geq i$ such that $f_{T_{i'}} = g_{T_{i'}}$.
\end{enumerate}
In other words, the category of standard fppf coverings of $T$ is
the colimit over $I$ of the categories of standard fppf coverings of $T_i$
\end{lemma}

\begin{proof}
By
Limits, Lemma \ref{limits-lemma-descend-finite-presentation}
the category of schemes of finite presentation over $T$ is the
colimit over $I$ of the categories of finite presentation over $T_i$. By
Limits, Lemmas \ref{limits-lemma-descend-affine-finite-presentation}
and \ref{limits-lemma-descend-flat-finite-presentation}
the same is true for category of schemes which are affine, flat and
of finite presentation over $T$.
To finish the proof of the lemma it suffices to show that if
$\{V_{j, i} \to T_i\}_{j = 1, \ldots, m}$ is a finite family of
flat finitely presented morphisms with $V_{j, i}$ affine, and the
base change $\coprod_j T \times_{T_i} V_{j, i} \to T$ is surjective,
then for some $i' \geq i$ the morphism
$\coprod T_{i'} \times_{T_i} V_{j, i} \to T_{i'}$ is surjective.
Denote $W_{i'} \subset T_{i'}$, resp.\ $W \subset T$ the image.
Of course $W = T$ by assumption.
Since the morphisms are flat and of finite presentation we see that
$W_i$ is a quasi-compact open of $T_i$, see
Morphisms, Lemma \ref{morphisms-lemma-fppf-open}.
Moreover, $W = T \times_{T_i} W_i$ (formation of image commutes
with base change). Hence by
Limits, Lemma \ref{limits-lemma-descend-opens}
we conclude that $W_{i'} = T_{i'}$ for some large enough $i'$
and we win.
\end{proof}

\begin{lemma}
\label{lemma-sheafify-finite-presentation}
Let $S$ be a scheme contained in $\Sch_{fppf}$.
Let $F : (\Sch/S)_{fppf}^{opp} \to \textit{Sets}$ be a functor.
If $F$ is locally of finite presentation over $S$ then its sheafification
$F^\#$ is locally of finite presentation over $S$.
\end{lemma}

\begin{proof}
Assume $F$ is locally of finite presentation.
It suffices to show that $F^+$ is locally of finite presentation, since
$F^\# = (F^+)^+$, see
Sites, Theorem \ref{sites-theorem-plus}.
Let $T$ be an affine scheme over $S$, and let $T = \lim T_i$ be written
as the directed limit of an inverse system of affine $S$ schemes.
Recall that $F^+(T)$ is the colimit of $\check H^0(\mathcal{V}, F)$
where the limit is over all coverings of $T$ in $(\Sch/S)_{fppf}$.
Any fppf covering of an affine scheme can be refined by a standard
fppf covering, see
Topologies, Lemma \ref{topologies-lemma-fppf-affine}.
Hence we can write
$$
F^+(T)
=
\colim_{\mathcal{V}\text{ standard covering }T}
\check H^0(\mathcal{V}, F).
$$
By
Lemma \ref{lemma-limit-fppf-topology}
we may rewrite this as
$$
\colim_{i \in I}
\colim_{\mathcal{V}_i\text{ standard covering }T_i}
\check H^0(T \times_{T_i}\mathcal{V}_i, F).
$$
(The order of the colimits is irrelevant by
Categories, Lemma \ref{categories-lemma-colimits-commute}.)
Given a standard fppf covering
$\mathcal{V}_i = \{V_j \to T_i\}_{j = 1, \ldots, m}$ of $T_i$ we see that
$$
T \times_{T_i} V_j = \lim_{i' \geq i} T_{i'} \times_T V_j
$$
by
Limits, Lemma \ref{limits-lemma-scheme-over-limit}, and similarly
$$
T \times_{T_i} (V_j \times_{T_i} V_{j'}) =
\lim_{i' \geq i} T_{i'} \times_T (V_j \times_{T_i} V_{j'}).
$$
As the presheaf $F$ is locally of finite presentation this means that
$$
\check H^0(T \times_{T_i}\mathcal{V}_i, F)
=
\colim_{i' \geq i}
\check H^0(T_{i'} \times_{T_i}\mathcal{V}_i, F).
$$
Hence the colimit expression for $F^+(T)$ above collapses to
$$
\colim_{i \in I} \colim_{\mathcal{V}\text{ standard covering }T_i}
\check H^0(\mathcal{V}, F).
=
\colim_{i \in I} F^+(T_i).
$$
In other words $F^+(T) = \colim_i F^+(T_i)$ and hence
the lemma holds.
\end{proof}

\begin{lemma}
\label{lemma-sheaf-finite-presentation}
Let $S$ be a scheme.
Let $F : (\Sch/S)_{fppf}^{opp} \to \textit{Sets}$ be a functor.
Assume that
\begin{enumerate}
\item $F$ is a sheaf, and
\item there exists an fppf covering $\{U_j \to S\}_{j \in J}$ such that
$F|_{(\Sch/U_j)_{fppf}}$ is locally of finite presentation.
\end{enumerate}
Then $F$ is locally of finite presentation.
\end{lemma}

\begin{proof}
Let $T$ be an affine scheme over $S$.
Let $I$ be a directed partially ordered set, and let
$T_i$ be an inverse system of affine schemes over $S$ such that
$T = \lim T_i$. We have to show that the canonical
map $\colim F(T_i) \to F(T)$ is bijective.

\medskip\noindent
Choose some $0 \in I$ and choose a standard fppf covering
$\{V_{0, k} \to T_{0}\}_{k = 1, \ldots, m}$ which refines
the pullback $\{U_j \times_S T_0 \to T_0\}$ of the given fppf covering of $S$.
For each $i \geq 0$ we set $V_{i, k} = T_i \times_{T_0} V_{0, k}$, and
we set $V_k = T \times_{T_0} V_{0, k}$. Note that
$V_k = \lim_{i \geq 0} V_{i, k}$, see
Limits, Lemma \ref{limits-lemma-scheme-over-limit}.

\medskip\noindent
Suppose that $x, x' \in \colim F(T_i)$ map to the same
element of $F(T)$. Say $x, x'$ are given by elements $x_i, x'_i \in F(T_i)$
for some $i \in I$ (we may choose the same $i$ for both as $I$ is directed).
By assumption (2) and the fact that $x_i, x'_i$ map to the same element
of $F(T)$ this implies that
$$
x_i|_{V_{i', k}} = x'_i|_{V_{i', k}}
$$
for some suitably large $i' \in I$. We can choose the same $i'$ for each
$k$ as $k \in \{1, \ldots, m\}$ ranges over a finite set.
Since $\{V_{i', k} \to T_{i'}\}$
is an fppf covering and $F$ is a sheaf this implies that
$x_i|_{T_{i'}} = x'_i|_{T_{i'}}$ as desired. This proves that the map
$\colim F(T_i) \to F(T)$ is injective.

\medskip\noindent
To show surjectivity we argue in a similar fashion.
Let $x \in F(T)$. By assumption (2) for each $k$ we
can choose a $i$ such that $x|_{V_k}$ comes from an
element $x_{i, k} \in F(V_{i, k})$. As before we may choose a
single $i$ which works for all $k$. By the injectivity
proved above we see that
$$
x_{i, k}|_{V_{i', k} \times_{T_{i'}} V_{i', l}}
=
x_{i, l}|_{V_{i', k} \times_{T_{i'}} V_{i', l}}
$$
for some large enough $i'$. Hence by the sheaf condition of $F$
the elements $x_{i, k}|_{V_{i', k}}$ glue to an element $x_{i'} \in F(T_{i'})$
as desired.
\end{proof}

\begin{lemma}
\label{lemma-sheafify-finite-presentation-map}
Let $S$ be a scheme contained in $\Sch_{fppf}$.
Let $F, G : (\Sch/S)_{fppf}^{opp} \to \textit{Sets}$ be functors.
If $a : F \to G$ is a transformation which is locally of finite
presentation, then the induced transformation of sheaves
$F^\# \to G^\#$ is of finite presentation.
\end{lemma}

\begin{proof}
Suppose that $T$ is a scheme and $y \in G^\#(T)$.
We have to show the functor
$F^\#_y : (\Sch/T)_{fppf}^{opp} \to \textit{Sets}$
constructed from $F^\# \to G^\#$ and $y$ as in
Definition \ref{definition-locally-finite-presentation}
is locally of finite presentation.
By Equation (\ref{equation-fibre-map-functors})
we see that $F^\#_y$ is a sheaf. Choose an fppf covering
$\{V_j \to T\}_{j \in J}$ such that $y|_{V_j}$ comes from
an element $y_j \in F(V_j)$.
Note that the restriction of $F^\#$ to $(\Sch/V_j)_{fppf}$
is just $F^\#_{y_j}$. If we can show that $F^\#_{y_j}$ is
locally of finite presentation then
Lemma \ref{lemma-sheaf-finite-presentation}
garantees that $F^\#_y$ is locally of finite presentation and
we win. This reduces us to the case $y \in G(T)$.

\medskip\noindent
Let $y \in G(T)$. In this case we claim that $F^\#_y = (F_y)^\#$.
This follows from
Equation (\ref{equation-fibre-map-functors}).
Thus this case follows from
Lemma \ref{lemma-sheafify-finite-presentation}.
\end{proof}

\begin{proposition}
\label{proposition-characterize-locally-finite-presentation}
Let $S$ be a scheme. Let $f : X \to Y$ be a morphism of algebraic
spaces over $S$. The following are equivalent:
\begin{enumerate}
\item The morphism $f$ is a morphism of algebraic spaces which is
locally of finite presentation, see
Morphisms of Spaces,
Definition \ref{spaces-morphisms-definition-locally-finite-presentation}.
\item The morphism $f : X \to Y$ is locally of finite presentation as
a transformation of functors, see
Definition \ref{definition-locally-finite-presentation}.
\end{enumerate}
\end{proposition}

\begin{proof}
Assume (1). Let $T$ be a scheme and let $y \in Y(T)$. We have to show that
$T \times_X Y$ is locally of finite presentation over $T$ in the sense of
Definition \ref{definition-locally-finite-presentation}.
Hence we are reduced to proving that if $X$ is an algebraic space which
is locally of finite presentation over $S$ as an algebraic space, then it
is locally of finite presentation as a functor
$X : (\Sch/S)_{fppf}^{opp} \to \textit{Sets}$.
To see this choose a presentation $X = U/R$, see
Spaces, Definition \ref{spaces-definition-presentation}.
It follows from
Morphisms of Spaces,
Definition \ref{spaces-morphisms-definition-locally-finite-presentation}
that both $U$ and $R$ are schemes which are locally of finite presentation
over $S$. Hence by
Limits, Proposition
\ref{limits-proposition-characterize-locally-finite-presentation}
we have
$$
U(T) = \colim U(T_i), \quad
R(T) = \colim R(T_i)
$$
whenever $T = \lim_i T_i$ in $(\Sch/S)_{fppf}$. It follows
that the presheaf
$$
(\Sch/S)_{fppf}^{opp} \longrightarrow \textit{Sets}, \quad
W \longmapsto U(W)/R(W)
$$
is locally of finite presentation. Hence by
Lemma \ref{lemma-sheafify-finite-presentation}
its sheafification $X = U/R$ is locally of finite presentation too.

\medskip\noindent
Assume (2). Choose a scheme $V$ and a surjective \'etale morphism
$V \to Y$. Next, choose a scheme $U$ and a surjective \'etale morphism
$U \to V \times_Y X$. By
Lemma \ref{lemma-base-change-locally-finite-presentation}
the transformation of functors $V \times_Y X \to V$ is locally of
finite presentation. By
Morphisms of Spaces,
Lemma \ref{spaces-morphisms-lemma-etale-locally-finite-presentation}
the morphism of algebraic spaces $U \to V \times_Y X$ is locally
of finite presentation, hence locally of finite presentation as
a transformation of functors by the first part of the proof. By
Lemma \ref{lemma-composition-locally-finite-presentation}
the composition $U \to V \times_Y X \to V$ is locally of
finite presentation as a transformation of functors. Hence
the morphism of schemes $U \to V$ is locally of finite presentation by
Limits, Proposition
\ref{limits-proposition-characterize-locally-finite-presentation}
(modulo a set theoretic remark, see last paragraph of the proof).
This means, by definition, that (1) holds.

\medskip\noindent
Set theoretic remark. Let $U \to V$ be a morphism of
$(\Sch/S)_{fppf}$. In the statement of
Limits, Proposition
\ref{limits-proposition-characterize-locally-finite-presentation}
we characterize $U \to V$ as being locally of finite presentation
if for {\it all} directed inverse systems $(T_i, f_{ii'})$ of affine schemes
over $V$ we have $U(T) = \colim V(T_i)$, but in the current setting
we may only consider affine schemes $T_i$ over $V$ which are (isomorphic to)
an object of $(\Sch/S)_{fppf}$. So we have to make sure that there
are enough affines in $(\Sch/S)_{fppf}$ to make the proof work.
Inspecting the proof of (2) $\Rightarrow$ (1) of
Limits, Proposition
\ref{limits-proposition-characterize-locally-finite-presentation}
we see that the question reduces to the case that $U$ and $V$ are affine.
Say $U = \Spec(A)$ and $V = \Spec(B)$. By construction
of $(\Sch/S)_{fppf}$ the spectrum of any ring of cardinality
$\leq |B|$ is isomorphic to an object of $(\Sch/S)_{fppf}$.
Hence it suffices to observe that in the "only if" part of the proof of
Algebra, Lemma \ref{algebra-lemma-characterize-finite-presentation}
only $A$-algebras of cardinality $\leq |B|$ are used.
\end{proof}

\begin{remark}
\label{remark-limit-preserving}
Here is an important special case of
Proposition \ref{proposition-characterize-locally-finite-presentation}.
Let $S$ be a scheme. Let $X$ be an algebraic space over $S$.
Then $X$ is locally of finite presentation over $S$ if and only
if $X$, as a functor $(\Sch/S)^{opp} \to \textit{Sets}$,
is limit preserving. Compare with
Limits, Remark \ref{limits-remark-limit-preserving}.
\end{remark}















\section{Limits of algebraic spaces}
\label{section-limits}

\noindent
The following lemma explains how we think of limits of algebraic
spaces in this chapter. We will use (without further mention) that the
base change of an affine morphism of algebraic spaces is affine (see
Morphisms of Spaces, Lemma \ref{spaces-morphisms-lemma-base-change-affine}).

\begin{lemma}
\label{lemma-directed-inverse-system-has-limit}
Let $S$ be a scheme. Let $I$ be a directed partially ordered set.
Let $(X_i, f_{ii'})$ be an inverse system over $I$
in the category of algebraic spaces over $S$.
If the morphisms $f_{ii'} : X_i \to X_{i'}$ are affine, then the
limit $X = \lim_i X_i$ (as an fppf sheaf) is an algebraic space.
Moreover,
\begin{enumerate}
\item each of the morphisms $f_i : X \to X_i$ is affine,
\item for any $i \in I$ and any morphism of algebraic spaces
$T \to X_i$ we have
$$
X \times_{X_i} T = \lim_{i' \geq i} X_{i'} \times_{X_i} T.
$$
as algebraic spaces over $S$.
\end{enumerate}
\end{lemma}

\begin{proof}
Part (2) is a formal consequence of the existence of the
limit $X = \lim X_i$ as an algebraic space over $S$.
Choose an element $0 \in I$ (this is possible as a directed partially
ordered set is nonempty). Choose a scheme $U_0$ and a surjective
\'etale morphism $U_0 \to X_0$. Set $R_0 = U_0 \times_{X_0} U_0$
so that $X_0 = U_0/R_0$. For $i \geq 0$ set
$U_i = X_i \times_{X_0} U_0$ and
$R_i = X_i \times_{X_0} R_0 = U_i \times_{X_i} U_i$.
By Limits, Lemma \ref{limits-lemma-directed-inverse-system-has-limit}
we see that $U = \lim_{i \geq 0} U_i$ and $R = \lim_{i \geq 0} R_i$
are schemes. Moreover, the two morphisms $s, t : R \to U$ are the base
change of the two projections $R_0 \to U_0$ by the morphism
$U \to U_0$ and the morphism $R \to U \times_S U$ is the base change
of the morphism $R_0 \to U_0 \times_S U_0$ by the morphism
$U \times_S U \to U_0 \times_S U_0$. Hence the morphism
$R \to U \times_S U$ is an \'etale equivalence relation. We claim that
the natural map
\begin{equation}
\label{equation-isomorphism-sheaves}
U/R \longrightarrow \lim X_i
\end{equation}
is an isomorphism of fppf sheaves on the category of schemes over $S$.
The claim implies $X = \lim X_i$ is an algebraic
space by Spaces, Theorem \ref{spaces-theorem-presentation}.

\medskip\noindent
Let $Z$ be a scheme and let $a : Z \to \lim X_i$ be a morphism.
Then $a = (a_i)$ where $a_i : Z \to X_i$. Set $W_0 = Z \times_{a_0, X_0} U_0$.
Note that $W_0 = Z \times_{a_i, X_i} U_i$ for all $i \geq 0$ by our
choice of $U_i \to X_i$ above. Hence we obtain a morphism
$W_0 \to \lim_{i \geq 0} U_i = U$. Since $W_0 \to Z$ is surjective
and \'etale, we conclude that (\ref{equation-isomorphism-sheaves})
is a surjective map of sheaves. Finally, suppose that
$Z$ is a scheme and that $a, b : Z \to U/R$ are two morphisms
which are equalized by (\ref{equation-isomorphism-sheaves}).
We have to show that $a = b$.
After replacing $Z$ by the members of an fppf covering
we may assume there exist morphisms $a', b' : Z \to U$ which
give rise to $a$ and $b$. The condition that $a, b$ are
equalized by (\ref{equation-isomorphism-sheaves}) means that
for each $i \geq 0$ the compositions $a_i', b_i' : Z \to U \to U_i$
are equal as morphisms into $U_i/R_i = X_i$. Hence
$(a_i', b_i') : Z \to U_i \times_S U_i$ factors through
$R_i$, say by some morphism $c_i : Z \to R_i$. Since
$R = \lim_{i \geq 0} R_i$ we see that $c = \lim c_i : Z \to R$
is a morphism which shows that $a, b$ are equal as morphisms
of $Z$ into $U/R$.

\medskip\noindent
Part (1) follows as we have seen above that
$U_i \times_{X_i} X = U$ and $U \to U_i$ is affine by
construction.
\end{proof}

\begin{lemma}
\label{lemma-space-over-limit}
Let $S$ be a scheme. Let $I$ be a directed partially ordered set.
Let $(X_i, f_{ii'})$ be an inverse system over $I$ of algebraic spaces
over $S$ with affine transition maps.
Let $X = \lim_i X_i$. Let $0 \in I$. Suppose that $T \to X_0$ is a
morphism of algebraic spaces. Then
$$
T \times_{X_0} X = \lim_{i \geq 0} T \times_{X_0} X_i
$$
as algebraic spaces over $S$.
\end{lemma}

\begin{proof}
The limit $X$ is an algebraic space by
Lemma \ref{lemma-directed-inverse-system-has-limit}.
The equality is formal, see
Categories, Lemma \ref{categories-lemma-colimits-commute}.
\end{proof}

\begin{lemma}
\label{lemma-descend-section}
Let $S$ be a scheme. Let $I$ be a directed partially ordered set.
Let $(X_i, f_{ii'})$ be an inverse system over $I$ of algebraic spaces
over $S$. Assume
\begin{enumerate}
\item the morphisms $f_{ii'} : X_i \to X_{i'}$ are affine,
\item each $X_i$ is quasi-compact and quasi-separated.
\end{enumerate}
Let $X = \lim_i X_i$. Let $0 \in I$. Suppose that $\mathcal{F}_0$ is a
quasi-coherent sheaf on $X_0$. Set $\mathcal{F}_i = f_{0i}^*\mathcal{F}_0$
for $i \geq 0$ and set $\mathcal{F} = f_0^*\mathcal{F}_0$. Then
$$
\Gamma(X, \mathcal{F}) = \colim_{i \geq 0} \Gamma(X_i, \mathcal{F}_i)
$$
\end{lemma}

\begin{proof}
Choose a surjective \'etale morphism $U_0 \to X_0$ where $U_0$ is an affine
scheme (Properties of Spaces, Lemma
\ref{spaces-properties-lemma-quasi-compact-affine-cover}).
Set $U_i = X_i \times_{X_0} U_0$.
Set $R_0 = U_0 \times_{X_0} U_0$ and $R_i = R_0 \times_{X_0} X_i$.
In the proof of Lemma \ref{lemma-directed-inverse-system-has-limit} we have
seen that there exists a presentation $X = U/R$ with
$U = \lim U_i$ and $R = \lim R_i$.
Note that $U_i$ and $U$ are affine and that $R_i$ and $R$ are
quasi-compact and separated (as $X_i$ is quasi-separated). Hence
Limits, Lemma \ref{limits-lemma-descend-section}
implies that
$$
\mathcal{F}(U) = \colim \mathcal{F}_i(U_i)
\quad\text{and}\quad
\mathcal{F}(R) = \colim \mathcal{F}_i(R_i).
$$
The lemma follows as
$\Gamma(X, \mathcal{F}) = \text{Ker}(\mathcal{F}(U) \to \mathcal{F}(R))$
and similarly
$\Gamma(X_i, \mathcal{F}_i) =
\text{Ker}(\mathcal{F}_i(U_i) \to \mathcal{F}_i(R_i))$
\end{proof}






\section{Descending relative objects}
\label{section-descending-relative}

\noindent
The following lemma is typical of the type of results in this section.

\begin{lemma}
\label{lemma-descend-finite-presentation}
Let $S$ be a scheme. Let $I$ be a directed partially ordered set.
Let $(X_i, f_{ii'})$ be an inverse system over $I$ of algebraic spaces
over $S$. Assume
\begin{enumerate}
\item the morphisms $f_{ii'} : X_i \to X_{i'}$ are affine,
\item the spaces $X_i$ are quasi-compact and quasi-separated.
\end{enumerate}
Let $X = \lim_i X_i$. Then the category of algebraic spaces
of finite presentation over $X$ is the colimit over $I$ of the
categories of algebraic spaces of finite presentation over $X_i$.
\end{lemma}

\begin{proof}
Pick $0 \in I$. Choose a surjective \'etale morphism $U_0 \to X_0$ where
$U_0$ is an affine scheme (Properties of Spaces, Lemma
\ref{spaces-properties-lemma-quasi-compact-affine-cover}).
Set $U_i = X_i \times_{X_0} U_0$. Set $R_0 = U_0 \times_{X_0} U_0$ and
$R_i = R_0 \times_{X_0} X_i$. Denote $s_i, t_i : R_i \to U_i$ and
$s, t : R \to U$ the two projections. In the proof of
Lemma \ref{lemma-directed-inverse-system-has-limit} we have
seen that there exists a presentation $X = U/R$ with
$U = \lim U_i$ and $R = \lim R_i$. Note that $U_i$ and $U$ are affine and
that $R_i$ and $R$ are quasi-compact and separated (as $X_i$ is
quasi-separated). Let $Y$ be an algebraic space over $S$ and let
$Y \to X$ be a morphism of finite presentation. Set $V = U \times_X Y$.
This is an algebraic space of finite presentation over $U$.
Choose an affine scheme $W$ and a surjective \'etale morphism $W \to V$.
Then $W \to Y$ is surjective \'etale as well. Set $R' = W \times_Y W$
so that $Y = W/R'$ (see Spaces, Section \ref{spaces-section-presentations}).
Note that $W$ is a scheme of finite presentation over $U$ and that $R'$
is a scheme of finite presentation over $R$ (details omitted).
By Limits, Lemma \ref{limits-lemma-descend-finite-presentation}
we can find an index $i$ and a morphism of schemes $W_i \to U_i$ of
finite presentation whose base change to $U$ gives $W \to U$. Similarly
we can find, after possibly increasing $i$, a scheme $R'_i$ of finite
presentation over $R_i$ whose base change to $R$ is $R'$.
The projection morphisms $s', t' : R' \to W$ are morphisms over
the projection morphisms $s, t : R \to U$. Hence we can view $s'$,
resp.\ $t'$ as a morphism between schemes of finite presentation over
$U$ (with structure morphism $R' \to U$ given by $R' \to R$ followed
by $s$, resp.\ $t$). Hence we can apply
Limits, Lemma \ref{limits-lemma-descend-finite-presentation}
again to see that, after possibly increasing $i$, there exist
morphisms $s'_i, t'_i : R'_i \to W_i$, whose base change to $U$
is $S', t'$. By Limits, Lemmas \ref{limits-lemma-descend-etale} and
\ref{limits-lemma-descend-monomorphism}
we may assume that $s'_i, t'_i$ are \'etale and that
$j'_i : R'_i \to W_i \times_{X_i} W_i$ is a monomorphism (here we
view $j'_i$ as a morphsm of schemes of finite presentation over $U_i$ via
one of the projections -- it doesn't matter which one). Setting
$Y_i = W_i/R'_i$ (see Spaces, Theorem \ref{spaces-theorem-presentation})
we obtain an algebraic space of finite presentation
over $X_i$ whose base change to $X$ is isomorphic to $Y$.

\medskip\noindent
This shows that every algebraic space of finite presentation over $X$ comes
from an algebraic space of finite presentation over some $X_i$, i.e.,
it shows that the functor of the lemma is essentially surjective. To
show that it is fully faithful, consider an index $0 \in I$ and two
algebraic spaces $Y_0, Z_0$ of finite presentation over $X_0$.
Set $Y_i = X_i \times_{X_0} Y_0$, $Y = X \times_{X_0} Y_0$,
$Z_i = X_i \times_{X_0} Z_0$, and $Z = X \times_{X_0} Z_0$. Let
$\alpha : Y \to Z$ be a morphism of algebraic spaces over $X$.
Choose a surjective \'etale morphism $V_0 \to Y_0$ where $V_0$ is
an affine scheme. Set $V_i = V_0 \times_{Y_0} Y_i$ and
$V = V_0 \times_{Y_0} Y$ which are affine schemes endowed with
surjective \'etale morphisms to $Y_i$ and $Y$. The composition
$V \to Y \to Z \to Z_0$ comes from a (essentially unique) morphism
$V_i \to Z_0$ for some $i \geq 0$ by
Proposition \ref{proposition-characterize-locally-finite-presentation}
(applied to $Z_0 \to X_0$ which is of finite presentation by assumption).
After increasing $i$ the two compositions
$$
V_i \times_{Y_i} V_i \to V_i \to Z_0
$$
are equal as this is true in the limit. Hence we obtain a (essentially unique)
morphism $Y_i \to Z_0$. Since this is a morphism over $X_0$
it induces a morphism into $Z_i = Z_0 \times_{X_0} X_i$ as desired.
\end{proof}

\begin{lemma}
\label{lemma-descend-etale}
With notation and assumptions as in
Lemma \ref{lemma-descend-finite-presentation}.
Let $0 \in I$. Suppose that $\varphi_0 : Y_0 \to Z_0$
is a morphism of algebraic spaces of finite presentation over $X_0$.
If the base change of $\varphi_0$ to $X$ is \'etale
then there exists an index $i \geq 0$ such that
the base change of $\varphi_0$ to $X_i$ is \'etale.
\end{lemma}

\begin{proof}
Choose an affine scheme $U_0$ and a surjective \'etale morphism
$U_0 \to X_0$. Choose an affine scheme $W_0$ and a surjective \'etale
morphism $W_0 \to U_0 \times_{X_0} Z_0$. Choose an affine scheme
$V_0$ and a surjective \'etale morphism $V_0 \to W_0 \times_{Z_0} Y_0$.
Diagram
$$
\xymatrix{
V_0 \ar[d] \ar[r] & W_0 \ar[d] \ar[r] & U_0 \ar[d] \\
Y_0 \ar[r] & Z_0 \ar[r] & X_0
}
$$
The vertical arrows are surjective and \'etale by construction.
Recall that $Y_0 \to Z_0$ is \'etale if and only if $V_0 \to W_0$ is
\'etale (see
Morphisms of Spaces, Lemma \ref{spaces-morphisms-lemma-etale-local}).
We can base change this diagram to $X_i$ or $X$ and the same equivalence
holds. Hence the lemma follows from the case of schemes, which is
Limits, Lemma \ref{limits-lemma-descend-etale}.
\end{proof}

\begin{lemma}
\label{lemma-descend-monomorphism}
With notation and assumptions as in
Lemma \ref{lemma-descend-finite-presentation}.
Let $0 \in I$. Suppose that $\varphi_0 : Y_0 \to Z_0$
is a morphism of schemes of finite presentation over $X_0$.
If the base change of $\varphi_0$ to $X$ is a monomorphism
then there exists an index $i \geq 0$ such that
the base change of $\varphi_0$ to $X_i$ is a monomorphism.
\end{lemma}

\begin{proof}
Recall that a morphism $Y \to Z$ is a monomorphism if and
only if the diagonal $Y \to Y \times_Z Y$ is an isomorphism
(Morphisms of Spaces, Lemma \ref{spaces-morphisms-lemma-monomorphism}).
Observe that $Y_0 \times_{Z_0} Y_0$ is of finite presentation over $Z_0$
because morphisms of finite presentation are preserved under base change and
composition, see Morphisms of Spaces, Section
\ref{spaces-morphisms-section-finite-presentation}.
Hence the lemma follows from Lemma \ref{lemma-descend-finite-presentation}
by considering the morphism $Y_0 \to Y_0 \times_{Z_0} Y_0$.
\end{proof}

\begin{lemma}
\label{lemma-descend-surjective}
With notation and assumptions as in
Lemma \ref{lemma-descend-finite-presentation}.
Let $0 \in I$. Suppose that $\varphi_0 : X_0 \to Y_0$
is a morphism of schemes of finite presentation over $S_0$.
If the base change of $\varphi_0$ to $S$ is surjective
then there exists an index $i \geq 0$ such that
the base change of $\varphi_0$ to $S_i$ is surjective.
\end{lemma}

\begin{proof}
Choose an affine scheme $U_0$ and a surjective \'etale morphism
$U_0 \to X_0$. Choose an affine scheme $W_0$ and a surjective \'etale
morphism $W_0 \to U_0 \times_{X_0} Z_0$. Choose an affine scheme
$V_0$ and a surjective \'etale morphism $V_0 \to W_0 \times_{Z_0} Y_0$.
Diagram
$$
\xymatrix{
V_0 \ar[d] \ar[r] & W_0 \ar[d] \ar[r] & U_0 \ar[d] \\
Y_0 \ar[r] & Z_0 \ar[r] & X_0
}
$$
The vertical arrows are surjective and \'etale by construction.
Since $V_0 \to  W_0 \times_{Z_0} Y_0$ is surjective we see that
$Y_0 \to Z_0$ is surjective if and only if $V_0 \to W_0$ is
surjective. We can base change this diagram to $X_i$ or $X$ and
the same equivalence holds. Hence the lemma follows from the case of
schemes, which is
Limits, Lemma \ref{limits-lemma-descend-surjective}.
\end{proof}

\begin{lemma}
\label{lemma-descend-modules-finite-presentation}
With notation and assumptions as in
Lemma \ref{lemma-descend-finite-presentation}.
The category of $\mathcal{O}_X$-modules of finite presentation is the
colimit over $I$ of the categories $\mathcal{O}_{X_i}$-modules of finite
presentation.
\end{lemma}

\begin{proof}
Choose $0 \in I$. Choose an affine scheme $U_0$ and a surjective
\'etale morphism $U_0 \to X_0$. Set $U_i = X_i \times_{X_0} U_0$.
Set $R_0 = U_0 \times_{X_0} U_0$ and $R_i = R_0 \times_{X_0} X_i$.
Denote $s_i, t_i : R_i \to U_i$ and $s, t : R \to U$ the two
projections. In the proof of
Lemma \ref{lemma-directed-inverse-system-has-limit} we have
seen that there exists a presentation $X = U/R$ with
$U = \lim U_i$ and $R = \lim R_i$. Note that $U_i$ and $U$ are affine and
that $R_i$ and $R$ are quasi-compact and separated (as $X_i$ is
quasi-separated). Moreover, it is also true that
$R \times_{s, U, t} R = \colim R_i \times_{s_i, U_i, t_i} R_i$.
Thus we know that $\textit{QCoh}(U) = \colim \textit{QCoh}(U_i)$,
$\textit{QCoh}(R) = \colim \textit{QCoh}(R_i)$, and
$\textit{QCoh}(R \times_{s, U, t} R) = \colim
\textit{QCoh}(R_i \times_{s_i, U_i, t_i} R_i)$ by
Limits, Lemma \ref{limits-lemma-descend-modules-finite-presentation}.
We have $\textit{QCoh}(X) = \textit{QCoh}(U, R, s, t, c)$ and
$\textit{QCoh}(X_i) = \textit{QCoh}(U_i, R_i, s_i, t_i, c_i)$, see
Properties of Spaces, Proposition
\ref{spaces-properties-proposition-quasi-coherent}.
Thus the result follows formally.
\end{proof}









\section{More on limits}
\label{section-more-limits}

\noindent
This section is a continuation of Section \ref{section-limits}.

\begin{lemma}
\label{lemma-limit-is-affine}
In the situation of Lemma \ref{lemma-directed-inverse-system-has-limit}
assume that
\begin{enumerate}
\item each $X_i$ is quasi-separated and quasi-compact, and
\item $X = \lim X_i$ is affine.
\end{enumerate}
Then there exists an $i$ such that $X_i$ is affine.
\end{lemma}

\begin{proof}
Choose $0 \in I$. Choose an affine scheme $U_0$ and a surjective
\'etale morphism $U_0 \to X_0$. Set $U = U_0 \times_{X_0} X$
and $U_i = U_0 \times_{X_0} X_i$ for $i \geq 0$.
Then $U \to X$ is an \'etale morphism of affine schemes. Hence we
can write $X = \Spec(A)$, $U = \Spec(B)$ and
$$
B = A[x_1, \ldots, x_n]/(g_1, \ldots, g_n)
$$
such that $\Delta = \det(\partial g_\lambda/\partial x_\mu)$ is invertible
in $B$, see Algebra, Lemma \ref{algebra-lemma-etale-standard-smooth}.
Set $A_i = \mathcal{O}_{X_i}(X_i)$. We have $A \colim A_i$ by
Lemma \ref{lemma-descend-section}. For some $i$ we can find
$g_{1, i}, \ldots, g_{n, i} \in A_i[x_1, \ldots, x_n]$ mapping to
$g_1, \ldots, g_n$. Set
$B_i = A_i[x_1, \ldots, x_n]/(g_{1, i}, \ldots, g_{n, i})$.
Increasing $i$ if necessary we may assume that
$\Delta_i = \det(\partial g_{\lambda, i}/\partial x_\mu)$ is invertible
in $B_i$. Thus $A_i \to B_i$ is an \'etale ring map.
After increasing $i$ we may assume also that
$\Spec(B_i) \to \Spec(A_i)$ is surjective, see
Limits, Lemma \ref{limits-lemma-descend-surjective}. Increasing
$i$ yet again we can find elements
$h_{1, i}, \ldots, h_{n, i} \in \mathcal{O}_{U_i}(U_i)$ which map to the
classes of $x_1, \ldots, x_n$ in $B = \mathcal{O}_U(U)$ and such that
$g_{\lambda, i}(h_{\nu, i}) = 0$ in $\mathcal{O}_{U_i}(U_i)$. Thus
we obtain a commutative diagram
$$
\xymatrix{
X_i \ar[d] & U_i \ar[l] \ar[d] \\
\Spec(A_i) & \Spec(B_i) \ar[l]
}
$$
By construction the base change of $\Spec(B_i) \to \Spec(A_i)$ to
$X = \Spec(A)$ is isomorphic to $U = \Spec(B)$. Hence by
Lemma \ref{lemma-descend-finite-presentation}
for large $i$ the morphism
$$
U_i \longrightarrow X_i \times_{\Spec(A_i)} \Spec(B_i)
$$
is an isomorphism. At this point
Descent, Lemma \ref{descent-lemma-descent-data-sheaves}
applied to the fppf covering $\{\Spec(B_i) \to \Spec(A_i)\}$
combined with Descent, Lemma \ref{descent-lemma-affine}
give that $X_i \to \Spec(A_i)$ is representable by a scheme
affine over $\Spec(A_i)$ as desired. (Of course it then also follows
that $X_i = \Spec(A_i)$ but we don't need this.)
\end{proof}

\begin{lemma}
\label{lemma-limit-is-scheme}
In the situation of Lemma \ref{lemma-directed-inverse-system-has-limit}
assume that
\begin{enumerate}
\item each $X_i$ is quasi-separated and quasi-compact, and
\item $X = \lim X_i$ is representable (by a scheme).
\end{enumerate}
Then there exists an $i$ such that $X_i$ is representable.
\end{lemma}

\begin{proof}
Choose a finite affine open covering $X = \bigcup W_j$.
By Lemma \ref{lemma-descend-finite-presentation}
we can find an $i \in I$ and morphisms $W_{j, i} \to X_i$
whose base change to $X$ is $W_j \to X$. By
Lemma \ref{lemma-limit-is-affine} we may assume that
each $W_{j, i}$ is an affine scheme. By
Lemmas \ref{lemma-descend-etale} and \ref{lemma-descend-monomorphism}
we may assume each $W_{j, i} \to X_i$ is \'etale and a monomorphism
hence an open immersion (see
Morphisms of Spaces, Lemma
\ref{spaces-morphisms-lemma-etale-universally-injective-open}).
By Lemma \ref{lemma-descend-surjective} we see that we may assume
$\coprod W_{j, i} \to X_i$ is surjective. This means that $X_i$
is a scheme (see for example
Properties of Spaces, Section \ref{spaces-properties-section-schematic}).
\end{proof}









\section{Absolute Noetherian approximation}
\label{section-approximation}

\noindent
We use the following result which is almost identical to
\cite[Proposition 5.7.8]{GruRay}.

\begin{lemma}
\label{lemma-filter-quasi-compact-quasi-separated}
Let $X$ be a quasi-compact and quasi-separated algebraic space over
$\Spec(\mathbf{Z})$. There exist an integer $n$ and open subspaces
$$
\emptyset = U_{n + 1} \subset
U_n \subset U_{n - 1} \subset \ldots \subset U_1 = X
$$
with the following property: setting $T_p = U_p \setminus U_{p + 1}$
(with reduced induced subspace structure) there exists a quasi-compact
separated scheme $V_p$ and a surjective \'etale morphism $f_p : V_p \to U_p$
such that $f_p^{-1}(T_p) \to T_p$ is an isomorphism.
\end{lemma}

\begin{proof}
The proof of this lemma is identical to the proof of
Decent Spaces, Lemma \ref{decent-spaces-lemma-filter-reasonable}.
Observe that a quasi-separated space is reasonable, see
Decent Spaces, Lemma \ref{decent-spaces-lemma-bounded-fibres} and
Decent Spaces, Definition \ref{decent-spaces-definition-very-reasonable}.
At the end of the argument we add that since $X$ is quasi-separated
the schemes $V \times_X \ldots \times_X V$ are all quasi-compact.
Hence the schemes $W_p$ are quasi-compact. Hence the schemes
$V_p = W_p/S_p$ are quasi-compact.
\end{proof}

\begin{proposition}
\label{proposition-approximate}
Let $X$ be a quasi-compact and quasi-separated algebraic space over
$\Spec(\mathbf{Z})$. There exist a directed partially ordered set $I$
and an inverse system of algebraic spaces $(X_i, f_{ii'})$ over $I$
such that
\begin{enumerate}
\item the transition morphisms $f_{ii'}$ are affine
\item each $X_i$ is quasi-separated and of finite type over
$\mathbf{Z}$, and
\item $X = \lim X_i$.
\end{enumerate}
\end{proposition}

\begin{proof}
Following \cite{CLO}. We apply Lemma
\ref{lemma-filter-quasi-compact-quasi-separated}
to get open subspaces $U_p \subset X$, schemes $V_p$, and morphisms
$f_p : V_p \to U_p$ with properties as stated. Note that
$f_n : V_n \to U_n$ is an \'etale morphism of algebraic spaces
whose restriction to the inverse image of $T_n = (V_n)_{red}$ is an
isomorphism. Hence $f_n$ is an isomorphism, for example by
Morphisms of Spaces, Lemma
\ref{spaces-morphisms-lemma-etale-universally-injective-open}.
In particular $U_n$ is a quasi-compact and separated scheme.
Thus we can write $U_n = \lim U_{n, i}$ as a directed limit
of schemes of finite type over $\mathbf{Z}$ with affine transition
morphisms, see Limits, Proposition \ref{limits-proposition-approximate}.
Thus, applying descending induction on $p$, we see that we have reduced
to the problem posed in the following paragraph.

\medskip\noindent
Here we have $U \subset X$, $U = \lim U_i$, $Z \subset X$, and
$f : V \to X$ with the following properties
\begin{enumerate}
\item $X$ is a quasi-compact and quasi-separated algebraic space,
\item $V$ is a quasi-compact and separated scheme,
\item $U \subset X$ is a quasi-compact open subspace,
\item $(U_i, g_{ii'})$ is a directed system of quasi-separated algebraic spaces
of finite type over $\mathbf{Z}$ with affine transition morphisms
whose limit is $U$,
\item $Z \subset X$ is a closed subspace such that $|X| = |U| \amalg |Z|$,
\item $f : V \to X$ is a surjective \'etale morphism such that
$f^{-1}(Z) \to Z$ is an isomorphism.
\end{enumerate}
Problem: Show that the conclusion of the proposition holds for $X$.

\medskip\noindent
Note that $W = f^{-1}(U) \subset V$ is a quasi-compact open subscheme
\'etale over $U$. Hence we may apply
Lemmas \ref{lemma-descend-finite-presentation} and \ref{lemma-descend-etale}
to find an index $0 \in I$ and an \'etale morphism $W_0 \to U_0$
of finite presentation whose base change to $U$ produces $W$. Setting
$W_i = W_0 \times_{U_0} U_i$ we see that $W = \lim_{i \geq 0} W_i$. After
increasing $0$ we may assume the $W_i$ are schemes, see
Lemma \ref{lemma-limit-is-scheme}.
Moreover, $W_i$ is of finite type over $\mathbf{Z}$.

\medskip\noindent
Apply Limits, Lemma \ref{limits-lemma-approximate} to
$W = \lim_{i \geq 0} W_i$ and the inclusion $W \subset V$. Replace $I$
by the directed partially ordered set $J$ found in that lemma. This allows us
to write $V$ as a directed limit $V = \lim V_i$ of finite type schemes over
$\mathbf{Z}$ with affine transition maps such that each $V_i$ contains
$W_i$ as an open subscheme (compatible with transition morphisms).
For each $i$ we can form the push out
$$
\xymatrix{
W_i \ar[r] \ar[d]_\Delta & V_i \ar[d] \\
W_i \times_{U_i} W_i \ar[r] & R_i
}
$$
in the category of schemes. Namely, the left vertical and upper horizontal
arrows are open immersions of schemes. In other words, we can construct
$R_i$ as the glueing of $V_i$ and $W_i \times_{U_i} W_i$ along the common open
$W_i$ (see Schemes, Section \ref{schemes-section-glueing-schemes}). Note that
the \'etale projection maps $W_i \times_{U_i} W_i \to W_i$ extend
to \'etale morphisms $s_i, t_i : R_i \to V_i$. It is clear that the
morphism $j_i = (t_i, s_i) : R_i \to V_i \times V_i$ is an \'etale
equivalence relation on $V_i$. Note that $W_i \times_{U_i} W_i$ is
quasi-compact (as $U_i$ is quasi-separated and $W_i$ quasi-compact)
and $V_i$ is quasi-compact, hence $R_i$ is quasi-compact. For
$i \geq i'$ the diagram
\begin{equation}
\label{equation-cartesian}
\vcenter{
\xymatrix{
R_i \ar[r] \ar[d]_{s_i} & R_{i'} \ar[d]^{s_{i'}} \\
V_i \ar[r] & V_{i'}
}
}
\end{equation}
is cartesian because
$$
(W_{i'} \times_{U_{i'}} W_{i'}) \times_{U_{i'}} U_i =
W_{i'} \times_{U_{i'}} U_i \times_{U_i} U_i \times_{U_{i'}} W_{i'} =
W_i \times_{U_i} W_i.
$$
Consider the algebraic space $X_i = V_i/R_i$ (see
Spaces, Theorem \ref{spaces-theorem-presentation}).
As $V_i$ is of finite type over $\mathbf{Z}$ and $R_i$ is quasi-compact
we see that $X_i$ is quasi-separated and of finite type over $\mathbf{Z}$
(see
Properties of Spaces, Lemma \ref{spaces-properties-lemma-quasi-separated}
and
Morphisms of Spaces, Lemmas
\ref{spaces-morphisms-lemma-surjection-from-quasi-compact} and
\ref{spaces-morphisms-lemma-finite-type-local}).
As the construction of $R_i$ above is compatible
with transition morphisms, we obtain morphisms of algebraic spaces
$X_i \to X_{i'}$ for $i \geq i'$. The commutative diagrams
$$
\xymatrix{
V_i \ar[r] \ar[d] & V_{i'} \ar[d] \\
X_i \ar[r] & X_{i'}
}
$$
are cartesian as (\ref{equation-cartesian}) is cartesian, see
Groupoids, Lemma \ref{groupoids-lemma-criterion-fibre-product}.
Since $V_i \to V_{i'}$ is affine, this implies that $X_i \to X_{i'}$
is affine, see
Morphisms of Spaces, Lemma \ref{spaces-morphisms-lemma-affine-local}.
Thus we can form the limit $X' = \lim X_i$ by
Lemma \ref{lemma-directed-inverse-system-has-limit}.
We claim that $X \cong X'$ which finishes the proof of the proposition.

\medskip\noindent
Proof of the claim. Set $R = \lim R_i$.
By construction the algebraic space $X'$ comes
equipped with a surjective \'etale morphism $V \to X'$ such that
$$
V \times_{X'} V \cong R
$$
(use Lemma \ref{lemma-directed-inverse-system-has-limit}).
By construction $\lim W_i \times_{U_i} W_i = W \times_U W$ and $V = \lim V_i$
so that $R$ is the union of $W \times_U W$ and $V$ glued along $W$.
Property (6) implies the projections $V \times_X V \to V$ are isomorphisms
over $f^{-1}(Z) \subset V$. Hence the scheme $V \times_X V$ is the union
of the opens $\Delta_{V/X}(V)$ and $W \times_U W$ which intersect
along $\Delta_{W/X}(W)$. We conclude that there exists a unique isomorphism
$R \cong V \times_X V$ compatible with the projections to $V$.
Since $V \to X$ and $V \to X'$ are surjective \'etale we see that
$$
X = V/ V \times_X V = V/R = V/V \times_{X'} V = X'
$$
by Spaces, Lemma \ref{spaces-lemma-space-presentation} and we win.
\end{proof}




\section{Applications}
\label{section-applications}

\noindent
The following lemma can also be deduced directly from
Lemma \ref{lemma-filter-quasi-compact-quasi-separated}
without passing through absolute Noetherian approximation.

\begin{lemma}
\label{lemma-colimit-finitely-presented}
Let $S$ be a scheme. Let $X$ be a quasi-compact and quasi-separated algebraic
space over $S$. Every quasi-coherent $\mathcal{O}_X$-module is a
filted colimit of finitely presented $\mathcal{O}_X$-modules.
\end{lemma}

\begin{proof}
We may view as an algebraic space over $\Spec(\mathbf{Z})$, see
Spaces, Definition \ref{spaces-definition-base-change} and
Properties of Spaces, Definition \ref{spaces-properties-definition-separated}.
Thus we may apply Proposition \ref{proposition-approximate}
and write $X = \lim X_i$ with $X_i$ of finite presentation over $\mathbf{Z}$.
Thus $X_i$ is a Noetherian algebraic space, see
Morphisms of Spaces, Lemma
\ref{spaces-morphisms-lemma-finite-presentation-noetherian}.
Also, the morphism $X \to X_i$ is affine, see
Lemma \ref{lemma-directed-inverse-system-has-limit}.
We conclude by
Cohomology of Spaces, Lemma
\ref{spaces-cohomology-lemma-direct-colimit-finite-presentation}.
\end{proof}










\section{Other chapters}

\begin{multicols}{2}
\begin{enumerate}
\item \hyperref[introduction-section-phantom]{Introduction}
\item \hyperref[conventions-section-phantom]{Conventions}
\item \hyperref[sets-section-phantom]{Set Theory}
\item \hyperref[categories-section-phantom]{Categories}
\item \hyperref[topology-section-phantom]{Topology}
\item \hyperref[sheaves-section-phantom]{Sheaves on Spaces}
\item \hyperref[algebra-section-phantom]{Commutative Algebra}
\item \hyperref[sites-section-phantom]{Sites and Sheaves}
\item \hyperref[homology-section-phantom]{Homological Algebra}
\item \hyperref[derived-section-phantom]{Derived Categories}
\item \hyperref[more-algebra-section-phantom]{More Algebra}
\item \hyperref[simplicial-section-phantom]{Simplicial Methods}
\item \hyperref[modules-section-phantom]{Sheaves of Modules}
\item \hyperref[sites-modules-section-phantom]{Modules on Sites}
\item \hyperref[injectives-section-phantom]{Injectives}
\item \hyperref[cohomology-section-phantom]{Cohomology of Sheaves}
\item \hyperref[sites-cohomology-section-phantom]{Cohomology on Sites}
\item \hyperref[hypercovering-section-phantom]{Hypercoverings}
\item \hyperref[schemes-section-phantom]{Schemes}
\item \hyperref[constructions-section-phantom]{Constructions of Schemes}
\item \hyperref[properties-section-phantom]{Properties of Schemes}
\item \hyperref[morphisms-section-phantom]{Morphisms of Schemes}
\item \hyperref[coherent-section-phantom]{Coherent Cohomology}
\item \hyperref[divisors-section-phantom]{Divisors}
\item \hyperref[limits-section-phantom]{Limits of Schemes}
\item \hyperref[varieties-section-phantom]{Varieties}
\item \hyperref[chow-section-phantom]{Chow Homology}
\item \hyperref[topologies-section-phantom]{Topologies on Schemes}
\item \hyperref[descent-section-phantom]{Descent}
\item \hyperref[more-morphisms-section-phantom]{More on Morphisms}
\item \hyperref[flat-section-phantom]{More on Flatness}
\item \hyperref[groupoids-section-phantom]{Groupoid Schemes}
\item \hyperref[more-groupoids-section-phantom]{More on Groupoid Schemes}
\item \hyperref[etale-section-phantom]{\'Etale Morphisms of Schemes}
\item \hyperref[etale-cohomology-section-phantom]{\'Etale Cohomology}
\item \hyperref[spaces-section-phantom]{Algebraic Spaces}
\item \hyperref[spaces-properties-section-phantom]{Properties of Algebraic Spaces}
\item \hyperref[spaces-morphisms-section-phantom]{Morphisms of Algebraic Spaces}
\item \hyperref[spaces-topologies-section-phantom]{Topologies on Algebraic Spaces}
\item \hyperref[spaces-descent-section-phantom]{Descent and Algebraic Spaces}
\item \hyperref[spaces-more-morphisms-section-phantom]{More on Morphisms of Spaces}
\item \hyperref[quot-section-phantom]{Quot and Hilbert Spaces}
\item \hyperref[stacks-section-phantom]{Stacks}
\item \hyperref[spaces-groupoids-section-phantom]{Groupoids in Algebraic Spaces}
\item \hyperref[spaces-more-groupoids-section-phantom]{More on Groupoids in Spaces}
\item \hyperref[bootstrap-section-phantom]{Bootstrap}
\item \hyperref[examples-stacks-section-phantom]{Examples of Stacks}
\item \hyperref[groupoids-quotients-section-phantom]{Quotients of Groupoids}
\item \hyperref[algebraic-section-phantom]{Algebraic Stacks}
\item \hyperref[criteria-section-phantom]{Criteria for Representability}
\item \hyperref[stacks-properties-section-phantom]{Properties of Algebraic Stacks}
\item \hyperref[stacks-morphisms-section-phantom]{Morphisms of Algebraic Stacks}
\item \hyperref[examples-section-phantom]{Examples}
\item \hyperref[exercises-section-phantom]{Exercises}
\item \hyperref[guide-section-phantom]{Guide to Literature}
\item \hyperref[desirables-section-phantom]{Desirables}
\item \hyperref[coding-section-phantom]{Coding Style}
\item \hyperref[fdl-section-phantom]{GNU Free Documentation License}
\item \hyperref[index-section-phantom]{Auto Generated Index}
\end{enumerate}
\end{multicols}


\bibliography{my}
\bibliographystyle{amsalpha}

\end{document}

