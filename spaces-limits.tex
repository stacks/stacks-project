\IfFileExists{stacks-project.cls}{%
\documentclass{stacks-project}
}{%
\documentclass{amsart}
}

% The following AMS packages are automatically loaded with
% the amsart documentclass:
%\usepackage{amsmath}
%\usepackage{amssymb}
%\usepackage{amsthm}

% For dealing with references we use the comment environment
\usepackage{verbatim}
\newenvironment{reference}{\comment}{\endcomment}
%\newenvironment{reference}{}{}
\newenvironment{slogan}{\comment}{\endcomment}
\newenvironment{history}{\comment}{\endcomment}

% For commutative diagrams you can use
% \usepackage{amscd}
\usepackage[all]{xy}

% We use 2cell for 2-commutative diagrams.
\xyoption{2cell}
\UseAllTwocells

% To put source file link in headers.
% Change "template.tex" to "this_filename.tex"
% \usepackage{fancyhdr}
% \pagestyle{fancy}
% \lhead{}
% \chead{}
% \rhead{Source file: \url{template.tex}}
% \lfoot{}
% \cfoot{\thepage}
% \rfoot{}
% \renewcommand{\headrulewidth}{0pt}
% \renewcommand{\footrulewidth}{0pt}
% \renewcommand{\headheight}{12pt}

\usepackage{multicol}

% For cross-file-references
\usepackage{xr-hyper}

% Package for hypertext links:
\usepackage{hyperref}

% For any local file, say "hello.tex" you want to link to please
% use \externaldocument[hello-]{hello}
\externaldocument[introduction-]{introduction}
\externaldocument[conventions-]{conventions}
\externaldocument[sets-]{sets}
\externaldocument[categories-]{categories}
\externaldocument[topology-]{topology}
\externaldocument[sheaves-]{sheaves}
\externaldocument[sites-]{sites}
\externaldocument[stacks-]{stacks}
\externaldocument[fields-]{fields}
\externaldocument[algebra-]{algebra}
\externaldocument[brauer-]{brauer}
\externaldocument[homology-]{homology}
\externaldocument[derived-]{derived}
\externaldocument[simplicial-]{simplicial}
\externaldocument[more-algebra-]{more-algebra}
\externaldocument[smoothing-]{smoothing}
\externaldocument[modules-]{modules}
\externaldocument[sites-modules-]{sites-modules}
\externaldocument[injectives-]{injectives}
\externaldocument[cohomology-]{cohomology}
\externaldocument[sites-cohomology-]{sites-cohomology}
\externaldocument[dga-]{dga}
\externaldocument[dpa-]{dpa}
\externaldocument[hypercovering-]{hypercovering}
\externaldocument[schemes-]{schemes}
\externaldocument[constructions-]{constructions}
\externaldocument[properties-]{properties}
\externaldocument[morphisms-]{morphisms}
\externaldocument[coherent-]{coherent}
\externaldocument[divisors-]{divisors}
\externaldocument[limits-]{limits}
\externaldocument[varieties-]{varieties}
\externaldocument[topologies-]{topologies}
\externaldocument[descent-]{descent}
\externaldocument[perfect-]{perfect}
\externaldocument[more-morphisms-]{more-morphisms}
\externaldocument[flat-]{flat}
\externaldocument[groupoids-]{groupoids}
\externaldocument[more-groupoids-]{more-groupoids}
\externaldocument[etale-]{etale}
\externaldocument[chow-]{chow}
\externaldocument[intersection-]{intersection}
\externaldocument[pic-]{pic}
\externaldocument[adequate-]{adequate}
\externaldocument[dualizing-]{dualizing}
\externaldocument[duality-]{duality}
\externaldocument[discriminant-]{discriminant}
\externaldocument[local-cohomology-]{local-cohomology}
\externaldocument[curves-]{curves}
\externaldocument[resolve-]{resolve}
\externaldocument[models-]{models}
\externaldocument[pione-]{pione}
\externaldocument[etale-cohomology-]{etale-cohomology}
\externaldocument[proetale-]{proetale}
\externaldocument[crystalline-]{crystalline}
\externaldocument[spaces-]{spaces}
\externaldocument[spaces-properties-]{spaces-properties}
\externaldocument[spaces-morphisms-]{spaces-morphisms}
\externaldocument[decent-spaces-]{decent-spaces}
\externaldocument[spaces-cohomology-]{spaces-cohomology}
\externaldocument[spaces-limits-]{spaces-limits}
\externaldocument[spaces-divisors-]{spaces-divisors}
\externaldocument[spaces-over-fields-]{spaces-over-fields}
\externaldocument[spaces-topologies-]{spaces-topologies}
\externaldocument[spaces-descent-]{spaces-descent}
\externaldocument[spaces-perfect-]{spaces-perfect}
\externaldocument[spaces-more-morphisms-]{spaces-more-morphisms}
\externaldocument[spaces-flat-]{spaces-flat}
\externaldocument[spaces-groupoids-]{spaces-groupoids}
\externaldocument[spaces-more-groupoids-]{spaces-more-groupoids}
\externaldocument[bootstrap-]{bootstrap}
\externaldocument[spaces-pushouts-]{spaces-pushouts}
\externaldocument[groupoids-quotients-]{groupoids-quotients}
\externaldocument[spaces-more-cohomology-]{spaces-more-cohomology}
\externaldocument[spaces-simplicial-]{spaces-simplicial}
\externaldocument[formal-spaces-]{formal-spaces}
\externaldocument[restricted-]{restricted}
\externaldocument[spaces-resolve-]{spaces-resolve}
\externaldocument[formal-defos-]{formal-defos}
\externaldocument[defos-]{defos}
\externaldocument[cotangent-]{cotangent}
\externaldocument[examples-defos-]{examples-defos}
\externaldocument[algebraic-]{algebraic}
\externaldocument[examples-stacks-]{examples-stacks}
\externaldocument[stacks-sheaves-]{stacks-sheaves}
\externaldocument[criteria-]{criteria}
\externaldocument[artin-]{artin}
\externaldocument[quot-]{quot}
\externaldocument[stacks-properties-]{stacks-properties}
\externaldocument[stacks-morphisms-]{stacks-morphisms}
\externaldocument[stacks-limits-]{stacks-limits}
\externaldocument[stacks-cohomology-]{stacks-cohomology}
\externaldocument[stacks-perfect-]{stacks-perfect}
\externaldocument[stacks-introduction-]{stacks-introduction}
\externaldocument[stacks-more-morphisms-]{stacks-more-morphisms}
\externaldocument[stacks-geometry-]{stacks-geometry}
\externaldocument[moduli-]{moduli}
\externaldocument[moduli-curves-]{moduli-curves}
\externaldocument[examples-]{examples}
\externaldocument[exercises-]{exercises}
\externaldocument[guide-]{guide}
\externaldocument[desirables-]{desirables}
\externaldocument[coding-]{coding}
\externaldocument[obsolete-]{obsolete}
\externaldocument[fdl-]{fdl}
\externaldocument[index-]{index}

% Theorem environments.
%
\theoremstyle{plain}
\newtheorem{theorem}[subsection]{Theorem}
\newtheorem{proposition}[subsection]{Proposition}
\newtheorem{lemma}[subsection]{Lemma}

\theoremstyle{definition}
\newtheorem{definition}[subsection]{Definition}
\newtheorem{example}[subsection]{Example}
\newtheorem{exercise}[subsection]{Exercise}
\newtheorem{situation}[subsection]{Situation}

\theoremstyle{remark}
\newtheorem{remark}[subsection]{Remark}
\newtheorem{remarks}[subsection]{Remarks}

\numberwithin{equation}{subsection}

% Macros
%
\def\lim{\mathop{\rm lim}\nolimits}
\def\colim{\mathop{\rm colim}\nolimits}
\def\Spec{\mathop{\rm Spec}}
\def\Hom{\mathop{\rm Hom}\nolimits}
\def\Ext{\mathop{\rm Ext}\nolimits}
\def\SheafHom{\mathop{\mathcal{H}\!{\it om}}\nolimits}
\def\SheafExt{\mathop{\mathcal{E}\!{\it xt}}\nolimits}
\def\Sch{\textit{Sch}}
\def\Mor{\mathop{\rm Mor}\nolimits}
\def\Ob{\mathop{\rm Ob}\nolimits}
\def\Sh{\mathop{\textit{Sh}}\nolimits}
\def\NL{\mathop{N\!L}\nolimits}
\def\proetale{{pro\text{-}\acute{e}tale}}
\def\etale{{\acute{e}tale}}
\def\QCoh{\textit{QCoh}}
\def\Ker{\mathop{\rm Ker}}
\def\Im{\mathop{\rm Im}}
\def\Coker{\mathop{\rm Coker}}
\def\Coim{\mathop{\rm Coim}}

%
% Macros for moduli stacks/spaces
%
\def\QCohstack{\mathcal{QC}\!{\it oh}}
\def\Cohstack{\mathcal{C}\!{\it oh}}
\def\Spacesstack{\mathcal{S}\!{\it paces}}
\def\Quotfunctor{{\rm Quot}}
\def\Hilbfunctor{{\rm Hilb}}
\def\Curvesstack{\mathcal{C}\!{\it urves}}
\def\Polarizedstack{\mathcal{P}\!{\it olarized}}
\def\Complexesstack{\mathcal{C}\!{\it omplexes}}
% \Pic is the operator that assigns to X its picard group, usage \Pic(X)
% \Picardstack_{X/B} denotes the Picard stack of X over B
% \Picardfunctor_{X/B} denotes the Picard functor of X over B
\def\Pic{\mathop{\rm Pic}\nolimits}
\def\Picardstack{\mathcal{P}\!{\it ic}}
\def\Picardfunctor{{\rm Pic}}
\def\Deformationcategory{\mathcal{D}\!{\it ef}}


% OK, start here.
%
\begin{document}

\title{Limits of Algebraic Spaces}


\maketitle

\phantomsection
\label{section-phantom}

\tableofcontents

\section{Introduction}
\label{section-introduction}

\noindent
In this chapter we work out absolute Noetherian approximation for
quasi-compact and quasi-separated algebraic spaces following
\cite{CLO}. Another approach is due to David Rydh (see
\cite{rydh_approx}) whose results
also cover absolute Noetherian approximation for certain algebraic stacks.


\section{Conventions}
\label{section-conventions}

\noindent
The standing assumption is that all schemes are contained in
a big fppf site $\Sch_{fppf}$. And all rings $A$ considered
have the property that $\Spec(A)$ is (isomorphic) to an
object of this big site.

\medskip\noindent
Let $S$ be a scheme and let $X$ be an algebraic space over $S$.
In this chapter and the following we will write $X \times_S X$
for the product of $X$ with itself (in the category of algebraic
spaces over $S$), instead of $X \times X$.




\section{Directed limits of algebraic spaces with affine transition maps}
\label{section-limits}

\noindent
The following lemma explains how we think of limits of algebraic
spaces in this chapter. We will use (without further mention) that the
base change of an affine morphism of algebraic spaces is affine (see
Morphisms of Spaces, Lemma \ref{spaces-morphisms-lemma-base-change-affine}).

\begin{lemma}
\label{lemma-directed-inverse-system-has-limit}
Let $S$ be a scheme. Let $I$ be a directed partially ordered set.
Let $(X_i, f_{ii'})$ be an inverse system over $I$
in the category of algebraic spaces over $S$.
If the morphisms $f_{ii'} : X_i \to X_{i'}$ are affine, then the
limit $X = \lim_i X_i$ (as an fppf sheaf) is an algebraic space.
Moreover,
\begin{enumerate}
\item each of the morphisms $f_i : X \to X_i$ is affine,
\item for any $i \in I$ and any morphism of algebraic spaces
$T \to X_i$ we have
$$
X \times_{X_i} T = \lim_{i' \geq i} X_{i'} \times_{X_i} T.
$$
as algebraic spaces over $S$.
\end{enumerate}
\end{lemma}

\begin{proof}
Part (2) is a formal consequence of the existence of the
limit $X = \lim X_i$ as an algebraic space over $S$.
Choose an element $0 \in I$ (this is possible as a directed partially
ordered set is nonempty). Choose a scheme $U_0$ and a surjective
\'etale morphism $U_0 \to X_0$. Set $R_0 = U_0 \times_{X_0} U_0$
so that $X_0 = U_0/R_0$. For $i \geq 0$ set
$U_i = X_i \times_{X_0} U_0$ and
$R_i = X_i \times_{X_0} R_0 = U_i \times_{X_i} U_i$.
By Limits, Lemma \ref{limits-lemma-directed-inverse-system-has-limit}
we see that $U = \lim_{i \geq 0} U_i$ and $R = \lim_{i \geq 0} R_i$
are schemes. Moreover, the two morphisms $s, t : R \to U$ are the base
change of the two projections $R_0 \to U_0$ by the morphism
$U \to U_0$ and the morphism $R \to U \times_S U$ is the base change
of the morphism $R_0 \to U_0 \times_S U_0$ by the morphism
$U \times_S U \to U_0 \times_S U_0$. Hence the morphism
$R \to U \times_S U$ is an \'etale equivalence relation. We claim that
the natural map
\begin{equation}
\label{equation-isomorphism-sheaves}
U/R \longrightarrow \lim X_i
\end{equation}
is an isomorphism of fppf sheaves on the category of schemes over $S$.
The claim implies $X = \lim X_i$ is an algebraic
space by Spaces, Theorem \ref{spaces-theorem-presentation}.

\medskip\noindent
Let $Z$ be a scheme and let $a : Z \to \lim X_i$ be a morphism.
Then $a = (a_i)$ where $a_i : Z \to X_i$. Set $W_0 = Z \times_{a_0, X_0} U_0$.
Note that $W_0 = Z \times_{a_i, X_i} U_i$ for all $i \geq 0$ by our
choice of $U_i \to X_i$ above. Hence we obtain a morphism
$W_0 \to \lim_{i \geq 0} U_i = U$. Since $W_0 \to Z$ is surjective
and \'etale, we conclude that (\ref{equation-isomorphism-sheaves})
is a surjective map of sheaves. Finally, suppose that
$Z$ is a scheme and that $a, b : Z \to U/R$ are two morphisms
which are equalized by (\ref{equation-isomorphism-sheaves}).
We have to show that $a = b$.
After replacing $Z$ by the members of an fppf covering
we may assume there exist morphisms $a', b' : Z \to U$ which
give rise to $a$ and $b$. The condition that $a, b$ are
equalized by (\ref{equation-isomorphism-sheaves}) means that
for each $i \geq 0$ the compositions $a_i', b_i' : Z \to U \to U_i$
are equal as morphisms into $U_i/R_i = X_i$. Hence
$(a_i', b_i') : Z \to U_i \times_S U_i$ factors through
$R_i$, say by some morphism $c_i : Z \to R_i$. Since
$R = \lim_{i \geq 0} R_i$ we see that $c = \lim c_i : Z \to R$
is a morphism which shows that $a, b$ are equal as morphisms
of $Z$ into $U/R$.

\medskip\noindent
Part (1) follows as we have seen above that
$U_i \times_{X_i} X = U$ and $U \to U_i$ is affine by
construction.
\end{proof}

\begin{lemma}
\label{lemma-descend-section}
Let $S$ be a scheme. Let $I$ be a directed partially ordered set.
Let $(X_i, f_{ii'})$ be an inverse system over $I$ of algebraic spaces
over $S$. Assume
\begin{enumerate}
\item the morphisms $f_{ii'} : X_i \to X_{i'}$ are affine,
\item each $X_i$ is quasi-compact and quasi-separated.
\end{enumerate}
Let $X = \lim_i X_i$. Let $0 \in I$. Suppose that $\mathcal{F}_0$ is a
quasi-coherent sheaf on $X_0$. Set $\mathcal{F}_i = f_{0i}^*\mathcal{F}_0$
for $i \geq 0$ and set $\mathcal{F} = f_0^*\mathcal{F}_0$. Then
$$
\Gamma(X, \mathcal{F}) = \colim_{i \geq 0} \Gamma(X_i, \mathcal{F}_i)
$$
\end{lemma}

\begin{proof}
Choose a surjective \'etale morphism $U_0 \to X_0$ where $U_0$ is an affine
scheme (Properties of Spaces, Lemma
\ref{spaces-properties-lemma-quasi-compact-affine-cover}).
Set $U_i = X_i \times_{X_0} U_0$.
Set $R_0 = U_0 \times_{X_0} U_0$ and $R_i = R_0 \times_{X_0} X_i$.
In the proof of Lemma \ref{lemma-directed-inverse-system-has-limit} we have
seen that there exists a presentation $X = U/R$ with
$U = \lim U_i$ and $R = \lim R_i$.
Note that $U_i$ and $U$ are affine and that $R_i$ and $R$ are
quasi-compact and separated (as $X_i$ is quasi-separated). Hence
Limits, Lemma \ref{lemma-descend-section}
implies that
$$
\mathcal{F}(U) = \colim \mathcal{F}_i(U_i)
\quad\text{and}\quad
\mathcal{F}(R) = \colim \mathcal{F}_i(R_i).
$$
The lemma follows as
$\Gamma(X, \mathcal{F}) = \text{Ker}(\mathcal{F}(U) \to \mathcal{F}(R))$
and similarly
$\Gamma(X_i, \mathcal{F}_i) =
\text{Ker}(\mathcal{F}_i(U_i) \to \mathcal{F}_i(R_i))$
\end{proof}





\section{Other chapters}

\begin{multicols}{2}
\begin{enumerate}
\item \hyperref[introduction-section-phantom]{Introduction}
\item \hyperref[conventions-section-phantom]{Conventions}
\item \hyperref[sets-section-phantom]{Set Theory}
\item \hyperref[categories-section-phantom]{Categories}
\item \hyperref[topology-section-phantom]{Topology}
\item \hyperref[sheaves-section-phantom]{Sheaves on Spaces}
\item \hyperref[algebra-section-phantom]{Commutative Algebra}
\item \hyperref[sites-section-phantom]{Sites and Sheaves}
\item \hyperref[homology-section-phantom]{Homological Algebra}
\item \hyperref[derived-section-phantom]{Derived Categories}
\item \hyperref[more-algebra-section-phantom]{More Algebra}
\item \hyperref[simplicial-section-phantom]{Simplicial Methods}
\item \hyperref[modules-section-phantom]{Sheaves of Modules}
\item \hyperref[sites-modules-section-phantom]{Modules on Sites}
\item \hyperref[injectives-section-phantom]{Injectives}
\item \hyperref[cohomology-section-phantom]{Cohomology of Sheaves}
\item \hyperref[sites-cohomology-section-phantom]{Cohomology on Sites}
\item \hyperref[hypercovering-section-phantom]{Hypercoverings}
\item \hyperref[schemes-section-phantom]{Schemes}
\item \hyperref[constructions-section-phantom]{Constructions of Schemes}
\item \hyperref[properties-section-phantom]{Properties of Schemes}
\item \hyperref[morphisms-section-phantom]{Morphisms of Schemes}
\item \hyperref[coherent-section-phantom]{Coherent Cohomology}
\item \hyperref[divisors-section-phantom]{Divisors}
\item \hyperref[limits-section-phantom]{Limits of Schemes}
\item \hyperref[varieties-section-phantom]{Varieties}
\item \hyperref[chow-section-phantom]{Chow Homology}
\item \hyperref[topologies-section-phantom]{Topologies on Schemes}
\item \hyperref[descent-section-phantom]{Descent}
\item \hyperref[more-morphisms-section-phantom]{More on Morphisms}
\item \hyperref[flat-section-phantom]{More on Flatness}
\item \hyperref[groupoids-section-phantom]{Groupoid Schemes}
\item \hyperref[more-groupoids-section-phantom]{More on Groupoid Schemes}
\item \hyperref[etale-section-phantom]{\'Etale Morphisms of Schemes}
\item \hyperref[etale-cohomology-section-phantom]{\'Etale Cohomology}
\item \hyperref[spaces-section-phantom]{Algebraic Spaces}
\item \hyperref[spaces-properties-section-phantom]{Properties of Algebraic Spaces}
\item \hyperref[spaces-morphisms-section-phantom]{Morphisms of Algebraic Spaces}
\item \hyperref[spaces-topologies-section-phantom]{Topologies on Algebraic Spaces}
\item \hyperref[spaces-descent-section-phantom]{Descent and Algebraic Spaces}
\item \hyperref[spaces-more-morphisms-section-phantom]{More on Morphisms of Spaces}
\item \hyperref[quot-section-phantom]{Quot and Hilbert Spaces}
\item \hyperref[stacks-section-phantom]{Stacks}
\item \hyperref[spaces-groupoids-section-phantom]{Groupoids in Algebraic Spaces}
\item \hyperref[spaces-more-groupoids-section-phantom]{More on Groupoids in Spaces}
\item \hyperref[bootstrap-section-phantom]{Bootstrap}
\item \hyperref[examples-stacks-section-phantom]{Examples of Stacks}
\item \hyperref[groupoids-quotients-section-phantom]{Quotients of Groupoids}
\item \hyperref[algebraic-section-phantom]{Algebraic Stacks}
\item \hyperref[criteria-section-phantom]{Criteria for Representability}
\item \hyperref[stacks-properties-section-phantom]{Properties of Algebraic Stacks}
\item \hyperref[stacks-morphisms-section-phantom]{Morphisms of Algebraic Stacks}
\item \hyperref[examples-section-phantom]{Examples}
\item \hyperref[exercises-section-phantom]{Exercises}
\item \hyperref[guide-section-phantom]{Guide to Literature}
\item \hyperref[desirables-section-phantom]{Desirables}
\item \hyperref[coding-section-phantom]{Coding Style}
\item \hyperref[fdl-section-phantom]{GNU Free Documentation License}
\item \hyperref[index-section-phantom]{Auto Generated Index}
\end{enumerate}
\end{multicols}


\bibliography{my}
\bibliographystyle{amsalpha}

\end{document}

