\IfFileExists{stacks-project.cls}{%
\documentclass{stacks-project}
}{%
\documentclass{amsart}
}

% The following AMS packages are automatically loaded with
% the amsart documentclass:
%\usepackage{amsmath}
%\usepackage{amssymb}
%\usepackage{amsthm}

% For dealing with references we use the comment environment
\usepackage{verbatim}
\newenvironment{reference}{\comment}{\endcomment}
%\newenvironment{reference}{}{}
\newenvironment{slogan}{\comment}{\endcomment}
\newenvironment{history}{\comment}{\endcomment}

% For commutative diagrams you can use
% \usepackage{amscd}
\usepackage[all]{xy}

% We use 2cell for 2-commutative diagrams.
\xyoption{2cell}
\UseAllTwocells

% To put source file link in headers.
% Change "template.tex" to "this_filename.tex"
% \usepackage{fancyhdr}
% \pagestyle{fancy}
% \lhead{}
% \chead{}
% \rhead{Source file: \url{template.tex}}
% \lfoot{}
% \cfoot{\thepage}
% \rfoot{}
% \renewcommand{\headrulewidth}{0pt}
% \renewcommand{\footrulewidth}{0pt}
% \renewcommand{\headheight}{12pt}

\usepackage{multicol}

% For cross-file-references
\usepackage{xr-hyper}

% Package for hypertext links:
\usepackage{hyperref}

% For any local file, say "hello.tex" you want to link to please
% use \externaldocument[hello-]{hello}
\externaldocument[introduction-]{introduction}
\externaldocument[conventions-]{conventions}
\externaldocument[sets-]{sets}
\externaldocument[categories-]{categories}
\externaldocument[topology-]{topology}
\externaldocument[sheaves-]{sheaves}
\externaldocument[sites-]{sites}
\externaldocument[stacks-]{stacks}
\externaldocument[fields-]{fields}
\externaldocument[algebra-]{algebra}
\externaldocument[brauer-]{brauer}
\externaldocument[homology-]{homology}
\externaldocument[derived-]{derived}
\externaldocument[simplicial-]{simplicial}
\externaldocument[more-algebra-]{more-algebra}
\externaldocument[smoothing-]{smoothing}
\externaldocument[modules-]{modules}
\externaldocument[sites-modules-]{sites-modules}
\externaldocument[injectives-]{injectives}
\externaldocument[cohomology-]{cohomology}
\externaldocument[sites-cohomology-]{sites-cohomology}
\externaldocument[dga-]{dga}
\externaldocument[dpa-]{dpa}
\externaldocument[hypercovering-]{hypercovering}
\externaldocument[schemes-]{schemes}
\externaldocument[constructions-]{constructions}
\externaldocument[properties-]{properties}
\externaldocument[morphisms-]{morphisms}
\externaldocument[coherent-]{coherent}
\externaldocument[divisors-]{divisors}
\externaldocument[limits-]{limits}
\externaldocument[varieties-]{varieties}
\externaldocument[topologies-]{topologies}
\externaldocument[descent-]{descent}
\externaldocument[perfect-]{perfect}
\externaldocument[more-morphisms-]{more-morphisms}
\externaldocument[flat-]{flat}
\externaldocument[groupoids-]{groupoids}
\externaldocument[more-groupoids-]{more-groupoids}
\externaldocument[etale-]{etale}
\externaldocument[chow-]{chow}
\externaldocument[intersection-]{intersection}
\externaldocument[pic-]{pic}
\externaldocument[adequate-]{adequate}
\externaldocument[dualizing-]{dualizing}
\externaldocument[duality-]{duality}
\externaldocument[discriminant-]{discriminant}
\externaldocument[local-cohomology-]{local-cohomology}
\externaldocument[curves-]{curves}
\externaldocument[resolve-]{resolve}
\externaldocument[models-]{models}
\externaldocument[pione-]{pione}
\externaldocument[etale-cohomology-]{etale-cohomology}
\externaldocument[proetale-]{proetale}
\externaldocument[crystalline-]{crystalline}
\externaldocument[spaces-]{spaces}
\externaldocument[spaces-properties-]{spaces-properties}
\externaldocument[spaces-morphisms-]{spaces-morphisms}
\externaldocument[decent-spaces-]{decent-spaces}
\externaldocument[spaces-cohomology-]{spaces-cohomology}
\externaldocument[spaces-limits-]{spaces-limits}
\externaldocument[spaces-divisors-]{spaces-divisors}
\externaldocument[spaces-over-fields-]{spaces-over-fields}
\externaldocument[spaces-topologies-]{spaces-topologies}
\externaldocument[spaces-descent-]{spaces-descent}
\externaldocument[spaces-perfect-]{spaces-perfect}
\externaldocument[spaces-more-morphisms-]{spaces-more-morphisms}
\externaldocument[spaces-flat-]{spaces-flat}
\externaldocument[spaces-groupoids-]{spaces-groupoids}
\externaldocument[spaces-more-groupoids-]{spaces-more-groupoids}
\externaldocument[bootstrap-]{bootstrap}
\externaldocument[spaces-pushouts-]{spaces-pushouts}
\externaldocument[groupoids-quotients-]{groupoids-quotients}
\externaldocument[spaces-more-cohomology-]{spaces-more-cohomology}
\externaldocument[spaces-simplicial-]{spaces-simplicial}
\externaldocument[formal-spaces-]{formal-spaces}
\externaldocument[restricted-]{restricted}
\externaldocument[spaces-resolve-]{spaces-resolve}
\externaldocument[formal-defos-]{formal-defos}
\externaldocument[defos-]{defos}
\externaldocument[cotangent-]{cotangent}
\externaldocument[examples-defos-]{examples-defos}
\externaldocument[algebraic-]{algebraic}
\externaldocument[examples-stacks-]{examples-stacks}
\externaldocument[stacks-sheaves-]{stacks-sheaves}
\externaldocument[criteria-]{criteria}
\externaldocument[artin-]{artin}
\externaldocument[quot-]{quot}
\externaldocument[stacks-properties-]{stacks-properties}
\externaldocument[stacks-morphisms-]{stacks-morphisms}
\externaldocument[stacks-limits-]{stacks-limits}
\externaldocument[stacks-cohomology-]{stacks-cohomology}
\externaldocument[stacks-perfect-]{stacks-perfect}
\externaldocument[stacks-introduction-]{stacks-introduction}
\externaldocument[stacks-more-morphisms-]{stacks-more-morphisms}
\externaldocument[stacks-geometry-]{stacks-geometry}
\externaldocument[moduli-]{moduli}
\externaldocument[moduli-curves-]{moduli-curves}
\externaldocument[examples-]{examples}
\externaldocument[exercises-]{exercises}
\externaldocument[guide-]{guide}
\externaldocument[desirables-]{desirables}
\externaldocument[coding-]{coding}
\externaldocument[obsolete-]{obsolete}
\externaldocument[fdl-]{fdl}
\externaldocument[index-]{index}

% Theorem environments.
%
\theoremstyle{plain}
\newtheorem{theorem}[subsection]{Theorem}
\newtheorem{proposition}[subsection]{Proposition}
\newtheorem{lemma}[subsection]{Lemma}

\theoremstyle{definition}
\newtheorem{definition}[subsection]{Definition}
\newtheorem{example}[subsection]{Example}
\newtheorem{exercise}[subsection]{Exercise}
\newtheorem{situation}[subsection]{Situation}

\theoremstyle{remark}
\newtheorem{remark}[subsection]{Remark}
\newtheorem{remarks}[subsection]{Remarks}

\numberwithin{equation}{subsection}

% Macros
%
\def\lim{\mathop{\rm lim}\nolimits}
\def\colim{\mathop{\rm colim}\nolimits}
\def\Spec{\mathop{\rm Spec}}
\def\Hom{\mathop{\rm Hom}\nolimits}
\def\Ext{\mathop{\rm Ext}\nolimits}
\def\SheafHom{\mathop{\mathcal{H}\!{\it om}}\nolimits}
\def\SheafExt{\mathop{\mathcal{E}\!{\it xt}}\nolimits}
\def\Sch{\textit{Sch}}
\def\Mor{\mathop{\rm Mor}\nolimits}
\def\Ob{\mathop{\rm Ob}\nolimits}
\def\Sh{\mathop{\textit{Sh}}\nolimits}
\def\NL{\mathop{N\!L}\nolimits}
\def\proetale{{pro\text{-}\acute{e}tale}}
\def\etale{{\acute{e}tale}}
\def\QCoh{\textit{QCoh}}
\def\Ker{\mathop{\rm Ker}}
\def\Im{\mathop{\rm Im}}
\def\Coker{\mathop{\rm Coker}}
\def\Coim{\mathop{\rm Coim}}

%
% Macros for moduli stacks/spaces
%
\def\QCohstack{\mathcal{QC}\!{\it oh}}
\def\Cohstack{\mathcal{C}\!{\it oh}}
\def\Spacesstack{\mathcal{S}\!{\it paces}}
\def\Quotfunctor{{\rm Quot}}
\def\Hilbfunctor{{\rm Hilb}}
\def\Curvesstack{\mathcal{C}\!{\it urves}}
\def\Polarizedstack{\mathcal{P}\!{\it olarized}}
\def\Complexesstack{\mathcal{C}\!{\it omplexes}}
% \Pic is the operator that assigns to X its picard group, usage \Pic(X)
% \Picardstack_{X/B} denotes the Picard stack of X over B
% \Picardfunctor_{X/B} denotes the Picard functor of X over B
\def\Pic{\mathop{\rm Pic}\nolimits}
\def\Picardstack{\mathcal{P}\!{\it ic}}
\def\Picardfunctor{{\rm Pic}}
\def\Deformationcategory{\mathcal{D}\!{\it ef}}


% OK, start here.
%
\begin{document}

\title{Fields}


\maketitle

\phantomsection
\label{section-phantom}

\tableofcontents


\section{Introduction}
\label{section-introduction}

\noindent
In this chapter, we shall discuss the theory of fields. Recall that a
{\it field} is a ring in which all nonzero elements are invertible.
Equivalently, the only two ideals of a field are $(0)$ and $(1)$
since any nonzero element is a unit. Consequently fields will be the
simplest cases of much of the theory developed later.

\medskip\noindent
The theory of field extensions has a different feel from standard commutative
algebra since, for instance, any morphism of fields is injective. Nonetheless,
it turns out that questions involving rings can often be reduced to questions
about fields. For instance, any domain can be embedded in a field
(its quotient field), and any {\it local ring} (that is, a ring with a unique
maximal ideal; we have not defined this term yet) has associated to it its
residue field (that is, its quotient by the maximal ideal).
A knowledge of field extensions will thus be useful.




\section{Basic definitions}
\label{section-definitions}

\noindent
Because we have placed this chapter before the chapter discussing
commutative algebra we need to introduce some of the basic definitions
here before we discuss these in greater detail in the algebra chapters.

\begin{definition}
\label{defition-field}
An {\it field} is a nonzero ring where every nonzero element is invertible.
Given a field a {\it subfield} is a subring that is itself a field.
\end{definition}

\noindent
For a field $k$, we write $k^*$ for the subset $k \setminus \{0\}$.
This generalizes the usual notation $R^*$ that refers to the group of
invertible elements in a ring $R$.

\begin{definition}
\label{definition-domain}
A {\it domain} or an {\it integral domain} is a nonzero ring where $0$
is the only zerodivisor.
\end{definition}



\section{Examples of fields}
\label{section-examples}

\noindent
To get started, let us begin by providing several examples of fields. The
reader should recall that if $R$ is a ring and $I \subset R$ an
ideal, then $R/I$ is a field precisely when $I$ is a maximal ideal.

\begin{example}[Rational numbers]
\label{example-rational-numbers}
The rational numbers form a field. It is called the
{\it field of rational numbers} and denoted $\mathbf{Q}$.
\end{example}

\begin{example}[Prime fields]
\label{example-prime-field}
If $p$ is a prime number, then $\mathbf{Z}/(p)$ is a field, denoted
$\mathbf{F}_p$. Indeed, $(p)$ is a
maximal ideal in $\mathbf{Z}$. Thus, fields may be finite: $\mathbf{F}_p$
contains $p$ elements.
\end{example}

\begin{example}
\label{example-quotient-polymial-ring}
In a principal ideal domain, an ideal generated by an irreducible element
is maximal. Now, if $k$ is a field, then the polynomial ring $k[x]$ is a
principal ideal domain. It follows that if $P \in k[x]$ is an irreducible
polynomial (that is, a nonconstant polynomial
that does not admit a factorization into terms of smaller degrees), then
$k[x]/(P)$ is a field. It contains a copy of $k$ in a natural way.
This is a very general way of constructing fields. For instance, the
complex numbers $\mathbf{C}$
can be constructed as $\mathbf{R}[x]/(x^2 + 1)$.
\end{example}

\begin{example}[Quotient fields]
\label{example-quotient-field}
Recall that, given a domain $A$, there is an imbedding $A \to K(A)$ into a
field $K(A)$ constructed from $A$ in exactly the same manner that
$\mathbf{Q}$ is constructed from $\mathbf{Z}$. Formally the elements
of $K(A)$ are (equivalence classes of) fractions $a/b$,
$a, b \in A$, $b \not = 0$. As usual $a/b = a'/b'$ if and only if $ab' = ba'$.
This is called the {\it quotient field} or {\it field of fractions}.
The quotient field has the following universal property: given an
injective ring map $\varphi : A \to K$ to a field $K$, there is a unique
map $\psi: K(A) \to K$ making
$$
\xymatrix{
K(A) \ar[r]_\psi & K \\
A \ar[u] \ar[ru]_\varphi
}
$$
commute. Indeed, it is clear how to define such a map: we set
$\psi(a/b) = \varphi(a)\varphi(b)^{-1}$ where injectivity of $\varphi$
assures that $\varphi(b) \not = 0$ if $ b \not = 0$.
\end{example}

\begin{example}[Field of rational functions]
\label{example-field-of-rational-functions}
If $k$ is a field, then we can consider the field $k(x)$ of rational
functions over $k$. This is the quotient field of the polynomial ring
$k[x]$. In other words, it is the set of quotients $F/G$ for
$F, G \in k[x]$, $G \not = 0$ with the obvious equivalence relation.
\end{example}

\begin{example}
\label{example-field-of-meromorphic-functions}
Let $X$ be a Riemann surface. Let $\mathbf{C}(X)$ denote the
set of meromorphic functions on $X$. Then $\mathbf{C}(X)$ is a ring under
multiplication and addition of functions. It turns out that in fact
$\mathbf{C}(X)$ is a field. Namely, if a nonzero function $f(z)$ is
meromorphic, so is $1/f(z)$. For example, let $S^2$ be the Riemann
sphere; then we know from complex analysis that the ring of meromorphic
functions $\mathbf{C}(S^2)$ is the field of rational functions $\mathbf{C}(z)$.
\end{example}



\section{Vector spaces}
\label{section-vector-spaces}

\noindent
One reason fields are so nice is that the theory of modules over fields
(i.e. vector spaces), is very simple.

\begin{lemma}
\label{lemma-vector-space-is-free}
If $k$ is a field, then every $k$-module is free.
\end{lemma}

\begin{proof}
Indeed, by linear algebra we know that a $k$-module (i.e. vector space)
$V$ has a {\it basis} $\mathcal{B} \subset V$, which defines an isomorphism
from the free vector space on $\mathcal{B}$ to $V$.
\end{proof}

\begin{lemma}
\label{lemma-field-semi-simple}
Every exact sequence of modules over a field splits.
\end{lemma}

\begin{proof}
This follows from Lemma \ref{lemma-vector-space-is-free} as every vector
space is a projective module.
\end{proof}

\noindent
This is another reason why much of the theory in future chapters will not say
very much about fields, since modules behave in such a simple manner.
Note that Lemma \ref{lemma-field-semi-simple} is a statement about the
{\it category} of $k$-modules (for $k$ a field), because the notion of
exactness is inherently arrow-theoretic, i.e., makes use of purely categorical
notions, and can in fact be phrased within a so-called {\it abelian category}.

\medskip\noindent
Henceforth, since the study of modules over a field is linear algebra, and
since the ideal theory of fields is not very interesting, we shall study what
this chapter is really about: {\it extensions} of fields.


\section{The characteristic of a field}
\label{section-more-fields}

\noindent
In the category of rings, there is an {\it initial object} $\mathbf{Z}$: any
ring $R$ has a map from $\mathbf{Z}$ into it in precisely one way. For fields,
there is no such initial object.
Nonetheless, there is a family of objects such that every field can be mapped
into in exactly one way by exactly one of them, and in no way by the others.

\medskip\noindent
Let $F$ be a field. Think of $F$ as a ring to get a ring map
$f : \mathbf{Z} \to F$. The image of this ring map is a domain
(as a subring of a field) hence the kernel of $f$ is a prime ideal
in $\mathbf{Z}$. Hence the kernel of $f$ is either $(0)$ or $(p)$ for
some prime number $p$.

\medskip\noindent
In the first case we see that $f$ is injective, and in this case
we think of $\mathbf{Z}$ as a subring of $F$. Moreover, since every
nonzero element of $F$ is invertible we see that it makes sense to
talk about $p/q \in F$ for $p, q \in \mathbf{Z}$ with $q \not = 0$.
Hence in this case we may and we do think of $\mathbf{Q}$ as a subring of $F$.
One can easily see that this is the smallest subfield of $F$ in this case.

\medskip\noindent
In the second case, i.e., when $\text{Ker}(f) = (p)$ we see that
$\mathbf{Z}/(p) = \mathbf{F}_p$ is a subring of $F$. Clearly it is the
smallest subfield of $F$.

\medskip\noindent
Arguing in this way we see that every field contains a smallest subfield
which is either $\mathbf{Q}$ or finite equal to $\mathbf{F}_p$ for some
prime number $p$.

\begin{definition}
\label{definition-characteristic}
The {\it characteristic} of a field $F$ is $0$ if
$\mathbf{Z} \subset F$, or is a prime $p$ if $p = 0$ in $F$.
The {\it prime subfield of $F$} is the smallest subfield of $F$
which is either $\mathbf{Q} \subset F$ if the characteristic is zero, or
$\mathbf{F}_p \subset F$ if the characteristic is $p > 0$.
\end{definition}

\noindent
It is easy to see that if $E \subset F$ is a subfield, then the
characteristic of $E$ is the same as the characteristic of $F$.

\begin{example}
\label{example-characteristic}
The characteristic of $\mathbf{F}_p$ is $p$, and that of $\mathbf{Q}$ is $0$.
\end{example}


\section{Field extensions}
\label{section-extensions}

\noindent
In general, though, we are interested not so much in fields by themselves but
in field {\it extensions}. This is perhaps analogous to studying not rings
but {\it algebras} over a fixed ring.
The nice thing for fields is that the notion of a ``field over another field''
just recovers the notion of a field extension, by the next result.

\begin{proposition}
\label{lemma-field-maps-injective}
If $F$ is a field and $R$ is a nonzero ring, then any ring homomorphism
$\varphi : F \to R$ is either injective.
\end{proposition}

\begin{proof}
Indeed, let $a \in \text{Ker}(\varphi)$ be a nonzero element. Then we have
$\varphi(1) = \varphi(a^{-1} a) = \varphi(a^{-1}) \varphi(a) = 0$.
Thus $1 = \varphi(1) = 0$ and $R$ is the zero ring.
\end{proof}

\begin{definition}
\label{definition-extension}
If $F$ is a field contained in a field $G$, then $G$ is said
to be a {\it field extension} of $F$. We shall write $G/F$ to indicate
that $G$ is an extension of $F$.
\end{definition}

\noindent
So if $F, F'$ are fields, and $F \to F'$ is any ring-homomorphism, we see by
Lemma \ref{lemma-field-maps-injective} that it is injective, and $F'$ can be
regarded as an extension of $F$, by a slight abuse of language. Alternatively,
a field extension of $F$ is just an $F$-algebra that happens to be a field.
This is completely different than the situation for general rings, since a
ring homomorphism is not necessarily injective.

\medskip\noindent
Let $k$ be a field. There is a {\it category} of field extensions of $k$.
An object of this category is an extension $E/k$, that is a
(necessarily injective) morphism of fields
$$
k \to E,
$$
while a morphism between extensions $E/k$ and $E'/k$ is a $k$-algebra
morphism $E \to E'$; alternatively, it is a commutative diagram
$$
\xymatrix{
E \ar[rr] & & E' \\
& k \ar[ru] \ar[lu] &
}
$$
The set of morphisms from $E \to E'$ in the category of extensions of $k$
will be denoted by $\Mor_k(E, E')$.

\begin{definition}
\label{definition-tower}
A {\it tower} of fields $E_n/E_{n - 1}/\ldots/E_0$ consists of a sequence of
extensions of fields
$E_n/E_{n - 1}$, $E_{n - 1}/E_{n - 2}$, $\ldots$, $E_1/E_0$.
\end{definition}

\noindent
Let us give a few examples of field extensions.

\begin{example}
\label{example-monogenic-extension}
Let $k$ be a field, and $P \in k[x]$ an irreducible polynomial. We have
seen that $k[x]/(P)$ is a field (Example \ref{example-quotient-polymial-ring}).
Since it is also a $k$-algebra in the obvious way, it is an extension of $k$.
\end{example}

\begin{example}
\label{example-field-of-meromorphic-functions-extension-C}
If $X$ is a Riemann surface, then the field of meromorphic functions
$\mathbf{C}(X)$ (Example \ref{example-field-of-meromorphic-functions})
is an extension field of $\mathbf{C}$, because any element of $\mathbf{C}$
induces a meromorphic --- indeed, holomorphic --- constant function on $X$.
\end{example}

\noindent
Let $F/k$ be a field extension. Let $S \subset F$ be any subset.
Then there is a {\it smallest} subextension of $F$ (that is, a subfield of
$F$ containing $k$) that contains $S$. To see this, consider the family of
subfields of $F $ containing $S$ and $k$, and take their intersection; one
checks that this is a field. By a standard argument one shows, in fact, that
this is the set of elements of $F$ that can be obtained via a finite number
of elementary algebraic operations (addition, multiplication, subtraction,
and division) involving elements of $k$ and $S$.

\begin{definition}
\label{definition-generated-by}
If $F/k$ is an extension and $S \subset F$, we write $k(S)$ for the smallest
subextension of $F$ containing $S$. We will say that $S$ {\it generates} the
extension $k(S)/k$.
\end{definition}

\noindent
For instance, $\mathbf{C}$ is generated by $i$ over $\mathbf{R}$.

\begin{exercise}
\label{exercise-C-not-countably-generated}
Show that $\mathbf{C}$ does not have a countable set of generators over
$\mathbf{Q}$.
\end{exercise}

\noindent
Let us now classify extensions generated by one element.

\begin{lemma}[Classification of simple extensions]
\label{lemma-field-extension-generated-by-one-element}
If a field extension $F/k$ is generated by one element, then it is $F$ is
$k$-isomorphic either to the rational function field $k(t)/k$ or to one
of the extensions $k[t]/(P)$ for $P \in k[t]$ irreducible.
\end{lemma}

\noindent
We will see that many of the most important cases of field extensions are
generated by one element, so this is actually useful.

\begin{proof}
Let $\alpha \in F$ be such that $F = k(\alpha)$; by assumption, such an
$\alpha$ exists. There is a morphism of rings
$$
k[t] \to F
$$
sending the indeterminate $t$ to $\alpha$. The image is a domain, so the
kernel is a prime ideal. Thus, it is either $(0)$ or $(P)$ for $P \in k[t]$
irreducible.

\medskip\noindent
If the kernel is $(P)$ for $P \in k[t]$ irreducible, then the map factors
through $k[t]/(P)$, and induces a morphism of fields $k[t]/(P) \to F$. Since
the image contains $\alpha$, we see easily that the map is surjective, hence
an isomorphism. In this case, $k[t]/(P) \simeq F$.

\medskip\noindent
If the kernel is trivial, then we have an injection $k[t] \to F$.
One may thus define a morphism of the quotient field $k(t)$ into $F$; given a
quotient $R(t)/Q(t)$ with $R(t), Q(t) \in k[t]$, we map this to
$R(\alpha)/Q(\alpha)$. The hypothesis that $k[t] \to F$ is injective implies
that $Q(\alpha) \neq 0$ unless $Q$ is the zero polynomial.
The quotient field of $k[t]$ is the rational function field $k(t)$, so we get
a morphism $k(t) \to F$
whose image contains $\alpha$. It is thus surjective, hence an isomorphism.
\end{proof}




\section{Finite extensions}
\label{section-finite-extensions}

\noindent
If $F/E$ is a field extension, then evidently $F$ is also a vector space
over $E$ (the scalar action is just multiplication in $F$).

\begin{definition}
\label{definition-degree}
Let $F/E$ be an extension of fields. The dimension of $F$ considered as an
$E$-vector space is called the {\it degree} of the extension and is
denoted $[F : E]$. If $[F : E]<\infty$ then $F$ is said to be a
{\it finite} extension of $E$.
\end{definition}

\begin{example}
\label{example-C-over-R}
The field $\mathbf{C}$ is a two dimensional vector space over $\mathbf{R}$
with basis $1, i$. Thus $\mathbf{C}$ is a finite extension of $\mathbf{R}$
of degree 2.
\end{example}

\noindent
Let us now consider the degree in the most important special example, that
given by Lemma \ref{lemma-field-extension-generated-by-one-element}, in the
next two examples.

\begin{example}[Degree of a rational function field]
\label{example-degree-rational-function-field}
If $k$ is any field, then the rational function field $k(t)$ is
{\it not} a finite extension. For example the elements
$\left\{t^n, n \in \mathbf{Z}\right\}$ are linearly independent over $k$.

\medskip\noindent
In fact, if $k$ is uncountable, then $k(t)$ is {\it uncountably} dimensional
as a $k$-vector space. To show this, we claim that the family of elements
$\{1/(t- \alpha), \alpha \in k\} \subset k(t)$ is linearly independent over
$k$. A nontrivial relation between them would lead to a contradiction: for
instance, if one works over $\mathbf{C}$, then this follows because
$\frac{1}{t-\alpha}$, when considered as a meromorphic function on
$\mathbf{C}$, has a pole at $\alpha$ and nowhere else.
Consequently any sum $\sum c_i \frac{1}{t - \alpha_i}$ for the $c_i \in k^*$,
and $\alpha_i \in k$ distinct, would have poles at each of the $\alpha_i$.
In particular, it could not be zero.

\medskip\noindent
Amusingly, this leads to a quick proof of the Hilbert Nullstellensatz over
the complex numbers. For a slightly more general result, see
Algebra, Theorem \ref{algebra-theorem-uncountable-nullstellensatz}. 
\end{example}

\begin{example}[Degree of a simple algebraic extension]
\label{example-degree-simple-algebraic-extension}
Consider a monogenic field extension $E/k$ of the form discussed in
Example \ref{example-field-of-rational-functions}.
In other words, $E = k[t]/(P)$ for $P \in k[t]$ an irreducible polynomial.
Then the degree $[E : k]$ is just the degree the degree $d = \deg(P)$ of the
polynomial $P$. Indeed, say
\begin{equation}
\label{equation-P}
P = a_d t^d + a_1 t^{d - 1} + \dots + a_0.
\end{equation}
with $a_d \not = 0$. Then the images of $1, t, \dots, t^{d - 1}$ in
$k[t]/(P)$ are linearly independent over $k$, because any relation involving
them would have degree strictly smaller than that of $P$, and $P$ is the
element of smallest degree in the ideal $(P)$.

\medskip\noindent
Conversely, the set $S = \{1, t, \dots, t^{d - 1}\}$ (or more
properly their images) spans $k[t]/(P)$ as a vector space.
Indeed, we have by (\ref{equation-P}) that $a_d t^d$ lies in the span of $S$.
Since $a_d$ is invertible, we see that $t^d$ is in the span of $S$.
Similarly, the relation $t P(t) = 0$ shows that the image of $t^{d + 1}$
lies in the span of $\{1, t, \dots, t^d\}$ --- by what was just shown, thus
in the span of $S$. Working upward inductively, we find
that the image of $t^n$ for $n \geq d$ lies in the span of $S$.
\end{example}

\noindent
This confirms the observation that $[\mathbf{C}: \mathbf{R}] = 2$, for
instance. More generally, if $k$ is a field, and $\alpha \in k$ is not a
square, then the irreducible polynomial $x^2 - \alpha \in k[x]$ allows one
to construct an extension $k[x]/(x^2 - \alpha)$ of degree two.
We shall write this as $k(\sqrt{\alpha})$. Such extensions will be called
{\it quadratic,} for obvious reasons.

\medskip\noindent
The basic fact about the degree is that it is {\it multiplicative in towers.}

\begin{lemma}[Multiplicativity]
\label{lemma-multiplicativity-degrees}
Suppose given a tower of fields $F/E/k$. Then
$$
[F:k] = [F:E][E:k]
$$
\end{lemma}

\begin{proof}
Let $\alpha_1, \dots, \alpha_n \in F$ be an $E$-basis for $F$. Let
$\beta_1, \dots, \beta_m \in E$ be a $k$-basis for $E$. Then the claim is
that the set of products
$\{\alpha_i \beta_j, 1 \leq i \leq n, 1 \leq j \leq m\}$
is a $k$-basis for $F$. Indeed, let us check first that they span $F$ over $k$.

\medskip\noindent
By assumption, the $\{\alpha_i\}$ span $F$ over $E$. So if
$f \in F$, there are $a_i \in E$ with
$$
f = \sum\nolimits_i a_i \alpha_i,
$$
and, for each $i$, we can write $a_i = \sum b_{ij} \beta_j$ for some
$b_{ij} \in k$. Putting these together, we find
$$
f = \sum\nolimits_{i,j} b_{ij} \alpha_i \beta_j,
$$
proving that the $\{\alpha_i \beta_j\}$ span $F$ over $k$.

\medskip\noindent
Suppose now that there existed a nontrivial relation
$$
\sum\nolimits_{i,j} c_{ij} \alpha_i \beta_j = 0
$$
for the $c_{ij} \in k$. In that case, we would have
$$
\sum\nolimits_i \alpha_i \left( \sum\nolimits_j c_{ij} \beta_j \right) = 0,
$$
and the inner terms lie in $E$ as the $\beta_j$ do. Now $E$-linear
independence of the $\{\alpha_i\}$ shows that the inner sums are all zero.
Then $k$-linear independence of the $\{\beta_j\}$ shows that the
$c_{ij}$ all vanish.
\end{proof}

\noindent
We sidetrack to a slightly tangential definition.

\begin{definition}
\label{definition-number-field}
A field $K$ is said to be a {\it number field} if it has characteristic
$0$ and the extension $\mathbf{Q} \subset K$ is finite.
\end{definition}

\noindent
Number fields are the basic objects in algebraic number theory. We shall see
later that,
for the analog of the integers $\mathbf{Z}$ in a number field, something kind
of like unique factorization still holds (though strict unique factorization
generally does not!).


\section{Algebraic extensions}
\label{section-algebraic-extensions}

\noindent
An important class of extensions are those where every element generates
a finite extension.

\begin{definition}
\label{definition-algebraic}
Consider a field extension $F/E$. An element $\alpha \in F$ is said to be
{\it algebraic} over $E$ if $\alpha$ is the root of some nonzero polynomial
with coefficients in $E$. If all elements of $F$ are algebraic then $F$ is
said to be an {\it algebraic extension} of $E$.
\end{definition}

\noindent
By Lemma \ref{lemma-field-extension-generated-by-one-element}, the
subextension $E(\alpha)$ is isomorphic either to the rational function
field $E(t)$ or to a quotient ring $E[t]/(P)$ for $P \in E[t]$ an
irreducible polynomial. In the latter case, $\alpha$ is algebraic over
$E$ (in fact, the proof of
Lemma \ref{lemma-field-extension-generated-by-one-element}
shows that we can pick $P$ such that $\alpha$ is a root of $P$);
in the former case, it is not.

\begin{example}
\label{example-C-algebraic-over-R}
The field $\mathbf{C}$ is algebraic over $\mathbf{R}$. Namely, if
$\alpha = a + ib$ in $\mathbf{C}$, then $\alpha^2 - 2a\alpha + a^2 + b^2 = 0$
is a polynomial equation for $\alpha$ over $\mathbf{R}$.
\end{example}

\begin{example}
\label{example-compact-riemann-surface-is-finite-over-P1}
Let $X$ be a compact Riemann surface, and let
$f \in \mathbf{C}(X) - \mathbf{C}$ any nonconstant meromorphic function
on $X$ (see \ref{example-field-of-meromorphic-functions}). Then it is
known that $\mathbf{C}(X)$ is algebraic over the subextension
$\mathbf{C}(f)$ generated by $f$. We shall not prove this.
\end{example}

\begin{lemma}
\label{lemma-algebraic-goes-up}
Let $K/E/F$ be a tower of field extensions.
\begin{enumerate}
\item If $\alpha \in K$ is algebraic over $F$, then $\alpha$ is algebraic
over $E$.
\item if $K$ is algebraic over $F$, then $K$ is algebraic over $E$.
\end{enumerate}
\end{lemma}

\begin{proof}
This is immediate from the definitions.
\end{proof}

\noindent
We now show that there is a deep connection between finiteness and being
algebraic.

\begin{lemma}
\label{lemma-finite-is-algebraic}
A finite extension is algebraic. In fact, an extension $E/k$ is algebraic
if and only if every subextension $k(\alpha)/k$ generated by some
$\alpha \in E$ is finite.
\end{lemma}

\noindent
In general, it is very false that an algebraic extension is finite.

\begin{proof}
Let $E/k$ be finite, say of degree $n$. Choose $\alpha \in E$. Then the
elements $\{1, \alpha, \dots, \alpha^n\}$ are linearly
dependent over $E$, or we would necessarily have $[E : k] > n$. A relation of
linear dependence now gives the desired polynomial that $\alpha$ must satisfy.

\medskip\noindent
For the last assertion, note that a monogenic extension $k(\alpha)/k$ is
finite if and only $\alpha$ is algebraic over $k$, by
Examples \ref{example-degree-rational-function-field} and
\ref{example-degree-simple-algebraic-extension}.
So if $E/k$ is algebraic, then each $k(\alpha)/k$, $\alpha \in E$, is a finite
extension, and conversely.
\end{proof}

\noindent
We can extract a lemma of the last proof (really of
Examples \ref{example-degree-rational-function-field} and
\ref{example-degree-simple-algebraic-extension}):
a monogenic extension is finite if and only if it is algebraic.
We shall use this observation in the next result.

\begin{lemma}
\label{lemma-algebraic-finitely-generated}
Let $k$ be a field, and let $\alpha_1, \alpha_2, \dots, \alpha_n$ be elements
of some extension field such that each $\alpha_i$ is finite over $k$. Then the
extension $k(\alpha_1, \dots, \alpha_n)/k$ is finite.
That is, a finitely generated algebraic extension is finite.
\end{lemma}

\begin{proof}
Indeed, each extension
$k(\alpha_{1}, \dots, \alpha_{i+1})/k(\alpha_1, \dots, \alpha_{i})$
is generated by one element and algebraic, hence finite.
By multiplicativity of degree (Lemma \ref{lemma-multiplicativity-degrees})
we obtain the result.
\end{proof}

\noindent
The set of complex numbers that are algebraic over $\mathbf{Q}$ are simply
called the {\it algebraic numbers.} For instance, $\sqrt{2}$ is algebraic,
$i$ is algebraic, but $\pi$ is not.
It is a basic fact that the algebraic numbers form a field, although it is not
obvious how to prove this from the definition that a number is algebraic
precisely when it satisfies a nonzero polynomial equation with rational
coefficients (e.g. by polynomial equations).

\begin{lemma}
\label{lemma-algebraic-elements}
Let $E/k$ be a field extension. Then the elements of $E$ algebraic over $k$
form a subextension of $E/k$.
\end{lemma}

\begin{proof}
Let $\alpha, \beta \in E$ be algebraic over $k$. Then $k(\alpha, \beta)/k$
is a finite extension by Lemma \ref{lemma-algebraic-finitely-generated}.
It follows that $k(\alpha + \beta) \subset k(\alpha, \beta)$ is a finite
extension, which implies that $\alpha + \beta$ is algebraic by
Lemma \ref{lemma-finite-is-algebraic}. Similarly for the difference,
product and quotient of $\alpha$ and $\beta$.
\end{proof}

\noindent
Many nice properties of field extensions, like those of rings, will have the
property that they will be preserved by towers and composita.

\begin{lemma}
\label{lemma-algebraic-permanence}
Let $E/k$ and $F/E$ be algebraic extensions of fields. Then $F/k$ is an
algebraic extension of fields.
\end{lemma}

\begin{proof}
Choose $\alpha \in F$. Then $\alpha$ is algebraic over $E$.
The key observation is that $\alpha$ is algebraic over a
finitely generated subextension of $k$.
That is, there is a finite set $S \subset E$ such that $\alpha $ is algebraic
over $k(S)$: this is clear because being algebraic means that a certain
polynomial in $E[x]$ that $\alpha$ satisfies exists, and as $S$ we can take the
coefficients of this polynomial. It follows that $\alpha$ is algebraic over
$k(S)$. In particular, the extension $k(S, \alpha)/ k(S)$ is finite.
Since $S$ is a finite set, and $k(S)/k$ is algebraic,
Lemma \ref{lemma-algebraic-finitely-generated} shows that
$k(S)/k$ is finite. Using multiplicativity
(Lemma \ref{lemma-multiplicativity-degrees})
we find that $k(S,\alpha)/k$ is finite, so $\alpha$ is algebraic over $k$.
\end{proof}

\noindent
The method of proof in the previous argument --- that being algebraic
over $E$ was a property that {\it descended} to a finitely generated
subextension of $E$ --- is an idea that recurs throughout algebra.
It often allows one to reduce general commutative algebra questions
to the Noetherian case for example.

\begin{lemma}
\label{lemma-size-algebraic-extension}
Let $E/F$ be an algebraic extension of fields. Then the cardinality $|E|$
of $E$ is at most $\max(\aleph_0, |F|)$.
\end{lemma}

\begin{proof}
Let $S$ be the set of nonconstant polynomials with coefficients in $F$.
For every $P \in S$ the set of roots
$r(P, E) = \{\alpha \in E \mid P(\alpha) = 0\}$
is finite (details omitted). Moreover, the fact that $E$ is algebraic
over $F$ implies that $E = \bigcup_{P \in S} r(P, E)$.
It is clear that $S$ has cardinality bounded by $\max(\aleph_0, |F|)$
because the cardinality of a finite product of copies of $F$ has
cardinality at most $\max(\aleph_0, |F|)$.
Thus so does $E$.
\end{proof}


\section{Minimal polynomials}
\label{section-minimal-polynomials}

\noindent
Let $E/k$ be a field extension, and let $\alpha \in E$ be algebraic over $k$.
Then $\alpha$ satisfies a (nontrivial) polynomial equation in $k[x]$.
Consider the set of polynomials $P(x) \in k[x]$ such that $P(\alpha) = 0$; by
hypothesis, this set does not just contain the zero polynomial.
It is easy to see that this set is an {\it ideal.} Indeed, it is the kernel
of the map
$$
k[x] \to E, \quad x \mapsto \alpha
$$
Since $k[x]$ is a PID, there is a {\it generator} $m(x) \in k[x]$ of this
ideal. If we assume $m$ monic, without loss of generality, then $m$ is
uniquely determined.

\begin{definition}
\label{definition-minimal-polynomial}
$m(x)$ as above is called the {\it minimal polynomial} of $\alpha$ over $k$.
\end{definition}

\noindent
The minimal polynomial has the following characterization: it is the monic
polynomial, of smallest degree, that annihilates $\alpha$. Any nonconstant
multiple of $m(x)$ will have larger degree, and only multiples of $m(x)$ can
annihilate $\alpha$. This explains the name {\it minimal.}

\medskip\noindent
Clearly the minimal polynomial is {\it irreducible}. This is equivalent to the
assertion that the ideal in $k[x]$ consisting of polynomials annihilating
$\alpha$ is prime. This follows from the fact that the map
$k[x] \to E, x \mapsto \alpha$ is a map into a domain (even a field), so the
kernel is a prime ideal.

\begin{lemma}
\label{lemma-degree-minimal-polynomial}
The degree of the minimal polynomial is $[k(\alpha):k]$.
\end{lemma}

\begin{proof}
This is just a restatement of the argument in
Lemma \ref{lemma-field-extension-generated-by-one-element}: the observation
is that if $m(x)$ is the minimal polynomial of $\alpha$, then the map
$$
k[x]/(m(x)) \to k(\alpha), \quad x \mapsto \alpha
$$
is an isomorphism as in the aforementioned proof, and we have counted the
degree of such an extension (see
Example \ref{example-degree-simple-algebraic-extension}).
\end{proof}

\noindent
So the observation of the above proof is that if $\alpha \in E$ is algebraic,
then $k(\alpha) \subset E$ is isomorphic to $k[x]/(m(x))$.


\section{Algebraic closure}
\label{section-algebraic-closure}

\noindent
The ``fundamental theorem of algebra'' states that $\mathbf{C}$ is
algebraically closed. A beautiful proof of this result uses
Liouville's theorem in complex analysis, we shall give another
proof (see Lemma \ref{lemma-C-algebraically-closed}).

\begin{definition}
\label{definition-algebraically-closed}
A field $F$ is said to be {\it algebraically closed} if every algebraic
extension $E/F$ is trivial, i.e., $E = F$.
\end{definition}

\noindent
This may not be the definition in every text. Here is the lemma comparing
it with the other one.

\begin{lemma}
\label{lemma-algebraically-closed}
Let $F$ be a field. The following are equivalent
\begin{enumerate}
\item $F$ is algebraically closed,
\item every irreducible polynomial over $F$ is linear,
\item every nonconstant polynomial over $F$ has a root,
\item every nonconstant polynomial over $F$ is a product of linear factors.
\end{enumerate}
\end{lemma}

\begin{proof}
If $F$ is algebraically closed, then every irreducible polynomial is linear.
Namely, if there exists an irreducible polynomial of degree $> 1$, then
this generates a nontrivial finite (hence algebraic) field extension, see
Example \ref{example-degree-simple-algebraic-extension}.
Thus (1) implies (2). If every irreducible polynomial
is linear, then every irreducible polynomial has a root, whence every
nonconstant polynomial has a root. Thus (2) implies (3).

\medskip\noindent
Assume every nonconstant polynomial has a root. Let $P \in F[x]$
be nonconstant. If $P(\alpha) = 0$ with $\alpha \in F$, then we see
that $P = (x - \alpha)Q$ for some $Q \in F[x]$ (by division with remainder).
Thus we can argue by induction on the degree that any nonconstant
polynomial can be written as a product $c \prod (x - \alpha_i)$.

\medskip\noindent
Finally, suppose that every nonconstant polynomial over $F$ is a product of
linear factors. Let $E/F$ be an algebraic extension. Then all the simple
subextensions $F(\alpha)/F$ of $E$ are necessarily trivial (because the
only irreducible polynomials are linear by assumption). Thus $E = F$.
We see that (4) implies (1) and we are done.
\end{proof}

\noindent
Now we want to define a ``universal'' algebraic extension of a field.
Actually, we should be careful: the algebraic closure is {\it not} a
universal object. That is, the algebraic closure is not unique up to
{\it unique} isomorphism: it is only unique up to isomorphism. But still,
it will be very handy, if not functorial.

\begin{definition}
\label{definition-algebraic-closure}
Let $F$ be a field. We say $F$ is {\it algebraically closed} if every
algebraic extension $E/F$ is trivial, i.e., $E = F$. An {\it algebraic closure}
of $F$ is a field $\overline{F}$ containing $F$ such that:
\begin{enumerate}
\item $\overline{F}$ is algebraic over $F$.
\item $\overline{F}$ is algebraically closed.
\end{enumerate}
\end{definition}

\noindent
If $F$ is algebraically closed, then $F$ is its own algebraic closure.
We now prove the basic existence result.

\begin{theorem}
\label{theorem-existence-algebraic-closure}
Every field has an algebraic closure.
\end{theorem}

\noindent
The proof will mostly be a red herring to the rest of the chapter. However, we
will want to know that it is {\it possible} to embed a field inside an
algebraically closed field, and we will often assume it done.

\begin{proof}
Let $F$ be a field. By Lemma \ref{lemma-size-algebraic-extension} the
cardinality of an algebraic extension of $F$ is bounded by
$\max(\aleph_0, |F|)$. Choose a set $S$ containing $F$ with
$|S| > \max(\aleph_0, |F|)$. Let's consider triples
$(E, \sigma_E, \mu_E)$ where
\begin{enumerate}
\item $E$ is a set with $F \subset E \subset S$, and
\item $\sigma_E : E \times E \to E$ and $\mu_E : E \times E \to E$
are maps of sets such that $(E, \sigma_E, \mu_E)$ defines the structure
of a field extension of $F$ (in particular $\sigma_E(a, b) = a +_F b$
for $a, b \in F$ and similarly for $\mu_E$), and
\item $F \subset E$ is an algebraic field extension.
\end{enumerate}
The collection of all triples $(E, \sigma_E, \mu_E)$ forms a set $I$.
For $i \in I$ we will denote $E_i = (E_i, \sigma_i, \mu_i)$ the
corresponding field extension fo $F$. We define a partial ordering on
$I$ by declaring $i \leq i'$ if and only if $E_i \subset E_{i'}$
(this makes sense as $E_i$ and $E_{i'}$ are subsets of the same set $S$)
and we have $\sigma_i = \sigma_{i'}|_{E_i \times E_i}$ and
$\mu_i = \mu_{i'}|_{E_i \times E_i}$, in other words, $E_{i'}$ is a field
extension of $E_i$.

\medskip\noindent
Let $T \subset I$ be a totally ordered subset. Then it is clear that
$E_T = \bigcup_{i \in T} E_i$ with induced maps $\sigma_T = \bigcup \sigma_i$
and $\mu_T = \bigcup \mu_i$ is another element of $I$. In other words
every totally order subset of $I$ has a upper bound in $I$. By Zorn's lemma
there exists a maximal element $(E, \sigma_E, \mu_E)$ in $I$. We claim that
$E$ is an algebraic closure. Since by definition of $I$ the extension
$E/F$ is algebraic, it suffices to show that $E$ is algebraically closed.

\medskip\noindent
To see this we argue by contradiction. Namely, suppose that $E$ is not
algebraically closed. Then there exists an irreducible polynomial
$P$ over $E$ of degree $> 1$, see Lemma \ref{lemma-algebraically-closed}.
By Lemma \ref{lemma-finite-is-algebraic} we obtain a nontrivial finite
extension $E' = E[x]/(P)$. Observe that $E'/F$ is algebraic by
Lemma \ref{lemma-algebraic-permanence}.
Thus the cardinality of $E'$ is $\leq \max(\aleph_0, |F|)$.
By elementary set theory we can extend the given injection
$E \subset S$ to an injection $E' \to S$. In other words, we may
think of $E'$ as an element of our set $I$ contradicting the
maximality of $E$. This contradiction completes the proof.
\end{proof}

\begin{lemma}
\label{lemma-map-into-algebraic-closure}
Let $F$ be a field. Let $\overline{F}$ be an algebraic closure of $F$.
Let $M/F$ be an algebraic extension. Then there is a morphism of
$F$-extensions $M \to \overline{F}$.
\end{lemma}

\begin{proof}
Consider the set $I$ of pairs $(E, \varphi)$ where $F \subset E \subset M$
is a subextension and $\varphi : E \to \overline{F}$ is a morphism of
$F$-extensions. We partially order the set $I$ by declaring
$(E, \varphi) \leq (E', \varphi')$ if and only if $E \subset E'$ and
$\varphi'|_E = \varphi$. If $T = \{(E_t,  \varphi_t)\} \subset I$
is a totally ordered subset, then
$\bigcup \varphi_t : \bigcup E_t \to \overline{F}$ is an element of $I$.
Thus every totally ordered subset of $I$ has an upper bound.
By Zorn's lemma there exists a maximal element $(E, \varphi)$ in $I$.
We claim that $E = M$, which will finish the proof. If not, then
pick $\alpha \in M$, $\alpha \not \in E$. The $\alpha$ is algebraic
over $E$, see Lemma \ref{lemma-algebraic-goes-up}.
Let $P$ be the minimal polynomial of $\alpha$ over $E$.
Let $P^\varphi$ be the image of $P$ by $\varphi$ in $\overline{F}[x]$.
Since $\overline{F}$ is algebraically closed there is a root $\beta$
of $P^\varphi$ in $\overline{F}$. Then we can extend $\varphi$ to
$\varphi' : E(\alpha) = E[x]/(P) \to \overline{F}$ by mapping
$x$ to $\beta$. This contradicts the maximality of $(E, \varphi)$
as desired.
\end{proof}

\begin{lemma}
\label{lemma-algebraic-closures-isomorphic}
Any two algebraic closures of a field are isomorphic.
\end{lemma}

\begin{proof}
Let $F$ be a field. If $M$ and $\overline{F}$ are algebraic closures of
$F$, then there exists a morphism of $F$-extensions
$\varphi : M \to \overline{F}$ by
Lemma \ref{lemma-map-into-algebraic-closure}.
Now the image $\varphi(M)$ is algebraically closed.
On the other hand, the extension $\varphi(M) \subset \overline{F}$
is algebraic by Lemma \ref{lemma-algebraic-goes-up}.
Thus $\varphi(M) = \overline{F}$.
\end{proof}

\noindent
Let $K$ be an algebraically closed field. Then the ring $K[x]$ has a very
simple ideal structure as we saw in Lemma \ref{lemma-algebraically-closed}.
In particular, every polynomial $P \in K[x]$ can be written as
$$
P = c(x - \alpha_1) \ldots (x - \alpha_n),
$$
where $c$ is the constant term and the $\alpha_1, \ldots, \alpha_n \in k$
are the roots of $P$ (counted with multiplicity). Clearly, he only irreducible
polynomials in $K[x]$ are the linear polynomials $c(x - \alpha)$,
$c, \alpha \in K$ (and $c \neq 0$).

\medskip\noindent
In particular, two polynomials in $K[x]$ are {\it relatively prime}
(i.e., generate the unit ideal) if and only if they have no common roots.
This follows because the maximal ideals of $K[x]$ are of the form
$(x - \alpha)$, $\alpha \in K$. So if $F, G \in K[x]$ have no common root,
then $(F, G)$ cannot be contained in any $(x - \alpha)$ (as then they would
have a common root at $\alpha$).

\medskip\noindent
If $k$ is {\it not} algebraically closed, then this still gives
information about when two polynomials in $k[x]$ generate the unit ideal.

\begin{definition}
\label{definition-relatively-prime}
If $k$ is any field, we say that two polynomials in $k[x]$ are
{\it relatively prime} if they generate the unit ideal in $k[x]$.
\end{definition}

\begin{lemma}
\label{lemma-relatively-prime-polynomials}
Two polynomials in $k[x]$ are relatively prime precisely when they
have no common roots in an algebraic closure $\overline{k}$ of $k$.
\end{lemma}

\begin{proof}
The claim is that any two polynomials $P, Q$ generate $(1)$ in $k[x]$ if and
only if they generate $(1)$ in $\overline{k}[x]$. This is a piece of
linear algebra: a system of linear equations with coefficients in $k$ has
a solution if and only if it has a solution in any extension of $k$.
Consequently, we can reduce to the case of an algebraically closed field, in
which case the result is clear from what we have already proved.
\end{proof}





\section{Separable extensions}
\label{section-separable-extensions}

\noindent
In characteristic $p$ something funny happens with irreducible polynomials
over fields. We try to explain this succintly in the following lemma.

\begin{lemma}
\label{lemma-irreducible-polynomials}
Let $F$ be a field. Let $P \in F[x]$ be an irreducible polynomial over $F$.
Let $P' = \text{d}P/\text{d}x$ be the derivative of $P$ with respect
to $x$. Then one of the following two cases happens
\begin{enumerate}
\item $P$ and $P'$ are relatively prime, or
\item $P'$ is the zero polynomial.
\end{enumerate}
Then second case can only happen if $F$ has characteristic $p > 0$.
In this case $P(x) = Q(x^q)$ where $q = p^f$ is a power of $p$ and
$Q \in F[x]$ is an irreducible polynomial such that $Q$ and $Q'$
are relatively prime.
\end{lemma}

\begin{proof}
Note that $P'$ has degree $< \deg(P)$. Hence if $P$ and $P'$ are not relatively
prime, then $(P, P') = (R)$ where $R$ is a polynomial of degree $< \deg(P)$
contradicting the irreducibily of $P$. This proves we have the dichotomy
between (1) and (2).

\medskip\noindent
Assume we are in case (2) and $P = a_d x^d + \ldots + a_0$. Then
$P' = da_d x^{d - 1} + \ldots + a_1$. In characteristic $0$ we see
that this forces $a_d, \ldots, a_1 = 0$ which would mean $P$ is constant
a contradiction. Thus we conclude that the characteristic $p$ is positive.
In this case the condition $P' = 0$ forces $a_i = 0$ whenever $p \not | i$.
In other words, $P(x) = P_1(x^p)$ for some nonconstant polynomial $P_1$.
Clearly, $P_1$ is irreducible as well. By induction on the degree we
see that $P_1(x) = Q(x^q)$ as in the statement of the lemma, hence
$P(x) = Q(x^{pq})$ and the lemma is proved.
\end{proof}

\begin{definition}
\label{definition-separable}
Let $F$ be a field. Let $K/F$ be an extension of fields.
\begin{enumerate}
\item We say an irreducible polynomial $P$ over $F$ is {\it separable}
if it is relatively prime to its derivative.
\item Given $\alpha \in K$ algebraic over $F$ we say $\alpha$ is
{\it separable} over $F$ if its minimal polynomial is separable over $F$.
\item If $K$ is an algebraic extension of $F$, we say $K$ is
{\it separable}\footnote{For nonalgebraic extensions
this definition does not make sense and is not the correct one.}
over $F$ if every element of $K$ is separable over $F$.
\end{enumerate}
\end{definition}

\noindent
By Lemma \ref{lemma-irreducible-polynomials} in characteristic $0$ every
irreducible polynomial is separable, every algebraic element in an extension
is separable, and every algebraic extension is separable.

\begin{lemma}
\label{lemma-separable-goes-up}
Let $K/E/F$ be a tower of algebraic field extensions.
\begin{enumerate}
\item If $\alpha \in K$ is separable over $F$, then $\alpha$ is separable
over $E$.
\item if $K$ is separable over $F$, then $K$ is separable over $E$.
\end{enumerate}
\end{lemma}

\begin{proof}
We will use Lemma \ref{lemma-irreducible-polynomials} without further mention.
Let $P$ be the minimal polynomial of $\alpha$ over $F$.
Let $Q$ be the minimal polynomial of $\alpha$ over $E$.
Then $Q$ divides $P$ in the polynomial ring $E[x]$, say $P = QR$.
Then $P' = Q'R + QR'$. Thus if $Q' = 0$, then $Q$ divides $P$ and $P'$
hence $P' = 0$ by the lemma. This proves (1). Part (2)
follows immediately from (1) and the definitions.
\end{proof}

\begin{lemma}
\label{lemma-recognize-separable}
Let $F$ be a field. An irreducible polynomial $P$ over $F$
is separable if and only if $P$ has pairwise distinct roots in a
algebraic closure of $F$.
\end{lemma}

\begin{proof}
Suppose that $\alpha \in F$ is a root of both $P$ and $P'$.
Then $P = (x - \alpha)Q$ for some polynomial $Q$. Taking derivatives
we obtain $P' = Q + (x - \alpha)Q'$. Thus $\alpha$ is a root of $Q$.
Hence we see that if $P$ and $P'$ have a common root, then $P$
does not have pairwise distinct roots. Conversely, if $P$ has
a repeated root, i.e., $(x - \alpha)^2$ divides $P$, then $\alpha$
is a root of both $P$ and $P'$. Combined with
Lemma \ref{lemma-relatively-prime-polynomials} this proves the lemma.
\end{proof}

\begin{lemma}
\label{lemma-nr-roots-unchanged}
Let $F$ be a field and let $\overline{F}$ be an algebraic closure of $F$.
Let $p > 0$ be the characteristic of $F$. Let $P$ be a polynomial
over $F$. Then the set of roots of $P$ and $P(x^p)$ in $\overline{F}$
have the same cardinality (not counting multiplicity).
\end{lemma}

\begin{proof}
Clearly, $\alpha$ is a root of $P(x^p)$ if and only if $\alpha^p$ is a
root of $P$. In other words, the roots of $P(x^p)$ are the roots of
$x^p - \beta$, where $\beta$ is a root of $P$. Thus it suffices to show
that the map $\overline{F} \to \overline{F}$, $\alpha \mapsto \alpha^p$
is bijective. It is surjective, as $\overline{F}$ is algebraically closed
which means that every element has a $p$th root. It is injective because
$\alpha^p = \beta^p$ implies $(\alpha - \beta)^p = 0$ because
the characteristic is $p$. And of course in a field $x^p = 0$ implies
$x = 0$.
\end{proof}

\noindent
Let $F$ be a field and let $P$ be an irreducible polynomial over $F$.
Then we know that $P = Q(x^q)$ for some separable irreducible polynomial $Q$
(Lemma \ref{lemma-irreducible-polynomials}) where $q$ is a power of
the characteristic $p$ (and if the characteristic is zero, then
$q = 1$\footnote{A good convention for this chapter is to set $0^0 = 1$.}
and $Q = P$). By Lemma \ref{lemma-nr-roots-unchanged} the number of
roots of $P$ and $Q$ in any algebraic closure of $F$ is the same.
By Lemma \ref{lemma-recognize-separable} this number is equal to the degree
of $Q$.

\begin{definition}
\label{definition-separable-degree}
Let $F$ be a field. Let $P$ be an irreducible polynomial over $F$.
The {\it separable degree} of $P$ is the cardinality of the
set of roots of $P$ in any algebraic closure of $F$ (see discussion
above). Notation $\deg_s(P)$.
\end{definition}

\noindent
The separable degree of $P$ always divides the degree and the quotient
is a power of the characteristic. If the characteristic is zero, then
$\deg_s(P) = \deg(P)$.

\begin{situation}
\label{situation-finitely-generated}
Here $F$ be a field and $K/F$ is a finite extension generated by elements
$\alpha_1, \ldots, \alpha_n \in K$. We set $K_0 = F$ and
$$
K_i = F(\alpha_1, \dots, \alpha_i)
$$
to obtain a tower of finite extensions
$K = K_r / K_{r - 1} / \ldots / K_0 = F$.
Denote $P_i$ the minimal polynomial of $\alpha_i$ over $K_{i - 1}$.
Finally, we fix an algebraic closure $\overline{F}$ of $F$.
\end{situation}

\noindent
Let $F$, $K$, $\alpha_i$, and $\overline{F}$ be as in
Situation \ref{situation-finitely-generated}.
Suppose that $\varphi : K \to \overline{F}$ is a morphism of extensions
of $F$. Then we obtain maps $\varphi_i : K_i \to \overline{F}$.
In particular, we can take the image of $P_i \in K_{i - 1}[x]$ by
$\varphi_{i - 1}$ to get a polynomial $P_i^\varphi \in \overline{F}[x]$.

\begin{lemma}
\label{lemma-count-embeddings}
In situation \ref{situation-finitely-generated} the correspondence
$$
\Mor_F(K, \overline{F})
\longrightarrow
\{(\beta_1, \ldots, \beta_n)\text{ as below}\},
\quad
\varphi \longmapsto (\varphi(\alpha_1), \ldots, \varphi(\alpha_n))
$$
is a bijection. Here the right hand side is the set of $n$-tuples
$(\beta_1, \ldots, \beta_n)$ of elements of $\overline{F}$
such that $\beta_i$ is a root of $P_i^\varphi$.
\end{lemma}

\begin{proof}
Let $(\beta_1, \ldots, \beta_n)$ be an element of the right hand side.
We construct a map of fields corresponding to it by induction.
Namely, we set $\varphi_0 : K_0 \to \overline{F}$ equal to the given
map $K_0 = F \subset \overline{F}$. Having constructed
$\varphi_{i - 1} : K_{i - 1} \to \overline{F}$ we observe that
$K_i = K_{i - 1}[x]/(P_i)$. Hence we can set $\varphi_i$ equal
to the unique map $K_i \to \overline{F}$ inducing $\varphi_{i - 1}$
on $K_{i - 1}$ and mapping $x$ to $\beta_i$. This works precisely
as $\beta_i$ is a root of $P_i^\varphi$. Uniqueness implies that
the two constructions are mutually inverse.
\end{proof}

\begin{lemma}
\label{lemma-count-embeddings-explicitly}
In situation \ref{situation-finitely-generated} we have
$|\Mor_F(K, \overline{F})| = \prod_{i = 1}^n \deg_s(P_i)$.
\end{lemma}

\begin{proof}
This follows immediately from Lemma \ref{lemma-count-embeddings}.
Observe that a key ingredient we are tacitly using here is the
well-definedness of the separable degree of an irreducible polynomial
which was observed just prior to
Definition \ref{definition-separable-degree}.
\end{proof}

\noindent
We now use the result above to characterize separable field extensions.

\begin{lemma}
\label{lemma-separably-generated-separable}
Assumptions and notation as in Situation \ref{situation-finitely-generated}.
If each $P_i$ is separable, i.e., each $\alpha_i$ is separable over
$K_{i - 1}$, then
$$
|\Mor_F(K, \overline{F})| = [K : F]
$$
and the field extension $K/F$ is separable. If one of the $\alpha_i$ is
not separable over $K_{i - 1}$, then
$|\Mor_F(K, \overline{F})| < [K : F]$.
\end{lemma}

\begin{proof}
If $\alpha_i$ is separable over $K_{i - 1}$ then
$\deg_s(P_i) = \deg(P_i) = [K_i : K_{i - 1}]$
(last equality by Example \ref{example-degree-simple-algebraic-extension}).
By multiplicativity (Lemma \ref{lemma-multiplicativity-degrees}) we have
$$
[K : F] = \prod [K_i : K_{i - 1}] = \prod \deg(P_i) =
\prod \deg_s(P_i) = |\Mor_F(K, \overline{F})|
$$
where the last equality is Lemma \ref{lemma-count-embeddings-explicitly}.
By the exact same argument we get the strict inequality
$|\Mor_F(K, \overline{F})| < [K : F]$ if one of the $\alpha_i$ is
not separable over $K_{i - 1}$.

\medskip\noindent
Finally, assume again that each $\alpha_i$ is separable over $K_{i - 1}$.
Let $\gamma = \gamma_1 \in K$ be arbitrary. Then we can find additional
elements $\gamma_2, \ldots, \gamma_m$ such that
$K = F(\gamma_1, \ldots, \gamma_m)$ (for example we could take
$\gamma_2 = \alpha_1, \ldots, \gamma_{n + 1} = \alpha_n$).
Then we see by the last part of the lemma (already proven above)
that if $\gamma$ is not separable over $F$ we would have the
strict inequality $|\Mor_F(K, \overline{F})| < [K : F]$
contradicting the very first part of the lemma (already prove above
as well).
\end{proof}

\begin{lemma}
\label{lemma-separable-permanence}
Let $E/k$ and $F/E$ be separable algebraic extensions of fields. Then $F/k$
is a separable extension of fields.
\end{lemma}

\begin{proof}
Choose $\alpha \in F$. Then $\alpha$ is separable algebraic over $E$.
Let $P = x^d + \sum_{i < d} a_i x^i$ be the minimal polynomial of
$\alpha$ over $E$. Each $a_i$ is separable algebraic over $k$.
Consider the tower of fields
$$
k \subset k(a_0) \subset k(a_0, a_1) \subset \ldots \subset
k(a_0, \ldots, a_{d - 1}) \subset k(a_0, \ldots, a_{d - 1}, \alpha)
$$
Because $a_i$ is separable algebraic over $k$ it is separable algebraic
over $k(a_0, \ldots, a_{i - 1})$ by Lemma \ref{lemma-separable-goes-up}.
Finally, $\alpha$ is separable algebraic over $k(a_0, \ldots, a_{d - 1})$
because it is a root of $P$ which is irreducible
(as it is irreducible over the possibly bigger field $E$)
and separable (as it is separable over $E$).
Thus $k(a_0, \ldots, a_{d - 1}, \alpha)$ is separable over $k$
by Lemma \ref{lemma-separably-generated-separable}
and we conclude that $\alpha$ is separable over $k$ as desired.
\end{proof}

\begin{lemma}
\label{lemma-separable-elements}
Let $E/k$ be a field extension. Then the elements of $E$ separable
over $k$ form a subextension of $E/k$.
\end{lemma}

\begin{proof}
Let $\alpha, \beta \in E$ be separable over $k$. Then $\beta$ is separable
over $k(\alpha)$ by Lemma \ref{lemma-separable-goes-up}.
Thus we can apply Lemma \ref{lemma-separable-goes-up} to $k(\alpha, \beta)$
to see that $k(\alpha, \beta)$ is separable over $k$.
\end{proof}






\section{Purely inseparable extensions}
\label{section-purely-inseparable}

\noindent
Purely inseparable extensions are the opposite of the separable
extensions defined in the previous section. These extensions only
show up in positive characteristic.

\begin{definition}
\label{definition-purely-inseparable}
Let $F$ be a field of characteristic $p > 0$. Let $K/F$ be an extension.
\begin{enumerate}
\item An element $\alpha \in K$ is {\it purely inseparable} over $F$
if there exists a power $q$ of $p$ such that $\alpha^q \in F$.
\item The extension $K/F$ is said to be {\it purely inseparable}
if and only if every element of $K$ is purely inseparable over $F$.
\end{enumerate}
\end{definition}

\noindent
Observe that a purely inseparable extension is necessarily algebraic.
Let $F$ be a field of characteristic $p > 0$.
An example of a purely inseparable extension is gotten by adjoining
the $p$th root of an element $t \in F$ which does not yet have one. Namely,
the lemma below shows that $P = x^p - t$ is irreducible, and hence
$$
K = F[x]/(P) = F[t^{1/p}]
$$
is a field. And $K$ is purely inseparable over $F$ because every element
$$
a_0 + a_1t^{1/p} + \ldots + a_{p - 1}t^{p - 1/p}, a_i \in F
$$
has $p$th power equal to
$$
(a_0 + a_1t^{1/p} + \ldots + a_{p - 1}t^{p - 1/p})^p =
a_0^p + a_1^p t + \ldots + a_{p - 1}^pt^{p - 1} \in F
$$
This situation occurs for the field
$\mathbf{F}_p(t)$ of rational functions over $\mathbf{F}_p$.

\begin{lemma}
\label{lemma-take-pth-root}
Let $p$ be a prime number. Let $F$ be a field of characteristic $p$.
Let $t \in F$ be an element which does not have a $p$th root in $F$.
Then the polynomial $x^p - t$ is irreducible over $F$.
\end{lemma}

\begin{proof}
To see this, suppose that we have a factorization
$x^p - t = f g$. Taking derivatives we get $f' g + f g' = 0$.
Note that neither $f' = 0$ nor $g' = 0$ as the degrees of $f$ and $g$
are smaller than $p$. Moreover, $\deg(f') < \deg(f)$ and $\deg(g') < \deg(g)$.
We conclude that $f$ and $g$ have a factor in common. Thus if $x^p - t$
is reducible, then it is of the form $x^p - t = c f^n$ for some irreducible
$f$, $c \in F^*$, and $n > 1$. Since $p$ is a prime number this
implies $n = p$ and $f$ linear, which would imply $x^p - t$ has a root
in $F$. Contradiction.
\end{proof}

\noindent
We will see that taking $p$th roots is a very important operation in
characteristic $p$.

\begin{lemma}
\label{lemma-purely-inseparable-permanence}
Let $E/k$ and $F/E$ be purely inseparable extensions of fields. Then $F/k$
is a purely inseparable extension of fields.
\end{lemma}

\begin{proof}
Say the characteristic of $k$ is $p$. Choose $\alpha \in F$. Then
$\alpha^q \in E$ for some $p$-power $q$. Whereupon $(\alpha^q)^{q'} \in k$
for some $p$-power $q'$. Hence $\alpha^{qq'} \in k$.
\end{proof}

\begin{lemma}
\label{lemma-purely-inseparable-elements}
Let $E/k$ be a field extension. Then the elements of $E$ purely-inseparable
over $k$ form a subextension of $E/k$.
\end{lemma}

\begin{proof}
Let $p$ be the characteristic of $k$.
Let $\alpha, \beta \in E$ be purely inseparable over $k$. Say
$\alpha^q \in k$ and $\beta^{q'} \in k$ for some $p$-powers $q, q'$.
If $q''$ is a $p$-power, then
$(\alpha + \beta)^{q''} = \alpha^{q''} + \beta^{q''}$.
Hence if $q'' \geq q, q'$, then we conclude that $\alpha + \beta$
is purely inseparable over $k$. Similarly for the difference,
product and quotient of $\alpha$ and $\beta$.
\end{proof}




\section{Galois theory}
\label{section-galois}


\begin{definition}
\label{definition-automorphisms}
Let $E/F$ be an extension of fields. Then $\text{Aut}(K/F)$ denotes the
subgroup of the group of automorphisms of $K$ (as a field) consisting of
those automorphisms which fix $F$ pointwise.
\end{definition}

\begin{definition}
By $\deg(K/F)$ we mean the dimension of $K$ as an $F$-vector
space. We denote $K_s/F$ the set of elements of $K$ whose minimal polynomials
over $F$ have distinct roots; by \ref{sep_subfield} this is a subfield, and
$\deg(K_s/F) = \deg_s(K/F)$ and $\deg(K/K_s) = \deg_i(K/F)$ by definition.
\label{def:sep}
\end{definition}

\subsection{Theorems}



\begin{lemma} For any $f \in \text{Emb}(K/F)$, the map $\text{Aut}(K/F) \to \text{Emb}(K/F)$ given
by $\sigma \mapsto f \circ \sigma$ is injective.
\label{aut_inj}
\end{lemma}

\begin{proof} This is immediate from the injectivity of $f$. \end{proof}

\begin{lemma} $\text{Aut}(K/F)$ is finite.
\label{aut_fin}
\end{lemma}

\begin{proof} By \ref{aut_inj}, $\text{Aut}(K/F)$ injects into $\text{Emb}(K/F)$, which by
\ref{emb_size} is finite. \end{proof}

\begin{proposition} The inequality
\begin{equation*}
|\text{Aut}(K/F)| \leq |\text{Emb}(K/F)|
\end{equation*}
is an equality if and only if the $q_i$ all split in $K$.
\label{aut_ineq}
\end{proposition}

\begin{proof} The inequality follows from \ref{aut_inj} and from
\ref{aut_fin}.
Since both sets are finite, equality holds if and only if the injection of
\ref{aut_inj} is surjective (for fixed $f \in \text{Emb}(K/F)$).

If surjectivity holds, let $\beta_1, \dots, \beta_n$ be arbitrary roots of
$q_1, \dots, q_n$ in the sense of \ref{emb_roots}, and extract an embedding $g
\colon K \to M$ with $g(\alpha_i) = \beta_i$. Since the correspondence $f
\mapsto f \circ \sigma$ ($\sigma \in \text{Aut}(K/F)$) is a bijection, there is some
$\sigma$ such that $g = f \circ \sigma$, and therefore $f$ and $g$ have the
same image. Therefore the image of $K$ in $M$ is canonical, and contains
$\beta_1, \dots, \beta_n$ for any choice thereof.

If the $q_i$ all split, let $g \in \text{Emb}(K/F)$ be arbitrary, so the
$g(\alpha_i)$ are roots of $q_i$ in $M$ as in \ref{emb_roots}. But the $q_i$
have all their roots in $K$, hence in the image $f(K)$, so $f$ and $g$ again
have the same image, and $f^{-1} \circ g \in \text{Aut}(K/F)$. Thus $g = f \circ
(f^{-1} \circ g)$ shows that the map of \ref{aut_inj} is surjective.
\end{proof}

\begin{lemma} Define
\begin{equation*}
D(K/F) = \prod_{i = 1}^n \deg_s(K_i/K_{i - 1}).
\end{equation*}
Then the chain of equalities and inequalities
\begin{equation*}
|\text{Aut}(K/F)| \leq |\text{Emb}(K/F)| = D(K/F) \leq \deg(K/F)
\end{equation*}
holds; the first inequality is an equality if and only if each $q_i$ splits in
$K$, and the second if and only if each $q_i$ is separable.
\label{large_aut_ineq}
\end{lemma}

\begin{proof} The statements concerning the first inequality are just
\ref{aut_ineq}; the interior equality is just \ref{emb_size}; the latter
inequality is obvious from the multiplicativity of the degrees of field
extensions; and the deduction for equality follows from the definition of
$\deg_s$. \end{proof}

\begin{lemma} The $q_i$ respectively split and are separable in $K$ if and
only
if the $Q_i$ do and are.
\label{absolute_sepsplit}
\end{lemma}

\begin{proof} The ordering of the $\alpha_i$ is irrelevant, so we may take
each $i = 1$ in turn. Then $Q_1 = q_1$ and if either of the equalities in
\ref{large_aut_ineq} holds then so does the corresponding statement here.
Conversely, clearly each $q_i$ divides $Q_i$, so splitting or separability
for the latter implies that for the former. \end{proof}

\begin{lemma} Let $\alpha \in K$ have minimal polynomial $q$; if the $Q_i$
are
respectively split, separable, and purely inseparable over $F$ then $q$ is as
well.
\label{global_sepsplit}
\end{lemma}

\begin{proof} We may take $\alpha$ as the first element of an alternative
generating set for $K/F$. The numerical statement of \ref{large_aut_ineq}
does not depend on the particular generating set, hence the conditions given
hold of the set containing $\alpha$ if and only if they hold of the canonical
set ${\alpha_1, \dots, \alpha_n}$.

For purely inseparable, if the $Q_i$ all have only one root then $|\text{Emb}(K/F)|
= 1$ by \ref{large_aut_ineq}, and taking $\alpha$ as the first element of a
generating set as above shows that $q$ must have only one root as well for
this to hold. \end{proof}

\begin{lemma} $K_s$ is a field and $\deg(K_s/F) = D(K/F)$.
\label{sep_subfield}
\end{lemma}

\begin{proof} Assume $\text{Char}{F} = p > 0$, for otherwise $K_s = K$. Using
\ref{sep_poly}, write each $Q_i = R_i(x^{p^{d_i}})$, and let $\beta_i =
\alpha_i^{p^{d_i}}$. Then the $\beta_i$ have $R_i$ as minimal polynomials and
the $\alpha_i$ satisfy $s_i = x^{p^{d_i}} - \beta_i$ over $K' = F(\beta_1,
\dots, \beta_n)$. Therefore the $\alpha_i$ have minimal polynomials over $K'$
dividing the $s_i$ and hence those polynomials have but one distinct root.

By \ref{global_sepsplit}, the elements of $K'$ are separable, and those of
$K'$ purely inseparable over $K'$. In particular, since these minimal
polynomials divide those over $F$, none of these elements is separable, so $K'
= K_s$.

The numerical statement follows by computation:
\begin{equation*}
\deg(K/K') = \prod_{i = 1}^n p^{d_i}
	= \prod_{i = 1}^n \frac{\deg(K_i/K_{i - 1})}{\deg_s(K_i/K_{i - 1})}
	= \frac{\deg(K/F)}{D(K/F)}.
	\end{equation*}
\end{proof}

\begin{theorem} The following inequality holds:
\begin{equation*}
|\text{Aut}(K/F)| \leq |\text{Emb}(K/F)| = \deg_s(K/F) \leq \deg(K/F).
\end{equation*}
Equality holds on the left if and only if $K/F$ is splitting; it holds on the
right if and only if $K/F$ is separable.
\label{galois_size}
\end{theorem}

\begin{proof} The numerical statement combines \ref{large_aut_ineq} and
\ref{sep_subfield}. The deductions combine \ref{absolute_sepsplit} and
\ref{global_sepsplit}. \end{proof}

\subsection{Definitions}

Throughout, we will denote as before $K/F$ a finite field extension, and $G =
\text{Aut}(K/F)$, $H$ a subgroup of $G$. $L/F$ is a subextension of $K/F$.

\begin{definition} When $K/F$ is separable and splitting, we say it is Galois
and
write $G = \text{Gal}(K/F)$, the Galois group of $K$ over $F$.
\label{defn:galois_extension}
\end{definition}

\begin{definition} The fixed field of $H$ is the field $K^H$ of elements fixed
by
the action of $H$ on $K$. Conversely, $G_L$ is the fixing subgroup of $L$,
the subgroup of $G$ whose elements fix $L$.
\label{defn:fixing}
\end{definition}

\subsection{Theorems}

\begin{lemma} A polynomial $q(x) \in K[x]$ which splits in $K$ lies in
$K^H[x]$ if and only if its roots are permuted by the action of $H$. In this
case, the sets of roots of the irreducible factors of $q$ over $K^H$ are the
orbits
of the action of $H$ on the roots of $q$ (counting multiplicity).
\label{root_action}
\end{lemma}

\begin{proof} Since $H$ acts by automorphisms, we have $\sigma q(x) = q(\sigma
x)$ as a functional equation on $K$, so $\sigma$ permutes the roots of $q$.
Conversely, since the coefficients of $\sigma$ are the elementary symmetric
polynomials in its roots, $H$ permuting the roots implies that it fixes the
coefficients.

Clearly $q$ is the product of the polynomials $q_i$ whose roots are the orbits
of the action of $H$ on the roots of $q$, counting multiplicities, so it
suffices to show that these polynomials are defined over $K^H$ and are
irreducible. Since $H$ acts on the roots of the $q_i$ by construction, the
former is satisfied. If some $q_i$ factored over $K^H$, its factors would
admit an action of $H$ on their roots by the previous paragraph. The roots of
$q_i$ are distinct by construction, so its factors do not share roots; hence
the action on the roots of $q_i$ would not be transitive, a contradiction.
\end{proof}

\begin{lemma} Let $q(x) \in K[x]$; if it is irreducible, then $H$ acts
transitively on its roots; conversely, if $q$ is separable and $H$ acts
transitively on its roots, then $q(x) \in K^H[x]$ is irreducible.
\label{sep_irred}
\end{lemma}

\begin{proof} Immediate from \ref{root_action}. \end{proof}

\begin{lemma} If $K/F$ is Galois, so is $K/L$, and $\text{Gal}(K/L) = G_L$..
\label{sub_galois}
\end{lemma}

\begin{proof} $K/F$ Galois means that the minimal polynomial over $F$ of every
element of $K$ is separable and splits in $K$; the minimal polynomials over $L
= K^H$ divide those over $F$, and therefore this is true of $K/L$ as well;
hence $K/L$ is likewise a Galois extension. $\text{Gal}(K/L) = \text{Aut}(K/L)$ consists
of those automorphisms $\sigma$ of $K$ which fix $L$; since $F \subset L$ we
have \emph{a fortiori} that $\sigma$ fixes $F$, hence $\text{Gal}(K/L) \subset G$
and consists of the subgroup which fixes $L$; i.e. $G_L$. \end{proof}

\begin{lemma} If $K/F$ and $L/F$ are Galois, then the action of $G$ on
elements of $L$
defines a surjection of $G$ onto $\text{Gal}(L/F)$. Thus $G_L$ is normal in $G$ and
$\text{Gal}(L/F) \cong G/G_L$. Conversely, if $N \subset G$ is normal, then $K^N/F$
is Galois.
\label{normal}
\end{lemma}

\begin{proof} $L/F$ is splitting, so by \ref{root_action} the elements of $G$
act as endomorphisms (hence automorphisms) of $L/F$, and the kernel of this
action is $G_L$. By
\ref{sub_galois}, we have $G_L = \text{Gal}(K/L)$, so $|G_L| = |\text{Gal}(K/L)| = [K : L]
= [K : F] / [L : F]$,
or rearranging and using that $K/F$ is Galois, we get $|G|/|G_L| = [L : F] =
|\text{Gal}(L/F)|$. Thus the map $G \to \text{Gal}(L/F)$ is surjective and thus the
induced map $G/G_L \to
\text{Gal}(L/F)$ is an isomorphism.

Conversely, let $N$ be normal and take $\alpha \in K^N$. For any conjugate
$\beta$ of $\alpha$, we
have $\beta = g(\alpha)$ for some $g \in G$; let $n \in N$. Then $n(\beta) =
(ng)(\alpha) =
g(g^{-1} n g)(\alpha) = g(\alpha) = \beta$, since $g^{-1} n g \in N$ by
normality of $N$. Thus
$\beta \in K^N$, so $K^N$ is splitting, i.e., Galois. \end{proof}

\begin{proposition} If $K/F$ is Galois and $H = G_L$, then $K^H = L$.
\label{fixed_field}
\end{proposition}

\begin{proof} By \ref{sub_galois}, $K/L$ and $K/K^H$ are both Galois. By
definition, $\text{Gal}(K/L) = G_L = H$; since $H$ fixes $K^H$ we certainly have
$H < \text{Gal}(K/K^H)$, but since $L \subset K^H$ we have \emph{a fortiori} that
$\text{Gal}(K/K^H) < \text{Gal}(K/L) = H$, so $\text{Gal}(K/K^H) = H$ as well. It follows
from \ref{galois_size} that $\deg(K/L) = |H| = \deg(K/K^H)$, so that $K^H =
L$. \end{proof}

\begin{lemma} If $K$ is a finite field, then $K^\ast$ is cyclic.
\label{fin_cyclic}
\end{lemma}

\begin{proof} $K$ is then a finite extension of $\mathbf{F}_p$ for $p =
\text{Char}{K}$, hence has order $p^n$, $n = \deg(K/\mathbf{F}_p)$. Thus
$\alpha^{p^n} = \alpha$ for all $\alpha \in K$, since $|K^\ast| = p^n - 1$.
It follows that every element of $K$ is a root of $q_n(x) = x^{p^n} - x$. For
any $d < n$, the elements of order at most $p^d - 1$ satisfy $q_d(x)$, which has
$p^d$ roots. It follows that there are at least $p^n(p - 1) > 0$ elements of
order exactly $p^n - 1$, so $K^\ast$ is cyclic. \end{proof}

\begin{lemma} If $K$ is a finite field, then $\text{Gal}(K/F)$ is cyclic,
generated by
the Frobenius automorphism.
\label{fin_gal_cyclic}
\end{lemma}

\begin{proof} First take $F = \mathbf{F}_p$. Then the map $f_i(\alpha) =
\alpha^{p^i}$ is an endomorphism, injective since $K$ is a field, and
surjective since it is finite, hence an automorphism. Since every $\alpha$
satisfies $\alpha^{p^n} = \alpha$, $f_n = 1$, but by \ref{fin_cyclic}, $f_{n -
1}$ is nontrivial (applied to the generator). Since $n = \deg(K/F)$, $f =
f_1$ generates $\text{Gal}(K/F)$.

If $F$ is now arbitrary, by \ref{fixed_field} we have $\text{Gal}(K/F) =
\text{Gal}(K/\mathbf{F}_p)_F$, and every subgroup of a cyclic group is cyclic.
\end{proof}

\begin{lemma} If $K$ is finite, $K/F$ is primitive.
\label{fin_prim_elt}
\end{lemma}

\begin{proof} No element of $G$ fixes the generator $\alpha$ of $K^\ast$, so
it cannot lie in any proper subfield. Therefore $F(\alpha) = K$. \end{proof}

\begin{proposition} If $F$ is infinite and $K/F$ has only finitely many
subextensions, then it is
primitive.
\label{gen_prim_elt}
\end{proposition}

\begin{proof} We proceed by induction on the number of generators of $K/F$.

If $K = F(\alpha)$ we are done. If not, $K = F(\alpha_1, \dots, \alpha_n) =
F(\alpha_1, \dots, \alpha_{n - 1})(\alpha_n) = F(\beta, \alpha_n)$ by
induction, so we may assume $n = 2$. There are infinitely many subfields
$F(\alpha_1 + t \alpha_2)$, with $t \in F$, hence two of them are equal, say
for $t_1$ and
$t_2$. Thus, $\alpha_1 + t_2 \alpha_2 \in F(\alpha_1 + t_1 \alpha_2)$. Then
$(t_2 - t_1)\alpha_2 \in F(\alpha_1 + t_1 \alpha_2)$, hence $\alpha_2$ lies in
this field, hence $\alpha_1$ does. Therefore $K = F(\alpha_1 + t_1
\alpha_2)$. \end{proof}

\begin{lemma} If $K/F$ is separable, it is primitive, and the generator may
be
taken to be a linear combination of any finite set of generators of $K/F$.
\label{prim_elt}
\end{lemma}

\begin{proof} We may embed $K/F$ in a Galois extension $M/F$ by adjoining all
the conjugates of its generators. Subextensions of $K/F$ are as well
subextensions
of $K'/F$ and by \ref{fixed_field} the map $H \mapsto (K')^H$ is a surjection
from the subgroups of $G$ to the subextensions of $K'/F$, which are hence
finite in number. By \ref{fin_prim_elt} we may assume $F$ is infinite. The
result now follows from \ref{gen_prim_elt}. \end{proof}

\begin{lemma}
 If $K/F$ is Galois and $H \subset G$, then if $L = K^H$, we have $H = G_L$.
 \label{fixing_subgroup}
\end{lemma}

\begin{proof}
 Let $\alpha$ be a primitive element for $K/L$. The polynomial $\prod_{h \in
H} (x - h(\alpha))$ is fixed by $H$, and therefore has coefficients in $L$, so
$\alpha$ has $|H|$ conjugate roots over $L$. But since $\alpha$ is primitive,
we have $K = L(\alpha)$, so the minimal polynomial of $\alpha$ has degree
$\deg(K/L)$, which is the same as the number of its roots. Thus $|H| =
\deg(K/L)$. Since $H \subset G_L$ and $|G_L| = \deg(K/L)$, we have equality.
\end{proof}


\begin{theorem} The correspondences $H \mapsto K^H$, $L \mapsto G_L$ define
inclusion-reversing inverse maps between the set of subgroups of $G$ and the
set of subextensions of $K/F$, such that normal subgroups and Galois subfields
correspond.
\label{fundamental_theorem}
\end{theorem}

\begin{proof} This combines \ref{fixed_field}, \ref{fixing_subgroup}, and
\ref{normal}.
\end{proof}


\section{Transcendental Extensions}


There is a distinguished type of transcendental extension: those that are
``purely transcendental.''
\begin{definition} A field extension $E'/E$ is purely transcendental if it is
obtained by adjoining a set $B$ of algebraically independent elements. A set of
elements is algebraically independent over $E$ if there is no nonzero
polynomial$P$
with coefficients in $E$ such
that $P(b_1,b_2,\cdots b_n)=0$ for any finite subset of elements $b_1, \dots,
b_n \in B$.
\end{definition}

\begin{example} The field $\mathbf{Q}(\pi)$ is purely transcendental; in
particular, $\mathbf{Q}(\pi)\cong\mathbf{Q}(x)$ with the isomorphism fixing
$\mathbf{Q}$. \end{example}
Similar to the degree of an algebraic extension, there is a way of keeping
track of the number of algebraically independent generators that are required to
generate a purely transcendental extension.
\begin{definition} Let $E'/E$ be a purely transcendental extension generated by
some set of algebraically independent elements $B$. Then the transcendence
degree $trdeg(E'/E)=\#(B)$ and $B$ is called a transcendence basis for $E'/E$
(we will see later that $trdeg(E'/E)$ is independent of choice of basis).
\end{definition}
In general, let $F/E$ be a field extension, we can always construct an
intermediate extension $F/E'/E$ such that $F/E'$ is algebraic and $E'/E$ is
purely transcendental. Then if $B$ is a transcendence basis for $E'$, it is
also called a transcendence basis for $F$. Similarly, $trdeg(F/E)$ is defined to
be
$trdeg(E'/E)$.
\begin{theorem} Let $F/E$ be a field extension, a transcendence basis exists.
\end{theorem}
\begin{proof} Let $A$ be an algebraically independent subset of $F$. Now pick a
subset $G\subset F$ that generates $F/E$, we can find a transcendence basis
$B$ such that $A\subset B\subset G$. Define a collection of algebraically
independent sets $\mathcal{B}$ whose members are subsets of $G$ that contain
$A$. The set can be partially ordered inclusion and contains at least one
element, $A$. The union of elements of $\mathcal{B}$ is algebraically
independent since any algebraic dependence relation would have occurred in one
of the elements of $\mathcal{B}$ since the polynomial is only allowed to be over
finitely many variables. The union also satisfies $A\subset
\bigcup\mathcal{B}\subset G$ so by Zorn's lemma, there is a maximal element
$B\in\mathcal{B}$. Now we claim $F$ is algebraic over $E(B)$. This is because
if it wasn't then there would be a transcendental element $f\in G$ (since
$E(G)=F$)such that $B\cup\{f\}$ wold be algebraically independent contradicting
the
maximality of $B$. Thus $B$ is our transcendence basis. \end{proof}
Now we prove that the transcendence degree of a field extension is independent
of choice of basis.
\begin{theorem} Let $F/E$ be a field extension. Any two transcendence bases for
$F/E$ have the same cardinality. This shows that the $trdeg(E/F)$ is well
defined. \end{theorem}
\begin{proof}
Let $B$ and $B'$ be two transcendence bases. Without loss of generality, we can
assume that $\#(B')\leq \#(B)$. Now we divide the proof into two cases: the
first case is that $B$ is an infinite set. Then for each $\alpha\in B'$, there
is a finite set $B_{\alpha}$ such that $\alpha$ is algebraic over
$E(B_{\alpha})$ since any algebraic dependence relation only uses finitely many
indeterminates. Then we define $B^*=\bigcup_{\alpha\in B'} B_{\alpha}$. By
construction, $B^*\subset B$, but we claim that in fact the two sets are
equal. To see this, suppose that they are not equal, say there is an element
$\beta\in B\setminus B^*$. We know $\beta$ is algebraic over $E(B')$ which is
algebraic over $E(B^*)$. Therefor $\beta$ is algebraic over $E(B^*)$, a
contradiction. So $\#(B)\leq \sum_{\alpha\in B'} \#(B_{\alpha})$. Now if $B'$ is
finite, then so is $B$ so we can assume $B'$ is infinite; this means
\begin{equation} \#(B)\leq \sum_{\alpha\in B'}\#(B_{\alpha})=\#(\coprod
B_{\alpha})\leq \#(B'\times\mathbf{Z})=\#(B')\end{equation} with the inequality
$\#(\coprod
B_{\alpha}) \leq \#(B'\times \mathbf{Z})$ given by the correspondence
$b_{\alpha_i}\mapsto (\alpha,i)\in B'\times \mathbf{Z}$ with $B_\alpha =
\{b_{\alpha_1},b_{\alpha_2}\cdots b_{\alpha_{n_\alpha}}\}$ Therefore in the
infinite case, $\#(B)=\#(B')$.

Now we need to look at the case where $B$ is finite. In this case, $B'$ is also
finite, so suppose $B=\{\alpha_1,\cdots\alpha_n\}$ and
$B'=\{\beta_1,\cdots\beta_m\}$ with $m\leq n$. We perform induction on $m$: if
$m=0$ then $F/E$ is algebraic so $B=\null$ so $n=0$, otherwise there is an
irreducible polynomial $f\in E[x,y_1,\cdots y_n]$ such that
$f(\beta_1,\alpha_1,\cdots \alpha_n) = 0$. Since $\beta_1$ is not algebraic over
$E$, $f$ must involve some $y_i$ so without loss of generality, assume $f$ uses
$y_1$. Let $B^*=\{\beta_1,\alpha_2,\cdots\alpha_n\}$. We claim that $B^*$ is a
basis for $F/E$. To prove this claim, we see that we have a tower of algebraic
extensions $F/E(B^*,\alpha_1)/E(B^*)$ since $\alpha_1$ is algebraic over
$E(B^*)$. Now we claim that $B^*$ (counting multiplicity of elements) is
algebraically independent over $E$ because if it weren't, then there would be an
irreducible $g\in E[x,y_2,\cdots y_n]$ such that
$g(\beta_1,\alpha_2,\cdots\alpha_n)=0$ which must involve $x$ making $\beta_1$
algebraic over $E(\alpha_2,\cdots \alpha_n)$ which would make $\alpha_1$
algebraic over $E(\alpha_2,\cdots \alpha_n)$ which is impossible. So this means
that $\{\alpha_2,\cdots\alpha_n\}$ and $\{\beta_2,\cdots\beta_m\}$ are bases for
$F$ over $E(\beta_1)$ which means by induction, $m=n$. \end{proof}

\begin{example} Consider the field extension $\mathbf{Q}(e,\pi)$ formed by
adjoining the numbers $e$ and $\pi$. This field extension has transcendence
degree at least $1$ since both $e$ and $\pi$ are transcendental over the
rationals. However, this field extension might have transcendence degree $2$ if
$e$ and $\pi$ are algebraically independent. Whether or not this is true is
unknown and the problem of determining $trdeg(\mathbf{Q}(e,\pi))$ is an open
problem.\end{example}

\begin{example} let $E$ be a field and $F=E(t)/E$. Then $\{t\}$ is a
transcendence basis since $F=E(t)$. However, $\{t^2\}$ is also a transcendence
basis since $E(t)/E(t^2)$ is algebraic. This illustrates that while we can
always decompose an extension $F/E$ into an algebraic extension $F/E'$ and a
purely transcendental extension $E'/E$, this decomposition is not unique and
depends on choice of transcendence basis. \end{example}

\begin{exercise} If we have a tower of fields $G/F/E$, then
$trdeg(G/E)=trdeg(F/E)+trdeg(G/F)$. \end{exercise}

\begin{example}
Let $X$ be a compact Riemann surface. Then the function field $\mathbf{C}(X)$
(see \ref{example-field-of-meromorphic-functions}) has transcendence degree one over $\mathbf{C}$. In
fact, \emph{any} finitely generated extension of $\mathbf{C}$ of transcendence
degree one arises from a Riemann surface. There is even an equivalence of
categories between the category of compact Riemann surfaces and
(non-constant) holomorphic maps
and the opposite category of finitely generated extensions of $\mathbf{C}$ and
morphisms of $\mathbf{C}$-algebras. See \cite{Forster}.

There is an algebraic version of the above statement as well. Given an
(irreducible) algebraic curve in projective space over an algebraically
closed field $k$ (e.g. the complex numbers), one can consider its ``field of
rational
functions:'' basically, functions that look like quotients of polynomials,
where the denominator does not identically vanish on the curve.
There is a similar anti-equivalence of categories between smooth projective
curves and
non-constant morphisms of curves and finitely generated extensions of $k$ of
transcendence degree one. See \cite{H}.
\end{example}


\subsection{Linearly Disjoint Field Extensions}
Let $k$ be a field, $K$ and $L$ field extensions of $k$. Suppose also that $K$
and $L$ are embedded in some larger field $\Omega$.

\begin{definition} The compositum of $K$ and $L$ written $KL$ is $k(K\cup
L)=L(K)=K(L)$.
\end{definition}



\begin{definition} $K$ and $L$ are said to be linearly disjoint over $k$ if the
following map is injective:
\begin{equation} \theta: K\otimes_k L\rightarrow KL \end{equation} defined by
$x\otimes y\mapsto xy$.
\end{definition}


\section{Other chapters}

\begin{multicols}{2}
\begin{enumerate}
\item \hyperref[introduction-section-phantom]{Introduction}
\item \hyperref[conventions-section-phantom]{Conventions}
\item \hyperref[sets-section-phantom]{Set Theory}
\item \hyperref[categories-section-phantom]{Categories}
\item \hyperref[topology-section-phantom]{Topology}
\item \hyperref[sheaves-section-phantom]{Sheaves on Spaces}
\item \hyperref[algebra-section-phantom]{Commutative Algebra}
\item \hyperref[sites-section-phantom]{Sites and Sheaves}
\item \hyperref[homology-section-phantom]{Homological Algebra}
\item \hyperref[derived-section-phantom]{Derived Categories}
\item \hyperref[more-algebra-section-phantom]{More Algebra}
\item \hyperref[simplicial-section-phantom]{Simplicial Methods}
\item \hyperref[modules-section-phantom]{Sheaves of Modules}
\item \hyperref[sites-modules-section-phantom]{Modules on Sites}
\item \hyperref[injectives-section-phantom]{Injectives}
\item \hyperref[cohomology-section-phantom]{Cohomology of Sheaves}
\item \hyperref[sites-cohomology-section-phantom]{Cohomology on Sites}
\item \hyperref[hypercovering-section-phantom]{Hypercoverings}
\item \hyperref[schemes-section-phantom]{Schemes}
\item \hyperref[constructions-section-phantom]{Constructions of Schemes}
\item \hyperref[properties-section-phantom]{Properties of Schemes}
\item \hyperref[morphisms-section-phantom]{Morphisms of Schemes}
\item \hyperref[coherent-section-phantom]{Coherent Cohomology}
\item \hyperref[divisors-section-phantom]{Divisors}
\item \hyperref[limits-section-phantom]{Limits of Schemes}
\item \hyperref[varieties-section-phantom]{Varieties}
\item \hyperref[chow-section-phantom]{Chow Homology}
\item \hyperref[topologies-section-phantom]{Topologies on Schemes}
\item \hyperref[descent-section-phantom]{Descent}
\item \hyperref[more-morphisms-section-phantom]{More on Morphisms}
\item \hyperref[flat-section-phantom]{More on Flatness}
\item \hyperref[groupoids-section-phantom]{Groupoid Schemes}
\item \hyperref[more-groupoids-section-phantom]{More on Groupoid Schemes}
\item \hyperref[etale-section-phantom]{\'Etale Morphisms of Schemes}
\item \hyperref[etale-cohomology-section-phantom]{\'Etale Cohomology}
\item \hyperref[spaces-section-phantom]{Algebraic Spaces}
\item \hyperref[spaces-properties-section-phantom]{Properties of Algebraic Spaces}
\item \hyperref[spaces-morphisms-section-phantom]{Morphisms of Algebraic Spaces}
\item \hyperref[spaces-topologies-section-phantom]{Topologies on Algebraic Spaces}
\item \hyperref[spaces-descent-section-phantom]{Descent and Algebraic Spaces}
\item \hyperref[spaces-more-morphisms-section-phantom]{More on Morphisms of Spaces}
\item \hyperref[quot-section-phantom]{Quot and Hilbert Spaces}
\item \hyperref[stacks-section-phantom]{Stacks}
\item \hyperref[spaces-groupoids-section-phantom]{Groupoids in Algebraic Spaces}
\item \hyperref[spaces-more-groupoids-section-phantom]{More on Groupoids in Spaces}
\item \hyperref[bootstrap-section-phantom]{Bootstrap}
\item \hyperref[examples-stacks-section-phantom]{Examples of Stacks}
\item \hyperref[groupoids-quotients-section-phantom]{Quotients of Groupoids}
\item \hyperref[algebraic-section-phantom]{Algebraic Stacks}
\item \hyperref[criteria-section-phantom]{Criteria for Representability}
\item \hyperref[stacks-properties-section-phantom]{Properties of Algebraic Stacks}
\item \hyperref[stacks-morphisms-section-phantom]{Morphisms of Algebraic Stacks}
\item \hyperref[examples-section-phantom]{Examples}
\item \hyperref[exercises-section-phantom]{Exercises}
\item \hyperref[guide-section-phantom]{Guide to Literature}
\item \hyperref[desirables-section-phantom]{Desirables}
\item \hyperref[coding-section-phantom]{Coding Style}
\item \hyperref[fdl-section-phantom]{GNU Free Documentation License}
\item \hyperref[index-section-phantom]{Auto Generated Index}
\end{enumerate}
\end{multicols}


\bibliography{my}
\bibliographystyle{amsalpha}

\end{document}
