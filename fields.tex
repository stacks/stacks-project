\IfFileExists{stacks-project.cls}{%
\documentclass{stacks-project}
}{%
\documentclass{amsart}
}

% The following AMS packages are automatically loaded with
% the amsart documentclass:
%\usepackage{amsmath}
%\usepackage{amssymb}
%\usepackage{amsthm}

% For dealing with references we use the comment environment
\usepackage{verbatim}
\newenvironment{reference}{\comment}{\endcomment}
%\newenvironment{reference}{}{}
\newenvironment{slogan}{\comment}{\endcomment}
\newenvironment{history}{\comment}{\endcomment}

% For commutative diagrams you can use
% \usepackage{amscd}
\usepackage[all]{xy}

% We use 2cell for 2-commutative diagrams.
\xyoption{2cell}
\UseAllTwocells

% To put source file link in headers.
% Change "template.tex" to "this_filename.tex"
% \usepackage{fancyhdr}
% \pagestyle{fancy}
% \lhead{}
% \chead{}
% \rhead{Source file: \url{template.tex}}
% \lfoot{}
% \cfoot{\thepage}
% \rfoot{}
% \renewcommand{\headrulewidth}{0pt}
% \renewcommand{\footrulewidth}{0pt}
% \renewcommand{\headheight}{12pt}

\usepackage{multicol}

% For cross-file-references
\usepackage{xr-hyper}

% Package for hypertext links:
\usepackage{hyperref}

% For any local file, say "hello.tex" you want to link to please
% use \externaldocument[hello-]{hello}
\externaldocument[introduction-]{introduction}
\externaldocument[conventions-]{conventions}
\externaldocument[sets-]{sets}
\externaldocument[categories-]{categories}
\externaldocument[topology-]{topology}
\externaldocument[sheaves-]{sheaves}
\externaldocument[sites-]{sites}
\externaldocument[stacks-]{stacks}
\externaldocument[fields-]{fields}
\externaldocument[algebra-]{algebra}
\externaldocument[brauer-]{brauer}
\externaldocument[homology-]{homology}
\externaldocument[derived-]{derived}
\externaldocument[simplicial-]{simplicial}
\externaldocument[more-algebra-]{more-algebra}
\externaldocument[smoothing-]{smoothing}
\externaldocument[modules-]{modules}
\externaldocument[sites-modules-]{sites-modules}
\externaldocument[injectives-]{injectives}
\externaldocument[cohomology-]{cohomology}
\externaldocument[sites-cohomology-]{sites-cohomology}
\externaldocument[dga-]{dga}
\externaldocument[dpa-]{dpa}
\externaldocument[hypercovering-]{hypercovering}
\externaldocument[schemes-]{schemes}
\externaldocument[constructions-]{constructions}
\externaldocument[properties-]{properties}
\externaldocument[morphisms-]{morphisms}
\externaldocument[coherent-]{coherent}
\externaldocument[divisors-]{divisors}
\externaldocument[limits-]{limits}
\externaldocument[varieties-]{varieties}
\externaldocument[topologies-]{topologies}
\externaldocument[descent-]{descent}
\externaldocument[perfect-]{perfect}
\externaldocument[more-morphisms-]{more-morphisms}
\externaldocument[flat-]{flat}
\externaldocument[groupoids-]{groupoids}
\externaldocument[more-groupoids-]{more-groupoids}
\externaldocument[etale-]{etale}
\externaldocument[chow-]{chow}
\externaldocument[intersection-]{intersection}
\externaldocument[pic-]{pic}
\externaldocument[adequate-]{adequate}
\externaldocument[dualizing-]{dualizing}
\externaldocument[duality-]{duality}
\externaldocument[discriminant-]{discriminant}
\externaldocument[local-cohomology-]{local-cohomology}
\externaldocument[curves-]{curves}
\externaldocument[resolve-]{resolve}
\externaldocument[models-]{models}
\externaldocument[pione-]{pione}
\externaldocument[etale-cohomology-]{etale-cohomology}
\externaldocument[proetale-]{proetale}
\externaldocument[crystalline-]{crystalline}
\externaldocument[spaces-]{spaces}
\externaldocument[spaces-properties-]{spaces-properties}
\externaldocument[spaces-morphisms-]{spaces-morphisms}
\externaldocument[decent-spaces-]{decent-spaces}
\externaldocument[spaces-cohomology-]{spaces-cohomology}
\externaldocument[spaces-limits-]{spaces-limits}
\externaldocument[spaces-divisors-]{spaces-divisors}
\externaldocument[spaces-over-fields-]{spaces-over-fields}
\externaldocument[spaces-topologies-]{spaces-topologies}
\externaldocument[spaces-descent-]{spaces-descent}
\externaldocument[spaces-perfect-]{spaces-perfect}
\externaldocument[spaces-more-morphisms-]{spaces-more-morphisms}
\externaldocument[spaces-flat-]{spaces-flat}
\externaldocument[spaces-groupoids-]{spaces-groupoids}
\externaldocument[spaces-more-groupoids-]{spaces-more-groupoids}
\externaldocument[bootstrap-]{bootstrap}
\externaldocument[spaces-pushouts-]{spaces-pushouts}
\externaldocument[groupoids-quotients-]{groupoids-quotients}
\externaldocument[spaces-more-cohomology-]{spaces-more-cohomology}
\externaldocument[spaces-simplicial-]{spaces-simplicial}
\externaldocument[spaces-duality-]{spaces-duality}
\externaldocument[formal-spaces-]{formal-spaces}
\externaldocument[restricted-]{restricted}
\externaldocument[spaces-resolve-]{spaces-resolve}
\externaldocument[formal-defos-]{formal-defos}
\externaldocument[defos-]{defos}
\externaldocument[cotangent-]{cotangent}
\externaldocument[examples-defos-]{examples-defos}
\externaldocument[algebraic-]{algebraic}
\externaldocument[examples-stacks-]{examples-stacks}
\externaldocument[stacks-sheaves-]{stacks-sheaves}
\externaldocument[criteria-]{criteria}
\externaldocument[artin-]{artin}
\externaldocument[quot-]{quot}
\externaldocument[stacks-properties-]{stacks-properties}
\externaldocument[stacks-morphisms-]{stacks-morphisms}
\externaldocument[stacks-limits-]{stacks-limits}
\externaldocument[stacks-cohomology-]{stacks-cohomology}
\externaldocument[stacks-perfect-]{stacks-perfect}
\externaldocument[stacks-introduction-]{stacks-introduction}
\externaldocument[stacks-more-morphisms-]{stacks-more-morphisms}
\externaldocument[stacks-geometry-]{stacks-geometry}
\externaldocument[moduli-]{moduli}
\externaldocument[moduli-curves-]{moduli-curves}
\externaldocument[examples-]{examples}
\externaldocument[exercises-]{exercises}
\externaldocument[guide-]{guide}
\externaldocument[desirables-]{desirables}
\externaldocument[coding-]{coding}
\externaldocument[obsolete-]{obsolete}
\externaldocument[fdl-]{fdl}
\externaldocument[index-]{index}

% Theorem environments.
%
\theoremstyle{plain}
\newtheorem{theorem}[subsection]{Theorem}
\newtheorem{proposition}[subsection]{Proposition}
\newtheorem{lemma}[subsection]{Lemma}

\theoremstyle{definition}
\newtheorem{definition}[subsection]{Definition}
\newtheorem{example}[subsection]{Example}
\newtheorem{exercise}[subsection]{Exercise}
\newtheorem{situation}[subsection]{Situation}

\theoremstyle{remark}
\newtheorem{remark}[subsection]{Remark}
\newtheorem{remarks}[subsection]{Remarks}

\numberwithin{equation}{subsection}

% Macros
%
\def\lim{\mathop{\mathrm{lim}}\nolimits}
\def\colim{\mathop{\mathrm{colim}}\nolimits}
\def\Spec{\mathop{\mathrm{Spec}}}
\def\Hom{\mathop{\mathrm{Hom}}\nolimits}
\def\Ext{\mathop{\mathrm{Ext}}\nolimits}
\def\SheafHom{\mathop{\mathcal{H}\!\mathit{om}}\nolimits}
\def\SheafExt{\mathop{\mathcal{E}\!\mathit{xt}}\nolimits}
\def\Sch{\mathit{Sch}}
\def\Mor{\operatorname{Mor}\nolimits}
\def\Ob{\mathop{\mathrm{Ob}}\nolimits}
\def\Sh{\mathop{\mathit{Sh}}\nolimits}
\def\NL{\mathop{N\!L}\nolimits}
\def\proetale{{pro\text{-}\acute{e}tale}}
\def\etale{{\acute{e}tale}}
\def\QCoh{\mathit{QCoh}}
\def\Ker{\mathop{\mathrm{Ker}}}
\def\Im{\mathop{\mathrm{Im}}}
\def\Coker{\mathop{\mathrm{Coker}}}
\def\Coim{\mathop{\mathrm{Coim}}}

%
% Macros for moduli stacks/spaces
%
\def\QCohstack{\mathcal{QC}\!\mathit{oh}}
\def\Cohstack{\mathcal{C}\!\mathit{oh}}
\def\Spacesstack{\mathcal{S}\!\mathit{paces}}
\def\Quotfunctor{\mathrm{Quot}}
\def\Hilbfunctor{\mathrm{Hilb}}
\def\Curvesstack{\mathcal{C}\!\mathit{urves}}
\def\Polarizedstack{\mathcal{P}\!\mathit{olarized}}
\def\Complexesstack{\mathcal{C}\!\mathit{omplexes}}
% \Pic is the operator that assigns to X its picard group, usage \Pic(X)
% \Picardstack_{X/B} denotes the Picard stack of X over B
% \Picardfunctor_{X/B} denotes the Picard functor of X over B
\def\Pic{\mathop{\mathrm{Pic}}\nolimits}
\def\Picardstack{\mathcal{P}\!\mathit{ic}}
\def\Picardfunctor{\mathrm{Pic}}
\def\Deformationcategory{\mathcal{D}\!\mathit{ef}}


% OK, start here.
%
\begin{document}

\title{Fields}


\maketitle

\phantomsection
\label{section-phantom}

\tableofcontents


\section{Introduction}
\label{section-introduction}

\noindent
In this chapter, we shall discuss the theory of fields. Recall that a
{\it field} is a ring in which all nonzero elements are invertible.
Equivalently, the only two ideals of a field are $(0)$ and $(1)$
since any nonzero element is a unit. Consequently fields will be the
simplest cases of much of the theory developed later.

\medskip\noindent
The theory of field extensions has a different feel from standard commutative
algebra since, for instance, any morphism of fields is injective. Nonetheless,
it turns out that questions involving rings can often be reduced to questions
about fields. For instance, any domain can be embedded in a field
(its quotient field), and any {\it local ring} (that is, a ring with a unique
maximal ideal; we have not defined this term yet) has associated to it its
residue field (that is, its quotient by the maximal ideal).
A knowledge of field extensions will thus be useful.




\section{Basic definitions}
\label{section-definitions}

\noindent
Because we have placed this chapter before the chapter discussing
commutative algebra we need to introduce some of the basic definitions
here before we discuss these in greater detail in the algebra chapters.

\begin{definition}
\label{definition-field}
An {\it field} is a nonzero ring where every nonzero element is invertible.
Given a field a {\it subfield} is a subring that is itself a field.
\end{definition}

\noindent
For a field $k$, we write $k^*$ for the subset $k \setminus \{0\}$.
This generalizes the usual notation $R^*$ that refers to the group of
invertible elements in a ring $R$.

\begin{definition}
\label{definition-domain}
A {\it domain} or an {\it integral domain} is a nonzero ring where $0$
is the only zerodivisor.
\end{definition}



\section{Examples of fields}
\label{section-examples}

\noindent
To get started, let us begin by providing several examples of fields. The
reader should recall that if $R$ is a ring and $I \subset R$ an
ideal, then $R/I$ is a field precisely when $I$ is a maximal ideal.

\begin{example}[Rational numbers]
\label{example-rational-numbers}
The rational numbers form a field. It is called the
{\it field of rational numbers} and denoted $\mathbf{Q}$.
\end{example}

\begin{example}[Prime fields]
\label{example-prime-field}
If $p$ is a prime number, then $\mathbf{Z}/(p)$ is a field, denoted
$\mathbf{F}_p$. Indeed, $(p)$ is a
maximal ideal in $\mathbf{Z}$. Thus, fields may be finite: $\mathbf{F}_p$
contains $p$ elements.
\end{example}

\begin{example}
\label{example-quotient-polymial-ring}
In a principal ideal domain, an ideal generated by an irreducible element
is maximal. Now, if $k$ is a field, then the polynomial ring $k[x]$ is a
principal ideal domain. It follows that if $P \in k[x]$ is an irreducible
polynomial (that is, a nonconstant polynomial
that does not admit a factorization into terms of smaller degrees), then
$k[x]/(P)$ is a field. It contains a copy of $k$ in a natural way.
This is a very general way of constructing fields. For instance, the
complex numbers $\mathbf{C}$
can be constructed as $\mathbf{R}[x]/(x^2 + 1)$.
\end{example}

\begin{example}[Quotient fields]
\label{example-quotient-field}
Recall that, given a domain $A$, there is an imbedding $A \to K(A)$ into a
field $K(A)$ constructed from $A$ in exactly the same manner that
$\mathbf{Q}$ is constructed from $\mathbf{Z}$. Formally the elements
of $K(A)$ are (equivalence classes of) fractions $a/b$,
$a, b \in A$, $b \not = 0$. As usual $a/b = a'/b'$ if and only if $ab' = ba'$.
This is called the {\it quotient field} or {\it field of fractions} or
the {\it fraction field} of $A$.
The quotient field has the following universal property: given an
injective ring map $\varphi : A \to K$ to a field $K$, there is a unique
map $\psi : K(A) \to K$ making
$$
\xymatrix{
K(A) \ar[r]_\psi & K \\
A \ar[u] \ar[ru]_\varphi
}
$$
commute. Indeed, it is clear how to define such a map: we set
$\psi(a/b) = \varphi(a)\varphi(b)^{-1}$ where injectivity of $\varphi$
assures that $\varphi(b) \not = 0$ if $ b \not = 0$.
\end{example}

\begin{example}[Field of rational functions]
\label{example-field-of-rational-functions}
If $k$ is a field, then we can consider the field $k(x)$ of rational
functions over $k$. This is the quotient field of the polynomial ring
$k[x]$. In other words, it is the set of quotients $F/G$ for
$F, G \in k[x]$, $G \not = 0$ with the obvious equivalence relation.
\end{example}

\begin{example}
\label{example-field-of-meromorphic-functions}
Let $X$ be a Riemann surface. Let $\mathbf{C}(X)$ denote the
set of meromorphic functions on $X$. Then $\mathbf{C}(X)$ is a ring under
multiplication and addition of functions. It turns out that in fact
$\mathbf{C}(X)$ is a field. Namely, if a nonzero function $f(z)$ is
meromorphic, so is $1/f(z)$. For example, let $S^2$ be the Riemann
sphere; then we know from complex analysis that the ring of meromorphic
functions $\mathbf{C}(S^2)$ is the field of rational functions $\mathbf{C}(z)$.
\end{example}



\section{Vector spaces}
\label{section-vector-spaces}

\noindent
One reason fields are so nice is that the theory of modules over fields
(i.e. vector spaces), is very simple.

\begin{lemma}
\label{lemma-vector-space-is-free}
If $k$ is a field, then every $k$-module is free.
\end{lemma}

\begin{proof}
Indeed, by linear algebra we know that a $k$-module (i.e. vector space)
$V$ has a {\it basis} $\mathcal{B} \subset V$, which defines an isomorphism
from the free vector space on $\mathcal{B}$ to $V$.
\end{proof}

\begin{lemma}
\label{lemma-field-semi-simple}
Every exact sequence of modules over a field splits.
\end{lemma}

\begin{proof}
This follows from Lemma \ref{lemma-vector-space-is-free} as every vector
space is a projective module.
\end{proof}

\noindent
This is another reason why much of the theory in future chapters will not say
very much about fields, since modules behave in such a simple manner.
Note that Lemma \ref{lemma-field-semi-simple} is a statement about the
{\it category} of $k$-modules (for $k$ a field), because the notion of
exactness is inherently arrow-theoretic, i.e., makes use of purely categorical
notions, and can in fact be phrased within a so-called {\it abelian category}.

\medskip\noindent
Henceforth, since the study of modules over a field is linear algebra, and
since the ideal theory of fields is not very interesting, we shall study what
this chapter is really about: {\it extensions} of fields.


\section{The characteristic of a field}
\label{section-more-fields}

\noindent
In the category of rings, there is an {\it initial object} $\mathbf{Z}$: any
ring $R$ has a map from $\mathbf{Z}$ into it in precisely one way. For fields,
there is no such initial object.
Nonetheless, there is a family of objects such that every field can be mapped
into in exactly one way by exactly one of them, and in no way by the others.

\medskip\noindent
Let $F$ be a field. Think of $F$ as a ring to get a ring map
$f : \mathbf{Z} \to F$. The image of this ring map is a domain
(as a subring of a field) hence the kernel of $f$ is a prime ideal
in $\mathbf{Z}$. Hence the kernel of $f$ is either $(0)$ or $(p)$ for
some prime number $p$.

\medskip\noindent
In the first case we see that $f$ is injective, and in this case
we think of $\mathbf{Z}$ as a subring of $F$. Moreover, since every
nonzero element of $F$ is invertible we see that it makes sense to
talk about $p/q \in F$ for $p, q \in \mathbf{Z}$ with $q \not = 0$.
Hence in this case we may and we do think of $\mathbf{Q}$ as a subring of $F$.
One can easily see that this is the smallest subfield of $F$ in this case.

\medskip\noindent
In the second case, i.e., when $\Ker(f) = (p)$ we see that
$\mathbf{Z}/(p) = \mathbf{F}_p$ is a subring of $F$. Clearly it is the
smallest subfield of $F$.

\medskip\noindent
Arguing in this way we see that every field contains a smallest subfield
which is either $\mathbf{Q}$ or finite equal to $\mathbf{F}_p$ for some
prime number $p$.

\begin{definition}
\label{definition-characteristic}
The {\it characteristic} of a field $F$ is $0$ if
$\mathbf{Z} \subset F$, or is a prime $p$ if $p = 0$ in $F$.
The {\it prime subfield of $F$} is the smallest subfield of $F$
which is either $\mathbf{Q} \subset F$ if the characteristic is zero, or
$\mathbf{F}_p \subset F$ if the characteristic is $p > 0$.
\end{definition}

\noindent
It is easy to see that if $E \subset F$ is a subfield, then the
characteristic of $E$ is the same as the characteristic of $F$.

\begin{example}
\label{example-characteristic}
The characteristic of $\mathbf{F}_p$ is $p$, and that of $\mathbf{Q}$ is $0$.
\end{example}


\section{Field extensions}
\label{section-extensions}

\noindent
In general, though, we are interested not so much in fields by themselves but
in field {\it extensions}. This is perhaps analogous to studying not rings
but {\it algebras} over a fixed ring.
The nice thing for fields is that the notion of a ``field over another field''
just recovers the notion of a field extension, by the next result.

\begin{lemma}
\label{lemma-field-maps-injective}
If $F$ is a field and $R$ is a nonzero ring, then any ring homomorphism
$\varphi : F \to R$ is injective.
\end{lemma}

\begin{proof}
Indeed, let $a \in \Ker(\varphi)$ be a nonzero element. Then we have
$\varphi(1) = \varphi(a^{-1} a) = \varphi(a^{-1}) \varphi(a) = 0$.
Thus $1 = \varphi(1) = 0$ and $R$ is the zero ring.
\end{proof}

\begin{definition}
\label{definition-extension}
If $F$ is a field contained in a field $E$, then $E$ is said
to be a {\it field extension} of $F$. We shall write $E/F$ to indicate
that $E$ is an extension of $F$.
\end{definition}

\noindent
So if $F, F'$ are fields, and $F \to F'$ is any ring-homomorphism, we see by
Lemma \ref{lemma-field-maps-injective} that it is injective, and $F'$ can be
regarded as an extension of $F$, by a slight abuse of language. Alternatively,
a field extension of $F$ is just an $F$-algebra that happens to be a field.
This is completely different than the situation for general rings, since a
ring homomorphism is not necessarily injective.

\medskip\noindent
Let $k$ be a field. There is a {\it category} of field extensions of $k$.
An object of this category is an extension $E/k$, that is a
(necessarily injective) morphism of fields
$$
k \to E,
$$
while a morphism between extensions $E/k$ and $E'/k$ is a $k$-algebra
morphism $E \to E'$; alternatively, it is a commutative diagram
$$
\xymatrix{
E \ar[rr] & & E' \\
& k \ar[ru] \ar[lu] &
}
$$
The set of morphisms from $E \to E'$ in the category of extensions of $k$
will be denoted by $\Mor_k(E, E')$.

\begin{definition}
\label{definition-tower}
A {\it tower} of fields $E_n/E_{n - 1}/\ldots/E_0$ consists of a sequence of
extensions of fields
$E_n/E_{n - 1}$, $E_{n - 1}/E_{n - 2}$, $\ldots$, $E_1/E_0$.
\end{definition}

\noindent
Let us give a few examples of field extensions.

\begin{example}
\label{example-monogenic-extension}
Let $k$ be a field, and $P \in k[x]$ an irreducible polynomial. We have
seen that $k[x]/(P)$ is a field (Example \ref{example-quotient-polymial-ring}).
Since it is also a $k$-algebra in the obvious way, it is an extension of $k$.
\end{example}

\begin{example}
\label{example-field-of-meromorphic-functions-extension-C}
If $X$ is a Riemann surface, then the field of meromorphic functions
$\mathbf{C}(X)$ (Example \ref{example-field-of-meromorphic-functions})
is an extension field of $\mathbf{C}$, because any element of $\mathbf{C}$
induces a meromorphic --- indeed, holomorphic --- constant function on $X$.
\end{example}

\noindent
Let $F/k$ be a field extension. Let $S \subset F$ be any subset.
Then there is a {\it smallest} subextension of $F$ (that is, a subfield of
$F$ containing $k$) that contains $S$. To see this, consider the family of
subfields of $F $ containing $S$ and $k$, and take their intersection; one
checks that this is a field. By a standard argument one shows, in fact, that
this is the set of elements of $F$ that can be obtained via a finite number
of elementary algebraic operations (addition, multiplication, subtraction,
and division) involving elements of $k$ and $S$.

\begin{definition}
\label{definition-generated-by}
Let $k$ be a field. If $F/k$ is an extension of fields and
$S \subset F$, we write $k(S)$ for the smallest subfield of $F$
containing $k$ and $S$. We will say that $S$ {\it generates the
field extension} $k(S)/k$. If $S = \{\alpha\}$ is a singleton, then we
write $k(\alpha)$ instead of $k(\{\alpha\})$. We say $F/k$ is a
{\it finitely generated field extension} if there exists a
finite subset $S \subset F$ with $F = k(S)$.
\end{definition}

\noindent
For instance, $\mathbf{C}$ is generated by $i$ over $\mathbf{R}$.

\begin{exercise}
\label{exercise-C-not-countably-generated}
Show that $\mathbf{C}$ does not have a countable set of generators over
$\mathbf{Q}$.
\end{exercise}

\noindent
Let us now classify extensions generated by one element.

\begin{lemma}[Classification of simple extensions]
\label{lemma-field-extension-generated-by-one-element}
If a field extension $F/k$ is generated by one element, then it is
$k$-isomorphic either to the rational function field $k(t)/k$ or to one
of the extensions $k[t]/(P)$ for $P \in k[t]$ irreducible.
\end{lemma}

\noindent
We will see that many of the most important cases of field extensions are
generated by one element, so this is actually useful.

\begin{proof}
Let $\alpha \in F$ be such that $F = k(\alpha)$; by assumption, such an
$\alpha$ exists. There is a morphism of rings
$$
k[t] \to F
$$
sending the indeterminate $t$ to $\alpha$. The image is a domain, so the
kernel is a prime ideal. Thus, it is either $(0)$ or $(P)$ for $P \in k[t]$
irreducible.

\medskip\noindent
If the kernel is $(P)$ for $P \in k[t]$ irreducible, then the map factors
through $k[t]/(P)$, and induces a morphism of fields $k[t]/(P) \to F$. Since
the image contains $\alpha$, we see easily that the map is surjective, hence
an isomorphism. In this case, $k[t]/(P) \simeq F$.

\medskip\noindent
If the kernel is trivial, then we have an injection $k[t] \to F$.
One may thus define a morphism of the quotient field $k(t)$ into $F$; given a
quotient $R(t)/Q(t)$ with $R(t), Q(t) \in k[t]$, we map this to
$R(\alpha)/Q(\alpha)$. The hypothesis that $k[t] \to F$ is injective implies
that $Q(\alpha) \neq 0$ unless $Q$ is the zero polynomial.
The quotient field of $k[t]$ is the rational function field $k(t)$, so we get
a morphism $k(t) \to F$
whose image contains $\alpha$. It is thus surjective, hence an isomorphism.
\end{proof}




\section{Finite extensions}
\label{section-finite-extensions}

\noindent
If $F/E$ is a field extension, then evidently $F$ is also a vector space
over $E$ (the scalar action is just multiplication in $F$).

\begin{definition}
\label{definition-degree}
Let $F/E$ be an extension of fields. The dimension of $F$ considered as an
$E$-vector space is called the {\it degree} of the extension and is
denoted $[F : E]$. If $[F : E]<\infty$ then $F$ is said to be a
{\it finite} extension of $E$.
\end{definition}

\begin{example}
\label{example-C-over-R}
The field $\mathbf{C}$ is a two dimensional vector space over $\mathbf{R}$
with basis $1, i$. Thus $\mathbf{C}$ is a finite extension of $\mathbf{R}$
of degree 2.
\end{example}

\begin{lemma}
\label{lemma-finite-goes-up}
Let $K/E/F$ be a tower of algebraic field extensions.
If $K$ is finite over $F$, then $K$ is finite over $E$.
\end{lemma}

\begin{proof}
Direct from the definition.
\end{proof}

\noindent
Let us now consider the degree in the most important special example, that
given by Lemma \ref{lemma-field-extension-generated-by-one-element}, in the
next two examples.

\begin{example}[Degree of a rational function field]
\label{example-degree-rational-function-field}
If $k$ is any field, then the rational function field $k(t)$ is
{\it not} a finite extension. For example the elements
$\left\{t^n, n \in \mathbf{Z}\right\}$ are linearly independent over $k$.

\medskip\noindent
In fact, if $k$ is uncountable, then $k(t)$ is {\it uncountably} dimensional
as a $k$-vector space. To show this, we claim that the family of elements
$\{1/(t- \alpha), \alpha \in k\} \subset k(t)$ is linearly independent over
$k$. A nontrivial relation between them would lead to a contradiction: for
instance, if one works over $\mathbf{C}$, then this follows because
$\frac{1}{t-\alpha}$, when considered as a meromorphic function on
$\mathbf{C}$, has a pole at $\alpha$ and nowhere else.
Consequently any sum $\sum c_i \frac{1}{t - \alpha_i}$ for the $c_i \in k^*$,
and $\alpha_i \in k$ distinct, would have poles at each of the $\alpha_i$.
In particular, it could not be zero.

\medskip\noindent
Amusingly, this leads to a quick proof of the Hilbert Nullstellensatz over
the complex numbers. For a slightly more general result, see
Algebra, Theorem \ref{algebra-theorem-uncountable-nullstellensatz}. 
\end{example}

\begin{lemma}
\label{lemma-finite-finitely-generated}
A finite extension of fields is a finitely generated field extension.
The converse is not true.
\end{lemma}

\begin{proof}
Let $F/E$ be a finite extension of fields. Let $\alpha_1, \ldots, \alpha_n$
be a basis of $F$ as a vector space over $E$. Then
$F = E(\alpha_1, \ldots, \alpha_n)$ hence $F/E$ is a finitely generated
field extension. The converse is not true as follows from
Example \ref{example-degree-rational-function-field}.
\end{proof}

\begin{example}[Degree of a simple algebraic extension]
\label{example-degree-simple-algebraic-extension}
Consider a monogenic field extension $E/k$ of the form discussed in
Example \ref{example-monogenic-extension}.
In other words, $E = k[t]/(P)$ for $P \in k[t]$ an irreducible polynomial.
Then the degree $[E : k]$ is just the degree $d = \deg(P)$ of the
polynomial $P$. Indeed, say
\begin{equation}
\label{equation-P}
P = a_d t^d + a_1 t^{d - 1} + \ldots + a_0.
\end{equation}
with $a_d \not = 0$. Then the images of $1, t, \ldots, t^{d - 1}$ in
$k[t]/(P)$ are linearly independent over $k$, because any relation involving
them would have degree strictly smaller than that of $P$, and $P$ is the
element of smallest degree in the ideal $(P)$.

\medskip\noindent
Conversely, the set $S = \{1, t, \ldots, t^{d - 1}\}$ (or more
properly their images) spans $k[t]/(P)$ as a vector space.
Indeed, we have by (\ref{equation-P}) that $a_d t^d$ lies in the span of $S$.
Since $a_d$ is invertible, we see that $t^d$ is in the span of $S$.
Similarly, the relation $t P(t) = 0$ shows that the image of $t^{d + 1}$
lies in the span of $\{1, t, \ldots, t^d\}$ --- by what was just shown, thus
in the span of $S$. Working upward inductively, we find
that the image of $t^n$ for $n \geq d$ lies in the span of $S$.
\end{example}

\noindent
This confirms the observation that $[\mathbf{C}: \mathbf{R}] = 2$, for
instance. More generally, if $k$ is a field, and $\alpha \in k$ is not a
square, then the irreducible polynomial $x^2 - \alpha \in k[x]$ allows one
to construct an extension $k[x]/(x^2 - \alpha)$ of degree two.
We shall write this as $k(\sqrt{\alpha})$. Such extensions will be called
{\it quadratic,} for obvious reasons.

\medskip\noindent
The basic fact about the degree is that it is {\it multiplicative in towers.}

\begin{lemma}[Multiplicativity]
\label{lemma-multiplicativity-degrees}
Suppose given a tower of fields $F/E/k$. Then
$$
[F:k] = [F:E][E:k]
$$
\end{lemma}

\begin{proof}
Let $\alpha_1, \ldots, \alpha_n \in F$ be an $E$-basis for $F$. Let
$\beta_1, \ldots, \beta_m \in E$ be a $k$-basis for $E$. Then the claim is
that the set of products
$\{\alpha_i \beta_j, 1 \leq i \leq n, 1 \leq j \leq m\}$
is a $k$-basis for $F$. Indeed, let us check first that they span $F$ over $k$.

\medskip\noindent
By assumption, the $\{\alpha_i\}$ span $F$ over $E$. So if
$f \in F$, there are $a_i \in E$ with
$$
f = \sum\nolimits_i a_i \alpha_i,
$$
and, for each $i$, we can write $a_i = \sum b_{ij} \beta_j$ for some
$b_{ij} \in k$. Putting these together, we find
$$
f = \sum\nolimits_{i,j} b_{ij} \alpha_i \beta_j,
$$
proving that the $\{\alpha_i \beta_j\}$ span $F$ over $k$.

\medskip\noindent
Suppose now that there existed a nontrivial relation
$$
\sum\nolimits_{i,j} c_{ij} \alpha_i \beta_j = 0
$$
for the $c_{ij} \in k$. In that case, we would have
$$
\sum\nolimits_i \alpha_i \left( \sum\nolimits_j c_{ij} \beta_j \right) = 0,
$$
and the inner terms lie in $E$ as the $\beta_j$ do. Now $E$-linear
independence of the $\{\alpha_i\}$ shows that the inner sums are all zero.
Then $k$-linear independence of the $\{\beta_j\}$ shows that the
$c_{ij}$ all vanish.
\end{proof}

\noindent
We sidetrack to a slightly tangential definition.

\begin{definition}
\label{definition-number-field}
A field $K$ is said to be a {\it number field} if it has characteristic
$0$ and the extension $\mathbf{Q} \subset K$ is finite.
\end{definition}

\noindent
Number fields are the basic objects in algebraic number theory. We shall see
later that,
for the analog of the integers $\mathbf{Z}$ in a number field, something kind
of like unique factorization still holds (though strict unique factorization
generally does not!).


\section{Algebraic extensions}
\label{section-algebraic-extensions}

\noindent
An important class of extensions are those where every element generates
a finite extension.

\begin{definition}
\label{definition-algebraic}
Consider a field extension $F/E$. An element $\alpha \in F$ is said to be
{\it algebraic} over $E$ if $\alpha$ is the root of some nonzero polynomial
with coefficients in $E$. If all elements of $F$ are algebraic then $F$ is
said to be an {\it algebraic extension} of $E$.
\end{definition}

\noindent
By Lemma \ref{lemma-field-extension-generated-by-one-element}, the
subextension $E(\alpha)$ is isomorphic either to the rational function
field $E(t)$ or to a quotient ring $E[t]/(P)$ for $P \in E[t]$ an
irreducible polynomial. In the latter case, $\alpha$ is algebraic over
$E$ (in fact, the proof of
Lemma \ref{lemma-field-extension-generated-by-one-element}
shows that we can pick $P$ such that $\alpha$ is a root of $P$);
in the former case, it is not.

\begin{example}
\label{example-C-algebraic-over-R}
The field $\mathbf{C}$ is algebraic over $\mathbf{R}$. Namely, if
$\alpha = a + ib$ in $\mathbf{C}$, then $\alpha^2 - 2a\alpha + a^2 + b^2 = 0$
is a polynomial equation for $\alpha$ over $\mathbf{R}$.
\end{example}

\begin{example}
\label{example-compact-riemann-surface-is-finite-over-P1}
Let $X$ be a compact Riemann surface, and let
$f \in \mathbf{C}(X) - \mathbf{C}$ any nonconstant meromorphic function
on $X$ (see Example \ref{example-field-of-meromorphic-functions}). Then it is
known that $\mathbf{C}(X)$ is algebraic over the subextension
$\mathbf{C}(f)$ generated by $f$. We shall not prove this.
\end{example}

\begin{lemma}
\label{lemma-algebraic-goes-up}
Let $K/E/F$ be a tower of field extensions.
\begin{enumerate}
\item If $\alpha \in K$ is algebraic over $F$, then $\alpha$ is algebraic
over $E$.
\item if $K$ is algebraic over $F$, then $K$ is algebraic over $E$.
\end{enumerate}
\end{lemma}

\begin{proof}
This is immediate from the definitions.
\end{proof}

\noindent
We now show that there is a deep connection between finiteness and being
algebraic.

\begin{lemma}
\label{lemma-finite-is-algebraic}
A finite extension is algebraic. In fact, an extension $E/k$ is algebraic
if and only if every subextension $k(\alpha)/k$ generated by some
$\alpha \in E$ is finite.
\end{lemma}

\noindent
In general, it is very false that an algebraic extension is finite.

\begin{proof}
Let $E/k$ be finite, say of degree $n$. Choose $\alpha \in E$. Then the
elements $\{1, \alpha, \ldots, \alpha^n\}$ are linearly
dependent over $E$, or we would necessarily have $[E : k] > n$. A relation of
linear dependence now gives the desired polynomial that $\alpha$ must satisfy.

\medskip\noindent
For the last assertion, note that a monogenic extension $k(\alpha)/k$ is
finite if and only $\alpha$ is algebraic over $k$, by
Examples \ref{example-degree-rational-function-field} and
\ref{example-degree-simple-algebraic-extension}.
So if $E/k$ is algebraic, then each $k(\alpha)/k$, $\alpha \in E$, is a finite
extension, and conversely.
\end{proof}

\noindent
We can extract a lemma of the last proof (really of
Examples \ref{example-degree-rational-function-field} and
\ref{example-degree-simple-algebraic-extension}):
a monogenic extension is finite if and only if it is algebraic.
We shall use this observation in the next result.

\begin{lemma}
\label{lemma-algebraic-finitely-generated}
\begin{slogan}
A finitely generated algebraic extension is finite.
\end{slogan}
Let $k$ be a field, and let $\alpha_1, \alpha_2, \ldots, \alpha_n$ be elements
of some extension field such that each $\alpha_i$ is algebraic over $k$. Then
the extension $k(\alpha_1, \ldots, \alpha_n)/k$ is finite.
That is, a finitely generated algebraic extension is finite.
\end{lemma}

\begin{proof}
Indeed, each extension
$k(\alpha_{1}, \ldots, \alpha_{i+1})/k(\alpha_1, \ldots, \alpha_{i})$
is generated by one element and algebraic, hence finite.
By multiplicativity of degree (Lemma \ref{lemma-multiplicativity-degrees})
we obtain the result.
\end{proof}

\noindent
The set of complex numbers that are algebraic over $\mathbf{Q}$ are simply
called the {\it algebraic numbers.} For instance, $\sqrt{2}$ is algebraic,
$i$ is algebraic, but $\pi$ is not.
It is a basic fact that the algebraic numbers form a field, although it is not
obvious how to prove this from the definition that a number is algebraic
precisely when it satisfies a nonzero polynomial equation with rational
coefficients (e.g. by polynomial equations).

\begin{lemma}
\label{lemma-algebraic-elements}
Let $E/k$ be a field extension. Then the elements of $E$ algebraic over $k$
form a subextension of $E/k$.
\end{lemma}

\begin{proof}
Let $\alpha, \beta \in E$ be algebraic over $k$. Then $k(\alpha, \beta)/k$
is a finite extension by Lemma \ref{lemma-algebraic-finitely-generated}.
It follows that $k(\alpha + \beta) \subset k(\alpha, \beta)$ is a finite
extension, which implies that $\alpha + \beta$ is algebraic by
Lemma \ref{lemma-finite-is-algebraic}. Similarly for the difference,
product and quotient of $\alpha$ and $\beta$.
\end{proof}

\noindent
Many nice properties of field extensions, like those of rings, will have the
property that they will be preserved by towers and composita.

\begin{lemma}
\label{lemma-algebraic-permanence}
Let $E/k$ and $F/E$ be algebraic extensions of fields. Then $F/k$ is an
algebraic extension of fields.
\end{lemma}

\begin{proof}
Choose $\alpha \in F$. Then $\alpha$ is algebraic over $E$.
The key observation is that $\alpha$ is algebraic over a
finitely generated subextension of $k$.
That is, there is a finite set $S \subset E$ such that $\alpha $ is algebraic
over $k(S)$: this is clear because being algebraic means that a certain
polynomial in $E[x]$ that $\alpha$ satisfies exists, and as $S$ we can take the
coefficients of this polynomial. It follows that $\alpha$ is algebraic over
$k(S)$. In particular, the extension $k(S, \alpha)/ k(S)$ is finite.
Since $S$ is a finite set, and $k(S)/k$ is algebraic,
Lemma \ref{lemma-algebraic-finitely-generated} shows that
$k(S)/k$ is finite. Using multiplicativity
(Lemma \ref{lemma-multiplicativity-degrees})
we find that $k(S,\alpha)/k$ is finite, so $\alpha$ is algebraic over $k$.
\end{proof}

\noindent
The method of proof in the previous argument --- that being algebraic
over $E$ was a property that {\it descended} to a finitely generated
subextension of $E$ --- is an idea that recurs throughout algebra.
It often allows one to reduce general commutative algebra questions
to the Noetherian case for example.

\begin{lemma}
\label{lemma-size-algebraic-extension}
Let $E/F$ be an algebraic extension of fields. Then the cardinality $|E|$
of $E$ is at most $\max(\aleph_0, |F|)$.
\end{lemma}

\begin{proof}
Let $S$ be the set of nonconstant polynomials with coefficients in $F$.
For every $P \in S$ the set of roots
$r(P, E) = \{\alpha \in E \mid P(\alpha) = 0\}$
is finite (details omitted). Moreover, the fact that $E$ is algebraic
over $F$ implies that $E = \bigcup_{P \in S} r(P, E)$.
It is clear that $S$ has cardinality bounded by $\max(\aleph_0, |F|)$
because the cardinality of a finite product of copies of $F$ has
cardinality at most $\max(\aleph_0, |F|)$.
Thus so does $E$.
\end{proof}

\begin{lemma}
\label{lemma-subalgebra-algebraic-extension-field}
Let $E/F$ be a finite or more generally an algebraic extension of fields.
Any subring $F \subset R \subset E$ is a field.
\end{lemma}

\begin{proof}
Let $\alpha \in R$ be nonzero. Then $1, \alpha, \alpha^2, \ldots$
are contained in $R$. By Lemma \ref{lemma-finite-is-algebraic}
we find a nontrivial relation
$a_0 + a_1 \alpha + \ldots + a_d \alpha^d = 0$.
We may assume $a_0 \not = 0$ because if not we can divide the relation
by $\alpha$ to decrease $d$. Then we see that
$$
a_0 = \alpha (- a_1  - \ldots - a_d \alpha^{d - 1})
$$
which proves that the inverse of $\alpha$ is the element
$a_0^{-1} (- a_1  - \ldots - a_d \alpha^{d - 1})$
of $R$.
\end{proof}

\begin{lemma}
\label{lemma-algebraic-extension-self-map}
Let $E/F$ an algebraic extension of fields. Any $F$-algebra map
$f : E \to E$ is an automorphism.
\end{lemma}

\begin{proof}
If $E/F$ is finite, then $f : E \to E$ is an $F$-linear 
injective map (Lemma \ref{lemma-field-maps-injective})
of finite dimensional vector spaces, and hence bijective.
In general we still see that $f$ is injective.
Let $\alpha \in E$ and let $P \in F[x]$ be a
polynomial such that $P(\alpha) = 0$.
Let $E' \subset E$ be the subfield of $E$ generated
by the roots $\alpha = \alpha_1, \ldots, \alpha_n$ of $P$ in $E$.
Then $E'$ is finite over $F$ by Lemma \ref{lemma-algebraic-finitely-generated}.
Since $f$ preserves the set of roots, we find that
$f|_{E'} : E' \to E'$. Hence $f|_{E'}$ is an isomorphism
by the first part of the proof and we conclude that $\alpha$
is in the image of $f$.
\end{proof}






\section{Minimal polynomials}
\label{section-minimal-polynomials}

\noindent
Let $E/k$ be a field extension, and let $\alpha \in E$ be algebraic over $k$.
Then $\alpha$ satisfies a (nontrivial) polynomial equation in $k[x]$.
Consider the set of polynomials $P \in k[x]$ such that $P(\alpha) = 0$; by
hypothesis, this set does not just contain the zero polynomial.
It is easy to see that this set is an {\it ideal.} Indeed, it is the kernel
of the map
$$
k[x] \to E, \quad x \mapsto \alpha
$$
Since $k[x]$ is a PID, there is a {\it generator} $P \in k[x]$ of this
ideal. If we assume $P$ monic, without loss of generality, then $P$ is
uniquely determined.

\begin{definition}
\label{definition-minimal-polynomial}
The polynomial $P$ above is called the {\it minimal polynomial}
of $\alpha$ over $k$.
\end{definition}

\noindent
The minimal polynomial has the following characterization: it is the monic
polynomial, of smallest degree, that annihilates $\alpha$. Any nonconstant
multiple of $P$ will have larger degree, and only multiples of $P$ can
annihilate $\alpha$. This explains the name {\it minimal}.

\medskip\noindent
Clearly the minimal polynomial is {\it irreducible}. This is equivalent to the
assertion that the ideal in $k[x]$ consisting of polynomials annihilating
$\alpha$ is prime. This follows from the fact that the map
$k[x] \to E, x \mapsto \alpha$ is a map into a domain (even a field), so the
kernel is a prime ideal.

\begin{lemma}
\label{lemma-degree-minimal-polynomial}
The degree of the minimal polynomial is $[k(\alpha) : k]$.
\end{lemma}

\begin{proof}
This is just a restatement of the argument in
Lemma \ref{lemma-field-extension-generated-by-one-element}: the observation
is that if $P$ is the minimal polynomial of $\alpha$, then the map
$$
k[x]/(P) \to k(\alpha), \quad x \mapsto \alpha
$$
is an isomorphism as in the aforementioned proof, and we have counted the
degree of such an extension (see
Example \ref{example-degree-simple-algebraic-extension}).
\end{proof}

\noindent
So the observation of the above proof is that if $\alpha \in E$ is algebraic,
then $k(\alpha) \subset E$ is isomorphic to $k[x]/(P)$.


\section{Algebraic closure}
\label{section-algebraic-closure}

\noindent
The ``fundamental theorem of algebra'' states that $\mathbf{C}$ is
algebraically closed. A beautiful proof of this result uses
Liouville's theorem in complex analysis, we shall give another
proof (see Lemma \ref{lemma-C-algebraically-closed}).

\begin{definition}
\label{definition-algebraically-closed}
A field $F$ is said to be {\it algebraically closed} if every algebraic
extension $E/F$ is trivial, i.e., $E = F$.
\end{definition}

\noindent
This may not be the definition in every text. Here is the lemma comparing
it with the other one.

\begin{lemma}
\label{lemma-algebraically-closed}
Let $F$ be a field. The following are equivalent
\begin{enumerate}
\item $F$ is algebraically closed,
\item every irreducible polynomial over $F$ is linear,
\item every nonconstant polynomial over $F$ has a root,
\item every nonconstant polynomial over $F$ is a product of linear factors.
\end{enumerate}
\end{lemma}

\begin{proof}
If $F$ is algebraically closed, then every irreducible polynomial is linear.
Namely, if there exists an irreducible polynomial of degree $> 1$, then
this generates a nontrivial finite (hence algebraic) field extension, see
Example \ref{example-degree-simple-algebraic-extension}.
Thus (1) implies (2). If every irreducible polynomial
is linear, then every irreducible polynomial has a root, whence every
nonconstant polynomial has a root. Thus (2) implies (3).

\medskip\noindent
Assume every nonconstant polynomial has a root. Let $P \in F[x]$
be nonconstant. If $P(\alpha) = 0$ with $\alpha \in F$, then we see
that $P = (x - \alpha)Q$ for some $Q \in F[x]$ (by division with remainder).
Thus we can argue by induction on the degree that any nonconstant
polynomial can be written as a product $c \prod (x - \alpha_i)$.

\medskip\noindent
Finally, suppose that every nonconstant polynomial over $F$ is a product of
linear factors. Let $E/F$ be an algebraic extension. Then all the simple
subextensions $F(\alpha)/F$ of $E$ are necessarily trivial (because the
only irreducible polynomials are linear by assumption). Thus $E = F$.
We see that (4) implies (1) and we are done.
\end{proof}

\noindent
Now we want to define a ``universal'' algebraic extension of a field.
Actually, we should be careful: the algebraic closure is {\it not} a
universal object. That is, the algebraic closure is not unique up to
{\it unique} isomorphism: it is only unique up to isomorphism. But still,
it will be very handy, if not functorial.

\begin{definition}
\label{definition-algebraic-closure}
Let $F$ be a field. We say $F$ is {\it algebraically closed} if every
algebraic extension $E/F$ is trivial, i.e., $E = F$. An {\it algebraic closure}
of $F$ is a field $\overline{F}$ containing $F$ such that:
\begin{enumerate}
\item $\overline{F}$ is algebraic over $F$.
\item $\overline{F}$ is algebraically closed.
\end{enumerate}
\end{definition}

\noindent
If $F$ is algebraically closed, then $F$ is its own algebraic closure.
We now prove the basic existence result.

\begin{theorem}
\label{theorem-existence-algebraic-closure}
Every field has an algebraic closure.
\end{theorem}

\noindent
The proof will mostly be a red herring to the rest of the chapter. However, we
will want to know that it is {\it possible} to embed a field inside an
algebraically closed field, and we will often assume it done.

\begin{proof}
Let $F$ be a field. By Lemma \ref{lemma-size-algebraic-extension} the
cardinality of an algebraic extension of $F$ is bounded by
$\max(\aleph_0, |F|)$. Choose a set $S$ containing $F$ with
$|S| > \max(\aleph_0, |F|)$. Let's consider triples
$(E, \sigma_E, \mu_E)$ where
\begin{enumerate}
\item $E$ is a set with $F \subset E \subset S$, and
\item $\sigma_E : E \times E \to E$ and $\mu_E : E \times E \to E$
are maps of sets such that $(E, \sigma_E, \mu_E)$ defines the structure
of a field extension of $F$ (in particular $\sigma_E(a, b) = a +_F b$
for $a, b \in F$ and similarly for $\mu_E$), and
\item $F \subset E$ is an algebraic field extension.
\end{enumerate}
The collection of all triples $(E, \sigma_E, \mu_E)$ forms a set $I$.
For $i \in I$ we will denote $E_i = (E_i, \sigma_i, \mu_i)$ the
corresponding field extension to $F$. We define a partial ordering on
$I$ by declaring $i \leq i'$ if and only if $E_i \subset E_{i'}$
(this makes sense as $E_i$ and $E_{i'}$ are subsets of the same set $S$)
and we have $\sigma_i = \sigma_{i'}|_{E_i \times E_i}$ and
$\mu_i = \mu_{i'}|_{E_i \times E_i}$, in other words, $E_{i'}$ is a field
extension of $E_i$.

\medskip\noindent
Let $T \subset I$ be a totally ordered subset. Then it is clear that
$E_T = \bigcup_{i \in T} E_i$ with induced maps $\sigma_T = \bigcup \sigma_i$
and $\mu_T = \bigcup \mu_i$ is another element of $I$. In other words
every totally order subset of $I$ has a upper bound in $I$. By Zorn's lemma
there exists a maximal element $(E, \sigma_E, \mu_E)$ in $I$. We claim that
$E$ is an algebraic closure. Since by definition of $I$ the extension
$E/F$ is algebraic, it suffices to show that $E$ is algebraically closed.

\medskip\noindent
To see this we argue by contradiction. Namely, suppose that $E$ is not
algebraically closed. Then there exists an irreducible polynomial
$P$ over $E$ of degree $> 1$, see Lemma \ref{lemma-algebraically-closed}.
By Lemma \ref{lemma-finite-is-algebraic} we obtain a nontrivial finite
extension $E' = E[x]/(P)$. Observe that $E'/F$ is algebraic by
Lemma \ref{lemma-algebraic-permanence}.
Thus the cardinality of $E'$ is $\leq \max(\aleph_0, |F|)$.
By elementary set theory we can extend the given injection
$E \subset S$ to an injection $E' \to S$. In other words, we may
think of $E'$ as an element of our set $I$ contradicting the
maximality of $E$. This contradiction completes the proof.
\end{proof}

\begin{lemma}
\label{lemma-map-into-algebraic-closure}
Let $F$ be a field. Let $\overline{F}$ be an algebraic closure of $F$.
Let $M/F$ be an algebraic extension. Then there is a morphism of
$F$-extensions $M \to \overline{F}$.
\end{lemma}

\begin{proof}
Consider the set $I$ of pairs $(E, \varphi)$ where $F \subset E \subset M$
is a subextension and $\varphi : E \to \overline{F}$ is a morphism of
$F$-extensions. We partially order the set $I$ by declaring
$(E, \varphi) \leq (E', \varphi')$ if and only if $E \subset E'$ and
$\varphi'|_E = \varphi$. If $T = \{(E_t,  \varphi_t)\} \subset I$
is a totally ordered subset, then
$\bigcup \varphi_t : \bigcup E_t \to \overline{F}$ is an element of $I$.
Thus every totally ordered subset of $I$ has an upper bound.
By Zorn's lemma there exists a maximal element $(E, \varphi)$ in $I$.
We claim that $E = M$, which will finish the proof. If not, then
pick $\alpha \in M$, $\alpha \not \in E$. The $\alpha$ is algebraic
over $E$, see Lemma \ref{lemma-algebraic-goes-up}.
Let $P$ be the minimal polynomial of $\alpha$ over $E$.
Let $P^\varphi$ be the image of $P$ by $\varphi$ in $\overline{F}[x]$.
Since $\overline{F}$ is algebraically closed there is a root $\beta$
of $P^\varphi$ in $\overline{F}$. Then we can extend $\varphi$ to
$\varphi' : E(\alpha) = E[x]/(P) \to \overline{F}$ by mapping
$x$ to $\beta$. This contradicts the maximality of $(E, \varphi)$
as desired.
\end{proof}

\begin{lemma}
\label{lemma-algebraic-closures-isomorphic}
Any two algebraic closures of a field are isomorphic.
\end{lemma}

\begin{proof}
Let $F$ be a field. If $M$ and $\overline{F}$ are algebraic closures of
$F$, then there exists a morphism of $F$-extensions
$\varphi : M \to \overline{F}$ by
Lemma \ref{lemma-map-into-algebraic-closure}.
Now the image $\varphi(M)$ is algebraically closed.
On the other hand, the extension $\varphi(M) \subset \overline{F}$
is algebraic by Lemma \ref{lemma-algebraic-goes-up}.
Thus $\varphi(M) = \overline{F}$.
\end{proof}





\section{Relatively prime polynomials}
\label{section-relatively-prime}

\noindent
Let $K$ be an algebraically closed field. Then the ring $K[x]$ has a very
simple ideal structure as we saw in Lemma \ref{lemma-algebraically-closed}.
In particular, every polynomial $P \in K[x]$ can be written as
$$
P = c(x - \alpha_1) \ldots (x - \alpha_n),
$$
where $c$ is the constant term and the $\alpha_1, \ldots, \alpha_n \in k$
are the roots of $P$ (counted with multiplicity). Clearly, the only irreducible
polynomials in $K[x]$ are the linear polynomials $c(x - \alpha)$,
$c, \alpha \in K$ (and $c \neq 0$).

\begin{definition}
\label{definition-relatively-prime}
If $k$ is any field, we say that two polynomials in $k[x]$ are
{\it relatively prime} if they generate the unit ideal in $k[x]$.
\end{definition}

\noindent
Continuing the discussion above, if $K$ is an algebraically closed field,
two polynomials in $K[x]$ are relatively prime if and only if they have no
common roots. This follows because the maximal ideals of $K[x]$ are of the form
$(x - \alpha)$, $\alpha \in K$. So if $F, G \in K[x]$ have no common root,
then $(F, G)$ cannot be contained in any $(x - \alpha)$ (as then they would
have a common root at $\alpha$).

\medskip\noindent
If $k$ is {\it not} algebraically closed, then this still gives
information about when two polynomials in $k[x]$ generate the unit ideal.

\begin{lemma}
\label{lemma-relatively-prime-polynomials}
Two polynomials in $k[x]$ are relatively prime precisely when they
have no common roots in an algebraic closure $\overline{k}$ of $k$.
\end{lemma}

\begin{proof}
The claim is that any two polynomials $P, Q$ generate $(1)$ in $k[x]$ if and
only if they generate $(1)$ in $\overline{k}[x]$. This is a piece of
linear algebra: a system of linear equations with coefficients in $k$ has
a solution if and only if it has a solution in any extension of $k$.
Consequently, we can reduce to the case of an algebraically closed field, in
which case the result is clear from what we have already proved.
\end{proof}





\section{Separable extensions}
\label{section-separable-extensions}

\noindent
In characteristic $p$ something funny happens with irreducible polynomials
over fields. We explain this in the following lemma.

\begin{lemma}
\label{lemma-irreducible-polynomials}
Let $F$ be a field. Let $P \in F[x]$ be an irreducible polynomial over $F$.
Let $P' = \text{d}P/\text{d}x$ be the derivative of $P$ with respect
to $x$. Then one of the following two cases happens
\begin{enumerate}
\item $P$ and $P'$ are relatively prime, or
\item $P'$ is the zero polynomial.
\end{enumerate}
Then second case can only happen if $F$ has characteristic $p > 0$.
In this case $P(x) = Q(x^q)$ where $q = p^f$ is a power of $p$ and
$Q \in F[x]$ is an irreducible polynomial such that $Q$ and $Q'$
are relatively prime.
\end{lemma}

\begin{proof}
Note that $P'$ has degree $< \deg(P)$. Hence if $P$ and $P'$ are not relatively
prime, then $(P, P') = (R)$ where $R$ is a polynomial of degree $< \deg(P)$
contradicting the irreducibility of $P$. This proves we have the dichotomy
between (1) and (2).

\medskip\noindent
Assume we are in case (2) and $P = a_d x^d + \ldots + a_0$. Then
$P' = da_d x^{d - 1} + \ldots + a_1$. In characteristic $0$ we see
that this forces $a_d, \ldots, a_1 = 0$ which would mean $P$ is constant
a contradiction. Thus we conclude that the characteristic $p$ is positive.
In this case the condition $P' = 0$ forces $a_i = 0$ whenever $p \not | i$.
In other words, $P(x) = P_1(x^p)$ for some nonconstant polynomial $P_1$.
Clearly, $P_1$ is irreducible as well. By induction on the degree we
see that $P_1(x) = Q(x^q)$ as in the statement of the lemma, hence
$P(x) = Q(x^{pq})$ and the lemma is proved.
\end{proof}

\begin{definition}
\label{definition-separable}
Let $F$ be a field. Let $K/F$ be an extension of fields.
\begin{enumerate}
\item We say an irreducible polynomial $P$ over $F$ is {\it separable}
if it is relatively prime to its derivative.
\item Given $\alpha \in K$ algebraic over $F$ we say $\alpha$ is
{\it separable} over $F$ if its minimal polynomial is separable over $F$.
\item If $K$ is an algebraic extension of $F$, we say $K$ is
{\it separable}\footnote{For nonalgebraic extensions
this definition does not make sense and is not the correct one.}
over $F$ if every element of $K$ is separable over $F$.
\end{enumerate}
\end{definition}

\noindent
By Lemma \ref{lemma-irreducible-polynomials} in characteristic $0$ every
irreducible polynomial is separable, every algebraic element in an extension
is separable, and every algebraic extension is separable.

\begin{lemma}
\label{lemma-separable-goes-up}
Let $K/E/F$ be a tower of algebraic field extensions.
\begin{enumerate}
\item If $\alpha \in K$ is separable over $F$, then $\alpha$ is separable
over $E$.
\item if $K$ is separable over $F$, then $K$ is separable over $E$.
\end{enumerate}
\end{lemma}

\begin{proof}
We will use Lemma \ref{lemma-irreducible-polynomials} without further mention.
Let $P$ be the minimal polynomial of $\alpha$ over $F$.
Let $Q$ be the minimal polynomial of $\alpha$ over $E$.
Then $Q$ divides $P$ in the polynomial ring $E[x]$, say $P = QR$.
Then $P' = Q'R + QR'$. Thus if $Q' = 0$, then $Q$ divides $P$ and $P'$
hence $P' = 0$ by the lemma. This proves (1). Part (2)
follows immediately from (1) and the definitions.
\end{proof}

\begin{lemma}
\label{lemma-recognize-separable}
Let $F$ be a field. An irreducible polynomial $P$ over $F$
is separable if and only if $P$ has pairwise distinct roots in an
algebraic closure of $F$.
\end{lemma}

\begin{proof}
Suppose that $\alpha \in F$ is a root of both $P$ and $P'$.
Then $P = (x - \alpha)Q$ for some polynomial $Q$. Taking derivatives
we obtain $P' = Q + (x - \alpha)Q'$. Thus $\alpha$ is a root of $Q$.
Hence we see that if $P$ and $P'$ have a common root, then $P$
does not have pairwise distinct roots. Conversely, if $P$ has
a repeated root, i.e., $(x - \alpha)^2$ divides $P$, then $\alpha$
is a root of both $P$ and $P'$. Combined with
Lemma \ref{lemma-relatively-prime-polynomials} this proves the lemma.
\end{proof}

\begin{lemma}
\label{lemma-nr-roots-unchanged}
Let $F$ be a field and let $\overline{F}$ be an algebraic closure of $F$.
Let $p > 0$ be the characteristic of $F$. Let $P$ be a polynomial
over $F$. Then the set of roots of $P$ and $P(x^p)$ in $\overline{F}$
have the same cardinality (not counting multiplicity).
\end{lemma}

\begin{proof}
Clearly, $\alpha$ is a root of $P(x^p)$ if and only if $\alpha^p$ is a
root of $P$. In other words, the roots of $P(x^p)$ are the roots of
$x^p - \beta$, where $\beta$ is a root of $P$. Thus it suffices to show
that the map $\overline{F} \to \overline{F}$, $\alpha \mapsto \alpha^p$
is bijective. It is surjective, as $\overline{F}$ is algebraically closed
which means that every element has a $p$th root. It is injective because
$\alpha^p = \beta^p$ implies $(\alpha - \beta)^p = 0$ because
the characteristic is $p$. And of course in a field $x^p = 0$ implies
$x = 0$.
\end{proof}

\noindent
Let $F$ be a field and let $P$ be an irreducible polynomial over $F$.
Then we know that $P = Q(x^q)$ for some separable irreducible polynomial $Q$
(Lemma \ref{lemma-irreducible-polynomials}) where $q$ is a power of
the characteristic $p$ (and if the characteristic is zero, then
$q = 1$\footnote{A good convention for this chapter is to set $0^0 = 1$.}
and $Q = P$). By Lemma \ref{lemma-nr-roots-unchanged} the number of
roots of $P$ and $Q$ in any algebraic closure of $F$ is the same.
By Lemma \ref{lemma-recognize-separable} this number is equal to the degree
of $Q$.

\begin{definition}
\label{definition-separable-degree}
Let $F$ be a field. Let $P$ be an irreducible polynomial over $F$.
The {\it separable degree} of $P$ is the cardinality of the
set of roots of $P$ in any algebraic closure of $F$ (see discussion
above). Notation $\deg_s(P)$.
\end{definition}

\noindent
The separable degree of $P$ always divides the degree and the quotient
is a power of the characteristic. If the characteristic is zero, then
$\deg_s(P) = \deg(P)$.

\begin{situation}
\label{situation-finitely-generated}
Here $F$ be a field and $K/F$ is a finite extension generated by elements
$\alpha_1, \ldots, \alpha_n \in K$. We set $K_0 = F$ and
$$
K_i = F(\alpha_1, \ldots, \alpha_i)
$$
to obtain a tower of finite extensions
$K = K_r / K_{r - 1} / \ldots / K_0 = F$.
Denote $P_i$ the minimal polynomial of $\alpha_i$ over $K_{i - 1}$.
Finally, we fix an algebraic closure $\overline{F}$ of $F$.
\end{situation}

\noindent
Let $F$, $K$, $\alpha_i$, and $\overline{F}$ be as in
Situation \ref{situation-finitely-generated}.
Suppose that $\varphi : K \to \overline{F}$ is a morphism of extensions
of $F$. Then we obtain maps $\varphi_i : K_i \to \overline{F}$.
In particular, we can take the image of $P_i \in K_{i - 1}[x]$ by
$\varphi_{i - 1}$ to get a polynomial $P_i^\varphi \in \overline{F}[x]$.

\begin{lemma}
\label{lemma-count-embeddings}
In Situation \ref{situation-finitely-generated} the correspondence
$$
\Mor_F(K, \overline{F})
\longrightarrow
\{(\beta_1, \ldots, \beta_n)\text{ as below}\},
\quad
\varphi \longmapsto (\varphi(\alpha_1), \ldots, \varphi(\alpha_n))
$$
is a bijection. Here the right hand side is the set of $n$-tuples
$(\beta_1, \ldots, \beta_n)$ of elements of $\overline{F}$
such that $\beta_i$ is a root of $P_i^\varphi$.
\end{lemma}

\begin{proof}
Let $(\beta_1, \ldots, \beta_n)$ be an element of the right hand side.
We construct a map of fields corresponding to it by induction.
Namely, we set $\varphi_0 : K_0 \to \overline{F}$ equal to the given
map $K_0 = F \subset \overline{F}$. Having constructed
$\varphi_{i - 1} : K_{i - 1} \to \overline{F}$ we observe that
$K_i = K_{i - 1}[x]/(P_i)$. Hence we can set $\varphi_i$ equal
to the unique map $K_i \to \overline{F}$ inducing $\varphi_{i - 1}$
on $K_{i - 1}$ and mapping $x$ to $\beta_i$. This works precisely
as $\beta_i$ is a root of $P_i^\varphi$. Uniqueness implies that
the two constructions are mutually inverse.
\end{proof}

\begin{lemma}
\label{lemma-count-embeddings-explicitly}
In Situation \ref{situation-finitely-generated} we have
$|\Mor_F(K, \overline{F})| = \prod_{i = 1}^n \deg_s(P_i)$.
\end{lemma}

\begin{proof}
This follows immediately from Lemma \ref{lemma-count-embeddings}.
Observe that a key ingredient we are tacitly using here is the
well-definedness of the separable degree of an irreducible polynomial
which was observed just prior to
Definition \ref{definition-separable-degree}.
\end{proof}

\noindent
We now use the result above to characterize separable field extensions.

\begin{lemma}
\label{lemma-separably-generated-separable}
Assumptions and notation as in Situation \ref{situation-finitely-generated}.
If each $P_i$ is separable, i.e., each $\alpha_i$ is separable over
$K_{i - 1}$, then
$$
|\Mor_F(K, \overline{F})| = [K : F]
$$
and the field extension $K/F$ is separable. If one of the $\alpha_i$ is
not separable over $K_{i - 1}$, then
$|\Mor_F(K, \overline{F})| < [K : F]$.
\end{lemma}

\begin{proof}
If $\alpha_i$ is separable over $K_{i - 1}$ then
$\deg_s(P_i) = \deg(P_i) = [K_i : K_{i - 1}]$
(last equality by Lemma \ref{lemma-degree-minimal-polynomial}).
By multiplicativity (Lemma \ref{lemma-multiplicativity-degrees}) we have
$$
[K : F] = \prod [K_i : K_{i - 1}] = \prod \deg(P_i) =
\prod \deg_s(P_i) = |\Mor_F(K, \overline{F})|
$$
where the last equality is Lemma \ref{lemma-count-embeddings-explicitly}.
By the exact same argument we get the strict inequality
$|\Mor_F(K, \overline{F})| < [K : F]$ if one of the $\alpha_i$ is
not separable over $K_{i - 1}$.

\medskip\noindent
Finally, assume again that each $\alpha_i$ is separable over $K_{i - 1}$.
Let $\gamma = \gamma_1 \in K$ be arbitrary. Then we can find additional
elements $\gamma_2, \ldots, \gamma_m$ such that
$K = F(\gamma_1, \ldots, \gamma_m)$ (for example we could take
$\gamma_2 = \alpha_1, \ldots, \gamma_{n + 1} = \alpha_n$).
Then we see by the last part of the lemma (already proven above)
that if $\gamma$ is not separable over $F$ we would have the
strict inequality $|\Mor_F(K, \overline{F})| < [K : F]$
contradicting the very first part of the lemma (already prove above
as well).
\end{proof}

\begin{lemma}
\label{lemma-separable-equality}
Let $K/F$ be a finite extension of fields. Let $\overline{F}$ be an
algebraic closure of $F$. Then we have
$$
|\Mor_F(K, \overline{F})| \leq [K : F]
$$
with equality if and only if $K$ is separable over $F$.
\end{lemma}

\begin{proof}
This is a corollary of Lemma \ref{lemma-separably-generated-separable}.
Namely, since $K/F$ is finite we can find finitely many elements
$\alpha_1, \ldots, \alpha_n \in K$ generating $K$ over $F$ (for example
we can choose the $\alpha_i$ to be a basis of $K$ over $F$).
If $K/F$ is separable, then each $\alpha_i$ is separable over
$F(\alpha_1, \ldots, \alpha_{i - 1})$ by Lemma \ref{lemma-separable-goes-up}
and we get equality by Lemma \ref{lemma-separably-generated-separable}.
On the other hand, if we have equality, then no matter how we choose
$\alpha_1, \ldots, \alpha_n$ we get that $\alpha_1$ is separable over
$F$ by Lemma \ref{lemma-separably-generated-separable}. Since we
can start the sequence with an arbitrary element of $K$ it follows
that $K$ is separable over $F$.
\end{proof}

\begin{lemma}
\label{lemma-separable-permanence}
Let $E/k$ and $F/E$ be separable algebraic extensions of fields. Then $F/k$
is a separable extension of fields.
\end{lemma}

\begin{proof}
Choose $\alpha \in F$. Then $\alpha$ is separable algebraic over $E$.
Let $P = x^d + \sum_{i < d} a_i x^i$ be the minimal polynomial of
$\alpha$ over $E$. Each $a_i$ is separable algebraic over $k$.
Consider the tower of fields
$$
k \subset k(a_0) \subset k(a_0, a_1) \subset \ldots \subset
k(a_0, \ldots, a_{d - 1}) \subset k(a_0, \ldots, a_{d - 1}, \alpha)
$$
Because $a_i$ is separable algebraic over $k$ it is separable algebraic
over $k(a_0, \ldots, a_{i - 1})$ by Lemma \ref{lemma-separable-goes-up}.
Finally, $\alpha$ is separable algebraic over $k(a_0, \ldots, a_{d - 1})$
because it is a root of $P$ which is irreducible
(as it is irreducible over the possibly bigger field $E$)
and separable (as it is separable over $E$).
Thus $k(a_0, \ldots, a_{d - 1}, \alpha)$ is separable over $k$
by Lemma \ref{lemma-separably-generated-separable}
and we conclude that $\alpha$ is separable over $k$ as desired.
\end{proof}

\begin{lemma}
\label{lemma-separable-elements}
Let $E/k$ be a field extension. Then the elements of $E$ separable
over $k$ form a subextension of $E/k$.
\end{lemma}

\begin{proof}
Let $\alpha, \beta \in E$ be separable over $k$. Then $\beta$ is separable
over $k(\alpha)$ by Lemma \ref{lemma-separable-goes-up}.
Thus we can apply Lemma \ref{lemma-separable-permanence} to $k(\alpha, \beta)$
to see that $k(\alpha, \beta)$ is separable over $k$.
\end{proof}






\section{Purely inseparable extensions}
\label{section-purely-inseparable}

\noindent
Purely inseparable extensions are the opposite of the separable
extensions defined in the previous section. These extensions only
show up in positive characteristic.

\begin{definition}
\label{definition-purely-inseparable}
Let $F$ be a field of characteristic $p > 0$. Let $K/F$ be an extension.
\begin{enumerate}
\item An element $\alpha \in K$ is {\it purely inseparable} over $F$
if there exists a power $q$ of $p$ such that $\alpha^q \in F$.
\item The extension $K/F$ is said to be {\it purely inseparable}
if and only if every element of $K$ is purely inseparable over $F$.
\end{enumerate}
\end{definition}

\noindent
Observe that a purely inseparable extension is necessarily algebraic.
Let $F$ be a field of characteristic $p > 0$.
An example of a purely inseparable extension is gotten by adjoining
the $p$th root of an element $t \in F$ which does not yet have one. Namely,
the lemma below shows that $P = x^p - t$ is irreducible, and hence
$$
K = F[x]/(P) = F[t^{1/p}]
$$
is a field. And $K$ is purely inseparable over $F$ because every element
$$
a_0 + a_1t^{1/p} + \ldots + a_{p - 1}t^{p - 1/p}, a_i \in F
$$
has $p$th power equal to
$$
(a_0 + a_1t^{1/p} + \ldots + a_{p - 1}t^{p - 1/p})^p =
a_0^p + a_1^p t + \ldots + a_{p - 1}^pt^{p - 1} \in F
$$
This situation occurs for the field
$\mathbf{F}_p(t)$ of rational functions over $\mathbf{F}_p$.

\begin{lemma}
\label{lemma-take-pth-root}
Let $p$ be a prime number. Let $F$ be a field of characteristic $p$.
Let $t \in F$ be an element which does not have a $p$th root in $F$.
Then the polynomial $x^p - t$ is irreducible over $F$.
\end{lemma}

\begin{proof}
To see this, suppose that we have a factorization
$x^p - t = f g$. Taking derivatives we get $f' g + f g' = 0$.
Note that neither $f' = 0$ nor $g' = 0$ as the degrees of $f$ and $g$
are smaller than $p$. Moreover, $\deg(f') < \deg(f)$ and $\deg(g') < \deg(g)$.
We conclude that $f$ and $g$ have a factor in common. Thus if $x^p - t$
is reducible, then it is of the form $x^p - t = c f^n$ for some irreducible
$f$, $c \in F^*$, and $n > 1$. Since $p$ is a prime number this
implies $n = p$ and $f$ linear, which would imply $x^p - t$ has a root
in $F$. Contradiction.
\end{proof}

\noindent
We will see that taking $p$th roots is a very important operation in
characteristic $p$.

\begin{lemma}
\label{lemma-purely-inseparable-permanence}
Let $E/k$ and $F/E$ be purely inseparable extensions of fields. Then $F/k$
is a purely inseparable extension of fields.
\end{lemma}

\begin{proof}
Say the characteristic of $k$ is $p$. Choose $\alpha \in F$. Then
$\alpha^q \in E$ for some $p$-power $q$. Whereupon $(\alpha^q)^{q'} \in k$
for some $p$-power $q'$. Hence $\alpha^{qq'} \in k$.
\end{proof}

\begin{lemma}
\label{lemma-purely-inseparable-elements}
Let $E/k$ be a field extension. Then the elements of $E$ purely-inseparable
over $k$ form a subextension of $E/k$.
\end{lemma}

\begin{proof}
Let $p$ be the characteristic of $k$.
Let $\alpha, \beta \in E$ be purely inseparable over $k$. Say
$\alpha^q \in k$ and $\beta^{q'} \in k$ for some $p$-powers $q, q'$.
If $q''$ is a $p$-power, then
$(\alpha + \beta)^{q''} = \alpha^{q''} + \beta^{q''}$.
Hence if $q'' \geq q, q'$, then we conclude that $\alpha + \beta$
is purely inseparable over $k$. Similarly for the difference,
product and quotient of $\alpha$ and $\beta$.
\end{proof}

\begin{lemma}
\label{lemma-finite-purely-inseparable}
Let $E/F$ be a finite purely inseparable field extension of
characteristic $p > 0$. Then there exists a sequence of elements
$\alpha_1, \ldots, \alpha_n \in E$ such that we obtain a tower
of fields
$$
E = F(\alpha_1, \ldots, \alpha_n) \supset
F(\alpha_1, \ldots, \alpha_{n - 1}) \supset
\ldots
\supset F(\alpha_1) \supset F
$$
such that each intermediate extension is of degree $p$ and comes
from adjoining a $p$th root. Namely,
$\alpha_i^p \in F(\alpha_1, \ldots, \alpha_{i - 1})$
is an element which does not have a $p$th root in
$F(\alpha_1, \ldots, \alpha_{i - 1})$ for $i = 1, \ldots, n$.
\end{lemma}

\begin{proof}
By induction on the degree of $E/F$. If the degree of the extension is $1$
then the result is clear (with $n = 0$). If not, then choose
$\alpha \in E$, $\alpha \not \in F$. Say $\alpha^{p^r} \in F$ for some
$r > 0$. Pick $r$ minimal and replace $\alpha$ by $\alpha^{p^{r - 1}}$.
Then $\alpha \not \in F$, but $\alpha^p \in F$. Then $t = \alpha^p$ is not
a $p$th power in $F$ (because that would imply $\alpha \in F$, see
Lemma \ref{lemma-nr-roots-unchanged} or its proof).
Thus $F \subset F(\alpha)$ is a subextension of degree $p$
(Lemma \ref{lemma-take-pth-root}). By induction we find
$\alpha_1, \ldots, \alpha_n \in E$ generating $E/F(\alpha)$
satisfying the conclusions of the lemma.
The sequence $\alpha, \alpha_1, \ldots, \alpha_n$ does the job
for the extension $E/F$.
\end{proof}

\begin{lemma}
\label{lemma-separable-first}
\begin{slogan}
Any algebraic field extension is uniquely a separable field extension
followed by a purely inseparable one.
\end{slogan}
Let $E/F$ be an algebraic field extension. There exists a unique subextension
$F \subset E_{sep} \subset E$ such that $E_{sep}/F$ is separable and
$E/E_{sep}$ is purely inseparable.
\end{lemma}

\begin{proof}
If the characteristic is zero we set $E_{sep} = E$. Assume the characteristic
if $p > 0$. Let $E_{sep}$ be the set of elements of $E$ which are separable
over $F$. This is a subextension by Lemma \ref{lemma-separable-elements}
and of course $E_{sep}$ is separable over $F$. Given an $\alpha$ in $E$
there exists a $p$-power $q$ such that $\alpha^q$ is separable over $F$.
Namely, $q$ is that power of $p$ such that the minimal polynomial of
$\alpha$ is of the form $P(x^q)$ with $P$ separable algebraic, see
Lemma \ref{lemma-irreducible-polynomials}. Hence $E/E_{sep}$ is purely
inseparable. Uniqueness is clear.
\end{proof}

\begin{definition}
\label{definition-insep-degree}
Let $E/F$ be an algebraic field extension. Let $E_{sep}$ be the subextension
found in Lemma \ref{lemma-separable-first}.
\begin{enumerate}
\item The integer $[E_{sep} : F]$ is called the {\it separable
degree} of the extension. Notation $[E : F]_s$.
\item The integer $[E : E_{sep}]$ is called the {\it inseparable
degree}, or the {\it degree of inseparability} of the extension.
Notation $[E : F]_i$.
\end{enumerate}
\end{definition}

\noindent
Of course in characteristic $0$ we have $[E : F] = [E : F]_s$ and
$[E : F]_i = 1$. By multipliciativity
(Lemma \ref{lemma-multiplicativity-degrees}) we have
$$
[E : F] = [E : F]_s [E : F]_i
$$
even in case some of these degrees are infinite. In fact, the separable
degree and the inseparable degree are multiplicative too (see
Lemma \ref{lemma-multiplicativity-all-degrees}).

\begin{lemma}
\label{lemma-separable-degree}
Let $K/F$ be a finite extension. Let $\overline{F}$ be an algebraic
closure of $F$. Then $[K : F]_s = |\Mor_F(K, \overline{F})|$.
\end{lemma}

\begin{proof}
We first prove this when $K/F$ is purely inseparable. Namely, we claim that
in this case there is a unique map $K \to \overline{F}$. This can be
seen by choosing a sequence of elements $\alpha_1, \ldots, \alpha_n \in K$
as in Lemma \ref{lemma-finite-purely-inseparable}. The irreducible polynmial
of $\alpha_i$ over $F(\alpha_1, \ldots, \alpha_{i - 1})$ is $x^p - \alpha_i^p$.
Applying Lemma \ref{lemma-count-embeddings-explicitly} we see that
$|\Mor_F(K, \overline{F})| = 1$. On the other hand, $[K : F]_s = 1$
in this case hence the equality holds.

\medskip\noindent
Let's return to a general finite extension $K/F$. In this case
choose $F \subset K_s \subset K$ as in Lemma \ref{lemma-separable-first}.
By Lemma \ref{lemma-separable-equality} we have
$|\Mor_F(K_s, \overline{F})| = [K_s : F] = [K : F]_s$.
On the other hand, every field map $\sigma' : K_s \to \overline{F}$
extends to a unique field map $\sigma : K \to \overline{F}$ by the
result of the previous paragraph. In other words
$|\Mor_F(K, \overline{F})| = |\Mor_F(K_s, \overline{F})|$
and the proof is done.
\end{proof}

\begin{lemma}[Multiplicativity]
\label{lemma-multiplicativity-all-degrees}
Suppose given a tower of algebraic field extensions $K/E/F$. Then
$$
[K : F]_s = [K : E]_s [E : F]_s
\quad\text{and}\quad
[K : F]_i = [K : E]_i [E : F]_i
$$
\end{lemma}

\begin{proof}
We first prove this in case $K$ is finite over $F$. Since we have
multiplicativity for the usual degree (by
Lemma \ref{lemma-multiplicativity-degrees}) it suffices to prove
one of the two formulas. By Lemma \ref{lemma-separable-degree} we have
$[K : F]_s = |\Mor_F(K, \overline{F})|$. By the same lemma,
given any $\sigma \in \Mor_F(E, \overline{F})$ the number of extensions
of $\sigma$ to a map $\tau : K \to \overline{F}$ is $[K : E]_s$.
Namely, via $E \cong \sigma(E) \subset \overline{F}$ we can view
$\overline{F}$ as an algebraic closure of $E$. Combined with the
fact that there are $[E : F]_s = |\Mor_F(E, \overline{F})|$ choices
for $\sigma$ we obtain the result.

\medskip\noindent
If the extensions are infinite one can write $K$ as the union
of all finite subextension $F \subset K' \subset K$. For each
$K'$ we set $E' = E \cap K'$. Then we have the formulas of the
lemma for $K'/E'/F$ by the first paragraph. Since
$[K : F]_s = \sup \{[K' : F]_s\}$ and similarly for the other
degrees (some details omitted) we obtain the result in general.
\end{proof}









\section{Normal extensions}
\label{section-normal}

\noindent
Let $P \in F[x]$ be a nonconstant polynomial over a field $F$. We say $P$
{\it splits completely into linear factors over $F$} or
{\it splits completely over $F$} if there exist
$c \in F^*$, $n \geq 1$, $\alpha_1, \ldots, \alpha_n \in F$ such that
$$
P = c(x - \alpha_1) \ldots (x - \alpha_n)
$$
in $F[x]$. Normal extensions are defined as follows.

\begin{definition}
\label{definition-normal}
Let $E/F$ be an algebraic field extension. We say $E$ is {\it normal}
over $F$ if for all $\alpha \in E$ the minimal polynomial $P$
of $\alpha$ over $F$ splits completely into linear factors over $E$.
\end{definition}

\noindent
As in the case of separable extensions, it takes a bit of work to establish
the basic properties of this notion.

\begin{lemma}
\label{lemma-normal-goes-up}
Let $K/E/F$ be a tower of algebraic field extensions.
If $K$ is normal over $F$, then $K$ is normal over $E$.
\end{lemma}

\begin{proof}
Let $\alpha \in K$. Let $P$ be the minimal polynomial of $\alpha$ over $F$.
Let $Q$ be the minimal polynomial of $\alpha$ over $E$.
Then $Q$ divides $P$ in the polynomial ring $E[x]$, say $P = QR$.
Hence, if $P$ splits completely over $K$, then so does $Q$.
\end{proof}

\begin{lemma}
\label{lemma-intersect-normal}
Let $F$ be a field. Let $M/F$ be an algebraic extension. Let
$F \subset E_i \subset M$, $i \in I$ be subextensions with
$E_i/F$ normal. Then $\bigcap E_i$ is normal over $F$.
\end{lemma}

\begin{proof}
Direct from the definitions.
\end{proof}

\begin{lemma}
\label{lemma-characterize-normal}
Let $E/F$ be an algebraic extension of fields. Let $\overline{F}$ be an
algebraic closure of $F$. The following are equivalent
\begin{enumerate}
\item $E$ is normal over $F$, and
\item for every pair $\sigma, \sigma' \in \Mor_F(E, \overline{F})$ we
have $\sigma(E) = \sigma'(E)$.
\end{enumerate}
\end{lemma}

\begin{proof}
Let $\mathcal{P}$ be the set of all minimal polynomials over $F$ of
all elements of $E$. Set
$$
T =
\{\beta \in \overline{F} \mid P(\beta) = 0\text{ for some }P \in \mathcal{P}\}
$$
It is clear that if $E$ is normal over $F$, then $\sigma(E) = T$
for all $\sigma \in \Mor_F(E, \overline{F})$. Thus we see that (1)
implies (2).

\medskip\noindent
Conversely, assume (2). Pick $\beta \in T$.
We can find a corresponding $\alpha \in E$ whose minimal polynomial
$P \in \mathcal{P}$ annihilates $\beta$. Because $F(\alpha) = F[x]/(P)$
we can find an element $\sigma_0 \in \Mor_F(F(\alpha), \overline{F})$ mapping
$\alpha$ to $\beta$. By Lemma \ref{lemma-map-into-algebraic-closure}
we can extend $\sigma_0$ to a $\sigma \in \Mor_F(E, \overline{F})$.
Whence we see that $\beta$ is in the common image of all embeddings
$\sigma : E \to \overline{F}$. It follows that $\sigma(E) = T$
for any $\sigma$. Fix a $\sigma$. Now let $P \in \mathcal{P}$. Then we
can write
$$
P = (x - \beta_1) \ldots (x - \beta_n)
$$
for some $n$ and $\beta_i \in \overline{F}$ by
Lemma \ref{lemma-algebraically-closed}. Observe that $\beta_i \in T$.
Thus $\beta_i = \sigma(\alpha_i)$ for some $\alpha_i \in E$. Thus
$P = (x - \alpha_1) \ldots (x - \alpha_n)$ splits completely over $E$.
This finishes the proof.
\end{proof}

\begin{lemma}
\label{lemma-normally-generated}
Let $E/F$ be an algebraic extension of fields.
If $E$ is generated by $\alpha_i \in E$, $i \in I$
over $F$ and if for each $i$ the minimal polynomial
of $\alpha_i$ over $F$ splits completely in $E$, then
$E/F$ is normal.
\end{lemma}

\begin{proof}
Let $P_i$ be the minimal polynomial of $\alpha_i$ over $F$.
Let $\alpha_i = \alpha_{i, 1}, \alpha_{i, 2}, \ldots, \alpha_{i, d_i}$
be the roots of $P_i$ over $E$. Given two embeddings
$\sigma, \sigma' : E \to \overline{F}$ over $F$ we see that
$$
\{\sigma(\alpha_{i, 1}), \ldots, \sigma(\alpha_{i, d_i})\} =
\{\sigma'(\alpha_{i, 1}), \ldots, \sigma'(\alpha_{i, d_i})\}
$$
because both sides are equal to the set of roots of $P_i$
in $\overline{F}$. The elements $\alpha_{i, j}$
generate $E$ over $F$ and we find that $\sigma(E) = \sigma'(E)$.
Hence $E/F$ is normal by Lemma \ref{lemma-characterize-normal}.
\end{proof}

\begin{lemma}
\label{lemma-lift-maps}
Let $L/M/K$ be a tower of algebraic extensions.
\begin{enumerate}
\item If $M/K$ is normal, then any automorphism $\tau$ of $L/K$
induces an automorphism $\tau|_M : M \to M$.
\item If $L/K$ is normal, then $K$-algebra map $\sigma : M \to L$
extends to an automorphism of $L$.
\end{enumerate}
\end{lemma}

\begin{proof}
Choose an algebraic closure $\overline{L}$ of $L$
(Theorem \ref{theorem-existence-algebraic-closure}).

\medskip\noindent
Let $\tau$ be as in (1). Then $\tau(M) = M$ as subfields of $\overline{L}$
by Lemma \ref{lemma-characterize-normal} and hence
$\tau|_M : M \to M$ is an automorphism.

\medskip\noindent
Let $\sigma : M \to L$ be as in (2).
By Lemma \ref{lemma-map-into-algebraic-closure}
we can extend $\sigma$ to a map
$\tau : L \to \overline{L}$, i.e., such that
$$
\xymatrix{
L \ar[r]_\tau & \overline{L} \\
M \ar[u] \ar[ru]_\sigma & K \ar[l] \ar[u]
}
$$
is commutative. By Lemma \ref{lemma-characterize-normal} we see that
$\tau(L) = L$. Hence $\tau : L \to L$ is an automorphism which
extends $\sigma$.
\end{proof}

\begin{definition}
\label{definition-automorphisms}
Let $E/F$ be an extension of fields. Then $\text{Aut}(E/F)$ or
$\text{Aut}_F(E)$ denotes the automorphism group of $E$ as an object
of the category of $F$-extensions. Elements of $\text{Aut}(E/F)$
are called {\it automorphisms of $E$ over $F$} or
{\it automorphisms of $E/F$}.
\end{definition}

\noindent
Here is a characterization of normal extensions in terms of automorphisms.

\begin{lemma}
\label{lemma-normal-and-automorphisms}
Let $E/F$ be a finite extension. We have
$$
|\text{Aut}(E/F)| \leq [E : F]_s
$$
with equality if and only if $E$ is normal over $F$.
\end{lemma}

\begin{proof}
Choose an algebraic closure $\overline{F}$ of $F$. Recall that
$[E : F]_s = |\Mor_F(E, \overline{F})|$. Pick an element
$\sigma_0 \in \Mor_F(E, \overline{F})$. Then the map
$$
\text{Aut}(E/F) \longrightarrow \Mor_F(E, \overline{F}),\quad
\tau \longmapsto \sigma_0 \circ \tau
$$
is injective. Thus the inequality. If equality holds, then
every $\sigma \in \Mor_F(E, \overline{F})$ is gotten by precomposing
$\sigma_0$ by an automorphism. Hence $\sigma(E) = \sigma_0(E)$.
Thus $E$ is normal over $F$ by Lemma \ref{lemma-characterize-normal}.

\medskip\noindent
Conversely, assume that $E/F$ is normal. Then by
Lemma \ref{lemma-characterize-normal} we have $\sigma(E) = \sigma_0(E)$
for all $\sigma \in \Mor_F(E, \overline{F})$.
Thus we get an automorphism of $E$ over $F$ by setting
$\tau = \sigma_0^{-1} \circ \sigma$. Whence the map displayed above
is surjective.
\end{proof}

\begin{lemma}
\label{lemma-normal-embeddings-differ-by-aut}
Let $L/K$ be an algebraic normal extension of fields.
Let $E/K$ be an extension of fields. Then either
there is no $K$-embedding from $L$ to $E$ or
there is one $\tau : L \to E$ and every other
one is of the form $\tau \circ \sigma$ where $\sigma \in \text{Aut}(L/K)$.
\end{lemma}

\begin{proof}
Given $\tau$ replace $L$ by $\tau(L) \subset E$ and apply
Lemma \ref{lemma-lift-maps}.
\end{proof}





\section{Splitting fields}
\label{section-splitting-fieds}

\noindent
The following lemma is a useful tool for constructing normal field extensions.

\begin{lemma}
\label{lemma-splitting-field}
Let $F$ be a field. Let $P \in F[x]$ be a nonconstant polynomial.
There exists a smallest field extension $E/F$ such that $P$
splits completely over $E$. Moreover, the field extension $E/F$ is normal
and unique up to (nonunique) isomorphism.
\end{lemma}

\begin{proof}
Choose an algebraic closure $\overline{F}$. Then we can write
$P = c (x - \beta_1) \ldots (x - \beta_n)$ in $\overline{F}[x]$, see
Lemma \ref{lemma-algebraically-closed}. Note that $c \in F^*$. Set
$E = F(\beta_1, \ldots, \beta_n)$. Then it is clear that $E$ is
minimal with the requirement that $P$ splits completely over $E$.

\medskip\noindent
Next, let $E'$ be another minimal field extension of $F$ such that
$P$ splits completely over $E'$. Write
$P = c (x - \alpha_1) \ldots (x - \alpha_n)$ with $c \in F$ and
$\alpha_i \in E'$. Again it follows from minimality that
$E' = F(\alpha_1, \ldots, \alpha_n)$. Moreover, if we pick
any $\sigma : E' \to \overline{F}$
(Lemma \ref{lemma-map-into-algebraic-closure})
then we immediately see that $\sigma(\alpha_i) = \beta_{\tau(i)}$
for some permutation $\tau : \{1, \ldots, n\} \to \{1, \ldots, n\}$.
Thus $\sigma(E') = E$. This implies that $E'$ is a normal extension
of $F$ by Lemma \ref{lemma-characterize-normal}
and that $E \cong E'$ as extensions of $F$ thereby finishing the proof.
\end{proof}

\begin{definition}
\label{definition-splitting-field}
Let $F$ be a field. Let $P \in F[x]$ be a nonconstant polynomial.
The field extension $E/F$ constructed in Lemma \ref{lemma-splitting-field}
is called the {\it splitting field of $P$ over $F$}.
\end{definition}

\begin{lemma}
\label{lemma-normal-closure}
Let $E/F$ be a finite extension of fields. There exists a unique
smallest finite extension $K/E$ such that $K$ is normal over $F$.
\end{lemma}

\begin{proof}
Choose generators $\alpha_1, \ldots, \alpha_n$ of $E$ over $F$.
Let $P_1, \ldots, P_n$ be the minimal polynomials of
$\alpha_1, \ldots, \alpha_n$ over $F$. Set $P = P_1 \ldots P_n$.
Observe that $(x - \alpha_1) \ldots (x - \alpha_n)$ divides $P$, since
each $(x - \alpha_i)$ divides $P_i$. Say
$P = (x - \alpha_1) \ldots (x - \alpha_n)Q$.
Let $K/E$ be the splitting field of $P$ over $E$.
We claim that $K$ is the splitting field of $P$ over $F$ as well
(which implies that $K$ is normal over $F$).
This is clear because $K/E$ is generated by the roots of
$Q$ over $E$ and $E$ is generated by the roots of
$(x - \alpha_1) \ldots (x - \alpha_n)$ over $F$, hence
$K$ is generated by the roots of $P$ over $F$.

\medskip\noindent
Uniqueness. Suppose that $K'/E$ is a second smallest extension such that
$K'/F$ is normal. Choose an algebraic closure $\overline{F}$ and an
embedding $\sigma_0 : E \to \overline{F}$. By
Lemma \ref{lemma-map-into-algebraic-closure}
we can extend $\sigma_0$ to $\sigma : K \to \overline{F}$ and
$\sigma' : K' \to \overline{F}$.
By Lemma \ref{lemma-intersect-normal} we see that
$\sigma(K) \cap \sigma'(K')$ is normal over $F$.
By minimality we conclude that $\sigma(K) = \sigma(K')$.
Thus $\sigma \circ (\sigma')^{-1} : K' \to K$ gives an isomorphism
of extensions of $E$.
\end{proof}

\begin{definition}
\label{definition-normal-closure}
Let $E/F$ be a finite extension of fields. The field extension $K/E$
constructed in Lemma \ref{lemma-normal-closure}
is called the {\it normal closure $E$ over $F$}.
\end{definition}

\noindent
One can construct the normal closure inside any given normal extension.

\begin{lemma}
\label{lemma-normal-closure-inside-normal}
Let $L/K$ be an algebraic normal extension.
\begin{enumerate}
\item If $L/M/K$ is a subextension with $M/K$ finite, then there exists
a tower $L/M'/M/K$ with $M'/K$ finite and normal.
\item If $L/M'/M/K$ is a tower with $M/K$ normal and $M'/M$ finite,
then there exists a tower $L/M''/M'/M/K$ with $M''/M$
finite and $M''/K$ normal.
\end{enumerate}
\end{lemma}

\begin{proof}
Proof of (1). Let $M'$ be the smallest subextension of $L/K$ containing $M$
which is normal over $K$. By Lemma \ref{lemma-normal-closure}
this is the normal closure of $M/K$ and is finite over $K$.

\medskip\noindent
Proof of (2). Let $\alpha_1, \ldots, \alpha_n \in M'$ generate $M'$ over $M$.
Let $P_1, \ldots, P_n$ be the minimal polynomials of
$\alpha_1, \ldots, \alpha_n$ over $K$. Let $\alpha_{i, j}$ be the roots
of $P_i$ in $L$. Let $M''  = M(\alpha_{i, j})$. It follows from
Lemma \ref{lemma-normally-generated}
(applied with the set of generators $M \cup \{\alpha_{i, j}\}$)
that $M''$ is normal over $K$.
\end{proof}











\section{Roots of unity}
\label{section-roots-of-1}

\noindent
Let $F$ be a field. For an integer $n \geq 1$ we set
$$
\mu_n(F) = \{\zeta \in F \mid \zeta^n = 1\}
$$
This is called the {\it group of $n$th roots of unity} or
{\it $n$th roots of $1$}. It is an abelian group under multiplication
with neutral element given by $1$.
Observe that in a field the number of roots of a polynomial of degree $d$
is always at most $d$. Hence we see that $|\mu_n(F)| \leq n$
as it is defined by a polynomial equation of degree $n$.
Of course every element of $\mu_n(F)$ has order dividing $n$.
Moreover, the subgroups
$$
\mu_d(F) \subset \mu_n(F),\quad d | n
$$
each have at most $d$ elements. This implies that $\mu_n(F)$ is cyclic.

\begin{lemma}
\label{lemma-cyclic}
Let $A$ be an abelian group of exponent dividing $n$ such that
$\{x \in A \mid dx = 0\}$ has cardinality at most $d$ for all $d | n$.
Then $A$ is cyclic of order dividing $n$.
\end{lemma}

\begin{proof}
The conditions imply that $|A| \leq n$, in particular $A$ is finite.
The structure of finite abelian groups shows that
$A = \mathbf{Z}/e_1\mathbf{Z} \oplus \ldots \oplus \mathbf{Z}/e_r\mathbf{Z}$
for some integers $1 < e_1 | e_2 | \ldots | e_r$. This would imply
that $\{x \in A \mid e_1 x = 0\}$ has cardinality $e_1^r$. Hence
$r = 1$.
\end{proof}

\noindent
Applying this to the field $\mathbf{F}_p$ we obtain the celebrated result
that the group $(\mathbf{Z}/p\mathbf{Z})^*$ is a cyclic group. More about
this in the section on finite fields.

\medskip\noindent
One more observation is often useful: If $F$ has characteristic
$p > 0$, then $\mu_{p^n}(F) = \{1\}$. This is true because raising
to the $p$th power is an injective map on fields of characteristic $p$
as we have seen in the proof of Lemma \ref{lemma-nr-roots-unchanged}.
(Of course, it also follows from the statement of that lemma itself.)






\section{Finite fields}
\label{section-finite}

\noindent
Let $F$ be a finite field. It is clear that $F$ has positive characteristic
as we cannot have an injection $\mathbf{Q} \to F$. Say the characteristic
of $F$ is $p$. The extension $\mathbf{F}_p \subset F$ is finite.
Hence we see that $F$ has $q = p^f$ elements for some $f \geq 1$.

\medskip\noindent
Let us think about the group of units $F^*$. This is a finite abelian
group, so it has some exponent $e$. Then $F^* = \mu_e(F)$ and we see
from the discussion in Section \ref{section-roots-of-1} that $F^*$
is a cyclic group of order $q - 1$. (A posteriori it follows that
$e = q - 1$ as well.) In particular, if $\alpha \in F^*$ is a generator
then it clearly is true that
$$
F = \mathbf{F}_p(\alpha)
$$
In other words, the extension $F/\mathbf{F}_p$ is generated by a single
element. Of course, the same thing is true for any extension of finite
fields $E/F$ (because $E$ is already generated by a single element over
the prime field).





\section{Primitive elements}
\label{section-primitive-element}

\noindent
Let $E/F$ be a finite extension of fields. An element $\alpha \in E$
is called a {\it primitive element of $E$ over $F$} if $E = F(\alpha)$.

\begin{lemma}[Primitive element]
\label{lemma-primitive-element}
Let $E/F$ be a finite extension of fields. The following are equivalent
\begin{enumerate}
\item there exists a primitive element for $E$ over $F$, and
\item there are finitely many subextensions $E/K/F$.
\end{enumerate}
Moreover, (1) and (2) hold if $E/F$ is separable.
\end{lemma}

\begin{proof}
Let $\alpha \in E$ be a primitive element. Let $P$ be the minimal
polynomial of $\alpha$ over $F$. Let $E \subset M$ be a splitting
field for $P$ over $E$, so that
$P(x) = (x - \alpha)(x - \alpha_2) \ldots (x - \alpha_n)$ over $M$.
For ease of notation we set $\alpha_1 = \alpha$.
Next, let $E/K/F$ be a subextension. Let $Q$ be the minimal
polynomial of $\alpha$ over $K$. Observe that $\deg(Q) = [E : K]$.
Writing $Q = x^d + \sum_{i < d} a_i x^i$ we claim that
$K$ is equal to $L = F(a_0, \ldots, a_{d - 1})$. Indeed $\alpha$ has degree
$d$ over $L$ and $L \subset K$. Hence $[E : L] = [E : K]$ and it follows
that $[K : L] = 1$, i.e., $K = L$.
Thus it suffices to show there are at most finitely many possibilities
for the polynomial $Q$. This is clear because we have a factorization
$P = QR$  in $K[x]$ in particular in $E[x]$. Since we have unique
factorization in $E[x]$ there are at most finitely many monic
factors of $P$ in $E[x]$.

\medskip\noindent
If $F$ is a finite field (equivalently $E$ is a finite field), then
$E/F$ has a primitive element by the discussion in
Section \ref{section-finite}.
Next, assume $F$ is infinite and there are at most finitely many proper
subfields $E/K/F$. List them, say $K_1, \ldots, K_N$. Then
each $K_i \subset E$ is a proper sub $F$-vector space. As $F$ is infinite
we can find a vector $\alpha \in E$ with $\alpha \not \in K_i$ for all $i$
(a finite union of proper subvector spaces is never a subvector space;
details omitted). Then $\alpha$ is a primitive element for $E$ over $F$.

\medskip\noindent
Having established the equivalence of (1) and (2) we now turn to
the final statement of the lemma. Choose an algebraic closure
$\overline{F}$ of $F$. Enumerate the elements
$\sigma_1, \ldots, \sigma_n \in \Mor_F(E, \overline{F})$.
Since $E/F$ is separable we have $n = [E : F]$ by
Lemma \ref{lemma-separable-equality}.
Note that if $i \not = j$, then
$$
V_{ij} = \Ker(\sigma_i - \sigma_j : E \longrightarrow \overline{F})
$$
is not equal to $E$. Hence arguing as in the preceding paragraph
we can find $\alpha \in E$ with $\alpha \not \in V_{ij}$ for all
$i \not = j$. It follows that $|\Mor_F(F(\alpha), \overline{F})| \geq n$.
On the other hand $[F(\alpha) : F] \leq [E : F]$. Hence equality
by Lemma \ref{lemma-separable-equality}
and we conclude that $E = F(\alpha)$.
\end{proof}






\section{Trace and norm}
\label{section-trace-pairing}

\noindent
Let $L/K$ be a finite extension of fields. By
Lemma \ref{lemma-vector-space-is-free}
we can choose an isomorphism $L \cong K^{\oplus n}$ of $K$-modules.
Of course $n = [L : K]$ is the degree of the field extension.
Using this isomorphism we get for a $K$-algebra map
$$
L \longrightarrow \text{Mat}(n \times n, K),\quad
\alpha \longmapsto \text{matrix of multiplication by }\alpha
$$
Thus given $\alpha \in L$ we can take the trace and the determinant
of the corresponding matrix. Of course these quantities are independent
of the choice of the basis chosen above. More canonically, simply thinking
of $L$ as a finite dimensional $K$-vector space we have
$\text{Trace}_K(\alpha : L \to L)$ and the determinant
$\text{Det}_K(\alpha : L \to L)$.

\begin{definition}
\label{definition-trace-norm}
Let $L/K$ be a finite extension of fields. For $\alpha \in L$ we define
the {\it trace}
$\text{Trace}_{L/K}(\alpha) = \text{Trace}_K(\alpha : L \to L)$
and the {\it norm}
$\text{Norm}_{L/K}(\alpha) = \text{Det}_K(\alpha : L \to L)$.
\end{definition}

\noindent
It is clear from the definition that
$\text{Trace}_{L/K}$ is $K$-linear and satisfies
$\text{Trace}_{L/K}(\alpha) = [L : K]\alpha$ for $\alpha \in K$.
Similarly $\text{Norm}_{L/K}$ is multiplicative and
$\text{Norm}_{L/K}(\alpha) = \alpha^{[L : K]}$ for $\alpha \in K$.
This is a special case of the more general construction discussed
in Exercises, Exercises \ref{exercises-exercise-trace-det} and
\ref{exercises-exercise-trace-det-rings}.

\begin{lemma}
\label{lemma-characteristic-vs-minimal-polynomial}
Let $L/K$ be a finite extension of fields. Let $\alpha \in L$ and let $P$
be the minimal polynomial of $\alpha$ over $K$. Then the characteristic
polynomial of the $K$-linear map $\alpha : L \to L$ is equal to
$P^e$ with $e \deg(P) = [L : K]$.
\end{lemma}

\begin{proof}
Choose a basis $\beta_1, \ldots, \beta_e$ of $L$ over $K(\alpha)$.
Then $e$ satisfies $e \deg(P) = [L : K]$ by
Lemmas \ref{lemma-degree-minimal-polynomial} and
\ref{lemma-multiplicativity-degrees}.
Then we see that $L = \bigoplus K(\alpha) \beta_i$ is a
direct sum decomposition into $\alpha$-invariant subspaces
hence the characteristic polynomial of $\alpha : L \to L$
is equal to the characteristic polynommial of
$\alpha : K(\alpha) \to K(\alpha)$ to the power $e$.

\medskip\noindent
To finish the proof we may assume that $L = K(\alpha)$.
In this case by Cayley-Hamilton we see that $\alpha$
is a root of the characteristic polynomial. And since the
characteristic polynomial has the same degree as the minimal
polynomial, we find that equality holds.
\end{proof}

\begin{lemma}
\label{lemma-trace-and-norm-from-minimal-polynomial}
Let $L/K$ be a finite extension of fields. Let $\alpha \in L$ and let
$P = x^d + a_1 x^{d - 1} + \ldots + a_d$
be the minimal polynomial of $\alpha$ over $K$. Then
$$
\text{Norm}_{L/K}(\alpha) = (-1)^{[L : K]} a_d^e
\quad\text{and}\quad
\text{Trace}_{L/K}(\alpha) = - e a_1
$$
where $e d = [L : K]$.
\end{lemma}

\begin{proof}
Follows immediately from Lemma \ref{lemma-characteristic-vs-minimal-polynomial}
and the definitions.
\end{proof}

\begin{lemma}
\label{lemma-trace-and-norm-linear}
Let $L/K$ be a finite extension of fields. Let $V$ be a finite dimensional
vector space over $L$. Let $\varphi : V \to V$ be an $L$-linear map.
Then
$$
\text{Trace}_K(\varphi : V \to V) =
\text{Trace}_{L/K}(\text{Trace}_L(\varphi : V \to V))
$$
and
$$
\text{Det}_K(\varphi : V \to V) =
\text{Norm}_{L/K}(\text{Det}_L(\varphi : V \to V))
$$
\end{lemma}

\begin{proof}
Choose an isomorphism $V = L^{\oplus n}$ so that $\varphi$ corresponds
to an $n \times n$ matrix. In the case of traces, both sides of the formula
are additive in $\varphi$. Hence we can assume that $\varphi$
corresponds to the matrix with exactly one nonzero entry in the $(i, j)$ spot.
In this case a direct computation shows both sides are equal.

\medskip\noindent
In the case of norms both sides are zero if $\varphi$ has a nonzero kernel.
Hence we may assume $\varphi$ corresponds to an element of
$\text{GL}_n(L)$. Both sides of the formula are multiplicative in $\varphi$.
Since every element of $\text{GL}_n(L)$ is a product of elementary
matrices we may assume that $\varphi$ either looks like
$$
E_{12}(\lambda) =
\left(
\begin{matrix}
1 & \lambda & \ldots \\
0 & 1 & \ldots \\
\ldots & \ldots & \ldots
\end{matrix}
\right)
\quad\text{or}\quad
E_1(a) =
\left(
\begin{matrix}
a & 0 & \ldots \\
0 & 1 & \ldots \\
\ldots & \ldots & \ldots
\end{matrix}
\right)
$$
(because we may also permute the basis elements if we like).
In both cases the fomula is easy to verify by direct computation.
\end{proof}

\begin{lemma}
\label{lemma-trace-and-norm-tower}
Let $M/L/K$ be a tower of finite extensions of fields. Then
$$
\text{Trace}_{M/K} = \text{Trace}_{L/K} \circ \text{Trace}_{M/L}
\quad\text{and}\quad
\text{Norm}_{M/K} = \text{Norm}_{L/K} \circ \text{Norm}_{M/L}
$$
\end{lemma}

\begin{proof}
Think of $M$ as a vector space over $L$ and apply
Lemma \ref{lemma-trace-and-norm-linear}.
\end{proof}

\noindent
The trace pairing is defined using the trace.

\begin{definition}
\label{definition-trace-pairing}
Let $L/K$ be a finite extension of fields. The {\it trace pairing}
for $L/K$ is the symmetric $K$-bilinear form
$$
Q_{L/K} : L \times L \longrightarrow K,\quad
(\alpha, \beta) \longmapsto \text{Trace}_{L/K}(\alpha\beta)
$$
\end{definition}

\noindent
It turns out that a finite extension of fields is separable if and only
if the trace pairing is nondegenerate.

\begin{lemma}
\label{lemma-separable-trace-pairing}
Let $L/K$ be a finite extension of fields. The following are equivalent:
\begin{enumerate}
\item $L/K$ is separable, and
\item the trace pairing $Q_{L/K}$ is nondegenerate.
\end{enumerate}
\end{lemma}

\begin{proof}
Observe that the trace pairing is nondegenerate if and only if
$\text{Trace}_{L/K}$ is not identically zero.

\medskip\noindent
Suppose that $K$ has characteristic $p$ and $L = K(\alpha)$ with
$\alpha \not \in K$ and $\alpha^p \in K$. Then $\text{Trace}_{L/K}(1) = p = 0$.
For $i = 1, \ldots, p - 1$ we see that $x^p - \alpha^{pi}$ is the minimal
polynomial for $\alpha^i$ over $K$ and we find
$\text{Trace}_{L/K}(\alpha^i) = 0$ by
Lemma \ref{lemma-trace-and-norm-from-minimal-polynomial}.
Hence for this kind of purely inseparable degree $p$ extension
we see that $\text{Trace}_{L/K}$ is identically zero.

\medskip\noindent
Assume that $L/K$ is not separable. Then there exists a subfield
$L/K'/K$ such that $L/K'$ is a purely inseparable degree $p$ extension
as in the previous paragraph, see
Lemmas \ref{lemma-separable-first} and \ref{lemma-finite-purely-inseparable}.
Hence by Lemma \ref{lemma-trace-and-norm-tower}
we see that $\text{Trace}_{L/K}$ is identically zero.

\medskip\noindent
Assume on the other hand that $\text{Trace}_{L/K}$ is not identically zero.
Let $L/K'/K$ be a maximal subfield separable over $K$. Then by
Lemma \ref{lemma-trace-and-norm-tower}
we see that $\text{Trace}_{L/K'}$ is not identically zero.
Then we pick $\alpha \in L$ such that
$\text{Trace}_{L/K'}(\alpha) \not = 0$.
Then by Lemma \ref{lemma-trace-and-norm-from-minimal-polynomial}
we see that $\alpha$ is separable over $K'$. If $\alpha \not \in K'$
then $K'$ is not maximal. If $\alpha \in K'$ then
Lemma \ref{lemma-trace-and-norm-from-minimal-polynomial}
shows that the characteristic of $K$ does not divide $[L : K']$
which implies that $L/K'$ is separable
(as the inseparable degree of $L/K'$ is forced to be $1$,
see Definition \ref{definition-insep-degree})
and hence trivial.
\end{proof}

\noindent
Let $K$ be a field and let $Q : V \times V \to K$ be a bilinear form
on a finite dimensional vector space over $K$. Say $\dim_K(V) = n$.
Then $Q$ defines a linear map $Q : V \to V^*$, $v \mapsto Q(v, -)$
where $V^* = \Hom_K(V, K)$ is the dual vector space. Hence a linear map
$$
\text{Det}(Q) : \wedge^n(V) \longrightarrow \wedge^n(V)^*
$$
If we pick a basis element $\omega \in \wedge^n(V)$, then we can
write $\text{Det}(Q)(\omega) = \lambda \omega^\wedge$, where $\omega^\wedge$
is the dual basis element in $\wedge^n(V)^*$. If we change our
choice of $\omega$ into $c \omega$ for some $c \in K^*$, then
$\omega^\wedge$ changes into $c^{-1} \omega^\wedge$ and therefore
$\lambda$ changes into $c^2 \lambda$. Thus the class of
$\lambda$ in $K/(K^*)^2$ is well defined and is called the
{\it discriminant of $Q$}. Unwinding the definitions we see that
$$
\lambda = \det(Q(v_i, v_j)_{1 \leq i, j \leq n})
$$
if $\{v_1, \ldots, v_n\}$ is a basis for $V$ over $K$. Observe that
the discriminant is nonzero if and only if $Q$ is nondegenerate.

\begin{definition}
\label{definition-discriminant}
Let $L/K$ be a finite extension of fields. The
{\it discriminant of $L/K$} is the discriminant of
the trace pairing $Q_{L/K}$.
\end{definition}

\noindent
By the discussion above and Lemma \ref{lemma-separable-trace-pairing}
we see that the discriminant is nonzero if and only
if $L/K$ is separable. For $a \in K$ we often say
``the discriminant is $a$'' when it would be more correct
to say the discriminant is the class of $a$ in $K/(K^*)^2$.

\begin{exercise}
\label{exercise-quadratic-discriminant}
Let $L/K$ be an extension of degree $2$. Show that exactly
one of the following happens
\begin{enumerate}
\item the discriminant is $0$, the characteristic of $K$ is $2$,
and $L/K$ is purely inseparable obtained by taking a square root
of an element of $K$,
\item the disriminant is $1$, the characteristic of $K$ is $2$, and
$L/K$ is separable of degree $2$,
\item the discriminant is not a square, the characteristic of $K$
is not $2$, and $L$ is obtained from $K$ by taking the square root
of the discriminant.
\end{enumerate}
\end{exercise}











\section{Galois theory}
\label{section-galois-theory}

\noindent
Here is the definition.

\begin{definition}
\label{definition-galois}
A field extension $E/F$ is called {\it Galois} if it is algebraic,
separable, and normal.
\end{definition}

\noindent
It turns out that a finite extension is Galois if and only if it has
the ``correct'' number of automorphisms.

\begin{lemma}
\label{lemma-finite-Galois}
Let $E/F$ be a finite extension of fields. Then $E$ is Galois over $F$
if and only if $|\text{Aut}(E/F)| = [E : F]$.
\end{lemma}

\begin{proof}
Assume $|\text{Aut}(E/F)| = [E : F]$. By
Lemma \ref{lemma-normal-and-automorphisms} this implies
that $E/F$ is separable and normal, hence Galois.
Conversely, if $E/F$ is separable then $[E : F] = [E : F]_s$ and
if $E/F$ is in addition normal, then
Lemma \ref{lemma-normal-and-automorphisms} implies that
$|\text{Aut}(E/F)| = [E : F]$.
\end{proof}

\noindent
Motivated by the lemma above we introduce the Galois group as follows.

\begin{definition}
\label{definition-galois-group}
If $E/F$ is a Galois extension, then the group $\text{Aut}(E/F)$ is
called the {\it Galois group} and it is denoted $\text{Gal}(E/F)$.
\end{definition}

\noindent
If $L/K$ is an infinite Galois extension, then one should think of the
Galois group as a topological group. We will return to this in
Section \ref{section-infinite-galois}.

\begin{lemma}
\label{lemma-galois-goes-up}
Let $K/E/F$ be a tower of algebraic field extensions.
If $K$ is Galois over $F$, then $K$ is Galois over $E$.
\end{lemma}

\begin{proof}
Combine Lemmas \ref{lemma-normal-goes-up} and \ref{lemma-separable-goes-up}.
\end{proof}

\noindent
Let $G$ be a group acting on a field $K$ (by field automorphisms).
We will often use the notation
$$
K^G = \{x \in K \mid \sigma(x) = x \ \forall \sigma \in G\}
$$
and we will call this the {\it fixed field} for the action of $G$ on $K$.

\begin{lemma}
\label{lemma-galois-over-fixed-field}
Let $K$ be a field. Let $G$ be a finite group acting faithfully on $K$.
Then the extension $K/K^G$ is Galois, we have $[K : K^G] = |G|$,
and the Galois group of the extension is $G$.
\end{lemma}

\begin{proof}
Given $\alpha \in K$ consider the orbit $G \cdot \alpha \subset K$
of $\alpha$ under the group action. Consider the polynomial
$$
P = \prod\nolimits_{\beta \in G \cdot \alpha} (x - \beta) \in K[x]
$$
The key to the whole lemma is that this polynomial is invariant
under the action of $G$ and hence has coefficients in $K^G$.
Namely, for $\tau \in G$ we have
$$
P^\sigma = \prod\nolimits_{\beta \in G \cdot \alpha} (x - \tau(\beta)) =
\prod\nolimits_{\beta \in G \cdot \alpha} (x - \beta) = P
$$
because the map $\beta \mapsto \tau(\beta)$ is a permutation of
the orbit $G \cdot \alpha$. Thus $P \in K^G[x]$. Since also
$P(\alpha) = 0$ as $\alpha$ is an element of its orbit
we conclude that the extension $K/K^G$ is algebraic. Moreover,
the minimal polynomial $Q$ of $\alpha$ over $K^G$ divides the
polynomial $P$ just constructed. Hence $Q$ is separable
(by Lemma \ref{lemma-recognize-separable} for example) and
we conclude that $K/K^G$ is separable. Thus $K/K^G$ is Galois.
To finish the proof it suffices to show that $[K : K^G] = |G|$
since then $G$ will be the Galois group by
Lemma \ref{lemma-finite-Galois}.

\medskip\noindent
Pick finitely many elements $\alpha_i \in K$, $i = 1, \ldots, n$
such that $\sigma(\alpha_i) = \alpha_i$ for $i = 1, \ldots, n$ implies
$\sigma$ is the neutral element of $G$. Set
$$
L = K^G(\{\sigma(\alpha_i); 1 \leq i \leq n, \sigma \in G\}) \subset K
$$
and observe that the action of $G$ on $K$ induces an action of $G$ on $L$.
We will show that $L$ has degree $|G|$ over $K^G$. This will finish the
proof, since if $L \subset K$ is proper, then we can add an element
$\alpha \in K$, $\alpha \not \in L$ to our list of elements
$\alpha_1, \ldots, \alpha_n$ without increasing $L$ which is absurd.
This reduces us to the case that $K/K^G$ is finite which is
treated in the next paragraph.

\medskip\noindent
Assume $K/K^G$ is finite. By Lemma \ref{lemma-primitive-element}
we can find $\alpha \in K$ such that $K = K^G(\alpha)$.
By the construction in the first paragraph of this proof we see
that $\alpha$ has degree at most $|G|$ over $K$. However, the
degree cannot be less than $|G|$ as $G$ acts faithfully on
$K^G(\alpha) = L$ by construction and the inequality of
Lemma \ref{lemma-normal-and-automorphisms}.
\end{proof}

\begin{theorem}[Fundamental theorem of Galois theory]
\label{theorem-galois-theory}
Let $L/K$ be a finite Galois extension with Galois group $G$.
Then we have $K = L^G$ and the map
$$
\{\text{subgroups of }G\}
\longrightarrow
\{\text{subextensions }K \subset M \subset L\},\quad
H \longmapsto L^H
$$
is a bijection whose inverse maps $M$ to $\text{Gal}(L/M)$.
The normal subgroups $H$ of $G$ correspond exactly to those
subextensions $M$ with $M/K$ Galois.
\end{theorem}

\begin{proof}
By Lemma \ref{lemma-galois-goes-up} given a subextension $L/M/K$
the extension $L/M$ is Galois. Of course $L/M$ is also finite
(Lemma \ref{lemma-finite-goes-up}). Thus $|\text{Gal}(L/M)| = [L : M]$
by Lemma \ref{lemma-finite-Galois}.
Conversely, if $H \subset G$ is a finite subgroup, then
$[L : L^H] = |H|$ by Lemma \ref{lemma-galois-over-fixed-field}.
It follows formally from these two observations that we obtain
a bijective correspondence as in the theorem.

\medskip\noindent
If $H \subset G$ is normal, then $L^H$ is fixed by the action of
$G$ and we obtain a canonical map $G/H \to \text{Aut}(L^H/K)$.
This map has to be injective as $\text{Gal}(L/L^H) = H$. Hence
$|G/H| = [L^H : K]$ and $L^H$ is Galois by
Lemma \ref{lemma-finite-Galois}.

\medskip\noindent
Conversely, assume that $K \subset M \subset L$ with $M/K$ Galois.
By Lemma \ref{lemma-lift-maps} we see that every
element $\tau \in \text{Gal}(L/K)$ induces an element
$\tau|_M \in \text{Gal}(M/K)$. This induces a homomorphism
of Galois groups $\text{Gal}(L/K) \to \text{Gal}(M/K)$ whose
kernel is $H$. Thus $H$ is a normal subgroup.
\end{proof}

\begin{lemma}
\label{lemma-ses-galois}
Let $L/M/K$ be a tower of fields. Assume $L/K$ and $M/K$ are finite Galois.
Then we obtain a short exact sequence
$$
1 \to \text{Gal}(L/M) \to \text{Gal}(L/K) \to \text{Gal}(M/K) \to 1
$$
of finite groups.
\end{lemma}

\begin{proof}
Namely, by Lemma \ref{lemma-lift-maps}
we see that every element $\tau \in \text{Gal}(L/K)$ induces an element
$\tau|_M \in \text{Gal}(M/K)$ which gives us the homomorphism on the
right. The map on the left identifies the left group with the kernel
of the right arrow. The sequence is exact because the sizes
of the groups work out correctly by multiplicativity of degrees
in towers of finite extensions
(Lemma \ref{lemma-multiplicativity-degrees}).
One can also use Lemma \ref{lemma-lift-maps}
directly to see that the map on the right is surjective.
\end{proof}






\section{Infinite Galois theory}
\label{section-infinite-galois}

\noindent
The Galois group comes with a canonical topology.

\begin{lemma}
\label{lemma-galois-profinite}
Let $E/F$ be a Galois extension. Endow $\text{Gal}(E/F)$ with the coarsest
topology such that
$$
\text{Gal}(E/F) \times E \longrightarrow E
$$
is continuous when $E$ is given the discrete topology. Then
\begin{enumerate}
\item for any topological space $X$ and map $X \to \text{Aut}(E/F)$
such that the action $X \times E \to E$ is continuous the induced map
$X \to \text{Gal}(E/F)$ is continuous,
\item this topology turns $\text{Gal}(E/F)$ into
a profinite topological group.
\end{enumerate}
\end{lemma}

\begin{proof}
Throughout this proof we think of $E$ as a discrete topological space.
Recall that the compact open topology on the set of self maps
$\text{Map}(E, E)$ is the universal topology such that the action
$\text{Map}(E, E) \times E \to E$ is continuous. See
Topology, Example \ref{topology-example-automorphisms-of-a-set}
for a precise statement. The topology of the lemma on
$\text{Gal}(E/F)$ is the induced topology coming from the
injective map $\text{Gal}(E/F) \to \text{Map}(E, E)$.
Hence the universal property (1) follows from the corresponding
universal property of the compact open topology.
Since the set of invertible self maps $\text{Aut}(E)$
endowed with the compact open topology forms a topological group, see
Topology, Example \ref{topology-example-automorphisms-of-a-set},
and since $\text{Gal}(E/F) = \text{Aut}(E/F) \to \text{Map}(E, E)$
factors through $\text{Aut}(E)$ we obtain a topological group.
In other words, we are using the injection
$$
\text{Gal}(E/F) \subset \text{Aut}(E)
$$
to endow $\text{Gal}(E/F)$ with the induced structure of a topological group
(see Topology, Section \ref{topology-section-topological-groups})
and by construction this is the coarsest structure of a topological
group such that the action $\text{Gal}(E/F) \times E \to E$ is continuous.

\medskip\noindent
To show that $\text{Gal}(E/F)$ is profinite we argue as follows
(our argument is necessarily nonstandard because we have defined
the topology before showing that the Galois group is an inverse
limit of finite groups).
By Topology, Lemma \ref{topology-lemma-profinite-group}
it suffices to show that the underlying
topological space of $\text{Gal}(E/F)$ is profinite.
For any subset $S \subset E$ consider the set
$$
G(S) = \{ f : S \to E \mid
\begin{matrix}
f(\alpha)\text{ is a root of the minimal polynomial}\\
\text{of }\alpha\text{ over }F\text{ for all }\alpha \in S
\end{matrix}
\}
$$
Since a polynomial has only a finite number of roots we see that
$G(S)$ is finite for all $S \subset E$ finite. If $S \subset S'$
then restriction gives a map $G(S') \to G(S)$. Also, observe
that if $\alpha \in S \cap F$ and $f \in G(S)$, then $f(\alpha) = \alpha$
because the minimal polynomial is linear in this case.
Consider the profinite topological space
$$
G = \lim_{S \subset E\text{ finite}} G(S)
$$
Consider the canonical map
$$
c : \text{Gal}(E/F) \longrightarrow G,\quad
\sigma \longmapsto (\sigma|_S : S \to E)_S
$$
This is injective and unwinding the definitions the
reader sees the topology on $\text{Gal}(E/F)$ as defined above
is the induced topology from $G$. An element $(f_S) \in G$ is in
the image of $c$ exactly if
(A) $f_S(\alpha) + f_S(\beta) = f_S(\alpha + \beta)$ and
(M) $f_S(\alpha)f_S(\beta) = f_S(\alpha\beta)$ whenever
this makes sense (i.e., 
$\alpha, \beta, \alpha + \beta, \alpha\beta \in S$).
Namely, this means
$\lim f_S : E \to E$ will be an $F$-algebra map
and hence an automorphism by
Lemma \ref{lemma-algebraic-extension-self-map}.
The conditions (A) and (M) for a given triple
$(S, \alpha, \beta)$ define a closed subset of
$G$ and hence $\text{Gal}(E/F)$ is homeomorphic
to a closed subset of a profinite space and therefore
profinite itself.
\end{proof}

\begin{lemma}
\label{lemma-galois-infinite}
Let $L/M/K$ be a tower of fields. Assume both $L/K$ and
$M/K$ are Galois. Then there is a canonical surjective continuous
homomorphism $c : \text{Gal}(L/K) \to \text{Gal}(M/K)$.
\end{lemma}

\begin{proof}
By Lemma \ref{lemma-lift-maps} given $\tau : L \to L$ in
$\text{Gal}(L/K)$ the restriction $\tau|_M : M \to M$
is an element of $\text{Gal}(M/K)$. This defines the homomorphism $c$.
Continuity follows from the universal property of the topology:
the action
$$
\text{Gal}(L/K) \times M \longrightarrow M,\quad
(\tau, x) \longmapsto \tau(x) = c(\tau)(x)
$$
is continuous as $M \subset L$ and the action
$\text{Gal}(L/K) \times L \to L$ is continuous.
Hence continuity of $c$ by part (1) of
Lemma \ref{lemma-galois-profinite}.
Lemma \ref{lemma-lift-maps} also
shows that the map is surjective.
\end{proof}

\noindent
Here is a more standard way to think about
the Galois group of an infinite Galois extension.

\begin{lemma}
\label{lemma-infinite-galois-limit}
Let $L/K$ be a Galois extension with Galois group $G$.
Let $\Lambda$ be the set of finite Galois subextensions,
i.e., $\lambda \in \Lambda$ corresponds to $L/L_\lambda/K$
with $L_\lambda/K$ finite Galois with Galois group $G_\lambda$.
Define a partial ordering on $\Lambda$ by the rule
$\lambda \geq \lambda'$ if and only if
$L_\lambda \supset L_{\lambda'}$. Then
\begin{enumerate}
\item $\Lambda$ is a directed partially ordered set,
\item $L_\lambda$ is a system of $K$-extensions over $\Lambda$
and $L = \colim L_\lambda$,
\item $G_\lambda$ is an inverse system of finite groups over $\Lambda$,
the transition maps are surjective, and
$$
G = \lim_{\lambda \in \Lambda} G_\lambda
$$
as a profinite group, and
\item each of the projections $G \to G_\lambda$ is continuous and surjective.
\end{enumerate}
\end{lemma}

\begin{proof}
Every subfield of $L$ containing $K$ is separable over $K$
(follows immediately from the definition). Let $S \subset L$
be a finite subset. Then $K(S)/K$ is finite and there exists
a tower $L/E/K(S)/K$ such that $E/K$ is finite Galois, see
Lemma \ref{lemma-normal-closure-inside-normal}.
Hence $E = L_\lambda$ for some $\lambda \in \Lambda$.
This certainly implies the set $\Lambda$ is not empty.
Also, given $\lambda_1, \lambda_2 \in \Lambda$ we can
write $L_{\lambda_i} = K(S_i)$ for finite sets
$S_1, S_2 \subset L$ (Lemma \ref{lemma-finite-finitely-generated}).
Then there exists a $\lambda \in \Lambda$ such that
$K(S_1 \cup S_2) \subset L_\lambda$. Hence
$\lambda \geq \lambda_1, \lambda_2$ and
$\Lambda$ is directed (Categories, Definition
\ref{categories-definition-directed-system}).
Finally, since every element in $L$ is contained in
$L_\lambda$ for some $\lambda \in \Lambda$, it follows
from the description of filtered colimits in
Categories, Section \ref{categories-section-directed-colimits}
that $\colim L_\lambda = L$.

\medskip\noindent
If $\lambda \geq \lambda'$ in $\Lambda$, then we obtain a
canonical surjective map $G_\lambda \to G_{\lambda'}$,
$\sigma \mapsto \sigma|_{L_{\lambda'}}$
by Lemma \ref{lemma-ses-galois}. Thus we get an inverse
system of finite groups with surjective transition maps.

\medskip\noindent
Recall that $G = \text{Aut}(L/K)$. By Lemma \ref{lemma-galois-infinite}
the restriction $\sigma|_{L_\lambda}$ of a $\sigma \in G$ to $L_\lambda$
is an element of $G_\lambda$. Moreover, this procedure gives a continuous
surjection $G \to G_\lambda$. Since the transition mappings
in the inverse system of $G_\lambda$ are given by restriction
also, it is clear that we obtain a canonical continuous map
$$
G \longrightarrow \lim_{\lambda \in \Lambda} G_\lambda
$$
Continuity by definition of limits in the category of topological
groups; recall that these limits commute with the forgetful functor
to the categories of sets and topological spaces by
Topology, Lemma \ref{topology-lemma-topological-group-limits}.
On the other hand, since $L = \colim L_\lambda$ it is clear
that any element of the inverse limit (viewed as a set) defines an
automorphism of $L$. Thus the map is bijective. Since the topology
on both sides is profinite, and since a bijective continuous map
of profinite spaces is a homeomorphism
(Topology, Lemma \ref{topology-lemma-bijective-map}), the proof is complete.
\end{proof}

\begin{theorem}[Fundamental theorem of infinite Galois theory]
\label{theorem-inifinite-galois-theory}
Let $L/K$ be a Galois extension. Let $G = \text{Gal}(L/K)$
be the Galois group viewed as a profinite topoological group
(Lemma \ref{lemma-galois-profinite}). Then we have $K = L^G$ and the map
$$
\{\text{closed subgroups of }G\}
\longrightarrow
\{\text{subextensions }K \subset M \subset L\},\quad
H \longmapsto L^H
$$
is a bijection whose inverse maps $M$ to $\text{Gal}(L/M)$.
The finite subextensions $M$ correspond exactly to the open
subgroups $H \subset G$. The normal closed subgroups $H$ of $G$
correspond exactly to subextensions $M$ Galois over $K$.
\end{theorem}

\begin{proof}
We will use the result of finite Galois theory
(Theorem \ref{theorem-galois-theory})
without further mention.
Let $S \subset L$ be a finite subset. There exists a tower
$L/E/K$ such that $K(S) \subset E$ and such that
$E/K$ is finite Galois, see Lemma \ref{lemma-normal-closure-inside-normal}.
In other words, we see that $L/K$ is the union of its finite
Galois subextensions.
For such an $E$, by Lemma \ref{lemma-galois-infinite}
the map $\text{Gal}(L/K) \to \text{Gal}(E/K)$ is surjective
and continuous, i.e., the kernel is open because the topology
on $\text{Gal}(E/K)$ is discrete.
In particular we see that no element of $M \setminus K$ is fixed by
$\text{Gal}(L/K)$ as $E^{\text{Gal}(E/K)} = K$.
This proves that $L^G = K$.

\medskip\noindent
Lemma \ref{lemma-galois-goes-up} given a subextension $L/M/K$
the extension $L/M$ is Galois. It is immediate from the definition
of the topology on $G$ that the subgroup $\text{Gal}(L/M)$ is closed.
By the above applied to $L/M$ we see that $L^{\text{Gal}(L/M)} = M$

\medskip\noindent
Conversely, let $H \subset G$ be a closed subgroup. We claim that
$H = \text{Gal}(L/L^H)$. The inclusion $H \subset \text{Gal}(L/L^H)$
is clear. Suppose that $g \in \text{Gal}(L/L^H)$. Let $S \subset L$
be a finite subset. We will show that the open neighbourhood
$U_S(g) = \{g' \in G \mid g'(s) = g(s)\}$ of $g$ meets $H$.
This implies that $g \in H$ because $H$ is closed.
Let $L/E/K$ be a finite Galois subextension containing $K(S)$
as in the first paragraph of the proof and consider the homomorphism
$c : \text{Gal}(L/K) \to \text{Gal}(E/K)$.
Then $L^H \cap E = E^{c(H)}$. Since $g$ fixes $L^H$ it fixes
$E^{c(H)}$ and hence $c(g) \in c(H)$ by finite Galois theory.
Pick $h \in H$ with $c(h) = c(g)$. Then $h \in U_S(g)$ as desired.

\medskip\noindent
At this point we have established the correspondence between closed
subgroups and subextensions.

\medskip\noindent
Assume $H \subset G$ is open. Arguing as above we find that
$H$ containes $\text{Gal}(E/K)$ for some large enough finite
Galois subextension $E$ and we find that $L^H$ is contained
in $E$ whence finite over $K$. Conversely, if $M$ is a finite
subextension, then $M$ is generated by a finite subset $S$
and the corresponding subgroup is the open subset $U_S(e)$
where $e \in G$ is the neutral element.

\medskip\noindent
Assume that $K \subset M \subset L$ with $M/K$ Galois.
By Lemma \ref{lemma-galois-infinite} there is a surjective
continuous homomorphism of Galois groups
$\text{Gal}(L/K) \to \text{Gal}(M/K)$ whose
kernel is $\text{Gal}(L/M)$. Thus $\text{Gal}(L/M)$ is a normal
closed subgroup.

\medskip\noindent
Finally, assume $N \subset G$ is normal and closed. For any
$L/E/K$ as in the first paragraph of the proof, the image
$c(N) \subset \text{Gal}(E/K)$ is a normal subgroup.
Hence $L^N = \bigcup E^{c(N)}$ is a union of Galois extensions
of $K$ (by finite Galois theory) whence Galois over $K$.
\end{proof}

\begin{lemma}
\label{lemma-ses-infinite-galois}
Let $L/M/K$ be a tower of fields. Assume $L/K$ and $M/K$ are Galois.
Then we obtain a short exact sequence
$$
1 \to \text{Gal}(L/M) \to \text{Gal}(L/K) \to \text{Gal}(M/K) \to 1
$$
of profinite topological groups.
\end{lemma}

\begin{proof}
This is a reformulation of Lemma \ref{lemma-galois-infinite}.
\end{proof}






\section{The complex numbers}
\label{section-complex-numbers}

\noindent
The fundamental theorem of algebra states that the field of complex numbers
is an algebraically closed field. In this section we discuss this
briefly.

\medskip\noindent
The first remark we'd like to make is that you need to use a little
bit of input from calculus in order to prove this. We will use the
intuitively clear fact that every odd degree polynomial over
the reals has a real root. Namely, let
$P(x) = a_{2k + 1} x^{2k + 1} + \ldots + a_0 \in \mathbf{R}[x]$
for some $k \geq 0$ and $a_{2k + 1} \not = 0$.
We may and do assume $a_{2k + 1} > 0$. Then for $x \in \mathbf{R}$
very large (positive) we see that $P(x) > 0$ as the term
$a_{2k + 1} x^{2k + 1}$ dominates all the other terms. Similarly,
if $x \ll 0$, then $P(x) < 0$ by the same reason (and this is where
we use that the degree is odd). Hence by the intermediate value
theorem there is an $x \in \mathbf{R}$ with $P(x) = 0$.

\medskip\noindent
A conclusion we can draw from the above is that $\mathbf{R}$ has
no nontrivial odd degree field extensions, as elements of such extensions
would have odd degree minimal polynomials.

\medskip\noindent
Next, let $K/\mathbf{R}$ be a finite Galois extension with Galois group $G$.
Let $P \subset G$ be a $2$-sylow subgroup. Then $K^P/\mathbf{R}$ is an
odd degree extension, hence by the above $K^P = K$, which in turn implies
$G = P$. (All of these arguments rely on Galois theory of course.)
Thus $G$ is a $2$-group. If $G$ is nontrivial, then we see that
$\mathbf{C} \subset K$ as $\mathbf{C}$ is (up to isomorphism) the only degree
degree $2$ extension of $\mathbf{R}$. If $G$ has more than $2$ elements
we would obtain a quadratic extension of $\mathbf{C}$.
This is absurd as every complex number has a square root.

\medskip\noindent
The conclusion: $\mathbf{C}$ is algebraically closed. Namely, if not
then we'd get a nontrivial finite extension $\mathbf{C} \subset K$
which we could assume normal (hence Galois) over $\mathbf{R}$ by
Lemma \ref{lemma-normal-closure}. But we've seen above that then
$K = \mathbf{C}$.

\begin{lemma}[Fundamental theorem of algebra]
\label{lemma-C-algebraically-closed}
The field $\mathbf{C}$ is algebraically closed.
\end{lemma}

\begin{proof}
See discussion above.
\end{proof}





\section{Kummer extensions}
\label{section-Kummer}

\noindent
Let $K$ be a field. Let $n \geq 2$ be an integer such that $K$ contains
a primitive $n$th root of $1$. Let $a \in K$. Let $L$ be an extension
of $K$ obtained by adjoining a root $b$ of the equation $x^n = a$.
Then $L/K$ is Galois. If $G = \text{Gal}(L/K)$ is the Galois group, then
the map
$$
G \longrightarrow \mu_n(K),\quad \sigma \longmapsto \sigma(b)/b
$$
is an injective homomorphism of groups. In particular, $G$ is cyclic
of order dividing $n$ as a subgroup of the cyclic group $\mu_n(K)$.
Kummer theory gives a converse.

\begin{lemma}[Kummer extensions]
\label{lemma-Kummer}
Let $K \subset L$ be a Galois extension of fields whose Galois group is
$\mathbf{Z}/n\mathbf{Z}$. Assume moreover that the characteristic of $K$
is prime to $n$ and that $K$ contains a primitive $n$th root of $1$.
Then $L = K[z]$ with $z^n \in K$.
\end{lemma}

\begin{proof}
Omitted.
\end{proof}




\section{Artin-Schreier extensions}
\label{section-Artin-Schreier}

\noindent
Let $K$ be a field of characteristic $p > 0$. Let $a \in K$. Let $L$ be an
extension of $K$ obtained by adjoining a root $b$ of the equation
$x^p - x = a$. Then $L/K$ is Galois. If $G = \text{Gal}(L/K)$ is the Galois
group, then the map
$$
G \longrightarrow \mathbf{Z}/p\mathbf{Z},\quad
\sigma \longmapsto \sigma(b) - b
$$
is an injective homomorphism of groups. In particular, $G$ is cyclic
of order dividing $p$ as a subgroup of $\mathbf{Z}/p\mathbf{Z}$.
The theory of Artin-Schreier extensions gives a converse.

\begin{lemma}[Artin-Schreier extensions]
\label{lemma-Artin-Schreier}
Let $K \subset L$ be a Galois extension of fields of characteristic $p > 0$
with Galois group $\mathbf{Z}/p\mathbf{Z}$. Then $L = K[z]$ with
$z^p - z \in K$.
\end{lemma}

\begin{proof}
Omitted.
\end{proof}



\section{Transcendence}
\label{section-transcendence}

\noindent
We recall the standard definitions.

\begin{definition}
\label{definition-transcendence}
Let $k \subset K$ be a field extension.
\begin{enumerate}
\item A collection of elements $\{x_i\}_{i \in I}$ of $K$ is called
{\it algebraically independent} over $k$ if the map
$$
k[X_i; i\in I] \longrightarrow K
$$
which maps $X_i$ to $x_i$ is injective.
\item The field of fractions of a polynomial ring
$k[x_i; i \in I]$ is denoted $k(x_i; i\in I)$.
\item A {\it purely transcendental extension} of $k$ is any
field extension $k \subset K$ isomorphic to the field of
fractions of a polynomial ring over $k$.
\item A {\it transcendence basis} of $K/k$ is a
collection of elements $\{x_i\}_{i \in I}$ which are
algebraically independent over $k$ and such that
the extension $k(x_i; i\in I) \subset K$ is algebraic.
\end{enumerate}
\end{definition}

\begin{example}
\label{example-pi-transcendental}
The field $\mathbf{Q}(\pi)$ is purely transcendental because
$\pi$ isn't the root of a nonzero polynomial with rational coefficients.
In particular, $\mathbf{Q}(\pi) \cong \mathbf{Q}(x)$.
\end{example}

\begin{lemma}
\label{lemma-transcendence-degree}
Let $E/F$ be a field extension. A transcendence basis of $E$ over $F$ exists.
Any two transcendence bases have the same cardinality.
\end{lemma}

\begin{proof}
Let $A$ be an algebraically independent subset of $E$. Let $G$ be a subset
of $E$ containing $A$ that generates $E/F$. We claim we can find a
transcendence basis $B$ such that $A \subset B \subset G$.
To prove this consider the collection of algebraically independent subsets
$\mathcal{B}$ whose members are subsets of $G$ that contain $A$.
Define a partial ordering on $\mathcal{B}$ using inclusion.
Then $\mathcal{B}$ contains at least one element $A$.
The union of the elements of a totally ordered subset $T$ of $\mathcal{B}$
is an algebraically independent subset of $E$ over $F$ since any algebraic
dependence relation would have occurred in one of the elements of $T$
(since polynomials only involve finitely many variables). The union also
contains $A$ and is contained in $G$. By Zorn's lemma, there is a maximal
element $B \in \mathcal{B}$. Now we claim $E$ is algebraic over $F(B)$.
This is because if it wasn't then there would be an element
$f \in G$ transcendental over $F(B)$ since $E(G) = F$. Then
$B \cup\{f\}$ wold be algebraically independent contradicting the
maximality of $B$. Thus $B$ is our transcendence basis.

\medskip\noindent
Let $B$ and $B'$ be two transcendence bases. Without loss of generality, we
can assume that $|B'| \leq |B|$. Now we divide the proof into two cases: the
first case is that $B$ is an infinite set. Then for each $\alpha \in B'$,
there is a finite set $B_{\alpha}$ such that $\alpha$ is algebraic over
$E(B_{\alpha})$ since any algebraic dependence relation only uses finitely many
indeterminates. Then we define $B^* = \bigcup_{\alpha\in B'} B_{\alpha}$.
By construction, $B^* \subset B$, but we claim that in fact the two sets are
equal. To see this, suppose that they are not equal, say there is an element
$\beta \in B \setminus B^*$. We know $\beta$ is algebraic over $E(B')$ which
is algebraic over $E(B^*)$. Therefore $\beta$ is algebraic over $E(B^*)$, a
contradiction. So $|B| \leq |\bigcup_{\alpha \in B'} B_{\alpha}|$.
Now if $B'$ is finite, then so is $B$ so we can assume $B'$ is infinite;
this means
$$
|B| \leq |\bigcup\nolimits_{\alpha \in B'} B_{\alpha}| = |B'|
$$
because each $B_\alpha$ is finite and $B'$ is infinite. Therefore in the
infinite case, $|B| = |B'|$.

\medskip\noindent
Now we need to look at the case where $B$ is finite.
In this case, $B'$ is also finite, so suppose
$B = \{\alpha_1, \ldots, \alpha_n\}$ and
$B' = \{\beta_1, \ldots, \beta_m\}$ with $m \leq n$.
We perform induction on $m$: if $m = 0$ then $E/F$ is algebraic so
$B = \emptyset$ so $n = 0$. If $m > 0$, there is an irreducible polynomial
$f \in E[x, y_1, \ldots, y_n]$ such that
$f(\beta_1, \alpha_1, \ldots, \alpha_n) = 0$ and such that $x$ occurs in $f$.
Since $\beta_1$ is not algebraic over $F$, $f$ must involve some $y_i$
so without loss of generality, assume $f$ uses $y_1$.
Let $B^* = \{\beta_1, \alpha_2, \ldots, \alpha_n\}$.
We claim that $B^*$ is a basis for $E/F$. To prove this claim, we see that
we have a tower of algebraic extensions
$$
E/ F(B^*, \alpha_1) / F(B^*)
$$
since $\alpha_1$ is algebraic over $F(B^*)$.
Now we claim that $B^*$ (counting multiplicity of elements) is
algebraically independent over $E$ because if it weren't, then there would be an
irreducible $g\in E[x, y_2, \ldots, y_n]$ such that
$g(\beta_1, \alpha_2, \ldots, \alpha_n) = 0$
which must involve $x$ making $\beta_1$
algebraic over $E(\alpha_2, \ldots, \alpha_n)$ which would make $\alpha_1$
algebraic over $E(\alpha_2, \ldots, \alpha_n)$ which is impossible.
So this means that $\{\alpha_2, \ldots, \alpha_n\}$ and
$\{\beta_2, \ldots, \beta_m\}$ are bases for $E$ over $F(\beta_1)$
which means by induction, $m = n$. 
\end{proof}

\begin{definition}
\label{definition-transcendence-degree}
Let $k \subset K$ be a field extension.
The {\it transcendence degree} of $K$ over $k$ is
the cardinality of a transcendence basis of $K$ over $k$.
It is denoted $\text{trdeg}_k(K)$.
\end{definition}

\begin{lemma}
\label{lemma-transcendence-degree-tower}
Let $k \subset K \subset L$ be field extensions.
Then
$$
\text{trdeg}_k(L) =
\text{trdeg}_K(L) +
\text{trdeg}_k(K).
$$
\end{lemma}

\begin{proof}
Choose a transcendence basis $A \subset K$ of $K$ over $k$.
Choose a transcendence basis $B \subset L$ of $L$ over $K$.
Then it is straightforward to see that $A \cup B$ is a transcendence
basis of $L$ over $k$.
\end{proof}

\begin{example}
\label{example-pi-e-transcendental}
Consider the field extension $\mathbf{Q}(e, \pi)$ formed by
adjoining the numbers $e$ and $\pi$. This field extension has transcendence
degree at least $1$ since both $e$ and $\pi$ are transcendental over the
rationals. However, this field extension might have transcendence
degree $2$ if $e$ and $\pi$ are algebraically independent. Whether or
not this is true is unknown and whence the problem of determining
$trdeg(\mathbf{Q}(e, \pi))$ is open.
\end{example}

\begin{example}
\label{example-function-field-not-unique-transcendence-basis}
Let $F$ be a field and $E = F(t)$. Then $\{t\}$ is a
transcendence basis since $E = F(t)$. However, $\{t^2\}$
is also a transcendence basis since $F(t)/F(t^2)$ is algebraic.
This illustrates that while we can always decompose an extension
$E/F$ into an algebraic extension $E/F'$ and a
purely transcendental extension $F'/F$, this decomposition is not unique and
depends on choice of transcendence basis.
\end{example}

\begin{example}
\label{example-riemann-surface-transcendence}
Let $X$ be a compact Riemann surface. Then the function field
$\mathbf{C}(X)$ (see Example \ref{example-field-of-meromorphic-functions})
has transcendence degree one over $\mathbf{C}$. In fact, {\it any}
finitely generated extension of $\mathbf{C}$ of transcendence degree
one arises from a Riemann surface. There is even an equivalence of
categories between the category of compact Riemann surfaces and
(non-constant) holomorphic maps and the opposite of the category of finitely
generated extensions of $\mathbf{C}$ of transcendence degree $1$
and morphisms of $\mathbf{C}$-algebras. See \cite{Forster}.

\medskip\noindent
There is an algebraic version of the above statement as well. Given an
(irreducible) algebraic curve in projective space over an algebraically
closed field $k$ (e.g. the complex numbers), one can consider its
``field of rational functions'': basically, functions that look like
quotients of polynomials, where the denominator does not identically
vanish on the curve. There is a similar anti-equivalence of categories
(insert future reference here) between smooth projective curves and
non-constant morphisms of curves and finitely generated extensions of $k$ of
transcendence degree one. See \cite{H}.
\end{example}

\begin{definition}
\label{definition-algebraically-closed-in}
Let $k \subset K$ be a field extension.
\begin{enumerate}
\item The {\it algebraic closure of $k$ in $K$} is the subfield
$k'$ of $K$ consisting of elements of $K$ which are algebraic over $k$.
\item We say $k$ is {\it algebraically closed in $K$} if
every element of $K$ which is algebraic over $k$ is
contained in $k$.
\end{enumerate}
\end{definition}

\begin{lemma}
\label{lemma-algebraic-closure-in-finitely-generated}
Let $k \subset K$ be a finitely generated field extension.
The algebraic closure of $k$ in $K$ is finite over $k$.
\end{lemma}

\begin{proof}
Let $x_1, \ldots, x_r \in K$ be a transcendence basis for $K$
over $k$. Then $n = [K : k(x_1, \ldots, x_r)] < \infty$.
Suppose that $k \subset k' \subset K$ with $k'/k$ finite.
In this case
$[k'(x_1, \ldots, x_r) : k(x_1, \ldots, x_r)] = [k' : k] < \infty$.
Hence
$$
[k' : k] = [k'(x_1, \ldots, x_r) : k(x_1, \ldots, x_r)]
< [K : k(x_1, \ldots, x_r)] = n.
$$
In other words, the degrees of finite subextensions are bounded
and the lemma follows.
\end{proof}






\section{Linearly disjoint extensions}
\label{section-linearly-disjoint}

\noindent
Let $k$ be a field, $K$ and $L$ field extensions of $k$.
Suppose also that $K$ and $L$ are embedded in some larger field $\Omega$.

\begin{definition}
\label{definition-compositum}
Consider a diagram
\begin{equation}
\label{equation-inside-omega}
\vcenter{
\xymatrix{
L \ar[r] & \Omega \\
k \ar[r] \ar[u] & K \ar[u]
}
}
\end{equation}
of field extensions. The {\it compositum of $K$ and $L$ in $\Omega$}
written $KL$ is the smallest subfield of $\Omega$ containing both
$L$ and $K$.
\end{definition}

\noindent
It is clear that $KL$ is generated by the set $K \cup L$ over $k$,
generated by the set $K$ over $L$, and generated by the set $L$ over $K$.

\medskip\noindent
{\bf Warning:} The (isomorphism class of the) composition depends on
the choice of the embeddings of $K$ and $L$ into $\Omega$. For example
consider the number fields $K = \mathbf{Q}(2^{1/8}) \subset \mathbf{R}$ and
$L = \mathbf{Q}(2^{1/12}) \subset \mathbf{R}$. The compositum inside
$\mathbf{R}$ is the field $\mathbf{Q}(2^{1/24})$ of degree $24$ over
$\mathbf{Q}$. However, if we embed $K = \mathbf{Q}[x]/(x^8 - 2)$ into
$\mathbf{C}$ by mapping $x$ to $2^{1/8}e^{2\pi i/8}$, then the compositum
$\mathbf{Q}(2^{1/12}, 2^{1/8}e^{2\pi i/8})$ contains $i = e^{2\pi i/4}$ and has
degree $48$ over $\mathbf{Q}$ (we omit showing the degree is $48$, but
the existence of $i$ certainly proves the two composita are not isomorphic). 

\begin{definition}
\label{definition-linearly-disjoint}
Consider a diagram of fields as in (\ref{equation-inside-omega}).
We say that $K$ and $L$ are {\it linearly disjoint over $k$ in $\Omega$}
if the map
$$
K \otimes_k L \longrightarrow KL,\quad
\sum x_i \otimes y_i \longmapsto \sum x_i y_i
$$
is injective.
\end{definition}

\noindent
The following lemma does not seem to fit anywhere else.

\begin{lemma}
\label{lemma-normal-case}
Let $E/F$ be a normal algebraic field extension. There exist subextensions
$E / E_{sep} /F$ and $E / E_{insep} / F$ such that
\begin{enumerate}
\item $F \subset E_{sep}$ is Galois and $E_{sep} \subset E$
is purely inseparable,
\item $F \subset E_{insep}$ is purely inseparable and $E_{insep} \subset E$
is Galois,
\item $E = E_{sep} \otimes_F E_{insep}$.
\end{enumerate}
\end{lemma}

\begin{proof}
We found the subfield $E_{sep}$ in Lemma \ref{lemma-separable-first}.
We set $E_{insep} = E^{\text{Aut}(E/F)}$. Details omitted.
\end{proof}









\section{Review}
\label{section-algebraic}

\noindent
In this section we give a quick review of what has transpired above.

\medskip\noindent
Let $k \subset K$ be a field extension. Let $\alpha \in K$. Then we have the
following possibilities:
\begin{enumerate}
\item The element $\alpha$ is transcendental over $k$.
\item The element $\alpha$ is algebraic over $k$. Denote
$P(T) \in k[T]$ its {\it minimal polynomial}. This is a monic polynomial
$P(T) = T^d + a_1 T^{d - 1} + \ldots + a_d$ with coefficients in
$k$. It is irreducible and $P(\alpha) = 0$. These properties
uniquely determine $P$, and the integer $d$ is called the
{\it degree of $\alpha$ over $k$}. There are two subcases:
\begin{enumerate}
\item The polynomial $\text{d}P/\text{d}T$ is not identically zero.
This is equivalent to the condition that
$P(T) = \prod_{i = 1, \ldots, d} (T - \alpha_i)$ for
pairwise distinct elements $\alpha_1, \ldots, \alpha_d$
in the algebraic closure of $k$.
In this case we say that $\alpha$ is {\it separable} over $k$.
\item The $\text{d}P/\text{d}T$ is identically zero. In this case the
characteristic $p$ of $k$ is $ > 0$, and $P$ is actually a polynomial
in $T^p$. Clearly there exists a largest power $q = p^e$ such that $P$ is
a polynomial in $T^q$. Then the element $\alpha^q$ is separable over $k$.
\end{enumerate}
\end{enumerate}

\begin{definition}
\label{definition-separable-algebraic}
Algebraic field extensions.
\begin{enumerate}
\item A field extension $k \subset K$ is called {\it algebraic}
if every element of $K$ is algebraic over $k$.
\item An algebraic extension $k \subset k'$ is called {\it separable}
if every $\alpha \in k'$ is separable over $k$.
\item An algebraic
extension $k \subset k'$ is called {\it purely inseparable} if
the characteristic of $k$ is $p > 0$ and for every element
$\alpha \in k'$ there exists a power $q$ of $p$ such that
$\alpha^q \in k$.
\item An algebraic extension $k \subset k'$ is called {\it normal}
if for every $\alpha \in k'$ the minimal polynomial $P(T) \in k[T]$
of $\alpha$ over $k$ splits completely into linear factors over $k'$.
\item An algebraic extension $k \subset k'$ is called {\it Galois}
if it is separable and normal.
\end{enumerate}
\end{definition}

\noindent
The following lemma does not seem to fit anywhere else.

\begin{lemma}
\label{lemma-pth-root}
Let $K$ be a field of characteristic $p > 0$. Let $K \subset L$ be a separable
algebraic extension. Let $\alpha \in L$.
\begin{enumerate}
\item If the coefficients of the minimal polynomial of $\alpha$
over $K$ are $p$th powers in $K$ then $\alpha$ is a $p$th
power in $L$.
\item More generally, if $P \in K[T]$ is a polynomial such that (a) $\alpha$
is a root of $P$, (b) $P$ has pairwise distinct roots in an algebraic closure,
and (c) all coefficients of $P$ are $p$th powers, then $\alpha$ is a
$p$th power in $L$.
\end{enumerate}
\end{lemma}

\begin{proof}
It follows from the definitions that (2) implies (1). Assume $P$ is as in (2).
Write $P(T) = \sum\nolimits_{i = 0}^d a_i T^{d - i}$ and $a_i = b_i^p$.
The polynomial $Q(T) = \sum\nolimits_{i = 0}^d b_i T^{d - i}$ has distinct
roots in an algebraic closure as well, because the roots of $Q$
are the $p$th roots of the roots of $P$. If $\alpha$ is not a $p$th power,
then $T^p - \alpha$ is an irreducible polynomial over $L$
(Lemma \ref{lemma-take-pth-root}).
Moreover $Q$ and $T^p - \alpha$ have a root in common in
an algebraic closure $\overline{L}$.
Thus $Q$ and $T^p - \alpha$ are not relatively prime, which
implies $T^p - \alpha | Q$ in $L[T]$. This contradicts the fact that
the roots of $Q$ are pairwise distinct.
\end{proof}












\begin{multicols}{2}[\section{Other chapters}]
\noindent
Preliminaries
\begin{enumerate}
\item \hyperref[introduction-section-phantom]{Introduction}
\item \hyperref[conventions-section-phantom]{Conventions}
\item \hyperref[sets-section-phantom]{Set Theory}
\item \hyperref[categories-section-phantom]{Categories}
\item \hyperref[topology-section-phantom]{Topology}
\item \hyperref[sheaves-section-phantom]{Sheaves on Spaces}
\item \hyperref[sites-section-phantom]{Sites and Sheaves}
\item \hyperref[stacks-section-phantom]{Stacks}
\item \hyperref[fields-section-phantom]{Fields}
\item \hyperref[algebra-section-phantom]{Commutative Algebra}
\item \hyperref[brauer-section-phantom]{Brauer Groups}
\item \hyperref[homology-section-phantom]{Homological Algebra}
\item \hyperref[derived-section-phantom]{Derived Categories}
\item \hyperref[simplicial-section-phantom]{Simplicial Methods}
\item \hyperref[more-algebra-section-phantom]{More on Algebra}
\item \hyperref[smoothing-section-phantom]{Smoothing Ring Maps}
\item \hyperref[modules-section-phantom]{Sheaves of Modules}
\item \hyperref[sites-modules-section-phantom]{Modules on Sites}
\item \hyperref[injectives-section-phantom]{Injectives}
\item \hyperref[cohomology-section-phantom]{Cohomology of Sheaves}
\item \hyperref[sites-cohomology-section-phantom]{Cohomology on Sites}
\item \hyperref[dga-section-phantom]{Differential Graded Algebra}
\item \hyperref[dpa-section-phantom]{Divided Power Algebra}
\item \hyperref[hypercovering-section-phantom]{Hypercoverings}
\end{enumerate}
Schemes
\begin{enumerate}
\setcounter{enumi}{24}
\item \hyperref[schemes-section-phantom]{Schemes}
\item \hyperref[constructions-section-phantom]{Constructions of Schemes}
\item \hyperref[properties-section-phantom]{Properties of Schemes}
\item \hyperref[morphisms-section-phantom]{Morphisms of Schemes}
\item \hyperref[coherent-section-phantom]{Cohomology of Schemes}
\item \hyperref[divisors-section-phantom]{Divisors}
\item \hyperref[limits-section-phantom]{Limits of Schemes}
\item \hyperref[varieties-section-phantom]{Varieties}
\item \hyperref[topologies-section-phantom]{Topologies on Schemes}
\item \hyperref[descent-section-phantom]{Descent}
\item \hyperref[perfect-section-phantom]{Derived Categories of Schemes}
\item \hyperref[more-morphisms-section-phantom]{More on Morphisms}
\item \hyperref[flat-section-phantom]{More on Flatness}
\item \hyperref[groupoids-section-phantom]{Groupoid Schemes}
\item \hyperref[more-groupoids-section-phantom]{More on Groupoid Schemes}
\item \hyperref[etale-section-phantom]{\'Etale Morphisms of Schemes}
\end{enumerate}
Topics in Scheme Theory
\begin{enumerate}
\setcounter{enumi}{40}
\item \hyperref[chow-section-phantom]{Chow Homology}
\item \hyperref[intersection-section-phantom]{Intersection Theory}
\item \hyperref[weil-section-phantom]{Weil Cohomology Theories}
\item \hyperref[pic-section-phantom]{Picard Schemes of Curves}
\item \hyperref[adequate-section-phantom]{Adequate Modules}
\item \hyperref[dualizing-section-phantom]{Dualizing Complexes}
\item \hyperref[duality-section-phantom]{Duality for Schemes}
\item \hyperref[discriminant-section-phantom]{Discriminants and Differents}
\item \hyperref[local-cohomology-section-phantom]{Local Cohomology}
\item \hyperref[algebraization-section-phantom]{Algebraic and Formal Geometry}
\item \hyperref[curves-section-phantom]{Algebraic Curves}
\item \hyperref[resolve-section-phantom]{Resolution of Surfaces}
\item \hyperref[models-section-phantom]{Semistable Reduction}
\item \hyperref[pione-section-phantom]{Fundamental Groups of Schemes}
\item \hyperref[etale-cohomology-section-phantom]{\'Etale Cohomology}
\item \hyperref[crystalline-section-phantom]{Crystalline Cohomology}
\item \hyperref[proetale-section-phantom]{Pro-\'etale Cohomology}
\item \hyperref[more-etale-section-phantom]{More \'Etale Cohomology}
\item \hyperref[trace-section-phantom]{The Trace Formula}
\end{enumerate}
Algebraic Spaces
\begin{enumerate}
\setcounter{enumi}{59}
\item \hyperref[spaces-section-phantom]{Algebraic Spaces}
\item \hyperref[spaces-properties-section-phantom]{Properties of Algebraic Spaces}
\item \hyperref[spaces-morphisms-section-phantom]{Morphisms of Algebraic Spaces}
\item \hyperref[decent-spaces-section-phantom]{Decent Algebraic Spaces}
\item \hyperref[spaces-cohomology-section-phantom]{Cohomology of Algebraic Spaces}
\item \hyperref[spaces-limits-section-phantom]{Limits of Algebraic Spaces}
\item \hyperref[spaces-divisors-section-phantom]{Divisors on Algebraic Spaces}
\item \hyperref[spaces-over-fields-section-phantom]{Algebraic Spaces over Fields}
\item \hyperref[spaces-topologies-section-phantom]{Topologies on Algebraic Spaces}
\item \hyperref[spaces-descent-section-phantom]{Descent and Algebraic Spaces}
\item \hyperref[spaces-perfect-section-phantom]{Derived Categories of Spaces}
\item \hyperref[spaces-more-morphisms-section-phantom]{More on Morphisms of Spaces}
\item \hyperref[spaces-flat-section-phantom]{Flatness on Algebraic Spaces}
\item \hyperref[spaces-groupoids-section-phantom]{Groupoids in Algebraic Spaces}
\item \hyperref[spaces-more-groupoids-section-phantom]{More on Groupoids in Spaces}
\item \hyperref[bootstrap-section-phantom]{Bootstrap}
\item \hyperref[spaces-pushouts-section-phantom]{Pushouts of Algebraic Spaces}
\end{enumerate}
Topics in Geometry
\begin{enumerate}
\setcounter{enumi}{76}
\item \hyperref[spaces-chow-section-phantom]{Chow Groups of Spaces}
\item \hyperref[groupoids-quotients-section-phantom]{Quotients of Groupoids}
\item \hyperref[spaces-more-cohomology-section-phantom]{More on Cohomology of Spaces}
\item \hyperref[spaces-simplicial-section-phantom]{Simplicial Spaces}
\item \hyperref[spaces-duality-section-phantom]{Duality for Spaces}
\item \hyperref[formal-spaces-section-phantom]{Formal Algebraic Spaces}
\item \hyperref[restricted-section-phantom]{Restricted Power Series}
\item \hyperref[spaces-resolve-section-phantom]{Resolution of Surfaces Revisited}
\end{enumerate}
Deformation Theory
\begin{enumerate}
\setcounter{enumi}{84}
\item \hyperref[formal-defos-section-phantom]{Formal Deformation Theory}
\item \hyperref[defos-section-phantom]{Deformation Theory}
\item \hyperref[cotangent-section-phantom]{The Cotangent Complex}
\item \hyperref[examples-defos-section-phantom]{Deformation Problems}
\end{enumerate}
Algebraic Stacks
\begin{enumerate}
\setcounter{enumi}{88}
\item \hyperref[algebraic-section-phantom]{Algebraic Stacks}
\item \hyperref[examples-stacks-section-phantom]{Examples of Stacks}
\item \hyperref[stacks-sheaves-section-phantom]{Sheaves on Algebraic Stacks}
\item \hyperref[criteria-section-phantom]{Criteria for Representability}
\item \hyperref[artin-section-phantom]{Artin's Axioms}
\item \hyperref[quot-section-phantom]{Quot and Hilbert Spaces}
\item \hyperref[stacks-properties-section-phantom]{Properties of Algebraic Stacks}
\item \hyperref[stacks-morphisms-section-phantom]{Morphisms of Algebraic Stacks}
\item \hyperref[stacks-limits-section-phantom]{Limits of Algebraic Stacks}
\item \hyperref[stacks-cohomology-section-phantom]{Cohomology of Algebraic Stacks}
\item \hyperref[stacks-perfect-section-phantom]{Derived Categories of Stacks}
\item \hyperref[stacks-introduction-section-phantom]{Introducing Algebraic Stacks}
\item \hyperref[stacks-more-morphisms-section-phantom]{More on Morphisms of Stacks}
\item \hyperref[stacks-geometry-section-phantom]{The Geometry of Stacks}
\end{enumerate}
Topics in Moduli Theory
\begin{enumerate}
\setcounter{enumi}{102}
\item \hyperref[moduli-section-phantom]{Moduli Stacks}
\item \hyperref[moduli-curves-section-phantom]{Moduli of Curves}
\end{enumerate}
Miscellany
\begin{enumerate}
\setcounter{enumi}{104}
\item \hyperref[examples-section-phantom]{Examples}
\item \hyperref[exercises-section-phantom]{Exercises}
\item \hyperref[guide-section-phantom]{Guide to Literature}
\item \hyperref[desirables-section-phantom]{Desirables}
\item \hyperref[coding-section-phantom]{Coding Style}
\item \hyperref[obsolete-section-phantom]{Obsolete}
\item \hyperref[fdl-section-phantom]{GNU Free Documentation License}
\item \hyperref[index-section-phantom]{Auto Generated Index}
\end{enumerate}
\end{multicols}


\bibliography{my}
\bibliographystyle{amsalpha}

\end{document}
