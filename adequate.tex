\IfFileExists{stacks-project.cls}{%
\documentclass{stacks-project}
}{%
\documentclass{amsart}
}

% The following AMS packages are automatically loaded with
% the amsart documentclass:
%\usepackage{amsmath}
%\usepackage{amssymb}
%\usepackage{amsthm}

% For dealing with references we use the comment environment
\usepackage{verbatim}
\newenvironment{reference}{\comment}{\endcomment}
%\newenvironment{reference}{}{}
\newenvironment{slogan}{\comment}{\endcomment}
\newenvironment{history}{\comment}{\endcomment}

% For commutative diagrams you can use
% \usepackage{amscd}
\usepackage[all]{xy}

% We use 2cell for 2-commutative diagrams.
\xyoption{2cell}
\UseAllTwocells

% To put source file link in headers.
% Change "template.tex" to "this_filename.tex"
% \usepackage{fancyhdr}
% \pagestyle{fancy}
% \lhead{}
% \chead{}
% \rhead{Source file: \url{template.tex}}
% \lfoot{}
% \cfoot{\thepage}
% \rfoot{}
% \renewcommand{\headrulewidth}{0pt}
% \renewcommand{\footrulewidth}{0pt}
% \renewcommand{\headheight}{12pt}

\usepackage{multicol}

% For cross-file-references
\usepackage{xr-hyper}

% Package for hypertext links:
\usepackage{hyperref}

% For any local file, say "hello.tex" you want to link to please
% use \externaldocument[hello-]{hello}
\externaldocument[introduction-]{introduction}
\externaldocument[conventions-]{conventions}
\externaldocument[sets-]{sets}
\externaldocument[categories-]{categories}
\externaldocument[topology-]{topology}
\externaldocument[sheaves-]{sheaves}
\externaldocument[sites-]{sites}
\externaldocument[stacks-]{stacks}
\externaldocument[fields-]{fields}
\externaldocument[algebra-]{algebra}
\externaldocument[brauer-]{brauer}
\externaldocument[homology-]{homology}
\externaldocument[derived-]{derived}
\externaldocument[simplicial-]{simplicial}
\externaldocument[more-algebra-]{more-algebra}
\externaldocument[smoothing-]{smoothing}
\externaldocument[modules-]{modules}
\externaldocument[sites-modules-]{sites-modules}
\externaldocument[injectives-]{injectives}
\externaldocument[cohomology-]{cohomology}
\externaldocument[sites-cohomology-]{sites-cohomology}
\externaldocument[dga-]{dga}
\externaldocument[dpa-]{dpa}
\externaldocument[hypercovering-]{hypercovering}
\externaldocument[schemes-]{schemes}
\externaldocument[constructions-]{constructions}
\externaldocument[properties-]{properties}
\externaldocument[morphisms-]{morphisms}
\externaldocument[coherent-]{coherent}
\externaldocument[divisors-]{divisors}
\externaldocument[limits-]{limits}
\externaldocument[varieties-]{varieties}
\externaldocument[topologies-]{topologies}
\externaldocument[descent-]{descent}
\externaldocument[perfect-]{perfect}
\externaldocument[more-morphisms-]{more-morphisms}
\externaldocument[flat-]{flat}
\externaldocument[groupoids-]{groupoids}
\externaldocument[more-groupoids-]{more-groupoids}
\externaldocument[etale-]{etale}
\externaldocument[chow-]{chow}
\externaldocument[intersection-]{intersection}
\externaldocument[pic-]{pic}
\externaldocument[adequate-]{adequate}
\externaldocument[dualizing-]{dualizing}
\externaldocument[duality-]{duality}
\externaldocument[discriminant-]{discriminant}
\externaldocument[local-cohomology-]{local-cohomology}
\externaldocument[curves-]{curves}
\externaldocument[resolve-]{resolve}
\externaldocument[models-]{models}
\externaldocument[pione-]{pione}
\externaldocument[etale-cohomology-]{etale-cohomology}
\externaldocument[proetale-]{proetale}
\externaldocument[crystalline-]{crystalline}
\externaldocument[spaces-]{spaces}
\externaldocument[spaces-properties-]{spaces-properties}
\externaldocument[spaces-morphisms-]{spaces-morphisms}
\externaldocument[decent-spaces-]{decent-spaces}
\externaldocument[spaces-cohomology-]{spaces-cohomology}
\externaldocument[spaces-limits-]{spaces-limits}
\externaldocument[spaces-divisors-]{spaces-divisors}
\externaldocument[spaces-over-fields-]{spaces-over-fields}
\externaldocument[spaces-topologies-]{spaces-topologies}
\externaldocument[spaces-descent-]{spaces-descent}
\externaldocument[spaces-perfect-]{spaces-perfect}
\externaldocument[spaces-more-morphisms-]{spaces-more-morphisms}
\externaldocument[spaces-flat-]{spaces-flat}
\externaldocument[spaces-groupoids-]{spaces-groupoids}
\externaldocument[spaces-more-groupoids-]{spaces-more-groupoids}
\externaldocument[bootstrap-]{bootstrap}
\externaldocument[spaces-pushouts-]{spaces-pushouts}
\externaldocument[groupoids-quotients-]{groupoids-quotients}
\externaldocument[spaces-more-cohomology-]{spaces-more-cohomology}
\externaldocument[spaces-simplicial-]{spaces-simplicial}
\externaldocument[formal-spaces-]{formal-spaces}
\externaldocument[restricted-]{restricted}
\externaldocument[spaces-resolve-]{spaces-resolve}
\externaldocument[formal-defos-]{formal-defos}
\externaldocument[defos-]{defos}
\externaldocument[cotangent-]{cotangent}
\externaldocument[examples-defos-]{examples-defos}
\externaldocument[algebraic-]{algebraic}
\externaldocument[examples-stacks-]{examples-stacks}
\externaldocument[stacks-sheaves-]{stacks-sheaves}
\externaldocument[criteria-]{criteria}
\externaldocument[artin-]{artin}
\externaldocument[quot-]{quot}
\externaldocument[stacks-properties-]{stacks-properties}
\externaldocument[stacks-morphisms-]{stacks-morphisms}
\externaldocument[stacks-limits-]{stacks-limits}
\externaldocument[stacks-cohomology-]{stacks-cohomology}
\externaldocument[stacks-perfect-]{stacks-perfect}
\externaldocument[stacks-introduction-]{stacks-introduction}
\externaldocument[stacks-more-morphisms-]{stacks-more-morphisms}
\externaldocument[stacks-geometry-]{stacks-geometry}
\externaldocument[moduli-]{moduli}
\externaldocument[moduli-curves-]{moduli-curves}
\externaldocument[examples-]{examples}
\externaldocument[exercises-]{exercises}
\externaldocument[guide-]{guide}
\externaldocument[desirables-]{desirables}
\externaldocument[coding-]{coding}
\externaldocument[obsolete-]{obsolete}
\externaldocument[fdl-]{fdl}
\externaldocument[index-]{index}

% Theorem environments.
%
\theoremstyle{plain}
\newtheorem{theorem}[subsection]{Theorem}
\newtheorem{proposition}[subsection]{Proposition}
\newtheorem{lemma}[subsection]{Lemma}

\theoremstyle{definition}
\newtheorem{definition}[subsection]{Definition}
\newtheorem{example}[subsection]{Example}
\newtheorem{exercise}[subsection]{Exercise}
\newtheorem{situation}[subsection]{Situation}

\theoremstyle{remark}
\newtheorem{remark}[subsection]{Remark}
\newtheorem{remarks}[subsection]{Remarks}

\numberwithin{equation}{subsection}

% Macros
%
\def\lim{\mathop{\rm lim}\nolimits}
\def\colim{\mathop{\rm colim}\nolimits}
\def\Spec{\mathop{\rm Spec}}
\def\Hom{\mathop{\rm Hom}\nolimits}
\def\Ext{\mathop{\rm Ext}\nolimits}
\def\SheafHom{\mathop{\mathcal{H}\!{\it om}}\nolimits}
\def\SheafExt{\mathop{\mathcal{E}\!{\it xt}}\nolimits}
\def\Sch{\textit{Sch}}
\def\Mor{\mathop{\rm Mor}\nolimits}
\def\Ob{\mathop{\rm Ob}\nolimits}
\def\Sh{\mathop{\textit{Sh}}\nolimits}
\def\NL{\mathop{N\!L}\nolimits}
\def\proetale{{pro\text{-}\acute{e}tale}}
\def\etale{{\acute{e}tale}}
\def\QCoh{\textit{QCoh}}
\def\Ker{\mathop{\rm Ker}}
\def\Im{\mathop{\rm Im}}
\def\Coker{\mathop{\rm Coker}}
\def\Coim{\mathop{\rm Coim}}

%
% Macros for moduli stacks/spaces
%
\def\QCohstack{\mathcal{QC}\!{\it oh}}
\def\Cohstack{\mathcal{C}\!{\it oh}}
\def\Spacesstack{\mathcal{S}\!{\it paces}}
\def\Quotfunctor{{\rm Quot}}
\def\Hilbfunctor{{\rm Hilb}}
\def\Curvesstack{\mathcal{C}\!{\it urves}}
\def\Polarizedstack{\mathcal{P}\!{\it olarized}}
\def\Complexesstack{\mathcal{C}\!{\it omplexes}}
% \Pic is the operator that assigns to X its picard group, usage \Pic(X)
% \Picardstack_{X/B} denotes the Picard stack of X over B
% \Picardfunctor_{X/B} denotes the Picard functor of X over B
\def\Pic{\mathop{\rm Pic}\nolimits}
\def\Picardstack{\mathcal{P}\!{\it ic}}
\def\Picardfunctor{{\rm Pic}}
\def\Deformationcategory{\mathcal{D}\!{\it ef}}


% OK, start here.
%
\begin{document}

\title{Adequate Modules}


\maketitle

\phantomsection
\label{section-phantom}

\tableofcontents

\section{Introduction}
\label{section-introduction}

\noindent
For any scheme $X$ the category $\textit{QCoh}(\mathcal{O}_X)$
of quasi-coherent modules is abelian and a Serre subcategory
of the abelian category of all $\mathcal{O}_X$-modules. The same
thing works for the catetegory of quasi-coherent modules on
an algebraic space $X$ viewed as a subcategory of the category
of all $\mathcal{O}_X$-modules on the small \'etale site of $X$.
Moreover, for a quasi-compact and quasi-separated morphism
$f : X \to Y$ the pushforward $f_*$ and higher direct images
preserve quasi-coherency.

\medskip\noindent
Next, let $X$ be a scheme and let $\mathcal{O}$ be the structure
sheaf on one of the big sites of $X$, say, the big fppf site.
The category of quasi-coherent $\mathcal{O}$-modules is abelian
(in fact it is equivalent to the category of usual quasi-coherent
$\mathcal{O}_X$-modules on the scheme $X$ we mentioned above)
but its imbedding into $\textit{Mod}(\mathcal{O})$ is not exact.
An example is the map of quasi-coherent modules
$$
\mathcal{O}_{\mathbf{A}^1_k}
\longrightarrow
\mathcal{O}_{\mathbf{A}^1_k}
$$
on $\mathbf{A}^1_k = \text{Spec}(k[x])$ given by multiplication by $x$.
In the abelian category of quasi-coherent sheaves this map is injective,
whereas in the abelian category of all $\mathcal{O}$-modules on the
big site of $\mathbf{A}^1_k$ this map has a nontrivial kernel as we
see by evaluating on sections over $\text{Spec}(k[x]/(x)) = \text{Spec}(k)$.
Moreover, for a quasi-compact and quasi-separated morphism
$f : X \to Y$ the functor $f_{big, *}$ does not preserve quasi-coherency.

\medskip\noindent
In this chapter we introduce a larger category of modules, closely related
to quasi-coherent modules, which ``fixes'' the two problems mentioned above.



\section{Adequate functors}
\label{section-quasi-coherent}

\noindent
In this section we discuss a topic closely related to
direct images of quasi-coherent sheaves. Most of this material
was taken from the paper \cite{Jaffe}.

\medskip\noindent
Let $A$ be a ring. We will consider covariant functors $F$ defined on the
category of $A$-algebras $B$. In this section will always assume that
\begin{enumerate}
\item $F(B)$ is endowed with the structure of a $B$-module, and
\item for any map $B \to B'$ of $A$-algebras the map
$F(B) \to F(B')$ is $B$-linear.
\end{enumerate}
Unless specifically mentioned otherwise, we will consider only
transformations of functors $\varphi : F \to G$
such that $F(B) \to G(B)$ is $B$-linear for all $B$. Note that in this case
$\text{Ker}(\varphi)$ and $\text{Coker}(\varphi)$ are functors of the
same kind. Thus the category of these functors is an abelian category
(for settheoretic issues, see
Remark \ref{remark-settheoretic}).

\medskip\noindent
An important special case of such a functor comes about as follows.
Let $M$ be an $A$-module. Then we will denote $\underline{M}$ the functor
$B \mapsto M \otimes_A B$ (with obvious $B$-module structure).
Note that if $M \to N$ is a map of $A$-modules then there is an
associated map $\underline{M} \to \underline{N}$ of functors.
Conversely, any map of functors $\underline{M} \to \underline{N}$
comes from an $A$-module map $M \to N$ as the reader can see by
evaluating on $B = A$.

\begin{definition}
\label{definition-adequate}
A functor $F$ as above is called
\begin{enumerate}
\item {\it adequate}\footnote{These are the module-quasi-coherent functors
of the paper \cite{Jaffe}, see \cite[Corollary 3.7]{Jaffe}.
The linearly adequate functors are called linearly representable
in that paper.} if there exists a
map of $A$-modules $M \to N$ such that $F$ is isomorphic to
$\text{Ker}(\underline{M} \to \underline{N})$.
\item {\it linearly adequate} if there exists an exact sequence
$0 \to F \to \underline{A^{\oplus n}} \to \underline{A^{\oplus m}}$.
\end{enumerate}
\end{definition}

\noindent
It is clear that given $\varphi : M \to N$ then
$\text{Coker}(\underline{M} \to \underline{N}) =
\underline{Q}$ where $Q = \text{Coker}(M \to N)$. But this isn't the case
for the kernel in general: for example an injective map of
$A$-modules need not be injective after base change.

\begin{lemma}
\label{lemma-adequate-finite-presentation}
Let $F$ be an adequate functor. If $B = \text{colim}\ B_i$ is a filtered
colimit of $A$-algebras, then $F(B) = \text{colim}\ F(B_i)$.
\end{lemma}

\begin{proof}
This holds because for any $A$-module $M$ we have
$M \otimes_A B = \text{colim}\ M \otimes_A B_i$ (see
Algebra, Lemma \ref{algebra-lemma-tensor-products-commute-with-limits})
and because filtered colimits commute with exact sequences, see
Algebra, Lemma \ref{algebra-lemma-directed-colimit-exact}.
\end{proof}

\begin{remark}
\label{remark-settheoretic}
Consider the category $\textit{FP}_A$ whose objects are $A$-algebras $B$
of the form $B = A[x_1, \ldots, x_n]/(f_1, \ldots, f_m)$ and whose morphisms
are $A$-algebra maps. Every $B$ algebra is a filtered colimit of finitely
presented $A$-algebra, i.e., a filtered colimit of objects of
$\textit{FP}_A$. By
Lemma \ref{lemma-adequate-finite-presentation}
we conclude every adequate functor $F$ is determined by its restriction to
$\textit{FP}_A$. This will be true of all the functors discussed in this
section. We can therefore restrict to the functors
which have the property of
Lemma \ref{lemma-adequate-finite-presentation}
or equivalently to functors on $\textit{FP}_A$. These functors are simply
presheaves of $\mathcal{O}$-modules on the opposite category
$\textit{FP}_A^{opp}$ where $\mathcal{O}(B) = B$.
\end{remark}

\begin{lemma}
\label{lemma-adequate-flat}
Let $F$ be an adequate functor. If $B \to B'$ is flat, then
$F(B) \otimes_B B' \to F(B')$ is an isomorphism.
\end{lemma}

\begin{proof}
Choose an exact sequence $0 \to F \to \underline{M} \to \underline{N}$.
This gives the diagram
$$
\xymatrix{
0 \ar[r] & F(B) \otimes_B B' \ar[r] \ar[d] &
(M \otimes_A B)\otimes_B B' \ar[r] \ar[d] &
(N \otimes_A B)\otimes_B B' \ar[d] \\
0 \ar[r] & F(B') \ar[r] &
M \otimes_A B' \ar[r] &
N \otimes_A B'
}
$$
where the rows are exact (the top one because $B \to B'$ is flat).
Since the right two vertical arrows are isomorphisms, so is the
left one.
\end{proof}

\begin{lemma}
\label{lemma-adequate-surjection-from-linear}
Let $F$ be an adequate functor. Then there exists a
surjection $L \to F$ with $L$ a direct sum of linearly adequate functors.
\end{lemma}

\begin{proof}
Choose an exact sequence $0 \to F \to \underline{M} \to \underline{N}$
where $\underline{M} \to \underline{N}$ is given by
$\varphi : M \to N$. By
Lemma \ref{lemma-adequate-finite-presentation}
it suffices to construct $L \to F$ such that $L(B) \to F(B)$ is surjective
for every finitely presented $A$-algebra $B$. Hence it suffices to construct,
given a finitely presented $A$-algebra $B$ and an element $\xi \in F(B)$
a map $L \to F$ with $L$ linearly adequate such that $\xi$ is in the image
of $L(B) \to F(B)$.
(Because there is a set worth of such pairs $(B, \xi)$ up to isomorphism.)

\medskip\noindent
To do this write $\sum_{i = 1, \ldots, n} m_i \otimes b_i$ the image of
$\xi$ in $\underline{M}(B) = M \otimes_A B$. We know that
$\sum \varphi(m_i) \otimes b_i = 0$ in $N \otimes_A B$.
As $N$ is a filtered colimit of finitely presented $A$-modules, we can
find a finitely presented $A$-module $N'$, a commutative diagram
of $A$-modules
$$
\xymatrix{
A^{\oplus n} \ar[r] \ar[d]_{m_1, \ldots, m_n} & N' \ar[d] \\
M \ar[r] & N
}
$$
such that $(b_1, \ldots, b_n)$ maps to zero in $N' \otimes_A B$.
Choose a presentation $A^{\oplus l} \to A^{\oplus k} \to N' \to 0$.
Choose a lift $A^{\oplus n} \to A^{\oplus k}$ of the map
$A^{\oplus n} \to N'$ of the diagram. Then we see that there exist
$(c_1, \ldots, c_l) \in B^{\oplus l}$ such that
$(b_1, \ldots, b_n, c_1, \ldots, c_l)$ maps to zero in $B^{\oplus k}$
under the map $B^{\oplus n} \oplus B^{\oplus l} \to B^{\oplus k}$.
Consider the commutative diagram
$$
\xymatrix{
A^{\oplus n} \oplus A^{\oplus l} \ar[r] \ar[d] & A^{\oplus k} \ar[d] \\
M \ar[r] & N
}
$$
where the left vertical arrow is zero on the summand $A^{\oplus l}$.
Then we see that $L$ equal to the kernel of $\underline{A^{\oplus n + l}}
\to \underline{A^{\oplus k}}$ works because the element
$(b_1, \ldots, b_n, c_1, \ldots, c_l) \in L(B)$ maps to $\xi$.
\end{proof}

\noindent
Consider a graded $A$-algebra $B = \bigoplus_{d \geq 0} B_d$. Then there are
two $A$-algebra maps $p, a : B \to B[t, t^{-1}]$, namely $p : b \mapsto b$ and
$a : b \mapsto t^{\deg(b)} b$ where $b$ is homogeneous. If $F$ is a
functor as above, then we define
\begin{equation}
\label{equation-weight-k}
F(B)^{(k)} = \{\xi \in F(B) \mid t^k F(p)(\xi) = F(a)(\xi)\}.
\end{equation}
For functors which behave well with respect to flat ring extensions
this gives a direct sum decompostion. This amounts to the fact that
representations of $\mathbf{G}_m$ are completely reducible.

\begin{lemma}
\label{lemma-flat-functor-split}
Let $F$ be a functor such that for $B \to B'$ flat the map
$F(B) \otimes_B B' \to F(B')$ is an isomorphism.
Let $B$ be a graded $A$-algebra. Then
\begin{enumerate}
\item $F(B) = \bigoplus_{k \in \mathbf{Z}} F(B)^{(k)}$, and
\item the map $B \to B_0 \to B$ induces map $F(B) \to F(B)$
whose image is contained in $F(B)^{(0)}$.
\end{enumerate}
\end{lemma}

\begin{proof}
Let $x \in F(B)$. The map $p : B \to B[t, t^{-1}]$ is free
hence we know that
$$
F(B[t, t^{-1}]) =
\bigoplus\nolimits_{k \in \mathbf{Z}} F(p)(F(B)) \cdot t^k =
\bigoplus\nolimits_{k \in \mathbf{Z}} F(B) \cdot t^k
$$
as indicated we drop the $F(p)$ in the rest of the proof.
Write $F(a)(x) = \sum t^k x_k$ for some $x_k \in F(B)$.
Denote $\epsilon : B[t, t^{-1}] \to B$
the $B$-algebra map $t \mapsto 1$. Note that the compositions
$\epsilon \circ p, \epsilon \circ a : B \to B[t, t^{-1}] \to B$ are
the identity. Hence we see that
$$
x = F(\epsilon)(F(a)(x)) = F(\epsilon)(\sum t^k x_k) = \sum x_k.
$$
On the other hand, we claim that $x_k \in F(B)^{(k)}$. Namely, consider
the commutative diagram
$$
\xymatrix{
B \ar[r]_a \ar[d]_{a'} &
B[t, t^{-1}] \ar[d]^f \\
B[s, s^{-1}] \ar[r]^-g &
B[t, s, t^{-1}, s^{-1}]
}
$$
where $a'(b) = s^{\deg(b)}b$, $f(b) = b$, $f(t) = st$ and
$g(b) = t^{\deg(b)}b$ and $g(s) = s$. Then
$$
F(g)(F(a'))(x) = F(g)(\sum s^k x_k) =
\sum s^k F(a)(x_k)
$$
and going the other way we see
$$
F(f)(F(a))(x) = F(f)(\sum t^k x_k) = \sum (st)^k x_k.
$$
Since $B \to B[s, t, s^{-1}, t^{-1}]$ is free we see that
$F(B[t, s, t^{-1}, s^{-1}]) =
\bigoplus_{k, l \in \mathbf{Z}} F(B) \cdot t^ks^l$ and
comparing coefficients in the expressions above we find
$F(a)(x_k) = t^k x_k$ as desired.

\medskip\noindent
Finally, the image of $F(B_0) \to F(B)$ is contained in $F(B)^{(0)}$
because $B_0 \to B \xrightarrow{a} B[t, t^{-1}]$ is equal to
$B_0 \to B \xrightarrow{p} B[t, t^{-1}]$.
\end{proof}

\noindent
As a particular case of
Lemma \ref{lemma-flat-functor-split}
note that
$$
\underline{M}(B)^{(k)} = M \otimes_A B_k
$$
where $B_k$ is the degree $k$ part of the graded $A$-algebra $B$.

\begin{lemma}
\label{lemma-lift-map}
Given a solid diagram
$$
\xymatrix{
0 \ar[r] &
L \ar[d]_\varphi \ar[r] &
\underline{A^{\oplus n}} \ar[r] \ar@{..>}[ld] &
\underline{A^{\oplus m}} \\
& \underline{M}
}
$$
with exact row there exists a dotted arrow making the diagram commute.
\end{lemma}

\begin{proof}
Suppose that the map $A^{\oplus n} \to A^{\oplus m}$ is given by the
$m \times n$-matrix $(a_{ij})$. Consider the ring
$B = A[x_1, \ldots, x_n]/(\sum a_{ij}x_j)$. The element
$(x_1, \ldots, x_n) \in \underline{A^{\oplus n}}(B)$ maps to zero in
$\underline{A^{\oplus m}}(B)$ hence is the image of a unique element
$\xi \in L(B)$. Note that $\xi$ has the following universal property:
for any $A$-algebra $C$ and any $\xi' \in L(C)$ there exists an $A$-algebra
map $B \to C$ such that $\xi$ maps to $\xi'$ via the map $L(B) \to L(C)$.

\medskip\noindent
Note that $B$ is a graded $A$-algebra, hence we can use
Lemmas \ref{lemma-flat-functor-split} and \ref{lemma-adequate-flat}
to decompose the values of our functors on $B$ into graded pieces.
Note that $\xi \in L(B)^{(1)}$ as $(x_1, \ldots, x_n)$ is an element
of degree one in $\underline{A^{\oplus n}}(B)$. Hence we see that
$\varphi(\xi) \in \underline{M}(B)^{(1)} = M \otimes_A B_1$.
Since $B_1$ is generated by $x_1, \ldots, x_n$ as an $A$-module we
can write $\varphi(\xi) = \sum m_i \otimes x_i$. Consider the map
$A^{\oplus n} \to M$ which maps the $i$th basis vector to $m_i$.
By construction the associated map
$\underline{A^{\oplus n}} \to \underline{M}$
maps the element $\xi$ to $\varphi(\xi)$. It follows from the
universal property mentioned above that the diagram commutes.
\end{proof}

\begin{lemma}
\label{lemma-cokernel-into-module}
Let $\varphi : F \to \underline{M}$ be a map of functors with $F$ adequate.
Then $\text{Coker}(\varphi)$ is adequate.
\end{lemma}

\begin{proof}
By
Lemma \ref{lemma-adequate-surjection-from-linear}
we may assume that $F = \bigoplus L_i$ is a direct sum of linearly adequate
functors. Choose exact sequences
$0 \to L_i \to \underline{A^{\oplus n_i}} \to \underline{A^{\oplus m_i}}$.
For each $i$ choose a map $A^{\oplus n_i} \to M$ as in
Lemma \ref{lemma-lift-map}.
Consider the diagram
$$
\xymatrix{
0 \ar[r] &
\bigoplus L_i  \ar[r] \ar[d] &
\bigoplus \underline{A^{\oplus n_i}} \ar[r] \ar[ld] &
\bigoplus \underline{A^{\oplus m_i}} \\
& \underline{M}
}
$$
Consider the $A$-modules
$$
Q =
\text{Coker}(\bigoplus A^{\oplus n_i} \to M \oplus \bigoplus A^{\oplus m_i})
\quad\text{and}\quad
P = \text{Coker}(\bigoplus A^{\oplus n_i} \to \bigoplus A^{\oplus m_i}).
$$
Then we see that $\text{Coker}(\varphi)$ is isomorphic to the
kernel of $\underline{Q} \to \underline{P}$.
\end{proof}

\begin{lemma}
\label{lemma-cokernel-adequate}
Let $\varphi : F \to G$ be a map of adequate functors.
Then $\text{Coker}(\varphi)$ is adequate.
\end{lemma}

\begin{proof}
Choose an injection $G \to \underline{M}$.
Then we have an injection $G/F \to \underline{M}/F$. By
Lemma \ref{lemma-cokernel-into-module}
we see that $\underline{M}/F$ is adequate, hence we can find an injection
$\underline{M}/F \to \underline{N}$.
Composing we obtain an injection $G/F \to \underline{N}$. By
Lemma \ref{lemma-cokernel-into-module}
the cokernel of the induced map $G \to \underline{N}$ is adequate
hence we can find an injection $\underline{N}/G \to \underline{K}$.
Then $0 \to G/F \to \underline{N} \to \underline{K}$ is exact and
we win.
\end{proof}

\begin{lemma}
\label{lemma-kernel-adequate}
Let $\varphi : F \to G$ be a map of adequate functors.
Then $\text{Ker}(\varphi)$ is adequate.
\end{lemma}

\begin{proof}
Choose an injection $F \to \underline{M}$ and an injection
$G \to \underline{N}$. Denote $F \to \underline{M \oplus N}$
the diagonal map so that
$$
\xymatrix{
F \ar[d] \ar[r] & G \ar[d] \\
\underline{M \oplus N} \ar[r] & \underline{N}
}
$$
commutes. By
Lemma \ref{lemma-cokernel-adequate}
we can find a module map $M \oplus N \to K$ such that
$F$ is the kernel of $\underline{M \oplus N} \to \underline{K}$.
Then $\text{Ker}(\varphi)$ is the kernel of
$\underline{M \oplus N} \to \underline{K \oplus N}$.
\end{proof}

\begin{lemma}
\label{lemma-colimit-adequate}
An arbitrary direct sum of adequate functors is adequate.
A colimit of adequate functors is adequate.
\end{lemma}

\begin{proof}
The statement on direct sums is immediate.
A general colimit can be written as a kernel of a map between
direct sums, see
Categories, Lemma \ref{categories-lemma-colimits-coproducts-coequalizers}.
Hence this follows from
Lemma \ref{lemma-kernel-adequate}.
\end{proof}

\begin{lemma}
\label{lemma-flat-linear-functor}
Let $F, G$ be functors as above. Let $\varphi : F \to G$ be a not
necessarily linear transformation of functors. Assume
\begin{enumerate}
\item $\varphi$ is additive,
\item for every $A$-algebra $B$ and $\xi \in F(B)$ and unit
$u \in B^*$ we have $\varphi(u\xi) = u\varphi(\xi)$ in $G(B)$, and
\item for any flat ring map $B \to B'$ we have
$G(B) \otimes_B B' = G(B')$.
\end{enumerate}
Then $\varphi$ is linear.
\end{lemma}

\begin{proof}
Let $B$ be an $A$-algebra, $\xi \in F(B)$, and $b \in B$. Consider the ring
map
$$
B \to B' = B[x, y, x^{-1}, y^{-1}]/(x + y - b).
$$
This ring map is faithfully flat, hence $G(B) \subset G(B')$. On the
other hand
$$
\varphi(b\xi) = \varphi((x + y)\xi) =
\varphi(x\xi) + \varphi(y\xi) = x\varphi(\xi) + y\varphi(\xi)
= (x + y)\varphi(\xi) = b\varphi(\xi)
$$
because $x, y$ are units in $B'$. Hence we win.
\end{proof}

\begin{lemma}
\label{lemma-extension-adequate-key}
Let $0 \to \underline{M} \to G \to L \to 0$ be a short exact sequence
of functors with $L$ linearly adequate. Then $G$ is adequate.
\end{lemma}

\begin{proof}
We first point out that for any flat $A$-algebra map
$B \to B'$ the map $G(B) \otimes_B B' \to G(B')$ is an isomorphism.
Namely, this holds for $\underline{M}$ and $L$, see
Lemma \ref{lemma-adequate-flat}
and hence follows for $G$ by the five lemma. In particular, by
Lemma \ref{lemma-flat-functor-split}
we see that $G(B) = \bigoplus_{k \in \mathbf{Z}} G(B)^{(k)}$
for any graded $A$-algebra $B$.

\medskip\noindent
Choose an exact sequence
$0 \to L \to \underline{A^{\oplus n}} \to \underline{A^{\oplus m}}$.
Suppose that the map $A^{\oplus n} \to A^{\oplus m}$ is given by the
$m \times n$-matrix $(a_{ij})$. Consider the graded $A$-algebra
$B = A[x_1, \ldots, x_n]/(\sum a_{ij}x_j)$. The element
$(x_1, \ldots, x_n) \in \underline{A^{\oplus n}}(B)$ maps to zero in
$\underline{A^{\oplus m}}(B)$ hence is the image of a unique element
$\xi \in L(B)$. Observe that $\xi \in L(B)^{(1)}$. The map
$$
\text{Hom}_A(B, C) \longrightarrow L(C),\quad
f \longmapsto L(f)(\xi)
$$
defines an isomorphism of functors. The reason is that $f$ is
determined by the images $c_i = f(x_i) \in C$ which have to
satisfy the relations $\sum a_{ij}c_j = 0$. And $L(C)$ is the
set of $n$-tuples $(c_1, \ldots, c_n)$ satisfying the relations
$\sum a_{ij} c_j = 0$.

\medskip\noindent
Since the value of each of the functors $\underline{M}$, $G$, $L$
on $B$ is a direct sum of its weight spaces (by the lemma mentioned
above) exactness of $0 \to \underline{M} \to G \to L \to 0$ implies
the sequence $0 \to \underline{M}(B)^{(1)} \to G(B)^{(1)} \to L(B)^{(1)} \to 0$
is exact. Thus we may choose an element $\theta \in G(B)^{(1)}$ mapping
to $\xi$.

\medskip\noindent
Consider the graded $A$-algebra
$$
C = A[x_1, \ldots, x_n, y_1, \ldots, y_n]/
(\sum a_{ij}x_j, \sum a_{ij}y_j)
$$
There are three graded $A$-algebra homomorphisms $p_1, p_2, m : B \to C$
defined by the rules
$$
p_1(x_i) = x_i,\quad
p_1(x_i) = y_i,\quad
m(x_i) = x_i + y_i.
$$
We will show that the element
$$
\tau = G(m)(\theta) - G(p_1)(\theta) - G(p_2)(\theta) \in G(C)
$$
is zero. First, $\tau$ maps to zero in $L(C)$ by a direct calculation.
Hence $\tau$ is an element of $\underline{M}(C)$.
Moreover, since $m$, $p_1$, $p_2$ are graded algebra maps we see
that $\tau \in G(C)^{(1)}$ and since $\underline{M} \subset G$
we conclude
$$
\tau \in \underline{M}(C)^{(1)} = M \otimes_A C_1.
$$
We may write uniquely
$\tau = \underline{M}(p_1)(\tau_1) + \underline{M}(p_2)(\tau_2)$ with
$\tau_i \in M \otimes_A B_1 = \underline{M}(B)^{(1)}$ because
$C_1 = p_1(B_1) \oplus p_2(B_1)$.
Consider the ring map $q_1 : C \to B$ defined by $x_i \mapsto x_i$ and
$y_i \mapsto 0$. Then
$\underline{M}(q_1)(\tau) =
\underline{M}(q_1)(\underline{M}(p_1)(\tau_1) + \underline{M}(p_2)(\tau_2)) =
\tau_1$.
On the other hand, because
$q_1 \circ m = q_1 \circ p_1$ we see that
$G(q_1)(\tau) = - G(q_1 \circ p_2)(\tau)$. Since $q_1 \circ p_2$ factors as
$B \to A \to B$ we see that $G(q_1 \circ p_2)(\tau)$ is in
$G(B)^{(0)}$, see
Lemma \ref{lemma-flat-functor-split}.
Hence $\tau_1 = 0$ because it is in
$G(B)^{(0)} \cap \underline{M}(B)^{(1)} \subset
G(B)^{(0)} \cap G(B)^{(1)} = 0$.
Similarly $\tau_2 = 0$, whence $\tau = 0$.

\medskip\noindent
Since $\theta \in G(B)$ we obtain a (not a priori linear)
transformation of functors
$$
\psi : L(-) = \text{Hom}_A(B, - ) \longrightarrow G(-)
$$
by mapping $f : B \to C$ to $G(f)(\theta)$. Since $\theta$ is a lift of
$\xi$ the map $\psi$ is a right inverse of $G \to L$. In terms of
$\psi$ the statements proved above have the following meaning:
$\tau = 0$ means that $\psi$ is additive and
$\theta \in G(B)^{(1)}$ implies that for any $A$-algebra $D$ we have
$\psi(ul) = u\psi(l)$ in $G(D)$ for $l \in L(D)$ and $u \in D^*$ a unit.
This implies that $\psi$ is a linear transformation of functors, see
Lemma \ref{lemma-flat-linear-functor}.
Clearly this implies that $G \cong \underline{M} \oplus L$ and we win.
\end{proof}

\begin{remark}
\label{remark-linearly-adequate}
The proof of
Lemma \ref{lemma-extension-adequate-key}
shows that any extension $0 \to \underline{M} \to E \to L \to 0$
with $L$ linearly adequate splits. It uses only the following properties
of the functor $F = \underline{M}$:
\begin{enumerate}
\item for a flat ring map $B \to B'$ the map $F(B) \otimes_B B' \to F(B')$
is an isomorphism, and
\item the fact that
$F(C)^{(1)} = F(p_1)(F(B)^{(1)}) \oplus F(p_2)(F(B)^{(1)})$
where $C = A[x_1, \ldots, x_n, y_1, \ldots, y_n]/
(\sum a_{ij}x_j, \sum a_{ij}y_j)$ and
$B = A[x_1, \ldots, x_n]/(\sum a_{ij}x_j)$.
\end{enumerate}
These two properties hold for any adequate functor $F$; details omitted.
Hence we see that $L$ is a projective object of the abelian category of
adequate functors.
\end{remark}

\begin{lemma}
\label{lemma-extension-adequate}
Let $0 \to F \to G \to H \to 0$ be a short exact sequence of functors.
If $F$ and $H$ are adequate, so is $G$.
\end{lemma}

\begin{proof}
Choose an exact sequence $0 \to F \to \underline{M} \to \underline{N}$.
If we can show that $(\underline{M} \oplus G)/F$ is adequate, then
$G$ is the kernel of the map of adequate functors
$(\underline{M} \oplus G)/F \to \underline{N}$, hence
adequate by
Lemma \ref{lemma-kernel-adequate}.
Thus we may assume $F = \underline{M}$.

\medskip\noindent
We can choose a surjection $L \to H$ where $L$ is a direct sum of
linearly adequate functors, see
Lemma \ref{lemma-adequate-surjection-from-linear}.
If we can show that the pullback $G \times_H L$ is adequate, then
$G$ is the cokernel of the map $\text{Ker}(L \to H) \to G \times_H L$
hence adequate by
Lemma \ref{lemma-cokernel-adequate}.
Thus we may assume that $H = \bigoplus L_i$ is a direct sum of
linearly adequate functors. By
Lemma \ref{lemma-extension-adequate-key}
each of the pullbacks $G \times_H L_i$ is adequate. By
Lemma \ref{lemma-colimit-adequate}
we see that $\bigoplus G \times_H L_i$ is adequate.
Then $G$ is the cokernel of
$$
\bigoplus\nolimits_{i \not = i'} F \longrightarrow
\bigoplus G \times_H L_i
$$
where $\xi$ in the summand $(i, i')$ maps to
$(0, \ldots, 0, \xi, 0, \ldots, 0,-\xi,0,\ldots, 0)$
with nonzero entries in the summands $i$ and $i'$.
Thus $G$ is adequate by
Lemma \ref{lemma-cokernel-adequate}.
\end{proof}

\begin{lemma}
\label{lemma-base-change-adequate}
Let $A \to A'$ be a ring map. If $F$ is an adequate functor on
$A$-algebras, then its restriction $F'$ to $A'$-algebras is adequate too.
\end{lemma}

\begin{proof}
Choose an exact sequence $0 \to F \to \underline{M} \to \underline{N}$.
Then $F'(B') = F(B') = \text{Ker}(M \otimes_A B' \to N \otimes_A B')$.
Since $M \otimes_A B' = M \otimes_A A' \otimes_{A'} B'$ and similarly
for $N$ we see that $F'$ is the kernel of
$\underline{M \otimes_A A'} \to \underline{N \otimes_A A'}$.
\end{proof}

\begin{lemma}
\label{lemma-pushforward-adequate}
Let $A \to A'$ be a ring map. If $F'$ is an adequate functor on
$A'$-algebras, then the functor
$F : B \mapsto F'(A' \otimes_A B)$ is adequate too.
\end{lemma}

\begin{proof}
Choose an exact sequence $0 \to F' \to \underline{M'} \to \underline{N'}$.
Then
\begin{align*}
F(B) & = F'(A' \otimes_A B) \\
& = \text{Ker}(M' \otimes_{A'} (
A' \otimes_A B) \to N' \otimes_{A'} (A' \otimes_A B)) \\
& = \text{Ker}(M' \otimes_A B \to N' \otimes_A B)
\end{align*}
Thus $F$ is the kernel of
$\underline{M} \to \underline{N}$
where $M = M'$ and $N = N'$ viewed as $A$-modules.
\end{proof}

\begin{lemma}
\label{lemma-adequate-product}
Let $A = A_1 \times \ldots \times A_n$ be a product of rings.
An adequate functor over $A$ is the same thing as a sequence
$F_1, \ldots, F_n$ of adequate functors $F_i$ over $A_i$.
\end{lemma}

\begin{proof}
This is true because an $A$-algebra $B$ is canonically a product
$B_1 \times \ldots \times B_n$ and the same thing holds for $A$-modules.
Setting $F(B) = \coprod F_i(B_i)$ gives the correspondence.
Details omitted.
\end{proof}

\begin{lemma}
\label{lemma-adjoint}
For every functor $F$ there exists a morphism $Q \to F$
such that (1) $Q$ is adequate and (2) for every adequate functor
$G$ the map $\text{Hom}(G, Q) \to \text{Hom}(G, F)$ is a bijection.
\end{lemma}

\begin{proof}
Choose a set $\{L_i\}_{i \in I}$ of linearly adequate functors such that
every linearly adequate functor is isomorphic to one of the $L_i$.
This is possible. Suppose that we can find $Q \to F$ with (1) and
(2)' or every $i \in I$ the map $\text{Hom}(L_i, Q) \to \text{Hom}(L_i, F)$
is a bijection. Then (2) holds. Namely, combining
Lemmas \ref{lemma-adequate-surjection-from-linear} and
\ref{lemma-kernel-adequate}
we see that every adequate functor $G$ sits in an exact sequence
$$
K \to L \to G \to 0
$$
with $K$ and $L$ direct sums of linearly adequate functors. Hence (2)'
implies that
$\text{Hom}(L, Q) \to \text{Hom}(L, F)$
and
$\text{Hom}(K, Q) \to \text{Hom}(K, F)$
are bijections, whence the same thing for $G$.

\medskip\noindent
Consider the category $\mathcal{I}$ whose objects are pairs
$(i, \varphi)$ where $i \in I$ and $\varphi : L_i \to F$ is a morphism.
A morphism $(i, \varphi) \to (i', \varphi')$ is a map
$\psi : L_i \to L_{i'}$ such that $\varphi' \circ \psi = \varphi$.
Set
$$
Q = \text{colim}_{(i, \varphi) \in \text{Ob}(\mathcal{I})}\ L_i
$$
There is a natural map $Q \to F$, by
Lemma \ref{lemma-colimit-adequate}
it is adequate, and by construction it has property (2)'.
\end{proof}










\section{Adequate modules}
\label{section-adequate}

\noindent
In this section we fix
$\tau \in \{Zar, \acute{e}tale, smooth, syntomic, fppf\}$
and we fix a big $\tau$-site $\textit{Sch}_\tau$ as in
Topologies, Section \ref{topologies-section-procedure}.
All schemes will be objects of $\textit{Sch}_\tau$ and all rings
$A$ will be such that $\text{Spec}(A)$ is (isomorphic to) an
object of $\textit{Sch}_\tau$.

\medskip\noindent
In
Section \ref{section-quasi-coherent-sheaves}
we have seen that quasi-coherent modules on a scheme $S$
are the same as quasi-coherent modules on any of the big
sites $(\textit{Sch}/S)_\tau$ associated to $S$. We have seen that there
are two issues with this identification:
\begin{enumerate}
\item $\textit{QCoh}(\mathcal{O}_S) \to
\textit{Mod}((\textit{Sch}/S)_\tau, \mathcal{O})$,
$\mathcal{F} \mapsto \mathcal{F}^a$ is not exact in general, and
\item given a quasi-compact and quasi-separated morphism $f : X \to S$
the functor $f_*$ does not preserve quasi-coherent sheaves on the
big sites in general.
\end{enumerate}
Part (1) means that we cannot define a triangulated subcategory
of $D(\mathcal{O})$ consisting of complexes whose cohomology sheaves
are quasi-coherent. Part (2) means that $Rf_*\mathcal{F}$ isn't a
complex with quasi-coherent cohomology sheaves even when $\mathcal{F}$
is quasi-coherent and $f$ is quasi-compact and quasi-separated.
Moreover, the examples given in the proofs of
Lemma \ref{lemma-equivalence-quasi-coherent-limits}
and
Proposition \ref{proposition-equivalence-quasi-coherent-functorial}
are not of a pathological nature.

\medskip\noindent
In this section we discuss a slightly larger category
of $\mathcal{O}$-modules on $(\textit{Sch}/S)_\tau$ with contains the
quasi-coherent modules, is abelian, and is preserved under $f_*$ when
$f$ is quasi-compact and quasi-separated.
To do this, suppose that $S$ is a scheme. Let $\mathcal{F}$ be a presheaf
of $\mathcal{O}$-modules on $(\textit{Sch}/S)_\tau$.
For any affine scheme $\text{Spec}(A) \to S$ over $S$ we can restrict
$\mathcal{F}$ to the category of affine schemes over $\text{Spec}(A)$.
Thus we obtain a functor
$$
F = F_{\mathcal{F}, A} :
A\text{-Alg} \longrightarrow \textit{Ab},
\quad
B \longmapsto \mathcal{F}(\text{Spec}(B))
$$
The value $F(B)$ is a $B$-module and for a ring map
$B \to B'$ the map $F(B) \to F(B')$ is $B$-linear. Thus this is exactly
the kind of functor studied in
More on Algebra, Section \ref{more-algebra-section-quasi-coherent}.
The rest of this section will scarcely make any sense without reading
a bit about the algebra involved in the reference above.

\begin{definition}
\label{definition-adequate}
A sheaf of $\mathcal{O}$-modules $\mathcal{F}$ on $(\textit{Sch}/S)_\tau$ is
{\it adequate} if there exists a $\tau$-covering
$\{\text{Spec}(A_i) \to S\}_{i \in I}$ such that $F_{\mathcal{F}, A_i}$ is
adequate for all $i \in I$.
The category of adequate $\mathcal{O}$-modules on $(\textit{Sch}/S)_\tau$
is denoted $\textit{Adeq}(\mathcal{O}_S)$.
\end{definition}

\noindent
We will see below that the category of adequate $\mathcal{O}$-modules
is independent of the chosen topology $\tau$, hence the notation makes sense.
We first prove some obligatory lemmas which show that the definition
makes sense.

\begin{lemma}
\label{lemma-adequate-descent}
Let $A \to A'$ be a ring map and let $F$ be a linear functor on
$A$-algebras such that
\begin{enumerate}
\item the restriction $F'$ of $F$ to the category of $A'$-algebras is
adequate, and
\item for any $A$-algebra $B$ the sequence
$$
0 \to F(B) \to F(B \otimes_A A') \to F(B \otimes_A A' \otimes_A A')
$$
is exact.
\end{enumerate}
Then $F$ is adequate.
\end{lemma}

\begin{proof}
The functors $B \to F(B \otimes_A A')$ and
$B \mapsto F(B \otimes_A A' \otimes_A A')$ are adequate, see
More on Algebra, Lemmas \ref{more-algebra-lemma-pushforward-adequate} and
\ref{more-algebra-lemma-base-change-adequate}.
Hence $F$ as a kernel of a map of adequate functors is adequate, see
More on Algebra, Lemma \ref{more-algebra-lemma-kernel-adequate}.
\end{proof}

\begin{lemma}
\label{lemma-adequate-local}
Let $\mathcal{F}$ be an adequate $\mathcal{O}$-module on
$(\textit{Sch}/S)_\tau$. For any affine scheme $\text{Spec}(A)$ over $S$
the functor $F_{\mathcal{F}, A}$ is adequate.
\end{lemma}

\begin{proof}
Let $\{\text{Spec}(A_i) \to S\}_{i \in I}$ be a $\tau$-covering
such that $F_{\mathcal{F}, A_i}$ is adequate for all $i \in I$.
We can find a standard affine $\tau$-covering
$\{\text{Spec}(A'_j) \to \text{Spec}(A)\}_{j = 1, \ldots, m}$
such that $\text{Spec}(A'_j) \to \text{Spec}(A) \to S$ factors
through $\text{Spec}(A_{i(j)})$ for some $i(j) \in I$. Then we see that
$F_{\mathcal{F}, A'_j}$ is the restriction of
$F_{\mathcal{F}, A_{i(j)}}$ to the category of $A'_j$-algebras.
Hence $F_{\mathcal{F}, A'_j}$ is adequate by
More on Algebra, Lemma \ref{more-algebra-lemma-base-change-adequate}.
By
More on Algebra, Lemma \ref{more-algebra-lemma-adequate-product}
the sequence
$F_{\mathcal{F}, A'_j}$ corresponds to an adequate ``product'' functor
$F'$ over $A' = A'_1 \times \ldots \times A'_m$. As $\mathcal{F}$ is a
sheaf (for the Zariski topology) this product functor $F'$ is equal
to $F_{\mathcal{F}, A'}$, i.e., is the restriction of $F$ to $A'$-algebras.
Finally,  $\{\text{Spec}(A') \to \text{Spec}(A)\}$ is a $\tau$-covering.
This reduces us to
Lemma \ref{lemma-adequate-descent}.
\end{proof}

\begin{lemma}
\label{lemma-pullback-adequate}
Let $f : T \to S$ be a morphism of schemes.
The pullback $f^*\mathcal{F}$ of an adequate $\mathcal{O}_S$-module
$\mathcal{F}$ on $(\textit{Sch}/S)_\tau$ is an adequate
$\mathcal{O}_T$-module on $(\textit{Sch}/T)_\tau$.
\end{lemma}

\begin{proof}
The pullback map
$f^* : \textit{Mod}((\textit{Sch}/S)_\tau, \mathcal{O}_S) \to
\textit{Mod}((\textit{Sch}/T)_\tau, \mathcal{O}_T)$
is given by restriction, i.e., $f^*\mathcal{F}(V) = \mathcal{F}(V)$
for any scheme $V$ over $T$. Hence this lemma follows immediately from
Lemma \ref{lemma-adequate-local}
and the definition.
\end{proof}

\noindent
Here is a characterization of the category of adequate $\mathcal{O}$-modules.
To understand the significance, consider a map
$\mathcal{G} \to \mathcal{H}$ of
quasi-coherent $\mathcal{O}_S$-modules on a scheme $S$.
The cokernel of the associated map $\mathcal{G}^a \to \mathcal{H}^a$
of $\mathcal{O}$-modules is quasi-coherent because it is equal to
$(\mathcal{H}/\mathcal{G})^a$. But the kernel of
$\mathcal{G}^a \to \mathcal{H}^a$ in general isn't
quasi-coherent. However, it is adequate.

\begin{lemma}
\label{lemma-adequate-characterize}
Let $S$ be a scheme. Let $\mathcal{F}$ be an $\mathcal{O}$-module on
$(\textit{Sch}/S)_\tau$. The following are equivalent
\begin{enumerate}
\item $\mathcal{F}$ is adequate,
\item there exists an affine open covering $S = \bigcup S_i$ and
maps of quasi-coherent modules $\mathcal{G}_i \to \mathcal{H}_i$
such that $\mathcal{F}|_{(\textit{Sch}/S_i)_\tau}$ is the
kernel of $\mathcal{G}_i^a \to \mathcal{H}_i^a$
\item there exists a $\tau$-covering $\{S_i \to S\}_{i \in I}$ and
maps of quasi-coherent modules $\mathcal{G}_i \to \mathcal{H}_i$
such that $\mathcal{F}|_{(\textit{Sch}/S_i)_\tau}$ is the
kernel of $\mathcal{G}_i^a \to \mathcal{H}_i^a$,
\item there exists a $\tau$-covering $\{f_i : S_i \to S\}_{i \in I}$
such that each $f_i^*\mathcal{F}$ is adequate,
\item for any affine scheme $U$ over $S$ the restriction
$\mathcal{F}|_{(\textit{Sch}/U)_\tau}$ is the kernel
of a map $\mathcal{G}^a \to \mathcal{H}^a$ of quasi-coherent
$\mathcal{O}_U$-modules.
\end{enumerate}
\end{lemma}

\begin{proof}
Let $U = \text{Spec}(A)$ be an affine scheme over $S$.
Set $F = F_{\mathcal{F}, A}$. By definition, the functor
$F$ is adequate if and only if there exists a map of $A$-modules
$M \to N$ such that $F(B) = \text{Ker}(M \otimes_A B \to N \otimes_A B)$.
Combining with
Lemma \ref{lemma-adequate-local}
we see that (1) and (5) are equivalent.

\medskip\noindent
It is clear that (5) implies (2) and (2) implies (3).
If (3) holds then we can refine the covering
$\{S_i \to S\}$ such that each $S_i = \text{Spec}(A_i)$ is affine.
Then we see, by the prelimiary remarks of the proof, that
$F_{\mathcal{F}, A_i}$ is adequate. Thus $\mathcal{F}$
is adequate by definition. Hence (3) implies (1).

\medskip\noindent
Finally, (4) is equivalent to (1) using
Lemma \ref{lemma-pullback-adequate}
for one direction and that
a composition of $\tau$-coverings is a $\tau$-covering for the other.
\end{proof}

\noindent
Just like is true for quasi-coherent sheaves the category of
adequate modules is independent of the topology.

\begin{lemma}
\label{lemma-adequate-fpqc}
Let $\mathcal{F}$ be an adequate $\mathcal{O}$-module on
$(\textit{Sch}/S)_\tau$. For any surjective flat morphism
$\text{Spec}(B) \to \text{Spec}(A)$ of affines over $S$
the extended {\v C}ech complex
$$
0 \to \mathcal{F}(\text{Spec}(A)) \to
\mathcal{F}(\text{Spec}(B)) \to
\mathcal{F}(\text{Spec}(B \otimes_A B)) \to \ldots
$$
is exact. In particular $\mathcal{F}$ satisfies the sheaf condition
for fpqc coverings, and is a sheaf of $\mathcal{O}$-modules
on $(\textit{Sch}/S)_{fppf}$.
\end{lemma}

\begin{proof}
With $A \to B$ as in the lemma let $F = F_{\mathcal{F}, A}$. This functor
is adequate by
Lemma \ref{lemma-adequate-local}.
By
More on Algebra, Lemma \ref{more-algebra-lemma-adequate-flat}
since $A \to B$, $A \to B \otimes_A B$, etc are flat we see that
$F(B) = F(A) \otimes_A B$,
$F(B \otimes_A B) = F(A) \otimes_A B \otimes_A B$, etc.
Exactness follows from
Lemma \ref{lemma-ff-exact}.

\medskip\noindent
Thus $\mathcal{F}$ satisfies the sheaf condition for
$\tau$-coverings (in particular Zariski coverings) and any faithfully
flat covering of an affine by an affine. Arguing as in the proofs of
Lemma \ref{lemma-standard-fpqc-covering}
and
Proposition \ref{proposition-fpqc-descent-quasi-coherent}
we conclude that $\mathcal{F}$ satisfies the sheaf condition for all
fpqc coverings (made out of objects of $(\textit{Sch}/S)_\tau$).
Details omitted.
\end{proof}

\begin{lemma}
\label{lemma-same-cohomology-adequate}
Let $S$ be a scheme. Let $\mathcal{F}$ be an adequate
$\mathcal{O}$-module on $(\textit{Sch}/S)_\tau$.
\begin{enumerate}
\item The restriction $\mathcal{F}|_{S_{Zar}}$ is a quasi-coherent
$\mathcal{O}_S$-module on the scheme $S$.
\item The restriction $\mathcal{F}|_{S_{\acute{e}tale}}$ is the
quasi-coherent module associated to $\mathcal{F}|_{S_{Zar}}$.
\item For any affine scheme $U$ over $S$ we have $H^q(U, \mathcal{F}) = 0$
for all $q > 0$.
\item There is a canonical isomorphism
$$
H^q(S, \mathcal{F}|_{S_{Zar}}) =
H^q((\textit{Sch}/S)_\tau, \mathcal{F}).
$$
\end{enumerate}
\end{lemma}

\begin{proof}
By
More on Algebra, Lemma \ref{more-algebra-lemma-adequate-flat}
and
Lemma \ref{lemma-adequate-local}
we see that for any flat morphism of affines $U \to V$ over $S$
we have
$\mathcal{F}(U) = \mathcal{F}(V) \otimes_{\mathcal{O}(V)} \mathcal{O}(U)$.
This works in particular if $U \subset V \subset S$ are affine opens of
$S$, hence $\mathcal{F}|_{S_{Zar}}$ is quasi-coherent.
Thus (1) holds.

\medskip\noindent
Let $S' \to S$ be an \'etale morphism of schemes.
Then for $U \subset S'$ affine open mapping into an affine open
$V \subset S$ we see that
$\mathcal{F}(U) = \mathcal{F}(V) \otimes_{\mathcal{O}(V)} \mathcal{O}(U)$
because $U \to V$ is \'etale, hence flat. Therefore
$\mathcal{F}|_{S'_{Zar}}$ is the pullback of $\mathcal{F}|_{S_{Zar}}$.
This proves (2).

\medskip\noindent
We are going to apply
Cohomology on Sites,
Lemma \ref{sites-cohomology-lemma-cech-vanish-collection}
to the site $(\textit{Sch}/S)_\tau$ with
$\mathcal{B}$ the set of affine schemes over $S$ and
$\text{Cov}$ the set of standard affine $\tau$-coverings.
Assumption (3) of the lemma is satisfied by
Lemma \ref{lemma-standard-covering-Cech}
and
Lemma \ref{lemma-adequate-fpqc}
for the case of a covering by a single affine.
Hence we conclude that $H^p(U, \mathcal{F}) = 0$ for every
affine scheme $U$ over $S$. This proves (3).
In exactly the same way as in the proof of
Proposition \ref{proposition-same-cohomology-quasi-coherent}
this implies the equality of cohomologies (4).
\end{proof}

\begin{remark}
\label{remark-compare}
Let $S$ be a scheme. We have functors
$u : \textit{QCoh}(\mathcal{O}_S) \to \textit{Adeq}(\mathcal{O}_S)$
and
$v : \textit{Adeq}(\mathcal{O}_S) \to \textit{QCoh}(\mathcal{O}_S)$.
Namely, the functor $u : \mathcal{F} \mapsto \mathcal{F}^a$
comes from taking the associated $\mathcal{O}$-module which is
adequate by
Lemma \ref{lemma-adequate-characterize}.
Conversely, the functor $v$ comes from restriction
$v : \mathcal{G} \mapsto \mathcal{G}|_{S_{Zar}}$, see
Lemma \ref{lemma-same-cohomology-adequate}.
Since $\mathcal{F}^a$ can be described as the pullback of
$\mathcal{F}$ under a morphism of ringed topoi
$((\textit{Sch}/S)_\tau, \mathcal{O}) \to (S_{Zar}, \mathcal{O}_S)$, see
Remark \ref{remark-change-topologies-ringed-sites}
and since restriction is the pushforward we see that $u$ and $v$
are adjoint as follows
$$
\textit{Hom}_{\mathcal{O}_S}(\mathcal{F}, v\mathcal{G})
=
\textit{Hom}_{\mathcal{O}}(u\mathcal{F}, \mathcal{G})
$$
where $\mathcal{O}$ denotes the structure sheaf on the big site.
It is immediate from the description that the adjunction mapping
$\mathcal{F} \to vu\mathcal{F}$ is an isomorphism for all quasi-coherent
sheaves.
\end{remark}

\begin{lemma}
\label{lemma-sheafification-adequate}
Let $S$ be a scheme. Let $\mathcal{F}$ be a presheaf of $\mathcal{O}$-modules
on $(\textit{Sch}/S)_\tau$. If for every affine scheme
$\text{Spec}(A)$ over $S$ the functor $F_{\mathcal{F}, A}$ is
adequate, then the sheafification of $\mathcal{F}$ is an adequate
$\mathcal{O}$-module.
\end{lemma}

\begin{proof}
Let $U = \text{Spec}(A)$ be an affine scheme over $S$.
Set $F = F_{\mathcal{F}, A}$.
The sheafification $\mathcal{F}^\# = (\mathcal{F}^+)^+$, see
Sites, Section \ref{sites-section-sheafification}.
By construction
$$
(\mathcal{F})^+(U) =
\text{colim}_{\mathcal{U}}\ \check{H}^0(\mathcal{U}, \mathcal{F})
$$
where the colimit is over coverings in the site $(\textit{Sch}/S)_\tau$.
Since $U$ is affine it suffices to take the limit over standard
affine $\tau$-coverings
$\mathcal{U} = \{U_i \to U\}_{i \in I} =
\{\text{Spec}(A_i) \to \text{Spec}(A)\}_{i \in I}$ of $U$.
Since each $A \to A_i$ and $A \to A_i \otimes_A A_j$ is flat we see that
$$
\check{H}^0(\mathcal{U}, \mathcal{F}) =
\text{Ker}(\prod F(A) \otimes_A A_i \to \prod F(A) \otimes_A A_i \otimes_A A_j)
$$
by
More on Algebra, Lemma \ref{more-algebra-lemma-adequate-flat}.
Since $A \to \prod A_i$ is faithfully flat we see that this always
is canonically isomorphic to $F(A)$ by
Lemma \ref{lemma-ff-exact}.
Thus the presheaf $(\mathcal{F})^+$ has the same value as
$\mathcal{F}$ on all affine schemes over $S$. Repeating the argument
once more we deduce the same thing for $\mathcal{F}^\# = ((\mathcal{F})^+)^+$.
Thus $F_{\mathcal{F}, A} = F_{\mathcal{F}^\#, A}$ and we conclude
that $\mathcal{F}^\#$ is adequate.
\end{proof}

\begin{lemma}
\label{lemma-abelian-adequate}
Let $S$ be a scheme.
\begin{enumerate}
\item The category $\textit{Adeq}(\mathcal{O}_S)$ is abelian.
\item The functor
$\textit{Adeq}(\mathcal{O}_S) \to
\textit{Mod}((\textit{Sch}/S)_\tau, \mathcal{O})$
is exact.
\item If $0 \to \mathcal{F}_1 \to \mathcal{F}_2 \to \mathcal{F}_3 \to 0$
is a short exact sequence of $\mathcal{O}$-modules and
$\mathcal{F}_1$ and $\mathcal{F}_3$ are adequate, then
$\mathcal{F}_2$ is adequate.
\item The category $\textit{Adeq}(\mathcal{O}_S)$ has colimits and
$\textit{Adeq}(\mathcal{O}_S) \to
\textit{Mod}((\textit{Sch}/S)_\tau, \mathcal{O})$
commutes with them.
\end{enumerate}
\end{lemma}

\begin{proof}
Let $\varphi : \mathcal{F} \to \mathcal{G}$ be a map of adequate
$\mathcal{O}$-modules. To prove (1) and (2) it suffices to show that
$\mathcal{K} = \text{Ker}(\varphi)$ and
$\mathcal{Q} = \text{Coker}(\varphi)$ computed in
$\textit{Mod}((\textit{Sch}/S)_\tau, \mathcal{O})$ are adequate.
Let $U = \text{Spec}(A)$ be an affine scheme over $S$.
Let $F = F_{\mathcal{F}, A}$ and $G = F_{\mathcal{G}, A}$.
By
More on Algebra, Lemmas \ref{more-algebra-lemma-kernel-adequate} and
\ref{more-algebra-lemma-cokernel-adequate}
the kernel $K$ and cokernel $Q$ of the induced map
$F \to G$ are adequate functors.
Because the kernel is computed on the level of presheaves, we see
that $K = F_{\mathcal{K}, A}$ and we conclude $\mathcal{K}$ is adequate.
To prove the result for the cokernel, denote $\mathcal{Q}'$ the presheaf
cokernel of $\varphi$. Then $Q = F_{\mathcal{Q}', A}$ and
$\mathcal{Q} = (\mathcal{Q}')^\#$. Hence $\mathcal{Q}$
is adequate by
Lemma \ref{lemma-sheafification-adequate}.

\medskip\noindent
Let $0 \to \mathcal{F}_1 \to \mathcal{F}_2 \to \mathcal{F}_3 \to 0$
is a short exact sequence of $\mathcal{O}$-modules and
$\mathcal{F}_1$ and $\mathcal{F}_3$ are adequate. 
Let $U = \text{Spec}(A)$ be an affine scheme over $S$.
Let $F_i = F_{\mathcal{F}_i, A}$. The sequence of functors
$$
0 \to F_1 \to F_2 \to F_3 \to 0
$$
is exact, because for $V = \text{Spec}(B)$ affine over $U$ we have
$H^1(V, \mathcal{F}_1) = 0$ by
Lemma \ref{lemma-same-cohomology-adequate}.
Since $F_1$ and $F_3$ are adequate functors by
Lemma \ref{lemma-adequate-local}
we see that $F_2$ is adequate by
More on Algebra, Lemma \ref{more-algebra-lemma-extension-adequate}.
Thus $\mathcal{F}_2$ is adequate.

\medskip\noindent
Let $\mathcal{I} \to \textit{Adeq}(\mathcal{O}_S)$, $i \mapsto \mathcal{F}_i$
be a diagram. Denote $\mathcal{F} = \text{colim}_i\ \mathcal{F}_i$
the colimit computed in
$\textit{Mod}((\textit{Sch}/S)_\tau, \mathcal{O})$.
To prove (4) it suffices to show that $\mathcal{F}$ is adequate.
Let $\mathcal{F}' = \text{colim}_i\ \mathcal{F}_i$ be the colimit computed
in presheaves of $\mathcal{O}$-modules. Then
$\mathcal{F} = (\mathcal{F}')^\#$.
Let $U = \text{Spec}(A)$ be an affine scheme over $S$.
Let $F_i = F_{\mathcal{F}_i, A}$. By
More on Algebra, Lemma \ref{more-algebra-lemma-colimit-adequate}
the functor $\text{colim}_i\ F_i = F_{\mathcal{F}', A}$ is adequate.
Lemma \ref{lemma-sheafification-adequate}
shows that $\mathcal{F}$ is adequate.
\end{proof}

\noindent
The following lemma tells us that the total direct image
$Rf_*\mathcal{F}$ of an adequate module under a quasi-compact and
quasi-separated morphism is a complex whose cohomology sheaves
are adequate.

\begin{lemma}
\label{lemma-pushforward-adequate}
Let $f : T \to S$ be a quasi-compact and quasi-separated morphism
of schemes. For any adequate $\mathcal{O}_T$-module on
$(\textit{Sch}/T)_\tau$ the pushforward
$f_*\mathcal{F}$ and the higher direct images $R^if_*\mathcal{F}$
are adequate $\mathcal{O}_S$-modules on $(\textit{Sch}/S)_\tau$.
\end{lemma}

\begin{proof}
First we explain how to compute the higher direct images.
Choose an injective resolution $\mathcal{F} \to \mathcal{I}^\bullet$.
Then $R^if_*\mathcal{F}$ is the $i$th cohomology sheaf of the
complex $f_*\mathcal{I}^\bullet$.
Hence $R^if_*\mathcal{F}$ is the sheaf associated to the presheaf
which associates to an object $U/S$ of $(\textit{Sch}/S)_\tau$
the module
\begin{align*}
\frac{\text{Ker}(f_*\mathcal{I}^i(U) \to f_*\mathcal{I}^{i + 1}(U))}
{\text{Im}(f_*\mathcal{I}^{i - 1}(U) \to f_*\mathcal{I}^i(U))}
& =
\frac{\text{Ker}(\mathcal{I}^i(U \times_S T) \to
\mathcal{I}^{i + 1}(U \times_S T))}
{\text{Im}(\mathcal{I}^{i - 1}(U \times_S T) \to \mathcal{I}^i(U \times_S T))}
\\
& =
H^i(U \times_S T, \mathcal{F}) \\
& = H^i((\textit{Sch}/U \times_S T)_\tau,
\mathcal{F}|_{(\textit{Sch}/U \times_S T)_\tau}) \\
& = H^i(U \times_S T, \mathcal{F}|_{(U \times_S T)_{Zar}})
\end{align*}
The first equality by
Topologies, Lemma \ref{topologies-lemma-morphism-big-fppf}
(and its analogues for other topologies),
the second equality by definition of cohomology of $\mathcal{F}$
over an object of $(\textit{Sch}/T)_\tau$,
the third equality by
Cohomology on Sites, Lemma \ref{sites-cohomology-lemma-cohomology-of-open},
and the last equality by
Lemma \ref{lemma-same-cohomology-adequate}.
Thus by
Lemma \ref{lemma-sheafification-adequate}
it suffices to prove the claim stated in the following paragraph.

\medskip\noindent
Let $A$ be a ring. Let $T$ be a scheme quasi-compact and quasi-separated
over $A$. Let $\mathcal{F}$ be an adequate $\mathcal{O}_T$-module on
$(\textit{Sch}/T)_\tau$. For an $A$-algebra $B$ set
$T_B = T \times_{\text{Spec}(A)} \text{Spec}(B)$ and denote
$\mathcal{F}_B = \mathcal{F}|_{(T_B)_{Zar}}$ the restriction of
$\mathcal{F}$ to the small Zariski site of $T_B$.
(Recall that this is a ``usual'' quasi-coherent sheaf on the scheme
$T_B$, see
Lemma \ref{lemma-same-cohomology-adequate}.)
Claim: The functor
$$
B \longmapsto H^q(T_B, \mathcal{F}_B)
$$
is adequate. We will prove the lemma by the usual
procedure of cutting $T$ into pieces.

\medskip\noindent
Case I: $T$ is affine. In this case the schemes $T_B$ are all affine
and $H^q(T_B, \mathcal{F}_B) = 0$ for all $q \geq 1$.
The functor $B \mapsto H^0(T_B, \mathcal{F}_B)$ is adequate by
More on Algebra, Lemma \ref{more-algebra-lemma-pushforward-adequate}.

\medskip\noindent
Case II: $T$ is separated. Let $n$ be the minimal number of affines needed
to cover $T$. We argue by induction on $n$. The base case is Case I.
Choose an affine open covering $T = V_1 \cup \ldots \cup V_n$.
Set $V = V_1 \cup \ldots \cup V_{n - 1}$ and $U = V_n$. Observe that
$$
U \cap V = (V_1 \cap V_n) \cup \ldots \cup (V_{n - 1} \cap V_n)
$$
is also a union of $n - 1$ affine opens as $T$ is separated, see
Schemes, Lemma \ref{schemes-lemma-characterize-separated}.
Note that for each $B$ the base changes $U_B$, $V_B$ and
$(U \cap V)_B = U_B \cap V_B$ behave in the same way. Hence we see that
for each $B$ we have a long exact sequence
$$
0 \to
H^0(T_B, \mathcal{F}_B) \to
H^0(U_B, \mathcal{F}_B) \oplus H^0(V_B, \mathcal{F}_B) \to
H^0((U \cap V)_B, \mathcal{F}_B) \to
H^1(T_B, \mathcal{F}_B) \to \ldots
$$
functorial in $B$, see
Cohomology, Lemma \ref{cohomology-lemma-mayer-vietoris}.
By induction hypothesis the functors
$B \mapsto H^q(U_B, \mathcal{F}_B)$,
$B \mapsto H^q(V_B, \mathcal{F}_B)$, and
$B \mapsto H^q((U \cap V)_B, \mathcal{F}_B)$
are adequate. Using
More on Algebra, Lemmas \ref{more-algebra-lemma-kernel-adequate} and
\ref{more-algebra-lemma-cokernel-adequate}
we see that our functor $B \mapsto H^q(T_B, \mathcal{F}_B)$ sits in the
middle of a short exact sequence whose outer terms are adequate.
Thus the claim follows from
More on Algebra, Lemma \ref{more-algebra-lemma-extension-adequate}.

\medskip\noindent
Case III: General quasi-compact and quasi-separated case.
The proof is again by induction on the number $n$ of affines needed to
cover $T$. The base case $n = 1$ is Case I. 
Choose an affine open covering $T = V_1 \cup \ldots \cup V_n$.
Set $V = V_1 \cup \ldots \cup V_{n - 1}$ and $U = V_n$. Note that
since $T$ is quasi-separated $U \cap V$ is a quasi-compact open of an
affine scheme, hence Case II applies to it. The rest of the argument
proceeds in exactly the same manner as in the paragraph above and is
omitted.
\end{proof}







\section{Adequate and quasi-coherent sheaves on affines}
\label{section-comparison}

\noindent
In this section we start comparing adequate modules and quasi-coherent
modules on an affine scheme $S = \text{Spec}(A)$. Note that the category
of quasi-coherent $\mathcal{O}_S$-modules is equivalent to the category
of $A$-modules. In spite of this we will continue to write
$\textit{QCoh}(\mathcal{O}_S)$ and not the shorter $\textit{Mod}_A$.

\medskip\noindent
Recall that there are functors
$u : \textit{QCoh}(\mathcal{O}_S) \to \textit{Adeq}(\mathcal{O}_S)$
and
$v : \textit{Adeq}(\mathcal{O}_S) \to \textit{QCoh}(\mathcal{O}_S)$
satisfying the adjunction
$$
\textit{Hom}_{\textit{QCoh}(\mathcal{O}_S)}(\mathcal{F}, v\mathcal{G})
=
\textit{Hom}_{\textit{Adeq}(\mathcal{O}_S)}(u\mathcal{F}, \mathcal{G})
$$
and such that $\mathcal{F} \to vu\mathcal{F}$ is an isomorphism for
every quasi-coherent sheaf $\mathcal{F}$, see
Remark \ref{remark-compare}.
The functor $v$ is exact but $u$ is not left exact in general.
Hence $u$ is a fully faithfull embedding and we can identify
$\textit{QCoh}(\mathcal{O}_S)$ with a full subcategory of
$\textit{Adeq}(\mathcal{O}_S)$. The exact functor $v$ induces
a functor
$$
D(\textit{Adeq}(\mathcal{O}_S))
\longrightarrow
D(\textit{QCoh}(\mathcal{O}_S))
$$
and similarly for bounded versions. Note that, almost by definition, every
adequate sheaf has an embedding into a quasi-coherent sheaf, see
Lemma \ref{lemma-adequate-characterize}.
Hence by
Derived Categories, Lemma \ref{derived-lemma-subcategory-right-resolution}
every bounded below complex of adequate modules is quasi-isomorphic
to a bounded below complex of quasi-coherent modules.
It follows that we get a factorization
$$
K^+(\textit{QCoh}(\mathcal{O}_S))
\longrightarrow
D^+(\textit{Adeq}(\mathcal{O}_S))
\longrightarrow
D^+(\textit{QCoh}(\mathcal{O}_S))
$$
where the first arrow is a localization of
$K^+(\textit{QCoh}(\mathcal{O}_S))$, see the material in
Derived Categories, Section \ref{derived-section-localization}.
The multiplicative subset is the collection of maps
$\mathcal{F}^\bullet \to \mathcal{G}^\bullet$ of bounded below
complexes of quasi-coherent sheaves on $S$ such that the associated
map $u\mathcal{F}^\bullet \to u\mathcal{G}^\bullet$ of complexes
of $\mathcal{O}$-modules on the big fppf site of $S$
is a quasi-isomorphism, see
Derived Categories, Lemma \ref{derived-lemma-triangle-functor-localize}.
A map of complexes of $A$-modules $K^\bullet \to L^\bullet$ is said to be a
{\it universal quasi-isomorphism} if for every $A$-module $M$ the map
$K^\bullet \otimes_A M \to L^\bullet \otimes_A M$ is a quasi-isomorphism.
Of course this is equivalent to requiring
$K^\bullet \otimes_A B \to L^\bullet \otimes_A B$ to be a quasi-isomorphism
for every $A$-algebra $B$. This proves the following.

\begin{lemma}
\label{lemma-describe-Dplus-adequate}
Let $S = \text{Spec}(A)$. The derived category
$D^+(\textit{Adeq}(\mathcal{O}_S))$ is the localization of
$K^+(\textit{Mod}_A)$ at the multiplicative subset of universal
quasi-isomorphisms.
\end{lemma}

\begin{proof}
See discussion above.
\end{proof}

\noindent
Next, we briefly discuss the relationship between quasi-coherent
modules and all modules on an affine scheme $S$. (This should be elaborated
on, generalized, and moved somewhere else.) A reference is
\cite[Appendix B]{TT}. By the discussion in
Schemes, Section \ref{schemes-section-quasi-coherent}
the embedding
$\textit{QCoh}(\mathcal{O}_S) \subset \textit{Mod}(\mathcal{O}_S)$
exhibits $\textit{QCoh}(\mathcal{O}_S)$ as a Serre subcategory of
the category of $\mathcal{O}_S$-modules. Denote
$D_{\textit{QCoh}}(\mathcal{O}_S) \subset D(\mathcal{O}_S)$ the
subcategory of complexes whose cohomology sheaves are quasi-coherent, see
Derived Categories, Section \ref{derived-section-triangulated-sub}.
Thus we obtain a canonical
functor
$$
D(\textit{QCoh}(\mathcal{O}_S))
\longrightarrow
D_{\textit{QCoh}}(\mathcal{O}_S)
$$
see
Derived Categories, Equation (\ref{derived-equation-compare}).
It is not hard to see that the
bounded below version
\begin{equation}
\label{equation-compare-bounded}
D^+(\textit{QCoh}(\mathcal{O}_S))
\longrightarrow
D^+_{\textit{QCoh}}(\mathcal{O}_S)
\end{equation}
is an equivalence. Namely, taking right derived functor of global sections
gives a functor
\begin{equation}
\label{equation-back}
R\Gamma(S, - ) :
D^+_{\textit{QCoh}}(\mathcal{O}_S)
\longrightarrow
D^+(\textit{Mod}_A) = D^+(A).
\end{equation}
By the vanishing of cohomology of quasi-coherent sheaves we see that
for any $A$-module $M$ with $\mathcal{F} = \widetilde{M}$ we have
$R\Gamma(S, \mathcal{F}) = M$, hence the composition of
(\ref{equation-compare-bounded}) and (\ref{equation-back}) is isomorphic
to the identity functor (details omitted).
Actually it is true that the comparison map
$D(\textit{QCoh}(\mathcal{O}_S)) \to D_{\textit{QCoh}}(\mathcal{O}_S)$
is an equivalence for any quasi-compact and (semi-)separated scheme (insert
future reference here).

\medskip\noindent
Next, we briefly discuss the relationship between adequate
modules and all modules on the big fppf site of an affine scheme $S$.
(This should be elaborated on, generalized, and moved somewhere else.)
First we prove the following lemma.

\begin{lemma}
\label{lemma-right-adjoint-adequate}
Let $S = \text{Spec}(A)$ be an affine scheme.
The inclusion
$\textit{Adeq}(\mathcal{O}_S) \to
\textit{Mod}((\textit{Sch}/S)_{fppf}, \mathcal{O})$
has a right adjoint $Q$. Moreover, the adjunction mapping
$Q(\mathcal{F}) \to \mathcal{F}$ is an isomorphism for every adequate module
$\mathcal{F}$.
\end{lemma}

\begin{proof}
By
Topologies, Lemma \ref{topologies-lemma-affine-big-site-fppf}
we may work with $\mathcal{O}$-modules on $(\textit{Aff}/S)_{fppf}$.
If we can show that the inclusion
$\textit{Adeq}(\mathcal{O}_S) \to
\textit{PMod}((\textit{Aff}/S)_{fppf}, \mathcal{O})$
into presheaves of $\mathcal{O}$-modules has an adjoint then we win.
Now a presheaf of $\mathcal{O}$-modules on $(\textit{Aff}/S)_{fppf}$
is exactly a functor as discussed in
More on Algebra, Section \ref{more-algebra-section-quasi-coherent}
and among these the adequate functors exactly correspond to the
adequate modules (by definition).
Hence the lemma follows from
More on Algebra, Lemma \ref{more-algebra-lemma-adjoint}.
\end{proof}

\noindent
Since $Q$ is a right adjoint it is left exact. Since the inclusion
functor is exact, see
Lemma \ref{lemma-abelian-adequate}
we conclude that $Q$ transforms injectives into injectives, and that
the category $\textit{Adeq}(\mathcal{O}_S)$ has enough injectives, see
Homology, Lemma \ref{homology-lemma-adjoint-enough-injectives}
and
Injectives, Theorem \ref{injectives-theorem-sheaves-modules-injectives}.
The following lemma will be improved on later (insert future reference here).

\begin{lemma}
\label{lemma-RQ-zero}
Let $S = \text{Spec}(A)$ be an affine scheme.
Let $0 \to \mathcal{F} \to \mathcal{G} \to \mathcal{H} \to 0$
be an exact sequence of $\mathcal{O}$-modules on
$(\textit{Sch}/S)_{fppf}$ with $\mathcal{F}$ adequate. Then
$0 \to Q(\mathcal{F}) \to Q(\mathcal{G}) \to Q(\mathcal{H}) \to 0$
is exact.
\end{lemma}

\begin{proof}
Recall that $Q(\mathcal{F}) = \mathcal{F}$. Consider the pullback
$$
0 \to \mathcal{F} \to \mathcal{E} \to Q(\mathcal{H}) \to 0
$$
of the given exact sequence by the adjunction map
$Q(\mathcal{H}) \to \mathcal{H}$. By
Lemma \ref{lemma-abelian-adequate}
the module $\mathcal{E}$ is adequate. It follows in a straightforward
manner that $\mathcal{E} = Q(\mathcal{G})$.
\end{proof}

\noindent
The lemma above means that $R^1Q(\mathcal{F})$ is zero for every
adequate module as one discovers upon embedding $\mathcal{F}$ into
an injective $\mathcal{O}$-module and applying the long exact sequence
involving the right derived functors of $Q$. However, it doesn't
immediately imply the same thing for $R^iQ(\mathcal{F})$, $i \geq 2$!

\begin{lemma}
\label{lemma-right-adjoint-adequate-compute}
Let $S = \text{Spec}(A)$ be an affine scheme. Let $\mathcal{I}$ be an
injective object of $\textit{Mod}((\textit{Sch}/S)_{fppf}, \mathcal{O})$.
Then $Q(\mathcal{I})$ is isomorphic to the
quasi-coherent sheaf associated to the $A$-module $\Gamma(S, \mathcal{I})$.
\end{lemma}

\begin{proof}
Denote $\Gamma(S, \mathcal{I})^a$ the quasi-coherent $\mathcal{O}$-module
on $(\textit{Sch}/S)_{fppf}$ associated to $\mathcal{F}$.
By the universal property of $Q$ there exists a commutative diagram
$$
\xymatrix{
\Gamma(S, \mathcal{I})^a \ar[rr]_\alpha \ar[rd] & &
Q(\mathcal{I}) \ar[ld] \\
& \mathcal{I}
}
$$
Let $\varphi : \mathcal{F} \to \mathcal{I}$ be a morphism of
$\mathcal{O}$-modules with $\mathcal{F}$ adequate. By definition there is an
embedding $\mathcal{F} \subset \mathcal{G}$ with $\mathcal{G}$
quasi-coherent. Because $\mathcal{I}$ is an injective $\mathcal{O}$-module,
there exists a map $\psi : \mathcal{G} \to \mathcal{I}$ extending
$\varphi$. By the universal property of $\widetilde{\ }$, see
Schemes, Lemma \ref{schemes-lemma-compare-constructions},
there exists a map
$\tilde\psi : \mathcal{G} \to \Gamma(S, \mathcal{I})^a$
factoring $\psi$. The restriction $\tilde\psi|_{\mathcal{F}}$ gives
a map $\mathcal{F} \to \Gamma(S, \mathcal{I})^a$ over
$\mathcal{I}$. If we apply this reasoning to $Q(\mathcal{I}) \to \mathcal{I}$
we find a map $\beta : Q(\mathcal{I}) \to \Gamma(S, \mathcal{I})^a$
over $\mathcal{I}$. The universal property of $Q(\mathcal{I}) \to \mathcal{I}$
implies that $\alpha \circ \beta = \text{id}_{Q(\mathcal{I})}$.
Thus
$$
\Gamma(S, \mathcal{I})^a =
\beta(Q(\mathcal{I})) \oplus \text{Ker}(\alpha).
$$
By
Lemma \ref{lemma-equivalence-quasi-coherent-limits}
we find that $Q(\mathcal{I})$ and $\text{Ker}(\alpha)$ are quasi-coherent
as the cokernel of a map
$\Gamma(S, \mathcal{I})^a \to \Gamma(S, \mathcal{I})^a$.
Since $\Gamma(S, \text{Ker}(\alpha)) = 0$ by the commutativity of the
first diagram we conclude that $\text{Ker}(\alpha) = 0$.
\end{proof}

\noindent
By the discussion in
Section \ref{section-adequate}
the embedding
$\textit{Adeq}(\mathcal{O}_S) \subset
\textit{Mod}((\textit{Sch}/S)_{fppf}, \mathcal{O})$
exhibits $\textit{Adeq}(\mathcal{O}_S)$ as a Serre subcategory of
the category of all fppf $\mathcal{O}$-modules. Denote
$$
D_{\textit{Adeq}}(\mathcal{O}) \subset
D(\mathcal{O}) = D(\textit{Mod}((\textit{Sch}/S)_{fppf}, \mathcal{O}))
$$
the subcategory of complexes whose cohomology sheaves are adequate, see
Derived Categories, Section \ref{derived-section-triangulated-sub}.
Thus we obtain a canonical functor
$$
D(\textit{Adeq}(\mathcal{O}_S))
\longrightarrow
D_{\textit{Adeq}}(\mathcal{O})
$$
see
Derived Categories, Equation (\ref{derived-equation-compare}).
We think the bounded below version
\begin{equation}
\label{equation-compare-bounded-adequate}
D^+(\textit{Adeq}(\mathcal{O}_S))
\longrightarrow
D^+_{\textit{Adeq}}(\mathcal{O})
\end{equation}
is an equivalence. Namely, a quasi-inverse to
(\ref{equation-compare-bounded-adequate}) should be the right derived
functor $RQ$. To show this works, we have to show that
$R^iQ(\mathcal{F}) = 0$ for all $i > 0$ and all adequate module
$\mathcal{F}$. To be continued...










\section{Other chapters}

\begin{multicols}{2}
\begin{enumerate}
\item \hyperref[introduction-section-phantom]{Introduction}
\item \hyperref[conventions-section-phantom]{Conventions}
\item \hyperref[sets-section-phantom]{Set Theory}
\item \hyperref[categories-section-phantom]{Categories}
\item \hyperref[topology-section-phantom]{Topology}
\item \hyperref[sheaves-section-phantom]{Sheaves on Spaces}
\item \hyperref[algebra-section-phantom]{Commutative Algebra}
\item \hyperref[sites-section-phantom]{Sites and Sheaves}
\item \hyperref[homology-section-phantom]{Homological Algebra}
\item \hyperref[derived-section-phantom]{Derived Categories}
\item \hyperref[more-algebra-section-phantom]{More Algebra}
\item \hyperref[simplicial-section-phantom]{Simplicial Methods}
\item \hyperref[modules-section-phantom]{Sheaves of Modules}
\item \hyperref[sites-modules-section-phantom]{Modules on Sites}
\item \hyperref[injectives-section-phantom]{Injectives}
\item \hyperref[cohomology-section-phantom]{Cohomology of Sheaves}
\item \hyperref[sites-cohomology-section-phantom]{Cohomology on Sites}
\item \hyperref[hypercovering-section-phantom]{Hypercoverings}
\item \hyperref[schemes-section-phantom]{Schemes}
\item \hyperref[constructions-section-phantom]{Constructions of Schemes}
\item \hyperref[properties-section-phantom]{Properties of Schemes}
\item \hyperref[morphisms-section-phantom]{Morphisms of Schemes}
\item \hyperref[coherent-section-phantom]{Coherent Cohomology}
\item \hyperref[divisors-section-phantom]{Divisors}
\item \hyperref[limits-section-phantom]{Limits of Schemes}
\item \hyperref[varieties-section-phantom]{Varieties}
\item \hyperref[chow-section-phantom]{Chow Homology}
\item \hyperref[topologies-section-phantom]{Topologies on Schemes}
\item \hyperref[descent-section-phantom]{Descent}
\item \hyperref[more-morphisms-section-phantom]{More on Morphisms}
\item \hyperref[flat-section-phantom]{More on Flatness}
\item \hyperref[groupoids-section-phantom]{Groupoid Schemes}
\item \hyperref[more-groupoids-section-phantom]{More on Groupoid Schemes}
\item \hyperref[etale-section-phantom]{\'Etale Morphisms of Schemes}
\item \hyperref[etale-cohomology-section-phantom]{\'Etale Cohomology}
\item \hyperref[spaces-section-phantom]{Algebraic Spaces}
\item \hyperref[spaces-properties-section-phantom]{Properties of Algebraic Spaces}
\item \hyperref[spaces-morphisms-section-phantom]{Morphisms of Algebraic Spaces}
\item \hyperref[spaces-topologies-section-phantom]{Topologies on Algebraic Spaces}
\item \hyperref[spaces-descent-section-phantom]{Descent and Algebraic Spaces}
\item \hyperref[spaces-more-morphisms-section-phantom]{More on Morphisms of Spaces}
\item \hyperref[quot-section-phantom]{Quot and Hilbert Spaces}
\item \hyperref[stacks-section-phantom]{Stacks}
\item \hyperref[spaces-groupoids-section-phantom]{Groupoids in Algebraic Spaces}
\item \hyperref[spaces-more-groupoids-section-phantom]{More on Groupoids in Spaces}
\item \hyperref[bootstrap-section-phantom]{Bootstrap}
\item \hyperref[examples-stacks-section-phantom]{Examples of Stacks}
\item \hyperref[groupoids-quotients-section-phantom]{Quotients of Groupoids}
\item \hyperref[algebraic-section-phantom]{Algebraic Stacks}
\item \hyperref[criteria-section-phantom]{Criteria for Representability}
\item \hyperref[stacks-properties-section-phantom]{Properties of Algebraic Stacks}
\item \hyperref[stacks-morphisms-section-phantom]{Morphisms of Algebraic Stacks}
\item \hyperref[examples-section-phantom]{Examples}
\item \hyperref[exercises-section-phantom]{Exercises}
\item \hyperref[guide-section-phantom]{Guide to Literature}
\item \hyperref[desirables-section-phantom]{Desirables}
\item \hyperref[coding-section-phantom]{Coding Style}
\item \hyperref[fdl-section-phantom]{GNU Free Documentation License}
\item \hyperref[index-section-phantom]{Auto Generated Index}
\end{enumerate}
\end{multicols}


\bibliography{my}
\bibliographystyle{amsalpha}

\end{document}
