\IfFileExists{stacks-project.cls}{%
\documentclass{stacks-project}
}{%
\documentclass{amsart}
}

% The following AMS packages are automatically loaded with
% the amsart documentclass:
%\usepackage{amsmath}
%\usepackage{amssymb}
%\usepackage{amsthm}

% For dealing with references we use the comment environment
\usepackage{verbatim}
\newenvironment{reference}{\comment}{\endcomment}
%\newenvironment{reference}{}{}
\newenvironment{slogan}{\comment}{\endcomment}
\newenvironment{history}{\comment}{\endcomment}

% For commutative diagrams you can use
% \usepackage{amscd}
\usepackage[all]{xy}

% We use 2cell for 2-commutative diagrams.
\xyoption{2cell}
\UseAllTwocells

% To put source file link in headers.
% Change "template.tex" to "this_filename.tex"
% \usepackage{fancyhdr}
% \pagestyle{fancy}
% \lhead{}
% \chead{}
% \rhead{Source file: \url{template.tex}}
% \lfoot{}
% \cfoot{\thepage}
% \rfoot{}
% \renewcommand{\headrulewidth}{0pt}
% \renewcommand{\footrulewidth}{0pt}
% \renewcommand{\headheight}{12pt}

\usepackage{multicol}

% For cross-file-references
\usepackage{xr-hyper}

% Package for hypertext links:
\usepackage{hyperref}

% For any local file, say "hello.tex" you want to link to please
% use \externaldocument[hello-]{hello}
\externaldocument[introduction-]{introduction}
\externaldocument[conventions-]{conventions}
\externaldocument[sets-]{sets}
\externaldocument[categories-]{categories}
\externaldocument[topology-]{topology}
\externaldocument[sheaves-]{sheaves}
\externaldocument[sites-]{sites}
\externaldocument[stacks-]{stacks}
\externaldocument[fields-]{fields}
\externaldocument[algebra-]{algebra}
\externaldocument[brauer-]{brauer}
\externaldocument[homology-]{homology}
\externaldocument[derived-]{derived}
\externaldocument[simplicial-]{simplicial}
\externaldocument[more-algebra-]{more-algebra}
\externaldocument[smoothing-]{smoothing}
\externaldocument[modules-]{modules}
\externaldocument[sites-modules-]{sites-modules}
\externaldocument[injectives-]{injectives}
\externaldocument[cohomology-]{cohomology}
\externaldocument[sites-cohomology-]{sites-cohomology}
\externaldocument[dga-]{dga}
\externaldocument[dpa-]{dpa}
\externaldocument[hypercovering-]{hypercovering}
\externaldocument[schemes-]{schemes}
\externaldocument[constructions-]{constructions}
\externaldocument[properties-]{properties}
\externaldocument[morphisms-]{morphisms}
\externaldocument[coherent-]{coherent}
\externaldocument[divisors-]{divisors}
\externaldocument[limits-]{limits}
\externaldocument[varieties-]{varieties}
\externaldocument[topologies-]{topologies}
\externaldocument[descent-]{descent}
\externaldocument[perfect-]{perfect}
\externaldocument[more-morphisms-]{more-morphisms}
\externaldocument[flat-]{flat}
\externaldocument[groupoids-]{groupoids}
\externaldocument[more-groupoids-]{more-groupoids}
\externaldocument[etale-]{etale}
\externaldocument[chow-]{chow}
\externaldocument[intersection-]{intersection}
\externaldocument[pic-]{pic}
\externaldocument[adequate-]{adequate}
\externaldocument[dualizing-]{dualizing}
\externaldocument[duality-]{duality}
\externaldocument[discriminant-]{discriminant}
\externaldocument[local-cohomology-]{local-cohomology}
\externaldocument[curves-]{curves}
\externaldocument[resolve-]{resolve}
\externaldocument[models-]{models}
\externaldocument[pione-]{pione}
\externaldocument[etale-cohomology-]{etale-cohomology}
\externaldocument[proetale-]{proetale}
\externaldocument[crystalline-]{crystalline}
\externaldocument[spaces-]{spaces}
\externaldocument[spaces-properties-]{spaces-properties}
\externaldocument[spaces-morphisms-]{spaces-morphisms}
\externaldocument[decent-spaces-]{decent-spaces}
\externaldocument[spaces-cohomology-]{spaces-cohomology}
\externaldocument[spaces-limits-]{spaces-limits}
\externaldocument[spaces-divisors-]{spaces-divisors}
\externaldocument[spaces-over-fields-]{spaces-over-fields}
\externaldocument[spaces-topologies-]{spaces-topologies}
\externaldocument[spaces-descent-]{spaces-descent}
\externaldocument[spaces-perfect-]{spaces-perfect}
\externaldocument[spaces-more-morphisms-]{spaces-more-morphisms}
\externaldocument[spaces-flat-]{spaces-flat}
\externaldocument[spaces-groupoids-]{spaces-groupoids}
\externaldocument[spaces-more-groupoids-]{spaces-more-groupoids}
\externaldocument[bootstrap-]{bootstrap}
\externaldocument[spaces-pushouts-]{spaces-pushouts}
\externaldocument[groupoids-quotients-]{groupoids-quotients}
\externaldocument[spaces-more-cohomology-]{spaces-more-cohomology}
\externaldocument[spaces-simplicial-]{spaces-simplicial}
\externaldocument[formal-spaces-]{formal-spaces}
\externaldocument[restricted-]{restricted}
\externaldocument[spaces-resolve-]{spaces-resolve}
\externaldocument[formal-defos-]{formal-defos}
\externaldocument[defos-]{defos}
\externaldocument[cotangent-]{cotangent}
\externaldocument[examples-defos-]{examples-defos}
\externaldocument[algebraic-]{algebraic}
\externaldocument[examples-stacks-]{examples-stacks}
\externaldocument[stacks-sheaves-]{stacks-sheaves}
\externaldocument[criteria-]{criteria}
\externaldocument[artin-]{artin}
\externaldocument[quot-]{quot}
\externaldocument[stacks-properties-]{stacks-properties}
\externaldocument[stacks-morphisms-]{stacks-morphisms}
\externaldocument[stacks-limits-]{stacks-limits}
\externaldocument[stacks-cohomology-]{stacks-cohomology}
\externaldocument[stacks-perfect-]{stacks-perfect}
\externaldocument[stacks-introduction-]{stacks-introduction}
\externaldocument[stacks-more-morphisms-]{stacks-more-morphisms}
\externaldocument[stacks-geometry-]{stacks-geometry}
\externaldocument[moduli-]{moduli}
\externaldocument[moduli-curves-]{moduli-curves}
\externaldocument[examples-]{examples}
\externaldocument[exercises-]{exercises}
\externaldocument[guide-]{guide}
\externaldocument[desirables-]{desirables}
\externaldocument[coding-]{coding}
\externaldocument[obsolete-]{obsolete}
\externaldocument[fdl-]{fdl}
\externaldocument[index-]{index}

% Theorem environments.
%
\theoremstyle{plain}
\newtheorem{theorem}[subsection]{Theorem}
\newtheorem{proposition}[subsection]{Proposition}
\newtheorem{lemma}[subsection]{Lemma}

\theoremstyle{definition}
\newtheorem{definition}[subsection]{Definition}
\newtheorem{example}[subsection]{Example}
\newtheorem{exercise}[subsection]{Exercise}
\newtheorem{situation}[subsection]{Situation}

\theoremstyle{remark}
\newtheorem{remark}[subsection]{Remark}
\newtheorem{remarks}[subsection]{Remarks}

\numberwithin{equation}{subsection}

% Macros
%
\def\lim{\mathop{\rm lim}\nolimits}
\def\colim{\mathop{\rm colim}\nolimits}
\def\Spec{\mathop{\rm Spec}}
\def\Hom{\mathop{\rm Hom}\nolimits}
\def\Ext{\mathop{\rm Ext}\nolimits}
\def\SheafHom{\mathop{\mathcal{H}\!{\it om}}\nolimits}
\def\SheafExt{\mathop{\mathcal{E}\!{\it xt}}\nolimits}
\def\Sch{\textit{Sch}}
\def\Mor{\mathop{\rm Mor}\nolimits}
\def\Ob{\mathop{\rm Ob}\nolimits}
\def\Sh{\mathop{\textit{Sh}}\nolimits}
\def\NL{\mathop{N\!L}\nolimits}
\def\proetale{{pro\text{-}\acute{e}tale}}
\def\etale{{\acute{e}tale}}
\def\QCoh{\textit{QCoh}}
\def\Ker{\mathop{\rm Ker}}
\def\Im{\mathop{\rm Im}}
\def\Coker{\mathop{\rm Coker}}
\def\Coim{\mathop{\rm Coim}}

%
% Macros for moduli stacks/spaces
%
\def\QCohstack{\mathcal{QC}\!{\it oh}}
\def\Cohstack{\mathcal{C}\!{\it oh}}
\def\Spacesstack{\mathcal{S}\!{\it paces}}
\def\Quotfunctor{{\rm Quot}}
\def\Hilbfunctor{{\rm Hilb}}
\def\Curvesstack{\mathcal{C}\!{\it urves}}
\def\Polarizedstack{\mathcal{P}\!{\it olarized}}
\def\Complexesstack{\mathcal{C}\!{\it omplexes}}
% \Pic is the operator that assigns to X its picard group, usage \Pic(X)
% \Picardstack_{X/B} denotes the Picard stack of X over B
% \Picardfunctor_{X/B} denotes the Picard functor of X over B
\def\Pic{\mathop{\rm Pic}\nolimits}
\def\Picardstack{\mathcal{P}\!{\it ic}}
\def\Picardfunctor{{\rm Pic}}
\def\Deformationcategory{\mathcal{D}\!{\it ef}}


% OK, start here.
%
\begin{document}

\title{Properties of schemes}

%\begin{abstract}
%\end{abstract}

\maketitle

\tableofcontents

\section{Introduction}
\label{section-introduction}

\noindent
In this chapter we introduce some absolute properties of schemes.
A basic reference is \cite{EGA}.





\section{Integral, irreducible, and reduced schemes}
\label{section-integral}

\begin{definition}
\label{definition-integral}
Let $X$ be a scheme. We say $X$ is {\it integral} if for every affine open
$\text{Spec}(R) = U \subset X$ the ring $R$ is an integral domain.
\end{definition}

\begin{lemma}
\label{lemma-characterize-reduced}
Let $X$ be a scheme.
The following are equivalent.
\begin{enumerate}
\item The scheme $X$ is reduced, see Schemes, Definition \ref{schemes-definition-reduced}.
\item There exists an affine open covering $X = \bigcup U_i$
such that each $\Gamma(U_i, \mathcal{O}_X)$ is reduced.
\item For every affine open $U \subset X$ the ring
$\mathcal{O}_X(U)$ is reduced.
\item For every open $U \subset X$ the ring $\mathcal{O}_X(U)$ is reduced.
\end{enumerate}
\end{lemma}

\begin{proof}
See Schemes, Lemmas \ref{schemes-lemma-reduced} and
\ref{schemes-lemma-affine-reduced}.
\end{proof}

\begin{lemma}
\label{lemma-characterize-integral}
A scheme $X$ is integral if and only if it is reduced and irreducible.
\end{lemma}

\begin{proof}
If $X$ is irreducible, then every affine open $\text{Spec}(R) = U \subset X$
is irreducible. If $X$ is reduced, then $R$ is reduced, by
Lemma \ref{lemma-characterize-reduced} above. Hence $R$ is reduced
and $(0)$ is a prime ideal, i.e., $R$ is an integral domain.

\medskip\noindent
If $X$ is integral, then for every affine open $\text{Spec}(R) = U \subset X$
the ring $R$ is reduced and hence $X$ is reduced by
Lemma \ref{lemma-characterize-reduced}. Moreover, every affine open
is irreducible. We will show that any two nonempty
affine opens $U, V$ of $X$ intersect,
which will imply that $X$ is irreducible. Namely, if $U \cap V$
is empty, then $U \cup V$ is an affine open of $X$, see
Schemes, Lemma \ref{schemes-lemma-disjoint-union-affines}.
But a disjoint union of nonempty affines is not irreducible,
a contradiction.
\end{proof}









\section{Noetherian schemes}
\label{section-noetherian}

\begin{definition}
\label{definition-noetherian}
Let $X$ be a scheme.
\begin{enumerate}
\item We say $X$ is {\it locally Noetherian} if for every affine open
$\text{Spec}(R) = U \subset X$ the ring $R$ is Noetherian.
\item We say $X$ is {\it Noetherian} if $X$ is Noetherian
and quasi-compact.
\end{enumerate}
\end{definition}

\begin{lemma}
\label{lemma-noetherian}
Let $X$ be a scheme.
If $X$ has an open affine covering $X = \bigcup U_i$ such that
each $U_i$ is the spectrum of a Noetherian ring, then
$X$ is locally Noetherian.
\end{lemma}

\begin{proof}
Let $X = \bigcup U_i$ be such an open covering.
Let $\text{Spec}(R) = V \subset X$ be an arbitrary affine open.
By Schemes, Lemma \ref{schemes-lemma-good-subcover}
there exists a covering of $V = \text{Spec}(R)$ by finitely many standard opens
$D(f_j)$ such that each ring $R_{f_j}$ is a principal localization of
one of the Noetherian rings $\mathcal{O}_X(U_i)$. Hence
$R_{f_j}$ is Noetherian, see
Algebra, Lemma \ref{algebra-lemma-Noetherian-permanence}.
Hence the result follows from Algebra, Lemma \ref{algebra-lemma-cover}.
\end{proof}

\begin{lemma}
\label{lemma-immersion-into-noetherian}
Any immersion $Z \to X$ with $X$ locally Noetherian is quasi-compact.
\end{lemma}

\begin{proof}
A closed immersion is clearly quasi-compact.
A composition of quasi-compact morphisms is quasi-compact,
see Topology, Lemma \ref{topology-lemma-composition-quasi-compact}.
Hence it suffices to show that an open immersion into
a locally Noetherian scheme is quasi-compact.
Using Schemes, Lemma \ref{schemes-lemma-quasi-compact-affine}
we reduce to the case where $X$ is affine.
Here we win because any open subset of the spectrum
of a Noetherian ring is quasi-compact (for example
combine Algebra, Lemma \ref{algebra-lemma-Noetherian-topology} and
Topology, Lemmas \ref{topology-lemma-Noetherian} and
\ref{topology-lemma-constructible-Noetherian-space}).
\end{proof}

\begin{lemma}
\label{lemma-locally-Noetherian-quasi-separated}
A locally Noetherian scheme is quasi-separated.
\end{lemma}

\begin{proof}
By Schemes, Lemma \ref{schemes-lemma-characterize-quasi-separated}
we have to show that the intersection $U \cap V$ of two
affine opens of $X$ is quasi-compact. This follows from
Lemma \ref{lemma-immersion-into-noetherian} above on
considering the open immersion $U \cap V \to U$ for example.
(But really it is just because any open of the spectrum of a
Noetherian ring is quasi-compact.)
\end{proof}








\section{Quasi-affine schemes}
\label{section-quasi-affine}

\begin{definition}
\label{definition-quasi-affine}
A scheme $X$ is called {\it quasi-affine} if it is quasi-compact
and isomorphic to an open subscheme of an affine scheme.
\end{definition}





\section{Ample invertible sheaves}
\label{section-ample}

\noindent
Recall from Modules, Lemma \ref{modules-lemma-s-open}
that given an invertible sheaf $\mathcal{L}$ on a locally ringed
space $X$, and given a global section $s$ of $\mathcal{L}$
the set $X_s = \{x \in X \mid s \not \in \mathfrak m_x\mathcal{L}_x\}$
is open.

\begin{definition}
\label{definition-ample}
Let $X$ be a scheme.
Let $\mathcal{L}$ be an invertible $\mathcal{O}_X$-module.
We say {\it $\mathcal{L}$ is ample} if $X$ is quasi-compact, and
if the open sets $X_s$, where $s \in \Gamma(X, \mathcal{L}^{\otimes n})$
and $n \geq 1$, form a basis for the topology on $X$.
\end{definition}








\section{Miscellaneous}
\label{section-misc}

\begin{lemma}
\label{lemma-maximal-points-affine}
Let $X$ be a quasi-separated scheme.
Let $Z_1, \ldots, Z_n$ be pairwise distinct irreducible components of $X$,
see Topology, Section \ref{topology-section-irreducible-components}.
Let $\eta_i \in Z_i$ be their generic points, see
Schemes, Lemma \ref{schemes-lemma-scheme-sober}.
There exist affine open neighbourhoods $\eta_i \in U_i$
such that $U_i \cap U_j = \emptyset$ for all $i \not = j$.
In particular, $U = U_1 \cup \ldots \cup U_n$ is an affine
open containing all of the points $\eta_1, \ldots, \eta_n$.
\end{lemma}

\begin{proof}
Let $V_i$ be any affine open containing $\eta_i$
and disjoint from the closed set $Z_1 \cup \ldots \hat Z_i \ldots \cup Z_n$.
Since $X$ is quasi-separated for each $i$ the union
$W_i = \bigcup_{j, j \not = i} V_i \cap V_j$ is a quasi-compact
open of $V_i$ not containing $\eta_i$. 
We can find open neighbourhoods $U_i \subset V_i$
containing $\eta_i$ and disjoint from $W_i$ by
Algebra, Lemma \ref{algebra-lemma-standard-open-containing-maximal-point}.
Finally, $U$ is affine since it is the spectrum of
the ring $R_1 \times \ldots \times R_n$ where $R_i = \mathcal{O}_X(U_i)$,
see Schemes, Lemma \ref{schemes-lemma-disjoint-union-affines}.
\end{proof}

\begin{remark}
\label{remark-maximal-points-affine}
Lemma \ref{lemma-maximal-points-affine} above is false if $X$
is not quasi-separated. Here is an example. Take
$R = \mathbf{Q}[x, y_1, y_2, \ldots]/((x-i)y_i)$.
Consider the minimal prime ideal $\mathfrak p = (y_1, y_2, \ldots)$
of $R$. Glue two copies of $\text{Spec}(R)$ along the
(not quasi-compact) open $\text{Spec}(R) \setminus V(\mathfrak p)$
to get a scheme $X$ (glueing as in
Schemes, Example \ref{schemes-example-affine-space-zero-doubled}).
Then the two maximal points of $X$ corresponding to $\mathfrak p$
are not contained in a common affine open. The reason is
that any open of $\text{Spec}(R)$ containing $\mathfrak p$
contains infinitely many of the ``lines'' $x = i$, $y_j = 0$,
$j \not = i$ with parameter $y_i$. Details omitted.
\end{remark}






\section{Other chapters}

\begin{multicols}{2}
\begin{enumerate}
\item \hyperref[introduction-section-phantom]{Introduction}
\item \hyperref[conventions-section-phantom]{Conventions}
\item \hyperref[sets-section-phantom]{Set Theory}
\item \hyperref[categories-section-phantom]{Categories}
\item \hyperref[topology-section-phantom]{Topology}
\item \hyperref[sheaves-section-phantom]{Sheaves on Spaces}
\item \hyperref[algebra-section-phantom]{Commutative Algebra}
\item \hyperref[sites-section-phantom]{Sites and Sheaves}
\item \hyperref[homology-section-phantom]{Homological Algebra}
\item \hyperref[derived-section-phantom]{Derived Categories}
\item \hyperref[more-algebra-section-phantom]{More Algebra}
\item \hyperref[simplicial-section-phantom]{Simplicial Methods}
\item \hyperref[modules-section-phantom]{Sheaves of Modules}
\item \hyperref[sites-modules-section-phantom]{Modules on Sites}
\item \hyperref[injectives-section-phantom]{Injectives}
\item \hyperref[cohomology-section-phantom]{Cohomology of Sheaves}
\item \hyperref[sites-cohomology-section-phantom]{Cohomology on Sites}
\item \hyperref[hypercovering-section-phantom]{Hypercoverings}
\item \hyperref[schemes-section-phantom]{Schemes}
\item \hyperref[constructions-section-phantom]{Constructions of Schemes}
\item \hyperref[properties-section-phantom]{Properties of Schemes}
\item \hyperref[morphisms-section-phantom]{Morphisms of Schemes}
\item \hyperref[coherent-section-phantom]{Coherent Cohomology}
\item \hyperref[divisors-section-phantom]{Divisors}
\item \hyperref[limits-section-phantom]{Limits of Schemes}
\item \hyperref[varieties-section-phantom]{Varieties}
\item \hyperref[chow-section-phantom]{Chow Homology}
\item \hyperref[topologies-section-phantom]{Topologies on Schemes}
\item \hyperref[descent-section-phantom]{Descent}
\item \hyperref[more-morphisms-section-phantom]{More on Morphisms}
\item \hyperref[flat-section-phantom]{More on Flatness}
\item \hyperref[groupoids-section-phantom]{Groupoid Schemes}
\item \hyperref[more-groupoids-section-phantom]{More on Groupoid Schemes}
\item \hyperref[etale-section-phantom]{\'Etale Morphisms of Schemes}
\item \hyperref[etale-cohomology-section-phantom]{\'Etale Cohomology}
\item \hyperref[spaces-section-phantom]{Algebraic Spaces}
\item \hyperref[spaces-properties-section-phantom]{Properties of Algebraic Spaces}
\item \hyperref[spaces-morphisms-section-phantom]{Morphisms of Algebraic Spaces}
\item \hyperref[spaces-topologies-section-phantom]{Topologies on Algebraic Spaces}
\item \hyperref[spaces-descent-section-phantom]{Descent and Algebraic Spaces}
\item \hyperref[spaces-more-morphisms-section-phantom]{More on Morphisms of Spaces}
\item \hyperref[quot-section-phantom]{Quot and Hilbert Spaces}
\item \hyperref[stacks-section-phantom]{Stacks}
\item \hyperref[spaces-groupoids-section-phantom]{Groupoids in Algebraic Spaces}
\item \hyperref[spaces-more-groupoids-section-phantom]{More on Groupoids in Spaces}
\item \hyperref[bootstrap-section-phantom]{Bootstrap}
\item \hyperref[examples-stacks-section-phantom]{Examples of Stacks}
\item \hyperref[groupoids-quotients-section-phantom]{Quotients of Groupoids}
\item \hyperref[algebraic-section-phantom]{Algebraic Stacks}
\item \hyperref[criteria-section-phantom]{Criteria for Representability}
\item \hyperref[stacks-properties-section-phantom]{Properties of Algebraic Stacks}
\item \hyperref[stacks-morphisms-section-phantom]{Morphisms of Algebraic Stacks}
\item \hyperref[examples-section-phantom]{Examples}
\item \hyperref[exercises-section-phantom]{Exercises}
\item \hyperref[guide-section-phantom]{Guide to Literature}
\item \hyperref[desirables-section-phantom]{Desirables}
\item \hyperref[coding-section-phantom]{Coding Style}
\item \hyperref[fdl-section-phantom]{GNU Free Documentation License}
\item \hyperref[index-section-phantom]{Auto Generated Index}
\end{enumerate}
\end{multicols}


\bibliography{my}
\bibliographystyle{alpha}

\end{document}
