\IfFileExists{stacks-project.cls}{%
\documentclass{stacks-project}
}{%
\documentclass{amsart}
}

% The following AMS packages are automatically loaded with
% the amsart documentclass:
%\usepackage{amsmath}
%\usepackage{amssymb}
%\usepackage{amsthm}

% For dealing with references we use the comment environment
\usepackage{verbatim}
\newenvironment{reference}{\comment}{\endcomment}
%\newenvironment{reference}{}{}
\newenvironment{slogan}{\comment}{\endcomment}
\newenvironment{history}{\comment}{\endcomment}

% For commutative diagrams you can use
% \usepackage{amscd}
\usepackage[all]{xy}

% We use 2cell for 2-commutative diagrams.
\xyoption{2cell}
\UseAllTwocells

% To put source file link in headers.
% Change "template.tex" to "this_filename.tex"
% \usepackage{fancyhdr}
% \pagestyle{fancy}
% \lhead{}
% \chead{}
% \rhead{Source file: \url{template.tex}}
% \lfoot{}
% \cfoot{\thepage}
% \rfoot{}
% \renewcommand{\headrulewidth}{0pt}
% \renewcommand{\footrulewidth}{0pt}
% \renewcommand{\headheight}{12pt}

\usepackage{multicol}

% For cross-file-references
\usepackage{xr-hyper}

% Package for hypertext links:
\usepackage{hyperref}

% For any local file, say "hello.tex" you want to link to please
% use \externaldocument[hello-]{hello}
\externaldocument[introduction-]{introduction}
\externaldocument[conventions-]{conventions}
\externaldocument[sets-]{sets}
\externaldocument[categories-]{categories}
\externaldocument[topology-]{topology}
\externaldocument[sheaves-]{sheaves}
\externaldocument[sites-]{sites}
\externaldocument[stacks-]{stacks}
\externaldocument[fields-]{fields}
\externaldocument[algebra-]{algebra}
\externaldocument[brauer-]{brauer}
\externaldocument[homology-]{homology}
\externaldocument[derived-]{derived}
\externaldocument[simplicial-]{simplicial}
\externaldocument[more-algebra-]{more-algebra}
\externaldocument[smoothing-]{smoothing}
\externaldocument[modules-]{modules}
\externaldocument[sites-modules-]{sites-modules}
\externaldocument[injectives-]{injectives}
\externaldocument[cohomology-]{cohomology}
\externaldocument[sites-cohomology-]{sites-cohomology}
\externaldocument[dga-]{dga}
\externaldocument[dpa-]{dpa}
\externaldocument[hypercovering-]{hypercovering}
\externaldocument[schemes-]{schemes}
\externaldocument[constructions-]{constructions}
\externaldocument[properties-]{properties}
\externaldocument[morphisms-]{morphisms}
\externaldocument[coherent-]{coherent}
\externaldocument[divisors-]{divisors}
\externaldocument[limits-]{limits}
\externaldocument[varieties-]{varieties}
\externaldocument[topologies-]{topologies}
\externaldocument[descent-]{descent}
\externaldocument[perfect-]{perfect}
\externaldocument[more-morphisms-]{more-morphisms}
\externaldocument[flat-]{flat}
\externaldocument[groupoids-]{groupoids}
\externaldocument[more-groupoids-]{more-groupoids}
\externaldocument[etale-]{etale}
\externaldocument[chow-]{chow}
\externaldocument[intersection-]{intersection}
\externaldocument[pic-]{pic}
\externaldocument[adequate-]{adequate}
\externaldocument[dualizing-]{dualizing}
\externaldocument[duality-]{duality}
\externaldocument[discriminant-]{discriminant}
\externaldocument[local-cohomology-]{local-cohomology}
\externaldocument[curves-]{curves}
\externaldocument[resolve-]{resolve}
\externaldocument[models-]{models}
\externaldocument[pione-]{pione}
\externaldocument[etale-cohomology-]{etale-cohomology}
\externaldocument[proetale-]{proetale}
\externaldocument[crystalline-]{crystalline}
\externaldocument[spaces-]{spaces}
\externaldocument[spaces-properties-]{spaces-properties}
\externaldocument[spaces-morphisms-]{spaces-morphisms}
\externaldocument[decent-spaces-]{decent-spaces}
\externaldocument[spaces-cohomology-]{spaces-cohomology}
\externaldocument[spaces-limits-]{spaces-limits}
\externaldocument[spaces-divisors-]{spaces-divisors}
\externaldocument[spaces-over-fields-]{spaces-over-fields}
\externaldocument[spaces-topologies-]{spaces-topologies}
\externaldocument[spaces-descent-]{spaces-descent}
\externaldocument[spaces-perfect-]{spaces-perfect}
\externaldocument[spaces-more-morphisms-]{spaces-more-morphisms}
\externaldocument[spaces-flat-]{spaces-flat}
\externaldocument[spaces-groupoids-]{spaces-groupoids}
\externaldocument[spaces-more-groupoids-]{spaces-more-groupoids}
\externaldocument[bootstrap-]{bootstrap}
\externaldocument[spaces-pushouts-]{spaces-pushouts}
\externaldocument[groupoids-quotients-]{groupoids-quotients}
\externaldocument[spaces-more-cohomology-]{spaces-more-cohomology}
\externaldocument[spaces-simplicial-]{spaces-simplicial}
\externaldocument[spaces-duality-]{spaces-duality}
\externaldocument[formal-spaces-]{formal-spaces}
\externaldocument[restricted-]{restricted}
\externaldocument[spaces-resolve-]{spaces-resolve}
\externaldocument[formal-defos-]{formal-defos}
\externaldocument[defos-]{defos}
\externaldocument[cotangent-]{cotangent}
\externaldocument[examples-defos-]{examples-defos}
\externaldocument[algebraic-]{algebraic}
\externaldocument[examples-stacks-]{examples-stacks}
\externaldocument[stacks-sheaves-]{stacks-sheaves}
\externaldocument[criteria-]{criteria}
\externaldocument[artin-]{artin}
\externaldocument[quot-]{quot}
\externaldocument[stacks-properties-]{stacks-properties}
\externaldocument[stacks-morphisms-]{stacks-morphisms}
\externaldocument[stacks-limits-]{stacks-limits}
\externaldocument[stacks-cohomology-]{stacks-cohomology}
\externaldocument[stacks-perfect-]{stacks-perfect}
\externaldocument[stacks-introduction-]{stacks-introduction}
\externaldocument[stacks-more-morphisms-]{stacks-more-morphisms}
\externaldocument[stacks-geometry-]{stacks-geometry}
\externaldocument[moduli-]{moduli}
\externaldocument[moduli-curves-]{moduli-curves}
\externaldocument[examples-]{examples}
\externaldocument[exercises-]{exercises}
\externaldocument[guide-]{guide}
\externaldocument[desirables-]{desirables}
\externaldocument[coding-]{coding}
\externaldocument[obsolete-]{obsolete}
\externaldocument[fdl-]{fdl}
\externaldocument[index-]{index}

% Theorem environments.
%
\theoremstyle{plain}
\newtheorem{theorem}[subsection]{Theorem}
\newtheorem{proposition}[subsection]{Proposition}
\newtheorem{lemma}[subsection]{Lemma}

\theoremstyle{definition}
\newtheorem{definition}[subsection]{Definition}
\newtheorem{example}[subsection]{Example}
\newtheorem{exercise}[subsection]{Exercise}
\newtheorem{situation}[subsection]{Situation}

\theoremstyle{remark}
\newtheorem{remark}[subsection]{Remark}
\newtheorem{remarks}[subsection]{Remarks}

\numberwithin{equation}{subsection}

% Macros
%
\def\lim{\mathop{\mathrm{lim}}\nolimits}
\def\colim{\mathop{\mathrm{colim}}\nolimits}
\def\Spec{\mathop{\mathrm{Spec}}}
\def\Hom{\mathop{\mathrm{Hom}}\nolimits}
\def\Ext{\mathop{\mathrm{Ext}}\nolimits}
\def\SheafHom{\mathop{\mathcal{H}\!\mathit{om}}\nolimits}
\def\SheafExt{\mathop{\mathcal{E}\!\mathit{xt}}\nolimits}
\def\Sch{\mathit{Sch}}
\def\Mor{\operatorname{Mor}\nolimits}
\def\Ob{\mathop{\mathrm{Ob}}\nolimits}
\def\Sh{\mathop{\mathit{Sh}}\nolimits}
\def\NL{\mathop{N\!L}\nolimits}
\def\proetale{{pro\text{-}\acute{e}tale}}
\def\etale{{\acute{e}tale}}
\def\QCoh{\mathit{QCoh}}
\def\Ker{\mathop{\mathrm{Ker}}}
\def\Im{\mathop{\mathrm{Im}}}
\def\Coker{\mathop{\mathrm{Coker}}}
\def\Coim{\mathop{\mathrm{Coim}}}

%
% Macros for moduli stacks/spaces
%
\def\QCohstack{\mathcal{QC}\!\mathit{oh}}
\def\Cohstack{\mathcal{C}\!\mathit{oh}}
\def\Spacesstack{\mathcal{S}\!\mathit{paces}}
\def\Quotfunctor{\mathrm{Quot}}
\def\Hilbfunctor{\mathrm{Hilb}}
\def\Curvesstack{\mathcal{C}\!\mathit{urves}}
\def\Polarizedstack{\mathcal{P}\!\mathit{olarized}}
\def\Complexesstack{\mathcal{C}\!\mathit{omplexes}}
% \Pic is the operator that assigns to X its picard group, usage \Pic(X)
% \Picardstack_{X/B} denotes the Picard stack of X over B
% \Picardfunctor_{X/B} denotes the Picard functor of X over B
\def\Pic{\mathop{\mathrm{Pic}}\nolimits}
\def\Picardstack{\mathcal{P}\!\mathit{ic}}
\def\Picardfunctor{\mathrm{Pic}}
\def\Deformationcategory{\mathcal{D}\!\mathit{ef}}


% OK, start here.
%
\begin{document}

\title{Deformation Problems}

\maketitle

\phantomsection
\label{section-phantom}

\tableofcontents

\section{Introduction}
\label{section-introduction}

\noindent
The goal of this chapter is to work out examples of the general theory
developed in the chapters Formal Deformation Theory,
Deformation Theory, The Cotangent Complex.

\medskip\noindent
Section 3 of the paper \cite{Sch} by Schlessinger discusses some
examples as well.






\section{Examples of deformation problems}
\label{section-examples}

\noindent
List of things that should go here:
\begin{enumerate}
\item Deformations of schemes:
\begin{enumerate}
\item The Rim-Schlessinger condition.
\item Computing the tangent space.
\item Computing the infinitesimal deformations.
\item The deformation category of an affine hypersurface.
\end{enumerate}
\item Deformations of sheaves (for example fix $X/S$, a finite type point
$s$ of $S$, and a quasi-coherent sheaf $\mathcal{F}_s$ over $X_s$).
\item Deformations of algebraic spaces (very similar to deformations
of schemes; maybe even easier?).
\item Deformations of maps (eg morphisms between schemes; you can fix
both or one of the target and/or source).
\item Add more here.
\end{enumerate}





\section{General outline}
\label{section-general}

\noindent
This section lays out the procedure for discussing the next few examples.

\medskip\noindent
Step I. For each section we fix a Noetherian ring $\Lambda$ and
we fix a finite ring map $\Lambda \to k$ where $k$ is a field.
As usual we let $\mathcal{C}_\Lambda = \mathcal{C}_{\Lambda, k}$
be our base category, see
Formal Deformation Theory,
Definition \ref{formal-defos-definition-CLambda}.

\medskip\noindent
Step II. In each section we define a category $\mathcal{F}$
cofibred in groupoids over $\mathcal{C}_\Lambda$. Occassionally
we will consider instead a functor
$F : \mathcal{C}_\Lambda \to \textit{Sets}$.

\medskip\noindent
Step III. We explain to what extend $\mathcal{F}$ satisfies
the Rim-Schlesssinger condition (RS) discussed in
Formal Deformation Theory, Section \ref{formal-defos-section-RS-condition}.
Similarly, we may discuss to what extend our $\mathcal{F}$
satisfies (S1) and (S2) or to what extend $F$ satisfies
the corresponding Schlessinger's conditions (H1) and (H2).
See Formal Deformation Theory, Section
\ref{formal-defos-section-schlessinger-conditions}.

\medskip\noindent
Step IV. Let $x_0$ be an object of $\mathcal{F}(k)$, in other words an object
of $\mathcal{F}$ over $k$. In this chapter we will use the notation
$$
\Deformationcategory_{x_0} = \mathcal{F}_{x_0}
$$
to denote the predeformation category constructed in
Formal Deformation Theory, Remark
\ref{formal-defos-remark-localize-cofibered-groupoid}.
If $\mathcal{F}$ satisfies (RS), then
$\Deformationcategory_{x_0}$ is a deformation category
(Formal Deformation Theory, Lemma
\ref{formal-defos-lemma-localize-RS})
and satisfies (S1) and (S2)
(Formal Deformation Theory, Lemma
\ref{formal-defos-lemma-RS-implies-S1-S2}).
If (S1) and (S2) are satisfied, then
an important question is whether the tangent space
$$
T\Deformationcategory_{x_0} = T_{x_0}\mathcal{F} = T\mathcal{F}_{x_0}
$$
(see Formal Deformation Theory, Remark
\ref{formal-defos-remark-tangent-space-cofibered-groupoid} and
Definition \ref{formal-defos-definition-tangent-space})
is finite dimensional. Namely, this insures that
$\Deformationcategory_{x_0}$ has a versal formal object
(Formal Deformation Theory, Lemma
\ref{formal-defos-lemma-versal-object-existence}).

\medskip\noindent
Step V. If $\mathcal{F}$ passes Step IV, then the next question is whether
the $k$-vector space
$$
\text{Inf}(\Deformationcategory_{x_0}) = \text{Inf}_{x_0}(\mathcal{F})
$$
of infinitesimal automorphisms of $x_0$ is finite dimensional.
Namely, if true, this implies that
$\Deformationcategory_{x_0}$ admits a presentation by a
smooth prorepresentable groupoid in functors on $\mathcal{C}_\Lambda$, see
Formal Deformation Theory, Theorem
\ref{formal-defos-theorem-presentation-deformation-groupoid}.




\section{Finite projective modules}
\label{section-finite-projective-modules}

\noindent
This section is just a warmup. Of course finite projective modules
should not have any ``moduli''.

\begin{example}[Finite projective modules]
\label{example-finite-projective-modules}
Let $\mathcal{F}$ be the category defined as follows
\begin{enumerate}
\item an object is a pair $(A, M)$ consisting of an
object $A$ of $\mathcal{C}_\Lambda$ and a
finite projective $A$-module $M$, and
\item a morphism $(f, g) : (B, N) \to (A, M)$ consists of
a morphism $f : B \to A$ in $\mathcal{C}_\Lambda$ together
with a map $g : N \to M$ which is $f$-linear and induces
an isomorpism $N \otimes_{B, f} A \cong M$.
\end{enumerate}
The functor $p : \mathcal{F} \to \mathcal{C}_\Lambda$ sends $(A, M)$ to $A$
and $(f, g)$ to $f$. It is clear that $p$ is cofibred in groupoids.
Given a finite dimensional $k$-vector space $V$,
let $x_0 = (k, V)$ be the corresponding object of $\mathcal{F}(k)$.
We set
$$
\Deformationcategory_V = \mathcal{F}_{x_0}
$$
\end{example}

\noindent
Since every finite projective module over a local ring is finite free
(Algebra, Lemma \ref{algebra-lemma-finite-projective})
we see that
$$
\begin{matrix}
\text{isomorphism classes} \\
\text{of objects of }\mathcal{F}(A)
\end{matrix}
= \coprod\nolimits_{n \geq 0} \{*\}
$$
Although this means that the deformation theory of $\mathcal{F}$
is essentially trivial, we still work through the steps outlined
in Section \ref{section-general} to provide an easy example.

\begin{lemma}
\label{lemma-finite-projective-modules-RS}
Example \ref{example-finite-projective-modules}
satisfies the Rim-Schlessinger condition (RS).
In particular, $\Deformationcategory_V$ is a deformation category
for any finite dimensional vector space $V$ over $k$.
\end{lemma}

\begin{proof}
Let $A_1 \to A$ and $A_2 \to A$ be morphisms of $\mathcal{C}_\Lambda$.
Assume $A_2 \to A$ is surjective. According to
Formal Deformation Theory, Lemma
\ref{formal-defos-lemma-RS-2-categorical}
it suffices to show that the functor
$\mathcal{F}(A_1 \times_A A_2) \to
\mathcal{F}(A_1) \times_{\mathcal{F}(A)} \mathcal{F}(A_2)$
is an equivalence of categories.

\medskip\noindent
Thus we have to show that the category of finite projective modules
over $A_1 \times_A A_2$ is equivalent to the fibre product
of the categories of finite projective modules over $A_1$ and $A_2$
over the category of finite projective modules over $A$.
This is a special case of More on Algebra, Lemma
\ref{more-algebra-lemma-finitely-presented-module-over-fibre-product}.
We recall that the inverse functor sends the triple
$(M_1, M_2, \varphi)$ where
$M_1$ is a finite projective $A_1$-module,
$M_2$ is a finite projective $A_2$-module, and
$\varphi : M_1 \otimes_{A_1} A \to M_2 \otimes_{A_2} A$
is an isomorphism of $A$-module, to the finite projective
$A_1 \times_A A_2$-module $M_1 \times_\varphi M_2$.
\end{proof}

\begin{lemma}
\label{lemma-finite-projective-modules-TI}
In Example \ref{example-finite-projective-modules}
let $V$ be a finite dimensional $k$-vector space. Then
$$
T\Deformationcategory_V = (0)
\quad\text{and}\quad
\text{Inf}(\Deformationcategory_V) = \text{End}_k(V)
$$
are finite dimensional.
\end{lemma}

\begin{proof}
With $\mathcal{F}$ as in Example \ref{example-finite-projective-modules}
set $x_0 = (k, V) \in \Ob(\mathcal{F}(k))$.
Recall that $T\Deformationcategory_V = T_{x_0}\mathcal{F}$
is the set of isomorphism
classes of pairs $(x, \alpha)$ consisting of an object $x$ of $\mathcal{F}
$ over the dual numbers $k[\epsilon]$ and a morphism
$\alpha : x \to x_0$ of $\mathcal{F}$ lying over $k[\epsilon] \to k$.

\medskip\noindent
Up to isomorphism, there is a unique pair $(M, \alpha)$ consisting of a
finite projective module $M$ over $k[\epsilon]$
and $k[\epsilon]$-linear map $\alpha : M \to V$
which induces an isomorphism $M \otimes_{k[\epsilon]} k \to V$.
For example, if $V = k^{\oplus n}$, then we take
$M = k[\epsilon]^{\oplus n}$ with the obvious map $\alpha$.

\medskip\noindent
Similarly, $\text{Inf}(\Deformationcategory_V) = \text{Inf}_{x_0}(\mathcal{F})$
is the set of automorphisms
of the trivial deformation $x'_0$ of $x_0$ over $k[\epsilon]$.
See Formal Deformation Theory, Definition
\ref{formal-defos-definition-infinitesimal-auts} for details.

\medskip\noindent
Given $(M, \alpha)$ as in the second paragraph, we see that an element of
$\text{Inf}_{x_0}(\mathcal{F})$ is an automorphism $\gamma : M \to M$ with
$\gamma \bmod \epsilon = \text{id}$. Then we can write
$\gamma = \text{id}_M + \epsilon \psi$ where
$\psi : M/\epsilon M \to M/\epsilon M$ is $k$-linear.
Using $\alpha$ we can think of $\psi$ as an element of
$\text{End}_k(V)$ and this finishes the proof.
\end{proof}


\section{Representations of a group}
\label{section-representations}

\noindent
The deformation theory of representations can be very interesting.

\begin{example}[Representations of a group]
\label{example-representations}
Let $\Gamma$ be a group.
Let $\mathcal{F}$ be the category defined as follows
\begin{enumerate}
\item an object is a triple $(A, M, \rho)$ consisting of an
object $A$ of $\mathcal{C}_\Lambda$, a finite projective $A$-module $M$,
and a homomorphism $\rho : \Gamma \to \text{GL}_A(M)$, and
\item a morphism $(f, g) : (B, N, \tau) \to (A, M, \rho)$ consists of
a morphism $f : B \to A$ in $\mathcal{C}_\Lambda$ together
with a map $g : N \to M$ which is $f$-linear and $\Gamma$-equivariant
and induces an isomorpism $N \otimes_{B, f} A \cong M$.
\end{enumerate}
The functor $p : \mathcal{F} \to \mathcal{C}_\Lambda$ sends $(A, M, \rho)$
to $A$ and $(f, g)$ to $f$. It is clear that $p$ is cofibred in groupoids.
Given a finite dimensional $k$-vector space $V$ and a representation
$\rho_0 : \Gamma \to \text{GL}_k(V)$,
let $x_0 = (k, V, \rho_0)$ be the corresponding object of $\mathcal{F}(k)$.
We set
$$
\Deformationcategory_{V, \rho_0} = \mathcal{F}_{x_0}
$$
\end{example}

\noindent
Since every finite projective module over a local ring is finite free
(Algebra, Lemma \ref{algebra-lemma-finite-projective})
we see that
$$
\begin{matrix}
\text{isomorphism classes} \\
\text{of objects of }\mathcal{F}(A)
\end{matrix}
=
\coprod\nolimits_{n \geq 0}\quad
\begin{matrix}
\text{GL}_n(A)\text{-conjugacy classes of}\\
\text{homomorphisms }\rho : \Gamma \to \text{GL}_n(A)
\end{matrix}
$$
This is already more interesting than the discussion in
Section \ref{section-finite-projective-modules}.

\begin{lemma}
\label{lemma-representations-RS}
Example \ref{example-representations}
satisfies the Rim-Schlessinger condition (RS).
In particular, $\Deformationcategory_{V, \rho_0}$ is a deformation category
for any finite dimensional representation
$\rho_0 : \Gamma \to \text{GL}_k(V)$.
\end{lemma}

\begin{proof}
Let $A_1 \to A$ and $A_2 \to A$ be morphisms of $\mathcal{C}_\Lambda$.
Assume $A_2 \to A$ is surjective. According to
Formal Deformation Theory, Lemma
\ref{formal-defos-lemma-RS-2-categorical}
it suffices to show that the functor
$\mathcal{F}(A_1 \times_A A_2) \to
\mathcal{F}(A_1) \times_{\mathcal{F}(A)} \mathcal{F}(A_2)$
is an equivalence of categories.

\medskip\noindent
Consider an object
$$
((A_1, M_1, \rho_1), (A_2, M_2, \rho_2), (\text{id}_A, \varphi))
$$
of the category $\mathcal{F}(A_1) \times_{\mathcal{F}(A)} \mathcal{F}(A_2)$.
Then, as seen in the proof of Lemma \ref{lemma-finite-projective-modules-RS},
we can consider the finite projective
$A_1 \times_A A_2$-module $M_1 \times_\varphi M_2$.
Since $\varphi$ is compatible with the given actions we obtain
$$
\rho_1 \times \rho_2 : \Gamma \longrightarrow
\text{GL}_{A_1 \times_A A_2}(M_1 \times_\varphi M_2)
$$
Then $(M_1 \times_\varphi M_2, \rho_1 \times \rho_2)$
is an object of $\mathcal{F}(A_1 \times_A A_2)$.
This construction determines a quasi-inverse to our functor.
\end{proof}

\begin{lemma}
\label{lemma-representations-TI}
In Example \ref{example-representations} let 
$\rho_0 : \Gamma \to \text{GL}_k(V)$
be a finite dimensional representation. Then
$$
T\Deformationcategory_{V, \rho_0} = \Ext^1_{k[\Gamma]}(V, V) =
H^1(\Gamma, \text{End}_k(V))
\quad\text{and}\quad
\text{Inf}(\Deformationcategory_{V, \rho_0}) = H^0(\Gamma, \text{End}_k(V))
$$
Thus $\text{Inf}(\Deformationcategory_{V, \rho_0})$
is always finite dimensional
and $T\Deformationcategory_{V, \rho_0}$ is finite dimensional
if $\Gamma$ is finitely generated.
\end{lemma}

\begin{proof}
We first deal with the infinitesimal automorphisms.
Let $M = V \otimes_k k[\epsilon]$ with induced action
$\rho_0' : \Gamma \to \text{GL}_n(M)$.
Then an infinitesimal automorphism, i.e., an element of
$\text{Inf}(\Deformationcategory_{V, \rho_0})$,
is given by an automorphism
$\gamma = \text{id} + \epsilon \psi : M \to M$
as in the proof of Lemma \ref{lemma-finite-projective-modules-TI},
where moreover $\psi$ has to commute
with the action of $\Gamma$ (given by $\rho_0$).
Thus we see that
$$
\text{Inf}(\Deformationcategory_{V, \rho_0}) = H^0(\Gamma, \text{End}_k(V))
$$
as predicted in the lemma.

\medskip\noindent
Next, let $(k[\epsilon], M, \rho)$ be an object of $\mathcal{F}$
over $k[\epsilon]$ and let $\alpha : M \to V$ be a $\Gamma$-equivariant map
inducing an isomorphism $M/\epsilon M \to V$.
Since $M$ is free as a $k[\epsilon]$-module we obtain
an extension of $\Gamma$-modules
$$
0 \to V \to M \xrightarrow{\alpha} V \to 0
$$
We omit the detailed construction of the map on the left.
Conversely, if we have an extension of $\Gamma$-modules as
above, then we can use this to make a $k[\epsilon]$-module
structure on $M$ and get an object of $\mathcal{F}(k[\epsilon])$
together with a map $\alpha$ as above.
It follows that
$$
T\Deformationcategory_{V, \rho_0} = \Ext^1_{k[\Gamma]}(V, V)
$$
as predicted in the lemma. This is equal to
$H^1(\Gamma, \text{End}_k(V))$ by
\'Etale Cohomology, Lemma \ref{etale-cohomology-lemma-ext-modules-hom}.

\medskip\noindent
The statement on dimensions follows from
\'Etale Cohomology, Lemma
\ref{etale-cohomology-lemma-finite-dim-group-cohomology}.
\end{proof}

\noindent
In Example \ref{example-representations} if $\Gamma$ is finitely generated
and $(V, \rho_0)$ is a finite dimensional representation of $\Gamma$
over $k$, then $\Deformationcategory_{V, \rho_0}$
admits a presentation by a smooth prorepresentable groupoid in functors
over $\mathcal{C}_\Lambda$
and a fortiori has a (minimal) versal formal object. This follows
from Lemmas \ref{lemma-representations-RS} and \ref{lemma-representations-TI}
and the general discussion in Section \ref{section-general}.

\begin{lemma}
\label{lemma-representations-hull}
In Example \ref{example-representations} assume $\Gamma$ finitely generated.
Let $\rho_0 : \Gamma \to \text{GL}_k(V)$ be a finite dimensional representation.
Assume $\Lambda$ is a complete local ring with residue field $k$
(the classical case). Then the functor
$$
F : \mathcal{C}_\Lambda \longrightarrow \textit{Sets},\quad
A \longmapsto \Ob(\Deformationcategory_{V, \rho_0}(A))/\cong
$$
of isomorphism classes of objects has a hull. If
$H^0(\Gamma, \text{End}_k(V)) = k$, then $F$ is
prorepresentable.
\end{lemma}

\begin{proof}
The existence of a hull follows from Lemmas \ref{lemma-representations-RS} and
\ref{lemma-representations-TI} and
Formal Deformation Theory, Lemma \ref{formal-defos-lemma-RS-implies-S1-S2}
and Remark \ref{formal-defos-remark-compose-minimal-into-iso-classes}.

\medskip\noindent
Assume $H^0(\Gamma, \text{End}_k(V)) = k$. To see that $F$
is prorepresentable it suffices to show that $F$ is a
deformation functor, see Formal Deformation Theory, Theorem
\ref{formal-defos-theorem-Schlessinger-prorepresentability}.
In other words, we have to show $F$ satisfies (RS).
For this we can use the criterion of Formal Deformation Theory, Lemma
\ref{formal-defos-lemma-RS-associated-functor}.
The required surjectivity of automorphism groups will follow if we
show that
$$
A \cdot \text{id}_M =
\text{End}_{A[\Gamma]}(M)
$$
for any object $(A, M, \rho)$ of $\mathcal{F}$ such that
$M \otimes_A k$ is isomorphic to $V$ as a representation of $\Gamma$.
Since the left hand side is contained in the right hand side,
it suffices to show
$\text{length}_A \text{End}_{A[\Gamma]}(M) \leq \text{length}_A A$.
Choose pairwise distinct ideals
$(0) = I_n \subset \ldots \subset I_1 \subset A$
with $n = \text{length}(A)$. By correspondingly filtering
$M$, we see that it suffices to prove $\Hom_{A[\Gamma]}(M, I_tM/I_{t + 1}M)$
has length $1$. Since $I_tM/I_{t + 1}M \cong M \otimes_A k$
and since any $A[\Gamma]$-module map $M \to M \otimes_A k$ factors
uniquely through the quotient map $M \to M \otimes_A k$
to give an element of
$$
\text{End}_{A[\Gamma]}(M \otimes_A k) = \text{End}_{k[\Gamma]}(V) = k
$$
we conclude.
\end{proof}



\section{Continuous representations}
\label{section-continuous-representations}

\noindent
A very interesting thing one can do is to take an infinite Galois
group and study the deformation theory of its representations, see
\cite{Mazur-deforming}.

\begin{example}[Representations of a topological group]
\label{example-continuous-representations}
Let $\Gamma$ be a topological group.
Let $\mathcal{F}$ be the category defined as follows
\begin{enumerate}
\item an object is a triple $(A, M, \rho)$ consisting of an
object $A$ of $\mathcal{C}_\Lambda$, a finite projective $A$-module $M$,
and a continuous homomorphism $\rho : \Gamma \to \text{GL}_A(M)$
where $\text{GL}_A(M)$ is given the discrete topology\footnote{An alternative
would be to require the $A$-module $M$ with $G$-action given by $\rho$
is an $A\text{-}G$-module as defined in \'Etale Cohomology, Definition
\ref{etale-cohomology-definition-G-module-continuous}. However,
since $M$ is a finite $A$-module, this is equivalent.}, and
\item a morphism $(f, g) : (B, N, \tau) \to (A, M, \rho)$ consists of
a morphism $f : B \to A$ in $\mathcal{C}_\Lambda$ together
with a map $g : N \to M$ which is $f$-linear and $\Gamma$-equivariant
and induces an isomorpism $N \otimes_{B, f} A \cong M$.
\end{enumerate}
The functor $p : \mathcal{F} \to \mathcal{C}_\Lambda$ sends $(A, M, \rho)$
to $A$ and $(f, g)$ to $f$. It is clear that $p$ is cofibred in groupoids.
Given a finite dimensional $k$-vector space $V$ and a
continuous representation $\rho_0 : \Gamma \to \text{GL}_k(V)$,
let $x_0 = (k, V, \rho_0)$ be the corresponding object of $\mathcal{F}(k)$.
We set
$$
\Deformationcategory_{V, \rho_0} = \mathcal{F}_{x_0}
$$
\end{example}

\noindent
Since every finite projective module over a local ring is finite free
(Algebra, Lemma \ref{algebra-lemma-finite-projective})
we see that
$$
\begin{matrix}
\text{isomorphism classes} \\
\text{of objects of }\mathcal{F}(A)
\end{matrix}
=
\coprod\nolimits_{n \geq 0}\quad
\begin{matrix}
\text{GL}_n(A)\text{-conjugacy classes of}\\
\text{continuous homomorphisms }\rho : \Gamma \to \text{GL}_n(A)
\end{matrix}
$$

\begin{lemma}
\label{lemma-continuous-representations-RS}
Example \ref{example-continuous-representations}
satisfies the Rim-Schlessinger condition (RS).
In particular, $\Deformationcategory_{V, \rho_0}$ is a deformation category
for any finite dimensional continuous representation
$\rho_0 : \Gamma \to \text{GL}_k(V)$.
\end{lemma}

\begin{proof}
The proof is exactly the same as the proof of
Lemma \ref{lemma-representations-RS}.
\end{proof}

\begin{lemma}
\label{lemma-continuous-representations-TI}
In Example \ref{example-continuous-representations} let
$\rho_0 : \Gamma \to \text{GL}_k(V)$ be a finite dimensional
continuous representation. Then
$$
T\Deformationcategory_{V, \rho_0} = H^1(\Gamma, \text{End}_k(V))
\quad\text{and}\quad
\text{Inf}(\Deformationcategory_{V, \rho_0}) = H^0(\Gamma, \text{End}_k(V))
$$
Thus $\text{Inf}(\Deformationcategory_{V, \rho_0})$
is always finite dimensional
and $T\Deformationcategory_{V, \rho_0}$ is finite dimensional
if $\Gamma$ is topologically finitely generated.
\end{lemma}

\begin{proof}
The proof is exactly the same as the proof of
Lemma \ref{lemma-representations-TI}.
\end{proof}

\noindent
In Example \ref{example-continuous-representations} if $\Gamma$
is topologically finitely generated
and $(V, \rho_0)$ is a finite dimensional continuous representation of $\Gamma$
over $k$, then $\Deformationcategory_{V, \rho_0}$
admits a presentation by a smooth prorepresentable groupoid in functors
over $\mathcal{C}_\Lambda$
and a fortiori has a (minimal) versal formal object. This follows
from Lemmas \ref{lemma-continuous-representations-RS} and
\ref{lemma-continuous-representations-TI}
and the general discussion in Section \ref{section-general}.

\begin{lemma}
\label{lemma-continuous-representations-hull}
In Example \ref{example-continuous-representations} assume $\Gamma$
is topologically finitely generated.
Let $\rho_0 : \Gamma \to \text{GL}_k(V)$ be a finite dimensional representation.
Assume $\Lambda$ is a complete local ring with residue field $k$
(the classical case). Then the functor
$$
F : \mathcal{C}_\Lambda \longrightarrow \textit{Sets},\quad
A \longmapsto \Ob(\Deformationcategory_{V, \rho_0}(A))/\cong
$$
of isomorphism classes of objects has a hull. If
$H^0(\Gamma, \text{End}_k(V)) = k$, then $F$ is
prorepresentable.
\end{lemma}

\begin{proof}
The proof is exactly the same as the proof of
Lemma \ref{lemma-representations-hull}.
\end{proof}



\section{Graded algebras}
\label{section-graded-algebras}

\noindent
We will use the example in this section in the proof that the stack of
polarized proper schemes is an algebraic stack. For this reason we will
consider commutative graded algebras whose homogeneous parts are
finite projective modules (sometimes called ``locally finite'').

\begin{example}[Graded algebras]
\label{example-graded-algebras}
Let $\mathcal{F}$ be the category defined as follows
\begin{enumerate}
\item an object is a pair $(A, P)$ consisting of an
object $A$ of $\mathcal{C}_\Lambda$ and a graded $A$-algebra $P$
such that $P_d$ is a finite projective $A$-module for all $d \geq 0$, and
\item a morphism $(f, g) : (B, Q) \to (A, P)$ consists of
a morphism $f : B \to A$ in $\mathcal{C}_\Lambda$ together
with a map $g : Q \to P$ which is $f$-linear and induces an
isomorpism $Q \otimes_{B, f} A \cong P$.
\end{enumerate}
The functor $p : \mathcal{F} \to \mathcal{C}_\Lambda$ sends $(A, P)$
to $A$ and $(f, g)$ to $f$. It is clear that $p$ is cofibred in groupoids.
Given a graded $k$-algebra $P$ with $\dim_k(P_d) < \infty$ for all
$d \geq 0$, let $x_0 = (k, P)$ be the corresponding object of $\mathcal{F}(k)$.
We set
$$
\Deformationcategory_P = \mathcal{F}_{x_0}
$$
\end{example}

\begin{lemma}
\label{lemma-graded-algebras-RS}
Example \ref{example-graded-algebras}
satisfies the Rim-Schlessinger condition (RS).
In particular, $\Deformationcategory_P$ is a deformation category
for any graded $k$-algebra $P$.
\end{lemma}

\begin{proof}
Let $A_1 \to A$ and $A_2 \to A$ be morphisms of $\mathcal{C}_\Lambda$.
Assume $A_2 \to A$ is surjective. According to
Formal Deformation Theory, Lemma
\ref{formal-defos-lemma-RS-2-categorical}
it suffices to show that the functor
$\mathcal{F}(A_1 \times_A A_2) \to
\mathcal{F}(A_1) \times_{\mathcal{F}(A)} \mathcal{F}(A_2)$
is an equivalence of categories.

\medskip\noindent
Consider an object
$$
((A_1, P_1), (A_2, P_2), (\text{id}_A, \varphi))
$$
of the category $\mathcal{F}(A_1) \times_{\mathcal{F}(A)} \mathcal{F}(A_2)$.
Then we consider $P_1 \times_\varphi P_2$. Since
$\varphi : P_1 \otimes_{A_1} A \to P_2 \otimes_{A_2} A$
is an isomorphism of graded algebras, we see that the graded pieces
of $P_1 \times_\varphi P_2$ are finite projective $A_1 \times_A A_2$-modules,
see proof of Lemma \ref{lemma-finite-projective-modules-RS}.
Thus $P_1 \times_\varphi P_2$ is an object of $\mathcal{F}(A_1 \times_A A_2)$.
This construction determines a quasi-inverse to our functor
and the proof is complete.
\end{proof}

\begin{lemma}
\label{lemma-graded-algebras-TI}
In Example \ref{example-graded-algebras} let $P$ be a graded $k$-algebra.
Then
$$
T\Deformationcategory_P
\quad\text{and}\quad
\text{Inf}(\Deformationcategory_P) = \text{Der}_k(P, P)
$$
are finite dimensional if $P$ is finitely generated over $k$.
\end{lemma}

\begin{proof}
We first deal with the infinitesimal automorphisms.
Let $Q = P \otimes_k k[\epsilon]$.
Then an element of $\text{Inf}(\Deformationcategory_P)$
is given by an automorphism
$\gamma = \text{id} + \epsilon \delta : Q \to Q$
as above where now $\delta : P \to P$.
The fact that $\gamma$ is graded implies that
$\delta$ is homogeneous of degree $0$.
The fact that $\gamma$ is $k$-linear implies that
$\delta$ is $k$-linear.
The fact that $\gamma$ is multiplicative implies that
$\delta$ is a $k$-derivation.
Conversely, given a $k$-derivation $\delta : P \to P$
homogeneous of degree $0$, we obtain an automorphism
$\gamma = \text{id} + \epsilon \delta$ as above.
Thus we see that
$$
\text{Inf}(\Deformationcategory_P) = \text{Der}_k(P, P)
$$
as predicted in the lemma.
Clearly, if $P$ is generated in degrees $P_i$,
$0 \leq i \leq N$, then $\delta$ is determined by
the linear maps $\delta_i : P_i \to P_i$ for
$0 \leq i \leq N$ and we see that
$$
\dim_k \text{Der}_k(P, P) < \infty
$$
as desired.

\medskip\noindent
To finish the proof of the lemma we show that there is a finite
dimensional deformation space. To do this we
choose a presentation
$$
k[X_1, \ldots, X_n]/(F_1, \ldots, F_m) \longrightarrow P
$$
of graded $k$-algebras where $\deg(X_i) = d_i$ and
$F_j$ is homogeneous of degree $e_j$.
Let $Q$ be any graded $k[\epsilon]$-algebra
finite free in each degree which comes with an isomorphsm
$\alpha : Q/\epsilon Q \to P$ so that $(Q, \alpha)$ defines
an element of $T\Deformationcategory_P$.
Choose a homogeneous element $q_i \in Q$ of degree $d_i$
mapping to the image of $X_i$ in $P$.
Then we obtain
$$
k[\epsilon][X_1, \ldots, X_n] \longrightarrow Q,\quad
X_i \longmapsto q_i
$$
and since $P = Q/\epsilon Q$ this map is surjective by Nakayama's lemma.
A small diagram chase shows we can choose homogeneous elements
$F_{\epsilon, j} \in k[\epsilon][X_1, \ldots, X_n]$ of degree $e_j$
mapping to zero in $Q$ and mapping to $F_j$ in $k[X_1, \ldots, X_n]$.
Then
$$
k[\epsilon][X_1, \ldots, X_n]/(F_{\epsilon, 1}, \ldots, F_{\epsilon, m})
\longrightarrow Q
$$
is a presentation of $Q$ by flatness of $Q$ over $k[\epsilon]$.
Write
$$
F_{\epsilon, j} =  F_j + \epsilon G_j
$$
There is some ambiguity in the vector $(G_1, \ldots, G_m)$.
First, using different choices of $F_{\epsilon, j}$
we can modify $G_j$ by an arbitrary element of degree $e_j$
in the kernel of $k[X_1, \ldots, X_n] \to P$.
Hence, instead of $(G_1, \ldots, G_m)$, we remember the
element
$$
(g_1, \ldots, g_m) \in P_{e_1} \oplus \ldots \oplus P_{e_m}
$$
where $g_j$ is the image of $G_j$ in $P_{e_j}$.
Moreover, if we change our choice of $q_i$ into $q_i + \epsilon p_i$
with $p_i$ of degree $d_i$ then a computation (omitted) shows
that $g_j$ changes into
$$
g_j^{new} = g_j - \sum\nolimits_{i = 1}^n p_i \partial F_j / \partial X_i
$$
We conclude that the isomorphism class of $Q$ is determined by the
image of the vector $(G_1, \ldots, G_m)$ in the $k$-vector space
$$
W  = \Coker(P_{d_1} \oplus \ldots \oplus P_{d_n}
\xrightarrow{(\frac{\partial F_j}{\partial X_i})}
P_{e_1} \oplus \ldots \oplus P_{e_m})
$$
In this way we see that we obtain an injection
$$
T\Deformationcategory_P \longrightarrow W
$$
Since $W$ visibly has finite dimension, we conclude that the lemma is true.
\end{proof}

\noindent
In Example \ref{example-graded-algebras} if $P$ is a finitely generated
graded $k$-algebra, then $\Deformationcategory_P$
admits a presentation by a smooth prorepresentable groupoid in functors
over $\mathcal{C}_\Lambda$
and a fortiori has a (minimal) versal formal object. This follows
from Lemmas \ref{lemma-graded-algebras-RS} and
\ref{lemma-graded-algebras-TI}
and the general discussion in Section \ref{section-general}.

\begin{lemma}
\label{lemma-graded-algebras-hull}
In Example \ref{example-graded-algebras} assume $P$ is a finitely generated
graded $k$-algebra. Assume $\Lambda$ is a complete local ring
with residue field $k$
(the classical case). Then the functor
$$
F : \mathcal{C}_\Lambda \longrightarrow \textit{Sets},\quad
A \longmapsto \Ob(\Deformationcategory_P(A))/\cong
$$
of isomorphism classes of objects has a hull.
\end{lemma}

\begin{proof}
This follows immediately from Lemmas \ref{lemma-graded-algebras-RS} and
\ref{lemma-graded-algebras-TI} and
Formal Deformation Theory, Lemma \ref{formal-defos-lemma-RS-implies-S1-S2}
and Remark \ref{formal-defos-remark-compose-minimal-into-iso-classes}.
\end{proof}







\section{Rings}
\label{section-rings}

\noindent
The deformation theory of rings is the same as the deformation theory
of affine schemes. For rings and schemes when we talk about deformations
it means we are thinking about {\it flat} deformations.

\begin{example}[Rings]
\label{example-rings}
Let $\mathcal{F}$ be the category defined as follows
\begin{enumerate}
\item an object is a pair $(A, P)$ consisting of an
object $A$ of $\mathcal{C}_\Lambda$ and a flat $A$-algebra $P$, and
\item a morphism $(f, g) : (B, Q) \to (A, P)$ consists of
a morphism $f : B \to A$ in $\mathcal{C}_\Lambda$ together
with a map $g : Q \to P$ which is $f$-linear and induces an
isomorpism $Q \otimes_{B, f} A \cong P$.
\end{enumerate}
The functor $p : \mathcal{F} \to \mathcal{C}_\Lambda$ sends $(A, P)$
to $A$ and $(f, g)$ to $f$. It is clear that $p$ is cofibred in groupoids.
Given a $k$-algebra $P$, let $x_0 = (k, P)$ be the corresponding object
of $\mathcal{F}(k)$. We set
$$
\Deformationcategory_P = \mathcal{F}_{x_0}
$$
\end{example}

\begin{lemma}
\label{lemma-rings-RS}
Example \ref{example-rings}
satisfies the Rim-Schlessinger condition (RS).
In particular, $\Deformationcategory_P$ is a deformation category
for any $k$-algebra $P$.
\end{lemma}

\begin{proof}
Let $A_1 \to A$ and $A_2 \to A$ be morphisms of $\mathcal{C}_\Lambda$.
Assume $A_2 \to A$ is surjective. According to
Formal Deformation Theory, Lemma
\ref{formal-defos-lemma-RS-2-categorical}
it suffices to show that the functor
$\mathcal{F}(A_1 \times_A A_2) \to
\mathcal{F}(A_1) \times_{\mathcal{F}(A)} \mathcal{F}(A_2)$
is an equivalence of categories.
This is a special case of More on Algebra, Lemma
\ref{more-algebra-lemma-properties-algebras-over-fibre-product}.
\end{proof}

\begin{lemma}
\label{lemma-rings-TI}
In Example \ref{example-rings} let $P$ be a $k$-algebra. Then
$$
T\Deformationcategory_P = \text{Ext}^1_P(\NL_{P/k}, P)
\quad\text{and}\quad
\text{Inf}(\Deformationcategory_P) = \text{Der}_k(P, P)
$$
\end{lemma}

\begin{proof}
Recall that $\text{Inf}(\Deformationcategory_P)$ is the set of
automorphisms of the trivial deformation
$P[\epsilon] = P \otimes_k k[\epsilon]$ of $P$ to $k[\epsilon]$
equal to the identity modulo $\epsilon$.
By Deformation Theory, Lemma \ref{defos-lemma-huge-diagram}
this is equal to $\Hom_P(\Omega_{P/k}, P)$ which in turn is
equal to $\text{Der}_k(P, P)$ by
Algebra, Lemma \ref{algebra-lemma-universal-omega}.

\medskip\noindent
Recall that $T\Deformationcategory_P$ is the set of isomorphism classes
of flat deformations $Q$ of $P$ to $k[\epsilon]$, more precisely,
the set of isomorphism classes of $\Deformationcategory_P(k[\epsilon])$.
Recall that a $k[\epsilon]$-algebra $Q$ with $Q/\epsilon Q = P$
is flat over $k[\epsilon]$ if and only if
$$
0 \to P \xrightarrow{\epsilon} Q \to P \to 0
$$
is exact. This is proven in More on Morphisms, Lemma
\ref{more-morphisms-lemma-deform} and more generally in
Deformation Theory, Lemma \ref{defos-lemma-deform-module}.
Thus we may apply
Deformation Theory, Lemma \ref{defos-lemma-choices}
to see that the set of isomorphism classes of such
deformations is equal to $\text{Ext}^1_P(\NL_{P/k}, P)$.
\end{proof}

\begin{lemma}
\label{lemma-smooth}
In Example \ref{example-rings} let $P$ be a smooth $k$-algebra. Then
$T\Deformationcategory_P = (0)$.
\end{lemma}

\begin{proof}
By Lemma \ref{lemma-rings-TI} we have to show
$\text{Ext}^1_P(\NL_{P/k}, P) = (0)$.
Since $k \to P$ is smooth $\NL_{P/k}$ is quasi-isomorphic to the
complex consisting of a finite projective
$P$-module placed in degree $0$.
\end{proof}

\begin{lemma}
\label{lemma-finite-type-rings-TI}
In Lemma \ref{lemma-rings-TI} if $P$ is a finite type $k$-algebra, then
\begin{enumerate}
\item $\text{Inf}(\Deformationcategory_P)$ is finite dimensional if and only if
$\dim(P) = 0$, and
\item $T\Deformationcategory_P$ is finite dimensional if
$\Spec(P) \to \Spec(k)$ is smooth except at a finite number of points.
\end{enumerate}
\end{lemma}

\begin{proof}
Proof of (1). We view $\text{Der}_k(P, P)$ as a $P$-module.
If it has finite dimension over $k$, then it has finite length
as a $P$-module, hence it is supported in finitely many
closed points of $\Spec(P)$
(Algebra, Lemma \ref{algebra-lemma-simple-pieces}).
Since $\text{Der}_k(P, P) = \Hom_P(\Omega_{P/k}, P)$
we see that
$\text{Der}_k(P, P)_\mathfrak p = \text{Der}_k(P_\mathfrak p, P_\mathfrak p)$
for any prime $\mathfrak p \subset P$
(this uses Algebra, Lemmas
\ref{algebra-lemma-differentials-localize},
\ref{algebra-lemma-differentials-finitely-presented}, and
\ref{algebra-lemma-hom-from-finitely-presented}).
Let $\mathfrak p$ be a minimal prime ideal of $P$
corresponding to an irreducible component of dimension $d > 0$.
Then $P_\mathfrak p$ is an Artinian local ring
essentially of finite type over $k$ with residue field
and $\Omega_{P_\mathfrak p/k}$ is nonzero for example by
Algebra, Lemma \ref{algebra-lemma-characterize-smooth-over-field}.
Any nonzero finite module over an Artinian local ring
has both a sub and a quotient module isomorphic to the residue field.
Thus we find that
$\text{Der}_k(P_\mathfrak p, P_\mathfrak p) =
\Hom_{P_\mathfrak p}(\Omega_{P_\mathfrak p/k}, P_\mathfrak p)$
is nonzero too. Combining all of the above we find that (1) is true.

\medskip\noindent
Proof of (2). For a prime $\mathfrak p$ of $P$ we will use that
$\NL_{P_\mathfrak p/k} = (\NL_{P/k})_\mathfrak p$
(Algebra, Lemma \ref{algebra-lemma-localize-NL})
and we will
use that
$\text{Ext}_P^1(\NL_{P/k}, P)_\mathfrak p =
\text{Ext}_{P_\mathfrak p}^1(\NL_{P_\mathfrak p/k}, P_\mathfrak p)$
(More on Algebra, Remark
\ref{more-algebra-remark-pseudo-coherence-and-base-change-ext}).
Given a prime $\mathfrak p \subset P$
then $k \to P$ is smooth at $\mathfrak p$ if and only if
$(\NL_{P/k})_\mathfrak p$ is quasi-isomorphic
to a finite projective module placed in degree $0$ (this follows
immediately from the definition of a smooth ring map but it also
follows from the stronger Algebra, Lemma \ref{algebra-lemma-smooth-at-point}).

\medskip\noindent
Assume that $P$ is smooth over $k$ at all but finitely many primes.
Then these ``bad'' primes are maximal ideals
$\mathfrak m_1, \ldots, \mathfrak m_n \subset P$ by
Algebra, Lemma \ref{algebra-lemma-finite-type-algebra-finite-nr-primes}
and the fact that the ``bad'' primes form a closed subset of $\Spec(P)$.
For $\mathfrak p \not \in \{\mathfrak m_1, \ldots, \mathfrak m_n\}$
we have $\text{Ext}^1_P(\NL_{P/k}, P)_\mathfrak p = 0$ by the results above.
Thus $\text{Ext}^1_P(\NL_{P/k}, P)$ is a finite $P$-module
whose support is contained in $\{\mathfrak m_1, \ldots, \mathfrak m_r\}$.
By Algebra, Proposition
\ref{algebra-proposition-minimal-primes-associated-primes}
for example, we find that the dimension over $k$ of
$\text{Ext}^1_P(\NL_{P/k}, P)$ is a finite integer combination
of $\dim_k \kappa(\mathfrak m_i)$ and hence finite by
the Hilbert Nullstellensatz
(Algebra, Theorem \ref{algebra-theorem-nullstellensatz}).
\end{proof}

\noindent
In Example \ref{example-rings}, let $P$ be a finite type
$k$-algebra. Then $\Deformationcategory_P$
admits a presentation by a smooth prorepresentable groupoid in functors
over $\mathcal{C}_\Lambda$ if and only if $\dim(P) = 0$.
Furthermore, $\Deformationcategory_P$ has a versal formal
object if $\Spec(P) \to \Spec(k)$ has finitely many
singular points. This follows from Lemmas \ref{lemma-rings-RS} and
\ref{lemma-finite-type-rings-TI}
and the general discussion in Section \ref{section-general}.

\begin{lemma}
\label{lemma-rings-hull}
In Example \ref{example-rings} assume $P$ is a finite type
$k$-algebra such that $\Spec(P) \to \Spec(k)$ is smooth except
at a finite number of points.
Assume $\Lambda$ is a complete local ring with residue field $k$
(the classical case). Then the functor
$$
F : \mathcal{C}_\Lambda \longrightarrow \textit{Sets},\quad
A \longmapsto \Ob(\Deformationcategory_P(A))/\cong
$$
of isomorphism classes of objects has a hull.
\end{lemma}

\begin{proof}
This follows immediately from Lemmas \ref{lemma-rings-RS} and
\ref{lemma-finite-type-rings-TI} and
Formal Deformation Theory, Lemma \ref{formal-defos-lemma-RS-implies-S1-S2}
and Remark \ref{formal-defos-remark-compose-minimal-into-iso-classes}.
\end{proof}

\begin{lemma}
\label{lemma-localization}
In Example \ref{example-rings} let $P$ be a $k$-algebra.
Let $S \subset P$ be a multiplicative subset. There is a natural functor
$$
\Deformationcategory_P \longrightarrow \Deformationcategory_{S^{-1}P}
$$
of deformation categories.
\end{lemma}

\begin{proof}
Given a deformation of $P$ we can take the localization
of it to get a deformation of the localization; this is
clear and we encourage the reader to skip the proof. More precisely,
let $(A, Q) \to (k, P)$ be a morphism in $\mathcal{F}$, i.e.,
an object of $\Deformationcategory_P$. Let $S_Q \subset Q$ be the
inverse image of $S$. Then
Hence $(A, S_Q^{-1}Q) \to (k, S^{-1}P)$
is the desired object of $\Deformationcategory_{S^{-1}P}$.
\end{proof}

\begin{lemma}
\label{lemma-henselization}
In Example \ref{example-rings} let $P$ be a $k$-algebra.
Let $J \subset P$ be an ideal.
Denote $(P^h, J^h)$ the henselization of the pair $(P, J)$.
There is a natural functor
$$
\Deformationcategory_P \longrightarrow \Deformationcategory_{P^h}
$$
of deformation categories.
\end{lemma}

\begin{proof}
Given a deformation of $P$ we can take the henselization
of it to get a deformation of the henselization; this is
clear and we encourage the reader to skip the proof. More precisely,
let $(A, Q) \to (k, P)$ be a morphism in $\mathcal{F}$, i.e.,
an object of $\Deformationcategory_P$. Denote $J_Q \subset Q$ the inverse
image of $J$ in $Q$. Let $(Q^h, J_Q^h)$ be the henselization of
the pair $(Q, J_Q)$. Recall that $Q \to Q^h$ is flat
(More on Algebra, Lemma \ref{more-algebra-lemma-henselization-flat})
and hence $Q^h$ is flat over $A$.
By More on Algebra, Lemma \ref{more-algebra-lemma-henselization-integral}
we see that the map $Q^h \to P^h$ induces an isomorphism
$Q^h \otimes_A k = Q^h \otimes_Q P = P^h$.
Hence $(A, Q^h) \to (k, P^h)$ is the desired object of
$\Deformationcategory_{P^h}$.
\end{proof}

\begin{lemma}
\label{lemma-strict-henselization}
In Example \ref{example-rings} let $P$ be a $k$-algebra.
Assume $P$ is a local ring and let $P^{sh}$ be a strict henselization of $P$.
There is a natural functor
$$
\Deformationcategory_P \longrightarrow \Deformationcategory_{P^{sh}}
$$
of deformation categories.
\end{lemma}

\begin{proof}
Given a deformation of $P$ we can take the strict henselization
of it to get a deformation of the strict henselization; this is
clear and we encourage the reader to skip the proof. More precisely,
let $(A, Q) \to (k, P)$ be a morphism in $\mathcal{F}$, i.e.,
an object of $\Deformationcategory_P$. Since the kernel of the surjection
$Q \to P$ is nilpotent, we find that $Q$ is a local ring with the
same residue field as $P$. Let $Q^{sh}$ be the strict henselization
of $Q$. Recall that $Q \to Q^{sh}$ is flat
(More on Algebra, Lemma \ref{more-algebra-lemma-dumb-properties-henselization})
and hence $Q^{sh}$ is flat over $A$.
By Algebra, Lemma \ref{algebra-lemma-quotient-strict-henselization}
we see that the map $Q^{sh} \to P^{sh}$ induces an isomorphism
$Q^{sh} \otimes_A k = Q^{sh} \otimes_Q P = P^{sh}$.
Hence $(A, Q^{sh}) \to (k, P^{sh})$ is the desired object of
$\Deformationcategory_{P^{sh}}$.
\end{proof}

\begin{lemma}
\label{lemma-completion}
In Example \ref{example-rings} let $P$ be a $k$-algebra.
Assume $P$ Noetherian and let $J \subset P$ be an ideal.
Denote $P^\wedge$ the $J$-adic completion.
There is a natural functor
$$
\Deformationcategory_P \longrightarrow \Deformationcategory_{P^\wedge}
$$
of deformation categories.
\end{lemma}

\begin{proof}
Given a deformation of $P$ we can take the completion
of it to get a deformation of the completion; this is
clear and we encourage the reader to skip the proof. More precisely,
let $(A, Q) \to (k, P)$ be a morphism in $\mathcal{F}$, i.e.,
an object of $\Deformationcategory_P$. Observe that $Q$ is a Noetherian
ring: the kernel of the surjective ring map $Q \to P$ is
nilpotent and finitely generated and $P$ is Noetherian; apply
Algebra, Lemma \ref{algebra-lemma-completion-Noetherian}.
Denote $J_Q \subset Q$ the inverse
image of $J$ in $Q$. Let $Q^\wedge$ be the $J_Q$-adic completion of $Q$.
Recall that $Q \to Q^\wedge$ is flat
(Algebra, Lemma \ref{algebra-lemma-completion-flat})
and hence $Q^\wedge$ is flat over $A$.
The induced map $Q^\wedge \to P^\wedge$ induces an isomorphism
$Q^\wedge \otimes_A k = Q^\wedge \otimes_Q P = P^\wedge$ by
Algebra, Lemma \ref{algebra-lemma-completion-tensor} for example.
Hence $(A, Q^\wedge) \to (k, P^\wedge)$
is the desired object of $\Deformationcategory_{P^\wedge}$.
\end{proof}

\begin{lemma}
\label{lemma-power-series-rings-TI}
In Lemma \ref{lemma-rings-TI} if $P = k[[x_1, \ldots, x_n]]/(f)$
for some nonzero $f \in (x_1, \ldots, x_n)^2$, then
\begin{enumerate}
\item $\text{Inf}(\Deformationcategory_P)$ is finite dimensional
if and only if $n = 1$, and
\item $T\Deformationcategory_P$ is finite dimensional if
$$
\sqrt{(f, \partial f/\partial x_1, \ldots,  \partial f/\partial x_n)} =
(x_1, \ldots, x_n)
$$
\end{enumerate}
\end{lemma}

\begin{proof}
Proof of (1). Consider the derivations $\partial/\partial x_i$ of
$k[[x_1, \ldots, x_n]]$ over $k$. Write $f_i = \partial f/\partial x_i$.
The derivation
$$
\theta = \sum h_i \partial/\partial x_i
$$
of $k[[x_1, \ldots, x_n]]$
induces a derivation of $P = k[[x_1, \ldots, x_n]]/(f)$
if and only if
$\sum h_i f_i \in (f)$. Moreover, the induced derivation of $P$
is zero if and only if $h_i \in (f)$ for $i = 1, \ldots, n$.
Thus we find
$$
\Ker((f_1, \ldots, f_n) : P^{\oplus n} \longrightarrow P) \subset
\text{Der}_k(P, P)
$$
The left hand side is a finite dimensional $k$-vector space only if
$n = 1$; we omit the proof. We also leave it to the reader to see
that the right hand side has finite dimension if $n = 1$.
This proves (1).

\medskip\noindent
Proof of (2). Let $Q$ be a flat deformation of $P$ over $k[\epsilon]$
as in the proof of Lemma \ref{lemma-rings-TI}. Choose lifts $q_i \in Q$
of the image of $x_i$ in $P$. Then $Q$ is a complete local ring
with maximal ideal generated by $q_1, \ldots, q_n$ and $\epsilon$
(small argument omitted). Thus we get a surjection
$$
k[\epsilon][[x_1, \ldots, x_n]] \longrightarrow Q,\quad
x_i \longmapsto q_i
$$
Choose an element of the form
$f + \epsilon g \in k[\epsilon][[x_1, \ldots, x_n]]$
mapping to zero in $Q$. Observe that $g$ is well defined modulo $(f)$.
Since $Q$ is flat over $k[\epsilon]$ we get
$$
Q = k[\epsilon][[x_1, \ldots, x_n]]/(f + \epsilon g)
$$
Finally, if we changing the choice of $q_i$ amounts to
changing the coordinates $x_i$ into $x_i + \epsilon h_i$
for some $h_i \in k[[x_1, \ldots, x_n]]$. Then
$f + \epsilon g$ changes into $f + \epsilon (g + \sum h_i f_i)$
where $f_i = \partial f/\partial x_i$. Thus we see that the
isomorphism class of the deformation $Q$ is determined
by an element of
$$
k[[x_1, \ldots, x_n]]/
(f, \partial f/\partial x_1, \ldots,  \partial f/\partial x_n)
$$
This has finite dimension over $k$ if and only if
its support is the closed point of $k[[x_1, \ldots, x_n]]$
if and only if
$\sqrt{(f, \partial f/\partial x_1, \ldots,  \partial f/\partial x_n)} =
(x_1, \ldots, x_n)$.
\end{proof}






\section{Schemes}
\label{section-schemes}

\noindent
The deformation theory of schemes.

\begin{example}[Schemes]
\label{example-schemes}
Let $\mathcal{F}$ be the category defined as follows
\begin{enumerate}
\item an object is a pair $(A, X)$ consisting of an
object $A$ of $\mathcal{C}_\Lambda$ and a scheme $X$ flat over $A$, and
\item a morphism $(f, g) : (B, Y) \to (A, X)$ consists of
a morphism $f : B \to A$ in $\mathcal{C}_\Lambda$ together
with a morphism $g : X \to Y$ such that
$$
\xymatrix{
X \ar[r]_g \ar[d] & Y \ar[d] \\
\Spec(A) \ar[r]^f & \Spec(B)
}
$$
is a cartesian commutative diagram of schemes.
\end{enumerate}
The functor $p : \mathcal{F} \to \mathcal{C}_\Lambda$ sends $(A, X)$
to $A$ and $(f, g)$ to $f$. It is clear that $p$ is cofibred in groupoids.
Given a scheme $X$ over $k$, let $x_0 = (k, X)$ be the corresponding object
of $\mathcal{F}(k)$. We set
$$
\Deformationcategory_X = \mathcal{F}_{x_0}
$$
\end{example}

\begin{lemma}
\label{lemma-schemes-RS}
Example \ref{example-schemes}
satisfies the Rim-Schlessinger condition (RS).
In particular, $\Deformationcategory_X$ is a deformation category
for any scheme $X$ over $k$.
\end{lemma}

\begin{proof}
Let $A_1 \to A$ and $A_2 \to A$ be morphisms of $\mathcal{C}_\Lambda$.
Assume $A_2 \to A$ is surjective. According to
Formal Deformation Theory, Lemma
\ref{formal-defos-lemma-RS-2-categorical}
it suffices to show that the functor
$\mathcal{F}(A_1 \times_A A_2) \to
\mathcal{F}(A_1) \times_{\mathcal{F}(A)} \mathcal{F}(A_2)$
is an equivalence of categories.
Observe that
$$
\xymatrix{
\Spec(A) \ar[r] \ar[d] & \Spec(A_2) \ar[d] \\
\Spec(A_1) \ar[r] &
\Spec(A_1 \times_A A_2)
}
$$
is a pushout diagram as in More on Morphisms, Lemma
\ref{more-morphisms-lemma-pushout-along-thickening}.
Thus the lemma is a special case of More on Morphisms, Lemma
\ref{more-morphisms-lemma-equivalence-categories-schemes-over-pushout-flat}.
\end{proof}

\begin{lemma}
\label{lemma-schemes-TI}
In Example \ref{example-schemes} let $X$ be a scheme over $k$. Then
$$
\text{Inf}(\Deformationcategory_X) =
\text{Ext}^0_{\mathcal{O}_X}(\NL_{X/k}, \mathcal{O}_X) =
\Hom_{\mathcal{O}_X}(\Omega_{X/k}, \mathcal{O}_X) =
\text{Der}_k(\mathcal{O}_X, \mathcal{O}_X)
$$
and
$$
T\Deformationcategory_X =
\text{Ext}^1_{\mathcal{O}_X}(\NL_{X/k}, \mathcal{O}_X)
$$
\end{lemma}

\begin{proof}
Recall that $\text{Inf}(\Deformationcategory_X)$ is the set of
automorphisms of the trivial deformation
$X' = X \times_{\Spec(k)} \Spec(k[\epsilon])$ of $X$ to $k[\epsilon]$
equal to the identity modulo $\epsilon$.
By Deformation Theory, Lemma \ref{defos-lemma-deform}
this is equal to $\text{Ext}^0_{\mathcal{O}_X}(\NL_{X/k}, \mathcal{O}_X)$.
The equality $\text{Ext}^0_{\mathcal{O}_X}(\NL_{X/k}, \mathcal{O}_X) =
\Hom_{\mathcal{O}_X}(\Omega_{X/k}, \mathcal{O}_X)$ follows from
More on Morphisms, Lemma
\ref{more-morphisms-lemma-netherlander-quasi-coherent}.
The equality
$\Hom_{\mathcal{O}_X}(\Omega_{X/k}, \mathcal{O}_X) =
\text{Der}_k(\mathcal{O}_X, \mathcal{O}_X)$
follows from Morphisms, Lemma
\ref{morphisms-lemma-universal-derivation-universal}.

\medskip\noindent
Recall that $T_{x_0}\Deformationcategory_X$ is the set of isomorphism classes
of flat deformations $X'$ of $X$ to $k[\epsilon]$, more precisely,
the set of isomorphism classes of $\Deformationcategory_X(k[\epsilon])$.
Thus the second statement of the lemma follows from
Deformation Theory, Lemma \ref{defos-lemma-deform}.
\end{proof}

\begin{lemma}
\label{lemma-proper-schemes-TI}
In Lemma \ref{lemma-schemes-TI} if $X$ is proper over $k$, then
$\text{Inf}(\Deformationcategory_X)$ and $T\Deformationcategory_X$ are
finite dimensional.
\end{lemma}

\begin{proof}
By the lemma we have to show
$\Ext^1_{\mathcal{O}_X}(\NL_{X/k}, \mathcal{O}_X)$ and
$\Ext^0_{\mathcal{O}_X}(\NL_{X/k}, \mathcal{O}_X)$ are finite
dimensional. By More on Morphisms, Lemma
\ref{more-morphisms-lemma-netherlander-fp}
and the fact that $X$ is Noetherian, we see that
$\NL_{X/k}$ has coherent cohomology sheaves zero except
in degrees $0$ and $-1$.
By Derived Categories of Schemes, Lemma \ref{perfect-lemma-ext-finite}
the displayed $\Ext$-groups are finite $k$-vector spaces
and the proof is complete.
\end{proof}

\noindent
In Example \ref{example-schemes} if $X$ is a proper scheme over $k$,
then $\Deformationcategory_X$
admits a presentation by a smooth prorepresentable groupoid in functors
over $\mathcal{C}_\Lambda$
and a fortiori has a (minimal) versal formal object. This follows
from Lemmas \ref{lemma-schemes-RS} and
\ref{lemma-proper-schemes-TI}
and the general discussion in Section \ref{section-general}.

\begin{lemma}
\label{lemma-schemes-hull}
In Example \ref{example-schemes} assume $X$ is a proper $k$-scheme.
Assume $\Lambda$ is a complete local ring with residue field $k$
(the classical case). Then the functor
$$
F : \mathcal{C}_\Lambda \longrightarrow \textit{Sets},\quad
A \longmapsto \Ob(\Deformationcategory_X(A))/\cong
$$
of isomorphism classes of objects has a hull. If
$\text{Der}_k(\mathcal{O}_X, \mathcal{O}_X) = 0$, then
$F$ is prorepresentable.
\end{lemma}

\begin{proof}
The existence of a hull follows immediately from
Lemmas \ref{lemma-schemes-RS} and \ref{lemma-proper-schemes-TI} and
Formal Deformation Theory, Lemma \ref{formal-defos-lemma-RS-implies-S1-S2}
and Remark \ref{formal-defos-remark-compose-minimal-into-iso-classes}.

\medskip\noindent
Assume $\text{Der}_k(\mathcal{O}_X, \mathcal{O}_X) = 0$. Then
$\Deformationcategory_X$ and $F$ are equivalent by
Formal Deformation Theory, Lemma \ref{formal-defos-lemma-infdef-trivial}.
Hence $F$ is a deformation functor (because $\Deformationcategory_X$ is a
deformation category) with finite tangent space and we can apply
Formal Deformation Theory, Theorem
\ref{formal-defos-theorem-Schlessinger-prorepresentability}.
\end{proof}

\begin{lemma}
\label{lemma-open}
In Example \ref{example-schemes} let $X$ be a scheme over $k$.
Let $U \subset X$ be an open subscheme.
There is a natural functor
$$
\Deformationcategory_X \longrightarrow \Deformationcategory_U
$$
of deformation categories.
\end{lemma}

\begin{proof}
Given a deformation of $X$ we can take the corresponding open
of it to get a deformation of $U$. We omit the details.
\end{proof}

\begin{lemma}
\label{lemma-affine}
In Example \ref{example-schemes} let $X = \Spec(P)$ be an
affine scheme over $k$. With $\Deformationcategory_P$ as in
Example \ref{example-rings} there is a natural equivalence
$$
\Deformationcategory_X \longrightarrow \Deformationcategory_P
$$
of deformation categories.
\end{lemma}

\begin{proof}
The functor sends $(A, Y)$ to $\Gamma(Y, \mathcal{O}_Y)$.
This works because
any deformation of $X$ is affine by
More on Morphisms, Lemma \ref{more-morphisms-lemma-thickening-affine-scheme}.
\end{proof}

\begin{lemma}
\label{lemma-local-ring}
In Example \ref{example-schemes} let $X$ be a scheme over $k$
Let $p \in X$ be a point. With $\Deformationcategory_{\mathcal{O}_{X, p}}$
as in Example \ref{example-rings} there is a natural functor
$$
\Deformationcategory_X
\longrightarrow
\Deformationcategory_{\mathcal{O}_{X, p}}
$$
of deformation categories.
\end{lemma}

\begin{proof}
Choose an affine open $U = \Spec(P) \subset X$ containing $p$.
Then $\mathcal{O}_{X, p}$ is a localization of $P$. We combine
the functors from
Lemmas \ref{lemma-open}, \ref{lemma-affine}, and \ref{lemma-localization}.
\end{proof}

\begin{situation}
\label{situation-glueing}
Let $\Lambda \to k$ be as in Section \ref{section-general}.
Let $X$ be a scheme over $k$ which has an affine open covering
$X = U_1 \cup U_2$ with $U_{12} = U_1 \cap U_2$ affine too.
Write $U_1 = \Spec(P_1)$, $U_2 = \Spec(P_2)$ and $U_{12} = \Spec(P_{12})$.
Let $\Deformationcategory_X$, $\Deformationcategory_{U_1}$,
$\Deformationcategory_{U_2}$, and $\Deformationcategory_{U_{12}}$
be as in Example \ref{example-schemes} and let
$\Deformationcategory_{P_1}$, $\Deformationcategory_{P_2}$, and
$\Deformationcategory_{P_{12}}$ be as in Example \ref{example-rings}.
\end{situation}

\begin{lemma}
\label{lemma-glueing}
In Situation \ref{situation-glueing}
there is an equivalence
$$
\Deformationcategory_X =
\Deformationcategory_{P_1}
\times_{\Deformationcategory_{P_{12}}}
\Deformationcategory_{P_2}
$$
of deformation categories, see Examples \ref{example-schemes} and
\ref{example-rings}.
\end{lemma}

\begin{proof}
It suffices to show that the functors of Lemma \ref{lemma-open}
define an equivalence
$$
\Deformationcategory_X \longrightarrow
\Deformationcategory_{U_1}
\times_{\Deformationcategory_{U_{12}}}
\Deformationcategory_{U_2}
$$
because then we can apply Lemma \ref{lemma-affine} to translate into rings.
To do this we construct a quasi-inverse. Denote
$F_i : \Deformationcategory_{U_i} \to \Deformationcategory_{U_{12}}$
the functor of Lemma \ref{lemma-open}.
An object of the RHS is given by an $A$ in $\mathcal{C}_\Lambda$,
objects $(A, V_1) \to (k, U_1)$ and $(A, V_2) \to (k, U_2)$, and
a morphism
$$
g : F_1(A, V_1) \to F_2(A, V_2)
$$
Now $F_i(A, V_i) = (A, V_{i, 3 - i})$ where $V_{i, 3 - i} \subset V_i$
is the open subscheme whose base change to $k$ is $U_{12} \subset U_i$.
The morphism $g$ defines an isomorphism
$V_{1, 2} \to V_{2, 1}$ of schemes over $A$ compatible
with $\text{id} : U_{12} \to U_{12}$ over $k$.
Thus $(\{1, 2\}, V_i, V_{i, 3 - i}, g, g^{-1})$ is a glueing
data as in Schemes, Section \ref{schemes-section-glueing-schemes}.
Let $Y$ be the glueing, see Schemes, Lemma \ref{schemes-lemma-glue}.
Then $Y$ is a scheme over $A$ and the
compatibilities mentioned above show that
there is a canonical isomorphism
$Y \times_{\Spec(A)} \Spec(k) = X$.
Thus $(A, Y) \to (k, X)$ is an object of $\Deformationcategory_X$.
We omit the verification that this construction is a functor
and is quasi-inverse to the given one.
\end{proof}






\section{Morphisms of Schemes}
\label{section-schemes-morphisms}

\noindent
The deformation theory of morphisms of schemes.
Of course this is just an example of
deformations of diagrams of schemes.

\begin{example}[Morphisms of schemes]
\label{example-schemes-morphisms}
Let $\mathcal{F}$ be the category defined as follows
\begin{enumerate}
\item an object is a pair $(A, X \to Y)$ consisting of an
object $A$ of $\mathcal{C}_\Lambda$ and a morphism
$X \to Y$ of schemes over $A$ with both $X$ and $Y$ flat over $A$, and
\item a morphism $(f, g, h) : (A', X' \to Y') \to (A, X \to Y)$ consists of
a morphism $f : A' \to A$ in $\mathcal{C}_\Lambda$ together
with morphisms of schemes $g : X \to X'$ and $h : Y \to Y'$ such that
$$
\xymatrix{
X \ar[r]_g \ar[d] & X' \ar[d] \\
Y \ar[r]_h \ar[d] & Y' \ar[d] \\
\Spec(A) \ar[r]^f & \Spec(A')
}
$$
is a commutative diagram of schemes where both squares are cartesian.
\end{enumerate}
The functor $p : \mathcal{F} \to \mathcal{C}_\Lambda$ sends $(A, X \to Y)$
to $A$ and $(f, g, h)$ to $f$. It is clear that $p$ is cofibred in groupoids.
Given a morphism of schemes $X \to Y$ over $k$, let $x_0 = (k, X \to Y)$
be the corresponding object of $\mathcal{F}(k)$. We set
$$
\Deformationcategory_{X \to Y} = \mathcal{F}_{x_0}
$$
\end{example}

\begin{lemma}
\label{lemma-schemes-morphisms-RS}
Example \ref{example-schemes-morphisms}
satisfies the Rim-Schlessinger condition (RS).
In particular, $\Deformationcategory_{X \to Y}$ is a deformation category
for any morphism of schemes $X \to Y$ over $k$.
\end{lemma}

\begin{proof}
Let $A_1 \to A$ and $A_2 \to A$ be morphisms of $\mathcal{C}_\Lambda$.
Assume $A_2 \to A$ is surjective. According to
Formal Deformation Theory, Lemma
\ref{formal-defos-lemma-RS-2-categorical}
it suffices to show that the functor
$\mathcal{F}(A_1 \times_A A_2) \to
\mathcal{F}(A_1) \times_{\mathcal{F}(A)} \mathcal{F}(A_2)$
is an equivalence of categories.
Observe that
$$
\xymatrix{
\Spec(A) \ar[r] \ar[d] & \Spec(A_2) \ar[d] \\
\Spec(A_1) \ar[r] &
\Spec(A_1 \times_A A_2)
}
$$
is a pushout diagram as in More on Morphisms, Lemma
\ref{more-morphisms-lemma-pushout-along-thickening}.
Thus the lemma follows immediately from
More on Morphisms, Lemma
\ref{more-morphisms-lemma-equivalence-categories-schemes-over-pushout-flat}
as this describes the category of schemes flat over $A_1 \times_A A_2$
as the fibre product of the category of schemes flat over $A_1$
with the category of schemes flat over $A_2$ over the category of
schemes flat over $A$.
\end{proof}

\begin{lemma}
\label{lemma-schemes-morphisms-TI}
In Example \ref{example-schemes} let $X \to Y$ be a morphism of schemes
over $k$. There is a canonical exact sequence of $k$-vector spaces
$$
\xymatrix{
0 \ar[r] &
\text{Inf}(\Deformationcategory_{X \to Y}) \ar[r] &
\text{Inf}(\Deformationcategory_X \times \Deformationcategory_Y) \ar[r] &
\text{Der}_k(\mathcal{O}_Y, f_*\mathcal{O}_X) \ar[lld] \\
& T\Deformationcategory_{X \to Y} \ar[r] &
T(\Deformationcategory_X \times \Deformationcategory_Y) \ar[r] &
\text{Ext}^1_{\mathcal{O}_X}(Lf^*\NL_{Y/k}, \mathcal{O}_X)
}
$$
\end{lemma}

\begin{proof}
The obvious map of deformation categories
$\Deformationcategory_{X \to Y} \to
\Deformationcategory_X \times \Deformationcategory_Y$
gives two of the arrows in the exact sequence of the lemma.
Denote $f : X \to Y$ the given morphism.
Recall that $\text{Inf}(\Deformationcategory_{X \to Y})$
is the set of automorphisms of the trivial deformation
$$
f' :  X' = X \times_{\Spec(k)} \Spec(k[\epsilon])
\xrightarrow{f \times \text{id}}
Y' = Y \times_{\Spec(k)} \Spec(k[\epsilon])
$$
of $X \to Y$ to $k[\epsilon]$ equal to the identity modulo $\epsilon$.
This is clearly the same thing as pairs
$(\alpha, \beta) \in
\text{Inf}(\Deformationcategory_X \times \Deformationcategory_Y)$
of infinitesimal automorphisms of $X$ and $Y$ compatible with $f'$, i.e.,
such that $f' \circ \alpha = \beta \circ f'$.
By Deformation Theory, Lemma \ref{defos-lemma-huge-diagram-ringed-spaces}
for an arbitrary pair $(\alpha, \beta)$ the difference between
the morphism $f' : X' \to Y'$ and the morphism
$\beta^{-1} \circ f' \circ \alpha : X' \to Y'$ defines an elment
in
$$
\text{Der}_k(\mathcal{O}_Y, f_*\mathcal{O}_X) =
\Hom_{\mathcal{O}_Y}(\Omega_{Y/k}, f_*\mathcal{O}_X)
$$
Equality by More on Morphisms, Lemma
\ref{more-morphisms-lemma-netherlander-quasi-coherent}.
This defines the last top horizontal arrow and shows exactness
in the first two places. For the map
$$
\text{Der}_k(\mathcal{O}_Y, f_*\mathcal{O}_X)
\to
T\Deformationcategory_{X \to Y}
$$
we interpret elements of the source as morphisms
$f_\epsilon : X' \to Y'$ over $\Spec(k[\epsilon])$
equal to $f$ modulo $\epsilon$
using Deformation Theory, Lemma \ref{defos-lemma-huge-diagram-ringed-spaces}.
We send $f_\epsilon$ to the isomorphism class of
$(f_\epsilon : X' \to Y')$ in $T\Deformationcategory_{X \to Y}$.
Note that $(f_\epsilon : X' \to Y')$ is isomorphic to the
trivial deformation $(f' : X' \to Y')$ exactly when
$f_\epsilon  = \beta^{-1} \circ f \circ \alpha$ for some
pair $(\alpha, \beta)$ which implies exactness in the third spot.
Clearly, if some first order deformation
$(f_\epsilon : X_\epsilon \to Y_\epsilon)$
maps to zero in $T(\Deformationcategory_X \times \Deformationcategory_Y)$,
then we can choose isomorphisms $X' \to X_\epsilon$ and $Y' \to Y_\epsilon$
and we conclude we are in the image of the south-west arrow.
Therefore we have exactness at the fourth spot.
Finally, given two first order deformations $X_\epsilon$, $Y_\epsilon$
of $X$, $Y$ there is an obstruction in
$$
ob(X_\epsilon, Y_\epsilon) \in
\text{Ext}^1_{\mathcal{O}_X}(Lf^*\NL_{Y/k}, \mathcal{O}_X)
$$
which vanishes if and only if $f : X \to Y$ lifts to
$X_\epsilon \to Y_\epsilon$, see
Deformation Theory, Lemma \ref{defos-lemma-huge-diagram-ringed-spaces}.
This finishes the proof.
\end{proof}

\begin{lemma}
\label{lemma-proper-schemes-morphisms-TI}
In Lemma \ref{lemma-schemes-morphisms-TI} if $X$ and $Y$ are both
proper over $k$, then
$\text{Inf}(\Deformationcategory_{X \to Y})$ and
$T\Deformationcategory_{X \to Y}$ are finite dimensional.
\end{lemma}

\begin{proof}
Omitted. Hint: argue as in Lemma \ref{lemma-proper-schemes-TI}
and use the exact sequence of the lemma.
\end{proof}

\noindent
In Example \ref{example-schemes-morphisms}
if $X \to Y$ is a morphism of proper schemes over $k$,
then $\Deformationcategory_{X \to Y}$
admits a presentation by a smooth prorepresentable groupoid in functors
over $\mathcal{C}_\Lambda$
and a fortiori has a (minimal) versal formal object. This follows
from Lemmas \ref{lemma-schemes-morphisms-RS} and
\ref{lemma-proper-schemes-morphisms-TI}
and the general discussion in Section \ref{section-general}.

\begin{lemma}
\label{lemma-schemes-morphisms-hull}
In Example \ref{example-schemes-morphisms} assume $X \to Y$
is a morphism of proper $k$-schemes.
Assume $\Lambda$ is a complete local ring with residue field $k$
(the classical case). Then the functor
$$
F : \mathcal{C}_\Lambda \longrightarrow \textit{Sets},\quad
A \longmapsto \Ob(\Deformationcategory_{X \to Y}(A))/\cong
$$
of isomorphism classes of objects has a hull. If
$\text{Der}_k(\mathcal{O}_X, \mathcal{O}_X) =
\text{Der}_k(\mathcal{O}_Y, \mathcal{O}_Y) = 0$, then
$F$ is prorepresentable.
\end{lemma}

\begin{proof}
The existence of a hull follows immediately from
Lemmas \ref{lemma-schemes-morphisms-RS} and
\ref{lemma-proper-schemes-morphisms-TI} and
Formal Deformation Theory, Lemma \ref{formal-defos-lemma-RS-implies-S1-S2}
and Remark \ref{formal-defos-remark-compose-minimal-into-iso-classes}.

\medskip\noindent
Assume $\text{Der}_k(\mathcal{O}_X, \mathcal{O}_X) =
\text{Der}_k(\mathcal{O}_Y, \mathcal{O}_Y) = 0$. Then
the exact sequence of Lemma \ref{lemma-schemes-morphisms-TI}
combined with Lemma \ref{lemma-schemes-TI}
shows that $\text{Inf}(\Deformationcategory_{X \to Y}) = 0$.
Then $\Deformationcategory_{X \to Y}$ and $F$ are equivalent by
Formal Deformation Theory, Lemma \ref{formal-defos-lemma-infdef-trivial}.
Hence $F$ is a deformation functor (because
$\Deformationcategory_{X \to Y}$ is a
deformation category) with finite tangent space and we can apply
Formal Deformation Theory, Theorem
\ref{formal-defos-theorem-Schlessinger-prorepresentability}.
\end{proof}

\begin{lemma}
\label{lemma-schemes-morphisms-smooth-to-base}
\begin{reference}
This is discussed in \cite[Section 5.3]{Ravi-Murphys-Law} and
\cite[Theorem 3.3]{Ran-deformations}.
\end{reference}
In Example \ref{example-schemes} let $f : X \to Y$ be a morphism of schemes
over $k$. If $f_*\mathcal{O}_X = \mathcal{O}_Y$ and $R^1f_*\mathcal{O}_X = 0$,
then the morphism of deformation categories
$$
\Deformationcategory_{X \to Y} \to \Deformationcategory_X
$$
is an equivalence.
\end{lemma}

\begin{proof}
We construct a quasi-inverse to the forgetful functor of the lemma.
Namely, suppose that $(A, U)$ is an object of $\Deformationcategory_X$.
The given map $X \to U$ is a finite order thickening and we can use
it to identify the underlying topological spaces of $U$ and $X$, see
More on Morphisms, Section \ref{more-morphisms-section-thickenings}.
Thus we may and do think of $\mathcal{O}_U$ as a sheaf of
$A$-algebras on $X$; moreover the fact that $U \to \Spec(A)$ is
flat, means that $\mathcal{O}_U$ is flat as a sheaf of $A$-modules.
In particular, we have a filtration
$$
0 = \mathfrak m_A^n\mathcal{O}_U \subset
\mathfrak m_A^{n - 1}\mathcal{O}_U \subset \ldots \subset
\mathfrak m_A^2\mathcal{O}_U \subset
\mathfrak m_A\mathcal{O}_U \subset \mathcal{O}_U
$$
with subquotients equal to
$\mathcal{O}_X \otimes_k \mathfrak m_A^i/\mathfrak m_A^{i + 1}$
by flatness, see More on Morphisms, Lemma \ref{more-morphisms-lemma-deform}
or the more general Deformation Theory, Lemma \ref{defos-lemma-deform-module}.
Set
$$
\mathcal{O}_V = f_*\mathcal{O}_U
$$
viewed as sheaf of $A$-algebras on $Y$. Since
$R^1f_*\mathcal{O}_X = 0$ we find by the description above that
$R^1f_*(\mathfrak m_A^i\mathcal{O}_U/\mathfrak m_A^{i + 1}\mathcal{O}_U) = 0$
for all $i$. This implies that the sequences
$$
0 \to
(f_*\mathcal{O}_X) \otimes_k \mathfrak m_A^i/\mathfrak m_A^{i + 1} \to
f_*(\mathcal{O}_U/\mathfrak m_A^{i + 1}\mathcal{O}_U) \to
f_*(\mathcal{O}_U/\mathfrak m_A^i\mathcal{O}_U) \to 0
$$
are exact for all $i$. Reading the references given above backwards
(and using induction) we find that $\mathcal{O}_V$ is a flat
sheaf of $A$-algebras with
$\mathcal{O}_V/\mathfrak m_A\mathcal{O}_V = \mathcal{O}_Y$.
Using More on Morphisms, Lemma
\ref{more-morphisms-lemma-first-order-thickening}
we find that $(Y, \mathcal{O}_V)$ is a scheme, call it $V$.
The equality $\mathcal{O}_V = f_*\mathcal{O}_U$ defines a
morphism of ringed spaces $U \to V$ which is easily seen to be
a morphism of schemes. This finishes the proof by the
flatness already esthablished.
\end{proof}







\section{Algebraic spaces}
\label{section-algebraic-spaces}

\noindent
The deformation theory of algebraic spaces.

\begin{example}[Algebraic spaces]
\label{example-spaces}
Let $\mathcal{F}$ be the category defined as follows
\begin{enumerate}
\item an object is a pair $(A, X)$ consisting of an
object $A$ of $\mathcal{C}_\Lambda$ and an algebraic space
$X$ flat over $A$, and
\item a morphism $(f, g) : (B, Y) \to (A, X)$ consists of
a morphism $f : B \to A$ in $\mathcal{C}_\Lambda$ together
with a morphism $g : X \to Y$ of algebraic spaces over $\Lambda$
such that
$$
\xymatrix{
X \ar[r]_g \ar[d] & Y \ar[d] \\
\Spec(A) \ar[r]^f & \Spec(B)
}
$$
is a cartesian commutative diagram of algebraic spaces.
\end{enumerate}
The functor $p : \mathcal{F} \to \mathcal{C}_\Lambda$ sends $(A, X)$
to $A$ and $(f, g)$ to $f$. It is clear that $p$ is cofibred in groupoids.
Given an algebraic space $X$ over $k$, let
$x_0 = (k, X)$ be the corresponding object of $\mathcal{F}(k)$. We set
$$
\Deformationcategory_X = \mathcal{F}_{x_0}
$$
\end{example}

\begin{lemma}
\label{lemma-spaces-RS}
Example \ref{example-spaces}
satisfies the Rim-Schlessinger condition (RS).
In particular, $\Deformationcategory_X$ is a deformation category
for any algebraic space $X$ over $k$.
\end{lemma}

\begin{proof}
Let $A_1 \to A$ and $A_2 \to A$ be morphisms of $\mathcal{C}_\Lambda$.
Assume $A_2 \to A$ is surjective. According to
Formal Deformation Theory, Lemma
\ref{formal-defos-lemma-RS-2-categorical}
it suffices to show that the functor
$\mathcal{F}(A_1 \times_A A_2) \to
\mathcal{F}(A_1) \times_{\mathcal{F}(A)} \mathcal{F}(A_2)$
is an equivalence of categories.
Observe that
$$
\xymatrix{
\Spec(A) \ar[r] \ar[d] & \Spec(A_2) \ar[d] \\
\Spec(A_1) \ar[r] &
\Spec(A_1 \times_A A_2)
}
$$
is a pushout diagram as in Pushouts of Spaces, Lemma
\ref{spaces-pushouts-lemma-pushout-along-thickening}.
Thus the lemma is a special case of Pushouts of Spaces, Lemma
\ref{spaces-pushouts-lemma-equivalence-categories-spaces-pushout-flat}.
\end{proof}

\begin{lemma}
\label{lemma-spaces-TI}
In Example \ref{example-spaces} let $X$ be an algebraic space over $k$. Then
$$
\text{Inf}(\Deformationcategory_X) =
\text{Ext}^0_{\mathcal{O}_X}(\NL_{X/k}, \mathcal{O}_X) =
\Hom_{\mathcal{O}_X}(\Omega_{X/k}, \mathcal{O}_X) =
\text{Der}_k(\mathcal{O}_X, \mathcal{O}_X)
$$
and
$$
T\Deformationcategory_X =
\text{Ext}^1_{\mathcal{O}_X}(\NL_{X/k}, \mathcal{O}_X)
$$
\end{lemma}

\begin{proof}
Recall that $\text{Inf}(\Deformationcategory_X)$ is the set of
automorphisms of the trivial deformation
$X' = X \times_{\Spec(k)} \Spec(k[\epsilon])$ of $X$ to $k[\epsilon]$
equal to the identity modulo $\epsilon$.
By Deformation Theory, Lemma \ref{defos-lemma-deform-spaces}
this is equal to $\text{Ext}^0_{\mathcal{O}_X}(\NL_{X/k}, \mathcal{O}_X)$.
The equality $\text{Ext}^0_{\mathcal{O}_X}(\NL_{X/k}, \mathcal{O}_X) =
\Hom_{\mathcal{O}_X}(\Omega_{X/k}, \mathcal{O}_X)$ follows from
More on Morphisms of Spaces, Lemma
\ref{spaces-more-morphisms-lemma-netherlander-quasi-coherent}.
The equality
$\Hom_{\mathcal{O}_X}(\Omega_{X/k}, \mathcal{O}_X) =
\text{Der}_k(\mathcal{O}_X, \mathcal{O}_X)$
follows from More on Morphisms of Spaces, Definition
\ref{spaces-more-morphisms-definition-sheaf-differentials} and
Modules on Sites, Definition
\ref{sites-modules-definition-module-differentials}.

\medskip\noindent
Recall that $T_{x_0}\Deformationcategory_X$ is the set of isomorphism classes
of flat deformations $X'$ of $X$ to $k[\epsilon]$, more precisely,
the set of isomorphism classes of $\Deformationcategory_X(k[\epsilon])$.
Thus the second statement of the lemma follows from
Deformation Theory, Lemma \ref{defos-lemma-deform-spaces}.
\end{proof}

\begin{lemma}
\label{lemma-proper-spaces-TI}
In Lemma \ref{lemma-spaces-TI} if $X$ is proper over $k$, then
$\text{Inf}(\Deformationcategory_X)$ and $T\Deformationcategory_X$ are
finite dimensional.
\end{lemma}

\begin{proof}
By the lemma we have to show
$\Ext^1_{\mathcal{O}_X}(\NL_{X/k}, \mathcal{O}_X)$ and
$\Ext^0_{\mathcal{O}_X}(\NL_{X/k}, \mathcal{O}_X)$ are finite
dimensional. By More on Morphisms of Spaces, Lemma
\ref{spaces-more-morphisms-lemma-netherlander-fp}
and the fact that $X$ is Noetherian, we see that
$\NL_{X/k}$ has coherent cohomology sheaves zero except
in degrees $0$ and $-1$.
By Derived Categories of Spaces, Lemma \ref{spaces-perfect-lemma-ext-finite}
the displayed $\Ext$-groups are finite $k$-vector spaces
and the proof is complete.
\end{proof}

\noindent
In Example \ref{example-spaces} if $X$ is a proper algebraic space over $k$,
then $\Deformationcategory_X$
admits a presentation by a smooth prorepresentable groupoid in functors
over $\mathcal{C}_\Lambda$
and a fortiori has a (minimal) versal formal object. This follows
from Lemmas \ref{lemma-spaces-RS} and
\ref{lemma-proper-spaces-TI}
and the general discussion in Section \ref{section-general}.

\begin{lemma}
\label{lemma-spaces-hull}
In Example \ref{example-spaces} assume $X$ is a proper algebraic space over $k$.
Assume $\Lambda$ is a complete local ring with residue field $k$
(the classical case). Then the functor
$$
F : \mathcal{C}_\Lambda \longrightarrow \textit{Sets},\quad
A \longmapsto \Ob(\Deformationcategory_X(A))/\cong
$$
of isomorphism classes of objects has a hull. If
$\text{Der}_k(\mathcal{O}_X, \mathcal{O}_X) = 0$, then
$F$ is prorepresentable.
\end{lemma}

\begin{proof}
The existence of a hull follows immediately from
Lemmas \ref{lemma-spaces-RS} and \ref{lemma-proper-spaces-TI} and
Formal Deformation Theory, Lemma \ref{formal-defos-lemma-RS-implies-S1-S2}
and Remark \ref{formal-defos-remark-compose-minimal-into-iso-classes}.

\medskip\noindent
Assume $\text{Der}_k(\mathcal{O}_X, \mathcal{O}_X) = 0$. Then
$\Deformationcategory_X$ and $F$ are equivalent by
Formal Deformation Theory, Lemma \ref{formal-defos-lemma-infdef-trivial}.
Hence $F$ is a deformation functor (because $\Deformationcategory_X$ is a
deformation category) with finite tangent space and we can apply
Formal Deformation Theory, Theorem
\ref{formal-defos-theorem-Schlessinger-prorepresentability}.
\end{proof}





\section{Deformations of completions}
\label{section-compare}

\noindent
In this section we compare the deformation problem posed
by an algebra and its completion.
We first discuss ``liftability''.

\begin{lemma}
\label{lemma-lift-equivalence-module-derived}
Let $A' \to A$ be a surjection of rings with nilpotent kernel.
Let $A' \to P'$ be a flat ring map.
Set $P = P' \otimes_{A'} A$.
Let $M$ be an $A$-flat $P$-module.
Then the following are equivalent
\begin{enumerate}
\item there is an $A'$-flat $P'$-module $M'$ with
$M' \otimes_{P'} P = M$, and
\item there is an object $K' \in D^-(P')$ with
$K' \otimes_{P'}^\mathbf{L} P = M$.
\end{enumerate}
\end{lemma}

\begin{proof}
Suppose that $M'$ is as in (1). Then
$$
M = M' \otimes_P P' = M' \otimes_{A'} A =
M' \otimes_A^\mathbf{L} A' = M' \otimes_{P'}^\mathbf{L} P
$$
The first two equalities are clear, the third holds because
$M'$ is flat over $A'$, and the fourth holds by
More on Algebra, Lemma \ref{more-algebra-lemma-base-change-comparison}.
Thus (2) holds. Conversely, suppose $K'$ is as in (2).
We may and do assume $M$ is nonzero.
Let $t$ be the largest integer such that $H^t(K')$ is nonzero
(exists because $M$ is nonzero).
Then $H^t(K') \otimes_{P'} P = H^t(K' \otimes_{P'}^\mathbf{L} P)$
is zero if $t > 0$. Since the kernel of $P' \to P$ is nilpotent
this implies $H^t(K') = 0$ by Nakayama's lemma a contradiction.
Hence $t = 0$ (the case $t < 0$ is absurd as well).
Then $M' = H^0(K')$ is a $P'$-module such that $M = M' \otimes_{P'} P$
and the spectral sequence for Tor gives an injective map
$$
\text{Tor}_1^{P'}(M', P) \to H^{-1}(M' \otimes_{P'}^\mathbf{L} P) = 0
$$
By the reference on derived base change above
$0 = \text{Tor}_1^{P'}(M', P) = \text{Tor}_1^{A'}(M', A)$.
We conclude that $M'$ is $A'$-flat by
Algebra, Lemma \ref{algebra-lemma-what-does-it-mean}.
\end{proof}

\begin{lemma}
\label{lemma-lift-equivalence-module}
Consider a commutative diagram of Noetherian rings
$$
\xymatrix{
A' \ar[d] \ar[r] &
P' \ar[d] \ar[r] &
Q' \ar[d] \\
A \ar[r] &
P \ar[r] &
Q
}
$$
with cartesian squares, with flat horizontal arrows, and with
surjective vertial arrows whose kernels are nilpotent.
Let $J' \subset P'$ be an ideal such that $P'/J' = Q'/J'Q'$.
Let $M$ be an $A$-flat $P$-module.
Assume for all $g \in J'$ there exists an $A'$-flat $(P')_g$-module
lifting $M_g$. Then the following are equivalent
\begin{enumerate}
\item $M$ has an $A'$-flat lift to a $P'$-module, and
\item $M \otimes_P Q$ has an $A'$-flat lift to a $Q'$-module.
\end{enumerate}
\end{lemma}

\begin{proof}
Let $I = \Ker(A' \to A)$. By induction on the integer $n > 1$
such that $I^n = 0$ we reduce to the case where $I$ is an ideal
of square zero; details omitted.
We translate the condition of liftability of
$M$ into the problem of finding an object of $D^-(P')$ as in
Lemma \ref{lemma-lift-equivalence-module-derived}.
The obstruction to doing this is the element
$$
\omega(M) \in \text{Ext}^2_P(M, M \otimes_P^\mathbf{L} IP) =
\text{Ext}^2_P(M, M \otimes_P IP)
$$
constructed in
Deformation Theory, Lemma \ref{defos-lemma-canonical-class-algebra}.
The equality in the displayed formula holds as
$M \otimes_P^\mathbf{L} IP = M \otimes_P IP$
since $M$ and $P$ are $A$-flat\footnote{Choose a resolution
$F_\bullet \to I$ by free $A$-modules. Since $A \to P$ is flat,
$P \otimes_A F_\bullet$ is a free resolution of $IP$.
Hence $M \otimes_P^\mathbf{L} IP$ is represented by
$M \otimes_P P \otimes_A F_\bullet = M \otimes_A F_\bullet$.
This only has cohomology in degree $0$ as $M$ is $A$-flat.}.
The obstruction for lifting $M \otimes_P Q$ is similarly
the element
$$
\omega(M \otimes_P Q) \in
\text{Ext}^2_Q(M \otimes_P Q, (M \otimes_P Q) \otimes_Q IQ)
$$
which is the image of $\omega(M)$ by the functoriality
of the construction $\omega(-)$ of
Deformation Theory, Lemma \ref{defos-lemma-canonical-class-algebra}.
By More on Algebra, Lemma \ref{more-algebra-lemma-base-change-RHom}
we have
$$
\text{Ext}^2_Q(M \otimes_P Q, (M \otimes_P Q) \otimes_Q IQ) =
\text{Ext}^2_P(M, M \otimes_P IP) \otimes_P Q
$$
here we use that $P$ is Noetherian and $M$ finite.
Our assumption on $P' \to Q'$ guarantees that for an $P$-module $E$
the map $E \to E \otimes_P Q$ is bijective on $J'$-power torsion, see
More on Algebra, Lemma
\ref{more-algebra-lemma-neighbourhood-equivalence}.
Thus we conclude that it suffices to show $\omega(M)$
is $J'$-power torsion. In other words, it suffices to show that
$\omega(M)$ dies in
$$
\text{Ext}^2_P(M, M \otimes_P IP)_g =
\text{Ext}^2_{P_g}(M_g, M_g \otimes_{P_g} IP_g)
$$
for all $g \in J'$. Howeover, by the compatibility of formation of $\omega(M)$
with base change again, we conclude that this is true as $M_g$
is assumed to have a lift (of course you have to use the whole
string of equivalences again).
\end{proof}

\begin{lemma}
\label{lemma-lift-equivalence}
Let $A' \to A$ be a surjective map of Noetherian rings with nilpotent kernel.
Let $A \to B$ be a finite type flat ring map.
Let $\mathfrak b \subset B$ be an ideal such that
$\Spec(B) \to \Spec(A)$ is syntomic on the complement of $V(\mathfrak b)$.
Then $B$ has a flat lift to $A'$ if and only if the $\mathfrak b$-adic
completion $B^\wedge$ has a flat lift to $A'$.
\end{lemma}

\begin{proof}
Choose an $A$-algebra surjection $P = A[x_1, \ldots, x_n] \to B$.
Let $\mathfrak p \subset P$ be the inverse image of $\mathfrak b$.
Set $P' = A'[x_1, \ldots, x_n]$ and denote $\mathfrak p' \subset P'$
the inverse image of $\mathfrak p$. (Of course $\mathfrak p$
and $\mathfrak p'$ do not designate prime ideals here.)
We will denote $P^\wedge$ and $(P')^\wedge$ the respective completions.

\medskip\noindent
Suppose $A' \to B'$ is a flat lift of $A \to B$, in other words,
$A' \to B'$ is flat and there is an $A$-algebra isomorphism
$B = B' \otimes_{A'} A$. Then we can choose an $A'$-algebra map
$P' \to B'$ lifting the given surjection $P \to B$.
By Nakayama's lemma (Algebra, Lemma \ref{algebra-lemma-NAK})
we find that $B'$ is a quotient of $P'$. In particular, we find
that we can endow $B'$ with an $A'$-flat $P'$-module structure
lifting $B$ as an $A$-flat $P$-module.
Conversely, if we can lift $B$ to a $P'$-module $M'$ flat over $A'$,
then $M'$ is a cyclic module $M' \cong P'/J'$ (using Nakayama again)
and setting $B' = P'/J'$ we find a flat lift of $B$ as an algebra.

\medskip\noindent
Set $C = B^\wedge$ and $\mathfrak c = \mathfrak bC$.
Suppose that $A' \to C'$ is a flat lift of $A \to C$.
Then $C'$ is complete with respect to the inverse image $\mathfrak c'$
of $\mathfrak c$
(Algebra, Lemma \ref{algebra-lemma-complete-modulo-nilpotent}).
We choose an $A'$-algebra map $P' \to C'$ lifting
the $A$-algebra map $P \to C$. These maps pass through
completions to give surjections $P^\wedge \to C$ and $(P')^\wedge \to C'$
(for the second again using Nakayama's lemma).
In particular, we find that we can endow $C'$ with an $A'$-flat
$(P')^\wedge$-module structure lifting $C$ as an $A$-flat $P^\wedge$-module.
Conversely, if we can lift $C$ to a $(P')^\wedge$-module $N'$ flat over $A'$,
then $N'$ is a cyclic module $N' \cong (P')^\wedge/\tilde J$
(using Nakayama again) and setting $C' = (P')^\wedge/\tilde J$
we find a flat lift of $C$ as an algebra.

\medskip\noindent
Observe that $P' \to (P')^\wedge$ is a flat ring map which
induces an isomorphism $P'/\mathfrak p' = (P')^\wedge/\mathfrak p'(P')^\wedge$.
We conclude that our lemma is a consequence of
Lemma \ref{lemma-lift-equivalence-module} provided we can
show that $B_g$ lifts to an $A'$-flat $P'_g$-module for
$g \in \mathfrak p'$. However, the ring map $A \to B_g$ is syntomic
and hence lifts to an $A'$-flat algebra $B'$ by
Smoothing Ring Maps, Proposition \ref{smoothing-proposition-lift-smooth}.
Since $A' \to P'_g$ is smooth, we can lift $P_g \to B_g$
to a surjective map $P'_g \to B'$ as before and we get what we want.
\end{proof}

\noindent
Notation. Let $A \to B$ be a ring map. Let $N$ be a $B$-module.
We denote $\text{Exal}_A(B, N)$ the set of isomorphism classes
of extensions
$$
0 \to N \to C \to B \to 0
$$
of $A$-algebras such that $N$ is an ideal of square zero in $C$.
Given a second such $0 \to N \to C' \to B \to 0$ an isomorphism
is a $A$-algebra isomorpism $C \to C'$ such that the diagram
$$
\xymatrix{
0 \ar[r] &
N \ar[r] \ar[d]_{\text{id}} &
C \ar[r] \ar[d] &
B \ar[r] \ar[d]_{\text{id}} & 0 \\
0 \ar[r] &
N \ar[r] &
C' \ar[r] &
B \ar[r] & 0
}
$$
commutes. The assignment $N \mapsto \text{Exal}_A(B, N)$
is a functor which transforms products into products.
Hence this is an additive functor and $\text{Exal}_A(B, N)$
has a natural $B$-module structure. In fact, by
Deformation Theory, Lemma \ref{defos-lemma-choices}
we have $\text{Exal}_A(B, N) = \text{Ext}^1_B(\NL_{B/A}, N)$.

\begin{lemma}
\label{lemma-first-order-completion}
Let $k$ be a field. Let $B$ be a finite type $k$-algebra.
Let $J \subset B$ be an ideal such that
$\Spec(B) \to \Spec(k)$ is smooth on the complement of $V(J)$.
Let $N$ be a finite $B$-module.
Then there is a canonical bijection
$$
\text{Exal}_k(B, N) \to \text{Exal}_k(B^\wedge, N^\wedge)
$$
Here $B^\wedge$ and $N^\wedge$ are the $J$-adic completions.
\end{lemma}

\begin{proof}
The map is given by completion: given $0 \to N \to C \to B \to 0$
in $\text{Exal}_k(B, N)$ we send it to the completion $C^\wedge$
of $C$ with respect to the inverse image of $J$. Compare with
the proof of Lemma \ref{lemma-completion}.

\medskip\noindent
Since $k \to B$ is of finite presentation the complex
$\NL_{B/k}$ can be represented by a complex
$N^{-1} \to N^0$ where $N^i$ is a finite $B$-module, see
Algebra, Section \ref{algebra-section-netherlander} and
in particular
Algebra, Lemma \ref{algebra-lemma-NL-homotopy}.
As $B$ is Noetherian, this means that $\NL_{B/k}$
is pseudo-coherent. For $g \in J$ the $k$-algebra $B_g$
is smooth and hence $(\NL_{B/k})_g = \NL_{B_g/k}$
is quasi-isomorphic to a finite projective $B$-module sitting in degree $0$.
Thus $\text{Ext}^i_B(\NL_{B/k}, N)_g = 0$ for $i \geq 1$
and any $B$-module $N$. By
More on Algebra, Lemma \ref{more-algebra-lemma-ext-annihilated-into}
we conclude that
$$
\text{Ext}^1_B(\NL_{B/k}, N) \longrightarrow
\lim_n \text{Ext}^1_B(\NL_{B/k}, N/J^n N)
$$
is an isomorphism for any finite $B$-module $N$.

\medskip\noindent
Injectivity of the map.
Suppose that $0 \to N \to C \to B \to 0$ is in $\text{Exal}_k(B, N)$
and maps to zero in $\text{Exal}_k(B^\wedge, N^\wedge)$.
Choose a splitting $C^\wedge = B^\wedge \oplus N^\wedge$.
Then the induced map $C \to C^\wedge \to N^\wedge$
gives maps $C \to N/J^nN$ for all $n$.
Hence we see that our element is in the kernel of the maps
$$
\text{Ext}^1_B(\NL_{B/k}, N) \to
\text{Ext}^1_B(\NL_{B/k}, N/J^n N)
$$
for all $n$. By the previous paragraph we conclude that
our element is zero.

\medskip\noindent
Surjectivity of the map. Let $0 \to N^\wedge \to C' \to B^\wedge \to 0$
be an element of $\text{Exal}_k(B^\wedge, N^\wedge)$.
Pulling back by $B \to B^\wedge$ we get an element
$0 \to N^\wedge \to C'' \to B \to 0$ in
$\text{Exal}_k(B, N^\wedge)$.
we have
$$
\text{Ext}^1_B(\NL_{B/k}, N^\wedge) =
\text{Ext}^1_B(\NL_{B/k}, N) \otimes_B B^\wedge =
\text{Ext}^1_B(\NL_{B/k}, N)
$$
The first equality as $N^\wedge = N \otimes_B B^\wedge$
(Algebra, Lemma \ref{algebra-lemma-completion-tensor})
and
More on Algebra, Remark \ref{more-algebra-remark-pseudo-coherence-and-ext}.
The second equality because $\text{Ext}^1_B(\NL_{B/k}, N)$
is $J$-power torsion (see above), $B \to B^\wedge$ is flat and induces
an isomorphism $B/J \to B^\wedge/JB^\wedge$, and
More on Algebra, Lemma \ref{more-algebra-lemma-neighbourhood-equivalence}.
Thus we can find a $C \in \text{Exal}_k(B, N)$ mapping to $C''$ in
$\text{Exal}_k(B, N^\wedge)$.
Thus
$$
0 \to N^\wedge \to C' \to B^\wedge \to 0
\quad\text{and}\quad
0 \to N^\wedge \to C^\wedge \to B^\wedge \to 0
$$
are two elements of $\text{Exal}_k(B^\wedge, N^\wedge)$
mapping to the same element of $\text{Exal}_k(B, N^\wedge)$.
Taking the difference we get an element
$0 \to N^\wedge \to C' \to B^\wedge \to 0$ of
$\text{Exal}_k(B^\wedge, N^\wedge)$
whose image in $\text{Exal}_k(B, N^\wedge)$ is zero.
This means there exists
$$
\xymatrix{
0 \ar[r] &
N^\wedge \ar[r] &
C' \ar[r] &
B^\wedge \ar[r] & 0 \\
& & B \ar[u]^\sigma \ar[ru]
}
$$
Let $J' \subset C'$ be the inverse image of $JB^\wedge \subset B^\wedge$.
To finish the proof it suffices to note that
$\sigma$ is continuous for the $J$-adic topology on $B$
and the $J'$-adic topology on $C'$ and that $C'$ is $J'$-adically complete by
Algebra, Lemma \ref{algebra-lemma-complete-modulo-nilpotent}
(here we also use that $C'$ is Noetherian; small detail omitted).
Namely, this means that $\sigma$ factors through the
completion $B^\wedge$ and $C' = 0$ in $\text{Exal}_k(B^\wedge, N^\wedge)$.
\end{proof}

\begin{lemma}
\label{lemma-smooth-completion}
In Example \ref{example-rings} let $P$ be a $k$-algebra.
Let $J \subset P$ be an ideal.
Denote $P^\wedge$ the $J$-adic completion. If
\begin{enumerate}
\item $k \to P$ is of finite type, and
\item $\Spec(P) \to \Spec(k)$ is smooth on the complement of $V(J)$.
\end{enumerate}
then the functor between deformation categories of
Lemma \ref{lemma-completion}
$$
\Deformationcategory_P \longrightarrow \Deformationcategory_{P^\wedge}
$$
is smooth and induces an isomorphism on tangent spaces.
\end{lemma}

\begin{proof}
We know that $\Deformationcategory_P$ and $\Deformationcategory_{P^\wedge}$
are deformation categories by Lemma \ref{lemma-rings-RS}.
Thus it suffices to check
our functor identifies tangent spaces and a correspondence
between liftability, see
Formal Deformation Theory, Lemma \ref{formal-defos-lemma-easy-check-smooth}.
The property on liftability is proven in
Lemma \ref{lemma-lift-equivalence}
and the isomorphism on tangent spaces is the special case of
Lemma \ref{lemma-first-order-completion} where $N = B$.
\end{proof}



\section{Deformations of localizations}
\label{section-compare-localization}

\noindent
In this section we compare the deformation problem posed
by an algebra and its localization at a multiplicative subset.
We first discuss ``liftability''.

\begin{lemma}
\label{lemma-lift-equivalence-localization}
Let $A' \to A$ be a surjective map of Noetherian rings with nilpotent kernel.
Let $A \to B$ be a finite type flat ring map.
Let $S \subset B$ be a multiplicative subset such that
if $\Spec(B) \to \Spec(A)$ is not syntomic at $\mathfrak q$,
then $S \cap \mathfrak q = \emptyset$.
Then $B$ has a flat lift to $A'$ if and only if
$S^{-1}B$ has a flat lift to $A'$.
\end{lemma}

\begin{proof}
This proof is the same as the proof of
Lemma \ref{lemma-lift-equivalence} but easier. We suggest the
reader to skip the proof.
Choose an $A$-algebra surjection $P = A[x_1, \ldots, x_n] \to B$.
Let $S_P \subset P$ be the inverse image of $S$.
Set $P' = A'[x_1, \ldots, x_n]$ and denote $S_{P'} \subset P'$
the inverse image of $S_P$.

\medskip\noindent
Suppose $A' \to B'$ is a flat lift of $A \to B$, in other words,
$A' \to B'$ is flat and there is an $A$-algebra isomorphism
$B = B' \otimes_{A'} A$. Then we can choose an $A'$-algebra map
$P' \to B'$ lifting the given surjection $P \to B$.
By Nakayama's lemma (Algebra, Lemma \ref{algebra-lemma-NAK})
we find that $B'$ is a quotient of $P'$. In particular, we find
that we can endow $B'$ with an $A'$-flat $P'$-module structure
lifting $B$ as an $A$-flat $P$-module.
Conversely, if we can lift $B$ to a $P'$-module $M'$ flat over $A'$,
then $M'$ is a cyclic module $M' \cong P'/J'$ (using Nakayama again)
and setting $B' = P'/J'$ we find a flat lift of $B$ as an algebra.

\medskip\noindent
Set $C = S^{-1}B$. Suppose that $A' \to C'$ is a flat lift of $A \to C$.
Elements of $C'$ which map to invertible elements of $C$ are invertible.
We choose an $A'$-algebra map $P' \to C'$ lifting
the $A$-algebra map $P \to C$. By the remark above
these maps pass through localizations to give surjections
$S_P^{-1}P \to C$ and $S_{P'}^{-1}P' \to C'$
(for the second use Nakayama's lemma).
In particular, we find that we can endow $C'$ with an $A'$-flat
$S_{P'}^{-1}P'$-module structure lifting $C$ as an $A$-flat
$S_P^{-1}P$-module. Conversely, if we can lift $C$ to a
$S_{P'}^{-1}P'$-module $N'$ flat over $A'$, then $N'$
is a cyclic module $N' \cong S_{P'}^{-1}P'/\tilde J$
(using Nakayama again) and setting $C' = S_{P'}^{-1}P'/\tilde J$
we find a flat lift of $C$ as an algebra.

\medskip\noindent
The syntomic locus of a morphism of schemes is open by definition.
Let $J_B \subset B$ be an ideal cutting out the set of points
in $\Spec(B)$ where $\Spec(B) \to \Spec(A)$ is not syntomic.
Denote $J_P \subset P$ and $J_{P'} \subset P'$ the corresponding
ideals. Observe that $P' \to S_{P'}^{-1}P'$ is a flat ring map which
induces an isomorphism $P'/J_{P'} = S_{P'}^{-1}P'/J_{P'}S_{P'}^{-1}P'$
by our assumption on $S$ in the lemma, namely, the assumption
in the lemma is exactly that $B/J_B = S^{-1}(B/J_B)$.
We conclude that our lemma is a consequence of
Lemma \ref{lemma-lift-equivalence-module} provided we can
show that $B_g$ lifts to an $A'$-flat $P'_g$-module for
$g \in J_B$. However, the ring map $A \to B_g$ is syntomic
and hence lifts to an $A'$-flat algebra $B'$ by
Smoothing Ring Maps, Proposition \ref{smoothing-proposition-lift-smooth}.
Since $A' \to P'_g$ is smooth, we can lift $P_g \to B_g$
to a surjective map $P'_g \to B'$ as before and we get what we want.
\end{proof}

\begin{lemma}
\label{lemma-first-order-localization}
Let $k$ be a field. Let $B$ be a finite type $k$-algebra.
Let $S \subset B$ be a multiplicative subset ideal such that
if $\Spec(B) \to \Spec(k)$ is not smooth at $\mathfrak q$
then $S \cap \mathfrak q = \emptyset$.
Let $N$ be a finite $B$-module.
Then there is a canonical bijection
$$
\text{Exal}_k(B, N) \to \text{Exal}_k(S^{-1}B, S^{-1}N)
$$
\end{lemma}

\begin{proof}
This proof is the same as the proof of
Lemma \ref{lemma-first-order-completion} but easier. We suggest the
reader to skip the proof.
The map is given by localization: given $0 \to N \to C \to B \to 0$
in $\text{Exal}_k(B, N)$ we send it to the localization $S_C^{-1}C$
of $C$ with respect to the inverse image $S_C \subset C$ of $S$.
Compare with the proof of Lemma \ref{lemma-localization}.

\medskip\noindent
The smooth locus of a morphism of schemes is open by definition.
Let $J \subset B$ be an ideal cutting out the set of points
in $\Spec(B)$ where $\Spec(B) \to \Spec(A)$ is not smooth.
Since $k \to B$ is of finite presentation the complex
$\NL_{B/k}$ can be represented by a complex
$N^{-1} \to N^0$ where $N^i$ is a finite $B$-module, see
Algebra, Section \ref{algebra-section-netherlander} and
in particular
Algebra, Lemma \ref{algebra-lemma-NL-homotopy}.
As $B$ is Noetherian, this means that $\NL_{B/k}$
is pseudo-coherent. For $g \in J$ the $k$-algebra $B_g$
is smooth and hence $(\NL_{B/k})_g = \NL_{B_g/k}$
is quasi-isomorphic to a finite projective $B$-module sitting in degree $0$.
Thus $\text{Ext}^i_B(\NL_{B/k}, N)_g = 0$ for $i \geq 1$
and any $B$-module $N$. Finally, we have
$$
\text{Ext}^1_{S^{-1}B}(\NL_{S^{-1}B/k}, S^{-1}N) =
\text{Ext}^1_B(\NL_{B/k}, N) \otimes_B S^{-1}B =
\text{Ext}^1_B(\NL_{B/k}, N)
$$
The first equality by
More on Algebra, Lemma \ref{more-algebra-lemma-base-change-RHom}
and Algebra, Lemma \ref{algebra-lemma-localize-NL}.
The second because $\text{Ext}^1_B(\NL_{B/k}, N)$ is $J$-power
torsion and elements of $S$ act invertibly on $J$-power torsion modules.
This concludes the proof by the description of $\text{Exal}_A(B, N)$
as $\text{Ext}^1_B(\NL_{B/A}, N)$ given just above
Lemma \ref{lemma-first-order-completion}.
\end{proof}

\begin{lemma}
\label{lemma-smooth-localization}
In Example \ref{example-rings} let $P$ be a $k$-algebra.
Let $S \subset P$ be a multiplicative subset. If
\begin{enumerate}
\item $k \to P$ is of finite type, and
\item $\Spec(P) \to \Spec(k)$ is smooth at all points of
$V(g)$ for all $g \in S$.
\end{enumerate}
then the functor between deformation categories of
Lemma \ref{lemma-localization}
$$
\Deformationcategory_P \longrightarrow \Deformationcategory_{S^{-1}P}
$$
is smooth and induces an isomorphism on tangent spaces.
\end{lemma}

\begin{proof}
We know that $\Deformationcategory_P$ and $\Deformationcategory_{S^{-1}P}$
are deformation categories by Lemma \ref{lemma-rings-RS}.
Thus it suffices to check
our functor identifies tangent spaces and a correspondence
between liftability, see
Formal Deformation Theory, Lemma \ref{formal-defos-lemma-easy-check-smooth}.
The property on liftability is proven in
Lemma \ref{lemma-lift-equivalence-localization}
and the isomorphism on tangent spaces is the special case of
Lemma \ref{lemma-first-order-localization} where $N = B$.
\end{proof}



\section{Deformations of henselizations}
\label{section-compare-henselization}

\noindent
In this section we compare the deformation problem posed
by an algebra and its completion.
We first discuss ``liftability''.

\begin{lemma}
\label{lemma-lift-equivalence-henselization}
Let $A' \to A$ be a surjective map of Noetherian rings with nilpotent kernel.
Let $A \to B$ be a finite type flat ring map.
Let $\mathfrak b \subset B$ be an ideal such that
$\Spec(B) \to \Spec(A)$ is syntomic on the complement of $V(\mathfrak b)$.
Let $(B^h, \mathfrak b^h)$ be the henselization of the pair $(B, \mathfrak b)$.
Then $B$ has a flat lift to $A'$ if and only if $B^h$ has a flat lift to $A'$.
\end{lemma}

\begin{proof}[First proof]
This proof is a cheat. Namely, if $B$ has a flat lift $B'$, then
taking the henselization $(B')^h$ we obtain a flat lift of $B^h$
(compare with the proof of Lemma \ref{lemma-henselization}).
Conversely, suppose that $C'$ is an $A'$-flat lift of $(B')^h$.
Then let $\mathfrak c' \subset C'$ be the inverse image of the
ideal $\mathfrak b^h$. Then the completion $(C')^\wedge$ of
$C'$ with respect to $\mathfrak c'$ is a lift of $B^\wedge$ (details omitted).
Hence we see that $B$ has a flat lift by
Lemma \ref{lemma-lift-equivalence}.
\end{proof}

\begin{proof}[Second proof]
Choose an $A$-algebra surjection $P = A[x_1, \ldots, x_n] \to B$.
Let $\mathfrak p \subset P$ be the inverse image of $\mathfrak b$.
Set $P' = A'[x_1, \ldots, x_n]$ and denote $\mathfrak p' \subset P'$
the inverse image of $\mathfrak p$. (Of course $\mathfrak p$
and $\mathfrak p'$ do not designate prime ideals here.)
We will denote $P^h$ and $(P')^h$ the respective henselizations.
We will use that taking henselizations is functorial and that
the henselization of a quotient is the corresponding quotient
of the henselization, see
More on Algebra, Lemmas
\ref{more-algebra-lemma-irreducible-henselian-pair-connected} and
\ref{more-algebra-lemma-henselization-integral}.

\medskip\noindent
Suppose $A' \to B'$ is a flat lift of $A \to B$, in other words,
$A' \to B'$ is flat and there is an $A$-algebra isomorphism
$B = B' \otimes_{A'} A$. Then we can choose an $A'$-algebra map
$P' \to B'$ lifting the given surjection $P \to B$.
By Nakayama's lemma (Algebra, Lemma \ref{algebra-lemma-NAK})
we find that $B'$ is a quotient of $P'$. In particular, we find
that we can endow $B'$ with an $A'$-flat $P'$-module structure
lifting $B$ as an $A$-flat $P$-module.
Conversely, if we can lift $B$ to a $P'$-module $M'$ flat over $A'$,
then $M'$ is a cyclic module $M' \cong P'/J'$ (using Nakayama again)
and setting $B' = P'/J'$ we find a flat lift of $B$ as an algebra.

\medskip\noindent
Set $C = B^h$ and $\mathfrak c = \mathfrak bC$.
Suppose that $A' \to C'$ is a flat lift of $A \to C$.
Then $C'$ is henselian with respect to the inverse image
$\mathfrak c'$ of $\mathfrak c$
(by More on Algebra, Lemma \ref{more-algebra-lemma-henselian-henselian-pair}
and the fact that the kernel of $C' \to C$ is nilpotent).
We choose an $A'$-algebra map $P' \to C'$ lifting
the $A$-algebra map $P \to C$. These maps pass through
henselizations to give surjections $P^h \to C$ and $(P')^h \to C'$
(for the second again using Nakayama's lemma).
In particular, we find that we can endow $C'$ with an $A'$-flat
$(P')^h$-module structure lifting $C$ as an $A$-flat $P^h$-module.
Conversely, if we can lift $C$ to a $(P')^h$-module $N'$ flat over $A'$,
then $N'$ is a cyclic module $N' \cong (P')^h/\tilde J$
(using Nakayama again) and setting $C' = (P')^h/\tilde J$
we find a flat lift of $C$ as an algebra.

\medskip\noindent
Observe that $P' \to (P')^h$ is a flat ring map which
induces an isomorphism $P'/\mathfrak p' = (P')^h/\mathfrak p'(P')^h$
(More on Algebra, Lemma \ref{more-algebra-lemma-henselization-flat}).
We conclude that our lemma is a consequence of
Lemma \ref{lemma-lift-equivalence-module} provided we can
show that $B_g$ lifts to an $A'$-flat $P'_g$-module for
$g \in \mathfrak p'$. However, the ring map $A \to B_g$ is syntomic
and hence lifts to an $A'$-flat algebra $B'$ by
Smoothing Ring Maps, Proposition \ref{smoothing-proposition-lift-smooth}.
Since $A' \to P'_g$ is smooth, we can lift $P_g \to B_g$
to a surjective map $P'_g \to B'$ as before and we get what we want.
\end{proof}

\begin{lemma}
\label{lemma-first-order-henselization}
Let $k$ be a field. Let $B$ be a finite type $k$-algebra.
Let $J \subset B$ be an ideal such that
$\Spec(B) \to \Spec(k)$ is smooth on the complement of $V(J)$.
Let $N$ be a finite $B$-module.
Then there is a canonical bijection
$$
\text{Exal}_k(B, N) \to \text{Exal}_k(B^h, N^h)
$$
Here $(B^h, J^h)$ is the henselization of $(B, J)$
and $N^h = N \otimes_B B^h$.
\end{lemma}

\begin{proof}
This proof is the same as the proof of
Lemma \ref{lemma-first-order-completion} but easier. We suggest the
reader to skip the proof.
The map is given by henselization: given $0 \to N \to C \to B \to 0$
in $\text{Exal}_k(B, N)$ we send it to the
henselization $C^h$
of $C$ with respect to the inverse image $J_C \subset C$ of $J$.
Compare with the proof of Lemma \ref{lemma-henselization}.

\medskip\noindent
Since $k \to B$ is of finite presentation the complex
$\NL_{B/k}$ can be represented by a complex
$N^{-1} \to N^0$ where $N^i$ is a finite $B$-module, see
Algebra, Section \ref{algebra-section-netherlander} and
in particular
Algebra, Lemma \ref{algebra-lemma-NL-homotopy}.
As $B$ is Noetherian, this means that $\NL_{B/k}$
is pseudo-coherent. For $g \in J$ the $k$-algebra $B_g$
is smooth and hence $(\NL_{B/k})_g = \NL_{B_g/k}$
is quasi-isomorphic to a finite projective $B$-module sitting in degree $0$.
Thus $\text{Ext}^i_B(\NL_{B/k}, N)_g = 0$ for $i \geq 1$
and any $B$-module $N$. Finally, we have
\begin{align*}
\text{Ext}^1_{B^h}(\NL_{B^h/k}, N^h)
& =
\text{Ext}^1_{B^h}(\NL_{B/k} \otimes_B B^h, N \otimes_B B^h) \\
& =
\text{Ext}^1_B(\NL_{B/k}, N) \otimes_B B^h \\
& =
\text{Ext}^1_B(\NL_{B/k}, N)
\end{align*}
The first equality by
More on Algebra, Lemma \ref{more-algebra-lemma-henselization-NL}
(or rather its analogue for henselizations of pairs).
The second by
More on Algebra, Lemma \ref{more-algebra-lemma-base-change-RHom}.
The third because $\text{Ext}^1_B(\NL_{B/k}, N)$ is $J$-power
torsion, the map $B \to B^h$ is flat and induces an isomorphism
$B/J \to B^h/JB^h$ (More on Algebra, Lemma
\ref{more-algebra-lemma-henselization-flat}), and
More on Algebra, Lemma \ref{more-algebra-lemma-neighbourhood-equivalence}.
This concludes the proof by the description of $\text{Exal}_A(B, N)$
as $\text{Ext}^1_B(\NL_{B/A}, N)$ given just above
Lemma \ref{lemma-first-order-completion}.
\end{proof}

\begin{lemma}
\label{lemma-smooth-henselization}
In Example \ref{example-rings} let $P$ be a $k$-algebra.
Let $J \subset P$ be an ideal.
Denote $(P^h, J^h)$ the henselization of the pair $(P, J)$. If
\begin{enumerate}
\item $k \to P$ is of finite type, and
\item $\Spec(P) \to \Spec(k)$ is smooth on the complement of $V(J)$,
\end{enumerate}
then the functor between deformation categories of
Lemma \ref{lemma-henselization}
$$
\Deformationcategory_P \longrightarrow \Deformationcategory_{P^h}
$$
is smooth and induces an isomorphism on tangent spaces.
\end{lemma}

\begin{proof}
We know that $\Deformationcategory_P$ and $\Deformationcategory_{P^h}$
are deformation categories by Lemma \ref{lemma-rings-RS}.
Thus it suffices to check
our functor identifies tangent spaces and a correspondence
between liftability, see
Formal Deformation Theory, Lemma \ref{formal-defos-lemma-easy-check-smooth}.
The property on liftability is proven in
Lemma \ref{lemma-lift-equivalence-henselization}
and the isomorphism on tangent spaces is the special case of
Lemma \ref{lemma-first-order-henselization} where $N = B$.
\end{proof}



\section{Application to isolated singularities}
\label{section-isolated}

\noindent
We apply the discussion above to study the deformation theory
of a finite type algebra with finitely many singular points.

\begin{lemma}
\label{lemma-isolated}
In Example \ref{example-rings} let $P$ be a $k$-algebra.
Assume that $k \to P$ is of finite type and that $\Spec(P) \to \Spec(k)$
is smooth except at the maximal ideals
$\mathfrak m_1, \ldots, \mathfrak m_n$ of $P$.
Let $P_{\mathfrak m_i}$, $P_{\mathfrak m_i}^h$, $P_{\mathfrak m_i}^\wedge$
be the local ring, henselization, completion.
Then the maps of deformation categories
$$
\Deformationcategory_P \to
\prod \Deformationcategory_{P_{\mathfrak m_i}} \to
\prod \Deformationcategory_{P_{\mathfrak m_i}^h} \to
\prod \Deformationcategory_{P_{\mathfrak m_i}^\wedge}
$$
are smooth and induce isomorphisms on their finite dimensional
tangent spaces.
\end{lemma}

\begin{proof}
The tangent space is finite dimensional by
Lemma \ref{lemma-finite-type-rings-TI}.
The functors between the categories are constructed
in Lemmas \ref{lemma-localization}, \ref{lemma-henselization}, and
\ref{lemma-completion} (we omit some verifications of the form:
the completion of the henselization is the completion).

\medskip\noindent
Set $J = \mathfrak m_1 \cap \ldots \cap \mathfrak m_n$ and apply
Lemma \ref{lemma-smooth-completion} to get that
$\Deformationcategory_P \to \Deformationcategory_{P^\wedge}$
is smooth and induces an isomorphism on tangent spaces
where $P^\wedge$ is the $J$-adic completion of $P$.
However, since $P^\wedge = \prod P_{\mathfrak m_i}^\wedge$
we see that the map $\Deformationcategory_P \to
\prod \Deformationcategory_{P_{\mathfrak m_i}^\wedge}$
is smooth and induces an isomorphism on tangent spaces.

\medskip\noindent
Let $(P^h, J^h)$ be the henselization of the pair $(P, J)$.
Then $P^h = \prod P_{\mathfrak m_i}^h$ (look at idempotents
and use More on Algebra, Lemma
\ref{more-algebra-lemma-characterize-henselian-pair}).
Hence we can apply Lemma \ref{lemma-smooth-henselization}
to conclude as in the case of completion.

\medskip\noindent
To get the final case it suffices to show that
$\Deformationcategory_{P_{\mathfrak m_i}} \to
\Deformationcategory_{P_{\mathfrak m_i}^\wedge}$
is smooth and induce isomorphisms on tangent spaces for each $i$ separately.
To do this, we may replace $P$ by a principal localization
whose only singular point is a maximal ideal $\mathfrak m$
(corresponding to $\mathfrak m_i$ in the original $P$).
Then we can apply
Lemma \ref{lemma-smooth-localization}
with multiplicative subset $S = P \setminus \mathfrak m$ to conclude.
Minor details omitted.
\end{proof}






\section{Unobstructed deformation problems}
\label{section-unobstructed}

\noindent
Let $p : \mathcal{F} \to \mathcal{C}_\Lambda$ be a
category cofibred in groupoids. Recall that we say $\mathcal{F}$
is {\it smooth} or {\it unobstructed} if $p$ is smooth.
This means that given a surjection $\varphi : A' \to A$ in
$\mathcal{C}_\Lambda$ and $x \in \Ob(\mathcal{F}(A))$
there exists a morphism $f : x' \to x$ in $\mathcal{F}$
with $p(f) = \varphi$.
See Formal Deformation Theory, Section \ref{formal-defos-section-smooth}.
In this section we give some geometrically meaningful examples.

\begin{lemma}
\label{lemma-lci-unobstructed}
In Example \ref{example-rings} let $P$ be a local complete
intersection over $k$ (Algebra, Definition \ref{algebra-definition-lci-field}).
Then $\Deformationcategory_P$ is unobstructed.
\end{lemma}

\begin{proof}
Let $(A, Q) \to (k, P)$ be an object of $\Deformationcategory_P$.
Then we see that $A \to Q$ is a syntomic ring map by
Algebra, Definition \ref{algebra-definition-lci}.
Hence for any surjection $A' \to A$ in $\mathcal{C}_\Lambda$
we see that there is a morphism $(A', Q') \to (A, Q)$
lifting $A' \to A$ by
Smoothing Ring Maps, Proposition \ref{smoothing-proposition-lift-smooth}.
This proves the lemma.
\end{proof}

\begin{lemma}
\label{lemma-glueing-smooth}
In Situation \ref{situation-glueing} if $U_{12} \to \Spec(k)$ is smooth,
then the morphism
$$
\Deformationcategory_X
\longrightarrow
\Deformationcategory_{U_1} \times \Deformationcategory_{U_2} =
\Deformationcategory_{P_1} \times \Deformationcategory_{P_2}
$$
is smooth. If in addition
$U_1$ is a local complete intersection over $k$, then
$$
\Deformationcategory_X
\longrightarrow
\Deformationcategory_{U_2} = \Deformationcategory_{P_2}
$$
is smooth.
\end{lemma}

\begin{proof}
The equality signs hold by Lemma \ref{lemma-affine}.
Let us think of $\mathcal{C}_\Lambda$ as a deformation
category over $\mathcal{C}_\Lambda$ as in
Formal Deformation Theory, Section \ref{formal-defos-section-smooth}.
Then
$$
\Deformationcategory_{P_1} \times \Deformationcategory_{P_2} =
\Deformationcategory_{P_1}
\times_{\mathcal{C}_\Lambda}
\Deformationcategory_{P_2},
$$
see Formal Deformation Theory, Remarks
\ref{formal-defos-remarks-cofibered-groupoids}
(\ref{formal-defos-item-product}).
Using
Lemma \ref{lemma-glueing}
the first statement is that the functor
$$
\Deformationcategory_{P_1}
\times_{\Deformationcategory_{P_{12}}}
\Deformationcategory_{P_2}
\longrightarrow
\Deformationcategory_{P_1}
\times_{\mathcal{C}_\Lambda}
\Deformationcategory_{P_2}
$$
is smooth. This follows from Formal Deformation Theory, Lemma
\ref{formal-defos-lemma-map-fibre-products-smooth} as long as
we can show that $T\Deformationcategory_{P_{12}} = (0)$.
This vanishing follows from Lemma \ref{lemma-smooth}
as $P_{12}$ is smooth over $k$.
For the second statement it suffices to show that
$\Deformationcategory_{P_1} \to \mathcal{C}_\Lambda$
is smooth, see Formal Deformation Theory, Lemma
\ref{formal-defos-lemma-smooth-properties}.
In other words, we have to show $\Deformationcategory_{P_1}$
is unobstructed, which is Lemma \ref{lemma-lci-unobstructed}.
\end{proof}

\begin{lemma}
\label{lemma-curve-isolated}
In Example \ref{example-schemes} let $X$ be a scheme over $k$. Assume
\begin{enumerate}
\item $X$ is separated, finite type over $k$ and $\dim(X) \leq 1$,
\item $X \to \Spec(k)$ is smooth except at the closed
points $p_1, \ldots, p_n \in X$.
\end{enumerate}
Let $\mathcal{O}_{X, p_1}$, $\mathcal{O}_{X, p_1}^h$,
$\mathcal{O}_{X, p_1}^\wedge$ be the local ring, henselization, completion.
Consider the maps of deformation categories
$$
\Deformationcategory_X
\longrightarrow
\prod \Deformationcategory_{\mathcal{O}_{X, p_i}}
\longrightarrow
\prod \Deformationcategory_{\mathcal{O}_{X, p_i}^h}
\longrightarrow
\prod \Deformationcategory_{\mathcal{O}_{X, p_i}^\wedge}
$$
The first arrow is smooth and the second and third arrows
are smooth and induce isomorphisms on tangent spaces.
\end{lemma}

\begin{proof}
Choose an affine open $U_2 \subset X$ containing
$p_1, \ldots, p_n$ and the generic point of every irreducible
component of $X$. This is possible by
Varieties, Lemma \ref{varieties-lemma-dim-1-quasi-projective}
and Properties, Lemma \ref{properties-lemma-ample-finite-set-in-affine}.
Then $X \setminus U_2$ is finite and we can choose an affine open
$U_1 \subset X \setminus \{p_1, \ldots, p_n\}$ such that
$X = U_1 \cup U_2$. Set $U_{12} = U_1 \cap U_2$.
Then $U_1$ and $U_{12}$ are smooth affine schemes over $k$.
We conclude that
$$
\Deformationcategory_X \longrightarrow \Deformationcategory_{U_2}
$$
is smooth by Lemma \ref{lemma-glueing-smooth}.
Applying Lemmas \ref{lemma-affine} and \ref{lemma-isolated} we win.
\end{proof}

\begin{lemma}
\label{lemma-curve-isolated-lci}
In Example \ref{example-schemes} let $X$ be a scheme over $k$. Assume
\begin{enumerate}
\item $X$ is separated, finite type over $k$ and $\dim(X) \leq 1$,
\item $X$ is a local complete intersection over $k$, and
\item $X \to \Spec(k)$ is smooth except at finitely many points.
\end{enumerate}
Then $\Deformationcategory_X$ is unobstructed.
\end{lemma}

\begin{proof}
Let $p_1, \ldots, p_n \in X$ be the points where $X \to \Spec(k)$
isn't smooth. Choose an affine open $U_2 \subset X$ containing
$p_1, \ldots, p_n$ and the generic point of every irreducible
component of $X$. This is possible by
Varieties, Lemma \ref{varieties-lemma-dim-1-quasi-projective}
and Properties, Lemma \ref{properties-lemma-ample-finite-set-in-affine}.
Then $X \setminus U_2$ is finite and we can choose an affine open
$U_1 \subset X \setminus \{p_1, \ldots, p_n\}$ such that
$X = U_1 \cup U_2$. Set $U_{12} = U_1 \cap U_2$.
Then $U_1$ and $U_{12}$ are smooth affine schemes over $k$.
We conclude that
$$
\Deformationcategory_X \longrightarrow \Deformationcategory_{U_2}
$$
is smooth by Lemma \ref{lemma-glueing-smooth}.
Applying Lemmas \ref{lemma-affine} and \ref{lemma-lci-unobstructed} we win.
\end{proof}




\section{Smoothings}
\label{section-smoothing}

\noindent
Suppose given a finite type scheme or algebraic space $X$ over a field $k$.
It is often useful to find a flat morphism of finite type $Y \to \Spec(k[[t]])$
whose generic fibre is smooth and whose special fibre is isomorphic to $X$.
Such a thing is called a smoothing of $X$. In this section we will find
a smoothing for $1$-dimensional separated $X$ which have isolated
local complete intersection singularities.

\begin{lemma}
\label{lemma-criterion-smoothing}
Let $k$ be a field. Set $S = \Spec(k[[t]])$ and
$S_n = \Spec(k[t]/(t^n))$. Let $Y \to S$ be a proper, flat morphism
of schemes whose special fibre $X$ is Cohen-Macaulay and
equidimensional of dimension $d$. Denote $X_n = Y \times_S S_n$.
If for some $n \geq 1$ the $d$the Fitting ideal of $\Omega_{X_n/S_n}$
contains $t^{n - 1}$, then the generic fibre of $Y \to S$ is smooth.
\end{lemma}

\begin{proof}
By More on Morphisms, Lemma
\ref{more-morphisms-lemma-flat-finite-presentation-CM-open}
we see that $Y \to S$ is a Cohen-Macaulay morphism.
By Morphisms, Lemma
\ref{morphisms-lemma-flat-finite-presentation-CM-fibres-relative-dimension}
we see that $Y \to S$ has relative dimension $d$.
By Divisors, Lemma \ref{divisors-lemma-d-fitting-ideal-omega-smooth}
the $d$th Fitting ideal $\mathcal{I} \subset \mathcal{O}_Y$
of $\Omega_{Y/S}$ cuts out the singular locus of the morphism $Y \to S$.
In other words, $V(\mathcal{I}) \subset Y$ is the closed subset
of points where $Y \to S$ is not smooth.
By Divisors, Lemma \ref{divisors-lemma-base-change-and-fitting-ideal-omega}
formation of this Fitting ideal commutes with base change.
By assumption we see that $t^{n - 1}$ is a section of
$\mathcal{I} + t^n\mathcal{O}_Y$. Thus for every
$x \in X = V(t) \subset Y$ we conclude that
$t^{n - 1} \in \mathcal{I}_x$ where $\mathcal{I}_x$ is the stalk at $x$.
This implies that $V(\mathcal{I}) \subset V(t)$ in an
open neighbourhood of $X$ in $Y$. Since $Y \to S$
is proper, this implies $V(\mathcal{I}) \subset V(t)$
as desired.
\end{proof}

\begin{lemma}
\label{lemma-jouanolou-type-thing}
Let $k$ be a field. Let $1 \leq c \leq n$ be integers.
Let $f_1, \ldots, f_c \in k[x_1, \ldots x_n]$ be elements.
Let $a_{ij}$, $0 \leq i \leq n$, $1 \leq j \leq c$ be
variables. Consider
$$
g_j = f_j + a_{0j} + a_{1j}x_1 + \ldots + a_{nj}x_n \in
k[a_{ij}][x_1, \ldots, x_n]
$$
Denote $Y \subset \mathbf{A}^{n + c(n + 1)}_k$
the closed subscheme cut out by $g_1, \ldots, g_c$.
Denote $\pi : Y \to \mathbf{A}^{c(n + 1)}_k$ the projection
onto the affine space with variables $a_{ij}$.
Then there is a nonempty Zariski open 
of $\mathbf{A}^{c(n + 1)}_k$ over which $\pi$ is smooth.
\end{lemma}

\begin{proof}
Recall that the set of points where $\pi$ is smooth is open.
Thus the complement, i.e., the singular locus, is closed.
By Chevalley's theorem (in the form of
Morphisms, Lemma \ref{morphisms-lemma-chevalley})
the image of the singular locus is constructible.
Hence if the generic point of $\mathbf{A}^{c(n + 1)}_k$
is not in the image of the singular locus, then
the lemma follows (by Topology, Lemma
\ref{topology-lemma-generic-point-in-constructible} for example).
Thus we have to show there is no point
$y \in Y$ where $\pi$ is not smooth mapping to
the generic point of $\mathbf{A}^{c(n + 1)}_k$.
Consider the matrix of partial derivatives
$$
(\frac{\partial g_j}{\partial x_i}) =
(\frac{\partial f_j}{\partial x_i} + a_{ij})
$$
The image of this matrix in $\kappa(y)$ must have rank $< c$
since otherwise $\pi$ would be smooth at $y$, see discussion in
Smoothing Ring Maps, Section \ref{smoothing-section-singular-ideal}.
Thus we can find $\lambda_1, \ldots, \lambda_c \in \kappa(y)$
not all zero such that the vector $(\lambda_1, \ldots, \lambda_c)$
is in the kernel of this matrix.
After renumbering we may assume $\lambda_1 \not = 0$.
Dividing by $\lambda_1$ we may assume our vector has
the form $(1, \lambda_2, \ldots, \lambda_c)$.
Then we obtain
$$
a_{i1} = -
\frac{\partial f_j}{\partial x_1} -
\sum\nolimits_{j = 2, \ldots, c} \lambda_j(\frac{\partial f_j}{\partial x_i} + a_{ij})
$$
in $\kappa(y)$ for $i = 1, \ldots, n$. Moreover, since $y \in Y$ we also
have
$$
a_{0j} = -f_j - a_{1j}x_1 - \ldots - a_{nj}x_n
$$
in $\kappa(y)$. This means that the subfield of $\kappa(y)$
generated by $a_{ij}$ is contained in the subfield of $\kappa(y)$
generated by the images of $x_1, \ldots, x_n, \lambda_2, \ldots, \lambda_c$,
and $a_{ij}$ except for $a_{i1}$ and $a_{0j}$.
We count and we see that the transcendence degree of this is
at most $c(n + 1) - 1$. Hence $y$ cannot map to the generic point
as desired.
\end{proof}

\begin{lemma}
\label{lemma-smoothing-affine-lci}
Let $k$ be a field. Let $A$ be a global complete interesection
over $k$. There exists a flat finite type ring map
$k[[t]] \to B$ with $B/tB \cong A$ such that
$B[1/t]$ is smooth over $k((t))$.
\end{lemma}

\begin{proof}
Write $A = k[x_1, \ldots, x_n]/(f_1, \ldots, f_c)$ as in
Algebra, Definition \ref{algebra-definition-lci-field}.
We are going to choose
$a_{ij} \in (t) \subset k[[t]]$ and set
$$
g_j = f_j + a_{0j} + a_{1j}x_1 + \ldots + a_{nj}x_n \in
k[[t]][x_1, \ldots, x_n]
$$
After doing this we take
$B = k[[t]][x_1, \ldots, x_n]/(g_1, \ldots, g_c)$.
We claim that $k[[t]] \to B$ is flat at every prime ideal
lying over $(t)$. Namely, the elements $f_1, \ldots, f_c$
form a regular sequence in the local ring at any prime ideal
$\mathfrak p$ of $k[x_1, \ldots, x_n]$ containing $f_1, \ldots, f_c$
(Algebra, Lemma \ref{algebra-lemma-lci}). Thus $g_1, \ldots, g_c$
is locally a lift of a regular sequence and we can apply
Algebra, Lemma \ref{algebra-lemma-grothendieck-regular-sequence}.
Flatness at primes lying over $(0) \subset k[[t]]$ is automatic
because $k((t)) = k[[t]]_{(0)}$ is a field. Thus $B$ is flat
over $k[[t]]$.

\medskip\noindent
All that remains is to show that for suitable choices
of $a_{ij}$ the generic fibre $B_{(0)}$ is smooth over
$k((t))$. For this we have to show that we can choose
our $a_{ij}$ so that the induced morphism
$$
(a_{ij}) : \Spec(k[[t]]) \longrightarrow \mathbf{A}^{c(n + 1)}_k
$$
maps into the nonempty Zariski open of
Lemma \ref{lemma-jouanolou-type-thing}.
This is clear because there is no nonzero polynomial in the
$a_{ij}$ which vanishes on $(t)^{\oplus c(n + 1)}$.
(We leave this as an exercise to the reader.)
\end{proof}

\begin{lemma}
\label{lemma-smoothing-artinian-lci}
Let $k$ be a field. Let $A$ be a finite dimensional $k$-algebra
which is a local complete intersection over $k$. Then there is
a finite flat $k[[t]]$-algebra $B$ with $B/tB \cong A$
and $B[1/t]$ \'etale over $k((t))$.
\end{lemma}

\begin{proof}
Since $A$ is Artinian
(Algebra, Lemma \ref{algebra-lemma-finite-dimensional-algebra}),
we can write $A$ as a product of local Artinian rings
(Algebra, Lemma \ref{algebra-lemma-artinian-finite-length}).
Thus it suffices to prove the lemma if $A$ is local
(this uses that being a local complete intersection is
preserved under taking principal localizations, see
Algebra, Lemma \ref{algebra-lemma-localize-lci}).
In this case $A$ is a global complete intersection.
Consider the algebra $B$ constructed in
Lemma \ref{lemma-smoothing-affine-lci}.
Then $k[[t]] \to B$ is quasi-finite at the unique prime of $B$
lying over $(t)$ (Algebra, Definition \ref{algebra-definition-quasi-finite}).
Observe that $k[[t]]$ is a henselian local ring
(Algebra, Lemma \ref{algebra-lemma-complete-henselian}).
Thus $B = B' \times C$ where $B'$ is finite over $k[[t]]$
and $C$ has no prime lying over $(t)$, see
Algebra, Lemma \ref{algebra-lemma-characterize-henselian}.
Then $B'$ is the ring we are looking for
(recall that \'etale is the same thing as
smooth of relative dimension $0$).
\end{proof}

\begin{lemma}
\label{lemma-smoothing-at-lci-point}
Let $k$ be a field. Let $A$ be a $k$-algebra. Assume
\begin{enumerate}
\item $A$ is a local ring essentially of finite type over $k$,
\item $A$ is a complete intersection over $k$
(Algebra, Definition \ref{algebra-definition-lci-local-ring}).
\end{enumerate}
Set $d = \dim(A) + \text{trdeg}_k(\kappa)$ where $\kappa$
is the residue field of $A$. Then there exists an integer $n$
and a flat, essentially of finite type ring map
$k[[t]] \to B$ with $B/tB \cong A$ such that $t^n$ is in the
$d$th Fitting ideal of $\Omega_{B/k[[t]]}$.
\end{lemma}

\begin{proof}
By Algebra, Lemma \ref{algebra-lemma-lci-local} we can write $A$ as the
localization at a prime $\mathfrak p$ of a global complete intersection $P$
over $k$. Observe that $\dim(P) = d$ by
Algebra, Lemma \ref{algebra-lemma-dimension-at-a-point-finite-type-field}.
By Lemma \ref{lemma-smoothing-affine-lci} we can find a
flat, finite type ring map $k[[t]] \to Q$ such that $P \cong Q/tQ$ and
such that $k((t)) \to Q[1/t]$ is smooth. It follows from the construction
of $Q$ in the lemma that $k[[t]] \to Q$ is a relative global
complete intersection of relative dimension $d$; alternatively,
Algebra, Lemma \ref{algebra-lemma-syntomic} tells us that $Q$ or a
suitable principal localization of $Q$ is such a global complete intersection.
Hence by Divisors, Lemma \ref{divisors-lemma-d-fitting-ideal-omega-smooth}
the $d$th Fitting ideal $I \subset Q$ of $\Omega_{Q/k[[t]]}$
cuts out the singular locus of $\Spec(Q) \to \Spec(k[[t]])$.
Thus $t^n \in I$ for some $n$.
Let $\mathfrak q \subset Q$
be the inverse image of $\mathfrak p$. Set $B = Q_\mathfrak q$.
The lemma is proved.
\end{proof}

\begin{lemma}
\label{lemma-smoothing-proper-curve-isolated-lci}
Let $X$ be a scheme over a field $k$. Assume
\begin{enumerate}
\item $X$ is proper over $k$,
\item $X$ is a local complete intersection over $k$,
\item $X$ has dimension $\leq 1$, and
\item $X \to \Spec(k)$ is smooth except at finitely many points.
\end{enumerate}
Then there exists a flat projective morphism $Y \to \Spec(k[[t]])$
whose generic fibre is smooth and whose special fibre is
isomorphic to $X$.
\end{lemma}

\begin{proof}
Observe that $X$ is Cohen-Macaulay, see
Algebra, Lemma \ref{algebra-lemma-lci-CM}.
Thus $X = X' \amalg X''$ with $\dim(X') = 0$
and $X''$ equidimensional of dimension $1$, see Morphisms, Lemma
\ref{morphisms-lemma-flat-finite-presentation-CM-fibres-relative-dimension}.
Since $X'$ is finite over $k$ (Varieties, Lemma
\ref{varieties-lemma-algebraic-scheme-dim-0})
we can find $Y' \to \Spec(k[[t]])$ with special
fibre $X'$ and generic fibre smooth by
Lemma \ref{lemma-smoothing-artinian-lci}.
Thus it suffices to prove the lemma for $X''$.
After replacing $X$ by $X''$ we have $X$ is
Cohen-Macaulay and equidimensional of dimension $1$.

\medskip\noindent
We are going to use deformation theory for the situation $\Lambda = k \to k$.
Let $p_1, \ldots, p_r \in X$ be the closed singular points of $X$, i.e.,
the points where $X \to \Spec(k)$ isn't smooth. For each $i$ we pick
an integer $n_i$ and a flat, essentially of finite type ring map
$$
k[[t]] \longrightarrow B_i
$$
with $B_i/tB_i \cong \mathcal{O}_{X, p_i}$ such that
$t^{n_i}$ is in the $1$st Fitting ideal of $\Omega_{B_i/k[[t]]}$.
This is possible by Lemma \ref{lemma-smoothing-at-lci-point}.
Observe that the system $(B_i/t^nB_i)$ defines a formal object of
$\Deformationcategory_{\mathcal{O}_{X, p_i}}$ over $k[[t]]$.
By Lemma \ref{lemma-curve-isolated} the map
$$
\Deformationcategory_X
\longrightarrow
\prod\nolimits_{i = 1, \ldots, r} \Deformationcategory_{\mathcal{O}_{X, p_i}}
$$
is a smooth map between deformation categories. Hence by
Formal Deformation Theory, Lemma
\ref{formal-defos-lemma-smooth-morphism-essentially-surjective}
there exists a formal object $(X_n)$ in $\Deformationcategory_X$
mapping to the formal object $\prod_i (B_i/t^n)$ by the arrow above.
By More on Morphisms of Spaces, Lemma
\ref{spaces-more-morphisms-lemma-formal-algebraic-space-proper-reldim-1}
there exists a projective scheme $Y$ over $k[[t]]$ and compatible
isomorphisms $Y \times_{\Spec(k[[t]])} \Spec(k[t]/(t^n)) \cong X_n$.
By More on Morphisms, Lemma
\ref{more-morphisms-lemma-check-flatness-on-infinitesimal-nbhds}
we see that $Y \to \Spec(k[[t]])$ is flat.
Since $X$ is Cohen-Macaulay and equidimensional of dimension $1$
we may apply Lemma \ref{lemma-criterion-smoothing}
to check $Y$ has smooth generic fibre\footnote{Warning: in general it is
{\bf not} true that the local ring of $Y$ at the point
$p_i$ is isomorphic to $B_i$. We only know that this is true after
dividing by $t^n$ on both sides!}.
Choose $n$ strictly larger than the maximum of the integers $n_i$ found above.
It we can show $t^{n - 1}$ is in the first Fitting ideal of
$\Omega_{X_n/S_n}$ with $S_n = \Spec(k[t]/(t^n))$, then the proof is done.
To do this it suffices to prove this is true in each of
the local rings of $X_n$ at closed points $p$.
However, if $p$ corresponds to a smooth point for $X \to \Spec(k)$,
then $\Omega_{X_n/S_n, p}$ is free of rank $1$ and the first Fitting
ideal is equal to the local ring. If $p = p_i$ for some $i$, then
$$
\Omega_{X_n/S_n, p_i} =
\Omega_{(B_i/t^nB_i)/(k[t]/(t^n))} =
\Omega_{B_i/k[[t]]}/t^n\Omega_{B_i/k[[t]]}
$$
Since taking Fitting ideals commutes with base change
(with already used this but in this algebraic setting
it follows from More on Algebra, Lemma
\ref{more-algebra-lemma-fitting-ideal-basics}),
and since $n - 1 \geq n_i$ we see that $t^{n - 1}$ is
in the Fitting ideal of this module over $B_i/t^nB_i$ as desired.
\end{proof}

\begin{lemma}
\label{lemma-smoothing-curve-isolated-lci}
Let $k$ be a field and let $X$ be a scheme over $k$. Assume
\begin{enumerate}
\item $X$ is separated, finite type over $k$ and $\dim(X) \leq 1$,
\item $X$ is a local complete intersection over $k$, and
\item $X \to \Spec(k)$ is smooth except at finitely many points.
\end{enumerate}
Then there exists a flat, separated, finite type morphism $Y \to \Spec(k[[t]])$
whose generic fibre is smooth and whose special fibre is
isomorphic to $X$.
\end{lemma}

\begin{proof}
If $X$ is reduced, then we can choose an embedding
$X \subset \overline{X}$ as in
Varieties, Lemma \ref{varieties-lemma-reduced-dim-1-projective-completion}.
Writing $X = \overline{X} \setminus \{x_1, \ldots, x_n\}$
we see that $\mathcal{O}_{\overline{X}, x_i}$ is a discrete
valuation ring and hence in particular a local complete intersection
(Algebra, Definition \ref{algebra-definition-lci-local-ring}).
Thus $\overline{X}$ is a local complete intersection
over $k$ because this holds over the open $X$ and
at the points $x_i$ by Algebra, Lemma \ref{algebra-lemma-lci-local}.
Thus we may apply Lemma \ref{lemma-smoothing-proper-curve-isolated-lci}
to find a projective flat morphism $\overline{Y} \to \Spec(k[[t]])$
whose generic fibre is smooth and whose special fibre
is $\overline{X}$. Then we remove $x_1, \ldots, x_n$
from $\overline{Y}$ to obtain $Y$.

\medskip\noindent
In the general case, write $X = X' \amalg X''$ where
with $\dim(X') = 0$ and $X''$ equidimensional of dimension $1$.
Then $X''$ is reduced and the first paragraph applies to it.
On the other hand, $X'$ can be dealt with
as in the proof of Lemma \ref{lemma-smoothing-proper-curve-isolated-lci}.
Some details omitted.
\end{proof}





\begin{multicols}{2}[\section{Other chapters}]
\noindent
Preliminaries
\begin{enumerate}
\item \hyperref[introduction-section-phantom]{Introduction}
\item \hyperref[conventions-section-phantom]{Conventions}
\item \hyperref[sets-section-phantom]{Set Theory}
\item \hyperref[categories-section-phantom]{Categories}
\item \hyperref[topology-section-phantom]{Topology}
\item \hyperref[sheaves-section-phantom]{Sheaves on Spaces}
\item \hyperref[sites-section-phantom]{Sites and Sheaves}
\item \hyperref[stacks-section-phantom]{Stacks}
\item \hyperref[fields-section-phantom]{Fields}
\item \hyperref[algebra-section-phantom]{Commutative Algebra}
\item \hyperref[brauer-section-phantom]{Brauer Groups}
\item \hyperref[homology-section-phantom]{Homological Algebra}
\item \hyperref[derived-section-phantom]{Derived Categories}
\item \hyperref[simplicial-section-phantom]{Simplicial Methods}
\item \hyperref[more-algebra-section-phantom]{More on Algebra}
\item \hyperref[smoothing-section-phantom]{Smoothing Ring Maps}
\item \hyperref[modules-section-phantom]{Sheaves of Modules}
\item \hyperref[sites-modules-section-phantom]{Modules on Sites}
\item \hyperref[injectives-section-phantom]{Injectives}
\item \hyperref[cohomology-section-phantom]{Cohomology of Sheaves}
\item \hyperref[sites-cohomology-section-phantom]{Cohomology on Sites}
\item \hyperref[dga-section-phantom]{Differential Graded Algebra}
\item \hyperref[dpa-section-phantom]{Divided Power Algebra}
\item \hyperref[hypercovering-section-phantom]{Hypercoverings}
\end{enumerate}
Schemes
\begin{enumerate}
\setcounter{enumi}{24}
\item \hyperref[schemes-section-phantom]{Schemes}
\item \hyperref[constructions-section-phantom]{Constructions of Schemes}
\item \hyperref[properties-section-phantom]{Properties of Schemes}
\item \hyperref[morphisms-section-phantom]{Morphisms of Schemes}
\item \hyperref[coherent-section-phantom]{Cohomology of Schemes}
\item \hyperref[divisors-section-phantom]{Divisors}
\item \hyperref[limits-section-phantom]{Limits of Schemes}
\item \hyperref[varieties-section-phantom]{Varieties}
\item \hyperref[topologies-section-phantom]{Topologies on Schemes}
\item \hyperref[descent-section-phantom]{Descent}
\item \hyperref[perfect-section-phantom]{Derived Categories of Schemes}
\item \hyperref[more-morphisms-section-phantom]{More on Morphisms}
\item \hyperref[flat-section-phantom]{More on Flatness}
\item \hyperref[groupoids-section-phantom]{Groupoid Schemes}
\item \hyperref[more-groupoids-section-phantom]{More on Groupoid Schemes}
\item \hyperref[etale-section-phantom]{\'Etale Morphisms of Schemes}
\end{enumerate}
Topics in Scheme Theory
\begin{enumerate}
\setcounter{enumi}{40}
\item \hyperref[chow-section-phantom]{Chow Homology}
\item \hyperref[intersection-section-phantom]{Intersection Theory}
\item \hyperref[weil-section-phantom]{Weil Cohomology Theories}
\item \hyperref[pic-section-phantom]{Picard Schemes of Curves}
\item \hyperref[adequate-section-phantom]{Adequate Modules}
\item \hyperref[dualizing-section-phantom]{Dualizing Complexes}
\item \hyperref[duality-section-phantom]{Duality for Schemes}
\item \hyperref[discriminant-section-phantom]{Discriminants and Differents}
\item \hyperref[local-cohomology-section-phantom]{Local Cohomology}
\item \hyperref[algebraization-section-phantom]{Algebraic and Formal Geometry}
\item \hyperref[curves-section-phantom]{Algebraic Curves}
\item \hyperref[resolve-section-phantom]{Resolution of Surfaces}
\item \hyperref[models-section-phantom]{Semistable Reduction}
\item \hyperref[pione-section-phantom]{Fundamental Groups of Schemes}
\item \hyperref[etale-cohomology-section-phantom]{\'Etale Cohomology}
\item \hyperref[crystalline-section-phantom]{Crystalline Cohomology}
\item \hyperref[proetale-section-phantom]{Pro-\'etale Cohomology}
\item \hyperref[more-etale-section-phantom]{More \'Etale Cohomology}
\item \hyperref[trace-section-phantom]{The Trace Formula}
\end{enumerate}
Algebraic Spaces
\begin{enumerate}
\setcounter{enumi}{59}
\item \hyperref[spaces-section-phantom]{Algebraic Spaces}
\item \hyperref[spaces-properties-section-phantom]{Properties of Algebraic Spaces}
\item \hyperref[spaces-morphisms-section-phantom]{Morphisms of Algebraic Spaces}
\item \hyperref[decent-spaces-section-phantom]{Decent Algebraic Spaces}
\item \hyperref[spaces-cohomology-section-phantom]{Cohomology of Algebraic Spaces}
\item \hyperref[spaces-limits-section-phantom]{Limits of Algebraic Spaces}
\item \hyperref[spaces-divisors-section-phantom]{Divisors on Algebraic Spaces}
\item \hyperref[spaces-over-fields-section-phantom]{Algebraic Spaces over Fields}
\item \hyperref[spaces-topologies-section-phantom]{Topologies on Algebraic Spaces}
\item \hyperref[spaces-descent-section-phantom]{Descent and Algebraic Spaces}
\item \hyperref[spaces-perfect-section-phantom]{Derived Categories of Spaces}
\item \hyperref[spaces-more-morphisms-section-phantom]{More on Morphisms of Spaces}
\item \hyperref[spaces-flat-section-phantom]{Flatness on Algebraic Spaces}
\item \hyperref[spaces-groupoids-section-phantom]{Groupoids in Algebraic Spaces}
\item \hyperref[spaces-more-groupoids-section-phantom]{More on Groupoids in Spaces}
\item \hyperref[bootstrap-section-phantom]{Bootstrap}
\item \hyperref[spaces-pushouts-section-phantom]{Pushouts of Algebraic Spaces}
\end{enumerate}
Topics in Geometry
\begin{enumerate}
\setcounter{enumi}{76}
\item \hyperref[spaces-chow-section-phantom]{Chow Groups of Spaces}
\item \hyperref[groupoids-quotients-section-phantom]{Quotients of Groupoids}
\item \hyperref[spaces-more-cohomology-section-phantom]{More on Cohomology of Spaces}
\item \hyperref[spaces-simplicial-section-phantom]{Simplicial Spaces}
\item \hyperref[spaces-duality-section-phantom]{Duality for Spaces}
\item \hyperref[formal-spaces-section-phantom]{Formal Algebraic Spaces}
\item \hyperref[restricted-section-phantom]{Restricted Power Series}
\item \hyperref[spaces-resolve-section-phantom]{Resolution of Surfaces Revisited}
\end{enumerate}
Deformation Theory
\begin{enumerate}
\setcounter{enumi}{84}
\item \hyperref[formal-defos-section-phantom]{Formal Deformation Theory}
\item \hyperref[defos-section-phantom]{Deformation Theory}
\item \hyperref[cotangent-section-phantom]{The Cotangent Complex}
\item \hyperref[examples-defos-section-phantom]{Deformation Problems}
\end{enumerate}
Algebraic Stacks
\begin{enumerate}
\setcounter{enumi}{88}
\item \hyperref[algebraic-section-phantom]{Algebraic Stacks}
\item \hyperref[examples-stacks-section-phantom]{Examples of Stacks}
\item \hyperref[stacks-sheaves-section-phantom]{Sheaves on Algebraic Stacks}
\item \hyperref[criteria-section-phantom]{Criteria for Representability}
\item \hyperref[artin-section-phantom]{Artin's Axioms}
\item \hyperref[quot-section-phantom]{Quot and Hilbert Spaces}
\item \hyperref[stacks-properties-section-phantom]{Properties of Algebraic Stacks}
\item \hyperref[stacks-morphisms-section-phantom]{Morphisms of Algebraic Stacks}
\item \hyperref[stacks-limits-section-phantom]{Limits of Algebraic Stacks}
\item \hyperref[stacks-cohomology-section-phantom]{Cohomology of Algebraic Stacks}
\item \hyperref[stacks-perfect-section-phantom]{Derived Categories of Stacks}
\item \hyperref[stacks-introduction-section-phantom]{Introducing Algebraic Stacks}
\item \hyperref[stacks-more-morphisms-section-phantom]{More on Morphisms of Stacks}
\item \hyperref[stacks-geometry-section-phantom]{The Geometry of Stacks}
\end{enumerate}
Topics in Moduli Theory
\begin{enumerate}
\setcounter{enumi}{102}
\item \hyperref[moduli-section-phantom]{Moduli Stacks}
\item \hyperref[moduli-curves-section-phantom]{Moduli of Curves}
\end{enumerate}
Miscellany
\begin{enumerate}
\setcounter{enumi}{104}
\item \hyperref[examples-section-phantom]{Examples}
\item \hyperref[exercises-section-phantom]{Exercises}
\item \hyperref[guide-section-phantom]{Guide to Literature}
\item \hyperref[desirables-section-phantom]{Desirables}
\item \hyperref[coding-section-phantom]{Coding Style}
\item \hyperref[obsolete-section-phantom]{Obsolete}
\item \hyperref[fdl-section-phantom]{GNU Free Documentation License}
\item \hyperref[index-section-phantom]{Auto Generated Index}
\end{enumerate}
\end{multicols}


\bibliography{my}
\bibliographystyle{amsalpha}

\end{document}
