\documentclass{amsart}

% The following AMS packages are automatically loaded with amsart 
% documentclass:
%\usepackage{amsmath}
%\usepackage{amssymb}
%\usepackage{amsthm}

% For commutative diagrams you can use
% \usepackage{amscd}
% but Jason prefers xypic
\usepackage[all]{xy}

% To put source file link in headers.
% Change "template.tex" to "this_filename.tex"
\usepackage{fancyhdr}
\pagestyle{fancy}
\lhead{}
\chead{}
\rhead{Source file: \url{src/injectives.tex}}
\lfoot{}
\cfoot{\thepage}
\rfoot{}
\renewcommand{\headrulewidth}{0pt}
\renewcommand{\footrulewidth}{0pt}
\renewcommand{\headheight}{12pt}

% For cross-file-references
\usepackage{xr-hyper}

% Package for hypertext links:
\usepackage[colorlinks=true]{hyperref}
% For any local file, say "hello.tex" you want to refer to please use
% \externaldocument[hello-]{hello}
\externaldocument[conventions-]{conventions}
\externaldocument[sets-]{sets}
\externaldocument[hypercovering-]{hypercovering}

% The macro \autoref uses the macros \figurename, etc.
% We list the default values and we change some of them
% to start with a captial.
% Figure	\figurename
% Table		\tablename
% Part		\partname
% Appendix	\appendixname
% Equation	\equationname
% item		\Itemname
% \renewcommand{\Itemname}{Item}
\renewcommand{\Itemautorefname}{Item}
% chapter	\Chaptername
% \renewcommand{\Chaptername}{Chapter}
% \renewcommand{\Chapterautorefname}{Chapter}
% section	\sectionname
\renewcommand{\sectionname}{Section}
\renewcommand{\sectionautorefname}{Section}
% subsection	\subsectionname
\renewcommand{\subsectionname}{Subsection}
\renewcommand{\subsectionautorefname}{Subsection}
% subsubsection	\subsubsectionname
\renewcommand{\subsubsectionname}{Subsubsection}
\renewcommand{\subsubsectionautorefname}{Subsubsection}
% paragraph	\paragraphname
\renewcommand{\paragraphname}{Paragraph}
\renewcommand{\paragraphautorefname}{Paragraph}
% footnote	\Hfootnotename
% \renewcommand{\Hfootnotename}{Footnote}
\renewcommand{\Hfootnoteautorefname}{Footnote}
% Equation	\AMSname
% Theorem	\theoremname


% Theorem environments.
%
\newtheorem{theorem}{Theorem}[subsection]
\newtheorem{proposition}[theorem]{Proposition}
\newtheorem{lemma}[theorem]{Lemma}

\theoremstyle{definition}
\newtheorem{definition}[theorem]{Definition}
\newtheorem{example}[theorem]{Example}
\newtheorem{exercise}[theorem]{Exercise}
\newtheorem{situation}[theorem]{Situation}

\theoremstyle{remark}
\newtheorem{remark}[theorem]{Remark}
\newtheorem{remarks}[theorem]{Remarks}

\numberwithin{equation}{subsection}


% OK, start here.
%
\begin{document}

\title{Injectives}

%\begin{abstract}
%\end{abstract}

\maketitle
\thispagestyle{fancy}

\tableofcontents

\section{Introduction}
\label{section-introduction}

\noindent
We will use the existence of sufficiently many injectives to
do cohomology of abelian sheaves on a site. So we briefly 
explain why there are enough injectives. At the end we explain
the more general story.

\section{Existence of injectives in special cases}
\label{section-foundational}

\noindent
Grothendieck proved the existence of injectives in great generality 
in the paper \cite{Tohoku}. We will prove this is true for abelian
(pre)sheaves on a site.

\subsection{Modules}
\label{subsection-injectives-modules}

\noindent 
As an example theorem let us try to prove that there are enough injective
modules over a ring $R$. We start with the fact that $\mathbf{Q}/\mathbf{Z}$ 
is an injective abelian group. This we leave to the reader.

\smallskip\noindent
For any ring $R$ and any $R$-module $M$ over $R$ we denote 
$M^\wedge = \text{Hom}(M,\mathbf{Q}/\mathbf{Z})$
with its natural $R$-module structure. Also we denote
$F(M)$ the free module with basis given by the elements 
of $M$ and we let $F(M)\to M$ be the natural surjection of $R$-modules.

\smallskip\noindent
Note that there is a canonical map $M \to (M^\wedge)^\wedge$.
This is injective; you can check this using that 
$\mathbf{Q}/\mathbf{Z}$ is injective. There is a canonical 
injection $(M^\wedge)^\wedge \to (F(M^\wedge))^\wedge$.
Set $J(M)=(F(M^\wedge))^\wedge$. The composition of the two maps
above gives $M \to J(M)$. This will be the desired injection of $M$ 
into an injective $R$-module.

\smallskip\noindent
Note that $J(M) \cong \prod_{m\in M} R^\wedge$ as an $R$-module.
As the product of injective modules is injective, it suffices to
show that $R^\wedge$ is injective. For this you use that 
$\text{Hom}_R(N, R^\wedge) = \text{Hom}_R(R, N^\wedge)$ and the
fact that $(-)^\wedge$ is an exact functor.

\smallskip\noindent
The proof above gives us the best possible situation. Not only does
every module inject into an injective module, but the construction is
completely functorial. Namely, for any map of $R$-modules $M \to N$ 
there is an associated morphism $J(M) \to J(N)$ making the following 
diagram commute:
$$
\xymatrix{
M \ar[d] \ar[r] & N \ar[d] \\
J(M) \ar[r] & J(N) }
$$
This the kind of construction we would like to have in general.

\subsubsection{Categories of modules}
\label{subsubsection-category-modules}

\noindent
As a consequence we obtain a category of modules with a canonical
resolution. For an ordinal $\alpha$ we denote $\text{Mod}_{R,\alpha}$
the category of modules contained in $V_\alpha$ (see Sets, 
\hyperref[sets-subsection-sets-hierarchy]%
{Subsection~\ref*{sets-subsection-sets-hierarchy}}).

\begin{lemma}
\label{lemma-injective-module-preserves-category}
For any given set of $R$-modules $\{M_i\}_{i\in I}$ there exists an ordinal
$\alpha$ such that $M_i \in \text{Ob}(\text{Mod}_{R,\alpha})$,
$\forall i\in I$ and such that for any
$M \in \text{Ob}(\text{Mod}_{R,\alpha})$ we have
$J(M) \in \text{Ob}(\text{Mod}_{R,\alpha})$.
\end{lemma}

\begin{proof}
Consider the formula $\phi(M)$: ``$M$ is an $R$-module and there exists
an $R$-module $N$ such that $N=J(M)$''. Apply the reflection principle to
$\phi(M)$, see 
\hyperref[sets-theorem-reflection-principle]%
{Theorem~\ref*{sets-theorem-reflection-principle}}. (Use $T = \{M_i\}$.)
The result follows.
\end{proof}

\noindent
Some remarks are in order. First we observe that the modules $J(M)$
are injective in the absolute sense, and not only injective in the
category $\text{Mod}_{R,\alpha}$. Second, in exactly the same way we
can make sure that $\text{Mod}_{R,\alpha}$ has all finite limits,
finite direct sums, or countable sums and products, etc. Of course
the category $\text{mod}_{R,\alpha}$ never has arbitrary direct sums,
which is why working with $\text{mod}_{R,\alpha}$ is somewhat cumbersome.

\subsubsection{Projective resolutions}
\label{subsubsection-projective-resolution}

\noindent
FIXME: Remove probably?

\noindent
For any set $S$ we let $F(S)$ denote the free $R$-module on $S$.
Then any left $R$-module has the following two step resolution
$$
F(M\times M) \oplus F(R\times M) \to F(M) \to M \to 0.
$$
The first map is given by the rule
$$
[m1,m2] \oplus [r,m] \mapsto [m1+m2]-[m1]-[m2]+[rm]-r[m].
$$
The nice thing about this is that it is absolutely canonical.
Sometimes we write $S_1 = M\times M \coprod R\times M$ and
$S_0=M$, so that the resolution is $F(S_1) \to F(S_0) \to M \to 0$.

\subsection{Abelian presheaves}
\label{subsection-injectives-presheaves}

\noindent
Let $\mathcal{C}$ be a category. On the one hand, consider abelian
presheaves on $\mathcal{C}$. On the other hand, consider
families of abelian groups indexed by elements of
$\text{Ob}(\mathcal{C})$; in other words presheaves on the discrete
category with underlying set of objects $\text{Ob}(\mathcal{C})$.
We will denote presheaves on $\mathcal{C}$ by $B$ and presheaves on
$\text{Ob}(\mathcal{C})$ by $A$. Consider the forgetful functor $v$,
denoted $B \mapsto vB$.

\smallskip\noindent
There is a functor $u$ that assigns a presheaf on $\mathcal{C}$
to a presheaf on $\text{Ob}(\mathcal{C})$. It is defined as follows:
$$
\Gamma(U, uA) = \prod\nolimits_{U' \to U} A(U').
$$
So an element is a family $(a_\phi)_\phi$ with $\phi$
ranging through all morphisms in $\mathcal{C}$ with target $U$.
The restriction map on $uA$ corresponding to $g : V \to U$
maps our element $(a_\phi)_\phi$ to the element 
$(a_{g \circ \psi})_\psi$. 

\smallskip\noindent
There is a canonical surjective map $vuA \to A$ and a canonical map
injective map $B \to uvB$. We leave it to the reader to show that
$$
\text{Mor}_{\text{PAb}(\text{Ob}(\mathcal{C}))}(B, uA) =
\text{Mor}_{\text{PAb}(\mathcal{C})}(vB, A).
$$
(Obvious notation.) Thus the pair $(u,v)$ is an example of a pair of adjoint
functors. FIXME: Discuss this somewhere. It is clear that $u$ and $v$ are exact functors. It is clear that any presheaf on $\text{Ob}(\mathcal{C})$ has an
injective hull. In fact there is a functor $J$ such that
$A \mapsto \big(A \to J(A)\big)$ is functorial as in
\autoref{subsection-injectives-modules}.
(Namely, $J(A)$ is the assignment $U\mapsto J(A(U))$, where
$J(A(U))$ is the functor constructed in
Subsection \ref{subsection-injectives-modules} for the ring $\mathbf{Z}$
applied to the $\mathbf{Z}$-module $A(U)$.)

\smallskip\noindent
Putting all of this together gives us the following procedure
for embedding objects $B$ of $\text{PAb}(\mathcal{C}))$ into
an injective object: $B \to uJ(vB)$.

\begin{proposition}
\label{proposition-presheaves-injectives}
For abelian presheaves on a category there is a functorial injective hull.
\end{proposition}

\subsubsection{Categories of presheaves of abelian groups}
\label{subsubsection-category-presheaves}

\noindent
Arguing as in the proof of
Lemma \ref{lemma-injective-module-preserves-category} we obtain a category
with an injective resolution functor. For any ordinal $\alpha$,
we use the notation $\text{PAb}(\mathcal{C})_\alpha$ to denote the category
of presheaves $\mathcal{F}$ of abelian groups with $\mathcal{F} \in V_\alpha$.
See Sets, \hyperref[sets-subsection-sets-hierarchy]%
{Subsection~\ref*{sets-subsection-sets-hierarchy}}.

\begin{lemma} 
\label{lemma-injective-presheaf-preserves-category}
Given any set of abelian presheaves $\mathcal{F}_i$, $i\in I$, there
exists an ordinal $\alpha$ such that $\text{PAb}(\mathcal{C})_\alpha$
contains all of the $\mathcal{F}_i$, and such that there is a functor
$\text{PAb}(\mathcal{C})_\alpha \to
\text{Arrows}(\text{PAb}(\mathcal{C})_\alpha)$
of the form $\mathcal{F} \mapsto (\mathcal{F} \to J(\mathcal{F}))$
with the property that $\mathcal{F} \to J(\mathcal{F})$ is an injective
hull for all $\mathcal{F} \in \text{PAb}(\mathcal{C})_\alpha$.
\end{lemma}

\begin{proof}
FIXME. Very similar to the corresponding lemma for modules.
\end{proof}

\subsection{Abelian Sheaves}
\label{subsection-injectives-sheaves}

\noindent
Let $\mathcal{C}$ be a site. In this section we prove that there are 
enough injectives for abelian sheaves on $\mathcal{C}$. 

\smallskip\noindent
Denote $i$ the forgetfull functor from sheaves to presheaves. Let
$\#$ denote the sheafification functor, see FIXME. In this subsection we
will use that $i(\mathcal{F})^\# = \mathcal{F}$, see FIXME.
Finally, let $\mathcal{F} \to J(\mathcal{F})$ denote the canonical
embedding into an injective presheaf we found in 
\autoref{subsection-injectives-presheaves}. 

\smallskip\noindent
For any sheaf $\mathcal{F}$ in $\text{Ab}(\mathcal{C})$ and
any ordinal $\beta$ we define a sheaf
$J_\beta(\mathcal{F})$ by transfinite induction.
FIXME: explain transfinite induction in \url{src/sets.tex}.
First we set $J_0(\mathcal{F})=\mathcal{F}$.
We define $J_1(\mathcal{F})=J(i(\mathcal{F}))^\#$;
there is a map $\mathcal{F}=i(\mathcal{F})^\# \to J(i\mathcal{F})^\#$
by functoriality of $\#$. This map $\mathcal{F} \to J_1(\mathcal{F})$
is injective as $\#$ is exact. We set $J_{\alpha+1}=J_1(J_\alpha)$, 
and for a limit ordinal $\beta$, we define
$$
J_\beta(B) = \lim_{\alpha < \beta} J_\alpha(B).
$$
FIXME: limit notation.

\begin{lemma}
\label{lemma-map-into-next-one}
With notation as above.
Suppose that $\mathcal{G}_1 \to \mathcal{G}_2$ is an injective
map of abelian sheaves on $\mathcal{C}$. Let $\alpha$ be an ordinal
and let $\mathcal{G}_1\to J_\alpha(\mathcal{F})$ be a morphism
of sheaves. There exists a morphism $\mathcal{G}_2 \to
J_{\alpha+1}(\mathcal{F})$ such that the following diagram commutes
$$
\xymatrix{
\mathcal{G}_1 \ar[d] \ar[r] & \mathcal{G}_2 \ar[d] \\
J_{\alpha}(\mathcal{F}) \ar[r] & J_{\alpha+1}(\mathcal{F}) }
$$
\end{lemma}

\begin{proof}
FIXME.
\end{proof}

\noindent
This lemma says that somehow the system $\{J_{\alpha}(\mathcal{F})\}$
is the injective hull of $\mathcal{F}$ that we are looking for. Of course
we cannot take the limit over all $\alpha$ because they form a class
and not a set. However, the idea is now that you don't have to check
injectivity on all injections $\mathcal{G}_1 \to \mathcal{G}_2$, plus
the following lemma.

\begin{lemma}
\label{lemma-map-into-smaller}
Suppose that $\mathcal{G}_i$, $i\in I$ is set of sheaves of abelian 
groups on $\mathcal{C}$. There exists an ordinal $\beta$ such that
for any sheaf $\mathcal{F}$, any $i\in I$, and any map $\varphi : 
\mathcal{G}_i \to J_\beta(\mathcal{F})$ there exists an 
$\alpha < \beta$ such that $ \varphi $ factors through 
$J_\alpha(\mathcal{F})$.
\end{lemma}

\begin{proof}
(You can reduce this to the case of a single sheaf $\mathcal{G}$
by taking the direct sum of all the $\mathcal{G}_i$.)
FIXME. Hint: First suppose that $T = \lim_{\alpha < \beta} T_\alpha$
is a limit of sets and that $\varphi : S \to T$ is a map of sets. 
Then $\varphi$ lifts to a map into $T_\alpha$ for some $\alpha < \beta$
provided that $\beta$ is not a limit of ordinals indexed by $S$.
In other words, you pick $\beta$ to be a cardinal with cofinality
$cf(\beta)$ bigger than the cardinality of $S$; for example you can take 
$\beta = \aleph_{\alpha+1}$. Reference? Use this and
some argument for equalizers to get through.
\end{proof}

\noindent
Recall that for an object $X$ of $\mathcal{C}$ we denote $\mathbf{Z}_X$ 
the presheaf of abelian groups $\Gamma(U, \mathbf{Z}_X) = 
\oplus_{U \to X} \mathbf{Z}$. FIXME: should be introduced in
\url{src/sites.tex}. The sheaf associated to this presheaf
is denoted $\mathbf{Z}_X^\#$.

\begin{lemma}
\label{lemma-characterize-injectives}
Suppose $\mathcal{J}$ is a sheaf of abelian groups with the following
property: For all $X\in \text{Ob}(\mathcal{C})$, for all subsheaves
$\mathcal{G} \subset \mathbf{Z}_X^\#$ and for all morphisms
$\varphi : \mathcal{G} \to \mathcal{J}$, there exists an morphism
$\mathbf{Z}_X^\# \to \mathcal{J}$ extending $\varphi$.
Then $\mathcal{J}$ is an injective sheaf of abelian groups.
\end{lemma}

\begin{proof}
FIXME.
\end{proof}

\begin{theorem}
\label{theorem-sheaves-injectives}
The category of abelian sheaves has enough injectives (in the
strongest possible sense).
\end{theorem}

\begin{proof}
FIXME. Idea: Let $\mathcal{G}_i$, $i\in I$ be a set of abelian
sheaves such that every subsheaf of every $\mathbf{Z}_X^\#$
occurs as one of the $\mathcal{G}_i$. Apply
Lemma \ref{lemma-map-into-smaller} to this collection to
get an ordinal $\beta$. We claim that for any sheaf of abelian
groups $\mathcal{F}$ the map $\mathcal{F} \to J_\beta(\mathcal{F})$
is an injection of $\mathcal{F}$ into an injective.
Note that by construction the assigment $\mathcal{F} \mapsto
\big(\mathcal{F} \to J_\beta(\mathcal{F})\big)$ is functorial.

\smallskip\noindent
The proof of the claim comes from the fact that by
Lemma \ref{lemma-characterize-injectives} it suffices to extend any
morphism $\gamma : \mathcal{G} \to J_\beta(\mathcal{F})$ 
from a subsheaf $\mathcal{G}$ of some $\mathbf{Z}_X^\#$ to all of
$\mathbf{Z}_X^\#$. Then by Lemma \ref{lemma-map-into-smaller} the
map $\gamma$ lifts into $J_\alpha(\mathcal{F})$ for some
$\alpha < \beta$. Finally, we apply Lemma \ref{lemma-map-into-next-one}
to get the desired extension of $\gamma$ to a morphism
into $J_{\alpha+1}(\mathcal{F}) \to J_\beta(\mathcal{F})$.
\end{proof}

\subsubsection{Categories of abelian sheaves}
\label{subsubsection-abelian-sheaves}

\noindent
Again we obtain a result concerning the existence of a category 
preserved by the functorial assigment $\mathcal{F} \mapsto
\big(\mathcal{F} \to J_\beta(\mathcal{F})\big)$ described in
Theorem \ref{theorem-sheaves-injectives}. As is usual, for an
ordinal $\alpha$, we denote $\text{Ab}(\mathcal{C})_\alpha$ the
category of abelian sheaves on $\mathcal{C}$ which are elements
of $V_\alpha$.

\begin{lemma}
\label{lemma-injective-sheaf-preserves-category}
Given any set of abelian sheaves $\mathcal{F}_i$, $i\in I$, there
exists an ordinal $\alpha$ such that $\text{Ab}(\mathcal{C})_\alpha$
contains all of the $\mathcal{F}_i$, and such that there is a functor
$\text{Ab}(\mathcal{C})_\alpha \to
\text{Arrows}(\text{Ab}(\mathcal{C})_\alpha)$
of the form $\mathcal{F} \mapsto (\mathcal{F} \to J(\mathcal{F}))$
with the property that $\mathcal{F} \to J(\mathcal{F})$ is an injective
hull for all $\mathcal{F} \in \text{Ab}(\mathcal{C})_\alpha$.
\end{lemma}

\begin{proof}
FIXME. Very similar to the corresponding lemma for modules.
\end{proof}

\section{Grothendieck categories and injectives}

\noindent
Here we can talk in general about generators, limits and
all that good stuff. This will possibly be needed later on
anyway. FIXME.

\smallskip\noindent
To continue reading, 
\begin{enumerate}

\item visit the next section: Hypercoverings,
\autoref{hypercovering-section-introduction}, or 

\item go back to the
table of contents: \url{index.html#contents}.

\end{enumerate}



\bibliography{my}
\bibliographystyle{alpha}

\end{document}
