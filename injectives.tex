\IfFileExists{stacks-project.cls}{%
\documentclass{stacks-project}
}{%
\documentclass{amsart}
}

% The following AMS packages are automatically loaded with
% the amsart documentclass:
%\usepackage{amsmath}
%\usepackage{amssymb}
%\usepackage{amsthm}

% For dealing with references we use the comment environment
\usepackage{verbatim}
\newenvironment{reference}{\comment}{\endcomment}
%\newenvironment{reference}{}{}
\newenvironment{slogan}{\comment}{\endcomment}
\newenvironment{history}{\comment}{\endcomment}

% For commutative diagrams you can use
% \usepackage{amscd}
\usepackage[all]{xy}

% We use 2cell for 2-commutative diagrams.
\xyoption{2cell}
\UseAllTwocells

% To put source file link in headers.
% Change "template.tex" to "this_filename.tex"
% \usepackage{fancyhdr}
% \pagestyle{fancy}
% \lhead{}
% \chead{}
% \rhead{Source file: \url{template.tex}}
% \lfoot{}
% \cfoot{\thepage}
% \rfoot{}
% \renewcommand{\headrulewidth}{0pt}
% \renewcommand{\footrulewidth}{0pt}
% \renewcommand{\headheight}{12pt}

\usepackage{multicol}

% For cross-file-references
\usepackage{xr-hyper}

% Package for hypertext links:
\usepackage{hyperref}

% For any local file, say "hello.tex" you want to link to please
% use \externaldocument[hello-]{hello}
\externaldocument[introduction-]{introduction}
\externaldocument[conventions-]{conventions}
\externaldocument[sets-]{sets}
\externaldocument[categories-]{categories}
\externaldocument[topology-]{topology}
\externaldocument[sheaves-]{sheaves}
\externaldocument[sites-]{sites}
\externaldocument[stacks-]{stacks}
\externaldocument[fields-]{fields}
\externaldocument[algebra-]{algebra}
\externaldocument[brauer-]{brauer}
\externaldocument[homology-]{homology}
\externaldocument[derived-]{derived}
\externaldocument[simplicial-]{simplicial}
\externaldocument[more-algebra-]{more-algebra}
\externaldocument[smoothing-]{smoothing}
\externaldocument[modules-]{modules}
\externaldocument[sites-modules-]{sites-modules}
\externaldocument[injectives-]{injectives}
\externaldocument[cohomology-]{cohomology}
\externaldocument[sites-cohomology-]{sites-cohomology}
\externaldocument[dga-]{dga}
\externaldocument[dpa-]{dpa}
\externaldocument[hypercovering-]{hypercovering}
\externaldocument[schemes-]{schemes}
\externaldocument[constructions-]{constructions}
\externaldocument[properties-]{properties}
\externaldocument[morphisms-]{morphisms}
\externaldocument[coherent-]{coherent}
\externaldocument[divisors-]{divisors}
\externaldocument[limits-]{limits}
\externaldocument[varieties-]{varieties}
\externaldocument[topologies-]{topologies}
\externaldocument[descent-]{descent}
\externaldocument[perfect-]{perfect}
\externaldocument[more-morphisms-]{more-morphisms}
\externaldocument[flat-]{flat}
\externaldocument[groupoids-]{groupoids}
\externaldocument[more-groupoids-]{more-groupoids}
\externaldocument[etale-]{etale}
\externaldocument[chow-]{chow}
\externaldocument[intersection-]{intersection}
\externaldocument[pic-]{pic}
\externaldocument[adequate-]{adequate}
\externaldocument[dualizing-]{dualizing}
\externaldocument[duality-]{duality}
\externaldocument[discriminant-]{discriminant}
\externaldocument[local-cohomology-]{local-cohomology}
\externaldocument[curves-]{curves}
\externaldocument[resolve-]{resolve}
\externaldocument[models-]{models}
\externaldocument[pione-]{pione}
\externaldocument[etale-cohomology-]{etale-cohomology}
\externaldocument[proetale-]{proetale}
\externaldocument[crystalline-]{crystalline}
\externaldocument[spaces-]{spaces}
\externaldocument[spaces-properties-]{spaces-properties}
\externaldocument[spaces-morphisms-]{spaces-morphisms}
\externaldocument[decent-spaces-]{decent-spaces}
\externaldocument[spaces-cohomology-]{spaces-cohomology}
\externaldocument[spaces-limits-]{spaces-limits}
\externaldocument[spaces-divisors-]{spaces-divisors}
\externaldocument[spaces-over-fields-]{spaces-over-fields}
\externaldocument[spaces-topologies-]{spaces-topologies}
\externaldocument[spaces-descent-]{spaces-descent}
\externaldocument[spaces-perfect-]{spaces-perfect}
\externaldocument[spaces-more-morphisms-]{spaces-more-morphisms}
\externaldocument[spaces-flat-]{spaces-flat}
\externaldocument[spaces-groupoids-]{spaces-groupoids}
\externaldocument[spaces-more-groupoids-]{spaces-more-groupoids}
\externaldocument[bootstrap-]{bootstrap}
\externaldocument[spaces-pushouts-]{spaces-pushouts}
\externaldocument[groupoids-quotients-]{groupoids-quotients}
\externaldocument[spaces-more-cohomology-]{spaces-more-cohomology}
\externaldocument[spaces-simplicial-]{spaces-simplicial}
\externaldocument[formal-spaces-]{formal-spaces}
\externaldocument[restricted-]{restricted}
\externaldocument[spaces-resolve-]{spaces-resolve}
\externaldocument[formal-defos-]{formal-defos}
\externaldocument[defos-]{defos}
\externaldocument[cotangent-]{cotangent}
\externaldocument[examples-defos-]{examples-defos}
\externaldocument[algebraic-]{algebraic}
\externaldocument[examples-stacks-]{examples-stacks}
\externaldocument[stacks-sheaves-]{stacks-sheaves}
\externaldocument[criteria-]{criteria}
\externaldocument[artin-]{artin}
\externaldocument[quot-]{quot}
\externaldocument[stacks-properties-]{stacks-properties}
\externaldocument[stacks-morphisms-]{stacks-morphisms}
\externaldocument[stacks-limits-]{stacks-limits}
\externaldocument[stacks-cohomology-]{stacks-cohomology}
\externaldocument[stacks-perfect-]{stacks-perfect}
\externaldocument[stacks-introduction-]{stacks-introduction}
\externaldocument[stacks-more-morphisms-]{stacks-more-morphisms}
\externaldocument[stacks-geometry-]{stacks-geometry}
\externaldocument[moduli-]{moduli}
\externaldocument[moduli-curves-]{moduli-curves}
\externaldocument[examples-]{examples}
\externaldocument[exercises-]{exercises}
\externaldocument[guide-]{guide}
\externaldocument[desirables-]{desirables}
\externaldocument[coding-]{coding}
\externaldocument[obsolete-]{obsolete}
\externaldocument[fdl-]{fdl}
\externaldocument[index-]{index}

% Theorem environments.
%
\theoremstyle{plain}
\newtheorem{theorem}[subsection]{Theorem}
\newtheorem{proposition}[subsection]{Proposition}
\newtheorem{lemma}[subsection]{Lemma}

\theoremstyle{definition}
\newtheorem{definition}[subsection]{Definition}
\newtheorem{example}[subsection]{Example}
\newtheorem{exercise}[subsection]{Exercise}
\newtheorem{situation}[subsection]{Situation}

\theoremstyle{remark}
\newtheorem{remark}[subsection]{Remark}
\newtheorem{remarks}[subsection]{Remarks}

\numberwithin{equation}{subsection}

% Macros
%
\def\lim{\mathop{\rm lim}\nolimits}
\def\colim{\mathop{\rm colim}\nolimits}
\def\Spec{\mathop{\rm Spec}}
\def\Hom{\mathop{\rm Hom}\nolimits}
\def\Ext{\mathop{\rm Ext}\nolimits}
\def\SheafHom{\mathop{\mathcal{H}\!{\it om}}\nolimits}
\def\SheafExt{\mathop{\mathcal{E}\!{\it xt}}\nolimits}
\def\Sch{\textit{Sch}}
\def\Mor{\mathop{\rm Mor}\nolimits}
\def\Ob{\mathop{\rm Ob}\nolimits}
\def\Sh{\mathop{\textit{Sh}}\nolimits}
\def\NL{\mathop{N\!L}\nolimits}
\def\proetale{{pro\text{-}\acute{e}tale}}
\def\etale{{\acute{e}tale}}
\def\QCoh{\textit{QCoh}}
\def\Ker{\mathop{\rm Ker}}
\def\Im{\mathop{\rm Im}}
\def\Coker{\mathop{\rm Coker}}
\def\Coim{\mathop{\rm Coim}}

%
% Macros for moduli stacks/spaces
%
\def\QCohstack{\mathcal{QC}\!{\it oh}}
\def\Cohstack{\mathcal{C}\!{\it oh}}
\def\Spacesstack{\mathcal{S}\!{\it paces}}
\def\Quotfunctor{{\rm Quot}}
\def\Hilbfunctor{{\rm Hilb}}
\def\Curvesstack{\mathcal{C}\!{\it urves}}
\def\Polarizedstack{\mathcal{P}\!{\it olarized}}
\def\Complexesstack{\mathcal{C}\!{\it omplexes}}
% \Pic is the operator that assigns to X its picard group, usage \Pic(X)
% \Picardstack_{X/B} denotes the Picard stack of X over B
% \Picardfunctor_{X/B} denotes the Picard functor of X over B
\def\Pic{\mathop{\rm Pic}\nolimits}
\def\Picardstack{\mathcal{P}\!{\it ic}}
\def\Picardfunctor{{\rm Pic}}
\def\Deformationcategory{\mathcal{D}\!{\it ef}}


% OK, start here.
%
\begin{document}

\title{Injectives}


\maketitle

\tableofcontents

\section{Introduction}
\label{section-introduction}

\noindent
We will use the existence of sufficiently many injectives to
do cohomology of abelian sheaves on a site. So we briefly 
explain why there are enough injectives. At the end we explain
the more general story.

\section{Abelian groups}
\label{section-abelian-groups}

\noindent
In this section we show the category of abelian groups has enough
injectives. Recall that an abelian group $M$ is {\it divisible}
if and only if for every $x \in M$ and every $n \in \mathbf{N}$
there exists a $y \in M$ such that $n y = x$.

\begin{lemma}
\label{lemma-injective-abelian}
An abelian group $J$ is an injective object in 
the category of abelian groups if and only if $J$
is divisible.
\end{lemma}

\begin{proof}
Suppose that $J$ is not divisible. Then there exists
an $x \in J$ and $n \in \mathbf{N}$ such that there
is no $y \in J$ with $n y = x$. Then the morphism
$\mathbf{Z} \to J$, $m \mapsto mx$ does not extend
to $\frac{1}{n}\mathbf{Z} \supset \mathbf{Z}$. Hence
$J$ is not injective.

\medskip\noindent
Let $A \subset B$ be abelian groups.
Assume that $J$ is a divisible abelian group.
Let $\varphi : A \to J$ be a morphism. 
Consider the set of homomorphisms $\varphi' : A' \to J$
with $A \subset A' \subset B$ and $\varphi'|_A = \varphi$.
Define $(A', \varphi') \geq (A'', \varphi'')$ if
and only if $A' \supset A''$ and $\varphi'|_{A''} = \varphi''$.
If $(A_i, \varphi_i)_{i \in I}$ is a totally
ordered collection of such pairs, then we obtain a map
$\bigcup_{i \in I} A_i \to J$ defined by $a \in A_i$
maps to $\varphi_i(a)$. Thus Zorn's lemma applies.
To conclude we have to show that if the pair
$(A', \varphi')$ is maximal then $A' = B$.
In other words, it suffices to show, given
any subgroup $A \subset B$, $A \not = B$ and
any $\varphi : A \to J$, then we can find
$\varphi' : A' \to J$ with $A \subset A' \subset B$
such that (a) the inclusion $A \subset A'$ is strict, and
(b) the morphism $\varphi'$ extends $\varphi$.

\medskip\noindent
To prove this, pick $x \in B$, $x \not \in A$.
If there exists no $n\in \mathbf{N}$ such that
$nx \in A$, then $A \oplus \mathbf{Z} \cong A + \mathbf{Z}x$.
Hence we can extend $\varphi$ to $A' = A + \mathbf{Z}x$
by using $\varphi$ on $A$ and mapping $x$ to zero for example.
If there does exist an $n \in \mathbf{N}$ such that
$nx \in A$, then let $n$ be the minimal such integer.
Let $z \in J$ be an element such that $nz = \varphi(nx)$.
Define a morphism $\tilde\varphi : A \oplus \mathbf{Z} \to J$ by
$(a, m) \mapsto \varphi(a) + mz$. By our choice of
$z$ the kernel of $\tilde \varphi$ contains the kernel
of the map $A \oplus \mathbf{Z} \to B$,
$(a, m) \mapsto a + mx$. Hence $\tilde \varphi$ factors
through the image $A' = A + \mathbf{Z}x$, and this extends the morphism
$\varphi$.
\end{proof}

\noindent
We can use this lemma to show that every abelian
group can be embbeded in a injective abelian
group. But this is a special case of the result of
the following section.

\section{Modules}
\label{section-injectives-modules}

\noindent 
As an example theorem let us try to prove that there are enough injective
modules over a ring $R$. We start with the fact that $\mathbf{Q}/\mathbf{Z}$ 
is an injective abelian group. This follows from
Lemma \ref{lemma-injective-abelian} above.

\begin{definition}
\label{definition-simple-functors}
Let $R$ be a ring.
\begin{enumerate}
\item For any $R$-module $M$ over $R$ we denote 
$M^\vee = \text{Hom}(M,\mathbf{Q}/\mathbf{Z})$
with its natural $R$-module structure. We think
of {\it $M \mapsto M^\vee$} as a contravariant functor
from the category of $R$-modules to itself.
\item For any $R$-module $M$ we denote
$$
F(M) = \bigoplus\nolimits_{m \in M} R[m]
$$
the {\it free module} with basis given by the elements $[m]$ with
$m \in M$. We let $F(M)\to M$, $\sum f_i [m_i] \mapsto \sum f_i m_i$
be the natural surjection of $R$-modules.
We think of $M \mapsto (F(M) \to M)$ as a functor from
the category of $R$-modules to the category of
arrows in $R$-modules.
\end{enumerate}
\end{definition}

\begin{lemma}
\label{lemma-vee-exact}
Let $R$ be a ring.
The functor $M \mapsto M^\vee$ is exact.
\end{lemma}

\begin{proof}
This because $\mathbf{Q}/\mathbf{Z}$
is an injective abelian group.
\end{proof}

\noindent
There is a canonical map $ev : M \to (M^\vee)^\vee$
given by evaluation: given $x \in M$ we let
$ev(x) \in (M^\vee)^\vee = \text{Hom}(M^\vee, \mathbf{Q}/\mathbf{Z})$
be the map $\varphi \mapsto \varphi(x)$.

\begin{lemma}
\label{lemma-ev-injective}
For any $R$-module $M$ the evaluation map
$ev : M \to (M^\vee)^\vee$ is injective.
\end{lemma}

\begin{proof}
You can check this using that $\mathbf{Q}/\mathbf{Z}$ is an injective
abelian group. Namely, if $x \in M$ is not zero, then let
$M' \subset M$ be the cyclic group it generates. There exists
a nonzero map $M' \to \mathbf{Q}/\mathbf{Z}$ which necessarily does
not annihilate $x$. This extends to
a map $\varphi : M \to \mathbf{Q}/\mathbf{Z}$
And then $ev(x)(\varphi) = \varphi(x) \not = 0$.
\end{proof}

\noindent
The canonical surjection $F(M) \to M$ of $R$-modules turns into a
a canonical injection, see above, of $R$-modules 
$$
(M^\vee)^\vee \longrightarrow (F(M^\vee))^\vee.
$$
Set $J(M) = (F(M^\vee))^\vee$. The composition of $ev$ with this
the displayed map gives $M \to J(M)$ functorially in $M$.

\begin{lemma}
\label{lemma-JM-injective}
Let $R$ be a ring. For every $R$-module $M$ the
$R$-module $J(M)$ is injective.
\end{lemma}

\begin{proof}
Note that $J(M) \cong \prod_{m\in M} R^\vee$ as an $R$-module.
As the product of injective modules is injective, it suffices to
show that $R^\vee$ is injective. For this we use that 
$$
\text{Hom}_R(N, R^\vee) =
\text{Hom}_R(N, \text{Hom}_{\mathbf{Z}}(R, \mathbf{Q}/\mathbf{Z})) =
N^\vee
$$
and the
fact that $(-)^\vee$ is an exact functor by Lemma
\ref{lemma-vee-exact}.
\end{proof}

\begin{lemma}
\label{lemma-injectives-modules}
Let $R$ be a ring.
The construction above defines a covariant functor
$M \mapsto (M \to J(M))$ from the category of
$R$-modules to the category of arrows of $R$-modules
such that for every module $M$ the output
$M \to J(M)$ is an injective map of $M$ into
an injective $R$-module $J(M)$.
\end{lemma}

\begin{proof}
Follows from the above.
\end{proof}

\noindent
In particular, for any map of $R$-modules $M \to N$ 
there is an associated morphism $J(M) \to J(N)$
making the following diagram commute:
$$
\xymatrix{
M \ar[d] \ar[r] & N \ar[d] \\
J(M) \ar[r] & J(N) }
$$
This the kind of construction we would like to have in general.
In Homology, Section \ref{homology-section-injectives}
we introduced terminology to express this. Namely,
we say this means that the category of $R$-modules
has functorial injective embeddings.

\section{Projective resolutions}
\label{section-projective-resolution}

\noindent
Totally unimportant. Skip this section.

\medskip\noindent
For any set $S$ we let $F(S)$ denote the free $R$-module on $S$.
Then any left $R$-module has the following two step resolution
$$
F(M\times M) \oplus F(R\times M) \to F(M) \to M \to 0.
$$
The first map is given by the rule
$$
[m_1, m_2] \oplus [r, m] \mapsto [m_1 + m_2] - [m_1] - [m_2] + [rm] - r[m].
$$


\section{Abelian presheaves}
\label{section-injectives-presheaves}

\noindent
Let $\mathcal{C}$ be a category. Recall that this means that
$\text{Ob}(\mathcal{C})$ is a set. On the one hand, consider abelian
presheaves on $\mathcal{C}$, see
Sites, Section \ref{sites-section-presheaves}.
On the other hand, consider families of abelian groups
indexed by elements of $\text{Ob}(\mathcal{C})$; in other
words presheaves on the discrete category with underlying set
of objects $\text{Ob}(\mathcal{C})$. Let us denote this
discrete category simply $\text{Ob}(\mathcal{C})$.
There is a natural functor
$$
i : \text{Ob}(\mathcal{C}) \longrightarrow \mathcal{C}
$$
and hence there is a natural restriction or forgetful functor
$$
v = i^p : 
\textit{PAb}(\mathcal{C})
\longrightarrow
\textit{PAb}(\text{Ob}(\mathcal{C}))
$$
compare Sites, Section \ref{sites-section-functoriality-PSh}.
We will denote presheaves
on $\mathcal{C}$ by $B$ and presheaves on
$\text{Ob}(\mathcal{C})$ by $A$.

\medskip\noindent
There is also a functor $u = i_p$ (compare
Sites, Section \ref{sites-section-functoriality-PSh})
that assigns an abelian presheaf on $\mathcal{C}$
to an abelian presheaf on $\text{Ob}(\mathcal{C})$.
In this particular case it is defined as follows:
$$
uA(U) = \prod\nolimits_{U' \to U} A(U').
$$
So an element is a family $(a_\phi)_\phi$ with $\phi$
ranging through all morphisms in $\mathcal{C}$ with target $U$.
The restriction map on $uA$ corresponding to $g : V \to U$
maps our element $(a_\phi)_\phi$ to the element 
$(a_{g \circ \psi})_\psi$. 

\medskip\noindent
There is a canonical surjective map $vuA \to A$ and a canonical
injective map $B \to uvB$. We leave it to the reader to show that
$$
\text{Mor}_{\textit{PAb}(\text{Ob}(\mathcal{C}))}(B, uA)
=
\text{Mor}_{\textit{PAb}(\mathcal{C})}(vB, A).
$$
in this simple case; the general case is in
Sites, Section \ref{sites-section-functoriality-PSh}.
Thus the pair $(u,v)$ is an example of a pair of adjoint
functors, see
Categories, Section \ref{categories-section-adjoint}.

\medskip\noindent
At this point we can list the following facts
about the situation above.
\begin{enumerate}
\item The functors $u$ and $v$ are exact. This follows from
the explicit description of these functors given above.
\item In particular the functor $v$ transforms injective maps
into injective maps.
\item The category $\textit{PAb}(\text{Ob}(\mathcal{C}))$
has enough injectives.
\item In fact there is a functorial injective embedding
$A \mapsto \big(A \to J(A)\big)$ as in
Homology, Definition \ref{homology-definition-functorial-injective-embedding}.
Namely, we can take $J(A)$ to be the
presheaf $U\mapsto J(A(U))$, where
$J(-)$ is the functor constructed in
Section \ref{section-injectives-modules} for the ring $\mathbf{Z}$.
\end{enumerate}
Putting all of this together gives us the following procedure
for embedding objects $B$ of $\text{PAb}(\mathcal{C}))$ into
an injective object: $B \to uJ(vB)$. See 
Homology, Lemma \ref{homology-lemma-adjoint-functorial-injectives}.

\begin{proposition}
\label{proposition-presheaves-injectives}
For abelian presheaves on a category there is a functorial injective
embedding.
\end{proposition}

\begin{proof}
See discussion above.
\end{proof}

\section{Abelian Sheaves}
\label{section-injectives-sheaves}

\noindent
Let $\mathcal{C}$ be a site. In this section we prove that there are 
enough injectives for abelian sheaves on $\mathcal{C}$. 

\medskip\noindent
Denote $i$ the forgetfull functor from abelian sheaves to
abelian presheaves
$$
i : \textit{Ab}(\mathcal{C}) \longrightarrow \textit{PAb}(\mathcal{C})
$$
Let ${}^\#$ denote the sheafification functor. Recall that
${}^\#$ is a left adjoint to $i$
and that  $i\mathcal{F}^\# = \mathcal{F}$ for any abelian
sheaf $\mathcal{F}$.
Finally, let $\mathcal{G} \to J(\mathcal{G})$ denote the canonical
embedding into an injective presheaf we found in 
Section \ref{section-injectives-presheaves}. 

\medskip\noindent
For any sheaf $\mathcal{F}$ in $\text{Ab}(\mathcal{C})$ and
any ordinal $\beta$ we define a sheaf
$J_\beta(\mathcal{F})$ by transfinite induction.
We set $J_0(\mathcal{F}) = \mathcal{F}$.
We define $J_1(\mathcal{F})=J(i\mathcal{F})^\#$.
There is a map
$$
\mathcal{F} \longrightarrow
i\mathcal{F}^\# \longrightarrow
J(i\mathcal{F})^\# = J_1(\mathcal{F})
$$
by functoriality of $\#$. This map $\mathcal{F} \to J_1(\mathcal{F})$
is injective as $\#$ is exact. We set
$J_{\alpha + 1}(\mathcal{F}) = J_1(J_\alpha(\mathcal{F}))$.
So that there are canonical injective maps
$J_\alpha(\mathcal{F}) \to J_{\alpha + 1}(\mathcal{F})$.
For a limit ordinal $\beta$, we define
$$
J_\beta(\mathcal{F}) = \text{colim}_{\alpha < \beta}\ J_\alpha(\mathcal{F}).
$$
Note that this is a directed colimit.

\begin{lemma}
\label{lemma-map-into-next-one}
With notation as above.
Suppose that $\mathcal{G}_1 \to \mathcal{G}_2$ is an injective
map of abelian sheaves on $\mathcal{C}$. Let $\alpha$ be an ordinal
and let $\mathcal{G}_1 \to J_\alpha(\mathcal{F})$ be a morphism
of sheaves. There exists a morphism $\mathcal{G}_2 \to
J_{\alpha+1}(\mathcal{F})$ such that the following diagram commutes
$$
\xymatrix{
\mathcal{G}_1 \ar[d] \ar[r] & \mathcal{G}_2 \ar[d] \\
J_{\alpha}(\mathcal{F}) \ar[r] & J_{\alpha+1}(\mathcal{F}) }
$$
\end{lemma}

\begin{proof}
This is because the map $i\mathcal{G}_1 \to i\mathcal{G}_2$ is injective
and hence $i\mathcal{G}_1 \to iJ_\alpha(\mathcal{F})$ extends to
$i\mathcal{G}_2 \to J(iJ_\alpha(\mathcal{F}))$ which gives the
desired map after applying the sheafification functor.
\end{proof}

\noindent
This lemma says that somehow the system $\{J_{\alpha}(\mathcal{F})\}$
is an injective embedding of $\mathcal{F}$. Of course
we cannot take the limit over all $\alpha$ because they form a class
and not a set. However, the idea is now that you don't have to check
injectivity on all injections $\mathcal{G}_1 \to \mathcal{G}_2$, plus
the following lemma.

\begin{lemma}
\label{lemma-map-into-smaller}
Suppose that $\mathcal{G}_i$, $i\in I$ is set of sheaves of abelian 
groups on $\mathcal{C}$. There exists an ordinal $\beta$ such that
for any sheaf $\mathcal{F}$, any $i\in I$, and any map $\varphi : 
\mathcal{G}_i \to J_\beta(\mathcal{F})$ there exists an 
$\alpha < \beta$ such that $ \varphi $ factors through 
$J_\alpha(\mathcal{F})$.
\end{lemma}

\begin{proof}
This reduces to the case of a single sheaf $\mathcal{G}$
by taking the direct sum of all the $\mathcal{G}_i$.

\medskip\noindent
First suppose that $T_\beta = \text{colim}_{\alpha < \beta}\ T_\alpha$
is a colimit of sets indexed by ordinals less than $\beta$,
say with injective transition maps.
Suppose that $\varphi : S \to T$ is a map of sets. 
Then $\varphi$ lifts to a map into $T_\alpha$ for some $\alpha < \beta$
provided that $\beta$ is not a limit of ordinals indexed by $S$.
In other words, if $\beta$ is an ordinal with cofinality
(see \cite[Chapter I, Definition 10.30]{Kunen}, or see
Sets, Section \ref{sets-section-cofinality})
strictly bigger than the cardinality of $S$ then this is the case.
 
\medskip\noindent
Now consider
$$
S = \coprod\nolimits_{U \in \text{Ob}(\mathcal{C})} \mathcal{G}(U).
$$
and 
$$
T_\beta
=
\coprod\nolimits_{U \in \text{Ob}(\mathcal{C})} J_\beta(\mathcal{F})(U)
$$
Then $T_\beta =  \text{colim}_{\alpha < \beta}\ T_\alpha$
with injective transition maps.
A morphism $\mathcal{G} \to J_\beta(\mathcal{F})$ factors
through $J_\alpha(\mathcal{F})$ if and only if
the associated map $S \to T_\beta$ factors through $T_\alpha$.
Applying the reasoning in the previous paragraph we see that if
the cofinality of $\beta$
is bigger than the cardinality
of $S$, then the result of the lemma is true. Hence the lemma
follows from the fact that there are ordinals with arbitrarily
large cofinality. For example, given a cardinal $\kappa$,
the cofinality of $2^\kappa$ is bigger than $\kappa$, see
\cite[Chapter I, Corollary 10.41]{Kunen}.
\end{proof}

\noindent
Recall that for an object $X$ of $\mathcal{C}$ we denote $\mathbf{Z}_X$ 
the presheaf of abelian groups $\Gamma(U, \mathbf{Z}_X) = 
\oplus_{U \to X} \mathbf{Z}$. The sheaf associated to this presheaf
is denoted $\mathbf{Z}_X^\#$. It can be characterized by
the property
$$
\text{Mor}_{\textit{Ab}(\mathcal{C})}(\mathbf{Z}_X^\#, \mathcal{G})
=
\mathcal{G}(X)
$$
where the element $\varphi$ of the left hand side is mapped
to $\varphi(1 \cdot \text{id}_X)$ in the right hand side.

\begin{lemma}
\label{lemma-characterize-injectives}
Suppose $\mathcal{J}$ is a sheaf of abelian groups with the following
property: For all $X\in \text{Ob}(\mathcal{C})$, for any abelian subsheaf
$\mathcal{S} \subset \mathbf{Z}_X^\#$ and any morphism
$\varphi : \mathcal{S} \to \mathcal{J}$, there exists a morphism
$\mathbf{Z}_X^\# \to \mathcal{J}$ extending $\varphi$.
Then $\mathcal{J}$ is an injective sheaf of abelian groups.
\end{lemma}

\begin{proof}
Let $\mathcal{F} \to \mathcal{G}$ be an injective map
of abelian sheaves. Suppose $\varphi : \mathcal{F} \to \mathcal{J}$
is a morphism. Arguing as in the proof of
Lemma \ref{lemma-injective-abelian} we see that it suffices
to prove that if $\mathcal{F} \not = \mathcal{G}$, then we
can find an abelian sheaf $\mathcal{F}'$,
$\mathcal{F} \subset \mathcal{F}' \subset \mathcal{G}$
such that (a) the inclusion $\mathcal{F} \subset \mathcal{F}'$ is strict,
and (b) $\varphi$ can be extended to $\mathcal{F}'$.
To see this, let $X$ be an object of $\mathcal{C}$ such
that the inclusion $\mathcal{F}(X) \subset \mathcal{G}(X)$
is strict. Pick $s \in \mathcal{G}(X)$, $s \not \in \mathcal{F}(X)$.
Let $\psi : \mathbf{Z}_X^\# \to \mathcal{G}$ be the morphism corresponding
to the section $s$ according to the formula just above the statement
of the lemma. Set
$$
\mathcal{S} = \psi^{-1}(\mathcal{G}).
$$
By assumption the morphism
$$
\mathcal{S} \xrightarrow{\psi} \mathcal{F} \xrightarrow{\varphi} \mathcal{J}
$$
can be extended to a morphism $\varphi' : \mathbf{Z}_X^\# \to \mathcal{J}$.
Note that $\varphi'$ is annihilates the kernel of $\psi$. Thus
$\varphi'$ gives rise to a morphism
$\varphi'' : \text{Im}(\psi) \to \mathcal{J}$
which agrees with $\varphi$ on the intersection
$\mathcal{F} \cap \text{Im}(\psi)$ by construction.
Thus $\varphi$ and $\varphi''$ glue to give an extension
of $\varphi$ to the strictly bigger subsheaf
$\mathcal{F} + \text{Im}(\psi)$.
\end{proof}

\begin{theorem}
\label{theorem-sheaves-injectives}
The category of sheaves of abelian groups on a
site has enough injectives. In fact there exists
a functorial injective embedding, see
Homology, Definition \ref{homology-definition-functorial-injective-embedding}.
\end{theorem}

\begin{proof}
Let $\mathcal{G}_i$, $i\in I$ be a set of abelian
sheaves such that every subsheaf of every $\mathbf{Z}_X^\#$
occurs as one of the $\mathcal{G}_i$. Apply
Lemma \ref{lemma-map-into-smaller} to this collection to
get an ordinal $\beta$. We claim that for any sheaf of abelian
groups $\mathcal{F}$ the map $\mathcal{F} \to J_\beta(\mathcal{F})$
is an injection of $\mathcal{F}$ into an injective.
Note that by construction the assigment
$\mathcal{F} \mapsto \big(\mathcal{F} \to J_\beta(\mathcal{F})\big)$
is indeed functorial.

\medskip\noindent
The proof of the claim comes from the fact that by
Lemma \ref{lemma-characterize-injectives} it suffices to extend any
morphism $\gamma : \mathcal{G} \to J_\beta(\mathcal{F})$ 
from a subsheaf $\mathcal{G}$ of some $\mathbf{Z}_X^\#$ to all of
$\mathbf{Z}_X^\#$. Then by Lemma \ref{lemma-map-into-smaller} the
map $\gamma$ lifts into $J_\alpha(\mathcal{F})$ for some
$\alpha < \beta$. Finally, we apply Lemma \ref{lemma-map-into-next-one}
to get the desired extension of $\gamma$ to a morphism
into $J_{\alpha+1}(\mathcal{F}) \to J_\beta(\mathcal{F})$.
\end{proof}


\section{$\mathcal{O}$-Modules}
\label{section-sheaves-modules}

\noindent
Let $\mathcal{C}$ be a site.
Let $\mathcal{O}$ be a sheaf of rings on $\mathcal{C}$.
By analogy with Section \ref{section-injectives-modules}
let us try to prove that there are enough injective
$\mathcal{O}$-modules. First of all, we pick an injective
embedding
$$
\bigoplus j_{U!}(\mathcal{O}|_U)/\mathcal{I} \longrightarrow \mathcal{J}
$$
where $\mathcal{J}$ is an injective abelian sheaf (which
exists by the previous section). Here the direct sum is
over all ideal sheaves $\mathcal{I} \subset j_{U!}(\mathcal{O})$
of all extensions by zero of restrictions of $\mathcal{O}$
top $U$.
Insert here: future reference to the functor of restriction and
extension by $0$ for $j_U : \mathcal{C}/U \to \mathcal{C}$.


\medskip\noindent
For any sheaf of $\mathcal{O}$-modules $\mathcal{F}$
denote
$$
\mathcal{F}^\vee
=
\textit{Hom}(\mathcal{F}, \mathcal{J})
$$
with its natural $\mathcal{O}$-module structure. 
Insert here future reference to internal hom.
We will also need
a canonical flat resolution of a sheaf of $\mathcal{O}$-modules.
This we can do as follows: For any $\mathcal{O}$-module
$\mathcal{F}$ we denote
$$
F(\mathcal{F})
=
\bigoplus\nolimits_{U \in \text{Ob}(\mathcal{C}), s \in \mathcal{F}(U)}
j_{U!}(\mathcal{O}|_U).
$$

\begin{lemma}
\label{lemma-vee-exact-sheaves}
The functor $\mathcal{F} \mapsto \mathcal{F}^\vee$ is exact.
\end{lemma}

\begin{proof}
This because $\mathcal{J}$ is an injective abelian sheaf.
\end{proof}

\noindent
There is a canonical map $ev : \mathcal{F} \to (\mathcal{F}^\vee)^\vee$
given by evaluation: given $x \in \mathcal{F}(U)$ we let
$ev(x) \in (\mathcal{F}^\vee)^\vee = 
\textit{Hom}(\mathcal{F}^\vee, \mathcal{J})$
be the map $\varphi \mapsto \varphi(x)$.

\begin{lemma}
\label{lemma-ev-injective-sheaves}
For any $\mathcal{O}$-module $\mathcal{F}$ the evaluation map
$ev : \mathcal{F} \to (\mathcal{F}^\vee)^\vee$ is injective.
\end{lemma}

\begin{proof}
You can check this using the definition of $\mathcal{J}$.
Namely, if $s \in \mathcal{F}(U)$ is not zero, then let
$j_{U!}(\mathcal{O}|_U) \to \mathcal{F}$ be the map of
$\mathcal{O}$-modules it generates. Then let $\mathcal{I}$
be the kernel of this map. There exists
a nonzero map $\mathcal{F} \supset j_{U!}(\mathcal{O}|_U)/\mathcal{I}
\to \mathcal{J}$ which does not annihilate $s$. This extends to a map
$\varphi : \mathcal{F} \to \mathcal{J}$
And then $ev(s)(\varphi) = \varphi(s) \not = 0$.
\end{proof}

\noindent
The canonical surjection
$F(\mathcal{F}) \to \mathcal{F}$ of $\mathcal{O}$-modules turns into a
a canonical injection, see above, of $\mathcal{O}$-modules 
$$
(\mathcal{F}^\vee)^\vee \longrightarrow (F(\mathcal{F}^\vee))^\vee.
$$
Set $J(\mathcal{F}) = (F(\mathcal{F}^\vee))^\vee$.
The composition of $ev$ with this
the displayed map gives
$\mathcal{F} \to J(\mathcal{F})$ functorially in $\mathcal{F}$.

\begin{lemma}
\label{lemma-JM-injective-sheaves}
Let $\mathcal{O}$ be a sheaf of rings.
For every $\mathcal{F}$-module $\mathcal{F}$ the
$\mathcal{F}$-module $J(\mathcal{F})$ is injective.
\end{lemma}

\begin{proof}
We have to show that the functor
$\text{Hom}_\mathcal{O}(\mathcal{G}, J(\mathcal{F}))$
is exact. Note that
\begin{eqnarray*}
\text{Hom}_\mathcal{O}(\mathcal{G}, J(\mathcal{F}))
& = &
\text{Hom}_\mathcal{O}(\mathcal{G}, (F(\mathcal{F}^\vee))^\vee) \\
& = &
\text{Hom}_\mathcal{O}
(\mathcal{G}, \textit{Hom}(F(\mathcal{F}^\vee), \mathcal{J})) \\
& = &
\text{Hom}(\mathcal{G} \otimes_{\mathcal{O}} F(\mathcal{F}^\vee), \mathcal{J})
\end{eqnarray*}
Thus what we want follows from the fact that $F(\mathcal{F}^\vee)$
is flat and $\mathcal{J}$ is injective.
\end{proof}

\begin{theorem}
\label{theorem-sheaves-modules-injectives}
Let $\mathcal{C}$ be a site.
Let $\mathcal{O}$ be a sheaf of rings on $\mathcal{C}$.
The category of sheaves of $\mathcal{O}$-modules on a
site has enough injectives. In fact there exists
a functorial injective embedding, see
Homology, Definition \ref{homology-definition-functorial-injective-embedding}.
\end{theorem}

\begin{proof}
From the discussion in this section.
\end{proof}




















\section{Grothendieck categories and injectives}

\noindent
Here we can talk in general about generators, limits and
all that good stuff. This will possibly be needed later on
anyway.

\medskip\noindent
Grothendieck proved the existence of injectives in great generality 
in the paper \cite{Tohoku}. 


\section{Other chapters}

\begin{multicols}{2}
\begin{enumerate}
\item \hyperref[introduction-section-phantom]{Introduction}
\item \hyperref[conventions-section-phantom]{Conventions}
\item \hyperref[sets-section-phantom]{Set Theory}
\item \hyperref[categories-section-phantom]{Categories}
\item \hyperref[topology-section-phantom]{Topology}
\item \hyperref[sheaves-section-phantom]{Sheaves on Spaces}
\item \hyperref[algebra-section-phantom]{Commutative Algebra}
\item \hyperref[sites-section-phantom]{Sites and Sheaves}
\item \hyperref[homology-section-phantom]{Homological Algebra}
\item \hyperref[derived-section-phantom]{Derived Categories}
\item \hyperref[more-algebra-section-phantom]{More Algebra}
\item \hyperref[simplicial-section-phantom]{Simplicial Methods}
\item \hyperref[modules-section-phantom]{Sheaves of Modules}
\item \hyperref[sites-modules-section-phantom]{Modules on Sites}
\item \hyperref[injectives-section-phantom]{Injectives}
\item \hyperref[cohomology-section-phantom]{Cohomology of Sheaves}
\item \hyperref[sites-cohomology-section-phantom]{Cohomology on Sites}
\item \hyperref[hypercovering-section-phantom]{Hypercoverings}
\item \hyperref[schemes-section-phantom]{Schemes}
\item \hyperref[constructions-section-phantom]{Constructions of Schemes}
\item \hyperref[properties-section-phantom]{Properties of Schemes}
\item \hyperref[morphisms-section-phantom]{Morphisms of Schemes}
\item \hyperref[coherent-section-phantom]{Coherent Cohomology}
\item \hyperref[divisors-section-phantom]{Divisors}
\item \hyperref[limits-section-phantom]{Limits of Schemes}
\item \hyperref[varieties-section-phantom]{Varieties}
\item \hyperref[chow-section-phantom]{Chow Homology}
\item \hyperref[topologies-section-phantom]{Topologies on Schemes}
\item \hyperref[descent-section-phantom]{Descent}
\item \hyperref[more-morphisms-section-phantom]{More on Morphisms}
\item \hyperref[flat-section-phantom]{More on Flatness}
\item \hyperref[groupoids-section-phantom]{Groupoid Schemes}
\item \hyperref[more-groupoids-section-phantom]{More on Groupoid Schemes}
\item \hyperref[etale-section-phantom]{\'Etale Morphisms of Schemes}
\item \hyperref[etale-cohomology-section-phantom]{\'Etale Cohomology}
\item \hyperref[spaces-section-phantom]{Algebraic Spaces}
\item \hyperref[spaces-properties-section-phantom]{Properties of Algebraic Spaces}
\item \hyperref[spaces-morphisms-section-phantom]{Morphisms of Algebraic Spaces}
\item \hyperref[spaces-topologies-section-phantom]{Topologies on Algebraic Spaces}
\item \hyperref[spaces-descent-section-phantom]{Descent and Algebraic Spaces}
\item \hyperref[spaces-more-morphisms-section-phantom]{More on Morphisms of Spaces}
\item \hyperref[quot-section-phantom]{Quot and Hilbert Spaces}
\item \hyperref[stacks-section-phantom]{Stacks}
\item \hyperref[spaces-groupoids-section-phantom]{Groupoids in Algebraic Spaces}
\item \hyperref[spaces-more-groupoids-section-phantom]{More on Groupoids in Spaces}
\item \hyperref[bootstrap-section-phantom]{Bootstrap}
\item \hyperref[examples-stacks-section-phantom]{Examples of Stacks}
\item \hyperref[groupoids-quotients-section-phantom]{Quotients of Groupoids}
\item \hyperref[algebraic-section-phantom]{Algebraic Stacks}
\item \hyperref[criteria-section-phantom]{Criteria for Representability}
\item \hyperref[stacks-properties-section-phantom]{Properties of Algebraic Stacks}
\item \hyperref[stacks-morphisms-section-phantom]{Morphisms of Algebraic Stacks}
\item \hyperref[examples-section-phantom]{Examples}
\item \hyperref[exercises-section-phantom]{Exercises}
\item \hyperref[guide-section-phantom]{Guide to Literature}
\item \hyperref[desirables-section-phantom]{Desirables}
\item \hyperref[coding-section-phantom]{Coding Style}
\item \hyperref[fdl-section-phantom]{GNU Free Documentation License}
\item \hyperref[index-section-phantom]{Auto Generated Index}
\end{enumerate}
\end{multicols}



\bibliography{my}
\bibliographystyle{alpha}

\end{document}
