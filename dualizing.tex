\IfFileExists{stacks-project.cls}{%
\documentclass{stacks-project}
}{%
\documentclass{amsart}
}

% The following AMS packages are automatically loaded with
% the amsart documentclass:
%\usepackage{amsmath}
%\usepackage{amssymb}
%\usepackage{amsthm}

% For dealing with references we use the comment environment
\usepackage{verbatim}
\newenvironment{reference}{\comment}{\endcomment}
%\newenvironment{reference}{}{}
\newenvironment{slogan}{\comment}{\endcomment}
\newenvironment{history}{\comment}{\endcomment}

% For commutative diagrams you can use
% \usepackage{amscd}
\usepackage[all]{xy}

% We use 2cell for 2-commutative diagrams.
\xyoption{2cell}
\UseAllTwocells

% To put source file link in headers.
% Change "template.tex" to "this_filename.tex"
% \usepackage{fancyhdr}
% \pagestyle{fancy}
% \lhead{}
% \chead{}
% \rhead{Source file: \url{template.tex}}
% \lfoot{}
% \cfoot{\thepage}
% \rfoot{}
% \renewcommand{\headrulewidth}{0pt}
% \renewcommand{\footrulewidth}{0pt}
% \renewcommand{\headheight}{12pt}

\usepackage{multicol}

% For cross-file-references
\usepackage{xr-hyper}

% Package for hypertext links:
\usepackage{hyperref}

% For any local file, say "hello.tex" you want to link to please
% use \externaldocument[hello-]{hello}
\externaldocument[introduction-]{introduction}
\externaldocument[conventions-]{conventions}
\externaldocument[sets-]{sets}
\externaldocument[categories-]{categories}
\externaldocument[topology-]{topology}
\externaldocument[sheaves-]{sheaves}
\externaldocument[sites-]{sites}
\externaldocument[stacks-]{stacks}
\externaldocument[fields-]{fields}
\externaldocument[algebra-]{algebra}
\externaldocument[brauer-]{brauer}
\externaldocument[homology-]{homology}
\externaldocument[derived-]{derived}
\externaldocument[simplicial-]{simplicial}
\externaldocument[more-algebra-]{more-algebra}
\externaldocument[smoothing-]{smoothing}
\externaldocument[modules-]{modules}
\externaldocument[sites-modules-]{sites-modules}
\externaldocument[injectives-]{injectives}
\externaldocument[cohomology-]{cohomology}
\externaldocument[sites-cohomology-]{sites-cohomology}
\externaldocument[dga-]{dga}
\externaldocument[dpa-]{dpa}
\externaldocument[hypercovering-]{hypercovering}
\externaldocument[schemes-]{schemes}
\externaldocument[constructions-]{constructions}
\externaldocument[properties-]{properties}
\externaldocument[morphisms-]{morphisms}
\externaldocument[coherent-]{coherent}
\externaldocument[divisors-]{divisors}
\externaldocument[limits-]{limits}
\externaldocument[varieties-]{varieties}
\externaldocument[topologies-]{topologies}
\externaldocument[descent-]{descent}
\externaldocument[perfect-]{perfect}
\externaldocument[more-morphisms-]{more-morphisms}
\externaldocument[flat-]{flat}
\externaldocument[groupoids-]{groupoids}
\externaldocument[more-groupoids-]{more-groupoids}
\externaldocument[etale-]{etale}
\externaldocument[chow-]{chow}
\externaldocument[intersection-]{intersection}
\externaldocument[pic-]{pic}
\externaldocument[adequate-]{adequate}
\externaldocument[dualizing-]{dualizing}
\externaldocument[duality-]{duality}
\externaldocument[discriminant-]{discriminant}
\externaldocument[local-cohomology-]{local-cohomology}
\externaldocument[curves-]{curves}
\externaldocument[resolve-]{resolve}
\externaldocument[models-]{models}
\externaldocument[pione-]{pione}
\externaldocument[etale-cohomology-]{etale-cohomology}
\externaldocument[proetale-]{proetale}
\externaldocument[crystalline-]{crystalline}
\externaldocument[spaces-]{spaces}
\externaldocument[spaces-properties-]{spaces-properties}
\externaldocument[spaces-morphisms-]{spaces-morphisms}
\externaldocument[decent-spaces-]{decent-spaces}
\externaldocument[spaces-cohomology-]{spaces-cohomology}
\externaldocument[spaces-limits-]{spaces-limits}
\externaldocument[spaces-divisors-]{spaces-divisors}
\externaldocument[spaces-over-fields-]{spaces-over-fields}
\externaldocument[spaces-topologies-]{spaces-topologies}
\externaldocument[spaces-descent-]{spaces-descent}
\externaldocument[spaces-perfect-]{spaces-perfect}
\externaldocument[spaces-more-morphisms-]{spaces-more-morphisms}
\externaldocument[spaces-flat-]{spaces-flat}
\externaldocument[spaces-groupoids-]{spaces-groupoids}
\externaldocument[spaces-more-groupoids-]{spaces-more-groupoids}
\externaldocument[bootstrap-]{bootstrap}
\externaldocument[spaces-pushouts-]{spaces-pushouts}
\externaldocument[groupoids-quotients-]{groupoids-quotients}
\externaldocument[spaces-more-cohomology-]{spaces-more-cohomology}
\externaldocument[spaces-simplicial-]{spaces-simplicial}
\externaldocument[formal-spaces-]{formal-spaces}
\externaldocument[restricted-]{restricted}
\externaldocument[spaces-resolve-]{spaces-resolve}
\externaldocument[formal-defos-]{formal-defos}
\externaldocument[defos-]{defos}
\externaldocument[cotangent-]{cotangent}
\externaldocument[examples-defos-]{examples-defos}
\externaldocument[algebraic-]{algebraic}
\externaldocument[examples-stacks-]{examples-stacks}
\externaldocument[stacks-sheaves-]{stacks-sheaves}
\externaldocument[criteria-]{criteria}
\externaldocument[artin-]{artin}
\externaldocument[quot-]{quot}
\externaldocument[stacks-properties-]{stacks-properties}
\externaldocument[stacks-morphisms-]{stacks-morphisms}
\externaldocument[stacks-limits-]{stacks-limits}
\externaldocument[stacks-cohomology-]{stacks-cohomology}
\externaldocument[stacks-perfect-]{stacks-perfect}
\externaldocument[stacks-introduction-]{stacks-introduction}
\externaldocument[stacks-more-morphisms-]{stacks-more-morphisms}
\externaldocument[stacks-geometry-]{stacks-geometry}
\externaldocument[moduli-]{moduli}
\externaldocument[moduli-curves-]{moduli-curves}
\externaldocument[examples-]{examples}
\externaldocument[exercises-]{exercises}
\externaldocument[guide-]{guide}
\externaldocument[desirables-]{desirables}
\externaldocument[coding-]{coding}
\externaldocument[obsolete-]{obsolete}
\externaldocument[fdl-]{fdl}
\externaldocument[index-]{index}

% Theorem environments.
%
\theoremstyle{plain}
\newtheorem{theorem}[subsection]{Theorem}
\newtheorem{proposition}[subsection]{Proposition}
\newtheorem{lemma}[subsection]{Lemma}

\theoremstyle{definition}
\newtheorem{definition}[subsection]{Definition}
\newtheorem{example}[subsection]{Example}
\newtheorem{exercise}[subsection]{Exercise}
\newtheorem{situation}[subsection]{Situation}

\theoremstyle{remark}
\newtheorem{remark}[subsection]{Remark}
\newtheorem{remarks}[subsection]{Remarks}

\numberwithin{equation}{subsection}

% Macros
%
\def\lim{\mathop{\rm lim}\nolimits}
\def\colim{\mathop{\rm colim}\nolimits}
\def\Spec{\mathop{\rm Spec}}
\def\Hom{\mathop{\rm Hom}\nolimits}
\def\Ext{\mathop{\rm Ext}\nolimits}
\def\SheafHom{\mathop{\mathcal{H}\!{\it om}}\nolimits}
\def\SheafExt{\mathop{\mathcal{E}\!{\it xt}}\nolimits}
\def\Sch{\textit{Sch}}
\def\Mor{\mathop{\rm Mor}\nolimits}
\def\Ob{\mathop{\rm Ob}\nolimits}
\def\Sh{\mathop{\textit{Sh}}\nolimits}
\def\NL{\mathop{N\!L}\nolimits}
\def\proetale{{pro\text{-}\acute{e}tale}}
\def\etale{{\acute{e}tale}}
\def\QCoh{\textit{QCoh}}
\def\Ker{\mathop{\rm Ker}}
\def\Im{\mathop{\rm Im}}
\def\Coker{\mathop{\rm Coker}}
\def\Coim{\mathop{\rm Coim}}

%
% Macros for moduli stacks/spaces
%
\def\QCohstack{\mathcal{QC}\!{\it oh}}
\def\Cohstack{\mathcal{C}\!{\it oh}}
\def\Spacesstack{\mathcal{S}\!{\it paces}}
\def\Quotfunctor{{\rm Quot}}
\def\Hilbfunctor{{\rm Hilb}}
\def\Curvesstack{\mathcal{C}\!{\it urves}}
\def\Polarizedstack{\mathcal{P}\!{\it olarized}}
\def\Complexesstack{\mathcal{C}\!{\it omplexes}}
% \Pic is the operator that assigns to X its picard group, usage \Pic(X)
% \Picardstack_{X/B} denotes the Picard stack of X over B
% \Picardfunctor_{X/B} denotes the Picard functor of X over B
\def\Pic{\mathop{\rm Pic}\nolimits}
\def\Picardstack{\mathcal{P}\!{\it ic}}
\def\Picardfunctor{{\rm Pic}}
\def\Deformationcategory{\mathcal{D}\!{\it ef}}


% OK, start here.
%
\begin{document}

\title{Dualizing Complexes}


\maketitle

\phantomsection
\label{section-phantom}

\tableofcontents

\section{Introduction}
\label{section-introduction}

\noindent
A reference is the book \cite{R+D}.

\medskip\noindent
The goals of this chapter are the following:
\begin{enumerate}
\item Define what it means to have a dualizing complex $\omega_A^\bullet$
over a Noetherian ring $A$, namely
\begin{enumerate}
\item we have $\omega_A^\bullet \in D^{+}(A)$,
\item the cohomology modules $H^i(\omega_A^\bullet)$ are
all finite $A$-modules,
\item $\omega_A^\bullet$ has finite injective dimension, and
\item we have $A \to R\Hom_A(\omega_A^\bullet, \omega_A^\bullet)$
is a quasi-isomorphism.
\end{enumerate}
\item List elementary properties of dualizing complexes.
\item Show a dualizing complex gives rise to a dimension function.
\item Show a dualizing complex gives rise to a good notion of a
reflexive hull.
\item Prove the finiteness theorem when a dualizing complex exists.
\end{enumerate}






\section{Essential surjections and injections}
\label{section-essential}

\noindent
We will mostly work in categories of modules, but we may as well make
the definition in general.

\begin{definition}
\label{definition-essential}
Let $\mathcal{A}$ be an abelian category.
\begin{enumerate}
\item An injection $A \subset B$ of $\mathcal{A}$ is {\it essential},
or we say that $B$ is an {\it essential extension of} $A$,
if every nonzero subobject $B' \subset B$ has nonzero intersection with $A$.
\item A surjection $f : A \to B$ of $\mathcal{A}$ is {\it essential}
if every proper subobject $A' \subset A$ we have $f(A') \not = B$.
\end{enumerate}
\end{definition}

\begin{lemma}
\label{lemma-union-essential-extensions}
Let $R$ be a ring. Let $M$ be an $R$-module. Let $E = \colim E_i$
be a filtered colimit of $R$-modules. Suppose given a compatible
system of injective $R$-module maps $M \to E_i$ which turn $E_i$
into essential extensions of $M$. Then $M \to E$ is an essential
extension of $M$.
\end{lemma}

\begin{proof}
Immediate from the definitions and the fact that filtered
colimits are exact (Algebra, Lemma \ref{algebra-lemma-directed-colimit-exact}).
\end{proof}

\begin{lemma}
\label{lemma-essential-extensions-in-injective}
Let $R$ be a ring. Let $I$ be an injective $R$-module. Let $E \subset I$
be a submodule. The following are equivalent
\begin{enumerate}
\item $E$ is injective, and
\item for all $E \subset E' \subset I$ with $E \subset E'$ essential
we have $E = E'$.
\end{enumerate}
In particular, an $R$-module is injective if and only if every essential
extension is trivial.
\end{lemma}

\begin{proof}
The final assertion follows from the first and the fact that the
category of $R$-modules has enough injectives
(Injectives, Section \ref{injectives-section-injectives-modules}).

\medskip\noindent
Assume (1). Let $E \subset E' \subset I$ as in (2).
Then the map $\text{id}_E : E \to E$ can be extended
to a map $\alpha : E' \to E$. The kernel of $\alpha$ has to be
zero because it intersects $E$ trivially and $E'$ is an essential
extension. Hence $E = E'$.

\medskip\noindent
Assume (2). Let $M \subset N$ be $R$-modules and let $\varphi : M \to E$
be an $R$-module map. In order to prove (1) we havve to show that
$\varphi$ extends to a morphism $N \to E$.
Consider the set $\mathcal{S}$ of pairs
$(M', \varphi')$ where $M \subset M' \subset N$ and $\varphi' : M' \to E$
is an $R$-module map agreeing with $\varphi$ on $M$. We define an ordering
on $\mathcal{S}$ by the rule $(M', \varphi') \leq (M'', \varphi'')$
if and only if $M' \subset M''$ and $\varphi''|_{M'} = \varphi'$.
It is clear that we can take the maximum of a totally ordered subset
of $\mathcal{S}$. Hence by Zorn's lemma we may assume $(M, \varphi)$
is a maximal element and our goal becomes to show that $M = N$.

\medskip\noindent
In this case choose an extension $\psi : N \to I$ of $\varphi$ composed
with the inclusion $E \to I$. This is possible as $I$ is injective.
If $\psi(N)$ is not contained in $E$, then by (2)
$\varphi(M) \subset \psi(N)$ is not an essential extension. hence
we can find a nonzero submodule $K \subset \psi(N)$ not meeting $\varphi(M)$.
This means that $\psi^{-1}(K) \cap M = \text{Ker}(\varphi)$.
Thus we can extend $\varphi$ to $M + \psi^{-1}(K)$ by using the
zero map on $\psi^{-1}(K)$. This contradicts the maximality of
$(M, \varphi)$.
\end{proof}







\section{Projective covers}
\label{section-projective-cover}

\noindent
In this section we briefly discuss projective covers.

\begin{definition}
\label{definition-projective-cover}
Let $R$ be a ring. A surjection $P \to M$ of $R$-modules is said
to be a {\it projective cover}, or sometimes a {\it projective envelope},
if $P$ is a projective $R$-module and $P \to M$ is an essential
surjection.
\end{definition}

\noindent
Projective covers do not always exist. For example, if $k$ is a field
and $R = k[x]$ is the polynomial ring over $k$, then the module $M = R/(x)$
does not have a projective cover. Namely, for any surjection $f : P \to M$
with $P$ projective over $R$, the proper submodule $(x - 1)P$ surjects
onto $M$. Hence $f$ is not essential.

\begin{lemma}
\label{lemma-projective-cover-unique}
Let $R$ be a ring and let $M$ be an $R$-module. If a projective cover
of $M$ exists, then it is unique up to isomorphism.
\end{lemma}

\begin{proof}
Let $P \to M$ and $P' \to M$ be projective covers. Because $P$ is a
projective $R$-module and $P' \to M$ is surjective, we can find an
$R$-module map $\alpha : P \to P'$ compatible with the maps to $M$.
Since $P' \to M$ is essential, we see that $\alpha$ is surjective.
As $P'$ is a projective $R$-module we can choose a direct sum decomposition
$P = \text{Ker}(\alpha) \oplus P'$. Since $P' \to M$ is surjective
and since $P \to M$ is essential we conclude that $\text{ker}(\alpha)$
is zero as desired.
\end{proof}

\noindent
Here is an example where projective covers exist.

\begin{lemma}
\label{lemma-projective-covers-local}
Let $(R, \mathfrak m, \kappa)$ be a local ring. Any finite $R$-module has
a projective cover.
\end{lemma}

\begin{proof}
Let $M$ be a finite $R$-module. Let $r = \dim_\kappa(M/\mathfrak m M)$.
Choose $x_1, \ldots, x_r \in M$ mapping to a basis of $M/\mathfrak m M$.
Consider the map $f : R^{\oplus r} \to M$. By Nakayama's lemma this is
a surjection (Algebra, Lemma \ref{algebra-lemma-NAK}). If
$N \subset R^{\oplus R}$ is a proper submodule, then
$N/\mathfrak m N \to \kappa^{\oplus r}$ is not surjective (by
Nakayama's lemma again) hence $N/\mathfrak m N \to M/\mathfrak m M$
is not surjective. Thus $f$ is an essential surjection.
\end{proof}







\section{Injective hulls}
\label{section-injective-hull}

\noindent
In this section we briefly discuss injective hulls.

\begin{definition}
\label{definition-injective hull}
Let $R$ be a ring. A injection $M \to I$ of $R$-modules is said
to be an {\it injective hull} if $I$ is a injective $R$-module and
$M \to I$ is an essential injection.
\end{definition}

\noindent
Injective hulls always exist.

\begin{lemma}
\label{lemma-injective-hull}
Let $R$ be a ring. Any $R$-module has an injective hull.
\end{lemma}

\begin{proof}
Let $M$ be an $R$-module. By
Injectives, Section \ref{injectives-section-injectives-modules}
the category of $R$-modules has enough injectives.
Choose an injection $M \to I$ with $I$ an injective $R$-module.
Consider the set $\mathcal{S}$ of submodules $M \subset E \subset I$
such that $E$ is an essential extension of $M$. We order $\mathcal{S}$
by inclusion. If $\{E_\alpha\}$ is a totally ordered subset
of $\mathcal{S}$, then $\bigcup E_\alpha$ is an essential extension of $M$
too (Lemma \ref{lemma-union-essential-extensions}).
Thus we can apply Zorn's lemma and find a maximal element
$E \in \mathcal{S}$. We claim $M \subset E$ is an injective hull, i.e.,
$E$ is an injective $R$-module. This follows from
Lemma \ref{lemma-essential-extensions-in-injective}.
\end{proof}

\begin{lemma}
\label{lemma-injective-hull-unique}
Let $R$ be a ring and let $M$ be an $R$-module. Injective hull of $M$
is unique up to isomorphism.
\end{lemma}

\begin{proof}
This lemma is dual to Lemma \ref{lemma-projective-cover-unique}.
\end{proof}






\section{Other chapters}

\begin{multicols}{2}
\begin{enumerate}
\item \hyperref[introduction-section-phantom]{Introduction}
\item \hyperref[conventions-section-phantom]{Conventions}
\item \hyperref[sets-section-phantom]{Set Theory}
\item \hyperref[categories-section-phantom]{Categories}
\item \hyperref[topology-section-phantom]{Topology}
\item \hyperref[sheaves-section-phantom]{Sheaves on Spaces}
\item \hyperref[algebra-section-phantom]{Commutative Algebra}
\item \hyperref[sites-section-phantom]{Sites and Sheaves}
\item \hyperref[homology-section-phantom]{Homological Algebra}
\item \hyperref[derived-section-phantom]{Derived Categories}
\item \hyperref[more-algebra-section-phantom]{More Algebra}
\item \hyperref[simplicial-section-phantom]{Simplicial Methods}
\item \hyperref[modules-section-phantom]{Sheaves of Modules}
\item \hyperref[sites-modules-section-phantom]{Modules on Sites}
\item \hyperref[injectives-section-phantom]{Injectives}
\item \hyperref[cohomology-section-phantom]{Cohomology of Sheaves}
\item \hyperref[sites-cohomology-section-phantom]{Cohomology on Sites}
\item \hyperref[hypercovering-section-phantom]{Hypercoverings}
\item \hyperref[schemes-section-phantom]{Schemes}
\item \hyperref[constructions-section-phantom]{Constructions of Schemes}
\item \hyperref[properties-section-phantom]{Properties of Schemes}
\item \hyperref[morphisms-section-phantom]{Morphisms of Schemes}
\item \hyperref[coherent-section-phantom]{Coherent Cohomology}
\item \hyperref[divisors-section-phantom]{Divisors}
\item \hyperref[limits-section-phantom]{Limits of Schemes}
\item \hyperref[varieties-section-phantom]{Varieties}
\item \hyperref[chow-section-phantom]{Chow Homology}
\item \hyperref[topologies-section-phantom]{Topologies on Schemes}
\item \hyperref[descent-section-phantom]{Descent}
\item \hyperref[more-morphisms-section-phantom]{More on Morphisms}
\item \hyperref[flat-section-phantom]{More on Flatness}
\item \hyperref[groupoids-section-phantom]{Groupoid Schemes}
\item \hyperref[more-groupoids-section-phantom]{More on Groupoid Schemes}
\item \hyperref[etale-section-phantom]{\'Etale Morphisms of Schemes}
\item \hyperref[etale-cohomology-section-phantom]{\'Etale Cohomology}
\item \hyperref[spaces-section-phantom]{Algebraic Spaces}
\item \hyperref[spaces-properties-section-phantom]{Properties of Algebraic Spaces}
\item \hyperref[spaces-morphisms-section-phantom]{Morphisms of Algebraic Spaces}
\item \hyperref[spaces-topologies-section-phantom]{Topologies on Algebraic Spaces}
\item \hyperref[spaces-descent-section-phantom]{Descent and Algebraic Spaces}
\item \hyperref[spaces-more-morphisms-section-phantom]{More on Morphisms of Spaces}
\item \hyperref[quot-section-phantom]{Quot and Hilbert Spaces}
\item \hyperref[stacks-section-phantom]{Stacks}
\item \hyperref[spaces-groupoids-section-phantom]{Groupoids in Algebraic Spaces}
\item \hyperref[spaces-more-groupoids-section-phantom]{More on Groupoids in Spaces}
\item \hyperref[bootstrap-section-phantom]{Bootstrap}
\item \hyperref[examples-stacks-section-phantom]{Examples of Stacks}
\item \hyperref[groupoids-quotients-section-phantom]{Quotients of Groupoids}
\item \hyperref[algebraic-section-phantom]{Algebraic Stacks}
\item \hyperref[criteria-section-phantom]{Criteria for Representability}
\item \hyperref[stacks-properties-section-phantom]{Properties of Algebraic Stacks}
\item \hyperref[stacks-morphisms-section-phantom]{Morphisms of Algebraic Stacks}
\item \hyperref[examples-section-phantom]{Examples}
\item \hyperref[exercises-section-phantom]{Exercises}
\item \hyperref[guide-section-phantom]{Guide to Literature}
\item \hyperref[desirables-section-phantom]{Desirables}
\item \hyperref[coding-section-phantom]{Coding Style}
\item \hyperref[fdl-section-phantom]{GNU Free Documentation License}
\item \hyperref[index-section-phantom]{Auto Generated Index}
\end{enumerate}
\end{multicols}


\bibliography{my}
\bibliographystyle{amsalpha}

\end{document}
