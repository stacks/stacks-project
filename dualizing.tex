\IfFileExists{stacks-project.cls}{%
\documentclass{stacks-project}
}{%
\documentclass{amsart}
}

% The following AMS packages are automatically loaded with
% the amsart documentclass:
%\usepackage{amsmath}
%\usepackage{amssymb}
%\usepackage{amsthm}

% For dealing with references we use the comment environment
\usepackage{verbatim}
\newenvironment{reference}{\comment}{\endcomment}
%\newenvironment{reference}{}{}
\newenvironment{slogan}{\comment}{\endcomment}
\newenvironment{history}{\comment}{\endcomment}

% For commutative diagrams you can use
% \usepackage{amscd}
\usepackage[all]{xy}

% We use 2cell for 2-commutative diagrams.
\xyoption{2cell}
\UseAllTwocells

% To put source file link in headers.
% Change "template.tex" to "this_filename.tex"
% \usepackage{fancyhdr}
% \pagestyle{fancy}
% \lhead{}
% \chead{}
% \rhead{Source file: \url{template.tex}}
% \lfoot{}
% \cfoot{\thepage}
% \rfoot{}
% \renewcommand{\headrulewidth}{0pt}
% \renewcommand{\footrulewidth}{0pt}
% \renewcommand{\headheight}{12pt}

\usepackage{multicol}

% For cross-file-references
\usepackage{xr-hyper}

% Package for hypertext links:
\usepackage{hyperref}

% For any local file, say "hello.tex" you want to link to please
% use \externaldocument[hello-]{hello}
\externaldocument[introduction-]{introduction}
\externaldocument[conventions-]{conventions}
\externaldocument[sets-]{sets}
\externaldocument[categories-]{categories}
\externaldocument[topology-]{topology}
\externaldocument[sheaves-]{sheaves}
\externaldocument[sites-]{sites}
\externaldocument[stacks-]{stacks}
\externaldocument[fields-]{fields}
\externaldocument[algebra-]{algebra}
\externaldocument[brauer-]{brauer}
\externaldocument[homology-]{homology}
\externaldocument[derived-]{derived}
\externaldocument[simplicial-]{simplicial}
\externaldocument[more-algebra-]{more-algebra}
\externaldocument[smoothing-]{smoothing}
\externaldocument[modules-]{modules}
\externaldocument[sites-modules-]{sites-modules}
\externaldocument[injectives-]{injectives}
\externaldocument[cohomology-]{cohomology}
\externaldocument[sites-cohomology-]{sites-cohomology}
\externaldocument[dga-]{dga}
\externaldocument[dpa-]{dpa}
\externaldocument[hypercovering-]{hypercovering}
\externaldocument[schemes-]{schemes}
\externaldocument[constructions-]{constructions}
\externaldocument[properties-]{properties}
\externaldocument[morphisms-]{morphisms}
\externaldocument[coherent-]{coherent}
\externaldocument[divisors-]{divisors}
\externaldocument[limits-]{limits}
\externaldocument[varieties-]{varieties}
\externaldocument[topologies-]{topologies}
\externaldocument[descent-]{descent}
\externaldocument[perfect-]{perfect}
\externaldocument[more-morphisms-]{more-morphisms}
\externaldocument[flat-]{flat}
\externaldocument[groupoids-]{groupoids}
\externaldocument[more-groupoids-]{more-groupoids}
\externaldocument[etale-]{etale}
\externaldocument[chow-]{chow}
\externaldocument[intersection-]{intersection}
\externaldocument[pic-]{pic}
\externaldocument[adequate-]{adequate}
\externaldocument[dualizing-]{dualizing}
\externaldocument[duality-]{duality}
\externaldocument[discriminant-]{discriminant}
\externaldocument[local-cohomology-]{local-cohomology}
\externaldocument[curves-]{curves}
\externaldocument[resolve-]{resolve}
\externaldocument[models-]{models}
\externaldocument[pione-]{pione}
\externaldocument[etale-cohomology-]{etale-cohomology}
\externaldocument[proetale-]{proetale}
\externaldocument[crystalline-]{crystalline}
\externaldocument[spaces-]{spaces}
\externaldocument[spaces-properties-]{spaces-properties}
\externaldocument[spaces-morphisms-]{spaces-morphisms}
\externaldocument[decent-spaces-]{decent-spaces}
\externaldocument[spaces-cohomology-]{spaces-cohomology}
\externaldocument[spaces-limits-]{spaces-limits}
\externaldocument[spaces-divisors-]{spaces-divisors}
\externaldocument[spaces-over-fields-]{spaces-over-fields}
\externaldocument[spaces-topologies-]{spaces-topologies}
\externaldocument[spaces-descent-]{spaces-descent}
\externaldocument[spaces-perfect-]{spaces-perfect}
\externaldocument[spaces-more-morphisms-]{spaces-more-morphisms}
\externaldocument[spaces-flat-]{spaces-flat}
\externaldocument[spaces-groupoids-]{spaces-groupoids}
\externaldocument[spaces-more-groupoids-]{spaces-more-groupoids}
\externaldocument[bootstrap-]{bootstrap}
\externaldocument[spaces-pushouts-]{spaces-pushouts}
\externaldocument[groupoids-quotients-]{groupoids-quotients}
\externaldocument[spaces-more-cohomology-]{spaces-more-cohomology}
\externaldocument[spaces-simplicial-]{spaces-simplicial}
\externaldocument[formal-spaces-]{formal-spaces}
\externaldocument[restricted-]{restricted}
\externaldocument[spaces-resolve-]{spaces-resolve}
\externaldocument[formal-defos-]{formal-defos}
\externaldocument[defos-]{defos}
\externaldocument[cotangent-]{cotangent}
\externaldocument[examples-defos-]{examples-defos}
\externaldocument[algebraic-]{algebraic}
\externaldocument[examples-stacks-]{examples-stacks}
\externaldocument[stacks-sheaves-]{stacks-sheaves}
\externaldocument[criteria-]{criteria}
\externaldocument[artin-]{artin}
\externaldocument[quot-]{quot}
\externaldocument[stacks-properties-]{stacks-properties}
\externaldocument[stacks-morphisms-]{stacks-morphisms}
\externaldocument[stacks-limits-]{stacks-limits}
\externaldocument[stacks-cohomology-]{stacks-cohomology}
\externaldocument[stacks-perfect-]{stacks-perfect}
\externaldocument[stacks-introduction-]{stacks-introduction}
\externaldocument[stacks-more-morphisms-]{stacks-more-morphisms}
\externaldocument[stacks-geometry-]{stacks-geometry}
\externaldocument[moduli-]{moduli}
\externaldocument[moduli-curves-]{moduli-curves}
\externaldocument[examples-]{examples}
\externaldocument[exercises-]{exercises}
\externaldocument[guide-]{guide}
\externaldocument[desirables-]{desirables}
\externaldocument[coding-]{coding}
\externaldocument[obsolete-]{obsolete}
\externaldocument[fdl-]{fdl}
\externaldocument[index-]{index}

% Theorem environments.
%
\theoremstyle{plain}
\newtheorem{theorem}[subsection]{Theorem}
\newtheorem{proposition}[subsection]{Proposition}
\newtheorem{lemma}[subsection]{Lemma}

\theoremstyle{definition}
\newtheorem{definition}[subsection]{Definition}
\newtheorem{example}[subsection]{Example}
\newtheorem{exercise}[subsection]{Exercise}
\newtheorem{situation}[subsection]{Situation}

\theoremstyle{remark}
\newtheorem{remark}[subsection]{Remark}
\newtheorem{remarks}[subsection]{Remarks}

\numberwithin{equation}{subsection}

% Macros
%
\def\lim{\mathop{\rm lim}\nolimits}
\def\colim{\mathop{\rm colim}\nolimits}
\def\Spec{\mathop{\rm Spec}}
\def\Hom{\mathop{\rm Hom}\nolimits}
\def\Ext{\mathop{\rm Ext}\nolimits}
\def\SheafHom{\mathop{\mathcal{H}\!{\it om}}\nolimits}
\def\SheafExt{\mathop{\mathcal{E}\!{\it xt}}\nolimits}
\def\Sch{\textit{Sch}}
\def\Mor{\mathop{\rm Mor}\nolimits}
\def\Ob{\mathop{\rm Ob}\nolimits}
\def\Sh{\mathop{\textit{Sh}}\nolimits}
\def\NL{\mathop{N\!L}\nolimits}
\def\proetale{{pro\text{-}\acute{e}tale}}
\def\etale{{\acute{e}tale}}
\def\QCoh{\textit{QCoh}}
\def\Ker{\mathop{\rm Ker}}
\def\Im{\mathop{\rm Im}}
\def\Coker{\mathop{\rm Coker}}
\def\Coim{\mathop{\rm Coim}}

%
% Macros for moduli stacks/spaces
%
\def\QCohstack{\mathcal{QC}\!{\it oh}}
\def\Cohstack{\mathcal{C}\!{\it oh}}
\def\Spacesstack{\mathcal{S}\!{\it paces}}
\def\Quotfunctor{{\rm Quot}}
\def\Hilbfunctor{{\rm Hilb}}
\def\Curvesstack{\mathcal{C}\!{\it urves}}
\def\Polarizedstack{\mathcal{P}\!{\it olarized}}
\def\Complexesstack{\mathcal{C}\!{\it omplexes}}
% \Pic is the operator that assigns to X its picard group, usage \Pic(X)
% \Picardstack_{X/B} denotes the Picard stack of X over B
% \Picardfunctor_{X/B} denotes the Picard functor of X over B
\def\Pic{\mathop{\rm Pic}\nolimits}
\def\Picardstack{\mathcal{P}\!{\it ic}}
\def\Picardfunctor{{\rm Pic}}
\def\Deformationcategory{\mathcal{D}\!{\it ef}}


% OK, start here.
%
\begin{document}

\title{Dualizing Complexes}


\maketitle

\phantomsection
\label{section-phantom}

\tableofcontents

\section{Introduction}
\label{section-introduction}

\noindent
A reference is the book \cite{R+D}.

\medskip\noindent
The goals of this chapter are the following:
\begin{enumerate}
\item Define what it means to have a dualizing complex $\omega_A^\bullet$
over a Noetherian ring $A$, namely
\begin{enumerate}
\item we have $\omega_A^\bullet \in D^{+}(A)$,
\item the cohomology modules $H^i(\omega_A^\bullet)$ are
all finite $A$-modules,
\item $\omega_A^\bullet$ has finite injective dimension, and
\item we have $A \to R\Hom_A(\omega_A^\bullet, \omega_A^\bullet)$
is a quasi-isomorphism.
\end{enumerate}
\item List elementary properties of dualizing complexes.
\item Show a dualizing complex gives rise to a dimension function.
\item Show a dualizing complex gives rise to a good notion of a
reflexive hull.
\item Prove the finiteness theorem when a dualizing complex exists.
\end{enumerate}






\section{Essential surjections and injections}
\label{section-essential}

\noindent
We will mostly work in categories of modules, but we may as well make
the definition in general.

\begin{definition}
\label{definition-essential}
Let $\mathcal{A}$ be an abelian category.
\begin{enumerate}
\item An injection $A \subset B$ of $\mathcal{A}$ is {\it essential},
or we say that $B$ is an {\it essential extension of} $A$,
if every nonzero subobject $B' \subset B$ has nonzero intersection with $A$.
\item A surjection $f : A \to B$ of $\mathcal{A}$ is {\it essential}
if for every proper subobject $A' \subset A$ we have $f(A') \not = B$.
\end{enumerate}
\end{definition}

\noindent
Some lemmas about this notion.

\begin{lemma}
\label{lemma-essential}
Let $\mathcal{A}$ be an abelian category.
\begin{enumerate}
\item If $A \subset B$ and $B \subset C$ are essential extensions, then
$A \subset C$ is an essential extension.
\item If $A \subset B$ is an essential extension and $C \subset B$
is a subobject, then $A \cap C \subset C$ is an essential extension.
\item If $A \to B$ and $B \to C$ are essential surjections, then
$A \to C$ is an essential surjection.
\item Given an essential surjection $f : A \to B$ and a surjection
$A \to C$ with kernel $K$, the morphism $C \to B/f(K)$ is an essential
surjection.
\end{enumerate}
\end{lemma}

\begin{proof}
Omitted.
\end{proof}

\begin{lemma}
\label{lemma-union-essential-extensions}
Let $R$ be a ring. Let $M$ be an $R$-module. Let $E = \colim E_i$
be a filtered colimit of $R$-modules. Suppose given a compatible
system of essential injections $M \to E_i$ of $R$-modules.
Then $M \to E$ is an essential extension of $M$.
\end{lemma}

\begin{proof}
Immediate from the definitions and the fact that filtered
colimits are exact (Algebra, Lemma \ref{algebra-lemma-directed-colimit-exact}).
\end{proof}

\begin{lemma}
\label{lemma-essential-extension}
Let $R$ be a ring. Let $M \subset N$ be $R$-modules. The following
are equivalent
\begin{enumerate}
\item $M \subset N$ is an essential extension,
\item for all $x \in N$ there exists an $f \in R$ such that $fx \in M$
and $fx \not = 0$.
\end{enumerate}
\end{lemma}

\begin{proof}
Assume (1) and let $x \in N$ be a nonzero element. By (1) we have
$Rx \cap M \not = 0$. This implies (2).

\medskip\noindent
Assume (2). Let $N' \subset N$ be a nonzero submodule. Pick $x \in N'$
nonzero. By (2) we can find $f \in $ with $fx \in N$ and $fx \not = 0$.
Thus $N' \cap M \not = 0$.
\end{proof}




\section{Injective modules}
\label{section-injective-modules}

\noindent
Some results about injective modules over rings.

\begin{lemma}
\label{lemma-product-injectives}
Let $R$ be a ring. Any product of injective $R$-modules is injective.
\end{lemma}

\begin{proof}
Special case of Homology, Lemma \ref{homology-lemma-product-injectives}.
\end{proof}

\begin{lemma}
\label{lemma-injective-flat}
Let $R \to S$ be a flat ring map. If $E$ is an injective $S$-module,
then $E$ is injective as an $R$-module.
\end{lemma}

\begin{proof}
This is true because $\Hom_R(M, E) = \Hom_S(M \otimes_R S, E)$
by Algebra, Lemma \ref{algebra-lemma-adjoint-tensor-restrict}
and the fact that tensoring with $S$ is exact.
\end{proof}

\begin{lemma}
\label{lemma-injective-epimorphism}
Let $R \to S$ be an epimorphism of rings. Let $E$ be an $S$-module.
If $E$ is injective as an $R$-module, then $E$ is an injective $S$-module.
\end{lemma}

\begin{proof}
This is true because $\Hom_R(N, E) = \Hom_S(N, E)$ for any $S$-module $N$,
see Algebra, Lemma \ref{algebra-lemma-epimorphism-modules}.
\end{proof}

\begin{lemma}
\label{lemma-hom-injective}
Let $R \to S$ be a ring map. If $E$ is an injective $R$-module,
then $\Hom_R(S, E)$ is an injective $S$-module.
\end{lemma}

\begin{proof}
This is true because $\Hom_S(N, \Hom_R(S, E)) = \Hom_R(N, E)$ by
Algebra, Lemma \ref{algebra-lemma-adjoint-hom-restrict}.
\end{proof}

\begin{lemma}
\label{lemma-essential-extensions-in-injective}
Let $R$ be a ring. Let $I$ be an injective $R$-module. Let $E \subset I$
be a submodule. The following are equivalent
\begin{enumerate}
\item $E$ is injective, and
\item for all $E \subset E' \subset I$ with $E \subset E'$ essential
we have $E = E'$.
\end{enumerate}
In particular, an $R$-module is injective if and only if every essential
extension is trivial.
\end{lemma}

\begin{proof}
The final assertion follows from the first and the fact that the
category of $R$-modules has enough injectives
(More on Algebra, Section \ref{more-algebra-section-injectives-modules}).

\medskip\noindent
Assume (1). Let $E \subset E' \subset I$ as in (2).
Then the map $\text{id}_E : E \to E$ can be extended
to a map $\alpha : E' \to E$. The kernel of $\alpha$ has to be
zero because it intersects $E$ trivially and $E'$ is an essential
extension. Hence $E = E'$.

\medskip\noindent
Assume (2). Let $M \subset N$ be $R$-modules and let $\varphi : M \to E$
be an $R$-module map. In order to prove (1) we have to show that
$\varphi$ extends to a morphism $N \to E$. Consider the set $\mathcal{S}$
of pairs
$(M', \varphi')$ where $M \subset M' \subset N$ and $\varphi' : M' \to E$
is an $R$-module map agreeing with $\varphi$ on $M$. We define an ordering
on $\mathcal{S}$ by the rule $(M', \varphi') \leq (M'', \varphi'')$
if and only if $M' \subset M''$ and $\varphi''|_{M'} = \varphi'$.
It is clear that we can take the maximum of a totally ordered subset
of $\mathcal{S}$. Hence by Zorn's lemma we may assume $(M, \varphi)$
is a maximal element.

\medskip\noindent
Choose an extension $\psi : N \to I$ of $\varphi$ composed
with the inclusion $E \to I$. This is possible as $I$ is injective.
If $\psi(N) \subset E$, then $\psi$ is the desired extension.
If $\psi(N)$ is not contained in $E$, then by (2) the inclusion
$E \subset E + \psi(N)$ is not essential. hence
we can find a nonzero submodule $K \subset E + \psi(N)$ meeting $E$ in $0$.
This means that $M' = \psi^{-1}(E + K)$ strictly contains $M$.
Thus we can extend $\varphi$ to $M'$ using
$$
M' \xrightarrow{\psi|_{M'}} E + K \to (E + K)/K = E
$$
This contradicts the maximality of $(M, \varphi)$.
\end{proof}

\begin{example}
\label{example-reduced-ring-injective}
Let $R$ be a reduced ring. Let $\mathfrak p \subset R$ be a minimal prime
so that $K = R_\mathfrak p$ is a field
(Algebra, Lemma \ref{algebra-lemma-minimal-prime-reduced-ring}).
Then $K$ is an injective $R$-module. Namely, we have
$\Hom_R(M, K) = \Hom_K(M_\mathfrak p, K)$ for any $R$-module
$M$. Since localization is an exact functor and taking duals is
an exact functor on $K$-vector spaces we conclude $\Hom_R(-, K)$
is an exact functor, i.e., $K$ is an injective $R$-module.
\end{example}

\begin{lemma}
\label{lemma-characterize-injective}
Let $R$ be a ring. Let $E$ be an $R$-module. The following are equivalent
\begin{enumerate}
\item $E$ is an injective $R$-module, and
\item given an ideal $I \subset R$ and a module map $\varphi: I \to E$
there exists an extension of $\varphi$ to and $R$-module map $R \to E$.
\end{enumerate}
\end{lemma}

\begin{proof}
The implication (1) $\Rightarrow$ (2) follows from the definitions.
Thus we assume (2) holds and we prove (1).
First proof: Since $R$ is a generator for the category of $R$-modules,
this follows from
Injectives, Lemma \ref{injectives-lemma-characterize-injective}.

\medskip\noindent
Second proof: We have to show that every essential extension $E \subset E'$
is trivial. Pick $x \in E'$ and set $I = \{f \in R \mid fx \in E\}$.
The map $I \to E$, $f \mapsto fx$ extends to $\psi : R \to E$ by (2).
Then $x' = x - \psi(1)$ is an element of $E'$ whose annihilator in
$E'/E$ is $I$ and which is annihilated by $I$ as an element of $E'$.
Thus $Rx' = (R/I)x'$ does not intersect $E$. Since $E \subset E'$
is an essential extension it follows that $x' \in E$ as desired.
\end{proof}

\begin{lemma}
\label{lemma-sum-injective-modules}
Let $R$ be a Noetherian ring. A direct sum of injective modules
is injective.
\end{lemma}

\begin{proof}
Let $E_i$ be a family of injective modules parametrized by a set $I$.
Set $E = \bigcup E_i$. To show that $E$ is injective we use
Lemma \ref{lemma-characterize-injective}.
Thus let $\varphi : I \to E$ be a module map from an ideal of $R$
into $E$. As $I$ is a finite $R$-module (because $R$ is Noetherian)
we can find finitely many elements $i_1, \ldots, i_r \in I$
such that $\varphi$ maps into $\bigcup_{j = 1, \ldots, r} E_{i_j}$.
Then we can extend $\varphi$ into $\bigcup_{j = 1, \ldots, r} E_{i_j}$
using the injectivity of the modules $E_{i_j}$.
\end{proof}

\begin{lemma}
\label{lemma-injective-module-divide}
Let $R$ be a Noetherian ring. Let $I$ be an injective $R$-module.
\begin{enumerate}
\item Let $f \in R$. Then $E = \{x \in I \mid \exists\ n,\ f^n x  = 0\}$
is an injective submodule of $I$.
\item Let $J \subset R$ be an ideal. Then the $J$-power torsion
submodule $I[J^\infty]$ is an injective submodule of $I$.
\end{enumerate}
\end{lemma}

\begin{proof}
We will use Lemma \ref{lemma-essential-extensions-in-injective}
to prove (1).
Suppose that $E \subset E' \subset I$ and that $E'$ is an essential
extension of $E$. We will show that $E' = E$. If not, then we can
find $x \in E'$ and $x \not \in E$. Let $J = \{a \in R \mid ax \in E'\}$.
Since $R$ is Noetherian we can choose $x$ with $J$ maximal.
Since $R$ is Noetherian we can write $J = (g_1, \ldots, g_t)$ for some
$g_i \in R$. Say $f^{n_i}$ annihilates $g_ix$. Set $n = \max\{n_i\}$.
Then $x' = f^n x$ is an element of $E'$ not in $E$ and is annihilated
by $J$. By maximality of $J$ we see that $R x' = (R/J)x'  \cap E = (0)$.
Hence $E'$ is not an essential extension of $E$ a contradiction.

\medskip\noindent
To prove (2) write $J = (f_1, \ldots, f_t)$. Then
$I[J^\infty]$ is equal to
$$
(\ldots((I[f_1^\infty])[f_2^\infty])\ldots)[f_t^\infty]
$$
and the result follows from (1) and induction.
\end{proof}



\section{Projective covers}
\label{section-projective-cover}

\noindent
In this section we briefly discuss projective covers.

\begin{definition}
\label{definition-projective-cover}
Let $R$ be a ring. A surjection $P \to M$ of $R$-modules is said
to be a {\it projective cover}, or sometimes a {\it projective envelope},
if $P$ is a projective $R$-module and $P \to M$ is an essential
surjection.
\end{definition}

\noindent
Projective covers do not always exist. For example, if $k$ is a field
and $R = k[x]$ is the polynomial ring over $k$, then the module $M = R/(x)$
does not have a projective cover. Namely, for any surjection $f : P \to M$
with $P$ projective over $R$, the proper submodule $(x - 1)P$ surjects
onto $M$. Hence $f$ is not essential.

\begin{lemma}
\label{lemma-projective-cover-unique}
Let $R$ be a ring and let $M$ be an $R$-module. If a projective cover
of $M$ exists, then it is unique up to isomorphism.
\end{lemma}

\begin{proof}
Let $P \to M$ and $P' \to M$ be projective covers. Because $P$ is a
projective $R$-module and $P' \to M$ is surjective, we can find an
$R$-module map $\alpha : P \to P'$ compatible with the maps to $M$.
Since $P' \to M$ is essential, we see that $\alpha$ is surjective.
As $P'$ is a projective $R$-module we can choose a direct sum decomposition
$P = \text{Ker}(\alpha) \oplus P'$. Since $P' \to M$ is surjective
and since $P \to M$ is essential we conclude that $\text{ker}(\alpha)$
is zero as desired.
\end{proof}

\noindent
Here is an example where projective covers exist.

\begin{lemma}
\label{lemma-projective-covers-local}
Let $(R, \mathfrak m, \kappa)$ be a local ring. Any finite $R$-module has
a projective cover.
\end{lemma}

\begin{proof}
Let $M$ be a finite $R$-module. Let $r = \dim_\kappa(M/\mathfrak m M)$.
Choose $x_1, \ldots, x_r \in M$ mapping to a basis of $M/\mathfrak m M$.
Consider the map $f : R^{\oplus r} \to M$. By Nakayama's lemma this is
a surjection (Algebra, Lemma \ref{algebra-lemma-NAK}). If
$N \subset R^{\oplus R}$ is a proper submodule, then
$N/\mathfrak m N \to \kappa^{\oplus r}$ is not surjective (by
Nakayama's lemma again) hence $N/\mathfrak m N \to M/\mathfrak m M$
is not surjective. Thus $f$ is an essential surjection.
\end{proof}







\section{Injective hulls}
\label{section-injective-hull}

\noindent
In this section we briefly discuss injective hulls.

\begin{definition}
\label{definition-injective-hull}
Let $R$ be a ring. A injection $M \to I$ of $R$-modules is said
to be an {\it injective hull} if $I$ is a injective $R$-module and
$M \to I$ is an essential injection.
\end{definition}

\noindent
Injective hulls always exist.

\begin{lemma}
\label{lemma-injective-hull}
Let $R$ be a ring. Any $R$-module has an injective hull.
\end{lemma}

\begin{proof}
Let $M$ be an $R$-module. By
More on Algebra, Section \ref{more-algebra-section-injectives-modules}
the category of $R$-modules has enough injectives.
Choose an injection $M \to I$ with $I$ an injective $R$-module.
Consider the set $\mathcal{S}$ of submodules $M \subset E \subset I$
such that $E$ is an essential extension of $M$. We order $\mathcal{S}$
by inclusion. If $\{E_\alpha\}$ is a totally ordered subset
of $\mathcal{S}$, then $\bigcup E_\alpha$ is an essential extension of $M$
too (Lemma \ref{lemma-union-essential-extensions}).
Thus we can apply Zorn's lemma and find a maximal element
$E \in \mathcal{S}$. We claim $M \subset E$ is an injective hull, i.e.,
$E$ is an injective $R$-module. This follows from
Lemma \ref{lemma-essential-extensions-in-injective}.
\end{proof}

\begin{lemma}
\label{lemma-injective-hull-unique}
Let $R$ be a ring. Let $M$, $N$ be $R$-modules and let $M \to E$
and $N \to E'$ be injective hulls. Then
\begin{enumerate}
\item for any $R$-module map $\varphi : M \to N$ there exists an
$R$-module map $\psi : E \to E'$ such that
$$
\xymatrix{
M \ar[r] \ar[d]_\varphi & E \ar[d]^\psi \\
N \ar[r] & E'
}
$$
commutes,
\item if $\varphi$ is injective, then $\psi$ is injective,
\item if $\varphi$ is an essential injection, then $\psi$ is an isomorphism,
\item if $\varphi$ is an isomorphism, then $\psi$ is an isomorphism,
\item if $M \to I$ is an embedding of $M$ into an injective $R$-module,
then there is an isomorphism $I \cong E \oplus I'$ compatible with
the embeddings of $M$,
\end{enumerate}
In particular, the injective hull $E$ of $M$ is unique up to isomorphism.
\end{lemma}

\begin{proof}
Part (1) follows from the fact that $E'$ is an injective $R$-module.
Part (2) follows as $\text{Ker}(\psi) \cap M = 0$
and $E$ is an essential extension of $M$.
Assume $\varphi$ is an essential injection. Then
$E \cong \psi(E) \subset E'$ by (2) which implies
$E' = \psi(E) \oplus E''$ because $E$ is injective.
Since $E'$ is an essential extension of
$M$ (Lemma \ref{lemma-essential}) we get $E'' = 0$.
Part (4) is a special case of (3).
Assume $M \to I$ as in (5).
Choose a map $\alpha : E \to I$ extending the map $M \to I$.
Arguing as before we see that $\alpha$ is injective.
Thus as before $\alpha(E)$ splits off from $I$.
This proves (5).
\end{proof}

\begin{example}
\label{example-injective-hull-domain}
Let $R$ be a domain with fraction field $K$. Then $R \subset K$ is an
injective hull of $R$. Namely, by
Example \ref{example-reduced-ring-injective} we see that $K$ is an injective
$R$-module and by Lemma \ref{lemma-essential-extension} we see that
$R \subset K$ is an essential extension.
\end{example}

\begin{definition}
\label{definition-indecomposable}
An object $X$ of an additive category is called {\it indecomposable}
if it is nonzero and if $X = Y \oplus Z$, then either $Y = 0$ or $Z = 0$.
\end{definition}

\begin{lemma}
\label{lemma-indecomposable-injective}
Let $R$ be a ring. Let $E$ be an indecomposable injective $R$-module.
Then
\begin{enumerate}
\item $E$ is the injective hull of any nonzero submodule of $E$,
\item the intersection of any two nonzero submodules of $E$ is nonzero,
\item $\text{End}_R(E, E)$ is a noncommutative local ring with maximal
ideal those $\varphi : E \to E$ whose kernel is nonzero, and
\item the set of zerodivisors on $E$ is a prime ideal $\mathfrak p$ of $R$
and $E$ is an injective $R_\mathfrak p$-module.
\end{enumerate}
\end{lemma}

\begin{proof}
Part (1) follows from Lemma \ref{lemma-injective-hull-unique}.
Part (2) follows from part (1) and the definition of injective hulls.

\medskip\noindent
Proof of (3). Set $A = \text{End}_R(E, E)$ and
$I = \{\varphi \in A \mid \text{ker}(f) \not = 0\}$.
The statement means that $I$ is a two sided ideal and
that any $\varphi \in A$, $\varphi \not \in I$ is invertible.
Suppose $\varphi$ and $\psi$ are not injective.
Then $\text{Ker}(\varphi) \cap \text{Ker}(\psi)$ is nonzero
by (2). Hence $\varphi + \psi \in I$. It follows that $I$
is a two sided ideal. If $\varphi \in A$, $\varphi \not \in I$,
then $E \cong \varphi(E) \subset E$ is an injective submodule,
hence $E = \varphi(E)$ because $E$ is indecomposable.

\medskip\noindent
Proof of (4). Consider the ring map $R \to A$ and let $\mathfrak p \subset R$
be the inverse image of the maximal ideal $I$. Then it is clear
that $\mathfrak p$ is a prime ideal and that $R \to A$ extends to
$R_\mathfrak p \to A$. Thus $E$ is an $R_\mathfrak p$-module.
It follows from Lemma \ref{lemma-injective-epimorphism} that $E$ is injective
as an $R_\mathfrak p$-module.
\end{proof}

\begin{lemma}
\label{lemma-injective-hull-indecomposable}
Let $\mathfrak p \subset R$ be a prime of a ring $R$.
Let $E$ be the injective hull of $R/\mathfrak p$. Then
\begin{enumerate}
\item $E$ is indecomposable,
\item $E$ is the injective hull of $\kappa(\mathfrak p)$,
\item $E$ is the injective hull of $\kappa(\mathfrak p)$
over the ring $R_\mathfrak p$.
\end{enumerate}
\end{lemma}

\begin{proof}
As $R/\mathfrak p \subset \kappa(\mathfrak p)$ we can extend the embedding
to a map $\kappa(\mathfrak p) \to E$. Hence (2) holds.
For $f \in R$, $f \not \in \mathfrak p$
the map $f : \kappa(\mathfrak p) \to \kappa(\mathfrak p)$ is an isomorphism
hence the map $f : E \to E$ is an isomorphism,
see Lemma \ref{lemma-injective-hull-unique}.
Thus $E$ is an $R_\mathfrak p$-module. It is injective
as an $R_\mathfrak p$-module by Lemma \ref{lemma-injective-epimorphism}.
Finally, let $E' \subset E$ be a nonzero injective $R$-submodule.
Then $J = (R/\mathfrak p) \cap E'$ is nonzero. After shrinking $E'$
we may assume that $E'$ is the injective hull of $J$ (see
Lemma \ref{lemma-injective-hull-unique} for example).
Observe that $R/\mathfrak p$ is an essential extension of $J$ for example by
Lemma \ref{lemma-essential-extension}. Hence $E' \to E$
is an isomorphism by Lemma \ref{lemma-injective-hull-unique} part (3).
Hence $E$ is indecomposable.
\end{proof}

\begin{lemma}
\label{lemma-indecomposable-injective-noetherian}
Let $R$ be a Noetherian ring. Let $E$ be an indecomposable injective
$R$-module. Then there exists a prime ideal $\mathfrak p$ of $R$ such that
$E$ is the injective hull of $\kappa(\mathfrak p)$.
\end{lemma}

\begin{proof}
Let $\mathfrak p$ be the prime ideal found in
Lemma \ref{lemma-indecomposable-injective}.
Say $\mathfrak p = (f_1, \ldots, f_r)$.
Pick a nonzero element $x \in \bigcap \text{Ker}(f_i : E \to E)$,
see Lemma \ref{lemma-indecomposable-injective}.
Then $(R_\mathfrak p)x$ is a module isomorphic to $\kappa(\mathfrak p)$
inside $E$. We conclude by Lemma \ref{lemma-indecomposable-injective}.
\end{proof}

\begin{proposition}[Structure injective modules over Noetherian rings]
\label{proposition-structure-injectives-noetherian}
Let $R$ be a Noetherian ring.
Every injective module is a direct sum of indecomposable injective modules.
Every indecomposable injective module is the injective hull of
the residue field at a prime.
\end{proposition}

\begin{proof}
The second statement is Lemma \ref{lemma-indecomposable-injective-noetherian}.
For the second statement, let $I$ be an injective $R$-module.
We will use transfinite induction to construct $I_\alpha \subset I$
for ordinals $\alpha$ which are direct sums of indecomposable injective
$R$-modules $E_{\beta + 1}$ for $\beta < \alpha$.
For $\alpha = 0$ we let $I_0 = 0$. Suppose given an ordinal $\alpha$
such that $I_\alpha$ has been constructed. Then $I_\alpha$ is an
injective $R$-module by Lemma \ref{lemma-sum-injective-modules}.
Hence $I \cong I_\alpha \oplus I'$. If $I' = 0$ we are done.
If not, then $I'$ has an associated prime by
Algebra, Lemma \ref{algebra-lemma-ass-zero}.
Thus $I'$ contains a copy of $R/\mathfrak p$ for some prime $\mathfrak p$.
Hence $I'$ contains an indecomposable submodule $E$ by
Lemmas \ref{lemma-injective-hull-unique} and
\ref{lemma-injective-hull-indecomposable}. Set
$I_{\alpha + 1} = I_\alpha \oplus E_\alpha$.
If $\alpha$ is a limit ordinal and $I_\beta$ has been constructed
for $\beta < \alpha$, then we set
$I_\alpha = \bigcup_{\beta < \alpha} I_\beta$.
Observe that $I_\alpha = \bigoplus_{\beta < \alpha} E_{\beta + 1}$.
This concludes the proof.
\end{proof}



\section{Duality over Artinian local rings}
\label{section-artinian}

\noindent
Let $(R, \mathfrak m, \kappa)$ be an artinian local ring.
Recall that this implies $R$ is Noetherian and that $R$ has finite
length as an $R$-module. Moreover an $R$-module is finite if and
only if it has finite length. We will use these facts without
further mention in this section. Please see
Algebra, Sections \ref{algebra-section-length} and
\ref{algebra-section-artinian}
and
Algebra, Proposition \ref{algebra-proposition-dimension-zero-ring}
for more details.

\begin{lemma}
\label{lemma-finite}
Let $(R, \mathfrak m, \kappa)$ be an artinian local ring.
Let $E$ be an injective hull of $\kappa$. For every finite
$R$-module $M$ we have
$$
\text{length}_R(M) = \text{length}_R(\Hom_R(M, E))
$$
In particular, the injective hull $E$ of $\kappa$ is a finite $R$-module.
\end{lemma}

\begin{proof}
Because $E$ is an essential extension of $\kappa$ we have
$\kappa = E[\mathfrak m]$ where $E[\mathfrak m]$ is the
$\mathfrak m$-torsion in $E$ (notation as in More on Algebra, Section
\ref{more-algebra-section-formal-glueing}).
Hence $\Hom_R(\kappa, E) \cong \kappa$ and the equality of lengths
holds for $M = \kappa$. We prove the displayed equality of the lemma
by induction on the length of $M$. If $M$ is nonzero there exists a surjection
$M \to \kappa$ with kernel $M'$. Since the functor $M \mapsto \Hom_R(M, E)$
is exact we obtain a short exact sequence
$$
0 \to \Hom_R(\kappa, E) \to \Hom_R(M, E) \to \Hom_R(M', E) \to 0.
$$
Additivity of length for this sequence and the sequence
$0 \to M' \to M \to \kappa \to 0$ and the equality for $M'$ (induction
hypothesis) and $\kappa$ implies the equality for $M$.
The final statement of the lemma follows as $E = \Hom_R(R, E)$.
\end{proof}

\begin{lemma}
\label{lemma-evaluate}
Let $(R, \mathfrak m, \kappa)$ be an artinian local ring.
Let $E$ be an injective hull of $\kappa$.
For any finite $R$-module $M$ the evaluation map
$$
M \longrightarrow \Hom_R(\Hom_R(M, E), E)
$$
is an isomorphism. In particular $R = \Hom_R(E, E)$.
\end{lemma}

\begin{proof}
Observe that the displayed arrow is injective. Namely, if $x \in M$ is
a nonzero element, then there is a nonzero map $Rx \to \kappa$ which
we can extend to a map $\varphi : M \to E$ that doesn't vanish on $x$.
Since the source and target of the arrow have the same length by
Lemma \ref{lemma-finite}
we conclude it is an isomorphism. The final statement follows
on taking $M = R$.
\end{proof}

\noindent
To state the next lemma, denote $\text{Mod}^{fg}_R$ the category of finite
$R$-modules over a ring $R$.

\begin{lemma}
\label{lemma-duality}
Let $(R, \mathfrak m, \kappa)$ be an artinian local ring.
Let $E$ be an injective hull of $\kappa$.
The functor $D(-) = \Hom_R(-, E)$ induces an exact anti-equivalence
$\text{Mod}^{fg}_R \to \text{Mod}^{fg}_R$ and
$D \circ D \cong \text{id}$.
\end{lemma}

\begin{proof}
We have seen that $D \circ D = \text{id}$ on $\text{Mod}^{fg}_R$
in Lemma \ref{lemma-evaluate}. It follows immediately that
$D$ is an anti-equivalence.
\end{proof}

\begin{lemma}
\label{lemma-duality-torsion-cotorsion}
Assumptions and notation as in Lemma \ref{lemma-duality}.
Let $I \subset R$ be an ideal and $M$ a finite $R$-module.
Then
$$
D(M[I]) = D(M)/ID(M) \quad\text{and}\quad D(M/IM) = D(M)[I]
$$
\end{lemma}

\begin{proof}
Say $I = (f_1, \ldots, f_t)$. Consider the map
$$
M^{\oplus t} \xrightarrow{f_1, \ldots, f_t} M
$$
with cokernel $M/IM$. Applying the exact functor $D$ we conclude that
$D(M/IM)$ is $D(M)[I]$. The other case is proved in the same way.
\end{proof}



\section{Injective hull of the residue field}
\label{section-hull-residue-field}

\noindent
Most of our results will be for Noetherian local rings in this section.

\begin{lemma}
\label{lemma-quotient}
Let $R \to S$ be a surjective map of local rings with kernel $I$.
Let $E$ be the injective hull of the residue field of $R$ over $R$.
Then $E[I]$ is the injective hull of the residue field of $S$ over $S$.
\end{lemma}

\begin{proof}
Observe that $E[I] = \Hom_R(S, E)$ as $S = R/I$. Hence $E[I]$ is an injective
$S$-module by Lemma \ref{lemma-hom-injective}. Since $E$ is an essential
extension of $\kappa = R/\mathfrak m_R$ it follows that $E[I]$ is an
essential extension of $\kappa$ as well. The result follows.
\end{proof}

\begin{lemma}
\label{lemma-torsion-submodule-sum-injective-hulls}
Let $(R, \mathfrak m, \kappa)$ be a local ring.
Let $E$ be the injective hull of $\kappa$.
Let $M$ be a $\mathfrak m$-power torsion $R$-module
with $n = \dim_\kappa(M[\mathfrak m]) < \infty$.
Then $M$ is isomorphic to a submodule of $E^{\oplus n}$.
\end{lemma}

\begin{proof}
Observe that $E^{\oplus n}$ is the injective hull of
$\kappa^{\oplus n} = M[\mathfrak m]$. Thus there is an $R$-module map
$M \to E^{\oplus n}$ which is injective on $M[\mathfrak m]$.
Since $M$ is $\mathfrak m$-power torsion the inclusion
$M[\mathfrak m] \subset M$ is an essential extension
(for example by Lemma \ref{lemma-essential-extension})
we conclude that the kernel of $M \to E^{\oplus n}$ is zero.
\end{proof}

\begin{lemma}
\label{lemma-union-artinian}
Let $(R, \mathfrak m, \kappa)$ be a Noetherian local ring.
Let $E$ be an injective hull of $\kappa$ over $R$.
Let $E_n$ be an injective hull of $\kappa$ over $R/\mathfrak m^n$.
Then $E = \bigcup E_n$ and $E_n = E[\mathfrak m^n]$.
\end{lemma}

\begin{proof}
We have $E_n = E[\mathfrak m^n]$ by Lemma \ref{lemma-quotient}.
We have $E = \bigcup E_n$ because $\bigcup E_n = E[\mathfrak m^\infty]$
is an injective $R$-submodule which contains $\kappa$, see
Lemma \ref{lemma-injective-module-divide}.
\end{proof}

\noindent
The following lemma tells us the injective hull of the residue
field of a Noetherian local ring only depends on the completion.

\begin{lemma}
\label{lemma-compare}
Let $R \to S$ be a flat local homomorphism of local Noetherian rings
such that $R/\mathfrak m_R \cong S/\mathfrak m_R S$.
Then the injective hull of the residue field
of $R$ is the injective hull of the residue field of $S$.
\end{lemma}

\begin{proof}
Set $\kappa = R/\mathfrak m_R = S/\mathfrak m_S$.
Let $E_R$ be the injective hull of $\kappa$ over $R$.
Let $E_S$ be the injective hull of $\kappa$ over $S$.
Observe that $E_S$ is an injective $R$-module by
Lemma \ref{lemma-injective-flat}.
Choose an extension $E_R \to E_S$ of the identifitaction of
residue fields. This map is an isomorphism by
Lemma \ref{lemma-union-artinian}
because $R \to S$ induces an isomorphism
$R/\mathfrak m_R^n \to S/\mathfrak m_S^n$ for all $n$.
\end{proof}

\begin{lemma}
\label{lemma-endos}
Let $(R, \mathfrak m, \kappa)$ be a Noetherian local ring.
Let $E$ be an injective hull of $\kappa$ over $R$. Then
$\Hom_R(E, E)$ is canonically isomorphic to the completion of $R$.
\end{lemma}

\begin{proof}
Write $E = \bigcup E_n$ with $E_n = E[\mathfrak m^n]$ as in
Lemma \ref{lemma-union-artinian}. Any endomorphism of $E$
preserves this filtration. Hence
$$
\Hom_R(E, E) = \lim \Hom_R(E_n, E_n)
$$
The lemma follows as
$\Hom_R(E_n, E_n) = \Hom_{R/\mathfrak m^n}(E_n, E_n) = R/\mathfrak m^n$
by Lemma \ref{lemma-evaluate}.
\end{proof}

\begin{lemma}
\label{lemma-injective-hull-has-dcc}
Let $(R, \mathfrak m, \kappa)$ be a Noetherian local ring.
Let $E$ be an injective hull of $\kappa$ over $R$. Then
$E$ satisfies the descending chain condition.
\end{lemma}

\begin{proof}
If $E \subset M_1 \subset M_2 \ldots$ is a sequence of submodules, then
$$
\Hom_R(E, E) \to \Hom_R(M_1, E) \to \Hom_R(M_2, E) \to \ldots
$$
is sequence of surjections. By Lemma \ref{lemma-endos} each of these is a
module over the completion $R^\wedge = \Hom_R(E, E)$.
Since $R^\wedge$ is Noetherian
(Algebra, Lemma \ref{algebra-lemma-completion-Noetherian-Noetherian})
the sequence stabilizes: $\Hom_R(M_n, E) = \Hom_R(M_{n + 1}, E) = \ldots$.
Since $E$ is injective, this can only happen if $\Hom_R(M_n/M_{n + 1}, E)$
is zero. However, if $M_n/M_{n + 1}$ is nonzero, then it contains a
nonzero element annihilated by $\mathfrak m$, because $E$ is
$\mathfrak m$-power torsion by Lemma \ref{lemma-union-artinian}.
In this case $M_n/M_{n + 1}$ has a nonzero map into $E$, contradicting
the assumed vanishing. This finishes the proof.
\end{proof}

\begin{lemma}
\label{lemma-describe-categories}
Let $(R, \mathfrak m, \kappa)$ be a Noetherian local ring.
Let $E$ be an injective hull of $\kappa$.
\begin{enumerate}
\item For an $R$-module $M$ the following are equivalent:
\begin{enumerate}
\item $M$ satisfies the ascending chain condition,
\item $M$ is a finite $R$-module, and
\item there exist $n, m$ and an exact sequence
$R^{\oplus m} \to R^{\oplus n} \to M \to 0$.
\end{enumerate}
\item For an $R$-module $M$ the following are equivalent:
\begin{enumerate}
\item $M$ satisfies the descending chain condition,
\item $M$ is $\mathfrak m$-power torsion and
$\dim_\kappa(M[\mathfrak m]) < \infty$, and
\item there exist $n, m$ and an exact sequence
$0 \to M \to E^{\oplus n} \to E^{\oplus m}$.
\end{enumerate}
\end{enumerate}
\end{lemma}

\begin{proof}
We omit the proof of (1).

\medskip\noindent
Let $M$ be an $R$-module with the descending chain condition. Let $x \in M$.
Then $\mathfrak m^n x$ is a descending chain of submodules, hence stabilizes.
Thus $\mathfrak m^nx = \mathfrak m^{n + 1}x$ for some $n$. By Nakayama's lemma
(Algebra, Lemma \ref{algebra-lemma-NAK}) this implies $\mathfrak m^n x = 0$,
i.e., $x$ is $\mathfrak m$-power torsion. Since $M[\mathfrak m]$ is a vector
space over $\kappa$ it has to be finite dimensional in order to have the
descending chain condition.

\medskip\noindent
Assume that $M$ is $\mathfrak m$-power torsion and has a finite dimensional
$\mathfrak m$-torsion submodule $M[\mathfrak m]$. By
Lemma \ref{lemma-torsion-submodule-sum-injective-hulls}
we see that $M$ is a submodule of $E^{\oplus n}$ for some $n$.
Consider the quotient $N = E^{\oplus n}/M$. By
Lemma \ref{lemma-injective-hull-has-dcc} the module $E$ has the
descending chain condition hence so do $E^{\oplus n}$ and $N$.
Therefore $N$ satisfies (2)(a) which implies $N$ satisfies
(2)(b) by the second paragraph of the proof. Thus by
Lemma \ref{lemma-torsion-submodule-sum-injective-hulls}
again we see that $N$ is a submodule of $E^{\oplus m}$ for some $m$.
Thus we have a short exact sequence
$0 \to M \to E^{\oplus n} \to E^{\oplus m}$.

\medskip\noindent
Assume we have a short exact sequence
$0 \to M \to E^{\oplus n} \to E^{\oplus m}$.
Since $E$ satisfies the descending chain condition by
Lemma \ref{lemma-injective-hull-has-dcc}
so does $M$.
\end{proof}

\begin{proposition}[Matlis duality]
\label{proposition-matlis}
Let $(R, \mathfrak m, \kappa)$ be a complete local Noetherian ring.
Let $E$ be an injective hull of $\kappa$ over $R$. The functor
$D(-) = \Hom_R(-, E)$ induces an anti-equivalence
$$
\left\{
\begin{matrix}
R\text{-modules with the} \\
\text{descending chain condition}
\end{matrix}
\right\}
\longleftrightarrow
\left\{
\begin{matrix}
R\text{-modules with the} \\
\text{ascending chain condition}
\end{matrix}
\right\}
$$
and we have $D \circ D = \text{id}$ on either side of the equivalence.
\end{proposition}

\begin{proof}
By Lemma \ref{lemma-endos} we have $R = \Hom_R(E, E) = D(E)$.
Of course we have $E = \Hom_R(R, E) = D(R)$. Since $E$ is injective
the functor $D$ is exact. The result now follows immediately from the
description of the categories in
Lemma \ref{lemma-describe-categories}.
\end{proof}




















\section{Local cohomology}
\label{section-local-cohomology}

\noindent
Let $A$ be a ring and let $I \subset A$ be a finitely generated ideal
(if $I$ is not finitely generated perhaps a different definition
should be used). Consider the functor
$$
\Gamma_Z : \text{Mod}_A \longrightarrow \text{Mod}_A,\quad
M \longmapsto \bigcup M[I^n]
$$
which sends $M$ to the submodule of $I$-power torsion.
This functor is left exact and has a derived extension
$R\Gamma_Z : D(A) \to D(A)$. Note that for every $K \in D(A)$
there is a canonical map $R\Gamma_Z(K) \to K$.

\begin{lemma}
\label{lemma-local-cohomology-closed}
With notation as above. The functors $\Gamma_Z$ and $R\Gamma_Z$
depend only on the closed subset $Z = V(I) \subset \Spec(A)$.
\end{lemma}

\begin{proof}
Let $M$ be an $A$-module. Let $x \in M$. If $x \in \Gamma_Z(M)$, then $x$
maps to zero in $M_f$ for all $f \in I$. Hence $x$ maps to zero in
$M_\mathfrak p$ for all $\mathfrak p \not \supset I$. Conversely, if $x$
maps to zero in $M_\mathfrak p$ for all $\mathfrak p \not \supset I$,
then $x$ maps to zero in $M_f$ for all $f \in I$. Hence if
$I = (f_1, \ldots, f_r)$, then $f_i^{n_i}x = 0$ for some $n_i \geq 1$.
It follows that $x \in M[I^{\sum n_i}]$. Thus $\Gamma_Z(M)$ is
the kernel of $M \to \prod_{\mathfrak p \not \in Z} M_\mathfrak p$.
\end{proof}

\begin{lemma}
\label{lemma-local-cohomology-ext}
With notation as above. For any $A$-module $M$ we have
$$
R^q\Gamma_Z(M) = \colim_n \text{Ext}_A^q(A/I^n, M)
$$
\end{lemma}

\begin{proof}
This follows immediately from the fact that $M[I^n] = \Hom_A(A/I^n, M)$.
\end{proof}

\begin{lemma}
\label{lemma-local-cohomology-if-annihilated-noetherian}
With notation as above. If $A$ is Noetherian and the $A$-module $M$
is $I$-power torsion, then the canonical map $R\Gamma_Z(M) \to M$ is an
isomorphism in $D(A)$.
\end{lemma}

\begin{proof}
By Lemma \ref{lemma-injective-module-divide}
we see that $M$ has an injective resolution $M \to J^0 \to J^1 \to \ldots$
with $J^p = J^p[I^\infty]$ completely $I$-power torsion. The result follows.
\end{proof}

\begin{lemma}
\label{lemma-compute-local-cohomology-noetherian}
If $A$ is a Noetherian ring and $I = (f_1, \ldots, f_r)$ an ideal.
There is a canonical isomorphism
$$
(A \to \prod\nolimits_{i_0} A_{f_{i_0}} \to
\prod\nolimits_{i_0 < i_1} A_{f_{i_0}f_{i_1}} \to
\ldots \to A_{f_1\ldots f_r}) \longrightarrow R\Gamma_Z(A)
$$
in $D(A)$.
\end{lemma}

\begin{proof}
There is a canonical map of complexes
$$
(A \to \prod\nolimits_{i_0} A_{f_{i_0}} \to
\prod\nolimits_{i_0 < i_1} A_{f_{i_0}f_{i_1}} \to
\ldots \to A_{f_1\ldots f_r}) \longrightarrow A.
$$
Applying $R\Gamma_Z$ we get a map
$$
R\Gamma_Z(A \to \prod\nolimits_{i_0} A_{f_{i_0}} \to
\prod\nolimits_{i_0 < i_1} A_{f_{i_0}f_{i_1}} \to
\ldots \to A_{f_1\ldots f_r}) \longrightarrow R\Gamma_Z(A).
$$
This map is an isomorphism as clearly $R\Gamma_Z(A_f) = 0$
for all $f \in I$. On the other hand, the homology modules
of the complex
$A \to \prod\nolimits_{i_0} A_{f_{i_0}} \to
\prod\nolimits_{i_0 < i_1} A_{f_{i_0}f_{i_1}} \to
\ldots \to A_{f_1\ldots f_r}$
are $I$-power torsion by
More on Algebra, Lemmas
\ref{more-algebra-lemma-extended-alternating-Cech-is-colimit-koszul} and
\ref{more-algebra-lemma-homotopy-koszul}. Thus the canonical map
$$
\xymatrix{
R\Gamma_Z(A \to \prod\nolimits_{i_0} A_{f_{i_0}} \to
\prod\nolimits_{i_0 < i_1} A_{f_{i_0}f_{i_1}} \to
\ldots \to A_{f_1\ldots f_r}) \ar[d] \\
(A \to \prod\nolimits_{i_0} A_{f_{i_0}} \to
\prod\nolimits_{i_0 < i_1} A_{f_{i_0}f_{i_1}} \to
\ldots \to A_{f_1\ldots f_r})
}
$$
is an isomorphism in $D(A)$ by
Lemma \ref{lemma-local-cohomology-if-annihilated-noetherian}
and the spectral sequence of
Derived Categories, Lemma \ref{derived-lemma-two-ss-complex-functor}.
\end{proof}

\begin{lemma}
\label{lemma-local-cohomology-change-rings}
If $A \to B$ is a homomorphism of Noetherian rings and $I \subset A$
is an ideal, then
$$
R\Gamma_Z(A) \otimes_A^\mathbf{L} B = R\Gamma_Y(B)
$$
where $Y = V(IB) \subset \Spec(B)$.
\end{lemma}

\begin{proof}
This is clear from Lemma \ref{lemma-compute-local-cohomology-noetherian}
and the fact that formation of the extended {\v C}ech complex
commutes with base change.
\end{proof}








\section{Other chapters}

\begin{multicols}{2}
\begin{enumerate}
\item \hyperref[introduction-section-phantom]{Introduction}
\item \hyperref[conventions-section-phantom]{Conventions}
\item \hyperref[sets-section-phantom]{Set Theory}
\item \hyperref[categories-section-phantom]{Categories}
\item \hyperref[topology-section-phantom]{Topology}
\item \hyperref[sheaves-section-phantom]{Sheaves on Spaces}
\item \hyperref[algebra-section-phantom]{Commutative Algebra}
\item \hyperref[sites-section-phantom]{Sites and Sheaves}
\item \hyperref[homology-section-phantom]{Homological Algebra}
\item \hyperref[derived-section-phantom]{Derived Categories}
\item \hyperref[more-algebra-section-phantom]{More Algebra}
\item \hyperref[simplicial-section-phantom]{Simplicial Methods}
\item \hyperref[modules-section-phantom]{Sheaves of Modules}
\item \hyperref[sites-modules-section-phantom]{Modules on Sites}
\item \hyperref[injectives-section-phantom]{Injectives}
\item \hyperref[cohomology-section-phantom]{Cohomology of Sheaves}
\item \hyperref[sites-cohomology-section-phantom]{Cohomology on Sites}
\item \hyperref[hypercovering-section-phantom]{Hypercoverings}
\item \hyperref[schemes-section-phantom]{Schemes}
\item \hyperref[constructions-section-phantom]{Constructions of Schemes}
\item \hyperref[properties-section-phantom]{Properties of Schemes}
\item \hyperref[morphisms-section-phantom]{Morphisms of Schemes}
\item \hyperref[coherent-section-phantom]{Coherent Cohomology}
\item \hyperref[divisors-section-phantom]{Divisors}
\item \hyperref[limits-section-phantom]{Limits of Schemes}
\item \hyperref[varieties-section-phantom]{Varieties}
\item \hyperref[chow-section-phantom]{Chow Homology}
\item \hyperref[topologies-section-phantom]{Topologies on Schemes}
\item \hyperref[descent-section-phantom]{Descent}
\item \hyperref[more-morphisms-section-phantom]{More on Morphisms}
\item \hyperref[flat-section-phantom]{More on Flatness}
\item \hyperref[groupoids-section-phantom]{Groupoid Schemes}
\item \hyperref[more-groupoids-section-phantom]{More on Groupoid Schemes}
\item \hyperref[etale-section-phantom]{\'Etale Morphisms of Schemes}
\item \hyperref[etale-cohomology-section-phantom]{\'Etale Cohomology}
\item \hyperref[spaces-section-phantom]{Algebraic Spaces}
\item \hyperref[spaces-properties-section-phantom]{Properties of Algebraic Spaces}
\item \hyperref[spaces-morphisms-section-phantom]{Morphisms of Algebraic Spaces}
\item \hyperref[spaces-topologies-section-phantom]{Topologies on Algebraic Spaces}
\item \hyperref[spaces-descent-section-phantom]{Descent and Algebraic Spaces}
\item \hyperref[spaces-more-morphisms-section-phantom]{More on Morphisms of Spaces}
\item \hyperref[quot-section-phantom]{Quot and Hilbert Spaces}
\item \hyperref[stacks-section-phantom]{Stacks}
\item \hyperref[spaces-groupoids-section-phantom]{Groupoids in Algebraic Spaces}
\item \hyperref[spaces-more-groupoids-section-phantom]{More on Groupoids in Spaces}
\item \hyperref[bootstrap-section-phantom]{Bootstrap}
\item \hyperref[examples-stacks-section-phantom]{Examples of Stacks}
\item \hyperref[groupoids-quotients-section-phantom]{Quotients of Groupoids}
\item \hyperref[algebraic-section-phantom]{Algebraic Stacks}
\item \hyperref[criteria-section-phantom]{Criteria for Representability}
\item \hyperref[stacks-properties-section-phantom]{Properties of Algebraic Stacks}
\item \hyperref[stacks-morphisms-section-phantom]{Morphisms of Algebraic Stacks}
\item \hyperref[examples-section-phantom]{Examples}
\item \hyperref[exercises-section-phantom]{Exercises}
\item \hyperref[guide-section-phantom]{Guide to Literature}
\item \hyperref[desirables-section-phantom]{Desirables}
\item \hyperref[coding-section-phantom]{Coding Style}
\item \hyperref[fdl-section-phantom]{GNU Free Documentation License}
\item \hyperref[index-section-phantom]{Auto Generated Index}
\end{enumerate}
\end{multicols}


\bibliography{my}
\bibliographystyle{amsalpha}

\end{document}
