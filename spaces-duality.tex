\IfFileExists{stacks-project.cls}{%
\documentclass{stacks-project}
}{%
\documentclass{amsart}
}

% The following AMS packages are automatically loaded with
% the amsart documentclass:
%\usepackage{amsmath}
%\usepackage{amssymb}
%\usepackage{amsthm}

% For dealing with references we use the comment environment
\usepackage{verbatim}
\newenvironment{reference}{\comment}{\endcomment}
%\newenvironment{reference}{}{}
\newenvironment{slogan}{\comment}{\endcomment}
\newenvironment{history}{\comment}{\endcomment}

% For commutative diagrams you can use
% \usepackage{amscd}
\usepackage[all]{xy}

% We use 2cell for 2-commutative diagrams.
\xyoption{2cell}
\UseAllTwocells

% To put source file link in headers.
% Change "template.tex" to "this_filename.tex"
% \usepackage{fancyhdr}
% \pagestyle{fancy}
% \lhead{}
% \chead{}
% \rhead{Source file: \url{template.tex}}
% \lfoot{}
% \cfoot{\thepage}
% \rfoot{}
% \renewcommand{\headrulewidth}{0pt}
% \renewcommand{\footrulewidth}{0pt}
% \renewcommand{\headheight}{12pt}

\usepackage{multicol}

% For cross-file-references
\usepackage{xr-hyper}

% Package for hypertext links:
\usepackage{hyperref}

% For any local file, say "hello.tex" you want to link to please
% use \externaldocument[hello-]{hello}
\externaldocument[introduction-]{introduction}
\externaldocument[conventions-]{conventions}
\externaldocument[sets-]{sets}
\externaldocument[categories-]{categories}
\externaldocument[topology-]{topology}
\externaldocument[sheaves-]{sheaves}
\externaldocument[sites-]{sites}
\externaldocument[stacks-]{stacks}
\externaldocument[fields-]{fields}
\externaldocument[algebra-]{algebra}
\externaldocument[brauer-]{brauer}
\externaldocument[homology-]{homology}
\externaldocument[derived-]{derived}
\externaldocument[simplicial-]{simplicial}
\externaldocument[more-algebra-]{more-algebra}
\externaldocument[smoothing-]{smoothing}
\externaldocument[modules-]{modules}
\externaldocument[sites-modules-]{sites-modules}
\externaldocument[injectives-]{injectives}
\externaldocument[cohomology-]{cohomology}
\externaldocument[sites-cohomology-]{sites-cohomology}
\externaldocument[dga-]{dga}
\externaldocument[dpa-]{dpa}
\externaldocument[hypercovering-]{hypercovering}
\externaldocument[schemes-]{schemes}
\externaldocument[constructions-]{constructions}
\externaldocument[properties-]{properties}
\externaldocument[morphisms-]{morphisms}
\externaldocument[coherent-]{coherent}
\externaldocument[divisors-]{divisors}
\externaldocument[limits-]{limits}
\externaldocument[varieties-]{varieties}
\externaldocument[topologies-]{topologies}
\externaldocument[descent-]{descent}
\externaldocument[perfect-]{perfect}
\externaldocument[more-morphisms-]{more-morphisms}
\externaldocument[flat-]{flat}
\externaldocument[groupoids-]{groupoids}
\externaldocument[more-groupoids-]{more-groupoids}
\externaldocument[etale-]{etale}
\externaldocument[chow-]{chow}
\externaldocument[intersection-]{intersection}
\externaldocument[pic-]{pic}
\externaldocument[adequate-]{adequate}
\externaldocument[dualizing-]{dualizing}
\externaldocument[duality-]{duality}
\externaldocument[discriminant-]{discriminant}
\externaldocument[local-cohomology-]{local-cohomology}
\externaldocument[curves-]{curves}
\externaldocument[resolve-]{resolve}
\externaldocument[models-]{models}
\externaldocument[pione-]{pione}
\externaldocument[etale-cohomology-]{etale-cohomology}
\externaldocument[proetale-]{proetale}
\externaldocument[crystalline-]{crystalline}
\externaldocument[spaces-]{spaces}
\externaldocument[spaces-properties-]{spaces-properties}
\externaldocument[spaces-morphisms-]{spaces-morphisms}
\externaldocument[decent-spaces-]{decent-spaces}
\externaldocument[spaces-cohomology-]{spaces-cohomology}
\externaldocument[spaces-limits-]{spaces-limits}
\externaldocument[spaces-divisors-]{spaces-divisors}
\externaldocument[spaces-over-fields-]{spaces-over-fields}
\externaldocument[spaces-topologies-]{spaces-topologies}
\externaldocument[spaces-descent-]{spaces-descent}
\externaldocument[spaces-perfect-]{spaces-perfect}
\externaldocument[spaces-more-morphisms-]{spaces-more-morphisms}
\externaldocument[spaces-flat-]{spaces-flat}
\externaldocument[spaces-groupoids-]{spaces-groupoids}
\externaldocument[spaces-more-groupoids-]{spaces-more-groupoids}
\externaldocument[bootstrap-]{bootstrap}
\externaldocument[spaces-pushouts-]{spaces-pushouts}
\externaldocument[groupoids-quotients-]{groupoids-quotients}
\externaldocument[spaces-more-cohomology-]{spaces-more-cohomology}
\externaldocument[spaces-simplicial-]{spaces-simplicial}
\externaldocument[formal-spaces-]{formal-spaces}
\externaldocument[restricted-]{restricted}
\externaldocument[spaces-resolve-]{spaces-resolve}
\externaldocument[formal-defos-]{formal-defos}
\externaldocument[defos-]{defos}
\externaldocument[cotangent-]{cotangent}
\externaldocument[examples-defos-]{examples-defos}
\externaldocument[algebraic-]{algebraic}
\externaldocument[examples-stacks-]{examples-stacks}
\externaldocument[stacks-sheaves-]{stacks-sheaves}
\externaldocument[criteria-]{criteria}
\externaldocument[artin-]{artin}
\externaldocument[quot-]{quot}
\externaldocument[stacks-properties-]{stacks-properties}
\externaldocument[stacks-morphisms-]{stacks-morphisms}
\externaldocument[stacks-limits-]{stacks-limits}
\externaldocument[stacks-cohomology-]{stacks-cohomology}
\externaldocument[stacks-perfect-]{stacks-perfect}
\externaldocument[stacks-introduction-]{stacks-introduction}
\externaldocument[stacks-more-morphisms-]{stacks-more-morphisms}
\externaldocument[stacks-geometry-]{stacks-geometry}
\externaldocument[moduli-]{moduli}
\externaldocument[moduli-curves-]{moduli-curves}
\externaldocument[examples-]{examples}
\externaldocument[exercises-]{exercises}
\externaldocument[guide-]{guide}
\externaldocument[desirables-]{desirables}
\externaldocument[coding-]{coding}
\externaldocument[obsolete-]{obsolete}
\externaldocument[fdl-]{fdl}
\externaldocument[index-]{index}

% Theorem environments.
%
\theoremstyle{plain}
\newtheorem{theorem}[subsection]{Theorem}
\newtheorem{proposition}[subsection]{Proposition}
\newtheorem{lemma}[subsection]{Lemma}

\theoremstyle{definition}
\newtheorem{definition}[subsection]{Definition}
\newtheorem{example}[subsection]{Example}
\newtheorem{exercise}[subsection]{Exercise}
\newtheorem{situation}[subsection]{Situation}

\theoremstyle{remark}
\newtheorem{remark}[subsection]{Remark}
\newtheorem{remarks}[subsection]{Remarks}

\numberwithin{equation}{subsection}

% Macros
%
\def\lim{\mathop{\rm lim}\nolimits}
\def\colim{\mathop{\rm colim}\nolimits}
\def\Spec{\mathop{\rm Spec}}
\def\Hom{\mathop{\rm Hom}\nolimits}
\def\Ext{\mathop{\rm Ext}\nolimits}
\def\SheafHom{\mathop{\mathcal{H}\!{\it om}}\nolimits}
\def\SheafExt{\mathop{\mathcal{E}\!{\it xt}}\nolimits}
\def\Sch{\textit{Sch}}
\def\Mor{\mathop{\rm Mor}\nolimits}
\def\Ob{\mathop{\rm Ob}\nolimits}
\def\Sh{\mathop{\textit{Sh}}\nolimits}
\def\NL{\mathop{N\!L}\nolimits}
\def\proetale{{pro\text{-}\acute{e}tale}}
\def\etale{{\acute{e}tale}}
\def\QCoh{\textit{QCoh}}
\def\Ker{\mathop{\rm Ker}}
\def\Im{\mathop{\rm Im}}
\def\Coker{\mathop{\rm Coker}}
\def\Coim{\mathop{\rm Coim}}

%
% Macros for moduli stacks/spaces
%
\def\QCohstack{\mathcal{QC}\!{\it oh}}
\def\Cohstack{\mathcal{C}\!{\it oh}}
\def\Spacesstack{\mathcal{S}\!{\it paces}}
\def\Quotfunctor{{\rm Quot}}
\def\Hilbfunctor{{\rm Hilb}}
\def\Curvesstack{\mathcal{C}\!{\it urves}}
\def\Polarizedstack{\mathcal{P}\!{\it olarized}}
\def\Complexesstack{\mathcal{C}\!{\it omplexes}}
% \Pic is the operator that assigns to X its picard group, usage \Pic(X)
% \Picardstack_{X/B} denotes the Picard stack of X over B
% \Picardfunctor_{X/B} denotes the Picard functor of X over B
\def\Pic{\mathop{\rm Pic}\nolimits}
\def\Picardstack{\mathcal{P}\!{\it ic}}
\def\Picardfunctor{{\rm Pic}}
\def\Deformationcategory{\mathcal{D}\!{\it ef}}


% OK, start here.
%
\begin{document}

\title{Duality for Spaces}


\maketitle

\phantomsection
\label{section-phantom}

\tableofcontents

\section{Introduction}
\label{section-introduction}

\noindent
This chapter is the analogue of the corresponding chapter for
schemes, see Duality for Schemes, Section \ref{duality-section-introduction}.
The development is similar to the development in the papers
\cite{Neeman-Grothendieck}, \cite{LN},
\cite{Lipman-notes}, and \cite{Neeman-improvement}.




\section{Dualizing complexes on algebraic spaces}
\label{section-dualizing-spaces}

\noindent
Let $U$ be a locally Noetherian scheme. Let $\mathcal{O}_\etale$
be the structure sheaf of $U$ on the small \'etale site of $U$.
We will say an object $K \in D_\QCoh(\mathcal{O}_\etale)$ is
a dualizing complex on $U$ if $K = \epsilon^*(\omega_U^\bullet)$
for some dualizing complex $\omega_U^\bullet$ in the sense of
Duality for Schemes, Section \ref{duality-section-dualizing-schemes}.
Here $\epsilon^* : D_\QCoh(\mathcal{O}_U) \to D_\QCoh(\mathcal{O}_\etale)$
is the equivalence of Derived Categories of Spaces, Lemma
\ref{spaces-perfect-lemma-derived-quasi-coherent-small-etale-site}.
Most of the properties of $\omega_U^\bullet$ studied in
Duality for Schemes, Section \ref{duality-section-dualizing-schemes}
are inherited by $K$ via the discussion in
Derived Categories of Spaces, Sections
\ref{spaces-perfect-section-derived-quasi-coherent-etale} and
\ref{spaces-perfect-section-spell-out}.

\medskip\noindent
We define a dualizing complex on a locally Noetherian algebraic space
to be a complex which \'etale locally comes from a dualizing
complex on the corresponding scheme.

\begin{lemma}
\label{lemma-equivalent-definitions}
Let $S$ be a scheme. Let $X$ be a locally Noetherian algebraic space over $S$.
Let $K$ be an object of $D_\QCoh(\mathcal{O}_X)$. The following are equivalent
\begin{enumerate}
\item For every \'etale morphism $U \to X$ where $U$ is a scheme
the restriction $K|_U$ is a dualizing complex for $U$ (as discussed above).
\item There exists a surjective \'etale morphism $U \to X$ where $U$ is a
scheme such that $K|_U$ is a dualizing complex for $U$.
\end{enumerate}
\end{lemma}

\begin{proof}
Assume $U \to X$ is surjective \'etale where $U$ is a scheme.
Let $V \to X$ be an \'etale morphism where $V$ is a scheme.
Then
$$
U \leftarrow U \times_X V \rightarrow V
$$
are \'etale morphisms of schemes with the arrow to $V$ surjective.
Hence we can use Duality for Schemes, Lemma \ref{duality-lemma-descent-ascent}
to see that if $K|_U$ is a dualizing complex for $U$, then
$K|_V$ is a dualizing complex for $V$.
\end{proof}

\begin{definition}
\label{definition-dualizing-scheme}
Let $S$ be a scheme.
Let $X$ be a locally Noetherian algebraic space over $S$.
An object $K$ of $D_\QCoh(\mathcal{O}_X)$ is called a
{\it dualizing complex} if $K$ satisfies the equivalent conditions of
Lemma \ref{lemma-equivalent-definitions}.
\end{definition}

\begin{lemma}
\label{lemma-affine-duality}
Let $A$ be a Noetherian ring and let $X = \Spec(A)$. Let
$\mathcal{O}_\etale$ be the structure sheaf of $X$ on the
small \'etale site of $X$. Let $K, L$ be objects of $D(A)$.
If $K \in D_{\textit{Coh}}(A)$ and $L$ has finite injective
dimension, then
$$
\epsilon^*\widetilde{R\Hom_A(K, L)} =
R\SheafHom_{\mathcal{O}_\etale}(\epsilon^*\widetilde{K},
\epsilon^*\widetilde{L})
$$
in $D(\mathcal{O}_\etale)$ where
$\epsilon : (X_\etale, \mathcal{O}_\etale) \to (X, \mathcal{O}_X)$
is as in Derived Categories of Spaces, Section
\ref{spaces-perfect-section-derived-quasi-coherent-etale}.
\end{lemma}

\begin{proof}
By Duality for Schemes, Lemma \ref{duality-lemma-affine-duality}
we have a canonical isomorphism
$$
\widetilde{R\Hom_A(K, L)} =
R\SheafHom_{\mathcal{O}_X}(\widetilde{K}, \widetilde{L})
$$
in $D(\mathcal{O}_X)$. There is a canonical map
$$
\epsilon^*R\Hom_{\mathcal{O}_X}(\widetilde{K}, \widetilde{L})
\longrightarrow
R\SheafHom_{\mathcal{O}_\etale}(\epsilon^*\widetilde{K},
\epsilon^*\widetilde{L})
$$
in $D(\mathcal{O}_\etale)$, see Cohomology on Sites, Remark
\ref{sites-cohomology-remark-prepare-fancy-base-change}.
We will show the left and right hand side of this arrow
have isomorphic cohomology sheaves, but we will omit the
verification that the isomorphism is given by this arrow.

\medskip\noindent
We may assume that $L$ is given by a finite complex $I^\bullet$
of injective $A$-modules. By induction on the length of $I^\bullet$
and compatibility of the constructions with distinguished triangles,
we reduce to the case that $L = I[0]$ where $I$ is an injective $A$-module.
Recall that the cohomology sheaves of
$R\SheafHom_{\mathcal{O}_\etale}(\epsilon^*\widetilde{K},
\epsilon^*\widetilde{L}))$
are the sheafifications of the presheaf sending $U$ \'etale
over $X$ to the $i$th ext group between the restrictions of
$\epsilon^*\widetilde{K}$ and $\epsilon^*\widetilde{L}$
to $U_\etale$. See
Cohomology on Sites, Lemma
\ref{sites-cohomology-lemma-section-RHom-over-U}.
If $U = \Spec(B)$ is affine, then this ext group
is equal to $\text{Ext}^i_B(K \otimes_A B, L \otimes_A B)$
by the equivalence of
Derived Categories of Spaces, Lemma
\ref{spaces-perfect-lemma-derived-quasi-coherent-small-etale-site} and
Derived Categories of Schemes, Lemma
\ref{perfect-lemma-affine-compare-bounded}
(this also uses the compatibilities detailed in
Derived Categories of Spaces, Remark
\ref{spaces-perfect-remark-match-total-direct-images}).
Since $A \to B$ is \'etale, we see that
$I \otimes_A B$ is an injective $B$-module
by Dualizing Complexes, Lemma \ref{dualizing-lemma-injective-goes-up}.
Hence we see that
\begin{align*}
\Ext^n_B(K \otimes_A B, I \otimes_A B)
& =
\Hom_B(H^{-n}(K \otimes_A B), I \otimes_A B) \\
& =
\Hom_{A_f}(H^{-n}(K) \otimes_A B, I \otimes_A B) \\
& =
\Hom_A(H^{-n}(K), I) \otimes_A B \\
& =
\text{Ext}^n_A(K, I) \otimes_A B
\end{align*}
The penultimate equality because $H^{-n}(K)$ is a finite $A$-module, see
More on Algebra, Remark
\ref{more-algebra-remark-pseudo-coherence-and-base-change-ext}.
Therefore the cohomology sheaves of the left and right hand
side of the equality in the lemma are the same.
\end{proof}

\begin{lemma}
\label{lemma-dualizing-spaces}
Let $S$ be a scheme. Let $X$ be a locally Noetherian algebraic space over $S$.
Let $K$ be a dualizing complex on $X$.
Then $K$ is an object of $D_{\textit{Coh}}(\mathcal{O}_X)$
and $D = R\SheafHom_{\mathcal{O}_X}(-, K)$ induces an anti-equivalence
$$
D :
D_{\textit{Coh}}(\mathcal{O}_X)
\longrightarrow
D_{\textit{Coh}}(\mathcal{O}_X)
$$
which comes equipped with a canonical isomorphism
$\text{id} \to D \circ D$. If $X$ is quasi-compact, then
$D$ exchanges $D^+_{\textit{Coh}}(\mathcal{O}_X)$ and
$D^-_{\textit{Coh}}(\mathcal{O}_X)$ and induces an equivalence
$D^b_{\textit{Coh}}(\mathcal{O}_X) \to D^b_{\textit{Coh}}(\mathcal{O}_X)$.
\end{lemma}

\begin{proof}
Let $U \to X$ be an \'etale morphism with $U$ affine. Say $U = \Spec(A)$ and
let $\omega_A^\bullet$ be a dualizing complex for $A$ corresponding to $K|_U$
as in Lemma \ref{lemma-equivalent-definitions} and
Duality for Schemes, Lemma \ref{duality-lemma-equivalent-definitions}.
By Lemma \ref{lemma-affine-duality} the diagram
$$
\xymatrix{
D_{\textit{Coh}}(A) \ar[r] \ar[d]_{R\Hom_A(-, \omega_A^\bullet)} &
D_{\textit{Coh}}(\mathcal{O}_\etale)
\ar[d]^{R\SheafHom_{\mathcal{O}_\etale}(-, K|_U)} \\
D_{\textit{Coh}}(A) \ar[r] &
D(\mathcal{O}_\etale)
}
$$
commutes where $\mathcal{O}_\etale$ is the structure sheaf of the
small \'etale site of $U$. Since formation of $R\SheafHom$ commutes
with restriction, we conclude that $D$ sends
$D_{\textit{Coh}}(\mathcal{O}_X)$ into
$D_{\textit{Coh}}(\mathcal{O}_X)$. Moreover, the canonical map
$$
L \longrightarrow
R\SheafHom_{\mathcal{O}_X}(R\SheafHom_{\mathcal{O}_X}(L, K), K)
$$
(Cohomology on Sites, Lemma \ref{sites-cohomology-lemma-internal-hom-evaluate})
is an isomorphism for all $L$ in $D_{\textit{Coh}}(\mathcal{O}_X)$
because this is true over all $U$ as above by
Dualizing Complexes, Lemma \ref{dualizing-lemma-dualizing}.
The statement on boundedness properties of the functor $D$
in the quasi-compact case also follows from the corresponding
statements of Dualizing Complexes, Lemma \ref{dualizing-lemma-dualizing}.
\end{proof}

\noindent
Let $(\mathcal{C}, \mathcal{O})$ be a ringed site.
We will say that an object $L$ of $D(\mathcal{O})$ is {\it invertible}
if for every object $U$ of $\mathcal{C}$
there is a covering $\{U_i \to U\}$ of $U$ in $\mathcal{C}$
such that $L|_{U_i} \cong \mathcal{O}_{U_i}[-n_i]$
for some integers $n_i$.

\medskip\noindent
Let $S$ be a scheme and let $X$ be an algebraic space over $S$.
If $L$ in $D(\mathcal{O}_X)$ is invertible, then there is a
disjoint union decomposition $X = \coprod_{n \in \mathbf{Z}} X_n$
such that $L|_{X_n}$ is an invertible module sitting in degree $n$.
In particular, it follows that $L = \bigoplus H^n(L)[-n]$
which gives a well defined complex of $\mathcal{O}_X$-modules
(with zero differentials) representing $L$.
Moreover, we see that $L$ is a perfect object of $D(\mathcal{O}_X)$.

\begin{lemma}
\label{lemma-dualizing-unique-spaces}
Let $S$ be a scheme.
Let $X$ be a locally Noetherian algebraic space over $S$.
If $K$ and $K'$ are dualizing complexes on $X$, then $K'$
is isomorphic to $K \otimes_{\mathcal{O}_X}^\mathbf{L} L$
for some invertible object $L$ of $D(\mathcal{O}_X)$.
\end{lemma}

\begin{proof}
Set
$$
L = R\SheafHom_{\mathcal{O}_X}(K, K')
$$
This is an invertible object of $D(\mathcal{O}_X)$, because affine locally
this is true. Use Lemma \ref{lemma-affine-duality} and
Dualizing Complexes, Lemma
\ref{dualizing-lemma-dualizing-unique} and its proof.
The evaluation map $L \otimes_{\mathcal{O}_X}^\mathbf{L} K \to K'$
is an isomorphism for the same reason.
\end{proof}

\begin{lemma}
\label{lemma-dimension-function-scheme}
Let $S$ be a scheme. Let $X$ be a locally Noetherian
quasi-separated algebraic space over $S$.
Let $\omega_X^\bullet$ be a dualizing complex on $X$. Then $X$ the function
$|X| \to \mathbf{Z}$ defined by
$$
x \longmapsto \delta(x)\text{ such that }
\omega_{X, \overline{x}}^\bullet[-\delta(x)]
\text{ is a normalized dualizing complex over }
\mathcal{O}_{X, \overline{x}}
$$
is a dimension function on $|X|$.
\end{lemma}

\begin{proof}
Let $U$ be a scheme and let $U \to X$ be a surjective \'etale morphism.
Let $\omega_U^\bullet$ be the dualizing complex on $U$ associated
to $\omega_X^\bullet|_U$.
If $u \in U$ maps to $x \in |X|$, then $\mathcal{O}_{X, \overline{x}}$
is the strict henselization of $\mathcal{O}_{U, u}$. By
Dualizing Complexes, Lemma \ref{dualizing-lemma-flat-unramified}
we see that if $\omega^\bullet$ is a normalized dualizing complex
for $\mathcal{O}_{U, u}$, then
$\omega^\bullet \otimes_{\mathcal{O}_{U, u}} \mathcal{O}_{X, \overline{x}}$
is a normalized dualizing complex for $\mathcal{O}_{X, \overline{x}}$.
Hence we see that the dimension function $U \to \mathbf{Z}$ of
Duality for Schemes, Lemma \ref{duality-lemma-dimension-function-scheme}
for the scheme $U$ and the complex
$\omega_U^\bullet$ is equal to the composition of $U \to |X|$ with $\delta$.
Using the specializations in $|X|$ lift to specializations in $U$
and that nontrivial specializations in $U$ map to
nontrivial specializations in $X$
(Decent Spaces, Lemmas \ref{decent-spaces-lemma-decent-specialization} and
\ref{decent-spaces-lemma-decent-no-specializations-map-to-same-point})
an easy topological argument shows that $\delta$ is a dimension function
on $|X|$.
\end{proof}









\section{Other chapters}

\begin{multicols}{2}
\begin{enumerate}
\item \hyperref[introduction-section-phantom]{Introduction}
\item \hyperref[conventions-section-phantom]{Conventions}
\item \hyperref[sets-section-phantom]{Set Theory}
\item \hyperref[categories-section-phantom]{Categories}
\item \hyperref[topology-section-phantom]{Topology}
\item \hyperref[sheaves-section-phantom]{Sheaves on Spaces}
\item \hyperref[algebra-section-phantom]{Commutative Algebra}
\item \hyperref[sites-section-phantom]{Sites and Sheaves}
\item \hyperref[homology-section-phantom]{Homological Algebra}
\item \hyperref[derived-section-phantom]{Derived Categories}
\item \hyperref[more-algebra-section-phantom]{More Algebra}
\item \hyperref[simplicial-section-phantom]{Simplicial Methods}
\item \hyperref[modules-section-phantom]{Sheaves of Modules}
\item \hyperref[sites-modules-section-phantom]{Modules on Sites}
\item \hyperref[injectives-section-phantom]{Injectives}
\item \hyperref[cohomology-section-phantom]{Cohomology of Sheaves}
\item \hyperref[sites-cohomology-section-phantom]{Cohomology on Sites}
\item \hyperref[hypercovering-section-phantom]{Hypercoverings}
\item \hyperref[schemes-section-phantom]{Schemes}
\item \hyperref[constructions-section-phantom]{Constructions of Schemes}
\item \hyperref[properties-section-phantom]{Properties of Schemes}
\item \hyperref[morphisms-section-phantom]{Morphisms of Schemes}
\item \hyperref[coherent-section-phantom]{Coherent Cohomology}
\item \hyperref[divisors-section-phantom]{Divisors}
\item \hyperref[limits-section-phantom]{Limits of Schemes}
\item \hyperref[varieties-section-phantom]{Varieties}
\item \hyperref[chow-section-phantom]{Chow Homology}
\item \hyperref[topologies-section-phantom]{Topologies on Schemes}
\item \hyperref[descent-section-phantom]{Descent}
\item \hyperref[more-morphisms-section-phantom]{More on Morphisms}
\item \hyperref[flat-section-phantom]{More on Flatness}
\item \hyperref[groupoids-section-phantom]{Groupoid Schemes}
\item \hyperref[more-groupoids-section-phantom]{More on Groupoid Schemes}
\item \hyperref[etale-section-phantom]{\'Etale Morphisms of Schemes}
\item \hyperref[etale-cohomology-section-phantom]{\'Etale Cohomology}
\item \hyperref[spaces-section-phantom]{Algebraic Spaces}
\item \hyperref[spaces-properties-section-phantom]{Properties of Algebraic Spaces}
\item \hyperref[spaces-morphisms-section-phantom]{Morphisms of Algebraic Spaces}
\item \hyperref[spaces-topologies-section-phantom]{Topologies on Algebraic Spaces}
\item \hyperref[spaces-descent-section-phantom]{Descent and Algebraic Spaces}
\item \hyperref[spaces-more-morphisms-section-phantom]{More on Morphisms of Spaces}
\item \hyperref[quot-section-phantom]{Quot and Hilbert Spaces}
\item \hyperref[stacks-section-phantom]{Stacks}
\item \hyperref[spaces-groupoids-section-phantom]{Groupoids in Algebraic Spaces}
\item \hyperref[spaces-more-groupoids-section-phantom]{More on Groupoids in Spaces}
\item \hyperref[bootstrap-section-phantom]{Bootstrap}
\item \hyperref[examples-stacks-section-phantom]{Examples of Stacks}
\item \hyperref[groupoids-quotients-section-phantom]{Quotients of Groupoids}
\item \hyperref[algebraic-section-phantom]{Algebraic Stacks}
\item \hyperref[criteria-section-phantom]{Criteria for Representability}
\item \hyperref[stacks-properties-section-phantom]{Properties of Algebraic Stacks}
\item \hyperref[stacks-morphisms-section-phantom]{Morphisms of Algebraic Stacks}
\item \hyperref[examples-section-phantom]{Examples}
\item \hyperref[exercises-section-phantom]{Exercises}
\item \hyperref[guide-section-phantom]{Guide to Literature}
\item \hyperref[desirables-section-phantom]{Desirables}
\item \hyperref[coding-section-phantom]{Coding Style}
\item \hyperref[fdl-section-phantom]{GNU Free Documentation License}
\item \hyperref[index-section-phantom]{Auto Generated Index}
\end{enumerate}
\end{multicols}


\bibliography{my}
\bibliographystyle{amsalpha}

\end{document}
