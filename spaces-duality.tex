\IfFileExists{stacks-project.cls}{%
\documentclass{stacks-project}
}{%
\documentclass{amsart}
}

% The following AMS packages are automatically loaded with
% the amsart documentclass:
%\usepackage{amsmath}
%\usepackage{amssymb}
%\usepackage{amsthm}

% For dealing with references we use the comment environment
\usepackage{verbatim}
\newenvironment{reference}{\comment}{\endcomment}
%\newenvironment{reference}{}{}
\newenvironment{slogan}{\comment}{\endcomment}
\newenvironment{history}{\comment}{\endcomment}

% For commutative diagrams you can use
% \usepackage{amscd}
\usepackage[all]{xy}

% We use 2cell for 2-commutative diagrams.
\xyoption{2cell}
\UseAllTwocells

% To put source file link in headers.
% Change "template.tex" to "this_filename.tex"
% \usepackage{fancyhdr}
% \pagestyle{fancy}
% \lhead{}
% \chead{}
% \rhead{Source file: \url{template.tex}}
% \lfoot{}
% \cfoot{\thepage}
% \rfoot{}
% \renewcommand{\headrulewidth}{0pt}
% \renewcommand{\footrulewidth}{0pt}
% \renewcommand{\headheight}{12pt}

\usepackage{multicol}

% For cross-file-references
\usepackage{xr-hyper}

% Package for hypertext links:
\usepackage{hyperref}

% For any local file, say "hello.tex" you want to link to please
% use \externaldocument[hello-]{hello}
\externaldocument[introduction-]{introduction}
\externaldocument[conventions-]{conventions}
\externaldocument[sets-]{sets}
\externaldocument[categories-]{categories}
\externaldocument[topology-]{topology}
\externaldocument[sheaves-]{sheaves}
\externaldocument[sites-]{sites}
\externaldocument[stacks-]{stacks}
\externaldocument[fields-]{fields}
\externaldocument[algebra-]{algebra}
\externaldocument[brauer-]{brauer}
\externaldocument[homology-]{homology}
\externaldocument[derived-]{derived}
\externaldocument[simplicial-]{simplicial}
\externaldocument[more-algebra-]{more-algebra}
\externaldocument[smoothing-]{smoothing}
\externaldocument[modules-]{modules}
\externaldocument[sites-modules-]{sites-modules}
\externaldocument[injectives-]{injectives}
\externaldocument[cohomology-]{cohomology}
\externaldocument[sites-cohomology-]{sites-cohomology}
\externaldocument[dga-]{dga}
\externaldocument[dpa-]{dpa}
\externaldocument[hypercovering-]{hypercovering}
\externaldocument[schemes-]{schemes}
\externaldocument[constructions-]{constructions}
\externaldocument[properties-]{properties}
\externaldocument[morphisms-]{morphisms}
\externaldocument[coherent-]{coherent}
\externaldocument[divisors-]{divisors}
\externaldocument[limits-]{limits}
\externaldocument[varieties-]{varieties}
\externaldocument[topologies-]{topologies}
\externaldocument[descent-]{descent}
\externaldocument[perfect-]{perfect}
\externaldocument[more-morphisms-]{more-morphisms}
\externaldocument[flat-]{flat}
\externaldocument[groupoids-]{groupoids}
\externaldocument[more-groupoids-]{more-groupoids}
\externaldocument[etale-]{etale}
\externaldocument[chow-]{chow}
\externaldocument[intersection-]{intersection}
\externaldocument[pic-]{pic}
\externaldocument[adequate-]{adequate}
\externaldocument[dualizing-]{dualizing}
\externaldocument[duality-]{duality}
\externaldocument[discriminant-]{discriminant}
\externaldocument[local-cohomology-]{local-cohomology}
\externaldocument[curves-]{curves}
\externaldocument[resolve-]{resolve}
\externaldocument[models-]{models}
\externaldocument[pione-]{pione}
\externaldocument[etale-cohomology-]{etale-cohomology}
\externaldocument[proetale-]{proetale}
\externaldocument[crystalline-]{crystalline}
\externaldocument[spaces-]{spaces}
\externaldocument[spaces-properties-]{spaces-properties}
\externaldocument[spaces-morphisms-]{spaces-morphisms}
\externaldocument[decent-spaces-]{decent-spaces}
\externaldocument[spaces-cohomology-]{spaces-cohomology}
\externaldocument[spaces-limits-]{spaces-limits}
\externaldocument[spaces-divisors-]{spaces-divisors}
\externaldocument[spaces-over-fields-]{spaces-over-fields}
\externaldocument[spaces-topologies-]{spaces-topologies}
\externaldocument[spaces-descent-]{spaces-descent}
\externaldocument[spaces-perfect-]{spaces-perfect}
\externaldocument[spaces-more-morphisms-]{spaces-more-morphisms}
\externaldocument[spaces-flat-]{spaces-flat}
\externaldocument[spaces-groupoids-]{spaces-groupoids}
\externaldocument[spaces-more-groupoids-]{spaces-more-groupoids}
\externaldocument[bootstrap-]{bootstrap}
\externaldocument[spaces-pushouts-]{spaces-pushouts}
\externaldocument[groupoids-quotients-]{groupoids-quotients}
\externaldocument[spaces-more-cohomology-]{spaces-more-cohomology}
\externaldocument[spaces-simplicial-]{spaces-simplicial}
\externaldocument[formal-spaces-]{formal-spaces}
\externaldocument[restricted-]{restricted}
\externaldocument[spaces-resolve-]{spaces-resolve}
\externaldocument[formal-defos-]{formal-defos}
\externaldocument[defos-]{defos}
\externaldocument[cotangent-]{cotangent}
\externaldocument[examples-defos-]{examples-defos}
\externaldocument[algebraic-]{algebraic}
\externaldocument[examples-stacks-]{examples-stacks}
\externaldocument[stacks-sheaves-]{stacks-sheaves}
\externaldocument[criteria-]{criteria}
\externaldocument[artin-]{artin}
\externaldocument[quot-]{quot}
\externaldocument[stacks-properties-]{stacks-properties}
\externaldocument[stacks-morphisms-]{stacks-morphisms}
\externaldocument[stacks-limits-]{stacks-limits}
\externaldocument[stacks-cohomology-]{stacks-cohomology}
\externaldocument[stacks-perfect-]{stacks-perfect}
\externaldocument[stacks-introduction-]{stacks-introduction}
\externaldocument[stacks-more-morphisms-]{stacks-more-morphisms}
\externaldocument[stacks-geometry-]{stacks-geometry}
\externaldocument[moduli-]{moduli}
\externaldocument[moduli-curves-]{moduli-curves}
\externaldocument[examples-]{examples}
\externaldocument[exercises-]{exercises}
\externaldocument[guide-]{guide}
\externaldocument[desirables-]{desirables}
\externaldocument[coding-]{coding}
\externaldocument[obsolete-]{obsolete}
\externaldocument[fdl-]{fdl}
\externaldocument[index-]{index}

% Theorem environments.
%
\theoremstyle{plain}
\newtheorem{theorem}[subsection]{Theorem}
\newtheorem{proposition}[subsection]{Proposition}
\newtheorem{lemma}[subsection]{Lemma}

\theoremstyle{definition}
\newtheorem{definition}[subsection]{Definition}
\newtheorem{example}[subsection]{Example}
\newtheorem{exercise}[subsection]{Exercise}
\newtheorem{situation}[subsection]{Situation}

\theoremstyle{remark}
\newtheorem{remark}[subsection]{Remark}
\newtheorem{remarks}[subsection]{Remarks}

\numberwithin{equation}{subsection}

% Macros
%
\def\lim{\mathop{\rm lim}\nolimits}
\def\colim{\mathop{\rm colim}\nolimits}
\def\Spec{\mathop{\rm Spec}}
\def\Hom{\mathop{\rm Hom}\nolimits}
\def\Ext{\mathop{\rm Ext}\nolimits}
\def\SheafHom{\mathop{\mathcal{H}\!{\it om}}\nolimits}
\def\SheafExt{\mathop{\mathcal{E}\!{\it xt}}\nolimits}
\def\Sch{\textit{Sch}}
\def\Mor{\mathop{\rm Mor}\nolimits}
\def\Ob{\mathop{\rm Ob}\nolimits}
\def\Sh{\mathop{\textit{Sh}}\nolimits}
\def\NL{\mathop{N\!L}\nolimits}
\def\proetale{{pro\text{-}\acute{e}tale}}
\def\etale{{\acute{e}tale}}
\def\QCoh{\textit{QCoh}}
\def\Ker{\mathop{\rm Ker}}
\def\Im{\mathop{\rm Im}}
\def\Coker{\mathop{\rm Coker}}
\def\Coim{\mathop{\rm Coim}}

%
% Macros for moduli stacks/spaces
%
\def\QCohstack{\mathcal{QC}\!{\it oh}}
\def\Cohstack{\mathcal{C}\!{\it oh}}
\def\Spacesstack{\mathcal{S}\!{\it paces}}
\def\Quotfunctor{{\rm Quot}}
\def\Hilbfunctor{{\rm Hilb}}
\def\Curvesstack{\mathcal{C}\!{\it urves}}
\def\Polarizedstack{\mathcal{P}\!{\it olarized}}
\def\Complexesstack{\mathcal{C}\!{\it omplexes}}
% \Pic is the operator that assigns to X its picard group, usage \Pic(X)
% \Picardstack_{X/B} denotes the Picard stack of X over B
% \Picardfunctor_{X/B} denotes the Picard functor of X over B
\def\Pic{\mathop{\rm Pic}\nolimits}
\def\Picardstack{\mathcal{P}\!{\it ic}}
\def\Picardfunctor{{\rm Pic}}
\def\Deformationcategory{\mathcal{D}\!{\it ef}}


% OK, start here.
%
\begin{document}

\title{Duality for Spaces}


\maketitle

\phantomsection
\label{section-phantom}

\tableofcontents

\section{Introduction}
\label{section-introduction}

\noindent
This chapter is the analogue of the corresponding chapter for
schemes, see Duality for Schemes, Section \ref{duality-section-introduction}.
The development is similar to the development in the papers
\cite{Neeman-Grothendieck}, \cite{LN},
\cite{Lipman-notes}, and \cite{Neeman-improvement}.




\section{Dualizing complexes on algebraic spaces}
\label{section-dualizing-spaces}

\noindent
Let $U$ be a locally Noetherian scheme. Let $\mathcal{O}_\etale$
be the structure sheaf of $U$ on the small \'etale site of $U$.
We will say an object $K \in D_\QCoh(\mathcal{O}_\etale)$ is
a dualizing complex on $U$ if $K = \epsilon^*(\omega_U^\bullet)$
for some dualizing complex $\omega_U^\bullet$ in the sense of
Duality for Schemes, Section \ref{duality-section-dualizing-schemes}.
Here $\epsilon^* : D_\QCoh(\mathcal{O}_U) \to D_\QCoh(\mathcal{O}_\etale)$
is the equivalence of Derived Categories of Spaces, Lemma
\ref{spaces-perfect-lemma-derived-quasi-coherent-small-etale-site}.
Most of the properties of $\omega_U^\bullet$ studied in
Duality for Schemes, Section \ref{duality-section-dualizing-schemes}
are inherited by $K$ via the discussion in
Derived Categories of Spaces, Sections
\ref{spaces-perfect-section-derived-quasi-coherent-etale} and
\ref{spaces-perfect-section-spell-out}.

\medskip\noindent
We define a dualizing complex on a locally Noetherian algebraic space
to be a complex which \'etale locally comes from a dualizing
complex on the corresponding scheme.

\begin{lemma}
\label{lemma-equivalent-definitions}
Let $S$ be a scheme. Let $X$ be a locally Noetherian algebraic space over $S$.
Let $K$ be an object of $D_\QCoh(\mathcal{O}_X)$. The following are equivalent
\begin{enumerate}
\item For every \'etale morphism $U \to X$ where $U$ is a scheme
the restriction $K|_U$ is a dualizing complex for $U$ (as discussed above).
\item There exists a surjective \'etale morphism $U \to X$ where $U$ is a
scheme such that $K|_U$ is a dualizing complex for $U$.
\end{enumerate}
\end{lemma}

\begin{proof}
Assume $U \to X$ is surjective \'etale where $U$ is a scheme.
Let $V \to X$ be an \'etale morphism where $V$ is a scheme.
Then
$$
U \leftarrow U \times_X V \rightarrow V
$$
are \'etale morphisms of schemes with the arrow to $V$ surjective.
Hence we can use Duality for Schemes, Lemma \ref{duality-lemma-descent-ascent}
to see that if $K|_U$ is a dualizing complex for $U$, then
$K|_V$ is a dualizing complex for $V$.
\end{proof}

\begin{definition}
\label{definition-dualizing-scheme}
Let $S$ be a scheme.
Let $X$ be a locally Noetherian algebraic space over $S$.
An object $K$ of $D_\QCoh(\mathcal{O}_X)$ is called a
{\it dualizing complex} if $K$ satisfies the equivalent conditions of
Lemma \ref{lemma-equivalent-definitions}.
\end{definition}

\begin{lemma}
\label{lemma-affine-duality}
Let $A$ be a Noetherian ring and let $X = \Spec(A)$. Let
$\mathcal{O}_\etale$ be the structure sheaf of $X$ on the
small \'etale site of $X$. Let $K, L$ be objects of $D(A)$.
If $K \in D_{\textit{Coh}}(A)$ and $L$ has finite injective
dimension, then
$$
\epsilon^*\widetilde{R\Hom_A(K, L)} =
R\SheafHom_{\mathcal{O}_\etale}(\epsilon^*\widetilde{K},
\epsilon^*\widetilde{L})
$$
in $D(\mathcal{O}_\etale)$ where
$\epsilon : (X_\etale, \mathcal{O}_\etale) \to (X, \mathcal{O}_X)$
is as in Derived Categories of Spaces, Section
\ref{spaces-perfect-section-derived-quasi-coherent-etale}.
\end{lemma}

\begin{proof}
By Duality for Schemes, Lemma \ref{duality-lemma-affine-duality}
we have a canonical isomorphism
$$
\widetilde{R\Hom_A(K, L)} =
R\SheafHom_{\mathcal{O}_X}(\widetilde{K}, \widetilde{L})
$$
in $D(\mathcal{O}_X)$. There is a canonical map
$$
\epsilon^*R\Hom_{\mathcal{O}_X}(\widetilde{K}, \widetilde{L})
\longrightarrow
R\SheafHom_{\mathcal{O}_\etale}(\epsilon^*\widetilde{K},
\epsilon^*\widetilde{L})
$$
in $D(\mathcal{O}_\etale)$, see Cohomology on Sites, Remark
\ref{sites-cohomology-remark-prepare-fancy-base-change}.
We will show the left and right hand side of this arrow
have isomorphic cohomology sheaves, but we will omit the
verification that the isomorphism is given by this arrow.

\medskip\noindent
We may assume that $L$ is given by a finite complex $I^\bullet$
of injective $A$-modules. By induction on the length of $I^\bullet$
and compatibility of the constructions with distinguished triangles,
we reduce to the case that $L = I[0]$ where $I$ is an injective $A$-module.
Recall that the cohomology sheaves of
$R\SheafHom_{\mathcal{O}_\etale}(\epsilon^*\widetilde{K},
\epsilon^*\widetilde{L}))$
are the sheafifications of the presheaf sending $U$ \'etale
over $X$ to the $i$th ext group between the restrictions of
$\epsilon^*\widetilde{K}$ and $\epsilon^*\widetilde{L}$
to $U_\etale$. See
Cohomology on Sites, Lemma
\ref{sites-cohomology-lemma-section-RHom-over-U}.
If $U = \Spec(B)$ is affine, then this ext group
is equal to $\text{Ext}^i_B(K \otimes_A B, L \otimes_A B)$
by the equivalence of
Derived Categories of Spaces, Lemma
\ref{spaces-perfect-lemma-derived-quasi-coherent-small-etale-site} and
Derived Categories of Schemes, Lemma
\ref{perfect-lemma-affine-compare-bounded}
(this also uses the compatibilities detailed in
Derived Categories of Spaces, Remark
\ref{spaces-perfect-remark-match-total-direct-images}).
Since $A \to B$ is \'etale, we see that
$I \otimes_A B$ is an injective $B$-module
by Dualizing Complexes, Lemma \ref{dualizing-lemma-injective-goes-up}.
Hence we see that
\begin{align*}
\Ext^n_B(K \otimes_A B, I \otimes_A B)
& =
\Hom_B(H^{-n}(K \otimes_A B), I \otimes_A B) \\
& =
\Hom_{A_f}(H^{-n}(K) \otimes_A B, I \otimes_A B) \\
& =
\Hom_A(H^{-n}(K), I) \otimes_A B \\
& =
\text{Ext}^n_A(K, I) \otimes_A B
\end{align*}
The penultimate equality because $H^{-n}(K)$ is a finite $A$-module, see
More on Algebra, Remark
\ref{more-algebra-remark-pseudo-coherence-and-base-change-ext}.
Therefore the cohomology sheaves of the left and right hand
side of the equality in the lemma are the same.
\end{proof}

\begin{lemma}
\label{lemma-dualizing-spaces}
Let $S$ be a scheme. Let $X$ be a locally Noetherian algebraic space over $S$.
Let $K$ be a dualizing complex on $X$.
Then $K$ is an object of $D_{\textit{Coh}}(\mathcal{O}_X)$
and $D = R\SheafHom_{\mathcal{O}_X}(-, K)$ induces an anti-equivalence
$$
D :
D_{\textit{Coh}}(\mathcal{O}_X)
\longrightarrow
D_{\textit{Coh}}(\mathcal{O}_X)
$$
which comes equipped with a canonical isomorphism
$\text{id} \to D \circ D$. If $X$ is quasi-compact, then
$D$ exchanges $D^+_{\textit{Coh}}(\mathcal{O}_X)$ and
$D^-_{\textit{Coh}}(\mathcal{O}_X)$ and induces an equivalence
$D^b_{\textit{Coh}}(\mathcal{O}_X) \to D^b_{\textit{Coh}}(\mathcal{O}_X)$.
\end{lemma}

\begin{proof}
Let $U \to X$ be an \'etale morphism with $U$ affine. Say $U = \Spec(A)$ and
let $\omega_A^\bullet$ be a dualizing complex for $A$ corresponding to $K|_U$
as in Lemma \ref{lemma-equivalent-definitions} and
Duality for Schemes, Lemma \ref{duality-lemma-equivalent-definitions}.
By Lemma \ref{lemma-affine-duality} the diagram
$$
\xymatrix{
D_{\textit{Coh}}(A) \ar[r] \ar[d]_{R\Hom_A(-, \omega_A^\bullet)} &
D_{\textit{Coh}}(\mathcal{O}_\etale)
\ar[d]^{R\SheafHom_{\mathcal{O}_\etale}(-, K|_U)} \\
D_{\textit{Coh}}(A) \ar[r] &
D(\mathcal{O}_\etale)
}
$$
commutes where $\mathcal{O}_\etale$ is the structure sheaf of the
small \'etale site of $U$. Since formation of $R\SheafHom$ commutes
with restriction, we conclude that $D$ sends
$D_{\textit{Coh}}(\mathcal{O}_X)$ into
$D_{\textit{Coh}}(\mathcal{O}_X)$. Moreover, the canonical map
$$
L \longrightarrow
R\SheafHom_{\mathcal{O}_X}(R\SheafHom_{\mathcal{O}_X}(L, K), K)
$$
(Cohomology on Sites, Lemma \ref{sites-cohomology-lemma-internal-hom-evaluate})
is an isomorphism for all $L$ in $D_{\textit{Coh}}(\mathcal{O}_X)$
because this is true over all $U$ as above by
Dualizing Complexes, Lemma \ref{dualizing-lemma-dualizing}.
The statement on boundedness properties of the functor $D$
in the quasi-compact case also follows from the corresponding
statements of Dualizing Complexes, Lemma \ref{dualizing-lemma-dualizing}.
\end{proof}

\noindent
Let $(\mathcal{C}, \mathcal{O})$ be a ringed site. Recall
that an object $L$ of $D(\mathcal{O})$ is {\it invertible}
if it is an invertible object for the symmetric monoidal
structure on $D(\mathcal{O}_X)$ given by derived tensor product. In
Cohomology on Sites, Lemma \ref{sites-cohomology-lemma-invertible-derived}
we we have seen this means $L$ is perfect and if $(\mathcal{C}, \mathcal{O})$
is a locally ringed site, then for every object $U$ of $\mathcal{C}$
there is a covering $\{U_i \to U\}$ of $U$ in $\mathcal{C}$
such that $L|_{U_i} \cong \mathcal{O}_{U_i}[-n_i]$
for some integers $n_i$.

\medskip\noindent
Let $S$ be a scheme and let $X$ be an algebraic space over $S$.
If $L$ in $D(\mathcal{O}_X)$ is invertible, then there is a
disjoint union decomposition $X = \coprod_{n \in \mathbf{Z}} X_n$
such that $L|_{X_n}$ is an invertible module sitting in degree $n$.
In particular, it follows that $L = \bigoplus H^n(L)[-n]$
which gives a well defined complex of $\mathcal{O}_X$-modules
(with zero differentials) representing $L$.

\begin{lemma}
\label{lemma-dualizing-unique-spaces}
Let $S$ be a scheme.
Let $X$ be a locally Noetherian algebraic space over $S$.
If $K$ and $K'$ are dualizing complexes on $X$, then $K'$
is isomorphic to $K \otimes_{\mathcal{O}_X}^\mathbf{L} L$
for some invertible object $L$ of $D(\mathcal{O}_X)$.
\end{lemma}

\begin{proof}
Set
$$
L = R\SheafHom_{\mathcal{O}_X}(K, K')
$$
This is an invertible object of $D(\mathcal{O}_X)$, because affine locally
this is true. Use Lemma \ref{lemma-affine-duality} and
Dualizing Complexes, Lemma
\ref{dualizing-lemma-dualizing-unique} and its proof.
The evaluation map $L \otimes_{\mathcal{O}_X}^\mathbf{L} K \to K'$
is an isomorphism for the same reason.
\end{proof}

\begin{lemma}
\label{lemma-dimension-function-scheme}
Let $S$ be a scheme. Let $X$ be a locally Noetherian
quasi-separated algebraic space over $S$.
Let $\omega_X^\bullet$ be a dualizing complex on $X$. Then $X$ the function
$|X| \to \mathbf{Z}$ defined by
$$
x \longmapsto \delta(x)\text{ such that }
\omega_{X, \overline{x}}^\bullet[-\delta(x)]
\text{ is a normalized dualizing complex over }
\mathcal{O}_{X, \overline{x}}
$$
is a dimension function on $|X|$.
\end{lemma}

\begin{proof}
Let $U$ be a scheme and let $U \to X$ be a surjective \'etale morphism.
Let $\omega_U^\bullet$ be the dualizing complex on $U$ associated
to $\omega_X^\bullet|_U$.
If $u \in U$ maps to $x \in |X|$, then $\mathcal{O}_{X, \overline{x}}$
is the strict henselization of $\mathcal{O}_{U, u}$. By
Dualizing Complexes, Lemma \ref{dualizing-lemma-flat-unramified}
we see that if $\omega^\bullet$ is a normalized dualizing complex
for $\mathcal{O}_{U, u}$, then
$\omega^\bullet \otimes_{\mathcal{O}_{U, u}} \mathcal{O}_{X, \overline{x}}$
is a normalized dualizing complex for $\mathcal{O}_{X, \overline{x}}$.
Hence we see that the dimension function $U \to \mathbf{Z}$ of
Duality for Schemes, Lemma \ref{duality-lemma-dimension-function-scheme}
for the scheme $U$ and the complex
$\omega_U^\bullet$ is equal to the composition of $U \to |X|$ with $\delta$.
Using the specializations in $|X|$ lift to specializations in $U$
and that nontrivial specializations in $U$ map to
nontrivial specializations in $X$
(Decent Spaces, Lemmas \ref{decent-spaces-lemma-decent-specialization} and
\ref{decent-spaces-lemma-decent-no-specializations-map-to-same-point})
an easy topological argument shows that $\delta$ is a dimension function
on $|X|$.
\end{proof}






\section{Right adjoint of pushforward}
\label{section-twisted-inverse-image}

\noindent
This is the analogue of Duality for Schemes, Section
\ref{duality-section-twisted-inverse-image}.

\begin{lemma}
\label{lemma-twisted-inverse-image}
\begin{reference}
This is almost the same as \cite[Example 4.2]{Neeman-Grothendieck}.
\end{reference}
Let $S$ be a scheme.
Let $f : X \to Y$ be a morphism between quasi-separated and quasi-compact
algebraic spaces over $S$. The functor $Rf_* : D_\QCoh(X) \to D_\QCoh(Y)$
has a right adjoint.
\end{lemma}

\begin{proof}
We will prove a right adjoint exists by verifying the hypotheses of
Derived Categories, Proposition \ref{derived-proposition-brown}.
First off, the category $D_\QCoh(\mathcal{O}_X)$ has direct sums, see
Derived Categories of Spaces, Lemma
\ref{spaces-perfect-lemma-quasi-coherence-direct-sums}.
The category $D_\QCoh(\mathcal{O}_X)$ is compactly generated by
Derived Categories of Spaces, Theorem
\ref{spaces-perfect-theorem-bondal-van-den-Bergh}.
Since $X$ and $Y$ are quasi-compact and quasi-separated, so is $f$, see
Morphisms of Spaces, Lemmas
\ref{spaces-morphisms-lemma-compose-after-separated} and
\ref{spaces-morphisms-lemma-quasi-compact-permanence}.
Hence the functor $Rf_*$ commutes with direct sums, see
Derived Categories of Spaces, Lemma
\ref{spaces-perfect-lemma-quasi-coherence-pushforward-direct-sums}.
This finishes the proof.
\end{proof}

\begin{lemma}
\label{lemma-twisted-inverse-image-bounded-below}
Notation and assumptions as in Lemma \ref{lemma-twisted-inverse-image}.
Let $a : D_\QCoh(\mathcal{O}_Y) \to D_\QCoh(\mathcal{O}_X)$ be the right
adjoint to $Rf_*$. Then $a$ maps
$D^+_\QCoh(\mathcal{O}_Y)$ into $D^+_\QCoh(\mathcal{O}_X)$.
In fact, there exists an integer $N$ such that
$H^i(K) = 0$ for $i \leq c$ implies $H^i(a(K)) = 0$ for $i \leq c - N$.
\end{lemma}

\begin{proof}
By Derived Categories of Spaces, Lemma
\ref{spaces-perfect-lemma-quasi-coherence-direct-image}
the functor $Rf_*$ has finite cohomological dimension. In other words,
there exist an integer $N$ such that
$H^i(Rf_*L) = 0$ for $i \geq N + c$ if $H^i(L) = 0$ for $i \geq c$.
Say $K \in D^+_\QCoh(\mathcal{O}_Y)$ has $H^i(K) = 0$ for $i \leq c$.
Then
$$
\Hom_{D(\mathcal{O}_X)}(\tau_{\leq c - N}a(K), a(K)) =
\Hom_{D(\mathcal{O}_Y)}(Rf_*\tau_{\leq c - N}a(K), K) = 0
$$
by what we said above. Clearly, this implies that
$H^i(a(K)) = 0$ for $i \leq c - N$.
\end{proof}

\noindent
Let $S$ be a scheme.
Let $f : X \to Y$ be a morphism of quasi-separated and quasi-compact
algebraic spaces over $S$.
Let $a$ denote the right adjoint to
$Rf_* : D_\QCoh(\mathcal{O}_X) \to D_\QCoh(\mathcal{O}_Y)$. For every
$K \in D_\QCoh(\mathcal{O}_Y)$ and $L \in D_\QCoh(\mathcal{O}_X)$
we obtain a canonical map
\begin{equation}
\label{equation-sheafy-trace}
Rf_*R\SheafHom_{\mathcal{O}_X}(L, a(K))
\longrightarrow
R\SheafHom_{\mathcal{O}_Y}(Rf_*L, K)
\end{equation}
Namely, this map is constructed as the composition
$$
Rf_*R\SheafHom_{\mathcal{O}_X}(L, a(K)) \to
R\SheafHom_{\mathcal{O}_Y}(Rf_*L, Rf_*a(K)) \to
R\SheafHom_{\mathcal{O}_Y}(Rf_*L, K)
$$
where the first arrow is 
Cohomology on Sites, Remark
\ref{sites-cohomology-remark-projection-formula-for-internal-hom}
and the second arrow is the counit $Rf_*a(K) \to K$ of the adjunction.

\begin{lemma}
\label{lemma-iso-on-RSheafHom}
Let $S$ be a scheme.
Let $f : X \to Y$ be a morphism of quasi-compact and quasi-separated
algebraic spaces over $S$.
Let $a$ be the right adjoint to
$Rf_* : D_\QCoh(\mathcal{O}_X) \to D_\QCoh(\mathcal{O}_Y)$.
Then (\ref{equation-sheafy-trace})
$$
Rf_*R\SheafHom_{\mathcal{O}_X}(L, a(K))
\longrightarrow
R\SheafHom_{\mathcal{O}_Y}(Rf_*L, K)
$$
is an isomorphism for all $L \in D_\QCoh(\mathcal{O}_X)$ and
$K \in D_\QCoh(\mathcal{O}_Y)$.
\end{lemma}

\begin{proof}
Let $M \in D_\QCoh(\mathcal{O}_Y)$. Then we have the following
\begin{align*}
\Hom_Y(M, Rf_*R\SheafHom_{\mathcal{O}_X}(L, a(K)))
& =
\Hom_X(Lf^*M, R\SheafHom_{\mathcal{O}_X}(L, a(K))) \\
& =
\Hom_X(Lf^*M \otimes_{\mathcal{O}_X}^\mathbf{L} L, a(K)) \\
& =
\Hom_Y(Rf_*(Lf^*M \otimes_{\mathcal{O}_X}^\mathbf{L} L), K) \\
& =
\Hom_Y(M \otimes_{\mathcal{O}_Y}^\mathbf{L} Rf_*L, K) \\
& =
\Hom_Y(M, R\SheafHom_{\mathcal{O}_Y}(Rf_*L, K))
\end{align*}
The first equality holds by Cohomology on Sites, Lemma
\ref{sites-cohomology-lemma-adjoint}.
The second equality by Cohomology on Sites, Lemma
\ref{sites-cohomology-lemma-internal-hom}.
The third equality by construction of $a$.
The fourth equality by Derived Categories of Spaces, Lemma
\ref{spaces-perfect-lemma-cohomology-base-change} (this is the important step).
The fifth by Cohomology on Sites, Lemma
\ref{sites-cohomology-lemma-internal-hom}.
Thus the result holds by the Yoneda lemma.
\end{proof}

\begin{lemma}
\label{lemma-iso-global-hom}
Let $S$ be a scheme.
Let $f : X \to Y$ be a morphism of quasi-separated and quasi-compact
algebraic spaces over $S$.
For all $L \in D_\QCoh(\mathcal{O}_X)$ and $K \in D_\QCoh(\mathcal{O}_Y)$
(\ref{equation-sheafy-trace}) induces an isomorphism
$R\Hom_X(L, a(K)) \to R\Hom_Y(Rf_*L, K)$ of global derived homs.
\end{lemma}

\begin{proof}
By construction (Cohomology on Sites, Section
\ref{sites-cohomology-section-global-RHom}) the complexes
$$
R\Hom_X(L, a(K)) =
R\Gamma(X, R\SheafHom_{\mathcal{O}_X}(L, a(K))) =
R\Gamma(Y, Rf_*R\SheafHom_{\mathcal{O}_X}(L, a(K)))
$$
and
$$
R\Hom_Y(Rf_*L, K) = R\Gamma(Y, R\SheafHom_{\mathcal{O}_X}(Rf_*L, a(K)))
$$
Thus the lemma is a consequence of Lemma \ref{lemma-iso-on-RSheafHom}.
\end{proof}









\section{Right adjoint of pushforward and base change, I}
\label{section-base-change-map}

\noindent
Let us define the base change map between right adjoints of pushforward.
Let $S$ be a scheme. Consider a cartesian diagram
\begin{equation}
\label{equation-base-change}
\vcenter{
\xymatrix{
X' \ar[r]_{g'} \ar[d]_{f'} & X \ar[d]^f \\
Y' \ar[r]^g & Y
}
}
\end{equation}
where $Y'$ and $X$ are {\bf Tor independent} over $Y$. Denote
$$
a  : D_\QCoh(\mathcal{O}_Y) \to D_\QCoh(\mathcal{O}_X)
\quad\text{and}\quad
a' : D_\QCoh(\mathcal{O}_{Y'}) \to D_\QCoh(\mathcal{O}_{X'})
$$
the right adjoints to $Rf_*$ and $Rf'_*$
(Lemma \ref{lemma-twisted-inverse-image}).
The base change map of
Cohomology on Sites, Remark \ref{sites-cohomology-remark-base-change}
gives a transformation of functors
$$
Lg^* \circ Rf_* \longrightarrow Rf'_* \circ L(g')^*
$$
on derived categories of sheaves with quasi-coherent cohomology.
Hence a transformation between the right adjoints in the opposite direction
$$
a \circ Rg_* \longleftarrow Rg'_* \circ a'
$$

\begin{lemma}
\label{lemma-flat-precompose-pus}
In diagram (\ref{equation-base-change}) the map
$a \circ Rg_* \leftarrow Rg'_* \circ a'$ is an isomorphism.
\end{lemma}

\begin{proof}
The base change map $Lg^* \circ Rf_* K \to Rf'_* \circ L(g')^*K$
is an isomorphism for every $K$ in $D_\QCoh(\mathcal{O}_X)$ by
Derived Categories of Spaces, Lemma
\ref{spaces-perfect-lemma-compare-base-change}
(this uses the assumption of Tor independence).
Thus the corresponding transformation between adjoint functors
is an isomorphism as well.
\end{proof}

\noindent
Then we can consider the
morphism of functors
$D_\QCoh(\mathcal{O}_Y) \to D_\QCoh(\mathcal{O}_{X'})$
given by the composition
\begin{equation}
\label{equation-base-change-map}
L(g')^* \circ a \to
L(g')^* \circ a \circ Rg_* \circ Lg^* \leftarrow
L(g')^* \circ Rg'_* \circ a' \circ Lg^* \to a' \circ Lg^*
\end{equation}
The first arrow comes from the adjunction map $\text{id} \to Rg_* Lg^*$
and the last arrow from the adjunction map $L(g')^*Rg'_* \to \text{id}$.
We need the assumption on Tor independence to invert the arrow
in the middle, see Lemma \ref{lemma-flat-precompose-pus}.
Alternatively, we can think of (\ref{equation-base-change-map}) by
adjointness of $L(g')^*$ and $R(g')_*$ as a natural transformation
$$
a \to a \circ Rg_* \circ Lg^* \leftarrow Rg'_* \circ a' \circ Lg^*
$$
were again the second arrow is invertible. If $M \in D_\QCoh(\mathcal{O}_X)$
and $K \in D_\QCoh(\mathcal{O}_Y)$
then on Yoneda functors this map is given by
\begin{align*}
\Hom_X(M, a(K))
& =
\Hom_Y(Rf_*M, K) \\
& \to
\Hom_Y(Rf_*M, Rg_* Lg^*K) \\
& =
\Hom_{Y'}(Lg^*Rf_*M, Lg^*K) \\
& \leftarrow
\Hom_{Y'}(Rf'_* L(g')^*M, Lg^*K) \\
& =
\Hom_{X'}(L(g')^*M, a'(Lg^*K)) \\
& =
\Hom_X(M, Rg'_*a'(Lg^*K))
\end{align*}
(were the arrow pointing left is invertible by the base
change theorem given in
Derived Categories of Spaces, Lemma
\ref{spaces-perfect-lemma-compare-base-change})
which makes things a little bit more explicit.

\medskip\noindent
In this section we first prove that the base change map satisfies
some natural compatibilities with regards to stacking squares as in
Cohomology on Sites, Remarks
\ref{sites-cohomology-remark-compose-base-change} and
\ref{sites-cohomology-remark-compose-base-change-horizontal}
for the usual base change map.
We suggest the reader skip the rest of this section on a first reading.

\begin{lemma}
\label{lemma-compose-base-change-maps}
Let $S$ be a scheme. Consider a commutative diagram
$$
\xymatrix{
X' \ar[r]_k \ar[d]_{f'} & X \ar[d]^f \\
Y' \ar[r]^l \ar[d]_{g'} & Y \ar[d]^g \\
Z' \ar[r]^m & Z
}
$$
of quasi-compact and quasi-separated algebraic spaces over $S$ where
both diagrams are cartesian and where $f$ and $l$
as well as $g$ and $m$ are Tor independent.
Then the maps (\ref{equation-base-change-map})
for the two squares compose to give the base
change map for the outer rectangle (see proof for a precise statement).
\end{lemma}

\begin{proof}
It follows from the assumptions that $g \circ f$ and $m$ are Tor
independent (details omitted), hence the statement makes sense.
In this proof we write $k^*$ in place of $Lk^*$ and $f_*$ instead
of $Rf_*$. Let $a$, $b$, and $c$ be the right adjoints of
Lemma \ref{lemma-twisted-inverse-image}
for $f$, $g$, and $g \circ f$ and similarly for the primed versions.
The arrow corresponding to the top square is the composition
$$
\gamma_{top} :
k^* \circ a \to k^* \circ a \circ l_* \circ l^*
\xleftarrow{\xi_{top}} k^* \circ k_* \circ a' \circ l^* \to a' \circ l^*
$$
where $\xi_{top} : k_* \circ a' \to a \circ l_*$
is an isomorphism (hence can be inverted)
and is the arrow ``dual'' to the base change map
$l^* \circ f_* \to f'_* \circ k^*$. The outer arrows come
from the canonical maps $1 \to l_* \circ l^*$ and $k^* \circ k_* \to 1$.
Similarly for the second square we have
$$
\gamma_{bot} :
l^* \circ b \to l^* \circ b \circ m_* \circ m^*
\xleftarrow{\xi_{bot}} l^* \circ l_* \circ b' \circ m^* \to b' \circ m^*
$$
For the outer rectangle we get
$$
\gamma_{rect} :
k^* \circ c \to k^* \circ c \circ m_* \circ m^*
\xleftarrow{\xi_{rect}} k^* \circ k_* \circ c' \circ m^* \to c' \circ m^*
$$
We have $(g \circ f)_* = g_* \circ f_*$ and hence
$c = a \circ b$ and similarly $c' = a' \circ b'$.
The statement of the lemma is that $\gamma_{rect}$
is equal to the composition
$$
k^* \circ c = k^* \circ a \circ b \xrightarrow{\gamma_{top}}
a' \circ l^* \circ b \xrightarrow{\gamma_{bot}}
a' \circ b' \circ m^* = c' \circ m^*
$$
To see this we contemplate the following diagram:
$$
\xymatrix{
& & k^* \circ a \circ b \ar[d] \ar[lldd] \\
& & k^* \circ a \circ l_* \circ l^* \circ b \ar[ld] \\
k^* \circ a \circ b \circ m_* \circ m^* \ar[r] &
k^* \circ a \circ l_* \circ l^* \circ b \circ m_* \circ m^* &
k^* \circ k_* \circ a' \circ l^* \circ b \ar[u]_{\xi_{top}} \ar[d] \ar[ld] \\
& k^*\circ k_* \circ a' \circ l^* \circ b \circ m_* \circ m^*
\ar[u]_{\xi_{top}} \ar[rd] &
a' \circ l^* \circ b \ar[d] \\
k^* \circ k_* \circ a' \circ b' \circ m^* \ar[uu]_{\xi_{rect}} \ar[ddrr] &
k^*\circ k_* \circ a' \circ l^* \circ l_* \circ b' \circ m^*
\ar[u]_{\xi_{bot}} \ar[l] \ar[dr] &
a' \circ l^* \circ b \circ m_* \circ m^* \\
& & a' \circ l^* \circ l_* \circ b' \circ m^* \ar[u]_{\xi_{bot}} \ar[d] \\
& & a' \circ b' \circ m^*
}
$$
Going down the right hand side we have the composition and going
down the left hand side we have $\gamma_{rect}$.
All the quadrilaterals on the right hand side of this diagram commute
by Categories, Lemma \ref{categories-lemma-properties-2-cat-cats}
or more simply the discussion preceding
Categories, Definition \ref{categories-definition-horizontal-composition}.
Hence we see that it suffices to show the diagram
$$
\xymatrix{
a \circ l_* \circ l^* \circ b \circ m_* &
a \circ b \circ m_* \ar[l] \\
k_* \circ a' \circ l^* \circ b \circ m_* \ar[u]_{\xi_{top}} & \\
k_* \circ a' \circ l^* \circ l_* \circ b' \ar[u]_{\xi_{bot}} \ar[r] &
k_* \circ a' \circ b' \ar[uu]_{\xi_{rect}}
}
$$
becomes commutative if we invert the arrows $\xi_{top}$, $\xi_{bot}$,
and $\xi_{rect}$ (note that this is different from asking the
diagram to be commutative). However, the diagram
$$
\xymatrix{
& a \circ l_* \circ l^* \circ b \circ m_* \\
a \circ l_* \circ l^* \circ l_* \circ b'
\ar[ru]^{\xi_{bot}} & &
k_* \circ a' \circ l^* \circ b \circ m_* \ar[ul]_{\xi_{top}} \\
& k_* \circ a' \circ l^* \circ l_* \circ b'
\ar[ul]^{\xi_{top}} \ar[ur]_{\xi_{bot}}
}
$$
commutes by Categories, Lemma \ref{categories-lemma-properties-2-cat-cats}.
Since the diagrams
$$
\vcenter{
\xymatrix{
a \circ l_* \circ l^* \circ b \circ m_* & a \circ b \circ m \ar[l] \\
a \circ l_* \circ l^* \circ l_* \circ b' \ar[u] &
a \circ l_* \circ b' \ar[l] \ar[u]
}
}
\quad\text{and}\quad
\vcenter{
\xymatrix{
a \circ l_* \circ l^* \circ l_* \circ b' \ar[r] & a \circ l_* \circ b' \\
k_* \circ a' \circ l^* \circ l_* \circ b' \ar[u] \ar[r] &
k_* \circ a' \circ b' \ar[u]
}
}
$$
commute (see references cited) and since the composition of
$l_* \to l_* \circ l^* \circ l_* \to l_*$ is the identity,
we find that it suffices to prove that
$$
k \circ a' \circ b' \xrightarrow{\xi_{bot}} a \circ l_* \circ b
\xrightarrow{\xi_{top}} a \circ b \circ m_*
$$
is equal to $\xi_{rect}$ (via the identifications $a \circ b = c$
and $a' \circ b' = c'$). This is the statement dual to
Cohomology on Sites, Remark \ref{sites-cohomology-remark-compose-base-change}
and the proof is complete.
\end{proof}

\begin{lemma}
\label{lemma-compose-base-change-maps-horizontal}
Let $S$ be a scheme. Consider a commutative diagram
$$
\xymatrix{
X'' \ar[r]_{g'} \ar[d]_{f''} & X' \ar[r]_g \ar[d]_{f'} & X \ar[d]^f \\
Y'' \ar[r]^{h'} & Y' \ar[r]^h & Y
}
$$
of quasi-compact and quasi-separated algebraic spaces over $S$ where
both diagrams are cartesian and where $f$ and $h$
as well as $f'$ and $h'$ are Tor independent.
Then the maps (\ref{equation-base-change-map})
for the two squares compose to give the base
change map for the outer rectangle (see proof for a precise statement).
\end{lemma}

\begin{proof}
It follows from the assumptions that $f$ and $h \circ h'$ are Tor
independent (details omitted), hence the statement makes sense.
In this proof we write $g^*$ in place of $Lg^*$ and $f_*$ instead
of $Rf_*$. Let $a$, $a'$, and $a''$ be the right adjoints of
Lemma \ref{lemma-twisted-inverse-image}
for $f$, $f'$, and $f''$. The arrow corresponding to the right
square is the composition
$$
\gamma_{right} :
g^* \circ a \to g^* \circ a \circ h_* \circ h^*
\xleftarrow{\xi_{right}} g^* \circ g_* \circ a' \circ h^* \to a' \circ h^*
$$
where $\xi_{right} : g_* \circ a' \to a \circ h_*$
is an isomorphism (hence can be inverted)
and is the arrow ``dual'' to the base change map
$h^* \circ f_* \to f'_* \circ g^*$. The outer arrows come
from the canonical maps $1 \to h_* \circ h^*$ and $g^* \circ g_* \to 1$.
Similarly for the left square we have
$$
\gamma_{left} :
(g')^* \circ a' \to (g')^* \circ a' \circ (h')_* \circ (h')^*
\xleftarrow{\xi_{left}}
(g')^* \circ (g')_* \circ a'' \circ (h')^* \to a'' \circ (h')^*
$$
For the outer rectangle we get
$$
\gamma_{rect} :
k^* \circ a \to
k^* \circ a \circ m_* \circ m^* \xleftarrow{\xi_{rect}}
k^* \circ k_* \circ a'' \circ m^* \to
a'' \circ m^*
$$
where $k = g \circ g'$ and $m = h \circ h'$.
We have $k^* = (g')^* \circ g^*$ and $m^* = (h')^* \circ h^*$.
The statement of the lemma is that $\gamma_{rect}$
is equal to the composition
$$
k^* \circ a =
(g')^* \circ g^* \circ a \xrightarrow{\gamma_{right}}
(g')^* \circ a' \circ h^* \xrightarrow{\gamma_{left}}
a'' \circ (h')^* \circ h^* = a'' \circ m^*
$$
To see this we contemplate the following diagram
$$
\xymatrix{
& (g')^* \circ g^* \circ a \ar[d] \ar[ddl] \\
& (g')^* \circ g^* \circ a \circ h_* \circ h^* \ar[ld] \\
(g')^* \circ g^* \circ a \circ h_* \circ (h')_* \circ (h')^* \circ h^* &
(g')^* \circ g^* \circ g_* \circ a' \circ h^*
\ar[u]_{\xi_{right}} \ar[d] \ar[ld] \\
(g')^* \circ g^* \circ g_* \circ a' \circ (h')_* \circ (h')^* \circ h^*
\ar[u]_{\xi_{right}} \ar[dr] &
(g')^* \circ a' \circ h^* \ar[d] \\
(g')^* \circ g^* \circ g_* \circ (g')_* \circ a'' \circ (h')^* \circ h^*
\ar[u]_{\xi_{left}} \ar[ddr] \ar[dr] &
(g')^* \circ a' \circ (h')_* \circ (h')^* \circ h^* \\
& (g')^*\circ (g')_* \circ a'' \circ (h')^* \circ h^*
\ar[u]_{\xi_{left}} \ar[d] \\
& a'' \circ (h')^* \circ h^*
}
$$
Going down the right hand side we have the composition and going
down the left hand side we have $\gamma_{rect}$.
All the quadrilaterals on the right hand side of this diagram commute
by Categories, Lemma \ref{categories-lemma-properties-2-cat-cats}
or more simply the discussion preceding
Categories, Definition \ref{categories-definition-horizontal-composition}.
Hence we see that it suffices to show that
$$
g_* \circ (g')_* \circ a'' \xrightarrow{\xi_{left}}
g_* \circ a' \circ (h')_* \xrightarrow{\xi_{right}}
a \circ h_* \circ (h')_*
$$
is equal to $\xi_{rect}$. This is the statement dual to
Cohomology, Remark \ref{cohomology-remark-compose-base-change-horizontal}
and the proof is complete.
\end{proof}

\begin{remark}
\label{remark-going-around}
Let $S$ be a scheme. Consider a commutative diagram
$$
\xymatrix{
X'' \ar[r]_{k'} \ar[d]_{f''} & X' \ar[r]_k \ar[d]_{f'} & X \ar[d]^f \\
Y'' \ar[r]^{l'} \ar[d]_{g''} & Y' \ar[r]^l \ar[d]_{g'} & Y \ar[d]^g \\
Z'' \ar[r]^{m'} & Z' \ar[r]^m & Z
}
$$
of quasi-compact and quasi-separated algebraic spaces over $S$ where
all squares are cartesian and where
$(f, l)$, $(g, m)$, $(f', l')$, $(g', m')$ are
Tor independent pairs of maps.
Let $a$, $a'$, $a''$, $b$, $b'$, $b''$ be the
right adjoints of Lemma \ref{lemma-twisted-inverse-image}
for $f$, $f'$, $f''$, $g$, $g'$, $g''$.
Let us label the squares of the diagram $A$, $B$, $C$, $D$
as follows
$$
\begin{matrix}
A & B \\
C & D
\end{matrix}
$$
Then the maps (\ref{equation-base-change-map})
for the squares are (where we use $k^* = Lk^*$, etc)
$$
\begin{matrix}
\gamma_A : (k')^* \circ a' \to a'' \circ (l')^* &
\gamma_B : k^* \circ a \to a' \circ l^* \\
\gamma_C : (l')^* \circ b' \to b'' \circ (m')^* &
\gamma_D : l^* \circ b \to b' \circ m^*
\end{matrix}
$$
For the $2 \times 1$ and $1 \times 2$ rectangles we have four further
base change maps
$$
\begin{matrix}
\gamma_{A + B} : (k \circ k')^* \circ a \to a'' \circ (l \circ l')^* \\
\gamma_{C + D} : (l \circ l')^* \circ b \to b'' \circ (m \circ m')^* \\
\gamma_{A + C} : (k')^* \circ (a' \circ b') \to (a'' \circ b'') \circ (m')^* \\
\gamma_{A + C} : k^* \circ (a \circ b) \to (a' \circ b') \circ m^*
\end{matrix}
$$
By Lemma \ref{lemma-compose-base-change-maps-horizontal} we have
$$
\gamma_{A + B} = \gamma_A \circ \gamma_B, \quad
\gamma_{C + D} = \gamma_C \circ \gamma_D
$$
and by Lemma \ref{lemma-compose-base-change-maps} we have
$$
\gamma_{A + C} = \gamma_C \circ \gamma_A, \quad
\gamma_{B + D} = \gamma_D \circ \gamma_B
$$
Here it would be more correct to write
$\gamma_{A + B} = (\gamma_A \star \text{id}_{l^*}) \circ
(\text{id}_{(k')^*} \star \gamma_B)$ with notation as in
Categories, Section \ref{categories-section-formal-cat-cat}
and similarly for the others. However, we continue the
abuse of notation used in the proofs of
Lemmas \ref{lemma-compose-base-change-maps} and
\ref{lemma-compose-base-change-maps-horizontal}
of dropping $\star$ products with identities as one can figure
out which ones to add as long as the source and target of the
transformation is known.
Having said all of this we find (a priori) two transformations
$$
(k')^* \circ k^* \circ a \circ b
\longrightarrow
a'' \circ b'' \circ (m')^* \circ m^*
$$
namely
$$
\gamma_C \circ \gamma_A \circ \gamma_D \circ \gamma_B =
\gamma_{A + C} \circ \gamma_{B + D}
$$
and
$$
\gamma_C \circ \gamma_D \circ \gamma_A \circ \gamma_B =
\gamma_{C + D} \circ \gamma_{A + B}
$$
The point of this remark is to point out that these transformations
are equal. Namely, to see this it suffices to show that
$$
\xymatrix{
(k')^* \circ a' \circ l^* \circ b \ar[r]_{\gamma_D} \ar[d]_{\gamma_A} &
(k')^* \circ a' \circ b' \circ m^* \ar[d]^{\gamma_A} \\
a'' \circ (l')^* \circ l^* \circ b \ar[r]^{\gamma_D} &
a'' \circ (l')^* \circ b' \circ m^*
}
$$
commutes. This is true by
Categories, Lemma \ref{categories-lemma-properties-2-cat-cats}
or more simply the discussion preceding
Categories, Definition \ref{categories-definition-horizontal-composition}.
\end{remark}








\section{Right adjoint of pushforward and base change, II}
\label{section-base-change-II}

\noindent
In this section we prove that the base change map of
Section \ref{section-base-change-map} is an isomorphism
in some cases.

\begin{lemma}
\label{lemma-more-base-change}
In diagram (\ref{equation-base-change}) assume in addition
$g : Y' \to Y$ is a morphism of affine schemes and $f : X \to Y$ is proper.
Then the base change map (\ref{equation-base-change-map}) induces an
isomorphism
$$
L(g')^*a(K) \longrightarrow a'(Lg^*K)
$$
in the following cases
\begin{enumerate}
\item for all $K \in D_\QCoh(\mathcal{O}_X)$ if $f$
is flat of finite presentation,
\item for all $K \in D_\QCoh(\mathcal{O}_X)$ if $f$
is perfect and $Y$ Noetherian,
\item for $K \in D_\QCoh^+(\mathcal{O}_X)$ if $g$ has finite Tor dimension
and $Y$ Noetherian.
\end{enumerate}
\end{lemma}

\begin{proof}
Write $Y = \Spec(A)$ and $Y' = \Spec(A')$. As a base change of an affine
morphism, the morphism $g'$ is affine. Let $M$ be a perfect generator
for $D_\QCoh(\mathcal{O}_X)$, see Derived Categories of Spaces, Theorem
\ref{spaces-perfect-theorem-bondal-van-den-Bergh}. Then $L(g')^*M$ is a
generator for $D_\QCoh(\mathcal{O}_{X'})$, see
Derived Categories of Spaces, Remark
\ref{spaces-perfect-remark-pullback-generator}.
Hence it suffices to show that (\ref{equation-base-change-map})
induces an isomorphism
\begin{equation}
\label{equation-iso}
R\Hom_{X'}(L(g')^*M, L(g')^*a(K))
\longrightarrow
R\Hom_{X'}(L(g')^*M, a'(Lg^*K))
\end{equation}
of global hom complexes, see
Cohomology on Sites, Section \ref{sites-cohomology-section-global-RHom},
as this will imply the cone of $L(g')^*a(K) \to a'(Lg^*K)$
is zero.
The structure of the proof is as follows: we will first show that
these Hom complexes are isomorphic and in the last part of the proof
we will show that the isomorphism is induced by (\ref{equation-iso}).

\medskip\noindent
The left hand side. Because $M$ is perfect, the canonical map
$$
R\Hom_X(M, a(K)) \otimes^\mathbf{L}_A A'
\longrightarrow
R\Hom_{X'}(L(g')^*M, L(g')^*a(K))
$$
is an isomorphism by Derived Categories of Spaces, Lemma
\ref{spaces-perfect-lemma-affine-morphism-and-hom-out-of-perfect}.
We can combine this with the isomorphism
$R\Hom_Y(Rf_*M, K) = R\Hom_X(M, a(K))$
of Lemma \ref{lemma-iso-global-hom}
to get that the left hand side equals
$R\Hom_Y(Rf_*M, K) \otimes^\mathbf{L}_A A'$.

\medskip\noindent
The right hand side. Here we first use the isomorphism
$$
R\Hom_{X'}(L(g')^*M, a'(Lg^*K)) = R\Hom_{Y'}(Rf'_*L(g')^*M, Lg^*K)
$$
of Lemma \ref{lemma-iso-global-hom}. Since $f$ and $g$ are
Tor independent the base change
map $Lg^*Rf_*M \to Rf'_*L(g')^*M$ is an isomorphism by
Derived Categories of Spaces, Lemma
\ref{spaces-perfect-lemma-compare-base-change}.
Hence we may rewrite this as $R\Hom_{Y'}(Lg^*Rf_*M, Lg^*K)$.
Since $Y$, $Y'$ are affine and $K$, $Rf_*M$ are in $D_\QCoh(\mathcal{O}_Y)$
(Derived Categories of Spaces, Lemma
\ref{spaces-perfect-lemma-quasi-coherence-direct-image})
we have a canonical map
$$
\beta :
R\Hom_Y(Rf_*M, K) \otimes^\mathbf{L}_A A'
\longrightarrow
R\Hom_{Y'}(Lg^*Rf_*M, Lg^*K)
$$
in $D(A')$. This is the arrow
More on Algebra, Equation (\ref{more-algebra-equation-base-change-RHom})
where we have used Derived Categories of Schemes, Lemmas
\ref{perfect-lemma-affine-compare-bounded} and
\ref{perfect-lemma-quasi-coherence-internal-hom}
to translate back and forth into algebra.
\begin{enumerate}
\item If $f$ is flat and of finite presentation, the complex $Rf_*M$
is perfect on $Y$ by Derived Categories of Spaces, Lemma
\ref{spaces-perfect-lemma-flat-proper-perfect-direct-image-general}
and $\beta$ is an isomorphism by
More on Algebra, Lemma \ref{more-algebra-lemma-base-change-RHom} part (1).
\item If $f$ is perfect and $Y$ Noetherian, the complex $Rf_*M$
is perfect on $Y$ by More on Morphisms of Spaces, Lemma
\ref{spaces-more-morphisms-lemma-perfect-proper-perfect-direct-image}
and $\beta$ is an isomorphism as before.
\item If $g$ has finite tor dimension and $Y$ is Noetherian,
the complex $Rf_*M$ is pseudo-coherent on $Y$
(Derived Categories of Spaces, Lemmas
\ref{spaces-perfect-lemma-direct-image-coherent} and
\ref{spaces-perfect-lemma-identify-pseudo-coherent-noetherian})
and $\beta$ is an isomorphism by
More on Algebra, Lemma \ref{more-algebra-lemma-base-change-RHom} part (4).
\end{enumerate}
We conclude that we obtain the same answer as in the previous paragraph.

\medskip\noindent
In the rest of the proof we show that the identifications of
the left and right hand side of (\ref{equation-iso})
given in the second and third paragraph are in fact given by
(\ref{equation-iso}). To make our formulas manageable
we will use $(-, -)_X = R\Hom_X(-, -)$, use $- \otimes A'$
in stead of $- \otimes_A^\mathbf{L} A'$, and we will abbreviate
$g^* = Lg^*$ and $f_* = Rf_*$. Consider the following
commutative diagram
$$
\xymatrix{
((g')^*M, (g')^*a(K))_{X'} \ar[d] &
(M, a(K))_X \otimes A' \ar[l]^-\alpha \ar[d] &
(f_*M, K)_Y \otimes A' \ar@{=}[l] \ar[d] \\
((g')^*M, (g')^*a(g_*g^*K))_{X'} &
(M, a(g_*g^*K))_X \otimes A' \ar[l]^-\alpha &
(f_*M, g_*g^*K)_Y \otimes A' \ar@{=}[l] \ar@/_4pc/[dd]_{\mu'} \\
((g')^*M, (g')^*g'_*a'(g^*K))_{X'} \ar[u] \ar[d] &
(M, g'_*a'(g^*K))_X \otimes A' \ar[u] \ar[l]^-\alpha \ar[ld]^\mu &
(f_*M, K) \otimes A' \ar[d]^\beta \\
((g')^*M, a'(g^*K))_{X'} &
(f'_*(g')^*M, g^*K)_{Y'} \ar@{=}[l] \ar[r] &
(g^*f_*M, g^*K)_{Y'}
}
$$
The arrows labeled $\alpha$ are the maps from
Derived Categories of Spaces, Lemma
\ref{spaces-perfect-lemma-affine-morphism-and-hom-out-of-perfect}
for the diagram with corners $X', X, Y', Y$.
The upper part of the diagram is commutative as the horizontal arrows are
functorial in the entries.
The middle vertical arrows come from the invertible transformation
$g'_* \circ a' \to a \circ g_*$  of Lemma \ref{lemma-flat-precompose-pus}
and therefore the middle square is commutative.
Going down the left hand side is (\ref{equation-iso}).
The upper horizontal arrows provide the identifications used in the
second paragraph of the proof.
The lower horizontal arrows including $\beta$ provide the identifications
used in the third paragraph of the proof. Given $E \in D(A)$,
$E' \in D(A')$, and $c : E \to E'$ in $D(A)$ we will denote
$\mu_c : E \otimes A' \to E'$ the map induced by $c$
and the adjointness of restriction and base change;
if $c$ is clear we write $\mu = \mu_c$, i.e., we
drop $c$ from the notation. The map $\mu$ in the diagram is of this
form with $c$ given by the identification
$(M, g'_*a(g^*K))_X = ((g')^*M, a'(g^*K))_{X'}$
; the triangle involving $\mu$ is commutative by
Derived Categories of Spaces, Remark
\ref{spaces-perfect-remark-multiplication-map}.

\medskip\noindent
Observe that
$$
\xymatrix{
(M, a(g_*g^*K))_X &
(f_*M, g_* g^*K)_Y \ar@{=}[l] &
(g^*f_*M, g^*K)_{Y'} \ar@{=}[l] \\
(M, g'_* a'(g^*K))_X \ar[u] &
((g')^*M, a'(g^*K))_{X'} \ar@{=}[l] &
(f'_*(g')^*M, g^*K)_{Y'} \ar@{=}[l] \ar[u]
}
$$
is commutative by the very definition of the transformation
$g'_* \circ a' \to a \circ g_*$. Letting $\mu'$ be as above
corresponding to the identification
$(f_*M, g_*g^*K)_X = (g^*f_*M, g^*K)_{Y'}$, then the
hexagon commutes as well. Thus it suffices to show that
$\beta$ is equal to the composition of
$(f_*M, K)_Y \otimes A' \to (f_*M, g_*g^*K)_X \otimes A'$
and $\mu'$. To do this, it suffices to prove the two induced maps
$(f_*M, K)_Y \to (g^*f_*M, g^*K)_{Y'}$ are the same.
In other words, it suffices to show the diagram
$$
\xymatrix{
R\Hom_A(E, K) \ar[rr]_{\text{induced by }\beta} \ar[rd] & &
R\Hom_{A'}(E \otimes_A^\mathbf{L} A', K \otimes_A^\mathbf{L} A') \\
& R\Hom_A(E, K \otimes_A^\mathbf{L} A') \ar[ru]
}
$$
commutes for all $E, K \in D(A)$. Since this is how $\beta$ is constructed in
More on Algebra, Section \ref{more-algebra-section-base-change-RHom}
the proof is complete.
\end{proof}









\section{Right adjoint of pushforward and trace maps}
\label{section-trace}

\noindent
Let $S$ be a scheme.
Let $f : X \to Y$ be a morphism of quasi-compact and quasi-separated
algebraic spaces over $S$.
Let $a : D_\QCoh(\mathcal{O}_Y) \to D_\QCoh(\mathcal{O}_X)$
be the right adjoint as in Lemma \ref{lemma-twisted-inverse-image}. By
Categories, Section \ref{categories-section-adjoint} we obtain a
transformation of functors
$$
\text{Tr}_f : Rf_* \circ a \longrightarrow \text{id}
$$
The corresponding map $\text{Tr}_{f, K} : Rf_*a(K) \longrightarrow K$
for $K \in D_\QCoh(\mathcal{O}_Y)$ is sometimes called the {\it trace map}.
This is the map which has the property that the bijection
$$
\Hom_X(L, a(K)) \longrightarrow \Hom_Y(Rf_*L, K)
$$
for $L \in D_\QCoh(\mathcal{O}_X)$ which characterizes the right adjoint
is given by
$$
\varphi \longmapsto \text{Tr}_{f, K} \circ Rf_*\varphi
$$
The isomorphism
$$
Rf_*R\SheafHom_{\mathcal{O}_X}(L, a(K))
\longrightarrow
R\SheafHom_{\mathcal{O}_Y}(Rf_*L, K)
$$
of Lemma \ref{lemma-iso-on-RSheafHom}
comes about by composition with $\text{Tr}_{f, K}$.
Every trace map we are going to consider in this section will be a
special case of this trace map. Before we discuss some special cases
we show that formation of the trace map commutes with base change.

\begin{lemma}[Trace map and base change]
\label{lemma-trace-map-and-base-change}
Suppose we have a diagram (\ref{equation-base-change}).
Then the maps
$1 \star \text{Tr}_f : Lg^* \circ Rf_* \circ a \to Lg^*$ and
$\text{Tr}_{f'} \star 1 : Rf'_* \circ a' \circ Lg^* \to Lg^*$
agree via the base change maps
$\beta : Lg^* \circ Rf_* \to Rf'_* \circ L(g')^*$
(Cohomology on Sites, Remark \ref{sites-cohomology-remark-base-change})
and $\alpha : L(g')^* \circ a \to a' \circ Lg^*$
(\ref{equation-base-change-map}).
More precisely, the diagram
$$
\xymatrix{
Lg^* \circ Rf_* \circ a
\ar[d]_{\beta \star 1} \ar[r]_-{1 \star \text{Tr}_f} &
Lg^* \\
Rf'_* \circ L(g')^* \circ a \ar[r]^{1 \star \alpha} &
Rf'_* \circ a' \circ Lg^* \ar[u]_{\text{Tr}_{f'} \star 1}
}
$$
of transformations of functors commutes.
\end{lemma}

\begin{proof}
In this proof we write $f_*$ for $Rf_*$ and $g^*$ for $Lg^*$ and we
drop $\star$ products with identities as one can figure out which ones
to add as long as the source and target of the transformation is known.
Recall that $\beta : g^* \circ f_* \to f'_* \circ (g')^*$ is an isomorphism
and that $\alpha$ is defined using
the isomorphism $\beta^\vee : g'_* \circ a' \to a \circ g_*$
which is the adjoint of $\beta$, see Lemma \ref{lemma-flat-precompose-pus}
and its proof. First we note that the top horizontal arrow
of the diagram in the lemma is equal to the composition
$$
g^* \circ f_* \circ a \to
g^* \circ f_* \circ a \circ g_* \circ g^* \to
g^* \circ g_* \circ g^* \to g^*
$$
where the first arrow is the unit for $(g^*, g_*)$, the second arrow
is $\text{Tr}_f$, and the third arrow is the counit for $(g^*, g_*)$.
This is a simple consequence of the fact that the composition
$g^* \to g^* \circ g_* \circ g^* \to g^*$ of unit and counit is the identity.
Consider the diagram
$$
\xymatrix{
& g^* \circ f_* \circ a \ar[ld]_\beta \ar[d] \ar[r]_{\text{Tr}_f} & g^* \\
f'_* \circ (g')^* \circ a \ar[dr] &
g^* \circ f_* \circ a \circ g_* \circ g^* \ar[d]_\beta \ar[ru] &
g^* \circ f_* \circ g'_* \circ a' \circ g^* \ar[l]_{\beta^\vee} \ar[d]_\beta &
f'_* \circ a' \circ g^* \ar[lu]_{\text{Tr}_{f'}} \\
& f'_* \circ (g')^* \circ a \circ g_* \circ g^* &
f'_* \circ (g')^* \circ g'_* \circ a' \circ g^* \ar[ru] \ar[l]_{\beta^\vee}
}
$$
In this diagram the two squares commute 
Categories, Lemma \ref{categories-lemma-properties-2-cat-cats}
or more simply the discussion preceding
Categories, Definition \ref{categories-definition-horizontal-composition}.
The triangle commutes by the discussion above. By
Categories, Lemma
\ref{categories-lemma-transformation-between-functors-and-adjoints}
the square
$$
\xymatrix{
g^* \circ f_* \circ g'_* \circ a' \ar[d]_{\beta^\vee} \ar[r]_-\beta &
f'_* \circ (g')^* \circ g'_* \circ a' \ar[d] \\
g^* \circ f_* \circ a \circ g_* \ar[r] &
\text{id}
}
$$
commutes which implies the pentagon in the big diagram commutes.
Since $\beta$ and $\beta^\vee$ are isomorphisms, and since going on
the outside of the big diagram equals
$\text{Tr}_f \circ \alpha \circ \beta$ by definition this proves the lemma.
\end{proof}

\noindent
Let $S$ be a scheme.
Let $f : X \to Y$ be a morphism of quasi-compact and quasi-separated
algebraic spaces over $S$.
Let $a : D_\QCoh(\mathcal{O}_Y) \to D_\QCoh(\mathcal{O}_X)$
be the right adjoint of $Rf_*$ as in
Lemma \ref{lemma-twisted-inverse-image}. By
Categories, Section \ref{categories-section-adjoint} we obtain a
transformation of functors
$$
\eta_f : \text{id} \to  a \circ Rf_*
$$
which is called the unit of the adjunction.

\begin{lemma}
\label{lemma-unit-and-base-change}
Suppose we have a diagram (\ref{equation-base-change}). Then the maps
$1 \star \eta_f : L(g')^* \to L(g')^* \circ a \circ Rf_*$ and
$\eta_{f'} \star 1 : L(g')^* \to a' \circ Rf'_* \circ L(g')^*$
agree via the base change maps
$\beta : Lg^* \circ Rf_* \to Rf'_* \circ L(g')^*$
(Cohomology on Sites, Remark \ref{sites-cohomology-remark-base-change})
and $\alpha : L(g')^* \circ a \to a' \circ Lg^*$
(\ref{equation-base-change-map}).
More precisely, the diagram
$$
\xymatrix{
L(g')^* \ar[r]_-{1 \star \eta_f} \ar[d]_{\eta_{f'} \star 1} &
L(g')^* \circ a \circ Rf_* \ar[d]^\alpha \\
a' \circ Rf'_* \circ L(g')^* &
a' \circ Lg^* \circ Rf_* \ar[l]_-\beta
}
$$
of transformations of functors commutes.
\end{lemma}

\begin{proof}
This proof is dual to the proof of Lemma \ref{lemma-trace-map-and-base-change}.
In this proof we write $f_*$ for $Rf_*$ and $g^*$ for $Lg^*$ and we
drop $\star$ products with identities as one can figure out which ones
to add as long as the source and target of the transformation is known.
Recall that $\beta : g^* \circ f_* \to f'_* \circ (g')^*$ is an isomorphism
and that $\alpha$ is defined using
the isomorphism $\beta^\vee : g'_* \circ a' \to a \circ g_*$
which is the adjoint of $\beta$, see Lemma \ref{lemma-flat-precompose-pus}
and its proof. First we note that the left vertical arrow
of the diagram in the lemma is equal to the composition
$$
(g')^* \to (g')^* \circ g'_* \circ (g')^* \to
(g')^* \circ g'_* \circ a' \circ f'_* \circ (g')^* \to
a' \circ f'_* \circ (g')^*
$$
where the first arrow is the unit for $((g')^*, g'_*)$, the second arrow
is $\eta_{f'}$, and the third arrow is the counit for $((g')^*, g'_*)$.
This is a simple consequence of the fact that the composition
$(g')^* \to (g')^* \circ (g')_* \circ (g')^* \to (g')^*$
of unit and counit is the identity. Consider the diagram
$$
\xymatrix{
& (g')^* \circ a \circ f_* \ar[r] &
(g')^* \circ a \circ g_* \circ g^* \circ f_*
\ar[ld]_\beta \\
(g')^* \ar[ru]^{\eta_f} \ar[dd]_{\eta_{f'}} \ar[rd] &
(g')^* \circ a \circ g_* \circ f'_* \circ (g')^* &
(g')^* \circ g'_* \circ a' \circ g^* \circ f_*
\ar[u]_{\beta^\vee} \ar[ld]_\beta \ar[d] \\
& (g')^* \circ g'_* \circ a' \circ f'_* \circ (g')^*
\ar[ld] \ar[u]_{\beta^\vee} &
a' \circ g^* \circ f_* \ar[lld]^\beta \\
a' \circ f'_* \circ (g')^*
}
$$
In this diagram the two squares commute 
Categories, Lemma \ref{categories-lemma-properties-2-cat-cats}
or more simply the discussion preceding
Categories, Definition \ref{categories-definition-horizontal-composition}.
The triangle commutes by the discussion above. By the dual of
Categories, Lemma
\ref{categories-lemma-transformation-between-functors-and-adjoints}
the square
$$
\xymatrix{
\text{id} \ar[r] \ar[d] &
g'_* \circ a' \circ g^* \circ f_* \ar[d]^\beta \\
g'_* \circ a' \circ g^* \circ f_* \ar[r]^{\beta^\vee} &
a \circ g_* \circ f'_* \circ (g')^*
}
$$
commutes which implies the pentagon in the big diagram commutes.
Since $\beta$ and $\beta^\vee$ are isomorphisms, and since going on
the outside of the big diagram equals
$\beta \circ \alpha \circ \eta_f$ by definition this proves the lemma.
\end{proof}





\section{Right adjoint of pushforward and pullback}
\label{section-compare-with-pullback}

\noindent
Let $S$ be a scheme.
Let $f : X \to Y$ be a morphism of quasi-compact and quasi-separated
algebraic spaces over $S$.
Let $a$ be the right adjoint of pushforward as in
Lemma \ref{lemma-twisted-inverse-image}. For $K, L \in D_\QCoh(\mathcal{O}_Y)$
there is a canonical map
$$
Lf^*K \otimes^\mathbf{L}_{\mathcal{O}_X} a(L)
\longrightarrow
a(K \otimes_{\mathcal{O}_Y}^\mathbf{L} L)
$$
Namely, this map is adjoint to a map
$$
Rf_*(Lf^*K \otimes^\mathbf{L}_{\mathcal{O}_X} a(L)) =
K \otimes^\mathbf{L}_{\mathcal{O}_Y} Rf_*(a(L))
\longrightarrow
K \otimes^\mathbf{L}_{\mathcal{O}_Y} L
$$
(equality by Derived Categories of Spaces, Lemma
\ref{spaces-perfect-lemma-cohomology-base-change})
for which we use the trace map $Rf_*a(L) \to L$.
When $L = \mathcal{O}_Y$ we obtain a map
\begin{equation}
\label{equation-compare-with-pullback}
Lf^*K \otimes^\mathbf{L}_{\mathcal{O}_X} a(\mathcal{O}_Y) \longrightarrow a(K)
\end{equation}
functorial in $K$ and compatible with distinguished triangles.

\begin{lemma}
\label{lemma-compare-with-pullback-perfect}
Let $S$ be a scheme.
Let $f : X \to Y$ be a morphism of quasi-compact and quasi-separated
algebraic spaces over $S$. The map
$Lf^*K \otimes^\mathbf{L}_{\mathcal{O}_X} a(L) \to
a(K \otimes_{\mathcal{O}_Y}^\mathbf{L} L)$
defined above for $K, L \in D_\QCoh(\mathcal{O}_Y)$
is an isomorphism if $K$ is perfect. In particular,
(\ref{equation-compare-with-pullback}) is an isomorphism if $K$ is perfect.
\end{lemma}

\begin{proof}
Let $K^\vee$ be the ``dual'' to $K$, see
Cohomology on Sites, Lemma \ref{sites-cohomology-lemma-dual-perfect-complex}.
For $M \in D_\QCoh(\mathcal{O}_X)$ we have
\begin{align*}
\Hom_{D(\mathcal{O}_Y)}(Rf_*M, K \otimes^\mathbf{L}_{\mathcal{O}_Y} L)
& =
\Hom_{D(\mathcal{O}_Y)}(
Rf_*M \otimes^\mathbf{L}_{\mathcal{O}_Y} K^\vee, L) \\
& =
\Hom_{D(\mathcal{O}_X)}(
M \otimes^\mathbf{L}_{\mathcal{O}_X} Lf^*K^\vee, a(L)) \\
& =
\Hom_{D(\mathcal{O}_X)}(M,
Lf^*K \otimes^\mathbf{L}_{\mathcal{O}_X} a(L))
\end{align*}
Second equality by the definition of $a$ and the projection formula
(Cohomology on Sites, Lemma
\ref{sites-cohomology-lemma-projection-formula})
or the more general Derived Categories of Spaces, Lemma
\ref{spaces-perfect-lemma-cohomology-base-change}.
Hence the result by the Yoneda lemma.
\end{proof}

\begin{lemma}
\label{lemma-restriction-compare-with-pullback}
Suppose we have a diagram (\ref{equation-base-change}).
Let $K \in D_\QCoh(\mathcal{O}_Y)$. The diagram
$$
\xymatrix{
L(g')^*(Lf^*K \otimes^\mathbf{L}_{\mathcal{O}_X} a(\mathcal{O}_Y))
\ar[r] \ar[d] & L(g')^*a(K) \ar[d] \\
L(f')^*Lg^*K \otimes_{\mathcal{O}_{X'}}^\mathbf{L} a'(\mathcal{O}_{Y'})
\ar[r] & a'(Lg^*K)
}
$$
commutes where the horizontal arrows are the maps
(\ref{equation-compare-with-pullback}) for $K$ and $Lg^*K$
and the vertical maps are constructed using
Cohomology on Sites, Remark \ref{sites-cohomology-remark-base-change} and
(\ref{equation-base-change-map}).
\end{lemma}

\begin{proof}
In this proof we will write $f_*$ for $Rf_*$ and $f^*$ for $Lf^*$, etc,
and we will write $\otimes$ for $\otimes^\mathbf{L}_{\mathcal{O}_X}$, etc.
Let us write (\ref{equation-compare-with-pullback}) as the composition
\begin{align*}
f^*K \otimes a(\mathcal{O}_Y)
& \to
a(f_*(f^*K \otimes a(\mathcal{O}_Y))) \\
& \leftarrow
a(K \otimes f_*a(\mathcal{O}_K)) \\
& \to
a(K \otimes \mathcal{O}_Y) \\
& \to
a(K)
\end{align*}
Here the first arrow is the unit $\eta_f$, the second arrow is $a$
applied to Cohomology on Sites, Equation
(\ref{sites-cohomology-equation-projection-formula-map}) which is an
isomorphism by Derived Categories of Spaces, Lemma
\ref{spaces-perfect-lemma-cohomology-base-change}, the third arrow is
$a$ applied to $\text{id}_K \otimes \text{Tr}_f$, and the fourth
arrow is $a$ applied to the isomorphism $K \otimes \mathcal{O}_Y = K$.
The proof of the lemma consists in showing that each of these
maps gives rise to a commutative square as in the statement of the lemma.
For $\eta_f$ and $\text{Tr}_f$ this is
Lemmas \ref{lemma-unit-and-base-change} and
\ref{lemma-trace-map-and-base-change}.
For the arrow using Cohomology on Sites, Equation
(\ref{sites-cohomology-equation-projection-formula-map})
this is Cohomology on Sites, Remark
\ref{sites-cohomology-remark-compatible-with-diagram}.
For the multiplication map it is clear. This finishes the proof.
\end{proof}











\section{Right adjoint of pushforward for proper flat morphisms}
\label{section-proper-flat}

\noindent
For proper, flat, and finitely presented morphisms of quasi-compact
and quasi-separated algebraic spaces the right adjoint of pushforward
enjoys some remarkable properties.

\begin{lemma}
\label{lemma-proper-flat}
Let $S$ be a scheme.
Let $Y$ be a quasi-compact and quasi-separated algebraic space over $S$.
Let $f : X \to Y$ be a morphism of algebraic spaces which is proper, flat, and
of finite presentation.
Let $a$ be the right adjoint for
$Rf_* : D_\QCoh(\mathcal{O}_X) \to D_\QCoh(\mathcal{O}_Y)$ of
Lemma \ref{lemma-twisted-inverse-image}. Then $a$ commutes with direct sums.
\end{lemma}

\begin{proof}
Let $P$ be a perfect object of $D(\mathcal{O}_X)$. By
Derived Categories of Spaces, Lemma
\ref{spaces-perfect-lemma-flat-proper-perfect-direct-image-general}
the complex $Rf_*P$ is perfect on $Y$.
Let $K_i$ be a family of objects of $D_\QCoh(\mathcal{O}_Y)$.
Then
\begin{align*}
\Hom_{D(\mathcal{O}_X)}(P, a(\bigoplus K_i))
& =
\Hom_{D(\mathcal{O}_Y)}(Rf_*P, \bigoplus K_i) \\
& =
\bigoplus \Hom_{D(\mathcal{O}_Y)}(Rf_*P, K_i) \\
& =
\bigoplus \Hom_{D(\mathcal{O}_X)}(P, a(K_i))
\end{align*}
because a perfect object is compact (Derived Categories of Spaces,
Proposition \ref{spaces-perfect-proposition-compact-is-perfect}).
Since $D_\QCoh(\mathcal{O}_X)$ has a perfect generator
(Derived Categories of Spaces, Theorem
\ref{spaces-perfect-theorem-bondal-van-den-Bergh})
we conclude that the map $\bigoplus a(K_i) \to a(\bigoplus K_i)$
is an isomorphism, i.e., $a$ commutes with direct sums.
\end{proof}

\begin{lemma}
\label{lemma-compare-with-pullback-flat-proper}
Let $S$ be a scheme.
Let $Y$ be a quasi-compact and quasi-separated algebraic space over $S$.
Let $f : X \to Y$ be a morphism of algebraic spaces which is proper, flat, and
of finite presentation.
The map (\ref{equation-compare-with-pullback}) is an isomorphism
for every object $K$ of $D_\QCoh(\mathcal{O}_Y)$.
\end{lemma}

\begin{proof}
By Lemma \ref{lemma-proper-flat} we know that $a$ commutes
with direct sums. Hence the collection of objects of
$D_\QCoh(\mathcal{O}_Y)$ for which (\ref{equation-compare-with-pullback})
is an isomorphism is a strictly full, saturated, triangulated
subcategory of $D_\QCoh(\mathcal{O}_Y)$ which is moreover
preserved under taking direct sums. Since $D_\QCoh(\mathcal{O}_Y)$
is a module category (Derived Categories of Spaces, Theorem
\ref{spaces-perfect-theorem-DQCoh-is-Ddga}) generated by a single
perfect object (Derived Categories of Spaces, Theorem
\ref{spaces-perfect-theorem-bondal-van-den-Bergh})
we can argue as in
More on Algebra, Remark \ref{more-algebra-remark-P-resolution}
to see that it suffices to prove (\ref{equation-compare-with-pullback})
is an isomorphism for a single perfect object.
However, the result holds for perfect objects, see
Lemma \ref{lemma-compare-with-pullback-perfect}.
\end{proof}

\begin{lemma}
\label{lemma-properties-relative-dualizing}
Let $Y$ be an affine scheme. Let $f : X \to Y$ be a morphism of
algebraic spaces which is proper, flat, and of finite presentation.
Let $a$ be the right adjoint for
$Rf_* : D_\QCoh(\mathcal{O}_X) \to D_\QCoh(\mathcal{O}_Y)$ of
Lemma \ref{lemma-twisted-inverse-image}.
Then
\begin{enumerate}
\item $a(\mathcal{O}_Y)$ is a $Y$-perfect object of $D(\mathcal{O}_X)$,
\item $Rf_*a(\mathcal{O}_Y)$ has vanishing cohomology sheaves
in positive degrees,
\item $\mathcal{O}_X \to
R\SheafHom_{\mathcal{O}_X}(a(\mathcal{O}_Y), a(\mathcal{O}_Y))$
is an isomorphism.
\end{enumerate}
\end{lemma}

\begin{proof}
We will repeatedly use that
$Rf_*R\SheafHom_{\mathcal{O}_X}(L, a(K)) =
R\SheafHom_{\mathcal{O}_Y}(Rf_*L, K)$, see
Lemma \ref{lemma-iso-on-RSheafHom}.
Let $E$ be a perfect object of $D(\mathcal{O}_X)$
with dual $E^\vee$, see Cohomology on Sites, Lemma
\ref{sites-cohomology-lemma-dual-perfect-complex}.
Then
$$
Rf_*(E \otimes_{\mathcal{O}_X}^\mathbf{L} a(\mathcal{O}_Y)) =
Rf_*R\SheafHom_{\mathcal{O}_X}(E^\vee, a(\mathcal{O}_Y)) =
R\SheafHom_{\mathcal{O}_Y}(Rf_*E^\vee, \mathcal{O}_Y)
$$
By Derived Categories of Spaces, Lemma
\ref{spaces-perfect-lemma-flat-proper-perfect-direct-image-general}
the complex $Rf_*E^\vee$ is perfect.
Hence the dual $R\SheafHom_{\mathcal{O}_Y}(Rf_*E^\vee, \mathcal{O}_Y)$
is perfect as well. We conclude that $a(\mathcal{O}_Y)$
is pseudo-coherent by
Derived Categories of Spaces, Lemma
\ref{spaces-perfect-lemma-perfect-enough} amd
More on Morphisms of Spaces, Lemma
\ref{spaces-more-morphisms-lemma-characterize-pseudo-coherent}.

\medskip\noindent
Let $\mathcal{F}$ be a quasi-coherent $\mathcal{O}_Y$-module. By
Lemma \ref{lemma-compare-with-pullback-flat-proper} we have
$$
a(\mathcal{F}) =
Lf^*\mathcal{F} \otimes_{\mathcal{O}_X}^\mathbf{L} a(\mathcal{O}_Y) =
f^{-1}\mathcal{F} \otimes_{f^{-1}\mathcal{O}_Y}^\mathbf{L} a(\mathcal{O}_Y)
$$
Second equality by Cohomology on Sites, Lemma
\ref{sites-cohomology-lemma-variant-derived-pullback}.
By Lemma \ref{lemma-twisted-inverse-image-bounded-below}
there exists an integer $N$ such that
$H^i(a(\mathcal{F})) = 0$ for $i \leq -N$.
Looking at stalks we conclude that $a(\mathcal{O}_Y)$ has finite
tor dimension (details omitted; hint: for $y \in Y$ any
$\mathcal{O}_{Y, y}$-module occurs as $\mathcal{F}_y$
for some quasi-coherent module on the affine scheme $Y$).

\medskip\noindent
Combining the results of the previous two paragraphs we find that
$a(\mathcal{O}_Y)$ is $Y$-perfect, see
More on Morphisms of Spaces, Definition
\ref{spaces-more-morphisms-definition-relatively-perfect}.
This proves (1).

\medskip\noindent
Let $M$ be an object of $D_\QCoh(\mathcal{O}_Y)$. Then
\begin{align*}
\Hom_Y(M, Rf_*a(\mathcal{O}_Y)) & =
\Hom_X(Lf^*M, a(\mathcal{O}_Y)) \\
& =
\Hom_Y(Rf_*Lf^*M, \mathcal{O}_Y) \\
& =
\Hom_Y(M \otimes_{\mathcal{O}_Y}^\mathbf{L} Rf_*\mathcal{O}_Y, \mathcal{O}_Y)
\end{align*}
The first equality holds by Cohomology on Sites, Lemma
\ref{sites-cohomology-lemma-adjoint}.
The second equality by construction of $a$.
The third equality by Derived Categories of Spaces, Lemma
\ref{spaces-perfect-lemma-cohomology-base-change}.
Recall $Rf_*\mathcal{O}_X$ is perfect of tor amplitude in $[0, N]$
for some $N$, see
Derived Categories of Spaces, Lemma
\ref{spaces-perfect-lemma-flat-proper-perfect-direct-image-general}.
Thus we can represent $Rf_*\mathcal{O}_X$ by a complex of
finite projective modules sitting in degrees $[0, N]$
(using More on Algebra, Lemma \ref{more-algebra-lemma-perfect}
and the fact that $Y$ is affine).
Hence if $M = \mathcal{O}_Y[-i]$ for some $i > 0$, then the last
group is zero. Since $Y$ is affine we conclude that
$H^i(Rf_*a(\mathcal{O}_Y)) = 0$ for $i > 0$.
This proves (2).

\medskip\noindent
Let $E$ be a perfect object of $D_\QCoh(\mathcal{O}_X)$. Then
we have
\begin{align*}
\Hom_X(E, R\SheafHom_{\mathcal{O}_X}(a(\mathcal{O}_Y), a(\mathcal{O}_Y))
& =
\Hom_X(E \otimes_{\mathcal{O}_X}^\mathbf{L} a(\mathcal{O}_Y),
a(\mathcal{O}_Y)) \\
& =
\Hom_Y(Rf_*(E \otimes_{\mathcal{O}_X}^\mathbf{L} a(\mathcal{O}_Y)),
\mathcal{O}_Y) \\
& =
\Hom_Y(Rf_*(R\SheafHom_{\mathcal{O}_X}(E^\vee, a(\mathcal{O}_Y))),
\mathcal{O}_Y) \\
& =
\Hom_Y(R\SheafHom_{\mathcal{O}_Y}(Rf_*E^\vee, \mathcal{O}_Y),
\mathcal{O}_Y) \\
& =
R\Gamma(Y, Rf_*E^\vee) \\
& =
\Hom_X(E, \mathcal{O}_X)
\end{align*}
The first equality holds by Cohomology on Sites, Lemma
\ref{sites-cohomology-lemma-internal-hom}.
The second equality is the definition of $a$.
The third equality comes from the construction of the dual perfect
complex $E^\vee$, see Cohomology on Sites, Lemma
\ref{sites-cohomology-lemma-dual-perfect-complex}.
The fourth equality is given in the first line of the proof.
The fifth equality holds by double duality for perfect complexes
(Cohomology on Sites, Lemma
\ref{sites-cohomology-lemma-dual-perfect-complex})
and the fact that $Rf_*E$ is perfect by
Derived Categories of Spaces, Lemma
\ref{spaces-perfect-lemma-flat-proper-perfect-direct-image-general}
The last equality is Leray for $f$.
This string of equalities essentially shows (3)
holds by the Yoneda lemma. Namely, the object
$R\SheafHom(a(\mathcal{O}_Y), a(\mathcal{O}_Y))$
is in $D_\QCoh(\mathcal{O}_X)$ by Derived Categories of Spaces, Lemma
\ref{spaces-perfect-lemma-quasi-coherence-internal-hom}.
Taking $E = \mathcal{O}_X$ in the above we get a map
$\alpha : \mathcal{O}_X \to
R\SheafHom_{\mathcal{O}_X}(a(\mathcal{O}_Y), a(\mathcal{O}_Y))$
corresponding to
$\text{id}_{\mathcal{O}_X} \in \Hom_X(\mathcal{O}_X, \mathcal{O}_X)$.
Since all the isomorphisms above are functorial in $E$ we
see that the cone on $\alpha$ is an object $C$ of $D_\QCoh(\mathcal{O}_X)$
such that $\Hom(E, C) = 0$ for all perfect $E$.
Since the perfect objects generate
(Derived Categories of Spaces, Theorem
\ref{spaces-perfect-theorem-bondal-van-den-Bergh})
we conclude that $\alpha$ is an isomorphism.
\end{proof}










\section{Relative dualizing complexes for proper flat morphisms}
\label{section-relative-dualizing-proper-flat}

\noindent
Motivated by Duality for Schemes, Sections
\ref{duality-section-proper-flat} and
\ref{duality-section-relative-dualizing-complexes}
and the material in
Section \ref{section-proper-flat}
we make the following definition.

\begin{definition}
\label{definition-relative-dualizing-proper-flat}
Let $S$ be a scheme. Let $f : X \to Y$ be a proper, flat morphism
of algebraic spaces over $S$ which is of finite presentation.
A {\it relative dualizing complex} for $X/Y$ is a pair
$(\omega_{X/Y}^\bullet, \tau)$ consisting of a
$Y$-perfect object $\omega_{X/Y}^\bullet$ of $D(\mathcal{O}_X)$
and a map
$$
\tau : Rf_*\omega_{X/Y}^\bullet \longrightarrow \mathcal{O}_Y
$$
such that for any cartesian square
$$
\xymatrix{
X' \ar[r]_{g'} \ar[d]_{f'} & X \ar[d]^f \\
Y' \ar[r]^g & Y
}
$$
where $Y'$ is an affine scheme the pair
$(L(g')^*\omega_{X/Y}^\bullet, Lg^*\tau)$
is isomorphic to the pair
$(a'(\mathcal{O}_{Y'}), \text{Tr}_{f', \mathcal{O}_{Y'}})$
studied in Sections
\ref{section-twisted-inverse-image},
\ref{section-base-change-map},
\ref{section-base-change-II},
\ref{section-trace},
\ref{section-compare-with-pullback}, and
\ref{section-proper-flat}.
\end{definition}

\noindent
There are several remarks we should make here.
\begin{enumerate}
\item In Definition \ref{definition-relative-dualizing-proper-flat}
one may drop the assumption that $\omega_{X/Y}^\bullet$ is $Y$-perfect.
Namely, running $Y'$ through the members of an \'etale covering of $Y$
by affines, we see from Lemma \ref{lemma-properties-relative-dualizing}
that the restrictions of $\omega_{X/Y}^\bullet$ to the members of
an \'etale covering of $X$ are $Y$-perfect, which implies
$\omega_{X/Y}^\bullet$ is $Y$-perfect, see
More on Morphisms of Spaces, Section
\ref{spaces-more-morphisms-section-relatively-perfect}.
\item Consider a relative dualizing complex
$(\omega_{X/Y}^\bullet, \tau)$ and a cartesian square as in
Definition \ref{definition-relative-dualizing-proper-flat}.
We are going to think of the existence of the isomorphism
$(L(g')^*\omega_{X/Y}^\bullet, Lg^*\tau) \cong
(a'(\mathcal{O}_{Y'}), \text{Tr}_{f', \mathcal{O}_{Y'}})$
as follows: it says that for any $M' \in D_\QCoh(\mathcal{O}_{X'})$
the map
$$
\Hom_X(M', L(g')^*\omega_{X/Y}^\bullet)
\longrightarrow
\Hom_Y(Rf'_*M', \mathcal{O}_{Y'}),\quad
\varphi' \longmapsto Lg^*\tau \circ Rf'_*\varphi'
$$
is an isomorphism. This follows from the definition of $a'$
and the discussion in Section \ref{section-trace}. In particular,
the Yoneda lemma guarantees that the isomorphism is unique.
\item If $Y$ is affine itself, then a relative dualizing complex
$(\omega_{X/Y}^\bullet, \tau)$ exists and is canonically isomorphic
to $(a(\mathcal{O}_Y), \text{Tr}_{f, \mathcal{O}_Y})$ where
$a$ is the right adjoint for $Rf_*$ as in
Lemma \ref{lemma-twisted-inverse-image}
and $\text{Tr}_f$ is as in Section \ref{section-trace}.
Namely, given a diagram as in the definition we get
an isomorphism $L(g')^*a(\mathcal{O}_Y) \to a'(\mathcal{O}_{Y'})$ by
Lemma \ref{lemma-more-base-change}
which is compatible with trace maps by
Lemma \ref{lemma-trace-map-and-base-change}.
\end{enumerate}
This produces exactly enough information to glue the locally given
relative dualizing complexes to global ones. We suggest the reader
skip the proofs of the following lemmas.

\begin{lemma}
\label{lemma-relative-dualizing-RHom}
Let $S$ be a scheme. Let $X \to Y$ be a proper, flat morphism of
algebraic spaces which is of finite presentation.
If $(\omega_{X/Y}^\bullet, \tau)$ is a relative dualizing complex,
then  $\mathcal{O}_X \to
R\SheafHom_{\mathcal{O}_X}(\omega_{X/Y}^\bullet, \omega_{X/Y}^\bullet)$
is an isomorphism and $Rf_*\omega_{X/Y}^\bullet$ has vanishing cohomology
sheaves in positive degrees.
\end{lemma}

\begin{proof}
It suffices to prove this after base change to an affine scheme \'etale
over $Y$ in which case it follows from
Lemma \ref{lemma-properties-relative-dualizing}.
\end{proof}

\begin{lemma}
\label{lemma-uniqueness-relative-dualizing}
Let $S$ be a scheme. Let $X \to Y$ be a proper, flat morphism of
algebraic spaces which is of finite presentation.
If $(\omega_j^\bullet, \tau_j)$, $j = 1, 2$
are two relative dualizing complexes on $X/Y$,
then there is a unique isomorphism
$(\omega_1^\bullet, \tau_1) \to (\omega_2^\bullet, \tau_2)$.
\end{lemma}

\begin{proof}
Consider $g : Y' \to Y$ \'etale with $Y'$ an affine scheme
and denote $X' = Y' \times_Y X$ the base change.
By Definition \ref{definition-relative-dualizing-proper-flat}
and the discussion following, there is a unique isomorphism
$\iota : (\omega_1^\bullet|_{X'}, \tau_1|_{Y'}) \to
(\omega_2^\bullet|_{X'}, \tau_2|_{Y'})$. If $Y'' \to Y'$
is a further \'etale morphism of affines and $X'' = Y'' \times_Y X$,
then $\iota|_{X''}$ is the unique isomorphism
$(\omega_1^\bullet|_{X''}, \tau_1|_{Y''}) \to
(\omega_2^\bullet|_{X''}, \tau_2|_{Y''})$ (by uniqueness).
Also we have
$$
\text{Ext}^p_{X'}(\omega_1^\bullet|_{X'}, \omega_2^\bullet|_{X'}) = 0,
\quad p < 0
$$
because
$\mathcal{O}_{X'} \cong
R\SheafHom_{\mathcal{O}_{X'}}(\omega_1^\bullet|_{X'}, \omega_1^\bullet|_{X'})
\cong
R\SheafHom_{\mathcal{O}_{X'}}(\omega_1^\bullet|_{X'}, \omega_2^\bullet|_{X'})$
by Lemma \ref{lemma-relative-dualizing-RHom}.

\medskip\noindent
Choose a \'etale hypercovering $b : V \to Y$ such that each
$V_n = \coprod_{i \in I_n} Y_{n, i}$ with $Y_{n, i}$ affine.
This is possible by Hypercoverings, Lemma
\ref{hypercovering-lemma-hypercovering-object} and
Remark \ref{hypercovering-remark-take-unions-hypercovering-X}
(to replace the hypercovering produced
in the lemma by the one having disjoint unions in each degree).
Denote $X_{n, i} = Y_{n, i} \times_Y X$ and $U_n = V_n \times_Y X$
so that we obtain an \'etale hypercovering
$a : U \to X$ (Hypercoverings, Lemma
\ref{hypercovering-lemma-hypercovering-morphism-sites})
with $U_n = \coprod X_{n, i}$.
The assumptions of Simplicial Spaces, Lemma
\ref{spaces-simplicial-lemma-fppf-neg-ext-zero-hom}
are satisfied for $a : U \to X$ and the complexes
$\omega_1^\bullet$ and $\omega_2^\bullet$.
Hence we obtain a unique morphism
$\iota : \omega_1^\bullet \to \omega_2^\bullet$
whose restriction to $X_{0, i}$ is the unique
isomorphism $(\omega_1^\bullet|_{X_{0, i}}, \tau_1|_{Y_{0, i}}) \to
(\omega_2^\bullet|_{X_{0, i}}, \tau_2|_{Y_{0, i}})$
We still have to see that the diagram
$$
\xymatrix{
Rf_*\omega_1^\bullet \ar[rd]_{\tau_1} \ar[rr]_{Rf_*\iota} & &
Rf_*\omega_1^\bullet \ar[ld]^{\tau_2} \\
& \mathcal{O}_Y
}
$$
is commutative. However, we know that $Rf_*\omega_1^\bullet$ and
$Rf_*\omega_2^\bullet$ have vanishing cohomology sheaves in positive
degrees (Lemma \ref{lemma-relative-dualizing-RHom})
thus this commutativity may be proved after
restricting to the affines $Y_{0, i}$ where it holds by construction.
\end{proof}

\begin{lemma}
\label{lemma-covering-enough}
Let $S$ be a scheme. Let $X \to Y$ be a proper, flat morphism of
algebraic spaces which is of finite presentation.
Let $(\omega^\bullet, \tau)$ be a pair consisting
of a $Y$-perfect object of $D(\mathcal{O}_X)$ and a map
$\tau : Rf_*\omega^\bullet \to \mathcal{O}_Y$.
Assume we have cartesian diagrams
$$
\xymatrix{
X_i \ar[r]_{g_i'} \ar[d]_{f_i} & X \ar[d]^f \\
Y_i \ar[r]^{g_i} & Y
}
$$
with $Y_i$ affine such that $\{g_i : Y_i \to Y\}$ is an \'etale covering
and isomorphisms of pairs $(\omega^\bullet|_{X_i}, \tau|_{Y_i})
\to (a_i(\mathcal{O}_{Y_i}), \text{Tr}_{f_i, \mathcal{O}_{Y_i}})$
as in Definition \ref{definition-relative-dualizing-proper-flat}.
Then $(\omega^\bullet, \tau)$ is a relative dualizing complex for $X$ over $Y$.
\end{lemma}

\begin{proof}
Let $g : Y' \to Y$ and $X', f', g', a'$ be as in
Definition \ref{definition-relative-dualizing-proper-flat}.
Set $((\omega')^\bullet, \tau') = (L(g')^*\omega^\bullet, Lg^*\tau)$.
We can find a finite \'etale covering
$\{Y'_j \to Y'\}$ by affines which refines $\{Y_i \times_Y Y' \to Y'\}$
(Topologies, Lemma \ref{topologies-lemma-etale-affine}).
Thus for each $j$ there is an $i_j$ and a morphism
$k_j : Y'_j \to Y_{i_j}$ over $Y$. Consider the fibre products
$$
\xymatrix{
X'_j \ar[r]_{h_j'} \ar[d]_{f'_j} &
X' \ar[d]^{f'} \\
Y'_j \ar[r]^{h_j} & Y'
}
$$
Denote $k'_j : X'_j \to X_{i_j}$ the induced morphism (base change
of $k_j$ by $f_{i_j}$). Restricting the given isomorphisms to $Y'_j$
via the morphism $k'_j$ we get isomorphisms of pairs
$((\omega')^\bullet|_{X'_j}, \tau'|_{Y'_j})
\to (a_j(\mathcal{O}_{Y'_j}), \text{Tr}_{f'_j, \mathcal{O}_{Y'_j}})$.
After replacing $f : X \to Y$ by $f' : X' \to Y'$ we reduce to the
problem solved in the next paragraph.

\medskip\noindent
Assume $Y$ is affine. Problem: show $(\omega^\bullet, \tau)$ is
isomorphic to $(\omega_{X/Y}^\bullet, \text{Tr}) =
(a(\mathcal{O}_Y), \text{Tr}_{f, \mathcal{O}_Y})$.
We may assume our covering $\{Y_i \to Y\}$ is given by a single
surjective \'etale morphism $\{g : Y' \to Y\}$ of affines.
Namely, we can first replace $\{g_i: Y_i \to Y\}$ by a finite
subcovering, and then we can set $g = \coprod g_i :  Y' = \coprod Y_i \to Y$;
some details omitted. Set $X' = Y' \times_Y X$ with maps
$f', g'$ as in Definition \ref{definition-relative-dualizing-proper-flat}.
Then all we're given is that we have an isomorphism
$$
(\omega^\bullet|_{X'}, \tau|_{Y'}) \to
(a'(\mathcal{O}_{Y'}), \text{Tr}_{f', \mathcal{O}_{Y'}})
$$
Since $(\omega_{X/Y}^\bullet, \text{Tr})$ is a relative dualizing complex
(see discussion following
Definition \ref{definition-relative-dualizing-proper-flat})
there is a unique isomorphism
$$
(\omega_{X/Y}^\bullet|_{X'}, \text{Tr}|_{Y'}) \to
(a'(\mathcal{O}_{Y'}), \text{Tr}_{f', \mathcal{O}_{Y'}})
$$
Uniqueness by Lemma \ref{lemma-uniqueness-relative-dualizing} for example.
Combining the displayed isomorphisms we find an isomorphism
$$
\alpha :
(\omega^\bullet|_{X'}, \tau|_{Y'}) \to
(\omega_{X/Y}^\bullet|_{X'}, \text{Tr}|_{Y'})
$$
Set $Y'' = Y' \times_Y Y'$ and $X'' = Y'' \times_Y X$ the two
pullbacks of $\alpha$ to $X''$ have to be the same by uniqueness again.
Since we have vanishing negative self exts for
$\omega_{X'/Y'}^\bullet$ over $X'$ (Lemma \ref{lemma-relative-dualizing-RHom})
and since this remains true after pulling back by any projection
$Y' \times_Y \ldots \times_Y Y' \to Y'$ (small detail omitted -- compare
with the proof of Lemma \ref{lemma-uniqueness-relative-dualizing}),
we find that $\alpha$ descends to an isomorphism
$\omega^\bullet \to \omega_{X/Y}^\bullet$
over $X$ by Simplicial Spaces, Lemma
\ref{spaces-simplicial-lemma-fppf-neg-ext-zero-hom}.
\end{proof}

\begin{lemma}
\label{lemma-existence-relative-dualizing}
Let $S$ be a scheme. Let $X \to Y$ be a proper, flat morphism of
algebraic spaces which is of finite presentation.
There exists a relative dualizing complex $(\omega_{X/Y}^\bullet, \tau)$.
\end{lemma}

\begin{proof}
Choose a \'etale hypercovering $b : V \to Y$ such that each
$V_n = \coprod_{i \in I_n} Y_{n, i}$ with $Y_{n, i}$ affine.
This is possible by Hypercoverings, Lemma
\ref{hypercovering-lemma-hypercovering-object} and
Remark \ref{hypercovering-remark-take-unions-hypercovering-X}
(to replace the hypercovering produced
in the lemma by the one having disjoint unions in each degree).
Denote $X_{n, i} = Y_{n, i} \times_Y X$ and $U_n = V_n \times_Y X$
so that we obtain an \'etale hypercovering
$a : U \to X$ (Hypercoverings, Lemma
\ref{hypercovering-lemma-hypercovering-morphism-sites})
with $U_n = \coprod X_{n, i}$.
For each $n, i$ there exists a relative dualizing complex
$(\omega_{n, i}^\bullet, \tau_{n, i})$ on $X_{n, i}/Y_{n, i}$.
See discussion following
Definition \ref{definition-relative-dualizing-proper-flat}.
For $\varphi : [m] \to [n]$ and $i \in I_n$ consider the morphisms
$g_{\varphi, i} : Y_{n, i} \to Y_{m, \alpha(\varphi)}$ and
$g'_{\varphi, i} : X_{n, i} \to X_{m, \alpha(\varphi)}$
which are part of the structure of the given hypercoverings
(Hypercoverings, Section \ref{hypercovering-section-hypercovering-sites}).
Then we have a unique isomorphisms
$$
\iota_{n, i, \varphi} :
(L(g'_{n, i})^*\omega_{n, i}^\bullet, Lg_{n, i}^*\tau_{n, i})
\longrightarrow
(\omega_{m, \alpha(\varphi)(i)}^\bullet, \tau_{m, \alpha(\varphi)(i)})
$$
of pairs, see discussion following
Definition \ref{definition-relative-dualizing-proper-flat}.
Observe that $\omega_{n, i}^\bullet$ has vanishing negative
self exts on $X_{n, i}$ by Lemma \ref{lemma-relative-dualizing-RHom}.
Denote $(\omega_n^\bullet, \tau_n)$ the pair on $U_n/V_n$ constructed using
the pairs $(\omega_{n, i}^\bullet, \tau_{n, i})$ for $i \in I_n$.
For $\varphi : [m] \to [n]$ and $i \in I_n$ consider the morphisms
$g_\varphi : V_n \to V_m$ and $g'_\varphi : U_n \to U_m$ which are part
of the structure of the simplicial algebraic spaces $V$ and $U$.
Then we have unique isomorphisms
$$
\iota_\varphi :
(L(g'_\varphi)^*\omega_n^\bullet, Lg_\varphi^*\tau_n)
\longrightarrow
(\omega_m^\bullet, \tau_m)
$$
of pairs constructed from the isomorphisms on the pieces.
The uniqueness guarantees that these isomorphisms satisfy
the transitivity condition as formulated in
Simplicial Spaces, Definition
\ref{spaces-simplicial-definition-cartesian-derived-modules}.
The assumptions of Simplicial Spaces, Lemma
\ref{spaces-simplicial-lemma-fppf-glue-neg-ext-zero}
are satisfied for $a : U \to X$, the complexes $\omega_n^\bullet$
and the isomorphisms $\iota_\varphi$\footnote{This lemma uses only
$\omega_0^\bullet$ and the two maps $\delta_1^1, \delta_0^1 : [1] \to [0]$.
The reader can skip the first few lines of the proof of the
referenced lemma
because here we actually are already given a simplicial system of
the derived category of modules.}.
Thus we obtain an object $\omega^\bullet$ of $D_\QCoh(\mathcal{O}_X)$
together with an isomorphism
$\iota_0 : \omega^\bullet|_{U_0} \to \omega_0^\bullet$
compatible with the two isomorphisms
$\iota_{\delta^1_0}$ and $\iota_{\delta^1_1}$.
Finally, we apply
Simplicial Spaces, Lemma
\ref{spaces-simplicial-lemma-fppf-neg-ext-zero-hom}
to find a unique morphism
$$
\tau : Rf_*\omega^\bullet \longrightarrow \mathcal{O}_Y
$$
whose restriction to $V_0$ agrees with $\tau_0$; some details omitted --
compare with the end of the  proof of
Lemma \ref{lemma-uniqueness-relative-dualizing}
for example to see why we have
the required vanishing of negative exts.
By Lemma \ref{lemma-covering-enough}
the pair $(\omega^\bullet, \tau)$ is a relative dualizing complex
and the proof is complete.
\end{proof}

\begin{lemma}
\label{lemma-base-change-relative-dualizing}
Let $S$ be a scheme. Consider a cartesian square
$$
\xymatrix{
X' \ar[d]_{f'} \ar[r]_{g'} & X \ar[d]^f \\
Y' \ar[r]^g & Y
}
$$
of algebraic spaces over $S$. Assume $X \to Y$ is proper, flat, and
of finite presentation. Let $(\omega_{X/Y}^\bullet, \tau)$ be a
relative dualizing complex for $f$. Then
$(L(g')^*\omega_{X/Y}^\bullet, Lg^*\tau)$ is a relative dualizing
complex for $f'$.
\end{lemma}

\begin{proof}
Observe that $L(g')^*\omega_{X/Y}^\bullet$ is $Y'$-perfect by
More on Morphisms of Spaces, Lemma
\ref{spaces-more-morphisms-lemma-base-change-relatively-perfect}.
The other condition of
Definition \ref{definition-relative-dualizing-proper-flat}
holds by transitivity of fibre products.
\end{proof}







\section{Comparison with the case of schemes}
\label{section-comparison}

\noindent
We should add a lot more in this section.

\begin{lemma}
\label{lemma-compare}
Let $S$ be a scheme. Let $f : X \to Y$ be a morphism of
quasi-compact and quasi-separated algebraic spaces over $S$.
Assume $X$ and $Y$ are representable and let $f_0 : X_0 \to Y_0$ be a
morphism of schemes representing $f$ (awkward but temporary notation).
Let $a : D_\QCoh(\mathcal{O}_Y) \to D_\QCoh(\mathcal{O}_X)$
be the right adjoint of $Rf_*$ from Lemma \ref{lemma-twisted-inverse-image}.
Let $a_0 : D_\QCoh(\mathcal{O}_{Y_0}) \to D_\QCoh(\mathcal{O}_{X_0})$
be the right adjoint of $Rf_*$ from
Duality for Schemes, Lemma \ref{duality-lemma-twisted-inverse-image}.
Then 
$$
\xymatrix{
D_\QCoh(\mathcal{O}_{X_0})
\ar@{=}[rrrrrr]_{\text{Derived Categories of Spaces, Lemma
\ref{spaces-perfect-lemma-derived-quasi-coherent-small-etale-site}}}
& & & & & &
D_\QCoh(\mathcal{O}_X) \\
D_\QCoh(\mathcal{O}_{Y_0}) \ar[u]^{a_0}
\ar@{=}[rrrrrr]^{\text{Derived Categories of Spaces, Lemma
\ref{spaces-perfect-lemma-derived-quasi-coherent-small-etale-site}}}
& & & & & &
D_\QCoh(\mathcal{O}_Y) \ar[u]_a
}
$$
is commutative.
\end{lemma}

\begin{proof}
Follows from uniqueness of adjoints and the compatibilities of
Derived Categories of Spaces, Remark
\ref{spaces-perfect-remark-match-total-direct-images}.
\end{proof}









\section{Other chapters}

\begin{multicols}{2}
\begin{enumerate}
\item \hyperref[introduction-section-phantom]{Introduction}
\item \hyperref[conventions-section-phantom]{Conventions}
\item \hyperref[sets-section-phantom]{Set Theory}
\item \hyperref[categories-section-phantom]{Categories}
\item \hyperref[topology-section-phantom]{Topology}
\item \hyperref[sheaves-section-phantom]{Sheaves on Spaces}
\item \hyperref[algebra-section-phantom]{Commutative Algebra}
\item \hyperref[sites-section-phantom]{Sites and Sheaves}
\item \hyperref[homology-section-phantom]{Homological Algebra}
\item \hyperref[derived-section-phantom]{Derived Categories}
\item \hyperref[more-algebra-section-phantom]{More Algebra}
\item \hyperref[simplicial-section-phantom]{Simplicial Methods}
\item \hyperref[modules-section-phantom]{Sheaves of Modules}
\item \hyperref[sites-modules-section-phantom]{Modules on Sites}
\item \hyperref[injectives-section-phantom]{Injectives}
\item \hyperref[cohomology-section-phantom]{Cohomology of Sheaves}
\item \hyperref[sites-cohomology-section-phantom]{Cohomology on Sites}
\item \hyperref[hypercovering-section-phantom]{Hypercoverings}
\item \hyperref[schemes-section-phantom]{Schemes}
\item \hyperref[constructions-section-phantom]{Constructions of Schemes}
\item \hyperref[properties-section-phantom]{Properties of Schemes}
\item \hyperref[morphisms-section-phantom]{Morphisms of Schemes}
\item \hyperref[coherent-section-phantom]{Coherent Cohomology}
\item \hyperref[divisors-section-phantom]{Divisors}
\item \hyperref[limits-section-phantom]{Limits of Schemes}
\item \hyperref[varieties-section-phantom]{Varieties}
\item \hyperref[chow-section-phantom]{Chow Homology}
\item \hyperref[topologies-section-phantom]{Topologies on Schemes}
\item \hyperref[descent-section-phantom]{Descent}
\item \hyperref[more-morphisms-section-phantom]{More on Morphisms}
\item \hyperref[flat-section-phantom]{More on Flatness}
\item \hyperref[groupoids-section-phantom]{Groupoid Schemes}
\item \hyperref[more-groupoids-section-phantom]{More on Groupoid Schemes}
\item \hyperref[etale-section-phantom]{\'Etale Morphisms of Schemes}
\item \hyperref[etale-cohomology-section-phantom]{\'Etale Cohomology}
\item \hyperref[spaces-section-phantom]{Algebraic Spaces}
\item \hyperref[spaces-properties-section-phantom]{Properties of Algebraic Spaces}
\item \hyperref[spaces-morphisms-section-phantom]{Morphisms of Algebraic Spaces}
\item \hyperref[spaces-topologies-section-phantom]{Topologies on Algebraic Spaces}
\item \hyperref[spaces-descent-section-phantom]{Descent and Algebraic Spaces}
\item \hyperref[spaces-more-morphisms-section-phantom]{More on Morphisms of Spaces}
\item \hyperref[quot-section-phantom]{Quot and Hilbert Spaces}
\item \hyperref[stacks-section-phantom]{Stacks}
\item \hyperref[spaces-groupoids-section-phantom]{Groupoids in Algebraic Spaces}
\item \hyperref[spaces-more-groupoids-section-phantom]{More on Groupoids in Spaces}
\item \hyperref[bootstrap-section-phantom]{Bootstrap}
\item \hyperref[examples-stacks-section-phantom]{Examples of Stacks}
\item \hyperref[groupoids-quotients-section-phantom]{Quotients of Groupoids}
\item \hyperref[algebraic-section-phantom]{Algebraic Stacks}
\item \hyperref[criteria-section-phantom]{Criteria for Representability}
\item \hyperref[stacks-properties-section-phantom]{Properties of Algebraic Stacks}
\item \hyperref[stacks-morphisms-section-phantom]{Morphisms of Algebraic Stacks}
\item \hyperref[examples-section-phantom]{Examples}
\item \hyperref[exercises-section-phantom]{Exercises}
\item \hyperref[guide-section-phantom]{Guide to Literature}
\item \hyperref[desirables-section-phantom]{Desirables}
\item \hyperref[coding-section-phantom]{Coding Style}
\item \hyperref[fdl-section-phantom]{GNU Free Documentation License}
\item \hyperref[index-section-phantom]{Auto Generated Index}
\end{enumerate}
\end{multicols}


\bibliography{my}
\bibliographystyle{amsalpha}

\end{document}
