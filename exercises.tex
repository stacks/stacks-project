\magnification\magstep1
\nopagenumbers

%%%%%%%%%%%Blackboardbold 
\font\gbbb=msbm10 scaled \magstep1
\font\bbbf=msbm10 
\font\sbbb=msbm6 
\font\ssbbb=msbm5 
\textfont6=\bbbf
\scriptfont6=\sbbb 
\scriptscriptfont6=\ssbbb 
\def\bbb{\fam6}
\def\mP{{\bbb P}} 
\def\mA{{\bbb A}} 
\def\mB{{\bbb B}} 
\def\mR{{\bbb R}}
\def\mZ{{\bbb Z}}

%%%%Gothic
\font\ggothic=eufm10 scaled \magstep1
\font\gothicf=eufm10
\font\sgothic=eufm7
\font\ssgothic=eufm5
\textfont5=\gothicf
\scriptfont5=\sgothic
\scriptscriptfont5=\ssgothic
\def\gothic{\fam5}


\font\Kopfont=cmbx12
\def\mapright#1{\smash{\mathop{\longrightarrow}\limits^{#1}}}
\def\mapdown#1{\Big\downarrow\rlap{$\vcenter{\hbox{$\scriptstyle#1$}}$}}
\def\downmap#1{\downarrow\rlap{$\vcenter{\hbox{$\scriptstyle#1$}}$}}
\def\mapup#1{\Big\uparrow\rlap{$\vcenter{\hbox{$\scriptstyle#1$}}$}}
\def\longlongrightarrow{\relbar \joinrel \longrightarrow}
\def\cC{{\cal C}}
\def\cD{{\cal D}}
\def\gp{{\gothic p}}
\def\gq{{\gothic q}}
\def\Spec{\mathop{\rm Spec}}

\centerline{\Kopfont Schemes}

\smallskip
\centerline{Excercises 1}

\bigskip\noindent
{\bf Sheaves}

\bigskip\item{\bf 1.} Carefully prove that a map of sheaves of {\bf sets}
is an epimorphisms (in the category of sheaves of sets) if and only if the
induced maps on all the stalks are surjective.

\medskip\item{\bf 2.} Let $f : X \to Y$ be a map of topological spaces.
Prove pushforward $f_\ast$ and pullback $f^{-1}$ for sheaves of {\bf sets}
form an adjoint pair of functors.

\medskip\item{\bf 3.} Let $j : U \to X$ be an open immersion. Show
that $j^{-1}$ has a left adjoint $j_{!}$ on the category of sheaves
of sets. Characterize the stalks of $j_{!}({\cal G})$. (Hint: $j_{!}$
is called extension by zero when you do this for abelian sheaves... )

\medskip\item{\bf 4.} Let ${\cal F}$ be an abelian sheaf on $X$. Show
that ${\cal F}$ is the quotient of a (possibly very large) direct sum
of sheaves all of whose terms are of the form
$$
j_{!}(\underline{{\bbb Z}}_U)
$$
where $U \subset X$ is open and $\underline{{\bbb Z}}_U$ denotes the
constant sheaf with value ${\bbb Z}$ on $U$.

\medskip\noindent
{\bf Remark.} In the category of abelian sheaves the direct sum of
a family of sheaves $\{{\cal F}_i\}_{i\in I}$ is the sheaf associated to
the presheaf $U \mapsto \oplus {\cal F}_i(U)$. Consequently the stalk of
the direct sum at a point $x$ is the direct sum of the stalks of the 
${\cal F}_i$ at $x$.

\bye
\magnification\magstep1
\nopagenumbers

%%%%%%%%%%%Blackboardbold 
\font\gbbb=msbm10 scaled \magstep1
\font\bbbf=msbm10 
\font\sbbb=msbm6 
\font\ssbbb=msbm5 
\textfont6=\bbbf
\scriptfont6=\sbbb 
\scriptscriptfont6=\ssbbb 
\def\bbb{\fam6}
\def\mP{{\bbb P}} 
\def\mA{{\bbb A}} 
\def\mB{{\bbb B}} 
\def\mR{{\bbb R}}
\def\mZ{{\bbb Z}}

%%%%Gothic
\font\ggothic=eufm10 scaled \magstep1
\font\gothicf=eufm10
\font\sgothic=eufm7
\font\ssgothic=eufm5
\textfont5=\gothicf
\scriptfont5=\sgothic
\scriptscriptfont5=\ssgothic
\def\gothic{\fam5}


\font\Kopfont=cmbx12
\def\mapright#1{\smash{\mathop{\longrightarrow}\limits^{#1}}}
\def\mapdown#1{\Big\downarrow\rlap{$\vcenter{\hbox{$\scriptstyle#1$}}$}}
\def\downmap#1{\downarrow\rlap{$\vcenter{\hbox{$\scriptstyle#1$}}$}}
\def\mapup#1{\Big\uparrow\rlap{$\vcenter{\hbox{$\scriptstyle#1$}}$}}
\def\longlongrightarrow{\relbar \joinrel \longrightarrow}
\def\cC{{\cal C}}
\def\cD{{\cal D}}
\def\gp{{\gothic p}}
\def\gq{{\gothic q}}
\def\Spec{\mathop{\rm Spec}}

\centerline{\Kopfont Schemes}

\smallskip
\centerline{Exercises 2}

\bigskip\noindent
{\bf Schemes -- examples are important}

\bigskip\noindent
Let ${\cal C}$ be the category of locally ringed spaces.
An affine scheme is an object in ${\cal C}$ isomorphic in ${\cal C}$ to
a pair of the form $(\Spec A, {\cal O}_A)$. A scheme is an
object $(X, {\cal O}_X)$ of ${\cal C}$ such that every point $x\in X$ 
has an open neighbourhood $U \subset X$ such that the pair
$(U, {\cal O}_X|_U)$ is an affine scheme.

\bigskip\item{\bf 1.} Suppose that $X$ is a scheme whose underlying 
topological space has 2 points. Show that $X$ is an affine scheme.

\medskip\item{\bf 2.} Give an example of an affine scheme $(X, {\cal O}_X)$
and an open $U \subset X$ such that $(U, {\cal O}_X|U)$ is not an affine
scheme.

\medskip\item{\bf 3.} Given an example of a pair of affine schemes
$(X, {\cal O}_X)$, $(Y, {\cal O}_Y)$, an open subscheme $(U, {\cal O}_X|_U)$
of $X$ and a morphism of schemes $(U, {\cal O}_X|_U) \to (Y, {\cal O}_Y)$
that does not extend to a morphism of schemes $(X, {\cal O}_X) \to
(Y, {\cal O}_Y)$.

\medskip\item{\bf 4.} Give an example of a scheme $X$, a field $K$, and a
morphism of ringed spaces $\Spec K \to X$ which is NOT a morphism of schemes.

\medskip\item{\bf 5.} Do all the exercises in Hartshorne, Chapter II,
Sections 1 and 2...\ \ Just kidding!

\medskip\noindent
{\bf Remark.} When $(X, {\cal O}_X)$ is a ringed space and $U \subset X$
is an open subset then $(U, {\cal O}_X|_U)$ is a ringed space. Notation:
${\cal O}_U = {\cal O}_X|_U$. There is a canonical morphisms of ringed spaces
$$
	j : (U, {\cal O}_U) \longrightarrow (X, {\cal O}_X).
$$
If $(X, {\cal O}_X)$ is a locally ringed space, so is $(U, {\cal O}_U)$ and
$j$ is a morphism of locally ringed spaces. If $(X, {\cal O}_X)$ is a scheme
so is $(U, {\cal O}_U)$ and $j$ is a morphism of schemes. We say that
$(U, {\cal O}_U)$ is an {\it open subscheme} of $(X, {\cal O}_X)$ and that
$j$ is an {\it open immersion}. More generally, any morphism
$j' : (V, {\cal O}_V) \to (X, {\cal O}_X)$ that is {\it isomorphic} to a
morphism $j : (U, {\cal O}_U) \to (X, {\cal O}_X)$ as above is called an
open immersion.

\bye
\magnification\magstep1
\nopagenumbers

%%%%%%%%%%%Blackboardbold 
\font\gbbb=msbm10 scaled \magstep1
\font\bbbf=msbm10 
\font\sbbb=msbm6 
\font\ssbbb=msbm5 
\textfont6=\bbbf
\scriptfont6=\sbbb 
\scriptscriptfont6=\ssbbb 
\def\bbb{\fam6}
\def\mP{{\bbb P}} 
\def\mA{{\bbb A}} 
\def\mB{{\bbb B}} 
\def\mR{{\bbb R}}
\def\mZ{{\bbb Z}}

%%%%Gothic
\font\ggothic=eufm10 scaled \magstep1
\font\gothicf=eufm10
\font\sgothic=eufm7
\font\ssgothic=eufm5
\textfont5=\gothicf
\scriptfont5=\sgothic
\scriptscriptfont5=\ssgothic
\def\gothic{\fam5}


\font\Kopfont=cmbx12
\def\mapright#1{\smash{\mathop{\longrightarrow}\limits^{#1}}}
\def\mapdown#1{\Big\downarrow\rlap{$\vcenter{\hbox{$\scriptstyle#1$}}$}}
\def\downmap#1{\downarrow\rlap{$\vcenter{\hbox{$\scriptstyle#1$}}$}}
\def\mapup#1{\Big\uparrow\rlap{$\vcenter{\hbox{$\scriptstyle#1$}}$}}
\def\longlongrightarrow{\relbar \joinrel \longrightarrow}
\def\cC{{\cal C}}
\def\cD{{\cal D}}
\def\gp{{\gothic p}}
\def\gq{{\gothic q}}
\def\Spec{\mathop{\rm Spec}}
\def\Proj{\mathop{\rm Proj}}

\centerline{\Kopfont Schemes}

\smallskip
\centerline{Exercises 3}

\bigskip\noindent
{\bf Schemes -- examples are important}

\bigskip\noindent
Please also argue that your examples are as required. Feel free to quote
results from Hartshorne or EGA.

\bigskip\item{\bf 1.} Give an example of a morphism of {\it integral}
schemes $f : X \to Y$ such that the induced maps ${\cal O}_{Y,f(x)}
\to {\cal O}_{X,x}$ are surjective for all $x\in X$, but $f$
is not a closed immersion.

\medskip\item{\bf 2.} Give examples of graded rings $S$ such that
\itemitem{\bf (a)} $\Proj(S)$ is affine and nonempty, and
\itemitem{\bf (b)} $\Proj(S)$ is integral, nonempty but not isomorphic
to ${\bbb P}^n_A$ for any $n\geq 0$, any ring $A$.

\medskip\item{\bf 3.} Give an example of a nonconstant morphism
of schemes ${\bbb P}^1_{\bbb C} \to {\bbb P}^5_{\bbb C}$ over
$\Spec({\bbb C})$.

\medskip\item{\bf 4.} Give an example of an isomorphism of schemes
${\bbb P}^1_{\bbb C} \to \Proj({\bbb C}[X_0,X_1,X_2]/(X_0^2+X_1^2+X_2^2))$.

\medskip\item{\bf 5.} Give an example of a morphism of schemes
$f : X \to {\bbb A}^1_{\bbb C}=\Spec({\bbb C}[T])$ such that the
(scheme theoretic) fibre of $f$ over $t \in {\bbb A}^1_{\bbb C}$ is (a)
isomorphic to ${\bbb P}^1_{\bbb C}$ when $t$ is a closed point not equal
to $0$, and (b) not isomorphic to ${\bbb P}^1_{\bbb C}$ when $t=0$. 

\medskip\noindent
{\bf Remark.} This can be done in many, many ways. Here are some additional restraints
you can impose: Can you do it with fibre at $t=0$ projective? Can you do it
with special fibre irreducible and projective? Can you do it with special
fibre integral and projective? Can you do it with fibre at $t=0$ smooth and
projective? What about similar questions when you replace
${\bbb P}^1_{\bbb C}$ with another variety over ${\bbb C}$?

\bye
\magnification\magstep1
\nopagenumbers

%%%%%%%%%%%Blackboardbold 
\font\gbbb=msbm10 scaled \magstep1
\font\bbbf=msbm10 
\font\sbbb=msbm6 
\font\ssbbb=msbm5 
\textfont6=\bbbf
\scriptfont6=\sbbb 
\scriptscriptfont6=\ssbbb 
\def\bbb{\fam6}
\def\mP{{\bbb P}} 
\def\mA{{\bbb A}} 
\def\mB{{\bbb B}} 
\def\mR{{\bbb R}}
\def\mZ{{\bbb Z}}

%%%%Gothic
\font\ggothic=eufm10 scaled \magstep1
\font\gothicf=eufm10
\font\sgothic=eufm7
\font\ssgothic=eufm5
\textfont5=\gothicf
\scriptfont5=\sgothic
\scriptscriptfont5=\ssgothic
\def\gothic{\fam5}


\font\Kopfont=cmbx12
\def\mapright#1{\smash{\mathop{\longrightarrow}\limits^{#1}}}
\def\mapdown#1{\Big\downarrow\rlap{$\vcenter{\hbox{$\scriptstyle#1$}}$}}
\def\downmap#1{\downarrow\rlap{$\vcenter{\hbox{$\scriptstyle#1$}}$}}
\def\mapup#1{\Big\uparrow\rlap{$\vcenter{\hbox{$\scriptstyle#1$}}$}}
\def\longlongrightarrow{\relbar \joinrel \longrightarrow}
\def\cC{{\cal C}}
\def\cD{{\cal D}}
\def\gp{{\gothic p}}
\def\gq{{\gothic q}}
\def\Spec{\mathop{\rm Spec}}
\def\Proj{\mathop{\rm Proj}}

\centerline{\Kopfont Schemes}

\smallskip
\centerline{Exercises 4}

\bigskip\noindent
{\bf Schemes -- examples are important}

\bigskip\noindent
Please also argue that your examples are as required. Feel free to quote
results from Hartshorne or EGA. It is fine to copy examples out of Hartshorne,
and reference the explanation if it is in Hartshorne.

\bigskip\item{\bf 1.} (Pretty hard. You can leave some of the verifications
out if you like.) Give an example of a fibre product
$X\times_S Y$ such that $X$ and $Y$ are affine but $X\times_S Y$ is not.

\medskip\noindent
{\bf Remark/Hint.} It turns out this cannot happen with $S$ separated.
Do you know why?

\medskip\item{\bf 2.} Give an example of a scheme
$V$ which is integral 1-dimensional scheme of finite type
over ${\bbb Q}$ such that $\Spec{\bbb C} \times_{\Spec{\bbb Q}} V$
is not integral.

\medskip\item{\bf 3.} Give an example of a scheme
$V$ which is integral 1-dimensional scheme of finite type
over a field $k$ such that $\Spec k' \times_{\Spec{k}} V$
is not reduced for some finite field extension $k \subset k'$.

\medskip\noindent
{\bf Remark.} If your scheme is affine then dimension is the
same as the Krull dimension of the underlying ring. So you can
use last semesters results to compute dimension.

\medskip\item{\bf 4.} Give an example of a surjective morphism
$X \to {\bbb P}^n_{\bbb C}$ with $X$ affine.

\medskip\item{\bf 5.} (For the number theorists.) Give an example 
of a closed subscheme
$$
Z \subset \Spec {\bbb Z}[x, {1 \over x(x-1)(2x-1)}]
$$
such that the morphism $Z \to \Spec {\bbb Z}$ is finite and surjective.

\medskip\noindent
{\bf Remark.} If you do not like number theory, you can try the 
variant where you look at
$$
\Spec {\bbb F}_p[t, x, {1 \over x(x-t)(tx-1)}]  \longrightarrow
\Spec {\bbb F}_p[t]
$$
and you try to find a closed subscheme of the top scheme
which maps finite surjectively to the bottom one. (There is a
theoretical reason for having a finite ground field here; allthough
it may not be necessary in this particular case.)


\bye
\nopagenumbers

%%%%%%%%%%%Blackboardbold 
\font\gbbb=msbm10 scaled \magstep1
\font\bbbf=msbm10 
\font\sbbb=msbm6 
\font\ssbbb=msbm5 
\textfont6=\bbbf
\scriptfont6=\sbbb 
\scriptscriptfont6=\ssbbb 
\def\bbb{\fam6}
\def\mP{{\bbb P}} 
\def\mA{{\bbb A}} 
\def\mB{{\bbb B}} 
\def\mR{{\bbb R}}
\def\mZ{{\bbb Z}}

%%%%Gothic
\font\ggothic=eufm10 scaled \magstep1
\font\gothicf=eufm10
\font\sgothic=eufm7
\font\ssgothic=eufm5
\textfont5=\gothicf
\scriptfont5=\sgothic
\scriptscriptfont5=\ssgothic
\def\gothic{\fam5}


\font\Kopfont=cmbx12
\def\mapright#1{\smash{\mathop{\longrightarrow}\limits^{#1}}}
\def\mapdown#1{\Big\downarrow\rlap{$\vcenter{\hbox{$\scriptstyle#1$}}$}}
\def\downmap#1{\downarrow\rlap{$\vcenter{\hbox{$\scriptstyle#1$}}$}}
\def\mapup#1{\Big\uparrow\rlap{$\vcenter{\hbox{$\scriptstyle#1$}}$}}
\def\longlongrightarrow{\relbar \joinrel \longrightarrow}
\def\cC{{\cal C}}
\def\cD{{\cal D}}
\def\gp{{\gothic p}}
\def\gq{{\gothic q}}
\def\Spec{\mathop{\rm Spec}}
\def\Proj{\mathop{\rm Proj}}

\centerline{\Kopfont Schemes}

\smallskip
\centerline{Exercises 5}

\bigskip\noindent
{\bf Schemes -- invertible sheaves are important}

\bigskip\noindent
Feel free to quote results from Hartshorne or EGA.

\bigskip\noindent
An invertible ${\cal O}_X$-module on a locally ringed space $(X,{\cal O}_X)$
is a sheaf of ${\cal O}_X$-modules ${\cal L}$ such that every point
has an open neighbourhood $U \subset X$ such that ${\cal L}|_U$
is isomorphic to ${\cal O}_U$ as ${\cal O}_U$-module.
We say that ${\cal L}$ is trivial if it is isomorphic to 
${\cal O}_X$ as a ${\cal O}_X$-module.

\bigskip\item{\bf 1.} General facts.
\itemitem{\bf (a)} Show that an invertible ${\cal O}_X$-module on 
a scheme $X$ is quasi-coherent.
\itemitem{\bf (b)} Suppose $X\to Y$ is a morphism of ringed spaces,
and ${\cal L}$ an invertible ${\cal O}_Y$-module.
Show that $f^\ast {\cal L}$ is an invertible ${\cal O}_X$ module.

\medskip\item{\bf 2.} Algebra.
\itemitem{\bf (a)} Show that an invertible ${\cal O}_X$-module on 
an affine scheme $\Spec A$ corresponds to an $A$-module $M$ which is
(i) finite, (ii) projective, (iii) locally free of rank 1,
and hence (iv) flat, and (v) finitely presented. (Feel free to
quote things from last semesters course; or from algebra books.)
\itemitem{\bf (b)} Suppose that $A$ is a domain and that $M$ is
a module as in (a). Show that $M$ is isomorphic as an $A$-module
to an ideal $I \subset A$ such that $IA_\gp$ is principal for
every prime $\gp$.

\medskip\item{\bf 3.} Simple examples.
\itemitem{\bf (a)} Let $k$ be a field. Let $A = k[x]$.
Show that $X=\Spec A$ has only trivial invertible ${\cal O}_X$-modules.
\itemitem{\bf (b)} Let $A$ be the ring
$$
A = \{ f\in k[x] \mid f(0)=f(1) \}.
$$
Show that $X = \Spec A$ has a nontrivial invertible ${\cal O}_X$-module,
unless $k={\bbb F}_2$. (Hint: Think about $\Spec A$ as identifying
$0$ and $1$ in ${\bbb A}^1_k=\Spec k[x]$.)
\itemitem{\bf (c)} Same question for the ring $A = k[x^2,x^3] \subset
k[x]$ (except now $k = {\bbb F}_2$ works as well).

\medskip\item{\bf 4.} Higher dimensions.
\itemitem{\bf (a)} Prove that every invertible sheaf on two dimensional
affine space is trivial. More precisely, let 
${\bbb A}^2_k = \Spec k[x,y]$ where $k$ is a field.
Show that every invertible sheaf on ${\bbb A}^2_k$ is trivial.
(Hint: One way to do this is to consider the corresponding
module $M$, to look at $M \otimes_{k[x,y]} k(x)[y]$, and
then use 3(a) to find a generator for this; then you still have to think.
Another way to is to use 2(b) and use what we know about ideals of the
polynomial ring: primes of height one are generated by an irreducible
polynomial; then you still have to think.)
\itemitem{\bf (b)} Prove that every invertible sheaf on any open
subscheme of two dimensional affine space is trivial. More precisely, let 
$U \subset {\bbb A}^2_k$ be an open subscheme where $k$ is a field.
Show that every invertible sheaf on $U$ is trivial. Hint: Show that every
invertible sheaf on $U$ extends to one on ${\bbb A}^2_k$. Not easy;
but you can find it in Hartshorne.
\itemitem{\bf (c)} Find an example of a nontrivial
invertible sheaf on a punctured cone over a field. More
precisely, let $k$ be a field and let $C = \Spec k[x,y,z]/(xy-z^2)$.
Let $U = C \setminus \{ (x,y,z) \}$. Find a nontrivial
invertible sheaf on $U$. Hint: It may be easier to compute the
group of isomorphism classes of invertible sheaves on $U$ than to
just find one. Note that $U$ is covered by the opens 
$\Spec k[x,y,z,1/x]/(xy-z^2) $ and $\Spec k[x,y,z,1/y]/(xy-z^2)$
which are ``easy'' to deal with.


\bye
\nopagenumbers

%%%%%%%%%%%Blackboardbold 
\font\gbbb=msbm10 scaled \magstep1
\font\bbbf=msbm10 
\font\sbbb=msbm6 
\font\ssbbb=msbm5 
\textfont6=\bbbf
\scriptfont6=\sbbb 
\scriptscriptfont6=\ssbbb 
\def\bbb{\fam6}
\def\mP{{\bbb P}} 
\def\mA{{\bbb A}} 
\def\mB{{\bbb B}} 
\def\mR{{\bbb R}}
\def\mZ{{\bbb Z}}

%%%%Gothic
\font\ggothic=eufm10 scaled \magstep1
\font\gothicf=eufm10
\font\sgothic=eufm7
\font\ssgothic=eufm5
\textfont5=\gothicf
\scriptfont5=\sgothic
\scriptscriptfont5=\ssgothic
\def\gothic{\fam5}


\font\Kopfont=cmbx12
\def\mapright#1{\smash{\mathop{\longrightarrow}\limits^{#1}}}
\def\mapdown#1{\Big\downarrow\rlap{$\vcenter{\hbox{$\scriptstyle#1$}}$}}
\def\downmap#1{\downarrow\rlap{$\vcenter{\hbox{$\scriptstyle#1$}}$}}
\def\mapup#1{\Big\uparrow\rlap{$\vcenter{\hbox{$\scriptstyle#1$}}$}}
\def\longlongrightarrow{\relbar \joinrel \longrightarrow}
\def\cC{{\cal C}}
\def\cD{{\cal D}}
\def\gp{{\gothic p}}
\def\gq{{\gothic q}}
\def\Spec{\mathop{\rm Spec}}
\def\Proj{\mathop{\rm Proj}}

\centerline{\Kopfont Schemes}

\smallskip
\centerline{Exercises 6}

\bigskip\noindent
{\bf Schemes -- Examples again}

\bigskip\noindent
Feel free to quote results from Hartshorne or EGA.

\bigskip\item{\bf 1.} {\v C}ech cohomology. Here $k$ is a field.
\itemitem{\bf (a)} Let $X$ be a scheme with an open covering
${\cal U} : X = U_1 \cup U_2$, with $U_1 = \Spec k[x]$, $U_2= \Spec k[y]$
with $U_1 \cap U_2 = \Spec k[z,1/z]$ and with open immersions
$U_1 \cap U_2 \to U_1$ resp.\ $U_1 \cap U_2 \to U_2$ determined
by $x \mapsto z$ resp.\ $y \mapsto z$ (and I really mean this).
(We've seen in the lectures that such an $X$ exists; it is the affine
line zith zero doubled.) Compute ${\mathaccent 20 H}^1({\cal U}, {\cal O})$;
eg.\ give a basis for it as a $k$-vectorspace.
\itemitem{\bf (b)} For each element in
${\mathaccent 20 H}^1({\cal U}, {\cal O})$
construct an exact sequence of sheaves of ${\cal O}_X$-modules
$$
0 \to {\cal O}_X \to E \to {\cal O}_X \to 0
$$ 
such that the boundary $\delta(1) \in {\mathaccent 20 H}^1({\cal U}, {\cal O})$
equals the given element. (Part of the problem is to make sense of this. 
It is also OK to show abstractly such a thing has to exist.)

\medskip\noindent
{\bf Definition of delta.} Suppose that 
$$
0 \to {\cal F}_1 \to {\cal F}_2 \to {\cal F}_3 \to 0
$$
is a short exact sequence of abelian sheaves on any topological space $X$.
The boundary map
$\delta : H^0(X, {\cal F}_3) \to {\mathaccent 20 H}^1(X, {\cal F}_1)$
is defined as follows. Take an element $\tau \in H^0(X, {\cal F}_3)$.
Choose an open covering ${\cal U} : X = \bigcup_{i\in I} U_i$ such
that for each $i$ there exists a section $\tilde \tau_i \in {\cal F}_2$
lifting the restriction of $\tau$ to $U_i$. Then consider the assignment
$$
(i_0, i_1) \longmapsto
\tilde \tau_{i_0}|_{U_{i_0i_1}} - \tilde \tau_{i_1}|_{U_{i_0i_1}}.
$$
This is clearly a 1-coboundary in the {\v C}ech complex
${\mathaccent 20 C}^\ast({\cal U}, {\cal F}_2)$. But we observe that
(thinking of ${\cal F}_1$ as a subsheaf of ${\cal F}_2$) the RHS
always is a section of ${\cal F}_1$ over $U_{i_0i_1}$. Hence we
see that the assignment defines a 1-cochain in the complex
${\mathaccent 20 C}^\ast({\cal U}, {\cal F}_2)$. The cohomology class of
this 1-cochain is by definition $\delta(\tau)$.

\medskip\item{\bf 2.} Algebra. (Silly and should be easy.)
\itemitem{\bf (a)} Give an example of a ring $A$ and a nonsplit
short exact sequence of $A$-modules
$$
0 \to M_1 \to M_2 \to M_3 \to 0.
$$
\itemitem{\bf (b)} Give an example of a nonsplit sequence of $A$-modules
as above and a faithfully flat $A \to B$ such that 
$$
0 \to M_1\otimes_AB \to M_2\otimes_AB \to M_3\otimes_AB \to 0.
$$
is split as a sequence of $B$-modules.

\medskip\item{\bf 3.} Maps of $\Proj$. Let $R$ and $S$ be graded rings. So
$R = \oplus_{d \geq 0} R_d$ and $R_a \cdot R_b \subset R_{a+b}$.
Suppose we have a ring map
$$
\varphi : R \to S
$$
such that there exists an integer $e \geq 1$ such that
$\varphi( R_d ) \subset S_{de}$.
\itemitem{\bf (a)} For which elements $\gp \in \Proj(S)$ is
there a well-defined corresponding point in $\Proj(R)$? In other words,
find a suitable open $U \subset \Proj(S)$ such that $\varphi$ defines
a continuous map $\Proj(\varphi) : U \to \Proj(R)$.
\itemitem{\bf (b)} Give an example where $U \not = \Proj(S)$.
\itemitem{\bf (c)} Give an example where $U = \Proj(S)$.
\itemitem{\bf (d)} (Do not write this down.) Convince yourself that 
the continuous map $U \to \Proj(R)$ comes canonically with
a map on sheaves so that $\Proj(\varphi)$ is a morphism of schemes:
$$
\Proj(S) \supset U \longrightarrow \Proj(R)
$$

\medskip\noindent{\bf Notation.} Let $R$ be a graded ring as above and
let $n \geq 0$ be an integer. Let $X = \Proj(R)$. Then there is a unique
quasi-coherent ${\cal O}_X$-module ${\cal O}_X(n)$ on $X$ such that
for every homogeneous element $f \in R$ of positive degree we have
${\cal O}_X |_{D_{+}(f)}$ is the quasi-coherent sheaf associated to the
$R_{(f)} = (R_f)_0$-module $(R_f)_n$ ($=$elements homogenous of degree
$n$ in $R_f = R[1/f]$). See Hartshorne, page 116+. Note that there are
natural maps
$$
{\cal O}_X(n_1) \otimes_{{\cal O}_X} {\cal O}_X(n_2) \longrightarrow
{\cal O}_X(n_1+n_2)
$$

\medskip\item{\bf 4.} Pathologies in $\Proj$. 
Give examples of $R$ as above such that
\itemitem{\bf (a)} ${\cal O}_X(1)$ is not an invertible ${\cal O}_X$-module.
\itemitem{\bf (b)} ${\cal O}_X(1)$ is invertible, but the
natural map ${\cal O}_X(1) \otimes_{{\cal O}_X} {\cal O}_X(1) \to
{\cal O}_X(2)$ is NOT an isomorphism.

\bye
\nopagenumbers

%%%%%%%%%%%Blackboardbold 
\font\gbbb=msbm10 scaled \magstep1
\font\bbbf=msbm10 
\font\sbbb=msbm6 
\font\ssbbb=msbm5 
\textfont6=\bbbf
\scriptfont6=\sbbb 
\scriptscriptfont6=\ssbbb 
\def\bbb{\fam6}
\def\mP{{\bbb P}} 
\def\mA{{\bbb A}} 
\def\mB{{\bbb B}} 
\def\mR{{\bbb R}}
\def\mZ{{\bbb Z}}

%%%%Gothic
\font\ggothic=eufm10 scaled \magstep1
\font\gothicf=eufm10
\font\sgothic=eufm7
\font\ssgothic=eufm5
\textfont5=\gothicf
\scriptfont5=\sgothic
\scriptscriptfont5=\ssgothic
\def\gothic{\fam5}


\font\Kopfont=cmbx12
\def\mapright#1{\smash{\mathop{\longrightarrow}\limits^{#1}}}
\def\mapdown#1{\Big\downarrow\rlap{$\vcenter{\hbox{$\scriptstyle#1$}}$}}
\def\downmap#1{\downarrow\rlap{$\vcenter{\hbox{$\scriptstyle#1$}}$}}
\def\mapup#1{\Big\uparrow\rlap{$\vcenter{\hbox{$\scriptstyle#1$}}$}}
\def\longlongrightarrow{\relbar \joinrel \longrightarrow}
\def\cC{{\cal C}}
\def\cD{{\cal D}}
\def\gp{{\gothic p}}
\def\gq{{\gothic q}}
\def\Spec{\mathop{\rm Spec}}
\def\Proj{\mathop{\rm Proj}}
\def\length{\mathop{\rm length}\nolimits}

\centerline{\Kopfont Schemes}

\smallskip
\centerline{Exercises 6}

\bigskip\noindent
{\bf Schemes -- Divisors}

\bigskip\noindent
Feel free to quote results from Hartshorne or EGA.

\bigskip\noindent
{\bf Definitions.} Throughout, let $X$ be a Noetherian, integral and
separated scheme.
\item{\bf (a)} A Weil divisor is a formal linear combination
$\Sigma n_i[Z_i]$ of prime divisors $Z_i$ with integer coefficients.
\item{\bf (b)} A prime divisor is a closed subscheme $Z \subset X$,
which is integral with generic point $\xi \in Z$ such that
${\cal O}_{X,\xi}$ has dimension $1$. We will use the notation 
${\cal O}_{X,Z} = {\cal O}_{X,\xi}$
when $\xi \in Z \subset X$ is as above. Note that ${\cal O}_{X,Z} \subset
K(X)$ is a subring of the function field of $X$.
\item{\bf (c)} The Weil divisor associated to a rational function
$f \in K(X)^\ast$ is the sum $\Sigma v_Z(f)[Z]$. Here $v_Z(f)$ is
defined as follows
\itemitem{(c1)} If $f \in {\cal O}_{X,Z}^\ast$ then $v_Z(f)=0$.
\itemitem{(c2)} If $f \in {\cal O}_{X,Z}$ then 
$$
v_Z(f)=\length_{{\cal O}_{X,Z}}({\cal O}_{X,Z}/(f)).
$$
\itemitem{(c3)} If $f = {a \over b}$ with $a,b \in {\cal O}_{X,Z}$
then 
$$
v_Z(f)=\length_{{\cal O}_{X,Z}}({\cal O}_{X,Z}/(a)) -
\length_{{\cal O}_{X,Z}}({\cal O}_{X,Z}/(b)).
$$
\item{\bf (d)} An effective Cartier divisor on {\it any} scheme $S$
is a closed subscheme $D \subset S$ such that every point $d\in D$
has an affine open neighbourhood $\Spec A = U \subset S$ in $S$
so that $D \cap U = \Spec A/(f)$ with $f \in A$ a nonzero divisor.
\item{\bf (e)} The Weil divisor $[D]$ associated to an effective
Cartier divisor $D \subset X$ of our Noetherian integral separated
scheme $X$ is defined as the sum $\Sigma v_Z(D)[Z]$ where
$v_Z(D)$ is defined as follows
\itemitem{(e1)} If the generic point $\xi$ of $Z$ is not in $D$
then $v_Z(D)=0$.
\itemitem{(e1)} If the generic point $\xi$ of $Z$ is in $D$
then 
$$
v_Z(D)=\length_{{\cal O}_{X,Z}}({\cal O}_{X,Z}/(f))
$$
where $f \in {\cal O}_{X,Z}={\cal O}_{X,\xi}$ is the nonzero divisor
which defines $D$ in an affine neighbourhood of $\xi$ (as in definition
{\bf (d)} above).
\item{\bf (f)} Let $S$ be {\it any} scheme. The sheaf of total quotient
rings ${\cal K}_S$ is the sheaf of ${\cal O}_S$-algebras which is
the sheafification of the pre-sheaf ${\cal K}'$ defined as follows.
For $U \subset S$ open we set ${\cal K}'(U) = S_U^{-1}{\cal O}_S(U)$
where $S_U \subset {\cal O}_S(U)$ is the multiplicative subset
consisting of sections $f \in {\cal O}_S(U)$ such that the germ
of $f$ in ${\cal O}_{S,u}$ is a nonzero divisor for every $u\in U$.
In particular the elements of $S_U$ are all nonzero divisors.
Thus ${\cal O}_S$ is a subsheaf of ${\cal K}_S$, and we get a
short exact sequence
$$
0 \to {\cal O}_S^\ast \to {\cal K}_S^\ast \to
{\cal K}_S^\ast/{\cal O}_S^\ast \to 0.
$$
\item{\bf (g)} A Cartier divisor on {\it any} scheme $S$ is a global
section of the quotient sheaf ${\cal K}_S^\ast/{\cal O}_S^\ast$.
\item{\bf (h)} The Weil divisor associated to a Cartier divisor
$\tau \in \Gamma(X, {\cal K}_X^\ast/{\cal O}_X^\ast)$ over our 
Noetherian integral separated scheme
$X$ is the sum $\Sigma v_Z(\tau)[Z]$ where $v_Z(\tau)$ is defined
as by the following recipe
\itemitem{(h1)} If the germ of $\tau$ at the generic point $\xi$
of $Z$ is zero -- in other words the image of $\tau$ in the stalk
$({\cal K}^\ast/{\cal O}^\ast)_\xi$ is ``zero'' -- then $v_Z(\tau)=0$.
\itemitem{(h2)} Find an affine open neighbourhood $\Spec A = U \subset X$
so that $\tau|_U$ is the image of a section $f \in {\cal K}(U)$
and moreover $f = a/b$ with $a,b \in A$. Then we set
$$
v_Z(f)=\length_{{\cal O}_{X,Z}}({\cal O}_{X,Z}/(a)) -
\length_{{\cal O}_{X,Z}}({\cal O}_{X,Z}/(b)).
$$

\medskip\noindent
{\bf Remarks.} (a) On a Noetherian integral separated scheme $X$ the 
sheaf ${\cal K}_X$ is constant with value the function field $K(X)$.

\noindent (b) To make sense out of the definitions above one needs
to show that
$$
\length_{\cal O}({\cal O}/(ab)) =
\length_{\cal O}({\cal O}/(a)) +
\length_{\cal O}({\cal O}/(b))
$$
for any pair $(a,b)$ of nonzero elements of a Noetherian 1-dimensional
local domain ${\cal O}$. This will be done in the lectures.

\goodbreak\bigskip\item{\bf 1.} Describe how to assign a Cartier divisor
to an effective Cartier divisor.

\medskip\noindent
The following questions have some logical dependencies; if you point them
out then you won't have to do all of them.

\medskip\item{\bf 2.} Give an example of a Weil divisor
(on a Noetherian integral separated scheme) which is not
the Weil divisor associated to a rational function.

\medskip\item{\bf 3.} Give an example of a Weil divisor
(on a Noetherian integral separated scheme) which is not
the Weil divisor associated to any effective Cartier divisor.

\medskip\item{\bf 4.} Give an example of a Weil divisor
(on a Noetherian integral separated scheme) which is not
the Weil divisor associated to any Cartier divisor.

\medskip\item{\bf 5.} Give an example of a Weil divisor $D$
(on a Noetherian integral separated scheme) which is not
the Weil divisor associated to any Cartier divisor but
such that $nD$ is the Weil divisor associate to a Cartier
divisor for some $n>1$.

\medskip\item{\bf 6.} Give an example of a Weil divisor $D$
(on a Noetherian integral separated scheme) which is not
the Weil divisor associated to any Cartier divisor and
such that $nD$ is NOT the Weil divisor associate to a Cartier
divisor for any $n>1$.

\medskip\item{\bf 7.}
Give an example of a Cartier divisor
which is not the difference of (the Cartier divisors associated
to) two effective Cartier divisors.\footnote{$\dagger$}{I do not know how to do this one myself, but I think this happens.}

\bye
\nopagenumbers

%%%%%%%%%%%Blackboardbold 
\font\gbbb=msbm10 scaled \magstep1
\font\bbbf=msbm10 
\font\sbbb=msbm6 
\font\ssbbb=msbm5 
\textfont6=\bbbf
\scriptfont6=\sbbb 
\scriptscriptfont6=\ssbbb 
\def\bbb{\fam6}
\def\mP{{\bbb P}} 
\def\mA{{\bbb A}} 
\def\mB{{\bbb B}} 
\def\mR{{\bbb R}}
\def\mZ{{\bbb Z}}

%%%%Gothic
\font\ggothic=eufm10 scaled \magstep1
\font\gothicf=eufm10
\font\sgothic=eufm7
\font\ssgothic=eufm5
\textfont5=\gothicf
\scriptfont5=\sgothic
\scriptscriptfont5=\ssgothic
\def\gothic{\fam5}


\font\Kopfont=cmbx12
\def\mapright#1{\smash{\mathop{\longrightarrow}\limits^{#1}}}
\def\mapdown#1{\Big\downarrow\rlap{$\vcenter{\hbox{$\scriptstyle#1$}}$}}
\def\downmap#1{\downarrow\rlap{$\vcenter{\hbox{$\scriptstyle#1$}}$}}
\def\mapup#1{\Big\uparrow\rlap{$\vcenter{\hbox{$\scriptstyle#1$}}$}}
\def\longlongrightarrow{\relbar \joinrel \longrightarrow}
\def\cC{{\cal C}}
\def\cD{{\cal D}}
\def\gp{{\gothic p}}
\def\gq{{\gothic q}}
\def\Spec{\mathop{\rm Spec}}
\def\Proj{\mathop{\rm Proj}}
\def\length{\mathop{\rm length}\nolimits}

\centerline{\Kopfont Schemes}

\bigskip\noindent
{\bf Definitions and results.} K\"ahler differentials.
\item{\bf (a)} Let $R \to A$ be a ring map. The module of K\"ahler
differentials of $A$ over $R$ is 
$$
\Omega^1_{A/R} = \bigoplus\nolimits_{a\in A} A \cdot {\rm d}a \Big/
\big\langle {\rm d}(a_1a_2)-a_1{\rm d}a_2-a_2{\rm d}a_1, {\rm d}r\big\rangle.
$$
The canonical universal $R$-derivation ${\rm d} : A \to \Omega^1_{A/R}$ 
maps $a\mapsto {\rm d}a$.
\item{\bf (b)} Consider the short exact sequence
$$
0 \to I \to A\otimes_R A \to A \to 0
$$
which defines the ideal $I$. There is a canonical derivation
${\rm d} : A \to I/I^2$ which maps $a$ to the class of
$a\otimes 1 - 1 \otimes a$. This is another presentation of
the module of derivations of $A$ over $R$, in other words
$$
(I/I^2, {\rm d}) \cong (\Omega^1_{A/R}, {\rm d}).
$$
\item{\bf (c)} For multiplicative subsets $S_R \subset R$ and
$S_A \subset A$ such that $S_R$ maps into $S_A$ we have
$$
\Omega^1_{S_A^{-1}A / S_R^{-1}R} =
S_A^{-1}\Omega^1_{A/R}.
$$
\item{\bf (d)} If $A$ is a finitely presented $R$-algebra then
$\Omega^1_{A/R}$ is a finitely presented $A$-module. Hence in
this case the {\it fitting} ideals of $\Omega^1_{A/R}$ are defined.
(See exercise set 6 of last semester.)
\item{\bf (e)} Let $f : X \to S$ be a morphism of schemes. There is
a quasi-coherent sheaf of ${\cal O}_X$-modules $\Omega^1_{X/S}$
and a ${\cal O}_S$-linear derivation 
$$
{\rm d} : {\cal O}_X \longrightarrow \Omega^1_{X/S}
$$
such that for any affine opens $\Spec A \subset X$, $\Spec R \subset S$
with $f(\Spec A) \subset \Spec R$ we have
$$
\Gamma(\Spec A, \Omega^1_{X/S}) = \Omega^1_{A/R}
$$
compatibly with ${\rm d}$.

\medskip\item{\bf 1.} Let $k[\epsilon]$ be the ring of dual numbers
over the field $k$, i.e., $\epsilon^2=0$.
\itemitem{\bf (a)} Consider the ring map
$$
R = k[\epsilon] \to A = k[x,\epsilon]/(\epsilon x)
$$
Show that the fitting ideals of $\Omega^1_{A/R}$ are (starting with the
zeroth fitting ideal)
$$
(\epsilon), A, A,\ldots
$$
\itemitem{\bf (b)} Consider the map $R=k[t] \to 
A=k[x,y,t]/(x(y-t)(y-1),x(x-t))$. Show that the fitting ideals of
of $\Omega^1_{A/R}$ in $A$ are (assume characteristic $k$ is zero
for simplicity)
$$
x(2x-t)(2y-t-1)A,\ (x,y,t)\cap (x,y-1,t),\ A,\ A,\ldots
$$
So the $0$-the fitting ideal is cut out by a single element of $A$,
the $1$st fitting ideal defines two closed points of $\Spec A$, and
the others are all trivial.
\itemitem{\bf (c)} Consider the map $R=k[t] \to A=k[x,y,t]/(xy-t^n)$.
Compute the fitting ideals of $\Omega^1_{A/R}$.

\medskip\noindent
{\bf Remark.} The $k$th fitting ideal of $\Omega^1_{X/S}$ is commonly used
to define the singular scheme of the morphism $X \to S$ when $X$ has relative
dimension $k$ over $S$. But as part (a) shows, you have to be careful doing
this when your family does not have ``constant'' fibre dimension, e.g., when 
it is not flat. As part (b) shows, flatness doesn't garantee it works either
(and yes this is a flat family). In ``good cases'' -- such as in (c) -- for
families of curves you expect the $0$-th fitting ideal to be zero and
the $1$st fitting ideal to define (scheme-theoretically) the singular locus.

\medskip\item{\bf 2.} Suppose that $R$ is a ring and 
$$
A = k[x_1,\ldots,x_n]/(f_1,\ldots,f_n).
$$
Note that we are assuming that $A$ is presented by the same
number of equations as variables. Thus the matrix of partial
derivatives
$$
( \partial f_i / \partial x_j )
$$
is $n\times n$, i.e., a square matrix. Assume that
its determinant is invertible as an element in $A$. Note that
this is exactly the condition that says that $\Omega^1_{A/R} = (0)$
in this case of $n$-generators and $n$ relations.
Let $\pi : B' \to B$ be a surjection of $R$-algebras 
whose kernel $J$ has square zero (as an ideal in $B'$).
Let $\varphi : A \to B$ be a homomorphism of $R$-algebras. 
Show there exists a unique homomorphism of $R$-algebras
$\varphi' : A \to B'$ such that $\varphi = \pi \circ \varphi'$.

\medskip\item{\bf 3.} Find a generalization
of the result of the previous exercise to the case where $A=R[x,y]/(f)$.

\bye
\nopagenumbers

%%%%%%%%%%%Blackboardbold 
\font\gbbb=msbm10 scaled \magstep1
\font\bbbf=msbm10 
\font\sbbb=msbm6 
\font\ssbbb=msbm5 
\textfont6=\bbbf
\scriptfont6=\sbbb 
\scriptscriptfont6=\ssbbb 
\def\bbb{\fam6}
\def\mA{{\bbb A}} 
\def\mB{{\bbb B}} 
\def\mC{{\bbb C}}
\def\mF{{\bbb F}}
\def\mN{{\bbb N}}
\def\mP{{\bbb P}} 
\def\mQ{{\bbb Q}} 
\def\mR{{\bbb R}}
\def\mZ{{\bbb Z}}

%%%%Gothic
\font\ggothic=eufm10 scaled \magstep1
\font\gothicf=eufm10
\font\sgothic=eufm7
\font\ssgothic=eufm5
\textfont5=\gothicf
\scriptfont5=\sgothic
\scriptscriptfont5=\ssgothic
\def\gothic{\fam5}


\font\Kopfont=cmbx12
\def\mapright#1{\smash{\mathop{\longrightarrow}\limits^{#1}}}
\def\mapdown#1{\Big\downarrow\rlap{$\vcenter{\hbox{$\scriptstyle#1$}}$}}
\def\downmap#1{\downarrow\rlap{$\vcenter{\hbox{$\scriptstyle#1$}}$}}
\def\mapup#1{\Big\uparrow\rlap{$\vcenter{\hbox{$\scriptstyle#1$}}$}}
\def\longlongrightarrow{\relbar \joinrel \longrightarrow}
\def\cC{{\cal C}}
\def\cD{{\cal D}}
\def\gp{{\gothic p}}
\def\gq{{\gothic q}}
\def\Spec{\mathop{\rm Spec}}
\def\Proj{\mathop{\rm Proj}}
\def\length{\mathop{\rm length}\nolimits}

\centerline{\Kopfont Schemes, Final Exam}

\bigskip\noindent
{\bf 0.} Definitions. Provide definitions of the following concepts.
\item{\bf (a)} $X$ is a {\it scheme} 
\item{\bf (b)} the morphism of schemes $f : X \to Y$ is {\it finite}
\item{\bf (c)} the morphisms of schemes $f : X \to Y$ is {\it of finite type}
\item{\bf (d)} the scheme $X$ is {\it Noetherian}
\item{\bf (e)} the ${\cal O}_X$-module ${\cal L}$ on the scheme $X$ is {\it invertible}
\item{\bf (f)} the {\it genus} of a nonsingular projective curve over an algebraically closed field

\medskip\noindent
{\bf 1.} Let $X = \Spec \mZ[x,y]$, and let ${\cal F}$ be a quasi-coherent
${\cal O}_X$-module. Suppose that ${\cal F}$ is zero when restricted to the
standard affine open $D(x)$.
\item{\bf (a)} Show that every global section $s$ of ${\cal F}$ is killed by some power of $x$, i.e., $x^ns=0$ for some $n\in \mN$.
\item{\bf (b)} Do you think the same is true if we do not assume that ${\cal F}$ is quasi-coherent?

\medskip\noindent
{\bf 2.} Suppose that $X \to \Spec(R)$ is a proper morphism and that
$R$ is a discrete valuation ring with residue field $k$. Suppose that
$X \times_{\Spec R} \Spec k$ is the empty scheme. Show that $X$ is the
empty scheme.

\medskip\noindent
{\bf 3.} Consider the projective\footnote{${}^\dagger$}{The projective embedding is $((X_0,X_1),(Y_0,Y_1))\mapsto (X_0Y_0,X_0Y_1,X_1Y_0,X_1Y_1)$ in other words $(x,y)\mapsto (1,y,x,xy)$.} variety
$$
\mP^1 \times \mP^1 = \mP^1_{\mC} \times_{\Spec \mC} \mP^1_\mC
$$
over the field of complex numbers $\mC$. It is covered by four affine pieces,
corresponding to pairs of standard affine pieces of $\mP^1_\mC$. For example,
suppose we use homogenous coordinates $X_0, X_1$ on the first factor and
$Y_0, Y_1$ on the second. Set $x=X_1/X_0$, and $y=Y_1/Y_0$. Then the 4 affine
open pieces are the spectra of the  rings
$$ 
\mC[x,y],\ 
\mC[x^{-1},y],\ 
\mC[x,y^{-1}],\ 
\mC[x^{-1},y^{-1}].
$$
Let $X \subset \mP^1 \times \mP^1$ be the closed subscheme which is the closure of the closed subset of the first affine piece given by the equation
$$
y^3(x^4+1) = x^4 -1.
$$
\item{\bf (a)} Show that $X$ is contained in the union of the first and
the last of the 4 affine open pieces.
\item{\bf (b)} Show that $X$ is a nonsingular projective curve.
\item{\bf (c)} Consider the morphism $pr_2 : X \to \mP^1$ (projection onto
the first factor). On the first affine piece it is the map $(x,y) \mapsto x$.
Briefly explain why it has degree $3$.
\item{\bf (d)} Compute the ramification points and ramification indices 
for the map $pr_2 : X \to \mP^1$.
\item{\bf (e)} Compute the genus of $X$.

\medskip\noindent
{\bf 4.} Let $X \to \Spec \mZ$ be a morphism of finite type. Suppose that
there is an infinite number of primes $p$ such that
$X\times_{\Spec \mZ} \Spec \mF_p$ is not empty. 
\item{\bf (a)} Show that $X \times_{\Spec \mZ}\Spec \mQ$ is not empty.
\item{\bf (b)} Do you think the same is true if we replace the condition
``finite type'' by the condition ``locally of finite type''?

\bye
