\IfFileExists{stacks-project.cls}{%
\documentclass{stacks-project}
}{%
\documentclass{amsart}
}

% The following AMS packages are automatically loaded with
% the amsart documentclass:
%\usepackage{amsmath}
%\usepackage{amssymb}
%\usepackage{amsthm}

% For dealing with references we use the comment environment
\usepackage{verbatim}
\newenvironment{reference}{\comment}{\endcomment}
%\newenvironment{reference}{}{}
\newenvironment{slogan}{\comment}{\endcomment}
\newenvironment{history}{\comment}{\endcomment}

% For commutative diagrams you can use
% \usepackage{amscd}
\usepackage[all]{xy}

% We use 2cell for 2-commutative diagrams.
\xyoption{2cell}
\UseAllTwocells

% To put source file link in headers.
% Change "template.tex" to "this_filename.tex"
% \usepackage{fancyhdr}
% \pagestyle{fancy}
% \lhead{}
% \chead{}
% \rhead{Source file: \url{template.tex}}
% \lfoot{}
% \cfoot{\thepage}
% \rfoot{}
% \renewcommand{\headrulewidth}{0pt}
% \renewcommand{\footrulewidth}{0pt}
% \renewcommand{\headheight}{12pt}

\usepackage{multicol}

% For cross-file-references
\usepackage{xr-hyper}

% Package for hypertext links:
\usepackage{hyperref}

% For any local file, say "hello.tex" you want to link to please
% use \externaldocument[hello-]{hello}
\externaldocument[introduction-]{introduction}
\externaldocument[conventions-]{conventions}
\externaldocument[sets-]{sets}
\externaldocument[categories-]{categories}
\externaldocument[topology-]{topology}
\externaldocument[sheaves-]{sheaves}
\externaldocument[sites-]{sites}
\externaldocument[stacks-]{stacks}
\externaldocument[fields-]{fields}
\externaldocument[algebra-]{algebra}
\externaldocument[brauer-]{brauer}
\externaldocument[homology-]{homology}
\externaldocument[derived-]{derived}
\externaldocument[simplicial-]{simplicial}
\externaldocument[more-algebra-]{more-algebra}
\externaldocument[smoothing-]{smoothing}
\externaldocument[modules-]{modules}
\externaldocument[sites-modules-]{sites-modules}
\externaldocument[injectives-]{injectives}
\externaldocument[cohomology-]{cohomology}
\externaldocument[sites-cohomology-]{sites-cohomology}
\externaldocument[dga-]{dga}
\externaldocument[dpa-]{dpa}
\externaldocument[hypercovering-]{hypercovering}
\externaldocument[schemes-]{schemes}
\externaldocument[constructions-]{constructions}
\externaldocument[properties-]{properties}
\externaldocument[morphisms-]{morphisms}
\externaldocument[coherent-]{coherent}
\externaldocument[divisors-]{divisors}
\externaldocument[limits-]{limits}
\externaldocument[varieties-]{varieties}
\externaldocument[topologies-]{topologies}
\externaldocument[descent-]{descent}
\externaldocument[perfect-]{perfect}
\externaldocument[more-morphisms-]{more-morphisms}
\externaldocument[flat-]{flat}
\externaldocument[groupoids-]{groupoids}
\externaldocument[more-groupoids-]{more-groupoids}
\externaldocument[etale-]{etale}
\externaldocument[chow-]{chow}
\externaldocument[intersection-]{intersection}
\externaldocument[pic-]{pic}
\externaldocument[adequate-]{adequate}
\externaldocument[dualizing-]{dualizing}
\externaldocument[duality-]{duality}
\externaldocument[discriminant-]{discriminant}
\externaldocument[local-cohomology-]{local-cohomology}
\externaldocument[curves-]{curves}
\externaldocument[resolve-]{resolve}
\externaldocument[models-]{models}
\externaldocument[pione-]{pione}
\externaldocument[etale-cohomology-]{etale-cohomology}
\externaldocument[proetale-]{proetale}
\externaldocument[crystalline-]{crystalline}
\externaldocument[spaces-]{spaces}
\externaldocument[spaces-properties-]{spaces-properties}
\externaldocument[spaces-morphisms-]{spaces-morphisms}
\externaldocument[decent-spaces-]{decent-spaces}
\externaldocument[spaces-cohomology-]{spaces-cohomology}
\externaldocument[spaces-limits-]{spaces-limits}
\externaldocument[spaces-divisors-]{spaces-divisors}
\externaldocument[spaces-over-fields-]{spaces-over-fields}
\externaldocument[spaces-topologies-]{spaces-topologies}
\externaldocument[spaces-descent-]{spaces-descent}
\externaldocument[spaces-perfect-]{spaces-perfect}
\externaldocument[spaces-more-morphisms-]{spaces-more-morphisms}
\externaldocument[spaces-flat-]{spaces-flat}
\externaldocument[spaces-groupoids-]{spaces-groupoids}
\externaldocument[spaces-more-groupoids-]{spaces-more-groupoids}
\externaldocument[bootstrap-]{bootstrap}
\externaldocument[spaces-pushouts-]{spaces-pushouts}
\externaldocument[groupoids-quotients-]{groupoids-quotients}
\externaldocument[spaces-more-cohomology-]{spaces-more-cohomology}
\externaldocument[spaces-simplicial-]{spaces-simplicial}
\externaldocument[formal-spaces-]{formal-spaces}
\externaldocument[restricted-]{restricted}
\externaldocument[spaces-resolve-]{spaces-resolve}
\externaldocument[formal-defos-]{formal-defos}
\externaldocument[defos-]{defos}
\externaldocument[cotangent-]{cotangent}
\externaldocument[examples-defos-]{examples-defos}
\externaldocument[algebraic-]{algebraic}
\externaldocument[examples-stacks-]{examples-stacks}
\externaldocument[stacks-sheaves-]{stacks-sheaves}
\externaldocument[criteria-]{criteria}
\externaldocument[artin-]{artin}
\externaldocument[quot-]{quot}
\externaldocument[stacks-properties-]{stacks-properties}
\externaldocument[stacks-morphisms-]{stacks-morphisms}
\externaldocument[stacks-limits-]{stacks-limits}
\externaldocument[stacks-cohomology-]{stacks-cohomology}
\externaldocument[stacks-perfect-]{stacks-perfect}
\externaldocument[stacks-introduction-]{stacks-introduction}
\externaldocument[stacks-more-morphisms-]{stacks-more-morphisms}
\externaldocument[stacks-geometry-]{stacks-geometry}
\externaldocument[moduli-]{moduli}
\externaldocument[moduli-curves-]{moduli-curves}
\externaldocument[examples-]{examples}
\externaldocument[exercises-]{exercises}
\externaldocument[guide-]{guide}
\externaldocument[desirables-]{desirables}
\externaldocument[coding-]{coding}
\externaldocument[obsolete-]{obsolete}
\externaldocument[fdl-]{fdl}
\externaldocument[index-]{index}

% Theorem environments.
%
\theoremstyle{plain}
\newtheorem{theorem}[subsection]{Theorem}
\newtheorem{proposition}[subsection]{Proposition}
\newtheorem{lemma}[subsection]{Lemma}

\theoremstyle{definition}
\newtheorem{definition}[subsection]{Definition}
\newtheorem{example}[subsection]{Example}
\newtheorem{exercise}[subsection]{Exercise}
\newtheorem{situation}[subsection]{Situation}

\theoremstyle{remark}
\newtheorem{remark}[subsection]{Remark}
\newtheorem{remarks}[subsection]{Remarks}

\numberwithin{equation}{subsection}

% Macros
%
\def\lim{\mathop{\rm lim}\nolimits}
\def\colim{\mathop{\rm colim}\nolimits}
\def\Spec{\mathop{\rm Spec}}
\def\Hom{\mathop{\rm Hom}\nolimits}
\def\Ext{\mathop{\rm Ext}\nolimits}
\def\SheafHom{\mathop{\mathcal{H}\!{\it om}}\nolimits}
\def\SheafExt{\mathop{\mathcal{E}\!{\it xt}}\nolimits}
\def\Sch{\textit{Sch}}
\def\Mor{\mathop{\rm Mor}\nolimits}
\def\Ob{\mathop{\rm Ob}\nolimits}
\def\Sh{\mathop{\textit{Sh}}\nolimits}
\def\NL{\mathop{N\!L}\nolimits}
\def\proetale{{pro\text{-}\acute{e}tale}}
\def\etale{{\acute{e}tale}}
\def\QCoh{\textit{QCoh}}
\def\Ker{\mathop{\rm Ker}}
\def\Im{\mathop{\rm Im}}
\def\Coker{\mathop{\rm Coker}}
\def\Coim{\mathop{\rm Coim}}

%
% Macros for moduli stacks/spaces
%
\def\QCohstack{\mathcal{QC}\!{\it oh}}
\def\Cohstack{\mathcal{C}\!{\it oh}}
\def\Spacesstack{\mathcal{S}\!{\it paces}}
\def\Quotfunctor{{\rm Quot}}
\def\Hilbfunctor{{\rm Hilb}}
\def\Curvesstack{\mathcal{C}\!{\it urves}}
\def\Polarizedstack{\mathcal{P}\!{\it olarized}}
\def\Complexesstack{\mathcal{C}\!{\it omplexes}}
% \Pic is the operator that assigns to X its picard group, usage \Pic(X)
% \Picardstack_{X/B} denotes the Picard stack of X over B
% \Picardfunctor_{X/B} denotes the Picard functor of X over B
\def\Pic{\mathop{\rm Pic}\nolimits}
\def\Picardstack{\mathcal{P}\!{\it ic}}
\def\Picardfunctor{{\rm Pic}}
\def\Deformationcategory{\mathcal{D}\!{\it ef}}


% OK, start here.
%
\begin{document}

\title{Decent Algebraic Spaces}


\maketitle

\phantomsection
\label{section-phantom}

\tableofcontents

\section{Introduction}
\label{section-introduction}

\noindent
In this chapter we talk study the more esoteric aspects of general
algebraic spaces, i.e., those algebraic spaces which aren't quasi-separated.
A reference for quasi-separated algebraic spaces is
\cite{K}.


\section{Conventions}
\label{section-conventions}

\noindent
The standing assumption is that all schemes are contained in
a big fppf site $\textit{Sch}_{fppf}$. And all rings $A$ considered
have the property that $\text{Spec}(A)$ is (isomorphic) to an
object of this big site.

\medskip\noindent
Let $S$ be a scheme and let $X$ be an algebraic space over $S$.
In this chapter and the following we will write $X \times_S X$
for the product of $X$ with itself (in the category of algebraic
spaces over $S$), instead of $X \times X$.



\section{Universally bounded fibres}
\label{section-universally-bounded}

\noindent
We briefly discuss what it means for a morphism from a scheme to an
algebraic space to have universally bounded fibres. Please refer to
Morphisms, Section \ref{morphisms-section-universally-bounded}
for similar definitions and results on morphisms of schemes.

\begin{definition}
\label{definition-universally-bounded}
Let $S$ be a scheme. Let $X$ be an algebraic space over $S$, and
let $U$ be a scheme over $S$. Let $f : U \to X$ be a morphism over $S$.
We say the {\it fibres of $f$ are universally bounded}\footnote{This is
probably nonstandard notation.}
if there exists an integer $n$ such that for all fields
$k$ and all morphisms $\text{Spec}(k) \to X$ the fibre
product $\text{Spec}(k) \times_X U$ is a finite scheme over $k$
whose degree over $k$ is $\leq n$.
\end{definition}

\noindent
This definition makes sense because the fibre product
$\text{Spec}(k) \times_Y X$ is a scheme. Moreover, if $Y$ is a scheme
we recover the notion of
Morphisms, Definition \ref{morphisms-definition-universally-bounded}
by virtue of
Morphisms, Lemma \ref{morphisms-lemma-characterize-universally-bounded}.

\begin{lemma}
\label{lemma-composition-universally-bounded}
Let $S$ be a scheme. Let $X$ be an algebraic space over $S$.
Let $V \to U$ be a morphism of schemes over $S$, and let
$U \to X$ be a morphism from $U$ to $X$. If the fibres of
$V \to U$ and $U \to X$ are universally bounded, then so
are the fibres of $V \to X$.
\end{lemma}

\begin{proof}
Let $n$ be an integer which works for $V \to U$, and let $m$ be
an integer which works for $U \to X$ in
Defintion \ref{definition-universally-bounded}.
Let $\text{Spec}(k) \to X$ be a morphism, where $k$ is a field.
Consider the morphisms
$$
\text{Spec}(k) \times_X V
\longrightarrow
\text{Spec}(k) \times_X U
\longrightarrow
\text{Spec}(k).
$$
By assumption the scheme $\text{Spec}(k) \times_X U$
is finite of degree at most $m$ over $k$, and $n$ is an integer which
bounds the degree of the fibres of the first morphism. Hence by
Morphisms, Lemma \ref{morphisms-lemma-composition-universally-bounded}
we conclude that $\text{Spec}(k) \times_X V$ is finite over $k$
of degree at most $nm$.
\end{proof}

\begin{lemma}
\label{lemma-base-change-universally-bounded}
Let $S$ be a scheme.
Let $Y \to X$ be a representable morphism of algebraic spaces over $S$.
Let $U \to X$ be a morphism from a scheme to $X$.
If the fibres of $U \to X$ are universally bounded, then the fibres
of $U \times_X Y \to Y$ are universally bounded.
\end{lemma}

\begin{proof}
This is clear from the definition, and properties of fibre products.
(Note that $U \times_X Y$ is a scheme
as we assumed $Y \to X$ representable, so the definition applies.)
\end{proof}

\begin{lemma}
\label{lemma-descent-universally-bounded}
Let $S$ be a scheme. Let $g : Y \to X$ be a representable morphism of
algebraic spaces over $S$. Let $f : U \to X$ be a morphism from a scheme
towards $X$. Let $f' : U \times_X Y \to Y$ be the base change of $f$.
If
$$
\text{Im}(|f| : |U| \to |X|) \subset \text{Im}(|g| : |Y| \to |X|)
$$
and $f'$ has universally bounded fibres, then $f$ has universally
bounded fibres.
\end{lemma}

\begin{proof}
Let $n \geq 0$ be an integer bounding the degrees of the fibre
products $\text{Spec}(k) \times_Y (U \times_X Y)$ as in
Definition \ref{definition-universally-bounded} for the morphism $f'$.
We claim that $n$ works for $f$ also. Namely, suppose that
$x : \text{Spec}(k) \to X$ is a morphism from the spectrum of
a field. Then either $\text{Spec}(k) \times_X U$ is empty (and there
is nothing to prove), or $x$ is in the image of $|f|$. By
Properties of Spaces,
Lemma \ref{spaces-properties-lemma-points-cartesian}
and the assumption of the lemma we see
that this means there exists a field extension $k \subset k'$ and a
commutative diagram
$$
\xymatrix{
\text{Spec}(k') \ar[r] \ar[d] & Y \ar[d] \\
\text{Spec}(k) \ar[r] & X
}
$$
Hence we see that
$$
\text{Spec}(k') \times_Y (U \times_X Y) =
\text{Spec}(k') \times_{\text{Spec}(k)} (\text{Spec}(k) \times_X U)
$$
Since the scheme $\text{Spec}(k') \times_Y (U \times_X Y)$ is assumed finite
of degree $\leq n$ over $k'$ it follows that also $\text{Spec}(k) \times_X U$
is finite of degree $\leq n$ over $k$ as desired. (Some details omitted.)
\end{proof}

\begin{lemma}
\label{lemma-universally-bounded-permanence}
Let $S$ be a scheme. Let $X$ be an algebraic space over $S$.
Consider a commutative diagram
$$
\xymatrix{
U \ar[rd]_g \ar[rr]_{f} & & V \ar[ld]^h \\
& X &
}
$$
where $U$ and $V$ are schemes. If $g$ has universally bounded fibres,
and $f$ is surjective and flat, then also $h$ has universally bounded fibres.
\end{lemma}

\begin{proof}
Assume $g$ has universally bounded fibres, and $f$ is surjective and flat.
Say $n \geq 0$. is an integer which bounds the degrees of the schemes
$\text{Spec}(k) \times_X U$ as in
Definition \ref{definition-universally-bounded}.
We claim $n$ also works for $h$.
Let $\text{Spec}(k) \to X$ be a morphism from the spectrum of a
field to $X$. Consider the morphism of schemes
$$
\text{Spec}(k) \times_X V \longrightarrow \text{Spec}(k) \times_X U
$$
It is flat and surjective. By assumption the scheme
on the left is finite of degree $\leq n$ over $\text{Spec}(k)$.
It follows from
Morphisms, Lemma \ref{morphisms-lemma-universally-bounded-permanence}
that the degree of the scheme on the right is also bounded by $n$
as desired.
\end{proof}

\begin{lemma}
\label{lemma-universally-bounded-finite-fibres}
Let $S$ be a scheme.
Let $X$ be an algebraic space over $S$, and let $U$ be a scheme over $S$.
Let $\varphi : U \to X$ be a morphism over $S$.
If the fibres of $\varphi$ are universally bounded, then there exists an
integer $n$ such that each fibre of $|U| \to |X|$ has at most
$n$ elements.
\end{lemma}

\begin{proof}
The integer $n$ of Definition \ref{definition-universally-bounded} works.
Namely, pick $x \in |X|$. Represent $x$ by a morphism
$x : \text{Spec}(k) \to X$. Then we get a commutative diagram
$$
\xymatrix{
\text{Spec}(k) \times_X U \ar[r] \ar[d] & U \ar[d] \\
\text{Spec}(k) \ar[r]^x & X
}
$$
which shows (via
Properties of Spaces,
Lemma \ref{spaces-properties-lemma-points-cartesian})
that the inverse image of $x$ in $|U|$ is the image of
the top horizontal arrow. Since $\text{Spec}(k) \times_X U$ is finite
of degree $\leq n$ over $k$ it has at most $n$ points.
\end{proof}








\section{Finiteness conditions and points}
\label{section-points-monomorphisms}

\noindent
In this section we elaborate on the question of when points can be represented
by monomorphisms from spectra of fields into the space.

\begin{remark}
\label{remark-recall}
Before we give the proof of the next lemma let us recall some facts
about \'etale morphisms of schemes:
\begin{enumerate}
\item An \'etale morphism is flat and hence generalizations lift along
an \'etale morphism
(Morphisms, Lemmas \ref{morphisms-lemma-etale-flat}
and \ref{morphisms-lemma-generalizations-lift-flat}).
\item An \'etale morphism is unramified, an unramified morphism is locally
quasi-finite, hence fibres are discrete
(Morphisms, Lemmas \ref{morphisms-lemma-flat-unramified-etale},
\ref{morphisms-lemma-unramified-quasi-finite}, and
\ref{morphisms-lemma-quasi-finite-at-point-characterize}).
\item A quasi-compact \'etale morphism is quasi-finite and in particular
has finite fibres
(Morphisms, Lemmas \ref{morphisms-lemma-quasi-finite-locally-quasi-compact} and
\ref{morphisms-lemma-quasi-finite}).
\item An \'etale scheme over a field $k$ is a disjoint union of spectra
of finite separable field extension of $k$
(Morphisms, Lemma \ref{morphisms-lemma-etale-over-field}).
\end{enumerate}
For a general discussion of \'etale morphisms, please see
\'Etale Morphisms of Schemes, Section \ref{etale-section-etale-morphisms}.
\end{remark}

\begin{lemma}
\label{lemma-U-finite-above-x}
Let $S$ be a scheme. Let $X$ be an algebraic space over $S$.
Let $x \in |X|$. The following are equivalent:
\begin{enumerate}
\item there exists a family of schemes $U_i$ and
\'etale morphisms $\varphi_i : U_i \to X$ such that
$\coprod \varphi_i : \coprod U_i \to X$ is surjective,
and such that for each $i$ the fibre of
$|U_i| \to |X|$ over $x$ is finite, and
\item for every affine scheme $U$ and \'etale morphism $\varphi : U \to X$
the fibre of $|U| \to |X|$ over $x$ is finite.
\end{enumerate}
\end{lemma}

\begin{proof}
The implication (2) $\Rightarrow$ (1) is trivial.
Let $\varphi_i : U_i \to X$ be a family of \'etale morphisms as in (1).
Let $\varphi : U \to X$ be an \'etale morphism from a scheme
towards $X$. Consider the fibre product diagrams
$$
\xymatrix{
U \times_X U_i \ar[r]_-{p_i} \ar[d]_{q_i} & U_i \ar[d]^{\varphi_i} \\
U \ar[r]^\varphi & X
}
\quad \quad
\xymatrix{
\coprod U \times_X U_i \ar[r]_-{\coprod p_i} \ar[d]_{\coprod q_i} &
\coprod U_i \ar[d]^{\coprod \varphi_i} \\
U \ar[r]^\varphi & X
}
$$
Since $q_i$ is \'etale it is open (see Remark \ref{remark-recall}).
Moreover, the morphism $\coprod q_i$ is surjective.
Hence there exist finitely many indices $i_1, \ldots, i_n$ and
a quasi-compact opens $W_{i_j} \subset U \times_X U_{i_j}$
which surject onto $U$.
The morphism $p_i$ is \'etale, hence locally quasi-finite (see remark on
\'etale morphisms above). Thus we may apply
Morphisms, Lemma
\ref{morphisms-lemma-locally-quasi-finite-qc-source-universally-bounded}
to see the fibres of $p_{i_j}|_{W_{i_j}} : W_{i_j} \to U_i$ are finite.
Hence by
Properties of Spaces,
Lemma \ref{spaces-properties-lemma-points-cartesian}
and the assumption on $\varphi_i$ we conclude that the fibre 
of $\varphi$ over $x$ is finite. In other words (2) holds.
\end{proof}

\begin{lemma}
\label{lemma-U-universally-bounded}
Let $S$ be a scheme. Let $X$ be an algebraic space over $S$.
The following are equivalent:
\begin{enumerate}
\item there exist schemes $U_i$ and \'etale morphisms
$U_i \to X$ such that $\coprod U_i \to X$ is surjective and
each $U_i \to X$ has universally bounded fibres, and
\item for every affine scheme $U$ and \'etale morphism $\varphi : U \to X$
the fibres of $U \to X$ are universally bounded.
\end{enumerate}
\end{lemma}

\begin{proof}
The implication (2) $\Rightarrow$ (1) is trivial.
Assume (1). Let $(\varphi_i : U_i \to X)_{i \in I}$ be a collection of
\'etale morphisms from schemes towards $X$, covering $X$, such that
each $\varphi_i$ has universally bounded fibres.
Let $\psi : U \to X$ be an \'etale morphism from an affine scheme towards $X$.
For each $i$ consider the fibre product diagram
$$
\xymatrix{
U \times_X U_i \ar[r]_{p_i} \ar[d]_{q_i} & U_i \ar[d]^{\varphi_i} \\
U \ar[r]^\psi & X
}
$$
Since $q_i$ is \'etale it is open (see Remark \ref{remark-recall}).
Moreover, we have $U = \bigcup \text{Im}(q_i)$, since the family
$(\varphi_i)_{i \in I}$ is surjective. Since $U$ is affine, hence quasi-compact
we can finite finitely many $i_1, \ldots, i_n \in I$ and quasi-compact
opens $W_j \subset U \times_X U_{i_j}$ such that
$U = \bigcup p_{i_j}(W_j)$.
The morphism $p_{i_j}$ is \'etale, hence locally quasi-finite
(see remark on \'etale morphisms above). Thus we may apply
Morphisms, Lemma
\ref{morphisms-lemma-locally-quasi-finite-qc-source-universally-bounded}
to see the fibres of $p_{i_j}|_{W_j} : W_j \to U_{i_j}$ are universally
bounded. Hence by
Lemma \ref{lemma-composition-universally-bounded}
we see that the fibres of $W_j \to X$ are universally bounded.
Thus also $\coprod_{j = 1, \ldots, n} W_j \to X$ has universally
bounded fibres. Since $\coprod_{j = 1, \ldots, n} W_j \to X$ factors
through the surjective \'etale map
$\coprod q_{i_j}|_{W_j} : \coprod_{j = 1, \ldots, n} W_j \to U$ we
see that the fibres of $U \to X$ are universally bounded by
Lemma \ref{lemma-universally-bounded-permanence}.
In other words (2) holds.
\end{proof}

\begin{lemma}
\label{lemma-R-finite-above-x}
Let $S$ be a scheme. Let $X$ be an algebraic space over $S$.
Let $x \in |X|$. The following are equivalent:
\begin{enumerate}
\item there exists a scheme $U$, an \'etale morphism
$\varphi : U \to X$, and points $u, u' \in U$ mapping to
$x$ such that setting $R = U \times_X U$ the fibre of
$$
|R| \to |U| \times_{|X|} |U|
$$
over $(u, u')$ is finite,
\item for every scheme $U$, \'etale morphism $\varphi : U \to X$ and
any points $u, u' \in U$ mapping to
$x$ setting $R = U \times_X U$ the fibre of
$$
|R| \to |U| \times_{|X|} |U|
$$
over $(u, u')$ is finite,
\item there exists a morphism $\text{Spec}(k) \to X$ with $k$ a field
in the equivalence class of $x$ such that the projections
$\text{Spec}(k) \times_X \text{Spec}(k) \to \text{Spec}(k)$ are
\'etale and quasi-compact, and
\item there exists a monomorphism $\text{Spec}(k) \to X$ with $k$ a field
in the equivalence class of $x$.
\end{enumerate}
\end{lemma}

\begin{proof}
Assume (1), i.e., let $\varphi : U \to X$ be an \'etale morphism from a scheme
towards $X$, and let $u, u'$ be points of $U$ lying over $x$
such that the fibre of $|R| \to |U| \times_{|X|} |U|$ over $(u, u')$
is a finite set. In this proof we think of a point $u = \text{Spec}(\kappa(u))$
as a scheme. Note that $u \to U$, $u' \to U$ are monomorphisms (see
Schemes, Lemma \ref{schemes-lemma-injective-points-surjective-stalks}),
hence $u \times_X u' \to R = U \times_X U$ is a monomorphism.
In this language the assumption really means that
$u \times_X u'$ is a scheme whose underlying topological space has
finitely many points.
Let $\psi : W \to X$ be an \'etale morphism from a scheme towards $X$.
Let $w, w' \in W$ be points of $W$ mapping to $x$.
We have to show that $w \times_X w'$ is a scheme whose underlying topological
space has finitely many points.
Consider the fibre product diagram
$$
\xymatrix{
W \times_X U \ar[r]_p \ar[d]_q & U \ar[d]^\varphi \\
W \ar[r]^\psi & X
}
$$
As $x$ is the image of $u$ and $u'$ we may pick points
$\tilde w, \tilde w'$ in $W \times_X U$ with $q(\tilde w) = w$,
$q(\tilde w') = w'$, $u = p(\tilde w)$ and $u' = p(\tilde w')$, see
Properties of Spaces,
Lemma \ref{spaces-properties-lemma-points-cartesian}.
As $p$, $q$ are \'etale the field extensions
$\kappa(w) \subset \kappa(\tilde w) \supset \kappa(u)$ and
$\kappa(w') \subset \kappa(\tilde w') \supset \kappa(u')$ are
finite separable, see Remark \ref{remark-recall}.
Then we get a commutative diagram
$$
\xymatrix{
w \times_X w' \ar[d] &
\tilde w \times_X \tilde w' \ar[l] \ar[d] \ar[r] &
u \times_X u' \ar[d] \\
w \times_X w' &
\tilde w \times_S \tilde w' \ar[l] \ar[r] &
u \times_S u'
}
$$
where the squares are fibre product squares. The lower horizontal
morpisms are \'etale and quasi-compact, as any scheme of the form
$\text{Spec}(k) \times_S \text{Spec}(k')$ is affine, and by our
observations about the field extensions above.
Thus we see that the top horizontal arrows are \'etale and quasi-compact
and hence have finite fibres.
We have seen above that $|u \times_X u'|$ is finite, so we conclude that
$|w \times_X w'|$ is finite. In other words, (2) holds.

\medskip\noindent
Assume (2). Let $U \to X$ be an \'etale morphism from a scheme $U$
such that $x$ is in the image of $|U| \to |X|$. Let $u \in U$ be
a point mapping to $x$. Then we have seen in the previous
paragraph that $u = \text{Spec}(\kappa(u)) \to X$ has the property that
$u \times_X u$ has a finite underlying topological space. On the other
hand, the projection maps $u \times_X u \to u$ are the composition
$$
u \times_X u \longrightarrow
u \times_X U \longrightarrow
u \times_X X = u,
$$
i.e., the composition of a monomorphism (the base change of the monomorphism
$u \to U$) by an \'etale morphism (the base change of the \'etale morphism
$U \to X$). Hence $u \times_X U$ is a disjoint union of spectra of fields
finite separable over $\kappa(u)$ (see
Remark \ref{remark-recall}). Since $u \times_X u$ is finite the image
of it in $u \times_X U$ is a finite disjoint union of spectra of fields
finite separable over $\kappa(u)$. By
Schemes, Lemma \ref{schemes-lemma-mono-towards-spec-field}
we conclude that $u \times_X u$ is a finite disjoint union of spectra
of fields finite separable over $\kappa(u)$. In other words, we see that
$u \times_X u \to u$ is quasi-compact and \'etale. This means that (3) holds.

\medskip\noindent
Let us prove that (3) implies (4). Let $\text{Spec}(k) \to X$ be a morphism
from the spectrum of a field into $X$, in the equivalence class of $x$
such that the two projections
$t, s : R = \text{Spec}(k) \times_X \text{Spec}(k)  \to \text{Spec}(k)$
are quasi-compact and \'etale.
This means in particular
that $R$ is an \'etale equivalence relation on $\text{Spec}(k)$.
By Spaces, Theorem \ref{spaces-theorem-presentation}
we know that the quotient sheaf
$X' = \text{Spec}(k)/R$ is an algebraic space. By
Groupoids, Lemma \ref{groupoids-lemma-quotient-groupoid-restrict}
the map $X' \to X$ is a monomorphism.
Since $s, t$ are quasi-compact, we see that $R$ is quasi-compact and hence
Properties of Spaces,
Lemma \ref{spaces-properties-lemma-point-like-spaces}
applies to $X'$, and we see that
$X' = \text{Spec}(k')$ for some field $k'$. Hence we get a factorization
$$
\text{Spec}(k) \longrightarrow
\text{Spec}(k') \longrightarrow X
$$
which shows that $\text{Spec}(k') \to X$ is a monomorphism mapping
to $x \in |X|$. In other words (4) holds.

\medskip\noindent
Finally, we prove that (4) implies (1). Let $\text{Spec}(k) \to X$
be a monomorphism with $k$ a field in the equivalence class of $x$.
Let $U \to X$ be a surjectve \'etale morphism from a scheme $U$ to $X$.
Let $u \in U$ be a point over $x$. Since $\text{Spec}(k) \times_X u$
is nonempty, and since $\text{Spec}(k) \times_X u \to u$ is a monomorphism
we conclude that $\text{Spec}(k) \times_X u = u$ (see
Schemes, Lemma \ref{schemes-lemma-mono-towards-spec-field}).
Hence $u \to U \to X$ factors through $\text{Spec}(k) \to X$, here is
a picture
$$
\xymatrix{
u \ar[r] \ar[d] & U \ar[d] \\
\text{Spec}(k) \ar[r] & X
}
$$
Since the right vertical arrow is \'etale this implies that
$k \subset \kappa(u)$ is a finite separable extension. Hence we conclude that
$$
u \times_X u = u \times_{\text{Spec}(k)} u
$$
is a finite scheme, and we win by the discussion of the meaning of property
(1) in the first paragraph of this proof.
\end{proof}

\begin{lemma}
\label{lemma-weak-UR-finite-above-x}
Let $S$ be a scheme. Let $X$ be an algebraic space over $S$.
Let $x \in |X|$.
Let $U$ be a scheme and let $\varphi : U \to X$ be an \'etale morphism.
The following are equivalent:
\begin{enumerate}
\item $x$ is in the image of $|U| \to |X|$, and
setting $R = U \times_X U$ the fibres of both
$$
|U| \longrightarrow |X|
\quad\text{and}\quad
|R| \longrightarrow |X|
$$
over $x$ are finite,
\item there exists a monomorphism $\text{Spec}(k) \to X$ with $k$ a field
in the equivalence class of $x$, and
the fibre product $\text{Spec}(k) \times_X U$ is
a finite nonempty scheme over $k$.
\end{enumerate}
\end{lemma}

\begin{proof}
Assume (1). This clearly implies the first condition of
Lemma \ref{lemma-R-finite-above-x} and hence we obtain a monomorphism
$\text{Spec}(k) \to X$ in the class of $x$. Taking the fibre product
we see that $\text{Spec}(k) \times_X U \to \text{Spec}(k)$ is a scheme
\'etale over $\text{Spec}(k)$ with finitely many points, hence a finite
nonempty scheme over $k$, i.e., (2) holds.

\medskip\noindent
Assume (2). By assumption $x$ is in the image of
$|U| \to |X|$. The finiteness of the fibre of
$|U| \to |X|$ over $x$ is clear since this fibre is equal to
$|\text{Spec}(k) \times_X U|$ by
Properties of Spaces,
Lemma \ref{spaces-properties-lemma-points-cartesian}.
The finiteness of the fibre of $|R| \to |X|$ above $x$ is also clear
since it is equal to the set underlying the scheme
$$
(\text{Spec}(k) \times_X U) \times_{\text{Spec}(k)} (\text{Spec}(k) \times_X U)
$$
which is finite over $k$. Thus (1) holds.
\end{proof}

\begin{lemma}
\label{lemma-UR-finite-above-x}
Let $S$ be a scheme. Let $X$ be an algebraic space over $S$.
Let $x \in |X|$. The following are equivalent:
\begin{enumerate}
\item for every affine scheme $U$, any \'etale morphism
$\varphi : U \to X$ setting $R = U \times_X U$ the fibres of both
$$
|U| \longrightarrow |X|
\quad\text{and}\quad
|R| \longrightarrow |X|
$$
over $x$ are finite,
\item there exists a monomorphism $\text{Spec}(k) \to X$ with $k$ a field
in the equivalence class of $x$, and for any affine scheme $U$ and \'etale
morphism $U \to X$ the fibre product $\text{Spec}(k) \times_X U$ is
a finite scheme over $k$, and
\item there exist schemes $U_i$ and \'etale morphisms
$U_i \to X$ such that $\coprod U_i \to X$ is surjective and for each
$i$, setting $R_i = U_i \times_X U_i$ the fibres of both
$$
|U_i| \longrightarrow |X|
\quad\text{and}\quad
|R_i| \longrightarrow |X|
$$
over $x$ are finite.
\end{enumerate}
\end{lemma}

\begin{proof}
The equivalence of (1) and (2) follows on applying
Lemma \ref{lemma-weak-UR-finite-above-x} to every \'etale morphism
$U \to X$ with $U$ affine. It is clear that (2) implies (3).
Assume $U_i \to X$ and $R_i$ are as in (3). We conclude from
Lemma \ref{lemma-U-finite-above-x}
that for any affine scheme $U$ and \'etale morphism $U \to X$
the fibre of $|U| \to |X|$ over $x$ is finite.
Say this fibre is $\{u_1, \ldots, u_n\}$.
Then, as
Lemma \ref{lemma-R-finite-above-x} (1)
applies to $U_i \to X$ for some $i$ such that $x$ is in the image of
$|U_i| \to |X|$, we see that the fibre of
$|R = U \times_X U| \to |U| \times_{|X|} |U|$
is finite over $(u_a, u_b)$, $a, b \in \{1, \ldots, n\}$.
Hence the fibre of $|R| \to |X|$ over $x$ is finite.
In this way we see that (1) holds.
\end{proof}







\section{Reasonable and decent algebraic spaces}
\label{section-reasonable-decent}

\noindent
The conditions in the following definition
are not exactly conditions on the diagonal of $X$, but they are in some
sense separation conditions on $X$.

\begin{definition}
\label{definition-very-reasonable}
Let $S$ be a scheme.
Let $X$ be an algebraic space over $S$.
\begin{enumerate}
\item We say $X$ is {\it decent} if for every point $x \in X$ the equivalent
conditions of Lemma \ref{lemma-UR-finite-above-x} hold, in other words
property $(\gamma)$ of
Lemma \ref{lemma-bounded-fibres}
holds.
\item We say $X$ is {\it reasonable} if the equivalent conditions of
Lemma \ref{lemma-U-universally-bounded}
hold, in other words property $(\delta)$ of
Lemma \ref{lemma-bounded-fibres}
holds.
\item We say $X$ is {\it very reasonable} if there exists a set of schemes
$U_i$ and morphisms $U_i \to X$ such that
\begin{enumerate}
\item each $U_i \to X$ is \'etale,
\item both projections $U_i \times_X U_i \to U_i$ are
quasi-compact, and
\item the morphism $\coprod U_i \to X$ is surjective (and \'etale).
\end{enumerate}
This property is denoted $(\epsilon)$ in
Lemma \ref{lemma-bounded-fibres}.
\end{enumerate}
\end{definition}

\noindent
The notion of a very reasonable algebraic space was introduced because
the assumption was sufficient to prove some of the results below, especially
Proposition \ref{proposition-very-reasonable-open-dense-scheme}
and
Proposition \ref{proposition-very-reasonable-sober}.
We hope (in the future) to strengthen these results to
the case where the space $X$ is reasonable or even just decent.
Condition (3)(b) means that $U_i \to X$ is quasi-compact onto its image;
see Lemma \ref{lemma-characterize-very-reasonable} and its proof.
In particular, if there exists a scheme $U$ and a surjective, quasi-compact
morphism $U \to X$, then $X$ is very reasonable. Namely, in this case both
projections $U \times_X U \to U$ are quasi-compact.

\begin{lemma}
\label{lemma-characterize-very-reasonable}
Let $S$ be a scheme.
Let $X$ be an algebraic space over $S$.
The following are equivalent:
\begin{enumerate}
\item $X$ is very reasonable, and
\item there exists a Zariski covering $X = \bigcup X_i$ and for
each $i$ a scheme $U_i$ and a quasi-compact surjective \'etale
morphism $U_i \to X_i$.
\end{enumerate}
\end{lemma}

\begin{proof}
If (2) holds then the morphisms $U_i \to X_i \to X$ are \'etale (combine
Morphisms, Lemma \ref{morphisms-lemma-composition-etale}
and
Spaces, Lemmas
\ref{spaces-lemma-morphism-schemes-gives-representable-transformation-property}
and
\ref{spaces-lemma-composition-representable-transformations-property}).
Moreover, as $U_i \times_X U_i = U_i \times_{X_i} U_i$,
both projections $U_i \times_X U_i \to U_i$ are quasi-compact.

\medskip\noindent
If $X$ is very reasonable then there exists a surjective \'etale morphism
$\coprod U_i \to X$, where each $U_i$ is a scheme, such that
the projections $U_i \times_X U_i \to U_i$ are quasi-compact.
Let $X_i \subset X$ be the open subspace corresponding to the image
of the open map $|U_i| \to |X|$ (use
Properties of Spaces,
Lemmas \ref{spaces-properties-lemma-open-subspaces} and
\ref{spaces-properties-lemma-topology-points}).
By
Properties of Spaces,
Lemma \ref{spaces-properties-lemma-factor-through-open-subspace}
we get morphisms $U_i \to X_i$ which are
surjective by
Properties of Spaces,
Lemma \ref{spaces-properties-lemma-characterize-surjective}.
Hence $U_i \to X_i$ is surjective \'etale, and the projections
$U_i \times_{X_i} U_i \to U_i$ are quasi-compact, again because
$U_i \times_{X_i} U_i = U_i \times_X U_i$. Thus by
Spaces, Lemma \ref{spaces-lemma-representable-morphisms-spaces-property}
the morphisms $U_i \to X_i$ are quasi-compact.
\end{proof}

\begin{lemma}
\label{lemma-scheme-very-reasonable}
A scheme is very reasonable.
\end{lemma}

\begin{proof}
This is true because the identity map is a quasi-compact, surjective
\'etale morphism.
\end{proof}

\begin{lemma}
\label{lemma-very-reasonable-Zariski-local}
Let $S$ be a scheme.
Let $X$ be an algebraic space over $S$.
If there exists a Zariski open covering $X = \bigcup X_i$ such that
each $X_i$ is very reasonable, then $X$ is very reasonable.
\end{lemma}

\begin{proof}
Assume there exists a Zariski open covering
$X = \bigcup X_i$, where each $X_i$ is very reasonable.
Then we can find sets $J_i$ and morphisms
$\varphi_{ij} : U_{ij} \to X_i$ such that each $\varphi_{ij}$
is \'etale, both projections $U_{ij} \times_{X_i} U_{ij} \to U_{ij}$
are quasi-compact, and $\coprod_{j \in J_i} U_{ij} \to X_i$ is surjective.
In this case the compositions $U_{ij} \to X_i \to X$ are \'etale
(combine
Morphisms, Lemmas
\ref{morphisms-lemma-composition-etale}
and
\ref{morphisms-lemma-open-immersion-etale}
and
Spaces, Lemmas
\ref{spaces-lemma-morphism-schemes-gives-representable-transformation-property}
and
\ref{spaces-lemma-composition-representable-transformations-property}).
Since $X_i \subset X$ is a subspace we see that
$U_{ij} \times_{X_i} U_{ij} = U_{ij} \times_X U_{ij}$, and hence the
condition on fibre products is preserved. And clearly
$\coprod_{i, j} U_{ij} \to X$ is surjective. Hence $X$ is very reasonable.
\end{proof}

\begin{lemma}
\label{lemma-quasi-separated-very-reasonable}
An algebraic space which is Zariski locally quasi-separated is very reasonable.
In particular any quasi-separated algebraic space is very reasonable.
\end{lemma}

\begin{proof}
By Lemma \ref{lemma-very-reasonable-Zariski-local}
it suffices to show that a quasi-separated algebraic space is very reasonable.

\medskip\noindent
Let $S$ be a scheme, and let $X$ be a quasi-separated algebraic space
over $S$. Let $U$ be a scheme and let $U \to X$ be a surjective \'etale
morphism. Let $U = \bigcup U_i$ be an affine open covering of $U$.
Each of the morphisms $U_i \to X$ is \'etale (combine
Morphisms, Lemma \ref{morphisms-lemma-composition-etale}
and
Spaces, Lemmas
\ref{spaces-lemma-morphism-schemes-gives-representable-transformation-property}
and
\ref{spaces-lemma-composition-representable-transformations-property}).
Hence $\coprod U_i \to X$ is surjective \'etale.
To finish the proof we show that $U_i \to X$ is quasi-compact, which
in particular implies that both projections $U_i \times_X U_i \to U_i$ are
quasi-compact. To do this we may by
Spaces, Lemma \ref{spaces-lemma-viewed-as-properties}
view $X$ as an algebraic space over $\text{Spec}(\mathbf{Z})$.
In other words, in the rest of the proof we may assume that
$S = \text{Spec}(\mathbf{Z})$.

\medskip\noindent
Let $T \to X$ be a morphism from a scheme to $X$. Then
$$
T \times_X U_i
=
(T \times_{\text{Spec}(\mathbf{Z})} U_i)
\times_{X, \Delta}
(X \times_{\text{Spec}(\mathbf{Z})} X)
$$
and hence $T \times_X U_i \to T$ is the composition
$$
(T \times_{\text{Spec}(\mathbf{Z})} U_i)
\times_{X, \Delta}
(X \times_{\text{Spec}(\mathbf{Z})} X)
\longrightarrow
T \times_{\text{Spec}(\mathbf{Z})} U_i
\longrightarrow T
$$
The first arrow is quasi-compact by our assumption that $X$ is
(absolutely) quasi-separated, and the second is quasi-compact because
it is affine (since $U_i$ was chosen to be affine, and
Morphisms, Lemma \ref{morphisms-lemma-base-change-affine}).
\end{proof}

\begin{lemma}
\label{lemma-representable-very-reasonable}
Let $S$ be a scheme.
Let $X$, $Y$ be algebraic spaces over $S$.
Let $Y \to X$ be a representable morphism.
If $X$ is very reasonable, so is $Y$.
\end{lemma}

\begin{proof}
We will repeatedlty use
Spaces, Lemma
\ref{spaces-lemma-base-change-representable-transformations-property}.
Let $U_i \to X$ be as in Definition \ref{definition-very-reasonable}.
Set $V_i = Y \times_X U_i$. The morphisms $V_i \to Y$ are \'etale,
and $\coprod V_i \to Y$ is surjective. Because
$V_i \times_Y V_i = Y \times_X (U_i \times_X U_i)$ we see
that the projections $V_i \times_Y V_i \to V_i$ are
base changes of the projections $U_i \times_X U_i \to U_i$, and so
quasi-compact as well. Hence $Y$ is very reasonable.
\end{proof}

\begin{remark}
\label{remark-very-reasonable-Zariski-locally-quasi-separated}
Very reasonable algebraic spaces form a stricly larger collection than
Zariski locally quasi-separated algebraic spaces. Consider
an algebraic space of the form $X = [U/G]$ (see
Spaces, Definition \ref{spaces-definition-quotient})
where $G$ is a finite group acting without fixed points on a
non-quasi-separated scheme $U$. Namely, in this case
$U \times_X U = U \times G$ and clearly both projections to $U$ are
quasi-compact, hence $X$ is very reasonable. On the other hand, the diagonal
$U \times_X U \to U \times U$ is not quasi-compact, hence this
algebraic space is not quasi-separated. Now, take $U$ the infinite
affine space over a field $k$ of characteristic $\not = 2$ with
zero doubled, see
Schemes, Example \ref{schemes-example-not-quasi-separated}.
Let $0_1, 0_2$ be the two zeros of $U$. Let $G = \{+1, -1\}$, and
let $-1$ act by $-1$ on all coordinates, and by switching
$0_1$ and $0_2$. Then $[U/G]$ is very reasonable but not Zariski locally
quasi-separated (details omitted).
\end{remark}

\begin{example}
\label{example-not-very-reasonable}
The algebraic space $[\mathbf{A}^1_{\mathbf{Q}}/\mathbf{Z}]$ constructed in
Spaces, Example \ref{spaces-example-affine-line-translation}
is not very reasonable.
\end{example}

\begin{remark}
\label{remark-reasonable}
Let $S$ be a scheme and let $X$ be an algebraic space over $S$.
Suppose that for any affine scheme $U$ and \'etale morphism
$\varphi : U \to X$ the fibres of $\varphi$ are universally bounded. In
Definition \ref{definition-very-reasonable}
we called such an algebraic space {\it reasonable}. Reasonable spaces are
technically easier to work with than very reasonable algebraic spaces
(this has to do with descent of the property; see
Remark \ref{remark-very-reasonable}).
On the other hand, we do not know whether a reasonable algebraic
space has an open dense subspace which is a scheme, and we also do not know
whether its underlying topological space is sober, whereas we do know that
very reasonable spaces have those properties (see
Proposition \ref{proposition-very-reasonable-open-dense-scheme}
and
Proposition
\ref{proposition-very-reasonable-sober}).
\end{remark}

\begin{remark}
\label{remark-fun-property-reasonable}
This is a continuation of Remark \ref{remark-reasonable}.
Observation: A reasonable space is a colimit of quasi-separated
algebraic spaces. We sketch the proof in the case $X = U/R$ with $U$ affine.
In this case, reasonable means $U \to X$ is universally bounded.
Hence there exists an integer $N$ such that the ``fibres'' of $U \to X$
have degree at most $N$, see
Definition \ref{definition-universally-bounded}.
Denote $s, t : R \to U$ and $c : R \times_{s, U, t} R \to R$ the
groupoid structural maps.
We claim that for every quasi-compact open $A \subset R$ there exists
an open $R' \subset R$ such that
\begin{enumerate}
\item $A \subset R'$,
\item $R'$ is quasi-compact, and
\item $(U, R', s|_{R'}, t|_{R'}, c|_{R' \times_{s, U, t} R'})$ is
a groupoid scheme.
\end{enumerate}
Note that $e : U \to R$ is open as it is a section of the \'etale morphism
$s : R \to U$, see
\'Etale Morphisms of Schemes,
Proposition \ref{etale-proposition-properties-sections}. Moreover
$U$ is affine hence quasi-compact. Hence we may replace $A$ by
$A \cup e(U) \subset R$, and assume that $A$ contains $e(U)$. Next, we
define inductively $A^1 = A$, and
$$
A^n = c(A^{n - 1} \times_{s, U, t} A) \subset R
$$
for $n \geq 2$. Arguing inductively, we see that $A^n$ is quasi-compact for
all $n \geq 2$, as the image of the quasi-compact fibre product
$A^{n - 1} \times_{s, U, t} A$. If $k$ is an algebraically
closed field over $S$, and we consider $k$-points then
$$
A^n(k) = \left\{(u, u') \in U(k)
:
\begin{matrix}
\text{there exist } u = u_1, u_2, \ldots, u_n \in U(k)\text{ with} \\
(u_i , u_{i + 1}) \in A \text{ for all }i = 1, \ldots, n - 1.
\end{matrix}
\right\}
$$
But as the fibres of $U(k) \to X(k)$ have size at most $N$ we see that if
$n > N$ then we get a repeat in the sequence above, and we can shorten it
proving $A^N = A^n$ for all $n \geq N$.
This implies that $R' = A^N$ gives a groupoid scheme
$(U, R', s|_{R'}, t|_{R'}, c|_{R' \times_{s, U, t} R'})$, proving the claim
above. Consider the map of sheaves on $(\textit{Sch}/S)_{fppf}$
$$
\text{colim}_{R' \subset R}\ U/R' \longrightarrow U/R
$$
where $R' \subset R$ runs over the quasi-compact open subschemes
of $R$ which give \'etale equivalence relations as above. Each of the
quotients $U/R'$ is an algebraic space
(see Spaces, Theorem \ref{spaces-theorem-presentation}).
Since $R'$ is quasi-compact, and $U$ affine the morphism
$R' \to U \times_{\text{Spec}(\mathbf{Z})} U$ is quasi-compact,
and hence $U/R'$ is quasi-compact. Finally, if $T$ is a quasi-compact
scheme, then
$$
\text{colim}_{R' \subset R}\ U(T)/R'(T) \longrightarrow U(T)/R(T)
$$
is a bijection, since every morphism from $T$ into $R$ ends up in one
of the open subrelations $R'$ by the claim above. This clearly implies
that the colimit of the sheaves $U/R'$ is $U/R$. In other words
the algebraic space $X = U/R$ is the colimit of the quasi-separated
algebraic spaces $U/R'$.
\end{remark}





\section{Conditions on algebraic spaces}
\label{section-conditions}

\noindent
In this section we discuss the relationship between various natural
conditions on algebraic spaces we have seen above.

\begin{lemma}
\label{lemma-bounded-fibres}
Let $S$ be a scheme. Let $X$ be an algebraic space over $S$.
Consider the following conditions on $X$:
\begin{enumerate}
\item[$(\alpha)$] For every $x \in |X|$, the equivalent conditions of
Lemma \ref{lemma-U-finite-above-x} hold.
\item[$(\beta)$] The map
$$
\{\text{Spec}(k) \to X \text{ monomorphism}\}
\longrightarrow
|X|
$$
is bijective, i.e., for every $x \in |X|$ the equivalent conditions of
Lemma \ref{lemma-R-finite-above-x} hold.
\item[$(\gamma)$] For every $x \in |X|$, the equivalent conditions of
Lemma \ref{lemma-UR-finite-above-x} hold, in other words $X$ is decent.
\item[$(\delta)$] The equivalent conditions of
Lemma \ref{lemma-U-universally-bounded}
hold, in other words $X$ is reasonable.
\item[$(\epsilon)$] The space $X$ is very reasonable.
\item[$(\zeta)$] The space $X$ is quasi-separated.
\item[$(\eta)$] The space $X$ is representable, i.e., $X$ is a scheme.
\item[$(\theta)$] The space $X$ is a quasi-separated scheme.
\end{enumerate}
We have
$$
\xymatrix{
& (\eta) \ar@{=>}[rd] & & & &  \\
(\theta) \ar@{=>}[ru] \ar@{=>}[rd] & & 
(\epsilon) \ar@{=>}[r] &
(\delta) \ar@{=>}[r] &
(\gamma) \ar@{<=>}[r] & (\alpha) + (\beta) \\
& (\zeta) \ar@{=>}[ru] & & & & 
}
$$
\end{lemma}

\begin{proof}
The implication $(\gamma) \Leftrightarrow (\alpha) + (\beta)$ is immediate.
The implications in the diamond on the left we have seen in
Section \ref{section-reasonable-decent}.

\medskip\noindent
Assume $(\delta)$. Let $U$ be an affine scheme, and let $U \to X$ be an \'etale
morphism. By assumption the fibres of the morphism $U \to X$ are universally
bounded. Thus also the fibres of both projections $R = U \times_X U \to U$
are universally bounded, see
Lemma \ref{lemma-base-change-universally-bounded}.
And by
Lemma \ref{lemma-composition-universally-bounded}
also the fibres of $R \to X$ are universally bounded.
Hence for any $x \in X$ the fibres of $|U| \to |X|$ and $|R| \to |X|$
over $x$ are finite, see
Lemma \ref{lemma-universally-bounded-finite-fibres}.
In other words, the equivalent conditions of
Lemma \ref{lemma-UR-finite-above-x}
hold. This proves that $(\delta) \Rightarrow (\gamma)$.

\medskip\noindent
Let us show that $(\epsilon)$ implies $(\delta)$.
Assume $(\epsilon)$. By
Lemma \ref{lemma-characterize-very-reasonable}
there exists
a Zariski open covering $X = \bigcup X_i$ such that for each $i$
there exists a scheme $U_i$ and a quasi-compact surjective \'etale morphism
$U_i \to X_i$. Choose an $i$ and an affine open subscheme $W \subset U_i$.
It suffices to show that $W \to X$ has universally bounded fibres, since then
the family of all these morphisms $W \to X$ covers $X$.
To do this we consider the diagram
$$
\xymatrix{
W \times_X U_i \ar[r]_-p \ar[d]_q & U_i \ar[d] \\
W \ar[r] & X
}
$$
Since $W \to X$ factors through $X_i$ we see that
$W \times_X U_i = W \times_{X_i} U_i$, and hence $q$ is quasi-compact.
Since $W$ is affine this implies that the scheme $W \times_X U_i$
is quasi-compact. Thus we may apply
Morphisms, Lemma
\ref{morphisms-lemma-locally-quasi-finite-qc-source-universally-bounded}
and we conclude that $p$ has universally bounded fibres.
We may apply
Lemma \ref{lemma-descent-universally-bounded}
to conclude that $W \to X$ has universally bounded fibres as well.
\end{proof}

\begin{lemma}
\label{lemma-properties-local}
Let $S$ be a scheme.
Let $\mathcal{P}$ be one of the properties
$(\alpha)$, $(\beta)$, $(\gamma)$, $(\delta)$, $(\epsilon)$, or
$(\eta)$ of algebraic spaces
listed in Lemma \ref{lemma-bounded-fibres}.
Then if $X$ is an algebraic space over $S$, and $X = \bigcup X_i$ is a
Zariski open covering such that each $X_i$ has $\mathcal{P}$,
then $X$ has $\mathcal{P}$.
\end{lemma}

\begin{proof}
Let $X$ be an algebraic space over $S$, and let $X = \bigcup X_i$ is a
Zariski open covering such that each $X_i$ has $\mathcal{P}$.
The condition $(\alpha)$ for $X_i$ can be formulated as the condition
that for every $x \in |X_i|$ and every affine scheme $U$, and \'etale morphism
$\varphi : U \to X_i$ the fibre of $\varphi : |U| \to |X_i|$
over $x$ are finite. Consider $x \in X$, an affine scheme $U$ and
an \'etale morphism $U \to X$. Since $X = \bigcup X_i$ is a
Zariski open covering there exits a finite affine open covering
$U = U_1 \cup \ldots \cup U_n$ such that each $U_j \to X$ factors through
some $X_{i_j}$. By assumption the fibres of $|U_j | \to |X_{i_j}|$
over $x$ are finite for $j = 1, \ldots, n$. Clearly this means that
the fibre of $|U| \to |X|$ over $x$ is finite.
This proves the result for $(\alpha)$.

\medskip\noindent
Note that $(\gamma) = (\alpha) + (\beta)$ by
Lemma \ref{lemma-bounded-fibres}
hence if we prove the lemma for $(\beta)$ then the lemma follows for
$(\gamma)$.

\medskip\noindent
The lemma for $(\beta)$ is immediate from the definition as $X_i \to X$ is
a monomorphism. The lemma for property $(\delta)$ is clear also since given
schemes $U_{ij}$ and \'etale morphisms $U_{ij} \to X_i$ with universally
bounded fibres which cover $X_i$, then these schemes also given an
\'etale surjective morphism $\coprod U_{ij} \to X$ and $U_{ij} \to X$
still has universally bounded fibres. For $(\epsilon)$, see
Lemma \ref{lemma-very-reasonable-Zariski-local}.
For $(\eta)$, see
Properties of Spaces,
Lemma \ref{spaces-properties-lemma-subscheme}.
\end{proof}

\begin{lemma}
\label{lemma-representable-properties}
Let $S$ be a scheme.
Let $\mathcal{P}$ be one of the properties
$(\beta)$, $(\gamma)$, $(\delta)$, $(\epsilon)$, or $(\eta)$
of algebraic spaces listed in Lemma \ref{lemma-bounded-fibres}.
Let $X$, $Y$ be algebraic spaces over $S$.
Let $X \to Y$ be a representable morphism.
If $Y$ has property $\mathcal{P}$, so does $X$.
\end{lemma}

\begin{proof}
Assume $f : X \to Y$ is a representable morphism of algebraic spaces,
and assume that $Y$ has $\mathcal{P}$. Let $x \in |X|$, and set
$y = f(x) \in |Y|$.

\medskip\noindent
If $\mathcal{P}$ is $(\beta)$, then there exists a monomorphism
$\text{Spec}(k) \to Y$ representing $y$. The fibre product
$X_y = \text{Spec}(k) \times_Y X$ is a scheme, and $x$ corresponds
to a point of $X_y$, i.e., to a monomorphism $\text{Spec}(k') \to X_y$.
As $X_y \to X$ is a monomorphism also we see that $x$ is represented
by the monomorphism $\text{Spec}(k') \to X_y \to X$. In other words
$(\beta)$ holds for $X$.

\medskip\noindent
Suppose $\mathcal{P}$ is $(\gamma)$. Since $(\gamma) \Rightarrow (\beta)$
we have seen in the preceding paragraph that $y$ and $x$ can be represented
by monomorphisms as in the following diagram
$$
\xymatrix{
\text{Spec}(k') \ar[r]_-x \ar[d] & X \ar[d] \\
\text{Spec}(k) \ar[r]^-y & Y
}
$$
Also, by definition of property $(\gamma)$ via
Lemma \ref{lemma-UR-finite-above-x} (3)
there exist schemes
$V_i$ and \'etale morphisms $V_i \to Y$ such that $\coprod V_i \to Y$
is surjective and for each $i$, setting $R_i = V_i \times_Y V_i$
the fibres of both
$$
|V_i| \longrightarrow |Y|
\quad\text{and}\quad
|R_i| \longrightarrow |Y|
$$
over $y$ are finite. This means that the schemes
$(V_i)_y$ and $(R_i)_y$ are finite schemes over $y = \text{Spec}(k)$.
As $X \to Y$ is representable, the fibre products $U_i = V_i \times_Y X$
are schemes. The morphisms $U_i \to X$ are \'etale, and
$\coprod U_i \to X$ is surjective. Finally, for each $i$ we have
$$
(U_i)_x =
(V_i \times_Y X)_x =
(V_i)_y \times_{\text{Spec}(k)} \text{Spec}(k')
$$
and
$$
(U_i \times_X U_i)_x =
\left((V_i \times_Y X) \times_X (V_i \times_Y X)\right)_x =
(R_i)_y \times_{\text{Spec}(k)} \text{Spec}(k')
$$
hence these are finite over $k'$ as base changes of the finite
schemes $(V_i)_y$ and $(R_i)_y$. This implies that $(\gamma)$ holds for $X$,
again via the third condition of
Lemma \ref{lemma-UR-finite-above-x}.

\medskip\noindent
Suppose $\mathcal{P}$ is $(\delta)$. Let $V \to Y$ be an \'etale morphism with
$V$ an affine scheme. Since $Y$ has property $(\delta)$ this morphism has
universally bounded fibres. By
Lemma \ref{lemma-base-change-universally-bounded}
the base change $V \times_Y X \to X$ also has universally bounded fibres.
Hence the first part of
Lemma \ref{lemma-U-universally-bounded}
applies and we see that $Y$ also has property $(\delta)$.

\medskip\noindent
In case $\mathcal{P} = (\epsilon)$ the result is
Lemma \ref{lemma-representable-very-reasonable}.
In case $\mathcal{P} = (\eta)$ the result is
Categories, Lemma \ref{categories-lemma-representable-over-representable}.
\end{proof}






\section{Points and specializations}
\label{section-specializations}

\begin{lemma}
\label{lemma-no-specializations-map-to-same-point}
Let $S$ be a scheme.
Let $X$ be an algebraic space over $S$.
Let $U \to X$ be an \'etale morphism from a scheme to $X$.
Assume $u, u' \in |U|$ map to the same point $x$ of $|X|$, and
$u' \leadsto u$. If the pair $(X, x)$ satisfies the
equivalent conditions of
Lemma \ref{lemma-U-finite-above-x}
then $u = u'$.
\end{lemma}

\begin{proof}
Assume the pair $(X, x)$ satisfies the
equivalent conditions for Lemma \ref{lemma-U-finite-above-x}.
Let $U$ be a scheme, $U \to X$ \'etale, and
let $u, u' \in |U|$ map to $x$ of $|X|$, and
$u' \leadsto u$. We may and do replace $U$ by an affine
neighbourhood of $u$. Let $t, s : R = U \times_X U \to U$
be the \'etale projection maps.

\medskip\noindent
We finish the proof as follows.
Pick a point $r \in R$ with $t(r) = u$ and $s(r) = u'$.
This is possible by
Properties of Spaces,
Lemma \ref{spaces-properties-lemma-points-presentation}.
Because generalizations lift along the \'etale morphism $t$
(Remark \ref{remark-recall}) we can find a specialization $r' \leadsto r$ with
$t(r') = u'$. Set $u'' = s(r')$. Then $u'' \leadsto u'$.
Thus we may repeat and find $r'' \leadsto r'$ with
$t(r'') = u''$. Set $u''' = s(r'')$, and so on.
Here is a picture:
$$
\xymatrix{
& r'' \ar[rd]^s \ar[ld]_t \ar@{~>}[d] & \\
u'' \ar@{~>}[d] & r' \ar[rd]^s \ar[ld]_t \ar@{~>}[d] & u''' \ar@{~>}[d] \\
u' \ar@{~>}[d] & r \ar[rd]^s \ar[ld]_t & u'' \ar@{~>}[d] \\
u & & u'
}
$$
In Remark \ref{remark-recall} we have seen that there are no specializations
among points in the fibres of the \'etale morphism $s$. Hence if
$u^{(n + 1)} = u^{(n)}$ for some $n$, then also $r^{(n)} = r^{(n - 1)}$ and
hence also (by taking $t$) $u^{(n)} = u^{(n - 1)}$. This then forces the
whole tower to collapse, in particular $u = u'$. Thus we see that if
$u \not = u'$, then all the specializations are strict and
$\{u, u', u'', \ldots\}$ is an infinite set of points in $U$ which map to the
point $x$ in $|X|$. As we chose $U$ affine this contradicts the second part of
Lemma \ref{lemma-U-finite-above-x}, as desired.
\end{proof}

\begin{lemma}
\label{lemma-specialization}
Let $S$ be an algebraic space.
Let $X$ be an algebraic space over $S$.
Let $x, x' \in |X|$ and assume $x' \leadsto x$, i.e., $x$ is a
specialization of $x'$.
Assume the pair $(X, x')$ satisfies the equivalent conditions
of Lemma \ref{lemma-UR-finite-above-x}. Then
for every \'etale morphism $\varphi : U \to X$ from a scheme $U$ and any
$u \in U$ with $\varphi(u) = x$, exists a point $u'\in U$,
$u' \leadsto u$ with $\varphi(u') = x'$.
\end{lemma}

\begin{proof}
We may replace $U$ by an affine open neighbourhood of $u$.
Hence we may assume that $U$ is affine. As $x$ is in the
image of the open map $|U| \to |X|$, so is $x'$. Thus we may
replace $X$ by the Zariski open subspace corresponding to
the image of $|U| \to |X|$, see
Properties of Spaces,
Lemma \ref{spaces-properties-lemma-open-subspaces}.
In other words we may assume that
$U \to X$ is surjective and \'etale.
Let $s, t : R = U \times_X U \to U$ be the projections.
By our assumption that $(X, x')$ satisfies the equivalent conditions
of Lemma \ref{lemma-UR-finite-above-x} we see that the fibres
of $|U| \to |X|$ and $|R| \to |X|$
over $x'$ are finite. Say $\{u'_1, \ldots, u'_n\} \subset U$ and
$\{r'_1, \ldots, r'_m\} \subset R$ form the complete inverse image
of $\{x'\}$.
Consider the closed sets
$$
T = \overline{\{u'_1\}} \cup \ldots \cup \overline{\{u'_n\}} \subset |U|,
\quad
T' = \overline{\{r'_1\}} \cup \ldots \cup \overline{\{r'_m\}} \subset |R|.
$$
Trivially we have $s(T') \subset T$. Because $R$ is an equivalence
relation we also have $t(T') = s(T')$ as the set $\{r_j'\}$
is invariant under the inverse of $R$ by construction. Let $w \in T$
be any point. Then $u'_i \leadsto w$ for some $i$. Choose $r \in R$
with $s(r) = w$. Since generalizations lift along $s : R \to U$, see
Remark \ref{remark-recall}, we can find $r' \leadsto r$ with
$s(r') = u_i'$. Then $r' = r'_j$ for some $j$ and we conclude that
$w \in s(T')$. Hence $T = s(T') = t(T')$ is an $|R|$-invariant closed
set in $|U|$. This means $T$ is the inverse image of a closed (!)
subset $T'' = \varphi(T)$ of $|X|$, see
Properties of Spaces,
Lemmas \ref{spaces-properties-lemma-points-presentation} and
\ref{spaces-properties-lemma-topology-points}.
Hence $T'' = \overline{\{x'\}}$.
Thus $T$ contains some point $u_1$ mapping to $x$ as $x \in T''$.
I.e., we see that for some $i$ there exists a specialization
$u'_i \leadsto u_1$ which maps to the given specialization
$x' \leadsto x$.

\medskip\noindent
To finish the proof, choose a point $r \in R$ such that
$s(r) = u$ and $t(r) = u_1$ (using
Properties of Spaces,
Lemma \ref{spaces-properties-lemma-points-cartesian}).
As generalizations lift along $t$, and $u'_i \leadsto u_1$
we can find a specialization $r' \leadsto r$ such that $t(r') = u'_i$.
Set $u' = s(r')$. Then $u' \leadsto u$ and $\varphi(u') = x'$ as
desired.
\end{proof}

\begin{lemma}
\label{lemma-kolmogorov}
Let $S$ be a scheme.
Let $X$ be an algebraic space over $S$.
Assume that for every $x \in |X|$ the equivalent conditions
of Lemma \ref{lemma-UR-finite-above-x} hold, i.e., $X$ is decent.
Then $|X|$ is Kolmogorov (see
Topology, Definition \ref{topology-definition-generic-point}).
\end{lemma}

\begin{proof}
Let $x_1, x_2 \in |X|$ with $x_1 \leadsto x_2$ and $x_2 \leadsto x_1$.
We have to show that $x_1 = x_2$. Pick a scheme $U$ and an \'etale morphism
$U \to X$ such that $x_1, x_2$ are both in the image of $|U| \to |X|$.
By Lemma \ref{lemma-specialization} we can find a specialization
$u_1 \leadsto u_2$ in $U$ mapping to $x_1 \leadsto x_2$.
By Lemma \ref{lemma-specialization} we can find
$u_2' \leadsto u_1$ mapping to $x_2 \leadsto x_1$. This means that
$u_2' \leadsto u_2$ is a specialization between points of $U$ mapping to
the same point of $X$, namely $x_2$. This is not possible, unless
$u_2' = u_2$, see
Lemma \ref{lemma-no-specializations-map-to-same-point}. Hence
also $u_1 = u_2$ as desired.
\end{proof}







\section{Schematic locus}
\label{section-schematic}


\begin{lemma}
\label{lemma-very-reasonable-quasi-compact-pieces}
Let $S$ be a scheme.
Let $X$ be a very reasonable algebraic space over $S$.
There exists a set of schemes
$U_i$ and morphisms $U_i \to X$ such that
\begin{enumerate}
\item each $U_i$ is a quasi-compact scheme,
\item each $U_i \to X$ is \'etale,
\item both projections $U_i \times_X U_i \to U_i$ are quasi-compact, and
\item the morphism $\coprod U_i \to X$ is surjective (and \'etale).
\end{enumerate}
\end{lemma}

\begin{proof}
Definition \ref{definition-very-reasonable}
says that there exist $U_i \to X$ such that (2), (3) and (4) hold.
Fix $i$, and set $R_i = U_i \times_X U_i$, and denote $s, t : R_i \to U_i$
the projections.
For any affine open $W \subset U_i$ the open $W' = t(s^{-1}(W)) \subset U_i$
is a quasi-compact $R_i$-invariant open (see
Groupoids, Lemma \ref{groupoids-lemma-constructing-invariant-opens}).
Hence $W'$ is a quasi-compact scheme, $W' \to X$ is \'etale, and
$W' \times_X W' = s^{-1}(W') = t^{-1}(W')$ so both projections
$W' \times_X W' \to W'$ are quasi-compact. This means the family of
$W' \to X$, where $W \subset U_i$ runs through the members of affine
open coverings of the $U_i$ gives what we want.
\end{proof}

\begin{proposition}
\label{proposition-very-reasonable-open-dense-scheme}
Let $S$ be a scheme.
Let $X$ be an algebraic space over $S$.
If $X$ is very reasonable, then there exists a dense open subspace
of $X$ which is a scheme.
\end{proposition}

\begin{proof}
By
Properties of Spaces,
Lemma \ref{spaces-properties-lemma-subscheme}
and
Lemma \ref{lemma-characterize-very-reasonable}
we may assume that there exists a scheme $U$ and a
surjective quasi-compact, \'etale morphism $U \to X$.
Set $R = U \times_X U$, and denote $s, t : R \to U$ the projections
as usual. Note that $s, t$ are surjective, quasi-compact and \'etale, hence
also quasi-finite (see
\'Etale Morphisms of Schemes, Section \ref{etale-section-etale-morphisms}).
By
More on Morphisms,
Lemma \ref{more-morphisms-lemma-quasi-finite-finite-over-dense-open}
there exists a dense open subscheme $W \subset U$ such that
$s^{-1}(W) \to W$ is finite. By
Descent, Lemma \ref{descent-lemma-descending-property-finite}
being finite is fpqc (and in particular \'etale) local on the target.
Hence we may apply
More on Groupoids, Lemma \ref{more-groupoids-lemma-property-invariant}
which says that the largest open $W \subset U$ over which $s$ is
finite is $R$-invariant. It is still dense of course.
The restriction $R_W$ of $R$ to $W$ equals $R_W = s^{-1}(W) = t^{-1}(W)$
(see Groupoids, Definition \ref{groupoids-definition-invariant-open}
and discussion following it).
By construction $s_W, t_W : R_W \to W$ are finite \'etale.
If we can show the open subspace $W/R_W \subset X$ (see
Spaces, Lemma \ref{spaces-lemma-finding-opens})
contains a dense open subspace which is a scheme, then the
proposition follows for $X$. This reduces us to
Properties of Spaces,
Lemma \ref{spaces-properties-lemma-finite-etale-cover-dense-open-scheme}.
\end{proof}









\section{Points on very reasonable spaces}
\label{section-points-very-reasonable}

\noindent
In this section we prove some properties of points on
very reasonable algebraic spaces.

\begin{lemma}
\label{lemma-very-reasonable-points-monomorphism}
Let $S$ be a scheme. Let $X$ be an algebraic space over $S$.
Consider the map
$$
\{\text{Spec}(k) \to X \text{ monomorphism}\}
\longrightarrow
|X|
$$
This map is always injective. If $X$ is very reasonable then this map
is a bijection.
\end{lemma}

\begin{proof}
We have seen in
Properties of Spaces,
Lemma \ref{spaces-properties-lemma-points-monomorphism}
that the map is an injection in general.
By Lemma \ref{lemma-bounded-fibres} it is surjective when $X$ is
very reasonable.
\end{proof}

\noindent
The following lemma is a tiny bit stronger than
Properties of Spaces,
Lemma \ref{spaces-properties-lemma-point-like-spaces}.
We will improve this lemma in Lemma \ref{lemma-when-field}.

\begin{lemma}
\label{lemma-very-reasonable-point-like-spaces}
Let $S$ be a scheme. Let $k$ be a field.
Let $X$ be an algebraic space over $S$ and assume that there exists
a surjective \'etale morphism $\text{Spec}(k) \to X$.
If $X$ is very reasonable, then $X \cong \text{Spec}(k')$
where $k' \subset k$ is a finite separable extension.
\end{lemma}

\begin{proof}
This can be proved directly by adding a few words to the proof of
Properties of Spaces,
Lemma \ref{spaces-properties-lemma-point-like-spaces},
but we think it is fun to deduce it from the results obtained so far.
By Lemma \ref{lemma-very-reasonable-points-monomorphism}
we see that $\text{Spec}(k) \to X$ factors as
$\text{Spec}(k) \to \text{Spec}(k') \to X$ where
$\text{Spec}(k') \to X$ is a monomorphism. But since $\text{Spec}(k) \to X$
is a surjection of sheaves on $(\textit{Sch}/S)_{fppf}$, we see that
also $\text{Spec}(k') \to X$ is surjective (as a map of sheaves). But a
map of sheaves which is both injective and surjective is an isomorphism.
Finally, the fact that $\text{Spec}(k) \to X$ is \'etale means that
$k \otimes_{k'} k$ is \'etale over $k$, which implies easily that
$k' \subset k$ is a finite separable extension.
\end{proof}

\noindent
The following lemma shows that specialization of points behaves in a
reasonable manner on very reasonable algebraic spaces.
Spaces, Example \ref{spaces-example-infinite-product}
shows that this is {\bf not} true in general.

\begin{lemma}
\label{lemma-very-reasonable-no-specializations-map-to-same-point}
Let $S$ be a scheme.
Let $X$ be a very reasonable algebraic space over $S$.
Let $U \to X$ be an \'etale morphism from a scheme to $X$.
If $u, u' \in |U|$ map to the same point of $|X|$, and
$u' \leadsto u$, then $u = u'$.
\end{lemma}

\begin{proof}
Combine Lemmas \ref{lemma-bounded-fibres} and
\ref{lemma-no-specializations-map-to-same-point}.
\end{proof}

\begin{lemma}
\label{lemma-very-reasonable-specialization}
Let $S$ be an algebraic space.
Let $X$ be an algebraic space over $S$.
Let $x, x' \in |X|$ and assume $x' \leadsto x$, i.e., $x$ is a
specialization of $x'$.
Assume $X$ is very reasonable. Then for every \'etale morphism
$\varphi : U \to X$ from a scheme $U$ and any $u \in U$ with
$\varphi(u) = x$, exists a point $u'\in U$, $u' \leadsto u$ with
$\varphi(u') = x'$.
\end{lemma}

\begin{proof}
Combine Lemmas \ref{lemma-bounded-fibres} and
\ref{lemma-specialization}.
\end{proof}

\begin{lemma}
\label{lemma-very-reasonable-kolmogorov}
Let $S$ be a scheme.
Let $X$ be a very reasonable algebraic space over $S$.
Then $|X|$ is Kolmogorov (see
Topology, Definition \ref{topology-definition-generic-point}).
\end{lemma}

\begin{proof}
Combine Lemmas \ref{lemma-bounded-fibres} and
\ref{lemma-kolmogorov}.
\end{proof}

\begin{proposition}
\label{proposition-very-reasonable-sober}
Let $S$ be a scheme.
Let $X$ be a very reasonable algebraic space over $S$.
Then the topological space $|X|$ is sober (see
Topology, Definition \ref{topology-definition-generic-point}).
\end{proposition}

\begin{proof}
We have seen in
Lemma \ref{lemma-very-reasonable-kolmogorov}
that $|X|$ is Kolmogorov.
Hence it remains to show that every irreducible closed subset
$T \subset |X|$ has a generic point. By
Properties of Spaces,
Lemma \ref{spaces-properties-lemma-reduced-closed-subspace}
there exists a closed subspace $Z \subset X$ with $|Z| = |T|$.
By definition this means that $Z \to X$ is a representable morphism
of algebraic spaces. Hence $Z$ is a very reasonable algebraic space
by Lemma \ref{lemma-representable-very-reasonable}. By
Proposition \ref{proposition-very-reasonable-open-dense-scheme}
we see that there exists an open dense subspace $Z' \subset Z$
which is a scheme. This means that $|Z'| \subset T$ is open dense.
Hence the topological space $|Z'|$ is irreducible, which means that
$Z'$ is an irreducible scheme. By
Schemes, Lemma \ref{schemes-lemma-scheme-sober}
we conclude that $|Z'|$ is the closure of a single point $\eta \in T$
and hence also $T = \overline{\{\eta\}}$, and we win.
\end{proof}













\section{Decent spaces}
\label{section-decent}

\noindent
In this section we collect some useful facts on decent spaces.

\begin{lemma}
\label{lemma-when-field}
Let $S$ be a scheme.
Let $X$ be a decent reduced algebraic space over $S$.
Assume that $|X|$ is a singleton.
Then $X \cong \text{Spec}(k)$ for some field $k$.
\end{lemma}

\begin{proof}
As $|X|$ is a singleton $X$ is quasi-compact, see
Properties of Spaces,
Lemma \ref{spaces-properties-lemma-quasi-compact-space}.
Let $U \to X$ be surjective \'etale with $U$ an affine scheme, see
Properties of Spaces,
Lemma \ref{spaces-properties-lemma-quasi-compact-affine-cover}.
Since $X$ is reduced we see that $U$ is reduced, see
Properties of Spaces,
Section \ref{spaces-properties-section-types-properties}.
As $X$ is decent there exists a monomorphism $\text{Spec}(k) \to X$
and $V = \text{Spec}(k) \times_X U$ is a scheme finite \'etale
over $\text{Spec}(k)$. Namely, this
follows from the definition of decent, see
Definition \ref{definition-very-reasonable},
which says that the equivalent conditions of
Lemma \ref{lemma-UR-finite-above-x}
hold at the unique point of $X$. Hence $V$ is a finite disjoint union of
spectra of finite separable field extensions of $k$, see
Morphisms, Lemma \ref{morphisms-lemma-etale-over-field}.
On the other hand $V \to U$ is a monomorphism (as $\text{Spec}(k) \to X$
is a monomorphism) and surjective (as $\text{Spec}(k) \to X$ is surjective by
Properties of Spaces,
Lemma \ref{spaces-properties-lemma-characterize-surjective}).
In particular $U$ has finitely many points.
By Lemma \ref{lemma-no-specializations-map-to-same-point}
there are no specializations among the points of $U$ (note that decent
implies the condition of that lemma are satisfied in view of
Lemma \ref{lemma-bounded-fibres}).
It follows that $U$ is a finite discrete topological space.
As $U$ is also reduced it follows that $U$ is a disjoint union
of spectra of fields. By
Schemes, Lemma \ref{schemes-lemma-mono-towards-spec-field}
we conclude that $V \to U$ is an isomorphism. Hence we see that
$U \to X$ factors through $\text{Spec}(k)$ which implies that
$\text{Spec}(k) \to X$ is also a surjection of sheaves, whence
an isomorphism as desired.
\end{proof}

\begin{remark}
\label{remark-one-point-decent-scheme}
We will see later (insert future reference here) that an algebraic space whose
reduction is a scheme is a scheme. Hence it follows from 
Lemma \ref{lemma-when-field}
that a decent algebraic space with one point is a scheme.
\end{remark}




\section{Valuative criterion}
\label{section-valuative-criterion-universally-closed}

\noindent
For a quasi-compact morphism from a decent space the valuative
criterion is necessary in order for the morphism to be
universally closed.


\begin{proposition}
\label{proposition-characterize-universally-closed}
Let $S$ be a scheme.
Let $f : X \to Y$ be a morphism of algebraic spaces over $S$.
Assume
\begin{enumerate}
\item $f$ is quasi-compact, and
\item $X$ has property $(\gamma)$ of
Lemma \ref{lemma-bounded-fibres}.
\end{enumerate}
Then $f$ is universally closed if and only if the
existence part of the valuative criterion holds.
\end{proposition}

\begin{proof}
In
Morphisms of Spaces,
Lemma \ref{spaces-morphisms-lemma-quasi-compact-existence-universally-closed}
we have seen one of the implications.
To prove the other, assume that $f$ is universally closed. Let
$$
\xymatrix{
\text{Spec}(K) \ar[r] \ar[d] & X \ar[d] \\
\text{Spec}(A) \ar[r] & Y
}
$$
be a diagram as in
Morphisms of Spaces,
Definition \ref{spaces-morphisms-definition-valuative-criterion}.
Let $X_A = \text{Spec}(A) \times_Y X$, so that we have
$$
\xymatrix{
\text{Spec}(K) \ar[r] \ar[rd] & X_A \ar[d] \\
 & \text{Spec}(A)
}
$$
By
Morphisms of Spaces,
Lemma \ref{spaces-morphisms-lemma-base-change-quasi-compact}
we see that
$X_A \to \text{Spec}(A)$ is quasi-compact. Since $X_A \to X$
is representable, we see that $X_A$ has property $(\gamma)$ also, see
Lemma \ref{lemma-representable-properties}.
Moreover, as $f$ is universally closed, we see that $X_A \to \text{Spec}(A)$
is universally closed.
Hence we may and do replace $X$ by $X_A$ and $Y$ by $\text{Spec}(A)$.

\medskip\noindent
Let $x' \in |X|$ be the equivalence class of
$\text{Spec}(K) \to X$. Let $y \in |Y| = |\text{Spec}(A)|$ be
the closed point. Set $y' = f(x')$; it is the generic point of
$\text{Spec}(A)$. Since $f$ is universally closed we see that
$f(\overline{\{x'\}})$ contains $\overline{\{y'\}}$, and hence
contains $y$. Let $x \in \overline{\{x'\}}$ be a point such that
$f(x) = y$. Let $U$ be a scheme, and $\varphi : U \to X$
an \'etale morphism such that there exists a $u \in U$ with
$\varphi(u) = x$. By
Lemma \ref{lemma-specialization}
and our assumption that $X$ has property $(\gamma)$
there exists a specialization $u' \leadsto u$ on $U$ with $\varphi(u') = x'$.
This means that there exists a common field extension
$K \subset K' \supset \kappa(u')$ such that
$$
\xymatrix{
\text{Spec}(K') \ar[r] \ar[d] & U \ar[d] \\
\text{Spec}(K) \ar[r] \ar[rd] & X \ar[d] \\
 & \text{Spec}(A)
}
$$
is commutative. This gives the following commutative diagram of rings
$$
\xymatrix{
K' & \mathcal{O}_{U, u} \ar[l] \\
K \ar[u] & \\
 & A \ar[lu] \ar[uu]
}
$$
By
Algebra, Lemma \ref{algebra-lemma-dominate}
we can find a valuation ring $A' \subset K'$ dominating the image of
$\mathcal{O}_{U, u}$ in $K'$. Since by construction $\mathcal{O}_{U, u}$
dominates $A$ we see that $A'$ dominates $A$ also. Hence we obtain a diagram
resembling the second diagram of
Morphisms of Spaces,
Definition \ref{spaces-morphisms-definition-valuative-criterion}
and the proposition is proved.
\end{proof}











\section{Relative conditions}
\label{section-relative-conditions}

\noindent
This is a (yet another) technical section dealing with conditions on
algebraic spaces having to do with points. It is probably a good idea
to skip this section.

\begin{definition}
\label{definition-relative-conditions}
Let $S$ be a scheme.
\begin{enumerate}
\item We say an algebraic space $X$ over $S$ {\it has
property $(\beta)$, $(\gamma)$, $(\delta)$, or $(\epsilon)$} if $X$
has the corresponding property of
Lemma \ref{lemma-bounded-fibres}.
\item An algebraic space which has $(\gamma)$ is called {\it decent}, see
Definition \ref{definition-very-reasonable}.
\item An algebraic space which has $(\delta)$ is called {\it reasonable}, see
Definition \ref{definition-very-reasonable}.
\item An algebraic space which has $(\epsilon)$ is called
{\it very reasonable}, see
Definition \ref{definition-very-reasonable}.
\item Let $f : X \to Y$ be a morphism of algebraic spaces over $S$.
We say $f$ {\it has property $(\beta)$, $(\gamma)$, $(\delta)$,
or $(\epsilon)$} if for any scheme $T$ and morphism $T \to Y$
the fibre product $T \times_Y X$ has the corresponding property.
\item A morphism $f$ which has property $(\gamma)$ is called a
{\it decent morphism}.
\item A morphism $f$ which has property $(\delta)$ is called a
{\it reasonable morphism}.
\item A morphism $f$ which has property $(\epsilon)$ is called a
{\it very reasonable morphism}.
\end{enumerate}
\end{definition}

\noindent
We refer to Remark \ref{remark-very-reasonable} for an informal discussion.
It will turn out that the class of very reasonable morphisms is not so
useful, but that the classes of decent and reasonable morphisms are useful.

\begin{lemma}
\label{lemma-properties-trivial-implications}
Let $S$ be a scheme.
Let $f : X \to Y$ be a morphism of algebraic spaces over $S$.
We have the following implications among the conditions on $f$:
$$
\xymatrix{
\text{representable} \ar@{=>}[rd] & & & & \\
& \text{very reasonable} \ar@{=>}[r] & \text{reasonable} \ar@{=>}[r] &
\text{decent} \ar@{=>}[r] & (\beta) \\
\text{quasi-separated} \ar@{=>}[ru] & & & &
}
$$
\end{lemma}

\begin{proof}
This is clear from the definitions,
Lemma \ref{lemma-bounded-fibres}
and
Morphisms of Spaces,
Lemma \ref{spaces-morphisms-lemma-separated-local}.
\end{proof}

\begin{lemma}
\label{lemma-base-change-relative-conditions}
Having property $(\beta)$, $(\gamma)$, $(\delta)$, or $(\epsilon)$
is preserved under arbitrary base change.
\end{lemma}

\begin{proof}
Omitted.
\end{proof}

\begin{lemma}
\label{lemma-composition-relative-conditions}
Having property $(\beta)$, $(\gamma)$, or $(\delta)$
is preserved under compositions.
\end{lemma}

\begin{proof}
Let $\omega \in \{\beta, \gamma, \delta\}$.
Let $f : X \to Y$ and $g : Y \to Z$ be morphisms of algebraic spaces
over the scheme $S$. Assume $f$ and $g$ both have property
$(\omega)$. Then we have to show
that for any scheme $T$ and morphism $T \to Z$ the space $T \times_Z X$
has $(\omega)$. By
Lemma \ref{lemma-base-change-relative-conditions}
this reduces us to the following claim: Suppose that $Y$ is an algebraic
space having property $(\omega)$, and that $f : X \to Y$ is a morphism
with $(\omega)$. Then $X$ has $(\omega)$.

\medskip\noindent
Let us prove the claim in case $\omega = \beta$. In this case we have to show
that any $x \in |X|$ is represented by a monomorphism from the spectrum
of a field into $X$. Let $y = f(x) \in |Y|$. By assumption there exists
a field $k$ and a monomorphism $\text{Spec}(k) \to Y$ representing $y$.
Then $x$ corresponds to a point $x'$ of $\text{Spec}(k) \times_Y X$.
By assumption $x'$ is represented by a monomorphism
$\text{Spec}(k') \to \text{Spec}(k) \times_Y X$. Clearly the composition
$\text{Spec}(k') \to X$ is a monomorphism representing $x$.

\medskip\noindent
Let us prove the claim in case $\omega = \gamma$.
Let $x \in |X|$ and $y = f(x) \in |Y|$. By the result of the preceding
paragraph we can choose a diagram
$$
\xymatrix{
\text{Spec}(k') \ar[r]_x \ar[d] & X \ar[d]^f \\
\text{Spec}(k) \ar[r]^y & Y
}
$$
with horizontal arrows monomorphisms. We are going to denote
fibre products of the form $\text{Spec}(k) \times_{y, Y} ?$,
resp.\ $\text{Spec}(k') \times_{x, X} ?$ by $?_y$, resp.\ $?_x$.
Choose an affine scheme $V$ and \'etale morphism $V \to Y$
such that $V_y$ is not empty. Choose an affine scheme $U$ and an
\'etale morphism $U \to V \times_Y X$ such that
$U_x$ is not empty. Picture:
$$
\xymatrix{
U \ar[r] \ar[rd] & V \times_Y X \ar[d] \ar[r] & X \ar[d]^f \\
 & V \ar[r] & Y
}
$$
The assumption $(\gamma)$ for $Y$ implies that $V_y$ is a finite scheme
over $k$ and the assumption $(\gamma)$ for $f$ (applied to the base change
of $f$ by $V_y \to Y$) implies the fibres of
$U_x \to \text{Spec}(k') \times_{\text{Spec}(k)} V_y = (V \times_Y X)_x$
are finite. Note that the morphism
$U_x \to \text{Spec}(k') \times_{\text{Spec}(k)} V_y$ is \'etale.
Hence the scheme $U_x$ is finite. Now the collection of all \'etale morphisms
$U \to X$ with $U$ affine which either are of the form above, or have
$U_x = \emptyset$ cover $X$. Hence this collection is a collection of
morphisms as in
Lemma \ref{lemma-UR-finite-above-x} part (3).
As the point $x$ was arbitrary this implies $(\gamma)$ holds for $X$.

\medskip\noindent
Let us prove the claim in case $\omega = \delta$.
Choose $V \to Y$ \'etale with $V$ an affine scheme.
Choose $U \to V \times_Y X$ \'etale with $U$ an affine scheme.
By assumption $V \to Y$ has universally bounded fibres. By
Lemma \ref{lemma-base-change-universally-bounded}
$V \times_Y X \to X$ has universally bounded fibres.
By assumption on $f$ we see that $U \to V \times_Y X$ has
universally bounded fibres. By
Lemma \ref{lemma-composition-universally-bounded}
the composition $U \to X$ has universally bounded fibres.
Hence there exists sufficiently many \'etale morphisms $U \to X$
from schemes with universally bounded fibres, and we conclude
that $X$ has property $(\delta)$.
\end{proof}

\begin{lemma}
\label{lemma-descent-conditions}
Let $S$ be a scheme.
Let $f : X \to Y$ be a morphism of algebraic spaces over $S$.
Let $\mathcal{P} \in \{(\beta), (\gamma), (\delta)\}$.
Assume
\begin{enumerate}
\item $f$ is quasi-compact,
\item $f$ is \'etale,
\item $|f| : |X| \to |Y|$ is surjective, and
\item the algebraic space $X$ has property $\mathcal{P}$.
\end{enumerate}
Then $Y$ has property $\mathcal{P}$.
\end{lemma}

\begin{proof}
Let us prove this in case $\mathcal{P} = (\beta)$. Let $y \in |Y|$ be
a point. We have to show that $y$ can be represented by a monomorphism
from a field. Choose a point $x \in |X|$ with $f(x) = y$.
By assumption we may represent $x$ by a monomorphism
$\text{Spec}(k) \to X$, with $k$ a field. By
Lemma \ref{lemma-R-finite-above-x}
it suffices to show that the projections
$\text{Spec}(k) \times_Y \text{Spec}(k) \to \text{Spec}(k)$
are \'etale and quasi-compact. We can factor the first projection as
$$
\text{Spec}(k) \times_Y \text{Spec}(k)
\longrightarrow
\text{Spec}(k) \times_Y X
\longrightarrow
\text{Spec}(k)
$$
The first morphism is a monomorphism, and the second is \'etale and
quasi-compact. By
Properties of Spaces,
Lemma \ref{spaces-properties-lemma-etale-over-field-scheme}
we see that $\text{Spec}(k) \times_Y X$ is a scheme. Hence it is a
finite disjoint union of spectra of finite separable field extensions
of $k$. By
Schemes, Lemma \ref{schemes-lemma-mono-towards-spec-field}
we see that the first arrow identifies
$\text{Spec}(k) \times_Y \text{Spec}(k)$ with a finite disjoint
union of spectra of finite separable field extensions of $k$.
Hence the projection morphism is \'etale and quasi-compact.

\medskip\noindent
Let us prove this in case $\mathcal{P} = (\gamma)$.
We have already seen in the first paragraph of the proof that this implies
that every $y \in |Y|$ can be represented by a monomorphism
$y : \text{Spec}(k) \to Y$. Pick such a $y$. Pick an affine
scheme $U$ and an \'etale morphism $U \to X$ such that the image
of $|U| \to |Y|$ contains $y$. By
Lemma \ref{lemma-UR-finite-above-x}
it suffices to show that $U_y$ is a finite scheme over $k$. The fibre
product $X_y = \text{Spec}(k) \times_Y X$ is a quasi-compact \'etale
algebraic space over $k$. Hence by
Properties of Spaces,
Lemma \ref{spaces-properties-lemma-etale-over-field-scheme}
it is a scheme. So it is a finite disjoint union of spectra of
finite separable extensions of $k$. Say $X_y = \{x_1, \ldots, x_n\}$
so $x_i$ is given by  $x_i : \text{Spec}(k_i) \to X$ with
$[k_i : k] < \infty$. By assumption $X$ has $(\gamma)$, so the schemes
$U_{x_i} = \text{Spec}(k_i) \times_X U$ is finite over $k_i$.
Finally, we note that $U_y = \coprod U_{x_i}$ as a scheme and we conclude
that $U_y$ is finite over $k$ as desired.

\medskip\noindent
Let us prove this in case $\mathcal{P} = (\delta)$.
Pick an affine scheme $V$ and an \'etale morphism $V \to Y$.
We have the show the fibres of $V \to Y$ are universally bounded.
The algebraic space $V \times_Y X$ is quasi-compact.
Thus we can find an affine scheme $W$ and a surjective \'etale morphism
$W \to V \times_Y X$, see
Properties of Spaces,
Lemma \ref{spaces-properties-lemma-quasi-compact-affine-cover}.
Here is a picture (solid diagram)
$$
\xymatrix{
W \ar[r]  \ar[rd] &
V \times_Y X \ar[r] \ar[d] &
X \ar[d]_f & \text{Spec}(k) \ar@{..>}[l]^x \ar@{..>}[ld]^y \\
 & V \ar[r] & Y
}
$$
The morphism $W \to X$ is universally bounded by our assumption that
the space $X$ has property $(\delta)$. Let $n$ be an integer bounding
the degrees of the fibres of $W \to X$. We claim that the same integer
works for bounding the fibres of $V \to Y$. Namely, suppose $y \in |Y|$
is a point. Then there exists a $x \in |X|$ with $f(x) = y$ (see above).
This means we can find a field $k$ and morphisms $x, y$ given as dotted
arrows in the diagram above. In particular we get a surjective \'etale
morphism
$$
\text{Spec}(k) \times_{x, X} W
\to
\text{Spec}(k) \times_{x, X} (V \times_Y X) = \text{Spec}(k) \times_{y, Y} V
$$
which shows that the degree of $\text{Spec}(k) \times_{y, Y} V$ over $k$
is less than or equal to the degree of $\text{Spec}(k) \times_{x, X} W$
over $k$, i.e., $\leq n$, and we win. (This last part of the argument
is the same as the argument in the proof of
Lemma \ref{lemma-descent-universally-bounded}.
Unfortunately that lemma is not general enough because it only applies
to representable morphisms.)
\end{proof}

\begin{lemma}
\label{lemma-relative-conditions-local}
Let $S$ be a scheme.
Let $f : X \to Y$ be a morphism of algebraic spaces over $S$.
Let $\mathcal{P} \in \{(\beta), (\gamma), (\delta), (\epsilon)\}$.
The following are equivalent
\begin{enumerate}
\item $f$ is $\mathcal{P}$,
\item for every affine scheme $Z$ and every morphism $Z \to Y$ the
base change $Z \times_Y X \to Z$ of $f$ is $\mathcal{P}$,
\item for every affine scheme $Z$ and every morphism $Z \to Y$ the
algebraic space $Z \times_Y X$ is $\mathcal{P}$, and
\item there exists a Zariski covering $Y = \bigcup Y_i$ such
that each morphism $f^{-1}(Y_i) \to Y_i$ has $\mathcal{P}$.
\end{enumerate}
If $\mathcal{P} \in \{(\beta), (\gamma), (\delta)\}$, then this is also
equivalent to 
\begin{enumerate}
\item[(4)] there exists a scheme $V$ and a surjective \'etale morphism
$V \to Y$ such that the base change $V \times_Y X \to V$ has
$\mathcal{P}$.
\end{enumerate}
\end{lemma}

\begin{proof}
The implications (1) $\Rightarrow$ (2) $\Rightarrow$ (3) $\Rightarrow$ (4)
are trivial.
The implication (3) $\Rightarrow$ (1) can be seen as follows.
Let $Z \to Y$ be a morphism whose source is a scheme over $S$.
Consider the algebraic space $Z \times_Y X$. If we assume (3), then
for any affine open $W \subset Z$, the open subspace
$W \times_Y X$ of $Z \times_Y X$ has property $\mathcal{P}$. Hence by
Lemma \ref{lemma-properties-local}
the space $Z \times_Y X$ has property $\mathcal{P}$, i.e., (1) holds.
A similar argument (omitted) shows that (4) implies (1).

\medskip\noindent
The implication (1) $\Rightarrow$ (5) is trivial. Let $V \to Y$ be
an \'etale morphism from a scheme as in (5). Let $Z$ be an affine scheme,
and let $Z \to Y$ be a morphism. Consider the diagram
$$
\xymatrix{
Z \times_Y V \ar[r]_q \ar[d]_p & V \ar[d] \\
Z \ar[r] & Y
}
$$
Since $p$ is \'etale, and hence open, we can choose finitely many affine open
subschemes $W_i \subset Z \times_Y V$ such that $Z = \bigcup p(W_i)$.
Consider the commutative diagram
$$
\xymatrix{
V \times_Y X \ar[d] &
(\coprod W_i) \times_Y X \ar[l] \ar[d] \ar[r] &
Z \times_Y X \ar[d] \\
V &
\coprod W_i \ar[l] \ar[r] &
Z
}
$$
We know $V \times_Y X$ has property $\mathcal{P}$. By 
Lemma \ref{lemma-representable-properties}
we see that $(\coprod W_i) \times_Y X$ has property $\mathcal{P}$.
Note that the morphism $(\coprod W_i) \times_Y X \to Z \times_Y X$
is \'etale and quasi-compact as the base change of $\coprod W_i \to Z$.
Hence by Lemma \ref{lemma-descent-conditions}
we conclude that $Z \times_Y X$ has property $\mathcal{P}$.
\end{proof}

\begin{remark}
\label{remark-very-reasonable}
Informally the properties of
Definition \ref{definition-relative-conditions}
mean the following:
\begin{enumerate}
\item Condition $(\beta)$ on a space means that points are always represented
by monomorphisms from spectra of fields.
\item Condition $(\gamma)$ on a space means that points are always represented
by monomorphisms from spectra of fields and that those monomorphisms are
quasi-compact (insert future reference here). Such a space is called
{\it decent}.
\item Condition $(\delta)$ means $(\gamma)$ $+$ locally on the space exist
\'etale coverings whose fibres are universally bounded.
\item Very reasonable means there exists a Zariski open covering whose
pieces have coverings $\varphi_i : U_i \to X_i$ which are quasi-compact.
Very reasonable implies $(\delta)$.
\item A morphism has one of these properties if (very) loosely speaking the
fibres of the morphism have the corresponding properties.
\end{enumerate}
Being decent is useful to prove things about specializations of
points on $|X|$. Condition $(\delta)$ is a bit stronger, and technically
quite easy to work with. Very reasonable is a good condition in the sense that
it implies that $X$ has a dense open subspace which is a scheme, and
that $|X|$ is a sober topological space. This is not clear for spaces
which have property $(\delta)$ and probably not true (although see
Remark \ref{remark-fun-property-reasonable}
for an interesting additional property of spaces of type $(\delta)$).
On the other hand, we do not know whether the class of very reasonable
morphisms is closed under composition, and we do not know whether
very reasonable spaces satisfy a descent property as the one in
Lemma \ref{lemma-descent-conditions} (even with $f$ assumed representable).
\end{remark}

\noindent
Here is the lemma we promised earlier.

\begin{lemma}
\label{lemma-re-characterize-universally-closed}
Let $S$ be a scheme.
Let $f : X \to Y$ be a morphism of algebraic spaces over $S$.
Assume $f$ is quasi-compact, and $f$ has property $(\gamma)$ (i.e., $f$
is decent).
(For example if $f$ is representable, or quasi-separated, see
Lemma \ref{lemma-properties-trivial-implications}.)
Then $f$ is universally closed if and only if the
existence part of the valuative criterion holds.
\end{lemma}

\begin{proof}
In
Morphisms of Spaces,
Lemma \ref{spaces-morphisms-lemma-quasi-compact-existence-universally-closed}
we proved that any quasi-compact morphism which satsifies the existence
part of the valuative criterion is universally closed.
To prove the other, assume that $f$ is universally closed.
In the proof of
Proposition \ref{proposition-characterize-universally-closed}
we have seen that it suffices to show, for any valuation ring $A$,
and any morphism $\text{Spec}(A) \to Y$, that the base change
$f_A : X_A \to \text{Spec}(A)$ satisfies the existence part of the valuative
criterion. By definition the algebraic space $X_A$ has property $(\gamma)$
and hence
Proposition \ref{proposition-characterize-universally-closed}
applies to the morphism $f_A$ and we win.
\end{proof}




















\section{Other chapters}

\begin{multicols}{2}
\begin{enumerate}
\item \hyperref[introduction-section-phantom]{Introduction}
\item \hyperref[conventions-section-phantom]{Conventions}
\item \hyperref[sets-section-phantom]{Set Theory}
\item \hyperref[categories-section-phantom]{Categories}
\item \hyperref[topology-section-phantom]{Topology}
\item \hyperref[sheaves-section-phantom]{Sheaves on Spaces}
\item \hyperref[algebra-section-phantom]{Commutative Algebra}
\item \hyperref[sites-section-phantom]{Sites and Sheaves}
\item \hyperref[homology-section-phantom]{Homological Algebra}
\item \hyperref[derived-section-phantom]{Derived Categories}
\item \hyperref[more-algebra-section-phantom]{More Algebra}
\item \hyperref[simplicial-section-phantom]{Simplicial Methods}
\item \hyperref[modules-section-phantom]{Sheaves of Modules}
\item \hyperref[sites-modules-section-phantom]{Modules on Sites}
\item \hyperref[injectives-section-phantom]{Injectives}
\item \hyperref[cohomology-section-phantom]{Cohomology of Sheaves}
\item \hyperref[sites-cohomology-section-phantom]{Cohomology on Sites}
\item \hyperref[hypercovering-section-phantom]{Hypercoverings}
\item \hyperref[schemes-section-phantom]{Schemes}
\item \hyperref[constructions-section-phantom]{Constructions of Schemes}
\item \hyperref[properties-section-phantom]{Properties of Schemes}
\item \hyperref[morphisms-section-phantom]{Morphisms of Schemes}
\item \hyperref[coherent-section-phantom]{Coherent Cohomology}
\item \hyperref[divisors-section-phantom]{Divisors}
\item \hyperref[limits-section-phantom]{Limits of Schemes}
\item \hyperref[varieties-section-phantom]{Varieties}
\item \hyperref[chow-section-phantom]{Chow Homology}
\item \hyperref[topologies-section-phantom]{Topologies on Schemes}
\item \hyperref[descent-section-phantom]{Descent}
\item \hyperref[more-morphisms-section-phantom]{More on Morphisms}
\item \hyperref[flat-section-phantom]{More on Flatness}
\item \hyperref[groupoids-section-phantom]{Groupoid Schemes}
\item \hyperref[more-groupoids-section-phantom]{More on Groupoid Schemes}
\item \hyperref[etale-section-phantom]{\'Etale Morphisms of Schemes}
\item \hyperref[etale-cohomology-section-phantom]{\'Etale Cohomology}
\item \hyperref[spaces-section-phantom]{Algebraic Spaces}
\item \hyperref[spaces-properties-section-phantom]{Properties of Algebraic Spaces}
\item \hyperref[spaces-morphisms-section-phantom]{Morphisms of Algebraic Spaces}
\item \hyperref[spaces-topologies-section-phantom]{Topologies on Algebraic Spaces}
\item \hyperref[spaces-descent-section-phantom]{Descent and Algebraic Spaces}
\item \hyperref[spaces-more-morphisms-section-phantom]{More on Morphisms of Spaces}
\item \hyperref[quot-section-phantom]{Quot and Hilbert Spaces}
\item \hyperref[stacks-section-phantom]{Stacks}
\item \hyperref[spaces-groupoids-section-phantom]{Groupoids in Algebraic Spaces}
\item \hyperref[spaces-more-groupoids-section-phantom]{More on Groupoids in Spaces}
\item \hyperref[bootstrap-section-phantom]{Bootstrap}
\item \hyperref[examples-stacks-section-phantom]{Examples of Stacks}
\item \hyperref[groupoids-quotients-section-phantom]{Quotients of Groupoids}
\item \hyperref[algebraic-section-phantom]{Algebraic Stacks}
\item \hyperref[criteria-section-phantom]{Criteria for Representability}
\item \hyperref[stacks-properties-section-phantom]{Properties of Algebraic Stacks}
\item \hyperref[stacks-morphisms-section-phantom]{Morphisms of Algebraic Stacks}
\item \hyperref[examples-section-phantom]{Examples}
\item \hyperref[exercises-section-phantom]{Exercises}
\item \hyperref[guide-section-phantom]{Guide to Literature}
\item \hyperref[desirables-section-phantom]{Desirables}
\item \hyperref[coding-section-phantom]{Coding Style}
\item \hyperref[fdl-section-phantom]{GNU Free Documentation License}
\item \hyperref[index-section-phantom]{Auto Generated Index}
\end{enumerate}
\end{multicols}


\bibliography{my}
\bibliographystyle{amsalpha}

\end{document}
