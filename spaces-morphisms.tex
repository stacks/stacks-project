\IfFileExists{stacks-project.cls}{%
\documentclass{stacks-project}
}{%
\documentclass{amsart}
}

% The following AMS packages are automatically loaded with
% the amsart documentclass:
%\usepackage{amsmath}
%\usepackage{amssymb}
%\usepackage{amsthm}

% For dealing with references we use the comment environment
\usepackage{verbatim}
\newenvironment{reference}{\comment}{\endcomment}
%\newenvironment{reference}{}{}
\newenvironment{slogan}{\comment}{\endcomment}
\newenvironment{history}{\comment}{\endcomment}

% For commutative diagrams you can use
% \usepackage{amscd}
\usepackage[all]{xy}

% We use 2cell for 2-commutative diagrams.
\xyoption{2cell}
\UseAllTwocells

% To put source file link in headers.
% Change "template.tex" to "this_filename.tex"
% \usepackage{fancyhdr}
% \pagestyle{fancy}
% \lhead{}
% \chead{}
% \rhead{Source file: \url{template.tex}}
% \lfoot{}
% \cfoot{\thepage}
% \rfoot{}
% \renewcommand{\headrulewidth}{0pt}
% \renewcommand{\footrulewidth}{0pt}
% \renewcommand{\headheight}{12pt}

\usepackage{multicol}

% For cross-file-references
\usepackage{xr-hyper}

% Package for hypertext links:
\usepackage{hyperref}

% For any local file, say "hello.tex" you want to link to please
% use \externaldocument[hello-]{hello}
\externaldocument[introduction-]{introduction}
\externaldocument[conventions-]{conventions}
\externaldocument[sets-]{sets}
\externaldocument[categories-]{categories}
\externaldocument[topology-]{topology}
\externaldocument[sheaves-]{sheaves}
\externaldocument[sites-]{sites}
\externaldocument[stacks-]{stacks}
\externaldocument[fields-]{fields}
\externaldocument[algebra-]{algebra}
\externaldocument[brauer-]{brauer}
\externaldocument[homology-]{homology}
\externaldocument[derived-]{derived}
\externaldocument[simplicial-]{simplicial}
\externaldocument[more-algebra-]{more-algebra}
\externaldocument[smoothing-]{smoothing}
\externaldocument[modules-]{modules}
\externaldocument[sites-modules-]{sites-modules}
\externaldocument[injectives-]{injectives}
\externaldocument[cohomology-]{cohomology}
\externaldocument[sites-cohomology-]{sites-cohomology}
\externaldocument[dga-]{dga}
\externaldocument[dpa-]{dpa}
\externaldocument[hypercovering-]{hypercovering}
\externaldocument[schemes-]{schemes}
\externaldocument[constructions-]{constructions}
\externaldocument[properties-]{properties}
\externaldocument[morphisms-]{morphisms}
\externaldocument[coherent-]{coherent}
\externaldocument[divisors-]{divisors}
\externaldocument[limits-]{limits}
\externaldocument[varieties-]{varieties}
\externaldocument[topologies-]{topologies}
\externaldocument[descent-]{descent}
\externaldocument[perfect-]{perfect}
\externaldocument[more-morphisms-]{more-morphisms}
\externaldocument[flat-]{flat}
\externaldocument[groupoids-]{groupoids}
\externaldocument[more-groupoids-]{more-groupoids}
\externaldocument[etale-]{etale}
\externaldocument[chow-]{chow}
\externaldocument[intersection-]{intersection}
\externaldocument[pic-]{pic}
\externaldocument[adequate-]{adequate}
\externaldocument[dualizing-]{dualizing}
\externaldocument[duality-]{duality}
\externaldocument[discriminant-]{discriminant}
\externaldocument[local-cohomology-]{local-cohomology}
\externaldocument[curves-]{curves}
\externaldocument[resolve-]{resolve}
\externaldocument[models-]{models}
\externaldocument[pione-]{pione}
\externaldocument[etale-cohomology-]{etale-cohomology}
\externaldocument[proetale-]{proetale}
\externaldocument[crystalline-]{crystalline}
\externaldocument[spaces-]{spaces}
\externaldocument[spaces-properties-]{spaces-properties}
\externaldocument[spaces-morphisms-]{spaces-morphisms}
\externaldocument[decent-spaces-]{decent-spaces}
\externaldocument[spaces-cohomology-]{spaces-cohomology}
\externaldocument[spaces-limits-]{spaces-limits}
\externaldocument[spaces-divisors-]{spaces-divisors}
\externaldocument[spaces-over-fields-]{spaces-over-fields}
\externaldocument[spaces-topologies-]{spaces-topologies}
\externaldocument[spaces-descent-]{spaces-descent}
\externaldocument[spaces-perfect-]{spaces-perfect}
\externaldocument[spaces-more-morphisms-]{spaces-more-morphisms}
\externaldocument[spaces-flat-]{spaces-flat}
\externaldocument[spaces-groupoids-]{spaces-groupoids}
\externaldocument[spaces-more-groupoids-]{spaces-more-groupoids}
\externaldocument[bootstrap-]{bootstrap}
\externaldocument[spaces-pushouts-]{spaces-pushouts}
\externaldocument[groupoids-quotients-]{groupoids-quotients}
\externaldocument[spaces-more-cohomology-]{spaces-more-cohomology}
\externaldocument[spaces-simplicial-]{spaces-simplicial}
\externaldocument[formal-spaces-]{formal-spaces}
\externaldocument[restricted-]{restricted}
\externaldocument[spaces-resolve-]{spaces-resolve}
\externaldocument[formal-defos-]{formal-defos}
\externaldocument[defos-]{defos}
\externaldocument[cotangent-]{cotangent}
\externaldocument[examples-defos-]{examples-defos}
\externaldocument[algebraic-]{algebraic}
\externaldocument[examples-stacks-]{examples-stacks}
\externaldocument[stacks-sheaves-]{stacks-sheaves}
\externaldocument[criteria-]{criteria}
\externaldocument[artin-]{artin}
\externaldocument[quot-]{quot}
\externaldocument[stacks-properties-]{stacks-properties}
\externaldocument[stacks-morphisms-]{stacks-morphisms}
\externaldocument[stacks-limits-]{stacks-limits}
\externaldocument[stacks-cohomology-]{stacks-cohomology}
\externaldocument[stacks-perfect-]{stacks-perfect}
\externaldocument[stacks-introduction-]{stacks-introduction}
\externaldocument[stacks-more-morphisms-]{stacks-more-morphisms}
\externaldocument[stacks-geometry-]{stacks-geometry}
\externaldocument[moduli-]{moduli}
\externaldocument[moduli-curves-]{moduli-curves}
\externaldocument[examples-]{examples}
\externaldocument[exercises-]{exercises}
\externaldocument[guide-]{guide}
\externaldocument[desirables-]{desirables}
\externaldocument[coding-]{coding}
\externaldocument[obsolete-]{obsolete}
\externaldocument[fdl-]{fdl}
\externaldocument[index-]{index}

% Theorem environments.
%
\theoremstyle{plain}
\newtheorem{theorem}[subsection]{Theorem}
\newtheorem{proposition}[subsection]{Proposition}
\newtheorem{lemma}[subsection]{Lemma}

\theoremstyle{definition}
\newtheorem{definition}[subsection]{Definition}
\newtheorem{example}[subsection]{Example}
\newtheorem{exercise}[subsection]{Exercise}
\newtheorem{situation}[subsection]{Situation}

\theoremstyle{remark}
\newtheorem{remark}[subsection]{Remark}
\newtheorem{remarks}[subsection]{Remarks}

\numberwithin{equation}{subsection}

% Macros
%
\def\lim{\mathop{\rm lim}\nolimits}
\def\colim{\mathop{\rm colim}\nolimits}
\def\Spec{\mathop{\rm Spec}}
\def\Hom{\mathop{\rm Hom}\nolimits}
\def\Ext{\mathop{\rm Ext}\nolimits}
\def\SheafHom{\mathop{\mathcal{H}\!{\it om}}\nolimits}
\def\SheafExt{\mathop{\mathcal{E}\!{\it xt}}\nolimits}
\def\Sch{\textit{Sch}}
\def\Mor{\mathop{\rm Mor}\nolimits}
\def\Ob{\mathop{\rm Ob}\nolimits}
\def\Sh{\mathop{\textit{Sh}}\nolimits}
\def\NL{\mathop{N\!L}\nolimits}
\def\proetale{{pro\text{-}\acute{e}tale}}
\def\etale{{\acute{e}tale}}
\def\QCoh{\textit{QCoh}}
\def\Ker{\mathop{\rm Ker}}
\def\Im{\mathop{\rm Im}}
\def\Coker{\mathop{\rm Coker}}
\def\Coim{\mathop{\rm Coim}}

%
% Macros for moduli stacks/spaces
%
\def\QCohstack{\mathcal{QC}\!{\it oh}}
\def\Cohstack{\mathcal{C}\!{\it oh}}
\def\Spacesstack{\mathcal{S}\!{\it paces}}
\def\Quotfunctor{{\rm Quot}}
\def\Hilbfunctor{{\rm Hilb}}
\def\Curvesstack{\mathcal{C}\!{\it urves}}
\def\Polarizedstack{\mathcal{P}\!{\it olarized}}
\def\Complexesstack{\mathcal{C}\!{\it omplexes}}
% \Pic is the operator that assigns to X its picard group, usage \Pic(X)
% \Picardstack_{X/B} denotes the Picard stack of X over B
% \Picardfunctor_{X/B} denotes the Picard functor of X over B
\def\Pic{\mathop{\rm Pic}\nolimits}
\def\Picardstack{\mathcal{P}\!{\it ic}}
\def\Picardfunctor{{\rm Pic}}
\def\Deformationcategory{\mathcal{D}\!{\it ef}}


% OK, start here.
%
\begin{document}

\title{Morphisms of algebraic spaces}


\maketitle

\phantomsection
\label{section-phantom}

\tableofcontents

\section{Introduction}
\label{section-introduction}

\noindent
In this chapter we introduce some types of morphisms of algebraic spaces.
A reference is \cite{Kn}.

\medskip\noindent
The goal is to extend the definition of each of the types of morphisms of
schemes defined in the chapters on schemes, and on morphisms of schemes
to the category of algebraic spaces. Each case is slightly different and
it seems best to treat them all separately.







\section{Properties of representable morphisms}
\label{section-representable}

\noindent
Let $S$ be a scheme.
Let $f : X \to Y$ be a representable morphism of algebraic spaces. In
Spaces, Section \ref{spaces-section-representable-properties}
we defined what it means for $f$ to
have property $\mathcal{P}$ in case $\mathcal{P}$ is a property
of morphisms of schemes which
\begin{enumerate}
\item is preserved under any base change,
see Schemes, Definition \ref{schemes-definition-preserved-by-base-change},
and
\item is fppf local on the base, see
Descent, Definition \ref{descent-definition-property-morphisms-local}.
\end{enumerate}
Namely, in this case we say $f$ has property $\mathcal{P}$ if and only
if for every scheme $U$ and any morphism $U \to Y$ the morphism of schemes
$X \times_Y U \to U$ has property $\mathcal{P}$.

\medskip\noindent
According to the lists in
Spaces, Section \ref{spaces-section-lists}
this applies to the following properties:
(1)(a) closed immersions,
(1)(b) open immersions,
(1)(c) quasi-compact immersions,
(2) quasi-compact,
(3) universally-closed,
(4) (quasi-)separated,
(5) monomorphism,
(6) surjective,
(7) radicial (or universally injective),
(8) affine,
(9) quasi-affine,
(10) (locally) of finite type,
(11) (locally) quasi-finite,
(12) (locally) of finite presentation,
(13) locally of finite type of relative dimension $d$,
(14) universally open,
(15) flat,
(16) syntomic,
(17) smooth,
(18) unramified,
(19) etale,
(20) proper,
(21) finite or integral,
(22) finite locally free, and
(23) immersion.

\medskip\noindent
In this chapter we will redefine these notions for not necessarily
representable morphisms of algebraic spaces. Whenever we do this we will make
sure that the new definition agrees with the old one, in order to avoid
ambiguity.

\medskip\noindent
Note that the definition above applies whenever $X$ is a scheme,
since a morphism from a scheme to an algebraic space is representable.
And in particular it applies when both $X$ and $Y$ are schemes.
In Spaces, Lemma
\ref{spaces-lemma-morphism-schemes-gives-representable-transformation-property}
we have seen that in this case the definitions
match, and no ambiguity arise.

\medskip\noindent
Furthermore, in Spaces,
Lemma \ref{spaces-lemma-base-change-representable-transformations-property}
we have seen that the property of
representable morphisms of algebraic spaces so defined is stable under
arbitrary base change by a morphism of algebraic spaces.
And finally, in Spaces, Lemmas
\ref{spaces-lemma-composition-representable-transformations-property}
and
\ref{spaces-lemma-product-representable-transformations-property}
we have seen that if $\mathcal{P}$ is stable under compositions,
which holds for the properties
(1)(a), (1)(b), (1)(c), (2) -- (23), except (13) above, then
taking products of representable morphisms preserves property $\mathcal{P}$ 
and compositions of representable morphisms preserves property $\mathcal{P}$.

\medskip\noindent
We will use these facts below, and whenever we do we will simply refer
to this section as a reference.





\section{Immersions}
\label{section-immersions}

\noindent
Open, closed and locally closed immersions of algebraic spaces were defined in
Spaces, Section \ref{spaces-section-Zariski}.
Namely, a morphism of algebraic spaces is a
{\it closed immersion} (resp. {\it open immersion}, resp.\ {\it immersion})
if it is representable and a closed immersion (resp.\ open immersion,
resp.\ immersion) in the sense of Section \ref{section-representable}.

\medskip\noindent
In particular these types of morphisms are stable under base change,
composition and products of morphisms in the category of algebraic
spaces over $S$, see
Spaces, Lemmas \ref{spaces-lemma-composition-immersions} and
\ref{spaces-lemma-base-change-immersions}.

\begin{lemma}
\label{lemma-characterize-closed-immersion}
Let $S$ be a scheme. Let $f : X \to Y$ be a morphism of algebraic spaces
over $S$. The following are equivalent:
\begin{enumerate}
\item $f$ is a closed immersion (resp.\ open immersion, resp.\ immersion),
\item for every scheme $Z$ and any morphism $Z \to Y$ the morphism
$Z \times_Y X \to Z$ is a closed immersion (resp.\ open immersion,
resp.\ immersion),
\item for every affine scheme $Z$ and any morphism
$Z \to Y$ the morphism $Z \times_Y X \to Z$ is a closed immersion
(resp.\ open immersion, resp.\ immersion), and
\item there exists a scheme $V$ and a surjective etale morphism
$V \to Y$ such that $V \times_Y X \to V$ is a closed immersion
(resp.\ open immersion, resp.\ immersion).
\end{enumerate}
\end{lemma}

\begin{proof}
Using that a base change of a
closed immersion (resp.\ open immersion, resp.\ immersion)
is another one it is clear that (1) implies (2) and (2) implies (3).
Also (3) implies (4) since we can take $V$ to be a disjoint union of
affines, see
Properties of Spaces,
Lemma \ref{spaces-properties-lemma-cover-by-union-affines}.

\medskip\noindent
Assume $V \to X$ is as in (4).
Let $\mathcal{P}$ be the property
closed immersion (resp.\ open immersion, resp.\ immersion)
of morphisms of schemes. Note that property $\mathcal{P}$
is preserved under any base change and fppf local on the
base (see Section \ref{section-representable}).
Moreover, morphisms of type $\mathcal{P}$ are separated and
locally quasi-finite (in each of the three cases, see
Schemes, Lemma \ref{schemes-lemma-immersions-monomorphisms}, and
Morphisms, Lemma \ref{morphisms-lemma-immersion-locally-quasi-finite}).
Hence
by
More on Morphisms, Lemma
\ref{more-morphisms-lemma-separated-locally-quasi-finite-morphisms-fppf-descend}
the morphisms of type $\mathcal{P}$ satisfy descent for fppf covering. Thus
Spaces, Lemma \ref{spaces-lemma-morphism-sheaves-with-P-effective-descent-etale}
applies and we see that $Z \to X$ is representable and has property
$\mathcal{P}$, in other words (1) holds.
\end{proof}



\section{Quasi-compact morphisms}
\label{section-quasi-compact}

\noindent
By Section \ref{section-representable} we know what it means for
a representable morphism of algebraic spaces to be quasi-compact.
In order to formulate the definition for a general morphism
of algebraic spaces we make the following observation.

\begin{lemma}
\label{lemma-characterize-representable-quasi-compact}
Let $S$ be a scheme.
Let $f : X \to Y$ be a representable morphism of algebraic spaces over $S$.
The following are equivalent:
\begin{enumerate}
\item $f$ is quasi-compact, and
\item for every quasi-compact algebraic space $Z$ and any morphism
$Z \to Y$ the algebraic space $Z \times_Y X$ is quasi-compact.
\end{enumerate}
\end{lemma}

\begin{proof}
Assume (1), and let $Z \to Y$ be a morphism of algebraic spaces with
$Z$ quasi-compact. By
Properties of Spaces,
Definition \ref{spaces-properties-definition-quasi-compact}
there exists a quasi-compact scheme $U$ and a surjective etale
morphism $U \to Z$. Since $f$ is representable and quasi-compact
we see by definition that $U \times_Y X$ is a scheme, and that
$U \times_Y X \to U$ is quasi-compact. Hence $U \times_Y X$ is
a quasi-compact scheme. The morphism $U \times_Y X \to Z \times_Y X$
is etale and surjective (as the base change of the representable
etale and surjective morphism $U \to Z$, see
Section \ref{section-representable}).
Hence by definition $Z \times_Y X$ is quasi-compact.

\medskip\noindent
Assume (2). Let $Z \to Y$ be a morphism, where $Z$ is a scheme.
We have to show that $p : Z \times_Y X \to Z$ is quasi-compact.
Let $U \subset Z$ be affine open. Then $p^{-1}(U) = U \times_Y Z$
and the scheme $U \times_Y Z$ is quasi-compact by assumption (2).
Hence $p$ is quasi-compact, see
Schemes, Section \ref{schemes-section-quasi-compact}.
\end{proof}

\noindent
This motivates the following definition.

\begin{definition}
\label{definition-quasi-compact}
Let $S$ be a scheme.
Let $f : X \to Y$ be a morphism of algebraic spaces over $S$.
We say $f$ is {\it quasi-compact} if for every quasi-compact
algebraic space $Z$ and morphism $Z \to Y$ the fibre product
$Z \times_Y X$ is quasi-compact.
\end{definition}

\noindent
By Lemma \ref{lemma-characterize-representable-quasi-compact}
above this agrees with the already existing notion
for representable morphisms of algebraic spaces.

\begin{lemma}
\label{lemma-base-change-quasi-compact}
The base change of a quasi-compact morphism of algebraic spaces
by any morphism of algebraic spaces is quasi-compact.
\end{lemma}

\begin{proof}
Omitted.
\end{proof}

\begin{lemma}
\label{lemma-composition-quasi-compact}
The composition of a pair of quasi-compact morphisms of algebraic spaces
is quasi-compact.
\end{lemma}

\begin{proof}
Omitted.
\end{proof}

\begin{lemma}
\label{lemma-characterize-quasi-compact}
Let $S$ be a scheme.
Let $f : X \to Y$ be a morphism of algebraic spaces over $S$.
The following are equivalent:
\begin{enumerate}
\item $f$ is quasi-compact,
\item for every scheme $Z$ and any morphism $Z \to Y$ the morphism of
algebraic spaces $Z \times_Y X \to Z$ is quasi-compact,
\item for every affine scheme $Z$ and any morphism
$Z \to Y$ the algebraic space $Z \times_Y X$ is quasi-compact, and
\item there exists a scheme $V$ and a surjective etale morphism
$V \to Y$ such that $V \times_Y X \to V$ is a quasi-compact morphism
of algebraic spaces.
\end{enumerate}
\end{lemma}

\begin{proof}
We will use Lemma \ref{lemma-base-change-quasi-compact}
without further mention.
It is clear that (1) implies (2) and that (2) implies (3).
Assume (3). Let $Z$ be a quasi-compact algebraic space over $S$,
and let $Z \to Y$ be a morphism. By
Properties of Spaces, Lemma
\ref{spaces-properties-lemma-quasi-compact-affine-cover}
there exists an affine scheme $U$ and a surjective etale morphism
$U \to Z$. Then $U \times_Y X \to Z \times_Y X$ is surjective and
etale. Hence by
Properties of Spaces, Lemma
\ref{spaces-properties-lemma-characterize-surjective}
the map $|U \times_Y X| \to |Z \times_Y X|$ is surjective.
By assumption $|U \times_Y X|$ is quasi-compact, so
we conclude that $|Z \times_Y X|$ is quasi-compact.
This proves that (3) implies (1).

\medskip\noindent
That (1) implies (4) is clear. Let $V \to Y$ be as in
condition (4). Let $Z$ be an
affine scheme, and let $Z \to Y$ be a morphism. Consider the diagram
$$
\xymatrix{
Z \times_Y V \ar[r] \ar[d] & V \ar[d] \\
Z \ar[r] & Y
}
$$
As the left vertical arrow is etale, hence open,
we can find a quasi-compact open $W \subset V \times_Y Z$ which
surjects onto $Z$. By assumption the space $W \times_V (V \times_Y X)$
is quasi-compact. Since $W \times_V (V \times_Y X) \to Z \times_Y X$
is surjective we conclude that $|Z \times_Y X|$ is quasi-compact as
before. Hence (3) holds.
\end{proof}




\section{Universally closed morphisms}
\label{section-universally-closed}

\noindent
For a representable morphism of algebraic spaces we have already defined (in
Section \ref{section-representable})
what it means to be universally closed. Hence before we give the natural
definition we check that it agrees with this in the representable case.

\begin{lemma}
\label{lemma-characterize-representable-universally-closed}
Let $S$ be a scheme. Let $f : X \to Y$ be a representable morphism of
algebraic spaces over $S$. The following are equivalent
\begin{enumerate}
\item $f$ is universally closed, and
\item for every morphism of algebraic spaces $Z \to Y$ the morphism of
topological spaces $|Z \times_Y X| \to |Z|$ is closed.
\end{enumerate}
\end{lemma}

\begin{proof}
Assume (1), and let $Z \to Y$ be as in (2). Choose a scheme $V$ and
a surjective etale morphism $V \to Y$. By assumption the morphism
of schemes $V \times_Y X \to V$ is universally closed. By
Properties of Spaces, Section \ref{spaces-properties-section-points}
in the commutative diagram
$$
\xymatrix{
|V \times_Y X| \ar[r] \ar[d] & |Z \times_Y X| \ar[d] \\
|V| \ar[r] & |Z|
}
$$
the horizontal arrows are open and surjective, and moreover
$$
|V \times_Y X| \longrightarrow |V| \times_{|Z|} |Z \times_Y X|
$$
is surjective. Hence as the left
vertical arrow is closed it follows that the right vertical arrow is
closed. This proves (2). The implication (2) $\Rightarrow$ (1) is
immediate from the definitions.
\end{proof}

\noindent
Thus we may use the following natural definition.

\begin{definition}
\label{definition-closed}
Let $S$ be a scheme. Let $f : X \to Y$ be a morphism of algebraic spaces
over $S$.
\begin{enumerate}
\item We say $f$ is {\it closed} if the map of topological
spaces $|X| \to |Y|$ is closed.
\item We say $f$ is {\it universally closed} if for every morphism
of algebraic spaces $Z \to Y$ the morphism of topological spaces
$$
|Z \times_Y X| \to |Z|
$$
is closed, i.e., the base change $Z \times_Y X \to Z$ is closed.
\end{enumerate}
\end{definition}

\begin{lemma}
\label{lemma-base-change-universally-closed}
The base change of a universally closed morphism of algebraic spaces
by any morphism of algebraic spaces is universally closed.
\end{lemma}

\begin{proof}
This is immediate from the definition.
\end{proof}

\begin{lemma}
\label{lemma-composition-universally-closed}
The composition of a pair of (universally) closed morphisms of algebraic spaces
is (universally) closed.
\end{lemma}

\begin{proof}
Omitted.
\end{proof}

\begin{lemma}
\label{lemma-characterize-universally-closed}
Let $S$ be a scheme. Let $f : X \to Y$ be a morphism of algebraic spaces
over $S$. The following are equivalent
\begin{enumerate}
\item $f$ is universally closed,
\item for every scheme $Z$ and every morphism $Z \to Y$
the projection $|Z \times_Y X| \to |Z|$ is closed,
\item for every affine scheme $Z$ and every morphism $Z \to Y$
the projection $|Z \times_Y X| \to |Z|$ is closed, and
\item there exists a scheme $V$ and a surjective etale morphism
$V \to Y$ such that $V \times_Y X \to V$ is a universally closed morpism
of algebraic spaces.
\end{enumerate}
\end{lemma}

\begin{proof}
We omit the proof that (1) implies (2), and that (2) implies (3).

\medskip\noindent
Assume (3). Choose a surjective etale morphism $V \to Y$.
We are going to show that $V \times_Y X \to V$ is a universally
closed morpism of algebraic spaces. Let $Z \to V$ be a morphism
from an algebraic space to $V$. Let $W \to Z$ be a surjective etale
morphism where $W = \coprod W_i$ is a disjoint union of affine schemes, see
Properties of Spaces,
Lemma \ref{spaces-properties-lemma-cover-by-union-affines}.
Then we have the following commutative diagram
$$
\xymatrix{
\coprod_i |W_i \times_Y X| \ar@{=}[r] \ar[d] &
|W \times_Y X| \ar[r] \ar[d] &
|Z \times_Y X| \ar[d] \ar@{=}[r] &
|Z \times_V (V \times_Y X)| \ar[ld] \\
\coprod |W_i| \ar@{=}[r] &
|W| \ar[r] &
|Z|
}
$$
We have to show the south-east arrow is closed. The middle horizontal
arrows are surjective and open
(Properties of Spaces, Lemma \ref{spaces-properties-lemma-etale-open}).
By assumption (3), and the fact that
$W_i$ is affine we see that the left vertical arrows are closed. Hence
it follows that the right vertical arrow is closed.

\medskip\noindent
Assume (4). We will show that $f$ is universally closed.
Let $Z \to Y$ be a morphism of algebraic spaces. Consider the
diagram
$$
\xymatrix{
|(V \times_Y Z) \times_V (V \times_Y X)| \ar@{=}[r] \ar[rd] &
|V \times_Y X| \ar[r] \ar[d] &
|Z \times_Y X| \ar[d] \\
 &
|V \times_Y Z| \ar[r] &
|Z|
}
$$
The south-west arrow is closed by assumption. The horizontal arrows are
surjective and open because the corresponding morphisms of
algebraic spaces are etale (see
Properties of spaces, Lemma \ref{spaces-properties-lemma-etale-open}).
It follows that the right vertical arrow is closed.
\end{proof}

\begin{example}
\label{example-strange-universally-closed}
Strange example of a universally closed morphism.
Let $\mathbf{Q} \subset k$ be a field of characteristic zero.
Let $X = [\mathbf{A}^1_k/\mathbf{Z}]$ as in
Spaces, Example \ref{spaces-example-affine-line-translation}.
We claim the structure morphism $p : X \to \text{Spec}(k)$
is universally closed.
Namely, if $Z/k$ is a scheme, and $T \subset |X \times_k Z|$ is closed,
then $T$ corresponds to a $\mathbf{Z}$-invariant closed subset of
$T' \subset |\mathbf{A}^1 \times Z|$. It is easy to see that
this implies that $T'$ is the inverse image of a subset $T''$ of
$Z$. By
Morphisms, Lemma \ref{morphisms-lemma-fpqc-quotient-topology}
we have that $T'' \subset Z$ is closed.
Of course $T''$ is the image of $T$. Hence $p$ is universally
closed by Lemma \ref{lemma-characterize-universally-closed}.
\end{example}




\section{Valuative criteria}
\label{section-valuative}

\noindent
The formulation of the existence part of the valuative criterion is
slightly different for morphisms of algebraic spaces, since it may be
necessary to extend the fraction field of the valuation ring.
See Example, \ref{example-finite-separable-needed}. Here is the definition.

\begin{definition}
\label{definition-valuative-criterion}
Let $S$ be a scheme.
Let $f : X \to Y$ be a morphism of algebraic spaces over $S$.
We say $f$ {\it satisfies the uniqueness part of the valuative criterion}
if given any commutative solid diagram
$$
\xymatrix{
\text{Spec}(K) \ar[r] \ar[d] & X \ar[d] \\
\text{Spec}(A) \ar[r] \ar@{-->}[ru] & Y
}
$$
where $A$ is a valuation ring with field of fractions $K$, there exists
at most one dotted arrow (without requiring existence).
We say $f$ {\it satisfies the existence part of the valuative criterion}
if given any solid diagram as above there exists an extension
$K \subset K'$ of fields, a valuation ring $A' \subset K'$ dominating
$A$ and a morphism $\text{Spec}(A') \to X$ such that the following
diagram commutes
$$
\xymatrix{
\text{Spec}(K') \ar[r] \ar[d] & \text{Spec}(K) \ar[r] & X \ar[d] \\
\text{Spec}(A') \ar[r] \ar[rru] &\text{Spec}(A) \ar[r] & Y
}
$$
We say $f$ {\it satisfies the valuative criterion}
if $f$ satisfies both the existence and uniqueness part.
\end{definition}

\noindent
It turns out that for algebraic spaces, it always sufffices to
take a finite separable extension $K \subset K'$ above.
See Lemma \ref{lemma-finite-separable-enough}.
Before we prove it we show that the criterion
is identical to the criterion as formulated for morphisms of schemes
in case the morphism of algebraic spaces is representable.

\begin{lemma}
\label{lemma-valuative-criterion-representable}
Let $S$ be a scheme.
Let $f : X \to Y$ be a morphism of algebraic spaces over $S$.
Assume $f$ is representable. The following are equivalent
\begin{enumerate}
\item $f$ satisfies the existence part of the valuation criterion
as in Definition \ref{definition-valuative-criterion} above, and
\item given any commutative solid diagram
$$
\xymatrix{
\text{Spec}(K) \ar[r] \ar[d] & X \ar[d] \\
\text{Spec}(A) \ar[r] \ar@{-->}[ru] & Y
}
$$
where $A$ is a valuation ring with field of fractions $K$, there exists
a dotted arrow, i.e., $f$ satisfies the existence part of the valuative
criterion as in
Schemes, Definition \ref{schemes-definition-valuative-criterion}.
\end{enumerate}
\end{lemma}

\begin{proof}
It suffices to show that given a commutative diagram of the form
$$
\xymatrix{
\text{Spec}(K') \ar[r] \ar[d] & \text{Spec}(K) \ar[r] & X \ar[d] \\
\text{Spec}(A') \ar[r] \ar[rru]^\varphi &\text{Spec}(A) \ar[r] & Y
}
$$
as in Definition \ref{definition-valuative-criterion}, then we can
find a morphism $\text{Spec}(A) \to X$ fitting into the diagram too.
Set $X_A = \text{Spec}(A) \times_Y Y$. As $f$ is representable we see
that $X_A$ is a scheme. The morphism $\varphi$ gives a morphism
$\varphi' : \text{Spec}(A') \to X_A$. Let $x \in X_A$ be the image of
the closed point of $\varphi' : \text{Spec}(A') \to X_A$. Then we
have the following commutative diagram of rings
$$
\xymatrix{
K' & K \ar[l] & \mathcal{O}_{X_A, x} \ar[l] \ar[lld] \\
A' \ar[u] & A \ar[l] & A \ar[l] \ar[u]
}
$$
Since $A$ is a valuation ring, and since $A'$ dominates $A$, we see
that $K \cap A' = A$. Hence the ring map $\mathcal{O}_{X_A, x} \to K$
has image contained in $A$. Whence a morphism $\text{Spec}(A) \to X_A$ (see
Schemes, Section \ref{schemes-section-points})
as desired.
\end{proof}

\begin{lemma}
\label{lemma-finite-separable-enough}
Let $S$ be a scheme.
Let $f : X \to Y$ be a morphism of algebraic spaces over $S$.
The following are equivalent
\begin{enumerate}
\item $f$ satisfies the existence part of the valuation criterion
as in Definition \ref{definition-valuative-criterion}, and
\item $f$ satisfies the existence part of the valuation criterion
as in Definition \ref{definition-valuative-criterion} modified by
requiring the extension $K \subset K'$ to be finite separable.
\end{enumerate}
\end{lemma}

\begin{proof}
We have to show that (1) implies (2). Suppose given a diagram
$$
\xymatrix{
\text{Spec}(K') \ar[r] \ar[d] & \text{Spec}(K) \ar[r] & X \ar[d] \\
\text{Spec}(A') \ar[r] \ar[rru] &\text{Spec}(A) \ar[r] & Y
}
$$
as in Definition \ref{definition-valuative-criterion} with $K \subset K'$
arbitrary. Choose a scheme $U$ and a surjective etale morphism $U \to X$.
Then
$$
\text{Spec}(A') \times_X U \longrightarrow \text{Spec}(A')
$$
is surjective etale. Let $p$ be a point of $\text{Spec}(A') \times_X U$
mapping to the closed point of $\text{Spec}(A')$. Let $p' \leadsto p$
be a generalization of $p$ mapping to the generic point of $\text{Spec}(A')$.
Such a generalization exists because generalizations lift along flat
morphisms of schemes, see
Morphisms, Lemma \ref{morphisms-lemma-generalizations-lift-flat}.
Then $p'$ corresponds to a point of the scheme $\text{Spec}(K') \times_X U$.
Note that
$$
\text{Spec}(K') \times_X U
=
\text{Spec}(K') \times_{\text{Spec}(K)} (\text{Spec}(K) \times_X U)
$$
Hence $p'$ maps to a point $q' \in \text{Spec}(K) \times_X U$ whose
residue field is a finite separable extension of $K$. Finally,
$p' \leadsto p$ maps to a specialization $u' \leadsto u$ on the
scheme $U$. With all this notation we get the following diagram of
rings
$$
\xymatrix{
\kappa(p') & & \kappa(q') \ar[ll] & \kappa(u') \ar[l] \\
& \mathcal{O}_{\text{Spec}(A') \times_X U, p} \ar[lu] & &
\mathcal{O}_{U, u} \ar[ll] \ar[u] \\
K' \ar[uu] & A' \ar[l] \ar[u] & A \ar[l] \ar'[u][uu]
}
$$
This means that the ring $B \subset \kappa(q')$ generated by
the images of $A$ and $\mathcal{O}_{U, u}$ maps to a subring
of $\kappa(p')$ contained in the image $B'$ of
$\mathcal{O}_{\text{Spec}(A') \times_X U, p} \to \kappa(p')$.
Note that $B'$ is a local ring. Let $\mathfrak m \subset B$
be the maximal ideal. By construction $A \cap \mathfrak m$,
(resp.\ $\mathcal{O}_{U, u} \cap \mathfrak m$, resp.\ $A' \cap \mathfrak m$)
is the maximal ideal of $A$ (resp.\ $\mathcal{O}_{U, u}$, resp.\ $A'$).
Set $\mathfrak q = B \cap \mathfrak m$. This is a
prime ideal such that $A \cap \mathfrak q$ is the maximal ideal of $A$.
Hence $B_{\mathfrak q} \subset \kappa(q')$ is a local ring dominating
$A$. By
Algebra, Lemma \ref{algebra-lemma-dominate}
we can find a valuation ring $A_1 \subset \kappa(q')$
with field of fractions $\kappa(q')$
dominating $B_{\mathfrak q}$. The (local) ring map
$\mathcal{O}_{U, u} \to A_1$ gives a morphism
$\text{Spec}(A_1) \to U \to X$
such that the diagram
$$
\xymatrix{
\text{Spec}(\kappa(q')) \ar[r] \ar[d] & \text{Spec}(K) \ar[r] & X \ar[d] \\
\text{Spec}(A_1) \ar[r] \ar[rru] &\text{Spec}(A) \ar[r] & Y
}
$$
is commutative. Since $f.f.(A_1) = \kappa(q') \supset K$ is finite
separable by construction the lemma is proved.
\end{proof}

\begin{example}
\label{example-finite-separable-needed}
Consider the algebraic space $X$ constructed in
Spaces, Example \ref{spaces-example-non-representable-descent}.
Recall that it is the affine line with zero doubled in a Galois twisted
relative to a degree two Galois extension $k \subset k'$.
As such it comes with a morphism
$$
\pi : X \longrightarrow S = \mathbf{A}^1_k
$$
which is quasi-compact. We claim that $\pi$ is universally closed.
Namely, after base change by $\text{Spec}(k') \to \text{Spec}(k)$
the morphism $\pi$ is identified with the morphism
$$
\text{affine line with zero doubled}
\longrightarrow
\text{affine line}
$$
which is universally closed (some details omitted). Since the morphism
$\text{Spec}(k') \to \text{Spec}(k)$ is universally closed and
surjective, a diagram chase shows that $\pi$ is universally closed.
On the other hand, consider the diagram
$$
\xymatrix{
\text{Spec}(k((x))) \ar[r] \ar[d] & X \ar[d]^\pi \\
\text{Spec}(k[[x]]) \ar[r] \ar@{..>}[ru] & \mathbf{A}^1_k
}
$$
Since the unique point of $X$ above $0 \in \mathbf{A}^1_k$
corresponds to a monomorphism $\text{Spec}(k') \to X$
it is clear there cannot exist a dotted arrow! This shows that
a finite separable field extension is needed in general.
\end{example}

\begin{lemma}
\label{lemma-base-change-valuative-criteria}
The base change of a morphism of algebraic spaces which satisfies the
existence part of (resp.\ uniqueness part of) the valuative criterion
by any morphism of algebraic spaces satisfies the
existence part of (resp.\ uniqueness part of) the valuative criterion.
\end{lemma}

\begin{proof}
Let $f : X \to Y$ be a morphism of algebraic spaces over the scheme $S$.
Let $Z \to Y$ be any morphism of algebraic spaces over $S$.
Consider a solid commutative diagram of the following shape
$$
\xymatrix{
\text{Spec}(K) \ar[r] \ar[d] & Z \times_Y X \ar[r] \ar[d] & X \ar[d] \\
\text{Spec}(A) \ar[r] \ar@{-->}[ru] \ar@{-->}[rru] & Z \ar[r] & Y
}
$$
Then the set of north-west dotted arrows making the diagram commute
is in 1-1 correspondence with the set of west-north-west dotted arrows
making the diagram commute. This proves the lemma in the case of
``uniqueness''. For the existence part, assume $f$ satisfies the existence
part of the valuative criterion. If we are given a solid commutative
diagram as above, then by assumption there exists an extension $K \subset K'$
of fields and a valuation ring $A' \subset K'$ dominating $A$ and
a morphism $\text{Spec}(A') \to X$ fitting into the following commutative
diagram
$$
\xymatrix{
\text{Spec}(K') \ar[r] \ar[d] &
\text{Spec}(K) \ar[r] & Z \times_Y X \ar[r] & X \ar[d] \\
\text{Spec}(A') \ar[r] \ar[rrru] & \text{Spec}(A) \ar[r] & Z \ar[r] & Y
}
$$
And by the remarks above the skew arrow corresponds to an arrow
$\text{Spec}(A') \to Z \times_Y X$ as desired.
\end{proof}

\begin{lemma}
\label{lemma-composition-valuative-criteria}
The composition of two morphisms of algebraic spaces which satisfy the
(existence part of, resp.\ uniqueness part of) the valuative criterion
satisfies the (existence part of, resp.\ uniqueness part of) the valuative
criterion.
\end{lemma}

\begin{proof}
Let $f : X \to Y$, $g : Y \to Z$ be morphisms of algebraic spaces over the
scheme $S$. Consider a solid commutative diagram of the following shape
$$
\xymatrix{
\text{Spec}(K) \ar[dd] \ar[r] & X \ar[d]^f \\
& Y \ar[d]^g \\
\text{Spec}(A) \ar[r] \ar@{-->}[ru] \ar@{-->}[ruu] & Z
}
$$
If we have the uniqueness part for $g$, then there exists at
most one north-west dotted arrow making the diagram commute.
If we also have the uniqueness part for $f$, then we have
at most one north-north-west dotted arrow making the diagram
commute. The proof in the existence case comes from contemplating
the following diagram
$$
\xymatrix{
\text{Spec}(K'') \ar[r] \ar[dd] &
\text{Spec}(K') \ar[r] &
\text{Spec}(K) \ar[r] &
X \ar[d]^f \\
& & & Y \ar[d]^g \\
\text{Spec}(A'') \ar[r] \ar[rrruu] &
\text{Spec}(A') \ar[r] \ar[rru] &
\text{Spec}(A) \ar[r] &
Z
}
$$
Namely, the existence part for $g$ gives us the extension $K'$, the
valuation ring $A'$ and the arrow $\text{Spec}(A') \to Y$, whereupon
the existence part for $f$ gives us the extension $K''$, the
valuation ring $A''$ and the arrow $\text{Spec}(A'') \to X$.
\end{proof}






\section{Valuative criterion for universal closedness}
\label{section-valuative-criterion-universally-closed}

\noindent
This is a little more involved than in the case of schemes, especially
since the most optimistic guess is wrong. See discussion below.

\begin{lemma}
\label{lemma-quasi-compact-existence-universally-closed}
Let $S$ be a scheme.
Let $f : X \to Y$ be a morphism of algebraic spaces over $S$.
Assume
\begin{enumerate}
\item $f$ is quasi-compact, and
\item $f$ satisfies the existence part of the valuative criterion.
\end{enumerate}
Then $f$ is universally closed.
\end{lemma}

\begin{proof}
By Lemmas \ref{lemma-base-change-quasi-compact}
and \ref{lemma-base-change-valuative-criteria}
properties (1) and (2) are preserved under
any base change. By Lemma \ref{lemma-characterize-universally-closed}
we only have to show that $T \times_Y X \to T$ is universally closed,
whenever $T$ is an affine scheme over $S$ mapping into $Y$. Hence it
suffices to prove: If $Y$ is an affine scheme, $f : X \to Y$ is quasi-compact
and satisfies the existence part of the valuative criterion, then
$f : |X| \to |Y|$ is closed. In this situation $X$ is a quasi-compact
algebraic space. By
Properties of Spaces,
Lemma \ref{spaces-properties-lemma-quasi-compact-affine-cover}
there exists an affine scheme $U$ and a surjective etale morphism
$\varphi : U \to X$. Let $T \subset |X|$ closed. The inverse image
$\varphi^{-1}(T) \subset U$ is closed, and hence is the set of points
of an affine closed subscheme $Z \subset U$. Thus, by
Algebra, Lemma \ref{algebra-lemma-image-stable-specialization-closed}
we see that $f(T) = f(\varphi(|Z|)) \subset |Y|$ is closed if it is
closed under specialization.

\medskip\noindent
Let $y' \leadsto y$ be a specialization in $Y$ with $y' \in f(T)$.
Choose a point $x' \in T \subset |X|$ mapping to $y'$ under $f$.
We may represent $x'$ by a morphism $\text{Spec}(K) \to X$
for some field $K$. Thus we have the following diagram
$$
\xymatrix{
\text{Spec}(K) \ar[r]_-{x'} \ar[d] & X \ar[d]^f \\
\text{Spec}(\mathcal{O}_{Y, y}) \ar[r] & Y,
}
$$
see
Schemes, Section \ref{schemes-section-points}
for the existence of the left vertical map.
Choose a valuation ring $A \subset K$ dominating the image of
the ring map $\mathcal{O}_{Y, y} \to K$ (this is possible since
the image is a local ring and not a field as $y' \not = y$, see
Algebra, Lemma \ref{algebra-lemma-dominate}).
By assumption there exists a field extension $K \subset K'$ and a
valuation ring $A' \subset K'$ dominating $A$, and a morphism
$\text{Spec}(A') \to X$ fitting into the commutative diagram.
Since $A'$ dominates $A$, and $A$ dominates $\mathcal{O}_{Y, y}$
we see that the closed point of $\text{Spec}(A')$ maps to
a point $x \in X$ with $f(x) = y$ which is a specialization of $x'$.
Hence $x \in T$ as $T$ is closed, and hence $y \in f(T)$ as desired.
\end{proof}

\noindent
We also want to prove the converse of
Lemma \ref{lemma-quasi-compact-existence-universally-closed}.
Namely, we would like to show, under additional conditions, that
a quasi-compact morphism is universallly closed if and only if
the existence part of the valuative criterion holds.
Example \ref{example-strange-universally-closed} shows that
$[\mathbf{A}^1_k/\mathbf{Z}] \to \text{Spec}(k)$ is universally
closed, but it is easy to see that the existence part of
the valuative criterion fails. Hence some
additional hypothesis is needed. (See also
Lemma \ref{lemma-re-characterize-universally-closed}
for a slight weakening of the hypothesis.)

\begin{proposition}
\label{proposition-characterize-universally-closed}
Let $S$ be a scheme.
Let $f : X \to Y$ be a morphism of algebraic spaces over $S$.
Assume
\begin{enumerate}
\item $f$ is quasi-compact, and
\item $X$ has property $(\gamma)$ of
Properties of Spaces, Lemma \ref{spaces-properties-lemma-bounded-fibres}.
\end{enumerate}
Then $f$ is universally closed if and only if the
existence part of the valuative criterion holds.
\end{proposition}

\begin{proof}
In Lemma \ref{lemma-quasi-compact-existence-universally-closed}
we have seen one of the implications.
To prove the other, assume that $f$ is universally closed. Let
$$
\xymatrix{
\text{Spec}(K) \ar[r] \ar[d] & X \ar[d] \\
\text{Spec}(A) \ar[r] & Y
}
$$
be a diagram as in
Definition \ref{definition-valuative-criterion}.
Let $X_A = \text{Spec}(A) \times_Y X$, so that we have
$$
\xymatrix{
\text{Spec}(K) \ar[r] \ar[rd] & X_A \ar[d] \\
 & \text{Spec}(A)
}
$$
By Lemma \ref{lemma-base-change-quasi-compact} we see that
$X_A \to \text{Spec}(A)$ is quasi-compact. Since $X_A \to X$
is representable, we see that $X_A$ has property $(\gamma)$ also, see
Properties of Spaces,
Lemma \ref{spaces-properties-lemma-representable-properties}.
Moreover, as $f$ is universally closed, we see that $X_A \to \text{Spec}(A)$
is universally closed.
Hence we may and do replace $X$ by $X_A$ and $Y$ by $\text{Spec}(A)$.

\medskip\noindent
Let $x' \in |X|$ be the equivalence class of
$\text{Spec}(K) \to X$. Let $y \in |Y| = |\text{Spec}(A)|$ be
the closed point. Set $y' = f(x')$; it is the generic point of
$\text{Spec}(A)$. Since $f$ is universally closed we see that
$f(\overline{\{x'\}})$ contains $\overline{\{y'\}}$, and hence
contains $y$. Let $x \in \overline{\{x'\}}$ be a point such that
$f(x) = y$. Let $U$ be a scheme, and $\varphi : U \to X$
an etale morphism such that there exists a $u \in U$ with
$\varphi(u) = x$. By
Properties of Spaces, Lemma \ref{spaces-properties-lemma-specialization}
and our assumption that $X$ has property $(\gamma)$
there exists a specialization $u' \leadsto u$ on $U$ with $\varphi(u') = x'$.
This means that there exists a common field extension
$K \subset K' \supset \kappa(u')$ such that
$$
\xymatrix{
\text{Spec}(K') \ar[r] \ar[d] & U \ar[d] \\
\text{Spec}(K) \ar[r] \ar[rd] & X \ar[d] \\
 & \text{Spec}(A)
}
$$
is commutative. This gives the following commutative diagram of rings
$$
\xymatrix{
K' & \mathcal{O}_{U, u} \ar[l] \\
K \ar[u] & \\
 & A \ar[lu] \ar[uu]
}
$$
By
Algebra, Lemma \ref{algebra-lemma-dominate}
we can find a valuation ring $A' \subset K'$ dominating the image of
$\mathcal{O}_{U, u}$ in $K'$. Since by construction $\mathcal{O}_{U, u}$
dominates $A$ we see that $A'$ dominates $A$ also. Hence we obtain a diagram
resembling the second diagram of
Definition \ref{definition-valuative-criterion}
and the proposition is proved.
\end{proof}













\section{Separation axioms}
\label{section-separation-axioms}

\noindent
It makes sense to list some a priori properties of the diagonal of
a morphism of algebraic spaces.

\begin{lemma}
\label{lemma-properties-diagonal}
Let $S$ be a scheme contained in $\textit{Sch}_{fppf}$.
Let $f : X \to Y$ be a morphism of algebraic spaces over $S$.
Let $\Delta_{X/Y} : X \to X \times_Y X$ be the diagonal morphism.
Then
\begin{enumerate}
\item $\Delta_{X/Y}$ is representable,
\item $\Delta_{X/Y}$ is locally of finite type,
\item $\Delta_{X/Y}$ is a monomorphism,
\item $\Delta_{X/Y}$ is separated, and
\item $\Delta_{X/Y}$ is locally quasi-finite.
\end{enumerate}
\end{lemma}

\begin{proof}
We are going to use the fact that $\Delta_{X/S}$ is
representable (by definition of an algebraic space) and that
it satisfies properties (2) -- (5), see
Spaces, Lemma \ref{spaces-lemma-properties-diagonal}.
Note that we have a factorization
$$
X
\longrightarrow
X \times_Y X
\longrightarrow
X \times_S X
$$
of the diagonal $\Delta_{X/S} : X \to X \times_S X$. Since
$X \times_Y X \to X \times_S X$ is a monomorphism, and since
$\Delta_{X/S}$ is representable, it follows formally that
$\Delta_{X/Y}$ is representable. In particular, the rest of
the statements now make sense, by the discussion in
Section \ref{section-representable}.

\medskip\noindent
Choose a surjective etale morphism $U \to X$, with $U$ a scheme.
Consider the diagram
$$
\xymatrix{
R = U \times_X U \ar[r] \ar[d] &
U \times_Y U \ar[d] \ar[r] &
U \times_S U \ar[d] \\
X \ar[r] & X \times_Y X \ar[r] & X \times_S X
}
$$
Both squares are cartesian, hence so is the outer retangle.
The top row consists of schemes, and the vertical arrows
are surjective etale morphisms. By
Spaces, Lemma \ref{spaces-lemma-representable-morphisms-spaces-property}
the properties (2) -- (5) for $\Delta_{X/Y}$ are equivalent to those of
$R \to U \times_Y U$. In the proof of
Spaces, Lemma \ref{spaces-lemma-properties-diagonal}
we have seen that $R \to U \times_S U$ has properties (2) -- (5).
The morphism $U \times_Y U \to U \times_S U$ is a monomorphism
of schemes. These facts imply that $R \to U \times_S U$ have
properties (2) -- (5).

\medskip\noindent
Namely: For (3), note that $R \to U \times_Y U$
is a monomorphism as the composition
$R \to U \times_S U$ is a monomorphism. For (2), note that
$R \to U \times_Y U$ is locally of finite type, as the
composition $R \to U \times_S U$ is locally of finite type
(Morphisms, Lemma \ref{morphisms-lemma-permanence-finite-type}).
A monomorphism which is locally of finite type is locally quasi-finite
because it has finite fibres
(Morphisms, Lemma \ref{morphisms-lemma-finite-fibre}), hence (5).
A monomorphism is separated
(Schemes, Lemma \ref{schemes-lemma-monomorphism-separated}), hence (4).
\end{proof}

\begin{definition}
\label{definition-separated}
Let $S$ be a scheme.
Let $f : X \to Y$ be a morphism of algebraic spaces over $S$.
Let $\Delta_{X/Y} : X \to X \times_Y X$ be the diagonal morphism.
\begin{enumerate}
\item We say $f$ is {\it separated} if $\Delta_{X/Y}$ is a closed immersion.
\item We say $f$ is {\it weakly locally separated}\footnote{This is probably
nonstandard notation.} if $\Delta_{X/Y}$ is an immersion.
\item We say $f$ is {\it locally separated} if $\Delta_{X/Y}$ is a
quasi-compact immersion.
\item We say $f$ is {\it quasi-separated} if $\Delta_{X/Y}$ is quasi-compact.
\end{enumerate}
\end{definition}

\noindent
This definition makes sense since $\Delta_{X/Y}$ is representable,
and hence we know what it means for it to have one of the properties
described in the definition. We will see below
(Lemma \ref{lemma-match-separated}) that this definition matches the ones
we already have for morphisms of schemes and representable morphisms.

\begin{lemma}
\label{lemma-trivial-implications}
Let $S$ be a scheme.
Let $f : X \to Y$ be a morphism of algebraic spaces over $S$.
We have the following implications among the separation axioms
of Definition \ref{definition-separated}:
\begin{enumerate}
\item $f$ separated implies all the others, and
\item $f$ locally separated implies $f$ quasi-separated and
$f$ weakly locally separated.
\end{enumerate}
\end{lemma}

\begin{proof}
Omitted.
\end{proof}

\begin{lemma}
\label{lemma-base-change-separated}
All of the separation axioms listed in Defintion \ref{definition-separated}
are stable under base change.
\end{lemma}

\begin{proof}
Let $f : X \to Y$ and $Y' \to Y$ be morphisms of algebraic spaces.
Let $f' : X' \to Y'$ be the base change of $f$ by $Y' \to Y$. Then
$\Delta_{X'/Y'}$ is the base change of $\Delta_{X/Y}$ by
the morphism $X' \times_{Y'} X' \to X \times_Y X$. By the results of
Section \ref{section-representable}
each of the properties of the diagonal used in
Definition \ref{definition-separated}
is stable under base change. Hence the lemma is true.
\end{proof}

\begin{lemma}
\label{lemma-fibre-product-after-map}
Let $S$ be a scheme. Let $f : X \to Z$, $g : Y \to Z$ and $Z \to T$
be morphisms of algebraic spaces over $S$. Consider the induced morphism
$i : X \times_Z Y \to X \times_T Y$. Then
\begin{enumerate}
\item $i$ is representable, locally of finite type, locally quasi-finite,
separated and a monomorphism,
\item if $Z \to T$ is weakly locally separated, then $i$ is an immersion,
\item if $Z \to T$ is locally separated, then $i$ is a quasi-compact immersion,
\item if $Z \to T$ is separated, then $i$ is a closed immersion, and
\item if $Z \to T$ is quasi-separated, then $i$ is quasi-compact.
\end{enumerate}
\end{lemma}

\begin{proof}
By general category theory the following diagram
$$
\xymatrix{
X \times_Z Y \ar[r]_i \ar[d] & X \times_T Y \ar[d] \\
Z \ar[r]^-{\Delta_{Z/T}} \ar[r] & Z \times_T Z
}
$$
is a fibre product diagram. Hence $i$ is the base change of the
diagonal morphism $\Delta_{Z/T}$. Thus the lemma follows
from Lemma \ref{lemma-properties-diagonal}, and the material in
Section \ref{section-representable}.
\end{proof}

\begin{lemma}
\label{lemma-semi-diagonal}
Let $S$ be a scheme. Let $T$ be an algebraic space over $S$.
Let $g : X \to Y$ be a morphism of algebraic spaces over $T$.
Consider the graph $i : X \to X \times_T Y$ of $g$. Then
\begin{enumerate}
\item $i$ is representable, locally of finite type, locally quasi-finite,
separated and a monomorphism,
\item if $Y \to T$ is weakly locally separated, then $i$ is an immersion,
\item if $Y \to T$ is locally separated, then $i$ is a quasi-compact immersion,
\item if $Y \to T$ is separated, then $i$ is a closed immersion, and
\item if $Y \to T$ is quasi-separated, then $i$ is quasi-compact.
\end{enumerate}
\end{lemma}

\begin{proof}
This is a special case of Lemma \ref{lemma-fibre-product-after-map}
applied to the morphism $X = X \times_Y Y \to X\times_T Y$.
\end{proof}

\begin{lemma}
\label{lemma-section-immersion}
Let $S$ be a schemes.
Let $f : X \to T$ be a morphism of algebraic spaces over $S$.
Let $s : T \to X$ be a section of $f$ (in a formula
$f \circ s = \text{id}_T$). Then
\begin{enumerate}
\item $s$ is representable, locally of finite type, locally quasi-finite,
separated and a monomorphism,
\item if $f$ is weakly locally separated, then $i$ is an immersion,
\item if $f$ is locally separated, then $i$ is a quasi-compact immersion,
\item if $f$ is separated, then $i$ is a closed immersion, and
\item if $f$ is quasi-separated, then $i$ is quasi-compact.
\end{enumerate}
\end{lemma}

\begin{proof}
This is a special case of Lemma \ref{lemma-semi-diagonal} applied to
$g = s$ so the morphism $i = s : T \to T \times_T X$.
\end{proof}

\begin{lemma}
\label{lemma-composition-separated}
All of the separation axioms listed in Defintion \ref{definition-separated}
are stable under composition of morphisms.
\end{lemma}

\begin{proof}
Let $f : X \to Y$ and $g : Y \to Z$ be morphisms of algebraic spaces
to which the axiom in question applies.
The diagonal $\Delta_{X/Z}$ is the composition
$$
X \longrightarrow X \times_Y X \longrightarrow X \times_Z X.
$$
Our separation axiom is defined by requiring the diagonal
to have some property $\mathcal{P}$. By
Lemma \ref{lemma-fibre-product-after-map} above we see that
the second arrow also has this property. Hence the lemma follows
since the composition of (representable) morphisms with property
$\mathcal{P}$ also is a morphism with property $\mathcal{P}$, see
Section \ref{section-representable}.
\end{proof}

\begin{lemma}
\label{lemma-compose-after-separated}
Let $S$ be a scheme.
Let $f : X \to Y$ and $g : Y \to Z$ be morphisms of algebraic spaces over $S$.
\begin{enumerate}
\item If $g \circ f$ is separated then so is $f$.
\item If $g \circ f$ is weakly locally separated then so is $f$.
\item If $g \circ f$ is locally separated then so is $f$.
\item If $g \circ f$ is quasi-separated then so is $f$.
\end{enumerate}
\end{lemma}

\begin{proof}
Consider the factorization
$$
X \to X \times_Y X \to X \times_Z X
$$
of the diagonal morphism of $g \circ f$. In any case the last morphism
is a monomorphism. Hence for any scheme $T$ and morphism
$T \to X \times_Y X$ we have the equality
$$
X \times_{(X \times_Y X)} T = X \times_{(X \times_Z X)} T.
$$
Hence the result is clear.
\end{proof}

\begin{lemma}
\label{lemma-characterize-separated}
Let $S$ be a scheme.
Let $f : X \to Y$ be a morphism of algebraic spaces over $S$.
Let $\mathcal{P}$ be any of the separation
axioms of Definition \ref{definition-separated}.
The following are equivalent
\begin{enumerate}
\item $f$ is $\mathcal{P}$,
\item for every scheme $Z$ and morphism $Z \to Y$ the
base change $Z \times_Y X \to Z$ of $f$ is $\mathcal{P}$,
\item for every affine scheme $Z$ and every morphism $Z \to Y$ the
base change $Z \times_Y X \to Z$ of $f$ is $\mathcal{P}$,
\item for every affine scheme $Z$ and every morphism $Z \to Y$ the
algebraic space $Z \times_Y X$ is $\mathcal{P}$ (see
Properties of Spaces, Definition \ref{spaces-properties-definition-separated}),
and
\item there exists a scheme $V$ and a surjective etale morphism
$V \to Y$ such that the base change $V \times_Y X \to V$ has
$\mathcal{P}$.
\end{enumerate}
\end{lemma}

\begin{proof}
We will repeatedly use
Lemma \ref{lemma-base-change-separated}
without further mention. In particular, it is clear that
(1) implies (2) and (2) implies (3).

\medskip\noindent
Let us prove that (3) and (4) are equivalent. Note that if $Z$ is an affine
scheme, then the morphism $Z \to \text{Spec}(\mathbf{Z})$ is a separated
morphism as a morphism of algebraic spaces over $\text{Spec}(\mathbf{Z})$.
If $Z \times_Y X \to Z$ is $\mathcal{P}$, then
$Z \times_Y X \to \text{Spec}(\mathbf{Z})$ is $\mathcal{P}$
as a composition (see
Lemma \ref{lemma-composition-separated}). Hence the algebraic
space $Z \times_Y X$ is $\mathcal{P}$. Conversely, if the algebraic
space $Z \times_Y X$ is $\mathcal{P}$, then
$Z \times_Y X \to \text{Spec}(\mathbf{Z})$ is $\mathcal{P}$, and
hence by
Lemma \ref{lemma-compose-after-separated}
we see that $Z \times_Y X \to Z$ is $\mathcal{P}$.

\medskip\noindent
Let us prove that (3) implies (5). Assume (3). Let $V$ be a scheme
and let $V \to Y$ be etale surjective. We have to show that
$V \times_Y X \to V$ has property $\mathcal{P}$. In other words,
we have to show that the morphism
$$
V \times_Y X \longrightarrow
(V \times_Y X) \times_V (V \times_Y X) = V \times_Y X \times_Y X
$$
has the corresponding property (i.e., is a closed immersion, immersion,
quasi-compact immersion, or quasi-compact). Let $V = \bigcup V_j$ be an
affine open covering of $V$. By assumption we know that each of the morphisms
$$
V_j \times_Y X \longrightarrow V_j \times_Y X \times_Y X
$$
does have the corresponding property. Since being a closed immersion,
immersion, quasi-compact immersion, or quasi-compact is Zariski local
on the target, and since the $V_j$ cover $V$ we get the desired conclusion.

\medskip\noindent
Let us prove that (5) implies (1). Let $V \to Y$ be as in (5).
Then we have the fibre product diagram
$$
\xymatrix{
V \times_Y X \ar[r] \ar[d] &
X \ar[d] \\
V \times_Y X \times_Y X \ar[r] &
X \times_Y X
}
$$
By assumption the left vertical arrow is a closed immersion,
immersion, quasi-compact immersion, or quasi-compact. It follows from
Spaces, Lemma \ref{spaces-lemma-descent-representable-transformations-property}
that also the right vertical arrow is a closed immersion,
immersion, quasi-compact immersion, or quasi-compact.
\end{proof}

\begin{lemma}
\label{lemma-match-separated}
Let $S$ be a scheme.
If $f : X \to Y$ is a representable morphism of algebraic spaces over $S$,
then $f$ is (quasi-)separated in the sense of
Definition \ref{definition-separated}
above if and only if $f$ is (quasi-)separated in the sense of
Section \ref{section-representable}. In particular,
if $f : X \to Y$ is a morphism of schemes over $S$, then
$f$ is (quasi-)separated in the sense of
Definition \ref{definition-separated}
if and only if $f$ is (quasi-)separated as a morphism of schemes.
\end{lemma}

\begin{proof}
This is the equivalence of (1) and (2) of
Lemma \ref{lemma-characterize-separated}.
\end{proof}

\begin{lemma}
\label{lemma-quasi-compact-permanence}
Let $S$ be a scheme.
Let $f : X \to Y$ and $g : Y \to Z$ be morphisms of algebraic spaces over $S$.
If $g \circ f$ is quasi-compact and $g$ is quasi-separated
then $f$ is quasi-compact.
\end{lemma}

\begin{proof}
This is true because $f$ equals the composition
$(1, f) : X \to X \times_Z Y \to Y$. The first map
is quasi-compact by Lemma \ref{lemma-section-immersion}
because it is a section of the quasi-separated morphism $X \times_Z Y \to X$
(a base change of $g$, see Lemma \ref{lemma-base-change-separated}).
The second map is quasi-compact as it
is the base change of $f$, see
Lemma \ref{lemma-base-change-quasi-compact}.
And compositions of quasi-compact
morphisms are quasi-compact, see Lemma \ref{lemma-composition-quasi-compact}.
\end{proof}






\section{Valuative criterion of separatedness}
\label{section-valuative-separatedness}

\begin{lemma}
\label{lemma-separated-implies-valuative}
Let $S$ be a scheme.
Let $f : X \to Y$ be a morphism of algebraic spaces over $S$.
If $f$ is separated, then $f$ satisfies the uniqueness
part of the valuative criterion.
\end{lemma}

\begin{proof}
Let a diagram as in Definition \ref{definition-valuative-criterion}
be given. Suppose there are two distinct morphisms
$a, b : \text{Spec}(A) \to X$ fitting into the diagram.
Let $Z \subset \text{Spec}(A)$ be the equalizer of $a$ and $b$.
Then $Z = \text{Spec}(A) \times_{(a, b), X \times_Y X, \Delta} X$.
If $f$ is separated, then $\Delta$ is a closed immersion, and
this is a closed subscheme of $\text{Spec}(A)$. By assumption it contains
the generic point of $\text{Spec}(A)$. Since $A$ is a domain
this implies $Z = \text{Spec}(A)$. Hence $a = b$ as desired.
\end{proof}

\begin{lemma}
\label{lemma-valuative-criterion-separatedness}
(Valuative criterion separatedness.)
Let $S$ be a scheme.
Let $f : X \to Y$ be a morphism of algebraic spaces over $S$.
Assume
\begin{enumerate}
\item the morphism $f$ is quasi-separated, and
\item the morphism $f$ satisfies the uniqueness
part of the valuative criterion.
\end{enumerate}
Then $f$ is separated.
\end{lemma}

\begin{proof}
Assumption (1) means $\Delta_{X/Y}$ is quasi-compact.
We claim the morphism
$\Delta_{X/Y} : X \to X \times_Y X$ satisfies the existence
part of the valuative criterion.
Let a solid commutative diagram
$$
\xymatrix{
\text{Spec}(K) \ar[r] \ar[d] & X \ar[d] \\
\text{Spec}(A) \ar[r] \ar@{-->}[ru] & X \times_Y X
}
$$
be given. The lower right arrow corresponds to a
pair of morphisms $a, b : \text{Spec}(A) \to X$ over $Y$.
By assumption (2) we see that $a = b$. Hence using $a$ as the dotted
arrow works. Hence
Lemma \ref{lemma-quasi-compact-existence-universally-closed}
applies, and we see that $\Delta_{X/Y}$ is universally closed.
Since always $\Delta_{X/Y}$ is locally of finite type and
separated, we conclude from
More on Morphisms, Lemma \ref{more-morphisms-lemma-characterize-finite}
that $\Delta_{X/Y}$ is a finite morphism (also, use the
general principle of
Spaces, Lemma
\ref{spaces-lemma-representable-transformations-property-implication}).
At this point $\Delta_{X/Y}$ is a representable, finite monomorphism,
hence a closed immersion by
Morphisms, Lemma \ref{morphisms-lemma-finite-monomorphism-closed}.
\end{proof}








\section{Relative conditions}
\label{section-relative-conditions}

\noindent
This is a (yet another) technical section dealing with conditions on
algebraic spaces having to do with points. It is probably a good idea
to skip this section.

\begin{definition}
\label{definition-relative-conditions}
Let $S$ be a scheme.
\begin{enumerate}
\item We say an algebraic space $X$ over $S$ {\it has
property $(\beta)$, $(\gamma)$, $(\delta)$, or $(\epsilon)$} if $X$
has the corresponding property of
Properties of Spaces, Lemma \ref{spaces-properties-lemma-bounded-fibres}.
\item An algebraic space which has $(\epsilon)$ is called a
{\it reasonable}, see 
Properties of Spaces, Definition \ref{spaces-properties-definition-reasonable}.
\item Let $f : X \to Y$ be a morphism of algebraic spaces over $S$.
We say $f$ {\it has property $(\beta)$, $(\gamma)$, $(\delta)$,
or $(\epsilon)$} if for any scheme $T$ and morphism $T \to Y$
the fibre product $T \times_Y X$ has the corresponding property.
\item A morphism $f$ which has property $(\epsilon)$ is called a
{\it reasonable morphism}.
\end{enumerate}
\end{definition}

\noindent
We refer to Remark \ref{remark-reasonable} for an informal discussion.

\begin{lemma}
\label{lemma-properties-trivial-implications}
Let $S$ be a scheme.
Let $f : X \to Y$ be a morphism of algebraic spaces over $S$.
We have the following implications among the conditions on $f$:
$$
\xymatrix{
\text{representable} \ar@{=>}[rd] & & & & \\
& \text{reasonable} \ar@{=>}[r] & (\delta) \ar@{=>}[r] &
(\gamma) \ar@{=>}[r] & (\beta) \\
\text{quasi-separated} \ar@{=>}[ru] & & & &
}
$$
\end{lemma}

\begin{proof}
This is clear from the definitions,
Properties of Spaces, Lemma \ref{spaces-properties-lemma-bounded-fibres}
and
Lemma \ref{lemma-characterize-separated}.
\end{proof}


\begin{lemma}
\label{lemma-base-change-relative-conditions}
Having property $(\beta)$, $(\gamma)$, $(\delta)$, or $(\epsilon)$
(being reasonable) is preserved under arbitrary base change.
\end{lemma}

\begin{proof}
Omitted.
\end{proof}

\begin{lemma}
\label{lemma-composition-relative-conditions}
Having property $(\beta)$, $(\gamma)$, or $(\delta)$
is preserved under compositions.
\end{lemma}

\begin{proof}
Let $\omega \in \{\beta, \gamma, \delta\}$.
Let $f : X \to Y$ and $g : Y \to Z$ be morphisms of algebraic spaces
over the scheme $S$. Assume $f$ and $g$ both have property
$(\omega)$. Then we have to show
that for any scheme $T$ and morphism $T \to Z$ the space $T \times_Z X$
has $(\omega)$. By
Lemma \ref{lemma-base-change-relative-conditions}
this reduces us to the following claim: Suppose that $Y$ is an algebraic
space having property $(\omega)$, and that $f : X \to Y$ is a morphism
with $(\omega)$. Then $X$ has $(\omega)$.

\medskip\noindent
Let us prove the claim in case $\omega = \beta$. In this case we have to show
that any $x \in |X|$ is represented by a monomorphism from the spectrum
of a field into $X$. Let $y = f(x) \in |Y|$. By assumption there exists
a field $k$ and a monomorphism $\text{Spec}(k) \to Y$ representing $y$.
Then $x$ corresponds to a point $x'$ of $\text{Spec}(k) \times_Y X$.
By assumption $x'$ is represented by a monomorphism
$\text{Spec}(k') \to \text{Spec}(k) \times_Y X$. Clearly the composition
$\text{Spec}(k') \to X$ is a monomorphism representing $x$.

\medskip\noindent
Let us prove the claim in case $\omega = \gamma$.
Let $x \in |X|$ and $y = f(x) \in |Y|$. By the result of the preceding
paragraph we can choose a diagram
$$
\xymatrix{
\text{Spec}(k') \ar[r]_x \ar[d] & X \ar[d]^f \\
\text{Spec}(k) \ar[r]^y & Y
}
$$
with horizontal arrows monomorphisms. We are going to denote
fibre products of the form $\text{Spec}(k) \times_{y, Y} ?$,
resp.\ $\text{Spec}(k') \times_{x, X} ?$ by $?_y$, resp.\ $?_x$.
Choose an affine scheme $V$ and etale morphism $V \to Y$
such that $V_y$ is not empty. Choose an affine scheme $U$ and an
etale morphism $U \to V \times_Y X$ such that
$U_x$ is not empty. Picture:
$$
\xymatrix{
U \ar[r] \ar[rd] & V \times_Y X \ar[d] \ar[r] & X \ar[d]^f \\
 & V \ar[r] & Y
}
$$
The assumption $(\gamma)$ for $Y$ implies that $V_y$ is a finite scheme
over $k$ and the assumption $(\gamma)$ for $f$ (applied to the base change
of $f$ by $V_y \to Y$) implies the fibres of
$U_x \to \text{Spec}(k') \times_{\text{Spec}(k)} V_y = (V \times_Y X)_x$
are finite. Note that the morphism
$U_x \to \text{Spec}(k') \times_{\text{Spec}(k)} V_y$ is etale.
Hence the scheme $U_x$ is finite. As the point $x$ was arbitrary
Properties of Spaces,
Lemma \ref{spaces-properties-lemma-UR-finite-above-x} part (3)
implies $(\gamma)$ holds for $X$.

\medskip\noindent
Let us prove the claim in case $\omega = \delta$.
Choose $V \to Y$ etale with $V$ an affine scheme.
Choose $U \to V \times_Y X$ etale with $U$ an affine scheme.
By assumption $V \to Y$ has universally bounded fibres. By
Properties of Spaces,
Lemma \ref{spaces-properties-lemma-base-change-universally-bounded}
$V \times_Y X \to X$ has universally bounded fibres.
By assumption on $f$ we see that $U \to V \times_Y X$ has
universally bounded fibres. By
Properties of Spaces,
Lemma \ref{spaces-properties-lemma-composition-universally-bounded}
the composition $U \to X$ has universally bounded fibres.
Hence there exists sufficiently many etale morphisms $U \to X$
from schemes with universally bounded fibres, and we conclude
that $X$ has property $(\delta)$.
\end{proof}

\begin{lemma}
\label{lemma-descent-conditions}
Let $S$ be a scheme.
Let $f : X \to Y$ be a morphism of algebraic spaces over $S$.
Let $\mathcal{P} \in \{(\beta), (\gamma), (\delta)\}$.
Assume
\begin{enumerate}
\item $f$ is quasi-compact,
\item $f$ is etale,
\item $|f| : |X| \to |Y|$ is surjective, and
\item the algebraic space $X$ has property $\mathcal{P}$.
\end{enumerate}
Then $Y$ has property $\mathcal{P}$.
\end{lemma}

\begin{proof}
Let us prove this in case $\mathcal{P} = (\beta)$. Let $y \in |Y|$ be
a point. We have to show that $y$ can be represented by a monomorphism
from a field. Choose a point $x \in |X|$ with $f(x) = y$.
By assumption we may represent $x$ by a monomorphism
$\text{Spec}(k) \to X$, with $k$ a field. By
Properties of Spaces,
Lemma \ref{spaces-properties-lemma-R-finite-above-x}
it suffices to show that the projections
$\text{Spec}(k) \times_Y \text{Spec}(k) \to \text{Spec}(k)$
are etale and quasi-compact. We can factor the first projection as
$$
\text{Spec}(k) \times_Y \text{Spec}(k)
\longrightarrow
\text{Spec}(k) \times_Y X
\longrightarrow
\text{Spec}(k)
$$
The first morphism is a monomorphism, and the second is etale and
quasi-compact. By
Properties of Spaces,
Lemma \ref{spaces-properties-lemma-etale-over-field-scheme}
we see that $\text{Spec}(k) \times_Y X$ is a scheme. Hence it is a
finite disjoint union of spectra of finite separable field extensions
of $k$. By
Schemes, Lemma \ref{schemes-lemma-mono-towards-spec-field}
we see that the first arrow identifies
$\text{Spec}(k) \times_Y \text{Spec}(k)$ with a finite disjoint
union of spectra of finite separable field extensions of $k$.
Hence the projection morphism is etale and quasi-compact.

\medskip\noindent
Let us prove this in case $\mathcal{P} = (\gamma)$.
We have already seen in the first paragraph of the proof that this implies
that every $y \in |Y|$ can be represented by a monomorphism
$y : \text{Spec}(k) \to Y$. Pick such a $y$. Pick an affine
scheme $U$ and an etale morphism $U \to X$ such that the image
of $|U| \to |Y|$ contains $y$. By
Properties of Spaces,
Lemma \ref{spaces-properties-lemma-UR-finite-above-x}
it suffices to show that $U_y$ is a finite scheme over $k$. The fibre
product $X_y = \text{Spec}(k) \times_Y X$ is a quasi-compact etale
algebraic space over $k$. Hence by
Properties of Spaces,
Lemma \ref{spaces-properties-lemma-etale-over-field-scheme}
it is a scheme. So it is a finite disjoint union of spectra of
finite separable extensions of $k$. Say $X_y = \{x_1, \ldots, x_n\}$
so $x_i$ is given by  $x_i : \text{Spec}(k_i) \to X$ with
$[k_i : k] < \infty$. By assumption $X$ has $(\gamma)$, so the schemes
$U_{x_i} = \text{Spec}(k_i) \times_X U$ is finite over $k_i$.
Finally, we note that $U_y = \coprod U_{x_i}$ as a scheme and we conclude
that $U_y$ is finite over $k$ as desired.

\medskip\noindent
Let us prove this in case $\mathcal{P} = (\delta)$.
Pick an affine scheme $V$ and an etale morphism $V \to Y$.
We have the show the fibres of $V \to Y$ are universally bounded.
The algebraic space $V \times_Y X$ is quasi-compact.
Thus we can find an affine scheme $W$ and a surjective etale morphism
$W \to V \times_Y X$, see
Properties of Spaces,
Lemma \ref{spaces-properties-lemma-quasi-compact-affine-cover}.
Here is a picture (solid diagram)
$$
\xymatrix{
W \ar[r]  \ar[rd] &
V \times_Y X \ar[r] \ar[d] &
X \ar[d]_f & \text{Spec}(k) \ar@{..>}[l]^x \ar@{..>}[ld]^y \\
 & V \ar[r] & Y
}
$$
The morphism $W \to X$ is universally bounded by our assumption that
the space $X$ has property $(\delta)$. Let $n$ be an integer bounding
the degrees of the fibres of $W \to X$. We claim that the same integer
works for bounding the fibres of $V \to Y$. Namely, suppose $y \in |Y|$
is a point. Then there exists a $x \in |X|$ with $f(x) = y$ (see above).
This means we can find a field $k$ and morphisms $x, y$ given as dotted
arrows in the diagram above. In particular we get a surjective etale
morphism
$$
\text{Spec}(k) \times_{x, X} W
\to
\text{Spec}(k) \times_{x, X} (V \times_Y X) = \text{Spec}(k) \times_{y, Y} V
$$
which shows that the degree of $\text{Spec}(k) \times_{y, Y} V$ over $k$
is less than or equal to the degree of $\text{Spec}(k) \times_{x, X} W$
over $k$, i.e., $\leq n$, and we win. (This last part of the argument
is the same as the argument in the proof of
Properties of Spaces,
Lemma \ref{spaces-properties-lemma-descent-universally-bounded}. Unfortunately
that lemma is not general enough because it only applies to representable
morphisms.)
\end{proof}

\begin{lemma}
\label{lemma-characterize-relative-conditions}
Let $S$ be a scheme.
Let $f : X \to Y$ be a morphism of algebraic spaces over $S$.
Let $\mathcal{P} \in \{(\beta), (\gamma), (\delta), (\epsilon)\}$.
The following are equivalent
\begin{enumerate}
\item $f$ is $\mathcal{P}$,
\item for every affine scheme $Z$ and every morphism $Z \to Y$ the
base change $Z \times_Y X \to Z$ of $f$ is $\mathcal{P}$,
\item for every affine scheme $Z$ and every morphism $Z \to Y$ the
algebraic space $Z \times_Y X$ is $\mathcal{P}$.
\end{enumerate}
If $\mathcal{P} \in \{(\beta), (\gamma), (\delta)\}$, then this is also
equivalent to 
\begin{enumerate}
\item[(4)] there exists a scheme $V$ and a surjective etale morphism
$V \to Y$ such that the base change $V \times_Y X \to V$ has
$\mathcal{P}$.
\end{enumerate}
\end{lemma}

\begin{proof}
The implications (1) $\Rightarrow$ (2) and (2) $\Rightarrow$ (3) are trivial.
The implication (3) $\Rightarrow$ (1) can be seen as follows.
Let $Z \to Y$ be a morphism whose source is a scheme over $S$.
Consider the algebraic space $Z \times_Y X$. If we assume (3), then
for any affine open $W \subset Z$, the open subspace
$W \times_Y X$ of $Z \times_Y X$ has property $\mathcal{P}$. Hence by
Properties of Spaces, Lemma \ref{spaces-properties-lemma-properties-local}
the space $Z \times_Y X$ has property $\mathcal{P}$, i.e., (1) holds.

\medskip\noindent
The implication (1) $\Rightarrow$ (4) is trivial. Let $V \to Y$ be
an etale morphism from a scheme as in (4). Let $Z$ be an affine scheme,
and let $Z \to Y$ be a morphism. Consider the diagram
$$
\xymatrix{
Z \times_Y V \ar[r]_q \ar[d]_p & V \ar[d] \\
Z \ar[r] & Y
}
$$
Since $p$ is etale, and hence open, we can choose finitely many affine open
subschemes $W_i \subset Z \times_Y V$ such that $Z = \bigcup p(W_i)$.
Consider the commutative diagram
$$
\xymatrix{
V \times_Y X \ar[d] &
(\coprod W_i) \times_Y X \ar[l] \ar[d] \ar[r] &
Z \times_Y X \ar[d] \\
V &
\coprod W_i \ar[l] \ar[r] &
Z
}
$$
We know $V \times_Y X$ has property $\mathcal{P}$. By 
Properties of Spaces,
Lemma \ref{spaces-properties-lemma-representable-properties}
we see that $(\coprod W_i) \times_Y X$ has property $\mathcal{P}$.
Note that the morphism $(\coprod W_i) \times_Y X \to Z \times_Y X$
is etale and quasi-compact as the base change of $\coprod W_i \to Z$.
Hence by Lemma \ref{lemma-descent-conditions}
we conclude that $Z \times_Y X$ has property $\mathcal{P}$.
\end{proof}

\begin{remark}
\label{remark-reasonable}
Informally the properties of
Definition \ref{definition-relative-conditions}
mean the following:
\begin{enumerate}
\item Condition $(\beta)$ on a space means that points are always represented
by monomorphisms from fields.
\item Condition $(\gamma)$ means $(\beta)$ $+$ locally on the space exist etale
coverings whose fibres are finite.
\item Condition $(\delta)$ means $(\gamma)$ $+$ locally on the space exist
etale coverings whose fibres are universally bounded.
\item Reasonable means there exists a Zariski open covering whose
pieces have coverings $\varphi_i : U_i \to X_i$ which are quasi-compact.
Reasonable implies $(\delta)$.
\item A morphism has one of these properties if (very) loosely speaking the
fibres of the morphism have the corresponding properties.
\end{enumerate}
Condition $(\gamma)$ is useful to prove things about specializations of
points on $|X|$. Condition $(\delta)$ is a bit stronger, and technically
quite easy to work with. Reasonable is a good condition in the sense that
it implies that $X$ has a dense open subspace which is a scheme, and
that $|X|$ is a sober topological space. This is not clear for spaces
which have property $(\delta)$ and probably not true (although see
Properties of Spaces,
Remark \ref{spaces-properties-remark-fun-property-almost-reasonable}
for an interesting additional property of spaces of type $(\delta)$).
On the other hand, we do not know whether the class of reasonable
morphisms is closed under composition, and we do not know whether
reasonable spaces satisfy a descent property as the one in
Lemma \ref{lemma-descent-conditions} (even with $f$ assumed representable).
\end{remark}

\noindent
Here is the lemma we promised earlier.

\begin{lemma}
\label{lemma-re-characterize-universally-closed}
Let $S$ be a scheme.
Let $f : X \to Y$ be a morphism of algebraic spaces over $S$.
Assume $f$ is quasi-compact, and $f$ has property $(\gamma)$.
(For example if $f$ is representable, or quasi-separated, see
Lemma \ref{lemma-properties-trivial-implications}.)
Then $f$ is universally closed if and only if the
existence part of the valuative criterion holds.
\end{lemma}

\begin{proof}
In Lemma \ref{lemma-quasi-compact-existence-universally-closed}
we proved that any quasi-compact morphism which satsifies the existence
part of the valuative criterion is universally closed.
To prove the other, assume that $f$ is universally closed.
In the proof of
Proposition \ref{proposition-characterize-universally-closed}
we have seen that it suffices to show, for any valuation ring $A$,
and any morphism $\text{Spec}(A) \to Y$, that the base change
$f_A : X_A \to \text{Spec}(A)$ satisfies the existence part of the valuative
criterion. By definition the algebraic space $X_A$ has property $(\gamma)$
and hence Proposition \ref{proposition-characterize-universally-closed}
applies to the morphism $f_A$ and we win.
\end{proof}









\section{Pushforward of quasi-coherent sheaves}
\label{section-pushforward}

\noindent
We first prove a simple lemma that relates pushforward of sheaves of modules
for a morphism of algebraic spaces to pushforward of sheaves of modules for
a morphism of schemes.

\begin{lemma}
\label{lemma-compute-pushforward}
Let $S$ be a scheme.
Let $f : X \to Y$ be a morphism of algebraic spaces over $S$.
Let $U \to X$ be a surjective etale morphism from a scheme to $X$.
Set $R = U \times_X U$ and denote $t, s : R \to U$ the projection
morphisms as usual. Denote $a : U \to Y$ and $b : R \to Y$ the induced
morphisms. For any $\mathcal{F} \in \text{Mod}(\mathcal{O}_X)$
there exists an exact sequence
$$
0 \to f_*\mathcal{F} \to a_*(\mathcal{F}|_U) \to b_*(\mathcal{F}|_R)
$$
where the second arrow is the difference $t^* - s^*$.
\end{lemma}

\begin{proof}
We denote $\mathcal{F}$ also its extension to a sheaf of modules on
$X_{spaces, etale}$, see
Properties of Spaces,
Remark \ref{spaces-properties-remark-explain-equivalence}.
Let $V \to Y$ be an object of $Y_{etale}$. Then $V \times_Y X$ is an
object of $X_{spaces, etale}$, and by definition
$f_*\mathcal{F}(V) = \mathcal{F}(V \times_Y X)$. Since $U \to X$ is
surjective etale, we see that $\{V \times_Y U \to V \times_Y X\}$
is a covering. Also, we have
$(V \times_Y U) \times_X (V \times_Y U) = V \times_Y R$.
Hence, by the sheaf condition of $\mathcal{F}$ on
$X_{spaces, etale}$ we have a short exact sequence
$$
0 \to \mathcal{F}(V \times_Y X)
\to \mathcal{F}(V \times_Y U) \to \mathcal{F}(V \times_Y R)
$$
where the second arrow is the difference of restricting via $t$ or $s$.
This exact sequence is functorial in $V$ and hence we obtain the lemma.
\end{proof}

\noindent
Here is the main result of this section. In its proof we have to be
a little careful since we are working with sheaves of modules
in the etale topology and we have not yet matched what happens for
these sheaves in the case of a morphism of representable algebraic
spaces with what happens for ``usual'' quasi-coherent sheaves on schemes.

\begin{lemma}
\label{lemma-pushforward}
Let $S$ be a scheme.
Let $f : X \to Y$ be a morphism of algebraic spaces over $S$.
If $f$ is quasi-compact and quasi-separated, then $f_*$ transforms
quasi-coherent $\mathcal{O}_X$-modules into
quasi-coherent $\mathcal{O}_Y$-modules.
\end{lemma}

\begin{proof}
Let $\mathcal{F}$ be a quasi-coherent sheaf on $X$. We have to show that
$f_*\mathcal{F}$ is a quasi-coherent sheaf on $Y$. For this it suffices
to show that for any affine scheme $V$ and etale morphism $V \to Y$
the restriction of $f_*\mathcal{F}$ to $V$ is quasi-coherent, see
Properties of Spaces,
Lemma \ref{spaces-properties-lemma-characterize-quasi-coherent}
Let $f' : V \times_Y X \to V$
be the base change of $f$ by $V \to Y$. Note that $f'$ is also
quasi-compact and quasi-separated, see
Lemmas \ref{lemma-base-change-quasi-compact} and
\ref{lemma-base-change-separated}.
By
Properties of Spaces,
Lemma \ref{spaces-properties-lemma-pushforward-etale-base-change-modules}
we know that the restriction of $f_*\mathcal{F}$ to $V$ is $f'_*$ of the
restriction of $\mathcal{F}$ to $V \times_Y X$. Hence by
we may replace $f$ by $f'$, and assume that $Y$ is an affine scheme.

\medskip\noindent
Assume $Y$ is an affine scheme. Since $f$ is quasi-compact we see that $X$
is quasi-compact. Thus we may choose an affine scheme $U$ and a surjective
etale morphism $U \to X$, see
Properties of Spaces,
Lemma \ref{spaces-properties-lemma-quasi-compact-affine-cover}.
By Lemma \ref{lemma-compute-pushforward} we get an exact sequence
$$
0 \to f_*\mathcal{F} \to a_*(\mathcal{F}|_U) \to b_*(\mathcal{F}|_R).
$$
As $X \to Y$ is quasi-separated we see that $R = U \times_X U \to U \times_Y U$
is a quasi-compact monomorphism. This implies that $R$ is a quasi-compact
separated scheme (as $U$ and $Y$ are affine at this point).
Hence the morphisms $a : U \to Y$ and $b : R \to Y$ are quasi-compact and
quasi-separated. In particular,
Descent,
Proposition \ref{descent-proposition-equivalence-quasi-coherent-functorial}
applies! This implies that $f_*\mathcal{F}$ is a kernel of quasi-coherent
modules, and hence itself quasi-coherent, see
Properties of Spaces,
Lemma \ref{spaces-properties-lemma-properties-quasi-coherent}.
\end{proof}

\noindent
At this point we point out that
Descent,
Proposition \ref{descent-proposition-equivalence-quasi-coherent-functorial}
(used in the proof above) also means that if $f : X \to Y$ is a
quasi-compact and quasi-separated morphism of algebraic spaces, and
if $X$ and $Y$ are representable, then the functor
$f_* : \text{QCoh}(X) \to \text{QCoh}(Y)$ agrees with the
usual functor if we think of $X$ and $Y$ as schemes.






\section{Closed immersions}
\label{section-closed-immersions}

\noindent
In this section we elucidate some of the results obtained previously on
immersions of algebraic spaces; it should parallel
Morphisms, Section \ref{morphisms-section-closed-immersions}.

\begin{lemma}
\label{lemma-closed-immersion-ideals}
Let $S$ be a scheme.
Let $X$ be an algebraic space over $S$.
For every closed immersion $i : Z \to X$ the sheaf
$i_*\mathcal{O}_Z$ is a quasi-coherent $\mathcal{O}_X$-module, the map
$i^\sharp : \mathcal{O}_X \to i_*\mathcal{O}_Z$ is surjective and its
kernel is a quasi-coherent sheaf of ideals. The rule
$Z \mapsto \text{Ker}(\mathcal{O}_X \to i_*\mathcal{O}_Z)$
defines an inclusion reversing bijection
$$
\begin{matrix}
\text{closed subschemes}\\
Z \subset X
\end{matrix}
\longrightarrow
\begin{matrix}
\text{quasi-coherent sheaves}\\
\text{of ideals }\mathcal{I} \subset \mathcal{O}_X
\end{matrix}
$$
Moreover, given a closed subscheme $Z$ corresponding to the quasi-coherent
sheaf of ideals $\mathcal{I} \subset \mathcal{O}_X$ a morphism of algebraic
spaces $h : Y \to X$ factors through $Z$ if and only if the map
$h^*\mathcal{I} \to h^*\mathcal{O}_X = \mathcal{O}_Y$ is zero.
\end{lemma}

\begin{proof}
Let $U \to X$ be a surjective etale morphism whose source is a scheme.
Consider the diagram
$$
\xymatrix{
U \times_X Z \ar[r] \ar[d]_{i'} & Z \ar[d]^i \\
U \ar[r] & X
}
$$
By
Lemma \ref{lemma-characterize-closed-immersion}
we see that $i$ is a closed immersion
if and only if $i'$ is a closed immersion. By
Properties of Spaces,
Lemma \ref{spaces-properties-lemma-pushforward-etale-base-change-modules}
we see that $i'_*\mathcal{O}_{U \times_X Z}$ is the restriction of
$i_*\mathcal{O}_Z$ to $U$. Hence the assertions on
$\mathcal{O}_X \to i_*\mathcal{O}_Z$ are equivalent to the
corresponding assertions on
$\mathcal{O}_U \to i'_*\mathcal{O}_{U \times_X Z}$.
And since $i'$ is a closed immersion of schemes, these results follow from
Morphisms, Lemma \ref{morphisms-lemma-closed-immersion}.

\medskip\noindent
Let us prove that given a quasi-coherent
sheaf of ideals $\mathcal{I} \subset \mathcal{O}_X$ the formula
$$
Z(T) = \{h : T \to X \mid h^*\mathcal{I} \to \mathcal{O}_T
\text{ is zero}\}
$$
defines a closed subspace of $X$. It is clearly a subfunctor of $X$.
To show that $Z \to X$ is representable by closed immersions, let
$\varphi : U \to X$ be a morphism from a scheme towards $X$. Then
$Z \times_X U$ is represented by the analogous subfunctor of $U$ corresponding
to the sheaf of ideals $\text{Im}(\varphi^*\mathcal{I} \to \mathcal{O}_U)$. By
Properties of Spaces,
Lemma \ref{spaces-properties-lemma-pullback-quasi-coherent}
the $\mathcal{O}_U$-module $\varphi^*\mathcal{I}$ is quasi-coherent on
on $U$, and hence $\text{Im}(\varphi^*\mathcal{I} \to \mathcal{O}_U)$
is a quasi-coherent sheaf of ideals on $U$. By
Schemes, Lemma \ref{schemes-lemma-characterize-closed-subspace}
we conclude that $Z \times_X U$ is represented by the closed subscheme
of $U$ associated to $\text{Im}(\varphi^*\mathcal{I} \to \mathcal{O}_U)$.
Thus $Z$ is a closed subspace of $X$.

\medskip\noindent
In the formula for $Z$ above the inputs $T$ are schemes since algebraic
spaces are sheaves on $(\textit{Sch}/S)_{fppf}$. We omit the verification
that the same formula remains true if $T$ is an algebraic space.
\end{proof}












\section{Surjective morphisms}
\label{section-surjective}

\noindent
We have already defined in Section \ref{section-representable}
what it means for a representable morphism of algebraic spaces
to be surjective.

\begin{lemma}
\label{lemma-surjective-representable}
Let $S$ be a scheme. Let $f : X \to Y$ be a representable
morphism of algebraic spaces over $S$. Then
$f$ is surjective if and only if $|f| : |X| \to |Y|$ is surjective.
\end{lemma}

\begin{proof}
Namely, if $f : X \to Y$ is representable, then it is surjective if and only if
for every scheme $T$ and every morphism $T \to Y$ the base change
$f_T : T \times_Y X \to T$ of $f$ is surjective. By
Properties of spaces, Lemma \ref{spaces-properties-lemma-points-cartesian}
the map $|T \times_Y X| \to |T| \times_{|Y|} |X|$ is always surjective,
Hence $|f_T| : |T \times_Y X| \to |T|$ is surjective if $|f| : |X| \to |Y|$
is surjective. Conversely, if $|f_T|$ is surjective for every
$T \to Y$, then by taking $T$ to be the spectrum of a field we conclude that
$|X| \to |Y|$ is surjective.
\end{proof}

\noindent
This clears the way for the following definition.

\begin{definition}
\label{definition-surjective}
Let $S$ be a scheme. Let $f : X \to Y$ be a morphism of algebraic
spaces over $S$. We say $f$ is {\it surjective}
if the map $|f| : |X| \to |Y|$ of associated topological spaces
is surjective.
\end{definition}

\begin{lemma}
\label{lemma-surjective-local}
Let $S$ be a scheme.
Let $f : X \to Y$ be a morphism of algebraic spaces over $S$.
The following are equivalent:
\begin{enumerate}
\item $f$ is surjective,
\item for every scheme $Z$ and any morphism $Z \to Y$ the morphism
$Z \times_Y X \to Z$ is surjective,
\item for every affine scheme $Z$ and any morphism
$Z \to Y$ the morphism $Z \times_Y X \to Z$ is surjective,
\item there exists a scheme $V$ and a surjective etale morphism
$V \to Y$ such that $V \times_Y X \to V$ is a surjective morphism,
\item there exists a scheme $U$ and a surjective etale morphism
$\varphi : U \to X$ such that the composition $f \circ \varphi$
is surjective, and
\item there exists a commutative diagram
$$
\xymatrix{
U \ar[d] \ar[r] & V \ar[d] \\
X \ar[r] & Y
}
$$
where $U$, $V$ are schemes and the vertical arrows are surjective etale
such that the horizontal arrow is surjective.
\end{enumerate}
\end{lemma}

\begin{proof}
Omitted.
\end{proof}

\begin{lemma}
\label{lemma-composition-surjective}
The composition of surjective morphisms is surjective.
\end{lemma}

\begin{proof}
Omitted.
\end{proof}

\begin{lemma}
\label{lemma-base-change-surjective}
The base change of a surjective morphism is surjective.
\end{lemma}

\begin{proof}
Omitted. Hint: Use
Properties of spaces, Lemma \ref{spaces-properties-lemma-points-cartesian}.
\end{proof}







\section{Radicial morphisms}
\label{section-radicial}

\noindent
We have already defined in Section \ref{section-representable}
what it means for a representable morphism of algebraic spaces
to be radicial. For a field $K$ over $S$ (recall this means that
we are given a structure morphism $\text{Spec}(K) \to S$) and an
algebraic space $X$ over $S$ we write
$X(K) = \text{Mor}_S(\text{Spec}(K), X)$.

\begin{lemma}
\label{lemma-radical-representable}
Let $S$ be a scheme. Let $f : X \to Y$ be a representable
morphism of algebraic spaces over $S$. Then
$f$ is radicial if and only if for all fields $K$ the
map $X(K) \to Y(K)$ is injective.
\end{lemma}

\begin{proof}
Suppose that $f$ is radicial. Then for any field $K$ and any morphism
$\text{Spec}(K) \to Y$ the map $\text{Spec}(K) \times_Y X \to \text{Spec}(K)$
is radicial. Hence there exists at most one section of the morphism
$\text{Spec}(K) \times_Y X \to \text{Spec}(K)$. Hence the map
$X(K) \to Y(K)$ is injective. Conversely, suppose that for every field $K$
the map $X(K) \to Y(K)$ is injective. Let $T \to Y$ be a morphism from a
scheme into $Y$, and consider the base change $f_T : T \times_Y X \to T$.
For any field $K$ we have
$$
(T \times_Y X)(K) = T(K) \times_{Y(K)} X(K)
$$
by definition of the fibre product, and hence the injectivity of
$X(K) \to Y(K)$ garantees the injectivity of
$(T \times_Y X)(K) \to T(K)$ which means that $f_T$ is radicial as desired.
\end{proof}

\noindent
This clears the way for the following definition.

\begin{definition}
\label{definition-radicial}
Let $S$ be a scheme. Let $f : X \to Y$ be a morphism of algebraic
spaces over $S$. We say $f$ is {\it radicial}
if the map $X(K) \to Y(K)$ is injective for every field $K$ over $S$.
\end{definition}

\noindent
We will see below that this is the same thing as being {\it universally
injective}.

\begin{lemma}
\label{lemma-base-change-radicial}
The base change of a radicial morphism is radicial.
\end{lemma}

\begin{proof}
Omitted. Hint: This is formal.
\end{proof}

\begin{lemma}
\label{lemma-radicial-local}
Let $S$ be a scheme.
Let $f : X \to Y$ be a morphism of algebraic spaces over $S$.
The following are equivalent:
\begin{enumerate}
\item $f$ is radicial,
\item the morphism $f$ is {\it universally injective}, i.e.,
for every morphism $Y' \to Y$ the induced map $|Y' \times_Y X| \to |Y'|$
is injective,
\item for every scheme $Z$ and any morphism $Z \to Y$ the morphism
$Z \times_Y X \to Z$ is radicial,
\item for every affine scheme $Z$ and any morphism
$Z \to Y$ the morphism $Z \times_Y X \to Z$ is radicial, and
\item there exists a scheme $Z$ and a surjective morphism
$Z \to Y$ such that $Z \times_Y X \to Z$ is radicial.
\end{enumerate}
\end{lemma}

\begin{proof}
By
Morphisms, Lemma \ref{morphisms-lemma-radicial-universally-injective}
we see that (2) implies (3). It is clear that (3) implies (4),
and (4) implies (5) since we can take $Z$ to be a disjoint union
of affines.

\medskip\noindent
Assume $g : Z \to Y$ as in (5). Let $y : \text{Spec}(K) \to Y$ be a
morphism from the spectrum of a field into $Y$. By assumption we
can find an extension field $\alpha : K \subset K'$ and a morphism
$z : \text{Spec}(K') \to Z$ such that $y \circ \alpha = g \circ z$
(with obvious abuse of notation). By assumption the
morphism $Z \times_Y X \to Z$ is radicial, hence there is at most one
lift of $g \circ z : \text{Spec}(K') \to Y$ to a morphism into $X$.
Since $\{\alpha : \text{Spec}(K') \to \text{Spec}(K)\}$ is a
fpqc covering this implies there is at most one lift of
$y : \text{Spec}(K) \to Y$ to a morphism into $X$, see
Properties of Spaces, Lemma \ref{spaces-properties-lemma-separated-fpqc}.
Thus we see that (1) holds.

\medskip\noindent
Finally, assume (1) holds. Let $g : Y' \to Y$ be a morphism of algebraic
spaces, and denote $f' : Y' \times_Y X \to Y'$ the base change of $f$.
This means that there exist fields $K_i$, $i = 1, 2$ and morphisms
$\varphi_i : \text{Spec}(K_i) \to Y' \times_Y X$
such that $f' \circ \varphi_1$ and $f' \circ \varphi_2$ define the
same element of $|Y'|$. By definition this means there exists a
field $\Omega$ and embeddings $\alpha_i : K_i \subset \Omega$ such that
the two morphisms
$f' \circ \varphi_i \circ \alpha_i : \text{Spec}(\Omega) \to Y'$ are equal.
In particular the compositions $g \circ f' \circ \varphi_i \circ \alpha_i$
are equal. As $f$ is radicial this implies that the morphism
$g' \circ \varphi_i \circ \alpha_i$ are equal, where $g' : Y' \times_Y X \to X$
is the projection. By the universal property of the fibre product we conclude
that the morphisms
$\varphi_i \circ \alpha_i : \text{Spec}(\Omega) \to Y' \times_Y X$ are
equal. In other words $\varphi_1$ and $\varphi_2$ define the same point
of $Y' \times_Y X$ as desired.
\end{proof}

\begin{lemma}
\label{lemma-composition-radicial}
A composition of radicial morphisms is radicial.
\end{lemma}

\begin{proof}
Omitted.
\end{proof}





\section{Affine morphisms}
\label{section-affine}

\noindent
We have already defined in Section \ref{section-representable}
what it means for a representable morphism of algebraic spaces
to be affine.

\begin{lemma}
\label{lemma-affine-representable}
Let $S$ be a scheme. Let $f : X \to Y$ be a representable
morphism of algebraic spaces over $S$. Then
$f$ is affine if and only if for all affine schemes $Z$
and morphisms $Z \to Y$ the scheme $X \times_Y Z$ is affine.
\end{lemma}

\begin{proof}
This follows directly from the definition of an affine morphism of schemes
(Morphisms, Definition \ref{morphisms-definition-affine}).
\end{proof}

\noindent
This clears the way for the following definition.

\begin{definition}
\label{definition-affine}
Let $S$ be a scheme.
Let $f : X \to Y$ be a morphism of algebraic spaces over $S$.
We say $f$ is {\it affine} if for every affine scheme $Z$ and
morphism $Z \to Y$ the algebraic space $X \times_Y Z$ is representable
by an affine scheme.
\end{definition}

\begin{lemma}
\label{lemma-affine-local}
Let $S$ be a scheme.
Let $f : X \to Y$ be a morphism of algebraic spaces over $S$.
The following are equivalent:
\begin{enumerate}
\item $f$ is representable and affine,
\item $f$ is affine,
\item there exists a scheme $V$ and a surjective etale morphism
$V \to Y$ such that $V \times_Y X \to V$ is affine.
\end{enumerate}
\end{lemma}

\begin{proof}
It is clear that (1) implies (2) and that (2) implies (3) by taking
$V$ to be a disjoint union of affines etale over $Y$, see
Properties of Spaces,
Lemma \ref{spaces-properties-lemma-cover-by-union-affines}.
Assume $V \to Y$ is as in (3). Then for every affine open $W$ of $V$ we see
that $W \times_Y X$ is an affine open of $V \times_Y X$. Hence by
Properties of Spaces, Lemma \ref{spaces-properties-lemma-subscheme}
we conclude that $V \times_Y X$ is a scheme. Moreover the morphism
$V \times_Y X \to V$ is affine. This means we can apply
Spaces,
Lemma \ref{spaces-lemma-morphism-sheaves-with-P-effective-descent-etale}
because the class of affine morphisms satisfies all the required
properties (see
Morphisms, Lemmas \ref{morphisms-lemma-base-change-affine} and
Descent, Lemmas \ref{descent-lemma-descending-property-affine}
and \ref{descent-lemma-affine}). The conclusion of applying this lemma
is that $f$ is representable and affine, i.e., (1) holds.
\end{proof}

\begin{lemma}
\label{lemma-composition-affine}
The composition of affine morphisms is affine.
\end{lemma}

\begin{proof}
Omitted.
\end{proof}

\begin{lemma}
\label{lemma-base-change-affine}
The base change of an affine morphism is affine.
\end{lemma}

\begin{proof}
Omitted.
\end{proof}









\section{Quasi-affine morphisms}
\label{section-quasi-affine}

\noindent
We have already defined in Section \ref{section-representable}
what it means for a representable morphism of algebraic spaces
to be quasi-affine.

\begin{lemma}
\label{lemma-quasi-affine-representable}
Let $S$ be a scheme. Let $f : X \to Y$ be a representable
morphism of algebraic spaces over $S$. Then
$f$ is quasi-affine if and only if for all affine schemes $Z$
and morphisms $Z \to Y$ the scheme $X \times_Y Z$ is quasi-affine.
\end{lemma}

\begin{proof}
This follows directly from the definition of a quasi-affine morphism
of schemes
(Morphisms, Definition \ref{morphisms-definition-quasi-affine}).
\end{proof}

\noindent
This clears the way for the following definition.

\begin{definition}
\label{definition-quasi-affine}
Let $S$ be a scheme.
Let $f : X \to Y$ be a morphism of algebraic spaces over $S$.
We say $f$ is {\it quasi-affine} if for every affine scheme $Z$ and
morphism $Z \to Y$ the algebraic space $X \times_Y Z$ is representable
by a quasi-affine scheme.
\end{definition}

\begin{lemma}
\label{lemma-quasi-affine-local}
Let $S$ be a scheme.
Let $f : X \to Y$ be a morphism of algebraic spaces over $S$.
The following are equivalent:
\begin{enumerate}
\item $f$ is representable and quasi-affine,
\item $f$ is quasi-affine,
\item there exists a scheme $V$ and a surjective etale morphism
$V \to Y$ such that $V \times_Y X \to V$ is quasi-affine.
\end{enumerate}
\end{lemma}

\begin{proof}
It is clear that (1) implies (2) and that (2) implies (3) by taking
$V$ to be a disjoint union of affines etale over $Y$, see
Properties of Spaces,
Lemma \ref{spaces-properties-lemma-cover-by-union-affines}.
Assume $V \to Y$ is as in (3). Then for every affine open $W$ of $V$ we see
that $W \times_Y X$ is a quasi-affine open of $V \times_Y X$. Hence by
Properties of Spaces, Lemma \ref{spaces-properties-lemma-subscheme}
we conclude that $V \times_Y X$ is a scheme. Moreover the morphism
$V \times_Y X \to V$ is quasi-affine. This means we can apply
Spaces,
Lemma \ref{spaces-lemma-morphism-sheaves-with-P-effective-descent-etale}
because the class of quasi-affine morphisms satisfies all the required
properties (see
Morphisms, Lemmas \ref{morphisms-lemma-base-change-quasi-affine} and
Descent, Lemmas \ref{descent-lemma-descending-property-quasi-affine}
and \ref{descent-lemma-quasi-affine}). The conclusion of applying this lemma
is that $f$ is representable and quasi-affine, i.e., (1) holds.
\end{proof}

\begin{lemma}
\label{lemma-composition-quasi-affine}
The composition of quasi-affine morphisms is quasi-affine.
\end{lemma}

\begin{proof}
Omitted.
\end{proof}

\begin{lemma}
\label{lemma-base-change-quasi-affine}
The base change of a quasi-affine morphism is quasi-affine.
\end{lemma}

\begin{proof}
Omitted.
\end{proof}








\section{Types of morphisms local on source and target}
\label{section-local-source-target}

\noindent
Given a property of morphisms of schemes which is local on the source
and the target in the etale topology we may use it to define a corresponding
property of morphisms of algebraic spaces, namely by imposing either of
the equivalent conditions of the lemma below. (We will not always
automatically do so; so in each instance we will explicitly restate
the definition elsewhere.)

\begin{lemma}
\label{lemma-local-source-target}
Let $S$ be a scheme.
Let $f : X \to Y$ be a morphism of algebraic spaces over $S$.
Let $\mathcal{P}$ be a property of morphisms of schemes
which is etale local on the source and the target, see
Descent, Definitions \ref{descent-definition-property-morphisms-local} and
\ref{descent-definition-property-morphisms-local-source}.
Consider commutative diagrams
$$
\xymatrix{
U \ar[d] \ar[r]_\psi & V \ar[d] \\
X \ar[r]^f & Y
}
$$
where $U$ and $V$ are schemes and the vertical arrows are etale.
The following are equivalent
\begin{enumerate}
\item for any diagram as above the morphism $\psi$ has property
$\mathcal{P}$, and
\item for some diagram as above with surjective vertical arrows
the morphism $\psi$ has property $\mathcal{P}$.
\end{enumerate}
If $X$ and $Y$ are representable, then this is also
equivalent to $f$ (as a morphisms of schemes) having property $\mathcal{P}$.
If $\mathcal{P}$ is also preserved under any base change, and
fppf local on the base, then for representable morphisms $f$ this
is also equivalent to $f$ having property $\mathcal{P}$ in the sense
of Section \ref{section-representable}.
\end{lemma}

\begin{proof}
Omitted. Hint: This requires drawing a bunch of commutative diagrams
and arguing formally.
\end{proof}







\section{Morphisms of finite type}
\label{section-finite-type}

\noindent
For a morphism of schemes the property of being locally of finite type is
stable under base change and fpqc local on
the target and fppf local on the source. See
Morphisms, Lemma \ref{morphisms-lemma-base-change-finite-type}, and
Descent, Lemmas \ref{descent-lemma-descending-property-finite-type} and
\ref{descent-lemma-locally-finite-type-fppf-local-source}.
Hence, by
Lemma \ref{lemma-local-source-target}
above, we may define what it means for a morphism of algebraic spaces
to be locally of finite type as
follows and it agrees with the already existing notion defined in
Section \ref{section-representable}
when the morphism is representable.

\begin{definition}
\label{definition-locally-finite-type}
Let $S$ be a scheme.
\begin{enumerate}
\item A morphism of algebraic spaces $f : X \to Y$ is
{\it locally of finite type} if the equivalent conditions of
Lemma \ref{lemma-local-source-target} hold with
$\mathcal{P} = \text{locally of finite type}$.
\item A morphism of  algebraic spaces $f : X \to Y$ is
{\it of finite type} if it is locally of finite type and quasi-compact.
\end{enumerate}
\end{definition}

\begin{lemma}
\label{lemma-composition-finite-type}
The composition of finite type morphisms is of finite type.
The same holds for locally of finite type.
\end{lemma}

\begin{proof}
Omitted.
\end{proof}

\begin{lemma}
\label{lemma-base-change-finite-type}
A base change of a finite type morphism is finite type.
The same holds for locally of finite type.
\end{lemma}

\begin{proof}
Omitted.
\end{proof}









\section{Quasi-finite morphisms}
\label{section-quasi-finite}

\noindent
For a morphism of schemes the property of being locally quasi-finite is
stable under base change and fpqc local on
the target and etale local on the source. See
Morphisms, Lemma \ref{morphisms-lemma-base-change-quasi-finite}, and
Descent, Lemmas \ref{descent-lemma-descending-property-quasi-finite} and
\ref{descent-lemma-locally-quasi-finite-etale-local-source}.
Hence, by
Lemma \ref{lemma-local-source-target}
above, we may define what it means for a morphism of algebraic spaces
to be locally quasi-finite as
follows and it agrees with the already existing notion defined in
Section \ref{section-representable}
when the morphism is representable.

\begin{definition}
\label{definition-locally-quasi-finite}
Let $S$ be a scheme.
\begin{enumerate}
\item A morphism of algebraic spaces $f : X \to Y$ is
{\it locally quasi-finite} if the equivalent conditions of
Lemma \ref{lemma-local-source-target} hold with
$\mathcal{P} = \text{locally quasi-finite}$.
\item A morphism of  algebraic spaces $f : X \to Y$ is
{\it quasi-finite} if it is locally quasi-finite and quasi-compact.
\end{enumerate}
\end{definition}

\begin{lemma}
\label{lemma-composition-quasi-finite}
The composition of quasi-finite morphisms is quasi-finite.
The same holds for locally quasi-finite.
\end{lemma}

\begin{proof}
Omitted.
\end{proof}

\begin{lemma}
\label{lemma-base-change-quasi-finite}
A base change of a quasi-finite morphism is quasi-finite.
The same holds for locally quasi-finite.
\end{lemma}

\begin{proof}
Omitted.
\end{proof}

\begin{lemma}
\label{lemma-immersion-quasi-finite}
An immersion is locally quasi-finite.
\end{lemma}

\begin{proof}
Omitted.
\end{proof}

\begin{lemma}
\label{lemma-permanence-quasi-finite}
Let $S$ be a scheme.
Let $X \to Y \to Z$ be morphisms of algebraic spaces over $S$.
If $X \to Z$ is locally quasi-finite, then $X \to Y$
is locally quasi-finite.
\end{lemma}

\begin{proof}
Choose a commutative diagram
$$
\xymatrix{
U \ar[d] \ar[r] & V \ar[d] \ar[r] & W \ar[d] \\
X \ar[r] & Y \ar[r] & Z
}
$$
with vertical arrows etale and surjective. (See
Spaces, Lemma \ref{spaces-lemma-lift-morphism-presentations}.)
Apply
Morphisms, Lemma \ref{morphisms-lemma-permanence-quasi-finite}
to the top row.
\end{proof}








\section{Morphisms of finite presentation}
\label{section-finite-presentation}

\noindent
For a morphism of schemes the property of being locally of finite presentation
is stable under base change and fpqc local on the target and
etale local on the source. See
Morphisms, Lemma \ref{morphisms-lemma-base-change-finite-presentation}, and
Descent,
Lemmas \ref{descent-lemma-descending-property-locally-finite-presentation}
and
\ref{descent-lemma-locally-finite-presentation-fppf-local-source}.
Hence, by
Lemma \ref{lemma-local-source-target}
above, we may define what it means for a morphism of algebraic spaces
to be locally quasi-finite as
follows and it agrees with the already existing notion defined in
Section \ref{section-representable}
when the morphism is representable.

\begin{definition}
\label{definition-locally-finite-presentation}
Let $S$ be a scheme.
\begin{enumerate}
\item A morphism of algebraic spaces $f : X \to Y$ is
{\it locally of finite presentation} if the equivalent conditions of
Lemma \ref{lemma-local-source-target} hold with
$\mathcal{P} =$``locally of finite presentation''.
\item A morphism of  algebraic spaces $f : X \to Y$ is
{\it of finite presentation}
if it is locally of finite presentation, quasi-compact and
quasi-separated.
\end{enumerate}
\end{definition}

\noindent
Note that a morphism of finite presentation is {\bf not} just a quasi-compact
morphisms which is locally of finite presentation.

\begin{lemma}
\label{lemma-composition-finite-presentation}
The composition of morphisms of finite presentation is of finite presentation.
The same holds for locally of finite presentation.
\end{lemma}

\begin{proof}
Omitted.
\end{proof}

\begin{lemma}
\label{lemma-base-change-finite-presentation}
A base change of a morphism of finite presentation is of finite presentation
The same holds for locally of finite presentation.
\end{lemma}

\begin{proof}
Omitted.
\end{proof}









\section{Open morphisms}
\label{section-open}

\noindent
For a representable morphism of algebraic spaces we have already defined (in
Section \ref{section-representable})
what it means to be universally open. Hence before we give the natural
definition we check that it agrees with this in the representable case.

\begin{lemma}
\label{lemma-characterize-representable-universally-open}
Let $S$ be a scheme. Let $f : X \to Y$ be a representable morphism of
algebraic spaces over $S$. The following are equivalent
\begin{enumerate}
\item $f$ is universally open, and
\item for every morphism of algebraic spaces $Z \to Y$ the morphism of
topological spaces $|Z \times_Y X| \to |Z|$ is open.
\end{enumerate}
\end{lemma}

\begin{proof}
Assume (1), and let $Z \to Y$ be as in (2). Choose a scheme $V$ and
a surjective etale morphism $V \to Y$. By assumption the morphism
of schemes $V \times_Y X \to V$ is universally open. By
Properties of Spaces, Section \ref{spaces-properties-section-points}
in the commutative diagram
$$
\xymatrix{
|V \times_Y X| \ar[r] \ar[d] & |Z \times_Y X| \ar[d] \\
|V| \ar[r] & |Z|
}
$$
the horizontal arrows are open and surjective, and moreover
$$
|V \times_Y X| \longrightarrow |V| \times_{|Z|} |Z \times_Y X|
$$
is surjective. Hence as the left
vertical arrow is open it follows that the right vertical arrow is
open. This proves (2). The implication (2) $\Rightarrow$ (1) is
immediate from the definitions.
\end{proof}

\noindent
Thus we may use the following natural definition.

\begin{definition}
\label{definition-open}
Let $S$ be a scheme. Let $f : X \to Y$ be a morphism of algebraic spaces
over $S$.
\begin{enumerate}
\item We say $f$ is {\it open} if the map of topological spaces
$|f| : |X| \to |Y|$ is open.
\item We say $f$ is {\it universally open} if for every morphism
of algebraic spaces $Z \to Y$ the morphism of topological spaces
$$
|Z \times_Y X| \to |Z|
$$
is open, i.e., the base change $Z \times_Y X \to Z$ is open.
\end{enumerate}
\end{definition}

\begin{lemma}
\label{lemma-base-change-universally-open}
The base change of a universally open morphism of algebraic spaces
by any morphism of algebraic spaces is universally open.
\end{lemma}

\begin{proof}
This is immediate from the definition.
\end{proof}

\begin{lemma}
\label{lemma-composition-universally-open}
The composition of a pair of (universally) open morphisms of algebraic spaces
is (universally) open.
\end{lemma}

\begin{proof}
Omitted.
\end{proof}

\begin{lemma}
\label{lemma-characterize-universally-open}
Let $S$ be a scheme. Let $f : X \to Y$ be a morphism of algebraic spaces
over $S$. The following are equivalent
\begin{enumerate}
\item $f$ is universally open,
\item for every scheme $Z$ and every morphism $Z \to Y$
the projection $|Z \times_Y X| \to |Z|$ is open,
\item for every affine scheme $Z$ and every morphism $Z \to Y$
the projection $|Z \times_Y X| \to |Z|$ is open, and
\item there exists a scheme $V$ and a surjective etale morphism
$V \to Y$ such that $V \times_Y X \to V$ is a universally open morpism
of algebraic spaces.
\end{enumerate}
\end{lemma}

\begin{proof}
We omit the proof that (1) implies (2), and that (2) implies (3).

\medskip\noindent
Assume (3). Choose a surjective etale morphism $V \to Y$.
We are going to show that $V \times_Y X \to V$ is a universally
open morphism of algebraic spaces. Let $Z \to V$ be a morphism
from an algebraic space to $V$. Let $W \to Z$ be a surjective etale
morphism where $W = \coprod W_i$ is a disjoint union of affine schemes, see
Properties of Spaces,
Lemma \ref{spaces-properties-lemma-cover-by-union-affines}.
Then we have the following commutative diagram
$$
\xymatrix{
\coprod_i |W_i \times_Y X| \ar@{=}[r] \ar[d] &
|W \times_Y X| \ar[r] \ar[d] &
|Z \times_Y X| \ar[d] \ar@{=}[r] &
|Z \times_V (V \times_Y X)| \ar[ld] \\
\coprod |W_i| \ar@{=}[r] &
|W| \ar[r] &
|Z|
}
$$
We have to show the south-east arrow is open. The middle horizontal
arrows are surjective and open
(Properties of Spaces, Lemma \ref{spaces-properties-lemma-etale-open}).
By assumption (3), and the fact that
$W_i$ is affine we see that the left vertical arrows are open. Hence
it follows that the right vertical arrow is open.

\medskip\noindent
Assume $V \to Y$ is as in (4). We will show that $f$ is universally open.
Let $Z \to Y$ be a morphism of algebraic spaces. Consider the
diagram
$$
\xymatrix{
|(V \times_Y Z) \times_V (V \times_Y X)| \ar@{=}[r] \ar[rd] &
|V \times_Y X| \ar[r] \ar[d] &
|Z \times_Y X| \ar[d] \\
 &
|V \times_Y Z| \ar[r] &
|Z|
}
$$
The south-west arrow is open by assumption. The horizontal arrows are
surjective and open because the corresponding morphisms of
algebraic spaces are etale (see
Properties of spaces, Lemma \ref{spaces-properties-lemma-etale-open}).
It follows that the right vertical arrow is open.
\end{proof}












\section{Flat morphisms}
\label{section-flat}

\noindent
For a morphism of schemes the property of being flat is
stable under base change and fpqc local on
both the target and the source. See
Morphisms, Lemma \ref{morphisms-lemma-base-change-flat}, and
Descent, Lemmas \ref{descent-lemma-descending-property-flat} and
\ref{descent-lemma-flat-fpqc-local-source}.
Hence, by
Lemma \ref{lemma-local-source-target}
above, we may define the notion of a flat morphism of algebraic spaces as
follows and it agrees with the already existing notion defined in
Section \ref{section-representable}
when the morphism is representable.

\begin{definition}
\label{definition-flat}
Let $S$ be a scheme.
A morphism of algebraic spaces $f : X \to Y$ is {\it flat} if
the equivalent conditions of Lemma \ref{lemma-local-source-target} hold with
$\mathcal{P} = \text{flat}$.
\end{definition}

\begin{lemma}
\label{lemma-flat-local}
Let $S$ be a scheme.
Let $f : X \to Y$ be a morphism of algebraic spaces over $S$.
The following are equivalent:
\begin{enumerate}
\item $f$ is flat,
\item for every scheme $Z$ and any morphism $Z \to Y$ the morphism
$Z \times_Y X \to Z$ is flat,
\item for every affine scheme $Z$ and any morphism
$Z \to Y$ the morphism $Z \times_Y X \to Z$ is flat, and
\item there exists a scheme $V$ and a surjective etale morphism
$V \to Y$ such that $V \times_Y X \to V$ is flat.
\end{enumerate}
\end{lemma}

\begin{proof}
Omitted.
\end{proof}

\begin{lemma}
\label{lemma-composition-flat}
The composition of flat morphisms is flat.
\end{lemma}

\begin{proof}
Omitted.
\end{proof}

\begin{lemma}
\label{lemma-base-change-flat}
The base change of a flat morphism is flat.
\end{lemma}

\begin{proof}
Omitted.
\end{proof}









\section{Syntomic morphisms}
\label{section-syntomic}

\noindent
For a morphism of schemes the property of being syntomic is
stable under base change
(Morphisms, Lemma \ref{morphisms-lemma-base-change-syntomic}),
fpqc local on the target
(Descent, Lemma \ref{descent-lemma-descending-property-syntomic}),
and syntomic local on the source
(Descent, Lemma \ref{descent-lemma-syntomic-syntomic-local-source}).
Hence, by
Lemma \ref{lemma-local-source-target}
above, we may define the notion of a syntomic morphism of algebraic spaces as
follows and it agrees with the already existing notion defined in
Section \ref{section-representable}
when the morphism is representable.

\begin{definition}
\label{definition-syntomic}
Let $S$ be a scheme.
A morphism of algebraic spaces $f : X \to Y$ is {\it syntomic} if
the equivalent conditions of Lemma \ref{lemma-local-source-target} hold with
$\mathcal{P} = \text{syntomic}$.
\end{definition}

\begin{lemma}
\label{lemma-composition-syntomic}
The composition of syntomic morphisms is syntomic.
\end{lemma}

\begin{proof}
Omitted.
\end{proof}

\begin{lemma}
\label{lemma-base-change-syntomic}
The base change of a syntomic morphism is syntomic.
\end{lemma}

\begin{proof}
Omitted.
\end{proof}

\begin{lemma}
\label{lemma-syntomic-local}
Let $S$ be a scheme.
Let $f : X \to Y$ be a morphism of algebraic spaces over $S$.
The following are equivalent:
\begin{enumerate}
\item $f$ is syntomic,
\item for every scheme $Z$ and any morphism $Z \to Y$ the morphism
$Z \times_Y X \to Z$ is syntomic,
\item for every affine scheme $Z$ and any morphism
$Z \to Y$ the morphism $Z \times_Y X \to Z$ is syntomic,
\item there exists a scheme $V$ and a surjective etale morphism
$V \to Y$ such that $V \times_Y X \to V$ is a syntomic morphism,
\item there exists a scheme $U$ and a surjective etale morphism
$\varphi : U \to X$ such that the composition $f \circ \varphi$
is syntomic, and
\item there exists a commutative diagram
$$
\xymatrix{
U \ar[d] \ar[r] & V \ar[d] \\
X \ar[r] & Y
}
$$
where $U$, $V$ are schemes and the vertical arrows are surjective etale
such that the horizontal arrow is syntomic.
\end{enumerate}
\end{lemma}

\begin{proof}
Omitted.
\end{proof}











\section{Etale morphisms of algebraic spaces}
\label{section-etale}

\noindent
The notion of an etale morphism of algebraic spaces was defined in
Properties of Spaces, Definition \ref{spaces-properties-definition-etale}.
In this section we work out some of its properties.

\begin{lemma}
\label{lemma-etale-local}
Let $S$ be a scheme.
Let $f : X \to Y$ be a morphism of algebraic spaces over $S$.
The following are equivalent:
\begin{enumerate}
\item $f$ is etale,
\item for every scheme $Z$ and any morphism $Z \to Y$ the morphism
$Z \times_Y X \to Z$ is etale,
\item for every affine scheme $Z$ and any morphism
$Z \to Y$ the morphism $Z \times_Y X \to Z$ is etale,
\item there exists a scheme $V$ and a surjective etale morphism
$V \to Y$ such that $V \times_Y X \to V$ is an etale morphism,
\item there exists a scheme $U$ and a surjective etale morphism
$\varphi : U \to X$ such that the composition $f \circ \varphi$
is etale, and
\item there exists a commutative diagram
$$
\xymatrix{
U \ar[d] \ar[r] & V \ar[d] \\
X \ar[r] & Y
}
$$
where $U$, $V$ are schemes and the vertical arrows are surjective etale
such that the horizontal arrow is etale.
\end{enumerate}
\end{lemma}

\begin{proof}
Combine
Properties of Spaces, Lemmas
\ref{spaces-properties-lemma-etale-local},
\ref{spaces-properties-lemma-base-change-etale} and
\ref{spaces-properties-lemma-composition-etale}.
Some details omitted.
\end{proof}



\begin{lemma}
\label{lemma-etale-locally-quasi-finite}
An etale morphism of algebraic spaces is locally quasi-finite.
\end{lemma}

\begin{proof}
Let $X \to Y$ be an etale morphism of algebraic spaces, see
Properties of Spaces, Definition \ref{spaces-properties-definition-etale}.
By
Properties of Spaces, Lemma \ref{spaces-properties-lemma-etale-local}
we see this means there exists a diagram as in
Lemma \ref{lemma-local-source-target}
with $\psi$ etale. By
Morphisms, Lemma \ref{morphisms-lemma-etale-locally-quasi-finite}
$\psi$ is locally quasi-finite. Hence $X \to Y$ is locally quasi-finite
by definition.
\end{proof}






\section{Bootstrap}
\label{section-bootstrap}

\noindent
In
Spaces, Section \ref{spaces-section-algebraic-spaces}
we defined an algebraic space as a sheaf in the fppf topology whose
diagonal is representable, and such that there exist a surjective etale
morphism from a scheme towards it. In this section we show that
a sheaf in the fppf topology whose diagonal is representable by algebraic
spaces and which has an etale surjective covering by an algebraic space
is also an algebraic space.
In other words, the category of algebraic spaces is an enlargement of the
category of schemes by certain fppf sheaves $F$ with certain conditions
on the diagonal and the existence of an etale covering by a scheme. The
result of this section says that doing the same process again starting with
the category of algebraic spaces, does not lead to yet another category.

\medskip\noindent
Another motivation for the material in this section is that it will guarantee
later that an algebraic stack whose inertia stack is trivial is equivalent
to an algebraic space.

\medskip\noindent
We begin with a result that is interesting in its own right.
It says that an algebraic space which is locally quasi-finite and
separated over a scheme is a scheme. But first...\ a lemma (which will
be obsoleted by
Proposition \ref{proposition-locally-quasi-finite-separated-over-scheme}).

\begin{lemma}
\label{lemma-neighbourhood-scheme}
Let $S$ be a scheme. Consider a commutative diagram
$$
\xymatrix{
V' \ar[r] \ar[rd] & T' \times_T X \ar[r] \ar[d] & X \ar[d] \\
& T' \ar[r] & T
}
$$
Assume
\begin{enumerate}
\item $T' \to T$ is an etale morphism of affine schemes,
\item $X$ is an algebraic space,
\item $X \to T$ is a separated, locally quasi-finite morphism,
\item $V'$ is an open subspace of $T' \times_T X$, and
\item $V' \to T'$ is quasi-affine.
\end{enumerate}
In this situation the image $U$ of $V'$ in $X$ is a quasi-compact
open subspace of $X$ which is representable.
\end{lemma}

\begin{proof}
We first make some trivial observations. 
Note that $V'$ is a representable by Lemma \ref{lemma-quasi-affine-local}.
It is also quasi-compact (as a quasi-affine scheme over an affine scheme, see
Morphisms, Lemma \ref{morphisms-lemma-quasi-affine-separated}).
Since $T' \times_T X \to X$ is etale
(Properties of Spaces, Lemma \ref{spaces-properties-lemma-base-change-etale})
the map $|T' \times_T X| \to |X|$ is open, see
Properties of Spaces, Lemma \ref{spaces-properties-lemma-etale-open}.
Let $U \subset X$ be the open subspace corresponding to the image of
$|V'|$, see
Properties of Spaces, Lemma \ref{spaces-properties-lemma-open-subspaces}.
As $|V'|$ is quasi-compact we see that $|U|$ is quasi-compact, hence
$U$ is a quasi-compact algebraic spaces, by 
Properties of Spaces, Lemma \ref{spaces-properties-lemma-quasi-compact-space}.

\medskip\noindent
By
Morphisms,
Lemma \ref{morphisms-lemma-locally-quasi-finite-qc-source-universally-bounded}
the morphism $T' \to T$ is universally bounded. Hence we can do induction on
the integer $n$ bounding the degree of the fibres of $T' \to T$, see
Morphisms, Lemma \ref{morphisms-lemma-etale-universally-bounded}
for a description of this integer in the case of an etale morphism.
If $n = 1$, then $T' \to T$ is an open immersion (see
Descent, Lemma \ref{descent-lemma-radicial-etale-open-immersion}),
and the result is clear. Assume $n > 1$.

\medskip\noindent
Consider the affine scheme $T'' = T' \times_T T'$.
As $T' \to T$ is etale we have a decomposition (into open and closed affine
subschemes) $T'' = \Delta(T') \amalg T^*$. Namely $\Delta = \Delta_{T'/T}$
is open by
Morphisms, Lemma \ref{morphisms-lemma-diagonal-unramfied-morphism}
and closed because $T' \to T$ is separated as a morphism of affines.
As a base change the degrees of the fibres of the second projection
$\text{pr}_1 : T' \times_T T' \to T'$ are bounded by $n$, see
Morphisms, Lemma \ref{morphisms-lemma-base-change-universally-bounded}.
On the other hand, $\text{pr}_1|_{\Delta(T')} : \Delta(T') \to T'$ is
an isomorphism and every fibre has exactly one point.
Thus, on applying
Morphisms, Lemma \ref{morphisms-lemma-etale-universally-bounded}
we conclude the degrees of the fibres of the restriction
$\text{pr}_1|_{T^*} : T^* \to T'$ are bounded by $n - 1$.
Hence the induction hypothesis applied to the diagram
$$
\xymatrix{
p_0^{-1}(V') \cap X^* \ar[r] \ar[rd] &
X^* \ar[r]_{p_1|_{X^*}} \ar[d] &
X' \ar[d] \\
& T^* \ar[r]^{\text{pr}_1|_{T^*}} & T'
}
$$
gives that $p_1(p_0^{-1}(V') \cap X^*)$
is a quasi-compact scheme. Here we set
$X'' = T'' \times_T X$, $X^* = T^* \times_T X$, and $X' = T' \times_T X$,
and $p_0, p_1 : X'' \to X'$ are the base changes of $\text{pr}_0, \text{pr}_1$.
Most of the hypotheses of the lemma imply
by base change the corresponding hypothesis for the diagram above.
For example $p_0^{-1}(V') = T'' \times_{T'} V'$
is a scheme quasi-affine over $T''$ as a base change. Some
verifications omitted.

\medskip\noindent
By
Properties of Spaces, Lemma \ref{spaces-properties-lemma-subscheme}
we conclude that
$$
p_1(p_0^{-1}(V')) =
V' \cup p_1(p_0^{-1}(V') \cap X^*)
$$
is a quasi-compact scheme. Moreover, it is clear that
$p_1(p_0^{-1}(V'))$ is the inverse image of the
quasi-compact open subspace $U \subset X$ discussed in the
first paragraph of the proof. In other words, $T' \times_T U$ is a scheme!
Note that $T' \times_T U$ is quasi-compact and
separated and locally quasi-finite over $T'$, as
$T' \times_T X \to T'$ is locally quasi-finite and separated
being a base change of the original morphism $X \to T$ (see
Lemmas \ref{lemma-base-change-separated} and
\ref{lemma-base-change-quasi-finite}).
This implies by
More on Morphisms,
Lemma \ref{more-morphisms-lemma-quasi-finite-separated-quasi-affine}
that $T' \times_T U \to T'$ is quasi-affine.

\medskip\noindent
By
Descent, Lemma \ref{descent-lemma-descent-data-sheaves}
this gives a descent datum on $T' \times_T U / T'$
relative to the etale covering $\{T' \to W\}$, where $W \subset T$
is the image of the morphism $T' \to T$.
Because $U'$ is quasi-affine over $T'$ we see from
Descent, Lemma \ref{descent-lemma-quasi-affine}
that this datum is effective, and by the last part of
Descent, Lemma \ref{descent-lemma-descent-data-sheaves}
this implies that $U$ is a scheme as desired.
Some minor details omitted.
\end{proof}

\begin{proposition}
\label{proposition-locally-quasi-finite-separated-over-scheme}
Let $S$ be a scheme.
Let $f : X \to T$ be a morphism of algebraic spaces.
Assume
\begin{enumerate}
\item $T$ is representable,
\item $f$ is locally quasi-finite, and
\item $f$ is separated.
\end{enumerate}
Then $X$ is representable.
\end{proposition}

\begin{proof}
Let $T = \bigcup T_i$ be an affine open covering of the scheme $T$.
If we can show that the open subspaces $X_i = f^{-1}(T_i)$ are
representable, then $X$ is representable, see
Properties of Spaces, Lemma \ref{spaces-properties-lemma-subscheme}.
Note that $X_i = T_i \times_T X$ and that locally quasi-finite and
separated are both stable under base change, see
Lemmas \ref{lemma-base-change-separated} and
\ref{lemma-base-change-quasi-finite}.
Hence we may assume $T$ is an affine scheme.

\medskip\noindent
By
Properties of Spaces,
Lemma \ref{spaces-properties-lemma-union-of-quasi-compact}
there exists a Zariski covering $X = \bigcup X_i$
such that each $X_i$ has a surjective etale covering by
an affine scheme. By
Properties of Spaces, Lemma \ref{spaces-properties-lemma-subscheme}
again it suffices to prove the propostion for each $X_i$.
Hence we may assume there exists an affine scheme $U$ and a
surjective etale morphism $U \to X$. This reduces us to the
situation in the next paragraph.

\medskip\noindent
Assume we have
$$
U \longrightarrow X \longrightarrow T
$$
where $U$ and $T$ are affine schemes, $U \to X$ is etale surjective, and
$X \to T$ is separated and locally quasi-finite. By
Lemmas \ref{lemma-etale-locally-quasi-finite} and
\ref{lemma-composition-quasi-finite}
the morphism $U \to T$ is locally quasi-finite.
Since $U$ and $T$ are affine it is quasi-finite.
Set $R = U \times_X U$. Then $X = U/R$, see
Spaces, Lemma \ref{spaces-lemma-space-presentation}.
As $X \to T$ is separated the
morphism $R \to U \times_T U$ is a closed immersion, see
Lemma \ref{lemma-fibre-product-after-map}.
In particular $R$ is an affine scheme also.
As $U \to X$ is etale the projection morphisms
$t, s : R \to U$ are etale as well. In particular $s$ and $t$ are
quasi-finite, flat and of finite presentation (see
Morphisms, Lemmas \ref{morphisms-lemma-etale-locally-quasi-finite},
\ref{morphisms-lemma-etale-flat} and
\ref{morphisms-lemma-etale-locally-finite-presentation}).

\medskip\noindent
Let $(U, R, s, t, c)$ be the groupoid associated to the etale
equivalence relation $R$ on $U$. Let $u \in U$ be a point, and
denote $p \in T$ its image. We are going to use
Groupoids, Lemma \ref{groupoids-lemma-quasi-finite-over-base-j-proper}
for the groupoid $(U, R, s, t, c)$ over the scheme $T$ with
points $p$ and $u$ as above.
By the discussion in the previous paragraph all the
assumptions (1) -- (7) of that lemma are satisfied.
Hence we get an etale neighbourhood
$(T', p') \to (T, p)$ and disjoint union decompositions
$$
U_{T'} = U' \amalg W, \quad
R_{T'} = R' \amalg W'
$$
and $u' \in U'$ satisfying conclusions
(a), (b), (c), (d), (e), (f), (g), and (h) of the aforementioned
Groupoids, Lemma \ref{groupoids-lemma-quasi-finite-over-base-j-proper}.
We may and do assume that $T'$ is affine (after possibly shrinking $T'$).
Conclusion (h) implies that $R' = U' \times_{X_{T'}} U'$ with projection
mappings identified with the restrictions of $s'$ and $t'$.
Thus $(U', R', s'|_{R'}, t'|_{R'}, c'|_{R' \times_{t', U', s'} R'})$ of
conclusion (g) is an etale equivalence relation. By
Spaces, Lemma \ref{spaces-lemma-finding-opens}
we conclude that $U'/R'$ is an open subspace of $X_{T'}$. By conclusion (d)
the schemes $U'$, $R'$ are affine and the morphisms
$s'|_{R'}, t'|_{R'}$ are finite etale. Hence
Groupoids, Lemma \ref{groupoids-proposition-finite-flat-equivalence}
kicks in and we see that $U'/R'$ is an affine scheme.

\medskip\noindent
We conclude that for every pair of points $(u, p)$ as above we can
find an etale neighbourhood $(T', p') \to (T, p)$ with
$\kappa(p) = \kappa(p')$ and a point $u' \in U_{T'}$ mapping to $u$
such that the image $x'$ of $u'$ in $|X_{T'}|$ has an open neighbourhood
$V'$ in $X_{T'}$ which is an affine scheme. We apply
Lemma \ref{lemma-neighbourhood-scheme}
to obtain an open subspace $W \subset X$ which is a scheme, and
which contains $x$ (the image of $u$ in $|X|$).
Since this works for every $x$ we see that $X$
is a scheme by
Properties of Spaces, Lemma \ref{spaces-properties-lemma-subscheme}.
This ends the proof.
\end{proof}

\noindent
Before we can state the main result of this section we need to introduce
a little more notation, very similar to the discussion preceding the
definition of algebraic spaces in the chapter on algebraic spaces.

\begin{lemma}
\label{lemma-representable-by-spaces}
Let $S$ be a scheme.
Let $a : F \to G$ be a map of presheaves on $(\textit{Sch}/S)_{fppf}$.
Suppose $a : F \to G$ is representable by algebraic spaces, see
Spaces, Definition \ref{spaces-definition-morphism-representable-by-spaces}.
If $X$ is an algebraic space over $S$, and $X \to G$ is a map of presheaves
then $X \times_G F$ is an algebraic space.
\end{lemma}

\begin{proof}
Follows immediately from
Spaces, Lemma \ref{spaces-lemma-representable-by-spaces-over-space}.
\end{proof}

\begin{definition}
\label{definition-surjective-transformation}
Let $S$ be a scheme.
Let $a : F \to G$ be a map of presheaves on $(\textit{Sch}/S)_{fppf}$, and
assume $a : F \to G$ is representable by algebraic spaces, see
Spaces, Definition \ref{spaces-definition-morphism-representable-by-spaces}.
We say $a$ is {\it surjective} (resp.\ {\it etale})
if for every scheme $T$ and map $\xi : T \to G$
the morphism of algebraic spaces $T \times_{\xi, G} F \to T$
is surjective (resp.\ etale).
\end{definition}

\noindent
By
Spaces, Lemma \ref{spaces-lemma-morphism-spaces-is-representable-by-spaces}
any morphism between algebraic spaces over $S$ is representable by algebraic
spaces. And by
Lemmas \ref{lemma-surjective-local}
and \ref{lemma-etale-local}
the definition of surjective (resp.\ etale)
above agrees with the already existing definition of morphisms
of algebraic spaces.

\begin{lemma}
\label{lemma-base-change-transformation}
Let $S$ be a scheme.
Let
$$
\xymatrix{
G' \times_G F \ar[r] \ar[d]^{a'} & F \ar[d]^a \\
G' \ar[r] & G
}
$$
be a fibre square of presheaves on $(\textit{Sch}/S)_{fppf}$.
\begin{enumerate}
\item If $a$ is representable by algebraic spaces so is $a'$,
\item if $a$ is representable by algebraic spaces and surjective so is $a'$,
and
\item if $a$ is representable by algebraic spaces and etale so is $a'$.
\end{enumerate}
\end{lemma}

\begin{proof}
Omitted. Hint: This is formal.
\end{proof}

\begin{lemma}
\label{lemma-composition-transformation}
Let $S$ be a scheme.
Let
$$
\xymatrix{
F \ar[r]^a & G \ar[r]^b & H
}
$$
be maps of presheaves on $(\textit{Sch}/S)_{fppf}$.
\begin{enumerate}
\item If $a$, $b$ are representable by algebraic spaces so is $b \circ a$,
\item if $a$, $b$ are representable by algebraic spaces and surjective
so is $b \circ a$, and
\item if $a$, $b$ are representable by algebraic spaces and etale so is
$b \circ a$.
\end{enumerate}
\end{lemma}

\begin{proof}
Omitted. Hint: Use
Spaces, Lemma \ref{spaces-lemma-representable-by-spaces-over-space}.
\end{proof}

\begin{lemma}
\label{lemma-representable-diagonal}
If $F$ is a presheaf on $(\textit{Sch}/S)_{fppf}$ such that
$\Delta_F : F \to F \times F$ is representable by algebraic spaces,
then for every algebraic space $X$ any map $X \to F$ is representable
by algebraic spaces.
\end{lemma}

\begin{proof}
Let $X \to F$ be as in the lemma. Let $T$ be a scheme, and let $T \to F$ be
a morphism. Then we have
$$
T \times_F X = (T \times_S X) \times_{F \times F, \Delta} F
$$
which is an algebraic space by
Lemma \ref{lemma-representable-by-spaces}
and the assumption.
\end{proof}

\noindent
In particular if $F$ is a presheaf as in the lemma, then for any
morphism $X \to F$ where $X$ is an algebraic space it makes sense to
say that $X \to F$ is surjective (resp.\ etale) by using
Definition \ref{definition-surjective-transformation}.
With these preliminaries out of the way we can now state the main
result of this section.

\begin{theorem}
\label{theorem-bootstrap}
Let $S$ be a scheme.
Let $F : (\textit{Sch}/S)_{fppf}^{opp} \to \textit{Sets}$ be a functor.
Assume that
\begin{enumerate}
\item the presheaf $F$ is a sheaf,
\item the diagonal morphism $F  \to F \times F$ is representable by
algebraic spaces, and
\item there exists an algebraic space $X$
and a map $X \to F$ which is surjective, and etale.
\end{enumerate}
Then $F$ is an algebraic space.
\end{theorem}

\begin{proof}
We will use the remarks directly below
Definition \ref{definition-surjective-transformation}
without further mention.
In the situation of the theorem, let $U \to X$ be a surjective etale morphism
from a scheme towards $X$.
By Lemma \ref{lemma-composition-transformation}
$U \to F$ is surjective and etale also.
Hence the theorem boils down to proving that
$\Delta_F$ is representable.

\medskip\noindent
Let $U$ be a scheme, and let $U \to F$ be surjective and etale.
Set $R = U \times_F U$, which is an algebraic space (see
Lemma \ref{lemma-representable-diagonal}).
The morphism of algebraic spaces $R \to U \times_S U$ is a monomorphism,
hence separated (as the diagonal of a monomorphism is an isomorphism).
Moreover, since $U \to F$ is etale, we see that $R \to U$ is etale, by
Lemma \ref{lemma-base-change-transformation}. In particular, we see
that $R \to U$ is locally quasi-finite, see
Lemma \ref{lemma-etale-locally-quasi-finite}.
We conclude that also $R \to U \times_S U$ is
locally quasi-finite by
Lemma \ref{lemma-permanence-quasi-finite}.
Hence
Proposition \ref{proposition-locally-quasi-finite-separated-over-scheme}
applies and $R$ is a scheme. Hence $F = U/R$ is an algebraic
space according to
Spaces, Theorem \ref{spaces-theorem-presentation}.
\end{proof}





























\section{Other chapters}

\begin{multicols}{2}
\begin{enumerate}
\item \hyperref[introduction-section-phantom]{Introduction}
\item \hyperref[conventions-section-phantom]{Conventions}
\item \hyperref[sets-section-phantom]{Set Theory}
\item \hyperref[categories-section-phantom]{Categories}
\item \hyperref[topology-section-phantom]{Topology}
\item \hyperref[sheaves-section-phantom]{Sheaves on Spaces}
\item \hyperref[algebra-section-phantom]{Commutative Algebra}
\item \hyperref[sites-section-phantom]{Sites and Sheaves}
\item \hyperref[homology-section-phantom]{Homological Algebra}
\item \hyperref[derived-section-phantom]{Derived Categories}
\item \hyperref[more-algebra-section-phantom]{More Algebra}
\item \hyperref[simplicial-section-phantom]{Simplicial Methods}
\item \hyperref[modules-section-phantom]{Sheaves of Modules}
\item \hyperref[sites-modules-section-phantom]{Modules on Sites}
\item \hyperref[injectives-section-phantom]{Injectives}
\item \hyperref[cohomology-section-phantom]{Cohomology of Sheaves}
\item \hyperref[sites-cohomology-section-phantom]{Cohomology on Sites}
\item \hyperref[hypercovering-section-phantom]{Hypercoverings}
\item \hyperref[schemes-section-phantom]{Schemes}
\item \hyperref[constructions-section-phantom]{Constructions of Schemes}
\item \hyperref[properties-section-phantom]{Properties of Schemes}
\item \hyperref[morphisms-section-phantom]{Morphisms of Schemes}
\item \hyperref[coherent-section-phantom]{Coherent Cohomology}
\item \hyperref[divisors-section-phantom]{Divisors}
\item \hyperref[limits-section-phantom]{Limits of Schemes}
\item \hyperref[varieties-section-phantom]{Varieties}
\item \hyperref[chow-section-phantom]{Chow Homology}
\item \hyperref[topologies-section-phantom]{Topologies on Schemes}
\item \hyperref[descent-section-phantom]{Descent}
\item \hyperref[more-morphisms-section-phantom]{More on Morphisms}
\item \hyperref[flat-section-phantom]{More on Flatness}
\item \hyperref[groupoids-section-phantom]{Groupoid Schemes}
\item \hyperref[more-groupoids-section-phantom]{More on Groupoid Schemes}
\item \hyperref[etale-section-phantom]{\'Etale Morphisms of Schemes}
\item \hyperref[etale-cohomology-section-phantom]{\'Etale Cohomology}
\item \hyperref[spaces-section-phantom]{Algebraic Spaces}
\item \hyperref[spaces-properties-section-phantom]{Properties of Algebraic Spaces}
\item \hyperref[spaces-morphisms-section-phantom]{Morphisms of Algebraic Spaces}
\item \hyperref[spaces-topologies-section-phantom]{Topologies on Algebraic Spaces}
\item \hyperref[spaces-descent-section-phantom]{Descent and Algebraic Spaces}
\item \hyperref[spaces-more-morphisms-section-phantom]{More on Morphisms of Spaces}
\item \hyperref[quot-section-phantom]{Quot and Hilbert Spaces}
\item \hyperref[stacks-section-phantom]{Stacks}
\item \hyperref[spaces-groupoids-section-phantom]{Groupoids in Algebraic Spaces}
\item \hyperref[spaces-more-groupoids-section-phantom]{More on Groupoids in Spaces}
\item \hyperref[bootstrap-section-phantom]{Bootstrap}
\item \hyperref[examples-stacks-section-phantom]{Examples of Stacks}
\item \hyperref[groupoids-quotients-section-phantom]{Quotients of Groupoids}
\item \hyperref[algebraic-section-phantom]{Algebraic Stacks}
\item \hyperref[criteria-section-phantom]{Criteria for Representability}
\item \hyperref[stacks-properties-section-phantom]{Properties of Algebraic Stacks}
\item \hyperref[stacks-morphisms-section-phantom]{Morphisms of Algebraic Stacks}
\item \hyperref[examples-section-phantom]{Examples}
\item \hyperref[exercises-section-phantom]{Exercises}
\item \hyperref[guide-section-phantom]{Guide to Literature}
\item \hyperref[desirables-section-phantom]{Desirables}
\item \hyperref[coding-section-phantom]{Coding Style}
\item \hyperref[fdl-section-phantom]{GNU Free Documentation License}
\item \hyperref[index-section-phantom]{Auto Generated Index}
\end{enumerate}
\end{multicols}


\bibliography{my}
\bibliographystyle{amsalpha}

\end{document}
