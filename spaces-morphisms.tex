\IfFileExists{stacks-project.cls}{%
\documentclass{stacks-project}
}{%
\documentclass{amsart}
}

% The following AMS packages are automatically loaded with
% the amsart documentclass:
%\usepackage{amsmath}
%\usepackage{amssymb}
%\usepackage{amsthm}

% For dealing with references we use the comment environment
\usepackage{verbatim}
\newenvironment{reference}{\comment}{\endcomment}
%\newenvironment{reference}{}{}
\newenvironment{slogan}{\comment}{\endcomment}
\newenvironment{history}{\comment}{\endcomment}

% For commutative diagrams you can use
% \usepackage{amscd}
\usepackage[all]{xy}

% We use 2cell for 2-commutative diagrams.
\xyoption{2cell}
\UseAllTwocells

% To put source file link in headers.
% Change "template.tex" to "this_filename.tex"
% \usepackage{fancyhdr}
% \pagestyle{fancy}
% \lhead{}
% \chead{}
% \rhead{Source file: \url{template.tex}}
% \lfoot{}
% \cfoot{\thepage}
% \rfoot{}
% \renewcommand{\headrulewidth}{0pt}
% \renewcommand{\footrulewidth}{0pt}
% \renewcommand{\headheight}{12pt}

\usepackage{multicol}

% For cross-file-references
\usepackage{xr-hyper}

% Package for hypertext links:
\usepackage{hyperref}

% For any local file, say "hello.tex" you want to link to please
% use \externaldocument[hello-]{hello}
\externaldocument[introduction-]{introduction}
\externaldocument[conventions-]{conventions}
\externaldocument[sets-]{sets}
\externaldocument[categories-]{categories}
\externaldocument[topology-]{topology}
\externaldocument[sheaves-]{sheaves}
\externaldocument[sites-]{sites}
\externaldocument[stacks-]{stacks}
\externaldocument[fields-]{fields}
\externaldocument[algebra-]{algebra}
\externaldocument[brauer-]{brauer}
\externaldocument[homology-]{homology}
\externaldocument[derived-]{derived}
\externaldocument[simplicial-]{simplicial}
\externaldocument[more-algebra-]{more-algebra}
\externaldocument[smoothing-]{smoothing}
\externaldocument[modules-]{modules}
\externaldocument[sites-modules-]{sites-modules}
\externaldocument[injectives-]{injectives}
\externaldocument[cohomology-]{cohomology}
\externaldocument[sites-cohomology-]{sites-cohomology}
\externaldocument[dga-]{dga}
\externaldocument[dpa-]{dpa}
\externaldocument[hypercovering-]{hypercovering}
\externaldocument[schemes-]{schemes}
\externaldocument[constructions-]{constructions}
\externaldocument[properties-]{properties}
\externaldocument[morphisms-]{morphisms}
\externaldocument[coherent-]{coherent}
\externaldocument[divisors-]{divisors}
\externaldocument[limits-]{limits}
\externaldocument[varieties-]{varieties}
\externaldocument[topologies-]{topologies}
\externaldocument[descent-]{descent}
\externaldocument[perfect-]{perfect}
\externaldocument[more-morphisms-]{more-morphisms}
\externaldocument[flat-]{flat}
\externaldocument[groupoids-]{groupoids}
\externaldocument[more-groupoids-]{more-groupoids}
\externaldocument[etale-]{etale}
\externaldocument[chow-]{chow}
\externaldocument[intersection-]{intersection}
\externaldocument[pic-]{pic}
\externaldocument[adequate-]{adequate}
\externaldocument[dualizing-]{dualizing}
\externaldocument[duality-]{duality}
\externaldocument[discriminant-]{discriminant}
\externaldocument[local-cohomology-]{local-cohomology}
\externaldocument[curves-]{curves}
\externaldocument[resolve-]{resolve}
\externaldocument[models-]{models}
\externaldocument[pione-]{pione}
\externaldocument[etale-cohomology-]{etale-cohomology}
\externaldocument[proetale-]{proetale}
\externaldocument[crystalline-]{crystalline}
\externaldocument[spaces-]{spaces}
\externaldocument[spaces-properties-]{spaces-properties}
\externaldocument[spaces-morphisms-]{spaces-morphisms}
\externaldocument[decent-spaces-]{decent-spaces}
\externaldocument[spaces-cohomology-]{spaces-cohomology}
\externaldocument[spaces-limits-]{spaces-limits}
\externaldocument[spaces-divisors-]{spaces-divisors}
\externaldocument[spaces-over-fields-]{spaces-over-fields}
\externaldocument[spaces-topologies-]{spaces-topologies}
\externaldocument[spaces-descent-]{spaces-descent}
\externaldocument[spaces-perfect-]{spaces-perfect}
\externaldocument[spaces-more-morphisms-]{spaces-more-morphisms}
\externaldocument[spaces-flat-]{spaces-flat}
\externaldocument[spaces-groupoids-]{spaces-groupoids}
\externaldocument[spaces-more-groupoids-]{spaces-more-groupoids}
\externaldocument[bootstrap-]{bootstrap}
\externaldocument[spaces-pushouts-]{spaces-pushouts}
\externaldocument[groupoids-quotients-]{groupoids-quotients}
\externaldocument[spaces-more-cohomology-]{spaces-more-cohomology}
\externaldocument[spaces-simplicial-]{spaces-simplicial}
\externaldocument[formal-spaces-]{formal-spaces}
\externaldocument[restricted-]{restricted}
\externaldocument[spaces-resolve-]{spaces-resolve}
\externaldocument[formal-defos-]{formal-defos}
\externaldocument[defos-]{defos}
\externaldocument[cotangent-]{cotangent}
\externaldocument[examples-defos-]{examples-defos}
\externaldocument[algebraic-]{algebraic}
\externaldocument[examples-stacks-]{examples-stacks}
\externaldocument[stacks-sheaves-]{stacks-sheaves}
\externaldocument[criteria-]{criteria}
\externaldocument[artin-]{artin}
\externaldocument[quot-]{quot}
\externaldocument[stacks-properties-]{stacks-properties}
\externaldocument[stacks-morphisms-]{stacks-morphisms}
\externaldocument[stacks-limits-]{stacks-limits}
\externaldocument[stacks-cohomology-]{stacks-cohomology}
\externaldocument[stacks-perfect-]{stacks-perfect}
\externaldocument[stacks-introduction-]{stacks-introduction}
\externaldocument[stacks-more-morphisms-]{stacks-more-morphisms}
\externaldocument[stacks-geometry-]{stacks-geometry}
\externaldocument[moduli-]{moduli}
\externaldocument[moduli-curves-]{moduli-curves}
\externaldocument[examples-]{examples}
\externaldocument[exercises-]{exercises}
\externaldocument[guide-]{guide}
\externaldocument[desirables-]{desirables}
\externaldocument[coding-]{coding}
\externaldocument[obsolete-]{obsolete}
\externaldocument[fdl-]{fdl}
\externaldocument[index-]{index}

% Theorem environments.
%
\theoremstyle{plain}
\newtheorem{theorem}[subsection]{Theorem}
\newtheorem{proposition}[subsection]{Proposition}
\newtheorem{lemma}[subsection]{Lemma}

\theoremstyle{definition}
\newtheorem{definition}[subsection]{Definition}
\newtheorem{example}[subsection]{Example}
\newtheorem{exercise}[subsection]{Exercise}
\newtheorem{situation}[subsection]{Situation}

\theoremstyle{remark}
\newtheorem{remark}[subsection]{Remark}
\newtheorem{remarks}[subsection]{Remarks}

\numberwithin{equation}{subsection}

% Macros
%
\def\lim{\mathop{\rm lim}\nolimits}
\def\colim{\mathop{\rm colim}\nolimits}
\def\Spec{\mathop{\rm Spec}}
\def\Hom{\mathop{\rm Hom}\nolimits}
\def\Ext{\mathop{\rm Ext}\nolimits}
\def\SheafHom{\mathop{\mathcal{H}\!{\it om}}\nolimits}
\def\SheafExt{\mathop{\mathcal{E}\!{\it xt}}\nolimits}
\def\Sch{\textit{Sch}}
\def\Mor{\mathop{\rm Mor}\nolimits}
\def\Ob{\mathop{\rm Ob}\nolimits}
\def\Sh{\mathop{\textit{Sh}}\nolimits}
\def\NL{\mathop{N\!L}\nolimits}
\def\proetale{{pro\text{-}\acute{e}tale}}
\def\etale{{\acute{e}tale}}
\def\QCoh{\textit{QCoh}}
\def\Ker{\mathop{\rm Ker}}
\def\Im{\mathop{\rm Im}}
\def\Coker{\mathop{\rm Coker}}
\def\Coim{\mathop{\rm Coim}}

%
% Macros for moduli stacks/spaces
%
\def\QCohstack{\mathcal{QC}\!{\it oh}}
\def\Cohstack{\mathcal{C}\!{\it oh}}
\def\Spacesstack{\mathcal{S}\!{\it paces}}
\def\Quotfunctor{{\rm Quot}}
\def\Hilbfunctor{{\rm Hilb}}
\def\Curvesstack{\mathcal{C}\!{\it urves}}
\def\Polarizedstack{\mathcal{P}\!{\it olarized}}
\def\Complexesstack{\mathcal{C}\!{\it omplexes}}
% \Pic is the operator that assigns to X its picard group, usage \Pic(X)
% \Picardstack_{X/B} denotes the Picard stack of X over B
% \Picardfunctor_{X/B} denotes the Picard functor of X over B
\def\Pic{\mathop{\rm Pic}\nolimits}
\def\Picardstack{\mathcal{P}\!{\it ic}}
\def\Picardfunctor{{\rm Pic}}
\def\Deformationcategory{\mathcal{D}\!{\it ef}}


% OK, start here.
%
\begin{document}

\title{Morphisms of algebraic spaces}


\maketitle

\phantomsection
\label{section-phantom}

\tableofcontents

\section{Introduction}
\label{section-introduction}

\noindent
In this chapter we introduce some types of morphisms of algebraic spaces.
A reference is \cite{Kn}.



\section{Properties of representable morphisms}
\label{section-representable}

\noindent
Let $S$ be a scheme.
Let $f : X \to Y$ be a representable morphism of algebraic spaces. In
Spaces, Section \ref{spaces-section-representable-properties}
we defined what it means for $f$ to
have property $\mathcal{P}$ in case $\mathcal{P}$ is a property
of morphisms of schemes which
\begin{enumerate}
\item is preserved under any base change,
see Schemes, Definition \ref{schemes-definition-preserved-by-base-change},
and
\item is fppf local on the base, see
Descent, Definition \ref{descent-definition-property-morphisms-local}.
\end{enumerate}
Namely, in this case we say $f$ has property $\mathcal{P}$ if and only
if for every scheme $U$ and any morphism $U \to Y$ the morphism of schemes
$X \times_Y U \to U$ has property $\mathcal{P}$.

\medskip\noindent
According to the lists in
Spaces, Section \ref{spaces-section-lists}
this applies to the following properties:
(1)(a) closed immersions,
(1)(b) open immersions,
(1)(c) quasi-compact immersions,
(2) quasi-compact,
(3) universally-closed,
(4) (quasi-)separated,
(5) monomorphism,
(6) surjective,
(7) radicial (or universally injective),
(8) affine,
(9) quasi-affine,
(10) (locally) of finite type,
(11) (locally) quasi-finite,
(12) (locally) of finite presentation,
(13) locally of finite type of relative dimension $d$,
(14) universally open,
(15) flat,
(16) syntomic,
(17) smooth,
(18) unramified,
(19) etale,
(20) proper,
(21) finite or integral,
(22) finite locally free, and
(23) immersion.

\medskip\noindent
In this chapter we will redefine these notions for not necessarily
representable morphisms of algebraic spaces. Whenever we do this we will make
sure that the new definition agrees with the old one, in order to avoid
ambiguity.

\medskip\noindent
Note that the definition above applies whenever $X$ is a scheme,
since a morphism from a scheme to an algebraic space is representatble.
And in particular it applies when both $X$ and $Y$ are schemes.
In Spaces, Lemma
\ref{spaces-lemma-morphism-schemes-gives-representable-transformation-property}
we have seen that in this case the definitions
match, and no ambiguity arise.

\medskip\noindent
Furthermore, in Spaces,
Lemma \ref{spaces-lemma-base-change-representable-transformations-property}
we have seen that the property of
representable morphisms of algebraic spaces so defined is stable under
arbitrary base change by a morphism of algebraic spaces.
And finally, in Spaces, Lemmas
\ref{spaces-lemma-composition-representable-transformations-property}
and
\ref{spaces-lemma-product-representable-transformations-property}
we have seen that if $\mathcal{P}$ is stable under compositions,
which holds for the properties
(1)(a), (1)(b), (1)(c), (2) -- (23), except (13) above, then
taking products of representable morphisms preserves property $\mathcal{P}$ 
and compositions of representable morphisms preserves property $\mathcal{P}$.

\medskip\noindent
We will use these facts below, and whenever we do we will simply refer
to this section as a reference.





\section{Immersions}
\label{section-immersions}

\noindent
Open, closed and locally closed immersions of algebraic spaces were defined in
Spaces, Section \ref{spaces-section-Zariski}.
Namely, such a morphism is by definition representable and has the
corresponding property as discussed in the previous section.

\medskip\noindent
In particular such a type of morphism is stable under base change,
composition and products of morphisms in the category of algebraic
spaces over $S$, see Section \ref{section-representable}.





\section{Quasi-compact morphisms}
\label{section-quasi-compact}

\noindent
By Section \ref{section-representable} we know what it means for
a representable morphism of algebraic spaces to be quasi-compact.
In order to formulate the definition for a general morphism
of algebraic spaces we make the following observation.

\begin{lemma}
\label{lemma-characterize-representable-quasi-compact}
Let $S$ be a scheme.
Let $f : X \to Y$ be a representable morphism of algebraic spaces over $S$.
The following are equivalent:
\begin{enumerate}
\item $f$ is quasi-compact, and
\item for every quasi-compact algebraic space $Z$ and any morphism
$Z \to Y$ the algebraic space $Z \times_Y X$ is quasi-compact.
\end{enumerate}
\end{lemma}

\begin{proof}
Assume (1), and let $Z \to Y$ be a morphism of algebraic spaces with
$Z$ quasi-compact. By
Properties of Spaces,
Definition \ref{spaces-properties-definition-quasi-compact}
there exists a quasi-compact scheme $U$ and a surjective etale
morphism $U \to Z$. Since $f$ is representable and quasi-compact
we see by definition that $U \times_Y X$ is a scheme, and that
$U \times_Y X \to U$ is quasi-compact. Hence $U \times_Y X$ is
a quasi-compact scheme. The morphism $U \times_Y X \to Z \times_Y X$
is etale and surjective (as the base change of the representable
etale and surjective morphism $U \to Z$, see
Section \ref{section-representable}).
Hence by definition $Z \times_Y X$ is quasi-compact.

\medskip\noindent
Assume (2). Let $Z \to Y$ be a morphism, where $Z$ is a scheme.
We have to show that $p : Z \times_Y X \to Z$ is quasi-compact.
Let $U \subset Z$ be affine open. Then $p^{-1}(U) = U \times_Y Z$
and the scheme $U \times_Y Z$ is quasi-compact by assumption (2).
Hence $p$ is quasi-compact, see
Schemes, Section \ref{schemes-section-quasi-compact}.
\end{proof}

\noindent
This motivates the following definition.

\begin{definition}
\label{definition-quasi-compact}
Let $S$ be a scheme.
Let $f : X \to Y$ be a morphism of algebraic spaces over $S$.
We say $f$ is {\it quasi-compact} if for every quasi-compact
algebraic space $Z$ and morphism $Z \to Y$ the fibre product
$Z \times_Y X$ is quasi-compact.
\end{definition}

\noindent
By Lemma \ref{lemma-characterize-representable-quasi-compact}
above this agrees with the already existing notion
for representable morphisms of algebraic spaces.

\begin{lemma}
\label{lemma-base-change-quasi-compact}
The base change of a quasi-compact morphism of algebraic spaces
by any morphism of algebraic spaces is quasi-compact.
\end{lemma}

\begin{proof}
Omitted.
\end{proof}

\begin{lemma}
\label{lemma-composition-quasi-compact}
The composition of a pair of quasi-compact morphisms of algebraic spaces
is quasi-compact.
\end{lemma}

\begin{proof}
Omitted.
\end{proof}





\section{Universally closed morphisms}
\label{section-universally-closed}

\noindent
Here is the natural definition.

\begin{definition}
\label{definition-universally-closed}
Let $S$ be a scheme. Let $f : X \to Y$ be a morphism of algebraic spaces
over $S$. We say $f$ is {\it universally closed} if for every morphism
of algebraic spaces $Z \to Y$ the morphism of topological spaces
$$
|Z \times_Y X| \to |Z|
$$
is closed.
\end{definition}

\begin{lemma}
\label{lemma-base-change-universally-closed}
The base change of a universally closed morphism of algebraic spaces
by any morphism of algebraic spaces is universally closed.
\end{lemma}

\begin{proof}
This is immediate from the definition.
\end{proof}

\begin{lemma}
\label{lemma-characterize-universally-closed}
Let $S$ be a scheme. Let $f : X \to Y$ be a morphism of algebraic spaces
over $S$. The following are equivalent
\begin{enumerate}
\item $f$ is universally closed,
\item for every scheme $Z$ and every morphism $Z \to Y$
the projection $|Z \times_Y X| \to |Z|$ is closed,
\item for every affine scheme $Z$ and every morphism $Z \to Y$
the projection $|Z \times_Y X| \to |Z|$ is closed, and
\item there exists a scheme $V$ and a surjective etale morphism
$V \to Y$ such that $V \times_Y X \to V$ is a universally closed morpism
of algebraic spaces.
\end{enumerate}
\end{lemma}

\begin{proof}
We omit the proof that (1) implies (2), and that (2) implies (3).

\medskip\noindent
Assume (3). Choose a surjective etale morphism $V \to Y$.
We are going to show that $V \times_Y X \to V$ is a universally
closed morpism of algebraic spaces. Let $Z \to V$ be a morphism
from an algebraic space to $V$. Let $W \to Z$ be a surjective etale
morphism where $W = \coprod W_i$ is a disjoint union of affine schemes, see
Properties of Spaces,
Lemma \ref{spaces-properties-lemma-cover-by-union-affines}.
Then we have the following commutative diagram
$$
\xymatrix{
\coprod_i |W_i \times_Y X| \ar@{=}[r] \ar[d] &
|W \times_Y X| \ar[r] \ar[d] &
|Z \times_Y X| \ar[d] \ar@{=}[r] &
|Z \times_V (V \times_Y X)| \ar[ld] \\
\coprod |W_i| \ar@{=}[r] &
|W| \ar[r] &
|Z|
}
$$
We have to show the south-east arrow is closed. The middle horizontal
arrows are surjective and open
(Properties of Spaces, Lemma \ref{spaces-properties-lemma-etale-open}).
By assumption (3), and the fact that
$W_i$ is affine we see that the left vertical arrows are closed. Hence
it follows that the right vertical arrow is closed.

\medskip\noindent
Assume (4). We will show that $f$ is universally closed.
Let $Z \to Y$ be a morphism of algebraic spaces. Consider the
diagram
$$
\xymatrix{
|(V \times_Y Z) \times_V (V \times_Y X)| \ar@{=}[r] \ar[rd] &
|V \times_Y X| \ar[r] \ar[d] &
|Z \times_Y X| \ar[d] \\
 &
|V \times_Y Z| \ar[r] &
|Z|
}
$$
The south-west arrow is closed by assumption. The horizontal arrows are
surjective and open because the corresponding morphisms of
algebraic spaces are etale (see
Properties of spaces, Lemma \ref{spaces-properties-lemma-etale-open}).
It follows that the right vertical arrow is closed.
\end{proof}

\begin{lemma}
\label{lemma-composition-universally-closed}
The composition of a pair of universally closed morphisms of algebraic spaces
is universally closed.
\end{lemma}

\begin{proof}
Omitted.
\end{proof}

\begin{example}
\label{example-strange-universally-closed}
Strange example of a universally closed morphism.
Let $\mathbf{Q} \subset k$ be a field of characteristic zero.
Let $X = [\mathbf{A}^1_k/\mathbf{Z}]$ as in
Spaces, Example \ref{spaces-example-affine-line-translation}.
We claim the structure morphism $p : X \to \text{Spec}(k)$
is universally closed.
Namely, if $Z/k$ is a scheme, and $T \subset |X \times_k Z|$ is closed,
then $T$ corresponds to a $\mathbf{Z}$-invariant closed subset of
$T' \subset |\mathbf{A}^1 \times Z|$. It is easy to see that
this implies that $T'$ is the inverse image of a subset $T''$ of
$Z$. By
Morphisms, Lemma \ref{morphisms-lemma-fpqc-quotient-topology}
we have that $T'' \subset Z$ is closed.
Of course $T''$ is the image of $T$. Hence $p$ is universally
closed by Lemma \ref{lemma-characterize-universally-closed}.
\end{example}




\section{Valuative criteria}
\label{section-valuative}


\begin{definition}
\label{definition-valuative-criterion}
Let $S$ be a scheme.
Let $f : X \to Y$ be a morphism of algebraic spaces over $S$.
We say $f$ {\it satisfies the uniqueness part of the valuative criterion}
if given any commutative solid diagram
$$
\xymatrix{
\text{Spec}(K) \ar[r] \ar[d] & X \ar[d] \\
\text{Spec}(A) \ar[r] \ar@{-->}[ru] & Y
}
$$
where $A$ is a valuation ring with field of fractions $K$, there exists
at most one dotted arrow (without requiring existence).
We say $f$ {\it satisfies the existence part of the valuative criterion}
if given any solid diagram as above there exists an extension
$K \subset K'$ of fields, a valuation ring $A' \subset K'$ dominating
$A$ and a morphism $\text{Spec}(A') \to X$ such that the following
diagram commutes
$$
\xymatrix{
\text{Spec}(K') \ar[r] \ar[d] & \text{Spec}(K) \ar[r] & X \ar[d] \\
\text{Spec}(A') \ar[r] \ar[rru] &\text{Spec}(A) \ar[r] & Y
}
$$
We say $f$ {\it satisfies the valuative criterion}
if $f$ satisfies both the existence and uniqueness part.
\end{definition}

\begin{lemma}
\label{lemma-valuative-criterion-representable}
Let $S$ be a scheme.
Let $f : X \to Y$ be a morphism of algebraic spaces over $S$.
Assume $f$ is representable. The following are equivalent
\begin{enumerate}
\item $f$ satisfies the existence part of the valuation criterion
as in Definition \ref{definition-valuative-criterion} above, and
\item given any commutative solid diagram
$$
\xymatrix{
\text{Spec}(K) \ar[r] \ar[d] & X \ar[d] \\
\text{Spec}(A) \ar[r] \ar@{-->}[ru] & Y
}
$$
where $A$ is a valuation ring with field of fractions $K$, there exists
a dotted arrow, i.e., $f$ satisfies the existence part of the valuative
criterion as in
Schemes, Definition \ref{schemes-definition-valuative-criterion}.
\end{enumerate}
\end{lemma}

\begin{proof}
It suffices to show that given a commutative diagram of the form
$$
\xymatrix{
\text{Spec}(K') \ar[r] \ar[d] & \text{Spec}(K) \ar[r] & X \ar[d] \\
\text{Spec}(A') \ar[r] \ar[rru]^\varphi &\text{Spec}(A) \ar[r] & Y
}
$$
as in Definition \ref{definition-valuative-criterion}, then we can
find a morphism $\text{Spec}(A) \to X$ fitting into the diagram too.
Set $X_A = \text{Spec}(A) \times_Y Y$. As $f$ is representable we see
that $X_A$ is a scheme. The morphism $\varphi$ gives a morphism
$\varphi' : \text{Spec}(A') \to X_A$. Let $x \in X_A$ be the image of
the closed point of $\varphi' : \text{Spec}(A') \to X_A$. Then we
have the following commutative diagram of rings
$$
\xymatrix{
K' & K \ar[l] & \mathcal{O}_{X_A, x} \ar[l] \ar[lld] \\
A' \ar[u] & A \ar[l] & A \ar[l] \ar[u]
}
$$
Since $A$ is a valuation ring, and since $A'$ dominates $A$, we see
that $K \cap A' = A$. Hence the ring map $\mathcal{O}_{X_A, x} \to K$
has image contained in $A$. Whence a morphism $\text{Spec}(A) \to X_A$ (see
Schemes, Section \ref{schemes-section-points})
as desired.
\end{proof}

\begin{lemma}
\label{lemma-base-change-valuative-criteria}
The base change of a morphism of algebraic spaces which satisfies the
(existence part of, resp.\ uniqueness part of) the valuative criterion
by any morphism of algebraic spaces satisfies the
(existence part of, resp.\ uniqueness part of) the valuative criterion.
\end{lemma}

\begin{proof}
Let $f : X \to Y$ be a morphism of algebraic spaces over the scheme $S$.
Let $Z \to Y$ be any morphism of algebraic spaces over $S$.
Consider a solid commutative diagram of the following shape
$$
\xymatrix{
\text{Spec}(K) \ar[r] \ar[d] & Z \times_Y X \ar[r] \ar[d] & X \ar[d] \\
\text{Spec}(A) \ar[r] \ar@{-->}[ru] \ar@{-->}[rru] & Z \ar[r] & Y
}
$$
Then the set of north-west dotted arrows making the diagram commute
is in 1-1 correspondence with the set of west-north-west dotted arrows
making the diagram commute. This proves the lemma in the case of
``uniqueness''. For the existence part, assume $f$ satisfies the existence
part of the valuative criterion. If we are given a solid commutative
diagram as above, then by assumption there exists an extension $K \subset K'$
of fields and a valuation ring $A' \subset K'$ dominating $A$ and
a morphism $\text{Spec}(A') \to X$ fitting into the following commutative
diagram
$$
\xymatrix{
\text{Spec}(K') \ar[r] \ar[d] &
\text{Spec}(K) \ar[r] & Z \times_Y X \ar[r] & X \ar[d] \\
\text{Spec}(A') \ar[r] \ar[rrru] & \text{Spec}(A) \ar[r] & Z \ar[r] & Y
}
$$
And by the remarks above the skew arrow corresponds to an arrow
$\text{Spec}(A') \to Z \times_Y X$ as desired.
\end{proof}

\begin{lemma}
\label{lemma-composition-valuative-criteria}
The composition of two morphisms of algebraic spaces which satisfy the
(existence part of, resp.\ uniqueness part of) the valuative criterion
satisfies the (existence part of, resp.\ uniqueness part of) the valuative
criterion.
\end{lemma}

\begin{proof}
Let $f : X \to Y$, $g : Y \to Z$ be morphisms of algebraic spaces over the
scheme $S$. Consider a solid commutative diagram of the following shape
$$
\xymatrix{
\text{Spec}(K) \ar[dd] \ar[r] & X \ar[d]^f \\
& Y \ar[d]^g \\
\text{Spec}(A) \ar[r] \ar@{-->}[ru] \ar@{-->}[ruu] & Z
}
$$
If we have the uniqueness part for $g$, then there exists at
most one north-west dotted arrow making the diagram commute.
If we also have the uniqueness part for $f$, then we have
at most one north-north-west dotted arrow making the diagram
commute. The proof in the existence case comes from contemplating
the following diagram
$$
\xymatrix{
\text{Spec}(K'') \ar[r] \ar[dd] &
\text{Spec}(K') \ar[r] &
\text{Spec}(K) \ar[r] &
X \ar[d]^f \\
& & & Y \ar[d]^g \\
\text{Spec}(A'') \ar[r] \ar[rrruu] &
\text{Spec}(A') \ar[r] \ar[rru] &
\text{Spec}(A) \ar[r] &
Z
}
$$
Namely, the existence part for $g$ gives us the extension $K'$, the
valuation ring $A'$ and the arrow $\text{Spec}(A') \to Y$, whereupon
the existence part for $f$ gives us the extension $K''$, the
valuation ring $A''$ and the arrow $\text{Spec}(A'') \to X$.
\end{proof}






\section{Separation axioms}
\label{section-separation-axioms}

\noindent
It makes sense to list some a priori properties of the diagonal of
a morphism of algebraic spaces.

\begin{lemma}
\label{lemma-properties-diagonal}
Let $S$ be a scheme contained in $\textit{Sch}_{fppf}$.
Let $f : X \to Y$ be a morphism of algebraic spaces over $S$.
Let $\Delta_{X/Y} : X \to X \times_Y X$ be the diagonal morphism.
Then
\begin{enumerate}
\item $\Delta_{X/Y}$ is representable,
\item $\Delta_{X/Y}$ is locally of finite type,
\item $\Delta_{X/Y}$ is a monomorphism,
\item $\Delta_{X/Y}$ is separated, and
\item $\Delta_{X/Y}$ is locally quasi-finite.
\end{enumerate}
\end{lemma}

\begin{proof}
We are going to use the fact that $\Delta_{X/S}$ is
representable (by definition of an algebraic space) and that
it satisfies properties (2) -- (5), see
Spaces, Lemma \ref{spaces-lemma-properties-diagonal}.
Note that we have a factorization
$$
X
\longrightarrow
X \times_Y X
\longrightarrow
X \times_S X
$$
of the diagonal $\Delta_{X/S} : X \to X \times_S X$. Since
$X \times_Y X \to X \times_S X$ is a monomorphism, and since
$\Delta_{X/S}$ is representable, it follows formally that
$\Delta_{X/Y}$ is representable. In particular, the rest of
the statements now make sense, by the discussion in
Section \ref{section-representable}.

\medskip\noindent
Choose a surjective etale morphism $U \to X$, with $U$ a scheme.
Consider the diagram
$$
\xymatrix{
R = U \times_X U \ar[r] \ar[d] &
U \times_Y U \ar[d] \ar[r] &
U \times_S U \ar[d] \\
X \ar[r] & X \times_Y X \ar[r] & X \times_S X
}
$$
Both squares are cartesian, hence so is the outer retangle.
The top row consists of schemes, and the vertical arrows
are surjective etale morphisms. By
Spaces, Lemma \ref{spaces-lemma-representable-morphisms-spaces-property}
the properties (2) -- (5) for $\Delta_{X/Y}$ are equivalent to those of
$R \to U \times_Y U$. In the proof of
Spaces, Lemma \ref{spaces-lemma-properties-diagonal}
we have seen that $R \to U \times_S U$ has properties (2) -- (5).
The morphism $U \times_Y U \to U \times_S U$ is a monomorphism
of schemes. These facts imply that $R \to U \times_S U$ have
properties (2) -- (5).

\medskip\noindent
Namely: For (3), note that $R \to U \times_Y U$
is a monomorphism as the composition
$R \to U \times_S U$ is a monomorphism. For (2), note that
$R \to U \times_Y U$ is locally of finite type, as the
composition $R \to U \times_S U$ is locally of finite type
(Morphisms, Lemma \ref{morphisms-lemma-permanence-finite-type}).
A monomorphism which is locally of finite type is locally quasi-finite
because it has finite fibres
(Morphisms, Lemma \ref{morphisms-lemma-finite-fibre}), hence (5).
A monomorphism is separated
(Schemes, Lemma \ref{schemes-lemma-monomorphism-separated}), hence (4).
\end{proof}

\begin{definition}
\label{definition-separated}
Let $S$ be a scheme.
Let $f : X \to Y$ be a morphism of algebraic spaces over $S$.
Let $\Delta_{X/Y} : X \to X \times_Y X$ be the diagonal morphism.
\begin{enumerate}
\item We say $f$ is {\it separated} if $\Delta_{X/Y}$ is a closed immersion.
\item We say $f$ is {\it weakly locally separated}\footnote{This is probably
nonstandard notation.} if $\Delta_{X/Y}$ is an immersion.
\item We say $f$ is {\it locally separated} if $\Delta_{X/Y}$ is a
quasi-compact immersion.
\item We say $f$ is {\it quasi-separated} if $\Delta_{X/Y}$ is quasi-compact.
\item We say $f$ is {\it Zariski locally quasi-separated}\footnote{This
definition was suggested by B.\ Conrad.} if there
exists a Zariski covering $X = \bigcup_{i \in I} X_i$ such that
each $\Delta_{X_i/Y}$ is quasi-separated.
\end{enumerate}
\end{definition}






























\section{Other chapters}

\begin{multicols}{2}
\begin{enumerate}
\item \hyperref[introduction-section-phantom]{Introduction}
\item \hyperref[conventions-section-phantom]{Conventions}
\item \hyperref[sets-section-phantom]{Set Theory}
\item \hyperref[categories-section-phantom]{Categories}
\item \hyperref[topology-section-phantom]{Topology}
\item \hyperref[sheaves-section-phantom]{Sheaves on Spaces}
\item \hyperref[algebra-section-phantom]{Commutative Algebra}
\item \hyperref[sites-section-phantom]{Sites and Sheaves}
\item \hyperref[homology-section-phantom]{Homological Algebra}
\item \hyperref[derived-section-phantom]{Derived Categories}
\item \hyperref[more-algebra-section-phantom]{More Algebra}
\item \hyperref[simplicial-section-phantom]{Simplicial Methods}
\item \hyperref[modules-section-phantom]{Sheaves of Modules}
\item \hyperref[sites-modules-section-phantom]{Modules on Sites}
\item \hyperref[injectives-section-phantom]{Injectives}
\item \hyperref[cohomology-section-phantom]{Cohomology of Sheaves}
\item \hyperref[sites-cohomology-section-phantom]{Cohomology on Sites}
\item \hyperref[hypercovering-section-phantom]{Hypercoverings}
\item \hyperref[schemes-section-phantom]{Schemes}
\item \hyperref[constructions-section-phantom]{Constructions of Schemes}
\item \hyperref[properties-section-phantom]{Properties of Schemes}
\item \hyperref[morphisms-section-phantom]{Morphisms of Schemes}
\item \hyperref[coherent-section-phantom]{Coherent Cohomology}
\item \hyperref[divisors-section-phantom]{Divisors}
\item \hyperref[limits-section-phantom]{Limits of Schemes}
\item \hyperref[varieties-section-phantom]{Varieties}
\item \hyperref[chow-section-phantom]{Chow Homology}
\item \hyperref[topologies-section-phantom]{Topologies on Schemes}
\item \hyperref[descent-section-phantom]{Descent}
\item \hyperref[more-morphisms-section-phantom]{More on Morphisms}
\item \hyperref[flat-section-phantom]{More on Flatness}
\item \hyperref[groupoids-section-phantom]{Groupoid Schemes}
\item \hyperref[more-groupoids-section-phantom]{More on Groupoid Schemes}
\item \hyperref[etale-section-phantom]{\'Etale Morphisms of Schemes}
\item \hyperref[etale-cohomology-section-phantom]{\'Etale Cohomology}
\item \hyperref[spaces-section-phantom]{Algebraic Spaces}
\item \hyperref[spaces-properties-section-phantom]{Properties of Algebraic Spaces}
\item \hyperref[spaces-morphisms-section-phantom]{Morphisms of Algebraic Spaces}
\item \hyperref[spaces-topologies-section-phantom]{Topologies on Algebraic Spaces}
\item \hyperref[spaces-descent-section-phantom]{Descent and Algebraic Spaces}
\item \hyperref[spaces-more-morphisms-section-phantom]{More on Morphisms of Spaces}
\item \hyperref[quot-section-phantom]{Quot and Hilbert Spaces}
\item \hyperref[stacks-section-phantom]{Stacks}
\item \hyperref[spaces-groupoids-section-phantom]{Groupoids in Algebraic Spaces}
\item \hyperref[spaces-more-groupoids-section-phantom]{More on Groupoids in Spaces}
\item \hyperref[bootstrap-section-phantom]{Bootstrap}
\item \hyperref[examples-stacks-section-phantom]{Examples of Stacks}
\item \hyperref[groupoids-quotients-section-phantom]{Quotients of Groupoids}
\item \hyperref[algebraic-section-phantom]{Algebraic Stacks}
\item \hyperref[criteria-section-phantom]{Criteria for Representability}
\item \hyperref[stacks-properties-section-phantom]{Properties of Algebraic Stacks}
\item \hyperref[stacks-morphisms-section-phantom]{Morphisms of Algebraic Stacks}
\item \hyperref[examples-section-phantom]{Examples}
\item \hyperref[exercises-section-phantom]{Exercises}
\item \hyperref[guide-section-phantom]{Guide to Literature}
\item \hyperref[desirables-section-phantom]{Desirables}
\item \hyperref[coding-section-phantom]{Coding Style}
\item \hyperref[fdl-section-phantom]{GNU Free Documentation License}
\item \hyperref[index-section-phantom]{Auto Generated Index}
\end{enumerate}
\end{multicols}


\bibliography{my}
\bibliographystyle{amsalpha}

\end{document}
