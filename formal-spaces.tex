\IfFileExists{stacks-project.cls}{%
\documentclass{stacks-project}
}{%
\documentclass{amsart}
}

% The following AMS packages are automatically loaded with
% the amsart documentclass:
%\usepackage{amsmath}
%\usepackage{amssymb}
%\usepackage{amsthm}

% For dealing with references we use the comment environment
\usepackage{verbatim}
\newenvironment{reference}{\comment}{\endcomment}
%\newenvironment{reference}{}{}
\newenvironment{slogan}{\comment}{\endcomment}
\newenvironment{history}{\comment}{\endcomment}

% For commutative diagrams you can use
% \usepackage{amscd}
\usepackage[all]{xy}

% We use 2cell for 2-commutative diagrams.
\xyoption{2cell}
\UseAllTwocells

% To put source file link in headers.
% Change "template.tex" to "this_filename.tex"
% \usepackage{fancyhdr}
% \pagestyle{fancy}
% \lhead{}
% \chead{}
% \rhead{Source file: \url{template.tex}}
% \lfoot{}
% \cfoot{\thepage}
% \rfoot{}
% \renewcommand{\headrulewidth}{0pt}
% \renewcommand{\footrulewidth}{0pt}
% \renewcommand{\headheight}{12pt}

\usepackage{multicol}

% For cross-file-references
\usepackage{xr-hyper}

% Package for hypertext links:
\usepackage{hyperref}

% For any local file, say "hello.tex" you want to link to please
% use \externaldocument[hello-]{hello}
\externaldocument[introduction-]{introduction}
\externaldocument[conventions-]{conventions}
\externaldocument[sets-]{sets}
\externaldocument[categories-]{categories}
\externaldocument[topology-]{topology}
\externaldocument[sheaves-]{sheaves}
\externaldocument[sites-]{sites}
\externaldocument[stacks-]{stacks}
\externaldocument[fields-]{fields}
\externaldocument[algebra-]{algebra}
\externaldocument[brauer-]{brauer}
\externaldocument[homology-]{homology}
\externaldocument[derived-]{derived}
\externaldocument[simplicial-]{simplicial}
\externaldocument[more-algebra-]{more-algebra}
\externaldocument[smoothing-]{smoothing}
\externaldocument[modules-]{modules}
\externaldocument[sites-modules-]{sites-modules}
\externaldocument[injectives-]{injectives}
\externaldocument[cohomology-]{cohomology}
\externaldocument[sites-cohomology-]{sites-cohomology}
\externaldocument[dga-]{dga}
\externaldocument[dpa-]{dpa}
\externaldocument[hypercovering-]{hypercovering}
\externaldocument[schemes-]{schemes}
\externaldocument[constructions-]{constructions}
\externaldocument[properties-]{properties}
\externaldocument[morphisms-]{morphisms}
\externaldocument[coherent-]{coherent}
\externaldocument[divisors-]{divisors}
\externaldocument[limits-]{limits}
\externaldocument[varieties-]{varieties}
\externaldocument[topologies-]{topologies}
\externaldocument[descent-]{descent}
\externaldocument[perfect-]{perfect}
\externaldocument[more-morphisms-]{more-morphisms}
\externaldocument[flat-]{flat}
\externaldocument[groupoids-]{groupoids}
\externaldocument[more-groupoids-]{more-groupoids}
\externaldocument[etale-]{etale}
\externaldocument[chow-]{chow}
\externaldocument[intersection-]{intersection}
\externaldocument[pic-]{pic}
\externaldocument[adequate-]{adequate}
\externaldocument[dualizing-]{dualizing}
\externaldocument[duality-]{duality}
\externaldocument[discriminant-]{discriminant}
\externaldocument[local-cohomology-]{local-cohomology}
\externaldocument[curves-]{curves}
\externaldocument[resolve-]{resolve}
\externaldocument[models-]{models}
\externaldocument[pione-]{pione}
\externaldocument[etale-cohomology-]{etale-cohomology}
\externaldocument[proetale-]{proetale}
\externaldocument[crystalline-]{crystalline}
\externaldocument[spaces-]{spaces}
\externaldocument[spaces-properties-]{spaces-properties}
\externaldocument[spaces-morphisms-]{spaces-morphisms}
\externaldocument[decent-spaces-]{decent-spaces}
\externaldocument[spaces-cohomology-]{spaces-cohomology}
\externaldocument[spaces-limits-]{spaces-limits}
\externaldocument[spaces-divisors-]{spaces-divisors}
\externaldocument[spaces-over-fields-]{spaces-over-fields}
\externaldocument[spaces-topologies-]{spaces-topologies}
\externaldocument[spaces-descent-]{spaces-descent}
\externaldocument[spaces-perfect-]{spaces-perfect}
\externaldocument[spaces-more-morphisms-]{spaces-more-morphisms}
\externaldocument[spaces-flat-]{spaces-flat}
\externaldocument[spaces-groupoids-]{spaces-groupoids}
\externaldocument[spaces-more-groupoids-]{spaces-more-groupoids}
\externaldocument[bootstrap-]{bootstrap}
\externaldocument[spaces-pushouts-]{spaces-pushouts}
\externaldocument[groupoids-quotients-]{groupoids-quotients}
\externaldocument[spaces-more-cohomology-]{spaces-more-cohomology}
\externaldocument[spaces-simplicial-]{spaces-simplicial}
\externaldocument[formal-spaces-]{formal-spaces}
\externaldocument[restricted-]{restricted}
\externaldocument[spaces-resolve-]{spaces-resolve}
\externaldocument[formal-defos-]{formal-defos}
\externaldocument[defos-]{defos}
\externaldocument[cotangent-]{cotangent}
\externaldocument[examples-defos-]{examples-defos}
\externaldocument[algebraic-]{algebraic}
\externaldocument[examples-stacks-]{examples-stacks}
\externaldocument[stacks-sheaves-]{stacks-sheaves}
\externaldocument[criteria-]{criteria}
\externaldocument[artin-]{artin}
\externaldocument[quot-]{quot}
\externaldocument[stacks-properties-]{stacks-properties}
\externaldocument[stacks-morphisms-]{stacks-morphisms}
\externaldocument[stacks-limits-]{stacks-limits}
\externaldocument[stacks-cohomology-]{stacks-cohomology}
\externaldocument[stacks-perfect-]{stacks-perfect}
\externaldocument[stacks-introduction-]{stacks-introduction}
\externaldocument[stacks-more-morphisms-]{stacks-more-morphisms}
\externaldocument[stacks-geometry-]{stacks-geometry}
\externaldocument[moduli-]{moduli}
\externaldocument[moduli-curves-]{moduli-curves}
\externaldocument[examples-]{examples}
\externaldocument[exercises-]{exercises}
\externaldocument[guide-]{guide}
\externaldocument[desirables-]{desirables}
\externaldocument[coding-]{coding}
\externaldocument[obsolete-]{obsolete}
\externaldocument[fdl-]{fdl}
\externaldocument[index-]{index}

% Theorem environments.
%
\theoremstyle{plain}
\newtheorem{theorem}[subsection]{Theorem}
\newtheorem{proposition}[subsection]{Proposition}
\newtheorem{lemma}[subsection]{Lemma}

\theoremstyle{definition}
\newtheorem{definition}[subsection]{Definition}
\newtheorem{example}[subsection]{Example}
\newtheorem{exercise}[subsection]{Exercise}
\newtheorem{situation}[subsection]{Situation}

\theoremstyle{remark}
\newtheorem{remark}[subsection]{Remark}
\newtheorem{remarks}[subsection]{Remarks}

\numberwithin{equation}{subsection}

% Macros
%
\def\lim{\mathop{\rm lim}\nolimits}
\def\colim{\mathop{\rm colim}\nolimits}
\def\Spec{\mathop{\rm Spec}}
\def\Hom{\mathop{\rm Hom}\nolimits}
\def\Ext{\mathop{\rm Ext}\nolimits}
\def\SheafHom{\mathop{\mathcal{H}\!{\it om}}\nolimits}
\def\SheafExt{\mathop{\mathcal{E}\!{\it xt}}\nolimits}
\def\Sch{\textit{Sch}}
\def\Mor{\mathop{\rm Mor}\nolimits}
\def\Ob{\mathop{\rm Ob}\nolimits}
\def\Sh{\mathop{\textit{Sh}}\nolimits}
\def\NL{\mathop{N\!L}\nolimits}
\def\proetale{{pro\text{-}\acute{e}tale}}
\def\etale{{\acute{e}tale}}
\def\QCoh{\textit{QCoh}}
\def\Ker{\mathop{\rm Ker}}
\def\Im{\mathop{\rm Im}}
\def\Coker{\mathop{\rm Coker}}
\def\Coim{\mathop{\rm Coim}}

%
% Macros for moduli stacks/spaces
%
\def\QCohstack{\mathcal{QC}\!{\it oh}}
\def\Cohstack{\mathcal{C}\!{\it oh}}
\def\Spacesstack{\mathcal{S}\!{\it paces}}
\def\Quotfunctor{{\rm Quot}}
\def\Hilbfunctor{{\rm Hilb}}
\def\Curvesstack{\mathcal{C}\!{\it urves}}
\def\Polarizedstack{\mathcal{P}\!{\it olarized}}
\def\Complexesstack{\mathcal{C}\!{\it omplexes}}
% \Pic is the operator that assigns to X its picard group, usage \Pic(X)
% \Picardstack_{X/B} denotes the Picard stack of X over B
% \Picardfunctor_{X/B} denotes the Picard functor of X over B
\def\Pic{\mathop{\rm Pic}\nolimits}
\def\Picardstack{\mathcal{P}\!{\it ic}}
\def\Picardfunctor{{\rm Pic}}
\def\Deformationcategory{\mathcal{D}\!{\it ef}}


% OK, start here.
%
\begin{document}

\title{Formal Algebraic Spaces}


\maketitle

\phantomsection
\label{section-phantom}

\tableofcontents

\section{Introduction}
\label{section-introduction}

\noindent
Formal schemes were introduced in \cite{EGA}. A more general version
of formal schemes was introduced in \cite{McQuillan} and another
in \cite{Yasuda}. Formal algebraic spaces were introduced in \cite{Kn}.
This chapter introduces the notion of formal algebraic spaces
we will work with. Our definition is general enough to allow most
classes of formal schemes/spaces in the literature as subcategories.














\section{Formal schemes \`a la EGA}
\label{section-formal-schemes-EGA}

\noindent
In this section we review the construction of formal schemes in \cite{EGA}.
This notion, although very useful in algebraic geometry,
may not always be the correct one to consider. Perhaps it is better to say
that in the setup of the theory a number choices are made, where for
different purposes others might work better. And indeed in the literature
one can find many different closely related theories
adapted to the problem the authors may want to consider. Still, one
of the major advantages of the theory as sketched here is that one
gets to work with definite geometric objects.

\medskip\noindent
Before we start we should point out an issue with the sheaf condition
for sheaves of topological rings or more generally sheaves of topological
spaces. Namely, the big categories
\begin{enumerate}
\item category of topological spaces,
\item category of topological abelian groups,
\item category of topological rings,
\item category of topological modules over a given topological ring,
\end{enumerate}
endowed with their natural forgetful functors to $\textit{Sets}$ are not
examples of types of algebraic structures as defined in
Sheaves, Section \ref{sheaves-section-algebraic-structures}.
Thus we cannot blithely use the machinery developed in that
chapter to them. On the other hand, each of the categories
listed above has products, so we can define the notion of a
sheaf as in Sheaves, Definition
\ref{sheaves-definition-sheaf-values-in-category}.
Moreover, products still commute with the forgetful functor
to $\textit{Sets}$ so the underlying presheaf of sets
of a sheaf of topological spaces (for example) is a sheaf of
sets. The key difference is that for an open covering
$U = \bigcup_{i \in I} U_i$ the diagram
$$
\xymatrix{
\mathcal{F}(U) \ar[r]
&
\prod\nolimits_{i\in I}
\mathcal{F}(U_i)
\ar@<1ex>[r] \ar@<-1ex>[r]
&
\prod\nolimits_{(i_0, i_1) \in I \times I}
\mathcal{F}(U_{i_0} \cap U_{i_1})
}
$$
has to be an equalizer diagram in the category of topological
spaces, groups, rings, modules, i.e., that the first map identifies
$\mathcal{F}(U)$ with a subspace of $\prod_{i \in I} \mathcal{F}(U_i)$
which is endowed with the product topology.

\medskip\noindent
The stalk $\mathcal{F}_x$ of a sheaf $\mathcal{F}$
of topological spaces, abelian groups, rings, or modules
at a point $x \in X$ is defined as the colimit over open neighbourhoods
$$
\mathcal{F}_x = \colim_{x\in U} \mathcal{F}(U)
$$
in the corresponding category, which is the same as taking
the colimit on the level of underlying sheaves of sets, groups,
rings, or modules but comes equipped with a topology (the colimit
topology). One can sheafify presheaves of topological spaces,
abelian groups, rings, or modules and taking stalks commuts
with this operation, see
Remark \ref{remark-sheafification-of-presheaves-in-top}.

\medskip\noindent
Let $f : X \to Y$ be a continuous map of topological spaces.
There is a functor $f_*$ from the category of sheaves of topological
spaces, abelian groups, rings, modules, to the corresponding category
of sheaves on $Y$ which is defined by setting
$f_*\mathcal{F}(V) = \mathcal{F}(f^{-1}V)$ as usual.
(We delay discussing the pullback in this setting till later.)
We define the notion of an $f$-map $\xi : \mathcal{G} \to \mathcal{F}$
between a sheaf of topoligical spaces $\mathcal{G}$ on $Y$ and
a sheaf of topological spaces $\mathcal{F}$ on $X$ in exactly the
same manner as in Sheaves, Definition \ref{sheaves-definition-f-map}
with the additional constraint that
$\xi_V : \mathcal{G}(V) \to \mathcal{F}(f^{-1}V)$ be continuous
for every open $V \subset Y$. We have
$$
\{f\text{-maps from }\mathcal{G}\text{ to }\mathcal{F}\} =
\text{Mor}_{\Sh(Y, \textit{Top})}(\mathcal{G}, f_*\mathcal{F})
$$
as in Sheaves, Lemma \ref{sheaves-lemma-f-map}. Similarly for
sheaves of topological abelian groups, rings, modules. Finally,
let $\xi : \mathcal{G} \to \mathcal{F}$ be an $f$-map as above.
Then given $x \in X$ with image $y = f(x)$ there is a continuous
map
$$
\xi_x : \mathcal{G}_y \longrightarrow \mathcal{F}_x
$$
of stalks defined in exactly the same manner as in the discussion
following Sheaves, Definition \ref{sheaves-definition-composition-f-maps}.

\medskip\noindent
Using the discussion above, we can define a category $LTRS$ of
``locally topologically ringed spaces''. An object is a pair
$(X, \mathcal{O}_X)$ consisting of a topological space
$X$ and a sheaf of topological rings $\mathcal{O}_X$ whose stalks
$\mathcal{O}_{X, x}$ are local rings (if one forgets about the topology).
A morphism $(X, \mathcal{O}_X) \to (Y, \mathcal{O}_Y)$ of
$LTRS$ is a pair $(f, f^\sharp)$ where $f : X \to Y$ is a continuous
map of topological spaces and $f^\sharp : \mathcal{O}_Y \to \mathcal{O}_X$
is an $f$-map such that for every $x \in X$ the induced map
$$
f^\sharp_x : \mathcal{O}_{Y, f(x)} \longrightarrow \mathcal{O}_{X, x}
$$
is a local homomorphism of local rings (forgetting about the topologies).
The composition works in exactly the same manner as composition of
morphisms of locally ringed spaces.

\medskip\noindent
Assume now that the topological space $X$ has a basis consisting
of quasi-compact opens. Given a sheaf $\mathcal{G}$ of sets, abelian groups,
rings, modules over a ring, one can endow $\mathcal{G}$ with
the structure of a sheaf of topological spaces, abelian groups,
rings, modules. Namely, if $U \subset X$ is quasi-compact open,
we endow $\mathcal{F}(U)$ with the discrete topology. If $U \subset X$
is arbitrary, then we choose an open covering $U = \bigcup_{i \in I} U_i$
by quasi-compact opens and we endow $\mathcal{F}(U)$ with
the induced topology from $\prod_{i \in I} \mathcal{F}(U_i)$
(as we should do according to our discussion above).
The reader may verify (omitted) that we obtain a sheaf of topological
spaces, groups, rings, modules in this fashion. Let us say
that a sheaf of topological spaces, groups, rings, modules is
{\it pseudo-discrete} if the topology on $\mathcal{F}(U)$ is
discrete for every quasi-compact open $U \subset X$. Then
the construction given above is an adjoint to the forgetful functor
and induces an equivalence between the category of sheaves
of sets and the category of pseudo-discrete sheaves of topological space
(similarly for groups, rings, modules).

\medskip\noindent
Grothendieck and Dieudonn\'e first define formal affine schemes.
These correspond to admissible topological rings $A$, see
More on Algebra, Definition \ref{more-algebra-definition-topological-ring}.
Namely, given $A$ one considers a fundamental system $I_\lambda$ of ideals
of definition for the ring $A$. (In any admissible topological ring
the family of all ideals of definition forms a fundamental system.)
For each $\lambda$ we can consider the
scheme $\Spec(A/I_\lambda)$. For $I_\lambda \subset I_\mu$ the induced
morphism
$$
\Spec(A/I_\mu) \to \Spec(A/I_\lambda)
$$
is a thickening because $I_\mu^n \subset I_\lambda$ for some $n$.
Another way to see this, is to notice that the image of each of the
maps
$$
\Spec(A/I_\lambda) \to \Spec(A)
$$
is a homeomorphism onto the set of open prime ideals of $A$.
This motivates the definition
$$
\text{Spf}(A) = \{\text{open prime ideals }\mathfrak p \subset A\}
$$
endowed with the topology coming from $\Spec(A)$. For each $\lambda$
we can consider the structure sheaf $\mathcal{O}_{\Spec(A/I_\lambda}$
as a sheaf on $\text{Spf}(A)$. Let $\mathcal{O}_\lambda$ be the corresponding
pseudo-discrete sheaf of topological rings, see above.
Then we set
$$
\mathcal{O}_{\text{Spf}(A)} = \lim \mathcal{O}_\lambda
$$
where the limit is taken in the category of sheaves of topological rings.
The pair $(\text{Spf}(A), \mathcal{O}_{\text{Spf}(A)})$ is called the
{\it formal spectrum} of $A$.

\medskip\noindent
At this point one should check several things. The first is that
the stalks $\mathcal{O}_{\text{Spf}(A), x}$ are local rings
(forgetting about the topology). The second is that given
$f \in A$, for the corresponding open $D(f) \cap \text{Spf}(A)$
we have
$$
\Gamma(D(f) \cap \text{Spf}(A), \mathcal{O}_{\text{Spf}(A)})
= A_{\{f\}} = \lim (A/I_\lambda)_f
$$
as topological rings where $I_\lambda$ is a fundamental system of ideals
of definition as above. Moreover, the ring $A_{\{f\}}$ is admissible too and
$(\text{Spf}(A_f), \mathcal{O}_{\text{Spf}(A_{\{f\}})})$
is isomorphic to
$(D(f) \cap \text{Spf}(A),
\mathcal{O}_{\text{Spf}(A)}|_{D(f) \cap \text{Spf}(A)})$.
Finally, given a pair of admissible topological rings $A, B$
we have
\begin{equation}
\label{equation-morphisms-affine-formal-schemes}
\Mor_{LTRS}((\text{Spf}(B), \mathcal{O}_{\text{Spf}(B)}),
(\text{Spf}(A), \mathcal{O}_{\text{Spf}(A)}))
= \Hom_{cont}(A, B)
\end{equation}
where $LTRS$ is the category of ``locally topologically ringed spaces''
as defined above.

\medskip\noindent
Having said this, in \cite{EGA} a {\it formal scheme} is defined as a pair
$(\mathfrak X, \mathcal{O}_\mathfrak X)$ where $\mathfrak X$
is a topological space and $\mathcal{O}_\mathfrak X$ is a sheaf
of topological rings such that every point has an open neighbourhood
isomorphic (in $LTRS$) to an affine formal scheme.
A {\it morphism of formal schemes}
$f : (\mathfrak X, \mathcal{O}_\mathfrak X) \to
(\mathfrak Y, \mathcal{O}_\mathfrak Y)$
is a morphism in the category $LTRS$.

\medskip\noindent
Let $A$ be a ring endowed with the discrete topology. Then $A$ is
admissible and the formal scheme $\text{Spf}(A)$ is equal to
$\Spec(A)$. The structure sheaf $\mathcal{O}_{\text{Spf}(A)}$
is the pseudo-discrete sheaf of topological rings associated
to $\mathcal{O}_{\Spec(A)}$, in other words, its underlying
sheaf of rings is equal to $\mathcal{O}_{\Spec(A)}$ and the
ring $\mathcal{O}_{\text{Spf}(A)}(U) = \mathcal{O}_{\Spec(A)}(U)$
over a quasi-compact open $U$ has the discrete topology,
but not in general. Thus we can associate to every affine scheme
a formal affine scheme. In exactly the same manner we can start
with a general scheme $(X, \mathcal{O}_X)$ and associate to
it $(X, \mathcal{O}'_X)$ where $\mathcal{O}'_X$ is the
pseudo-discrete sheaf of topological rings whose underlying
sheaf of rings is $\mathcal{O}_X$. This construction is
compatible with morphisms and defines a functor
\begin{equation}
\label{equation-compare-schemes-formal-schemes}
\textit{Schemes} \longrightarrow \textit{Formal Schemes}
\end{equation}
It follows in a straightforward manner from
(\ref{equation-morphisms-affine-formal-schemes})
that this functor is fully faithful.

\medskip\noindent
Let $\mathfrak X$ be a formal scheme. Let us define the {\it size}
of the formal scheme by the formula
$\text{size}(\mathfrak X) = \max(\aleph_0, \kappa_1, \kappa_2)$
where $\kappa_1$ is the cardinality of the formal affine opens of
$\mathfrak X$ and $\kappa_2$ is the supremum of the cardinalities
of $\mathcal{O}_\mathfrak X(\mathfrak U)$ where
$\mathfrak U \subset \mathfrak X$ is such a formal affine open.

\begin{lemma}
\label{lemma-fully-faithful}
Choose a category of schemes $\Sch_\alpha$
as in Sets, Lemma \ref{sets-lemma-construct-category}.
Given a formal scheme $\mathfrak X$ let
$$
h_\mathfrak X : (\Sch_\alpha)^{opp} \longrightarrow \textit{Sets},\quad
h_\mathfrak X(S) = \Mor_{\textit{Formal Schemes}}(S, \mathfrak X)
$$
be its functor of points. Then we have
$$
\Mor_{\textit{Formal Schemes}}(\mathfrak X, \mathfrak Y) =
\Mor_{\textit{PSh}(\Sch_\alpha)}(h_\mathfrak X, h_\mathfrak Y)
$$
provided the size of $\mathfrak X$ is not too large.
\end{lemma}

\begin{proof}
First we observe that $h_\mathfrak X$ satisfies the sheaf property for
the Zariski topology for any formal scheme $\mathfrak X$ (see
Schemes, Definition \ref{schemes-definition-representable-by-open-immersions}).
This follows from the local nature of morphisms in the category
of formal schemes. Also, for an open immersion
$\mathfrak V \to \mathfrak W$ of formal schemes,
the corresponding tranformation of functors $h_\mathfrak V \to h_\mathfrak W$
is injective and representable by open immersions (see
Schemes, Definition \ref{schemes-definition-representable-by-open-immersions}).
Choose an open covering $\mathfrak X = \bigcup \mathfrak U_i$
of a formal scheme by affine formal schemes $\mathfrak U_i$.
Then the collection of functors
$h_{\mathfrak U_i}$ covers $h_\mathfrak X$ (see
Schemes, Definition \ref{schemes-definition-representable-by-open-immersions}).
Finally, note that
$$
h_{\mathfrak U_i} \times_{h_\mathfrak X} h_{\mathfrak U_j} =
h_{\mathfrak U_i \cap \mathfrak U_j}
$$
Hence in order to give a map $h_\mathfrak X \to h_\mathfrak Y$
is equivalent to giving a family of maps
$h_{\mathfrak U_i} \to h_\mathfrak Y$ which agree on overlaps.
Thus we can reduce the bijectivity (resp.\ injectivity) of the map
of the lemma to bijectivity (resp.\ injectivity) for the pairs
$(\mathfrak U_i, \mathfrak Y)$
and injectivity (resp.\ nothing)
for $(\mathfrak U_i \cap \mathfrak U_j, \mathfrak Y)$.
In this way we reduce to the case where $\mathfrak X$ is an
affine formal scheme. Say $\mathfrak X = \text{Spf}(A)$
for some admissible topological ring $A$. Also, choose a
fundamental system of ideals of definition $I_\lambda \subset A$.

\medskip\noindent
We can also localize on $\mathfrak Y$.
Namely, suppose that $\mathfrak V \subset \mathfrak Y$ is an
open formal subscheme and $\varphi : h_\mathfrak X \to h_\mathfrak Y$.
Then
$$
h_\mathfrak V \times_{h_\mathfrak Y, \varphi} h_\mathfrak X \to h_\mathfrak X
$$
is representable by open immersions. Pulling back to
$\Spec(A/I_\lambda)$ for all $\lambda$ we find an open subscheme
$U_\lambda \subset \Spec(A/I_\lambda)$. However, for
$I_\lambda \subset I_\mu$ the morphism $\Spec(A/I_\lambda) \to \Spec(A/I_\mu)$
pulls back $U_\mu$ to $U_\lambda$. Thus these glue to give
an open formal subscheme $\mathfrak U \subset \mathfrak X$.
A straightforward argument (omitted) shows that
$$
h_\mathfrak U = h_\mathfrak V \times_{h_\mathfrak Y} h_\mathfrak X
$$
In this way we see that given an open covering
$\mathfrak Y = \bigcup \mathfrak V_j$ and a transformation
of functors $\varphi :  h_\mathfrak X \to h_\mathfrak Y$
we obtain a corresponding open covering of $\mathfrak X$.
Since $\mathfrak X$ is affine, we can refine this covering by
a finite open covering
$\mathfrak X = \mathfrak U_1 \cup \ldots \cup \mathfrak U_n$
by affine formal subschemes. In other words, for each $i$ there
is a $j$ and a map $\varphi_i : h_{\mathfrak U_i} \to h_{\mathfrak V_j}$
such that
$$
\xymatrix{
h_{\mathfrak U_i} \ar[r]_{\varphi_i} \ar[d] & h_{\mathfrak V_j} \ar[d] \\
h_{\mathfrak X} \ar[r]^\varphi & h_\mathfrak Y
}
$$
commutes. With a few additional arguments (which we omit) this implies
that it suffices to prove the bijectivity of the lemma in case
both $\mathfrak X$ and $\mathfrak Y$ are affine formal schemes.

\medskip\noindent
Assume $\mathfrak X$ and $\mathfrak Y$ are affine formal schemes.
Say $\mathfrak X = \text{Spf}(A)$ and $\mathfrak Y = \text{Spf}(B)$.
Let $\varphi : h_\mathfrak X \to h_\mathfrak Y$ be a transformation
of functors. Let $I_\lambda \subset A$ be a fundamental system of
ideals of definition. The canonical inclusion morphsm
$i_\lambda : \Spec(A/I_\lambda) \to \mathfrak X$ maps to a morphism
$\varphi(i_\lambda) : \Spec(A/I_\lambda) \to \mathfrak Y$.
By (\ref{equation-morphisms-affine-formal-schemes}) this corresponds
to a continuous map $\chi_\lambda : B \to A/I_\lambda$.
Since $\varphi$ is a transformation of functors it follows
that for $I_\lambda \subset I_\mu$ the composition
$B \to A/I_\lambda \to A/I_\mu$ is equal to $\chi_\mu$.
In other words we obtain a ring map
$$
\chi = \lim \chi_\lambda : B \longrightarrow \lim A/I_\lambda = A
$$
This is a continuous homomorphism because the inverse image
of $I_\lambda$ is open for all $\lambda$ (as $A/I_\lambda$ has the discrete
topology and $\chi_\lambda$ is continuous). Thus we obtain
a morphism $\text{Spf}(\chi) : \mathfrak X \to \mathfrak Y$ by
(\ref{equation-morphisms-affine-formal-schemes}).
We omit the verification that this construction is the inverse
to the map of the lemma in this case.

\medskip\noindent
Set theoretic remarks. To make this work on the given category
of schemes $\Sch_\alpha$ we just have to make sure all the
schemes used in the proof above are isomorphic to objects of $\Sch_\alpha$.
In fact, a careful analysis shows that it suffices if the
schemes $\Spec(A/I_\lambda)$ occuring above are isomorphic to
objects of $\Sch_\alpha$. For this it certainly suffices to assume
the size of $\mathfrak X$ is at most the size of
a scheme contained in $\Sch_\alpha$.
\end{proof}

\begin{lemma}
\label{lemma-formal-scheme-sheaf-fppf}
Let $\mathfrak X$ be a formal scheme. The functor of points
$h_\mathfrak X$ (see Lemma \ref{lemma-fully-faithful})
satisfies the sheaf condition for fpqc coverings.
\end{lemma}

\begin{proof}
Topologies, Lemma \ref{topologies-lemma-sheaf-property-fpqc}
reduces us to the case of a Zariski covering and a covering
$\{\Spec(S) \to \Spec(R)\}$ with $R \to S$ faithfully flat.
We observed in the proof of Lemma \ref{lemma-fully-faithful} 
that $h_\mathfrak X$ satisfies the sheaf condition for Zariski coverings.

\medskip\noindent
Suppose that $R \to S$ is a faithfully flat ring map.
Denote $\pi : \Spec(S) \to \Spec(R)$ the
corresponding morphism of schemes. It is surjective and flat.
Let $f : \Spec(S) \to \mathfrak X$ be a morphism
such that $f \circ \text{pr}_1 = f \circ \text{pr}_2$
as maps $\Spec(S \otimes_R S) \to \mathfrak X$.
By Descent, Lemma \ref{descent-lemma-equiv-fibre-product}
we see that as a map on the underlying
sets $f$ is of the form $f = g \circ \pi$ for some
(set theoretic) map $g : \Spec(R) \to \mathfrak X$.
By Morphisms, Lemma \ref{morphisms-lemma-fpqc-quotient-topology}
and the fact that $f$ is continuous we see that $g$
is continuous.

\medskip\noindent
Pick $y \in \Spec(R)$. Choose $\mathfrak U \subset \mathfrak X$
an affine formal open subscheme containing $g(y)$.
Say $\mathfrak U = \text{Spf}(A)$ for some admissible topological
ring $A$. By the above we may choose an $r \in R$ such that
$x \in D(r) \subset g^{-1}(\mathfrak U)$.
The restriction of $f$ to $\pi^{-1}(D(r))$ into $\mathfrak U$
corresponds to a continuous ring map $A \to S_r$ by
(\ref{equation-morphisms-affine-formal-schemes}). The two induced ring maps
$A \to S_r \otimes_{R_r} S_r = (S \otimes_R S)_r$ are equal
by assumption on $f$.
Note that $R_r \to S_r$ is faithfully flat.
By Descent, Lemma \ref{descent-lemma-ff-exact} the equalizer of
the two arrows $S_r \to S_r \otimes_{R_r} S_r$ is $R_r$.
We conclude that $A \to S_r$ factors uniquely through a map $A \to S_r$
which is also continuous as it has the same (open) kernel as the
map $A \to S_r$. This map in turn gives a morphism $D(r) \to \mathfrak U$ by
(\ref{equation-morphisms-affine-formal-schemes}).

\medskip\noindent
What have we proved so far? We have shown that for any $y \in \Spec(R)$
there exists a standard affine open
$y \in D(r) \subset \Spec(R)$ such that the morphism
$f|_{\pi^{-1}(D(r))} : \pi^{-1}(D(r)) \to \mathfrak X$ factors uniquely
though some morphism $D(r) \to \mathfrak X$. We omit the
verification that these morphisms glue to the desired
morphism $\Spec(R) \to \mathfrak X$.
\end{proof}

\begin{remark}[McQuillan's variant]
\label{remark-mcquillan}
There is a variant of the construction of formal schemes due to
McQuillan, see \cite{McQuillan}.
He suggests a slight weakening of the condition of admissibility.
Namely, recall that an admissible topological ring is a complete
(and separated by our conventions) topological ring $A$
which is linearly topologized such that there exists an
ideal of defintion: an
open ideal $I$ such that any neighbourhood of $0$ contains $I^n$.
McQuillan works with what might be called {\it weakly admissible}
topological rings. A weakly admissible topological ring $A$ is a
complete (and separated by our conventions) topological ring
which is linearly topologized such that there exists an
{\it weak ideal of definition}: an open ideal $I$ such that
for all $f \in I$ we have
$f^n \to 0$ for $n \to \infty$. Similarly to the admissible case,
if $I$ is a weak ideal of definition and $J \subset A$ is an
open ideal, then $I \cap J$ is a weak ideal of definition.
Thus the weak ideals of definition form a fundamental system and
one can proceed along much the same route as above
to define a larger category of formal schemes based
on this notion. The analogues of Lemmas \ref{lemma-fully-faithful} and
\ref{lemma-formal-scheme-sheaf-fppf}
still hold in this setting (with the same proof).
\end{remark}

\begin{remark}[Sheafification of presheaves of topological spaces]
\label{remark-sheafification-of-presheaves-in-top}
\begin{reference}
\cite{Gray}
\end{reference}
In this remark we briefly discuss sheafification of presheaves
of topological spaces. The exact same arguments work for
presheaves of topological abelian groups, topological rings, and
topological modules (over a given topological ring). In order to
do this in the correct generality let us work over a site
$\mathcal{C}$. The reader who is interested in the case of (pre)sheaves
over a topological space $X$ should think of objects of $\mathcal{C}$
as the opens of $X$, of morphisms of $\mathcal{C}$ as inclusions of
opens, and of coverings in $\mathcal{C}$ as coverings in $X$, see
Sites, Example \ref{sites-example-site-topological}.
Denote $\Sh(\mathcal{C}, \textit{Top})$ the category of sheaves
of topological spaces on $\mathcal{C}$ and denote
$\textit{PSh}(\mathcal{C}, \textit{Top})$ the category of presheaves
of topological spaces on $\mathcal{C}$.
Let $\mathcal{F}$ be a presheaf of topological spaces on $\mathcal{C}$.
The sheafification $\mathcal{F}^\#$ should satisfy the formula
$$
\Mor_{\textit{PSh}(\mathcal{C}, \textit{Top})}(\mathcal{F}, \mathcal{G})
=
\Mor_{\Sh(\mathcal{C}, \textit{Top})}(\mathcal{F}^\#, \mathcal{G})
$$
functorially in $\mathcal{G}$ from $\Sh(\mathcal{C}, \textit{Top})$.
In other words, we are trying to construct the left adjoint
to the inclusion fuctor
$\Sh(\mathcal{C}, \textit{Top}) \to \Sh(\mathcal{C}, \textit{Top})$.
We first claim that $\Sh(\mathcal{C}, \textit{Top})$ has limits
and that the inclusion functor commutes with them.
Namely, given a category $\mathcal{I}$ and a functor
$i \mapsto \mathcal{G}_i$ into $\Sh(\mathcal{C}, \textit{Top})$
we simply define
$$
(\lim \mathcal{G}_i)(U) = \lim \mathcal{G}_i(U)
$$
where we take the limit in the category of toplogical spaces
(Topology, Lemma \ref{topology-lemma-limits}). This defines a sheaf
because limits commute with limits
(Categories, Lemma \ref{categories-lemma-colimits-commute})
and in particular products and equalizers (which are the
operations used in the sheaf axiom). Finally, a morphism
of presheaves from $\mathcal{F} \to \lim \mathcal{G}_i$ is
clearly the same thing as a compatible system of morphisms
$\mathcal{F} \to \mathcal{G}_i$. In other words, the object
$\lim \mathcal{G}_i$ is the limit in the category
of presheaves of topological spaces and a fortiori in the
category of sheaves of topological spaces.
Our second claim is that any morphism of presheaves
$\mathcal{F} \to \mathcal{G}$ with $\mathcal{G}$ an object of
$\Sh(\mathcal{C}, \textit{Top})$ factors through a subsheaf
$\mathcal{G}' \subset \mathcal{G}$ whose size is bounded.
Here we define the {\it size} $|\mathcal{H}|$
of a sheaf of topological spaces $\mathcal{H}$ to be the cardinal
$\sup_{U \in \Ob(\mathcal{C})} |\mathcal{H}(U)|$.
To prove our claim we let
$$
\mathcal{G}'(U) =
\left\{
\quad
s \in \mathcal{G}(U)
\quad \Big | \quad
\begin{matrix}
\text{there exists a covering }\{U_i \to U\}_{i \in I} \\
\text{such that }
s|_{U_i} \in \Im(\mathcal{F}(U_i) \to \mathcal{G}(U_i))
\end{matrix}
\quad
\right\}
$$
We endow $\mathcal{G}'(U)$ with the induced topology.
Then $\mathcal{G}'$ is a sheaf of topological spaces (details omitted)
and $\mathcal{G}' \to \mathcal{G}$ is a morphism through which
the given map $\mathcal{F} \to \mathcal{G}$ factors. Moreover,
the size of $\mathcal{G}'$ is bounded by some cardinal
$\kappa$ depending only on $\mathcal{C}$ and the presheaf $\mathcal{F}$
(hint: use that coverings in $\mathcal{C}$
form a set by our conventions). Putting everything together we see
that the assumptions of Categories, Theorem
\ref{categories-theorem-adjoint-functor}
are satisfied and we obtain sheafification as the left
adjoint of the inclusion functor from sheaves to presheaves.
Finally, let $p$ be a point of the
site $\mathcal{C}$ given by a functor $u : \mathcal{C} \to \textit{Sets}$,
see Sites, Definition \ref{sites-definition-point}.
For a topological space $M$ the presheaf defined by the rule
$$
U \mapsto \text{Map}(u(U), M) = \prod\nolimits_{x \in u(U)} M
$$
endowed with the product topology is a sheaf of topological spaces.
Hence the exact same argument as given in the proof of
Sites, Lemma \ref{sites-lemma-point-pushforward-sheaf} shows that
$\mathcal{F}_p = \mathcal{F}^\#_p$, in other words, sheafification
commutes with taking stalks at a point.
\end{remark}




\section{Conventions and notation}
\label{section-conventions}

\noindent
The conventions from now on will be similar to the conventions in
Properties of Spaces, Section \ref{spaces-properties-section-conventions}.
Thus from now on the standing assumption is that all schemes are contained
in a big fppf site $\Sch_{fppf}$. And all rings $A$ considered have the
property that $\Spec(A)$ is (isomorphic) to an object of this big site.
For topological rings $A$ we assume only that all discrete quotients have
this property (but usually we assume more, compare with
Remark \ref{remark-set-theoretic}).

\medskip\noindent
Let $S$ be a scheme and let $X$ be a ``space'' over $S$, i.e., a sheaf on
$(\Sch/S)_{fppf}$. In this chapter we will write $X \times_S X$ for the
product of $X$ with itself in the category of sheaves on $(\Sch/S)_{fppf}$
instead of $X \times X$. Moreover, if $X$ and $Y$ are ``spaces'' then
we say "let $f : X \to Y$ be a morphism" to indicate that $f$ is a
natural transformation of functors, i.e., a map of sheaves on
$(\Sch/S)_{fppf}$. Similarly, if $U$ is a scheme over $S$ and
$X$ is a ``space'' over $S$, then we say
"let $f : U \to X$ be a morphism" or
"let $g : X \to U$ be a morphism" to indicate that $f$ or $g$
is a map of sheaves $h_U \to X$ or $X \to h_U$ where $h_U$ is as in
Categories, Example \ref{categories-example-hom-functor}.





\section{Affine formal algebraic spaces}
\label{section-affine-formal-algebraic-spaces}

\noindent
In this section we introduce affine formal algebraic spaces.
These will in fact be the same as what are called affine
formal schemes in \cite{BVGD}. However, we will call
them affine formal algebraic spaces, in order to prevent confusion with
the notion of an affine formal scheme as defined in \cite{EGA}.

\medskip\noindent
Recall that a thickening of schemes is a closed
immersion which induces a bijection on underlying topological
spaces, see More on Morphisms, Definition
\ref{more-morphisms-definition-thickening}.

\begin{definition}
\label{definition-affine-formal-algebraic-space}
Let $S$ be a scheme. We say a sheaf $X$ on $(\Sch/S)_{fppf}$ is an
{\it affine formal algebraic space} if there exist
\begin{enumerate}
\item a directed partially ordered set $\Lambda$,
\item a system $(X_\lambda, f_{\lambda \mu})$ over $\Lambda$
in $(\Sch/S)_{fppf}$ where
\begin{enumerate}
\item each $X_\lambda$ is affine,
\item each $f_{\lambda \mu} : X_\lambda \to X_\mu$ is a thickening,
\end{enumerate}
\end{enumerate}
such that
$$
X \cong \colim_{\lambda \in \Lambda} X_\lambda
$$
as fppf sheaves and $X$ satisfies a set theoretic condition
(see Remark \ref{remark-set-theoretic}). A
{\it morphism of affine formal algebraic spaces}
over $S$ is a map of sheaves.
\end{definition}

\noindent
Observe that the system $(X_\lambda, f_{\lambda \mu})$ is not
part of the data. Suppose that $U$ is a quasi-compact scheme over $S$.
Since the transition maps are monomorphisms, we see that
$$
X(U) = \colim X_\lambda(U)
$$
by Sites, Lemma \ref{sites-lemma-directed-colimits-sections}.
Thus the fppf sheafification inherent in the colimit of the
definition is a Zariski sheafification which does not do
anything for quasi-compact schemes.

\begin{lemma}
\label{lemma-diagonal-affine-formal-algebraic-space}
Let $S$ be a scheme. If $X$ is an affine formal algebraic space over
$S$, then the diagonal morphism $\Delta : X \to X \times_S X$
is representable and a closed immersion.
\end{lemma}

\begin{proof}
Suppose given $U \to X$ and $V \to X$ where $U, V$ are schemes over $S$.
Let us show that $U \times_X V$ is representable. Write $X = \colim X_\lambda$
as in Definition \ref{definition-affine-formal-algebraic-space}.
The discussion above shows that Zariski locally on $U$ and $V$ the morphisms
factors through some $X_\lambda$. In this case
$U \times_X V = U \times_{X_\lambda} V$ which is a scheme.
Thus the diagonal is representable, see
Spaces, Lemma \ref{spaces-lemma-representable-diagonal}.
Given $(a, b) : W \to X \times_S X$ where $W$ is a scheme over $S$
consider the map $X \times_{\Delta, X \times_S X, (a, b)} W \to W$.
As before locally on $W$ the morphisms $a$ and $b$ map into
the affine scheme $X_\lambda$ for some $\lambda$ and then
we get the morphism
$X_\lambda
\times_{\Delta_\lambda, X_\lambda \times_S X_\lambda, (a, b)} W \to W$.
This is the base change of
$\Delta_\lambda : X_\lambda \to X_\lambda \times_S X_\lambda$
which is a closed immersion as $X_\lambda \to S$ is separated
(because $X_\lambda$ is affine).
Thus $X \to X \times_S X$ is a closed immersion.
\end{proof}

\noindent
A morphism of schemes $X \to X'$ is a thickening if it is
a closed immersion and induces a bijection on underlying sets of points, see
(More on Morphisms, Definition
\ref{more-morphisms-definition-thickening}).
Observe that $X \to X'$ is a thickening if and only if
it is a closed immersion, surjective, and universally injective.
In this way we see that the property of being a thickening
is preserved under arbitrary base change and fpqc local on the target, see
Spaces, Section \ref{spaces-section-lists}.
Thus Spaces, Definition \ref{spaces-definition-relative-representable-property}
applies to ``thickening'' and we know what it means for a
representable transformation $F \to G$ of
presheaves on $(\Sch/S)_{fppf}$ to be a thickening.
We observe that this does not clash with our definition
(More on Morphisms of Spaces, Definition
\ref{spaces-more-morphisms-definition-thickening})
of thickenings in case $F$ and $G$ are algebraic spaces.

\begin{lemma}
\label{lemma-covering-by-thickenings}
Let $X_\lambda, \lambda \in \Lambda$ and $X = \colim X_\lambda$
be as in Definition \ref{definition-affine-formal-algebraic-space}.
Then $X_\lambda \to X$ is representable and a thickening.
\end{lemma}

\begin{proof}
The statement makes sense by the discussion in
Spaces, Section \ref{spaces-section-representable} and
\ref{spaces-section-representable-properties}.
By Lemma \ref{lemma-diagonal-affine-formal-algebraic-space}
the morphisms $X_\lambda \to X$ are representable.
Given $U \to X$ where $U$ is a scheme,
then the discussion following
Definition \ref{definition-affine-formal-algebraic-space}
shows that Zariski locally on $U$ the
morphism factors through some $X_\mu$ with $\lambda \leq \mu$.
In this case $U \times_X X_\lambda = U \times_{X_\mu} X_\lambda$
so that $U \times_X X_\lambda \to U$ is a base change of
the thickening $X_\lambda \to X_\mu$.
\end{proof}

\begin{lemma}
\label{lemma-factor-through-thickening}
Let $X_\lambda, \lambda \in \Lambda$ and $X = \colim X_\lambda$
be as in Definition \ref{definition-affine-formal-algebraic-space}.
If $Y$ is a quasi-compact algebraic space over $S$, then any
morphism $Y \to X$ factors through an $X_\lambda$.
\end{lemma}

\begin{proof}
Choose an affine scheme $V$ and a surjective \'etale morphism
$V \to Y$. The composition $V \to Y \to X$ factors through
$X_\lambda$ for some $\lambda$ by the discussion following
Definition \ref{definition-affine-formal-algebraic-space}.
Since $V \to Y$ is a surjection of sheaves, we conclude.
\end{proof}

\begin{lemma}
\label{lemma-characterize-affine-formal-algebraic-space}
Let $S$ be a scheme. Let $X$ be a sheaf on $(\Sch/S)_{fppf}$.
Then $X$ is an affine formal algebraic space if and only if
the following hold
\begin{enumerate}
\item any morphism $U \to X$ where $U$ is an affine scheme over $S$
factors through a morphism $T \to X$ which is representable and a
thickening with $T$ an affine scheme over $S$, and
\item a set theoretic condition as in Remark \ref{remark-set-theoretic}.
\end{enumerate}
\end{lemma}

\begin{proof}
It follows from Lemmas \ref{lemma-covering-by-thickenings} and
\ref{lemma-factor-through-thickening} that an affine formal algebraic space
satisfies (1) and (2). In order to prove the converse we may
assume $X$ is not empty.
Let $\Lambda$ be the category of representable morphisms $T \to X$ which are
thickenings where $T$ is an affine scheme over $S$. This category
is directed. Since $X$ is not empty, $\Lambda$ contains at least one
object. If $T \to X$ and $T' \to X$ are in $\Lambda$, then we can
factor $T \amalg T' \to X$ through $T'' \to X$ in $\Lambda$. Between
any two objects of $\Lambda$ there is a unique arrow or none. Thus
$\Lambda$ is a directed partially ordered set and by assumption
$X = \colim_{T \to X\text{ in }\Lambda} T$.
\end{proof}

\noindent
For a general affine formal algebraic space $X$ there is no guarantee
that $X$ has enough functions to separate points (for example).
To characterize those that do we offer the following lemma.

\begin{lemma}
\label{lemma-mcquillan-affine-formal-algebraic-space}
Let $S$ be a scheme. Let $X$ be an fppf sheaf on $(\Sch/S)_{fppf}$
which satisfies the set theoretic condition of
Remark \ref{remark-set-theoretic}.
The following are equivalent:
\begin{enumerate}
\item there exists a weakly admissible topological ring $A$ over $S$
(see Remark \ref{remark-mcquillan}) such that
$X = \colim_{I \subset A\text{ weak ideal of definition}} \Spec(A/I)$,
\item $X$ is an affine formal algebraic space and
there exists an $S$-algebra $A$ and a map $X \to \Spec(A)$
such that for a closed immersion $T \to X$ with $T$ an affine scheme
the composition $T \to \Spec(A)$ is a closed immersion,
\item $X$ is an affine formal algebraic space and
there exists an $S$-algebra $A$ and a map $X \to \Spec(A)$
such that for a closed immersion $T \to X$ with $T$ a scheme
the composition $T \to \Spec(A)$ is a closed immersion,
\item $X$ is an affine formal algebraic space and
for some choice of $X = \colim X_\lambda$ as in
Definition \ref{definition-affine-formal-algebraic-space}
the projections $\lim \Gamma(X_\lambda, \mathcal{O}_{X_\lambda})
\to \Gamma(X_\lambda, \mathcal{O}_{X_\lambda})$ are surjective,
\item $X$ is an affine formal algebraic space and for any choice
of $X = \colim X_\lambda$ as in
Definition \ref{definition-affine-formal-algebraic-space}
the projections $\lim \Gamma(X_\lambda, \mathcal{O}_{X_\lambda})
\to \Gamma(X_\lambda, \mathcal{O}_{X_\lambda})$ are surjective.
\end{enumerate}
Moreover, the weakly admissible topological ring is
$A = \lim \Gamma(X_\lambda, \mathcal{O}_{X_\lambda})$
endowed with its limit topology and the weak ideals of definition
classify exactly the morphisms $T \to X$ which are representable
and thickenings.
\end{lemma}

\begin{proof}
It is clear that (5) implies (4).

\medskip\noindent
Assume (4) for $X = \colim X_\lambda$ as in
Definition \ref{definition-affine-formal-algebraic-space}.
Set $A = \lim \Gamma(X_\lambda, \mathcal{O}_{X_\lambda})$.
Let $T \to X$ be a closed immersion with $T$ a scheme
(note that $T \to X$ is representable by
Lemma \ref{lemma-diagonal-affine-formal-algebraic-space}).
Since $X_\lambda \to X$ is a thickening, so is
$X_\lambda \times_X T \to T$. On the other hand,
$X_\lambda \times_X T \to X_\lambda$ is a closed immersion,
hence $X_\lambda \times_X T$ is affine. Hence $T$ is affine
by Limits, Proposition \ref{limits-proposition-affine}.
Then $T \to X$ factors through $X_\lambda$ for some $\lambda$
by Lemma \ref{lemma-factor-through-thickening}.
Thus $A \to \Gamma(X_\lambda, \mathcal{O}) \to \Gamma(T, \mathcal{O})$
is surjective. In this way we see that (3) holds.

\medskip\noindent
It is clear that (3) implies (2).

\medskip\noindent
Assume (2) for $A$ and $X \to \Spec(A)$. Write $X = \colim X_\lambda$
as in Definition \ref{definition-affine-formal-algebraic-space}.
Then $A_\lambda = \Gamma(X_\lambda, \mathcal{O})$ is a quotient
of $A$ by assumption (2). Hence $A^\wedge = \lim A_\lambda$
is a complete topological ring, see discussion in
More on Algebra, Section \ref{more-algebra-section-topological-ring}.
The maps $A^\wedge \to A_\lambda$ are surjective as $A \to A_\lambda$ is.
We claim that for any $\lambda$ the kernel $I_\lambda \subset A^\wedge$ of
$A^\wedge \to A_\lambda$ is a weak ideal of definition.
Namely, it is open by definition of the limit topology.
If $f \in I_\lambda$, then for any $\mu \in \Lambda$
the image of $f$ in $A_\mu$ is zero in all the residue fields
of the points of $X_\mu$. Hence it is a nilpotent element
of $A_\mu$. Hence some power $f^n \in I_\mu$. Thus $f^n \to 0$
as $n \to 0$. Thus $A^\wedge$ is weakly admissible.
Finally, suppose that $I \subset A^\wedge$ is a weak ideal
of definition. Then $I \subset A^\wedge$ is open and hence there exists
some $\lambda$ such that $I \supset I_\lambda$. Thus we obtain a morphism
$\Spec(A^\wedge/I) \to \Spec(A_\lambda) \to X$.
Then it follows that $X = \colim \Spec(A^\wedge/I)$ where now
the colimit is over all weak ideals of definition.
Thus (1) holds.

\medskip\noindent
Assume (1). In this case it is clear that $X$ is an affine formal
algebraic space. Let $X = \colim X_\lambda$ be any presentation as in
Definition \ref{definition-affine-formal-algebraic-space}.
For each $\lambda$ we can find a weak ideal of definition
$I \subset A$ such that $X_\lambda \to X$ factors through
$\Spec(A/I) \to X$, see Lemma \ref{lemma-factor-through-thickening}.
Then $X_\lambda = \Spec(A/I_\lambda)$ with $I \subset I_\lambda$.
Conversely, for any weak ideal of definition $I \subset A$
the morphism $\Spec(A/I) \to X$ factors through $X_\lambda$
for some $\lambda$, i.e., $I_\lambda \subset I$.
It follows that each $I_\lambda$ is a weak ideal of definition
and that they form a cofinal subset of the set of weak ideals
of definition. Hence $A = \lim A/I = \lim A/I_\lambda$
and we see that (5) is true and moreover that
$A = \lim \Gamma(X_\lambda, \mathcal{O}_{X_\lambda})$.
\end{proof}

\noindent
With this lemma in hand we can make the following definition.

\begin{definition}
\label{definition-types-affine-formal-algebraic-space}
Let $S$ be a scheme. Let $X$ be an affine formal algebraic space over $S$.
We say
\begin{enumerate}
\item $X$ is {\it McQuillan} if $X$ satisfies the equivalent conditions
of Lemma \ref{lemma-mcquillan-affine-formal-algebraic-space}. Let $A$
be the weakly admissible topological ring associated to $X$.
\item $X$ is {\it classical} if $X$ is McQuillan and $A$ is admissible.
\item $X$ is {\it adic} if $X$ is McQuillan and $A$ is adic.
\item $X$ is {\it Noetherian} if $X$ is McQuillan and $A$ is
both Noetherian and adic.
\end{enumerate}
\end{definition}

\begin{remark}
\label{remark-compare-with-affine-formal-schemes}
The classical affine formal algebraic spaces correspond to the
affine formal schemes considered in EGA (\cite{EGA}). To explain this
we assume our base scheme is $\Spec(\mathbf{Z})$. Let
$\mathfrak X = \text{Spf}(A)$ be an affine formal scheme.
Let $h_\mathfrak X$ be its functor of points as in
Lemma \ref{lemma-fully-faithful}.
Then $h_\mathfrak X = \colim h_{\Spec(A/I)}$ where the colimit
is over the collection of ideals of definition of the admissible
topological ring $A$. This follows from
(\ref{equation-morphisms-affine-formal-schemes})
when evaluating on affine schemes and it suffices to check
on affine schemes as both sides are fppf sheaves, see
Lemma \ref{lemma-formal-scheme-sheaf-fppf}.
Thus $h_\mathfrak X$ is an affine formal algebraic space.
In fact, it is a classical affine formal algebraic space
by Definition \ref{definition-types-affine-formal-algebraic-space}.
Thus Lemma \ref{lemma-fully-faithful} tells us
the category of affine formal schemes is equivalent to the category
of classical affine formal algebraic spaces.
\end{remark}

\begin{lemma}
\label{lemma-presentation-representable}
Let $S$ be a scheme. Let $f : X \to Y$ be a map
of presheaves on $(\Sch/S)_{fppf}$. If $X$ is an affine formal algebraic
space and $f$ is representable by algebraic spaces and locally quasi-finite,
then $f$ is representable (by schemes).
\end{lemma}

\begin{proof}
Let $T$ be a scheme over $S$ and $T \to Y$ a map. We have to show that
the algebraic space $X \times_Y T$ is a scheme. Write $X = \colim X_\lambda$
as in Definition
\ref{definition-affine-formal-algebraic-space}.
Let $W \subset X \times_Y T$
be a quasi-compact open subspace. The restriction of the projection
$X \times_Y T \to X$ to $W$ factors through $X_\lambda$ for some $\lambda$.
Then
$$
W \to X_\lambda \times_S T
$$
is a monomorphism (hence separated) and locally quasi-finite (because
$W \to X \times_Y T \to T$ is locally quasi-finite by our assumption
on $X \to Y$, see Morphisms of Spaces, Lemma
\ref{spaces-morphisms-lemma-permanence-quasi-finite}).
Hence $W$ is a scheme by
Morphisms of Spaces, Proposition
\ref{spaces-morphisms-proposition-locally-quasi-finite-separated-over-scheme}.
Thus $X \times_Y T$ is a scheme by
Properties of Spaces, Lemma \ref{spaces-properties-lemma-subscheme}.
\end{proof}

\begin{lemma}
\label{lemma-countable-affine-formal-algebraic-space}
Let $S$ be a scheme. Let $X$ be an affine formal algebraic space over $S$.
The following are equivalent
\begin{enumerate}
\item there exists a system $X_1 \to X_2 \to X_3 \to \ldots$
of thickenings of affine schemes over $S$ such that $X = \colim X_n$,
\item there exists a choice $X = \colim X_\lambda$ as in
Definition \ref{definition-affine-formal-algebraic-space}
such that $\Lambda$ is countable.
\end{enumerate}
\end{lemma}

\begin{proof}
This follows from the observation that a countable directed partially
ordered set has a cofinal subset isomorphic to $(\mathbf{N}, \geq)$.
See proof of Algebra, Lemma \ref{algebra-lemma-ML-limit-nonempty}.
\end{proof}

\begin{definition}
\label{definition-countable}
Let $S$ be a scheme. Let $X$ be an affine formal algebraic space over $S$.
We say $X$ is {\it countably indexed} if the equivalent conditions of
Lemma \ref{lemma-countable-affine-formal-algebraic-space} are satisfied.
\end{definition}

\noindent
In the language of \cite{BVGD} this is expressed by saying that
$X$ is an $\aleph_0$-ind scheme.

\begin{lemma}
\label{lemma-implications-between-types}
Let $S$ be a scheme. Let $X$ be an affine formal algebraic space over $S$.
\begin{enumerate}
\item If $X$ is countably indexed, then $X$ is McQuillan.
\item If $X$ is adic, then $X$ is countably indexed.
\end{enumerate}
\end{lemma}

\begin{proof}
Proof of (1). Write $X = \lim X_n$ for some system
$X_1 \to X_2 \to X_3 \to \ldots$ of thickenings of affine schemes over $S$.
Then
$$
A = \lim \Gamma(X_n, \mathcal{O}_{X_n})
$$
surjects onto each $\Gamma(X_n, \mathcal{O}_{X_n})$ because the transition
maps are surjections as the morphisms $X_n \to X_{n + 1}$ are closed
immersions.

\medskip\noindent
Proof of (2). By definition there exists an adic topological ring $A$
such that $X = \colim \Spec(A/I)$ where the colimit is over the ideals
of definition of $A$. As $A$ is adic, there exits an ideal $I$
such that $\{I^n\}$ forms a fundamental system of neighbourhoods of $0$.
Then each $I^n$ is an ideal of definition and $X = \colim \Spec(A/I^n)$.
Thus $X$ is countably indexed.
\end{proof}







\section{Formal algebraic spaces}
\label{section-formal-algebraic-spaces}

\noindent
We take a break from our habit of introducing new concepts first
for rings, then for schemes, and then for algebraic spaces, by
introducing formal algebraic spaces without first introducing
formal schemes. The general idea will be that a formal algebraic space
is a sheaf in the fppf topology which \'etale locally is an
affine formal scheme in the sense of \cite{BVGD}.
Related material can be found in \cite{Yasuda}.

\medskip\noindent
In the definition of a formal algebraic space we are going to
borrow some terminology from
Bootstrap, Sections
\ref{bootstrap-section-morphism-representable-by-spaces} and
\ref{bootstrap-section-representable-by-spaces-properties}.

\begin{definition}
\label{definition-formal-algebraic-space}
Let $S$ be a scheme. We say a sheaf $X$ on $(\Sch/S)_{fppf}$ is a
{\it formal algebraic space} if there exist a family of maps
$\{X_i \to X\}_{i \in I}$ of sheaves such that
\begin{enumerate}
\item $X_i$ is an affine formal algebraic space,
\item $X_i \to X$ is representable by algebraic spaces and \'etale,
\item $\coprod X_i \to X$ is surjective as a map of sheaves
\end{enumerate}
and $X$ satisfies a set theoretic condition
(see Remark \ref{remark-set-theoretic}). A
{\it morphism of formal algebraic spaces}
over $S$ is a map of sheaves.
\end{definition}

\noindent
Discussion. Sanity check: an affine formal algebraic space is
a formal algebraic space. In the situation of the definition
the morphisms $X_i \to X$ are representable (by schemes), see
Lemma \ref{lemma-presentation-representable}.
By Bootstrap, Lemma
\ref{bootstrap-lemma-surjective-flat-locally-finite-presentation}
we could instead of asking $\coprod X_i \to X$
to be surjective as a map of sheaves, require that it be
surjective (which makes sense because it is representable).

\medskip\noindent
Our notion of a formal algebraic space is {\bf very general}.
In fact, even affine formal algebraic spaces as defined above
are very nasty objects. However, they do have an underlying
reduced algebraic space as the following lemma demonstrates.

\begin{lemma}
\label{lemma-reduction-formal-algebraic-space}
Let $S$ be a scheme. Let $X$ be a formal algebraic space over $S$.
There exists a reduced algebraic space $X_{red}$ and a representable
morphism $X_{red} \to X$ which is a thickening. A morphism $U \to X$
with $U$ a reduced algebraic space factors uniquely through $X_{red}$.
\end{lemma}

\begin{proof}
First assume that $X$ is an affine formal algebraic space.
Say $X = \colim X_\lambda$ as in
Definition \ref{definition-affine-formal-algebraic-space}.
Since the transition morphisms are thickenings, the affine
schemes $X_\lambda$ all have isomorphic reductions $X_{red}$.
The morphism $X_{red} \to X$ is representable and a thickening
by Lemma \ref{lemma-covering-by-thickenings} and the fact that
compositions of thickenings are thickenings. We omit the
verification of the universal
property (use Schemes, Definition
\ref{schemes-definition-reduced-induced-scheme},
Schemes, Lemma \ref{schemes-lemma-map-into-reduction},
Properties of Spaces, Definition
\ref{spaces-properties-definition-reduced-induced-space}, and
Properties of Spaces, Lemma \ref{spaces-properties-lemma-map-into-reduction}).

\medskip\noindent
Let $X$ and $\{X_i \to X\}_{i \in I}$ be as in
Definition \ref{definition-formal-algebraic-space}.
For each $i$ let $X_{i, red} \to X_i$ be the reduction as
constructed above. For $i, j \in I$ the projection
$X_{i, red} \times_X X_j \to X_{i, red}$ is an \'etale (by assumption)
morphism of schemes (by Lemma \ref{lemma-presentation-representable}).
Hence $X_{i, red} \times_X X_j$ is reduced (see
Descent, Lemma \ref{descent-lemma-reduced-local-smooth}).
Thus the projection $X_{i, red} \times_X X_j \to X_j$ factors
through $X_{j, red}$ by the universal property. We conclude that
$$
R_{ij} = X_{i, red} \times_X X_j = X_{i, red} \times_X X_{j, red} =
X_i \times_X X_{j, red}
$$
because the morphisms $X_{i, red} \to X_i$ are injections of sheaves.
Set $U = \coprod X_{i, red}$, set
$R = \coprod R_{ij}$, and denote $s, t : R \to U$ the two
projections. As a sheaf $R = U \times_X U$ and $s$ and $t$
are \'etale. Then $(t, s) : R \to U$ defines an \'etale equivalence
relation by our observations above. Thus $X_{red} = U/R$ is an
algebraic space by Spaces, Theorem \ref{spaces-theorem-presentation}.
By construction the diagram
$$
\xymatrix{
\coprod X_{i, red} \ar[r] \ar[d] & \coprod X_i \ar[d] \\
X_{red} \ar[r] & X
}
$$
is cartesian. Since the right vertical arrow is \'etale surjective
and the top horizontal arrow is representable and a thickening
we conclude that $X_{red} \to X$ is representable by
Bootstrap, Lemma \ref{bootstrap-lemma-after-fppf-sep-lqf}
(to verify the assumptions of the lemma use that a surjective
\'etale morphism is surjective, flat, and locally of finite
presentation and use that thickenings are separated and locally quasi-finite).
Then we can use Spaces, Lemma
\ref{spaces-lemma-descent-representable-transformations-property}
to conclude that $X_{red} \to X$ is a thickening
(use that being a thickening is equivalent to being
a surjective, universally injective, closed immersion).

\medskip\noindent
Finally, suppose that $U \to X$ is a morphism with
$U$ a reduced algebraic space over $S$. Then each $X_i \times_X U$
is \'etale over $U$ and therefore reduced (by our defintion of
reduced algebraic spaces in Properties of Spaces, Section
\ref{spaces-properties-section-types-properties}).
Then $X_i \times_X U \to X_i$ factors through $X_{i, red}$.
Hence $U \to X$ factors through $X_{red}$ because
$\{X_i \times_X U \to U\}$ is an \'etale covering.
\end{proof}

\begin{lemma}
\label{lemma-diagonal-formal-algebraic-space}
Let $S$ be a scheme. If $X$ is a formal algebraic space over
$S$, then the diagonal morphism $\Delta : X \to X \times_S X$
is representable, a monomorphism, locally quasi-finite,
locally of finite type, and separated.
\end{lemma}

\begin{proof}
Suppose given $U \to X$ and $V \to X$ with $U, V$ schemes over $S$.
Then $U \times_X V$ is a sheaf. Choose $\{X_i \to X\}$ as in
Definition \ref{definition-formal-algebraic-space}.
For every $i$ the morphism
$$
(U \times_X X_i) \times_{X_i} (V \times_X X_i)
= (U \times_X V) \times_X X_i \to U \times_X V
$$
is representable and \'etale as a base change of $X_i \to X$
and its source is a scheme (use
Lemmas \ref{lemma-diagonal-affine-formal-algebraic-space} and
\ref{lemma-presentation-representable}). These maps are jointly surjective
hence $U \times_X V$ is an algebraic space by
Bootstrap, Theorem \ref{bootstrap-theorem-final-bootstrap}.
The morphism $U \times_X V \to U \times_S V$ is a monomorphism.
It is also locally quasi-finite, because on precomposing with
the morphism displayed above we obtain the composition
$$
(U \times_X X_i) \times_{X_i} (V \times_X X_i)
\to (U \times_X X_i) \times_S (V \times_X X_i)
\to U \times_S V
$$
which is locally quasi-finite as a composition of a closed
immersion (Lemma \ref{lemma-diagonal-affine-formal-algebraic-space})
and an \'etale morphism, see
Descent on Spaces, Lemma
\ref{spaces-descent-lemma-locally-quasi-finite-etale-local-source}.
Hence we conclude that $U \times_X V$ is a scheme by
Morphisms of Spaces, Proposition
\ref{spaces-morphisms-proposition-locally-quasi-finite-separated-over-scheme}.
Thus $\Delta$ is representable, see
Spaces, Lemma \ref{spaces-lemma-representable-diagonal}.

\medskip\noindent
Let $W \to X \times_S X$ be a morphism where $W$ is a scheme over $S$.
For each $i$ consider the diagram
$$
\xymatrix{
W \times_{(X \times_S X)} X_i \ar[d] \ar[r] &
W \times_{(X \times_S X)} (X_i \times_S X_i) \ar[d] \\
W \times_{(X \times_S X)} X \ar[r] & W
}
$$
The vertical arrows are \'etale because $X_i \to X$ is \'etale.
The top horizontal arrow is a closed immersion by
Lemma \ref{lemma-diagonal-affine-formal-algebraic-space}.
Because the maps $X_i \to X$ are jointly surjective, this certainly
implies that the lower horizontal arrow is locally of finite type, see
Morphisms of Spaces, Lemma \ref{spaces-morphisms-lemma-finite-type-local}.
Since it is clearly a monomorphism, we conclude that
it is locally quasi-finite (Morphisms of Spaces, Lemma
\ref{spaces-morphisms-lemma-monomorphism-loc-finite-type-loc-quasi-finite}).
It is separated because any monomorphism is separated
(Morphisms of Spaces, Lemma
\ref{spaces-morphisms-lemma-monomorphism-separated}). This finishes
the proof.
\end{proof}

\begin{remark}
\label{remark-compare-with-formal-schemes}
Modulo set theoretic issues the category of formal schemes \`a la EGA
(see Section \ref{section-formal-schemes-EGA}) is equivalent to a full
subcategory of the category of formal algebraic spaces. To explain this
we assume our base scheme is $\Spec(\mathbf{Z})$. By
Lemma \ref{lemma-formal-scheme-sheaf-fppf} the functor of points
$h_\mathfrak X$ associated to a formal scheme $\mathfrak X$ is a sheaf
in the fppf topology. By Lemma \ref{lemma-fully-faithful}
the assignment $\mathfrak X \mapsto h_\mathfrak X$ is a fully faithful
embedding of the category of formal schemes into the category of
fppf sheaves. Given a formal scheme $\mathfrak X$ we choose an open covering
$\mathfrak X = \bigcup \mathfrak X_i$ with $\mathfrak X_i$
affine formal schemes. Then $h_{\mathfrak X_i}$
is an affine formal algebraic space by
Remark \ref{remark-compare-with-affine-formal-schemes}.
The morphisms $h_{\mathfrak X_i} \to h_\mathfrak X$ are representable
and open immersions. Thus $\{h_{\mathfrak X_i} \to h_\mathfrak X\}$
is a family as in Definition \ref{definition-formal-algebraic-space}
and we see that $h_\mathfrak X$ is a formal algebraic space.
\end{remark}

\begin{remark}
\label{remark-set-theoretic}
Let $S$ be a scheme and let $(\Sch/S)_{fppf}$ be a big fppf site as
in Topologies, Definition \ref{topologies-definition-big-small-fppf}.
As our set theoretic condition on $X$ in
Definitions \ref{definition-affine-formal-algebraic-space} and
\ref{definition-formal-algebraic-space} we take:
there exist objects $U, R$ of $(\Sch/S)_{fppf}$, a
morphism $U \to X$ which is a surjection of fppf sheaves, and
a morphism $R \to U \times_X U$ which is a surjection of fppf sheaves.
In other words, we require our sheaf to be a coequalizer of
two maps between representable sheaves.
Here are some observations which imply this notion behaves
reasonably well:
\begin{enumerate}
\item Suppose $X = \colim_{\lambda \in \Lambda} X_\lambda$
and the system satisfies conditions (1) and (2) of
Definition \ref{definition-affine-formal-algebraic-space}. Then
$U = \coprod_{\lambda \in \Lambda} X_\lambda \to X$ is a surjection
of fppf sheaves. Moreover, $U \times_X U$ is a closed subscheme
of $U \times_S U$ by Lemma \ref{lemma-diagonal-affine-formal-algebraic-space}.
Hence if $U$ is representable by an object of $(\Sch/S)_{fppf}$
then $U \times_S U$ is too (see Sets, Lemma \ref{sets-lemma-what-is-in-it})
and the set theoretic condition is satisfied. This is always the case
if $\Lambda$ is countable, see Sets, Lemma \ref{sets-lemma-what-is-in-it}.
\item Sanity check. Let $\{X_i \to X\}_{i \in I}$ be as in
Definition \ref{definition-formal-algebraic-space}
(with the set theoretic condition as formulated above)
and assume that each $X_i$ is actually an affine scheme.
Then $X$ is an algebraic space. Namely, if we choose a larger
big fppf site $(\Sch'/S)_{fppf}$ such that $U' = \coprod X_i$
and $R' = \coprod X_i \times_X X_j$ are representable by objects
in it, then $X' = U'/R'$ will be an object of the category
of algebraic spaces for this choice. Then an application of
Spaces, Lemma \ref{spaces-lemma-fully-faithful} shows that
$X$ is an algebraic space for $(\Sch/S)_{fppf}$.
\item Let $\{X_i \to X\}_{i \in I}$ be a family of maps of sheaves
satisfying conditions (1), (2), (3) of
Definition \ref{definition-formal-algebraic-space}.
For each $i$ we can pick $U_i \in \Ob((\Sch/S)_{fppf})$
and $U_i \to X_i$ which is a surjection of sheaves.
Thus if $I$ is not too large (for example countable) then
$U = \coprod U_i \to X$ is a surjection of sheaves and
$U$ is representable by an object of $(\Sch/S)_{fppf}$.
To get $R \in \Ob((\Sch/S)_{fppf})$ surjecting onto $U \times_X U$
it suffices to assume the diagonal $\Delta : X \to X \times_S X$ is not
too wild, for example this always works if the diagonal of $X$ is
quasi-compact, i.e., $X$ is quasi-separated.
\end{enumerate}
\end{remark}








\section{Completion along a closed subset}
\label{section-completion}

\noindent
Our notion of a formal algebraic space is well adapted to taking the
completion along a closed.

\begin{lemma}
\label{lemma-completion-affine-is-affine-formal-algebraic-space}
Let $S$ be a scheme. Let $X$ be an affine scheme over $S$.
Let $T \subset |X|$ be a closed subset. Then the functor
$$
(\Sch/S)_{fppf} \longrightarrow \textit{Sets},\quad
U \longmapsto \{f : U \to X \mid f(|U|) \subset T\}
$$
is a McQuillan affine formal algebraic space.
\end{lemma}

\begin{proof}
Say $X = \Spec(A)$ and $T$ corresponds to the radical ideal $I \subset A$.
Let $U = \Spec(B)$ be an affine scheme over $S$ and let
$f : U \to X$ be an element of $F(U)$. Then $f$ corresponds to a
ring map $\varphi : A \to B$ such that every prime of $B$ contains
$\varphi(I) B$. Thus every element of $\varphi(I)$ is nilpotent in $B$, see
Algebra, Lemma \ref{algebra-lemma-Zariski-topology}.
Setting $J = \Ker(\varphi)$ we conclude that $I/J$ is a locally nilpotent
ideal in $A/J$. Equivalently, $V(J) = V(I) = T$.
In other words, the functor of the lemma equals
$\colim \Spec(A/J)$ where the colimit is over the
collection of ideals $J$ with $V(J) = T$.
Thus our functor is an affine formal algebraic space. It is McQuillan
(Definition \ref{definition-types-affine-formal-algebraic-space})
because the maps $A \to A/J$ are surjective
and hence $A^\wedge = \lim A/J \to A/J$ is surjective, see
Lemma \ref{lemma-mcquillan-affine-formal-algebraic-space}.
\end{proof}

\begin{lemma}
\label{lemma-completion-is-formal-algebraic-space}
Let $S$ be a scheme. Let $X$ be an algebraic space over $S$.
Let $T \subset |X|$ be a closed subset. Then the functor
$$
(\Sch/S)_{fppf} \longrightarrow \textit{Sets},\quad
U \longmapsto \{f : U \to X \mid f(|U|) \subset T\}
$$
is a formal algebraic space.
\end{lemma}

\begin{proof}
Denote $F$ the functor. Let $\{U_i \to U\}$ be an fppf covering.
Then $\coprod |U_i| \to |U|$ is surjective. Since $X$ is an fppf
sheaf, it follows that $F$ is an fppf sheaf.

\medskip\noindent
Let $\{g_i : X_i \to X\}$ be an \'etale covering such that $X_i$ is affine
for all $i$, see Properties of Spaces, Lemma
\ref{spaces-properties-lemma-cover-by-union-affines}.
The morphisms $F \times_X X_i \to F$ are \'etale
(see Spaces, Lemma
\ref{spaces-lemma-base-change-representable-transformations-property})
and the map $\coprod F \times_X X_i \to F$ is a surjection of sheaves.
Thus it suffices to prove that $F \times_X X_i$ is an affine formal
algebraic space. A $U$-valued point of $F \times_X X_i$ is a
morphism $U \to X_i$ whose image is contained in the closed subset
$g_i^{-1}(T) \subset |X_i|$. Thus this follows from
Lemma \ref{lemma-completion-affine-is-affine-formal-algebraic-space}.
\end{proof}










\section{Other chapters}

\begin{multicols}{2}
\begin{enumerate}
\item \hyperref[introduction-section-phantom]{Introduction}
\item \hyperref[conventions-section-phantom]{Conventions}
\item \hyperref[sets-section-phantom]{Set Theory}
\item \hyperref[categories-section-phantom]{Categories}
\item \hyperref[topology-section-phantom]{Topology}
\item \hyperref[sheaves-section-phantom]{Sheaves on Spaces}
\item \hyperref[algebra-section-phantom]{Commutative Algebra}
\item \hyperref[sites-section-phantom]{Sites and Sheaves}
\item \hyperref[homology-section-phantom]{Homological Algebra}
\item \hyperref[derived-section-phantom]{Derived Categories}
\item \hyperref[more-algebra-section-phantom]{More Algebra}
\item \hyperref[simplicial-section-phantom]{Simplicial Methods}
\item \hyperref[modules-section-phantom]{Sheaves of Modules}
\item \hyperref[sites-modules-section-phantom]{Modules on Sites}
\item \hyperref[injectives-section-phantom]{Injectives}
\item \hyperref[cohomology-section-phantom]{Cohomology of Sheaves}
\item \hyperref[sites-cohomology-section-phantom]{Cohomology on Sites}
\item \hyperref[hypercovering-section-phantom]{Hypercoverings}
\item \hyperref[schemes-section-phantom]{Schemes}
\item \hyperref[constructions-section-phantom]{Constructions of Schemes}
\item \hyperref[properties-section-phantom]{Properties of Schemes}
\item \hyperref[morphisms-section-phantom]{Morphisms of Schemes}
\item \hyperref[coherent-section-phantom]{Coherent Cohomology}
\item \hyperref[divisors-section-phantom]{Divisors}
\item \hyperref[limits-section-phantom]{Limits of Schemes}
\item \hyperref[varieties-section-phantom]{Varieties}
\item \hyperref[chow-section-phantom]{Chow Homology}
\item \hyperref[topologies-section-phantom]{Topologies on Schemes}
\item \hyperref[descent-section-phantom]{Descent}
\item \hyperref[more-morphisms-section-phantom]{More on Morphisms}
\item \hyperref[flat-section-phantom]{More on Flatness}
\item \hyperref[groupoids-section-phantom]{Groupoid Schemes}
\item \hyperref[more-groupoids-section-phantom]{More on Groupoid Schemes}
\item \hyperref[etale-section-phantom]{\'Etale Morphisms of Schemes}
\item \hyperref[etale-cohomology-section-phantom]{\'Etale Cohomology}
\item \hyperref[spaces-section-phantom]{Algebraic Spaces}
\item \hyperref[spaces-properties-section-phantom]{Properties of Algebraic Spaces}
\item \hyperref[spaces-morphisms-section-phantom]{Morphisms of Algebraic Spaces}
\item \hyperref[spaces-topologies-section-phantom]{Topologies on Algebraic Spaces}
\item \hyperref[spaces-descent-section-phantom]{Descent and Algebraic Spaces}
\item \hyperref[spaces-more-morphisms-section-phantom]{More on Morphisms of Spaces}
\item \hyperref[quot-section-phantom]{Quot and Hilbert Spaces}
\item \hyperref[stacks-section-phantom]{Stacks}
\item \hyperref[spaces-groupoids-section-phantom]{Groupoids in Algebraic Spaces}
\item \hyperref[spaces-more-groupoids-section-phantom]{More on Groupoids in Spaces}
\item \hyperref[bootstrap-section-phantom]{Bootstrap}
\item \hyperref[examples-stacks-section-phantom]{Examples of Stacks}
\item \hyperref[groupoids-quotients-section-phantom]{Quotients of Groupoids}
\item \hyperref[algebraic-section-phantom]{Algebraic Stacks}
\item \hyperref[criteria-section-phantom]{Criteria for Representability}
\item \hyperref[stacks-properties-section-phantom]{Properties of Algebraic Stacks}
\item \hyperref[stacks-morphisms-section-phantom]{Morphisms of Algebraic Stacks}
\item \hyperref[examples-section-phantom]{Examples}
\item \hyperref[exercises-section-phantom]{Exercises}
\item \hyperref[guide-section-phantom]{Guide to Literature}
\item \hyperref[desirables-section-phantom]{Desirables}
\item \hyperref[coding-section-phantom]{Coding Style}
\item \hyperref[fdl-section-phantom]{GNU Free Documentation License}
\item \hyperref[index-section-phantom]{Auto Generated Index}
\end{enumerate}
\end{multicols}


\bibliography{my}
\bibliographystyle{amsalpha}

\end{document}

