\IfFileExists{stacks-project.cls}{%
\documentclass{stacks-project}
}{%
\documentclass{amsart}
}

% The following AMS packages are automatically loaded with
% the amsart documentclass:
%\usepackage{amsmath}
%\usepackage{amssymb}
%\usepackage{amsthm}

% For dealing with references we use the comment environment
\usepackage{verbatim}
\newenvironment{reference}{\comment}{\endcomment}
%\newenvironment{reference}{}{}
\newenvironment{slogan}{\comment}{\endcomment}
\newenvironment{history}{\comment}{\endcomment}

% For commutative diagrams you can use
% \usepackage{amscd}
\usepackage[all]{xy}

% We use 2cell for 2-commutative diagrams.
\xyoption{2cell}
\UseAllTwocells

% To put source file link in headers.
% Change "template.tex" to "this_filename.tex"
% \usepackage{fancyhdr}
% \pagestyle{fancy}
% \lhead{}
% \chead{}
% \rhead{Source file: \url{template.tex}}
% \lfoot{}
% \cfoot{\thepage}
% \rfoot{}
% \renewcommand{\headrulewidth}{0pt}
% \renewcommand{\footrulewidth}{0pt}
% \renewcommand{\headheight}{12pt}

\usepackage{multicol}

% For cross-file-references
\usepackage{xr-hyper}

% Package for hypertext links:
\usepackage{hyperref}

% For any local file, say "hello.tex" you want to link to please
% use \externaldocument[hello-]{hello}
\externaldocument[introduction-]{introduction}
\externaldocument[conventions-]{conventions}
\externaldocument[sets-]{sets}
\externaldocument[categories-]{categories}
\externaldocument[topology-]{topology}
\externaldocument[sheaves-]{sheaves}
\externaldocument[sites-]{sites}
\externaldocument[stacks-]{stacks}
\externaldocument[fields-]{fields}
\externaldocument[algebra-]{algebra}
\externaldocument[brauer-]{brauer}
\externaldocument[homology-]{homology}
\externaldocument[derived-]{derived}
\externaldocument[simplicial-]{simplicial}
\externaldocument[more-algebra-]{more-algebra}
\externaldocument[smoothing-]{smoothing}
\externaldocument[modules-]{modules}
\externaldocument[sites-modules-]{sites-modules}
\externaldocument[injectives-]{injectives}
\externaldocument[cohomology-]{cohomology}
\externaldocument[sites-cohomology-]{sites-cohomology}
\externaldocument[dga-]{dga}
\externaldocument[dpa-]{dpa}
\externaldocument[hypercovering-]{hypercovering}
\externaldocument[schemes-]{schemes}
\externaldocument[constructions-]{constructions}
\externaldocument[properties-]{properties}
\externaldocument[morphisms-]{morphisms}
\externaldocument[coherent-]{coherent}
\externaldocument[divisors-]{divisors}
\externaldocument[limits-]{limits}
\externaldocument[varieties-]{varieties}
\externaldocument[topologies-]{topologies}
\externaldocument[descent-]{descent}
\externaldocument[perfect-]{perfect}
\externaldocument[more-morphisms-]{more-morphisms}
\externaldocument[flat-]{flat}
\externaldocument[groupoids-]{groupoids}
\externaldocument[more-groupoids-]{more-groupoids}
\externaldocument[etale-]{etale}
\externaldocument[chow-]{chow}
\externaldocument[intersection-]{intersection}
\externaldocument[pic-]{pic}
\externaldocument[adequate-]{adequate}
\externaldocument[dualizing-]{dualizing}
\externaldocument[duality-]{duality}
\externaldocument[discriminant-]{discriminant}
\externaldocument[local-cohomology-]{local-cohomology}
\externaldocument[curves-]{curves}
\externaldocument[resolve-]{resolve}
\externaldocument[models-]{models}
\externaldocument[pione-]{pione}
\externaldocument[etale-cohomology-]{etale-cohomology}
\externaldocument[proetale-]{proetale}
\externaldocument[crystalline-]{crystalline}
\externaldocument[spaces-]{spaces}
\externaldocument[spaces-properties-]{spaces-properties}
\externaldocument[spaces-morphisms-]{spaces-morphisms}
\externaldocument[decent-spaces-]{decent-spaces}
\externaldocument[spaces-cohomology-]{spaces-cohomology}
\externaldocument[spaces-limits-]{spaces-limits}
\externaldocument[spaces-divisors-]{spaces-divisors}
\externaldocument[spaces-over-fields-]{spaces-over-fields}
\externaldocument[spaces-topologies-]{spaces-topologies}
\externaldocument[spaces-descent-]{spaces-descent}
\externaldocument[spaces-perfect-]{spaces-perfect}
\externaldocument[spaces-more-morphisms-]{spaces-more-morphisms}
\externaldocument[spaces-flat-]{spaces-flat}
\externaldocument[spaces-groupoids-]{spaces-groupoids}
\externaldocument[spaces-more-groupoids-]{spaces-more-groupoids}
\externaldocument[bootstrap-]{bootstrap}
\externaldocument[spaces-pushouts-]{spaces-pushouts}
\externaldocument[groupoids-quotients-]{groupoids-quotients}
\externaldocument[spaces-more-cohomology-]{spaces-more-cohomology}
\externaldocument[spaces-simplicial-]{spaces-simplicial}
\externaldocument[formal-spaces-]{formal-spaces}
\externaldocument[restricted-]{restricted}
\externaldocument[spaces-resolve-]{spaces-resolve}
\externaldocument[formal-defos-]{formal-defos}
\externaldocument[defos-]{defos}
\externaldocument[cotangent-]{cotangent}
\externaldocument[examples-defos-]{examples-defos}
\externaldocument[algebraic-]{algebraic}
\externaldocument[examples-stacks-]{examples-stacks}
\externaldocument[stacks-sheaves-]{stacks-sheaves}
\externaldocument[criteria-]{criteria}
\externaldocument[artin-]{artin}
\externaldocument[quot-]{quot}
\externaldocument[stacks-properties-]{stacks-properties}
\externaldocument[stacks-morphisms-]{stacks-morphisms}
\externaldocument[stacks-limits-]{stacks-limits}
\externaldocument[stacks-cohomology-]{stacks-cohomology}
\externaldocument[stacks-perfect-]{stacks-perfect}
\externaldocument[stacks-introduction-]{stacks-introduction}
\externaldocument[stacks-more-morphisms-]{stacks-more-morphisms}
\externaldocument[stacks-geometry-]{stacks-geometry}
\externaldocument[moduli-]{moduli}
\externaldocument[moduli-curves-]{moduli-curves}
\externaldocument[examples-]{examples}
\externaldocument[exercises-]{exercises}
\externaldocument[guide-]{guide}
\externaldocument[desirables-]{desirables}
\externaldocument[coding-]{coding}
\externaldocument[obsolete-]{obsolete}
\externaldocument[fdl-]{fdl}
\externaldocument[index-]{index}

% Theorem environments.
%
\theoremstyle{plain}
\newtheorem{theorem}[subsection]{Theorem}
\newtheorem{proposition}[subsection]{Proposition}
\newtheorem{lemma}[subsection]{Lemma}

\theoremstyle{definition}
\newtheorem{definition}[subsection]{Definition}
\newtheorem{example}[subsection]{Example}
\newtheorem{exercise}[subsection]{Exercise}
\newtheorem{situation}[subsection]{Situation}

\theoremstyle{remark}
\newtheorem{remark}[subsection]{Remark}
\newtheorem{remarks}[subsection]{Remarks}

\numberwithin{equation}{subsection}

% Macros
%
\def\lim{\mathop{\rm lim}\nolimits}
\def\colim{\mathop{\rm colim}\nolimits}
\def\Spec{\mathop{\rm Spec}}
\def\Hom{\mathop{\rm Hom}\nolimits}
\def\Ext{\mathop{\rm Ext}\nolimits}
\def\SheafHom{\mathop{\mathcal{H}\!{\it om}}\nolimits}
\def\SheafExt{\mathop{\mathcal{E}\!{\it xt}}\nolimits}
\def\Sch{\textit{Sch}}
\def\Mor{\mathop{\rm Mor}\nolimits}
\def\Ob{\mathop{\rm Ob}\nolimits}
\def\Sh{\mathop{\textit{Sh}}\nolimits}
\def\NL{\mathop{N\!L}\nolimits}
\def\proetale{{pro\text{-}\acute{e}tale}}
\def\etale{{\acute{e}tale}}
\def\QCoh{\textit{QCoh}}
\def\Ker{\mathop{\rm Ker}}
\def\Im{\mathop{\rm Im}}
\def\Coker{\mathop{\rm Coker}}
\def\Coim{\mathop{\rm Coim}}

%
% Macros for moduli stacks/spaces
%
\def\QCohstack{\mathcal{QC}\!{\it oh}}
\def\Cohstack{\mathcal{C}\!{\it oh}}
\def\Spacesstack{\mathcal{S}\!{\it paces}}
\def\Quotfunctor{{\rm Quot}}
\def\Hilbfunctor{{\rm Hilb}}
\def\Curvesstack{\mathcal{C}\!{\it urves}}
\def\Polarizedstack{\mathcal{P}\!{\it olarized}}
\def\Complexesstack{\mathcal{C}\!{\it omplexes}}
% \Pic is the operator that assigns to X its picard group, usage \Pic(X)
% \Picardstack_{X/B} denotes the Picard stack of X over B
% \Picardfunctor_{X/B} denotes the Picard functor of X over B
\def\Pic{\mathop{\rm Pic}\nolimits}
\def\Picardstack{\mathcal{P}\!{\it ic}}
\def\Picardfunctor{{\rm Pic}}
\def\Deformationcategory{\mathcal{D}\!{\it ef}}


% OK, start here.
%
\begin{document}

\title{Formal Algebraic Spaces}


\maketitle

\phantomsection
\label{section-phantom}

\tableofcontents

\section{Introduction}
\label{section-introduction}

\noindent
Formal schemes were introduced in \cite{EGA}. A more general version
of formal schemes was introduced in \cite{McQuillan} and another
in \cite{Yasuda}. Formal algebraic spaces were introduced in \cite{Kn}.
Related material and much besides can be found in
\cite{Abbes} and \cite{Fujiwara-Kato}.
This chapter introduces the notion of formal algebraic spaces
we will work with. Our definition is general enough to allow most
classes of formal schemes/spaces in the literature as full subcategories.

\medskip\noindent
Although we do discuss the comparison of some of these alternative theories
with ours, we do not always give full details when it is not necessary for
the logical development of the theory.











\section{Formal schemes \`a la EGA}
\label{section-formal-schemes-EGA}

\noindent
In this section we review the construction of formal schemes in \cite{EGA}.
This notion, although very useful in algebraic geometry,
may not always be the correct one to consider. Perhaps it is better to say
that in the setup of the theory a number choices are made, where for
different purposes others might work better. And indeed in the literature
one can find many different closely related theories
adapted to the problem the authors may want to consider. Still, one
of the major advantages of the theory as sketched here is that one
gets to work with definite geometric objects.

\medskip\noindent
Before we start we should point out an issue with the sheaf condition
for sheaves of topological rings or more generally sheaves of topological
spaces. Namely, the big categories
\begin{enumerate}
\item category of topological spaces,
\item category of topological abelian groups,
\item category of topological rings,
\item category of topological modules over a given topological ring,
\end{enumerate}
endowed with their natural forgetful functors to $\textit{Sets}$ are not
examples of types of algebraic structures as defined in
Sheaves, Section \ref{sheaves-section-algebraic-structures}.
Thus we cannot blithely use the machinery developed in that
chapter to them. On the other hand, each of the categories
listed above has products, so we can define the notion of a
sheaf as in Sheaves, Definition
\ref{sheaves-definition-sheaf-values-in-category}.
Moreover, products still commute with the forgetful functor
to $\textit{Sets}$ so the underlying presheaf of sets
of a sheaf of topological spaces (for example) is a sheaf of
sets. The key difference is that for an open covering
$U = \bigcup_{i \in I} U_i$ the diagram
$$
\xymatrix{
\mathcal{F}(U) \ar[r]
&
\prod\nolimits_{i\in I}
\mathcal{F}(U_i)
\ar@<1ex>[r] \ar@<-1ex>[r]
&
\prod\nolimits_{(i_0, i_1) \in I \times I}
\mathcal{F}(U_{i_0} \cap U_{i_1})
}
$$
has to be an equalizer diagram in the category of topological
spaces, groups, rings, modules, i.e., that the first map identifies
$\mathcal{F}(U)$ with a subspace of $\prod_{i \in I} \mathcal{F}(U_i)$
which is endowed with the product topology.

\medskip\noindent
The stalk $\mathcal{F}_x$ of a sheaf $\mathcal{F}$
of topological spaces, abelian groups, rings, or modules
at a point $x \in X$ is defined as the colimit over open neighbourhoods
$$
\mathcal{F}_x = \colim_{x\in U} \mathcal{F}(U)
$$
in the corresponding category, which is the same as taking
the colimit on the level of underlying sheaves of sets, groups,
rings, or modules but comes equipped with a topology (the colimit
topology). One can sheafify presheaves of topological spaces,
abelian groups, rings, or modules and taking stalks commuts
with this operation, see
Remark \ref{remark-sheafification-of-presheaves-in-top}.

\medskip\noindent
Let $f : X \to Y$ be a continuous map of topological spaces.
There is a functor $f_*$ from the category of sheaves of topological
spaces, abelian groups, rings, modules, to the corresponding category
of sheaves on $Y$ which is defined by setting
$f_*\mathcal{F}(V) = \mathcal{F}(f^{-1}V)$ as usual.
(We delay discussing the pullback in this setting till later.)
We define the notion of an $f$-map $\xi : \mathcal{G} \to \mathcal{F}$
between a sheaf of topoligical spaces $\mathcal{G}$ on $Y$ and
a sheaf of topological spaces $\mathcal{F}$ on $X$ in exactly the
same manner as in Sheaves, Definition \ref{sheaves-definition-f-map}
with the additional constraint that
$\xi_V : \mathcal{G}(V) \to \mathcal{F}(f^{-1}V)$ be continuous
for every open $V \subset Y$. We have
$$
\{f\text{-maps from }\mathcal{G}\text{ to }\mathcal{F}\} =
\text{Mor}_{\Sh(Y, \textit{Top})}(\mathcal{G}, f_*\mathcal{F})
$$
as in Sheaves, Lemma \ref{sheaves-lemma-f-map}. Similarly for
sheaves of topological abelian groups, rings, modules. Finally,
let $\xi : \mathcal{G} \to \mathcal{F}$ be an $f$-map as above.
Then given $x \in X$ with image $y = f(x)$ there is a continuous
map
$$
\xi_x : \mathcal{G}_y \longrightarrow \mathcal{F}_x
$$
of stalks defined in exactly the same manner as in the discussion
following Sheaves, Definition \ref{sheaves-definition-composition-f-maps}.

\medskip\noindent
Using the discussion above, we can define a category $LTRS$ of
``locally topologically ringed spaces''. An object is a pair
$(X, \mathcal{O}_X)$ consisting of a topological space
$X$ and a sheaf of topological rings $\mathcal{O}_X$ whose stalks
$\mathcal{O}_{X, x}$ are local rings (if one forgets about the topology).
A morphism $(X, \mathcal{O}_X) \to (Y, \mathcal{O}_Y)$ of
$LTRS$ is a pair $(f, f^\sharp)$ where $f : X \to Y$ is a continuous
map of topological spaces and $f^\sharp : \mathcal{O}_Y \to \mathcal{O}_X$
is an $f$-map such that for every $x \in X$ the induced map
$$
f^\sharp_x : \mathcal{O}_{Y, f(x)} \longrightarrow \mathcal{O}_{X, x}
$$
is a local homomorphism of local rings (forgetting about the topologies).
The composition works in exactly the same manner as composition of
morphisms of locally ringed spaces.

\medskip\noindent
Assume now that the topological space $X$ has a basis consisting
of quasi-compact opens. Given a sheaf $\mathcal{G}$ of sets, abelian groups,
rings, modules over a ring, one can endow $\mathcal{G}$ with
the structure of a sheaf of topological spaces, abelian groups,
rings, modules. Namely, if $U \subset X$ is quasi-compact open,
we endow $\mathcal{F}(U)$ with the discrete topology. If $U \subset X$
is arbitrary, then we choose an open covering $U = \bigcup_{i \in I} U_i$
by quasi-compact opens and we endow $\mathcal{F}(U)$ with
the induced topology from $\prod_{i \in I} \mathcal{F}(U_i)$
(as we should do according to our discussion above).
The reader may verify (omitted) that we obtain a sheaf of topological
spaces, groups, rings, modules in this fashion. Let us say
that a sheaf of topological spaces, groups, rings, modules is
{\it pseudo-discrete} if the topology on $\mathcal{F}(U)$ is
discrete for every quasi-compact open $U \subset X$. Then
the construction given above is an adjoint to the forgetful functor
and induces an equivalence between the category of sheaves
of sets and the category of pseudo-discrete sheaves of topological space
(similarly for groups, rings, modules).

\medskip\noindent
Grothendieck and Dieudonn\'e first define formal affine schemes.
These correspond to admissible topological rings $A$, see
More on Algebra, Definition \ref{more-algebra-definition-topological-ring}.
Namely, given $A$ one considers a fundamental system $I_\lambda$ of ideals
of definition for the ring $A$. (In any admissible topological ring
the family of all ideals of definition forms a fundamental system.)
For each $\lambda$ we can consider the
scheme $\Spec(A/I_\lambda)$. For $I_\lambda \subset I_\mu$ the induced
morphism
$$
\Spec(A/I_\mu) \to \Spec(A/I_\lambda)
$$
is a thickening because $I_\mu^n \subset I_\lambda$ for some $n$.
Another way to see this, is to notice that the image of each of the
maps
$$
\Spec(A/I_\lambda) \to \Spec(A)
$$
is a homeomorphism onto the set of open prime ideals of $A$.
This motivates the definition
$$
\text{Spf}(A) = \{\text{open prime ideals }\mathfrak p \subset A\}
$$
endowed with the topology coming from $\Spec(A)$. For each $\lambda$
we can consider the structure sheaf $\mathcal{O}_{\Spec(A/I_\lambda}$
as a sheaf on $\text{Spf}(A)$. Let $\mathcal{O}_\lambda$ be the corresponding
pseudo-discrete sheaf of topological rings, see above.
Then we set
$$
\mathcal{O}_{\text{Spf}(A)} = \lim \mathcal{O}_\lambda
$$
where the limit is taken in the category of sheaves of topological rings.
The pair $(\text{Spf}(A), \mathcal{O}_{\text{Spf}(A)})$ is called the
{\it formal spectrum} of $A$.

\medskip\noindent
At this point one should check several things. The first is that
the stalks $\mathcal{O}_{\text{Spf}(A), x}$ are local rings
(forgetting about the topology). The second is that given
$f \in A$, for the corresponding open $D(f) \cap \text{Spf}(A)$
we have
$$
\Gamma(D(f) \cap \text{Spf}(A), \mathcal{O}_{\text{Spf}(A)})
= A_{\{f\}} = \lim (A/I_\lambda)_f
$$
as topological rings where $I_\lambda$ is a fundamental system of ideals
of definition as above. Moreover, the ring $A_{\{f\}}$ is admissible too and
$(\text{Spf}(A_f), \mathcal{O}_{\text{Spf}(A_{\{f\}})})$
is isomorphic to
$(D(f) \cap \text{Spf}(A),
\mathcal{O}_{\text{Spf}(A)}|_{D(f) \cap \text{Spf}(A)})$.
Finally, given a pair of admissible topological rings $A, B$
we have
\begin{equation}
\label{equation-morphisms-affine-formal-schemes}
\Mor_{LTRS}((\text{Spf}(B), \mathcal{O}_{\text{Spf}(B)}),
(\text{Spf}(A), \mathcal{O}_{\text{Spf}(A)}))
= \Hom_{cont}(A, B)
\end{equation}
where $LTRS$ is the category of ``locally topologically ringed spaces''
as defined above.

\medskip\noindent
Having said this, in \cite{EGA} a {\it formal scheme} is defined as a pair
$(\mathfrak X, \mathcal{O}_\mathfrak X)$ where $\mathfrak X$
is a topological space and $\mathcal{O}_\mathfrak X$ is a sheaf
of topological rings such that every point has an open neighbourhood
isomorphic (in $LTRS$) to an affine formal scheme.
A {\it morphism of formal schemes}
$f : (\mathfrak X, \mathcal{O}_\mathfrak X) \to
(\mathfrak Y, \mathcal{O}_\mathfrak Y)$
is a morphism in the category $LTRS$.

\medskip\noindent
Let $A$ be a ring endowed with the discrete topology. Then $A$ is
admissible and the formal scheme $\text{Spf}(A)$ is equal to
$\Spec(A)$. The structure sheaf $\mathcal{O}_{\text{Spf}(A)}$
is the pseudo-discrete sheaf of topological rings associated
to $\mathcal{O}_{\Spec(A)}$, in other words, its underlying
sheaf of rings is equal to $\mathcal{O}_{\Spec(A)}$ and the
ring $\mathcal{O}_{\text{Spf}(A)}(U) = \mathcal{O}_{\Spec(A)}(U)$
over a quasi-compact open $U$ has the discrete topology,
but not in general. Thus we can associate to every affine scheme
a formal affine scheme. In exactly the same manner we can start
with a general scheme $(X, \mathcal{O}_X)$ and associate to
it $(X, \mathcal{O}'_X)$ where $\mathcal{O}'_X$ is the
pseudo-discrete sheaf of topological rings whose underlying
sheaf of rings is $\mathcal{O}_X$. This construction is
compatible with morphisms and defines a functor
\begin{equation}
\label{equation-compare-schemes-formal-schemes}
\textit{Schemes} \longrightarrow \textit{Formal Schemes}
\end{equation}
It follows in a straightforward manner from
(\ref{equation-morphisms-affine-formal-schemes})
that this functor is fully faithful.

\medskip\noindent
Let $\mathfrak X$ be a formal scheme. Let us define the {\it size}
of the formal scheme by the formula
$\text{size}(\mathfrak X) = \max(\aleph_0, \kappa_1, \kappa_2)$
where $\kappa_1$ is the cardinality of the formal affine opens of
$\mathfrak X$ and $\kappa_2$ is the supremum of the cardinalities
of $\mathcal{O}_\mathfrak X(\mathfrak U)$ where
$\mathfrak U \subset \mathfrak X$ is such a formal affine open.

\begin{lemma}
\label{lemma-fully-faithful}
Choose a category of schemes $\Sch_\alpha$
as in Sets, Lemma \ref{sets-lemma-construct-category}.
Given a formal scheme $\mathfrak X$ let
$$
h_\mathfrak X : (\Sch_\alpha)^{opp} \longrightarrow \textit{Sets},\quad
h_\mathfrak X(S) = \Mor_{\textit{Formal Schemes}}(S, \mathfrak X)
$$
be its functor of points. Then we have
$$
\Mor_{\textit{Formal Schemes}}(\mathfrak X, \mathfrak Y) =
\Mor_{\textit{PSh}(\Sch_\alpha)}(h_\mathfrak X, h_\mathfrak Y)
$$
provided the size of $\mathfrak X$ is not too large.
\end{lemma}

\begin{proof}
First we observe that $h_\mathfrak X$ satisfies the sheaf property for
the Zariski topology for any formal scheme $\mathfrak X$ (see
Schemes, Definition \ref{schemes-definition-representable-by-open-immersions}).
This follows from the local nature of morphisms in the category
of formal schemes. Also, for an open immersion
$\mathfrak V \to \mathfrak W$ of formal schemes,
the corresponding tranformation of functors $h_\mathfrak V \to h_\mathfrak W$
is injective and representable by open immersions (see
Schemes, Definition \ref{schemes-definition-representable-by-open-immersions}).
Choose an open covering $\mathfrak X = \bigcup \mathfrak U_i$
of a formal scheme by affine formal schemes $\mathfrak U_i$.
Then the collection of functors
$h_{\mathfrak U_i}$ covers $h_\mathfrak X$ (see
Schemes, Definition \ref{schemes-definition-representable-by-open-immersions}).
Finally, note that
$$
h_{\mathfrak U_i} \times_{h_\mathfrak X} h_{\mathfrak U_j} =
h_{\mathfrak U_i \cap \mathfrak U_j}
$$
Hence in order to give a map $h_\mathfrak X \to h_\mathfrak Y$
is equivalent to giving a family of maps
$h_{\mathfrak U_i} \to h_\mathfrak Y$ which agree on overlaps.
Thus we can reduce the bijectivity (resp.\ injectivity) of the map
of the lemma to bijectivity (resp.\ injectivity) for the pairs
$(\mathfrak U_i, \mathfrak Y)$
and injectivity (resp.\ nothing)
for $(\mathfrak U_i \cap \mathfrak U_j, \mathfrak Y)$.
In this way we reduce to the case where $\mathfrak X$ is an
affine formal scheme. Say $\mathfrak X = \text{Spf}(A)$
for some admissible topological ring $A$. Also, choose a
fundamental system of ideals of definition $I_\lambda \subset A$.

\medskip\noindent
We can also localize on $\mathfrak Y$.
Namely, suppose that $\mathfrak V \subset \mathfrak Y$ is an
open formal subscheme and $\varphi : h_\mathfrak X \to h_\mathfrak Y$.
Then
$$
h_\mathfrak V \times_{h_\mathfrak Y, \varphi} h_\mathfrak X \to h_\mathfrak X
$$
is representable by open immersions. Pulling back to
$\Spec(A/I_\lambda)$ for all $\lambda$ we find an open subscheme
$U_\lambda \subset \Spec(A/I_\lambda)$. However, for
$I_\lambda \subset I_\mu$ the morphism $\Spec(A/I_\lambda) \to \Spec(A/I_\mu)$
pulls back $U_\mu$ to $U_\lambda$. Thus these glue to give
an open formal subscheme $\mathfrak U \subset \mathfrak X$.
A straightforward argument (omitted) shows that
$$
h_\mathfrak U = h_\mathfrak V \times_{h_\mathfrak Y} h_\mathfrak X
$$
In this way we see that given an open covering
$\mathfrak Y = \bigcup \mathfrak V_j$ and a transformation
of functors $\varphi :  h_\mathfrak X \to h_\mathfrak Y$
we obtain a corresponding open covering of $\mathfrak X$.
Since $\mathfrak X$ is affine, we can refine this covering by
a finite open covering
$\mathfrak X = \mathfrak U_1 \cup \ldots \cup \mathfrak U_n$
by affine formal subschemes. In other words, for each $i$ there
is a $j$ and a map $\varphi_i : h_{\mathfrak U_i} \to h_{\mathfrak V_j}$
such that
$$
\xymatrix{
h_{\mathfrak U_i} \ar[r]_{\varphi_i} \ar[d] & h_{\mathfrak V_j} \ar[d] \\
h_{\mathfrak X} \ar[r]^\varphi & h_\mathfrak Y
}
$$
commutes. With a few additional arguments (which we omit) this implies
that it suffices to prove the bijectivity of the lemma in case
both $\mathfrak X$ and $\mathfrak Y$ are affine formal schemes.

\medskip\noindent
Assume $\mathfrak X$ and $\mathfrak Y$ are affine formal schemes.
Say $\mathfrak X = \text{Spf}(A)$ and $\mathfrak Y = \text{Spf}(B)$.
Let $\varphi : h_\mathfrak X \to h_\mathfrak Y$ be a transformation
of functors. Let $I_\lambda \subset A$ be a fundamental system of
ideals of definition. The canonical inclusion morphsm
$i_\lambda : \Spec(A/I_\lambda) \to \mathfrak X$ maps to a morphism
$\varphi(i_\lambda) : \Spec(A/I_\lambda) \to \mathfrak Y$.
By (\ref{equation-morphisms-affine-formal-schemes}) this corresponds
to a continuous map $\chi_\lambda : B \to A/I_\lambda$.
Since $\varphi$ is a transformation of functors it follows
that for $I_\lambda \subset I_\mu$ the composition
$B \to A/I_\lambda \to A/I_\mu$ is equal to $\chi_\mu$.
In other words we obtain a ring map
$$
\chi = \lim \chi_\lambda : B \longrightarrow \lim A/I_\lambda = A
$$
This is a continuous homomorphism because the inverse image
of $I_\lambda$ is open for all $\lambda$ (as $A/I_\lambda$ has the discrete
topology and $\chi_\lambda$ is continuous). Thus we obtain
a morphism $\text{Spf}(\chi) : \mathfrak X \to \mathfrak Y$ by
(\ref{equation-morphisms-affine-formal-schemes}).
We omit the verification that this construction is the inverse
to the map of the lemma in this case.

\medskip\noindent
Set theoretic remarks. To make this work on the given category
of schemes $\Sch_\alpha$ we just have to make sure all the
schemes used in the proof above are isomorphic to objects of $\Sch_\alpha$.
In fact, a careful analysis shows that it suffices if the
schemes $\Spec(A/I_\lambda)$ occuring above are isomorphic to
objects of $\Sch_\alpha$. For this it certainly suffices to assume
the size of $\mathfrak X$ is at most the size of
a scheme contained in $\Sch_\alpha$.
\end{proof}

\begin{lemma}
\label{lemma-formal-scheme-sheaf-fppf}
Let $\mathfrak X$ be a formal scheme. The functor of points
$h_\mathfrak X$ (see Lemma \ref{lemma-fully-faithful})
satisfies the sheaf condition for fpqc coverings.
\end{lemma}

\begin{proof}
Topologies, Lemma \ref{topologies-lemma-sheaf-property-fpqc}
reduces us to the case of a Zariski covering and a covering
$\{\Spec(S) \to \Spec(R)\}$ with $R \to S$ faithfully flat.
We observed in the proof of Lemma \ref{lemma-fully-faithful} 
that $h_\mathfrak X$ satisfies the sheaf condition for Zariski coverings.

\medskip\noindent
Suppose that $R \to S$ is a faithfully flat ring map.
Denote $\pi : \Spec(S) \to \Spec(R)$ the
corresponding morphism of schemes. It is surjective and flat.
Let $f : \Spec(S) \to \mathfrak X$ be a morphism
such that $f \circ \text{pr}_1 = f \circ \text{pr}_2$
as maps $\Spec(S \otimes_R S) \to \mathfrak X$.
By Descent, Lemma \ref{descent-lemma-equiv-fibre-product}
we see that as a map on the underlying
sets $f$ is of the form $f = g \circ \pi$ for some
(set theoretic) map $g : \Spec(R) \to \mathfrak X$.
By Morphisms, Lemma \ref{morphisms-lemma-fpqc-quotient-topology}
and the fact that $f$ is continuous we see that $g$
is continuous.

\medskip\noindent
Pick $y \in \Spec(R)$. Choose $\mathfrak U \subset \mathfrak X$
an affine formal open subscheme containing $g(y)$.
Say $\mathfrak U = \text{Spf}(A)$ for some admissible topological
ring $A$. By the above we may choose an $r \in R$ such that
$x \in D(r) \subset g^{-1}(\mathfrak U)$.
The restriction of $f$ to $\pi^{-1}(D(r))$ into $\mathfrak U$
corresponds to a continuous ring map $A \to S_r$ by
(\ref{equation-morphisms-affine-formal-schemes}). The two induced ring maps
$A \to S_r \otimes_{R_r} S_r = (S \otimes_R S)_r$ are equal
by assumption on $f$.
Note that $R_r \to S_r$ is faithfully flat.
By Descent, Lemma \ref{descent-lemma-ff-exact} the equalizer of
the two arrows $S_r \to S_r \otimes_{R_r} S_r$ is $R_r$.
We conclude that $A \to S_r$ factors uniquely through a map $A \to S_r$
which is also continuous as it has the same (open) kernel as the
map $A \to S_r$. This map in turn gives a morphism $D(r) \to \mathfrak U$ by
(\ref{equation-morphisms-affine-formal-schemes}).

\medskip\noindent
What have we proved so far? We have shown that for any $y \in \Spec(R)$
there exists a standard affine open
$y \in D(r) \subset \Spec(R)$ such that the morphism
$f|_{\pi^{-1}(D(r))} : \pi^{-1}(D(r)) \to \mathfrak X$ factors uniquely
though some morphism $D(r) \to \mathfrak X$. We omit the
verification that these morphisms glue to the desired
morphism $\Spec(R) \to \mathfrak X$.
\end{proof}

\begin{remark}[McQuillan's variant]
\label{remark-mcquillan}
There is a variant of the construction of formal schemes due to
McQuillan, see \cite{McQuillan}.
He suggests a slight weakening of the condition of admissibility.
Namely, recall that an admissible topological ring is a complete
(and separated by our conventions) topological ring $A$
which is linearly topologized such that there exists an
ideal of defintion: an
open ideal $I$ such that any neighbourhood of $0$ contains $I^n$.
McQuillan works with what might be called {\it weakly admissible}
topological rings. A weakly admissible topological ring $A$ is a
complete (and separated by our conventions) topological ring
which is linearly topologized such that there exists an
{\it weak ideal of definition}: an open ideal $I$ such that
for all $f \in I$ we have
$f^n \to 0$ for $n \to \infty$. Similarly to the admissible case,
if $I$ is a weak ideal of definition and $J \subset A$ is an
open ideal, then $I \cap J$ is a weak ideal of definition.
Thus the weak ideals of definition form a fundamental system and
one can proceed along much the same route as above
to define a larger category of formal schemes based
on this notion. The analogues of Lemmas \ref{lemma-fully-faithful} and
\ref{lemma-formal-scheme-sheaf-fppf}
still hold in this setting (with the same proof).
\end{remark}

\begin{remark}[Sheafification of presheaves of topological spaces]
\label{remark-sheafification-of-presheaves-in-top}
\begin{reference}
\cite{Gray}
\end{reference}
In this remark we briefly discuss sheafification of presheaves
of topological spaces. The exact same arguments work for
presheaves of topological abelian groups, topological rings, and
topological modules (over a given topological ring). In order to
do this in the correct generality let us work over a site
$\mathcal{C}$. The reader who is interested in the case of (pre)sheaves
over a topological space $X$ should think of objects of $\mathcal{C}$
as the opens of $X$, of morphisms of $\mathcal{C}$ as inclusions of
opens, and of coverings in $\mathcal{C}$ as coverings in $X$, see
Sites, Example \ref{sites-example-site-topological}.
Denote $\Sh(\mathcal{C}, \textit{Top})$ the category of sheaves
of topological spaces on $\mathcal{C}$ and denote
$\textit{PSh}(\mathcal{C}, \textit{Top})$ the category of presheaves
of topological spaces on $\mathcal{C}$.
Let $\mathcal{F}$ be a presheaf of topological spaces on $\mathcal{C}$.
The sheafification $\mathcal{F}^\#$ should satisfy the formula
$$
\Mor_{\textit{PSh}(\mathcal{C}, \textit{Top})}(\mathcal{F}, \mathcal{G})
=
\Mor_{\Sh(\mathcal{C}, \textit{Top})}(\mathcal{F}^\#, \mathcal{G})
$$
functorially in $\mathcal{G}$ from $\Sh(\mathcal{C}, \textit{Top})$.
In other words, we are trying to construct the left adjoint
to the inclusion fuctor
$\Sh(\mathcal{C}, \textit{Top}) \to \Sh(\mathcal{C}, \textit{Top})$.
We first claim that $\Sh(\mathcal{C}, \textit{Top})$ has limits
and that the inclusion functor commutes with them.
Namely, given a category $\mathcal{I}$ and a functor
$i \mapsto \mathcal{G}_i$ into $\Sh(\mathcal{C}, \textit{Top})$
we simply define
$$
(\lim \mathcal{G}_i)(U) = \lim \mathcal{G}_i(U)
$$
where we take the limit in the category of toplogical spaces
(Topology, Lemma \ref{topology-lemma-limits}). This defines a sheaf
because limits commute with limits
(Categories, Lemma \ref{categories-lemma-colimits-commute})
and in particular products and equalizers (which are the
operations used in the sheaf axiom). Finally, a morphism
of presheaves from $\mathcal{F} \to \lim \mathcal{G}_i$ is
clearly the same thing as a compatible system of morphisms
$\mathcal{F} \to \mathcal{G}_i$. In other words, the object
$\lim \mathcal{G}_i$ is the limit in the category
of presheaves of topological spaces and a fortiori in the
category of sheaves of topological spaces.
Our second claim is that any morphism of presheaves
$\mathcal{F} \to \mathcal{G}$ with $\mathcal{G}$ an object of
$\Sh(\mathcal{C}, \textit{Top})$ factors through a subsheaf
$\mathcal{G}' \subset \mathcal{G}$ whose size is bounded.
Here we define the {\it size} $|\mathcal{H}|$
of a sheaf of topological spaces $\mathcal{H}$ to be the cardinal
$\sup_{U \in \Ob(\mathcal{C})} |\mathcal{H}(U)|$.
To prove our claim we let
$$
\mathcal{G}'(U) =
\left\{
\quad
s \in \mathcal{G}(U)
\quad \Big | \quad
\begin{matrix}
\text{there exists a covering }\{U_i \to U\}_{i \in I} \\
\text{such that }
s|_{U_i} \in \Im(\mathcal{F}(U_i) \to \mathcal{G}(U_i))
\end{matrix}
\quad
\right\}
$$
We endow $\mathcal{G}'(U)$ with the induced topology.
Then $\mathcal{G}'$ is a sheaf of topological spaces (details omitted)
and $\mathcal{G}' \to \mathcal{G}$ is a morphism through which
the given map $\mathcal{F} \to \mathcal{G}$ factors. Moreover,
the size of $\mathcal{G}'$ is bounded by some cardinal
$\kappa$ depending only on $\mathcal{C}$ and the presheaf $\mathcal{F}$
(hint: use that coverings in $\mathcal{C}$
form a set by our conventions). Putting everything together we see
that the assumptions of Categories, Theorem
\ref{categories-theorem-adjoint-functor}
are satisfied and we obtain sheafification as the left
adjoint of the inclusion functor from sheaves to presheaves.
Finally, let $p$ be a point of the
site $\mathcal{C}$ given by a functor $u : \mathcal{C} \to \textit{Sets}$,
see Sites, Definition \ref{sites-definition-point}.
For a topological space $M$ the presheaf defined by the rule
$$
U \mapsto \text{Map}(u(U), M) = \prod\nolimits_{x \in u(U)} M
$$
endowed with the product topology is a sheaf of topological spaces.
Hence the exact same argument as given in the proof of
Sites, Lemma \ref{sites-lemma-point-pushforward-sheaf} shows that
$\mathcal{F}_p = \mathcal{F}^\#_p$, in other words, sheafification
commutes with taking stalks at a point.
\end{remark}




\section{Conventions and notation}
\label{section-conventions}

\noindent
The conventions from now on will be similar to the conventions in
Properties of Spaces, Section \ref{spaces-properties-section-conventions}.
Thus from now on the standing assumption is that all schemes are contained
in a big fppf site $\Sch_{fppf}$. And all rings $A$ considered have the
property that $\Spec(A)$ is (isomorphic) to an object of this big site.
For topological rings $A$ we assume only that all discrete quotients have
this property (but usually we assume more, compare with
Remark \ref{remark-set-theoretic}).

\medskip\noindent
Let $S$ be a scheme and let $X$ be a ``space'' over $S$, i.e., a sheaf on
$(\Sch/S)_{fppf}$. In this chapter we will write $X \times_S X$ for the
product of $X$ with itself in the category of sheaves on $(\Sch/S)_{fppf}$
instead of $X \times X$. Moreover, if $X$ and $Y$ are ``spaces'' then
we say "let $f : X \to Y$ be a morphism" to indicate that $f$ is a
natural transformation of functors, i.e., a map of sheaves on
$(\Sch/S)_{fppf}$. Similarly, if $U$ is a scheme over $S$ and
$X$ is a ``space'' over $S$, then we say
"let $f : U \to X$ be a morphism" or
"let $g : X \to U$ be a morphism" to indicate that $f$ or $g$
is a map of sheaves $h_U \to X$ or $X \to h_U$ where $h_U$ is as in
Categories, Example \ref{categories-example-hom-functor}.





\section{Affine formal algebraic spaces}
\label{section-affine-formal-algebraic-spaces}

\noindent
In this section we introduce affine formal algebraic spaces.
These will in fact be the same as what are called affine
formal schemes in \cite{BVGD}. However, we will call
them affine formal algebraic spaces, in order to prevent confusion with
the notion of an affine formal scheme as defined in \cite{EGA}.

\medskip\noindent
Recall that a thickening of schemes is a closed
immersion which induces a surjection on underlying topological
spaces, see More on Morphisms, Definition
\ref{more-morphisms-definition-thickening}.

\begin{definition}
\label{definition-affine-formal-algebraic-space}
Let $S$ be a scheme. We say a sheaf $X$ on $(\Sch/S)_{fppf}$ is an
{\it affine formal algebraic space} if there exist
\begin{enumerate}
\item a directed partially ordered set $\Lambda$,
\item a system $(X_\lambda, f_{\lambda \mu})$ over $\Lambda$
in $(\Sch/S)_{fppf}$ where
\begin{enumerate}
\item each $X_\lambda$ is affine,
\item each $f_{\lambda \mu} : X_\lambda \to X_\mu$ is a thickening,
\end{enumerate}
\end{enumerate}
such that
$$
X \cong \colim_{\lambda \in \Lambda} X_\lambda
$$
as fppf sheaves and $X$ satisfies a set theoretic condition
(see Remark \ref{remark-set-theoretic}). A
{\it morphism of affine formal algebraic spaces}
over $S$ is a map of sheaves.
\end{definition}

\noindent
Observe that the system $(X_\lambda, f_{\lambda \mu})$ is not
part of the data. Suppose that $U$ is a quasi-compact scheme over $S$.
Since the transition maps are monomorphisms, we see that
$$
X(U) = \colim X_\lambda(U)
$$
by Sites, Lemma \ref{sites-lemma-directed-colimits-sections}.
Thus the fppf sheafification inherent in the colimit of the
definition is a Zariski sheafification which does not do
anything for quasi-compact schemes.

\begin{lemma}
\label{lemma-diagonal-affine-formal-algebraic-space}
Let $S$ be a scheme. If $X$ is an affine formal algebraic space over
$S$, then the diagonal morphism $\Delta : X \to X \times_S X$
is representable and a closed immersion.
\end{lemma}

\begin{proof}
Suppose given $U \to X$ and $V \to X$ where $U, V$ are schemes over $S$.
Let us show that $U \times_X V$ is representable. Write $X = \colim X_\lambda$
as in Definition \ref{definition-affine-formal-algebraic-space}.
The discussion above shows that Zariski locally on $U$ and $V$ the morphisms
factors through some $X_\lambda$. In this case
$U \times_X V = U \times_{X_\lambda} V$ which is a scheme.
Thus the diagonal is representable, see
Spaces, Lemma \ref{spaces-lemma-representable-diagonal}.
Given $(a, b) : W \to X \times_S X$ where $W$ is a scheme over $S$
consider the map $X \times_{\Delta, X \times_S X, (a, b)} W \to W$.
As before locally on $W$ the morphisms $a$ and $b$ map into
the affine scheme $X_\lambda$ for some $\lambda$ and then
we get the morphism
$X_\lambda
\times_{\Delta_\lambda, X_\lambda \times_S X_\lambda, (a, b)} W \to W$.
This is the base change of
$\Delta_\lambda : X_\lambda \to X_\lambda \times_S X_\lambda$
which is a closed immersion as $X_\lambda \to S$ is separated
(because $X_\lambda$ is affine).
Thus $X \to X \times_S X$ is a closed immersion.
\end{proof}

\noindent
A morphism of schemes $X \to X'$ is a thickening if it is a closed immersion
and induces a surjection on underlying sets of points, see
(More on Morphisms, Definition
\ref{more-morphisms-definition-thickening}).
Hence the property of being a thickening is preserved under arbitrary
base change and fpqc local on the target, see
Spaces, Section \ref{spaces-section-lists}.
Thus Spaces, Definition \ref{spaces-definition-relative-representable-property}
applies to ``thickening'' and we know what it means for a
representable transformation $F \to G$ of
presheaves on $(\Sch/S)_{fppf}$ to be a thickening.
We observe that this does not clash with our definition
(More on Morphisms of Spaces, Definition
\ref{spaces-more-morphisms-definition-thickening})
of thickenings in case $F$ and $G$ are algebraic spaces.

\begin{lemma}
\label{lemma-covering-by-thickenings}
Let $X_\lambda, \lambda \in \Lambda$ and $X = \colim X_\lambda$
be as in Definition \ref{definition-affine-formal-algebraic-space}.
Then $X_\lambda \to X$ is representable and a thickening.
\end{lemma}

\begin{proof}
The statement makes sense by the discussion in
Spaces, Section \ref{spaces-section-representable} and
\ref{spaces-section-representable-properties}.
By Lemma \ref{lemma-diagonal-affine-formal-algebraic-space}
the morphisms $X_\lambda \to X$ are representable.
Given $U \to X$ where $U$ is a scheme,
then the discussion following
Definition \ref{definition-affine-formal-algebraic-space}
shows that Zariski locally on $U$ the
morphism factors through some $X_\mu$ with $\lambda \leq \mu$.
In this case $U \times_X X_\lambda = U \times_{X_\mu} X_\lambda$
so that $U \times_X X_\lambda \to U$ is a base change of
the thickening $X_\lambda \to X_\mu$.
\end{proof}

\begin{lemma}
\label{lemma-factor-through-thickening}
Let $X_\lambda, \lambda \in \Lambda$ and $X = \colim X_\lambda$
be as in Definition \ref{definition-affine-formal-algebraic-space}.
If $Y$ is a quasi-compact algebraic space over $S$, then any
morphism $Y \to X$ factors through an $X_\lambda$.
\end{lemma}

\begin{proof}
Choose an affine scheme $V$ and a surjective \'etale morphism
$V \to Y$. The composition $V \to Y \to X$ factors through
$X_\lambda$ for some $\lambda$ by the discussion following
Definition \ref{definition-affine-formal-algebraic-space}.
Since $V \to Y$ is a surjection of sheaves, we conclude.
\end{proof}

\begin{lemma}
\label{lemma-characterize-affine-formal-algebraic-space}
Let $S$ be a scheme. Let $X$ be a sheaf on $(\Sch/S)_{fppf}$.
Then $X$ is an affine formal algebraic space if and only if
the following hold
\begin{enumerate}
\item any morphism $U \to X$ where $U$ is an affine scheme over $S$
factors through a morphism $T \to X$ which is representable and a
thickening with $T$ an affine scheme over $S$, and
\item a set theoretic condition as in Remark \ref{remark-set-theoretic}.
\end{enumerate}
\end{lemma}

\begin{proof}
It follows from Lemmas \ref{lemma-covering-by-thickenings} and
\ref{lemma-factor-through-thickening} that an affine formal algebraic space
satisfies (1) and (2). In order to prove the converse we may
assume $X$ is not empty.
Let $\Lambda$ be the category of representable morphisms $T \to X$ which are
thickenings where $T$ is an affine scheme over $S$. This category
is directed. Since $X$ is not empty, $\Lambda$ contains at least one
object. If $T \to X$ and $T' \to X$ are in $\Lambda$, then we can
factor $T \amalg T' \to X$ through $T'' \to X$ in $\Lambda$. Between
any two objects of $\Lambda$ there is a unique arrow or none. Thus
$\Lambda$ is a directed partially ordered set and by assumption
$X = \colim_{T \to X\text{ in }\Lambda} T$.
\end{proof}

\noindent
For a general affine formal algebraic space $X$ there is no guarantee
that $X$ has enough functions to separate points (for example).
To characterize those that do we offer the following lemma.

\begin{lemma}
\label{lemma-mcquillan-affine-formal-algebraic-space}
Let $S$ be a scheme. Let $X$ be an fppf sheaf on $(\Sch/S)_{fppf}$
which satisfies the set theoretic condition of
Remark \ref{remark-set-theoretic}.
The following are equivalent:
\begin{enumerate}
\item there exists a weakly admissible topological ring $A$ over $S$
(see Remark \ref{remark-mcquillan}) such that
$X = \colim_{I \subset A\text{ weak ideal of definition}} \Spec(A/I)$,
\item $X$ is an affine formal algebraic space and
there exists an $S$-algebra $A$ and a map $X \to \Spec(A)$
such that for a closed immersion $T \to X$ with $T$ an affine scheme
the composition $T \to \Spec(A)$ is a closed immersion,
\item $X$ is an affine formal algebraic space and
there exists an $S$-algebra $A$ and a map $X \to \Spec(A)$
such that for a closed immersion $T \to X$ with $T$ a scheme
the composition $T \to \Spec(A)$ is a closed immersion,
\item $X$ is an affine formal algebraic space and
for some choice of $X = \colim X_\lambda$ as in
Definition \ref{definition-affine-formal-algebraic-space}
the projections $\lim \Gamma(X_\lambda, \mathcal{O}_{X_\lambda})
\to \Gamma(X_\lambda, \mathcal{O}_{X_\lambda})$ are surjective,
\item $X$ is an affine formal algebraic space and for any choice
of $X = \colim X_\lambda$ as in
Definition \ref{definition-affine-formal-algebraic-space}
the projections $\lim \Gamma(X_\lambda, \mathcal{O}_{X_\lambda})
\to \Gamma(X_\lambda, \mathcal{O}_{X_\lambda})$ are surjective.
\end{enumerate}
Moreover, the weakly admissible topological ring is
$A = \lim \Gamma(X_\lambda, \mathcal{O}_{X_\lambda})$
endowed with its limit topology and the weak ideals of definition
classify exactly the morphisms $T \to X$ which are representable
and thickenings.
\end{lemma}

\begin{proof}
It is clear that (5) implies (4).

\medskip\noindent
Assume (4) for $X = \colim X_\lambda$ as in
Definition \ref{definition-affine-formal-algebraic-space}.
Set $A = \lim \Gamma(X_\lambda, \mathcal{O}_{X_\lambda})$.
Let $T \to X$ be a closed immersion with $T$ a scheme
(note that $T \to X$ is representable by
Lemma \ref{lemma-diagonal-affine-formal-algebraic-space}).
Since $X_\lambda \to X$ is a thickening, so is
$X_\lambda \times_X T \to T$. On the other hand,
$X_\lambda \times_X T \to X_\lambda$ is a closed immersion,
hence $X_\lambda \times_X T$ is affine. Hence $T$ is affine
by Limits, Proposition \ref{limits-proposition-affine}.
Then $T \to X$ factors through $X_\lambda$ for some $\lambda$
by Lemma \ref{lemma-factor-through-thickening}.
Thus $A \to \Gamma(X_\lambda, \mathcal{O}) \to \Gamma(T, \mathcal{O})$
is surjective. In this way we see that (3) holds.

\medskip\noindent
It is clear that (3) implies (2).

\medskip\noindent
Assume (2) for $A$ and $X \to \Spec(A)$. Write $X = \colim X_\lambda$
as in Definition \ref{definition-affine-formal-algebraic-space}.
Then $A_\lambda = \Gamma(X_\lambda, \mathcal{O})$ is a quotient
of $A$ by assumption (2). Hence $A^\wedge = \lim A_\lambda$
is a complete topological ring, see discussion in
More on Algebra, Section \ref{more-algebra-section-topological-ring}.
The maps $A^\wedge \to A_\lambda$ are surjective as $A \to A_\lambda$ is.
We claim that for any $\lambda$ the kernel $I_\lambda \subset A^\wedge$ of
$A^\wedge \to A_\lambda$ is a weak ideal of definition.
Namely, it is open by definition of the limit topology.
If $f \in I_\lambda$, then for any $\mu \in \Lambda$
the image of $f$ in $A_\mu$ is zero in all the residue fields
of the points of $X_\mu$. Hence it is a nilpotent element
of $A_\mu$. Hence some power $f^n \in I_\mu$. Thus $f^n \to 0$
as $n \to 0$. Thus $A^\wedge$ is weakly admissible.
Finally, suppose that $I \subset A^\wedge$ is a weak ideal
of definition. Then $I \subset A^\wedge$ is open and hence there exists
some $\lambda$ such that $I \supset I_\lambda$. Thus we obtain a morphism
$\Spec(A^\wedge/I) \to \Spec(A_\lambda) \to X$.
Then it follows that $X = \colim \Spec(A^\wedge/I)$ where now
the colimit is over all weak ideals of definition.
Thus (1) holds.

\medskip\noindent
Assume (1). In this case it is clear that $X$ is an affine formal
algebraic space. Let $X = \colim X_\lambda$ be any presentation as in
Definition \ref{definition-affine-formal-algebraic-space}.
For each $\lambda$ we can find a weak ideal of definition
$I \subset A$ such that $X_\lambda \to X$ factors through
$\Spec(A/I) \to X$, see Lemma \ref{lemma-factor-through-thickening}.
Then $X_\lambda = \Spec(A/I_\lambda)$ with $I \subset I_\lambda$.
Conversely, for any weak ideal of definition $I \subset A$
the morphism $\Spec(A/I) \to X$ factors through $X_\lambda$
for some $\lambda$, i.e., $I_\lambda \subset I$.
It follows that each $I_\lambda$ is a weak ideal of definition
and that they form a cofinal subset of the set of weak ideals
of definition. Hence $A = \lim A/I = \lim A/I_\lambda$
and we see that (5) is true and moreover that
$A = \lim \Gamma(X_\lambda, \mathcal{O}_{X_\lambda})$.
\end{proof}

\noindent
With this lemma in hand we can make the following definition.

\begin{definition}
\label{definition-types-affine-formal-algebraic-space}
Let $S$ be a scheme. Let $X$ be an affine formal algebraic space over $S$.
We say $X$ is {\it McQuillan} if $X$ satisfies the equivalent conditions
of Lemma \ref{lemma-mcquillan-affine-formal-algebraic-space}. Let $A$
be the weakly admissible topological ring associated to $X$. We say
\begin{enumerate}
\item $X$ is {\it classical} if $X$ is McQuillan and $A$ is admissible,
\item $X$ is {\it adic} if $X$ is McQuillan and $A$ is adic,
\item $X$ is {\it adic*} if $X$ is McQuillan, $A$ is adic, and $A$
has a finitely generated ideal of definition, and
\item $X$ is {\it Noetherian} if $X$ is McQuillan and $A$ is
both Noetherian and adic.
\end{enumerate}
\end{definition}

\noindent
In \cite{Fujiwara-Kato} they use the terminology ``of finite ideal type''
for the property that an adic topological ring $A$ contains a finitely
generated ideal of definition.

\begin{remark}
\label{remark-compare-with-affine-formal-schemes}
The classical affine formal algebraic spaces correspond to the
affine formal schemes considered in EGA (\cite{EGA}). To explain this
we assume our base scheme is $\Spec(\mathbf{Z})$. Let
$\mathfrak X = \text{Spf}(A)$ be an affine formal scheme.
Let $h_\mathfrak X$ be its functor of points as in
Lemma \ref{lemma-fully-faithful}.
Then $h_\mathfrak X = \colim h_{\Spec(A/I)}$ where the colimit
is over the collection of ideals of definition of the admissible
topological ring $A$. This follows from
(\ref{equation-morphisms-affine-formal-schemes})
when evaluating on affine schemes and it suffices to check
on affine schemes as both sides are fppf sheaves, see
Lemma \ref{lemma-formal-scheme-sheaf-fppf}.
Thus $h_\mathfrak X$ is an affine formal algebraic space.
In fact, it is a classical affine formal algebraic space
by Definition \ref{definition-types-affine-formal-algebraic-space}.
Thus Lemma \ref{lemma-fully-faithful} tells us
the category of affine formal schemes is equivalent to the category
of classical affine formal algebraic spaces.
\end{remark}

\noindent
Having made the connection with affine formal schemes above,
it seems natural to make the following definition.

\begin{definition}
\label{definition-affine-formal-spectrum}
Let $S$ be a scheme. Let $A$ be a (weakly) admissible topological ring over
$S$, see More on Algebra, Definition
\ref{more-algebra-definition-topological-ring} (Remark \ref{remark-mcquillan}).
The {\it formal spectrum} of $A$ is the affine formal algebraic space
$$
\text{Spf}(A) = \colim \Spec(A/I)
$$
where the colimit is over the set of (weak) ideals of definion of $A$
and taken in the category $\Sh((\Sch/S)_{fppf})$.
\end{definition}

\noindent
To be sure, in our theory there exist affine formal algebraic spaces
which are not the formal spectrum of any weakly admissible topological ring.

\begin{lemma}
\label{lemma-presentation-representable}
Let $S$ be a scheme. Let $f : X \to Y$ be a map
of presheaves on $(\Sch/S)_{fppf}$. If $X$ is an affine formal algebraic
space and $f$ is representable by algebraic spaces and locally quasi-finite,
then $f$ is representable (by schemes).
\end{lemma}

\begin{proof}
Let $T$ be a scheme over $S$ and $T \to Y$ a map. We have to show that
the algebraic space $X \times_Y T$ is a scheme. Write $X = \colim X_\lambda$
as in Definition
\ref{definition-affine-formal-algebraic-space}.
Let $W \subset X \times_Y T$
be a quasi-compact open subspace. The restriction of the projection
$X \times_Y T \to X$ to $W$ factors through $X_\lambda$ for some $\lambda$.
Then
$$
W \to X_\lambda \times_S T
$$
is a monomorphism (hence separated) and locally quasi-finite (because
$W \to X \times_Y T \to T$ is locally quasi-finite by our assumption
on $X \to Y$, see Morphisms of Spaces, Lemma
\ref{spaces-morphisms-lemma-permanence-quasi-finite}).
Hence $W$ is a scheme by
Morphisms of Spaces, Proposition
\ref{spaces-morphisms-proposition-locally-quasi-finite-separated-over-scheme}.
Thus $X \times_Y T$ is a scheme by
Properties of Spaces, Lemma \ref{spaces-properties-lemma-subscheme}.
\end{proof}






\section{Countably indexed affine formal algebraic spaces}
\label{section-countably-indexed}

\noindent
These are the affine formal algebraic spaces as in the following lemma.

\begin{lemma}
\label{lemma-countable-affine-formal-algebraic-space}
Let $S$ be a scheme. Let $X$ be an affine formal algebraic space over $S$.
The following are equivalent
\begin{enumerate}
\item there exists a system $X_1 \to X_2 \to X_3 \to \ldots$
of thickenings of affine schemes over $S$ such that $X = \colim X_n$,
\item there exists a choice $X = \colim X_\lambda$ as in
Definition \ref{definition-affine-formal-algebraic-space}
such that $\Lambda$ is countable.
\end{enumerate}
\end{lemma}

\begin{proof}
This follows from the observation that a countable directed partially
ordered set has a cofinal subset isomorphic to $(\mathbf{N}, \geq)$.
See proof of Algebra, Lemma \ref{algebra-lemma-ML-limit-nonempty}.
\end{proof}

\begin{definition}
\label{definition-countable}
Let $S$ be a scheme. Let $X$ be an affine formal algebraic space over $S$.
We say $X$ is {\it countably indexed} if the equivalent conditions of
Lemma \ref{lemma-countable-affine-formal-algebraic-space} are satisfied.
\end{definition}

\noindent
In the language of \cite{BVGD} this is expressed by saying that
$X$ is an $\aleph_0$-ind scheme.

\begin{lemma}
\label{lemma-implications-between-types}
Let $X$ be an affine formal algebraic space over a scheme $S$.
\begin{enumerate}
\item If $X$ is Noetherian, then $X$ is adic*.
\item If $X$ is adic*, then $X$ is adic.
\item If $X$ is adic, then $X$ is countably indexed.
\item If $X$ is countably indexed, then $X$ is McQuillan.
\end{enumerate}
\end{lemma}

\begin{proof}
Parts (1) and (2) are immediate from the definitions.

\medskip\noindent
Proof of (3). By definition there exists an adic topological ring $A$
such that $X = \colim \Spec(A/I)$ where the colimit is over the ideals
of definition of $A$. As $A$ is adic, there exits an ideal $I$
such that $\{I^n\}$ forms a fundamental system of neighbourhoods of $0$.
Then each $I^n$ is an ideal of definition and $X = \colim \Spec(A/I^n)$.
Thus $X$ is countably indexed.

\medskip\noindent
Proof of (4). Write $X = \lim X_n$
for some system $X_1 \to X_2 \to X_3 \to \ldots$ of thickenings of affine
schemes over $S$. Then
$$
A = \lim \Gamma(X_n, \mathcal{O}_{X_n})
$$
surjects onto each $\Gamma(X_n, \mathcal{O}_{X_n})$ because the transition
maps are surjections as the morphisms $X_n \to X_{n + 1}$ are closed
immersions.
\end{proof}

\begin{lemma}
\label{lemma-characterize-noetherian-affine}
Let $S$ be a scheme. Let $X$ be an affine formal algebraic space.
The following are equivalent
\begin{enumerate}
\item $X$ is Noetherian,
\item $X$ is adic* and for some choice of $X = \colim X_\lambda$ as in
Definition \ref{definition-affine-formal-algebraic-space}
the schemes $X_\lambda$ are Noetherian,
\item $X$ is adic* and for a closed immersion $T \to X$ with $T$
a scheme, $T$ is Noetherian.
\end{enumerate}
\end{lemma}

\begin{proof}
This follows from the fact that if $A$ is a ring complete with
respect to a finitely generated ideal $I$, then $A$ is Noetherian
if and only if $A/I$ is Noetherian, see
Algebra, Lemma \ref{algebra-lemma-completion-Noetherian}.
Details omitted.
\end{proof}





\section{Formal algebraic spaces}
\label{section-formal-algebraic-spaces}

\noindent
We take a break from our habit of introducing new concepts first
for rings, then for schemes, and then for algebraic spaces, by
introducing formal algebraic spaces without first introducing
formal schemes. The general idea will be that a formal algebraic space
is a sheaf in the fppf topology which \'etale locally is an
affine formal scheme in the sense of \cite{BVGD}.
Related material can be found in \cite{Yasuda}.

\medskip\noindent
In the definition of a formal algebraic space we are going to
borrow some terminology from
Bootstrap, Sections
\ref{bootstrap-section-morphism-representable-by-spaces} and
\ref{bootstrap-section-representable-by-spaces-properties}.

\begin{definition}
\label{definition-formal-algebraic-space}
Let $S$ be a scheme. We say a sheaf $X$ on $(\Sch/S)_{fppf}$ is a
{\it formal algebraic space} if there exist a family of maps
$\{X_i \to X\}_{i \in I}$ of sheaves such that
\begin{enumerate}
\item $X_i$ is an affine formal algebraic space,
\item $X_i \to X$ is representable by algebraic spaces and \'etale,
\item $\coprod X_i \to X$ is surjective as a map of sheaves
\end{enumerate}
and $X$ satisfies a set theoretic condition
(see Remark \ref{remark-set-theoretic}). A
{\it morphism of formal algebraic spaces}
over $S$ is a map of sheaves.
\end{definition}

\noindent
Discussion. Sanity check: an affine formal algebraic space is
a formal algebraic space. In the situation of the definition
the morphisms $X_i \to X$ are representable (by schemes), see
Lemma \ref{lemma-presentation-representable}.
By Bootstrap, Lemma
\ref{bootstrap-lemma-surjective-flat-locally-finite-presentation}
we could instead of asking $\coprod X_i \to X$
to be surjective as a map of sheaves, require that it be
surjective (which makes sense because it is representable).

\medskip\noindent
Our notion of a formal algebraic space is {\bf very general}.
In fact, even affine formal algebraic spaces as defined above
are very nasty objects. However, they do have an underlying
reduced algebraic space as the following lemma demonstrates.

\begin{lemma}
\label{lemma-reduction-formal-algebraic-space}
Let $S$ be a scheme. Let $X$ be a formal algebraic space over $S$.
There exists a reduced algebraic space $X_{red}$ and a representable
morphism $X_{red} \to X$ which is a thickening. A morphism $U \to X$
with $U$ a reduced algebraic space factors uniquely through $X_{red}$.
\end{lemma}

\begin{proof}
First assume that $X$ is an affine formal algebraic space.
Say $X = \colim X_\lambda$ as in
Definition \ref{definition-affine-formal-algebraic-space}.
Since the transition morphisms are thickenings, the affine
schemes $X_\lambda$ all have isomorphic reductions $X_{red}$.
The morphism $X_{red} \to X$ is representable and a thickening
by Lemma \ref{lemma-covering-by-thickenings} and the fact that
compositions of thickenings are thickenings. We omit the
verification of the universal
property (use Schemes, Definition
\ref{schemes-definition-reduced-induced-scheme},
Schemes, Lemma \ref{schemes-lemma-map-into-reduction},
Properties of Spaces, Definition
\ref{spaces-properties-definition-reduced-induced-space}, and
Properties of Spaces, Lemma \ref{spaces-properties-lemma-map-into-reduction}).

\medskip\noindent
Let $X$ and $\{X_i \to X\}_{i \in I}$ be as in
Definition \ref{definition-formal-algebraic-space}.
For each $i$ let $X_{i, red} \to X_i$ be the reduction as
constructed above. For $i, j \in I$ the projection
$X_{i, red} \times_X X_j \to X_{i, red}$ is an \'etale (by assumption)
morphism of schemes (by Lemma \ref{lemma-presentation-representable}).
Hence $X_{i, red} \times_X X_j$ is reduced (see
Descent, Lemma \ref{descent-lemma-reduced-local-smooth}).
Thus the projection $X_{i, red} \times_X X_j \to X_j$ factors
through $X_{j, red}$ by the universal property. We conclude that
$$
R_{ij} = X_{i, red} \times_X X_j = X_{i, red} \times_X X_{j, red} =
X_i \times_X X_{j, red}
$$
because the morphisms $X_{i, red} \to X_i$ are injections of sheaves.
Set $U = \coprod X_{i, red}$, set
$R = \coprod R_{ij}$, and denote $s, t : R \to U$ the two
projections. As a sheaf $R = U \times_X U$ and $s$ and $t$
are \'etale. Then $(t, s) : R \to U$ defines an \'etale equivalence
relation by our observations above. Thus $X_{red} = U/R$ is an
algebraic space by Spaces, Theorem \ref{spaces-theorem-presentation}.
By construction the diagram
$$
\xymatrix{
\coprod X_{i, red} \ar[r] \ar[d] & \coprod X_i \ar[d] \\
X_{red} \ar[r] & X
}
$$
is cartesian. Since the right vertical arrow is \'etale surjective
and the top horizontal arrow is representable and a thickening
we conclude that $X_{red} \to X$ is representable by
Bootstrap, Lemma \ref{bootstrap-lemma-after-fppf-sep-lqf}
(to verify the assumptions of the lemma use that a surjective
\'etale morphism is surjective, flat, and locally of finite
presentation and use that thickenings are separated and locally quasi-finite).
Then we can use Spaces, Lemma
\ref{spaces-lemma-descent-representable-transformations-property}
to conclude that $X_{red} \to X$ is a thickening
(use that being a thickening is equivalent to being
a surjective closed immersion).

\medskip\noindent
Finally, suppose that $U \to X$ is a morphism with
$U$ a reduced algebraic space over $S$. Then each $X_i \times_X U$
is \'etale over $U$ and therefore reduced (by our defintion of
reduced algebraic spaces in Properties of Spaces, Section
\ref{spaces-properties-section-types-properties}).
Then $X_i \times_X U \to X_i$ factors through $X_{i, red}$.
Hence $U \to X$ factors through $X_{red}$ because
$\{X_i \times_X U \to U\}$ is an \'etale covering.
\end{proof}

\begin{lemma}
\label{lemma-diagonal-formal-algebraic-space}
Let $S$ be a scheme. If $X$ is a formal algebraic space over
$S$, then the diagonal morphism $\Delta : X \to X \times_S X$
is representable, a monomorphism, locally quasi-finite,
locally of finite type, and separated.
\end{lemma}

\begin{proof}
Suppose given $U \to X$ and $V \to X$ with $U, V$ schemes over $S$.
Then $U \times_X V$ is a sheaf. Choose $\{X_i \to X\}$ as in
Definition \ref{definition-formal-algebraic-space}.
For every $i$ the morphism
$$
(U \times_X X_i) \times_{X_i} (V \times_X X_i)
= (U \times_X V) \times_X X_i \to U \times_X V
$$
is representable and \'etale as a base change of $X_i \to X$
and its source is a scheme (use
Lemmas \ref{lemma-diagonal-affine-formal-algebraic-space} and
\ref{lemma-presentation-representable}). These maps are jointly surjective
hence $U \times_X V$ is an algebraic space by
Bootstrap, Theorem \ref{bootstrap-theorem-final-bootstrap}.
The morphism $U \times_X V \to U \times_S V$ is a monomorphism.
It is also locally quasi-finite, because on precomposing with
the morphism displayed above we obtain the composition
$$
(U \times_X X_i) \times_{X_i} (V \times_X X_i)
\to (U \times_X X_i) \times_S (V \times_X X_i)
\to U \times_S V
$$
which is locally quasi-finite as a composition of a closed
immersion (Lemma \ref{lemma-diagonal-affine-formal-algebraic-space})
and an \'etale morphism, see
Descent on Spaces, Lemma
\ref{spaces-descent-lemma-locally-quasi-finite-etale-local-source}.
Hence we conclude that $U \times_X V$ is a scheme by
Morphisms of Spaces, Proposition
\ref{spaces-morphisms-proposition-locally-quasi-finite-separated-over-scheme}.
Thus $\Delta$ is representable, see
Spaces, Lemma \ref{spaces-lemma-representable-diagonal}.

\medskip\noindent
Let $W \to X \times_S X$ be a morphism where $W$ is a scheme over $S$.
For each $i$ consider the diagram
$$
\xymatrix{
W \times_{(X \times_S X)} X_i \ar[d] \ar[r] &
W \times_{(X \times_S X)} (X_i \times_S X_i) \ar[d] \\
W \times_{(X \times_S X)} X \ar[r] & W
}
$$
The vertical arrows are \'etale because $X_i \to X$ is \'etale.
The top horizontal arrow is a closed immersion by
Lemma \ref{lemma-diagonal-affine-formal-algebraic-space}.
Because the maps $X_i \to X$ are jointly surjective, this certainly
implies that the lower horizontal arrow is locally of finite type, see
Morphisms of Spaces, Lemma \ref{spaces-morphisms-lemma-finite-type-local}.
Since it is clearly a monomorphism, we conclude that
it is locally quasi-finite (Morphisms of Spaces, Lemma
\ref{spaces-morphisms-lemma-monomorphism-loc-finite-type-loc-quasi-finite}).
It is separated because any monomorphism is separated
(Morphisms of Spaces, Lemma
\ref{spaces-morphisms-lemma-monomorphism-separated}). This finishes
the proof.
\end{proof}

\begin{lemma}
\label{lemma-space-to-formal-space}
Let $S$ be a scheme. Let $f : X \to Y$ be a morphism from an
algebraic space over $S$ to a formal algebraic space over $S$.
Then $f$ is representable by algebraic spaces.
\end{lemma}

\begin{proof}
Let $Z \to Y$ be a morphism where $Z$ is a scheme over $S$.
We have to show that $X \times_Y Z$ is an algebraic space.
Choose a scheme $U$ and a surjective \'etale morphism $U \to X$.
Then $U \times_Y Z \to X \times_Y Z$ is representabel surjective \'etale
(Spaces, Lemma
\ref{spaces-lemma-base-change-representable-transformations-property})
and $U \times_Y Z$ is a scheme by
Lemma \ref{lemma-diagonal-formal-algebraic-space}.
Hence the result by
Bootstrap, Theorem \ref{bootstrap-theorem-final-bootstrap}.
\end{proof}

\begin{remark}
\label{remark-compare-with-formal-schemes}
Modulo set theoretic issues the category of formal schemes \`a la EGA
(see Section \ref{section-formal-schemes-EGA}) is equivalent to a full
subcategory of the category of formal algebraic spaces. To explain this
we assume our base scheme is $\Spec(\mathbf{Z})$. By
Lemma \ref{lemma-formal-scheme-sheaf-fppf} the functor of points
$h_\mathfrak X$ associated to a formal scheme $\mathfrak X$ is a sheaf
in the fppf topology. By Lemma \ref{lemma-fully-faithful}
the assignment $\mathfrak X \mapsto h_\mathfrak X$ is a fully faithful
embedding of the category of formal schemes into the category of
fppf sheaves. Given a formal scheme $\mathfrak X$ we choose an open covering
$\mathfrak X = \bigcup \mathfrak X_i$ with $\mathfrak X_i$
affine formal schemes. Then $h_{\mathfrak X_i}$
is an affine formal algebraic space by
Remark \ref{remark-compare-with-affine-formal-schemes}.
The morphisms $h_{\mathfrak X_i} \to h_\mathfrak X$ are representable
and open immersions. Thus $\{h_{\mathfrak X_i} \to h_\mathfrak X\}$
is a family as in Definition \ref{definition-formal-algebraic-space}
and we see that $h_\mathfrak X$ is a formal algebraic space.
\end{remark}

\begin{remark}
\label{remark-set-theoretic}
Let $S$ be a scheme and let $(\Sch/S)_{fppf}$ be a big fppf site as
in Topologies, Definition \ref{topologies-definition-big-small-fppf}.
As our set theoretic condition on $X$ in
Definitions \ref{definition-affine-formal-algebraic-space} and
\ref{definition-formal-algebraic-space} we take:
there exist objects $U, R$ of $(\Sch/S)_{fppf}$, a
morphism $U \to X$ which is a surjection of fppf sheaves, and
a morphism $R \to U \times_X U$ which is a surjection of fppf sheaves.
In other words, we require our sheaf to be a coequalizer of
two maps between representable sheaves.
Here are some observations which imply this notion behaves
reasonably well:
\begin{enumerate}
\item Suppose $X = \colim_{\lambda \in \Lambda} X_\lambda$
and the system satisfies conditions (1) and (2) of
Definition \ref{definition-affine-formal-algebraic-space}. Then
$U = \coprod_{\lambda \in \Lambda} X_\lambda \to X$ is a surjection
of fppf sheaves. Moreover, $U \times_X U$ is a closed subscheme
of $U \times_S U$ by Lemma \ref{lemma-diagonal-affine-formal-algebraic-space}.
Hence if $U$ is representable by an object of $(\Sch/S)_{fppf}$
then $U \times_S U$ is too (see Sets, Lemma \ref{sets-lemma-what-is-in-it})
and the set theoretic condition is satisfied. This is always the case
if $\Lambda$ is countable, see Sets, Lemma \ref{sets-lemma-what-is-in-it}.
\item Sanity check. Let $\{X_i \to X\}_{i \in I}$ be as in
Definition \ref{definition-formal-algebraic-space}
(with the set theoretic condition as formulated above)
and assume that each $X_i$ is actually an affine scheme.
Then $X$ is an algebraic space. Namely, if we choose a larger
big fppf site $(\Sch'/S)_{fppf}$ such that $U' = \coprod X_i$
and $R' = \coprod X_i \times_X X_j$ are representable by objects
in it, then $X' = U'/R'$ will be an object of the category
of algebraic spaces for this choice. Then an application of
Spaces, Lemma \ref{spaces-lemma-fully-faithful} shows that
$X$ is an algebraic space for $(\Sch/S)_{fppf}$.
\item Let $\{X_i \to X\}_{i \in I}$ be a family of maps of sheaves
satisfying conditions (1), (2), (3) of
Definition \ref{definition-formal-algebraic-space}.
For each $i$ we can pick $U_i \in \Ob((\Sch/S)_{fppf})$
and $U_i \to X_i$ which is a surjection of sheaves.
Thus if $I$ is not too large (for example countable) then
$U = \coprod U_i \to X$ is a surjection of sheaves and
$U$ is representable by an object of $(\Sch/S)_{fppf}$.
To get $R \in \Ob((\Sch/S)_{fppf})$ surjecting onto $U \times_X U$
it suffices to assume the diagonal $\Delta : X \to X \times_S X$ is not
too wild, for example this always works if the diagonal of $X$ is
quasi-compact, i.e., $X$ is quasi-separated.
\end{enumerate}
\end{remark}






\section{Colimits of algebraic spaces along thickenings}
\label{section-global-colimits}

\noindent
A special type of formal algebraic space is one which can globally
be written as a cofiltered colimit of algebraic spaces along
thickenings as in the following lemma. We will see later
(in Section \ref{section-quasi-compact-quasi-separated})
that any quasi-compact and quasi-separated formal algebraic space
is such a global colimit.

\begin{lemma}
\label{lemma-colimit-is-formal}
Let $S$ be a scheme. Suppose given a directed partially ordered set
$\Lambda$ and a system of algebraic spaces $(X_\lambda, f_{\lambda \mu})$
over $\Lambda$ where each $f_{\lambda \mu} : X_\lambda \to X_\mu$ is a
thickening. Then $X = \colim_{\lambda \in \Lambda} X_\lambda$
is a formal algebraic space over $S$.
\end{lemma}

\begin{proof}
Since we take the colimit in the category of fppf sheaves, we
see that $X$ is a sheaf. Choose and fix $\lambda \in \Lambda$. Choose an
\'etale covering $\{X_{i, \lambda} \to X_\lambda\}$ where $X_i$ is an affine
scheme over $S$, see Properties of Spaces, Lemma
\ref{spaces-properties-lemma-cover-by-union-affines}.
For each $\mu \geq \lambda$ there exists a cartesian diagram
$$
\xymatrix{
X_{i, \lambda} \ar[r] \ar[d] & X_{i, \mu} \ar[d] \\
X_\lambda \ar[r] & X_\mu
}
$$
with \'etale vertical arrows, see
More on Morphisms of Spaces, Theorem
\ref{spaces-more-morphisms-theorem-topological-invariance}
(this also uses that a thickening is a surjective closed immersion which
satisfies the conditions of the theorem). Moreover, these diagrams are
unique up to unique isomorphism and hence
$X_{i, \mu} = X_\mu \times_{X_{\mu'}} X_{i, \mu'}$ for
$\mu' \geq \mu$. The morphisms $X_{i, \mu} \to X_{i, \mu'}$
is a thickening as a base change of a thickening. Each $X_{i, \mu}$
is an affine scheme by Limits of Spaces, Proposition
\ref{spaces-limits-proposition-affine} and the fact that
$X_{i, \lambda}$ is affine.
Set $X_i = \colim_{\mu \geq \lambda} X_{i, \mu}$. Then $X_i$ is
an affine formal algebraic space. The morphism $X_i \to X$
is \'etale because given an affine scheme $U$ any $U \to X$
factors through $X_\mu$ for some $\mu \geq \lambda$ (details omitted).
In this way we see that $X$ is a formal algebraic space.
\end{proof}

\noindent
Let $S$ be a scheme. Let $X$ be a formal algebraic space over $S$.
How does one prove or check that $X$ is a global colimit as in
Lemma \ref{lemma-colimit-is-formal}? To do this we look for maps
$i : Z \to X$ where $Z$ is an algebraic space over $S$ and $i$ is
surjective and a closed immersion, in other words, $i$ is a thickening.
This makes sense as $i$ is representable by algebraic spaces
(Lemma \ref{lemma-space-to-formal-space}) and we can use
Bootstrap, Definition \ref{bootstrap-definition-property-transformation}
as before.

\begin{remark}[Weak ideals of definition]
\label{remark-weak-ideals-of-definition}
Let $\mathfrak X$ be a formal scheme in the sense of McQuillan, see
Remark \ref{remark-mcquillan}. An {\it weak ideal of definition}
for $\mathfrak X$ is an ideal sheaf
$\mathcal{I} \subset \mathcal{O}_\mathfrak X$ such that
for all $\mathfrak U \subset \mathfrak X$ affine formal open subscheme
the ideal
$\mathcal{I}(\mathfrak U) \subset \mathcal{O}_\mathfrak X(\mathfrak U)$
is a weak ideal of definition of the weakly admissible topological ring
$\mathcal{O}_\mathfrak X(\mathfrak U)$.
It suffices to check the condition on the members of an affine open covering.
There is a one-to-one correspondence
$$
\{\text{weak ideals of definition for }\mathfrak X\}
\leftrightarrow
\{\text{thickenings }i : Z \to h_\mathfrak X\text{ as above}\}
$$
This correspondence associates to $\mathcal{I}$ the scheme
$Z = (\mathfrak X, \mathcal{O}_\mathfrak X/\mathcal{I})$
together with the obvious morphism to $\mathfrak X$.
A {\it fundamental system of weak ideals of definiton}
is a collection of weak ideals of definition
$\mathcal{I}_\lambda$ such that on every affine open
formal subscheme $\mathfrak U \subset \mathfrak X$ the
ideals
$$
I_\lambda = \mathcal{I}_\lambda(\mathfrak U) \subset
A = \Gamma(\mathfrak U, \mathcal{O}_\mathfrak X)
$$
form a fundamental system of weak ideals of definition of the
weakly admissible topological ring $A$. It suffices to check
on the members of an affine open covering. We conclude that
the formal algebraic space $h_\mathfrak X$ associated to
the McQuillan formal scheme $\mathfrak X$ is a colimit of schemes as
in Lemma \ref{lemma-colimit-is-formal} if and only if
there exists a fundamental system of weak ideals of definition
for $\mathfrak X$.
\end{remark}

\begin{remark}[Ideals of definition]
\label{remark-ideals-of-definition}
Let $\mathfrak X$ be a formal scheme \`a la EGA.
An {\it ideal of definition} for $\mathfrak X$ is an ideal sheaf
$\mathcal{I} \subset \mathcal{O}_\mathfrak X$ such that
for all $\mathfrak U \subset \mathfrak X$ affine formal open subscheme
the ideal
$\mathcal{I}(\mathfrak U) \subset \mathcal{O}_\mathfrak X(\mathfrak U)$
is an ideal of definition of the admissible topological ring
$\mathcal{O}_\mathfrak X(\mathfrak U)$.
It suffices to check the condition on the members of an affine open covering.
We do {\bf not} get the same correspondence between ideals of definition
and thickenings $Z \to \mathfrak X$ as in
Remark \ref{remark-weak-ideals-of-definition}.
A {\it fundamental system of ideals of definiton}
is a collection of ideals of definition
$\mathcal{I}_\lambda$ such that on every affine open
formal subscheme $\mathfrak U \subset \mathfrak X$ the
ideals
$$
I_\lambda = \mathcal{I}_\lambda(\mathfrak U) \subset
A = \Gamma(\mathfrak U, \mathcal{O}_\mathfrak X)
$$
form a fundamental system of ideals of definition of the
admissible topological ring $A$. It suffices to check
on the members of an affine open covering. Suppose that $\mathfrak X$
is quasi-compact and that $\{\mathcal{I}_\lambda\}_{\lambda \in \Lambda}$
is a fundamental system of weak ideals of definition.
If $A$ is an admissible topological ring $A$ then all
sufficiently small open ideals are ideals of definition
(namely any open ideal contained in an ideal of definition
is an ideal of definition). Thus since we only need to check
on the finitely many members of an affine open covering
we see that $\mathcal{I}_\lambda$ is an ideal of definition
for $\lambda$ sufficiently large. Using the discussion in
Remark \ref{remark-weak-ideals-of-definition} we conclude that
the formal algebraic space $h_\mathfrak X$ associated to
the quasi-compact formal scheme $\mathfrak X$ \`a la EGA
is a colimit of schemes as in Lemma \ref{lemma-colimit-is-formal}
if and only if there exists a fundamental system of ideals of definition
for $\mathfrak X$.
\end{remark}






\section{Completion along a closed subset}
\label{section-completion}

\noindent
Our notion of a formal algebraic space is well adapted to taking the
completion along a closed.

\begin{lemma}
\label{lemma-completion-affine-is-affine-formal-algebraic-space}
Let $S$ be a scheme. Let $X$ be an affine scheme over $S$.
Let $T \subset |X|$ be a closed subset. Then the functor
$$
(\Sch/S)_{fppf} \longrightarrow \textit{Sets},\quad
U \longmapsto \{f : U \to X \mid f(|U|) \subset T\}
$$
is a McQuillan affine formal algebraic space.
\end{lemma}

\begin{proof}
Say $X = \Spec(A)$ and $T$ corresponds to the radical ideal $I \subset A$.
Let $U = \Spec(B)$ be an affine scheme over $S$ and let
$f : U \to X$ be an element of $F(U)$. Then $f$ corresponds to a
ring map $\varphi : A \to B$ such that every prime of $B$ contains
$\varphi(I) B$. Thus every element of $\varphi(I)$ is nilpotent in $B$, see
Algebra, Lemma \ref{algebra-lemma-Zariski-topology}.
Setting $J = \Ker(\varphi)$ we conclude that $I/J$ is a locally nilpotent
ideal in $A/J$. Equivalently, $V(J) = V(I) = T$.
In other words, the functor of the lemma equals
$\colim \Spec(A/J)$ where the colimit is over the
collection of ideals $J$ with $V(J) = T$.
Thus our functor is an affine formal algebraic space. It is McQuillan
(Definition \ref{definition-types-affine-formal-algebraic-space})
because the maps $A \to A/J$ are surjective
and hence $A^\wedge = \lim A/J \to A/J$ is surjective, see
Lemma \ref{lemma-mcquillan-affine-formal-algebraic-space}.
\end{proof}

\begin{lemma}
\label{lemma-completion-is-formal-algebraic-space}
Let $S$ be a scheme. Let $X$ be an algebraic space over $S$.
Let $T \subset |X|$ be a closed subset. Then the functor
$$
(\Sch/S)_{fppf} \longrightarrow \textit{Sets},\quad
U \longmapsto \{f : U \to X \mid f(|U|) \subset T\}
$$
is a formal algebraic space.
\end{lemma}

\begin{proof}
Denote $F$ the functor. Let $\{U_i \to U\}$ be an fppf covering.
Then $\coprod |U_i| \to |U|$ is surjective. Since $X$ is an fppf
sheaf, it follows that $F$ is an fppf sheaf.

\medskip\noindent
Let $\{g_i : X_i \to X\}$ be an \'etale covering such that $X_i$ is affine
for all $i$, see Properties of Spaces, Lemma
\ref{spaces-properties-lemma-cover-by-union-affines}.
The morphisms $F \times_X X_i \to F$ are \'etale
(see Spaces, Lemma
\ref{spaces-lemma-base-change-representable-transformations-property})
and the map $\coprod F \times_X X_i \to F$ is a surjection of sheaves.
Thus it suffices to prove that $F \times_X X_i$ is an affine formal
algebraic space. A $U$-valued point of $F \times_X X_i$ is a
morphism $U \to X_i$ whose image is contained in the closed subset
$g_i^{-1}(T) \subset |X_i|$. Thus this follows from
Lemma \ref{lemma-completion-affine-is-affine-formal-algebraic-space}.
\end{proof}




\section{Fibre products}
\label{section-fibre-products}

\noindent
Obligatory section about fibre products of formal algebraic spaces.

\begin{lemma}
\label{lemma-etale-covering-by-formal-algebraic-spaces}
Let $S$ be a scheme. Let $\{X_i \to X\}_{i \in I}$ be a family of maps
of sheaves on $(\Sch/S)_{fppf}$. Assume (a) $X_i$ is a
formal algebraic space over $S$, (b) $X_i \to X$ is representable
by algebraic spaces and \'etale, and (c) $\coprod X_i \to X$
is a surjection of sheaves. Then $X$ is a formal algebraic space
over $S$.
\end{lemma}

\begin{proof}
For each $i$ pick $\{X_{ij} \to X_i\}_{j \in J_i}$ as in
Definition \ref{definition-formal-algebraic-space}.
Then $\{X_{ij} \to X\}_{i \in I, j \in J_i}$ is a family
as in Definition \ref{definition-formal-algebraic-space}
for $X$.
\end{proof}

\begin{lemma}
\label{lemma-fibre-products-general}
Let $S$ be a scheme. Let $X, Y$ be formal algebraic spaces over $S$
and let $Z$ be a sheaf whose diagonal is representable by
algebraic spaces. Then $X \times_Z Y$ is a formal algebraic space.
\end{lemma}

\begin{proof}
Choose $\{X_i \to X\}$ and $\{Y_j \to Y\}$ as in
Definition \ref{definition-formal-algebraic-space}.
Then $\{X_i \times_Z Y_j \to X \times_Z Y\}$ is a family
of maps which are representable by algebraic spaces and \'etale.
Thus Lemma \ref{lemma-etale-covering-by-formal-algebraic-spaces}
tells us it suffices to show that $X \times_Z Y$ is a formal
algebraic space when $X$ and $Y$ are affine formal algebraic spaces.

\medskip\noindent
Assume $X$ and $Y$ are affine formal algebraic spaces.
Write $X = \colim X_\lambda$ and $Y = \colim Y_\mu$ as
in Definition \ref{definition-affine-formal-algebraic-space}.
Then $X \times_Z Y = \colim X_\lambda \times_Z Y_\mu$.
Each $X_\lambda \times_Z Y_\mu$ is an algebraic space.
For $\lambda \leq \lambda'$ and $\mu \leq \mu'$ the morphism
$$
X_\lambda \times_Z Y_\mu \to
X_\lambda \times_Z Y_{\mu'} \to
X_{\lambda'} \times_Z Y_{\mu'}
$$
is a thickening as a composition of base changes of thickenings.
Thus we conclude by applying Lemma \ref{lemma-colimit-is-formal}.
\end{proof}

\begin{lemma}
\label{lemma-fibre-products}
Let $S$ be a scheme. The category of formal algebraic spaces over $S$
has fibre products.
\end{lemma}

\begin{proof}
Special case of Lemma \ref{lemma-fibre-products-general}
because formal algebraic spaces have representable diagonals, see
Lemma \ref{lemma-diagonal-formal-algebraic-space}.
\end{proof}







\section{Separation axioms for formal algebraic spaces}
\label{section-separation}

\noindent
This section is about ``absolute'' separation conditions on formal algebraic
spaces. We will discuss separation conditions for morphisms of formal\
algebraic spaces later.

\begin{lemma}
\label{lemma-characterize-quasi-separated}
Let $S$ be a scheme. Let $X$ be a formal algebraic space over $S$.
The following are equivalent
\begin{enumerate}
\item the reduction of $X$
(Lemma \ref{lemma-reduction-formal-algebraic-space}) is a
quasi-separated algebraic space,
\item for $U \to X$, $V \to X$ with $U$, $V$ quasi-compact schemes
the fibre product $U \times_X V$ is quasi-compact,
\item for $U \to X$, $V \to X$ with $U$, $V$ affine
the fibre product $U \times_X V$ is quasi-compact.
\end{enumerate}
\end{lemma}

\begin{proof}
Observe that $U \times_X V$ is a scheme by
Lemma \ref{lemma-diagonal-formal-algebraic-space}.
Let $U_{red}, V_{red}, X_{red}$ be the reduction of $U, V, X$.
Then
$$
U_{red} \times_{X_{red}} V_{red} = U_{red} \times_X V_{red} \to U \times_X V
$$
is a thickening of schemes. From this the equivalence of (1) and (2)
is clear, keeping in mind the analgous lemma for algebraic spaces, see
Properties of Spaces, Lemma
\ref{spaces-properties-lemma-characterize-quasi-separated}.
We omit the proof of the equivalence of (2) and (3).
\end{proof}

\begin{lemma}
\label{lemma-characterize-separated}
Let $S$ be a scheme. Let $X$ be a formal algebraic space over $S$.
The following are equivalent
\begin{enumerate}
\item the reduction of $X$
(Lemma \ref{lemma-reduction-formal-algebraic-space}) is a separated
algebraic space,
\item for $U \to X$, $V \to X$ with $U$, $V$ affine
the fibre product $U \times_X V$ is affine and
$$
\mathcal{O}(U) \otimes_\mathbf{Z} \mathcal{O}(V)
\longrightarrow
\mathcal{O}(U \times_X V)
$$
is surjective.
\end{enumerate}
\end{lemma}

\begin{proof}
If (2) holds, then $X_{red}$ is a separated algebraic space
by applying Properties of Spaces, Lemma
\ref{spaces-properties-lemma-characterize-quasi-separated}
to morphisms $U \to X_{red}$ and $V \to X_{red}$
with $U, V$ affine and using that $U \times_{X_{red}} V = U \times_X V$.

\medskip\noindent
Assume (1). Let $U \to X$ and $V \to X$ be as in (2).
Observe that $U \times_X V$ is a scheme by
Lemma \ref{lemma-diagonal-formal-algebraic-space}.
Let $U_{red}, V_{red}, X_{red}$ be the reduction of $U, V, X$.
Then
$$
U_{red} \times_{X_{red}} V_{red} = U_{red} \times_X V_{red} \to U \times_X V
$$
is a thickening of schemes. It follows that
$(U \times_X V)_{red} = (U_{red} \times_{X_{red}} V_{red})_{red}$.
In particular, we see that $(U \times_X V)_{red}$ is an affine scheme
and that
$$
\mathcal{O}(U) \otimes_\mathbf{Z} \mathcal{O}(V)
\longrightarrow
\mathcal{O}((U \times_X V)_{red})
$$
is surjective, see Properties of Spaces, Lemma
\ref{spaces-properties-lemma-characterize-quasi-separated}.
Then $U \times_X V$ is affine by
Limits of Spaces, Proposition \ref{spaces-limits-proposition-affine}.
On the other hand, the morphism $U \times_X V \to U \times V$
of affine schemes is the composition
$$
U \times_X V = X \times_{(X \times_S X)} (U \times_S V)
\to U \times_S V \to U \times V
$$
The first morphism is a monomorphism and locally of finite type
(Lemma \ref{lemma-diagonal-formal-algebraic-space}).
The second morphism is an immersion
(Schemes, Lemma \ref{schemes-lemma-fibre-product-after-map}).
Hence the composition is a monomorphism which is locally of finite type.
On the other hand, the composition is integral as the map on
underlying reduced affine schemes is a closed immersion
by the above and hence universally closed (use
Morphisms, Lemma \ref{morphisms-lemma-integral-universally-closed}).
Thus the ring map
$$
\mathcal{O}(U) \otimes_\mathbf{Z} \mathcal{O}(V)
\longrightarrow
\mathcal{O}(U \times_X V)
$$
is an epimorphism which is integral of finite type
hence finite hence surjective (use
Morphisms, Lemma \ref{morphisms-lemma-finite-integral}
and
Algebra, Lemma \ref{algebra-lemma-finite-epimorphism-surjective}).
\end{proof}

\begin{definition}
\label{definition-separated}
Let $S$ be a scheme. Let $X$ be a formal algebraic space over $S$.
We say
\begin{enumerate}
\item $X$ is {\it quasi-separated} if the equivalent conditions of
Lemma \ref{lemma-characterize-quasi-separated} are satisfied.
\item $X$ is {\it separated} if the equivalent conditions of
Lemma \ref{lemma-characterize-separated} are satisfied.
\end{enumerate}
\end{definition}






\section{Quasi-compact formal algebraic spaces}
\label{section-quasi-compact}

\noindent
Here is the characterization of quasi-compact formal algebraic
spaces.

\begin{lemma}
\label{lemma-characterize-quasi-compact}
Let $S$ be a scheme. Let $X$ be a formal algebraic space over $S$.
The following are equivalent
\begin{enumerate}
\item the reduction of $X$
(Lemma \ref{lemma-reduction-formal-algebraic-space}) is a quasi-compact
algebraic space,
\item we can find $\{X_i \to X\}_{i \in I}$ as in
Definition \ref{definition-formal-algebraic-space} with $I$ finite,
\item there exists a morphism $Y \to X$ representable by algebraic
spaces which is \'etale and surjective and where
$Y$ is an affine formal algebraic space.
\end{enumerate}
\end{lemma}

\begin{proof}
Omitted.
\end{proof}

\begin{definition}
\label{definition-quasi-compact}
Let $S$ be a scheme. Let $X$ be a formal algebraic space over $S$.
We say $X$ is {\it quasi-compact} if the equivalent conditions of
Lemma \ref{lemma-characterize-quasi-compact} are satisfied.
\end{definition}

\begin{lemma}
\label{lemma-characterize-quasi-compact-morphism}
Let $S$ be a scheme. Let $f : X \to Y$ be a morphism of formal algebraic
spaces over $S$. The following are equivalent
\begin{enumerate}
\item the induced map $f_{red} : X_{red} \to Y_{red}$ between reductions
(Lemma \ref{lemma-reduction-formal-algebraic-space}) is a quasi-compact
morphism of algebraic spaces,
\item for every quasi-compact scheme $T$ and morphism $T \to Y$
the fibre product $X \times_Y T$ is a quasi-compact formal
algebraic space, and
\item for every affine scheme $T$ and morphism $T \to Y$
the fibre product $X \times_Y T$ is a quasi-compact formal
algebraic space.
\end{enumerate}
\end{lemma}

\begin{proof}
Omitted.
\end{proof}

\begin{definition}
\label{definition-quasi-compact-morphism}
Let $S$ be a scheme. Let $f : X \to Y$ be a morphism of
formal algebraic spaces over $S$.
We say $f$ is {\it quasi-compact} if the equivalent conditions of
Lemma \ref{lemma-characterize-quasi-compact-morphism} are satisfied.
\end{definition}







\section{Quasi-compact and quasi-separated formal algebraic spaces}
\label{section-quasi-compact-quasi-separated}

\noindent
The following result is due to Yasuda, see
\cite[Proposition 3.32]{Yasuda}.

\begin{lemma}
\label{lemma-structure-quasi-compact-quasi-separated}
\begin{reference}
\cite[Proposition 3.32]{Yasuda}
\end{reference}
Let $S$ be a scheme. Let $X$ be a quasi-compact and quasi-separated
formal algebraic space over $S$. Then $X = \colim X_\lambda$
for a system of algebraic spaces $(X_\lambda, f_{\lambda \mu})$
over a directed partially ordered set $\Lambda$ where each
$f_{\lambda \mu} : X_\lambda \to X_\mu$ is a thickening.
\end{lemma}

\begin{proof}
By Lemma \ref{lemma-characterize-quasi-compact} we may choose an
affine formal algebraic space $Y$ and a representable surjective
\'etale morphism $Y \to X$. Write $Y = \colim Y_\lambda$ as in
Definition \ref{definition-affine-formal-algebraic-space}.

\medskip\noindent
Pick $\lambda \in \Lambda$. Then $Y_\lambda \times_X Y$ is a scheme by
Lemma \ref{lemma-presentation-representable}. The reduction
(Lemma \ref{lemma-reduction-formal-algebraic-space})
of $Y_\lambda \times_X Y$ is equal to the reduction of
$Y_{red} \times_{X_{red}} Y_{red}$ which is quasi-compact as $X$
is quasi-separated and $Y_{red}$ is affine.
Therefore $Y_\lambda \times_X Y$ is a quasi-compact scheme.
Hence there exists a $\mu \geq \lambda$ such that
$\text{pr}_2 : Y_\lambda \times_X Y \to Y$ factors
through $Y_\mu$, see Lemma \ref{lemma-factor-through-thickening}.
Let $Z_\lambda$ be the scheme theoretic image of the morphism
$\text{pr}_2 : Y_\lambda \times_X Y \to Y_\mu$.
This is independent of the choice of $\mu$ and we can and
will think of $Z_\lambda \subset Y$ as the scheme theoretic
image of the morphism $\text{pr}_2 : Y_\lambda \times_X Y \to Y$.
Observe that $Z_\lambda$ is also equal to the scheme theoretic image
of the morphism $\text{pr}_1 : Y \times_X Y_\lambda \to Y$ since
this is isomorphic to the morphism used to define $Z_\lambda$.
We claim that $Z_\lambda \times_X Y = Y \times_X Z_\lambda$ as subfunctors
of $Y \times_X Y$. Namely, since $Y \to X$ is \'etale we see that
$Z_\lambda \times_X Y$ is the scheme theoretic image of the morphism
$$
\text{pr}_{13} = \text{pr}_1 \times \text{id}_Y :
Y \times_X Y_\lambda \times_X Y \longrightarrow Y \times_X Y
$$
by Morphisms of Spaces, Lemma
\ref{spaces-morphisms-lemma-quasi-compact-scheme-theoretic-image}.
By the same token, $Y \times_X Z_\lambda$ is the scheme theoretic image
of the morphism
$$
\text{pr}_{13} = \text{id}_Y \times \text{pr}_2 : 
Y \times_X Y_\lambda \times_X Y \longrightarrow Y \times_X Y
$$
The claim follows. Then
$R_\lambda = Z_\lambda \times_X Y = Y \times_X Z_\lambda$
together with the morphism $R_\lambda \to Z_\lambda \times_S Z_\lambda$
defines an \'etale equivalence relation. In this way we obtain an algebraic
space $X_\lambda = Z_\lambda/R_\lambda$. By construction the diagram
$$
\xymatrix{
Z_\lambda \ar[r] \ar[d] & Y \ar[d] \\
X_\lambda \ar[r] & X
}
$$
is cartesian (because $X$ is the coequalizer of the two projections
$R = Y \times_X Y \to Y$, because $Z_\lambda \subset Y$ is $R$-invariant,
and because $R_\lambda$ is the restriction of $R$ to $Z_\lambda$).
Hence $X_\lambda \to X$ is representable and a closed immersion, see
Spaces, Lemma
\ref{spaces-lemma-morphism-sheaves-with-P-effective-descent-etale}.
On the other hand, since $Y_\lambda \subset Z_\lambda$ we see that
$(X_\lambda)_{red} = X_{red}$, in other words, $X_\lambda \to X$
is a thickening. Finally, we claim that
$$
X = \colim X_\lambda
$$
We have $Y \times_X X_\lambda = Z_\lambda \supset Y_\lambda$. Every
morphism $T \to X$ where $T$ is a scheme over $S$ lifts \'etale locally
to a morphism into $Y$ which lifts \'etale locally into a morphism
into some $Y_\lambda$. Hence $T \to X$ lifts \'etale locally on
$T$ to a morphism into $X_\lambda$. This finishes the proof.
\end{proof}

\begin{remark}
\label{remark-structure-quasi-compact-quasi-separated}
In this remark we translate the statement and proof of
Lemma \ref{lemma-structure-quasi-compact-quasi-separated}
into the language of formal schemes \`a la EGA.
Looking at Remark \ref{remark-ideals-of-definition} we see
that the lemma can be translated as follows
\begin{itemize}
\item[$(*)$] Every quasi-compact and quasi-separated formal
scheme has a fundamental system of ideals of definition.
\end{itemize}
To prove this we first use the induction principle (reformulated for
quasi-compact and quasi-separated formal schemes) of
Cohomology of Schemes, Lemma \ref{coherent-lemma-induction-principle}
to reduce to the the following situation:
$\mathfrak X = \mathfrak U \cup \mathfrak V$
with $\mathfrak U$, $\mathfrak V$ open formal subschemes,
with $\mathfrak V$ affine, and the result is true for $\mathfrak U$,
$\mathfrak V$, and $\mathfrak U \cap \mathfrak V$. Pick any ideals
of definition $\mathcal{I} \subset \mathcal{O}_\mathfrak U$
and $\mathcal{J} \subset \mathcal{O}_\mathfrak V$.
By our assumption that we have a fundamental system of ideals
of definition on $\mathfrak U$ and $\mathfrak V$ and because
$\mathfrak U \cap \mathfrak V$ is quasi-compact, we can find
ideals of definition $\mathcal{I}' \subset \mathcal{I}$
and $\mathcal{J}' \subset \mathcal{J}$
such that
$$
\mathcal{I}'|_{\mathfrak U \cap \mathfrak V} \subset
\mathcal{J}|_{\mathfrak U \cap \mathfrak V}
\quad\text{and}\quad
\mathcal{J}'|_{\mathfrak U \cap \mathfrak V} \subset
\mathcal{I}|_{\mathfrak U \cap \mathfrak V}
$$
Let $U \to U' \to \mathfrak U$ and $V \to V' \to \mathfrak V$ be the
closed immersions determined by the ideals of definition
$\mathcal{I}' \subset \mathcal{I} \subset \mathcal{O}_\mathfrak U$
and
$\mathcal{J}' \subset \mathcal{J} \subset \mathcal{O}_\mathfrak U$.
Let $\mathfrak U \cap V$ denote the open subscheme of $V$ whose
underlying topological space is that of $\mathfrak U \cap \mathfrak V$.
Similarly for $U \cap \mathfrak V$. Then we consider
$$
Z_U = \text{scheme theoretic image of }
U \amalg (\mathfrak U \cap V) \longrightarrow U'
$$
and
$$
Z_V = \text{scheme theoretic image of }
(U \cap \mathfrak V) \amalg V \longrightarrow V'
$$
Since taking scheme theoretic images of quasi-compact morphisms
commutes with restriction to opens (Morphisms, Lemma
\ref{morphisms-lemma-quasi-compact-scheme-theoretic-image})
we see that $Z_U \cap \mathfrak V = \mathfrak U \cap Z_V$.
Thus $Z_U$ and $Z_V$ glue to a scheme $Z$ which comes equipped
with a morphism $Z \to \mathfrak X$. Analogous to the discussion in
Remark \ref{remark-weak-ideals-of-definition}
we see that $Z$ corresponds to a weak ideal
of definition $\mathcal{I}_Z \subset \mathcal{O}_\mathfrak X$.
Note that $U \subset Z_U \subset Z$ and that
$V \subset Z_V \subset Z$. Thus the collection of all $\mathcal{I}_Z$
constructed in this manner forms a fundamental system of weak
ideals of definition. Hence a subfamily gives a fundamental system of ideals
of definition, see Remark \ref{remark-ideals-of-definition}.
\end{remark}






\section{Morphisms representable by algebraic spaces}
\label{section-representable}

\noindent
Let $f : X \to Y$ be a morphism of formal algebraic spaces which
is representable by algebraic spaces. For these types of morphisms
we have a lot of theory at our disposal, thanks to the work done
in the chapters on algebraic spaces.

\begin{lemma}
\label{lemma-algebraic-space-over-affine-formal}
Let $S$ be a scheme. Let $Y$ be an affine formal algebraic space over $S$.
Let $f : X \to Y$ be a map of sheaves on $(\Sch/S)_{fppf}$ which is
representable by algebraic spaces. Then $X$ is a formal
algebraic space.
\end{lemma}

\begin{proof}
Write $Y = \colim Y_\lambda$ as in
Definition \ref{definition-affine-formal-algebraic-space}.
For each $\lambda$ the fibre product
$X \times_Y Y_\lambda$ is an algebraic space.
Hence $X = \colim X \times_Y Y_\lambda$ is a formal
algebraic space by Lemma \ref{lemma-colimit-is-formal}.
\end{proof}

\begin{lemma}
\label{lemma-representable-by-algebraic-spaces}
Let $S$ be a scheme. Let $Y$ be a formal algebraic space over $S$.
Let $f : X \to Y$ be a map of sheaves on $(\Sch/S)_{fppf}$ which is
representable by algebraic spaces. Then $X$ is a formal
algebraic space.
\end{lemma}

\begin{proof}
Let $\{Y_i \to Y\}$ be as in
Definition \ref{definition-formal-algebraic-space}.
Then $X \times_Y Y_i \to X$ is a family of morphisms
representable by algebraic spaces, \'etale, and jointly
surjective. Thus it suffices to show that
$X \times_Y Y_i$ is a formal algebraic space, see
Lemma \ref{lemma-etale-covering-by-formal-algebraic-spaces}.
This follows from Lemma \ref{lemma-algebraic-space-over-affine-formal}.
\end{proof}

\begin{lemma}
\label{lemma-affine-representable-by-algebraic-spaces}
Let $S$ be a scheme. Let $f : X \to Y$ be a morphism of
affine formal algebraic spaces which is representable by
algebraic spaces. Then $f$ is representable (by schemes) and affine.
\end{lemma}

\begin{proof}
The first assertion follows from Lemma \ref{lemma-presentation-representable}.
Write $Y = \colim Y_\mu$ and $X = \colim X_\lambda$ as in
Definition \ref{definition-affine-formal-algebraic-space}.
For the second, let $T \to Y$ be a morphism where $T$ is a scheme
over $S$. We have to show that $X \times_Y T \to T$ is affine, see
Spaces, Definition \ref{spaces-definition-relative-representable-property}.
To do this we may assume that $T$ is affine and we have to prove
that $X \times_Y T$ is affine. In this case $T \to Y$ factors
through $Y_\mu \to Y$ for some $\mu$, see
Lemma \ref{lemma-factor-through-thickening}.
Since $f$ is quasi-compact we see that $X \times_Y T$ is
quasi-compact (Lemma \ref{lemma-characterize-quasi-compact-morphism}).
Hence $X \times_Y T \to X$ factors through $X_\lambda$ for some
$\lambda$. Similarly $X_\lambda \to Y$ factors through $Y_\mu$
after increasing $\mu$. Then
$X \times_Y T = X_\lambda \times_{Y_\mu} T$.
We conclude as fibre products of affine schemes are affine.
\end{proof}

\begin{lemma}
\label{lemma-property-goes-up-affine-morphism}
Let $S$ be a scheme. Let $Y$ be an affine formal algebraic space.
Let $f : X \to Y$ be a map of sheaves on $(\Sch/S)_{fppf}$ which
is representable and affine. Then
\begin{enumerate}
\item $X$ is an affine formal algebraic space.
\item if $Y$ is countably indexed, then $X$ is countably indexed.
\item if $Y$ is adic*, then $X$ is adic*,
\item if $Y$ is Noetherian and $f$ is (locally) of finite type, then
$X$ is Noetherian.
\end{enumerate}
\end{lemma}

\begin{proof}
Proof of (1). Write $Y = \colim_{\lambda \in \Lambda} Y_\lambda$ as in
Definition \ref{definition-affine-formal-algebraic-space}.
Since $f$ is representable and affine, the fibre products
$X_\lambda = Y_\lambda \times_Y X$ are affine. And
$X = \colim Y_\lambda \times_Y X$.
Thus $X$ is an affine formal algebraic space.

\medskip\noindent
Proof of (2). If $Y$ is countably indexed, then in the argument above
we may assume $\Lambda$ is countable.
Then we immediately see that $X$ is countably indexed too.

\medskip\noindent
Proof of (3). Assume $Y$ is adic*. Then $Y = \text{Spf}(B)$ for some adic
topological ring $B$ which has a finitely generated
ideal $J$ such that $\{J^n\}$ is a fundamental system of open ideals.
Of course, then $Y = \colim \Spec(B/J^n)$.
The schemes $X \times_Y \Spec(B/J^n)$ are affine
and we can write $X \times_Y \Spec(B/J^n) = \Spec(A_n)$.
Then $X = \colim \Spec(A_n)$. The $B$-algebra maps $A_{n + 1} \to A_n$
are surjective and induce isomorphisms $A_{n + 1}/J^nA_{n + 1} \to A_n$.
By Algebra, Lemma \ref{algebra-lemma-limit-complete} the ring
$A = \lim A_n$ is $J$-adically complete and $A/J^n A = A_n$.
Hence $X = \text{Spf}(A^\wedge)$ is adic*.

\medskip\noindent
Proof of (4). Combining (3) with Lemma \ref{lemma-implications-between-types}
we see that $X$ is adic*. Thus we can use the criterion of
Lemma \ref{lemma-characterize-noetherian-affine}.
First, it tells us the affine schemes $Y_\lambda$ are Noetherian.
Then $X_\lambda \to Y_\lambda$ is of finite type, hence $X_\lambda$
is Noetherian too (Morphisms, Lemma
\ref{morphisms-lemma-finite-type-noetherian}).
Then the criterion tells us $X$ is Noetherian and the proof is
complete.
\end{proof}

\begin{lemma}
\label{lemma-property-goes-up-affine}
Let $S$ be a scheme. Let $f : X \to Y$ be a morphism of affine formal
algebraic spaces which is representable by algebraic spaces. Then
\begin{enumerate}
\item if $Y$ is countably indexed, then $X$ is countably indexed.
\item if $Y$ is adic*, then $X$ is adic*,
\item if $Y$ is Noetherian and $f$ is (locally) of finite type, then
$X$ is Noetherian.
\end{enumerate}
\end{lemma}

\begin{proof}
Combine Lemmas \ref{lemma-affine-representable-by-algebraic-spaces} and
\ref{lemma-property-goes-up-affine-morphism}.
\end{proof}

\noindent
We will sometimes use
Bootstrap, Definition \ref{bootstrap-definition-property-transformation}
to transport properties of morphisms of algebraic spaces
to properties of morphisms of formal algebraic spaces.
If the morphism is representable (by schemes), then this is the
same as using
Spaces, Definition \ref{spaces-definition-relative-representable-property}.

\begin{lemma}
\label{lemma-characterize-finite-type}
Let $S$ be a scheme. Let $f : X \to Y$ be a morphism of formal algebraic
spaces over $S$ which is representable by algebraic spaces.
The following are equivalent
\begin{enumerate}
\item $f$ is of finite type in the sense of
Bootstrap, Definition \ref{bootstrap-definition-property-transformation},
\item $f$ is quasi-compact (Definition \ref{definition-quasi-compact-morphism})
and is locally of finite type in the sense of
Bootstrap, Definition \ref{bootstrap-definition-property-transformation}.
\end{enumerate}
\end{lemma}

\begin{proof}
Omitted.
\end{proof}







\section{Types of formal algebraic spaces}
\label{section-types}

\noindent
In this section we define Noetherian, adic*, and
countably indexed formal algebraic spaces. The types
adic, classical, and McQuillan are missing as we do not
know how to prove the analogue of the following lemmas
for those cases.

\begin{lemma}
\label{lemma-iff-countably-indexed}
Let $S$ be a scheme. Let $X \to Y$ be a morphism of affine
formal schemes which is representable by algebraic spaces,
surjective, and flat. Then $X$ is countably indexed if and only
if $Y$ is countably indexed.
\end{lemma}

\begin{proof}
Assume $X$ is countably indexed. We write $X = \colim X_n$ as in
Lemma \ref{lemma-countable-affine-formal-algebraic-space}.
Write $Y = \colim Y_\lambda$ as in
Definition \ref{definition-affine-formal-algebraic-space}.
For every $n$ we can pick a $\lambda_n$ such that
$X_n \to Y$ factors through $Y_{\lambda_n}$, see
Lemma \ref{lemma-factor-through-thickening}.
On the other hand, for every $\lambda$ the scheme
$Y_\lambda \times_Y X$ is affine
(Lemma \ref{lemma-affine-representable-by-algebraic-spaces})
and hence $Y_\lambda \times_Y X \to X$ factors through
$X_n$ for some $n$ (Lemma \ref{lemma-factor-through-thickening}).
Picture
$$
\xymatrix{
Y_\mu \times_Y X \ar[r] \ar[d] \ar@/^2pc/[rr]_{} &
Y_\lambda \times_Y X \ar[r] \ar[d] & X_n \ar[r] \ar[d] & X \ar[d] \\
Y_\mu \ar[r]\ar@/_2pc/[rr]^{} &
Y_\lambda \ar@{..>}[r] & Y_{\lambda_n} \ar[r] & Y
}
$$
If we can show the dotted arrow exists, then we conclude that
$Y = \colim Y_{\lambda_n}$ and $Y$ is countably indexed. To do this we pick a
$\mu$ such that we have the solid arrows in the diagram. Say
$Y_\mu = \Spec(B_\mu)$, the closed subscheme $Y_\lambda$ corresponds to
$J \subset B_\mu$, and the closed subscheme $Y_{\lambda_n}$ corresponds to
$J' \subset B_\mu$. We are trying to show that $J \subset J'$.
By the diagram above we know $JA_\mu \subset J'A_\mu$
where $Y_\mu \times_Y X = \Spec(A_\mu)$.
Since $X \to Y$ is surjective and flat the morphism
$Y_\lambda \times_Y X \to Y_\lambda$ is a faithfully flat morphism
of affine schemes, hence $B_\mu \to A_\mu$ is
faithfully flat. Thus $J \subset J'$ as desired.

\medskip\noindent
Assume $Y$ is countably indexed. Then $X$ is countably indexed
by Lemma \ref{lemma-property-goes-up-affine}.
\end{proof}

\begin{lemma}
\label{lemma-iff-adic*}
Let $S$ be a scheme. Let $X \to Y$ be a morphism of affine
formal schemes which is representable by algebraic spaces,
surjective, and flat. Then $X$ is adic* if and only if $Y$ is adic*.
\end{lemma}

\begin{proof}
Assume $Y$ is adic*. Then $X$ is adic* by
Lemma \ref{lemma-property-goes-up-affine}.

\medskip\noindent
Assume $X$ is adic*. Write $X = \text{Spf}(A)$ for some adic ring $A$
which has a finitely generated ideal $I$ such that $\{I^n\}$ is a
fundamental system of open ideals. By
Lemmas \ref{lemma-iff-countably-indexed},
\ref{lemma-countable-affine-formal-algebraic-space}, and
\ref{lemma-implications-between-types}
we can write $Y = \text{Spf}(B)$ where $B$ is a weakly admissible
topological ring with a countable fundamental system $\{J_m\}$ of
weak ideals of definition. Set $Y_m = \Spec(B/J_m)$ so that
$Y = \colim Y_m$. The scheme $Y_m \times_Y X$ is affine
(Lemma \ref{lemma-affine-representable-by-algebraic-spaces})
and we have $X = \colim Y_m \times_Y X$. Say $Y_m \times_Y X = \Spec(A_m)$
so that $B/J_m \to A_m$ is a faithfully flat ring map.
Choose $n \geq 1$. There exists an $m$ such that $\Spec(A/I^n) \to Y$
factors through $Y_m$. In terms of ideals
\begin{equation}
\label{equation-first}
\forall n\ \exists m,\ J_m A \subset I^n.
\end{equation}
Choose $m \geq 1$. We can find an $n$ such that the morphism
$\Spec(A_m) \to X$ factors through $\Spec(A/I^n)$. In terms of ideals
\begin{equation}
\label{equation-second}
\forall m\ \exists n,\ I^n \subset \Ker(A \to A_m).
\end{equation}
Moreover, the squares
$$
\xymatrix{
A_{m'} \ar[r] & A_m \\
B/J_{m'} \ar[r] \ar[u] & B/J_m \ar[u]
}
$$
are cocartesian for all $m' \geq m$. Fixing $m$ we can pick
$m' \gg n \gg m$ such that we have a factorization
$A_{m'} \to A/I^n \to A_m$. Then the diagram being cocartesian
implies that $\Ker(A \to A_m) = J_mA + I^n$ for all $n \gg 0$.
In other words, $\Ker(A \to A_m)$ is the closure of $J_mA$.

\medskip\noindent
Fix an $m$. Pick $n$ such that $I^n \subset \Ker(A \to A_m)$
(\ref{equation-second}). Choose generators $f_1, \ldots, f_r$
of $I$. For any $E = (e_1, \ldots, e_r)$ with $|E| = \sum e_i = n$ write
$$
f_1^{e_1} \ldots f_r^{e_r} =
\sum g_{E, j} a_{E, j} + \delta_E
$$
with $g_{E, j} \in J_m$, $a_{E, j} \in A$, and $\delta_E \in I^{n + 1}$
(possible by the above). Let $J = (g_{E, j}) \subset B$.
Then we see that
$$
I^n \subset J A + I^{n + 1}
$$
As $I$ is contained in the radical of $A$ and $I^n$ is finitely generated
we see that $I^n \subset JA$ by Algebra, Lemma \ref{algebra-lemma-NAK}.

\medskip\noindent
We first apply what we just proved as follows:
since $I^n \subset J_mA$ we see that
$J_mA$ is open in $A$, hence closed, hence $\Ker(A \to A_m) = J_mA$,
in other words, $A_m = A/J_mA$. This holds for every $m$.

\medskip\noindent
Next, we pick $m$ with $J_mA \subset I$ (\ref{equation-first}).
Then choose $J \subset J_m$ with $I^n \subset JA \subset I$ as above.
For every $k \geq 1$ we define $\mathfrak b_k = \Ker(B \to A/J^kA)$.
For every $k$ there exists
an $m'$ with $J_{m'} \subset \mathfrak b_k$ as we have
$I^{nk} \subset J^kA$ and we can apply (\ref{equation-first}).
On the other hand, for every $m'$ there exists a $k$ such that
$I^k \subset J_{m'}A$ because $J_{m'}A$ is open. Then
$\mathfrak b_k$ maps to zero in $A/J_{m'}A$ which is faithfully
flat over $B/J_{m'}$. Hence $\mathfrak b_k \subset J_{m'}$.
In other words, we see that the topology on $B$ is defined by the
sequence of ideals $\mathfrak b_k$. Note that $J^k \subset \mathfrak b_k$
which implies that $\mathfrak b_k A = J^kA$.
In other words, we have reduced the problem to the
situation discussed in the following paragraph.

\medskip\noindent
We are given a ring map $B \to A$ where
\begin{enumerate}
\item $B$ is a weakly admissible topological ring with a fundamental system
$J_1 \supset J_2 \supset J_3 \supset \ldots$ of ideals of definition,
\item $A$ is a ring complete with respect to a finitely generated
ideal $I$,
\item we have $J_k A = I^k$ for all $k$, and
\item $B/J_k \to A/I^k$ is faithfully flat.
\end{enumerate}
Pick $g_1, \ldots, g_r \in J_1$
whose images in $A/I^2$ generate $I/I^2$; this is possible because
$J_1A/J_2A = I/I^2$. Then for all $k \geq 1$ we see that the elements
$g^E = g_1^{e_1} \ldots g_r^{e_r}$ with $|E| = k$ are in $J_k$
and their classes in $J_k/J_{k + 1}$ map to generators of
$I^k/I^{k + 1}$. Since $B/J_{k + 1} \to A/I^{k + 1}$
and $B/J_k \to A/I^k$ are flat we see that
$$
J_k/J_{k + 1} \otimes_{B/J_1} A/I =
J_k/J_{k + 1} \otimes_{B/J_{k + 1}} A/I^{k + 1} \to I^k/I^{k + 1}
$$
is an isomorphism (see More on Morphisms, Lemma
\ref{more-morphisms-lemma-deform}). Since $B/J_1 \to A/I$ is
faithfully flat, we conclude that the classes of the elements
$g^E$, $|E| = k$ generate $J_k/J_{k + 1}$. We claim that
$J_k = (g^E, |E| = k)$. Namely, suppose that $x_k \in J_k$.
By the above we can write
$$
x_k = \sum\nolimits_{|E| = k} b_{E, 0} g^E + x_{k + 1}
$$
with $x_{k + 1} \in J_{k + 1}$ and some $b_{E, 0} \in B$. Now we can write
$x_{k + 1}$ as follows
$$
x_{k + 1} =
\sum\nolimits_{|E| = k}
\left(\sum\nolimits_{|E'| = 1} b_{E, E'}g^{E'}\right) g^E + x_{k + 2}
$$
because every multi-index of degree $k + 1$ is a sum of a multi-index
of degree $k$ and a multi-index of degree $1$. Continuing in this
manner we can find $b_{E, E'} \in B$ such that for every $l > 1$
we have
$$
x_k = \sum\nolimits_{|E| = k}
\left(\sum\nolimits_{0 \leq |E'| < l} b_{E, E'} g^{E'}\right) g^E + x_{k + l}
$$
with some $x_{k + l} \in J_{k + l}$. Then we can finally define
$$
b_E = \sum\nolimits_{E'} b_{E, E'} g^{E'}
$$
as an element in $B$ and we see that $x_k = \sum b_E g^E$ as desired.
This finishes the proof as now $J_1$ is finitely generated and $J_k = J_1^k$
for all $k \geq 1$.
\end{proof}

\begin{lemma}
\label{lemma-iff-noetherian}
Let $S$ be a scheme. Let $X \to Y$ be a morphism of affine
formal schemes which is representable by algebraic spaces,
surjective, flat, and (locally) of finite type. Then $X$ is Noetherian
if and only if $Y$ is Noetherian.
\end{lemma}

\begin{proof}
Observe that a Noetherian affine formal algebraic space is adic*, see
Lemma \ref{lemma-implications-between-types}. Thus by
Lemma \ref{lemma-iff-adic*} we may assume that both $X$ and $Y$
are adic*. We will use the criterion of
Lemma \ref{lemma-characterize-noetherian-affine}
to see that the lemma holds. Namely, write $Y = \colim Y_n$
as in Lemma \ref{lemma-countable-affine-formal-algebraic-space}.
For each $n$ set $X_n = Y_n \times_Y X$. Then $X_n$ is an
affine scheme (Lemma \ref{lemma-affine-representable-by-algebraic-spaces})
and $X = \colim X_n$. Each of the morphisms $X_n \to Y_n$ is
faithfully flat and of finite type. Thus the lemma follows from the
fact that in this situation $X_n$ is Noetherian if and only if $Y_n$
is Noetherian, see
Algebra, Lemma \ref{algebra-lemma-descent-Noetherian} (to go down)
and
Algebra, Lemma \ref{algebra-lemma-Noetherian-permanence} (to go up).
\end{proof}

\begin{lemma}
\label{lemma-type-local}
Let $S$ be a scheme. Let $P \in \{countably\ indexed, adic*, Noetherian\}$.
Let $X$ be a formal algebraic space over $S$.
The following are equivalent
\begin{enumerate}
\item if $Y$ is an affine formal algebraic space and
$f : Y \to X$ is representable by algebraic spaces and \'etale,
then $Y$ has property $P$,
\item for some $\{X_i \to X\}_{i \in I}$ as in
Definition \ref{definition-formal-algebraic-space}
each $X_i$ has property $P$.
\end{enumerate}
\end{lemma}

\begin{proof}
It is clear that (1) implies (2). Assume (2) and let
$Y \to X$ be as in (1). Since the fibre products $X_i \times_Y X$
are formal algebraic spaces (Lemma \ref{lemma-fibre-products-general})
we can pick coverings $\{X_{ij} \to X_i \times_X Y\}$ as in
Definition \ref{definition-formal-algebraic-space}.
Since $Y$ is quasi-compact, there exist
$(i_1, j_1), \ldots, (i_n, j_n)$ such that
$$
X_{i_1 j_1} \amalg \ldots \amalg X_{i_n j_n} \longrightarrow Y
$$
is surjective. Then $X_{i_kj_k} \to X_{i_k}$ is representable
by algebraic spaces and \'etale hence $X_{i_kj_k}$ has property $P$ by
Lemma \ref{lemma-property-goes-up-affine}.
Then $X_{i_1 j_1} \amalg \ldots \amalg X_{i_n j_n}$ is an
affine formal algebraic space with property $P$ (small detail
omitted on finite disjoint unions of affine formal algebraic spaces).
Hence we conclude by applying one of
Lemmas \ref{lemma-iff-countably-indexed},
\ref{lemma-iff-adic*}, and
\ref{lemma-iff-noetherian}.
\end{proof}

\noindent
The previous lemma clears the way for the following definition.

\begin{definition}
\label{definition-types-formal-algebraic-spaces}
Let $S$ be a scheme. Let $X$ be a formal algebraic space over $S$.
We say $X$ is {\it locally countably indexed},
{\it locally adic*}, or {\it locally Noetherian}
if the equivalent conditions of Lemma \ref{lemma-type-local}
hold for the corresponding property.
\end{definition}











\section{Restricted power series}
\label{section-restricted-power-series}

\noindent
Let $A$ be a topological ring complete with respect to a linear
topology (More on Algebra, Definition
\ref{more-algebra-definition-topological-ring}).
Let $I_\lambda$ be a fundamental system of open ideals.
Let $r \geq 0$ be an integer. In this setting one often
denotes
$$
A\{x_1, \ldots, x_r\} =
\lim_\lambda A/I_\lambda[x_1, \ldots, x_r] =
\lim_\lambda (A[x_1, \ldots, x_r]/I_\lambda A[x_1, \ldots, x_r])
$$
the completion of the polynomial ring with respect to the
ideals $I_\lambda$. We can think of elements of $A\{x_1, \ldots, x_r\}$ as
power series
$$
f = \sum\nolimits_{E = (e_1, \ldots, e_r)} a_E x_1^{e_1} \ldots x_r^{e_r}
$$
in $x_1, \ldots, x_r$ with coefficients $a_E \in A$ which tend
to zero in the topology of $A$. In other words, for any $\lambda$
all but a finite number of $a_E$ are in $I_\lambda$.
For this reason elements of $A\{x_1, \ldots, x_r\}$ are sometimes
called {\it restricted power series}.
Sometimes this ring is denoted $A\langle x_1, \ldots, x_r\rangle$;
we will refrain from using this notation.

\begin{remark}[Universal property restricted power series]
\label{remark-universal-property}
\begin{reference}
\cite[Chapter 0, 7.5.3]{EGA}
\end{reference}
Let $A \to C$ be a continuous map of complete linearly topologized rings.
Then any $A$-algebra map $A[x_1, \ldots x_r] \to C$ extends uniquely to a
continuous map $A\{x_1, \ldots, x_r\} \to C$ on restricted power series.
\end{remark}

\begin{remark}
\label{remark-I-adic-completion-and-restricted-power-series}
Let $A$ be a ring and let $I \subset A$ be an ideal.
If $A$ is $I$-adically complete, then the $I$-adic completion
$A[x_1, \ldots, x_r]^\wedge$ of $A[x_1, \ldots, x_r]$ is the
restricted power series ring over $A$ as a ring. However, it is not clear
that $A[x_1, \ldots, x_r]^\wedge$ is $I$-adically complete
unless $I$ is finitely generated. One way to think about this, is
to think of the topology on $A\{x_1, \ldots, x_r\}$ as the limit
topology (which is always complete) and the topology on
$A[x_1, \ldots, x_r]^\wedge$ as the $I$-adic topology (not always complete).
\end{remark}

\noindent
Let $A$ be a topological ring complete with respect to a linear
topology. Let $I_\lambda$ be a fundamental system of open ideals.
Let $\mathcal{C}$ be the category of systems $(B_\lambda)$ where
\begin{enumerate}
\item $B_\lambda$ is a finite type $A/I_\lambda$-algebra, and
\item $B_\mu \to B_\lambda$ is an $A/I_\mu$-algebra homomorphism
which induces an isomorphism $B_\mu/I_\lambda B_\mu \to B_\lambda$.
\end{enumerate}
Morphisms in $\mathcal{C}$ are given by systems of homomorphisms.

\begin{lemma}
\label{lemma-category-affine-over}
Let $S$ be a scheme. Let $X$ be an affine formal scheme over $S$.
Assume $X$ is McQuillan and let $A$ be the weakly admissible topological
ring associated to $X$. Then there is an anti-equivalence of categories
between
\begin{enumerate}
\item the category $\mathcal{C}$ introduced above, and
\item the category of arrows $Y \to X$ of affine formal algebraic
spaces which are representable by algebraic spaces and of finite type.
\end{enumerate}
\end{lemma}

\begin{proof}
Let $I_\lambda$ be a fundamental system of weakly admissible ideals
of definition in $A$. Then $Y \times_X \Spec(A/I_\lambda)$ is
affine (Lemma \ref{lemma-affine-representable-by-algebraic-spaces}).
Say $Y \times_X \Spec(A/I_\lambda) = \Spec(B_\lambda)$.
Then $(B_\lambda)$ is an object of $\mathcal{C}$.
Conversely, given a system $(B_\lambda)$ we can set
$Y = \colim \Spec(B_\lambda)$. Some details omitted.
\end{proof}

\begin{remark}[Finite type]
\label{remark-finite-type}
Let $f : Y \to X$ be a morphism of formal algebraic spaces.
We could\footnote{Due to how things
are setup in the Stacks project, this is really the correct
thing to do and stronger notions should have a different name.}
say $f$ is {\it locally of finite type}
if $f$ is representable by algebraic spaces and is locally
of finite type in the sense of
Bootstrap, Definition \ref{bootstrap-definition-property-transformation}.
On the other hand, suppose that $A$ is a weakly admissible
topological ring and $X = \text{Spf}(A)$. Let $I_\lambda$ be a
fundamental system of weak ideals of definition, so that
$A = \lim A/I_\lambda$. Finally, assume $Y$ is an affine formal
algebraic space. Then a natural condition to consider would be:
$Y = \text{Spf}(B)$ where $B$ is a weakly admissible topological ring
which is a quotient of the restricted power series ring
$A\{x_1, \ldots, x_n\}$ by a closed ideal. However,
the first definition given above gives $Y = \colim \Spec(B_\lambda)$ where
$(B_\lambda)$ is an object of $\mathcal{C}$ by
Lemma \ref{lemma-category-affine-over}.
In this case it is true that there exists a fixed integer $r$ such
that $B_\lambda$ is generated by $r$ elements over $A/I_\lambda$ for
all $\lambda$ (hint: use
Algebra, Lemma \ref{algebra-lemma-surjective-mod-locally-nilpotent}).
However, it is not clear (and we think not true in general) that we can recover
$B_\lambda$ as the quotient of $B = \lim B_\lambda$ by $I_\lambda$
nor is it clear that $B$ is a quotient of $A\{x_1, \ldots, x_r\}$.
In particular, it is not clear that $Y$ is a McQuillan affine
formal algebraic space.
\end{remark}






\section{Two categories}
\label{section-two-categories}

\noindent
Let $A$ be a ring and let $I \subset A$ be an ideal.
In this section ${}^\wedge$ will mean $I$-adic completion.
Set $A_n = A/I^n$ so that the $I$-adic completion of $A$ is
$A^\wedge = \lim A_n$. Let $\mathcal{C}$ be the
category
\begin{equation}
\label{equation-C}
\mathcal{C} =
\left\{
\begin{matrix}
\text{systems }(B_n, B_{n + 1} \to B_n)_{n \in \mathbf{N}}\text{ where }\\
B_n\text{ is a finite type }A_n\text{-algebra,}\\
B_{n + 1} \to B_n\text{ is an }A_{n + 1}\text{-algebra map}\\
\text{which induces }B_{n + 1}/I^nB_{n + 1} \cong B_n
\end{matrix}
\right\}
\end{equation}
Morphisms in $\mathcal{C}$ are given by systems of homomorphisms.
Let $\mathcal{C}'$ be the category
\begin{equation}
\label{equation-C-prime}
\mathcal{C}' =
\left\{
\begin{matrix}
A\text{-algebras }B\text{ which are }I\text{-adically complete}\\
\text{such that }B/IB\text{ is of finite type over }A/I
\end{matrix}
\right\}
\end{equation}
Morphisms in $\mathcal{C}'$ are $A$-algebra maps. There is a functor
\begin{equation}
\label{equation-from-complete-to-systems}
\mathcal{C}' \longrightarrow \mathcal{C},\quad
B \longmapsto (B/I^nB)
\end{equation}
Indeed, since $B/IB$ is of finite type over $A/I$ the ring maps
$A_n = A/I^n \to B/I^nB$ are of finite type
(apply Algebra, Lemma \ref{algebra-lemma-NAK}
to a ring map $A/I^n[x_1, \ldots, x_r] \to B/I^nB$
such that the images of $x_1, \ldots, x_r$ generate $B/IB$
over $A/I$).

\begin{lemma}
\label{lemma-topologically-finite-type}
Let $A$ be a ring and let $I \subset A$ be a finitely generated ideal.
The functor
$$
\mathcal{C} \longrightarrow \mathcal{C}',\quad
(B_n) \longmapsto B = \lim B_n
$$
is a quasi-inverse to (\ref{equation-from-complete-to-systems}).
The completions $A[x_1, \ldots, x_r]^\wedge$ are in $\mathcal{C}'$ and
any object of $\mathcal{C}'$ is of the form
$$
B = A[x_1, \ldots, x_r]^\wedge / J
$$
for some ideal $J \subset A[x_1, \ldots, x_r]^\wedge$.
\end{lemma}

\begin{proof}
Let $(B_n)$ be an object of $\mathcal{C}$. By
Algebra, Lemma \ref{algebra-lemma-limit-complete}
we see that $B = \lim B_n$ is $I$-adically complete
and $B/I^nB = B_n$. Hence we see that $B$ is an object of
$\mathcal{C}'$ and that we can recover the object $(B_n)$ 
by taking the quotients.
Conversely, if $B$ is an object of $\mathcal{C}'$, then
$B = \lim B/I^nB$ by assumption. Thus $B \mapsto (B/I^nB)$ is a quasi-inverse
to the functor of the lemma.

\medskip\noindent
Since $A[x_1, \ldots, x_r]^\wedge = \lim A_n[x_1, \ldots, x_r]$
it is an object of $\mathcal{C}'$ by the first statement of the lemma.
Finally, let $B$ be an object of $\mathcal{C}'$. Choose
$b_1, \ldots, b_r \in B$ whose images in $B/IB$ generate
$B/IB$ as an algebra over $A/I$. Since $B$ is $I$-adically
complete, the $A$-algebra map $A[x_1, \ldots, x_r] \to B$, $x_i \mapsto b_i$
extends to an $A$-algebra map $A[x_1, \ldots, x_r]^\wedge \to B$.
To finish the proof we have to show this map is surjective
which follows from Algebra, Lemma \ref{algebra-lemma-completion-generalities}
as our map $A[x_1, \ldots, x_r] \to B$ is surjective modulo $I$
and as $B = B^\wedge$.
\end{proof}

\noindent
We warn the reader that, in case $A$ is not Noetherian, the
quotient of an object of $\mathcal{C}'$ may not be an object
of $\mathcal{C}'$. See Examples, Lemma
\ref{examples-lemma-noncomplete-quotient}.
Next we show this does not happen when $A$ is Noetherian.

\begin{lemma}
\label{lemma-topologically-finite-type-Noetherian}
\begin{reference}
\cite[Proposition 7.5.5]{EGA1}
\end{reference}
Let $A$ be a Noetherian ring and let $I \subset A$ be an ideal. Then
\begin{enumerate}
\item every object of the category $\mathcal{C}'$, in particular the
completion $A[x_1, \ldots, x_r]^\wedge$, is Noetherian,
\item if $B$ is an object of $\mathcal{C}'$ and $J \subset B$ is an
ideal, then $B/J$ is an object of $\mathcal{C}'$.
\end{enumerate}
\end{lemma}

\begin{proof}
To see (1) by Lemma \ref{lemma-topologically-finite-type}
we reduce to the case of the completion of the polynomial ring.
This case follows from
Algebra, Lemma \ref{algebra-lemma-completion-Noetherian-Noetherian}
as $A[x_1, \ldots, x_r]$ is Noetherian
(Algebra, Lemma \ref{algebra-lemma-Noetherian-permanence}).
Part (2) follows from Algebra, Lemma \ref{algebra-lemma-completion-tensor}
which tells us that ever finite $B$-module is
$IB$-adically complete.
\end{proof}

\begin{remark}[Base change]
\label{remark-base-change}
Let $\varphi : A_1 \to A_2$ be a ring map and let
$I_i \subset A_i$ be ideals such that $\varphi(I_1^c) \subset I_2$
for some $c \geq 1$. This induces ring maps
$A_{1, cn} = A_1/I_1^{cn} \to A_2/I_2^n = A_{2, n}$ for all $n \geq 1$.
Let $\mathcal{C}_i$ be the category (\ref{equation-C}) for $(A_i, I_i)$.
There is a base change functor
\begin{equation}
\label{equation-base-change-systems}
\mathcal{C}_1 \longrightarrow \mathcal{C}_2,\quad
(B_n) \longmapsto (B_{cn} \otimes_{A_{1, cn}} A_{2, n})
\end{equation}
Let $\mathcal{C}_i'$ be the category (\ref{equation-C-prime}) for $(A_i, I_i)$.
If $I_2$ is finitely generated, then there is a base change functor
\begin{equation}
\label{equation-base-change-complete}
\mathcal{C}_1' \longrightarrow \mathcal{C}_2',\quad
B \longmapsto (B \otimes_{A_1} A_2)^\wedge
\end{equation}
because in this case the completion is complete
(Algebra, Lemma \ref{algebra-lemma-hathat-finitely-generated}).
If both $I_1$ and $I_2$ are finitely generated, then
the two base change functors agree via the functors
(\ref{equation-from-complete-to-systems})
which are equivalences by Lemma \ref{lemma-topologically-finite-type}.
\end{remark}

\begin{remark}[Base change by closed immersion]
\label{remark-take-bar}
Let $A$ be a Noetherian ring and $I \subset A$ an ideal.
Let $\mathfrak a \subset A$ be an ideal. Denote $\bar A = A/\mathfrak a$.
Let $\bar I \subset \bar A$ be an ideal such that
$I^c \bar A \subset \bar I$ and $\bar I^d \subset I\bar A$
for some $c, d \geq 1$. In this case the base change functor
(\ref{equation-base-change-complete}) for $(A, I)$ to $(\bar A, \bar I)$
is given by $B \mapsto \bar B = B/\mathfrak aB$. Namely, we have
\begin{equation}
\label{equation-base-change-to-closed}
\bar B = (B \otimes_A \bar A)^\wedge = (B/\mathfrak a B)^\wedge =
B/\mathfrak a B
\end{equation}
the last equality because any finite $B$-module is $I$-adically complete by
Algebra, Lemma \ref{algebra-lemma-completion-tensor}
and if annihilated by $\mathfrak a$ also $\bar I$-adically complete by
Algebra, Lemma \ref{algebra-lemma-change-ideal-completion}.
\end{remark}







\section{A naive cotangent complex}
\label{section-naive-cotangent-complex}

\noindent
Let $A$ be a Noetherian ring and let $I \subset A$ be a ideal.
Let $B$ be an $A$-algebra which is $I$-adically complete such
that $A/I \to B/IB$ is of finite type, i.e., an object of
(\ref{equation-C-prime}).
By Lemma \ref{lemma-topologically-finite-type-Noetherian} we can write
$$
B = A[x_1, \ldots, x_r]^\wedge / J
$$
for some finitely generated ideal $J$. For a choice of presentation as
above we define the naive cotangent complex in this setting by the formula
\begin{equation}
\label{equation-NL}
\NL^\wedge_{B/A} = (J/J^2 \longrightarrow \bigoplus B\text{d}x_i)
\end{equation}
with terms sitting in degrees $-1$ and $0$
where the map sends the residue class of $g \in J$ to the differential
$\text{d}g = \sum (\partial g/\partial x_i) \text{d}x_i$. Here
the partial derivative is taken by thinking of $g$ as a power series.
The following lemma shows that $\NL^\wedge_{B/A}$ is well defined
in $D(B)$, i.e., independent of the chosen presentation, although this
could be shown directly by comparing presentations as in
Algebra, Section \ref{algebra-section-netherlander}.

\begin{lemma}
\label{lemma-NL-is-limit}
Let $A$ be a Noetherian ring and let $I \subset A$ be a ideal.
Let $B$ be an object of (\ref{equation-C-prime}). Then
$\NL^\wedge_{B/A} = R\lim \NL_{B_n/A_n}$ in $D(B)$.
\end{lemma}

\begin{proof}
In fact, the presentation $B = A[x_1, \ldots, x_r]^\wedge / J$
defines presentations
$$
B_n = B/I^nB = A_n[x_1, \ldots, x_r]/J_n
$$
where
$$
J_n = JA_n[x_1, \ldots, x_r] =
J/(J \cap I^nA[x_1, \ldots, x_r]^\wedge)
$$
By Artin-Rees (Algebra, Lemma \ref{algebra-lemma-Artin-Rees})
in the Noetherian ring $A[x_1, \ldots, x_r]^\wedge$
(Lemma \ref{lemma-topologically-finite-type-Noetherian})
we see that we have canonical surjections
$$
J/I^nJ \to J_n \to J/I^{n - c}J,\quad n \geq c
$$
for some $c \geq 0$.
It follows that $\lim J_n/J_n^2 = J/J^2$ as any finite
$A[x_1, \ldots, x_r]^\wedge$-module is $I$-adically complete
(Algebra, Lemma \ref{algebra-lemma-completion-tensor}).
Thus
$$
\NL^\wedge_{B/A} =
\lim (J_n/J_n^2 \longrightarrow \bigoplus B_n \text{d}x_i)
$$
(termwise limit)
and the transition maps in the system are termwise surjective.
The two term complex $J_n/J_n^2 \longrightarrow \bigoplus B_n \text{d}x_i$
represents $\NL_{B_n/A_n}$ by
Algebra, Section \ref{algebra-section-netherlander}.
It follows that $\NL^\wedge_{B/A}$ represents
$R\lim \NL_{B_n/A_n}$ in the derived category by
More on Algebra, Lemma \ref{more-algebra-lemma-compute-Rlim-modules}.
\end{proof}

\begin{lemma}
\label{lemma-exact-sequence-NL}
Let $A$ be a Noetherian ring and let $I \subset A$ be a ideal.
Let $B \to C$ be morphism of (\ref{equation-C-prime}). Then
there is an exact sequence
$$
\xymatrix{
H^{-1}(\NL^\wedge_{B/A} \otimes_B C) \ar[r] &
H^{-1}(\NL^\wedge_{C/A}) \ar[r] &
H^{-1}(\NL^\wedge_{C/B}) \ar[lld] \\
C \otimes_B H^0(\NL^\wedge_{B/A}) \ar[r] &
H^0(\NL^\wedge_{C/A}) \ar[r] &
H^0(\NL^\wedge_{C/B}) \ar[r] & 0
}
$$
\end{lemma}

\begin{proof}
Choose a presentation $B = A[x_1, \ldots, x_r]^\wedge/J$.
Note that $(B, IB)$ is a pair consisting of a Noetherian ring
and an ideal, and $C$ is in the corresponding category (\ref{equation-C-prime})
for this pair. Hence we can choose a presentation
$C = B[y_1, \ldots, y_s]^\wedge/J'$. Combinging these presentations
gives a presentation
$$
C = A[x_1, \ldots, x_r, y_1, \ldots, y_s]^\wedge/K
$$
Then the reader verifies that we obtain a commutative diagram\
$$
\xymatrix{
0 \ar[r] &
\bigoplus C \text{d}x_i \ar[r] &
\bigoplus C \text{d}x_i \oplus \bigoplus C \text{d}y_j \ar[r] &
\bigoplus C \text{d}y_j \ar[r] &
0 \\
&
J/J^2 \otimes_B C \ar[r] \ar[u] &
K/K^2 \ar[r] \ar[u] &
J'/(J')^2 \ar[r] \ar[u] &
0
}
$$
with exact rows. Note that the vertical arrow on the left hand side
is the tensor product of the arrow defining $\NL^\wedge_{B/A}$ with
$\text{id}_C$. The lemma follows by applying the snake lemma
(Algebra, Lemma \ref{algebra-lemma-snake}).
\end{proof}

\noindent
Maps in the derived category out of a complex such as (\ref{equation-NL})
are easy to understand by the result of the following lemma.

\begin{lemma}
\label{lemma-map-out-of-almost-free}
Let $R$ be a ring. Let $M^\bullet$ be a complex of modules over $R$
with $M^i = 0$ for $i > 0$ and $M^0$ a projective $R$-module.
Let $K^\bullet$ be a second complex.
\begin{enumerate}
\item If $K^i = 0$ for $i \leq -2$, then
$\Hom_{D(R)}(M^\bullet, K^\bullet) = \Hom_{K(R)}(M^\bullet, K^\bullet)$,
\item If $K^i = 0$ for $i \leq -3$ and
$\alpha \in \Hom_{D(R)}(M^\bullet, K^\bullet)$ composed with
$K^\bullet \to K^{-2}[2]$ comes from an $R$-module map
$a : M^{-2} \to K^{-2}$ with $a \circ d_M^{-3} = 0$, then
$\alpha$ can be represented by a map of complexes
$a^\bullet : M^\bullet \to K^\bullet$ with $a^{-2} = a$.
\item In (2) for any second map of complexes
$(a')^\bullet : M^\bullet \to K^\bullet$
representing $\alpha$ with $a = (a')^{-2}$
there exist $h' : M^0 \to K^{-1}$ and
$h : M^{-1} \to K^{-2}$ such that
$$
h \circ d_M^{-2} = 0, \quad
(a')^{-1} = a^{-1} + d_K^{-2} \circ h + h' \circ d_M^{-1},\quad
(a')^0 = a^0 + d_K^{-1} \circ h'
$$
\end{enumerate}
\end{lemma}

\begin{proof}
Set $F^0 = M^0$.
Choose a free $R$-module $F^{-1}$ and a surjection $F^{-1} \to M^{-1}$.
Choose a free $R$-module $F^{-2}$ and a surjection
$F^{-2} \to M^{-2} \times_{M^{-1}} F^{-1}$. Continuing in this
way we obtain a quasi-isomorphism $p^\bullet : F^\bullet \to M^\bullet$
which is termwise surjective and with $F^i$ free for all $i$.

\medskip\noindent
Proof of (1). By
Derived Categories, Lemma \ref{derived-lemma-morphisms-from-projective-complex}
we have
$$
\Hom_{D(R)}(M^\bullet, K^\bullet) = \Hom_{K(R)}(F^\bullet, K^\bullet)
$$
If $K^i = 0$ for $i \leq -2$, then any morphism of complexes
$F^\bullet \to K^\bullet$ factors through $p^\bullet$. Similarly, any
homotopy $\{h^i : F^i \to K^{i - 1}\}$ factors through $p^\bullet$.
Thus (1) holds.

\medskip\noindent
Proof of (2). Choose $b^\bullet : F^\bullet \to K^\bullet$ representing
$\alpha$. The composition of $\alpha$ with $K^\bullet \to K^{-2}[2]$ is
represented by $b^{-2} : F^{-2} \to K^{-2}$. As this is homotopic to
$a \circ p^{-2} : F^{-2} \to M^{-2} \to K^{-2}$, there is a map
$h : F^{-1} \to K^{-2}$ such that $b^{-2} = a \circ p^{-2} + h \circ d_F^{-2}$.
Adjusting $b^\bullet$ by $h$ viewed as a homotopy from $F^\bullet$
to $K^\bullet$, we find that $b^{-2} = a \circ p^{-2}$. Hence $b^{-2}$
factors through $p^{-2}$. Since $F^0 = M^0$ the kernel of $p^{-2}$
surjects onto the kernel of $p^{-1}$ (for example because the kernel
of $p^\bullet$ is an acyclic complex or by a diagram chase). Hence $b^{-1}$
necessarily factors through $p^{-1}$ as well and we see that (2)
holds for these factorizations and $a^0 = b^0$.

\medskip\noindent
Proof of (3) is omitted. Hint: There is a homotopy between
$a^\bullet \circ p^\bullet$ and $(a')^\bullet \circ p^\bullet$
and we argue as before that this homotopy factors through $p^\bullet$.
\end{proof}

\begin{lemma}
\label{lemma-zero-in-derived}
Let $R$ be a ring. Let $M^\bullet$ be a two term complex $M^{-1} \to M^0$
over $R$. If $\varphi, \psi \in \text{End}_{D(R)}(M^\bullet)$
are zero on $H^i(M^\bullet)$, then $\varphi \circ \psi = 0$.
\end{lemma}

\begin{proof}
Apply
Derived Categories, Lemma \ref{derived-lemma-trick-vanishing-composition}
to see that $\varphi \circ \psi$ factors through $\tau_{\leq -2}M^\bullet = 0$.
\end{proof}



\section{Rig-\'etale maps}
\label{section-rig-etale}

\noindent
In this and the following few sections we will study ring maps as
in Lemma \ref{lemma-equivalent-with-artin}. Condition
(\ref{item-condition-artin}) is one of the conditions used in
\cite{ArtinII} to define modifications. Ring maps like this are sometimes
called rig-\'etale ring maps in the literature. These and
the analogously defined rig-smooth ring maps were studied
in \cite{Elkik}. A detailed exposition can also be found in
\cite{Abbes}. Our main goal will be to show that rig-\'etale
ring maps are completions of finite type algebras, a result
very similar to results found in Elkik's paper \cite{Elkik}.

\begin{lemma}
\label{lemma-equivalent-with-artin}
Let $A$ be a Noetherian ring and let $I \subset A$ be an ideal.
Let $B$ be an object of (\ref{equation-C-prime}). The following are equivalent
\begin{enumerate}
\item
\label{item-zero-on-NL}
there exists a $c \geq 0$ such that multiplication by $a$
on $\NL^\wedge_{B/A}$ is zero in $D(B)$ for all $a \in I^c$,
\item
\label{item-zero-on-cohomology-NL}
there exits a $c \geq 0$ such that $H^i(\NL^\wedge_{B/A})$, $i = -1, 0$ is
annihilated by $I^c$,
\item
\label{item-zero-on-cohomology-NL-truncations}
there exists a $c \geq 0$ such that $H^i(\NL_{B_n/A_n})$, $i = -1, 0$ is
annihlated by $I^c$ for all $n \geq 1$,
\item
\label{item-condition-artin}
$B = A[x_1, \ldots, x_r]^\wedge/J$ and
for every $a \in I$ there exists a $c \geq 0$ such that
\begin{enumerate}
\item $a^c$ annihilates $H^0(\NL^\wedge_{B/A})$, and
\item there exist $f_1, \ldots, f_r \in J$ such that
$a^c J \subset (f_1, \ldots, f_r) + J^2$.
\end{enumerate}
\end{enumerate}
\end{lemma}

\begin{proof}
The equivalence of (1) and (2) follows from
Lemma \ref{lemma-zero-in-derived}.
The equivalence of (1) $+$ (2) and (3) follows from
Lemma \ref{lemma-NL-is-limit}. Some details omitted.

\medskip\noindent
Assume the equivalent conditions (1), (2), (3) holds and let
$B = A[x_1, \ldots, x_r]^\wedge/J$ be a presentation
(see Lemma \ref{lemma-topologically-finite-type}). Let $a \in I$.
Let $c$ be such that multplication by $a^c$ is zero on $\NL^\wedge_{B/A}$
which exists by (1). By Lemma \ref{lemma-zero-in-derived} there exists a
a map $\alpha : \bigoplus B\text{d}x_i \to J/J^2$ such that
$\text{d} \circ \alpha$ and $\alpha \circ \text{d}$ are both
multiplication by $a^c$. Let $f_i \in J$ be an element whose
class modulo $J^2$ is equal to $\alpha(\text{d}x_i)$.
Then we see that (\ref{item-condition-artin})(a), (b) hold.

\medskip\noindent
Assume (\ref{item-condition-artin}) holds. Say $I = (a_1, \ldots, a_t)$.
Let $c_i \geq 0$ be the integer such that (\ref{item-condition-artin})(a), (b)
hold for $a_i^{c_i}$. Then we see that $I^{\sum c_i}$ annihilates
$H^0(\NL^\wedge_{B/A})$. Let $f_{i, 1}, \ldots, f_{i, r} \in J$
be as in (\ref{item-condition-artin})(b) for $a_i$.
Consider the composition
$$
B^{\oplus r} \to J/J^2 \to \bigoplus B\text{d}x_i
$$
where the $j$th basis vector is mapped to the class of $f_{i, j}$ in $J/J^2$.
By (\ref{item-condition-artin})(a) and (b) the cokernel of the composition
is annihilated by $a_i^{2c_i}$. Thus this map is surjective after inverting
$a_i^{c_i}$, and hence an isomorphism (Algebra, Lemma \ref{algebra-lemma-fun}).
Thus the kernel of $B^{\oplus r} \to \bigoplus B\text{d}x_i$ is
$a_i$-power torsion, and hence
$H^{-1}(\NL^\wedge_{B/A}) = \Ker(J/J^2 \to \bigoplus B\text{d}x_i)$
is $a_i$-power torsion. Since $B$ is Noetherian
(Lemma \ref{lemma-topologically-finite-type-Noetherian}),
all modules including $H^{-1}(\NL^\wedge_{B/A})$ are finite.
Thus $a_i^{d_i}$ annihilates $H^{-1}(\NL^\wedge_{B/A})$ for some $d_i \geq 0$.
It follows that $I^{\sum d_i}$ annihilates $H^{-1}(\NL^\wedge_{B/A})$
and we see that (2) holds.
\end{proof}

\begin{lemma}
\label{lemma-rig-etale}
Let $A$ be a Noetherian ring and let $I$ be an ideal.
Let $B$ be a finite type $A$-algebra.
\begin{enumerate}
\item If $\Spec(B) \to \Spec(A)$ is \'etale over $\Spec(A) \setminus V(I)$,
then $B^\wedge$ satisfies the equivalent conditions of
Lemma \ref{lemma-equivalent-with-artin}.
\item If $B^\wedge$ satisfies the equivalent conditions of
Lemma \ref{lemma-equivalent-with-artin},
then there exists $g \in 1 + IB$ such that $\Spec(B_g)$ is \'etale
over $\Spec(A) \setminus V(I)$.
\end{enumerate}
\end{lemma}

\begin{proof}
Assume $B^\wedge$ satisfies the equivalent conditions of
Lemma \ref{lemma-equivalent-with-artin}.
The naive cotangent complex $\NL_{B/A}$ is a complex of finite type
$B$-modules and hence $H^{-1}$ and $H^0$ are finite $B$-modules.
Completion is an exact functor on finite $B$-modules (Algebra,
Lemma \ref{algebra-lemma-completion-flat}) and $\NL^\wedge_{B^\wedge/A}$
is the completion of the complex $\NL_{B/A}$ (this is easy to see
by choosing presentations).
Hence the assumption implies there exists a $c \geq 0$ such that
$H^{-1}/I^nH^{-1}$ and $H^0/I^nH^0$ are annihilated by $I^c$
for all $n$. By Nakayama's lemma (Algebra, Lemma \ref{algebra-lemma-NAK})
this means that $I^cH^{-1}$ and $I^cH^0$ are annihilated by an element
of the form $g = 1 + x$ with $x \in IB$. After inverting $g$
(which does not change the quotients $B/I^nB$)
we see that $\NL_{B/A}$ has cohomology annihilated by $I^c$. Thus
$A \to B$ is \'etale at any prime of $B$ not lying over $V(I)$
by the definition of \'etale ring maps, see
Algebra, Definition \ref{algebra-definition-etale}.

\medskip\noindent
Conversely, assume that $\Spec(B) \to \Spec(A)$ is \'etale over
$\Spec(A) \setminus V(I)$. Then for every $a \in I$ there exists
a $c \geq 0$ such that multiplication by $a^c$ is zero $\NL_{B/A}$.
Since $\NL^\wedge_{B^\wedge/A}$ is the derived completion of
$\NL_{B/A}$ (see Lemma \ref{lemma-NL-is-limit}) it follows that
$B^\wedge$ satisfies the equivalent conditions of
Lemma \ref{lemma-equivalent-with-artin}.
\end{proof}

\begin{lemma}
\label{lemma-zero-after-modding-out}
Assume the map $(A_1, I_1) \to (A_2, I_2)$ is as in
Remark \ref{remark-base-change} with $A_1$ and $A_2$ Noetherian.
Let $B_1$ be in (\ref{equation-C-prime}) for $(A_1, I_1)$.
Let $B_2$ be the base change of $B_1$.
If multiplication by $f_1 \in B_1$ on $\NL^\wedge_{B_1/A_1}$
is zero in $D(B_1)$, then multiplication by
the image $f_2 \in B_2$ on $\NL^\wedge_{B_2/A_2}$ is zero
in $D(B_2)$.
\end{lemma}

\begin{proof}
Choose a presentation $B_1 = A_1[x_1, \ldots, x_r]^\wedge/J_1$.
Since
$A_2/I_2^n[x_1, \ldots, x_r] =
A_1/I_1^{cn}[x_1, \ldots, x_r] \otimes_{A_1/I_1^{cn}} A_2/I_2^n$
we have
$$
A_2[x_1, \ldots, x_r]^\wedge =
(A_1[x_1, \ldots, x_r]^\wedge \otimes_{A_1} A_2)^\wedge
$$
where we use $I_2$-adic completion on both sides (but of course
$I_1$-adic completion for $A_1[x_1, \ldots, x_r]^\wedge$).
Set $J_2 = J_1 A_2[x_1, \ldots, x_r]^\wedge$. Arguing similarly
we get the presentation
\begin{align*}
B_2
& =
(B_1 \otimes_{A_1} A_2)^\wedge \\
& =
\lim \frac{A_1/I_1^{cn}[x_1, \ldots, x_r]}{J_1(A_1/I_1^{cn}[x_1, \ldots, x_r])}
\otimes_{A_1/I_1^{cn}} A_2/I_2^n \\
& =
\lim \frac{A_2/I_2^n[x_1, \ldots, x_r]}{J_2(A_2/I_2^n[x_1, \ldots, x_r])} \\
& =
A_2[x_1, \ldots, x_r]^\wedge/J_2
\end{align*}
for $B_2$ over $A_2$. Consider the commutative diagram
$$
\xymatrix{
\NL^\wedge_{B_1/A_1} : \ar[d] &
J_1/J_1^2 \ar[r]_-{\text{d}} \ar[d] & \bigoplus B_1\text{d}x_i \ar[d] \\
\NL^\wedge_{B_2/A_2} : &
J_2/J_2^2 \ar[r] & \bigoplus B_2\text{d}x_i
}
$$
The induced arrow $J_1/J_1^2 \otimes_{B_1} B_2 \to J_2/J_2^2$
is surjective because $J_2$ is generated by the image of $J_1$.
By Lemma \ref{lemma-zero-in-derived}
there is a map $\alpha_1 : \bigoplus B\text{d}x_i \to J_1/J_1^2$
such that $f_1 \text{id}_{\bigoplus B_1\text{d}x_i} = \text{d} \circ \alpha_1$
and $f_1 \text{id}_{J_1/J_1^2} = \alpha_1 \circ \text{d}$. We define
$\alpha_2 : \bigoplus B_1\text{d}x_i \to J_2/J_2^2$
by mapping $\text{d}x_i$ to the image of $\alpha_1(\text{d}x_i)$
in $J_2/J_2^2$. Because the image of the vertical arrows
contains generators of the modules $J_2/J_2^2$ and $\bigoplus B_2 \text{d}x_i$
it follows that $\alpha_2$ also defines a homotopy between
multiplication by $f_2$ and the zero map.
\end{proof}

\begin{lemma}
\label{lemma-quotient-by-idempotent}
Let $A$ be a Noetherian ring and $I$ an ideal. Let $B$
be a finite type $A$-algebra. Let $B^\wedge \to C$ be a surjective
ring map with kernel $J$. If $J/J^2$ is annihilated by $I^c$
for some $c \geq 0$, then $C$ is isomorphic to the completion
of a finite type $A$-algebra.
\end{lemma}

\begin{proof}
Since $B^\wedge$ is Noetherian (Lemma
\ref{lemma-topologically-finite-type-Noetherian}),
we see that $J$ is a finitely generated
ideal. Hence we conclude from
Algebra, Lemma \ref{algebra-lemma-ideal-is-squared-union-connected}
that
$$
\Spec(C) \setminus V(IC) \longrightarrow \Spec(B^\wedge) \setminus V(IB^\wedge)
$$
is an open and closed immersion. Let
$V \subset \Spec(B^\wedge) \setminus V(IB^\wedge)$ be the complement
of the image viewed as an open and closed subscheme.
Let $Z \subset \Spec(B^\wedge)$ be the scheme
theoretic closure of $V$. Write $Z = \Spec(C')$. Then
$$
\Spec(C \times C') = \Spec(C) \amalg Z \longrightarrow \Spec(B^\wedge)
$$
is a finite morphism of schemes which is an isomorphism away from
$V(IB^\wedge)$. Hence the corresponding ring map $B^\wedge \to C \times C'$
is finite and becomes an isomorphism on inverting any element of $I$. By
More on Algebra, Proposition \ref{more-algebra-proposition-equivalence}
and Remark \ref{more-algebra-remark-formal-glueing-algebras}
applied to $B \to B^\wedge$ and the finitely generated ideal $IB$, we
conclude that $C \times C'$ is isomorphic to $D \otimes_B B^\wedge$
for some finite $B$-algebra $D$. (The reader can also prove this using
Pushouts of Spaces, Lemma \ref{spaces-pushouts-lemma-equivalence-on-affine}.)
Then $D/ID \cong C/IC \times C'/IC'$. Let $\overline{e} \in D/ID$
be the idempotent corresponding to the factor $C/IC$.
By More on Algebra, Lemma
\ref{more-algebra-lemma-lift-idempotent-upstairs} there exists an
\'etale ring map $B \to B'$ which induces an isomorphism
$B/IB \to B'/IB'$ such that $D' = D \otimes_B B'$ contains an
idempotent $e$ lifting $\overline{e}$. Since $C \times C'$
is $I$-adically complete the pair $(C \times C', IC \times IC')$
is henselian
(More on Algebra, Lemma \ref{more-algebra-lemma-complete-henselian}).
Thus we can factor the map $B \to C \times C'$ through $B'$.
DOing so we may replace $B$ by $B'$ and $D$ by $D'$. Then
we find that $D = D_e \times D_{1 - e} = D/(1 - e) \times D/(e)$
is a product of finite type $A$-algebras and the completion of the
first part is $C$ and the completion of the second part is $C'$.
\end{proof}

\begin{lemma}
\label{lemma-fully-faithful-etale-over-complement}
Let $A$ be a Noetherian ring. Let $I \subset A$ be an ideal.
Let $C, B$ be finite type $A$-algebras. Assume $\Spec(B) \to \Spec(A)$ is
\'etale over $\Spec(A) \setminus V(I)$. Then any $A$-algebra
map $B^\wedge \to C^\wedge$ of $I$-adic completions
comes from a unique $A$-algebra map
$$
B \longrightarrow C^h
$$
where $C^h$ is the henselization of the pair $(C, IC)$ as
in More on Algebra, Lemma \ref{more-algebra-lemma-henselization}.
Moreover, any $A$-algebra homomorphism $B \to C^h$ factors through
some \'etale $C$-algebra $C'$ such that $C/IC \to C'/IC'$ is an isomorphism.
\end{lemma}

\begin{proof}
Uniqueness follows from the fact that $C^h$ is a subring of
$C^\wedge$, see for example
More on Algebra, Lemma \ref{more-algebra-lemma-henselization-Noetherian-pair}.
The final assertion follows from the fact that $C^h$ is the filtered colimit
of these $C$-algebras $C'$, see proof of
More on Algebra, Lemma \ref{more-algebra-lemma-henselization}.
Having said this we now turn to the proof of existence.

\medskip\noindent
Let $\varphi : B^\wedge \to C^\wedge$ be the given map.
This defines a section
$$
\sigma : (B \otimes_A C)^\wedge \longrightarrow C^\wedge
$$
of the completion of the map $C \to B \otimes_A C$. We may
replace $(A, I, B, C, \varphi)$ by $(C, IC, B \otimes_A C, C, \sigma)$.
In this way we see that we may assume that $A = C$.

\medskip\noindent
Proof of existence in the case $A = C$. In this case the map
$\varphi : B^\wedge \to A^\wedge$ is necessarily surjective.
By Lemmas \ref{lemma-rig-etale} and \ref{lemma-exact-sequence-NL}
we see that the cohomology groups of
$\NL^\wedge_{A^\wedge/\!_\varphi B^\wedge}$
are annihilated by a power of $I$. Since $\varphi$ is surjective,
this implies that $\Ker(\varphi)/\Ker(\varphi)^2$ is annihilated
by a power of $I$. Hence $\varphi : B^\wedge \to A^\wedge$
is the completion of a finite type $B$-algebra $B \to D$, see
Lemma \ref{lemma-quotient-by-idempotent}. Hence $A \to D$ is
a finite type algebra map which induces an isomorphism
$A^\wedge \to D^\wedge$. By
Lemma \ref{lemma-rig-etale} we may replace $D$ by a localization
and assume that $A \to D$ is \'etale away from $V(I)$.
Since $A^\wedge \to D^\wedge$ is an isomorphism, we see that
$\Spec(D) \to \Spec(A)$ is flat and unramified along $V(ID)$
as well, hence \'etale everywhere. Therefore $D$ maps to
$A^h$ and the lemma is proved.
\end{proof}










\section{Glueing rings along a principal ideal}
\label{section-approximation-principal}

\noindent
In this situation we prove some results about the categories
$\mathcal{C}$ and $\mathcal{C}'$ of
Section \ref{section-two-categories}
in case $A$ is a Noetherian ring and $I = (a)$ is a principal
ideal.

\begin{remark}[Linear approximation]
\label{remark-linear-approximation}
Let $A$ be a ring and $I \subset A$ be a finitely generated ideal.
Let $C$ be an $I$-adically complete $A$-algebra.
Let $\psi : A[x_1, \ldots, x_r]^\wedge \to C$ be a continuous
$A$-algebra map. Suppose given $\delta_i \in C$, $i = 1, \ldots, r$.
Then we can consider
$$
\psi' : A[x_1, \ldots, x_r]^\wedge \to C,\quad
x_i \longmapsto \psi(x_i) + \delta_i
$$
see Remark \ref{remark-universal-property}. Then we have
$$
\psi'(g) = \psi(g) + \sum \psi(\partial g/\partial x_i)\delta_i + \xi
$$
with error term $\xi \in (\delta_i\delta_j)$. This follows by
writing $g$ as a power series and working term by term. Convergence
is automatic as the coefficients of $g$ tend to zero.
Details omitted.
\end{remark}

\begin{lemma}
\label{lemma-get-morphism-nonzerodivisor}
Let $A$ be a Noetherian ring and $I = (a)$ a principal ideal.
Let $B$, $C$ be objects of (\ref{equation-C-prime}).
Let $c \geq 0$ be an integer such that
multiplication by $a^c$ on $\NL^\wedge_{B/A}$ is zero in $D(B)$.
Assume $a$ is a nonzerodivisor on $C$.
Let $n > 2c$. For any $A_n$-algebra map $\psi_n : B_n \to C_n$
there exists a morphism $\varphi : B \to C$ such
that $\psi_n \bmod a^{n - c} = \varphi \bmod a^{n - c}$.
\end{lemma}

\begin{proof}
Choose a presentation $B = A[x_1, \ldots, x_r]^\wedge/J$. Choose
a lift
$$
\psi : A[x_1, \ldots, x_r]^\wedge \to C
$$
of $\psi_n$. Then $\psi(J) \subset a^nC$ and $\psi(J^2) \subset a^{2n}C$
which determines a linear map
$$
J/J^2 \longrightarrow a^nC/a^{2n}C,\quad g \longmapsto \psi(g)
$$
By assumption and Lemma \ref{lemma-zero-in-derived}
there is a $B$-module map
$\bigoplus B\text{d}x_i \to a^nC/a^{2n}C$,
$\text{d}x_i \mapsto \delta_i$ such that
$a^c \psi(g) = \sum \psi(\partial g/\partial x_i) \delta_i$
for all $g \in J$. Write $\delta_i = - a^c \delta'_i$ for some
$\delta'_i \in a^{n - c}C$. Since $a$ is a nonzerodivisor
on $C$ we see that $\psi(g) = - \sum \psi(\partial g/\partial x_i) \delta'_i$
in $C/a^{2n - c}C$.
Then we look at the map
$$
\psi' : A[x_1, \ldots, x_r]^\wedge \to C,\quad
x_i \longmapsto \psi(x_i) + \delta'_i
$$
A computation with power series (see Remark \ref{remark-linear-approximation})
shows that $\psi'(J) \subset a^{2n - 2c}C$. Since $n > 2c$
we see that $n' = 2n - 2c = n + (n - 2c) > n$. Thus we obtain a morphism
$\psi_{n'} : B_{n'} \to C_{n'}$ agreeing with $\psi_n$ modulo
$a^{n - c}$. Continuing in this fashion and taking a limit we
obtain the lemma.
\end{proof}

\begin{lemma}
\label{lemma-get-morphism-principal}
Let $A$ be a Noetherian ring and $I = (a)$ a principal ideal.
Let $B$, $C$ be objects of (\ref{equation-C-prime}).
Let $c \geq 0$ be an integer such that
multiplication by $a^c$ on $\NL^\wedge_{B/A}$ is zero in $D(B)$.
Let $d \geq 0$ be an integer such that
$C[a^\infty] \cap a^dC = 0$; such an integer always exists.
Let $n > \max(2c, c + d)$. For any $A_n$-algebra map $\psi_n : B_n \to C_n$
there exists a morphism $\varphi : B \to C$ of $\mathcal{C}'$ such
that $\psi_n \bmod a^{n - c} = \varphi \bmod a^{n - c}$.
\end{lemma}

\begin{proof}
Because $C$ is Noetherian we have $C[a^\infty] = C[a^e]$ for some
$e \geq 0$. By Artin-Rees there exists an integer $f$ such that
$a^nC \cap C[a^\infty] \subset a^{n - f}C[a^\infty]$ for all $n \geq f$.
Thus $d = e + f$ works.

\medskip\noindent
Let $C \to C'$ be the quotient of $C$ by $C[a^\infty]$. For $m > d$ the
diagram
$$
\xymatrix{
0 \ar[r] &
C[a^\infty] \ar[r] \ar[d] &
C \ar[r] \ar[d] & C' \ar[r] \ar[d] & 0 \\
0 \ar[r] &
C[a^\infty] \ar[r] &
C/a^m C \ar[r] & C'/a^m C' \ar[r] & 0
}
$$
has exact rows. Thus $C$ is the fibre product of $C'$ and
$C_m$ over $C'_m$. Thus the lemma now follows formally from
the lifting result of Lemma \ref{lemma-get-morphism-nonzerodivisor}.
\end{proof}

\begin{lemma}
\label{lemma-approximate-principal}
\begin{reference}
The rig-\'etale case of \cite[III Theorem 7]{Elkik}
which handles the rig-smooth case.
\end{reference}
Let $A$ be a Noetherian ring and $I = (a)$ a principal ideal.
Let $B$ be an object of (\ref{equation-C-prime}).
Let $c \geq 0$ be an integer such that
multiplication by $a^c$ on $\NL^\wedge_{B/A}$ is zero in $D(B)$.
Then there exists a finite type $A$-algebra $C$ and an
isomorphism $B \cong C^\wedge$.
\end{lemma}

\begin{proof}
Choose a presentation $B = A[x_1, \ldots, x_r]^\wedge/J$.
By Lemma \ref{lemma-zero-in-derived} we can find a map
$\alpha : \bigoplus B\text{d}x_i \to J/J^2$ such that
$\text{d} \circ \alpha$ and $\alpha \circ \text{d}$ are both
multiplication by $a^c$. Pick an element $f_i \in J$ whose
class modulo $J^2$ is equal to $\alpha(\text{d}x_i)$.
Then we see that $\text{d}f_i = a^c \text{d}x_i$ in $\bigoplus \text{d}x_i$.
In particular we have a ring map
$$
A[x_1, \ldots, x_r]^\wedge/
(f_1, \ldots, f_r, \Delta(f_1, \ldots, f_r) - a^{rc})
\longrightarrow B
$$
where $\Delta(f_1, \ldots, f_r) \in A[x_1, \ldots, x_r]^\wedge$
is the determinant of the matrix of partial derivatives of the $f_i$.

\medskip\noindent
Pick a large integer $N$. Pick $F_1, \ldots, F_r \in A[x_1, \ldots, x_r]$
such that $F_i - f_i \in I^NA[x_1, \ldots, x_r]^\wedge$. Set
$$
C = A[x_1, \ldots, x_r, z]/
(F_1, \ldots, F_r, z\Delta(F_1, \ldots, F_r) - a^{rc})
$$
We claim that multplication by $a^{2rc}$ is zero on $\NL_{C/A}$ in $D(C)$.
Namely, the determinant of the matrix of the partial derivatives
of the $r + 1$ generators of the ideal of $C$ with respect to the variables
$x_1, \ldots, x_{r + 1}, z$ is $\Delta(F_1, \ldots, F_r)^2$. Since
$\Delta(F_1, \ldots, F_r)$ divides $a^{rc}$ we in $C$ the
claim follows for example from
Algebra, Lemma \ref{algebra-lemma-matrix-left-inverse}.
Let $C^\wedge$ be the $I$-adic completion of $C$. Since
$\NL^\wedge_{C^\wedge/A}$ is the $I$-adic completion of $\NL_{C/A}$
we conclude that multiplication by $a^{2rc}$ is zero on
$\NL^\wedge_{C^\wedge/A}$ as well.

\medskip\noindent
By construction there is a (surjective) map $\psi_N : C/I^NC \to B/I^NB$
sending $x_i$ to $x_i$ and $z$ to $1$. By
Lemma \ref{lemma-get-morphism-principal} (with the roles of $B$ and $C$
reversed) for $N$ large enough we get a map $\varphi : C^\wedge \to B$ which
agrees with $\psi_N$ modulo $I^{N - 2rc}$.

\medskip\noindent
Since $\varphi : C^\wedge \to B$ is surjective modulo $I$ we see that it is
surjective (for example use
Algebra, Lemma \ref{algebra-lemma-completion-generalities}).
By construction and assumption the naive cotangent complexes
$\NL^\wedge_{C^\wedge/A}$ and $\NL^\wedge_{B/A}$
have cohomology annihilated by a fixed power of $a$. Thus the same thing
is true for $\NL^\wedge_{B/C^\wedge}$ by Lemma \ref{lemma-exact-sequence-NL}.
Since $\varphi$ is surjective we conclude that
$\Ker(\varphi)/\Ker(\varphi)^2$ is annihilated by a power of $a$.
The result of the lemma now follows from
Lemma \ref{lemma-quotient-by-idempotent}.
\end{proof}









\section{Glueing rings along an ideal}
\label{section-approximation}

\noindent
Let $A$ be a Noetherian ring. Let $I \subset A$ be an ideal.
In this section we study $I$-adically complete $A$-algebras
which are, in some vague sense, \'etale over the complement of
$V(I)$ in $\Spec(A)$.

\begin{lemma}
\label{lemma-get-morphism-general}
Let $A$ be a Noetherian ring. Let $I \subset A$ be an ideal.
Let $t$ be the minimal number of generators for $I$.
Let $C$ be an object of (\ref{equation-C-prime}).
There exists an integer $d \geq 0$ depending only on
$I \subset A \to C$ with the following property: given
\begin{enumerate}
\item $c \geq 0$ and $B$ in (\ref{equation-C-prime}) such that for $a \in I^c$
multiplication by $a$ on $\NL^\wedge_{B/A}$ is zero in $D(B)$,
\item an integer $n > 2t\max(c, d)$,
\item an $A_n$-algebra map $\psi_n : B_n \to C_n$,
\end{enumerate}
there exists a morphism $\varphi : B \to C$ of (\ref{equation-C-prime}) such
that $\psi_n \bmod I^{m - c} = \varphi \bmod I^{m - c}$
with $m = \lfloor \frac{n}{t} \rfloor$.
\end{lemma}

\begin{proof}
We prove this lemma by induction on the number of generators of $I$.
Say $I = (a_1, \ldots, a_t)$. If $t = 0$, then $I = 0$ and there
is nothing to prove. If $t = 1$, then the lemma follows from
Lemma \ref{lemma-get-morphism-principal} because
$2\max(c, d) \geq \max(2c, c + d)$. Assume $t > 1$.

\medskip\noindent
Set $m = \lfloor \frac{n}{t} \rfloor$ as in the lemma.
Set $\bar A = A/(a_t^m)$. Consider the ideal
$\bar I = (\bar a_1, \ldots, \bar a_{t - 1})$ in $\bar A$.
Let $\bar C = C/(a_t^m)$ be the base change of $C$ for the
map $(A, I) \to (\bar A, \bar I)$, see (\ref{equation-base-change-to-closed}).
Let $\bar d$ be the integer for $\bar I \subset \bar A \to \bar C$
which exists by induction hypothesis.

\medskip\noindent
Let $d_1 \geq 0$ be the integer found in
Lemma \ref{lemma-get-morphism-principal} for the ring $B$
and the element $a_t \in A$.

\medskip\noindent
We claim the lemma holds with $d = \max(\bar d, d_1)$.
To see this, let $c, B, n, \psi_n$ be as in the lemma.

\medskip\noindent
Note that $\bar I \subset I\bar A$. Hence by
Lemma \ref{lemma-zero-after-modding-out}
multiplication by an element of $\bar I^c$
on the cotangent complex of $\bar B = B/(a_t^m)$
is zero in $D(\bar B)$. Also, we have
$$
\bar I^{n - m + 1} \supset I^n \bar A
$$
Thus $\psi_n$ gives rise to a map
$$
\bar \psi_{n - m + 1} : \bar B_{n - m + 1} \to \bar C_{n - m + 1}
$$
Since $n > 2t\max(c, d)$ and $d \geq \bar d$ we see that
$$
n - m + 1 \geq (t - 1)n/t > 2(t - 1)\max(c, d) \geq 2(t - 1)\max(c, \bar d)
$$
Hence we can find a morphism $\varphi_m : \bar B \to \bar C$
agreeing with $\bar \psi_{n - m + 1}$ modulo the ideal
$\bar I^{m' - c}$ where $m' = \lfloor \frac{n - m + 1}{t - 1} \rfloor$.

\medskip\noindent
Since $m \geq n/t > 2\max(c, d) \geq 2\max(c, d_1) \geq \max(2c, c+ d_1)$,
we can apply Lemma \ref{lemma-get-morphism-principal} for
the ring map $A \to B$ and the ideal $(a_t)$ to
find a morphism $\varphi : B \to C$ agreeing modulo
$a_t^{m - c}$ with $\varphi_m$.

\medskip\noindent
All in all we find $\varphi : B \to C$ which agrees with
$\psi_n$ modulo
$$
(a_t^{m - c}) + (a_1, \ldots, a_{t - 1})^{m' - c}
\subset I^{\min(m - c, m' - c)}
$$
We leave it to the reader to see that
$\min(m - c, m' - c) = m - c$. This concludes the proof.
\end{proof}

\begin{lemma}
\label{lemma-lift-approximation}
Let $A$ be a Noetherian ring and $I \subset A$ an ideal.
Let $J \subset A$ be a nilpotent ideal. Consider a diagram
$$
\xymatrix{
C \ar[r] & C/JC \\
& B_0 \ar[u] \\
A \ar[r] \ar[uu] & A/J \ar[u]
}
$$
whose vertical arrows are of finite type such that
\begin{enumerate}
\item $\Spec(C) \to \Spec(A)$ is \'etale over $\Spec(A) \setminus V(I)$,
\item $\Spec(B_0) \to \Spec(A/J)$ is \'etale over
$\Spec(A/J) \setminus V((I + J)/J)$, and
\item $B_0 \to C/JC$ is \'etale and induces an isomorphism
$B_0/IB_0 = C/(I + J)C$.
\end{enumerate}
Then we can fill in the diagram
$$
\xymatrix{
C \ar[r] & C/JC \\
B \ar[u] \ar[r] & B_0 \ar[u] \\
A \ar[r] \ar[u] & A/J \ar[u]
}
$$
with $A \to B$ of finite type, $B/JB = B_0$, $B \to C$ \'etale, and
$\Spec(B) \to \Spec(A)$ \'etale over $\Spec(A) \setminus V(I)$.
\end{lemma}

\begin{proof}
By induction on the smallest $n$ such that $J^n = 0$ we reduce
to the case $J^2 = 0$. Denote by a subscript zero the base change
of objects to $A_0 = A/J$. Since $J^2 = 0$ we see that $JC$
is a $C_0$-module.

\medskip\noindent
Consider the canonical map
$$
\gamma : J \otimes_{A_0} C_0 \longrightarrow JC
$$
Since $\Spec(C) \to \Spec(A)$ is \'etale over the complement
of $V(I)$ (and hence flat) we see that $\gamma$ is an isomorphism
away from $V(IC_0)$, see
More on Morphisms of Schemes, Lemma \ref{more-morphisms-lemma-deform}.
In particular, the kernel and cokernel of $\gamma$ are annihilated by
a power of $I$ (use that $C_0$ is Noetherian and that the modules in
question are finite). Observe that $J \otimes_{A_0} C_0 =
(J \otimes_{A_0} B_0) \otimes_{B_0} C_0$. Hence by
More on Algebra, Lemma \ref{more-algebra-lemma-application-formal-glueing}
there exists a unique $B_0$-module homomorphism
$$
c : J \otimes_{A_0} B_0 \to N
$$
with $c \otimes \text{id}_{C_0} = \gamma$ and $\Ker(\gamma) = \Ker(c)$
and $\Coker(\gamma) = \Coker(c)$. Moreover, $N$ is a finite $B_0$-module, see
More on Algebra, Remark \ref{more-algebra-remark-formal-glueing-algebras}.

\medskip\noindent
Choose a presentation $B_0 = A[x_1, \ldots, x_r]/K$. To construct $B$
we try to find the dotted arrow $m$ fitting into the following
pushout diagram
$$
\xymatrix{
0 \ar[r] & N \ar@{..>}[r] & B \ar@{..>}[r] & B_0 \ar[r] & 0 \\
0 \ar[r] & K/K^2 \ar[r] \ar@{..>}[u]_m &
A[x_1, \ldots, x_r]/K^2 \ar[r] \ar@{..>}[u] &
A[x_1, \ldots, x_r]/K \ar@{=}[u] \ar[r] & 0 \\
& J \otimes_{A_0} B_0 \ar[u] \ar@/^2pc/[uu] |!{[lu];[u]}\hole
}
$$
where the curved arrow is the map $c$ constructed above and the
map $J \otimes_{A_0} B_0 \to K/K^2$ is the obvious one.

\medskip\noindent
As $B_0 \to C_0$ is \'etale we can write
$C_0 = B_0[y_1, \ldots, y_r]/(g_{0, 1}, \ldots, g_{0, r})$
such that the determinant of the partial derivatives of the $g_{0, j}$
is invertible in $C_0$, see
Algebra, Lemma \ref{algebra-lemma-etale-standard-smooth}.
We combine this with the chosen presentation of $B_0$ to
get a presentation $C_0 = A[x_1, \ldots, x_r, y_1, \ldots, y_s]/L$.
Choose a lift $\psi : A[x_i, y_j] \to C$
of the map to $C_0$. Then it is the case that $C$ fits into the diagram
$$
\xymatrix{
0 \ar[r] & JC \ar[r] & C \ar[r] & C_0 \ar[r] & 0 \\
0 \ar[r] & L/L^2 \ar[r] \ar[u]_\mu &
A[x_i, y_j]/L^2 \ar[r] \ar[u] &
A[x_i, y_j]/L \ar@{=}[u] \ar[r] & 0 \\
& J \otimes_{A_0} C_0 \ar[u] \ar@/^2pc/[uu] |!{[lu];[u]}\hole
}
$$
where the curved arrow is the map $\gamma$ constructed above and the
map $J \otimes_{A_0} C_0 \to L/L^2$ is the obvious one.
By our choice of presentations and the fact that $C_0$ is
a complete intersection over $B_0$ we have
$$
L/L^2 = K/K^2 \otimes_{B_0} C_0 \oplus \bigoplus C_0 g_j
$$
where $g_j \in L$ is any lift of $g_{0, j}$, see
More on Algebra, Lemma \ref{more-algebra-lemma-transitive-lci-at-end}.

\medskip\noindent
Consider the complex
$$
K^\bullet : J \otimes_{A_0} B_0 \to K/K^2 \to \bigoplus B_0 \text{d}x_i
$$
where the second arrow is the differential in the naive cotangent
complex of $B_0$ over $A$ for the given presentation and the last
term is placed in degree $0$. Since
$\Spec(B_0) \to \Spec(A_0)$ is \'etale away from $V(I)$
the cohomology modules of this complex are supported on
$V(IB_0)$. Namely, for $a \in I$ after inverting $a$
we can apply 
More on Algebra, Lemma \ref{more-algebra-lemma-transitive-lci-at-end}
for the ring maps $A_a \to A_{0, a} \to B_{0, a}$
and use that $\NL_{A_{0, a}/A_a} = J_a$ and
$\NL_{B_{0, a}/A_{0, a}} = 0$ (some details omitted).
Hence these cohomology groups are annihilated by a power of $I$.

\medskip\noindent
Similarly, consider the complex
$$
L^\bullet : J \otimes_{A_0} C_0 \to L/L^2 \to
\bigoplus C_0 \text{d}x_i \oplus \bigoplus C_0 \text{d}y_j
$$
By our direct sum decomposition of $L/L^2$ above and the fact
that the the determinant of the partial derivatives of the $g_{0, j}$
is invertible in $C_0$ we see that the natural map
$K^\bullet \to L^\bullet$ induces a quasi-isomorphism
$$
K^\bullet \otimes_{B_0} C_0 \longrightarrow L^\bullet
$$
The maps $\text{id}_{J \otimes_{A_0} C_0}$ and $\mu$ define
an element in
$$
\Hom_{D(C_0)}(L^\bullet, (J \otimes_{A_0} C_0 \to  JC))
$$
(the target two term complex is placed in degree $-2$ and $-1$)
such that the composition with the map to $J \otimes_{A_0} C_0[2]$
is the element in $\Hom_{D(C_0)}(L^\bullet, J \otimes_{A_0} C_0[2])$
corresponding to $\text{id}_{J \otimes_{A_0} C_0}$.
By
More on Algebra, Remark \ref{more-algebra-remark-neighbourhood-extensions}
(FIXME: remark needs to be revamped) we obtain a unique element
$$
\xi \in \Hom_{D(B_0)}(K^\bullet, (J \otimes_{A_0} B_0 \to N))
$$
whose composition with the map to $J \otimes_{A_0} B_0[2]$
is the element in $\Hom_{D(C_0)}(K^\bullet, J \otimes_{A_0} B_0[2])$
corresponding to $\text{id}_{J \otimes_{A_0} B_0}$. By
Lemma \ref{lemma-map-out-of-almost-free}
we can find a map of complexes $K^\bullet \to (J \otimes_{A_0} B_0 \to N)$
representing $\xi$ and equal to $\text{id}_{J \otimes_{A_0} B_0}$
in degree $-2$. Denote $m : K/K^2 \to N$ the degree $-1$ part
of this map.

\medskip\noindent
By Lemma \ref{lemma-map-out-of-almost-free} there exist maps
$h : L^{-1} \to J \otimes_{A_0} C_0$ and $h' : L^0 \to JC$
which define a homotopy between $(1, \mu, 0)$ and
$(1, m \otimes \oplus 0, 0)$ (FIXME: elucidate).
Since $h$ precomposed with $d_L^{-2}$ is zero it defines
an element in $\Hom_{D(C_0)}(L^\bullet, J \otimes_{A_0} C_0)$
which comes from a unique element $\chi$ of
$\Hom_{D(B_0)}(K^\bullet, J \otimes_{A_0} B_0)$
by More on Algebra, Remark \ref{more-algebra-remark-neighbourhood-extensions}
(FIXME: remark needs to be revamped).
Applying Lemma \ref{lemma-map-out-of-almost-free} again we represent $\chi$
by a map $g : K^{-1} \to J \otimes_{A_0} B_0$.
Then the base change $g \otimes \text{id}_{C_0}$ and $h$ differ
by a homotopy $h'' : L^0 \to J \otimes_{A_0} C$.
Hence if we modify $m$ into $m + c \circ h$, then
we find that $m \otimes \oplus 0$ and $\mu$ just differ by
a map $h' : L^0 \to JC$.

\medskip\noindent
Changing our choice of the map $\psi : A[x_i, y_j] \to C$
by sending $x_i$ to $\psi(x_i) + h'(\text{d}x_i)$ and sending
$y_j$ to $\psi(y_j) + h'(\text{d}y_j)$, we find a commutative
diagram
$$
\xymatrix{
N \ar[r] & JC \\
K/K^2 \ar[r] \ar[u]_m & L/L^2 \ar[u]_\mu \\
J \otimes_{A_0} B_0 \ar[u] \ar@/^2pc/[uu]^c \ar[r] &
J \otimes_{A_0} C_0 \ar[u] \ar@/_2pc/[uu]_\gamma
}
$$
At this point we can define $B$ as the pushout in the first
commutative diagram of the proof. The commutativity of the
diagram just displayed, shows that there is an $A$-algebra
map $B \to C$ compatible with the given map $N = JB \to JC$.
As $N \otimes_{B_0} C_) = JC$ it follows from
More on Morphisms, Lemma \ref{more-morphisms-lemma-deform}
that $B \to C$ is flat.
From this it easily follows that it is \'etale.
We omit the proof of the other properties as they are mostly
self evident at this point.
\end{proof}

\begin{lemma}
\label{lemma-approximate}
Let $A$ be a Noetherian ring. Let $I \subset A$ be an ideal.
Let $B$ be an object of (\ref{equation-C-prime}).
Let $c \geq 0$ be an integer such that for $a \in I^c$
multiplication by $a$ on $\NL^\wedge_{B/A}$ is zero in $D(B)$.
Then there exists a finite type $A$-algebra $C$ and an
isomorphism $B \cong C^\wedge$.
\end{lemma}

\begin{proof}
We prove this lemma by induction on the number of generators of $I$.
Say $I = (a_1, \ldots, a_t)$. If $t = 0$, then $I = 0$ and there
is nothing to prove. If $t = 1$, then the lemma follows from
Lemma \ref{lemma-approximate-principal}. Assume $t > 1$.

\medskip\noindent
For any $m \geq 1$ set $\bar A_m = A/(a_t^m)$. Consider the ideal
$\bar I_m = (\bar a_1, \ldots, \bar a_{t - 1})$ in $\bar A_m$.
Let $B_m = B/(a_t^m)$ be the base change of $B$ for the
map $(A, I) \to (\bar A_m, \bar I_m)$, see
(\ref{equation-base-change-to-closed}).
By Lemma \ref{lemma-zero-after-modding-out}
the assumption of the lemma holds for
$\bar I_m \subset \bar A_m \to B_m$.
Hence by induction hypothesis we find a finite type
$\bar A_m$-algebra $C_m$ and an isomorphism $B_m \cong C_m^\wedge$
where the completion is with respect to $\bar I_m$.
By Lemma \ref{lemma-rig-etale} we may assume that
$\Spec(C_m) \to \Spec(\bar A_m)$ is \'etale
over $\Spec(\bar A_m) \setminus V(\bar I_m)$.
By Lemma \ref{lemma-fully-faithful-etale-over-complement}
there exists an \'etale extension $C_m \to C'_m$ which induces
an isomorphism modulo $I$ and an $A$-algebra map $C_{m + 1} \to C'_m$
inducing the given map on completions.
By Lemma \ref{lemma-lift-approximation} this means we can succesively lift
$C_1$ to algebras $\tilde C_m$.
Finally, apply Lemma \ref{lemma-approximate-principal}
to the sequence $(\tilde C_m)$ to finish the proof.
FIXME: add lots more detail.
\end{proof}

\begin{proof}[Proof of Lemma \ref{lemma-approximate} in case $A$ is a G-ring]
This proof is easier in that it does not depend on the somewhat
delicate deformation theory argument given in the proof of
Lemma \ref{lemma-lift-approximation}, but of course it requires
a very strong assumption on the Noetherian ring $A$.

\medskip\noindent
Choose a presentation $B = A[x_1, \ldots, x_r]^\wedge/J$.
Choose generators $g_1, \ldots, g_m \in J$.
Choose generators $k_1, \ldots, k_t$ of the module
of relations between $g_1, \ldots, g_m$, i.e., such that
$$
(A[x_1, \ldots, x_r]^\wedge)^{\oplus t} \xrightarrow{k_1, \ldots, k_t}
(A[x_1, \ldots, x_r]^\wedge)^{\oplus m} \xrightarrow{g_1, \ldots, g_m}
A[x_1, \ldots, x_r]^\wedge
$$
is exact in the middle. Write $k_i = (k_{i1}, \ldots, k_{im})$ so that we have
\begin{equation}
\label{equation-relations-straight-up}
\sum k_{ij}g_j = 0
\end{equation}
for $i = 1, \ldots, t$.
Let $I^c = (a_1, \ldots, a_s)$. For each $l \in \{1, \ldots, s\}$
we know that multiplication by $a_l$ on $\NL^\wedge_{B/A}$ is zero
in $D(B)$. By Lemma \ref{lemma-zero-in-derived} we can find a map
$\alpha_l : \bigoplus B\text{d}x_i \to J/J^2$ such that
$\text{d} \circ \alpha_l$ and $\alpha_l \circ \text{d}$ are both
multiplication by $a_l$. Pick an element $f_{l, i} \in J$ whose
class modulo $J^2$ is equal to $\alpha_l(\text{d}x_i)$.
Then we have for all $l = 1, \ldots, s$ and $i = 1, \ldots, r$ that
\begin{equation}
\label{equation-derivatives}
\sum\nolimits_{i'} (\partial f_{l, i}/ \partial x_{i'}) \text{d}x_{i'} =
a_l \text{d}x_i + \sum h_{l, i}^{j', i'} g_{j'} \text{d}x_{i'}
\end{equation}
for some $h_{l, i}^{j', i'} \in A[x_1, \ldots, x_r]^\wedge$.
We also have for $j = 1, \ldots, m$ and $l = 1, \ldots, s$ that
\begin{equation}
\label{equation-ci}
a_l g_j = \sum h_{l, j}^if_{l, i} + \sum h_{l, j}^{j', j''}g_{j'} g_{j''}
\end{equation}
for some $h_{l, j}^i$ and $h_{l, j}^{j', j''}$ in
$A[x_1, \ldots, x_r]^\wedge$. Of course, since $f_{l, i} \in J$
we can write for $l = 1, \ldots, s$ and $i = 1, \ldots, r$
\begin{equation}
\label{equation-in-ideal}
f_{l, i} = \sum h_{l, i}^jg_j
\end{equation}
for some $h_{l, i}^j$ in $A[x_1, \ldots, x_r]^\wedge$.

\medskip\noindent
Let $A[x_1, \ldots, x_r]^h$ be the henselization of the
pair $(A[x_1, \ldots, x_r], IA[x_1, \ldots, x_r])$, see
More on Algebra, Lemma \ref{more-algebra-lemma-henselization}.
Since $A$ is a Noetherian G-ring, so is $A[x_1, \ldots, x_r]$, see
More on Algebra, Proposition
\ref{more-algebra-proposition-finite-type-over-G-ring}.
Hence we have approximation for the map
$A[x_1, \ldots, x_r]^h \to A[x_1, \ldots, x_r]^\wedge$
with respect to the ideal generated by $I$, see
Smoothing Ring Maps, Lemma \ref{smoothing-lemma-henselian-pair}.
Choose a large integer $M$. Choose
$$
G_j, K_{ij}, F_{l, i}, H_{l, j}^i, H_{l, j}^{j', j''}, H_{l, i}^j
\in A[x_1, \ldots, x_r]^h
$$
such that analogues of equations (\ref{equation-relations-straight-up}),
(\ref{equation-ci}), and (\ref{equation-in-ideal})
hold for these elements in $A[x_1, \ldots, x_r]^h$, i.e.,
$$
\sum K_{ij}G_j = 0,\quad
a_l G_j = \sum H_{l, j}^iF_{l, i} +
\sum H_{l, j}^{j', j''} G_{j'} G_{j''},\quad
F_{l, i} = \sum H_{l, i}^j G_j
$$
and such that we have
$$
G_j - g_j, K_{ij} - k_{ij}, F_{l, i} - f_{l, i},
H_{l, j}^i - h_{l, j}^i, H_{l, j}^{j', j''} - h_{l, j}^{j', j''},
H_{l, i}^j - h_{l, i}^j
\in I^M A[x_1, \ldots, x_r]^h
$$
where we take liberty of thinking of $A[x_1, \ldots, x_r]^h$ as a
subring of $A[x_1, \ldots, x_r]^\wedge$.
Note that we cannot guarantee that the analogue of
(\ref{equation-derivatives}) holds
in $A[x_1, \ldots, x_r]^h$, because it is not a polynomial equation.
But since taking partial derivatives is $A$-linear, we do get
the analogue modulo $I^M$. More precisely, we see that
\begin{equation}
\label{equation-derivatives-analogue}
\sum\nolimits_{i'} (\partial F_{l, i}/ \partial x_{i'}) \text{d}x_{i'}
- a_l \text{d}x_i - \sum h_{l, i}^{j', i'} G_{j'} \text{d}x_{i'}
\in I^MA[x_1, \ldots, x_r]^\wedge
\end{equation}
for $l = 1, \ldots, s$ and $i = 1, \ldots, r$.

\medskip\noindent
With these choices, consider the ring
$$
C^h = A[x_1, \ldots, x_r]^h/(G_1, \ldots, G_r)
$$
and denote $C^\wedge$ its $I$-adic completion, namely
$$
C^\wedge = A[x_1, \ldots, x_r]^\wedge/J',\quad
J' = (G_1, \ldots, G_r)A[x_1, \ldots, x_r]^\wedge
$$
In the following paragraphs we esthablish the fact that $C^\wedge$
is isomorphic to $B$. Then in the final paragraph we deal with
show that $C^h$ comes from a finite type algebra
over $A$ as in the statement of the lemma.

\medskip\noindent
First consider the cokernel
$$
\Omega = \Coker(J'/(J')^2 \longrightarrow \bigoplus C^\wedge \text{d}x_i)
$$
This $C^\wedge$ module is generated by the images of the elements
$\text{d}x_i$. Since $F_{l, i} \in J'$ by the analogue of
(\ref{equation-in-ideal}) we see from
(\ref{equation-derivatives-analogue}) we see
that $a_l \text{d}x_i \in I^M\Omega$. As $I^c = (a_l)$ we see that
$I^c \Omega \subset I^M \Omega$. Since $M > c$ we conclude that
$I^c \Omega = 0$ by Algebra, Lemma \ref{algebra-lemma-NAK}.

\medskip\noindent
Next, consider the kernel
$$
H_1 = \Ker(J'/(J')^2 \longrightarrow \bigoplus C^\wedge \text{d}x_i)
$$
By the analogue of (\ref{equation-ci}) we see that
$a_l J' \subset (F_{l, i}) + (J')^2$. On the other hand, the
determinant $\Delta_l$ of the matrix $(\partial F_{l, i}/ \partial x_{i'})$
satisfies $\Delta_l = a_l^r \bmod I^M C^\wedge$ by
(\ref{equation-derivatives-analogue}). It follows that
$a_l^{r + 1} H_1 \subset I^M H_1$ (some details omitted; use
Algebra, Lemma \ref{algebra-lemma-matrix-left-inverse}).
Now $(a_1^{r + 1}, \ldots, a_s^{r + 1}) \supset I^{(sr + 1)c}$.
Hence $I^{(sr + 1)c}H_1 \subset I^M H_1$ and since $M > (sr + 1)c$
we conclude that $I^{(sr + 1)c}H_1 = 0$.

\medskip\noindent
By Lemma \ref{lemma-zero-in-derived}
we conclude that multiplication by an element
of $I^{2(sr + 1)c}$ on $\NL^\wedge_{C^\wedge/A}$ is zero
(note that the bound does not depend on $M$ or the choice
of the approximation, as long as $M$ is large enough).
Since $G_j - g_j$ is in the ideal generated by $I^M$
we see that there is an isomorphism
$$
\psi_M : C^\wedge/I^MC^\wedge \to B/I^MB
$$
As $M$ is large enough we can use
Lemma \ref{lemma-get-morphism-general}
with $d = d(I \subset A \to B)$,
with $C^\wedge$ playing the role of $B$,
with $2(rs + 1)c$ instead of $c$,
to find a morphism
$$
\psi : C^\wedge \longrightarrow B
$$
which agrees with $\psi_M$ modulo $I^{q - 2(rs + 1)c}$ where
$q$ is the quotent of $M$ by the number of generators of $I$.
We claim $\psi$ is an isomorphism. Since $C^\wedge$ and $B$
are $I$-adically complete the map $\psi$ is surjective
because it is surjective modulo $I$ (see
Algebra, Lemma \ref{algebra-lemma-completion-generalities}).
On the other hand, as $M$ is large enough we see that
$$
\text{Gr}_I(C^\wedge) \cong \text{Gr}_I(B)
$$
as graded $\text{Gr}_I(A[x_1, \ldots, x_r]^\wedge)$-modules
by More on Algebra, Lemma \ref{more-algebra-lemma-approximate-complex-graded}.
Since $\psi$ is compatible with this isomorphism as it
agrees with $\psi_M$ modulo $I$, this means that $\text{Gr}_I(\psi)$ is an
isomorphism. As $C^\wedge$ and $B$ are
$I$-adically complete, it follows that $\psi$ is an isomorphism.

\medskip\noindent
This paragraph serves to deal with the issue that $C^h$
is not of finite type over $A$. Namely, the ring
$A[x_1, \ldots, x_r]^h$ is a filtered colimit of
\'etale $A[x_1, \ldots, x_r]$ algebras $A'$ such that
$A/I[x_1, \ldots, x_r] \to A'/IA'$ is an isomorphism
(see proof of More on Algebra, Lemma \ref{more-algebra-lemma-henselization}).
Pick an $A'$ such that $G_1, \ldots, G_m$ are the
images of $G'_1, \ldots, G'_m \in A'$.
Setting $C = A'/(G'_1, \ldots, G'_m)$ we get the finite
type algebra we were looking for.
\end{proof}

\begin{lemma}
\label{lemma-approximate-by-etale-over-complement}
Let $A$ be a Noetherian G-ring. Let $I \subset A$ be an ideal.
Let $B$ be an $I$-adically complete $A$-algebra with $A/I \to B/IB$
of finite type. The equivalent conditions of
Lemma \ref{lemma-equivalent-with-artin} are also equivalent to
\begin{enumerate}
\item[(5)]
\label{item-algebraize}
there exists a finite type $A$-algebra $C$ with
$\Spec(C) \to \Spec(A)$ is \'etale over $\Spec(A) \setminus V(I)$
such that $B \cong C^\wedge$.
\end{enumerate}
\end{lemma}

\begin{proof}
First, assume conditions (1) -- (4) hold. Then there exists
a finite type $A$-algebra $C$ with such that $B \cong C^\wedge$
by Lemma \ref{lemma-approximate}. In other words, $B_n = C/I^nC$.
The naive cotangent complex $\NL_{C/A}$ is a complex of finite type
$C$-modules and hence $H^{-1}$ and $H^0$ are finite $C$-modules.
By assumption there exists a $c \geq 0$ such that
$H^{-1}/I^nH^{-1}$ and $H^0/I^nH^0$ are annihilated by $I^c$
for some $n$. By Nakayama's lemma this means that
$I^cH^{-1}$ and $I^cH^0$ are annihilated by an element of the
form $f = 1 + x$ with $x \in IC$. After inverting $f$
(which does not change the quotients $B_n = C/I^nC$)
we see that $\NL_{C/A}$ has cohomology annihilated by $I^c$. Thus
$A \to C$ is \'etale at any prime of $C$ not lying over $V(I)$
by the definition of \'etale ring maps, see
Algebra, Definition \ref{algebra-definition-etale}.

\medskip\noindent
Conversely, assume that $A \to C$ of finite type is given such that
$\Spec(C) \to \Spec(A)$ is \'etale over $\Spec(A) \setminus V(I)$.
Then for every $a \in I$ there exists an $c \geq 0$ such that
multiplication by $a^c$ is zero $\NL_{C/A}$.
Since $\NL^\wedge_{C^\wedge/A}$ is the derived completion of
$\NL_{C/A}$ (see Lemma \ref{lemma-NL-is-limit}) it follows that
$B = C^\wedge$ satisfies the equivalent conditions of
Lemma \ref{lemma-equivalent-with-artin}.
\end{proof}

\begin{lemma}
\label{lemma-fully-faithfulness}
Let $A$ be a Noetherian G-ring. Let $I \subset A$ be an ideal.
Let $B, C$ be finite type $A$-algebras. For any $A$-algebra map
$\varphi : B^\wedge \to C^\wedge$ of $I$-adic completions and any
$N \geq 1$ there exist
\begin{enumerate}
\item an \'etale ring map $C \to C'$ which induces
an isomorphism $C/IC \to C'/IC'$,
\item an $A$-algebra map $\varphi : B \to C'$
\end{enumerate}
such that $\varphi$ and $\psi$ agree modulo $I^N$
into $C^\wedge = (C')^\wedge$.
\end{lemma}

\begin{proof}
The statement of the lemma makes sense as $C \to C'$ is flat
(Algebra, Lemma \ref{algebra-lemma-etale}) hence induces an isomorphism
$C/I^nC \to C'/I^nC'$ for all $n$
(More on Algebra, Lemma \ref{more-algebra-lemma-neighbourhood-isomorphism})
and hence an isomorphism on completions.
Let $C^h$ be the henselization of the pair $(C, IC)$, see
More on Algebra, Lemma \ref{more-algebra-lemma-henselization}.
Then $C^h$ is the filtered colimit of the algebras $C'$
and the maps
$C \to C' \to C^h$ induce isomorphism on completions (More on Algebra,
Lemma \ref{more-algebra-lemma-henselization-Noetherian-pair}).
Thus it suffices to prove there exists an $A$-algebra map
$B \to C^h$ which is congruent to $\psi$ modulo $I^N$.
Write $B = A[x_1, \ldots, x_n]/(f_1, \ldots, f_m)$.
The ring map $\psi$ corresponds to elements
$\hat c_1, \ldots, \hat c_n \in C^\wedge$ with
$f_j(\hat c_1, \ldots, \hat c_n) = 0$ for $j = 1, \ldots, m$.
Namely, as $A$ is a Noetherian G-ring, so is $C$, see
More on Algebra, Proposition
\ref{more-algebra-proposition-finite-type-over-G-ring}.
Thus Smoothing Ring Maps,
Lemma \ref{smoothing-lemma-henselian-pair}
applies to give elements $c_1, \ldots, c_n \in C^h$ such
that $f_j(c_1, \ldots, c_n) = 0$ for $j = 1, \ldots, m$
and such that $\hat c_i - c_i \in I^NC^h$.
This determines the map $B \to C^h$ as desired.
\end{proof}










\section{Other chapters}

\begin{multicols}{2}
\begin{enumerate}
\item \hyperref[introduction-section-phantom]{Introduction}
\item \hyperref[conventions-section-phantom]{Conventions}
\item \hyperref[sets-section-phantom]{Set Theory}
\item \hyperref[categories-section-phantom]{Categories}
\item \hyperref[topology-section-phantom]{Topology}
\item \hyperref[sheaves-section-phantom]{Sheaves on Spaces}
\item \hyperref[algebra-section-phantom]{Commutative Algebra}
\item \hyperref[sites-section-phantom]{Sites and Sheaves}
\item \hyperref[homology-section-phantom]{Homological Algebra}
\item \hyperref[derived-section-phantom]{Derived Categories}
\item \hyperref[more-algebra-section-phantom]{More Algebra}
\item \hyperref[simplicial-section-phantom]{Simplicial Methods}
\item \hyperref[modules-section-phantom]{Sheaves of Modules}
\item \hyperref[sites-modules-section-phantom]{Modules on Sites}
\item \hyperref[injectives-section-phantom]{Injectives}
\item \hyperref[cohomology-section-phantom]{Cohomology of Sheaves}
\item \hyperref[sites-cohomology-section-phantom]{Cohomology on Sites}
\item \hyperref[hypercovering-section-phantom]{Hypercoverings}
\item \hyperref[schemes-section-phantom]{Schemes}
\item \hyperref[constructions-section-phantom]{Constructions of Schemes}
\item \hyperref[properties-section-phantom]{Properties of Schemes}
\item \hyperref[morphisms-section-phantom]{Morphisms of Schemes}
\item \hyperref[coherent-section-phantom]{Coherent Cohomology}
\item \hyperref[divisors-section-phantom]{Divisors}
\item \hyperref[limits-section-phantom]{Limits of Schemes}
\item \hyperref[varieties-section-phantom]{Varieties}
\item \hyperref[chow-section-phantom]{Chow Homology}
\item \hyperref[topologies-section-phantom]{Topologies on Schemes}
\item \hyperref[descent-section-phantom]{Descent}
\item \hyperref[more-morphisms-section-phantom]{More on Morphisms}
\item \hyperref[flat-section-phantom]{More on Flatness}
\item \hyperref[groupoids-section-phantom]{Groupoid Schemes}
\item \hyperref[more-groupoids-section-phantom]{More on Groupoid Schemes}
\item \hyperref[etale-section-phantom]{\'Etale Morphisms of Schemes}
\item \hyperref[etale-cohomology-section-phantom]{\'Etale Cohomology}
\item \hyperref[spaces-section-phantom]{Algebraic Spaces}
\item \hyperref[spaces-properties-section-phantom]{Properties of Algebraic Spaces}
\item \hyperref[spaces-morphisms-section-phantom]{Morphisms of Algebraic Spaces}
\item \hyperref[spaces-topologies-section-phantom]{Topologies on Algebraic Spaces}
\item \hyperref[spaces-descent-section-phantom]{Descent and Algebraic Spaces}
\item \hyperref[spaces-more-morphisms-section-phantom]{More on Morphisms of Spaces}
\item \hyperref[quot-section-phantom]{Quot and Hilbert Spaces}
\item \hyperref[stacks-section-phantom]{Stacks}
\item \hyperref[spaces-groupoids-section-phantom]{Groupoids in Algebraic Spaces}
\item \hyperref[spaces-more-groupoids-section-phantom]{More on Groupoids in Spaces}
\item \hyperref[bootstrap-section-phantom]{Bootstrap}
\item \hyperref[examples-stacks-section-phantom]{Examples of Stacks}
\item \hyperref[groupoids-quotients-section-phantom]{Quotients of Groupoids}
\item \hyperref[algebraic-section-phantom]{Algebraic Stacks}
\item \hyperref[criteria-section-phantom]{Criteria for Representability}
\item \hyperref[stacks-properties-section-phantom]{Properties of Algebraic Stacks}
\item \hyperref[stacks-morphisms-section-phantom]{Morphisms of Algebraic Stacks}
\item \hyperref[examples-section-phantom]{Examples}
\item \hyperref[exercises-section-phantom]{Exercises}
\item \hyperref[guide-section-phantom]{Guide to Literature}
\item \hyperref[desirables-section-phantom]{Desirables}
\item \hyperref[coding-section-phantom]{Coding Style}
\item \hyperref[fdl-section-phantom]{GNU Free Documentation License}
\item \hyperref[index-section-phantom]{Auto Generated Index}
\end{enumerate}
\end{multicols}


\bibliography{my}
\bibliographystyle{amsalpha}

\end{document}

