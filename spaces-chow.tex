\IfFileExists{stacks-project.cls}{%
\documentclass{stacks-project}
}{%
\documentclass{amsart}
}

% The following AMS packages are automatically loaded with
% the amsart documentclass:
%\usepackage{amsmath}
%\usepackage{amssymb}
%\usepackage{amsthm}

% For dealing with references we use the comment environment
\usepackage{verbatim}
\newenvironment{reference}{\comment}{\endcomment}
%\newenvironment{reference}{}{}
\newenvironment{slogan}{\comment}{\endcomment}
\newenvironment{history}{\comment}{\endcomment}

% For commutative diagrams you can use
% \usepackage{amscd}
\usepackage[all]{xy}

% We use 2cell for 2-commutative diagrams.
\xyoption{2cell}
\UseAllTwocells

% To put source file link in headers.
% Change "template.tex" to "this_filename.tex"
% \usepackage{fancyhdr}
% \pagestyle{fancy}
% \lhead{}
% \chead{}
% \rhead{Source file: \url{template.tex}}
% \lfoot{}
% \cfoot{\thepage}
% \rfoot{}
% \renewcommand{\headrulewidth}{0pt}
% \renewcommand{\footrulewidth}{0pt}
% \renewcommand{\headheight}{12pt}

\usepackage{multicol}

% For cross-file-references
\usepackage{xr-hyper}

% Package for hypertext links:
\usepackage{hyperref}

% For any local file, say "hello.tex" you want to link to please
% use \externaldocument[hello-]{hello}
\externaldocument[introduction-]{introduction}
\externaldocument[conventions-]{conventions}
\externaldocument[sets-]{sets}
\externaldocument[categories-]{categories}
\externaldocument[topology-]{topology}
\externaldocument[sheaves-]{sheaves}
\externaldocument[sites-]{sites}
\externaldocument[stacks-]{stacks}
\externaldocument[fields-]{fields}
\externaldocument[algebra-]{algebra}
\externaldocument[brauer-]{brauer}
\externaldocument[homology-]{homology}
\externaldocument[derived-]{derived}
\externaldocument[simplicial-]{simplicial}
\externaldocument[more-algebra-]{more-algebra}
\externaldocument[smoothing-]{smoothing}
\externaldocument[modules-]{modules}
\externaldocument[sites-modules-]{sites-modules}
\externaldocument[injectives-]{injectives}
\externaldocument[cohomology-]{cohomology}
\externaldocument[sites-cohomology-]{sites-cohomology}
\externaldocument[dga-]{dga}
\externaldocument[dpa-]{dpa}
\externaldocument[hypercovering-]{hypercovering}
\externaldocument[schemes-]{schemes}
\externaldocument[constructions-]{constructions}
\externaldocument[properties-]{properties}
\externaldocument[morphisms-]{morphisms}
\externaldocument[coherent-]{coherent}
\externaldocument[divisors-]{divisors}
\externaldocument[limits-]{limits}
\externaldocument[varieties-]{varieties}
\externaldocument[topologies-]{topologies}
\externaldocument[descent-]{descent}
\externaldocument[perfect-]{perfect}
\externaldocument[more-morphisms-]{more-morphisms}
\externaldocument[flat-]{flat}
\externaldocument[groupoids-]{groupoids}
\externaldocument[more-groupoids-]{more-groupoids}
\externaldocument[etale-]{etale}
\externaldocument[chow-]{chow}
\externaldocument[intersection-]{intersection}
\externaldocument[pic-]{pic}
\externaldocument[adequate-]{adequate}
\externaldocument[dualizing-]{dualizing}
\externaldocument[duality-]{duality}
\externaldocument[discriminant-]{discriminant}
\externaldocument[local-cohomology-]{local-cohomology}
\externaldocument[curves-]{curves}
\externaldocument[resolve-]{resolve}
\externaldocument[models-]{models}
\externaldocument[pione-]{pione}
\externaldocument[etale-cohomology-]{etale-cohomology}
\externaldocument[proetale-]{proetale}
\externaldocument[crystalline-]{crystalline}
\externaldocument[spaces-]{spaces}
\externaldocument[spaces-properties-]{spaces-properties}
\externaldocument[spaces-morphisms-]{spaces-morphisms}
\externaldocument[decent-spaces-]{decent-spaces}
\externaldocument[spaces-cohomology-]{spaces-cohomology}
\externaldocument[spaces-limits-]{spaces-limits}
\externaldocument[spaces-divisors-]{spaces-divisors}
\externaldocument[spaces-over-fields-]{spaces-over-fields}
\externaldocument[spaces-topologies-]{spaces-topologies}
\externaldocument[spaces-descent-]{spaces-descent}
\externaldocument[spaces-perfect-]{spaces-perfect}
\externaldocument[spaces-more-morphisms-]{spaces-more-morphisms}
\externaldocument[spaces-flat-]{spaces-flat}
\externaldocument[spaces-groupoids-]{spaces-groupoids}
\externaldocument[spaces-more-groupoids-]{spaces-more-groupoids}
\externaldocument[bootstrap-]{bootstrap}
\externaldocument[spaces-pushouts-]{spaces-pushouts}
\externaldocument[groupoids-quotients-]{groupoids-quotients}
\externaldocument[spaces-more-cohomology-]{spaces-more-cohomology}
\externaldocument[spaces-simplicial-]{spaces-simplicial}
\externaldocument[formal-spaces-]{formal-spaces}
\externaldocument[restricted-]{restricted}
\externaldocument[spaces-resolve-]{spaces-resolve}
\externaldocument[formal-defos-]{formal-defos}
\externaldocument[defos-]{defos}
\externaldocument[cotangent-]{cotangent}
\externaldocument[examples-defos-]{examples-defos}
\externaldocument[algebraic-]{algebraic}
\externaldocument[examples-stacks-]{examples-stacks}
\externaldocument[stacks-sheaves-]{stacks-sheaves}
\externaldocument[criteria-]{criteria}
\externaldocument[artin-]{artin}
\externaldocument[quot-]{quot}
\externaldocument[stacks-properties-]{stacks-properties}
\externaldocument[stacks-morphisms-]{stacks-morphisms}
\externaldocument[stacks-limits-]{stacks-limits}
\externaldocument[stacks-cohomology-]{stacks-cohomology}
\externaldocument[stacks-perfect-]{stacks-perfect}
\externaldocument[stacks-introduction-]{stacks-introduction}
\externaldocument[stacks-more-morphisms-]{stacks-more-morphisms}
\externaldocument[stacks-geometry-]{stacks-geometry}
\externaldocument[moduli-]{moduli}
\externaldocument[moduli-curves-]{moduli-curves}
\externaldocument[examples-]{examples}
\externaldocument[exercises-]{exercises}
\externaldocument[guide-]{guide}
\externaldocument[desirables-]{desirables}
\externaldocument[coding-]{coding}
\externaldocument[obsolete-]{obsolete}
\externaldocument[fdl-]{fdl}
\externaldocument[index-]{index}

% Theorem environments.
%
\theoremstyle{plain}
\newtheorem{theorem}[subsection]{Theorem}
\newtheorem{proposition}[subsection]{Proposition}
\newtheorem{lemma}[subsection]{Lemma}

\theoremstyle{definition}
\newtheorem{definition}[subsection]{Definition}
\newtheorem{example}[subsection]{Example}
\newtheorem{exercise}[subsection]{Exercise}
\newtheorem{situation}[subsection]{Situation}

\theoremstyle{remark}
\newtheorem{remark}[subsection]{Remark}
\newtheorem{remarks}[subsection]{Remarks}

\numberwithin{equation}{subsection}

% Macros
%
\def\lim{\mathop{\rm lim}\nolimits}
\def\colim{\mathop{\rm colim}\nolimits}
\def\Spec{\mathop{\rm Spec}}
\def\Hom{\mathop{\rm Hom}\nolimits}
\def\Ext{\mathop{\rm Ext}\nolimits}
\def\SheafHom{\mathop{\mathcal{H}\!{\it om}}\nolimits}
\def\SheafExt{\mathop{\mathcal{E}\!{\it xt}}\nolimits}
\def\Sch{\textit{Sch}}
\def\Mor{\mathop{\rm Mor}\nolimits}
\def\Ob{\mathop{\rm Ob}\nolimits}
\def\Sh{\mathop{\textit{Sh}}\nolimits}
\def\NL{\mathop{N\!L}\nolimits}
\def\proetale{{pro\text{-}\acute{e}tale}}
\def\etale{{\acute{e}tale}}
\def\QCoh{\textit{QCoh}}
\def\Ker{\mathop{\rm Ker}}
\def\Im{\mathop{\rm Im}}
\def\Coker{\mathop{\rm Coker}}
\def\Coim{\mathop{\rm Coim}}

%
% Macros for moduli stacks/spaces
%
\def\QCohstack{\mathcal{QC}\!{\it oh}}
\def\Cohstack{\mathcal{C}\!{\it oh}}
\def\Spacesstack{\mathcal{S}\!{\it paces}}
\def\Quotfunctor{{\rm Quot}}
\def\Hilbfunctor{{\rm Hilb}}
\def\Curvesstack{\mathcal{C}\!{\it urves}}
\def\Polarizedstack{\mathcal{P}\!{\it olarized}}
\def\Complexesstack{\mathcal{C}\!{\it omplexes}}
% \Pic is the operator that assigns to X its picard group, usage \Pic(X)
% \Picardstack_{X/B} denotes the Picard stack of X over B
% \Picardfunctor_{X/B} denotes the Picard functor of X over B
\def\Pic{\mathop{\rm Pic}\nolimits}
\def\Picardstack{\mathcal{P}\!{\it ic}}
\def\Picardfunctor{{\rm Pic}}
\def\Deformationcategory{\mathcal{D}\!{\it ef}}


% OK, start here.
%
\begin{document}

\title{Chow Groups of Spaces}

\maketitle

\phantomsection
\label{section-phantom}


\tableofcontents


\section{Introduction}
\label{section-introduction}

\noindent
In this chapter we first discuss Chow groups of algebraic spaces.
Having defined these, we define chern classes of vector bundles as
operators on these chow groups. The strategy will be entirely
the same as the strategy in the case of schemes. Therefore we
urge the reader to take a look at the introduction
(Chow Homology, Section \ref{chow-section-introduction})
of the corresponding chapter for schemes.

\medskip\noindent
Some related papers: \cite{edidin-graham} and \cite{kresch_cycle}.



\section{Setup}
\label{section-setup}

\noindent
We first fix the category of algebraic spaces we will be working with.
Please keep in mind throughout this chapter that
``decent $+$ locally Noetherian'' is the same as
``quasi-separated $+$ locally Noetherian'' according to
Decent Spaces, Lemma
\ref{decent-spaces-lemma-locally-Noetherian-decent-quasi-separated}.

\begin{situation}
\label{situation-setup}
Here $S$ is a scheme and $B$ is an algebraic space over $S$.
We assume $B$ is quasi-separated, locally Noetherian, and
universally catenary (Decent Spaces, Definition
\ref{decent-spaces-definition-universally-catenary}).
Moreover, we assume given a dimension function
$\delta : |B| \longrightarrow \mathbf{Z}$.
We say $X/B$ is {\it good} if $X$ is an algebraic space
over $B$ whose structure morphism $f : X \to B$ is
quasi-separated and locally of finite type.
In this case we define
$$
\delta = \delta_{X/B} : |X| \longrightarrow \mathbf{Z}
$$
as the map sending $x$ to $\delta(f(x))$ plus the transcendence degree
of $x/f(x)$ (Morphisms of Spaces, Definition
\ref{spaces-morphisms-definition-dimension-fibre}).
This is a dimension function by
More on Morphisms of Spaces, Lemma
\ref{spaces-more-morphisms-lemma-universally-catenary-dimension-function}.
\end{situation}

\noindent
A special case is when $S = B$ is a scheme and $(S, \delta)$ is as in
Chow Homology, Situation \ref{chow-situation-setup}. Thus $B$ might be
the spectrum of a field (Chow Homology, Example \ref{chow-example-field})
or $B = \Spec(\mathbf{Z})$
(Chow Homology, Example \ref{chow-example-domain-dimension-1}).

\medskip\noindent
Many lemma, proposition, theorems, definitions on algebraic spaces
are easier in the setting of Situation \ref{situation-setup} because
the algebraic spaces we are working with are quasi-separated
(and thus a fortiori decent) and locally Noetherian. We will sprinkle
this chapter with remarks such as the following to point this out.

\begin{remark}
\label{remark-sober}
In Situation \ref{situation-setup} if $X/B$ is good, then
$|X|$ is a sober topological space. See
Properties of Spaces, Lemma \ref{spaces-properties-lemma-quasi-separated-sober}
or Decent Spaces, Proposition \ref{decent-spaces-proposition-reasonable-sober}.
We will use this without further mention
to choose generic points of closed subsets.
\end{remark}

\begin{remark}
\label{remark-integral}
In Situation \ref{situation-setup} if $X/B$ is good, then
$X$ is integral (Spaces over Fields, Definition
\ref{spaces-over-fields-definition-integral-algebraic-space})
if and only if $X$ is reduced and $|X|$ is irreducible.
Moreover, for any point $\xi \in |X|$ there is a unique integral closed
subspace $Z \subset X$ such that $\xi$ is the generic point
of the closed subset $|Z| \subset |X|$, namely, we can take
$Z$ to be the reduced induced algebraic space structure on
$\overline{\{\xi\}}$ of Properties of Spaces, Definition
\ref{spaces-properties-definition-reduced-induced-space}
and this will be an integral algebraic space by what we just said.
\end{remark}

\medskip\noindent
If $B$ is Jacobson and $\delta$ sends closed points to zero, then $\delta$
is the function sending a point to the dimension of its closure.

\begin{lemma}
\label{lemma-delta-is-dimension}
In Situation \ref{situation-setup} assume $B$ is Jacobson
and that $\delta(b) = 0$ for every closed point $b$ of $|B|$.
Let $X/B$ be good. If $Z \subset X$ is an integral closed subspace
with generic point $\xi \in |Z|$, then the following integers are the same:
\begin{enumerate}
\item $\delta(\xi) = \delta_{X/B}(\xi)$,
\item $\dim(|Z|)$,
\item $\text{codim}(\{z\}, |Z|)$ for $z \in |Z|$ closed,
\item the dimension of the local ring of $Z$ at $z$ for
$z \in |Z|$ closed, and
\item $\dim(\mathcal{O}_{Z, \overline{z}})$ for $z \in |Z|$ closed.
\end{enumerate}
\end{lemma}

\begin{proof}
Let $X$, $Z$, $\xi$ be as in the lemma.
Since $X$ is locally of finite type over $B$ we see that $X$ is Jacobson, see
Decent Spaces, Lemma
\ref{decent-spaces-lemma-Jacobson-universally-Jacobson}.
Hence $X_{\text{ft-pts}} \subset |X|$ is the set of closed points
by Decent Spaces, Lemma \ref{decent-spaces-lemma-decent-Jacobson-ft-pts}.
Given a chain $T_0 \supset \ldots \supset T_e$
of irreducible closed subsets of $|Z|$ we have
$T_e \cap X_{\text{ft-pts}}$ nonempty by
Morphisms of Spaces, Lemma
\ref{spaces-morphisms-lemma-enough-finite-type-points}.
Thus we can always assume such a chain ends
with $T_e = \{z\}$ for some $z \in |Z|$ closed.
It follows that $\dim(Z) = \sup_z \text{codim}(\{z\}, |Z|)$
where $z$ runs over the closed points of $|Z|$.
We have $\text{codim}(\{z\}, Z) = \delta(\xi) - \delta(z)$
by Topology, Lemma \ref{topology-lemma-dimension-function-catenary}.
By Morphisms of Spaces, Lemma
\ref{spaces-morphisms-lemma-finite-type-points-morphism}
the image of $z$ is a finite type point of $B$, i.e.,
a closed point of $|B|$. By
Morphisms of Spaces, Lemma
\ref{spaces-morphisms-lemma-jacobson-finite-type-points}
the transcendence degree of $z/b$ is $0$.
We conclude that $\delta(z) = \delta(b) = 0$ by assumption.
Thus we obtain equality
$$
\dim(|Z|) = \text{codim}(\{z\}, Z) = \delta(\xi)
$$
for all $z \in |Z|$ closed. Finally, we have that
$\text{codim}(\{z\}, Z)$ is equal to the dimension of the
local ring of $Z$ at $z$ by
Decent Spaces, Lemma \ref{decent-spaces-lemma-codimension-local-ring}
which in turn is equal to
$\dim(\mathcal{O}_{Z, \overline{z}})$ by
Properties of Spaces, Lemma \ref{spaces-properties-lemma-dimension-local-ring}.
\end{proof}

\noindent
In the situation of the lemma above the value of $\delta$
at the generic point of a closed irreducible subset
is the dimension of the irreducible closed subset.
This motivates the following definition.

\begin{definition}
\label{definition-delta-dimension}
In Situation \ref{situation-setup} for any good $X/B$
and any irreducible closed subset $T \subset |X|$ we define
$$
\dim_\delta(T) = \delta(\xi)
$$
where $\xi \in T$ is the generic point of $T$.
We will call this the {\it $\delta$-dimension of $Z$}.
If $Z$ is a closed subspace of $X$, then we define
$\dim_\delta(Z)$ as the supremum of the $\delta$-dimensions
of the irreducible components of $|Z|$.
\end{definition}








\section{Cycles}
\label{section-cycles}

\noindent
Since we are not assuming our spaces are quasi-compact we have
to be a little careful when defining cycles. We have to allow
infinite sums because a rational function may have infinitely many
poles for example. In any case, if $X$ is quasi-compact then a
cycle is a finite sum as usual.

\begin{definition}
\label{definition-cycles}
In Situation \ref{situation-setup} let $X/B$ be good.
Let $k \in \mathbf{Z}$.
\begin{enumerate}
\item A {\it cycle on $X$} is a formal sum
$$
\alpha = \sum n_Z [Z]
$$
where the sum is over integral closed subspaces $Z \subset X$,
each $n_Z \in \mathbf{Z}$, and
$\{|Z|; n_Z \not = 0\}$ is a locally finite
collection of subsets of $|X|$
(Topology, Definition \ref{topology-definition-locally-finite}).
\item A {\it $k$-cycle} on $X$ is
a cycle
$$
\alpha = \sum n_Z [Z]
$$
where $n_Z \not = 0 \Rightarrow \dim_\delta(Z) = k$.
\item The abelian group of all $k$-cycles on $X$ is denoted $Z_k(X)$.
\end{enumerate}
\end{definition}

\noindent
In other words, a $k$-cycle on $X$ is a locally finite formal
$\mathbf{Z}$-linear combination of integral closed subspaces
(Remark \ref{remark-integral}) of $\delta$-dimension $k$.
Addition of $k$-cycles $\alpha = \sum n_Z[Z]$ and
$\beta = \sum m_Z[Z]$ is given by
$$
\alpha + \beta = \sum (n_Z + m_Z)[Z],
$$
i.e., by adding the coefficients.












\section{Other chapters}

\begin{multicols}{2}
\begin{enumerate}
\item \hyperref[introduction-section-phantom]{Introduction}
\item \hyperref[conventions-section-phantom]{Conventions}
\item \hyperref[sets-section-phantom]{Set Theory}
\item \hyperref[categories-section-phantom]{Categories}
\item \hyperref[topology-section-phantom]{Topology}
\item \hyperref[sheaves-section-phantom]{Sheaves on Spaces}
\item \hyperref[algebra-section-phantom]{Commutative Algebra}
\item \hyperref[sites-section-phantom]{Sites and Sheaves}
\item \hyperref[homology-section-phantom]{Homological Algebra}
\item \hyperref[derived-section-phantom]{Derived Categories}
\item \hyperref[more-algebra-section-phantom]{More Algebra}
\item \hyperref[simplicial-section-phantom]{Simplicial Methods}
\item \hyperref[modules-section-phantom]{Sheaves of Modules}
\item \hyperref[sites-modules-section-phantom]{Modules on Sites}
\item \hyperref[injectives-section-phantom]{Injectives}
\item \hyperref[cohomology-section-phantom]{Cohomology of Sheaves}
\item \hyperref[sites-cohomology-section-phantom]{Cohomology on Sites}
\item \hyperref[hypercovering-section-phantom]{Hypercoverings}
\item \hyperref[schemes-section-phantom]{Schemes}
\item \hyperref[constructions-section-phantom]{Constructions of Schemes}
\item \hyperref[properties-section-phantom]{Properties of Schemes}
\item \hyperref[morphisms-section-phantom]{Morphisms of Schemes}
\item \hyperref[coherent-section-phantom]{Coherent Cohomology}
\item \hyperref[divisors-section-phantom]{Divisors}
\item \hyperref[limits-section-phantom]{Limits of Schemes}
\item \hyperref[varieties-section-phantom]{Varieties}
\item \hyperref[chow-section-phantom]{Chow Homology}
\item \hyperref[topologies-section-phantom]{Topologies on Schemes}
\item \hyperref[descent-section-phantom]{Descent}
\item \hyperref[more-morphisms-section-phantom]{More on Morphisms}
\item \hyperref[flat-section-phantom]{More on Flatness}
\item \hyperref[groupoids-section-phantom]{Groupoid Schemes}
\item \hyperref[more-groupoids-section-phantom]{More on Groupoid Schemes}
\item \hyperref[etale-section-phantom]{\'Etale Morphisms of Schemes}
\item \hyperref[etale-cohomology-section-phantom]{\'Etale Cohomology}
\item \hyperref[spaces-section-phantom]{Algebraic Spaces}
\item \hyperref[spaces-properties-section-phantom]{Properties of Algebraic Spaces}
\item \hyperref[spaces-morphisms-section-phantom]{Morphisms of Algebraic Spaces}
\item \hyperref[spaces-topologies-section-phantom]{Topologies on Algebraic Spaces}
\item \hyperref[spaces-descent-section-phantom]{Descent and Algebraic Spaces}
\item \hyperref[spaces-more-morphisms-section-phantom]{More on Morphisms of Spaces}
\item \hyperref[quot-section-phantom]{Quot and Hilbert Spaces}
\item \hyperref[stacks-section-phantom]{Stacks}
\item \hyperref[spaces-groupoids-section-phantom]{Groupoids in Algebraic Spaces}
\item \hyperref[spaces-more-groupoids-section-phantom]{More on Groupoids in Spaces}
\item \hyperref[bootstrap-section-phantom]{Bootstrap}
\item \hyperref[examples-stacks-section-phantom]{Examples of Stacks}
\item \hyperref[groupoids-quotients-section-phantom]{Quotients of Groupoids}
\item \hyperref[algebraic-section-phantom]{Algebraic Stacks}
\item \hyperref[criteria-section-phantom]{Criteria for Representability}
\item \hyperref[stacks-properties-section-phantom]{Properties of Algebraic Stacks}
\item \hyperref[stacks-morphisms-section-phantom]{Morphisms of Algebraic Stacks}
\item \hyperref[examples-section-phantom]{Examples}
\item \hyperref[exercises-section-phantom]{Exercises}
\item \hyperref[guide-section-phantom]{Guide to Literature}
\item \hyperref[desirables-section-phantom]{Desirables}
\item \hyperref[coding-section-phantom]{Coding Style}
\item \hyperref[fdl-section-phantom]{GNU Free Documentation License}
\item \hyperref[index-section-phantom]{Auto Generated Index}
\end{enumerate}
\end{multicols}


\bibliography{my}
\bibliographystyle{amsalpha}

\end{document}
