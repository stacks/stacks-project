\IfFileExists{stacks-project.cls}{%
\documentclass{stacks-project}
}{%
\documentclass{amsart}
}

% The following AMS packages are automatically loaded with
% the amsart documentclass:
%\usepackage{amsmath}
%\usepackage{amssymb}
%\usepackage{amsthm}

% For dealing with references we use the comment environment
\usepackage{verbatim}
\newenvironment{reference}{\comment}{\endcomment}
%\newenvironment{reference}{}{}
\newenvironment{slogan}{\comment}{\endcomment}
\newenvironment{history}{\comment}{\endcomment}

% For commutative diagrams you can use
% \usepackage{amscd}
\usepackage[all]{xy}

% We use 2cell for 2-commutative diagrams.
\xyoption{2cell}
\UseAllTwocells

% To put source file link in headers.
% Change "template.tex" to "this_filename.tex"
% \usepackage{fancyhdr}
% \pagestyle{fancy}
% \lhead{}
% \chead{}
% \rhead{Source file: \url{template.tex}}
% \lfoot{}
% \cfoot{\thepage}
% \rfoot{}
% \renewcommand{\headrulewidth}{0pt}
% \renewcommand{\footrulewidth}{0pt}
% \renewcommand{\headheight}{12pt}

\usepackage{multicol}

% For cross-file-references
\usepackage{xr-hyper}

% Package for hypertext links:
\usepackage{hyperref}

% For any local file, say "hello.tex" you want to link to please
% use \externaldocument[hello-]{hello}
\externaldocument[introduction-]{introduction}
\externaldocument[conventions-]{conventions}
\externaldocument[sets-]{sets}
\externaldocument[categories-]{categories}
\externaldocument[topology-]{topology}
\externaldocument[sheaves-]{sheaves}
\externaldocument[sites-]{sites}
\externaldocument[stacks-]{stacks}
\externaldocument[fields-]{fields}
\externaldocument[algebra-]{algebra}
\externaldocument[brauer-]{brauer}
\externaldocument[homology-]{homology}
\externaldocument[derived-]{derived}
\externaldocument[simplicial-]{simplicial}
\externaldocument[more-algebra-]{more-algebra}
\externaldocument[smoothing-]{smoothing}
\externaldocument[modules-]{modules}
\externaldocument[sites-modules-]{sites-modules}
\externaldocument[injectives-]{injectives}
\externaldocument[cohomology-]{cohomology}
\externaldocument[sites-cohomology-]{sites-cohomology}
\externaldocument[dga-]{dga}
\externaldocument[dpa-]{dpa}
\externaldocument[hypercovering-]{hypercovering}
\externaldocument[schemes-]{schemes}
\externaldocument[constructions-]{constructions}
\externaldocument[properties-]{properties}
\externaldocument[morphisms-]{morphisms}
\externaldocument[coherent-]{coherent}
\externaldocument[divisors-]{divisors}
\externaldocument[limits-]{limits}
\externaldocument[varieties-]{varieties}
\externaldocument[topologies-]{topologies}
\externaldocument[descent-]{descent}
\externaldocument[perfect-]{perfect}
\externaldocument[more-morphisms-]{more-morphisms}
\externaldocument[flat-]{flat}
\externaldocument[groupoids-]{groupoids}
\externaldocument[more-groupoids-]{more-groupoids}
\externaldocument[etale-]{etale}
\externaldocument[chow-]{chow}
\externaldocument[intersection-]{intersection}
\externaldocument[pic-]{pic}
\externaldocument[adequate-]{adequate}
\externaldocument[dualizing-]{dualizing}
\externaldocument[duality-]{duality}
\externaldocument[discriminant-]{discriminant}
\externaldocument[local-cohomology-]{local-cohomology}
\externaldocument[curves-]{curves}
\externaldocument[resolve-]{resolve}
\externaldocument[models-]{models}
\externaldocument[pione-]{pione}
\externaldocument[etale-cohomology-]{etale-cohomology}
\externaldocument[proetale-]{proetale}
\externaldocument[crystalline-]{crystalline}
\externaldocument[spaces-]{spaces}
\externaldocument[spaces-properties-]{spaces-properties}
\externaldocument[spaces-morphisms-]{spaces-morphisms}
\externaldocument[decent-spaces-]{decent-spaces}
\externaldocument[spaces-cohomology-]{spaces-cohomology}
\externaldocument[spaces-limits-]{spaces-limits}
\externaldocument[spaces-divisors-]{spaces-divisors}
\externaldocument[spaces-over-fields-]{spaces-over-fields}
\externaldocument[spaces-topologies-]{spaces-topologies}
\externaldocument[spaces-descent-]{spaces-descent}
\externaldocument[spaces-perfect-]{spaces-perfect}
\externaldocument[spaces-more-morphisms-]{spaces-more-morphisms}
\externaldocument[spaces-flat-]{spaces-flat}
\externaldocument[spaces-groupoids-]{spaces-groupoids}
\externaldocument[spaces-more-groupoids-]{spaces-more-groupoids}
\externaldocument[bootstrap-]{bootstrap}
\externaldocument[spaces-pushouts-]{spaces-pushouts}
\externaldocument[groupoids-quotients-]{groupoids-quotients}
\externaldocument[spaces-more-cohomology-]{spaces-more-cohomology}
\externaldocument[spaces-simplicial-]{spaces-simplicial}
\externaldocument[formal-spaces-]{formal-spaces}
\externaldocument[restricted-]{restricted}
\externaldocument[spaces-resolve-]{spaces-resolve}
\externaldocument[formal-defos-]{formal-defos}
\externaldocument[defos-]{defos}
\externaldocument[cotangent-]{cotangent}
\externaldocument[examples-defos-]{examples-defos}
\externaldocument[algebraic-]{algebraic}
\externaldocument[examples-stacks-]{examples-stacks}
\externaldocument[stacks-sheaves-]{stacks-sheaves}
\externaldocument[criteria-]{criteria}
\externaldocument[artin-]{artin}
\externaldocument[quot-]{quot}
\externaldocument[stacks-properties-]{stacks-properties}
\externaldocument[stacks-morphisms-]{stacks-morphisms}
\externaldocument[stacks-limits-]{stacks-limits}
\externaldocument[stacks-cohomology-]{stacks-cohomology}
\externaldocument[stacks-perfect-]{stacks-perfect}
\externaldocument[stacks-introduction-]{stacks-introduction}
\externaldocument[stacks-more-morphisms-]{stacks-more-morphisms}
\externaldocument[stacks-geometry-]{stacks-geometry}
\externaldocument[moduli-]{moduli}
\externaldocument[moduli-curves-]{moduli-curves}
\externaldocument[examples-]{examples}
\externaldocument[exercises-]{exercises}
\externaldocument[guide-]{guide}
\externaldocument[desirables-]{desirables}
\externaldocument[coding-]{coding}
\externaldocument[obsolete-]{obsolete}
\externaldocument[fdl-]{fdl}
\externaldocument[index-]{index}

% Theorem environments.
%
\theoremstyle{plain}
\newtheorem{theorem}[subsection]{Theorem}
\newtheorem{proposition}[subsection]{Proposition}
\newtheorem{lemma}[subsection]{Lemma}

\theoremstyle{definition}
\newtheorem{definition}[subsection]{Definition}
\newtheorem{example}[subsection]{Example}
\newtheorem{exercise}[subsection]{Exercise}
\newtheorem{situation}[subsection]{Situation}

\theoremstyle{remark}
\newtheorem{remark}[subsection]{Remark}
\newtheorem{remarks}[subsection]{Remarks}

\numberwithin{equation}{subsection}

% Macros
%
\def\lim{\mathop{\rm lim}\nolimits}
\def\colim{\mathop{\rm colim}\nolimits}
\def\Spec{\mathop{\rm Spec}}
\def\Hom{\mathop{\rm Hom}\nolimits}
\def\Ext{\mathop{\rm Ext}\nolimits}
\def\SheafHom{\mathop{\mathcal{H}\!{\it om}}\nolimits}
\def\SheafExt{\mathop{\mathcal{E}\!{\it xt}}\nolimits}
\def\Sch{\textit{Sch}}
\def\Mor{\mathop{\rm Mor}\nolimits}
\def\Ob{\mathop{\rm Ob}\nolimits}
\def\Sh{\mathop{\textit{Sh}}\nolimits}
\def\NL{\mathop{N\!L}\nolimits}
\def\proetale{{pro\text{-}\acute{e}tale}}
\def\etale{{\acute{e}tale}}
\def\QCoh{\textit{QCoh}}
\def\Ker{\mathop{\rm Ker}}
\def\Im{\mathop{\rm Im}}
\def\Coker{\mathop{\rm Coker}}
\def\Coim{\mathop{\rm Coim}}

%
% Macros for moduli stacks/spaces
%
\def\QCohstack{\mathcal{QC}\!{\it oh}}
\def\Cohstack{\mathcal{C}\!{\it oh}}
\def\Spacesstack{\mathcal{S}\!{\it paces}}
\def\Quotfunctor{{\rm Quot}}
\def\Hilbfunctor{{\rm Hilb}}
\def\Curvesstack{\mathcal{C}\!{\it urves}}
\def\Polarizedstack{\mathcal{P}\!{\it olarized}}
\def\Complexesstack{\mathcal{C}\!{\it omplexes}}
% \Pic is the operator that assigns to X its picard group, usage \Pic(X)
% \Picardstack_{X/B} denotes the Picard stack of X over B
% \Picardfunctor_{X/B} denotes the Picard functor of X over B
\def\Pic{\mathop{\rm Pic}\nolimits}
\def\Picardstack{\mathcal{P}\!{\it ic}}
\def\Picardfunctor{{\rm Pic}}
\def\Deformationcategory{\mathcal{D}\!{\it ef}}


% OK, start here.
%
\begin{document}

\title{Homological Algebra}

%\begin{abstract}
%\end{abstract}

\maketitle

\tableofcontents

\section{Introduction}
\label{section-introduction}

\noindent
Basic homological algebra will be explained in this document.
A reference is \cite{Maclane}.

\section{Basic notions}
\label{section-topology-basic}

\noindent
The following notions are considered basic and will not be defined,
and or proved. This does not mean they are all necessarily easy or 
well known.

\begin{enumerate}
\item Nothing yet.
\end{enumerate}


\section{Abelian categories}
\label{section-abelian-categories}

\noindent
An abelian category will be a category satisfying
just enough axioms so the snake lemma holds.

\begin{definition}
\label{definition-preadditive}
A category $\mathcal{A}$ is called {\it preadditive} if each
morphism set $\text{Mor}_{\mathcal{A}}(x, y)$ is endowed
with the structure of an abelian group such that the
compositions
$$
\text{Mor}(x, y) \times \text{Mor}(y, z)
\longrightarrow
\text{Mor}(x, z)
$$
are bilinear. A functor $F : \mathcal{A} \to \mathcal{B}$ of
preadditive categories is called {\it additive} if and only
if $F : \text{Mor}(x, y) \to \text{Mor}(F(x), F(y))$
is a homomorphism of abelian groups for all
$x, y \in \text{Ob}(\mathcal{A})$.
\end{definition}

\noindent
In particular for every $x, y$ there exists at least
one morphism $x \to y$, namely the zero map.

\begin{lemma}
\label{lemma-preadditive-zero}
Let $\mathcal{A}$ be a preadditive category.
If $\mathcal{A}$ has an initial object, then that object
is also a final object. If $\mathcal{A}$ has a final object, then that object
is also a initial object. Furthermore, if such an object $0$ exists,
then a morphism $\alpha : x \to y$ factors through $0$ if and only if
$\alpha = 0$.
\end{lemma}

\begin{proof}
Omitted.
\end{proof}

\begin{definition}
\label{definition-zero-object}
In a preadditive category $\mathcal{A}$ we call
{\it zero object}, and we denote it $0$
any final and initial object as in Lemma \ref{lemma-preadditive-zero} above.
\end{definition}

\begin{lemma}
\label{lemma-preadditive-direct-sum}
Let $\mathcal{A}$ be a preadditive category.
Let $x, y \in \text{Ob}(\mathcal{A})$.
If the product $x \times y$ exists, then so does
the coproduct $x \coprod y$.
If the product $x \times y$ exists, then so does
the coproduct $x \coprod y$. In this case
also $x \coprod y \cong x \times y$.
\end{lemma}

\begin{proof}
Suppose that $z = x \times y$ with projections
$p : z \to x$ and $q : z \to y$. Denote $i : x \to z$
the morphism corresponding to $(1, 0)$. Denote $j : y \to z$
the morphism corresponding to $(0, 1)$. Thus we have the
commutative diagram
$$
\xymatrix{
x \ar[rr]^1 \ar[rd]^i & & x \\
& z \ar[ru]^p \ar[rd]^q & \\
y \ar[rr]^1 \ar[ru]^j & & y
}
$$
where the diagonal compositions are zero.
Suppose given morphisms $a : x \to w$ and $b : y \to w$.
Then we can form the map $a \circ p + b \circ q : z \to w$.
In this way we get a bijection $\text{Mor}(z, w)
= \text{Mor}(x, w) \times \text{Mor}(y, w)$ which
show that $z = x \coprod y$.

\medskip\noindent
We leave it to the reader to construct the morphisms
$p,q$ given a coproduct $x \coprod y$ instead of a
product.
\end{proof}

\begin{definition}
\label{definition-direct-sum}
Given a pair of objects $x, y$
in a preadditive category $\mathcal{A}$ we call
{\it direct sum}, and we denote it $x \oplus y$ the
product $x \times y$ endowed with the morphisms
$i,j,p,q$ as in Lemma \ref{lemma-preadditive-direct-sum} above.
\end{definition}

\begin{definition}
\label{definition-additive-category}
A category $\mathcal{A}$ is called {\it additive}
if it is preadditive and finite products exist, in other
words it has a zero object and direct sums.
\end{definition}

\noindent
Namely the empty product is a finite product and
if it exists, then it is a final object.

\begin{definition}
\label{definition-kernel}
Let $\mathcal{A}$ be a preadditive category.
Let $f : x \to y$ be a morphism.
\begin{enumerate}
\item A {\it kernel} of $f$ is a morphism
$i : z \to x$ such that (a) $f \circ i = 0$ and (b)
for any $i' : z' \to x$ such that $f \circ i' = 0$ there
exists a unique morphism $g : z' \to z$ such that
$i' = i \circ g$.
\item If the kernel of $f$ exists, then we denote
this $\text{Ker}(f) \to x$.
\item A {\it cokernel} of $f$ is a morphism
$p : y \to z$ such that (a) $p \circ f = 0$ and (b)
for any $p' : y \to z'$ such that $p' \circ f = 0$ there
exists a unique morphism $g : z \to z'$ such that
$p' = g \circ p$.
\item If a cokernel of $f$ exists we denote this
$y \to \text{Coker}(f)$.
\item If a kernel of $f$ exists, then a {\it coimage
of $f$} is a cokernel for the morphism $\text{Ker}(f) \to x$.
\item If a kernel and coimage exist then we denote this
$x \to \text{Coim}(f)$.
\item If a cokernel of $f$ exists, then the {\it image of
$f$} is a kernel of the morphism $y \to \text{Coker}(f)$.
\item If a cokernel and image of $f$ exist then we denote
this $\text{Im}(f) \to y$.
\end{enumerate}
\end{definition}

\begin{lemma}
\label{lemma-coim-im-map}
Let $f : x \to y$ be a morphism in a preadditive category
such that the kernel, cokernel, image and coimage all exist.
Then $f$ can be factored uniquely as
$x \to \text{Coim}(f) \to \text{Im}(f) \to y$.
\end{lemma}

\begin{proof}
There is a canonical morphism $\text{Coim}(f) \to y$
because $\text{Ker}(f) \to x \to y$ is zero.
The composition $\text{Coim}(f) \to y \to \text{Coker}(f)$
is zero, because it is the unique morphism which gives
rise to the morphism $x \to y \to \text{Coker}(f)$ which
is zero. Hence $\text{Coim}(f) \to y$ factors uniquely through
$\text{Im}(f) \to y$, which gives us the desired map.
\end{proof}

\begin{definition}
\label{definition-abelian-category}
A category $\mathcal{A}$ is {\it abelian} if
it is additive, if all kernels and cokernels exist,
and if the natural map $\text{Coim}(f) \to \text{Im}(f)$
is an isomorphism for all morphisms $f$ of
$\mathcal{A}$.
\end{definition}

\begin{definition}
\label{definition-injective-surjective}
Let $f : x \to y$ be a morphism in an abelian category.
\begin{enumerate}
\item We say $f$ is {\it injective} if $\text{Ker}(f) = 0$.
\item We say $f$ is {\it surjective} if $\text{Coker}(f) = 0$.
\end{enumerate}
\end{definition}

\begin{lemma}
\label{lemma-characterize-injective}
Let $f : x \to y$ be a morphism in an abelian category. Then
\begin{enumerate}
\item $f$ is injective if and only if $f$ is a monomorphism, and
\item $f$ is surjective if and only if $f$ is an epimorphism.
\end{enumerate}
\end{lemma}

\begin{proof}
Omitted.
\end{proof}

\noindent
In an abelian category, if $K \subset M$ is a subobject,
then we denote
$$
M/K = \text{Coker}(K \to M).
$$

\begin{lemma}
\label{lemma-colimit-abelian-category}
Let $\mathcal{A}$ be an abelian category.
All finite limits and finite colimits exist in $\mathcal{A}$.
\end{lemma}

\begin{proof}
To show that finite limits exist it suffices to show
that finite products and equalizers exist, see
Categories, Lemma \ref{categories-lemma-finite-limits-exist}.
Finite products exist
by definition and the equalizer of $a,b : x \to y$ is
the kernel of $a - b$. The argument for finite colimits
is similar but dual to this.
\end{proof}

\begin{definition}
\label{definition-exact}
Let $\mathcal{A}$ be an abelian category.
We say a sequence of morphisms
$$
\ldots \to x \to y \to z \to \ldots
$$
in $\mathcal{A}$
is a {\it complex} if the composition of any two (drawn)
arrows is zero. We say a sequence as above is {\it exact at $y$} if
$\text{Im}(x \to y) = \text{Ker}(y \to z)$. We say it is {\it exact}
if it is exact at every object.
\end{definition}

\begin{lemma}
\label{lemma-snake}
Let $\mathcal{A}$ be an abelian category.
Suppose given a commutative diagram
$$
\xymatrix{
& x \ar[r] \ar[d]^\alpha &
y \ar[r] \ar[d]^\beta &
z \ar[r] \ar[d]^\gamma &
0 \\
0 \ar[r] & u \ar[r] & v \ar[r] & w
}
$$
with exact rows, then there is a canonical exact sequence
$$
\text{Ker}(\alpha) \to \text{Ker}(\beta) \to \text{Ker}(\gamma)
\to
\text{Coker}(\alpha) \to \text{Coker}(\beta) \to \text{Coker}(\gamma)
$$
Moreover, if $x \to y$ is injective, then the first map is
injective, and if $v \to w$ is surjective, then the last
map is surjective.
\end{lemma}

\begin{proof}
Shall we really prove this straight from the
axioms? Omitted.
\end{proof}








\section{Complexes}
\label{section-complexes}

\noindent
Of course the notions of a chain complex and a cochain complex
are dual and you only have to read one of the two parts of
this section. So pick the one you like.

\medskip\noindent
A {\it chain complex $A_\bullet$} in an abelian category $\mathcal{A}$
is a complex
$$
\ldots \to
A_{n + 1} \xrightarrow{d_{n + 1}}
A_n \xrightarrow{d_n}
A_{n - 1} \to
\ldots
$$
of $\mathcal{A}$. In other words, we are given an object $A_i$ of
$\mathcal{A}$ for all $i \in \mathbf{Z}$ and for
all $i \in \mathbf{Z}$ a morphism $d_i : A_i \to A_{i - 1}$ such that
$d_{i - 1} \circ d_i = 0$ for all $i$. A {\it morphism of chain
complexes $f : A_\bullet \to B_\bullet$} is given by a
family of morphisms $f_i : A_i \to B_i$ such that all
the diagrams
$$
\xymatrix{
A_i \ar[r]_{d_i} \ar[d]_{f_i} & A_{i - 1} \ar[d]^{f_{i - 1}} \\
B_i \ar[r]^{d_i} & B_{i - 1}
}
$$
commute. The {\it category of chain complexes of $\mathcal{A}$}
is denoted $\text{Ch}(\mathcal{A})$. The full subcategory consisting
of objects of the form
$$
\ldots \to A_2 \to A_1 \to A_0 \to 0 \to 0 \to \ldots
$$
is denoted $\text{Ch}_{\geq 0}(\mathcal{A})$.
In other words, a chain complex $A_\bullet$ belongs to
$\text{Ch}_{\geq 0}(\mathcal{A})$ if and only if
$A_i = 0$ for all $i < 0$.
A {\it homotopy $h$} between a pair of morphisms
of chain complexes $f, g : A_\bullet \to B_\bullet$ is
is a collection of morphisms $h_i : A_i \to B_{i + 1}$
such that we have
$$
f_i - g_i = d_{i + 1} \circ h_i + h_{i - 1} \circ d_i
$$
for all $i$. Clearly, the notions of chain complex, morphism of
chain complexes, and homotopies between morphisms of chain complexes
makes sense over any preadditive category. 

\begin{lemma}
\label{lemma-cat-chain-abelian}
Let $\mathcal{A}$ be an abelian category.
\begin{enumerate}
\item The category of chain complexes in $\mathcal{A}$ is
abelian.
\item A morphism of complexes
$f : A_\bullet \to B_\bullet$ is injective
if and only if each $f_n : A_n \to B_n$ is injective.
\item A morphism of complexes
$f : A_\bullet \to B_\bullet$ is surjective
if and only if each $f_n : A_n \to B_n$ is surjective.
\item A sequence of chain complexes
$$
A_\bullet \xrightarrow{f} B_\bullet \xrightarrow{g} C_\bullet
$$
is exact at $B_\bullet$ if and only if each sequence
$$
A_i \xrightarrow{f_i} B_i \xrightarrow{g_i} C_i
$$
is exact at $B_i$.
\end{enumerate}
\end{lemma}

\begin{proof}
Omitted.
\end{proof}

\noindent
For any $i \in \mathbf{Z}$ the $i$th {\it homology group}
of a chain complex $A_\bullet$ is defined by
the following formula
$$
H_i(A_\bullet) = \text{Ker}(d_i)/\text{Im}(d_{i + 1}).
$$
If $f : A_\bullet \to B_\bullet$ is a morphism of chain
complexes of $\mathcal{A}$ then we get an induced
morphism $H_i(f) : H_i(A_\bullet) \to H_i(B_\bullet)$
because clearly
$f_i(\text{Ker}(d_i : A_i \to A_{i - 1})) \subset
\text{Ker}(d_i : B_i \to B_{i - 1})$, and similarly
for $\text{Im}(d_{i + 1})$.
Thus we obtain a functor
$$
H_i : \text{Ch}(\mathcal{A}) \longrightarrow \mathcal{A}.
$$

\begin{definition}
\label{definition-quasi-isomorphism}
Let $\mathcal{A}$ be an abelian category.
\begin{enumerate}
\item A morphism of chain complexes $f : A_\bullet \to B_\bullet$
is called a {\it quasi-isomorphism} if the induced
maps $H_i(f) : H_i(A_\bullet) \to H_i(B_\bullet)$
is an isomorphism for all $i \in \mathbf{Z}$.
\item A chain complex $A_\bullet$ is called
{\it acyclic} if all of its homology objects
$H_i(A_\bullet)$ are zero.
\end{enumerate}
\end{definition}

\begin{lemma}
\label{lemma-map-homology-homotopy}
Let $\mathcal{A}$ be an abelian category.
If the maps $f, g : A_\bullet \to B_\bullet$ are
homotopic, then the induced maps $H_i(f)$ and $H_i(g)$
are equal.
\end{lemma}

\begin{proof}
Omitted.
\end{proof}

\begin{lemma}
\label{lemma-long-exact-sequence-chain}
Let $\mathcal{A}$ be an abelian category.
Suppose that
$$
0 \to
A_\bullet \to
B_\bullet \to
C_\bullet \to
0
$$
is a short exact sequence of chain complexes of $\mathcal{A}$.
Then there is a canonical long exact homology sequence
$$
\xymatrix{
\ldots & \ldots & \dots \ar[lld] \\
H_i(A_\bullet) \ar[r] & H_i(B_\bullet) \ar[r] & H_i(C_\bullet) \ar[lld] \\
H_{i - 1}(A_\bullet) \ar[r] &
H_{i - 1}(B_\bullet) \ar[r] &
H_{i - 1}(C_\bullet) \ar[lld] \\
\ldots & \ldots & \dots \\
}
$$
\end{lemma}

\begin{proof}
Omitted. The maps come from the Snake Lemma \ref{lemma-snake}.
\end{proof}

\noindent
A {\it cochain complex $A_\bullet$} in an abelian category $\mathcal{A}$
is a complex
$$
\ldots \to
A^{n - 1} \xrightarrow{d^{n - 1}}
A^n \xrightarrow{d^n}
A^{n + 1} \to
\ldots
$$
of $\mathcal{A}$. In other words, we are given an object $A^i$ of
$\mathcal{A}$ for all $i \in \mathbf{Z}$ and for
all $i \in \mathbf{Z}$ a morphism $d^i : A^i \to A^{i + 1}$ such that
$d^{i + 1} \circ d^i = 0$ for all $i$. A {\it morphism of cochain
complexes $f : A^\bullet \to B^\bullet$} is given by a
family of morphisms $f^i : A^i \to B^i$ such that all
the diagrams
$$
\xymatrix{
A^i \ar[r]_{d^i} \ar[d]_{f^i} & A^{i + 1} \ar[d]^{f^{i + 1}} \\
B^i \ar[r]^{d^i} & B^{i + 1}
}
$$
commute. The {\it category of cochain complexes of $\mathcal{A}$}
is denoted $\text{CoCh}(\mathcal{A})$. The full subcategory consisting
of objects of the form
$$
\ldots \to 0 \to 0 \to A^0 \to A^1 \to A^2 \to \ldots
$$
is denoted $\text{CoCh}_{\geq 0}(\mathcal{A})$.
In other words, a cochain complex $A^\bullet$ belongs to
$\text{CoCh}_{\geq 0}(\mathcal{A})$ if and only if
$A^i = 0$ for all $i < 0$.
A {\it homotopy $h$} between a pair of morphisms
of cochain complexes $f, g : A^\bullet \to B^\bullet$ is
is a collection of morphisms $h^i : A^i \to B^{i - 1}$
such that we have
$$
f^i - g^i = d^{i - 1} \circ h^i + h^{i + 1} \circ d^i
$$
for all $i$. Clearly, the notions of cochain complex, morphism of
cochain complexes, and homotopies between morphisms of cochain complexes
makes sense over any preadditive category.

\begin{lemma}
\label{lemma-cat-cochain-abelian}
Let $\mathcal{A}$ be an abelian category.
\begin{enumerate}
\item The category of cochain complexes in $\mathcal{A}$ is
abelian.
\item A morphism of cochain complexes
$f : A^\bullet \to B^\bullet$ is injective
if and only if each $f^n : A^n \to B^n$ is injective.
\item A morphism of cochain complexes
$f : A^\bullet \to B^\bullet$ is surjective
if and only if each $f^n : A^n \to B^n$ is surjective.
\item A sequence of cochain complexes
$$
A^\bullet \xrightarrow{f} B^\bullet \xrightarrow{g} C^\bullet
$$
is exact at $B^\bullet$ if and only if each sequence
$$
A^i \xrightarrow{f^i} B^i \xrightarrow{g^i} C^i
$$
is exact at $B^i$.
\end{enumerate}
\end{lemma}

\begin{proof}
Omitted.
\end{proof}

\noindent
For any $i \in \mathbf{Z}$ the $i$th {\it cohomology group}
of a cochain complex $A^\bullet$ is defined by
the following formula
$$
H^i(A^\bullet) = \text{Ker}(d^i)/\text{Im}(d^{i - 1}).
$$
If $f : A^\bullet \to B^\bullet$ is a morphism of cochain
complexes of $\mathcal{A}$ then we get an induced
morphism $H^i(f) : H^i(A^\bullet) \to H^i(B^\bullet)$
because clearly
$f^i(\text{Ker}(d^i : A^i \to A^{i + 1})) \subset
\text{Ker}(d^i : B^i \to B^{i + 1})$, and similarly
for $\text{Im}(d^{i - 1})$.
Thus we obtain a functor
$$
H^i : \text{CoCh}(\mathcal{A}) \longrightarrow \mathcal{A}.
$$

\begin{definition}
\label{definition-quasi-isomorphism-cochain}
Let $\mathcal{A}$ be an abelian category.
\begin{enumerate}
\item A morphism of cochain complexes $f : A^\bullet \to B^\bullet$
is called a {\it quasi-isomorphism} if the induced
maps $H^i(f) : H^i(A^\bullet) \to H^i(B^\bullet)$
is an isomorphism for all $i \in \mathbf{Z}$.
\item A cochain complex $A^\bullet$ is called
{\it acyclic} if all of its cohomology objects
$H^i(A^\bullet)$ are zero.
\end{enumerate}
\end{definition}

\begin{lemma}
\label{lemma-map-cohomology-homotopy}
Let $\mathcal{A}$ be an abelian category.
If the maps $f, g : A^\bullet \to B^\bullet$ are
homotopic, then the induced maps $H^i(f)$ and $H^i(g)$
are equal.
\end{lemma}

\begin{proof}
Omitted.
\end{proof}

\begin{lemma}
\label{lemma-long-exact-sequence-cochain}
Let $\mathcal{A}$ be an abelian category.
Suppose that
$$
0 \to
A^\bullet \to
B^\bullet \to
C^\bullet \to
0
$$
is a short exact sequence of chain complexes of $\mathcal{A}$.
Then there is a canonical long exact homology sequence
$$
\xymatrix{
\ldots & \ldots & \dots \ar[lld] \\
H^i(A^\bullet) \ar[r] &
H^i(B^\bullet) \ar[r] &
H^i(C^\bullet) \ar[lld] \\
H^{i + 1}(A^\bullet) \ar[r] &
H^{i + 1}(B^\bullet) \ar[r] &
H^{i + 1}(C^\bullet) \ar[lld] \\
\ldots & \ldots & \dots \\
}
$$
\end{lemma}

\begin{proof}
Omitted. The maps come from the Snake Lemma \ref{lemma-snake}.
\end{proof}
















\section{Injectives}
\label{section-injectives}

\begin{definition}
\label{definition-injective}
Let $\mathcal{A}$ be an abelian category.
An object $J \in \text{Ob}(\mathcal{A})$ is
called {\it injective} if for every injection
$A \hookrightarrow B$ and every morphism
$A \to J$ there exists a morphism $B \to J$ making
the following diagram commute
$$
\xymatrix{
A \ar[r] \ar[d] & B \ar@{-->}[ld] \\
J & 
}
$$
\end{definition}

\begin{definition}
\label{definition-enough-injectives}
Let $\mathcal{A}$ be an abelian category.
We say $\mathcal{A}$ has {\it enough injectives}
if every object $A$ has an injective morphism
$A \to J$ into an injective object $J$.
\end{definition}

\begin{definition}
\label{definition-functorial-injective-embedding}
Let $\mathcal{A}$ be an abelian category.
We say that {\it $\mathcal{A}$ has functorial injective embeddings}
if there exists a functor
$$
J : \mathcal{A} \longrightarrow \text{Arrows}(\mathcal{A})
$$
such that
\begin{enumerate}
\item $s \circ J = \text{id}_{\mathcal{A}}$,
\item for any object $A \in \text{Ob}(\mathcal{A})$
the morphism $J(A)$ is injective, and
\item for any object $A \in \text{Ob}(\mathcal{A})$
the object $t(J(A))$ is an injective object of $\mathcal{A}$.
\end{enumerate}
We will denote such a functor by
$A \mapsto (A \to J(A))$.
\end{definition}

\section{Projectives}
\label{section-projectives}

\begin{definition}
\label{definition-projective}
Let $\mathcal{A}$ be an abelian category.
An object $P \in \text{Ob}(\mathcal{A})$ is
called {\it projective} if for every surjection
$A \rightarrow B$ and every morphism
$P \to B$ there exists a morphism $P \to A$ making
the following diagram commute
$$
\xymatrix{
A \ar[r] & B \\
P \ar@{-->}[u] \ar[ru] & 
}
$$
\end{definition}

\begin{definition}
\label{definition-enough-projectives}
Let $\mathcal{A}$ be an abelian category.
We say $\mathcal{A}$ has {\it enough projectives}
if every object $A$ has an surjective morphism
$P \to A$ from an projective object $P$ onto it.
\end{definition}

\begin{definition}
\label{definition-functorial-projective-surjections}
Let $\mathcal{A}$ be an abelian category.
We say that {\it $\mathcal{A}$ has functorial projective surjections}
if there exists a functor
$$
P : \mathcal{A} \longrightarrow \text{Arrows}(\mathcal{A})
$$
such that
\begin{enumerate}
\item $t \circ J = \text{id}_{\mathcal{A}}$,
\item for any object $A \in \text{Ob}(\mathcal{A})$
the morphism $P(A)$ is surjective, and
\item for any object $A \in \text{Ob}(\mathcal{A})$
the object $s(P(A))$ is an projective object of $\mathcal{A}$.
\end{enumerate}
We will denote such a functor by
$A \mapsto (P(A) \to A)$.
\end{definition}


\section{Injectives and adjoint functors}
\label{section-adjoint}

\begin{lemma}
\label{lemma-adjoint-preserve-injectives}
Let $\mathcal{A}$ and $\mathcal{B}$ be abelian categories.
Let $u : \mathcal{A} \to \mathcal{B}$ and
$v : \mathcal{B} \to \mathcal{A}$ be additive functors.
Assume $u$ is right adjoint to $v$.
Assume that $v$ transforms injective maps into injective maps.
Then $u$ transforms injectives into injectives.
\end{lemma}

\begin{proof}
Adjointness means there are transformations of functors
$s : \text{id}_{\mathcal{B}} \to uv$ and $t : vu \to \text{id}_{\mathcal{A}}$
which give rise to the equality
$$
\text{Mor}_{\mathcal{B}}(B, uA)
=
\text{Mor}_{\mathcal{A}}(vB, A)
$$
via $\varphi : B \to uA$ corresponds to
$t \circ v\varphi : vB \to vuA \to A$ and
$\psi : vB \to A$ corresponds to $u\psi \circ s : B \to uvB \to uA$.
Let $J$ be an injective object of $\mathcal{A}$.
Let $i : N \hookrightarrow M$ be an injective morphism
in $\mathcal{B}$. Let $\alpha : N \to uJ$ be a morphism.
Since $vi : vN \hookrightarrow vM$ is injective we can extend the morphism
$t \circ v\alpha : vN \to vuJ \to J$ to a morphism $\beta : vM \to J$.
In a formula $\beta \circ vi = t \circ v\alpha$.
Consider $u\beta \circ s : M \to uvM \to uJ$. Then
$u\beta \circ s \circ i$ is equal to $\alpha$ because
via the adjointness formulas above we get
$t \circ vu\beta \circ vs \circ vi = \beta \circ vi = t \circ v\alpha$
by our choice of $\beta$.
\end{proof}

\begin{lemma}
\label{lemma-adjoint-enough-injectives}
Let $\mathcal{A}$ and $\mathcal{B}$ be abelian categories.
Let $u : \mathcal{A} \to \mathcal{B}$ and
$v : \mathcal{B} \to \mathcal{A}$ be additive functors.
Assume $u$ is right adjoint to $v$.
Assume that $v$ transforms injective maps into injective maps.
Assume that $\mathcal{A}$ has enough injectives.
Then $\mathcal{B}$ has enough injectives.
\end{lemma}

\begin{proof}
Pick $B \in \text{Ob}(\mathcal{B})$.
Pick an injection $vB \to J$ for $J$
an injective object of $\mathcal{A}$.
Consider $B \to uJ$. By Lemma \ref{lemma-adjoint-preserve-injectives}
the object $uJ$ is injective. Let $K \to B$ be the
kernel of $B \to uJ$. Because
$K \to B \to uJ$ is zero, then
$vK \to J$ is zero, hence $vK \to vB$ is zero,
hence $K \to B$ is zero because $v$
transforms injective maps into injective maps
by assumption. Thus $K = 0$ and we win.
\end{proof}

\begin{lemma}
\label{lemma-adjoint-functorial-injectives}
Let $\mathcal{A}$ and $\mathcal{B}$ be abelian categories.
Let $u : \mathcal{A} \to \mathcal{B}$ and
$v : \mathcal{B} \to \mathcal{A}$ be additive functors.
Assume $u$ is right adjoint to $v$.
Assume that $v$ transforms injective maps into injective maps.
Assume that $\mathcal{A}$ has functorial injective hulls.
Then $\mathcal{B}$ has functorial injective hulls.
\end{lemma}

\begin{proof}
Let $A \mapsto (A \to J(A))$ be a functorial
injective hull on $\mathcal{A}$. Then
$B \mapsto (B \to uJ(vB))$ is a functorial
injective hull on $\mathcal{B}$. Compare with the
proof of Lemma \ref{lemma-adjoint-enough-injectives}.
\end{proof}










\section{Other chapters}

\begin{multicols}{2}
\begin{enumerate}
\item \hyperref[introduction-section-phantom]{Introduction}
\item \hyperref[conventions-section-phantom]{Conventions}
\item \hyperref[sets-section-phantom]{Set Theory}
\item \hyperref[categories-section-phantom]{Categories}
\item \hyperref[topology-section-phantom]{Topology}
\item \hyperref[sheaves-section-phantom]{Sheaves on Spaces}
\item \hyperref[algebra-section-phantom]{Commutative Algebra}
\item \hyperref[sites-section-phantom]{Sites and Sheaves}
\item \hyperref[homology-section-phantom]{Homological Algebra}
\item \hyperref[derived-section-phantom]{Derived Categories}
\item \hyperref[more-algebra-section-phantom]{More Algebra}
\item \hyperref[simplicial-section-phantom]{Simplicial Methods}
\item \hyperref[modules-section-phantom]{Sheaves of Modules}
\item \hyperref[sites-modules-section-phantom]{Modules on Sites}
\item \hyperref[injectives-section-phantom]{Injectives}
\item \hyperref[cohomology-section-phantom]{Cohomology of Sheaves}
\item \hyperref[sites-cohomology-section-phantom]{Cohomology on Sites}
\item \hyperref[hypercovering-section-phantom]{Hypercoverings}
\item \hyperref[schemes-section-phantom]{Schemes}
\item \hyperref[constructions-section-phantom]{Constructions of Schemes}
\item \hyperref[properties-section-phantom]{Properties of Schemes}
\item \hyperref[morphisms-section-phantom]{Morphisms of Schemes}
\item \hyperref[coherent-section-phantom]{Coherent Cohomology}
\item \hyperref[divisors-section-phantom]{Divisors}
\item \hyperref[limits-section-phantom]{Limits of Schemes}
\item \hyperref[varieties-section-phantom]{Varieties}
\item \hyperref[chow-section-phantom]{Chow Homology}
\item \hyperref[topologies-section-phantom]{Topologies on Schemes}
\item \hyperref[descent-section-phantom]{Descent}
\item \hyperref[more-morphisms-section-phantom]{More on Morphisms}
\item \hyperref[flat-section-phantom]{More on Flatness}
\item \hyperref[groupoids-section-phantom]{Groupoid Schemes}
\item \hyperref[more-groupoids-section-phantom]{More on Groupoid Schemes}
\item \hyperref[etale-section-phantom]{\'Etale Morphisms of Schemes}
\item \hyperref[etale-cohomology-section-phantom]{\'Etale Cohomology}
\item \hyperref[spaces-section-phantom]{Algebraic Spaces}
\item \hyperref[spaces-properties-section-phantom]{Properties of Algebraic Spaces}
\item \hyperref[spaces-morphisms-section-phantom]{Morphisms of Algebraic Spaces}
\item \hyperref[spaces-topologies-section-phantom]{Topologies on Algebraic Spaces}
\item \hyperref[spaces-descent-section-phantom]{Descent and Algebraic Spaces}
\item \hyperref[spaces-more-morphisms-section-phantom]{More on Morphisms of Spaces}
\item \hyperref[quot-section-phantom]{Quot and Hilbert Spaces}
\item \hyperref[stacks-section-phantom]{Stacks}
\item \hyperref[spaces-groupoids-section-phantom]{Groupoids in Algebraic Spaces}
\item \hyperref[spaces-more-groupoids-section-phantom]{More on Groupoids in Spaces}
\item \hyperref[bootstrap-section-phantom]{Bootstrap}
\item \hyperref[examples-stacks-section-phantom]{Examples of Stacks}
\item \hyperref[groupoids-quotients-section-phantom]{Quotients of Groupoids}
\item \hyperref[algebraic-section-phantom]{Algebraic Stacks}
\item \hyperref[criteria-section-phantom]{Criteria for Representability}
\item \hyperref[stacks-properties-section-phantom]{Properties of Algebraic Stacks}
\item \hyperref[stacks-morphisms-section-phantom]{Morphisms of Algebraic Stacks}
\item \hyperref[examples-section-phantom]{Examples}
\item \hyperref[exercises-section-phantom]{Exercises}
\item \hyperref[guide-section-phantom]{Guide to Literature}
\item \hyperref[desirables-section-phantom]{Desirables}
\item \hyperref[coding-section-phantom]{Coding Style}
\item \hyperref[fdl-section-phantom]{GNU Free Documentation License}
\item \hyperref[index-section-phantom]{Auto Generated Index}
\end{enumerate}
\end{multicols}


\bibliography{my}
\bibliographystyle{alpha}

\end{document}
