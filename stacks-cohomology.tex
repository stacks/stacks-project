\IfFileExists{stacks-project.cls}{%
\documentclass{stacks-project}
}{%
\documentclass{amsart}
}

% The following AMS packages are automatically loaded with
% the amsart documentclass:
%\usepackage{amsmath}
%\usepackage{amssymb}
%\usepackage{amsthm}

% For dealing with references we use the comment environment
\usepackage{verbatim}
\newenvironment{reference}{\comment}{\endcomment}
%\newenvironment{reference}{}{}
\newenvironment{slogan}{\comment}{\endcomment}
\newenvironment{history}{\comment}{\endcomment}

% For commutative diagrams you can use
% \usepackage{amscd}
\usepackage[all]{xy}

% We use 2cell for 2-commutative diagrams.
\xyoption{2cell}
\UseAllTwocells

% To put source file link in headers.
% Change "template.tex" to "this_filename.tex"
% \usepackage{fancyhdr}
% \pagestyle{fancy}
% \lhead{}
% \chead{}
% \rhead{Source file: \url{template.tex}}
% \lfoot{}
% \cfoot{\thepage}
% \rfoot{}
% \renewcommand{\headrulewidth}{0pt}
% \renewcommand{\footrulewidth}{0pt}
% \renewcommand{\headheight}{12pt}

\usepackage{multicol}

% For cross-file-references
\usepackage{xr-hyper}

% Package for hypertext links:
\usepackage{hyperref}

% For any local file, say "hello.tex" you want to link to please
% use \externaldocument[hello-]{hello}
\externaldocument[introduction-]{introduction}
\externaldocument[conventions-]{conventions}
\externaldocument[sets-]{sets}
\externaldocument[categories-]{categories}
\externaldocument[topology-]{topology}
\externaldocument[sheaves-]{sheaves}
\externaldocument[sites-]{sites}
\externaldocument[stacks-]{stacks}
\externaldocument[fields-]{fields}
\externaldocument[algebra-]{algebra}
\externaldocument[brauer-]{brauer}
\externaldocument[homology-]{homology}
\externaldocument[derived-]{derived}
\externaldocument[simplicial-]{simplicial}
\externaldocument[more-algebra-]{more-algebra}
\externaldocument[smoothing-]{smoothing}
\externaldocument[modules-]{modules}
\externaldocument[sites-modules-]{sites-modules}
\externaldocument[injectives-]{injectives}
\externaldocument[cohomology-]{cohomology}
\externaldocument[sites-cohomology-]{sites-cohomology}
\externaldocument[dga-]{dga}
\externaldocument[dpa-]{dpa}
\externaldocument[hypercovering-]{hypercovering}
\externaldocument[schemes-]{schemes}
\externaldocument[constructions-]{constructions}
\externaldocument[properties-]{properties}
\externaldocument[morphisms-]{morphisms}
\externaldocument[coherent-]{coherent}
\externaldocument[divisors-]{divisors}
\externaldocument[limits-]{limits}
\externaldocument[varieties-]{varieties}
\externaldocument[topologies-]{topologies}
\externaldocument[descent-]{descent}
\externaldocument[perfect-]{perfect}
\externaldocument[more-morphisms-]{more-morphisms}
\externaldocument[flat-]{flat}
\externaldocument[groupoids-]{groupoids}
\externaldocument[more-groupoids-]{more-groupoids}
\externaldocument[etale-]{etale}
\externaldocument[chow-]{chow}
\externaldocument[intersection-]{intersection}
\externaldocument[pic-]{pic}
\externaldocument[adequate-]{adequate}
\externaldocument[dualizing-]{dualizing}
\externaldocument[duality-]{duality}
\externaldocument[discriminant-]{discriminant}
\externaldocument[local-cohomology-]{local-cohomology}
\externaldocument[curves-]{curves}
\externaldocument[resolve-]{resolve}
\externaldocument[models-]{models}
\externaldocument[pione-]{pione}
\externaldocument[etale-cohomology-]{etale-cohomology}
\externaldocument[proetale-]{proetale}
\externaldocument[crystalline-]{crystalline}
\externaldocument[spaces-]{spaces}
\externaldocument[spaces-properties-]{spaces-properties}
\externaldocument[spaces-morphisms-]{spaces-morphisms}
\externaldocument[decent-spaces-]{decent-spaces}
\externaldocument[spaces-cohomology-]{spaces-cohomology}
\externaldocument[spaces-limits-]{spaces-limits}
\externaldocument[spaces-divisors-]{spaces-divisors}
\externaldocument[spaces-over-fields-]{spaces-over-fields}
\externaldocument[spaces-topologies-]{spaces-topologies}
\externaldocument[spaces-descent-]{spaces-descent}
\externaldocument[spaces-perfect-]{spaces-perfect}
\externaldocument[spaces-more-morphisms-]{spaces-more-morphisms}
\externaldocument[spaces-flat-]{spaces-flat}
\externaldocument[spaces-groupoids-]{spaces-groupoids}
\externaldocument[spaces-more-groupoids-]{spaces-more-groupoids}
\externaldocument[bootstrap-]{bootstrap}
\externaldocument[spaces-pushouts-]{spaces-pushouts}
\externaldocument[groupoids-quotients-]{groupoids-quotients}
\externaldocument[spaces-more-cohomology-]{spaces-more-cohomology}
\externaldocument[spaces-simplicial-]{spaces-simplicial}
\externaldocument[formal-spaces-]{formal-spaces}
\externaldocument[restricted-]{restricted}
\externaldocument[spaces-resolve-]{spaces-resolve}
\externaldocument[formal-defos-]{formal-defos}
\externaldocument[defos-]{defos}
\externaldocument[cotangent-]{cotangent}
\externaldocument[examples-defos-]{examples-defos}
\externaldocument[algebraic-]{algebraic}
\externaldocument[examples-stacks-]{examples-stacks}
\externaldocument[stacks-sheaves-]{stacks-sheaves}
\externaldocument[criteria-]{criteria}
\externaldocument[artin-]{artin}
\externaldocument[quot-]{quot}
\externaldocument[stacks-properties-]{stacks-properties}
\externaldocument[stacks-morphisms-]{stacks-morphisms}
\externaldocument[stacks-limits-]{stacks-limits}
\externaldocument[stacks-cohomology-]{stacks-cohomology}
\externaldocument[stacks-perfect-]{stacks-perfect}
\externaldocument[stacks-introduction-]{stacks-introduction}
\externaldocument[stacks-more-morphisms-]{stacks-more-morphisms}
\externaldocument[stacks-geometry-]{stacks-geometry}
\externaldocument[moduli-]{moduli}
\externaldocument[moduli-curves-]{moduli-curves}
\externaldocument[examples-]{examples}
\externaldocument[exercises-]{exercises}
\externaldocument[guide-]{guide}
\externaldocument[desirables-]{desirables}
\externaldocument[coding-]{coding}
\externaldocument[obsolete-]{obsolete}
\externaldocument[fdl-]{fdl}
\externaldocument[index-]{index}

% Theorem environments.
%
\theoremstyle{plain}
\newtheorem{theorem}[subsection]{Theorem}
\newtheorem{proposition}[subsection]{Proposition}
\newtheorem{lemma}[subsection]{Lemma}

\theoremstyle{definition}
\newtheorem{definition}[subsection]{Definition}
\newtheorem{example}[subsection]{Example}
\newtheorem{exercise}[subsection]{Exercise}
\newtheorem{situation}[subsection]{Situation}

\theoremstyle{remark}
\newtheorem{remark}[subsection]{Remark}
\newtheorem{remarks}[subsection]{Remarks}

\numberwithin{equation}{subsection}

% Macros
%
\def\lim{\mathop{\rm lim}\nolimits}
\def\colim{\mathop{\rm colim}\nolimits}
\def\Spec{\mathop{\rm Spec}}
\def\Hom{\mathop{\rm Hom}\nolimits}
\def\Ext{\mathop{\rm Ext}\nolimits}
\def\SheafHom{\mathop{\mathcal{H}\!{\it om}}\nolimits}
\def\SheafExt{\mathop{\mathcal{E}\!{\it xt}}\nolimits}
\def\Sch{\textit{Sch}}
\def\Mor{\mathop{\rm Mor}\nolimits}
\def\Ob{\mathop{\rm Ob}\nolimits}
\def\Sh{\mathop{\textit{Sh}}\nolimits}
\def\NL{\mathop{N\!L}\nolimits}
\def\proetale{{pro\text{-}\acute{e}tale}}
\def\etale{{\acute{e}tale}}
\def\QCoh{\textit{QCoh}}
\def\Ker{\mathop{\rm Ker}}
\def\Im{\mathop{\rm Im}}
\def\Coker{\mathop{\rm Coker}}
\def\Coim{\mathop{\rm Coim}}

%
% Macros for moduli stacks/spaces
%
\def\QCohstack{\mathcal{QC}\!{\it oh}}
\def\Cohstack{\mathcal{C}\!{\it oh}}
\def\Spacesstack{\mathcal{S}\!{\it paces}}
\def\Quotfunctor{{\rm Quot}}
\def\Hilbfunctor{{\rm Hilb}}
\def\Curvesstack{\mathcal{C}\!{\it urves}}
\def\Polarizedstack{\mathcal{P}\!{\it olarized}}
\def\Complexesstack{\mathcal{C}\!{\it omplexes}}
% \Pic is the operator that assigns to X its picard group, usage \Pic(X)
% \Picardstack_{X/B} denotes the Picard stack of X over B
% \Picardfunctor_{X/B} denotes the Picard functor of X over B
\def\Pic{\mathop{\rm Pic}\nolimits}
\def\Picardstack{\mathcal{P}\!{\it ic}}
\def\Picardfunctor{{\rm Pic}}
\def\Deformationcategory{\mathcal{D}\!{\it ef}}


% OK, start here.
%
\begin{document}

\title{Cohomology of Algebraic Stacks}

\maketitle

\phantomsection
\label{section-phantom}

\tableofcontents




\section{Introduction}
\label{section-introduction}

\noindent
In this chapter we write about cohomology of algebraic stacks.
This mean in particular cohomology of quasi-coherent sheaves, i.e.,
we prove analogues of the results in the chapter entitled
``Coherent Cohomology''. The results in this chapter are different
from those in \cite{LM-B} mainly because we consistently use the
``big sites''. Before reading this chapter please take a quick look at
the chapter ``Sheaves on Algebraic Stacks'' in order to become
familiar with the terminology introduced there.



\section{Conventions and abuse of language}
\label{section-conventions}

\noindent
We continue to use the conventions and the abuse of language
introduced in
Properties of Stacks, Section \ref{stacks-properties-section-conventions}.











\section{Notation}
\label{section-notation}

\noindent
Different topologies. If we indicate an algebraic stack by a calligraphic
letter, such as $\mathcal{X}, \mathcal{Y}, \mathcal{Z}$, then the notation
$\mathcal{X}_{Zar}, \mathcal{X}_{\acute{e}tale}, \mathcal{X}_{smooth},
\mathcal{X}_{syntomic}, \mathcal{X}_{fppf}$ indicates the site introduced
in
Sheaves on Stacks, Definition
\ref{stacks-sheaves-definition-inherited-topologies}.
(Think ``big site''.) Correspondingly the structure sheaf of
$\mathcal{X}$ is a sheaf on $\mathcal{X}_{fppf}$.
On the other hand, algebraic spaces and schemes
are usually indicated by roman capitals, such as $X, Y, Z$, and in this case
$X_{\acute{e}tale}$ indicates the small \'etale site of $X$ (as
defined in
Topologies, Definition
\ref{topologies-definition-big-small-etale}
or
Properties of Spaces, Definition
\ref{spaces-properties-definition-etale-site}).
It seems that the distinction should be clear enough.

\medskip\noindent
The default topology is the fppf topology. Hence we will sometimes
say ``sheaf on $\mathcal{X}$'' or ``sheaf of $\mathcal{O}_\mathcal{X}$''
modules when we mean sheaf on $\mathcal{X}_{fppf}$ or object of
$\textit{Mod}(\mathcal{X}_{fppf}, \mathcal{O}_\mathcal{X})$.

\medskip\noindent
If $f : \mathcal{X} \to \mathcal{Y}$ is a morphism of algebraic
stacks, then the functors $f_*$ and $f^{-1}$ defined on presheaves
preserves sheaves for any of the topologies mentioned above. In particular
when we discuss the pushforward or pullback of a sheaf we don't have to
mention which topology we are working with. The same isn't true
when we compute cohomology groups and/or higher direct images. In this
case we will always mention which topology we are working with.

\medskip\noindent
Suppose that $f : X \to \mathcal{Y}$ is a morphism from an algebraic
space $X$ to an algebraic stack $\mathcal{Y}$. Let $\mathcal{G}$ be
a sheaf on $\mathcal{Y}_\tau$ for some topology $\tau$. In this case
$f^{-1}\mathcal{G}$ is a sheaf for the $\tau$ topology on $\mathcal{S}_X$
(the algebraic stack associated to $X$) because (by our conventions) $f$
really is a $1$-morphism $f : \mathcal{S}_X \to \mathcal{Y}$.
If $\tau = \acute{e}tale$ or stronger, then we write
$f^{-1}\mathcal{G}|_{X_{\acute{e}tale}}$
to denote the restriction to the \'etale site of $X$, see
Sheaves on Stacks, Section \ref{stacks-sheaves-section-compare}.
If $\mathcal{G}$ is an $\mathcal{O}_\mathcal{X}$-module we sometimes
write $f^*\mathcal{G}$ and $f^*\mathcal{G}|_{X_{\acute{e}tale}}$
instead.







\section{Pushforward of quasi-coherent modules}
\label{section-pushforward-quasi-coherent}

\noindent
Let $f : \mathcal{X} \to \mathcal{Y}$ be a morphism of algebraic stacks.
Consider the pushforward
$$
f_* :
\textit{Mod}(\mathcal{O}_\mathcal{X})
\longrightarrow
\textit{Mod}(\mathcal{O}_\mathcal{Y})
$$
It turns out that this functor almost never preserves the subcategories
of quasi-coherent sheaves. For example, consider the morphism of schemes
$$
j : X = \mathbf{A}^2_k \setminus \{0\} \longrightarrow \mathbf{A}^2_k = Y.
$$
Associated to this we have the corresponding morphism of algebraic stacks
$$
f = j_{big} : \mathcal{X} = (\Sch/X)_{fppf} \to
(\Sch/Y)_{fppf} = \mathcal{Y}
$$
The pushforward $f_*\mathcal{O}_\mathcal{X}$ of the structure sheaf has
global sections $k[x, y]$. Hence if $f_*\mathcal{O}_\mathcal{X}$ is
quasi-coherent on $\mathcal{Y}$ then we would have
$f_*\mathcal{O}_\mathcal{X} = \mathcal{O}_\mathcal{Y}$. However,
consider $T = \Spec(k) \to \mathbf{A}^2_k = Y$ mapping to $0$.
Then $\Gamma(T, f_*\mathcal{O}_\mathcal{X}) = 0$ because
$X \times_Y T = \emptyset$ whereas $\Gamma(T, \mathcal{O}_\mathcal{Y}) = k$.
On the positive side, we know from
Coherent, Lemma \ref{coherent-lemma-flat-base-change-cohomology}
that for any flat morphism $T \to Y$ we have the equality
$\Gamma(T, f_*\mathcal{O}_\mathcal{X}) = \Gamma(T, \mathcal{O}_\mathcal{Y})$
(this uses that $j$ is quasi-compact and quasi-separated).

\medskip\noindent
Let $f : \mathcal{X} \to \mathcal{Y}$ be a quasi-compact and
quasi-separated morphism of algebraic stacks. Here are
three key observations that will allows us to get a good theory anyway:
\begin{enumerate}
\item $f_*$ does preserve the category of locally quasi-coherent modules, and
\item $f_*$ transforms a quasi-coherent sheaf into a locally quasi-coherent
sheaf whose comparison maps, see
Sheaves on Stacks, Equation (\ref{stacks-sheaves-equation-comparison-modules})
are isomorphisms for maps $\varphi$ lying over flat maps, and
\item locally quasi-coherent $\mathcal{O}_\mathcal{Y}$-modules as in (2)
give rise to modules on a presentation of $\mathcal{Y}$ and hence
quasi-coherent modules on $\mathcal{Y}$, see
Sheaves on Stacks, Section
\ref{stacks-sheaves-section-quasi-coherent-algebraic-stacks}.
\end{enumerate}
Once we have worked out the details, we will obtain a functor
$$
f_{\textit{QCoh}, *} :
\textit{QCoh}(\mathcal{O}_\mathcal{X})
\longrightarrow
\textit{QCoh}(\mathcal{O}_\mathcal{Y})
$$
which is a right adjoint to
$f^* : \textit{QCoh}(\mathcal{O}_\mathcal{Y}) \to
\textit{QCoh}(\mathcal{O}_\mathcal{X})$
such that moreover
$$
\Gamma(y, f_*\mathcal{F}) = \Gamma(y, f_{\textit{QCoh}, *}\mathcal{F})
$$
for any $y \in \Ob(\mathcal{Y})$ such that the associated
$1$-morphism $y : V \to \mathcal{Y}$ is flat, see (insert future
reference here).
Moreover, a similar construction will produce functors $R^if_*$.
However, these results will not be sufficient to produce a
total direct image functor (of complexes with quasi-coherent
cohomology sheaves).




\section{The key lemma}
\label{section-key}

\noindent
The following lemma is the basis for our understanding of
higher direct images of certain types of sheaves of modules.
There are two verions: one for the \'etale topology and
one for the fppf topology.

\begin{lemma}
\label{lemma-general-pushforward}
Let $\mathcal{M}$ be a rule which associates to every algebraic stack
$\mathcal{X}$ a subcategory $\mathcal{M}_\mathcal{X}$ of
$\textit{Mod}(\mathcal{X}_{\acute{e}tale}, \mathcal{O}_\mathcal{X})$
such that
\begin{enumerate}
\item $\mathcal{M}_\mathcal{X}$ is a weak Serre subcategory
of $\textit{Mod}(\mathcal{X}_{\acute{e}tale}, \mathcal{O}_\mathcal{X})$
(see Homology, Definition \ref{homology-definition-serre-subcategory})
for all algebraic stacks $\mathcal{X}$,
\item for a smooth morphism of algebraic stacks
$f : \mathcal{Y} \to \mathcal{X}$ the functor $f^*$ maps
$\mathcal{M}_\mathcal{X}$ into $\mathcal{M}_\mathcal{Y}$,
\item if $f_i : \mathcal{X}_i \to \mathcal{X}$ is a family of smooth
morphisms of algebraic stacks with
$|\mathcal{X}| = \bigcup |f_i|(|\mathcal{X}_i|)$, then an object
$\mathcal{F}$ of
$\textit{Mod}(\mathcal{X}_{\acute{e}tale}, \mathcal{O}_\mathcal{X})$
is in $\mathcal{M}_\mathcal{X}$ if and only if
$f_i^*\mathcal{F}$ is in $\mathcal{M}_{\mathcal{X}_i}$ for all $i$, and
\item if $f : \mathcal{Y} \to \mathcal{X}$ is a morphism of algebraic
stacks such that $\mathcal{X}$ and $\mathcal{Y}$ are representable
by affine schemes, then $R^if_*$ maps $\mathcal{M}_\mathcal{Y}$
into $\mathcal{M}_\mathcal{X}$.
\end{enumerate}
Then for any quasi-compact and quasi-separated morphism 
$f : \mathcal{Y} \to \mathcal{X}$ of algebraic stacks
$R^if_*$ maps $\mathcal{M}_\mathcal{Y}$
into $\mathcal{M}_\mathcal{X}$. (Higher direct images computed in \'etale
topology.)
\end{lemma}

\begin{proof}
Let $f : \mathcal{Y} \to \mathcal{X}$ be a quasi-compact and quasi-separated
morphism of algebraic stacks and let $\mathcal{F}$ be an object of
$\mathcal{M}_\mathcal{Y}$. Choose a surjective smooth morphism
$\mathcal{U} \to \mathcal{X}$ where $\mathcal{U}$ is representable by
a scheme. By
Sheaves on Stacks, Lemma
\ref{stacks-sheaves-lemma-base-change-higher-direct-images}
taking higher direct images commutes with base change.
Assumption (2) shows that the pullback of $\mathcal{F}$
to $\mathcal{U} \times_\mathcal{X} \mathcal{Y}$ is in
$\mathcal{M}_{\mathcal{U} \times_\mathcal{X} \mathcal{Y}}$
because the projection
$\mathcal{U} \times_\mathcal{X} \mathcal{Y} \to \mathcal{Y}$
is smooth as a base change of a smooth morphism. Hence (3) shows we may
replace $\mathcal{Y} \to \mathcal{X}$ by the projection
$\mathcal{U} \times_\mathcal{X} \mathcal{Y} \to \mathcal{U}$.
In other words, we may assume that $\mathcal{X}$
is representable by a scheme.
Using (3) once more, we see that the question is Zariski local on
$\mathcal{X}$, hence we may assume that $\mathcal{X}$ is representable by
an affine scheme. Since $f$ is quasi-compact this implies that also
$\mathcal{Y}$ is quasi-compact. Thus we may choose a surjective smooth
morphism $g : \mathcal{V} \to \mathcal{Y}$ where $\mathcal{V}$ is representable
by an affine scheme.

\medskip\noindent
In this situation we have the spectral sequence
$$
E_2^{p, q} = R^q(f \circ g_p)_*g_p^*\mathcal{F}
\Rightarrow
R^{p + q}f_*\mathcal{F}
$$
of
Sheaves on Stacks, Proposition
\ref{stacks-sheaves-proposition-smooth-covering-compute-direct-image}.
Recall that this is the spectral sequence associated to a double
complex. By assumption (1) we may use
Homology, Remark \ref{homology-remark-weak-serre-subcategory}.
Note that the morphisms
$$
g_p : \mathcal{V}_p =
\mathcal{V} \times_\mathcal{Y} \ldots \times_\mathcal{Y} \mathcal{V}
\longrightarrow
\mathcal{Y}
$$
are smooth as compositions of base changes of the smooth morphism $g$.
Thus the sheaves $g_p^*\mathcal{F}$ are in
$\mathcal{M}_{\mathcal{V}_p}$ by (2). Hence it suffices to prove that the
higher direct images of objects of $\mathcal{M}_{\mathcal{V}_p}$ under
the morphisms
$$
\mathcal{V}_p =
\mathcal{V} \times_\mathcal{Y} \ldots \times_\mathcal{Y} \mathcal{V}
\longrightarrow
\mathcal{X}
$$
are in $\mathcal{M}_\mathcal{X}$. The algebraic stacks $\mathcal{V}_p$
are quasi-compact and quasi-separated by
Morphisms of Stacks, Lemma
\ref{stacks-morphisms-lemma-quasi-compact-quasi-separated-permanence}.
Of course each $\mathcal{V}_p$ is representable by an algebraic space
(the diagonal of the algebraic stack $\mathcal{Y}$ is representable
by algebraic spaces). This reduces us to the case where
$\mathcal{Y}$ is representable by an algebraic space and $\mathcal{X}$
is representable by an affine scheme.

\medskip\noindent
In the situation where $\mathcal{Y}$ is representable by an algebraic
space and $\mathcal{X}$ is representable by an affine scheme, we choose
anew a surjective smooth morphism $\mathcal{V} \to \mathcal{Y}$ where
$\mathcal{V}$ is representable by an affine scheme. Going through the
argument above once again we once again reduce to the morphisms
$\mathcal{V}_p \to \mathcal{X}$. But in the current situation the algebraic
stacks $\mathcal{V}_p$ are representable by quasi-compact and quasi-separated
schemes (becase the diagonal of an algebraic space is representable by
schemes).

\medskip\noindent
Thus we may assume $\mathcal{Y}$ is representable by a scheme and
$\mathcal{X}$ is representable by an affine scheme. Choose (again)
a surjective smooth morphism $\mathcal{V} \to \mathcal{Y}$ where
$\mathcal{V}$ is representable by an affine scheme. In this case all
the algebraic stacks $\mathcal{V}_p$ are representable by separated
schemes (because the diagonal of a scheme is separated).

\medskip\noindent
Thus we may assume $\mathcal{Y}$ is representable by a separated scheme and
$\mathcal{X}$ is representable by an affine scheme. Choose (yet again)
a surjective smooth morphism $\mathcal{V} \to \mathcal{Y}$ where
$\mathcal{V}$ is representable by an affine scheme. In this case all
the algebraic stacks $\mathcal{V}_p$ are representable by affine schemes
(because the diagonal of a separated scheme is a closed immersion hence affine)
and this case is handled by assumption (4).
This finishes the proof.
\end{proof}

\noindent
Here is the version for the fppf topology.

\begin{lemma}
\label{lemma-general-pushforward-fppf}
Let $\mathcal{M}$ be a rule which associates to every algebraic stack
$\mathcal{X}$ a subcategory $\mathcal{M}_\mathcal{X}$ of
$\textit{Mod}(\mathcal{O}_\mathcal{X})$
such that
\begin{enumerate}
\item $\mathcal{M}_\mathcal{X}$ is a weak Serre subcategory
of $\textit{Mod}(\mathcal{O}_\mathcal{X})$
for all algebraic stacks $\mathcal{X}$,
\item for a smooth morphism of algebraic stacks
$f : \mathcal{Y} \to \mathcal{X}$ the functor $f^*$ maps
$\mathcal{M}_\mathcal{X}$ into $\mathcal{M}_\mathcal{Y}$,
\item if $f_i : \mathcal{X}_i \to \mathcal{X}$ is a family of smooth
morphisms of algebraic stacks with
$|\mathcal{X}| = \bigcup |f_i|(|\mathcal{X}_i|)$, then an object
$\mathcal{F}$ of $\textit{Mod}(\mathcal{O}_\mathcal{X})$
is in $\mathcal{M}_\mathcal{X}$ if and only if
$f_i^*\mathcal{F}$ is in $\mathcal{M}_{\mathcal{X}_i}$ for all $i$, and
\item if $f : \mathcal{Y} \to \mathcal{X}$ is a morphism of algebraic
stacks and $\mathcal{X}$ and $\mathcal{Y}$ are representable
by affine schemes, then $R^if_*$ maps $\mathcal{M}_\mathcal{Y}$
into $\mathcal{M}_\mathcal{X}$.
\end{enumerate}
Then for any quasi-compact and quasi-separated morphism 
$f : \mathcal{Y} \to \mathcal{X}$ of algebraic stacks
$R^if_*$ maps $\mathcal{M}_\mathcal{Y}$
into $\mathcal{M}_\mathcal{X}$. (Higher direct images computed in fppf
topology.)
\end{lemma}

\begin{proof}
Identical to the proof of Lemma \ref{lemma-general-pushforward}.
\end{proof}


\section{Locally quasi-coherent modules}
\label{section-locally-quasi-coherent}

\noindent
Let $\mathcal{X}$ be an algebraic stack. Let $\mathcal{F}$ be a presheaf
of $\mathcal{O}_\mathcal{X}$-modules. We can ask whether $\mathcal{F}$
is {\it locally quasi-coherent}, see
Sheaves on Stacks, Definition
\ref{stacks-sheaves-definition-locally-quasi-coherent}.
Briefly, this means $\mathcal{F}$ is an $\mathcal{O}_\mathcal{X}$-module
for the \'etale topology such that for any morphism $f : U \to \mathcal{X}$
the restriction $f^*\mathcal{F}|_{U_{\acute{e}tale}}$ is quasi-coherent
on $U_{\acute{e}tale}$. (The actual definition is slightly different, but
equivalent.) A useful fact is that
$$
\textit{LQCoh}(\mathcal{O}_\mathcal{X}) \subset
\textit{Mod}(\mathcal{X}_{\acute{e}tale}, \mathcal{O}_\mathcal{X})
$$
is a weak Serre subcategory, see
Sheaves on Stacks, Lemma \ref{stacks-sheaves-lemma-lqc-colimits}.

\begin{lemma}
\label{lemma-check-lqc-on-etale-covering}
Let $\mathcal{X}$ be an algebraic stack. Let
$f_j : \mathcal{X}_j \to \mathcal{X}$ be a family of smooth
morphisms of algebraic stacks with
$|\mathcal{X}| =\bigcup |f_j|(|\mathcal{X}_j|)$.
Let $\mathcal{F}$ be a sheaf of $\mathcal{O}_\mathcal{X}$ modules
on $\mathcal{X}_{\acute{e}tale}$. If each $f_j^{-1}\mathcal{F}$
is locally quasi-coherent, then so is $\mathcal{F}$.
\end{lemma}

\begin{proof}
We may replace each of the algebraic stacks $\mathcal{X}_j$ by
a scheme $U_j$ (using that any algebraic stack has a smooth covering by
a scheme and that compositions of smooth morphisms are smooth, see
Morphisms of Stacks, Lemma \ref{stacks-morphisms-lemma-composition-smooth}).
The pullback of $\mathcal{F}$ to $(\Sch/U_j)_{\acute{e}tale}$ is still
locally quasi-coherent, see
Sheaves on Stacks, Lemma \ref{stacks-sheaves-lemma-pullback-lqc}.
Then $f = \coprod f_j : U = \coprod U_j \to \mathcal{X}$ is a smooth surjective
morphism. Let $x$ be an object of $\mathcal{X}$. By
Sheaves on Stacks, Lemma
\ref{stacks-sheaves-lemma-surjective-flat-locally-finite-presentation}
there exists an \'etale covering $\{x_i \to x\}_{i \in I}$
such that each $x_i$ lifts to an object $u_i$ of $(\Sch/U)_{\acute{e}tale}$.
This just means that $x$, $x_i$ live over a schemes $V$, $V_i$, that
$\{V_i \to V\}$ is an \'etale covering, and that $x_i$ comes from
a morphism $u_i : V_i \to U$. The restriction
$x_i^*\mathcal{F}|_{V_{i, \acute{e}tale}}$ is equal to the restriction
of $f^*\mathcal{F}$ to $V_{i, \acute{e}tale}$, see
Sheaves on Stacks, Lemma \ref{stacks-sheaves-lemma-comparison}.
Hence $x^*\mathcal{F}|_{V_{\acute{e}tale}}$
is a sheaf on the small \'etale site of $V$ which is quasi-coherent
when restricted to $V_{i, \acute{e}tale}$ for each $i$.
This implies that it is quasi-coherent (as desired), for example by
Properties of Spaces, Lemma
\ref{spaces-properties-lemma-characterize-quasi-coherent}.
\end{proof}

\begin{lemma}
\label{lemma-pushforward-locally-quasi-coherent}
Let $f : \mathcal{X} \to \mathcal{Y}$ be a quasi-compact and
quasi-separated morphism of algebraic stacks. Let 
$\mathcal{F}$ be a locally quasi-coherent
$\mathcal{O}_\mathcal{X}$-module on $\mathcal{X}_{\acute{e}tale}$.
Then $R^if_*\mathcal{F}$ (computed in the \'etale topology) is
a locally quasi-coherent on $\mathcal{Y}_{\acute{e}tale}$.
\end{lemma}

\begin{proof}
We will use
Lemma \ref{lemma-general-pushforward}
to prove this. We will check its assumptions (1) -- (4).
Parts (1) and (2) follows from
Sheaves on Stacks, Lemma \ref{stacks-sheaves-lemma-lqc-colimits}.
Part (3) follows from
Lemma \ref{lemma-check-lqc-on-etale-covering}.
Thus it suffices to show (4).

\medskip\noindent
Suppose $f : \mathcal{X} \to \mathcal{Y}$ is a morphism of algebraic stacs
such that $\mathcal{X}$ and $\mathcal{Y}$ are representable by affine
schemes $X$ and $Y$. Choose any object $y$ of $\mathcal{Y}$ lying over a
scheme $V$. For clarity, denote $\mathcal{V} = (\Sch/V)_{fppf}$ the
algebraic stack corresponding to $V$. Consider the cartesian diagram
$$
\xymatrix{
\mathcal{Z} \ar[d] \ar[r]_g \ar[d]_{f'} & \mathcal{X} \ar[d]^f \\
\mathcal{V} \ar[r]^y & \mathcal{Y}
}
$$
Thus $\mathcal{Z}$ is representable by the scheme $Z = V \times_Y X$
and $f'$ is quasi-compact and separated (even affine). By
Sheaves on Stacks, Lemma
\ref{stacks-sheaves-lemma-compare-representable-morphism-cohomology}
we have
$$
R^if_*\mathcal{F}|_{V_{\acute{e}tale}} =
R^if'_{small, *}\big(g^*\mathcal{F}|_{Z_{\acute{e}tale}}\big)
$$
The right hand side is a quasi-coherent sheaf on $V_{\acute{e}tale}$ by
Cohomology of Spaces, Lemma
\ref{spaces-cohomology-lemma-higher-direct-image}.
This implies the left hand side is quasi-coherent which is what
we had to prove.
\end{proof}







\section{Flat comparison maps}
\label{section-flat-comparison}

\noindent
Let $\mathcal{X}$ be an algebraic stack and let $\mathcal{F}$ be an object
of $\textit{Mod}(\mathcal{X}_{\acute{e}tale}, \mathcal{O}_\mathcal{X})$.
Given an object $x$ of $\mathcal{X}$ lying over the scheme $U$ the
restriction $\mathcal{F}|_{U_{\acute{e}tale}}$ is the restriction of
$x^{-1}\mathcal{F}$ to the small \'etale site of $U$, see
Sheaves on Stacks, Definition \ref{stacks-sheaves-definition-pullback}.
Next, let $\varphi : x \to x'$ be a morphism of $\mathcal{X}$ lying
over a morphism of schemes $f : U \to U'$. Thus a $2$-commutative diagram
$$
\xymatrix{
U \ar[rd]_x \ar[rr]_f & & U' \ar[ld]^{x'} \\
& \mathcal{X}
}
$$
Associated to $\varphi$ we obtain a comparison map between restrictions
\begin{equation}
\label{equation-comparison-modules}
c_\varphi :
f_{small}^*(\mathcal{F}|_{U'_{\acute{e}tale}})
\longrightarrow
\mathcal{F}|_{U_{\acute{e}tale}}
\end{equation}
see Sheaves on Stacks, Equation
(\ref{stacks-sheaves-equation-comparison-modules}).
In this situation we can consider the following property
of $\mathcal{F}$.

\begin{definition}
\label{definition-flat-base-change}
Let $\mathcal{X}$ be an algebraic stack and let $\mathcal{F}$ in
$\textit{Mod}(\mathcal{X}_{\acute{e}tale}, \mathcal{O}_\mathcal{X})$.
We say $\mathcal{F}$ has the {\it flat base change property}\footnote{This
may be nonstandard notation.}
if and only if $c_\varphi$ is an isomorphism whenever $f$ is flat.
\end{definition}

\noindent
Here is a lemma with some properties of this notion.

\begin{lemma}
\label{lemma-check-flat-comparison-on-etale-covering}
Let $\mathcal{X}$ be an algebraic stack. Let $\mathcal{F}$
be a sheaf of $\mathcal{O}_\mathcal{X}$-modules on
$\mathcal{X}_{\acute{e}tale}$.
\begin{enumerate}
\item If $\mathcal{F}$ has the flat base change property then for any morphism
$g : \mathcal{Y} \to \mathcal{X}$ of algebraic stacks, the
pullback $g^*\mathcal{F}$ does too.
\item The full subcategory of
$\textit{Mod}(\mathcal{X}_{\acute{e}tale}, \mathcal{O}_\mathcal{X})$
consisting of modules with the flat base change property
is a weak Serre subcategory.
\item  Let $f_i : \mathcal{X}_i \to \mathcal{X}$ be a family of
smooth morphisms of algebraic stacks such that
$|\mathcal{X}| = \bigcup_i |f_i|(|\mathcal{X}_i|)$. If each
$f_i^*\mathcal{F}$ has the flat base change property then so does
$\mathcal{F}$.
\end{enumerate}
\end{lemma}

\begin{proof}
Let $g : \mathcal{Y} \to \mathcal{X}$ be as in (1).
Let $y$ be an object of $\mathcal{Y}$ lying over a scheme $V$. By
Sheaves on Stacks, Lemma \ref{stacks-sheaves-lemma-comparison}
we have
$(g^*\mathcal{F})|_{V_{\acute{e}tale}} = \mathcal{F}|_{V_{\acute{e}tale}}$.
Moreover a comparison mapping for the sheaf $g^*\mathcal{F}$ on $\mathcal{Y}$
is a special case of a comparison map for the sheaf $\mathcal{F}$ on
$\mathcal{X}$, see
Sheaves on Stacks, Lemma \ref{stacks-sheaves-lemma-comparison}.
In this way (1) is clear.

\medskip\noindent
Proof of (2). We use the characterization of weak Serre subcategories of
Homology, Lemma \ref{homology-lemma-characterize-weak-serre-subcategory}.
Kernels and cokernels of
maps between sheaves having the flat base change property
also have the flat base change property. This is clear because
$f_{small}^*$ is exact for a flat morphism of schemes and since the
restriction functors $(-)|_{U_{\acute{e}tale}}$ are exact (because we
are working in the \'etale topology). Finally, if
$0 \to \mathcal{F}_1 \to \mathcal{F}_2 \to \mathcal{F}_3 \to 0$
is a short exact sequence of
$\textit{Mod}(\mathcal{X}_{\acute{e}tale}, \mathcal{O}_\mathcal{X})$
and the outer two sheaves have the flat base change property then
the middle one does as well, again because of the exactness of
$f_{small}^*$ and the restriction functors (and the 5 lemma).

\medskip\noindent
Let $f_i : \mathcal{X}_i \to \mathcal{X}$ be a jointly surjective family of
smooth morphisms of algebraic stacks and assume each $f_i^*\mathcal{F}$
has the flat base change property. By part (1), the definition of
an algebraic stack, and the fact that compositions of smooth morphisms
are smooth (see
Morphisms of Stacks, Lemma \ref{stacks-morphisms-lemma-composition-smooth})
we may assume that each $\mathcal{X}_i$ is representable by a scheme.
Let $\varphi : x \to x'$ be a morphism of $\mathcal{X}$ lying over
a flat morphism $a : U \to U'$ of schemes. By
Sheaves on Stacks, Lemma
\ref{stacks-sheaves-lemma-surjective-flat-locally-finite-presentation}
there exists a jointly surjective family of \'etale morphisms
$U'_i \to U'$ such that $U' \to U' \to \mathcal{X}$ factors through
$\mathcal{X}_i$. Thus we obtain commutative diagrams
$$
\xymatrix{
U_i = U \times_{U'} U_i' \ar[r]_-{a_i} \ar[d] &
U_i' \ar[r]_{x_i'} \ar[d] & \mathcal{X}_i \ar[d]^{f_i} \\
U \ar[r]^a & U' \ar[r]^{x'} & \mathcal{X}
}
$$
Note that each $a_i$ is a flat morphism of schemes as a base change of $a$.
Denote $\psi_i : x_i \to x'_i$ the morphism of $\mathcal{X}_i$ lying over
$a_i$ with target $x_i'$. By assumption the comparison maps
$c_{\psi_i} :
(a_i)_{small}^*\big(f_i^*\mathcal{F}|_{(U'_i)_{\acute{e}tale}}\big)
\to f_i^*\mathcal{F}|_{(U_i)_{\acute{e}tale}}$ is an isomorphism.
Because the vertical arrows $U_i' \to U'$ and $U_i \to U$ are \'etale,
the sheaves $f_i^*\mathcal{F}|_{(U_i')_{\acute{e}tale}}$ and
$f_i^*\mathcal{F}|_{(U_i)_{\acute{e}tale}}$ are the restrictions of
$\mathcal{F}|_{U'_{\acute{e}tale}}$ and $\mathcal{F}|_{U_{\acute{e}tale}}$
and the map $c_{\psi_i}$ is the restriction of $c_\varphi$ to
$(U_i)_{\acute{e}tale}$, see
Sheaves on Stacks, Lemma \ref{stacks-sheaves-lemma-comparison}.
Since $\{U_i \to U\}$ is an \'etale covering, this implies
that the comparison map $c_\varphi$ is an isomorphism which is what
we wanted to prove.
\end{proof}

\begin{lemma}
\label{lemma-flat-comparison}
Let $f : \mathcal{X} \to \mathcal{Y}$ be a quasi-compact and
quasi-separated morphism of algebraic stacks. Let 
$\mathcal{F}$ be an object of
$\textit{Mod}(\mathcal{X}_{\acute{e}tale}, \mathcal{O}_\mathcal{X})$
which is locally quasi-coherent and has the flat base change property.
Then each $R^ig_*\mathcal{F}$ (computed in the \'etale topology)
has the flat base change property.
\end{lemma}

\begin{proof}
We will use
Lemma \ref{lemma-general-pushforward}
to prove this. For every algebraic stack $\mathcal{X}$ let
$\mathcal{M}_\mathcal{X}$ denote the full subcategory of
$\textit{Mod}(\mathcal{X}_{\acute{e}tale}, \mathcal{O}_\mathcal{X})$
consisting of locally quasi-coherent sheaves with the flat base
change property. Once we verify conditions (1) -- (4) of
Lemma \ref{lemma-general-pushforward}
the lemma will follow. Properties (1), (2), and (3) follow from
Sheaves on Stacks, Lemmas \ref{stacks-sheaves-lemma-pullback-lqc} and
\ref{stacks-sheaves-lemma-lqc-colimits}
and
Lemmas \ref{lemma-check-lqc-on-etale-covering} and
\ref{lemma-check-flat-comparison-on-etale-covering}.
Thus it suffices to show part (4).

\medskip\noindent
Suppose $f : \mathcal{X} \to \mathcal{Y}$ is a morphism of algebraic stacs
such that $\mathcal{X}$ and $\mathcal{Y}$ are representable by affine
schemes $X$ and $Y$. In this case, suppose that
$\psi : y \to y'$ is a morphism of $\mathcal{Y}$ lying over
a flat morphism $b : V \to V'$ of schemes. For clarity denote
$\mathcal{V} = (\Sch/V)_{fppf}$ and $\mathcal{V}' = (\Sch/V')_{fppf}$
the corresponding algebraic stacks. Consider the diagram
of algebraic stacks
$$
\xymatrix{
\mathcal{Z} \ar[d]_{f''} \ar[r]_a &
\mathcal{Z}' \ar[r]_{x'} \ar[d]_{f'} & \mathcal{X} \ar[d]^f \\
\mathcal{V} \ar[r]^b & \mathcal{V}' \ar[r]^{y'} & \mathcal{Y}
}
$$
with both squares cartesian. As $f$ is representable by schemes
(and quasi-compact and separated -- even affine) we see that $\mathcal{Z}$ and
$\mathcal{Z}'$ are representable by schemes $Z$ and $Z'$ and in
fact $Z = V \times_{V'} Z'$. Since $\mathcal{F}$ has the flat
base change property we see that
$$
a_{small}^*\big(\mathcal{F}|_{Z'_{\acute{e}tale}}\big)
\longrightarrow
\mathcal{F}|_{Z_{\acute{e}tale}}
$$
is an isomorphism. Moreover,
$$
R^if_*\mathcal{F}|_{V'_{\acute{e}tale}} =
R^i(f')_{small, *}\big(\mathcal{F}|_{Z'_{\acute{e}tale}}\big)
$$
and
$$
R^if_*\mathcal{F}|_{V_{\acute{e}tale}} =
R^i(f'')_{small, *}\big(\mathcal{F}|_{Z_{\acute{e}tale}}\big)
$$
by
Sheaves on Stacks, Lemma
\ref{stacks-sheaves-lemma-compare-representable-morphism-cohomology}.
Hence we see that the comparision map
$$
c_\psi :
b_{small}^*(R^if_*\mathcal{F}|_{V'_{\acute{e}tale}})
\longrightarrow
R^if_*\mathcal{F}|_{V_{\acute{e}tale}}
$$
is an isomorphism by
Cohomology of Spaces, Lemma
\ref{spaces-cohomology-lemma-flat-base-change-cohomology}.
Thus $R^if_*\mathcal{F}$ has the flat base change property.
Since $R^if_*\mathcal{F}$ is locally quasi-coherent by
Lemma \ref{lemma-pushforward-locally-quasi-coherent}
we win.
\end{proof}

\begin{proposition}
\label{proposition-lcq-flat-base-change}
Summary of results on locally quasi-coherent modules having the flat
base change property.
\begin{enumerate}
\item Let $\mathcal{X}$ be an algebraic stack.
If $\mathcal{F}$ is an object of
$\textit{Mod}(\mathcal{X}_{\acute{e}tale}, \mathcal{O}_\mathcal{X})$
which is locally quasi-coherent and has the flat base change property,
then $\mathcal{F}$ is a sheaf for the fppf topology, i.e., it is
an object of $\textit{Mod}(\mathcal{O}_\mathcal{X})$.
\item The category of modules which are locally quasi-coherent
and have the flat base change property is a weak Serre subcategory
$\mathcal{M}_\mathcal{X}$ of both $\textit{Mod}(\mathcal{O}_\mathcal{X})$
and $\textit{Mod}(\mathcal{X}_{\acute{e}tale}, \mathcal{O}_\mathcal{X})$.
\item Pullback $f^*$ along any morphism of algebraic stacks
$f : \mathcal{X} \to \mathcal{Y}$ induces a functor
$f^* : \mathcal{M}_\mathcal{Y} \to \mathcal{M}_\mathcal{X}$.
\item If $f : \mathcal{X} \to \mathcal{Y}$ is a
quasi-compact and quasi-separated morphism of algebraic stacks
and $\mathcal{F}$ is an object of $\mathcal{M}_\mathcal{X}$, then
\begin{enumerate}
\item the derived direct image $Rf_*\mathcal{F}$ and the higher direct
images $R^if_*\mathcal{F}$ can be computed in either the \'etale or the
fppf topology with the same result, and
\item each $R^if_*\mathcal{F}$ is an object of $\mathcal{M}_\mathcal{Y}$.
\end{enumerate}
\end{enumerate}
\end{proposition}

\begin{proof}
Part (1) is
Sheaves on Stacks, Lemma
\ref{stacks-sheaves-lemma-lqc-flat-base-change-fppf-sheaf}.

\medskip\noindent
Part (2) for the embedding $\mathcal{M}_\mathcal{X} \subset
\textit{Mod}(\mathcal{X}_{\acute{e}tale}, \mathcal{O}_\mathcal{X})$
we have seen in the proof of
Lemma \ref{lemma-flat-comparison}.
Let us prove (2) for the embedding
$\mathcal{M}_\mathcal{X} \subset \textit{Mod}(\mathcal{O}_\mathcal{X})$.
Let $\varphi : \mathcal{F} \to \mathcal{G}$ be a morphism between
objects of $\mathcal{M}_\mathcal{X}$. Since $\text{Ker}(\varphi)$
is the same whether computed in the \'etale or the fppf
topology, we see that $\text{Ker}(\varphi)$ is in
$\mathcal{M}_\mathcal{X}$ by the \'etale case. On the other hand,
the cokernel computed in the fppf topology is the fppf sheafification
of the cokernel computed in the \'etale topology. However, this
\'etale cokernel is in $\mathcal{M}_\mathcal{X}$ hence an fppf sheaf
by (1) and we see that the cokernel is in $\mathcal{M}_\mathcal{X}$.
Finally, suppose that
$$
0 \to \mathcal{F}_1 \to \mathcal{F}_2 \to \mathcal{F}_3 \to 0
$$
is an exact sequence in $\textit{Mod}(\mathcal{O}_\mathcal{X})$
(i.e., using the fppf topology) with $\mathcal{F}_1$, $\mathcal{F}_2$
in $\mathcal{M}_\mathcal{X}$. In order to show that $\mathcal{F}_2$
is an object of $\mathcal{M}_\mathcal{X}$ it suffices to show that
the sequence is also exact in the \'etale topology. To do this it
suffices to show that any element of $H^1_{fppf}(x, \mathcal{F}_1)$
becomes zero on the members of an \'etale covering of $x$ (for any
object $x$ of $\mathcal{X}$). This is true because
$H^1_{fppf}(x, \mathcal{F}_1) = H^1_{\acute{e}tale}(x, \mathcal{F}_1)$ by
Sheaves on Stacks, Lemma \ref{stacks-sheaves-lemma-compare-fppf-etale}
and because of locality of cohohomology, see
Cohomology on Sites, Lemma
\ref{sites-cohomology-lemma-kill-cohomology-class-on-covering}.
This proves (2).

\medskip\noindent
Part (3) follows from
Lemma \ref{lemma-check-flat-comparison-on-etale-covering}
and
Sheaves on Stacks, Lemma \ref{stacks-sheaves-lemma-pullback-lqc}.

\medskip\noindent
Part (4)(b) for $R^if_*\mathcal{F}$ computed in the \'etale cohomology
follows from Lemma \ref{lemma-flat-comparison}.
Whereupon part (4)(a) follows from
Sheaves on Stacks, Lemma \ref{stacks-sheaves-lemma-compare-fppf-etale}
combined with (1) above.
\end{proof}






\section{Parasitic modules}
\label{section-parasitic}

\noindent
The following definition is compatible with
Adequate, Definition \ref{adequate-definition-parasitic}.

\begin{definition}
\label{definition-parasitic}
Let $\mathcal{X}$ be an algebraic stack.
A presheaf of $\mathcal{O}_\mathcal{X}$-modules $\mathcal{F}$ is
{\it parasitic} if we have $\mathcal{F}(x) = 0$ for any object $x$
of $\mathcal{X}$ which lies over a scheme $U$ such that the corresponding
morphism $x : U \to \mathcal{X}$ is flat.
\end{definition}

\noindent
Here is a lemma with some properties of this notion.

\begin{lemma}
\label{lemma-parasitic}
Let $\mathcal{X}$ be an algebraic stack. Let $\mathcal{F}$
be a presheaf of $\mathcal{O}_\mathcal{X}$-modules.
\begin{enumerate}
\item If $\mathcal{F}$ is parasitic and
$g : \mathcal{Y} \to \mathcal{X}$ is a flat morphism of algebraic stacks,
then $g^*\mathcal{F}$ is parasitic.
\item For $\tau \in \{Zariski, \acute{e}tale, smooth, syntomic, fppf\}$
we have
\begin{enumerate}
\item the $\tau$ sheafification of a parasitic presheaf of modules is
parasitic, and
\item the full subcategory of
$\textit{Mod}(\mathcal{X}_\tau, \mathcal{O}_\mathcal{X})$
consisting of parasitic modules is a Serre subcategory.
\end{enumerate}
\item Suppose $\mathcal{F}$ is a sheaf for the \'etale topology.
Let $f_i : \mathcal{X}_i \to \mathcal{X}$ be a family of
smooth morphisms of algebraic stacks such that
$|\mathcal{X}| = \bigcup_i |f_i|(|\mathcal{X}_i|)$. If each
$f_i^*\mathcal{F}$ is parasitic then so is $\mathcal{F}$.
\item Suppose $\mathcal{F}$ is a sheaf for the fppf topology.
Let $f_i : \mathcal{X}_i \to \mathcal{X}$ be a family of
flat and locally finitely presented morphisms of algebraic stacks such that
$|\mathcal{X}| = \bigcup_i |f_i|(|\mathcal{X}_i|)$. If each
$f_i^*\mathcal{F}$ is parasitic then so is $\mathcal{F}$.
\end{enumerate}
\end{lemma}

\begin{proof}
To see part (1) let $y$ be an object of $\mathcal{Y}$ which lies
over a scheme $V$ such that the corresponding morphism $y : V \to \mathcal{Y}$
is flat. Then $g(y) : V \to \mathcal{Y} \to \mathcal{X}$ is flat
as a composition of flat morphisms (see
Morphisms of Stacks, Lemma \ref{stacks-morphisms-lemma-composition-flat})
hence $\mathcal{F}(g(y))$ is zero by assumption. Since
$g^*\mathcal{F} = g^{-1}\mathcal{F}(y) = \mathcal{F}(g(y))$ we conclude
$g^*\mathcal{F}$ is parasitic.

\medskip\noindent
To see part (2)(a) note that if $\{x_i \to x\}$ is a $\tau$-covering
of $\mathcal{X}$, then each of the morphisms $x_i \to x$ lies
over a flat morphism of schemes. Hence if $x$ lies over a scheme
$U$ such that $x : U \to \mathcal{X}$ is flat, so do all of the
objects $x_i$. Hence the presheaf $\mathcal{F}^+$ (see
Sites, Section \ref{sites-section-sheafification})
is parasitic if the presheaf $\mathcal{F}$ is
parasitic. This proves (2)(a) as the sheafification of $\mathcal{F}$
is $(\mathcal{F}^+)^+$.

\medskip\noindent
Let $\mathcal{F}$ be a parasitic $\tau$-module. It is immediate from the
definitions that any submodule of $\mathcal{F}$ is parasitic. On the other
hand, if $\mathcal{F}' \subset \mathcal{F}$ is a submodule, then it is
equally clear that the presheaf
$x \mapsto \mathcal{F}(x)/\mathcal{F}'(x)$
is parasitic. Hence the quotient $\mathcal{F}/\mathcal{F}'$ is a parasitic
module by (2)(a). Finally, we have to show that given a short exact sequence
$0 \to \mathcal{F}_1 \to \mathcal{F}_2 \to \mathcal{F}_3 \to 0$
with $\mathcal{F}_1$ and $\mathcal{F}_3$ parasitic, then $\mathcal{F}_2$
is parasitic. This follows immediately on evaluating on $x$ lying
over a scheme flat over $\mathcal{X}$. This proves (2)(b), see
Homology, Lemma \ref{homology-lemma-characterize-serre-subcategory}.

\medskip\noindent
Let $f_i : \mathcal{X}_i \to \mathcal{X}$ be a jointly surjective family of
smooth morphisms of algebraic stacks and assume each $f_i^*\mathcal{F}$
is parasitic. Let $x$ be an object of $\mathcal{X}$ which lies over a
scheme $U$ such that $x : U \to \mathcal{X}$ is flat. Consider a surjective
smooth covering $W_i \to U \times_{x, \mathcal{X}} \mathcal{X}_i$.
Denote $y_i : W_i \to \mathcal{X}_i$ the projection. It follows
that $\{f_i(y_i) \to x\}$ is a covering for the smooth topology
on $\mathcal{X}$. Since a composition of flat morphisms is flat we see that
$f_i^*\mathcal{F}(y_i) = 0$. On the other hand, as we saw in the proof
of (1), we have $f_i^*\mathcal{F}(y_i) = \mathcal{F}(f_i(y_i))$.
Hence we see that for some smooth covering $\{x_i \to x\}_{i \in I}$
in $\mathcal{X}$ we have $\mathcal{F}(x_i) = 0$. This implies
$\mathcal{F}(x) = 0$ because the smooth topology is the same as
as the \'etale topology, see
More on Morphisms, Lemma \ref{more-morphisms-lemma-etale-dominates-smooth}.
Namely, $\{x_i \to x\}_{i \in I}$ lies over a smooth covering
$\{U_i \to U\}_{i \in I}$ of schemes. By the lemma just referenced
there exists an \'etale covering $\{V_j \to U\}_{j \in J}$ which
refines $\{U_i \to U\}_{i \in I}$. Denote $x'_j = x|_{V_j}$.
Then $\{x'_j \to x\}$ is an \'etale covering in $\mathcal{X}$
refining $\{x_i \to x\}_{i \in I}$. This means the map
$\mathcal{F}(x) \to \prod_{j \in J} \mathcal{F}(x'_j)$, which is
injective as $\mathcal{F}$ is a sheaf in the \'etale topology,
factors through $\mathcal{F}(x) \to \prod_{i \in I} \mathcal{F}(x_i)$
which is zero. Hence $\mathcal{F}(x) = 0$ as desired.

\medskip\noindent
Proof of (4): omitted. Hint: similar, but simpler, than the proof of (3).
\end{proof}

\medskip\noindent
Parasitic modules are preserved under absolutely any pushforward.

\begin{lemma}
\label{lemma-pushforward-parasitic}
Let $\tau \in \{\acute{e}tale, fppf\}$.
Let $\mathcal{X}$ be an algebraic stack.
Let $\mathcal{F}$ be a parasitic object of
$\textit{Mod}(\mathcal{X}_\tau, \mathcal{O}_\mathcal{X})$.
\begin{enumerate}
\item $H^i_\tau(\mathcal{X}, \mathcal{F}) = 0$ for all $i$.
\item Let $f : \mathcal{X} \to \mathcal{Y}$ be a morphism of algebraic stacks.
Then $R^if_*\mathcal{F}$ (computed in $\tau$-topology) is a
parasitic object of $\textit{Mod}(\mathcal{Y}_\tau, \mathcal{O}_\mathcal{Y})$.
\end{enumerate}
\end{lemma}

\begin{proof}
We first reduce (2) to (1).
By Sheaves on Stacks, Lemma \ref{stacks-sheaves-lemma-pushforward-restriction}
we see that $R^if_*\mathcal{F}$ is the sheaf associated to the presheaf
$$
y \longmapsto
H^i_\tau\Big(V \times_{y, \mathcal{Y}} \mathcal{X},
\ \text{pr}^{-1}\mathcal{F}\Big)
$$
Here $y$ is a typical object of $\mathcal{Y}$ lying over the scheme $V$.
By Lemma \ref{lemma-parasitic} it suffices to show that
these cohomology groups are zero when $y : V \to \mathcal{Y}$ is flat.
Note that $\text{pr} : V \times_{y, \mathcal{Y}} \mathcal{X} \to \mathcal{X}$
is flat as a base change of $y$. Hence by
Lemma \ref{lemma-parasitic} we see that $\text{pr}^{-1}\mathcal{F}$
is parasitic. Thus it suffices to prove (1).

\medskip\noindent
To see (1) we can use the spectral sequence of
Sheaves on Stacks, Lemma
\ref{stacks-sheaves-proposition-smooth-covering-compute-cohomology}
to reduce this to the the case where $\mathcal{X}$
is an algebraic stack representable by an algebraic space.
Note that in the spectral sequence each
$f_p^{-1}\mathcal{F} = f_p^*\mathcal{F}$ is a parasitic module by
Lemma \ref{lemma-parasitic} because the morphisms
$f_p : \mathcal{U}_p =
\mathcal{U} \times_\mathcal{X} \ldots
\times_\mathcal{X} \mathcal{U} \to \mathcal{X}$ are flat.
Reusing this spectral sequence one more time (as in the
proof of the key Lemma \ref{lemma-general-pushforward})
we reduce to the case where the
algebraic stack $\mathcal{X}$ is representable by a scheme $X$.
Then $H^i_\tau(\mathcal{X}, \mathcal{F}) = H^i((\Sch/X)_\tau, \mathcal{F})$.
In this case the vanishing follows easily from an argument
with {\v C}ech coverings, see
Adequate, Lemma \ref{adequate-lemma-cohomology-parasitic}.
\end{proof}






\section{Quasi-coherent sheaves}
\label{section-quasi-coherent}

\noindent
We have seen that the category of quasi-coherent sheaves on an algebraic stack
is equivalent to the category of quasi-coherent modules on a presentation, see
Sheaves on Stacks, Section
\ref{stacks-sheaves-section-quasi-coherent-algebraic-stacks}.
This fact is the basis for the following.

\begin{lemma}
\label{lemma-adjoint}
Let $\mathcal{X}$ be an algebraic stack. Let $\mathcal{M}_\mathcal{X}$
be the category of locally quasi-coherent modules with the
flat base change property, see
Proposition \ref{proposition-lcq-flat-base-change}.
Then the inclusion functor
$i : \textit{QCoh}(\mathcal{O}_\mathcal{X}) \to \mathcal{M}_\mathcal{X}$
has a right adjoint
$Q : \mathcal{M}_\mathcal{X} \to \textit{QCoh}(\mathcal{O}_\mathcal{X})$
such that $Q \circ i$ is the identity functor.
\end{lemma}

\begin{proof}
Coming soon.
\end{proof}








\section{Other chapters}

\begin{multicols}{2}
\begin{enumerate}
\item \hyperref[introduction-section-phantom]{Introduction}
\item \hyperref[conventions-section-phantom]{Conventions}
\item \hyperref[sets-section-phantom]{Set Theory}
\item \hyperref[categories-section-phantom]{Categories}
\item \hyperref[topology-section-phantom]{Topology}
\item \hyperref[sheaves-section-phantom]{Sheaves on Spaces}
\item \hyperref[algebra-section-phantom]{Commutative Algebra}
\item \hyperref[sites-section-phantom]{Sites and Sheaves}
\item \hyperref[homology-section-phantom]{Homological Algebra}
\item \hyperref[derived-section-phantom]{Derived Categories}
\item \hyperref[more-algebra-section-phantom]{More Algebra}
\item \hyperref[simplicial-section-phantom]{Simplicial Methods}
\item \hyperref[modules-section-phantom]{Sheaves of Modules}
\item \hyperref[sites-modules-section-phantom]{Modules on Sites}
\item \hyperref[injectives-section-phantom]{Injectives}
\item \hyperref[cohomology-section-phantom]{Cohomology of Sheaves}
\item \hyperref[sites-cohomology-section-phantom]{Cohomology on Sites}
\item \hyperref[hypercovering-section-phantom]{Hypercoverings}
\item \hyperref[schemes-section-phantom]{Schemes}
\item \hyperref[constructions-section-phantom]{Constructions of Schemes}
\item \hyperref[properties-section-phantom]{Properties of Schemes}
\item \hyperref[morphisms-section-phantom]{Morphisms of Schemes}
\item \hyperref[coherent-section-phantom]{Coherent Cohomology}
\item \hyperref[divisors-section-phantom]{Divisors}
\item \hyperref[limits-section-phantom]{Limits of Schemes}
\item \hyperref[varieties-section-phantom]{Varieties}
\item \hyperref[chow-section-phantom]{Chow Homology}
\item \hyperref[topologies-section-phantom]{Topologies on Schemes}
\item \hyperref[descent-section-phantom]{Descent}
\item \hyperref[more-morphisms-section-phantom]{More on Morphisms}
\item \hyperref[flat-section-phantom]{More on Flatness}
\item \hyperref[groupoids-section-phantom]{Groupoid Schemes}
\item \hyperref[more-groupoids-section-phantom]{More on Groupoid Schemes}
\item \hyperref[etale-section-phantom]{\'Etale Morphisms of Schemes}
\item \hyperref[etale-cohomology-section-phantom]{\'Etale Cohomology}
\item \hyperref[spaces-section-phantom]{Algebraic Spaces}
\item \hyperref[spaces-properties-section-phantom]{Properties of Algebraic Spaces}
\item \hyperref[spaces-morphisms-section-phantom]{Morphisms of Algebraic Spaces}
\item \hyperref[spaces-topologies-section-phantom]{Topologies on Algebraic Spaces}
\item \hyperref[spaces-descent-section-phantom]{Descent and Algebraic Spaces}
\item \hyperref[spaces-more-morphisms-section-phantom]{More on Morphisms of Spaces}
\item \hyperref[quot-section-phantom]{Quot and Hilbert Spaces}
\item \hyperref[stacks-section-phantom]{Stacks}
\item \hyperref[spaces-groupoids-section-phantom]{Groupoids in Algebraic Spaces}
\item \hyperref[spaces-more-groupoids-section-phantom]{More on Groupoids in Spaces}
\item \hyperref[bootstrap-section-phantom]{Bootstrap}
\item \hyperref[examples-stacks-section-phantom]{Examples of Stacks}
\item \hyperref[groupoids-quotients-section-phantom]{Quotients of Groupoids}
\item \hyperref[algebraic-section-phantom]{Algebraic Stacks}
\item \hyperref[criteria-section-phantom]{Criteria for Representability}
\item \hyperref[stacks-properties-section-phantom]{Properties of Algebraic Stacks}
\item \hyperref[stacks-morphisms-section-phantom]{Morphisms of Algebraic Stacks}
\item \hyperref[examples-section-phantom]{Examples}
\item \hyperref[exercises-section-phantom]{Exercises}
\item \hyperref[guide-section-phantom]{Guide to Literature}
\item \hyperref[desirables-section-phantom]{Desirables}
\item \hyperref[coding-section-phantom]{Coding Style}
\item \hyperref[fdl-section-phantom]{GNU Free Documentation License}
\item \hyperref[index-section-phantom]{Auto Generated Index}
\end{enumerate}
\end{multicols}


\bibliography{my}
\bibliographystyle{amsalpha}

\end{document}
