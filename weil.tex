\IfFileExists{stacks-project.cls}{%
\documentclass{stacks-project}
}{%
\documentclass{amsart}
}

% The following AMS packages are automatically loaded with
% the amsart documentclass:
%\usepackage{amsmath}
%\usepackage{amssymb}
%\usepackage{amsthm}

% For dealing with references we use the comment environment
\usepackage{verbatim}
\newenvironment{reference}{\comment}{\endcomment}
%\newenvironment{reference}{}{}
\newenvironment{slogan}{\comment}{\endcomment}
\newenvironment{history}{\comment}{\endcomment}

% For commutative diagrams you can use
% \usepackage{amscd}
\usepackage[all]{xy}

% We use 2cell for 2-commutative diagrams.
\xyoption{2cell}
\UseAllTwocells

% To put source file link in headers.
% Change "template.tex" to "this_filename.tex"
% \usepackage{fancyhdr}
% \pagestyle{fancy}
% \lhead{}
% \chead{}
% \rhead{Source file: \url{template.tex}}
% \lfoot{}
% \cfoot{\thepage}
% \rfoot{}
% \renewcommand{\headrulewidth}{0pt}
% \renewcommand{\footrulewidth}{0pt}
% \renewcommand{\headheight}{12pt}

\usepackage{multicol}

% For cross-file-references
\usepackage{xr-hyper}

% Package for hypertext links:
\usepackage{hyperref}

% For any local file, say "hello.tex" you want to link to please
% use \externaldocument[hello-]{hello}
\externaldocument[introduction-]{introduction}
\externaldocument[conventions-]{conventions}
\externaldocument[sets-]{sets}
\externaldocument[categories-]{categories}
\externaldocument[topology-]{topology}
\externaldocument[sheaves-]{sheaves}
\externaldocument[sites-]{sites}
\externaldocument[stacks-]{stacks}
\externaldocument[fields-]{fields}
\externaldocument[algebra-]{algebra}
\externaldocument[brauer-]{brauer}
\externaldocument[homology-]{homology}
\externaldocument[derived-]{derived}
\externaldocument[simplicial-]{simplicial}
\externaldocument[more-algebra-]{more-algebra}
\externaldocument[smoothing-]{smoothing}
\externaldocument[modules-]{modules}
\externaldocument[sites-modules-]{sites-modules}
\externaldocument[injectives-]{injectives}
\externaldocument[cohomology-]{cohomology}
\externaldocument[sites-cohomology-]{sites-cohomology}
\externaldocument[dga-]{dga}
\externaldocument[dpa-]{dpa}
\externaldocument[hypercovering-]{hypercovering}
\externaldocument[schemes-]{schemes}
\externaldocument[constructions-]{constructions}
\externaldocument[properties-]{properties}
\externaldocument[morphisms-]{morphisms}
\externaldocument[coherent-]{coherent}
\externaldocument[divisors-]{divisors}
\externaldocument[limits-]{limits}
\externaldocument[varieties-]{varieties}
\externaldocument[topologies-]{topologies}
\externaldocument[descent-]{descent}
\externaldocument[perfect-]{perfect}
\externaldocument[more-morphisms-]{more-morphisms}
\externaldocument[flat-]{flat}
\externaldocument[groupoids-]{groupoids}
\externaldocument[more-groupoids-]{more-groupoids}
\externaldocument[etale-]{etale}
\externaldocument[chow-]{chow}
\externaldocument[intersection-]{intersection}
\externaldocument[pic-]{pic}
\externaldocument[adequate-]{adequate}
\externaldocument[dualizing-]{dualizing}
\externaldocument[duality-]{duality}
\externaldocument[discriminant-]{discriminant}
\externaldocument[local-cohomology-]{local-cohomology}
\externaldocument[curves-]{curves}
\externaldocument[resolve-]{resolve}
\externaldocument[models-]{models}
\externaldocument[pione-]{pione}
\externaldocument[etale-cohomology-]{etale-cohomology}
\externaldocument[proetale-]{proetale}
\externaldocument[crystalline-]{crystalline}
\externaldocument[spaces-]{spaces}
\externaldocument[spaces-properties-]{spaces-properties}
\externaldocument[spaces-morphisms-]{spaces-morphisms}
\externaldocument[decent-spaces-]{decent-spaces}
\externaldocument[spaces-cohomology-]{spaces-cohomology}
\externaldocument[spaces-limits-]{spaces-limits}
\externaldocument[spaces-divisors-]{spaces-divisors}
\externaldocument[spaces-over-fields-]{spaces-over-fields}
\externaldocument[spaces-topologies-]{spaces-topologies}
\externaldocument[spaces-descent-]{spaces-descent}
\externaldocument[spaces-perfect-]{spaces-perfect}
\externaldocument[spaces-more-morphisms-]{spaces-more-morphisms}
\externaldocument[spaces-flat-]{spaces-flat}
\externaldocument[spaces-groupoids-]{spaces-groupoids}
\externaldocument[spaces-more-groupoids-]{spaces-more-groupoids}
\externaldocument[bootstrap-]{bootstrap}
\externaldocument[spaces-pushouts-]{spaces-pushouts}
\externaldocument[groupoids-quotients-]{groupoids-quotients}
\externaldocument[spaces-more-cohomology-]{spaces-more-cohomology}
\externaldocument[spaces-simplicial-]{spaces-simplicial}
\externaldocument[formal-spaces-]{formal-spaces}
\externaldocument[restricted-]{restricted}
\externaldocument[spaces-resolve-]{spaces-resolve}
\externaldocument[formal-defos-]{formal-defos}
\externaldocument[defos-]{defos}
\externaldocument[cotangent-]{cotangent}
\externaldocument[examples-defos-]{examples-defos}
\externaldocument[algebraic-]{algebraic}
\externaldocument[examples-stacks-]{examples-stacks}
\externaldocument[stacks-sheaves-]{stacks-sheaves}
\externaldocument[criteria-]{criteria}
\externaldocument[artin-]{artin}
\externaldocument[quot-]{quot}
\externaldocument[stacks-properties-]{stacks-properties}
\externaldocument[stacks-morphisms-]{stacks-morphisms}
\externaldocument[stacks-limits-]{stacks-limits}
\externaldocument[stacks-cohomology-]{stacks-cohomology}
\externaldocument[stacks-perfect-]{stacks-perfect}
\externaldocument[stacks-introduction-]{stacks-introduction}
\externaldocument[stacks-more-morphisms-]{stacks-more-morphisms}
\externaldocument[stacks-geometry-]{stacks-geometry}
\externaldocument[moduli-]{moduli}
\externaldocument[moduli-curves-]{moduli-curves}
\externaldocument[examples-]{examples}
\externaldocument[exercises-]{exercises}
\externaldocument[guide-]{guide}
\externaldocument[desirables-]{desirables}
\externaldocument[coding-]{coding}
\externaldocument[obsolete-]{obsolete}
\externaldocument[fdl-]{fdl}
\externaldocument[index-]{index}

% Theorem environments.
%
\theoremstyle{plain}
\newtheorem{theorem}[subsection]{Theorem}
\newtheorem{proposition}[subsection]{Proposition}
\newtheorem{lemma}[subsection]{Lemma}

\theoremstyle{definition}
\newtheorem{definition}[subsection]{Definition}
\newtheorem{example}[subsection]{Example}
\newtheorem{exercise}[subsection]{Exercise}
\newtheorem{situation}[subsection]{Situation}

\theoremstyle{remark}
\newtheorem{remark}[subsection]{Remark}
\newtheorem{remarks}[subsection]{Remarks}

\numberwithin{equation}{subsection}

% Macros
%
\def\lim{\mathop{\rm lim}\nolimits}
\def\colim{\mathop{\rm colim}\nolimits}
\def\Spec{\mathop{\rm Spec}}
\def\Hom{\mathop{\rm Hom}\nolimits}
\def\Ext{\mathop{\rm Ext}\nolimits}
\def\SheafHom{\mathop{\mathcal{H}\!{\it om}}\nolimits}
\def\SheafExt{\mathop{\mathcal{E}\!{\it xt}}\nolimits}
\def\Sch{\textit{Sch}}
\def\Mor{\mathop{\rm Mor}\nolimits}
\def\Ob{\mathop{\rm Ob}\nolimits}
\def\Sh{\mathop{\textit{Sh}}\nolimits}
\def\NL{\mathop{N\!L}\nolimits}
\def\proetale{{pro\text{-}\acute{e}tale}}
\def\etale{{\acute{e}tale}}
\def\QCoh{\textit{QCoh}}
\def\Ker{\mathop{\rm Ker}}
\def\Im{\mathop{\rm Im}}
\def\Coker{\mathop{\rm Coker}}
\def\Coim{\mathop{\rm Coim}}

%
% Macros for moduli stacks/spaces
%
\def\QCohstack{\mathcal{QC}\!{\it oh}}
\def\Cohstack{\mathcal{C}\!{\it oh}}
\def\Spacesstack{\mathcal{S}\!{\it paces}}
\def\Quotfunctor{{\rm Quot}}
\def\Hilbfunctor{{\rm Hilb}}
\def\Curvesstack{\mathcal{C}\!{\it urves}}
\def\Polarizedstack{\mathcal{P}\!{\it olarized}}
\def\Complexesstack{\mathcal{C}\!{\it omplexes}}
% \Pic is the operator that assigns to X its picard group, usage \Pic(X)
% \Picardstack_{X/B} denotes the Picard stack of X over B
% \Picardfunctor_{X/B} denotes the Picard functor of X over B
\def\Pic{\mathop{\rm Pic}\nolimits}
\def\Picardstack{\mathcal{P}\!{\it ic}}
\def\Picardfunctor{{\rm Pic}}
\def\Deformationcategory{\mathcal{D}\!{\it ef}}


% OK, start here.
%
\begin{document}

\title{Weil Cohomology Theories, UNDER CONSTRUCTION}


\maketitle

\phantomsection
\label{section-phantom}

\tableofcontents

\section{Introduction}
\label{section-introduction}

\noindent
In this chapter we discuss Weil cohomology theories for smooth
projective schemes over any base field. In the case of an algebraically
closed base field, our notion is the same as the notion introduced
in \cite{Kleiman-cycles}, see (insert future reference here).





\section{Conventions and notation}
\label{section-conventions}

\noindent
Let $R$ be a ring. A {\it graded commutative $R$-algebra} $A$ is a
strictly commutative differential graded $R$-algebra
(Differential Graded Algebra, Definitions \ref{dga-definition-dga} and
\ref{dga-definition-cdga}) whose differential is zero. Thus $A$
is an $R$-module endowed with a grading
$A = \bigoplus_{n \in \mathbf{Z}} A^n$ by
$R$-submodules. The $R$-bilinear multiplication
$$
A^n \times A^m \longrightarrow A^{n + m},\quad
\alpha \times \beta \longmapsto \alpha \cup \beta
$$
will be called the {\it cup product} in this chapter.
The commutativity constraint is
$\alpha \cup \beta = (-1)^{nm} \beta \cup \alpha$ if
$\alpha \in A^n$ and $\beta \in A^m$. Finally, there is
a multiplicative unit $1 \in A^0$, or equivalently, there is an
additive and multiplicative map $R \to A^0$ which is compatible the
$R$-module structure on $A$.

\medskip\noindent
Let $k$ be a field and let $X$ be a scheme of finite type over $k$.
The Chow groups $A_k(X)$ of $X$ have been defined in
Chow Homology, Definition \ref{chow-definition-rational-equivalence}.
Given a proper morphism $f : X \to Y$ of schemes of finite
type over $k$ there is a pushforward map $f_* : A_k(X) \to A_k(Y)$,
see Chow Homology, Section \ref{chow-section-proper-pushforward} and
Lemma \ref{chow-lemma-proper-pushforward-rational-equivalence}.
If $X$ is smooth over $k$ and equidimensional of dimension $d$, then
we have
$$
A^i(X) = A_{d - i}(X)
$$
see Chow Homology, Section \ref{chow-lemma-identify-chow-for-smooth}
and recall that the isomorphism sends $c \in A^i(X)$ to
$c \cap [X]_d \in A_{d - i}(X)$.
If $X$ smooth over $k$ and quasi-compact, then we can write canonically
$$
X = X_0 \amalg X_1 \amalg X_2 \amalg \ldots \amalg X_n
$$
into open and closed subschemes $X_d$ which are
equidimensional of dimension $d$. Set $[X] = \sum [X_d]_d$
as an element of $A_*(X)$. Since
$A_k(X) = \prod A_k(X_d)$ and $A^i(X) = \prod A^i(X_d)$
the map $c \mapsto c \cap [X]$ still defines an isomorphism
$$
A^*(X) = A_*(X)
$$
but it is no longer compatible with gradings, namely,
$$
A^i(X) = \bigoplus\nolimits_d A_{d - i}(X_d)
$$
The reader may use this as an alternative definition of $A^i(X)$.
There is an intersection product
$(\alpha, \beta) \mapsto \alpha \cdot \beta$ on $A_*(X)$
with the property that it sends $A^i(X) \times A^j(X)$ into $A^{i + j}(X)$,
see Chow Homology, Section \ref{chow-section-intersection-product}.
If $f : Y \to X$ is a morphism of schemes smooth over $k$, then
there is a pullback map
$$
f^* : A^i(X) \to A^i(Y),\quad
\alpha \mapsto f^*\alpha
$$
which is compatible with intersection products, see
Chow Homology, Lemma \ref{}.
Moreover, if $f$ is also proper, then
we have $f_*(\alpha \cdot f^*\beta) = f_*\alpha \cdot \beta$, see
Chow Homology, Lemma \ref{}.
We have $\alpha \cdot \beta = \Delta^*(\alpha \times \beta)$
if $\alpha, \beta$ are cycles on $X$ smooth over $k$.










\section{Weil cohomology theories}
\label{section-axioms}

\noindent
In this section we state precisely what data constitutes a
Weil cohomology theory, see \cite[Section 1.2]{Kleiman-cycles}.
We have modified the conditions to allow for nonalgebraically closed
base fields, but we have otherwise stuck to the original as
closely as possible.

\medskip\noindent
We fix a field $k$ (the base field).
We fix a field $F$ of characteristic $0$ (the coefficient field).
A Weil cohomology theory is given by data (D1), (D2), and (D3)
subject to axioms (A), (B), (C).

\medskip\noindent
The data is given by:
\begin{enumerate}
\item[(D1)] A contravariant functor $H^*$ from the category
of smooth projective schemes over $k$ to the category of
graded commutative $F$-algebras.
\item[(D2)] A $1$-dimensional $F$-vector space $F(1)$.
\item[(D3)] For every smooth projective scheme $X$ over $k$
a group homomorphism $\gamma : A^i(X) \to H^{2i}(X)(i)$.
\item[(D4)] For every smooth projective scheme $X$ over $k$
which is equidimensional of dimension $d$ a map
$\int_X : H^{2d}(X)(d) \to F$.
\end{enumerate}
We make some remarks to explain what this means and to introduce
some terminology associated with this.

\medskip\noindent
Remarks on (D1).
Given a smooth projective scheme $X$ over $k$ we say that $H^*(X)$
is the {\it cohomology} of $X$. Given a morphism $f : X \to Y$
of smooth projective schemes over $k$ we denote $f^* : H^*(Y) \to H^*(X)$
the map $H^*(f)$ and we call it the {\it pullback map}.

\medskip\noindent
Remarks on (D2).
The vector space $F(1)$ gives rise to {\it Tate twists} on the category of
$F$-vector spaces. Namely, for $n \in \mathbf{Z}$ we set
$F(n) = F(1)^{\otimes n}$ if $n \geq 0$, we set $F(-1) = \Hom_F(F(1), F)$,
and we set $F(n) = F(-1)^{\otimes - n}$ if $n < 0$. Please compare
with More on Algebra, Section \ref{more-algebra-section-picard}.
For an $F$-vector space $V$ we define then $n$th Tate twist
$$
V(n) = V \otimes_F F(n)
$$
We will use obvious notation, e.g., given $F$-vector spaces $U$, $V$
and $W$ and a linear map $U \otimes_F V \to W$ we obtain a linear
map $U(n) \otimes_F V(m) \to W(n + m)$ for $n, m \in \mathbf{Z}$.

\medskip\noindent
Remarks on (D3). The map $\gamma$ is called the {\it cycle class map}.
We say that $\gamma(\alpha)$ is the {\it cohomology class} of $\alpha$.
If $Z \subset Y \subset X$ are closed subschemes with $Y$ and $X$
smooth projective over $k$ and $Z$ integral, then $[Z]$ could
mean the class of the cycle $[Z]$ in $A^*(Y)$ or in $A^*(X)$.
In this case the notation $\gamma([Z])$ is abiguous and the true meaning
has to be deduced from context.

\medskip\noindent
Remarks on (D4). The map $\int_X$ is sometimes called the
{\it trace map} and is sometimes denoted $\text{Tr}_X$.

\medskip\noindent
The first axiom is often called {\it Poincar\'e duality}
\begin{enumerate}
\item[(A)] Let $X$ be a smooth projective scheme over $k$
which is equidimensional of dimension $d$. Then
\begin{enumerate}
\item $H^i(X) = 0$ unless $i \in [0, 2d]$.
\item $\dim_F H^i(X) < \infty$ for all $i$,
\item $H^i(X) \times H^{2d - i}(X)(d) \rightarrow
H^{2d}(X)(d) \rightarrow F$
is a perfect pairing for all $i$ where the final
map is the trace map $\int_X$.
\end{enumerate}
\end{enumerate}
Let $f : X \to Y$ be a morphism of smooth projective schemes with $X$
equidimensional of dimension $d$ and $Y$ is equidimensional of dimension $e$.
Using Poincar\'e duality we can define a {\it pushforward}
$$
f_* : H^{2d - i}(X)(d) \longrightarrow H^{2e - i}(Y)(e)
$$
as the contragredient of the linear map $f^* : H^i(Y) \to H^i(X)$. In a
formula, for $a \in H^{2d - i}(X)(d)$, the element $f_*a \in H^{2e - i}(Y)(e)$
is characterized by
$$
\int_X f^*b \cup a = \int_Y b \cup f_*a
$$
for all $b \in H^i(Y)$.

\begin{lemma}
\label{lemma-pushforward}
Assume given (D1), (D2), and (D4) satisfying (A). For $f : X \to Y$
a morphism of equidimensional smooth projective schemes over $k$ we have
$f_*(f^*b \cup a) = b \cup f_*a$. If $g : Y \to Z$ is a second morphism
with $Z$ smooth projective and equidimensional, then
$g_* \circ f_* = (g \circ f)_*$.
\end{lemma}

\begin{proof}
The first equality holds because
$$
\int_Y c \cup b \cup f_*a =
\int_X f^*c \cup f^*b \cup a =
\int_Y c \cup f_*(f^*b \cup a).
$$
The second equality holds because
$$
\int_Z c \cup (g \circ f)_*a = \int_X (g \circ f)^*c \cup a =
\int_X f^* g^* c \cup a = \int_Y g^*c \cup f_*a = \int_Z c \cup g_*f_*a
$$
This ends the proof.
\end{proof}

\noindent
The second axiom says that $H^*$ respects the monoidal structure
given by products via the {\it K\"unneth formula}
\begin{enumerate}
\item[(B)] Let $X$ and $Y$ be smooth projective schemes over $k$.
\begin{enumerate}
\item $\int_{\Spec(k)} : H^0(\Spec(k)) \to F$ sends $1$ to $1$,
\item $H^*(X) \otimes_F H^*(Y) \to H^*(X \times Y)$,
$\alpha \otimes \beta \mapsto \text{pr}_1^*\alpha \cup \text{pr}_2^*\beta$
is an isomorphism,
\item $H^*(X \amalg Y) \to H^*(X) \times H^*(Y)$,
$\alpha \longmapsto (\alpha|_X, \alpha|_Y)$ is an isomorphism.
\end{enumerate}
\end{enumerate}
If $H^*(-)$ is a cohomology theory defined only for smooth projective
varieties over $k$, then we can use axiom (B)(c) to
(uniquely) extend the functor to all smooth projective schemes over $k$.

\medskip\noindent
The third axiom concerns the cycle class maps
\begin{enumerate}
\item[(C)] The cycle class maps satisfy the following rules
\begin{enumerate}
\item for a morphism $f : X \to Y$ of smooth projective schemes over
$k$ we have $\gamma(f^*\beta) = f^*\gamma(\beta)$ for $\beta \in A^*(Y)$,
\item for a morphism $f : X \to Y$ of equidimensional smooth projective schemes
over $k$ we have
$\gamma(f_*\alpha) = f_*\gamma(\alpha)$ for $\alpha \in A^*(X)$,
\item for any smooth projective scheme $X$ over $k$ we have
$\gamma(\alpha \cdot \beta) = \gamma(\alpha) \cup \gamma(\beta)$
for $\alpha, \beta \in A^*(X)$, and
\item if $X$ is a smooth projective scheme equidimensional of dimension $d$
over $k$, then the image of $\gamma : A^d(X) \to H^{2d}(X)(d)$ generates
$H^{2d}(X)(d)$ over $H^0(X)$.
\end{enumerate}
\end{enumerate}
Let us elucidate condition (b). Namely, say that $X$ is equidimensional
of dimension $d$ and $Y$ is equidimensional of dimension $e$. Then we
see that pushforward on Chow groups gives
$$
f_* : A^{d - i}(X) = A_i(X) \to A_i(Y) = A^{e - i}(Y)
$$
Say $\alpha \in A^{d - i}(X)$. On the one hand, we have
$f_*\alpha \in A^{e - i}(Y)$ and hence
$\gamma(f_*\alpha) \in H^{2e - 2i}(Y)(e - i)$.
On the other hand, we have
$\gamma(\alpha) \in H^{2d - 2i}(X)(d - i)$ and hence
$f_*\gamma(\alpha) \in H^{2e - 2i}(Y)(e - i)$ as well.
Thus the condition $\gamma(f_*\alpha) = f_*\gamma(\alpha)$ makes sense.

\begin{remark}
\label{remark-replace-cup-product}
Let $X$ be a smooth projective scheme over $k$. We obtain a map
$$
H^*(X) \otimes_F H^*(X) \longrightarrow H^*(X \times X)
\xrightarrow{\Delta^*} H^*(X)
$$
where $\Delta^*$ is pullback along the diagonal morphism
$\Delta : X \to X \times X$. The composition is the cup product.
(Hints: pullback is an algebra homomorphism and
$\Delta^* \circ p^* = \text{id}$ and $\Delta^* \circ 1^* = \text{id}$.)
On the other hand, the intersection product
$\alpha \cdot \beta$ of cycles $\alpha, \beta$ on $X$ is the
pullback of the exterior product $\alpha \times \beta$ on $X \times X$.
It follows that in order to prove axiom (C)(c) it suffices to show
that $\gamma$ is compatible with exterior product (we leave the
precise formulation to the reader).
\end{remark}

\begin{definition}
\label{definition-weil-cohomology-theory}
Let $k$ be a field. Let $F$ be a field of characteristic $0$.
A {\it Weil cohomology theory} over $k$ with coefficients in $F$
is given by data (D1), (D2), (D3), and (D4) satisfying
Poincar\'e duality, the K\"unneth formula, and compatibility
with cycle classes, more precisely, satisfying (A), (B), and (C).
\end{definition}

\noindent
We give some consequences of these axioms.

\begin{lemma}
\label{lemma-base}
Let $H^*$ be a Weil cohomology theory
(Definition \ref{definition-weil-cohomology-theory}).
Set $S = \Spec(k)$.
Then $H^i(S) = 0$ for $i \not = 0$ and there is a
unique $F$-algebra isomorphism $F = H^0(S)$.
We have $\gamma([S]) = 1$ and
$\int_S 1 = 1$.
\end{lemma}

\begin{proof}
Observe that $S \times S = S$. By axiom (B)(a) we have $H^*(S)$
is nonzero and by axiom (B)(b) we have
$H^*(S) \otimes_F H^*(S) = H^*(S)$. Calculating dimensions on both sides
we see that $H^i(S) = 0$ unless $i = 0$ and we have
$F = H^0(S)$ via the unique isomorphism given by mapping $1 \in F$ to
$1 \in H^0(S)$. The element $\gamma([S]) \in H^0(S)$ is nonzero by
axiom (C)(d). Observe that $[S] \cdot [S] = [S]$. By axiom (C)(c)
we see that $\gamma([S])^2 = \gamma([S])$. Since $\gamma([S])$
is nonzero we conclude that $\gamma([S]) = 1$.
Finally, we have $\int_X 1 = 1$ by axiom (B)(a).
\end{proof}

\begin{lemma}
\label{lemma-unit}
Let $H^*$ be a Weil cohomology theory
(Definition \ref{definition-weil-cohomology-theory}).
Let $X$ be a nonempty smooth projective scheme over $k$.
Then $\gamma([X]) = 1$ and $1 \not = 0$ in $H^0(X)$.
\end{lemma}

\begin{proof}
Observe that $[X]$ is the pullback of $[S]$ by the structure morphism
$p : X \to S = \Spec(k)$. Hence we get $\gamma([X]) = 1$ by axiom (C)(a).
To see that $1 \not = 0$ it suffices to show that $H^*(X)$ is nonzero.
By axiom (B)(c) we may decompose $X$ into its irreducible components.
Hence we may and do assume $X$ is equidimensional of dimension $d$.

\medskip\noindent
Let $x \in X$ be a closed point whose residue field $k'$ is separable over $k$,
see Varieties, Lemma \ref{varieties-lemma-smooth-separable-closed-points-dense}.
Set $S' = \Spec(k')$ and let
$i : S' \to X$ be the inclusion morphism. Observe that
$p_*i_*[S'] = [k' : k][S]$ in $A_0(S) = A^0(S)$ is nonzero.
Using axiom (C)(b) twice and Lemma \ref{lemma-base}
we conclude that
$$
p_*i_*\gamma([S']) = \gamma([k' : k][S]) = [k' : k] \in H^0(S)
$$
is nonzero. Thus $i_*\gamma([S']) \in H^{2d}(X)(d)$ is nonzero
(because it maps to something nonzero via $p_*$). This concludes the proof.
\end{proof}

\begin{lemma}
\label{lemma-push-unit}
Let $H^*$ be a Weil cohomology theory
(Definition \ref{definition-weil-cohomology-theory}). Let $i : X \to Y$
be a closed immersion of smooth projective equidimensional schemes over $k$.
Then $\gamma([X]) = i_*1$ in $H^{2c}(Y)(c)$ where $c = \dim(Y) - \dim(X)$.
\end{lemma}

\begin{proof}
This is true because $1 = \gamma([X])$ in $H^0(X)$ by Lemma \ref{lemma-unit}
and then we can apply axiom (C)(b).
\end{proof}

\begin{lemma}
\label{lemma-dim-0}
Let $H^*$ be a Weil cohomology theory
(Definition \ref{definition-weil-cohomology-theory}).
Let $X$ be a smooth projective scheme of dimension zero over $k$.
Then
\begin{enumerate}
\item $H^i(X) = 0$ for $i \not = 0$,
\item $H^0(X)$ is a finite separable algebra over $F$,
\item $\dim_F H^0(X) = \deg(X \to \Spec(F))$,
\item $\int_X : H^0(X) \to F$ is the trace map,
\item $\gamma([X]) = 1$, and
\item $\int_X \gamma([X]) = \deg(X \to \Spec(k))$.
\end{enumerate}
\end{lemma}

\begin{proof}
We can write $X = \Spec(k')$ where $k'$ is a finite separable
algebra over $k$. Observe that $\deg(X \to \Spec(k)) = [k' : k]$.

\medskip\noindent
Choose a finite Galois extension $k''/k$ containing each of the
factors of $k'$. (Recall that a finite separable $k$-algebra is
a product of finite separable field extension of $k$.) Then we get
$$
k' \otimes_k k'' = \prod\nolimits_{\sigma \in \Hom_k(k', k'')} k''
$$
Setting $Y = \Spec(k'')$ axioms (B)(b) and (B)(c) give
$$
H^*(X) \otimes_F H^*(Y) =
\bigoplus\nolimits_{\sigma \in \Hom_k(k', k'')} H^*(Y)
$$
as graded commutative $F$-algebras. By Lemma \ref{lemma-unit} the
$F$-algebra $H^*(Y)$ is nonzero. Comparing dimensions on either side
of the displayed equation we conclude that $H^*(X)$ sits only in degree $0$ and
$\dim_F H^0(X) = [k' : k]$. Since
$$
H^0(X) \otimes_F H^0(Y) = H^0(Y) \times \ldots \times H^0(Y)
$$
as $F$-algebras, it follows that $H^0(X)$ is a separable $F$-algebra
because we may check this after the faithfully flat base change
$F \to H^0(Y)$. By Lemma \ref{lemma-unit} we have $\gamma([X]) = 1$.
Denote $f : X \to \Spec(k)$ the structure morphism. Then we have
$$
\int_X \gamma([X]) = \int_{\Spec(k)} f_*\gamma([X]) =
\int_{\Spec(k)} \gamma(f_*[X]) = [k' : k]
$$
The first equality is the definition of $f_*$ on cohomology.
The second equality follows from axiom (C)(b). The third equality
follows from $f_*[X] = [k' : k] [\Spec(k)]$ and the results
of Lemma \ref{lemma-base}.
\end{proof}

\begin{lemma}
\label{lemma-generated-by-separable}
Let $k$ be a field. Let $X$ be a smooth projective scheme over $k$.
Then $A_0(X)$ is generated by classes of closed points whose residue
fields are separable over $k$.
\end{lemma}

\begin{proof}
The lemma is immediate if $k$ has characteristic $0$ or is perfect.
Thus we may assume $k$ is an infinite field of characteristic $p > 0$.

\medskip\noindent
We may assume $X$ is irreducible of dimension $d$.
Observe that $k' = H^0(X, \mathcal{O}_X)$ is a finite separable
extension of $k$ and that $X$ is geometrically integral over $k'$.
We may and do replace $k$ by $k'$ and assume that $X$ is
geometrically integral.

\medskip\noindent
Let $x \in X$ be a closed point. Choose a sufficiently ample invertible
$\mathcal{O}_X$-module $\mathcal{L}$. Choose a trivialization
$\mathcal{L}_x = \mathcal{O}_{X, x}$. Set
$$
V = \{s \in H^0(X, \mathcal{L}) \mid s(x) = 0 \}
$$
The map $V \to \mathfrak m_x/\mathfrak m_x^2$ is surjective because
$\mathcal{L}$ is sufficiently ample. Consider the set
$$
V^d \supset U =
\{
(s_1, \ldots, s_d) \in V^d \mid s_1, \ldots, s_d
\text{ generate }
\mathfrak m_x/\mathfrak m_x^2
\text{ over }\kappa(x)
\}
$$
For $(s_1, \ldots, s_d) \in U$ set $H_i = Z(s_i)$. Since
$s_1, \ldots, s_d$ generate $\mathfrak m_x$ we see that
$$
H_1 \cap \ldots \cap H_d = x \amalg Z
$$
scheme theoretically.
We claim that for sufficiently general $(s_1, \ldots, s_d) \in U$
the scheme $Z$ is finite \'etale over $\Spec(k)$. This will finish
the proof as it shows that $[x] \sim_{rat} - [Z] + [Z']$
where $Z' = H'_1 \cap \ldots \cap H'_d$ is a general complete
intersection of vanishing loci of sufficiently general sections
of $\mathcal{L}$.

\medskip\noindent
To see that the claim is true, we may assume $\mathcal{L}$ is such that
there exists one choice of $s_1, \ldots, s_d$ such that
$H_1 \cap \ldots \cap H_d = x \amalg x' \amalg Z''$ (scheme theoretically)
for some closed point $x' \in X$ whose residue field is separable over $k$.
Namely, we can find a point $x'$ whose residue field is separable over
$k$ by
Varieties, Lemma \ref{varieties-lemma-smooth-separable-closed-points-dense}.
Then we choose $s_1, \ldots, s_d$ as above, but also vanishing at
$x'$ and generating $\mathfrak m_{x'}$.
The existence of this shows that the projection $I \to U$ from
incidence correspondence
$$
U \times X \setminus \{x\} \supset
I = \{((s_1, \ldots, s_d), x') \mid x' \in H_1 \cap \ldots \cap H_d\}
$$
is \'etale at some point. On the other hand, since $X \setminus \{x\}$
is geometrically integral, and since the geometric fibres of
the flat finite type morphism
$I \to X$ are open subschemes of linear spaces, we conclude that 
$I$ is geometrically integral over $k$.
Then we conclude that $I \to U$ is \'etale over (!)
a dense open subscheme $U' \subset U$. Since $k$-rational
points are dense in $U'$ we conclude.
Details omitted.
\end{proof}

\begin{lemma}
\label{lemma-degrees-cycles}
Let $H^*$ be a Weil cohomology theory
(Definition \ref{definition-weil-cohomology-theory}).
Let $X$ be a smooth projective scheme equidimensional of dimension $d$
over $k$. The diagram
$$
\xymatrix{
A^d(X) \ar[r]_-\gamma \ar@{=}[d] &
H^{2d}(X)(d) \ar[d]^{\int_X} \\
A_0(X) \ar[r]^\deg & F
}
$$
commutes where $\deg : A_0(X) \to \mathbf{Z}$ is the degree of
zero cycles discussed in Chow Homology, Section
\ref{chow-section-degree-zero-cycles}.
\end{lemma}

\begin{proof}
Let $x$ be a closed point of $X$ whose residue field is separable
over $k$. View $x$ as a scheme and denote
$i : x \to X$ the inclusion morphism. To avoid confusion denote
$\gamma' : A_0(x) \to H^0(x)$ the cycle class map for $x$.
Then we have
$$
\int_X \gamma([x]) = \int_X \gamma(i_*[x]) =
\int_X i_*\gamma'([x]) = \int_x \gamma'([x]) = \deg(x \to \Spec(k))
$$
The second equality is axiom (C)(b) and the third equality is
the definition of $i_*$ on cohomology. This proves the lemma
because $A^0(X)$ is generated by the classes of points $x$ as above
by Lemma \ref{lemma-generated-by-separable}.
\end{proof}

\begin{lemma}
\label{lemma-trace-disjoint-union}
Let $H^*$ be a Weil cohomology theory
(Definition \ref{definition-weil-cohomology-theory}).
Let $X, Y$ be smooth projective schemes equidimensional of dimension $d$
over $k$. Then $\int_{X \amalg Y} = \int_X + \int_Y$.
\end{lemma}

\begin{proof}
Denote $i : X \to X \amalg Y$ and $j : Y \to X \amalg Y$ be the coprojections.
By axiom (B)(c) the map
$(i^*, j^*) : H^*(X \amalg Y) \to H^*(X) \times H^*(Y)$ is an isomorphism.
The statement of the lemma means that under the isomorphism
$H^{2d}(X \amalg Y)(d) \to H^{2d}(X)(d) \oplus H^{2d}(Y)(d)$
the map $\int_X + \int_Y$ is tranformed into $\int_{X \amalg Y}$.

\medskip\noindent
By Lemma \ref{lemma-push-unit} we have $i_*1 = \gamma([X])$ and
$j_*1 = \gamma([Y])$. Hence $i_*1 + j_*1 = \gamma([X]) + \gamma([Y]) =
\gamma([X] + [Y]) = \gamma([X \amalg Y]) = 1$. Thus we have
$$
\int_{X \amalg Y} a =
\int_{X \amalg Y} a(i_*1 + j_*1) =
\int_{X \amalg Y} i_*(i^*a) + j_*(j^*a) =
\int_X i^*a +
\int_Y j^*a
$$
as desired. We have used Lemma \ref{lemma-pushforward}.
\end{proof}

\begin{lemma}
\label{lemma-dim-0-trace}
Let $H^*$ be a Weil cohomology theory
(Definition \ref{definition-weil-cohomology-theory}).
Let $X$ be a smooth projective scheme of dimension zero over $k$.
Then $\int_X : H^0(X) \to F$ is the trace map.
\end{lemma}

\begin{proof}
Write $X = \Spec(k')$ as in the proof of Lemma \ref{lemma-dim-0}.
Choose $k''/k$ finite Galois such that every factor of $k'$
maps into $k''$ as in the proof of Lemma \ref{lemma-dim-0}.
Then $\Sigma = \Hom_k(k', k'')$ has $[k' : k]$ elements and
we have seen in the proof of Lemma \ref{lemma-dim-0} that the map
$$
H^0(X) \otimes_F H^0(Y) \longrightarrow
\prod\nolimits_{\sigma \in \Sigma} H^0(Y),\quad
a \otimes b \longmapsto \prod\nolimits_\sigma \Spec(\sigma)^*a \cup b
$$
is an isomorphism. Write $H^0(Y) = \prod F_i$ as a finite product of
finite separable extensions of $F$. Define the $F$-algebra maps
$$
\tau_{i, \sigma} :
H^0(X) \xrightarrow{\Spec(\sigma)^*} H^0(Y) \to F_i
$$
We get $H^0(X) \otimes_F F_i = \prod\nolimits_{\sigma \in \Sigma} F_i$
via the maps $\tau_{i, \sigma}$. Hence $\tau_{i, \sigma}$
must be pairwise distinct and must exhaust the set
$T_i = \Hom_F(H^0(X), F_i)$ which must have cardinality $[k' : k]$.
It follows that $\sum_\sigma \tau_{i, \sigma} = \tau_i \circ \text{Tr}$
where $\text{Tr} : H^0(X) \to F$ is the trace map and
$\tau_i : F \to F_i$ is some linear map. Hence
$\sum_\sigma \Spec(\sigma)^* = \tau \circ \text{Tr}$ where
$\tau  = \sum \tau_i : F \to H^0(Y)$.
We conclude that
$$
\int_{X \times Y} a \otimes 1 =
\sum\nolimits_\sigma \int_Y \Spec(\sigma)^*a =
\int_Y \sum\nolimits_\sigma \Spec(\sigma)^*a =
\int_Y \tau(\text{Tr}(a))
$$
On the other hand, if $p : X \times Y \to X$ is the first projection,
we have
$$
\int_{X \times Y} a \otimes 1 = \int_X p_*(a \otimes 1) =
\int_X a \cup p_*1 = [k'' : k] \int_X a
$$
The first equality is the definition of $p_*$. The second equality
by Lemma \ref{lemma-pushforward}. The third equality because
$$
p_*1 = p_*\gamma([X \times Y]) =
\gamma(p_*[X \times Y]) = \gamma([k'' : k][Y]) =
[k'' : k] \gamma([Y]) = [k'' : k]
$$
where we've used Lemma \ref{lemma-unit}.
All in all we conclude that $\int_X a = \lambda(\text{Tr}(a))$
for some linear map $\lambda : F \to F$.
Using that $\int_X 1 = \deg(X \to \Spec(k))$ found in Lemma \ref{lemma-dim-0}
we conclude.
\end{proof}

\begin{lemma}
\label{lemma-trace-product}
Let $H^*$ be a Weil cohomology theory
(Definition \ref{definition-weil-cohomology-theory}).
Let $X$ and $Y$ be equidimensional smooth projective schemes over $k$.
Then $\int_{X \times Y} = \int_X \otimes \int_Y$.
\end{lemma}

\begin{proof}
Say $X$ has dimension $d$ and $Y$ has dimension $e$.
Denote $p : X \times Y \to X$ and $q : X \times Y \to Y$ the
projections. The statement means that under the isomorphism
$H^{2d}(X)(d) \otimes_F H^{2e}(Y)(e) \to H^{2d + 2e}(X \times Y)(d + e)$
the map $\int_{X \times Y}$ is tranformed into $\int_X \otimes \int_Y$.

\medskip\noindent
Say $X$ and $Y$ have dimension $0$. Then the result holds by
Lemma \ref{lemma-dim-0-trace} as the trace map has the corresponding property.

\medskip\noindent
General case. By Lemma \ref{lemma-generated-by-separable} and axiom (C)(d)
it suffices to check this on element of the form
$$
p^*(a \cup \gamma([x])) \cup q^*(b \cup \gamma([y])) =
p^*a \cup q^*b \cup \gamma([x \times y])
$$
where $x \in X$, $y \in Y$ are closed points whose residue fields are
separable over $k$ and $a \in H^0(X)$, $b \in H^0(Y)$. To see the
equality use axioms (C)(a) and (C)(c) and use that the intersection
product of $[x \times Y]$ and $[X \times y]$ is $[x \times y]$.
Denote $i : x \times y \to X \times Y$ the inclusion morphism.
Denote $f = p \circ i : x \times y \to X$ and
$g = q \circ i : x \times y \to y$. We compute
\begin{align*}
\int_{X \times Y} p^*a \cup q^*b \cup \gamma([x \times y])
& =
\int_{X \times Y} p^*a \cup q^*b \cup i_*1 \\
& =
\int_{X \times Y} i_*(f^*a \cup g^*b) \\
& =
\int_{x \times y} f^*a \cup g^*b \\
& =
(\int_x f^*a) (\int_y g^*b)
\end{align*}
The first equality because $i_*1 = \gamma([x \times y])$, see
Lemma \ref{lemma-push-unit}. The second equality by
Lemma \ref{lemma-pushforward}. The third is the definition of $i_*$.
The fourth is the case of dimension $0$.
We conclude because
$\int_x f^*a = \int_X a \cup \gamma([x])$ and
$\int_y g^*b = \int_Y b \cup \gamma([y])$
by a similar argument which we omit.
\end{proof}

\begin{lemma}
\label{lemma-class-diagonal}
Let $H^*$ be a Weil cohomology theory
(Definition \ref{definition-weil-cohomology-theory}).
Let $X$ be a smooth projective scheme which is equidimensional
of dimension $d$ over $k$. Choose a basis
$e_{i, j}, j = 1, \ldots, \beta_i$ of $H^i(X)$ over $F$.
Using K\"unneth write
$$
\gamma([\Delta]) =
\sum\nolimits_{i = 0, \ldots, 2d}
\sum\nolimits_j e_{i, j} \otimes e'_{2d - i , j}
\quad\text{in}\quad
\bigoplus\nolimits_i H^i(X) \otimes_F H^{2d - i}(X)(d)
$$
with $e'_{2d - i, j} \in H^{2d - i}(X)(d)$.
Then $\int_X e_{i, j} \cup e'_{2d - i, j'} = \delta_{jj'}$.
\end{lemma}

\begin{proof}
Recall that $\Delta^* : H^*(X \times X) \to H^*(X)$ is equal to the
cup product map $H^*(X) \otimes_F H^*(X) \to H^*(X)$. On the other
hand, recall that $\gamma([\Delta]) = \Delta_*1$ (Lemma \ref{lemma-push-unit})
and hence
$$
\int_{X \times X} \gamma([\Delta]) \cup a \otimes b =
\int_{X \times X} i_*1 \cup a \otimes b =
\int_X a \cup b
$$
On the other hand, we have
$$
\int_{X \times X} (\sum e_{i, j} \otimes e'_{2d -i , j}) \cup a \otimes b =
\sum (\int_X e_{i, j} \cup a)(\int_X e'_{2d - i, j} \cup b)
$$
by Lemma \ref{}. Thus if we choose $a$ such that $\int_X e_{i, j} \cup a = 1$
and all other pairings equal to zero, then we conclude that
$\int_X e'_{2d - i, j} \cup b = \int_X a \cup b$ for all $b$, i.e.,
$e'_{2d - i, j} = a$. This proves the lemma.
\end{proof}

\begin{lemma}
\label{lemma-square-diagonal}
Let $H^*$ be a Weil cohomology theory
(Definition \ref{definition-weil-cohomology-theory}).
Let $X$ be a smooth projective scheme over $k$ which is equidimensional
of dimension $d$. We have
$$
\sum\nolimits_{i = 0, \ldots, 2d} \dim_F H^i(X) =
\deg(\Delta \cdot \Delta) = \deg(c_d(\mathcal{T}_{X/k}))
$$
\end{lemma}

\begin{proof}
The equality on the right holds by Chow Homology, Lemma \ref{}.
By Lemma \ref{} we have
$$
\deg(\Delta \cdot \Delta) =
\int_{X \times X} \gamma([\Delta]) \cup \gamma([\Delta])
$$
Write $\gamma([\Delta]) = \sum  e_{i, j} \otimes e'_{2d - i , j}$
as in Lemma \ref{}.
Denote $\sigma : X \times X \to X \times X$ the automorphism switching
the factors. Observe that $a \otimes b = \text{pr}_1^*(a) \cup \text{pr}_2^*b$
and hence
$$
\sigma^*(a \otimes b) =  \text{pr}_2^*(a) \cup \text{pr}_1^*b =
(-1)^{\deg(a)\deg(b)} b \otimes a
$$
Since $\sigma(\Delta) = \Delta$ we find
$\gamma([\Delta]) = \sum (-1)^i e'_{2d - i , j} \otimes e_{i, j}$.
Using notation as in Lemma \ref{} we compute
\begin{align*}
\gamma([\Delta]) \cup \gamma([\Delta])
& =
\sum (-1)^{i'}
e_{i, j} \otimes e'_{2d - i , j} \cup e'_{2d - i', j'} \otimes e_{i', j'} \\
& =
\sum (-1)^{i'} e_{i, j} \cup e'_{2d - i', j'} \otimes 
e_{i', j'} \cup e'_{2d - i , j}
\end{align*}
(Note that we made two switches of order in the second step.)
If we apply $\int_{X \times X} = \int_X \otimes \int_X$ to this
we obtain $(-1)^i$ exactly when $i = i'$ and $j = j'$ and zero else.
This proves the lemma.
\end{proof}

\begin{lemma}
\label{lemma-H-0-separable}
Let $H^*$ be a Weil cohomology theory
(Definition \ref{definition-weil-cohomology-theory}).
Let $X$ be a smooth projective scheme over $k$. Then
$H^0(X)$ is a finite separable algebra over $F$.
\end{lemma}

\begin{proof}
By decomposing $X$ into its irreducible components we may
assume $X$ is irreducible and in particular equidimensional
of dimension $d$. Combining the finite dimensionality of
$H^{2d}(X)(d)$ of axiom (A)(b), with axiom (C)(d) and
Lemma \ref{lemma-generated-by-separable}
we can find finitely many closed points
$x_1, \ldots, x_r \in X$ whose residue fields are
separable over $k$ such that $\gamma([x_1]), \ldots, \gamma([x_r])$
generate $H^{2d}(X)(d)$ over $H^0(X)$.
Recall that $\gamma([x_i]) = i_{i, *}1$ where
$i_i : x_i \to X$ is the inclusion morphism, see
Lemma \ref{lemma-push-unit}. Thus $a \cup \gamma([x_i]) = i_{i, *}(i_i^*a)$ by
Lemma \ref{lemma-pushforward}. In other words, we see that
$$
(i_{1, *}, \ldots, i_{r, *}) :
H^0(x_1) \times \ldots \times H^0(x_r)
\longrightarrow
H^{2d}(X)(d)
$$
is surjective. Using Poincar\'e duality in the form of axiom (A)(c),
this implies that the $F$-algebra map
$$
(i_1^*, \ldots, i_r^*) :
H^0(X)
\longrightarrow
H^0(x_1) \times \ldots \times H^0(x_r)
$$
is injective. Using Lemma \ref{lemma-dim-0} we conclude that
$H^0(X)$ is a separable $F$-algebra.
\end{proof}

\begin{lemma}
\label{lemma-relations-classes-points}
Let $H^*$ be a Weil cohomology theory
(Definition \ref{definition-weil-cohomology-theory}).
Let $k'/k$ be a finite separable extension.
Let $X$ be a smooth projective scheme over $k'$.
Let $x, x' \in X$ be $k'$-rational points.
If $\gamma(x) \not = \gamma(x')$, then
$x - x'$ is not divisible by any integer $n > 1$ in $A_0(X)$.
\end{lemma}

\begin{proof}
We may assume $X$ is equidimensional of dimension $d$.
Say $x - x' = n(\sum m_i x_i)$ for some $x_i \in X$ closed points
whose residue fields are separable over $k$ (Lemma \ref{}).
Then
$$
\gamma(x) - \gamma(x') = n (\sum m_i \gamma(x_i))
$$
in $H^{2d}(X)(d)$. Denote $i^*, (i')^*, i_i^*$ the pullback maps
$H^0(X) \to H^0(x)$, $H^0(X) \to H^0(x')$, $H^0(X) \to H^0(x_i)$.
Recall that $\int_x : H^0(x) \to F$ is the trace map, which
we will denote $\text{Tr}_x$. Similarly for $x'$ and $x_i$.
Then by Poincar\'e duality the equation above is dual to
$$
\text{Tr}_x \circ i^* - \text{Tr}_{x'} \circ (i')^* =
\sum nm_i \text{Tr}_{x_i} \circ i_i^*
$$
which takes place in $\Hom_F(H^0(X), F)$.

\medskip\noindent
In order to see what is going on, momentarily we assume $F$ is separably closed.
Then each of $H^0(\Spec(k'))$, $H^0(X)$, $H^0(x)$, $H^0(x')$, $H^0(x_i)$
is a product of copies of $F$ by Lemma \ref{}.
Let $e \in H^0(X)$ be an idempotent generating a factor.
Note that since $x$ and $x'$ are rational points, the composition
$H^0(\Spec(k')) \to H^0(X) \to H^0(x)$ is an isomorphism.
This implies $i^*e$ is either a minimal idempotent (i.e.,
generates a factor $F$) or zero. Similarly for $(i')^*$.
If $\gamma(x) \not = \gamma(x')$, then $i^* \not = (i')^*$ and
hence we can choose $e$ such that $i^*e \not = 0$ and
$(i')^*e = 0$. Then $\text{Tr}_x(i^*e) = 1$ and $\text{Tr}_{x'}((i')^*e) = 0$.
On the other hand, $i_i^*e$ is an idempotent
and hence $\text{Tr}_{x_i}(i_i^*e) = r_i$ is an integer.
We conclude that
$$
1 = \sum n m_i r_i = n (\sum m_i r_i)
$$
which is impossible.

\medskip\noindent
If $F$ is not separably algebraically closed, then we replace
$H^*( - )$ by $H^*( - ) \otimes_F F^{sep}$ and we conclude.
Details omitted.
\end{proof}

\begin{lemma}
\label{lemma-divide-difference-points}
Let $k$ be a field. Let $X$ be a geometrically connected
smooth projective scheme over $k$. Let $x, x' \in X$ be $k$-rational points.
Then there exists a finite separable extension $k'/k$ such that
the pullback of $x - x'$ to $X_{k'}$
is divisible by an integer $n > 1$ in $A_0(X_{k'})$.
\end{lemma}

\begin{proof}
Using Bertini we can choose a smooth curve $C \subset X$
such that $x, x' \in C$. (One may have to extend $k$
a little bit here.) This reduces us to the case where
$X$ is a curve. Next, we see that $\mathcal{O}_X(x - x')$
defines a $k$-rational point of $\Pic_{X/k}$.
Choose a prime $\ell$ invertible in $k$. Since
$[\ell] : \Pic_{X/k} \to \Pic_{X/k}$ is finite \'etale,
we see that after replacing $k$ by a finite separable extension
the invertible module $\mathcal{O}_X(x - x')$ has an
$\ell$th root in $\Pic(X)$. This is what we had to show.
\end{proof}

\begin{lemma}
\label{lemma-H-0}
Let $H^*$ be a Weil cohomology theory
(Definition \ref{definition-weil-cohomology-theory}).
Let $X$ be a smooth projective scheme over $k$. Set
$k' = H^0(X, \mathcal{O}_X)$. This is a finite separable algebra
over $k$ and $X \to \Spec(k')$ has geometrically irreducible
fibres. Finally, $H^0(\Spec(k')) \to H^0(X)$ is an isomorphism.
\end{lemma}

\begin{proof}
We omit the proof of the geometric statements. Decomposing
$X$ into irreducible components, we may assume $X$ is irreducible
and $k'$ is a field. Let $k''/k'$ be a finite separable field
extension. Then
$$
H^0(X) \otimes_{H^0(\Spec(k'))} H^0(\Spec(k'')) =
H^0(X \times_{\Spec(k')} \Spec(k''))
$$
by K\"unneth and some arguments we omit. Thus, in order to prove that
$H^0(\Spec(k')) \to H^0(X)$ is an isomorphism, we may always replace
$k'$ by a finite separable extension.
Choose $x_1, \ldots, x_r \in X$ closed
points such that $H^0(X) \to \prod H^0(x_i)$ is injective, see
proof of Lemma \ref{}.
After extending $k'$ we may assume $x_1, \ldots, x_r$
are $k'$-rational points (but this replaces each $x_i$ with multiple
points, so $r$ is increased in this step). Next, by Lemma \ref{}
we may assume the differences $x_i - x_j$ are divisible by
positive integers in $A_0(X)$. Then we see that
$\gamma(x_1) = \gamma(x_2) = \ldots = \gamma(x_r)$ by Lemma \ref{}.
In other words all the maps $H^0(X) \to H^0(x_i)$
are the same. This ends the proof.
\end{proof}

\begin{remark}
\label{remark-betti-numbers-in-some-sense}
Let $H^*$ be a Weil cohomology theory
(Definition \ref{definition-weil-cohomology-theory}).
Let $X$ be a geometrically irreducible smooth projective scheme
over a finite separable extension $k'/k$. Suppose that
$$
H^0(\Spec(k')) = F_1 \times \ldots \times F_r
$$
for some fields $F_i$. Then we accordingly can write
$$
H^*(X) = \prod\nolimits_{i = 1, \ldots, r}
H^*(X) \otimes_{H^0(\Spec(k'))} F_i
$$
Now, Lemma \ref{} tells us that $H^0(X)$ is free of rank $1$
over $\prod F_i$. In other words, each of the factors
$H^0(X) \otimes_{H^0(\Spec(k'))} F_i$ has dimension $1$ over $F_i$.
Poincar\'e duality then tells us that the same is true for
cohomology in degree $2d$.
What isn't clear however is that the same holds in other degrees.
Namely, we don't know that given $0 < n < \dim(X)$ the integers
$$
\dim_{F_i} H^n(X) \otimes_{H^0(\Spec(k'))} F_i
$$
are independent of $i$! This question appears to be closely related
to the question of whether betti numbers of smooth projective
varieties are independent of the choice of a Weil cohomology theory.
\end{remark}








\section{Other chapters}

\begin{multicols}{2}
\begin{enumerate}
\item \hyperref[introduction-section-phantom]{Introduction}
\item \hyperref[conventions-section-phantom]{Conventions}
\item \hyperref[sets-section-phantom]{Set Theory}
\item \hyperref[categories-section-phantom]{Categories}
\item \hyperref[topology-section-phantom]{Topology}
\item \hyperref[sheaves-section-phantom]{Sheaves on Spaces}
\item \hyperref[algebra-section-phantom]{Commutative Algebra}
\item \hyperref[sites-section-phantom]{Sites and Sheaves}
\item \hyperref[homology-section-phantom]{Homological Algebra}
\item \hyperref[derived-section-phantom]{Derived Categories}
\item \hyperref[more-algebra-section-phantom]{More Algebra}
\item \hyperref[simplicial-section-phantom]{Simplicial Methods}
\item \hyperref[modules-section-phantom]{Sheaves of Modules}
\item \hyperref[sites-modules-section-phantom]{Modules on Sites}
\item \hyperref[injectives-section-phantom]{Injectives}
\item \hyperref[cohomology-section-phantom]{Cohomology of Sheaves}
\item \hyperref[sites-cohomology-section-phantom]{Cohomology on Sites}
\item \hyperref[hypercovering-section-phantom]{Hypercoverings}
\item \hyperref[schemes-section-phantom]{Schemes}
\item \hyperref[constructions-section-phantom]{Constructions of Schemes}
\item \hyperref[properties-section-phantom]{Properties of Schemes}
\item \hyperref[morphisms-section-phantom]{Morphisms of Schemes}
\item \hyperref[coherent-section-phantom]{Coherent Cohomology}
\item \hyperref[divisors-section-phantom]{Divisors}
\item \hyperref[limits-section-phantom]{Limits of Schemes}
\item \hyperref[varieties-section-phantom]{Varieties}
\item \hyperref[chow-section-phantom]{Chow Homology}
\item \hyperref[topologies-section-phantom]{Topologies on Schemes}
\item \hyperref[descent-section-phantom]{Descent}
\item \hyperref[more-morphisms-section-phantom]{More on Morphisms}
\item \hyperref[flat-section-phantom]{More on Flatness}
\item \hyperref[groupoids-section-phantom]{Groupoid Schemes}
\item \hyperref[more-groupoids-section-phantom]{More on Groupoid Schemes}
\item \hyperref[etale-section-phantom]{\'Etale Morphisms of Schemes}
\item \hyperref[etale-cohomology-section-phantom]{\'Etale Cohomology}
\item \hyperref[spaces-section-phantom]{Algebraic Spaces}
\item \hyperref[spaces-properties-section-phantom]{Properties of Algebraic Spaces}
\item \hyperref[spaces-morphisms-section-phantom]{Morphisms of Algebraic Spaces}
\item \hyperref[spaces-topologies-section-phantom]{Topologies on Algebraic Spaces}
\item \hyperref[spaces-descent-section-phantom]{Descent and Algebraic Spaces}
\item \hyperref[spaces-more-morphisms-section-phantom]{More on Morphisms of Spaces}
\item \hyperref[quot-section-phantom]{Quot and Hilbert Spaces}
\item \hyperref[stacks-section-phantom]{Stacks}
\item \hyperref[spaces-groupoids-section-phantom]{Groupoids in Algebraic Spaces}
\item \hyperref[spaces-more-groupoids-section-phantom]{More on Groupoids in Spaces}
\item \hyperref[bootstrap-section-phantom]{Bootstrap}
\item \hyperref[examples-stacks-section-phantom]{Examples of Stacks}
\item \hyperref[groupoids-quotients-section-phantom]{Quotients of Groupoids}
\item \hyperref[algebraic-section-phantom]{Algebraic Stacks}
\item \hyperref[criteria-section-phantom]{Criteria for Representability}
\item \hyperref[stacks-properties-section-phantom]{Properties of Algebraic Stacks}
\item \hyperref[stacks-morphisms-section-phantom]{Morphisms of Algebraic Stacks}
\item \hyperref[examples-section-phantom]{Examples}
\item \hyperref[exercises-section-phantom]{Exercises}
\item \hyperref[guide-section-phantom]{Guide to Literature}
\item \hyperref[desirables-section-phantom]{Desirables}
\item \hyperref[coding-section-phantom]{Coding Style}
\item \hyperref[fdl-section-phantom]{GNU Free Documentation License}
\item \hyperref[index-section-phantom]{Auto Generated Index}
\end{enumerate}
\end{multicols}


\bibliography{my}
\bibliographystyle{amsalpha}

\end{document}
