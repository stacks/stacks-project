\IfFileExists{stacks-project.cls}{%
\documentclass{stacks-project}
}{%
\documentclass{amsart}
}

% The following AMS packages are automatically loaded with
% the amsart documentclass:
%\usepackage{amsmath}
%\usepackage{amssymb}
%\usepackage{amsthm}

% For dealing with references we use the comment environment
\usepackage{verbatim}
\newenvironment{reference}{\comment}{\endcomment}
%\newenvironment{reference}{}{}
\newenvironment{slogan}{\comment}{\endcomment}
\newenvironment{history}{\comment}{\endcomment}

% For commutative diagrams you can use
% \usepackage{amscd}
\usepackage[all]{xy}

% We use 2cell for 2-commutative diagrams.
\xyoption{2cell}
\UseAllTwocells

% To put source file link in headers.
% Change "template.tex" to "this_filename.tex"
% \usepackage{fancyhdr}
% \pagestyle{fancy}
% \lhead{}
% \chead{}
% \rhead{Source file: \url{template.tex}}
% \lfoot{}
% \cfoot{\thepage}
% \rfoot{}
% \renewcommand{\headrulewidth}{0pt}
% \renewcommand{\footrulewidth}{0pt}
% \renewcommand{\headheight}{12pt}

\usepackage{multicol}

% For cross-file-references
\usepackage{xr-hyper}

% Package for hypertext links:
\usepackage{hyperref}

% For any local file, say "hello.tex" you want to link to please
% use \externaldocument[hello-]{hello}
\externaldocument[introduction-]{introduction}
\externaldocument[conventions-]{conventions}
\externaldocument[sets-]{sets}
\externaldocument[categories-]{categories}
\externaldocument[topology-]{topology}
\externaldocument[sheaves-]{sheaves}
\externaldocument[sites-]{sites}
\externaldocument[stacks-]{stacks}
\externaldocument[fields-]{fields}
\externaldocument[algebra-]{algebra}
\externaldocument[brauer-]{brauer}
\externaldocument[homology-]{homology}
\externaldocument[derived-]{derived}
\externaldocument[simplicial-]{simplicial}
\externaldocument[more-algebra-]{more-algebra}
\externaldocument[smoothing-]{smoothing}
\externaldocument[modules-]{modules}
\externaldocument[sites-modules-]{sites-modules}
\externaldocument[injectives-]{injectives}
\externaldocument[cohomology-]{cohomology}
\externaldocument[sites-cohomology-]{sites-cohomology}
\externaldocument[dga-]{dga}
\externaldocument[dpa-]{dpa}
\externaldocument[hypercovering-]{hypercovering}
\externaldocument[schemes-]{schemes}
\externaldocument[constructions-]{constructions}
\externaldocument[properties-]{properties}
\externaldocument[morphisms-]{morphisms}
\externaldocument[coherent-]{coherent}
\externaldocument[divisors-]{divisors}
\externaldocument[limits-]{limits}
\externaldocument[varieties-]{varieties}
\externaldocument[topologies-]{topologies}
\externaldocument[descent-]{descent}
\externaldocument[perfect-]{perfect}
\externaldocument[more-morphisms-]{more-morphisms}
\externaldocument[flat-]{flat}
\externaldocument[groupoids-]{groupoids}
\externaldocument[more-groupoids-]{more-groupoids}
\externaldocument[etale-]{etale}
\externaldocument[chow-]{chow}
\externaldocument[intersection-]{intersection}
\externaldocument[pic-]{pic}
\externaldocument[adequate-]{adequate}
\externaldocument[dualizing-]{dualizing}
\externaldocument[duality-]{duality}
\externaldocument[discriminant-]{discriminant}
\externaldocument[local-cohomology-]{local-cohomology}
\externaldocument[curves-]{curves}
\externaldocument[resolve-]{resolve}
\externaldocument[models-]{models}
\externaldocument[pione-]{pione}
\externaldocument[etale-cohomology-]{etale-cohomology}
\externaldocument[proetale-]{proetale}
\externaldocument[crystalline-]{crystalline}
\externaldocument[spaces-]{spaces}
\externaldocument[spaces-properties-]{spaces-properties}
\externaldocument[spaces-morphisms-]{spaces-morphisms}
\externaldocument[decent-spaces-]{decent-spaces}
\externaldocument[spaces-cohomology-]{spaces-cohomology}
\externaldocument[spaces-limits-]{spaces-limits}
\externaldocument[spaces-divisors-]{spaces-divisors}
\externaldocument[spaces-over-fields-]{spaces-over-fields}
\externaldocument[spaces-topologies-]{spaces-topologies}
\externaldocument[spaces-descent-]{spaces-descent}
\externaldocument[spaces-perfect-]{spaces-perfect}
\externaldocument[spaces-more-morphisms-]{spaces-more-morphisms}
\externaldocument[spaces-flat-]{spaces-flat}
\externaldocument[spaces-groupoids-]{spaces-groupoids}
\externaldocument[spaces-more-groupoids-]{spaces-more-groupoids}
\externaldocument[bootstrap-]{bootstrap}
\externaldocument[spaces-pushouts-]{spaces-pushouts}
\externaldocument[groupoids-quotients-]{groupoids-quotients}
\externaldocument[spaces-more-cohomology-]{spaces-more-cohomology}
\externaldocument[spaces-simplicial-]{spaces-simplicial}
\externaldocument[formal-spaces-]{formal-spaces}
\externaldocument[restricted-]{restricted}
\externaldocument[spaces-resolve-]{spaces-resolve}
\externaldocument[formal-defos-]{formal-defos}
\externaldocument[defos-]{defos}
\externaldocument[cotangent-]{cotangent}
\externaldocument[examples-defos-]{examples-defos}
\externaldocument[algebraic-]{algebraic}
\externaldocument[examples-stacks-]{examples-stacks}
\externaldocument[stacks-sheaves-]{stacks-sheaves}
\externaldocument[criteria-]{criteria}
\externaldocument[artin-]{artin}
\externaldocument[quot-]{quot}
\externaldocument[stacks-properties-]{stacks-properties}
\externaldocument[stacks-morphisms-]{stacks-morphisms}
\externaldocument[stacks-limits-]{stacks-limits}
\externaldocument[stacks-cohomology-]{stacks-cohomology}
\externaldocument[stacks-perfect-]{stacks-perfect}
\externaldocument[stacks-introduction-]{stacks-introduction}
\externaldocument[stacks-more-morphisms-]{stacks-more-morphisms}
\externaldocument[stacks-geometry-]{stacks-geometry}
\externaldocument[moduli-]{moduli}
\externaldocument[moduli-curves-]{moduli-curves}
\externaldocument[examples-]{examples}
\externaldocument[exercises-]{exercises}
\externaldocument[guide-]{guide}
\externaldocument[desirables-]{desirables}
\externaldocument[coding-]{coding}
\externaldocument[obsolete-]{obsolete}
\externaldocument[fdl-]{fdl}
\externaldocument[index-]{index}

% Theorem environments.
%
\theoremstyle{plain}
\newtheorem{theorem}[subsection]{Theorem}
\newtheorem{proposition}[subsection]{Proposition}
\newtheorem{lemma}[subsection]{Lemma}

\theoremstyle{definition}
\newtheorem{definition}[subsection]{Definition}
\newtheorem{example}[subsection]{Example}
\newtheorem{exercise}[subsection]{Exercise}
\newtheorem{situation}[subsection]{Situation}

\theoremstyle{remark}
\newtheorem{remark}[subsection]{Remark}
\newtheorem{remarks}[subsection]{Remarks}

\numberwithin{equation}{subsection}

% Macros
%
\def\lim{\mathop{\rm lim}\nolimits}
\def\colim{\mathop{\rm colim}\nolimits}
\def\Spec{\mathop{\rm Spec}}
\def\Hom{\mathop{\rm Hom}\nolimits}
\def\Ext{\mathop{\rm Ext}\nolimits}
\def\SheafHom{\mathop{\mathcal{H}\!{\it om}}\nolimits}
\def\SheafExt{\mathop{\mathcal{E}\!{\it xt}}\nolimits}
\def\Sch{\textit{Sch}}
\def\Mor{\mathop{\rm Mor}\nolimits}
\def\Ob{\mathop{\rm Ob}\nolimits}
\def\Sh{\mathop{\textit{Sh}}\nolimits}
\def\NL{\mathop{N\!L}\nolimits}
\def\proetale{{pro\text{-}\acute{e}tale}}
\def\etale{{\acute{e}tale}}
\def\QCoh{\textit{QCoh}}
\def\Ker{\mathop{\rm Ker}}
\def\Im{\mathop{\rm Im}}
\def\Coker{\mathop{\rm Coker}}
\def\Coim{\mathop{\rm Coim}}

%
% Macros for moduli stacks/spaces
%
\def\QCohstack{\mathcal{QC}\!{\it oh}}
\def\Cohstack{\mathcal{C}\!{\it oh}}
\def\Spacesstack{\mathcal{S}\!{\it paces}}
\def\Quotfunctor{{\rm Quot}}
\def\Hilbfunctor{{\rm Hilb}}
\def\Curvesstack{\mathcal{C}\!{\it urves}}
\def\Polarizedstack{\mathcal{P}\!{\it olarized}}
\def\Complexesstack{\mathcal{C}\!{\it omplexes}}
% \Pic is the operator that assigns to X its picard group, usage \Pic(X)
% \Picardstack_{X/B} denotes the Picard stack of X over B
% \Picardfunctor_{X/B} denotes the Picard functor of X over B
\def\Pic{\mathop{\rm Pic}\nolimits}
\def\Picardstack{\mathcal{P}\!{\it ic}}
\def\Picardfunctor{{\rm Pic}}
\def\Deformationcategory{\mathcal{D}\!{\it ef}}


% OK, start here.
%
\begin{document}

\title{Weil Cohomology Theories, UNDER CONSTRUCTION}


\maketitle

\phantomsection
\label{section-phantom}

\tableofcontents

\section{Introduction}
\label{section-introduction}

\noindent
In this chapter we discuss Weil cohomology theories for smooth
projective schemes over any base field. In the case of an algebraically
closed base field, our notion is the same as the notion introduced
in \cite{Kleiman-cycles}, see (insert future reference here).





\section{Conventions and notation}
\label{section-conventions}

\noindent
Let $R$ be a ring. In this chapter a
{\it graded commutative $R$-algebra} $A$ is a
commutative differential graded $R$-algebra
(Differential Graded Algebra, Definitions \ref{dga-definition-dga} and
\ref{dga-definition-cdga}) whose differential is zero. Thus $A$
is an $R$-module endowed with a grading
$A = \bigoplus_{n \in \mathbf{Z}} A^n$ by
$R$-submodules. The $R$-bilinear multiplication
$$
A^n \times A^m \longrightarrow A^{n + m},\quad
\alpha \times \beta \longmapsto \alpha \cup \beta
$$
will be called the {\it cup product} in this chapter.
The commutativity constraint is
$\alpha \cup \beta = (-1)^{nm} \beta \cup \alpha$ if
$\alpha \in A^n$ and $\beta \in A^m$. Finally, there is
a multiplicative unit $1 \in A^0$, or equivalently, there is an
additive and multiplicative map $R \to A^0$ which is compatible the
$R$-module structure on $A$.

\medskip\noindent
Let $k$ be a field and let $X$ be a scheme of finite type over $k$.
The Chow groups $\CH_k(X)$ of $X$ have been defined in
Chow Homology, Definition \ref{chow-definition-rational-equivalence}.
Given a proper morphism $f : X \to Y$ of schemes of finite
type over $k$ there is a pushforward map $f_* : \CH_k(X) \to \CH_k(Y)$,
see Chow Homology, Section \ref{chow-section-proper-pushforward} and
Lemma \ref{chow-lemma-proper-pushforward-rational-equivalence}.
If $X$ is smooth over $k$ and equidimensional of dimension $d$, then
we have
$$
\CH^i(X) = \CH_{d - i}(X)
$$
see Chow Homology, Section \ref{chow-lemma-identify-chow-for-smooth}
and recall that the isomorphism sends $c \in \CH^i(X)$ to
$c \cap [X]_d \in \CH_{d - i}(X)$.
If $X$ smooth over $k$ and quasi-compact, then we can write canonically
$$
X = X_0 \amalg X_1 \amalg X_2 \amalg \ldots \amalg X_n
$$
into open and closed subschemes $X_d$ which are
equidimensional of dimension $d$. Set $[X] = \sum [X_d]_d$
as an element of $\CH_*(X)$. Since
$\CH_k(X) = \prod \CH_k(X_d)$ and $\CH^i(X) = \prod \CH^i(X_d)$
the map $c \mapsto c \cap [X]$ still defines an isomorphism
$$
\CH^*(X) = \CH_*(X)
$$
but it is no longer compatible with gradings, namely,
$$
\CH^i(X) = \bigoplus\nolimits_d \CH_{d - i}(X_d)
$$
The reader may use this as an alternative definition of $\CH^i(X)$.
There is an intersection product
$(\alpha, \beta) \mapsto \alpha \cdot \beta$ on $\CH_*(X)$
with the property that it sends $\CH^i(X) \times \CH^j(X)$ into $\CH^{i + j}(X)$,
see Chow Homology, Section \ref{chow-section-intersection-product}.
If $f : Y \to X$ is a morphism of schemes smooth over $k$, then
there is a pullback map
$$
f^* : \CH^i(X) \to \CH^i(Y),\quad
\alpha \mapsto f^*\alpha
$$
which is compatible with intersection products, see
Chow Homology, Lemma \ref{}.
Moreover, if $f$ is also proper, then
we have $f_*(\alpha \cdot f^*\beta) = f_*\alpha \cdot \beta$, see
Chow Homology, Lemma \ref{}.
We have $\alpha \cdot \beta = \Delta^*(\alpha \times \beta)$
if $\alpha, \beta$ are cycles on $X$ smooth over $k$.






\section{Monoidal categories}
\label{section-monoidal}

\noindent
Let $\mathcal{C}$ be a category. Suppose we are given a functor
$$
\otimes : \mathcal{C} \times \mathcal{C} \longrightarrow \mathcal{C}
$$
We often want to know whether $\otimes$ satisfies an associative rule
and whether there is a unit for $\otimes$.

\medskip\noindent
An {\it associativity constraint} for $(\mathcal{C}, \otimes)$ is
a functorial isomorphism
$$
\phi_{X, Y, Z} : X \otimes (Y \otimes Z) \to (X \otimes Y) \otimes Z
$$
such that for all objects $X, Y, Z, W$ the diagram
$$
\xymatrix{
X \otimes (Y \otimes ( Z \otimes W)) \ar[r] \ar[d] &
(X \otimes Y) \otimes (Z \otimes W) \ar[r] &
((X \otimes Y) \otimes Z) \otimes W \ar[d]  \\
X \otimes ((Y \otimes Z) \otimes W) \ar[rr] & &
(X \otimes (Y \otimes Z)) \otimes W
}
$$
is commutative where every arrow is determined by a suitable application
of $\phi$ and functoriality of $\otimes$. Given an associativity constraint
there are well defined functors
$$
\mathcal{C} \times \ldots \times \mathcal{C} \longrightarrow \mathcal{C},
\quad
(X_1, \ldots, X_n) \longmapsto X_1 \otimes \ldots \otimes X_n
$$
for all $n \geq 1$.

\medskip\noindent
Let $\phi$ be an associativity constraint. A {\it unit} for
$(\mathcal{C}, \otimes, \phi)$ is an object $\mathbf{1}$
of $\mathcal{C}$ together with functorial isomorphisms
$$
\mathbf{1} \otimes X \to X
\quad\text{and}\quad
X \otimes \mathbf{1} \to X
$$
such that for all objects $X, Y$ the diagram
$$
\xymatrix{
X \otimes (\mathbf{1} \otimes Y) \ar[rr]_\phi \ar[rd] & &
(X \otimes \mathbf{1}) \otimes Y \ar[ld] \\
& X \otimes Y
}
$$
is commutative where the diagonal arrows are given by the isomorphisms
introduced above.

\medskip\noindent
An equivalent definition would be that a unit is a pair
$(\mathbf{1}, 1)$ where $\mathbf{1}$ is an object of $\mathcal{C}$ and
$1 : \mathbf{1} \otimes \mathbf{1} \to \mathbf{1}$
is an isomorphism such that the functors $L : X \mapsto \mathbf{1} \otimes X$
and $R : X \mapsto X \otimes \mathbf{1}$ are equivalences.
Certainly, given a unit as above we get the isomorphism
$1 : \mathbf{1} \otimes \mathbf{1} \to \mathbf{1}$ for free
and $L$ and $R$ are equivalences as they are isomorphic to the
identity functor. Conversely, given $(\mathbf{1}, 1)$ such that
$L$ and $R$ are equivalences, we obtain functorial isomorphisms
$l : \mathbf{1} \otimes X \to X$ and $r : X \otimes \mathbf{1} \to X$
characterized by $L(l) = 1 \otimes \text{id}_X$ and
$R(r) = \text{id}_X \otimes 1$. Then we can use $r$ and $l$
in the notion of unit as above.

\medskip\noindent
A unit is unique up to unique isomorphism if it exists (exercise).

\begin{definition}
\label{definition-monoidal-category}
A triple $(\mathcal{C}, \otimes, \phi)$ where $\mathcal{C}$ is a category,
$\otimes : \mathcal{C} \times \mathcal{C} \to \mathcal{C}$ is a functor,
and $\phi$ is an associativity constraint is called a {\it monoidal category}
if there exists a unit $\mathbf{1}$.
\end{definition}

\noindent
We always write $\mathbf{1}$ to denote a unit of a monoidal category;
it is determined up to unique isomorphism there is no harm in choosing one.
From now on we no longer write the brackets when taking tensor
products in monoidal categories and we always identify
$X \otimes \mathbf{1}$ and $\mathbf{1} \otimes X$ with $X$.
Moreover, we will say ``let $\mathcal{C}$ be a monoidal category''
with $\otimes, \phi, \mathbf{1}$ understood.

\begin{definition}
\label{definition-functor-monoidal-categories}
Let $\mathcal{C}$ and $\mathcal{C}'$ be monoidal categories.
A {\it functor of monoidal categories} $F : \mathcal{C} \to \mathcal{C}'$
is given by a functor $F$ as indicated and a natural transformation
$$
F(X) \otimes F(Y) \to F(X \otimes Y)
$$
such that for all objects $X, Y, Z$ the diagram
$$
\xymatrix{
F(X) \otimes (F(Y) \otimes F(Z)) \ar[r] \ar[d] &
F(X) \otimes F(Y \otimes Z) \ar[r] &
F(X \otimes (Y \otimes Z)) \ar[d] \\
(F(X) \otimes F(Y)) \otimes F(Z) \ar[r] &
F(X \otimes Y) \otimes F(Z) \ar[r] &
F((X \otimes Y) \otimes Z)
}
$$
commutes and such that $F(\mathbf{1})$ is a unit in $\mathcal{C}'$.
\end{definition}

\noindent
By our conventions about units, we may always assume
$F(\mathbf{1}) = \mathbf{1}$ if $F$ is a functor of monoidal categories.
As an example, if $A \to B$ is a ring homomorphism, then
the functor $M \mapsto M \otimes_A B$ is functor of monoidal
categories from $\text{Mod}_A$ to $\text{Mod}_B$.

\begin{lemma}
\label{lemma-invertible}
Let $\mathcal{C}$ be a monoidal category. Let $X$ be an object of
$\mathcal{C}$. The following are equivalent
\begin{enumerate}
\item the functor $L : Y \mapsto X \otimes Y$ is an equivalence,
\item the functor $R : Y \mapsto Y \otimes X$ is an equivalence,
\item there exists an object $X'$ such that
$X \otimes X' \cong X' \otimes X \cong \mathbf{1}$.
\end{enumerate}
\end{lemma}

\begin{proof}
Assume (1). Choose $X'$ such that $L(X') = \mathbf{1}$, i.e.,
$X \otimes X' \cong \mathbf{1}$. Denote $L'$ and $R'$ the functors
corresponding to $X'$. The equation $X \otimes X' \cong \mathbf{1}$
implies $L \circ L' \cong \text{id}$. Thus $L'$ must be the quasi-inverse
to $L$ (which exists by assumption). Hence $L' \circ L \cong \text{id}$.
Hence $X' \otimes X \cong \mathbf{1}$. Thus (3) holds.

\medskip\noindent
The proof of (2) $\Rightarrow$ (3) is dual to what we just said.

\medskip\noindent
Assume (3). Then it is clear that $L'$ and $L$ are quasi-inverse
to each other and it is clear that $R'$ and $R$ are quasi-inverse
to each other. Thus (1) and (2) hold.
\end{proof}

\begin{definition}
\label{definition-invertible}
Let $\mathcal{C}$ be a monoidal category. An object $X$ of $\mathcal{C}$
is called {\it invertible} if any (or all) of the equivalent conditions of
Lemma \ref{lemma-invertible} hold.
\end{definition}

\noindent
Observe that if $F : \mathcal{C} \to \mathcal{C}'$ is a functor of
monoidal categories, then $F$ sends invertible objects to invertible
objects.

\begin{definition}
\label{definition-dual}
Given a monoidal category $(\mathcal{C}, \otimes, \phi)$
and an object $X$ a {\it left dual} is an object $Y$ together with
morphisms $\eta : \mathbf{1} \to X \otimes Y$ and
$\epsilon : Y \otimes X \to \mathbf{1}$
such that the diagrams
$$
\vcenter{
\xymatrix{
X \ar[rd]_1 \ar[r]_-{\eta \otimes 1} &
X \otimes Y \otimes X  \ar[d]^{1 \otimes \epsilon} \\
& X
}
}
\quad\text{and}\quad
\vcenter{
\xymatrix{
Y \ar[rd]_1 \ar[r]_-{1 \otimes \eta} &
Y \otimes X \otimes Y  \ar[d]^{\epsilon \otimes 1} \\
& Y
}
}
$$
commute. In this situation we say that $X$ is a {\it right dual} of $Y$.
\end{definition}

\noindent
Observe that if $F : \mathcal{C} \to \mathcal{C}'$ is a functor of
monoidal categories, then $F(Y)$ is a left dual of $F(X)$ if
$Y$ is a left dual of $X$.

\begin{lemma}
\label{lemma-left-dual}
Let $\mathcal{C}$ be a monoidal category. If $Y$ is a left dual to $X$,
then
$$
\Mor(Z' \otimes X, Z) = \Mor(Z', Z \otimes Y)
\quad\text{and}\quad
\Mor(Y \otimes Z', Z) = \Mor(Z', X \otimes Z)
$$
functorially in $Z$ and $Z'$.
\end{lemma}

\begin{proof}
Consider the maps
$$
\Mor(Z' \otimes X, Z) \to
\Mor(Z' \otimes X \otimes Y, Z \otimes Y) \to
\Mor(Z', Z \otimes Y)
$$
where we use $\eta$ in the second arrow
and the sequence of maps
$$
\Mor(Z', Z \otimes Y) \to
\Mor(Z' \otimes X, Z \otimes Y \otimes X) \to
\Mor(Z' \otimes X, Z)
$$
where we use $\epsilon$ in the second arrow. A straightforward calculation
using the properties of $\eta$ and $\epsilon$
shows that the compositions of these are mutually inverse.
Similarly for the other equality.
\end{proof}

\begin{remark}
\label{remark-left-dual-adjoint}
Lemma \ref{lemma-left-dual} says in particular that $Z \mapsto Z \otimes Y$
is the right adjoint of $Z' \mapsto Z' \otimes X$. Conversely, if this is
true, then we get $\eta : \mathbf{1} \to X \otimes Y$ by evaluating
the unit of the adjunction on $\mathbf{1}$
and $\epsilon : Y \otimes X \to \mathbf{1}$ by evaluating the counit
of the adjunction on $\mathbf{1}$
(Categories, Section \ref{categories-section-adjoint}). Thus the
requirement that $Z \mapsto Z \otimes Y$ be the right adjoint of
$Z' \mapsto Z' \otimes X$ is an equivalent formulation of the
property of being a left dual. Uniqueness of adjoint functors
guarantees that a left dual of $X$, if it exists, is unique up
to unique isomorphism.
\end{remark}

\begin{lemma}
\label{lemma-tensor-dual}
Let $\mathcal{C}$ be a monoidal category. If $Y_i$, $i = 1, 2$
are left duals of $X_i$, $i = 1, 2$, then $Y_2 \otimes Y_1$ is
a left dual of $X_1 \otimes X_2$.
\end{lemma}

\begin{proof}
Follows from uniqueness of adjoints and Remark \ref{remark-left-dual-adjoint}.
\end{proof}

\begin{lemma}
\label{lemma-additive-dual}
Let $\mathcal{C}$ be an additive monoidal category.
If $Y_i$, $i = 1, 2$ are left duals of $X_i$, $i = 1, 2$, then
$Y_1 \oplus Y_2$ is a left dual of $X_1 \oplus X_2$.
\end{lemma}

\begin{proof}
Follows from uniqueness of adjoints and Remark \ref{remark-left-dual-adjoint}.
\end{proof}

\begin{lemma}
\label{lemma-Karoubian-dual}
In an additive Karoubian monoidal category every summand
of an object which has a left dual has a left dual.
\end{lemma}

\begin{proof}
We will use Lemma \ref{lemma-left-dual} without further mention.
Let $X$ be an object which has a left dual $Y$. We have
$$
\Hom(X, X) = \Hom(\mathbf{1}, X \otimes Y) = \Hom(Y, Y)
$$
If $a : X \to X$ corresponds to $b : Y \to Y$ then $b$ is the unique
endomorphism of $Y$ such that precomposing by $a$ on
$$
\Hom(Z' \otimes X, Z) = \Hom(Z', Z \otimes Y)
$$
is the same as postcomposing by $1 \otimes b$.
Hence the bijection $\Hom(X, X) \to \Hom(Y, Y)$, $a \mapsto b$
is an isomorphism of the opposite of the algebra $\Hom(X, X)$ with
the algebra $\Hom(Y, Y)$. In particular, if $X = X_1 \oplus X_2$, then
the corresponding projectors $e_1, e_2$ are mapped to idempotents
in $\Hom(Y, Y)$. If $Y = Y_1 \oplus Y_2$ is the corresponding direct
sum decomposition of $Y$ (Homology, Section \ref{homology-section-karoubian})
then we see that under the bijection
$\Hom(Z' \otimes X, Z) = \Hom(Z', Z \otimes Y)$
we have $\Hom(Z' \otimes X_i, Z) = \Hom(Z', Z \otimes Y_i)$
functorially as subgroups for $i = 1, 2$.
It follows that $Y_i$ is the left dual of
$X_i$ by the discussion in Remark \ref{remark-left-dual-adjoint}.
\end{proof}

\begin{lemma}
\label{lemma-left-dual-graded-vector-spaces}
Let $F$ be a field. Let $\mathcal{C}$ be the category of graded
$F$-vector spaces viewed as a monoidal category
with the usual tensor structure. If $V$ in $\mathcal{C}$
has a left dual $W$, then $\sum_i \dim_F V^i < \infty$
and the map $\epsilon$ defines nondegenerate pairings
$W^{-i} \times V^i \to F$.
\end{lemma}

\begin{proof}
Omitted.
\end{proof}

\noindent
A {\it commutativity constraint} for $(\mathcal{C}, \otimes)$ is a
functorial isomorphism
$$
\psi : X \otimes Y \longrightarrow Y \otimes X
$$
such that the composition
$$
X \otimes Y \xrightarrow{\psi} Y \otimes X \xrightarrow{\psi} X \otimes Y
$$
is the identity. We say $\psi$ is {\it compatible} with a given associativity
constraint $\phi$ if for all objects $X, Y, Z$ the diagram
$$
\xymatrix{
X \otimes (Y \otimes Z) \ar[r]_\phi \ar[d]^\psi &
(X \otimes Y) \otimes Z \ar[r]_\psi &
Z \otimes (X \otimes Y) \ar[d]^\phi \\
X \otimes (Z \otimes Y) \ar[r]^\phi &
(X \otimes Z) \otimes Y \ar[r]^\psi &
(Z \otimes X) \otimes Y
}
$$
commutes.

\begin{definition}
\label{definition-symmetric-monoidal-category}
A quadruple $(\mathcal{C}, \otimes, \phi, \psi)$ where
$\mathcal{C}$ is a category,
$\otimes : \mathcal{C} \otimes \mathcal{C} \to \mathcal{C}$ is a functor,
$\phi$ is an associativity constraint, and
$\psi$ is a commutativity constraint compatible with $\phi$
is called a {\it symmetric monoidal category} if there exists
a unit.
\end{definition}

\noindent
To be sure, if $(\mathcal{C}, \otimes, \phi, \psi)$ is a
symmetric monoidal category, then $(\mathcal{C}, \otimes, \phi)$
is a monoidal category.

\begin{example}
\label{example-graded-vector-spaces}
Let $F$ be a field. Let $\mathcal{C}$ be the category of graded
$F$-vector spaces with its usual tensor structure. There are
two commutativity constraints on $\mathcal{C}$ which turn $\mathcal{C}$
into a symmetric monoidal category: one involves the
intervention of signs and the other does not. In this chapter we will
use the one that does. To be explicit, if $V$ and $W$ are graded $F$-vector
spaces we will use the isomorphism
$$
\psi : V \otimes_F W \longrightarrow W \otimes_F V
$$
which sends $v \otimes w$ to $(-1)^{ab}w \otimes v$ if
$v \in V^a$ and $w \in W^b$.
\end{example}

\begin{definition}
\label{definition-functor-symmetric-monoidal-categories}
Let $\mathcal{C}$ and $\mathcal{C}'$ be symmetric monoidal categories.
A {\it functor of symmetric monoidal categories}
$F : \mathcal{C} \to \mathcal{C}'$
is given by a functor $F$ as indicated and a natural transformation
$$
F(X) \otimes F(Y) \to F(X \otimes Y)
$$
such that $F$ is a functor of monoidal categories and such that
for all objects $X, Y$ the diagram
$$
\xymatrix{
F(X) \otimes F(Y) \ar[r] \ar[d] &
F(X \otimes Y) \ar[d] \\
F(Y) \otimes F(X) \ar[r] &
F(Y \otimes X)
}
$$
commutes.
\end{definition}








\section{Correspondences}
\label{section-correspondences}

\noindent
Let $k$ be a field. In this section we construct the graded category over
$\mathbf{Q}$ whose objects are smooth projective schemes over $k$ and whose
morphisms are correspondences.

\medskip\noindent
Let $X$ and $Y$ be smooth projective schemes over $k$.
Let $X = \coprod X_d$ be the decomposition of $X$ into
the open and closed subschemes which are equidimensional
with $\dim(X_d) = d$. We define the $\mathbf{Q}$-vector space
{\it of correspondences of degree $r$ from $X$ to $Y$}
by the formula:
$$
\text{Corr}^r(X, Y) =
\bigoplus\nolimits_d \CH^{d + r}(X_d \times Y) \otimes \mathbf{Q}
\subset
\CH^*(X \times Y) \otimes \mathbf{Q}
$$
Given $c \in \text{Corr}^r(X, Y)$ and $\beta \in \CH_k(Y) \otimes \mathbf{Q}$
we can define the {\it pullback} of $\beta$ by $c$ using the formula
$$
c^*(\beta) = \text{pr}_{1, *}(c \cap \text{pr}_2^*\beta)
\quad\text{in}\quad
\CH_{k - r}(X) \otimes \mathbf{Q}
$$
This makes sense because $\text{pr}_2$ is flat of relative dimension
$d$ on $X_d \times Y$, hence $\text{pr}_2^*\beta$ is a cycle of
dimension $d + k$ on $X_d \times Y$, hence $c \cap \text{pr}_2^*\alpha$
is a cycle of dimension $k - r$ on $X_d \times Y$ whose pushforward
by the proper morphism $\text{pr}_1$ is a cycle of the same dimension.
Similarly, switching to grading by codimension,
given $\alpha \in \CH^i(X) \otimes \mathbf{Q}$ we can define the
{\it pushforward} of $\alpha$ by $c$ using the formula
$$
c_*(\alpha) = \text{pr}_{2, *}(c \cap \text{pr}_1^*\alpha)
\quad\text{in}\quad
\CH^{i + r}(Y) \otimes \mathbf{Q}
$$
This makes sense because $\text{pr}_1^*\alpha$ is a cycle of codimension
$i$ on $X \times Y$, hence $c \cap \text{pr}_1^*\alpha$ is a cycle
of codimension $i + d + r$ on $X_d \times Y$, which pushes forward
to a cycle of codimension $i + r$ on $Y$.

\medskip\noindent
Given a three smooth projective schemes $X, Y, Z$ over $k$ we define a
{\it composition of correspondences}
$$
\text{Corr}^s(Y, Z)
\times
\text{Corr}^r(X, Y)
\longrightarrow
\text{Corr}^{r + s}(X, Z)
$$
by the rule
$$
(c', c)
\longmapsto
c' \circ c =
\text{pr}_{13, *}(\text{pr}_{12}^*c \cdot \text{pr}_{23}^*c')
$$
where $\text{pr}_{12} : X \times Y \times Z \to X \times Y$
is the projection and similarly for $\text{pr}_{13}$ and $\text{pr}_{23}$.

\begin{lemma}
\label{lemma-composition-correspondences}
We have the following for correspondences:
\begin{enumerate}
\item composition of correspondences is $\mathbf{Q}$-bilinear
and associative,
\item there is a canonical isomorphism
$$
\CH_{-r}(X) = \text{Corr}^r(X, \Spec(k))
$$
such that pullback by correspondences corresponds to composition,
\item there is a canonical isomorphism
$$
\CH^r(X) \otimes \mathbf{Q} = \text{Corr}^r(\Spec(k), X)
$$
such that pushforward by correspondences corresponds to composition,
\item composition of correspondences is compatible with pushforward and
pullback of cycles.
\end{enumerate}
\end{lemma}

\begin{proof}
Bilinearity follows immediately from the linearity of pushforward
and pullback and the bilinearity of the intersection product.
To prove associativity, say we have
$X, Y, Z, W$ and $c \in \text{Corr}(X, Y)$, $c' \in \text{Corr}(Y, Z)$, and
$c'' \in \text{Corr}(Z, W)$. Then we have
\begin{align*}
c'' \circ (c' \circ c)
& =
\text{pr}^{134}_{14, *}(
\text{pr}^{134, *}_{13}
\text{pr}^{123}_{13, *}(\text{pr}^{123, *}_{12}c \cdot
\text{pr}^{123, *}_{23}c')
\cdot \text{pr}^{134, *}_{34}c'') \\
& =
\text{pr}^{134}_{14, *}(
\text{pr}^{1234}_{134, *}
\text{pr}^{1234, *}_{123}(\text{pr}^{123, *}_{12}c \cdot
\text{pr}^{123, *}_{23}c')
\cdot \text{pr}^{134, *}_{34}c'') \\
& =
\text{pr}^{134}_{14, *}(
\text{pr}^{1234}_{134, *}
(\text{pr}^{1234, *}_{12}c \cdot
\text{pr}^{1234, *}_{23}c')
\cdot \text{pr}^{134, *}_{34}c'') \\
& =
\text{pr}^{134}_{14, *}
\text{pr}^{1234}_{134, *}
((\text{pr}^{1234, *}_{12}c \cdot
\text{pr}^{1234, *}_{23}c')
\cdot \text{pr}^{1234, *}_{34}c'') \\
& =
\text{pr}^{1234}_{14, *}(
(\text{pr}^{1234, *}_{12}c \cdot
\text{pr}^{1234, *}_{23}c') \cdot
\text{pr}^{1234, *}_{34}c'')
\end{align*}
Since intersection product is associative this concludes the proof
of associativity of composition of correspondences.

\medskip\noindent
We omit the proofs of (2) and (3).

\medskip\noindent
The statement on pushforward and pullback of cycles
means that $(c' \circ c)^*(\alpha) = c^*((c')^*(\alpha))$ and
$(c' \circ c)_*(\alpha) = (c')_*(c_*(\alpha))$.
This follows on combining (1), (2), and (3).
\end{proof}

\begin{example}
\label{example-graph-correspondence}
Let $f : Y \to X$ be a morphism of smooth projective schemes over $k$.
Denote $\Gamma_f \subset X \times Y$ the graph of $f$. More precisely,
$\Gamma_f$ is the image of the closed immersion
$$
(f, \text{id}_Y) : Y \longrightarrow X \times Y
$$
If $X$ is equidimensional of dimension $d$, then $\Gamma_f$
has pure codimension $d$. Hence
$[\Gamma_f] \in \CH^*(X \times Y) \otimes \mathbf{Q}$
is contained in $\text{Corr}^0(X \times Y)$, i.e., $[\Gamma_f]$
is a correspondence of degree $0$ from $X$ to $Y$.
\end{example}

\begin{lemma}
\label{lemma-identity-correspondence}
Let $X$ be a smooth projective scheme over $k$. Consider the diagonal
$\Delta \subset X \times X$ of $X$. The class of $\Delta$ is an
unit element of the graded algebra $\text{Corr}^*(X, X)$.
\end{lemma}

\begin{proof}
Omitted.
\end{proof}

\begin{lemma}
\label{lemma-contravariant-functor}
There is a contravariant functor from the category of smooth
projective schemes over $k$ to the category of correspondences
which is the identity on objects and sends $f : Y \to X$ to
the element $[\Gamma_f] \in \text{Corr}^0(X, Y)$.
\end{lemma}

\begin{proof}
To see this we have to show if $g : Z \to Y$ is another morphism of
smooth projective schemes over $k$, then we have
$[\Gamma_g] \circ [\Gamma_f] = [\Gamma_{g \circ f}]$ in
$\text{Corr}^0(X, Z)$. Details omitted.
\end{proof}

\begin{remark}
\label{remark-transpose}
Let $X$ and $Y$ be smooth projective schemes over $k$.
Assume $X$ is equidimensional of dimension $d$ and
$Y$ is equidimensional of dimension $e$. Then the isomorphism
$X \times Y \to Y \times X$ switching the factors determines
an isomorphism
$$
\text{Corr}^r(X, Y) \longrightarrow \text{Corr}^{d - e + r}(Y, X),\quad
c \longmapsto c^t
$$
called the {\it transpose}. It acts on cycles as well as cycle classes.
An example which is sometimes useful, is the transpose
$[\Gamma_f]^t = [\Gamma_f^t]$ of the graph of a morphism $f : Y \to X$.
\end{remark}

\begin{lemma}
\label{lemma-functor-and-cycles}
Let $f : Y \to X$ be a morphism of smooth projective schemes over $k$.
Let $[\Gamma_f] \in \text{Corr}^0(X, Y)$ be as in
Example \ref{example-graph-correspondence}. Then
\begin{enumerate}
\item pushforward of cycles by the correspondence $[\Gamma_f]$
corresponds to pullback of cycles by $f$,
\item pullback of cycles by the correspondence $[\Gamma_f]$
corresponds to pushforward of cycles by $f$,
\item if $X$ and $Y$ are equidimensional of dimensions $d$ and $e$,
then
\begin{enumerate}
\item pushforward of cycles by the correspondence
$[\Gamma_f^t]$ of Remark \ref{remark-transpose}
corresponds to pushforward of cycles by $f$, and
\item pullback of cycles by the correspondence
$[\Gamma_f^t]$ of Remark \ref{remark-transpose}
corresponds to pullback of cycles by $f$.
\end{enumerate}
\end{enumerate}
\end{lemma}

\begin{proof}
Omitted.
\end{proof}

\begin{example}
\label{example-decompose-P1}
Let $X = \mathbf{P}^1_k$. Then we have
$$
\text{Corr}^0(X, X) = \CH^1(X \times X) = \CH_1(X \times X)
$$
Choose a $k$-rational point $x \in X$ and
consider the cycles $c_0 = [x \times X]$ and $c_2 = [X \times x]$.
A computation shows that $1 = [\Delta] = c_0 + c_2$ in $\text{Corr}^0(X, X)$
and that we have the following rules for composition
$c_0 \circ c_0 = c_0$,
$c_0 \circ c_2 = 0$,
$c_2 \circ c_0 = 0$, and
$c_2 \circ c_2 = c_2$.
In other words, $c_0$ and $c_2$ are orthogonal idempotents in
the algebra $\text{Corr}^0(X, X)$ and in fact we get
$$
\text{Corr}^0(X, X) = \mathbf{Q} \times \mathbf{Q}
$$
as a $\mathbf{Q}$-algebra.
\end{example}

\noindent
The category of correspondences is a symmetric monoidal category.
Given smooth projective schemes $X$ and $Y$ over $k$, we define
$X \otimes Y = X \times Y$. As associativity constraint
$$
X \otimes (Y \otimes Z) = (X \otimes Y) \otimes Z
$$
we use the usual associativity constraint on products of schemes.
The unit object will be $\Spec(k)$. The commutativity will be
given by the isomorphism $X \times Y \to Y \times X$ switching the factors.
Given four smooth projective schemes $X, X', Y, Y'$ over $k$
we define a tensor product
$$
\otimes :
\text{Corr}^r(X, Y) \times \text{Corr}^{r'}(X', Y')
\longrightarrow
\text{Corr}^{r + r'}(X \times X', Y \times Y')
$$
by the rule
$$
(c, c') \longmapsto
c \otimes c' = \text{pr}_{13}^*c \cdot \text{pr}_{24}^*c'
$$
where $\text{pr}_{13} : X \times X' \times Y \times Y' \to X \times Y$
and $\text{pr}_{24} : X \times X' \times Y \times Y' \to X' \times Y'$
are the projections.

\begin{lemma}
\label{lemma-tensor-product}
The tensor product of correspondences defined above turns the category of
correspondences into a symmetric monoidal category with unit $\Spec(k)$.
\end{lemma}

\begin{proof}
Omitted.
\end{proof}

\begin{lemma}
\label{lemma-prep-dual}
Let $f : Y \to X$ be a morphism of smooth projective schemes over $k$.
Assume $X$ and $Y$ equidimensional of dimensions $d$ and $e$.
Denote $a = [\Gamma_f] \in \text{Corr}^0(X, Y)$ and
$a^t = [\Gamma_f^t] \in \text{Corr}^{d - e}(Y, X)$.
Set
$\eta_X = [\Gamma_{X \to X \times X}] \in \text{Corr}^0(X \times X, X)$,
$\eta_Y = [\Gamma_{Y \to Y \times Y}] \in \text{Corr}^0(Y \times Y, Y)$,
$[X] \in \text{Corr}^d(X, \Spec(k))$, and
$[Y] \in \text{Corr}^e(Y, \Spec(k))$. The diagram
$$
\xymatrix{
X \otimes Y \ar[r]_{a^t \otimes \text{id}} \ar[d]_{\text{id} \otimes a} &
Y \otimes Y \ar[r]_{\eta_Y} &
Y \ar[d]^{[Y]} \\
X \otimes X \ar[r]^{\eta_X} &
X \ar[r]^{[X]} &
\Spec(k)
}
$$
is commutative in the category of correspondences.
\end{lemma}

\begin{proof}
Going either way around the diagram a computation shows that we
obtain the element of $\text{Corr}^d(X \times Y, \Spec(k))$
corresponding to the cycle $\Gamma_f \subset X \times Y$.
\end{proof}







\section{Chow motives}
\label{section-chow-motives}

\noindent
We fix a base field $k$. We want to define an additive
Karoubian $\mathbf{Q}$-linear category $M_k$ endowed
with a monoidal structure and a contravariant functor
$$
h : \{\text{smooth projective schemes over }k\} \longrightarrow M_k
$$
which maps products to tensor products and disjoint unions to direct sums.
Finally, we want objects of $M_k$ to have (left) duals.

\medskip\noindent
A {\it motive} or a {\it Chow motive} over $k$ will be a triple
$(X, p, m)$ where
\begin{enumerate}
\item $X$ is a smooth projective scheme over $k$,
\item $p \in \text{Corr}^0(X, X)$ satisfies $p \circ p = p$,
\item $m \in \mathbf{Z}$.
\end{enumerate}
Given a second motive $(Y, q, n)$ we define a
{\it morphism of motives} or a {\it morphism of Chow motives}
to be an element of
$$
\Hom((X, p, m), (Y, q, n)) =
q \circ \text{Corr}^{n - m}(X, Y) \circ p \subset \text{Corr}^{n - m}(X, Y)
$$
Composition of morphisms of motives is defined using the composition of
correspondences defined above.

\begin{lemma}
\label{lemma-motives}
The category $M_k$ whose objects are motives over $k$ and morphisms
are morphisms of motives over $k$ is a $\mathbf{Q}$-linear category.
There is a contravariant functor
$$
h : \{\text{smooth projective schemes over }k\} \longrightarrow M_k
$$
defined by $h(X) = (X, 1, 0)$ and $h(f) = [\Gamma_f]$.
\end{lemma}

\begin{proof}
Follows immediately from Lemma \ref{lemma-contravariant-functor}.
\end{proof}

\begin{lemma}
\label{lemma-Karoubian}
The category $M_k$ is Karoubian.
\end{lemma}

\begin{proof}
Let $M = (X, p, m)$ be a motive and let $a \in \Mor(M, M)$
be a projector. Then $a = a \circ a$ both in $\Mor(M, M)$
as well as in $\text{Corr}^0(X, X)$. Set $N = (X, a, m)$.
Since we have $a = p \circ a \circ a$ in $\text{Corr}^0(X, X)$
we see that $a : N \to M$ is a morphism of $M_k$.
Next, suppose that $b : (Y, q, n) \to M$ is a morphism
such that $(1 - a) \circ b = 0$. Then $b = a \circ b$ as well as
$b = b \circ q$. Hence $b$ is a morphism $b : (Y, q, n) \to N$.
Thus we see that the projector $1 - a$ has a kernel, namely $N$
and we find that $M_k$ is Karoubian, see
Homology, Definition \ref{homology-definition-karoubian}.
\end{proof}

\noindent
We define a functor
$$
\otimes : M_k \times M_k \longrightarrow M_k
$$
On objects we use the formula
$$
(X, p, m) \otimes (Y, q, n) = (X \times Y, p \otimes q, m + n)
$$
On morphisms, we use
$$
\xymatrix{
\Mor((X, p, m), (Y, q, n)) \times
\Mor((X', p', m'), (Y', q', n')) \ar[d] \\
\Mor(
(X \times X', p \otimes p', m + m'),
(Y \times Y', q \otimes q', n + n'))
}
$$
given by the simple rule $(a, a') \longmapsto a \otimes a'$ where
$\otimes$ on correspondences is as in Section \ref{section-correspondences}.
This makes sense: by definition of morphisms of motives
we can write $a = q \circ c \circ p$ and $a' = q' \circ c' \circ p'$
with $c \in \text{Corr}^{n - m}(X, Y)$ and
$c' \in \text{Corr}^{n' - m'}(X', Y')$
and then we obtain
$$
a \otimes a' =
(q \circ c \circ p) \otimes (q' \circ c' \circ p') =
(q \otimes q') \circ (c \otimes c') \circ (p \otimes p')
$$
which is indeed a morphism of motives from
$(X \times X', p \otimes p', m + m')$ to
$(Y \times Y', q \otimes q', n + n')$.

\begin{lemma}
\label{lemma-motives-monoidal}
The category $M_k$ with tensor product defined as above
is symmetric monoidal with the obvious associativity and commutativity
constraints and with unit $\mathbf{1} = (\Spec(k), 1, 0)$.
\end{lemma}

\begin{proof}
Follows readily from Lemma \ref{lemma-tensor-product}. Details omitted.
\end{proof}

\noindent
The motives $\mathbf{1}(n) = (\Spec(k), 1, n)$ are useful. Observe that
$$
\mathbf{1} = \mathbf{1}(0)
\quad\text{and}\quad
\mathbf{1}(n + m) = \mathbf{1}(n) \otimes \mathbf{1}(m)
$$
Thus tensoring with $\mathbf{1}(1)$ is an autoequivalence of the
category of motives. Given a motive $M$ we sometimes write
$M(n) = M \otimes \mathbf{1}(n)$. Observe that if $M = (X, p, m)$,
then $M(n) = (X, p, m + n)$.

\begin{lemma}
\label{lemma-inverse-h2}
With notation as in Example \ref{example-decompose-P1}
\begin{enumerate}
\item
the motive $(X, c_0, 0)$ is isomorphic to the motive
$\mathbf{1} = (\Spec(k), 1, 0)$.
\item
the motive $(X, c_2, 0)$ is isomorphic to the motive
$\mathbf{1}(-1) = (\Spec(k), 1, -1)$.
\end{enumerate}
\end{lemma}

\begin{proof}
We will use Lemma \ref{lemma-contravariant-functor} without further mention.
The structure morphism $X \to \Spec(k)$ gives a correspondence
$a \in \text{Corr}^0(\Spec(k), X)$. On the other hand, the rational
point $x$ is a morphism $\Spec(k) \to X$ which gives a correspondence
$b \in \text{Corr}^0(X, \Spec(k))$. We have $b \circ a = 1$ as a
correspondence on $\Spec(k)$. The composition $a \circ b$ corresponds
to the graph of the composition $X \to x \to X$ which is
$c_0 = [x \times X]$. It follows that $a = c_0 \circ a$ and
$b = b \circ c_0$. Hence, unwinding the definitions, we see that
$a$ and $b$ are mutually inverse morphisms
$a : (\Spec(k), 1, 0) \to (X, c_0, 0)$ and
$b : (X, c_0, 0) \to (\Spec(k), 1, 0)$.

\medskip\noindent
We will proceed exactly as above to prove the second statement.
Denote
$$
a' \in \text{Corr}^1(\Spec(k), X) = \CH^1(X)
$$
the class of the point $x$. Denote
$$
b' \in \text{Corr}^{-1}(X, \Spec(k)) = \CH^0(X)
$$
the class of $[X]$. Then $b' \circ a' = 1$ as a correspondence on $\Spec(k)$.
Computing the intersection product
$\text{pr}_{12}^*b' \cdot \text{pr}_{23}^*a'$
on $X \times \Spec(k) \times X$ gives the cycle
$X \times \Spec(k) \times x$. Hence
the composition $a' \circ b'$ is equal to $c_2$ as a
correspondence on $X$. It follows that $a' = c_0 \circ a'$ and
$b' = b' \circ c_0$. Recall that
$$
\Mor((\Spec(k), 1, -1), (X, c_2, 0)) =
c_2 \circ \text{Corr}^1(\Spec(k), X)
\subset
\text{Corr}^1(\Spec(k), X)
$$
and
$$
\Mor((X, c_2, 0), (\Spec(k), 1, -1)) =
\text{Corr}^{-1}(X, \Spec(k)) \circ c_2
\subset
\text{Corr}^{-1}(X, \Spec(k))
$$
Hence, we see that $a'$ and $b'$ are mutually inverse morphisms
$a' : (\Spec(k), 1, -1) \to (X, c_0, 0)$ and
$b' : (X, c_0, 0) \to (\Spec(k), 1, -1)$.
\end{proof}

\begin{lemma}
\label{lemma-additive}
The category $M_k$ is additive.
\end{lemma}

\begin{proof}
Let $(Y, p, m)$ and $(Z, q, n)$ be motives. If $n = m$, then a
direct sum is given by $(Y \amalg Z, p + q, m)$, with obvious notation.
Details omitted.

\medskip\noindent
Suppose that $n < m$. Let $X$, $c_2$ be as in
Example \ref{example-decompose-P1}. Then we consider
\begin{align*}
(Z, q, n)
& =
(Z, q, m) \otimes (\Spec(k), 1, -1) \otimes \ldots \otimes
(\Spec(k), 1, -1) \\
& \cong
(Z, q, m) \otimes (X, c_2, 0) \otimes \ldots \otimes (X, c_2, 0) \\
& \cong
(Z \times X^{m - n}, q \otimes c_2 \otimes \ldots \otimes c_2, m)
\end{align*}
where we have used Lemma \ref{lemma-inverse-h2}.
This reduces us to the case discussed in the first paragraph.
\end{proof}

\begin{lemma}
\label{lemma-decompose-P1}
In $M_k$ we have $h(\mathbf{P}^1_k) \cong \mathbf{1} \oplus \mathbf{1}(-1)$.
\end{lemma}

\begin{proof}
This follows from Example \ref{example-decompose-P1} and
Lemma \ref{lemma-inverse-h2}.
\end{proof}

\begin{lemma}
\label{lemma-characterize-motives}
Let $X$, $c_2$ be as in Example \ref{example-decompose-P1}.
Let $\mathcal{C}$ be a Karoubian symmetric monoidal category.
Any functor
$$
F :
\left\{
\begin{matrix}
\text{smooth projective schemes over }k\\
\text{morphisms are correspondence of degree }0
\end{matrix}
\right\}
\longrightarrow
\mathcal{C}
$$
of symmetric monoidal categories such that the image of $F(c_2)$ on
$F(X)$ is an invertible object, factors uniquely through a functor
$F : M_k \to \mathcal{C}$ of symmetric monoidal categories.
\end{lemma}

\begin{proof}
Denote $U$ in $\mathcal{C}$ the invertible object which is assumed to exist
in the statement of the lemma. We extend $F$ to motives by setting
$$
F(X, p, m) = \left(\text{the image of
the projector }F(p)\text{ in }F(X)\right) \otimes U^{\otimes -m}
$$
which makes sense because $U$ is invertible and because $\mathcal{C}$
is Karoubian. An important feature of this choice is that
$F(X, c_2, 0) = U$. Observe that
\begin{align*}
F((X, p, m) \otimes (Y, q, n))
& =
F(X \times Y, p \otimes q, m + n) \\
& =
\left(\text{the image of }F(p \otimes q)\text{ in }F(X \times Y)\right)
\otimes U^{\otimes -m - n} \\
& =
F(X, p, m) \otimes F(Y, q, n)
\end{align*}
Thus we see that our rule is compatible with tensor products on
the level of objects (details omitted).

\medskip\noindent
Next, we extend $F$ to morphisms of motives. Suppose that
$$
a \in
\Hom((Y, p, m), (Z, q, n)) =
q \circ \text{Corr}^{n - m}(Y, Z) \circ p \subset \text{Corr}^{n - m}(Y, Z)
$$
is a morphism. If $n = m$, then $a$ is a correspondence of degree $0$
and we can use $F(a) : F(Y) \to F(Z)$ to get the desired map
$F(Y, p, m) \to F(Z, q, n)$. If $n < m$ we get canonical identifications
\begin{align*}
s : F((Z, q, n))
& \to
F(Z, q, m) \otimes U^{m - n} \\
& \to
F(Z, q, m) \otimes F(X, c_2, 0) \otimes \ldots \otimes F(X, c_2, 0) \\
& \to
F((Z, q, m) \otimes (X, c_2, 0) \otimes \ldots \otimes (X, c_2, 0)) \\
& \to
F((Z \times X^{m - n}, q \otimes c_2 \otimes \ldots \otimes c_2, m))
\end{align*}
Namely, for the first isomorphism we use the definition of $F$ on motives
above. For the second, we use the choice of $U$. For the third we use
the compatibility of $F$ on tensor products of motives. The fourth
is the definition of tensor products on motives. On the other hand, since
we similarly have an isomorphism
$$
\sigma : (Z, q, n) \to
(Z \times X^{m - n}, q \otimes c_2 \otimes \ldots \otimes c_2, m)
$$
(see proof of Lemma \ref{lemma-additive}). Composing $a$ with this isomorphism
gives
$$
\sigma \circ a \in
\Hom((Y, p, m),
(Z \times X^{m - n}, q \otimes c_2 \otimes \ldots \otimes c_2, m))
$$
Putting everything together we obtain
$$
s^{-1} \circ F(\sigma \circ a) :
F(Y, p, m) \to
F(Z, q, n)
$$
If $n > m$ we similarly define isomorphisms
$$
t : F((Y, p, m)) \to
F((Y \times X^{n - m}, p \otimes c_2 \otimes \ldots \otimes c_2, n))
$$
and
$$
\tau : (Y, p, m)) \to
(Y \times X^{n - m}, p \otimes c_2 \otimes \ldots \otimes c_2, n)
$$
and we set $F(a) = F(a \circ \tau^{-1}) \circ t$.
We omit the verification that this construction defines a functor
of symmetric monoidal categories.
\end{proof}

\begin{lemma}
\label{lemma-dual}
Let $X$ be a smooth projective scheme over $k$ which is equidimensional
of dimension $d$. Then $h(X)(d)$ is a left dual to $h(X)$ in $M_k$.
\end{lemma}

\begin{proof}
We will use Lemma \ref{lemma-composition-correspondences}
without further mention. We compute
$$
\Hom(\mathbf{1}, h(X) \otimes h(X)(d)) =
\text{Corr}^d(\Spec(k), X \times X) = \CH^d(X \times X)
$$
Here we have $\eta = [\Delta]$. On the other hand, we have
$$
\Hom(h(X)(d) \otimes h(X), \mathbf{1}) =
\text{Corr}^{-d}(X \times X, \Spec(k)) = \CH_d(X \times X)
$$
and here we have the class $\epsilon = [\Delta]$
of the diagonal as well. The composition of the correspondence
$[\Delta] \otimes 1$ with $1 \otimes [\Delta]$ either way
is the correspondence $[\Delta] = 1$. This proves the lemma.
\end{proof}

\begin{lemma}
\label{lemma-dual-general}
Every object of $M_k$ has a left dual.
\end{lemma}

\begin{proof}
Let $M = (X, p, m)$ be an object of $M_k$. Then $M$ is a summand of
$(X, 0, m) = h(X)(m)$.
By Lemma \ref{lemma-Karoubian-dual} it suffices to show that
$h(X)(m) = h(X) \otimes \mathbf{1}(m)$ has a dual.
By construction $\mathbf{1}(-m)$ is a left dual of $\mathbf{1}(m)$.
Hence it suffices to show that $h(X)$ has a left dual, see
Lemma \ref{lemma-tensor-dual}.
Let $X = \coprod X_i$ be the decomposition of $X$ into
irreducible components. Then $h(X) = \bigoplus h(X_i)$
and it suffices to show that $h(X_i$ has a left dual, see
Lemma \ref{lemma-additive-dual}.
This follows from Lemma \ref{lemma-dual}.
\end{proof}






\section{Chow groups of motives}
\label{section-chow-groups-motives}

\noindent
We define the Chow groups of a motive as follows.

\begin{definition}
\label{definition-chow-group-motives}
Let $k$ be a base field. Let $M = (X, p, m)$ be a Chow motive over $k$.
For $i \in \mathbf{Z}$ we define the {\it $i$th Chow group of $M$}
by the formula
$$
\CH^i(M) = p\left(\CH^{i + m}(X) \otimes \mathbf{Q}\right)
$$
\end{definition}

\noindent
We have $\CH^i(h(X)) = \CH^i(X) \otimes \mathbf{Q}$
if $X$ is a smooth projective scheme over $k$.

\medskip\noindent
Observe that $\CH^i(-)$ is a functor from $M_k$ to $\mathbf{Q}$-vector spaces.
Indeed, if $c : M \to N$ is a morphism of motives
$M = (X, p, m)$ and $N = (Y, q, n)$, then $c$ is a correspondence of
degree $n - m$ from $X$ to $Y$ and hence pushforward along $c$
(Section \ref{section-correspondences}) is a family of maps
$$
c_* :
\CH^{i + m}(X) \otimes \mathbf{Q}
\longrightarrow
\CH^{i + n}(Y) \otimes \mathbf{Q}
$$
Since $c = q \circ c \circ p$ by definition of morphisms of motives,
we see that indeed we obtain
$$
c_* : \CH^i(M) \to \CH^i(N)
$$
for all $i \in \mathbf{Z}$. We omit the verification that this
is compatible with compositions of morphisms of motives (this follows
from Lemma \ref{lemma-composition-correspondences}).
This functoriality of Chow groups can also be deduced from the following
lemma.

\begin{lemma}
\label{lemma-chow-groups-representable}
Let $k$ be a base field. The functor $\CH^i(-)$ on the category
of motives $M_k$ is representable by $\mathbf{1}(i)$, i.e., we
have
$$
\CH^i(M) = \Hom_{M_k}(\mathbf{1}(-i), M)
$$
functorially in $M$ in $M_k$.
\end{lemma}

\begin{proof}
Immediate from the definitions.
\end{proof}

\noindent
The reader can imagine that we can use this, the Yoneda lemma, and
the duality in Lemma \ref{lemma-dual} to obtain the following.

\begin{lemma}[Manin]
\label{lemma-manin}
Let $k$ be a base field. Let $c : M \to N$ be a morphism of motives.
If for every smooth projective scheme $X$ over $k$ the map
$c \otimes 1 : M \otimes h(X) \to N \otimes h(X)$ induces an isomorphism on
Chow groups, then $c$ is an isomorphism.
\end{lemma}

\begin{proof}
Any object $L$ of $M_k$ is a summand of $h(X)(m)$ for some smooth projective
scheme $X$ over $k$ and some $m \in \mathbf{Z}$. Observe that the Chow groups
of $M \otimes h(X)(m)$ are the same as the Chow groups of of $M \otimes h(X)$
up to a shift in degrees. Hence our assumption implies
that $c \otimes 1 : M \otimes L \to N \otimes L$ induces an isomorphism on
Chow gruops for every object $L$ of $M_k$. By
Lemma \ref{lemma-chow-groups-representable}
we see that
$$
\Hom_{M_k}(\mathbf{1}, M \otimes L) \to
\Hom_{M_k}(\mathbf{1}, N \otimes L)
$$
is an isomorphism for every $L$. Since every object of $M_k$ has a left dual
(Lemma \ref{lemma-dual-general}) we conclude that
$$
\Hom_{M_k}(K, M) \to \Hom_{M_k}(K, N)
$$
is an isomorphism for every object $K$ of $M_k$, see
Lemma \ref{lemma-left-dual}. We conclude by the Yoneda lemma
(Categories, Lemma \ref{categories-lemma-yoneda}).
\end{proof}




\section{Projective space bundle formula}
\label{section-projective-space-bundle}

\noindent
Let $k$ be a base field. Let $X$ be a smooth projective scheme over $k$.
Let $\mathcal{E}$ be a locally free $\mathcal{O}_X$-module of rank $r$.
Our convention is that the {\it projective bundle associated to
$\mathcal{E}$} is the morphism
$$
\xymatrix{
P = \mathbf{P}(\mathcal{E}) =
\underline{\text{Proj}}_X(\text{Sym}^*(\mathcal{E}))
\ar[r]^-p
& X
}
$$
over $X$ with $\mathcal{O}_P(1)$ normalized so that
$p_*(\mathcal{O}_P(1)) = \mathcal{E}$.
Denote $\xi = c_1(\mathcal{O}_P(1)) \cap [P] \in \CH^1(P)$ the first
chern class. For $i = 0, \ldots, r - 1$ consider the correspondences
$$
c_i = [\Gamma_p] \cdot \text{pr}_2^*\xi^i \in
\CH^{d + i}(X \times P) = \text{Corr}^i(X, P)
$$
We may and do think of $c_i$ as a morphism $h(X)(i) \to h(P)$.

\begin{lemma}[Projective space bundle formula]
\label{lemma-projective-space-bundle-formula}
In the situation above, the map
$$
\sum\nolimits_{i = 0, \ldots, r - 1} c_i :
\bigoplus\nolimits_{i = 0, \ldots, r - 1} h(X)(i)
\longrightarrow
h(P)
$$
is an isomorphism in the category of motives.
\end{lemma}

\begin{proof}
By Lemma \ref{lemma-manin} it suffices to show that
our map defines an isomorphism on Chow groups of motives
after taking the product with any smooth projective scheme $Z$.
Observe that $P \times Z \to X \times Z$ is the projective
bundle associated to the pullback of $\mathcal{E}$ to $X \times Z$.
Hence the statement on Chow groups is true
by the projective space bundle formula given
in Chow Homology, Lemma \ref{chow-lemma-chow-ring-projective-bundle}.
\end{proof}

\noindent
In the situation above, for $j = 0, \ldots, r - 1$ consider
the correspondences
$$
c'_j = \text{pr}_1^*\xi^{r - 1 - j} \cdot [\Gamma_p^t] \in
\CH^{d + r - 1 - j}(P \times X) = \text{Corr}^{-j}(P, X)
$$
For $i, j \in \{0, \ldots, r - 1\}$ we have
$$
c'_j \circ c_i =
\text{pr}_{13, *}(
\text{pr}_{12}^*[\Gamma_p] \cdot \text{pr}_{23}^*[\Gamma_p^t]
\cdot \text{pr}_2^*\xi^{i + r - 1 - j})
$$
The cycles $\text{pr}_{12}^{-1}\Gamma_p$ and 
$\text{pr}_{23}^{-1}\Gamma_p^t$ intersect transversally and
with intersection equal to the image of
$(p, 1, p) : P \to X \times P \times X$.
Observe that the fibres of
$(p, p) : \text{pr}_{13} \circ (p, 1, p) P \to X \times X$
have dimension $r - 1$. We immediately conclude
$c'_j \circ c_i = 0$ for $i + r - 1 - j < r - 1 \Leftrightarrow i < j$.
On the other hand, by the projective space bundle formula
(Chow Homology, Lemma \ref{chow-lemma-chow-ring-projective-bundle})
the cycle $\xi^{r - 1}$ maps
to $[X]$ in $X$. Hence for $i = j$ the pushforward above
gives the class of the diagonal and hence
we see that
$$
c'_i \circ c_i = 1 \in \text{Corr}^0(X, X)
$$
for all $i \in \{0, \ldots, r - 1\}$. Thus we see that the matrix
of the composition
$$
\bigoplus h(X)(i)
\xrightarrow{\bigoplus c_i}
h(P)
\xrightarrow{\bigoplus c'_j}
\bigoplus h(X)(j)
$$
is invertible (upper triangular with $1$s on the diagonal).
We conclude from the projective space bundle formula
(Lemma \ref{lemma-projective-space-bundle-formula})
that also the composition the other way around is
invertible, but it seems a bit harder to prove this directly.

\begin{lemma}
\label{lemma-diagonal-projective-bundle}
Let $p : P \to X$ be as in Lemma \ref{lemma-projective-space-bundle-formula}.
The class $[\Delta_P]$ of the diagonal of $P$ in $\CH^*(P \times P)$
can be written
as
$$
[\Delta_P] =
\left(\sum\nolimits_{i = 0, \ldots, r - 1}
\text{pr}_1^*\alpha_i \cdot \text{pr}_2^*\xi^i\right)
(p \times p)^*[\Delta_X]
$$
for some $\alpha_i \in \CH^{r - 1 - i}(P)$ with $\alpha_{r - 1} = 1$.
\end{lemma}

\begin{proof}
Denote $q_i : P \times_X P \to P$ the projections. Observe that we have
the transversal intersection
$\Delta_{P/X} = (p \times p)^{-1}\Delta_X \cap (P \times_X P)$
in $P \times P$. Thus it suffices to show that the class of
$\Delta_{P/X} \subset P \times_X P$ is of the form
$$
\left(\sum\nolimits_{i = 0, \ldots, r - 1}
q_1^*\alpha_i \cdot q_2^*\xi^i\right)
$$
for some $\alpha_i \in \CH^{r - 1 - i}(P)$ with $\alpha_{r - 1} = 1$.
Denote $\mathcal{S}$ the kernel of the canonical surjective map
$p^*\mathcal{E} \to \mathcal{O}_P(1)$. Also set
$q = p \circ q_1 = p \circ q_2 : P \times_X P \to X$.
Next, consider the maps
$$
q_1^*\mathcal{S} \otimes q_2^*\mathcal{O}_P(-1) \to
q^*\mathcal{E} \otimes q^*\mathcal{E}^\vee \to
\mathcal{O}_{P \times_X P}
$$
The source is a module locally free of rank $r - 1$ and a local calculation
shows that this map vanishes exactly along $\Delta_{P/X}$.
By Chow Homology, Lemma \ref{chow-lemma-top-chern-class}
the class $[\Delta_{P/X}]$ is the top chern class of the dual
$$
q_1^*\mathcal{S}^\vee \otimes q_2^*\mathcal{O}_P(1)
$$
The lemma follows from Chow Homology, Lemma
\ref{chow-lemma-chern-classes-E-tensor-L}.
\end{proof}








\section{Classical Weil cohomology theories}
\label{section-axioms-classical}

\noindent
In this section we state precisely what a classical Weil cohomology
theory amounts to, exactly as in \cite[Section 1.2]{Kleiman-cycles}.

\medskip\noindent
We fix an algebrically closed field $k$ (the base field).
In this section {\it variety} will mean a variety over $k$, see
Varieties, Section \ref{varieties-section-varieties}.
We fix a field $F$ of characteristic $0$ (the coefficient field).
A Weil cohomology theory is given by data (D1), (D2), and (D3)
subject to axioms (A), (B), (C).

\medskip\noindent
The data is given by:
\begin{enumerate}
\item[(D1)] A contravariant functor $H^*$ from the category
of smooth projective varieties to the category of
graded commutative $F$-algebras.
\item[(D2)] For every smooth projective variety $X$
a group homomorphism $\gamma : \CH^i(X) \to H^{2i}(X)$.
\item[(D3)] For every smooth projective variety $X$ of dimension $d$
a map $\int_X : H^{2d}(X) \to F$.
\end{enumerate}
We make some remarks to explain what this means and to introduce
some terminology associated with this.

\medskip\noindent
Remarks on (D1). Given a smooth projective variety $X$
we say that $H^*(X)$ is the {\it cohomology} of $X$. Given a morphism
$f : X \to Y$ of smooth projective varieties we denote
$f^* : H^*(Y) \to H^*(X)$ the map $H^*(f)$ and we call it the
{\it pullback map}.

\medskip\noindent
Remarks on (D2). The map $\gamma$ is called the {\it cycle class map}.
We say that $\gamma(\alpha)$ is the {\it cohomology class} of $\alpha$.
If $Z \subset Y \subset X$ are closed subschemes with $Y$ and $X$
smooth projective varieties and $Z$ integral, then $[Z]$ could
mean the class of the cycle $[Z]$ in $\CH^*(Y)$ or in $\CH^*(X)$.
In this case the notation $\gamma([Z])$ is abiguous and the true meaning
has to be deduced from context.

\medskip\noindent
Remarks on (D3). The map $\int_X$ is sometimes called the
{\it trace map} and is sometimes denoted $\text{Tr}_X$.

\medskip\noindent
The first axiom is often called {\it Poincar\'e duality}
\begin{enumerate}
\item[(A)] Let $X$ be a smooth projective variety of dimension $d$. Then
\begin{enumerate}
\item $H^i(X) = 0$ unless $i \in [0, 2d]$,
\item $\dim_F H^i(X) < \infty$ for all $i$,
\item $H^i(X) \times H^{2d - i}(X) \rightarrow H^{2d}(X) \rightarrow F$
is a perfect pairing for all $i$ where the final
map is the trace map $\int_X$, and
\item $\int_X : H^{2d}(X) \to F$ is an isomorphism.
\end{enumerate}
\end{enumerate}
Let $f : X \to Y$ be a morphism of smooth projective varieties
with $\dim(X) = d$ and $\dim(Y) = e$. Using Poincar\'e duality
we can define a {\it pushforward}
$$
f_* : H^{2d - i}(X) \longrightarrow H^{2e - i}(Y)
$$
as the contragredient of the linear map $f^* : H^i(Y) \to H^i(X)$. In a
formula, for $a \in H^{2d - i}(X)$, the element $f_*a \in H^{2e - i}(Y)$
is characterized by
$$
\int_X f^*b \cup a = \int_Y b \cup f_*a
$$
for all $b \in H^i(Y)$.

\begin{lemma}
\label{lemma-pushforward-classical}
Assume given (D1) and (D3) satisfying (A). For $f : X \to Y$
a morphism of smooth projective varieties we have
$f_*(f^*b \cup a) = b \cup f_*a$. If $g : Y \to Z$ is a second morphism
of smooth projective varieties, then $g_* \circ f_* = (g \circ f)_*$.
\end{lemma}

\begin{proof}
The first equality holds because
$$
\int_Y c \cup b \cup f_*a =
\int_X f^*c \cup f^*b \cup a =
\int_Y c \cup f_*(f^*b \cup a).
$$
The second equality holds because
$$
\int_Z c \cup (g \circ f)_*a = \int_X (g \circ f)^*c \cup a =
\int_X f^* g^* c \cup a = \int_Y g^*c \cup f_*a = \int_Z c \cup g_*f_*a
$$
This ends the proof.
\end{proof}

\noindent
The second axiom says that $H^*$ respects the monoidal structure
given by products via the {\it K\"unneth formula}
\begin{enumerate}
\item[(B)] Let $X$ and $Y$ be smooth projective varieties. The map
$$
H^*(X) \otimes_F H^*(Y) \to H^*(X \times Y),\quad
a \otimes b \mapsto \text{pr}_1^*a \cup \text{pr}_2^*b
$$
is an isomorphism.
\end{enumerate}

\medskip\noindent
The third axiom concerns the cycle class maps
\begin{enumerate}
\item[(C)] The cycle class maps satisfy the following rules
\begin{enumerate}
\item for a morphism $f : X \to Y$ of smooth projective varieties
we have $\gamma(f^*\beta) = f^*\gamma(\beta)$ for $\beta \in \CH^*(Y)$,
\item for a morphism $f : X \to Y$ of smooth projective varieties we have
$\gamma(f_*\alpha) = f_*\gamma(\alpha)$ for $\alpha \in \CH^*(X)$,
\item for any smooth projective variety $X$ we have
$\gamma(\alpha \cdot \beta) = \gamma(\alpha) \cup \gamma(\beta)$
for $\alpha, \beta \in \CH^*(X)$, and
\item $\int_{\Spec(k)} \gamma([\Spec(k)]) = 1$.
\end{enumerate}
\end{enumerate}

\begin{remark}
\label{remark-replace-cup-product-classical}
Let $X$ be a smooth projective variety. We obtain a map
$$
H^*(X) \otimes_F H^*(X) \longrightarrow H^*(X \times X)
\xrightarrow{\Delta^*} H^*(X)
$$
where $\Delta^*$ is pullback along the diagonal morphism
$\Delta : X \to X \times X$. The composition is the cup product.
(Hints: pullback is an algebra homomorphism and
$\text{pr}_i \circ \Delta = \text{id}$.)
On the other hand, the intersection product
$\alpha \cdot \beta$ of cycles $\alpha, \beta$ on $X$ is the
pullback of the exterior product $\alpha \times \beta$ on $X \times X$.
It follows that in order to prove axiom (C)(c) it suffices to show
that $\gamma$ is compatible with exterior product. This is how
axiom (C)(c) is formulated in \cite{Kleiman-cycles}.
\end{remark}

\begin{definition}
\label{definition-weil-cohomology-theory-classical}
Let $k$ be an algebraically closed field.
Let $F$ be a field of characteristic $0$.
A {\it classical Weil cohomology theory} over $k$ with coefficients in $F$
is given by data (D1), (D2), and (D3) satisfying
Poincar\'e duality, the K\"unneth formula, and compatibility
with cycle classes, more precisely, satisfying (A), (B), and (C).
\end{definition}

\noindent
We do a tiny bit of work.

\begin{lemma}
\label{lemma-degrees-cycles-classical}
Let $H^*$ be a classical Weil cohomology theory
(Definition \ref{definition-weil-cohomology-theory-classical}).
Let $X$ be a smooth projective variety of dimension $d$. The diagram
$$
\xymatrix{
\CH^d(X) \ar[r]_-\gamma \ar@{=}[d] &
H^{2d}(X) \ar[d]^{\int_X} \\
\CH_0(X) \ar[r]^\deg & F
}
$$
commutes where $\deg : \CH_0(X) \to \mathbf{Z}$ is the degree of
zero cycles discussed in Chow Homology, Section
\ref{chow-section-degree-zero-cycles}.
\end{lemma}

\begin{proof}
The result holds for $\Spec(k)$ by axiom (C)(d). Let $x : \Spec(k) \to X$
be a closed point of $X$. Then we have $\gamma([x]) = x_*\gamma([\Spec(k)])$
in $H^{2d}(X)$ by axiom (C)(b). Hence $\int_X \gamma([x]) = 1$ by the
definition of $x_*$.
\end{proof}

\begin{lemma}
\label{lemma-trace-product-classical}
Let $H^*$ be a classical Weil cohomology theory
(Definition \ref{definition-weil-cohomology-theory-classical}).
Let $X$ and $Y$ be smooth projective varieties.
Then $\int_{X \times Y} = \int_X \otimes \int_Y$.
\end{lemma}

\begin{proof}
Say $\dim(X) = d$ and $\dim(Y) = e$. By axiom (B) we have
$H^{2d + 2e}(X \times Y) = H^{2d}(X) \otimes H^{2e}(Y)$
and by axiom (A)(d) this is $1$-dimensional.
By Lemma \ref{lemma-degrees-cycles-classical}
this $1$-dimensional vector space generated by the
class $\gamma([x \times y])$ of a closed point $(x, y)$ and
$\int_{X \times Y} \gamma([x \times y]) = 1$.
Since $\gamma([x \times y]) = \gamma([x]) \otimes \gamma([y])$
by axioms (C)(a) and (C)(c) and since $\int_X \gamma([x]) = 1$ and
$\int_Y \gamma([y]) = 1$ we conclude.
\end{proof}

\begin{lemma}
\label{lemma-pr2star-classical}
Let $H^*$ be a classical Weil cohomology theory
(Definition \ref{definition-weil-cohomology-theory-classical}).
Let $X$ and $Y$ be smooth projective varieties.
Then $\text{pr}_{2, *} : H^*(X \times Y) \to H^*(Y)$
sends $a \otimes b$ to $(\int_X a) b$.
\end{lemma}

\begin{proof}
This is equivalent to the result of Lemma \ref{lemma-trace-product-classical}.
\end{proof}

\begin{lemma}
\label{lemma-class-diagonal-classical}
Let $H^*$ be a classical Weil cohomology theory
(Definition \ref{definition-weil-cohomology-theory-classical}).
Let $X$ be a smooth projective variety of dimension $d$.
Choose a basis $e_{i, j}, j = 1, \ldots, \beta_i$ of $H^i(X)$ over $F$.
Using K\"unneth write
$$
\gamma([\Delta]) =
\sum\nolimits_{i = 0, \ldots, 2d}
\sum\nolimits_j e_{i, j} \otimes e'_{2d - i , j}
\quad\text{in}\quad
\bigoplus\nolimits_i H^i(X) \otimes_F H^{2d - i}(X)
$$
with $e'_{2d - i, j} \in H^{2d - i}(X)$.
Then $\int_X e_{i, j} \cup e'_{2d - i, j'} = (-1)^i\delta_{jj'}$.
\end{lemma}

\begin{proof}
Recall that $\Delta^* : H^*(X \times X) \to H^*(X)$ is equal to the
cup product map $H^*(X) \otimes_F H^*(X) \to H^*(X)$, see
Remark \ref{remark-replace-cup-product-classical}. On the other hand we have
$\gamma([\Delta]) = \Delta_*\gamma([X]) = \Delta_*1$ by
axiom (C)(c) and the fact that $\gamma([X]) = 1$. Namely,
$[X] \cdot [X] = [X]$ hence by axiom (C)(c) $\gamma([X])$ is $0$ or $1$ in
the $1$-dimensional $F$-algebra $H^0(X)$. But $\gamma([X])$
cannot be zero as $[X] \cdot [x] = [x]$ for a closed point $x$ of $X$
and we have the nonvanishing of $\gamma([x])$ by
Lemma \ref{lemma-degrees-cycles-classical}.
Hence
$$
\int_{X \times X} \gamma([\Delta]) \cup a \otimes b =
\int_{X \times X} \Delta_*1 \cup a \otimes b =
\int_X a \cup b
$$
by the definition of $\Delta_*$. On the other hand, we have
$$
\int_{X \times X} (\sum e_{i, j} \otimes e'_{2d -i , j}) \cup a \otimes b =
(\int_X a \cup e_{i, j})(\int_X e'_{2d - i, j} \cup b)
$$
by Lemma \ref{lemma-trace-product-classical}; note that we made
two switches of order so that the sign is $1$.
Thus if we choose $a$ such that $\int_X a \cup e_{i, j} = 1$
and all other pairings equal to zero, then we conclude that
$\int_X e'_{2d - i, j} \cup b = \int_X a \cup b$ for all $b$, i.e.,
$e'_{2d - i, j} = a$. This proves the lemma.
\end{proof}

\begin{lemma}
\label{lemma-square-diagonal-classical}
Let $H^*$ be a classical Weil cohomology theory
(Definition \ref{definition-weil-cohomology-theory-classical}).
Let $X$ be a smooth projective variety. We have
$$
\sum\nolimits_{i = 0, \ldots, 2\dim(X)} (-1)^i\dim_F H^i(X) =
\deg(\Delta \cdot \Delta) = \deg(c_d(\mathcal{T}_X))
$$
\end{lemma}

\begin{proof}
The equality on the right holds by
Chow Homology, Lemma \ref{chow-lemma-gysin-fundamental}.
By Lemma \ref{lemma-degrees-cycles-classical} we have
\begin{align*}
\deg(\Delta \cdot \Delta)
& =
\int_{X \times X} \gamma([\Delta]) \cup \gamma([\Delta]) \\
& =
\int_{X \times X} \Delta_*1 \cup \gamma([\Delta]) \\
& =
\int_{X \times X} \Delta_*(\Delta^*\gamma([\Delta])) \\
& =
\int_X \Delta^*\gamma([\Delta])
\end{align*}
Write $\gamma([\Delta]) = \sum  e_{i, j} \otimes e'_{2d - i , j}$
as in Lemma \ref{lemma-class-diagonal-classical}.
Recalling that $\Delta^*$ is given by cup product we obtain
$$
\int_X \sum\nolimits_{i, j} e_{i, j} \cup e'_{2d - i, j} =
\sum\nolimits_{i, j} \int_X e_{i, j} \cup e'_{2d - i, j} =
\sum\nolimits_{i, j} (-1)^i = \sum (-1)^i\beta_i
$$
as desired.
\end{proof}




\noindent
We will now tie classical Weil cohomology theories in with motives as follows.

\begin{lemma}
\label{lemma-from-functor-to-weil-classical}
Let $k$ be an algebraically closed field. Let $F$ be a field of
characteristic $0$. Consider a $\mathbf{Q}$-linear functor
$$
G : M_k \longrightarrow \text{graded }F\text{-vector spaces}
$$
of symmetric monoidal categories such that $G(\mathbf{1}(1))$
is nonzero only in degree $-2$. Then we obtain data (D1), (D2), (D3)
satisfying all of (A), (B), (C) except for possibly (A)(a) and (A)(d).
\end{lemma}

\begin{proof}
We obtain a contravariant functor from the category of smooth
projective varieties to the category of graded $F$-vector spaces
by setting $H^*(X) = G(h(X))$. By assumption we have a canonical
isomorphism
$$
H^*(X \times Y) = G(h(X \times Y)) = G(h(X) \otimes h(Y)) =
G(h(X)) \otimes G(h(Y)) = H^*(X) \otimes H^*(Y)
$$
compatible with pullbacks. Using pullback along the diagonal
$\Delta : X \to X \times X$ we obtain a canonical map
$$
H^*(X) \otimes H^*(X) = H^*(X \times X) \to H^*(X)
$$
of graded vector spaces compatible with pullbacks.
This defines a functorial graded $F$-algebra structure on
$H^*(X)$. Since $\Delta$ commutes with the commutativity
constraint $h(X) \otimes h(X) \to h(X) \otimes h(X)$ (switching the factors)
and since $G$ is a functor of symmetric monoidal categories (so compatible with
commutativity constraints), and by our convention in
Example \ref{example-graded-vector-spaces}
we conclude that $H^*(X)$ is a graded
commutative algebra! Hence we get our datum (D1).

\medskip\noindent
Since $\mathbf{1}(1)$ is invertible in the category of motives
we see that $G(\mathbf{1}(1))$ is invertible in the category of
graded $F$-vector spaces. Thus$\sum_i \dim_F G^i(\mathbf{1}(1)) = 1$.
By assumption we only get something nonzero in degree $2$ and we may
choose an isomorphism $F[2] \to G(\mathbf{1}(1))$ of graded $F$-vector spaces.
Here and below $F[n]$ means the graded $F$-vector space which has
$F$ in degree $-n$ and zero elsewhere. Using compatibility with
tensor products, we find for all $n \in \mathbf{Z}$ an isomorphism
$F[2n] \to G(\mathbf{1}(n))$ compatible with tensor products.

\medskip\noindent
Let $X$ be a smooth projective variety. By
Lemma \ref{lemma-composition-correspondences} we have
$$
\CH^r(X) \otimes \mathbf{Q} = \text{Corr}^r(\Spec(k), X) =
\Hom(\mathbf{1}(-r), h(X))
$$
Applying the functor $G$ we obtain
$$
\gamma :
\CH^r(X) \otimes \mathbf{Q} \longrightarrow
\Hom(G(\mathbf{1}(-r)), H^*(X)) = H^{2r}(X)
$$
This is the datum (D2).

\medskip\noindent
Let $X$ be a smooth projective variety of dimension $d$. By
Lemma \ref{lemma-composition-correspondences} we have
$$
\Mor(h(X)(d), \mathbf{1}) = \Mor((X, 1, d), (\Spec(k), 1, 0)) =
\text{Corr}^{-d}(X, \Spec(k)) = \CH_d(X)
$$
Thus the class of the cycle $[X]$ in $\CH_d(X)$ defines a morphism
$h(X)(d) \to \mathbf{1}$. Applying $G$ we obtain
$$
H^*(X) \otimes F[-2d] = G(h(X)(d)) \longrightarrow G(\mathbf{1}) = F
$$
This map is zero except in degree $0$ where we obtain
$\int_X : H^{2d}(X) \to F$. This is the datum (D3).

\medskip\noindent
Let $X$ be equidimensional of dimension $d$. By Lemma \ref{lemma-dual}
we know that $h(X)(d)$ is a left dual to $h(X)$. Hence
$G(h(X)(d)) = H^*(X) \otimes F[-2d]$ is a left dual to
$H^*(X)$ in the category of graded $F$-vector spaces.
By Lemma \ref{lemma-left-dual-graded-vector-spaces}
we find that $\sum_i \dim_F H^i(X) < \infty$ and that
$\epsilon : h(X)(d) \otimes h(X) \to \mathbf{1}$ produces
nondegenerate pairings $H^{2d - i}(X) \otimes_F H^i(X) \to F$.
In the proof of Lemma \ref{lemma-dual} we have seen that
$\epsilon$ is given by $[\Delta]$ via the identifications
$$
\Hom(h(X)(d) \otimes h(X), \mathbf{1}) =
\text{Corr}^{-d}(X \times X, \Spec(k)) =
\CH_d(X \times X)
$$
Thus $\epsilon$ is the composition of $[X] : h(X)(d) \to \mathbf{1}$
and $h(\Delta)(d) : h(X)(d) \otimes h(X) \to h(X)(d)$. It follows
that the pairings above are given by cup product followed by
$\int_X$. This proves axiom (A) parts (b) and (c).

\medskip\noindent
Axiom (B) follows from the assumption that $G$ is compatible
with tensor structures and our construction of the cup product above.

\medskip\noindent
Axiom (C). Our construction of $\gamma$ takes a cycle $\alpha$ on $X$,
interprets it a correspondence $a$ from $\Spec(k)$ to $X$ of some degree,
and then applies $G$. If $f : Y \to X$ is a morphism of smooth projective
varieties, then $f^*\alpha$ is the pushforward (!) of $\alpha$
by the correspondence $[\Gamma_f]$ from $X$ to $Y$, see
Lemma \ref{lemma-functor-and-cycles}. Hence
$f^*\alpha$ viewed as a correspondence from $\Spec(k)$ to $Y$
is equal to $a \circ [\Gamma_f]$, see
Lemma \ref{lemma-composition-correspondences}.
Since $G$ is a functor, we conclude
$\gamma$ is compatible with pullbacks, i.e., axiom (C)(a) holds.

\medskip\noindent
Let $f : Y \to X$ be a morphism of smooth projective varieties and
let $\beta \in \CH^r(Y)$ be a cycle on $Y$. We have to show that
$$
\int_Y \gamma(\beta) \cup f^*c = \int_X \gamma(f_*\beta) \cup c
$$
for all $c \in H^*(X)$. Let $a, a^t, \eta_X, \eta_Y, [X], [Y]$
be as in Lemma \ref{lemma-prep-dual}.
Let $b$ be $\beta$ viewed as a correspondence from $\Spec(k)$ to $Y$
of degree $r$. Then $f_*\beta$ viewed as a correspondence from
$\Spec(k)$ to $X$ is equal to $a^t \circ b$, see
Lemmas \ref{lemma-functor-and-cycles} and
\ref{lemma-composition-correspondences}.
The displayed equality above holds if we can show that
$$
h(X) = \mathbf{1} \otimes h(X)
\xrightarrow{b \otimes 1}
h(Y)(r) \otimes h(X)
\xrightarrow{1 \otimes a}
h(Y)(r) \otimes h(Y)
\xrightarrow{\eta_Y}
h(Y)(r)
\xrightarrow{[Y]}
\mathbf{1}(r - e)
$$
is equal to
$$
h(X) = \mathbf{1} \otimes h(X)
\xrightarrow{a^t \circ b \otimes 1}
h(X)(r + d - e) \otimes h(X)
\xrightarrow{\eta_X}
h(X)(r + d - e)
\xrightarrow{[X]}
\mathbf{1}(r - e)
$$
This follows immediately from Lemma \ref{lemma-prep-dual}.
Thus we have axiom (C)(b).

\medskip\noindent
To prove axiom (C)(c) we use the discussion in
Remark \ref{remark-replace-cup-product-classical}.
Hence it suffices to prove that $\gamma$ is compatible with
exterior products. Let $X$, $Y$ be smooth projective varieties and
let $\alpha$, $\beta$ be cycles on them. Denote
$a$, $b$ the corresponding correspondences from $\Spec(k)$ to
$X$, $Y$. Then $\alpha \times \beta$ corresponds to the
correspondence $a \otimes b$ from $\Spec(k)$ to $X \otimes Y = X \times Y$.
Hence the requirement follows from the fact that $G$ is
compatible with the tensor structures on both sides.

\medskip\noindent
Axiom (C)(d) follows because the cycle $[\Spec(k)]$
corresponds to the identity morphism on $h(\Spec(k))$.
This finishes the proof of the lemma.
\end{proof}

\begin{lemma}
\label{lemma-from-weil-to-functor-classical}
Let $k$ be an algebraically closed field. Let $F$ be a field of
characteristic $0$. Let $H^*$ be a classical Weil cohomology theory.
Then we can construct a $\mathbf{Q}$-linear functor
$$
G : M_k \longrightarrow \text{graded }F\text{-vector spaces}
$$
of symmetric monoidal categories such that $H^*(X) = G(h(X))$.
\end{lemma}

\begin{proof}
By Lemma \ref{lemma-characterize-motives} it suffices to construct a functor
$G$ on the category of smooth projective schemes over $k$
with morphisms given by correspondences of degree $0$ such that
the image of $G(c_2)$ on $G(\mathbf{P}^1)$ is an invertible graded
$F$-vector space.
Since every smooth projective scheme is canonically a disjoint
union of smooth projective varieties, it suffices to construct
$G$ on the category whose objects are smooth projective varieties
and whose morphisms are correspondences of degree $0$. (Some details
omitted.)

\medskip\noindent
Given a smooth projective variety $X$ we set $G(X) = H^*(X)$.

\medskip\noindent
Given a correspondence $c \in \text{Corr}^0(X, Y)$ between smooth
projective varieties we consider the map
$G(c) : G(X) = H^*(X) \to G(Y) = H^*(Y)$ given by the rule
$$
a \longmapsto
G(c)(a) = \text{pr}_{2, *}(\gamma(c) \cup \text{pr}_1^*a)
$$
It is clear that $G(c)$ is additive in $c$ and hence $\mathbf{Q}$-linear.
Compatibility of $\gamma$ with pullbacks, pushforwards, and
intersection products given by axioms (C)(a), (C)(b), and (C)(c)
shows that we have
$G(c' \circ c) = G(c') \circ G(c)$ if $c' \in \text{Corr}^0(Y, Z)$.
Namely, for $a \in H^*(X)$ we have
\begin{align*}
(G(c') \circ G(c))(a)
& =
\text{pr}^{23}_{3, *}(\gamma(c') \cup
\text{pr}^{23, *}_2(\text{pr}^{12}_{2, *}(\gamma(c) \cup
\text{pr}^{12, *}_1a))) \\
& =
\text{pr}^{23}_{3, *}(\gamma(c') \cup
\text{pr}^{123}_{23, *}(\text{pr}^{123, *}_{12}(\gamma(c) \cup
\text{pr}^{12, *}_1 a))) \\
& =
\text{pr}^{23}_{3, *}
\text{pr}^{123}_{23, *}(
\text{pr}^{123, *}_{23}\gamma(c') \cup
\text{pr}^{123, *}_{12}\gamma(c) \cup
\text{pr}^{123, *}_1 a) \\
& =
\text{pr}^{23}_{3, *}
\text{pr}^{123}_{23, *}(
\gamma(\text{pr}^{123, *}_{23}c') \cup
\gamma(\text{pr}^{123, *}_{12}c) \cup
\text{pr}^{123, *}_1 a) \\
& =
\text{pr}^{13}_{3, *}
\text{pr}^{123}_{13, *}(
\gamma(\text{pr}^{123, *}_{23}c' \cdot \text{pr}^{123, *}_{12}c) \cup
\text{pr}^{123, *}_1 a) \\
& =
\text{pr}^{13}_{3, *}(
\gamma(\text{pr}^{123}_{13, *}(
\text{pr}^{123, *}_{23}c' \cdot \text{pr}^{123, *}_{12}c)) \cup
\text{pr}^{13, *}_1 a) \\
& =
G(c' \circ c)(a)
\end{align*}
with obvious notation. To finish the proof that $G$ is a functor,
we have to show identities are preserved. In other words, if
$c = [\Delta] \in \text{Corr}^0(X, X)$ is the unit
(Lemma \ref{lemma-identity-correspondence}), then we have to show that
$G(c) = \text{id}$. This follows from the determination
of $\gamma([\Delta])$ in Lemma \ref{lemma-class-diagonal-classical}
and Lemma \ref{lemma-pr2star-classical}.
This finishes the construction of $G$ as a functor on
smooth projective varieties and correspondences of degree $0$.

\medskip\noindent
It follows from axioms (A)(a) and (A)(d) that
$G(\Spec(k)) = H^*(\Spec(k))$ is canonically isomorphic to $F$
as an $F$-algebra.
The K\"unneth axiom (B) shows our functor is compatible with tensor products.
Thus our functor is a functor of symmetric monoidal categories.

\medskip\noindent
We still have to check that the image of $G(c_2)$ on $G(\mathbf{P}^1)$
is an invertible graded $F$-vector space (in particular we don't know yet
that $G$ extends to $M_k$).
By axiom (A)(d) the map $\int_{\mathbf{P}^1} : H^2(\mathbf{P}^1) \to F$
is an isomorphism. By axiom (A)(c) we see that $\dim_F H^0(\mathbf{P}^1) = 1$.
By Lemma \ref{lemma-square-diagonal-classical} and axiom (A)(a)
we obtain $2 - \dim_F H^1(\mathbf{P}^1) = c_1(T_{\mathbf{P}^1}) = 2$.
Hence $H^1(\mathbf{P}^1) = 0$. Thus
$$
G(\mathbf{P}^1) = H^0(\mathbf{P}^1) \oplus H^2(\mathbf{P}^1)
$$
Recall that $1 = c_0 + c_2$ is a decomposition of the identity
into a sum of orthogonal idempotents in
$\text{Corr}^0(\mathbf{P}^1, \mathbf{P}^1)$, see
Example \ref{example-decompose-P1}. We have $c_0 = a \circ b$ where
$a \in \text{Corr}^0(\Spec(k), \mathbf{P}^1)$ and
$b \in \text{Corr}^0(\mathbf{P}^1, \Spec(k))$ and where
$b \circ a = 1$ in $\text{Corr}^0(\Spec(k), \Spec(k))$, see proof of
Lemma \ref{lemma-inverse-h2}. Since $F = G(\Spec(k))$, it follows from
functoriality that $G(c_0)$ is the projector onto the summand
$H^0(\mathbf{P}^1) \subset G(\mathbf{P}^1)$. Hence
$G(c_2)$ must necessarily be the projection onto $H^2(\mathbf{P}^1)$
and the proof is complete.
\end{proof}

\begin{proposition}
\label{proposition-weil-cohomology-theory-classical}
Let $k$ be an algebraically closed field. Let $F$ be a field of
characteristic $0$. A classical Weil cohomology theory is the same thing
as a $\mathbf{Q}$-linear functor
$$
G : M_k \longrightarrow \text{graded }F\text{-vector spaces}
$$
of symmetric monoidal categories together with an isomorphism
$F[2] \to G(\mathbf{1}(1))$ of graded $F$-vector spaces such that
in addition
\begin{enumerate}
\item $G(h(X))$ lives in nonnegative degrees, and
\item $\dim_F G^0(h(X)) = 1$
\end{enumerate}
for any smooth projective variety $X$.
\end{proposition}

\begin{proof}
Given $G$ and $F[2] \to G(\mathbf{1}(1))$ by setting $H^*(X) = G(h(X))$
we obtain data (D1), (D2), (D3) satisfying all of (A), (B), (C)
except for possibly (A)(a) and (A)(d), see
Lemma \ref{lemma-from-functor-to-weil-classical} and its proof.
Observe that assumptions (1) and (2) imply axioms (A)(a) and (A)(d)
in the presence of the known axioms (A)(b) and (A)(c).

\medskip\noindent
Conversely, given $H^*$ we get a functor $G$ by the construction of
Lemma \ref{lemma-from-weil-to-functor-classical}. The isomorphism
$F[2] \to G(\mathbf{1}(1))$ comes from the isomorpihsm
$\int_{\mathbf{P}^1} : H^2(\mathbf{P}^1) \to F$ of axiom (A)(d).
Since $G(h(X)) = H^*(X)$ assumptions (1) and (2) follow from axiom (A).
\end{proof}











\section{Cycles over non-closed fields}
\label{section-cycles-nonclosed}

\noindent
Some lemmas which will help us in our study of motives
over base fields which are not algebraically closed.

\begin{lemma}
\label{lemma-generated-by-separable}
Let $k$ be a field. Let $X$ be a smooth projective scheme over $k$.
Then $\CH_0(X)$ is generated by classes of closed points whose residue
fields are separable over $k$.
\end{lemma}

\begin{proof}
The lemma is immediate if $k$ has characteristic $0$ or is perfect.
Thus we may assume $k$ is an infinite field of characteristic $p > 0$.

\medskip\noindent
We may assume $X$ is irreducible of dimension $d$.
Observe that $k' = H^0(X, \mathcal{O}_X)$ is a finite separable
extension of $k$ and that $X$ is geometrically integral over $k'$.
We may and do replace $k$ by $k'$ and assume that $X$ is
geometrically integral.

\medskip\noindent
Let $x \in X$ be a closed point. Choose a sufficiently ample invertible
$\mathcal{O}_X$-module $\mathcal{L}$. Choose a trivialization
$\mathcal{L}_x = \mathcal{O}_{X, x}$. Set
$$
V = \{s \in H^0(X, \mathcal{L}) \mid s(x) = 0 \}
$$
The map $V \to \mathfrak m_x/\mathfrak m_x^2$ is surjective because
$\mathcal{L}$ is sufficiently ample. Consider the set
$$
V^d \supset U =
\{
(s_1, \ldots, s_d) \in V^d \mid s_1, \ldots, s_d
\text{ generate }
\mathfrak m_x/\mathfrak m_x^2
\text{ over }\kappa(x)
\}
$$
For $(s_1, \ldots, s_d) \in U$ set $H_i = Z(s_i)$. Since
$s_1, \ldots, s_d$ generate $\mathfrak m_x$ we see that
$$
H_1 \cap \ldots \cap H_d = x \amalg Z
$$
scheme theoretically.
We claim that for sufficiently general $(s_1, \ldots, s_d) \in U$
the scheme $Z$ is finite \'etale over $\Spec(k)$. This will finish
the proof as it shows that $[x] \sim_{rat} - [Z] + [Z']$
where $Z' = H'_1 \cap \ldots \cap H'_d$ is a general complete
intersection of vanishing loci of sufficiently general sections
of $\mathcal{L}$.

\medskip\noindent
To see that the claim is true, we may assume $\mathcal{L}$ is such that
there exists one choice of $s_1, \ldots, s_d$ such that
$H_1 \cap \ldots \cap H_d = x \amalg x' \amalg Z''$ (scheme theoretically)
for some closed point $x' \in X$ whose residue field is separable over $k$.
Namely, we can find a point $x'$ whose residue field is separable over
$k$ by
Varieties, Lemma \ref{varieties-lemma-smooth-separable-closed-points-dense}.
Then we choose $s_1, \ldots, s_d$ as above, but also vanishing at
$x'$ and generating $\mathfrak m_{x'}$.
The existence of this shows that the projection $I \to U$ from
incidence correspondence
$$
U \times X \setminus \{x\} \supset
I = \{((s_1, \ldots, s_d), x') \mid x' \in H_1 \cap \ldots \cap H_d\}
$$
is \'etale at some point. On the other hand, since $X \setminus \{x\}$
is geometrically integral, and since the geometric fibres of
the flat finite type morphism
$I \to X$ are open subschemes of linear spaces, we conclude that 
$I$ is geometrically integral over $k$.
Then we conclude that $I \to U$ is \'etale over (!)
a dense open subscheme $U' \subset U$. Since $k$-rational
points are dense in $U'$ we conclude.
Details omitted.
\end{proof}

\begin{lemma}
\label{lemma-divide-difference-points}
Let $k$ be a field. Let $X$ be a geometrically irreducible
smooth projective scheme over $k$. Let $x, x' \in X$ be $k$-rational points.
Then there exists a finite separable extension $k'/k$ such that
the pullback of $[x] - [x']$ to $X_{k'}$
is divisible by an integer $n > 1$ in $\CH_0(X_{k'})$.
\end{lemma}

\begin{proof}
Using Bertini we can choose a smooth curve $C \subset X$
which is geometrically irreducible over $k$
such that $x, x' \in C$. (One may have to extend $k$
a little bit here.) This reduces us to the case where
$X$ is a curve. Next, we see that $\mathcal{O}_X(x - x')$
defines a $k$-rational point of $\Pic_{X/k}$.
Choose a prime $\ell$ invertible in $k$. Since
$[\ell] : \Pic_{X/k} \to \Pic_{X/k}$ is finite \'etale,
we see that after replacing $k$ by a finite separable extension
the invertible module $\mathcal{O}_X(x - x')$ has an
$\ell$th root in $\Pic(X)$. This is what we had to show.
\end{proof}

\begin{lemma}[Voevodsky]
\label{lemma-smash-nilpotence}
\begin{reference}
\cite{nilpotence}
\end{reference}
Let $k$ be a field. Let $X$ be a geometrically irreducible
smooth projective scheme over $k$. Let $x, x' \in X$ be $k$-rational points.
Then there exists an $n \geq 1$ such that
$$
([x] - [x']) \times \ldots \times ([x] - [x']) \in
\CH_0(X^n)
$$
is torsion.
\end{lemma}

\begin{proof}
If we can show this after base change to the algebraic closure of $k$,
then the result follows over $k$. (The kernel of pullback on zero cycles
is torsion.) Hence we may and do assume $k$ is algebraically closed.
Using Bertini we can choose a smooth curve $C \subset X$ passing through
$x$ and $x'$. Hence we may assume $X$ is a curve.

\medskip\noindent
Write $S^n(X) = \underline{\Hilbfunctor}^n_{X/k}$ with notation as in
Picard Schemes of Curves, Sections \ref{pic-section-hilbert-scheme-points}
and \ref{pic-section-divisors}. There is a canonical morphism
$$
\pi : X^n \longrightarrow S^n(X)
$$
which sends the $k$-rational point $(x_1, \ldots, x_n)$ to the $k$-rational
point corresponding to the divisor $[x_1] + \ldots + [x_n]$ on $X$.
There is a faithful action of the symmetric group $S_n$ on $X^n$.
The morphism $\pi$ is $S_n$-invariant and the fibres of $\pi$ are
$S_n$-orbits (set theoretically). Finally, $\pi$ is finite flat of
degree $n!$, see Picard Schemes of Curves, Lemma
\ref{pic-lemma-universal-object}.

\medskip\noindent
Let $\alpha_n$ be the zero cycle on $X^n$ given by the formula in the
statement of the lemma. Let $\mathcal{L} = \mathcal{O}_X(x - x')$. Then
$c_1(\mathcal{L}) \cap [X] = [x] - [x']$. Thus
$$
\alpha_n = c_1(\mathcal{L}_1) \cap \ldots \cap c_1(\mathcal{L}_n) \cap [X^n]
$$
where $\mathcal{L}_i = \text{pr}_i^*\mathcal{L}$ and $\text{pr}_i : X^n \to X$
is the $i$th projection. By either
Divisors, Lemma \ref{divisors-lemma-finite-locally-free-has-norm} or
Divisors, Lemma \ref{divisors-lemma-norm-in-normal-case}
there is a norm for $\pi$. Set
$$
\mathcal{N} = \text{Norm}_\pi(\mathcal{L}_1)
$$
See Divisors, Lemma \ref{divisors-lemma-norm-invertible}. We have
$$
\pi^*\mathcal{N} =
(\otimes_{i = 1, \ldots, n} \mathcal{L}_i)^{\otimes (n - 1)!}
$$
in $\Pic(X^n)$ by a calculation. Deails omitted; hint: this follows from
the fact that
$\text{Norm}_\pi : \pi_*\mathcal{O}_{X^n} \to \mathcal{O}_{S^n(X)}$
composed with the natural map $\pi_*\mathcal{O}_{S^n(X)} \to \mathcal{O}_{X^n}$
is equal to the product over all $\sigma \in S_n$ of the action of
$\sigma$ on $\pi_*\mathcal{O}_{X^n}$. Consider
$$
\beta_n = c_1(\mathcal{N})^n \cap [S^n(X)]
$$
in $\CH_0(S^n(X))$. Observe that
$c_1(\mathcal{L}_i) \cap c_1(\mathcal{L}_i) = 0$
because $\mathcal{L}_i$ is pulled back from a curve. Thus we see that
\begin{align*}
\pi^*\beta_n
& =
((n - 1)!)^n
(\sum\nolimits_{i = 1, \ldots, n} c_1(\mathcal{L}_i))^n \cap [X^n] \\
& =
((n - 1)!)^n n^n 
c_1(\mathcal{L}_1) \cap \ldots \cap c_1(\mathcal{L}_n) \cap [X^n] \\
& =
(n!)^n \alpha_n
\end{align*}
Thus it suffices to show that $\beta_n$ is torsion.

\medskip\noindent
There is a canonical morphism
$$
f : S^n(X) \longrightarrow \underline{\Picardfunctor}^n_{X/k}
$$
See Picard Groups of Curves, Lemma \ref{pic-lemma-picard-pieces}.
For $n \geq 2g - 1$ this morphism is a projective space bundle.
The invertible sheaf $\mathcal{N}$ is trivial on the
fibres of $f$ (hint: because $\mathcal{L}$ is algebraically
equivalent to zero, so is $\mathcal{N}$ and the fibres are
projective spaces). Thus by the projective space bundle formula
(Chow Homology, Lemma \ref{chow-lemma-chow-ring-projective-bundle})
we see that $\mathcal{N} = f^*\mathcal{M}$ for some invertible
module $\mathcal{M}$ on $\underline{\Picardfunctor}^n_{X/k}$.
Of course, then we see that
$$
c_1(\mathcal{N})^n = f^*(c_1(\mathcal{M})^n)
$$
is zero because $n > g = \dim(\underline{\Picardfunctor}^n_{X/k})$.
This finishes the proof.
\end{proof}















\section{Weil cohomology theories, I}
\label{section-axioms}

\noindent
Let $k$ be an arbitrary field. Motivated by
Proposition \ref{proposition-weil-cohomology-theory-classical}
in this section we will list the necessary data and axioms which will
correspond to functors $G$ of symmetric monoidal categories from
the category $M_k$ of motives over $k$ to the category of
graded $F$-vector spaces such that $G(\mathbf{1}(1))$ sits in
degree $-2$. In Section \ref{section-old} we will finally use this material
to define a Weil cohomology theory.

\medskip\noindent
We fix a field $k$ (the base field).
We fix a field $F$ of characteristic $0$ (the coefficient field).
The data is given by:
\begin{enumerate}
\item[(D1)] A contravariant functor $H^*$ from the category
of smooth projective schemes over $k$ to the category of
graded commutative $F$-algebras.
\item[(D2)] A $1$-dimensional $F$-vector space $F(1)$.
\item[(D3)] For every smooth projective scheme $X$ over $k$
a group homomorphism $\gamma : \CH^i(X) \to H^{2i}(X)(i)$.
\item[(D4)] For every nonempty smooth projective scheme $X$ over $k$
which is equidimensional of dimension $d$ a map
$\int_X : H^{2d}(X)(d) \to F$.
\end{enumerate}
We make some remarks to explain what this means and to introduce
some terminology associated with this.

\medskip\noindent
Remarks on (D1).
Given a smooth projective scheme $X$ over $k$ we say that $H^*(X)$
is the {\it cohomology} of $X$. Given a morphism $f : X \to Y$
of smooth projective schemes over $k$ we denote $f^* : H^*(Y) \to H^*(X)$
the map $H^*(f)$ and we call it the {\it pullback map}.

\medskip\noindent
Remarks on (D2).
The vector space $F(1)$ gives rise to {\it Tate twists} on the category of
$F$-vector spaces. Namely, for $n \in \mathbf{Z}$ we set
$F(n) = F(1)^{\otimes n}$ if $n \geq 0$, we set $F(-1) = \Hom_F(F(1), F)$,
and we set $F(n) = F(-1)^{\otimes - n}$ if $n < 0$. Please compare
with More on Algebra, Section \ref{more-algebra-section-picard}.
For an $F$-vector space $V$ we define then $n$th Tate twist
$$
V(n) = V \otimes_F F(n)
$$
We will use obvious notation, e.g., given $F$-vector spaces $U$, $V$
and $W$ and a linear map $U \otimes_F V \to W$ we obtain a linear
map $U(n) \otimes_F V(m) \to W(n + m)$ for $n, m \in \mathbf{Z}$.

\medskip\noindent
Remarks on (D3). The map $\gamma$ is called the {\it cycle class map}.
We say that $\gamma(\alpha)$ is the {\it cohomology class} of $\alpha$.
If $Z \subset Y \subset X$ are closed subschemes with $Y$ and $X$
smooth projective over $k$ and $Z$ integral, then $[Z]$ could
mean the class of the cycle $[Z]$ in $\CH^*(Y)$ or in $\CH^*(X)$.
In this case the notation $\gamma([Z])$ is abiguous and the true meaning
has to be deduced from context.

\medskip\noindent
Remarks on (D4). The map $\int_X$ is sometimes called the
{\it trace map} and is sometimes denoted $\text{Tr}_X$.

\medskip\noindent
The first axiom is often called {\it Poincar\'e duality}
\begin{enumerate}
\item[(A)] Let $X$ be a nonempty smooth projective scheme over $k$
which is equidimensional of dimension $d$. Then
\begin{enumerate}
\item $\dim_F H^i(X) < \infty$ for all $i$,
\item $H^i(X) \times H^{2d - i}(X)(d) \rightarrow
H^{2d}(X)(d) \rightarrow F$
is a perfect pairing for all $i$ where the final
map is the trace map $\int_X$.
\end{enumerate}
\end{enumerate}
Let $f : X \to Y$ be a morphism of nonempty smooth projective schemes with $X$
equidimensional of dimension $d$ and $Y$ is equidimensional of dimension $e$.
Using Poincar\'e duality we can define a {\it pushforward}
$$
f_* : H^{2d - i}(X)(d) \longrightarrow H^{2e - i}(Y)(e)
$$
as the contragredient of the linear map $f^* : H^i(Y) \to H^i(X)$. In a
formula, for $a \in H^{2d - i}(X)(d)$, the element $f_*a \in H^{2e - i}(Y)(e)$
is characterized by
$$
\int_X f^*b \cup a = \int_Y b \cup f_*a
$$
for all $b \in H^i(Y)$.

\begin{lemma}
\label{lemma-pushforward}
Assume given (D1), (D2), and (D4) satisfying (A). For $f : X \to Y$
a morphism of nonempty equidimensional smooth projective schemes over $k$
we have $f_*(f^*b \cup a) = b \cup f_*a$. If $g : Y \to Z$ is a second morphism
with $Z$ nonempty smooth projective and equidimensional, then
$g_* \circ f_* = (g \circ f)_*$.
\end{lemma}

\begin{proof}
The first equality holds because
$$
\int_Y c \cup b \cup f_*a =
\int_X f^*c \cup f^*b \cup a =
\int_Y c \cup f_*(f^*b \cup a).
$$
The second equality holds because
$$
\int_Z c \cup (g \circ f)_*a = \int_X (g \circ f)^*c \cup a =
\int_X f^* g^* c \cup a = \int_Y g^*c \cup f_*a = \int_Z c \cup g_*f_*a
$$
This ends the proof.
\end{proof}

\noindent
The second axiom says that $H^*$ respects the monoidal structure
given by products via the {\it K\"unneth formula}
\begin{enumerate}
\item[(B)] Let $X$ and $Y$ be smooth projective schemes over $k$.
\begin{enumerate}
\item $H^*(X) \otimes_F H^*(Y) \to H^*(X \times Y)$,
$\alpha \otimes \beta \mapsto \text{pr}_1^*\alpha \cup \text{pr}_2^*\beta$
is an isomorphism,
\item if $X$ and $Y$ are nonempty and equidimensional, then
$\int_{X \times Y} = \int_X \otimes \int_Y$ via (a).
\end{enumerate}
\end{enumerate}
Using axiom (B)(b) we can compute pushforwards along projections.

\begin{lemma}
\label{lemma-pr2star}
Assume given (D1), (D2), and (D4) satisfying (A) and (B).
Let $X$ and $Y$ be smooth projective schemes over $k$.
If $X$ and $Y$ are nonempty and equidimensional of dimensions $d$ and $e$, then
$\text{pr}_{2, *} : H^*(X \times Y)(d + e) \to H^*(Y)(e)$ sends
$a \otimes b$ to $(\int_X a) b$.
\end{lemma}

\begin{proof}
This follows from axioms (B)(a) and (B)(b).
\end{proof}

\noindent
The third axiom concerns the cycle class maps
\begin{enumerate}
\item[(C)] The cycle class maps satisfy the following rules
\begin{enumerate}
\item for a morphism $f : X \to Y$ of smooth projective schemes over
$k$ we have $\gamma(f^*\beta) = f^*\gamma(\beta)$ for $\beta \in \CH^*(Y)$,
\item for a morphism $f : X \to Y$ of nonempty
equidimensional smooth projective schemes over $k$ we have
$\gamma(f_*\alpha) = f_*\gamma(\alpha)$ for $\alpha \in \CH^*(X)$,
\item for any smooth projective scheme $X$ over $k$ we have
$\gamma(\alpha \cdot \beta) = \gamma(\alpha) \cup \gamma(\beta)$
for $\alpha, \beta \in \CH^*(X)$, and
\item $\int_{\Spec(k)} \gamma([\Spec(k)]) = 1$.
\end{enumerate}
\end{enumerate}
Let us elucidate axiom (C)(b). Namely, say $f : X \to Y$ is
as in (C)(b) with $\dim(X) = d$ and $\dim(Y) = e$. Then we
see that pushforward on Chow groups gives
$$
f_* : \CH^{d - i}(X) = \CH_i(X) \to \CH_i(Y) = \CH^{e - i}(Y)
$$
Say $\alpha \in \CH^{d - i}(X)$. On the one hand, we have
$f_*\alpha \in \CH^{e - i}(Y)$ and hence
$\gamma(f_*\alpha) \in H^{2e - 2i}(Y)(e - i)$.
On the other hand, we have
$\gamma(\alpha) \in H^{2d - 2i}(X)(d - i)$ and hence
$f_*\gamma(\alpha) \in H^{2e - 2i}(Y)(e - i)$ as well.
Thus the condition $\gamma(f_*\alpha) = f_*\gamma(\alpha)$ makes sense.

\begin{remark}
\label{remark-replace-cup-product}
Assume given (D1), (D2), (D3) and (D4) satisfying (A), (B), and (C)(a).
Let $X$ be a smooth projective scheme over $k$. We obtain a map
$$
H^*(X) \otimes_F H^*(X) \longrightarrow H^*(X \times X)
\xrightarrow{\Delta^*} H^*(X)
$$
where $\Delta^*$ is pullback along the diagonal morphism
$\Delta : X \to X \times X$. The composition is the cup product.
(Hints: pullback is an algebra homomorphism and
$\text{pr}_i \circ \Delta = \text{id}$.)
On the other hand, the intersection product
$\alpha \cdot \beta$ of cycles $\alpha, \beta$ on $X$ is the
pullback of the exterior product $\alpha \times \beta$ on $X \times X$.
It follows that in order to prove axiom (C)(c) it suffices to show that
$$
\gamma(\alpha \times \beta) =
\text{pr}_1^*\gamma(\alpha) \cup \text{pr}_2^*\gamma(\beta)
$$
Conversely, if $\gamma$ satisfies (C)(a) and (C)(c), then this
formula always holds as
$\alpha \times \beta = \text{pr}_1^*\alpha \cdot \text{pr}_2^*\beta$
in the Chow ring of $X \times Y$.
\end{remark}

\begin{lemma}
\label{lemma-base}
Assume given (D1), (D2), (D3) and (D4) satisfying (A), (B), and (C).
Then $H^i(\Spec(k)) = 0$ for $i \not = 0$ and there is a
unique $F$-algebra isomorphism $F = H^0(\Spec(k))$.
We have $\gamma([\Spec(k)]) = 1$ and $\int_{\Spec(k)} 1 = 1$.
\end{lemma}

\begin{proof}
By axiom (C)(d) we see that $H^0(\Spec(k))$ is nonzero and even
$\gamma([\Spec(k)])$ is nonzero.
Since $\Spec(k) \times \Spec(k) = \Spec(k)$ we get
$$
H^*(\Spec(k)) \otimes_F H^*(\Spec(k)) = H^*(\Spec(k))
$$
by axiom (B)(a) which implies (look at dimensions) that only
$H^0$ is nonzero and moreover has dimension $1$. Thus
$F = H^0(\Spec(k))$ via the unique $F$-algebra isomorphism
given by mapping $1 \in F$ to $1 \in H^0(\Spec(k))$.
Since $[\Spec(k)] \cdot [\Spec(k)] = [\Spec(k)]$ in the
Chow ring of $\Spec(k)$ we conclude that
$\gamma([\Spec(k)) \cup \gamma([\Spec(k)]) = \gamma([\Spec(k)])$
by axiom (C)(c). Since we already know that $\gamma([\Spec(k)])$ is nonzero
we conclude that it has to be equal to $1$.
Finally, we have $\int_{\Spec(k)} 1 = 1$ since
$\int_{\Spec(k)} \gamma([\Spec(k)]) = 1$
by axiom (C)(d).
\end{proof}

\begin{lemma}
\label{lemma-unit}
Assume given (D1), (D2), (D3) and (D4) satisfying (A), (B), and (C).
Let $X$ be a smooth projective scheme over $k$.
If $X = \emptyset$, then $H^*(X) = 0$.
If $X$ is nonempty, then $\gamma([X]) = 1$ and $1 \not = 0$ in $H^0(X)$.
\end{lemma}

\begin{proof}
First assume $X$ is nonempty.
Observe that $[X]$ is the pullback of $[\Spec(k)]$ by the structure morphism
$p : X \to \Spec(k)$. Hence we get $\gamma([X]) = 1$ by axiom (C)(a)
and Lemma \ref{lemma-base}. Let $X' \subset X$ be an irreducible component.
By functoriality it suffices to show $1 \not = 0$ in $H^0(X')$.
Thus we may and do assume $X$ is irreducible, and in particular
nonempty and equidimensional, say of dimension $d$.
To see that $1 \not = 0$ it suffices to show that $H^*(X)$ is nonzero.

\medskip\noindent
Let $x \in X$ be a closed point whose residue field $k'$
is separable over $k$, see
Varieties, Lemma \ref{varieties-lemma-smooth-separable-closed-points-dense}.
Let $i : \Spec(k') \to X$ be the inclusion morphism.
Denote  $p : X \to \Spec(k)$ is the structure morphism.
Observe that
$p_*i_*[\Spec(k')] = [k' : k][\Spec(k)]$ in $\CH_0(\Spec(k))$.
Using axiom (C)(b) twice and Lemma \ref{lemma-base}
we conclude that
$$
p_*i_*\gamma([\Spec(k')]) = \gamma([k' : k][\Spec(k)]) = [k' : k]
\in F = H^0(\Spec(k))
$$
is nonzero. Thus $i_*\gamma([\Spec(k)]) \in H^{2d}(X)(d)$ is nonzero
(because it maps to something nonzero via $p_*$). This concludes the proof
in case $X$ is nonempty.

\medskip\noindent
Finally, we consider the case of the empty scheme. Axiom (B)(a) gives
$H^*(\emptyset) \otimes H^*(\emptyset) = H^*(\emptyset)$ and
we get that $H^*(\emptyset)$ is either zero or $1$-dimensional
in degree $0$. Then axiom (B)(a) again shows that
$H^*(\emptyset) \otimes H^*(X) = H^*(\emptyset)$ for
all smooth projective schemes $X$ over $k$. Using axiom (A)(b)
and the nonvanishing of $H^0(X)$ we've seen above
we find that $H^*(X)$ is nonzero in at least two degrees
if $\dim(X) > 0$. This then forces $H^*(\emptyset)$ to be zero.
\end{proof}

\begin{lemma}
\label{lemma-push-unit}
Assume given (D1), (D2), (D3) and (D4) satisfying (A), (B), and (C).
Let $i : X \to Y$ be a closed immersion of nonempty smooth projective
equidimensional schemes over $k$. Then
$\gamma([X]) = i_*1$ in $H^{2c}(Y)(c)$ where $c = \dim(Y) - \dim(X)$.
\end{lemma}

\begin{proof}
This is true because $1 = \gamma([X])$ in $H^0(X)$ by Lemma \ref{lemma-unit}
and then we can apply axiom (C)(b).
\end{proof}

\begin{lemma}
\label{lemma-class-diagonal}
Assume given (D1), (D2), (D3) and (D4) satisfying (A), (B), and (C).
Let $X$ be a nonempty smooth projective scheme which is equidimensional
of dimension $d$ over $k$. Choose a basis
$e_{i, j}, j = 1, \ldots, \beta_i$ of $H^i(X)$ over $F$.
Using K\"unneth write
$$
\gamma([\Delta]) =
\sum\nolimits_{i = 0, \ldots, 2d}
\sum\nolimits_j e_{i, j} \otimes e'_{2d - i , j}
\quad\text{in}\quad
\bigoplus\nolimits_i H^i(X) \otimes_F H^{2d - i}(X)(d)
$$
with $e'_{2d - i, j} \in H^{2d - i}(X)(d)$.
Then $\int_X e_{i, j} \cup e'_{2d - i, j'} = (-1)^i\delta_{jj'}$.
\end{lemma}

\begin{proof}
Recall that $\Delta^* : H^*(X \times X) \to H^*(X)$ is equal to the
cup product map $H^*(X) \otimes_F H^*(X) \to H^*(X)$, see
Remark \ref{remark-replace-cup-product}. On the other
hand, recall that $\gamma([\Delta]) = \Delta_*1$ (Lemma \ref{lemma-push-unit})
and hence
$$
\int_{X \times X} \gamma([\Delta]) \cup a \otimes b =
\int_{X \times X} i_*1 \cup a \otimes b =
\int_X a \cup b
$$
by Lemma \ref{lemma-pushforward}.
On the other hand, we have
$$
\int_{X \times X} (\sum e_{i, j} \otimes e'_{2d -i , j}) \cup a \otimes b =
\sum (\int_X a \cup e_{i, j})(\int_X e'_{2d - i, j} \cup b)
$$
by axiom (B)(b); note that we made two switches of order so that the sign
for each term is $1$. Thus if we choose $a$ such that
$\int_X a \cup e_{i, j} = 1$ and all other pairings equal to zero, then
we conclude that $\int_X e'_{2d - i, j} \cup b = \int_X a \cup b$
for all $b$, i.e., $e'_{2d - i, j} = a$. This proves the lemma.
\end{proof}

\begin{lemma}
\label{lemma-cohomology-P1}
Assume given (D1), (D2), (D3) and (D4) satisfying (A), (B), and (C).
Then $H^*(\mathbf{P}^1_k)$ is $1$-dimensional in dimensions $0$ and $2$
and zero in other degrees.
\end{lemma}

\begin{proof}
Let $x \in \mathbf{P}^1_k$ be a $k$-rational point. Observe that
$\Delta = \text{pr}_1^*x + \text{pr}_2^*x$ as divisors on
$\mathbf{P}^1_k \times \mathbf{P}^1_k$. Using axiom (C)(a)
and additivity of $\gamma$ we see that
$$
\gamma([\Delta]) =
\text{pr}_1^*\gamma([x]) +
\text{pr}_2^*\gamma([x]) =
\gamma([x]) \otimes 1 + 1 \otimes \gamma([x])
$$
in $H^*(\mathbf{P}^1_k \times \mathbf{P}^1_k) =
H^*(\mathbf{P}^1_k) \otimes_F H^*(\mathbf{P}^1_k)$.
However, by Lemma \ref{lemma-class-diagonal}
we know that $\gamma([\Delta])$ cannot be written
as a sum of fewer than $\sum \beta_i$ pure tensors
where $\beta_i = \dim_F H^i(\mathbf{P}^1_k)$.
Thus we see that $\sum \beta_i \leq 2$.
By Lemma \ref{lemma-unit} we have $H^0(\mathbf{P}^1_k) \not = 0$.
By Poincar\'e duality, more precisely axiom (A)(b),
we have $\beta_0 = \beta_2$. Therefore the lemma holds.
\end{proof}

\begin{lemma}
\label{lemma-weil-additive}
Assume given (D1), (D2), (D3) and (D4) satisfying (A), (B), and (C).
If $X$ and $Y$ are smooth projective schemes over $k$, then
$H^*(X \amalg Y) \to H^*(X) \times H^*(Y)$,
$a \mapsto (i^*a, j^*a)$ is an isomorphism where $i$, $j$
are the coprojections.
\end{lemma}

\begin{proof}
If $X$ or $Y$ is empty, then this is true because
$H^*(\emptyset) = 0$ by Lemma \ref{lemma-unit}.
Thus we may assume both $X$ and $Y$ are nonempty.

\medskip\noindent
We first show that the map is injective. First, observe that
we can find morphisms $X' \to X$ and $Y' \to Y$
of smooth projective schemes so that $X'$ and $Y'$ are
equidimensional of the same dimension and such that
$X' \to X$ and $Y' \to Y$ each have a section. Namely,
decompose $X = \coprod X_d$ and $Y = \coprod Y_e$
into open and closed subschemes equidimensional of
dimension $d$ and $e$. Then take
$X' = \coprod X_d \times \mathbf{P}^{n - d}$
and $Y' = \coprod Y_e \times \mathbf{P}^{n - e}$ for some
$n$ sufficiently large. Thus pullback by
$X' \amalg Y' \to X \amalg Y$ is injective
(because there is a section) and
it suffices to show the injectivity for $X', Y'$
as we do in the next parapgrah.

\medskip\noindent
Let us show the map is injective when $X$ and $Y$ are equidimensional
of the same dimension $d$.
Observe that $[X \amalg Y] = [X] + [Y]$ in $\CH^0(X \amalg Y)$
and that $[X]$ and $[Y]$ are orthogonal idempotents in $\CH^0(X \amalg Y)$.
Thus
$$
1 = \gamma([X \amalg Y] = \gamma([X]) + \gamma([Y]) = i_*1 + j_*1
$$
is a decomposition into orthogonal idempotents. Here we have used
Lemmas \ref{lemma-unit} and \ref{lemma-push-unit} and axiom (C)(c).
Then we see that
$$
a = a \cup 1 = a \cup i_*1 + a \cup j_*1 =
i_*(i^*a) + j_*(j^*a)
$$
by the projection formula (Lemma \ref{lemma-pushforward}) and hence the map
is injective.

\medskip\noindent
We show the map is surjective. Write $e = \gamma([X])$ and $f = \gamma([Y])$
viewed as elements in $H^0(X \amalg Y)$. We have
$i^*e = 1$, $i^*f = 0$, $j^*e = 0$, and $j^*f = 1$ by axiom (C)(a).
Hence if $i^* : H^*(X \amalg Y) \to  H^*(X)$
and $j^* : H^*(X \amalg Y) \to H^*(Y)$ are surjective, then
so is $(i^*, j^*)$. Namely, for $a, a' \in H^*(X \amalg Y)$
we have
$$
(i^*a, j^*a') = (i^*(a \cup e + a' \cup f), j^*(a \cup e + a' \cup f))
$$
By symmetry it suffices to show $i^* : H^*(X \amalg Y) \to  H^*(X)$
is surjective. If there is a morphism $Y \to X$, then there is a morphism
$g : X \amalg Y \to X$ with $g \circ i = \text{id}_X$ and we conclude.
To finish the proof, observe that in order to prove
$i^*$ is surjective, it suffices to do so after tensoring
by a nonzero graded $F$-vector space. Hence by axiom (B)(b)
and nonvanishing of cohomology (Lemma \ref{lemma-unit})
it suffices to prove $i^*$ is surjective after replacing
$X$ and $Y$ by $X \times \Spec(k')$ and $Y \times \Spec(k')$
for some finite separable extension $k'/k$.
If we choose $k'$ such that there exists a closed point
$x \in X$ with $\kappa(x) = k'$ (and this is possible by
Varieties, Lemma \ref{varieties-lemma-smooth-separable-closed-points-dense})
then there is a morphism $Y \times \Spec(k') \to X \times \Spec(k')$
and we find that the proof is complete.
\end{proof}

\begin{lemma}
\label{lemma-from-functor-to-weil}
Let $k$ be a field. Let $F$ be a field of characteristic $0$.
Assume given a $\mathbf{Q}$-linear functor
$$
G : M_k \longrightarrow \text{graded }F\text{-vector spaces}
$$
of symmetric monoidal categories such that $G(\mathbf{1}(1))$
is nonzero only in degree $-2$. Then we obtain data (D1), (D2), (D3), (D4)
satisfying all of (A), (B), (C) above.
\end{lemma}

\begin{proof}
This proof is the same as the proof of
Lemma \ref{lemma-from-functor-to-weil-classical};
we urge the reader to read the proof of that lemma instead.
We obtain a contravariant functor from the category of smooth
projective schemes over $k$ to the category of graded $F$-vector spaces
by setting $H^*(X) = G(h(X))$. By assumption we have a canonical
isomorphism
$$
H^*(X \times Y) = G(h(X \times Y)) = G(h(X) \otimes h(Y)) =
G(h(X)) \otimes G(h(Y)) = H^*(X) \otimes H^*(Y)
$$
compatible with pullbacks. Using pullback along the diagonal
$\Delta : X \to X \times X$ we obtain a canonical map
$$
H^*(X) \otimes H^*(X) = H^*(X \times X) \to H^*(X)
$$
of graded vector spaces compatible with pullbacks.
This defines a functorial graded $F$-algebra structure on
$H^*(X)$. Since $\Delta$ commutes with the commutativity
constraint $h(X) \otimes h(X) \to h(X) \otimes h(X)$ (switching the factors)
and since $G$ is a functor of symmetric monoidal categories (so compatible with
commutativity constraints), and by our convention in
Example \ref{example-graded-vector-spaces}
we conclude that $H^*(X)$ is a graded
commutative algebra! Hence we get our datum (D1).

\medskip\noindent
Our datum (D2) is the vector space $F(1) = G^{-2}(\mathbf{1}(1))$.
Since $G$ is a symmetric monoidal functor we see that
$F(n) = G^{-2n}(\mathbf{1}(n))$ for all $n$.
It follows that
$$
H^{2r}(X)(r) = G^{2r}(h(X)) \otimes G^{-2r}(\mathbf{1}(r)) =
G^0(h(X)(r))
$$
a formula we will frequently use below.

\medskip\noindent
Let $X$ be a smooth projective scheme over $k$. By
Lemma \ref{lemma-composition-correspondences} we have
$$
\CH^r(X) \otimes \mathbf{Q} = \text{Corr}^r(\Spec(k), X) =
\Hom(\mathbf{1}(-r), h(X)) = \Hom(\mathbf{1}, h(X)(r))
$$
Applying the functor $G$ this maps into
$\Hom(G(\mathbf{1}), G(h(X)(r)))$.
By taking the image of $1$ in $G^0(\mathbf{1}) = F$
into $G^0(h(X)(r)) = H^{2r}(X)(r)$ we obtain
$$
\gamma :
\CH^r(X) \otimes \mathbf{Q} \longrightarrow H^{2r}(X)(r)
$$
This is the datum (D2).

\medskip\noindent
Let $X$ be a nonempty smooth projective scheme over $k$
which is equidimensional of dimension $d$. By
Lemma \ref{lemma-composition-correspondences} we have
$$
\Mor(h(X)(d), \mathbf{1}) = \Mor((X, 1, d), (\Spec(k), 1, 0)) =
\text{Corr}^{-d}(X, \Spec(k)) = \CH_d(X)
$$
Thus the class of the cycle $[X]$ in $\CH_d(X)$ defines a morphism
$h(X)(d) \to \mathbf{1}$. Applying $G$ and taking degree $0$
parts we obtain
$$
H^{2d}(X)(d) = G^0(h(X)(d)) \longrightarrow G^0(\mathbf{1}) = F
$$
This map $\int_X : H^{2d}(X)(d) \to F$ is the datum (D3).

\medskip\noindent
Let $X$ be a smooth projective scheme over $k$ which is
nonempty and equidimensional of dimension $d$. By Lemma \ref{lemma-dual}
we know that $h(X)(d)$ is a left dual to $h(X)$. Hence
$G(h(X)(d)) = H^*(X) \otimes_F F(d)[2d]$
is a left dual to $H^*(X)$ in the category of graded $F$-vector spaces.
Here $[n]$ is the shift functor on graded vector spaces.
By Lemma \ref{lemma-left-dual-graded-vector-spaces}
we find that $\sum_i \dim_F H^i(X) < \infty$ and that
$\epsilon : h(X)(d) \otimes h(X) \to \mathbf{1}$ produces
nondegenerate pairings $H^{2d - i}(X)(d) \otimes_F H^i(X) \to F$.
In the proof of Lemma \ref{lemma-dual} we have seen that
$\epsilon$ is given by $[\Delta]$ via the identifications
$$
\Hom(h(X)(d) \otimes h(X), \mathbf{1}) =
\text{Corr}^{-d}(X \times X, \Spec(k)) =
\CH_d(X \times X)
$$
Thus $\epsilon$ is the composition of $[X] : h(X)(d) \to \mathbf{1}$
and $h(\Delta)(d) : h(X)(d) \otimes h(X) \to h(X)(d)$. It follows
that the pairings above are given by cup product followed by
$\int_X$. This proves axiom (A).

\medskip\noindent
Axiom (B) follows from the assumption that $G$ is compatible
with tensor structures and our construction of the cup product above.

\medskip\noindent
Axiom (C). Our construction of $\gamma$ takes a cycle $\alpha$ on $X$,
interprets it a correspondence $a$ from $\Spec(k)$ to $X$ of some degree,
and then applies $G$. If $f : Y \to X$ is a morphism of nonempty
equidimensional smooth projective schemes over $k$, then
$f^*\alpha$ is the pushforward (!) of $\alpha$
by the correspondence $[\Gamma_f]$ from $X$ to $Y$, see
Lemma \ref{lemma-functor-and-cycles}. Hence
$f^*\alpha$ viewed as a correspondence from $\Spec(k)$ to $Y$
is equal to $a \circ [\Gamma_f]$, see
Lemma \ref{lemma-composition-correspondences}.
Since $G$ is a functor, we conclude
$\gamma$ is compatible with pullbacks, i.e., axiom (C)(a) holds.

\medskip\noindent
Let $f : Y \to X$ be a morphism of nonempty equidimensional
smooth projective schemes over $k$ and
let $\beta \in \CH^r(Y)$ be a cycle on $Y$. We have to show that
$$
\int_Y \gamma(\beta) \cup f^*c = \int_X \gamma(f_*\beta) \cup c
$$
for all $c \in H^*(X)$. Let $a, a^t, \eta_X, \eta_Y, [X], [Y]$
be as in Lemma \ref{lemma-prep-dual}.
Let $b$ be $\beta$ viewed as a correspondence from $\Spec(k)$ to $Y$
of degree $r$. Then $f_*\beta$ viewed as a correspondence from
$\Spec(k)$ to $X$ is equal to $a^t \circ b$, see
Lemmas \ref{lemma-functor-and-cycles} and
\ref{lemma-composition-correspondences}.
The displayed equality above holds if we can show that
$$
h(X) = \mathbf{1} \otimes h(X)
\xrightarrow{b \otimes 1}
h(Y)(r) \otimes h(X)
\xrightarrow{1 \otimes a}
h(Y)(r) \otimes h(Y)
\xrightarrow{\eta_Y}
h(Y)(r)
\xrightarrow{[Y]}
\mathbf{1}(r - e)
$$
is equal to
$$
h(X) = \mathbf{1} \otimes h(X)
\xrightarrow{a^t \circ b \otimes 1}
h(X)(r + d - e) \otimes h(X)
\xrightarrow{\eta_X}
h(X)(r + d - e)
\xrightarrow{[X]}
\mathbf{1}(r - e)
$$
This follows immediately from Lemma \ref{lemma-prep-dual}.
Thus we have axiom (C)(b).

\medskip\noindent
To prove axiom (C)(c) we use the discussion in
Remark \ref{remark-replace-cup-product-classical}.
Hence it suffices to prove that $\gamma$ is compatible with
exterior products. Let $X$, $Y$ be nonempty smooth projective
schemes over $k$ and let $\alpha$, $\beta$ be cycles on them. Denote
$a$, $b$ the corresponding correspondences from $\Spec(k)$ to
$X$, $Y$. Then $\alpha \times \beta$ corresponds to the
correspondence $a \otimes b$ from $\Spec(k)$ to $X \otimes Y = X \times Y$.
Hence the requirement follows from the fact that $G$ is
compatible with the tensor structures on both sides.

\medskip\noindent
Axiom (C)(d) follows because the cycle $[\Spec(k)]$
corresponds to the identity morphism on $h(\Spec(k))$.
This finishes the proof of the lemma.
\end{proof}

\begin{lemma}
\label{lemma-from-weil-to-functor}
Let $k$ be a field. Let $F$ be a field of characteristic $0$. Given
(D1), (D2), (D3), and (D4) satisfying (A), (B), (C)
we can construct a $\mathbf{Q}$-linear functor
$$
G : M_k \longrightarrow \text{graded }F\text{-vector spaces}
$$
of symmetric monoidal categories such that $H^*(X) = G(h(X))$.
\end{lemma}

\begin{proof}
The proof of this lemma is the same as the proof of
Lemma \ref{lemma-from-weil-to-functor-classical};
we urge the reader to read the proof of that lemma instead.
By Lemma \ref{lemma-characterize-motives} it suffices to construct a functor
$G$ on the category of smooth projective schemes over $k$
with morphisms given by correspondences of degree $0$ such that
the image of $G(c_2)$ on $G(\mathbf{P}^1_k)$ is an invertible graded
$F$-vector space.

\medskip\noindent
Let $X$ be a smooth projective scheme over $k$. There is a canonical
decomposition
$$
X = \coprod\nolimits_{0 \leq d \le \dim(X)} X_d
$$
into open and closed subschemes such that $X_d$ is equidimensional
of dimension $d$. By Lemma \ref{lemma-weil-additive} we have correspondingly
$$
H^*(X) \longrightarrow \prod\nolimits_{0 \leq d \le \dim(X)} H^*(X_d)
$$
If $Y$ is a second smooth projective scheme over $k$
and we similarly decompose $Y = \coprod Y_e$, then
$$
\text{Corr}^0(X, Y) = \bigoplus \text{Corr}^0(X_d, Y_e)
$$
As well we have $X \otimes Y = \coprod X_d \otimes Y_e$ in the
category of correspondences. From these observations it follows
that it suffices to construct $G$ on the category whose objects
are equidimensional smooth projective schemes over $k$
and whose morphisms are correspondences of degree $0$. (Some details
omitted.)

\medskip\noindent
Given an equdimensional smooth projective scheme
$X$ over $k$ we set $G(X) = H^*(X)$. Observe that $G(X) = 0$
if $X = \emptyset$ (Lemma \ref{lemma-unit}). Thus maps
from and to $G(\emptyset)$ are zero and we may and do
assume our schemes are nonempty in the discussions below.

\medskip\noindent
Given a correspondence $c \in \text{Corr}^0(X, Y)$ between
nonempty equidmensional smooth projective schemes over $k$
we consider the map $G(c) : G(X) = H^*(X) \to G(Y) = H^*(Y)$
given by the rule
$$
a \longmapsto
G(c)(a) = \text{pr}_{2, *}(\gamma(c) \cup \text{pr}_1^*a)
$$
It is clear that $G(c)$ is additive in $c$ and hence $\mathbf{Q}$-linear.
Compatibility of $\gamma$ with pullbacks, pushforwards, and
intersection products given by axioms (C)(a), (C)(b), and (C)(c)
shows that we have
$G(c' \circ c) = G(c') \circ G(c)$ if $c' \in \text{Corr}^0(Y, Z)$.
Namely, for $a \in H^*(X)$ we have
\begin{align*}
(G(c') \circ G(c))(a)
& =
\text{pr}^{23}_{3, *}(\gamma(c') \cup
\text{pr}^{23, *}_2(\text{pr}^{12}_{2, *}(\gamma(c) \cup
\text{pr}^{12, *}_1a))) \\
& =
\text{pr}^{23}_{3, *}(\gamma(c') \cup
\text{pr}^{123}_{23, *}(\text{pr}^{123, *}_{12}(\gamma(c) \cup
\text{pr}^{12, *}_1 a))) \\
& =
\text{pr}^{23}_{3, *}
\text{pr}^{123}_{23, *}(
\text{pr}^{123, *}_{23}\gamma(c') \cup
\text{pr}^{123, *}_{12}\gamma(c) \cup
\text{pr}^{123, *}_1 a) \\
& =
\text{pr}^{23}_{3, *}
\text{pr}^{123}_{23, *}(
\gamma(\text{pr}^{123, *}_{23}c') \cup
\gamma(\text{pr}^{123, *}_{12}c) \cup
\text{pr}^{123, *}_1 a) \\
& =
\text{pr}^{13}_{3, *}
\text{pr}^{123}_{13, *}(
\gamma(\text{pr}^{123, *}_{23}c' \cdot \text{pr}^{123, *}_{12}c) \cup
\text{pr}^{123, *}_1 a) \\
& =
\text{pr}^{13}_{3, *}(
\gamma(\text{pr}^{123}_{13, *}(\text{pr}^{123, *}_{23}c' \cdot
\text{pr}^{123, *}_{12}c)) \cup
\text{pr}^{13, *}_1 a) \\
& =
G(c' \circ c)(a)
\end{align*}
with obvious notation. To finish the proof that $G$ is a functor,
we have to show identities are preserved. In other words, if
$c = [\Delta] \in \text{Corr}^0(X, X)$ is the unit
(Lemma \ref{lemma-identity-correspondence}), then we have to show that
$G(c) = \text{id}$. This follows from the determination
of $\gamma([\Delta])$ in Lemma \ref{lemma-class-diagonal}
and Lemma \ref{lemma-pr2star}.
This finishes the construction of $G$ as a functor on
smooth projective schemes over $k$ and correspondences of degree $0$.

\medskip\noindent
By Lemma \ref{lemma-base} we have that
$G(\Spec(k)) = H^*(\Spec(k))$ is canonically isomorphic to $F$
as an $F$-algebra. The K\"unneth axiom (B)(a)
shows our functor is compatible with tensor products.
Thus our functor is a functor of symmetric monoidal categories.

\medskip\noindent
We still have to check that the image of $G(c_2)$ on
$G(\mathbf{P}^1_k) = H^*(\mathbf{P}^1_k)$
is an invertible graded $F$-vector space (in particular we don't know yet
that $G$ extends to $M_k$). By Lemma \ref{lemma-cohomology-P1}
we only have nonzero cohomology in degrees $0$ and $2$
both of dimension $1$. We have $1 = c_0 + c_2$ is a decomposition
of the identity into a sum of orthogonal idempotents in
$\text{Corr}^0(\mathbf{P}^1_k, \mathbf{P}^1_k)$, see
Example \ref{example-decompose-P1}. Further we have $c_0 = a \circ b$ where
$a \in \text{Corr}^0(\Spec(k), \mathbf{P}^1_k)$ and
$b \in \text{Corr}^0(\mathbf{P}^1_k, \Spec(k))$ and where
$b \circ a = 1$ in $\text{Corr}^0(\Spec(k), \Spec(k))$, see proof of
Lemma \ref{lemma-inverse-h2}. Thus $G(c_0)$ is the projector
onto the degree $0$ part. It follows that $G(c_2)$ must
be the projector onto the degree $2$ part and the proof is complete.
\end{proof}

\begin{proposition}
\label{proposition-weil-cohomology-theory}
Let $k$ be a field. Let $F$ be a field of characteristic $0$. There is a
$1$-to-$1$ correspondence between the following
\begin{enumerate}
\item data (D1), (D2), (D3), and (D4) satisfying (A), (B), (C), and
\item $\mathbf{Q}$-linear symmetric monoidal functors
$$
G : M_k \longrightarrow \text{graded }F\text{-vector spaces}
$$
such that $G(\mathbf{1}(1))$ is nonzero only in degree $-2$.
\end{enumerate}
\end{proposition}

\begin{proof}
Given $G$ by setting $H^*(X) = G(h(X))$ we obtain data (D1), (D2), (D3)
satisfying (A), (B), (C), see Lemma \ref{lemma-from-functor-to-weil}
and its proof.

\medskip\noindent
Conversely, given data (D1), (D2), (D3)
satisfying (A), (B), (C) we get a functor $G$ by the construction of
the proof of Lemma \ref{lemma-from-weil-to-functor}.

\medskip\noindent
We omit the detailed proof that these constructions are inverse
to each other.
\end{proof}











\section{Further properties}
\label{section-further}

\noindent
In this section we prove a few more results one obtains if
given data (D1), (D2), (D3) satisfying (A), (B), (C) as in
Section \ref{section-axioms}.

\begin{lemma}
\label{lemma-trace-disjoint-union}
Assume given (D1), (D2), (D3) and (D4) satisfying (A), (B), and (C).
Let $X, Y$ be nonempty smooth projective schemes both equidimensional
of dimension $d$ over $k$. Then $\int_{X \amalg Y} = \int_X + \int_Y$.
\end{lemma}

\begin{proof}
Denote $i : X \to X \amalg Y$ and $j : Y \to X \amalg Y$ be the coprojections.
By Lemma \ref{lemma-weil-additive} the map
$(i^*, j^*) : H^*(X \amalg Y) \to H^*(X) \times H^*(Y)$ is an isomorphism.
The statement of the lemma means that under the isomorphism
$(i^*, j^*) : H^{2d}(X \amalg Y)(d) \to H^{2d}(X)(d) \oplus H^{2d}(Y)(d)$
the map $\int_X + \int_Y$ is tranformed into $\int_{X \amalg Y}$.
This is true because
$$
\int_{X \amalg Y} a =
\int_{X \amalg Y} i_*(i^*a) + j_*(j^*a) =
\int_X i^*a + \int_Y j^*a
$$
where the equality $a = i_*(i^*a) + j_*(j^*a)$ was shown in
the proof of Lemma \ref{lemma-weil-additive}.
\end{proof}

\begin{lemma}
\label{lemma-dim-0}
Assume given (D1), (D2), (D3) and (D4) satisfying (A), (B), and (C).
Let $X$ be a smooth projective scheme of dimension zero over $k$.
Then
\begin{enumerate}
\item $H^i(X) = 0$ for $i \not = 0$,
\item $H^0(X)$ is a finite separable algebra over $F$,
\item $\dim_F H^0(X) = \deg(X \to \Spec(F))$,
\item $\int_X : H^0(X) \to F$ is the trace map,
\item $\gamma([X]) = 1$, and
\item $\int_X \gamma([X]) = \deg(X \to \Spec(k))$.
\end{enumerate}
\end{lemma}

\begin{proof}
We can write $X = \Spec(k')$ where $k'$ is a finite separable
algebra over $k$. Observe that $\deg(X \to \Spec(k)) = [k' : k]$.
Choose a finite Galois extension $k''/k$ containing each of the
factors of $k'$. (Recall that a finite separable $k$-algebra is
a product of finite separable field extension of $k$.)
Set $\Sigma = \Hom_k(k', k'')$. Then we get
$$
k' \otimes_k k'' = \prod\nolimits_{\sigma \in \Sigma} k''
$$
Setting $Y = \Spec(k'')$ axioms (B)(a) and Lemma \ref{lemma-weil-additive} give
$$
H^*(X) \otimes_F H^*(Y) =
\bigoplus\nolimits_{\sigma \in \Sigma} H^*(Y)
$$
as graded commutative $F$-algebras. By Lemma \ref{lemma-unit} the
$F$-algebra $H^*(Y)$ is nonzero. Comparing dimensions on either side
of the displayed equation we conclude that $H^*(X)$ sits only in degree $0$
and $\dim_F H^0(X) = [k' : k]$. Applying this to $Y$ we get
$H^*(Y) = H^0(Y)$. Since
$$
H^0(X) \otimes_F H^0(Y) = H^0(Y) \times \ldots \times H^0(Y)
$$
as $F$-algebras, it follows that $H^0(X)$ is a separable $F$-algebra
because we may check this after the faithfully flat base change
$F \to H^0(Y)$.

\medskip\noindent
The displayed isomorphism above is given by the map
$$
H^0(X) \otimes_F H^0(Y) \longrightarrow
\prod\nolimits_{\sigma \in \Sigma} H^0(Y),\quad
a \otimes b \longmapsto \prod\nolimits_\sigma \Spec(\sigma)^*a \cup b
$$
Via this isomorphism we have $\int_{X \times Y} = \sum_\sigma \int_Y$ by
Lemma \ref{lemma-trace-disjoint-union}. Thus
$$
\int_X a = \text{pr}_{1, *}(a \otimes 1) = \sum \Spec(\sigma)^*a
$$
in $H^0(Y)$; the first equality by Lemma \ref{lemma-pr2star}
and the second by the observation we just made. Choose an
algebraic closure $\overline{F}$ and
a $F$-algebra map $\tau : H^0(Y) \to \overline{F}$.
The isomorphism above base changes to the isomorphism
$$
H^0(X) \otimes_F \overline{F} \longrightarrow
\prod\nolimits_{\sigma \in \Sigma} \overline{F},\quad
a \otimes b \longmapsto \prod\nolimits_\sigma \tau(\Spec(\sigma)^*a) b
$$
It follows that $a \mapsto \tau(\Spec(\sigma)^*a)$ is a full set
of embeddings of $H^0(X)$ into $\overline{F}$. Applying $\tau$
to the formula for $\int_X a$ obtained above we conclude
that $\int_X$ is the trace map.
By Lemma \ref{lemma-unit} we have $\gamma([X]) = 1$.
Finally, we have $\int_X \gamma([X]) = \deg(X \to \Spec(k))$
because $\gamma([X]) = 1$ and the trace of $1$ is equal to $[k' : k]$
\end{proof}

\begin{lemma}
\label{lemma-degrees-cycles}
Assume given (D1), (D2), (D3) and (D4) satisfying (A), (B), and (C).
Let $X$ be a nonempty smooth projective scheme
equidimensional of dimension $d$ over $k$. The diagram
$$
\xymatrix{
\CH^d(X) \ar[r]_-\gamma \ar@{=}[d] &
H^{2d}(X)(d) \ar[d]^{\int_X} \\
\CH_0(X) \ar[r]^\deg & F
}
$$
commutes where $\deg : \CH_0(X) \to \mathbf{Z}$ is the degree of
zero cycles discussed in Chow Homology, Section
\ref{chow-section-degree-zero-cycles}.
\end{lemma}

\begin{proof}
Let $x$ be a closed point of $X$ whose residue field is separable
over $k$. View $x$ as a scheme and denote $i : x \to X$ the inclusion morphism.
To avoid confusion denote $\gamma' : \CH_0(x) \to H^0(x)$ the cycle class map
for $x$. Then we have
$$
\int_X \gamma([x]) = \int_X \gamma(i_*[x]) =
\int_X i_*\gamma'([x]) = \int_x \gamma'([x]) = \deg(x \to \Spec(k))
$$
The second equality is axiom (C)(b) and the third equality is
the definition of $i_*$ on cohomology. The final equality is
Lemma \ref{lemma-dim-0}. This proves the lemma
because $\CH_0(X)$ is generated by the classes of points $x$ as above
by Lemma \ref{lemma-generated-by-separable}.
\end{proof}

\begin{lemma}
\label{lemma-square-diagonal}
Assume given (D1), (D2), (D3) and (D4) satisfying (A), (B), and (C).
Let $X$ be a nonempty smooth projective scheme over $k$ which is
equidimensional of dimension $d$. We have
$$
\sum\nolimits_{i = 0, \ldots, 2d} (-1)^i\dim_F H^i(X) =
\deg(\Delta \cdot \Delta) = \deg(c_d(\mathcal{T}_{X/k}))
$$
\end{lemma}

\begin{proof}
The equality on the right holds by
Chow Homology, Lemma \ref{chow-lemma-gysin-fundamental}.
By Lemma \ref{lemma-degrees-cycles} we have
\begin{align*}
\deg(\Delta \cdot \Delta)
& =
\int_{X \times X} \gamma([\Delta]) \cup \gamma([\Delta]) \\
& =
\int_{X \times X} \Delta_*1 \cup \gamma([\Delta]) \\
& =
\int_{X \times X} \Delta_*(\Delta^*\gamma([\Delta])) \\
& =
\int_X \Delta^*\gamma([\Delta])
\end{align*}
We have used Lemmas \ref{lemma-push-unit} and
\ref{lemma-pushforward}.
Write $\gamma([\Delta]) = \sum  e_{i, j} \otimes e'_{2d - i , j}$
as in Lemma \ref{lemma-class-diagonal-classical}.
Recalling that $\Delta^*$ is given by cup product
(Remark \ref{remark-replace-cup-product}) we obtain
$$
\int_X \sum\nolimits_{i, j} e_{i, j} \cup e'_{2d - i, j} =
\sum\nolimits_{i, j} \int_X e_{i, j} \cup e'_{2d - i, j} =
\sum\nolimits_{i, j} (-1)^i = \sum (-1)^i\beta_i
$$
as desired.
\end{proof}

\begin{lemma}
\label{lemma-algebra-relations}
Let $F$ be a field of characteristic $0$.
Let $F'$ and $F_i$, $i = 1, \ldots, r$
be finite separable $F$-algebras. Let $A$ be a finite $F$-algebra.
Let $\sigma, \sigma' : A \to F'$ and $\sigma_i : A \to F_i$
be $F$-algebra maps. Assume $\sigma$ and $\sigma'$ surjective.
If there is a relation
$$
\text{Tr}_{F'/F} \circ \sigma - \text{Tr}_{F'/F} \circ \sigma' =
n(\sum m_i \text{Tr}_{F_i/F} \circ \sigma_i)
$$
where $n > 1$ and $m_i$ are integers, then $\sigma = \sigma'$.
\end{lemma}

\begin{proof}
Let $A \to A'$ be the maximal separable $F$-algebra quotient.
The maps $\sigma, \sigma', \sigma_i$ each factor through
$A \to A'$. After replacing $A$ by $A'$ we may assume
$A$ is a finnite separable $F$-algebra.

\medskip\noindent
Choose an algebraic closure $\overline{F}$. Set
$\overline{A} = A \otimes_F \overline{F}$,
$\overline{F}' = F' \otimes_F \overline{F}$, and
$\overline{F}_i = F_i \otimes_F \overline{F}$.
We can base change $\sigma$, $\sigma'$, $\sigma_i$
to get $\overline{F}$ algebra maps $\overline{A} \to \overline{F}'$
and $\overline{A} \to \overline{F}_i$. Moreover
$\text{Tr}_{\overline{F}'/\overline{F}}$ is the base
change of $\text{Tr}_{F'/F}$ and similarly for
$\text{Tr}_{F_i/F}$. Thus we may replace
$F$ by $\overline{F}$ and we reduce to the case discussed in
the next paragraph.

\medskip\noindent
Assume $F$ is algebraically closed. Then each of $A$, $F'$, $F_i$
is a product of copies of $F$. Let us say an element $e$ of a product
$F \times \ldots \times F$ of copies of $F$ is a minimal idempotent
if it generates one of the factors, i.e., if
$e = (0, \ldots, 0, 1, 0, \ldots, 0)$. Let $e \in A$ be a minimal idempotent.
Since $\sigma$ and $\sigma'$ 
are surjective, we see that $\sigma(e)$ and $\sigma'(e)$ are minimal
idempotents or zero. If $\sigma \not = \sigma'$, then we can choose
a minimal idempotent $e \in A$ such that $\sigma(e) = 0$ and
$\sigma'(e) \not = 0$ or vice versa. Then
$\text{Tr}_{F'/F}(\sigma(e)) = 0$ and
$\text{Tr}_{F'/F}(\sigma'(e)) = 1$ or vice versa.
On the other hand, $\sigma_i(e)$ is an idempotent
and hence $\text{Tr}_{F_i/F}(\sigma_i(e)) = r_i$ is an integer.
We conclude that
$$
-1 = \sum n m_i r_i = n (\sum m_i r_i)
\quad\text{or}\quad
1 = \sum n m_i r_i = n (\sum m_i r_i)
$$
which is impossible.
\end{proof}

\begin{lemma}
\label{lemma-relations-classes-points}
Assume given (D1), (D2), (D3) and (D4) satisfying (A), (B), and (C).
Let $k'/k$ be a finite separable extension.
Let $X$ be a smooth projective scheme over $k'$.
Let $x, x' \in X$ be $k'$-rational points.
If $\gamma(x) \not = \gamma(x')$, then
$[x] - [x']$ is not divisible by any integer $n > 1$ in $\CH_0(X)$.
\end{lemma}

\begin{proof}
If $x$ and $x'$ lie on distinct irreducible components of $X$, then
the result is obvious. Thus we may $X$ irreducible of dimension $d$.
Say $[x] - [x']$ is divisible by $n > 1$ in $\CH_0(X)$.
We may write $[x] - [x'] = n(\sum m_i [x_i])$ in $\CH_0(X)$
for some $x_i \in X$ closed points
whose residue fields are separable over $k$ by
Lemma \ref{lemma-generated-by-separable}.
Then
$$
\gamma([x]) - \gamma([x']) = n (\sum m_i \gamma([x_i]))
$$
in $H^{2d}(X)(d)$. Denote $i^*, (i')^*, i_i^*$ the pullback maps
$H^0(X) \to H^0(x)$, $H^0(X) \to H^0(x')$, $H^0(X) \to H^0(x_i)$.
Recall that $H^0(x)$ is a finite separable $F$-algebra
and that $\int_x : H^0(x) \to F$ is the trace map
(Lemma \ref{lemma-dim-0}) which we will denote $\text{Tr}_x$.
Similarly for $x'$ and $x_i$. Then by Poincar\'e duality in the form of
axiom (A)(b) the equation above is dual to
$$
\text{Tr}_x \circ i^* - \text{Tr}_{x'} \circ (i')^* =
n(\sum m_i \text{Tr}_{x_i} \circ i_i^*)
$$
which takes place in $\Hom_F(H^0(X), F)$. Finally, observe that
$i^*$ and $(i')^*$ are surjective as $x$ and $x'$ are $k'$-rational
points and hence the compositions $H^0(\Spec(k')) \to H^0(X) \to H^0(x)$
and $H^0(\Spec(k')) \to H^0(X) \to H^0(x')$ are isomorphisms.
By Lemma \ref{lemma-algebra-relations} we conclude that $i^* = (i')^*$
which contradicts the assumption that $\gamma([x]) \not = \gamma([x'])$.
\end{proof}

\begin{lemma}
\label{lemma-classes-points}
Assume given (D1), (D2), (D3) and (D4) satisfying (A), (B), and (C).
Let $k'/k$ be a finite separable extension. Let $X$ be a geometrically
irreducible smooth projective scheme over $k'$ of dimension $d$.
Then $\gamma : \CH_0(X) \to H^{2d}(X)(d)$ factors through
$\deg : \CH_0(X) \to \mathbf{Z}$.
\end{lemma}

\begin{proof}
By Lemma \ref{lemma-generated-by-separable} it suffices to show: given
closed points $x, x' \in X$ whose residue fields are separable over $k$
we have $\deg(x') \gamma([x]) = \deg(x) \gamma([x'])$.

\medskip\noindent
We first reduce to the case of $k'$-rational points. Let $k''/k'$ be a
Galois extension such that $\kappa(x)$ and $\kappa(x')$ embed into $k''$
over $k$. Set $Y = X \times_{\Spec(k')} \Spec(k'')$ and denote $p : Y \to X$
the projection. By our choice of $k''/k'$ there exists a
$k''$-rational point $y$, resp.\ $y'$ on $Y$ mapping to $x$, resp.\ $x'$.
Then $p_*[y] = [k'' : \kappa(x)][x]$ and
$p_*[y'] = [k'' : \kappa(x')][x']$ in $\CH_0(X)$.
By compatibility with pushforwards given in axiom (C)(b)
it suffices to prove $\gamma([y]) = \gamma([y'])$ in $\CH^{2d}(Y)(d)$.
This reduces us to the discussion in the next paragraph.

\medskip\noindent
Assume $x$ and $x'$ are $k'$-rational points. By
Lemma \ref{lemma-divide-difference-points} there
exists a finite separable extension $k''/k'$ of fields
such that the pullback $[y] - [y']$
of the difference $[x] - [x']$ becomes divisible
by an integer $n > 1$ on $Y = X \times_{\Spec(k')} \Spec(k'')$.
(Note that $y, y' \in Y$ are $k''$-rational points.)
By Lemma \ref{lemma-relations-classes-points} we have
$\gamma([y]) = \gamma([y'])$ in $H^{2d}(Y)(d)$.
By compatibility with pushforward in axiom (C)(b)
we conclude the same for $x$ and $x'$.
\end{proof}

\begin{lemma}
\label{lemma-injective}
Assume given (D1), (D2), (D3) and (D4) satisfying (A), (B), and (C). Let
$f : X \to Y$ be a dominant morphism of irreducible smooth projective schemes
over $k$. Then $H^*(Y) \to H^*(X)$ is injective.
\end{lemma}

\begin{proof}
There exists an integral closed subscheme $Z \subset X$ of the same
dimension as $Y$ mapping onto $Y$. Thus $f_*[Z] = m[Y]$ for some $m > 0$.
Then $f_* \gamma([Z]) = m$ in $H^*(Y)$. Hence by the projection formula
(Lemma \ref{lemma-pushforward})
we have $f_*(f^*a \cup \gamma([Z])) = m a$ and we conclude.
\end{proof}

\begin{lemma}
\label{lemma-otimes}
Assume given (D1), (D2), (D3) and (D4) satisfying (A), (B), and (C). Let
$k''/k'/k$ be finite separable algebras and let $X$ be a
smooth projective scheme over $k'$. Then
$$
H^*(X) \otimes_{H^0(\Spec(k'))} H^0(\Spec(k'')) =
H^*(X \times_{\Spec(k')} \Spec(k''))
$$
\end{lemma}

\begin{proof}
We will use the results of Lemma \ref{lemma-dim-0} without further mention.
Write
$$
k' \otimes_k k'' = k'' \times l
$$
for some finite separable $k'$-algebra $l$. Write
$F' = H^0(\Spec(k'))$, $F'' = H^0(\Spec(k''))$, and $G = H^0(\Spec(l))$.
Since $\Spec(k') \times \Spec(k'') = \Spec(k'') \amalg \Spec(l)$ we
deduce from axiom (B)(a) and Lemma \ref{lemma-weil-additive}
that we have
$$
F' \otimes_F F'' = F'' \times G
$$
The map from left to right identifies $F''$ with $F' \otimes_{F'} F''$.
By the same token we have
$$
H^*(X) \otimes_F F'' = H^*(X \times_{\Spec(k')} \Spec(k''))
\times H^*(X \times_{\Spec(k')} \Spec(l))
$$
as modules over $F' \otimes_F F'' = F'' \times G$. This proves the lemma.
\end{proof}






















\section{Weil cohomology theories, II}
\label{section-old}

\noindent
For us a Weil cohomology theory will be the analogue of a
classical Weil cohomology theory (Section \ref{section-axioms-classical})
when the ground field $k$ is not algebraically closed.
In Section \ref{section-axioms} we listed axioms which guarantee
our cohomology theory comes from a symmetric monoidal functor
on the category of motives over $k$. Missing from our axioms so
far are the condition $H^i(X) = 0$ for $i < 0$ and
a condition on $H^{2d}(X)(d)$ for $X$ equidimensional of dimension $d$
corresponding to the classical axioms (A)(a) and (A)(d).
Let us first convince the reader that it is necessary to impose
such conditions.

\begin{example}
\label{example-weird-weil}
Let $k = \mathbf{C}$ and $F = \mathbf{C}$ both be equal to the field
of complex numbers. For $X$ smooth projective over $k$ denote
$H^{p, q}(X) = H^q(X, \Omega^p_{X/k})$. Let $(H')^*$ be the functor
which sends $X$ to $(H')^*(X) = \bigoplus H^{p, q}(X)$ with the
usual cup product.
This is a classical Weil cohomology theory (insert future reference here).
By Proposition \ref{proposition-weil-cohomology-theory-classical}
we obtain a $\mathbf{Q}$-linear symmetric monoidal functor $G'$ from $M_k$
to the category of graded $F$-vector spaces. Of course, in this case
for every $M$ in $M_k$ the value $G'(M)$ is naturally bigraded, i.e.,
we have
$$
(G')(M) = \bigoplus (G')^{p, q}(M),\quad
(G')^n = \bigoplus\nolimits_{n = p + q} (G')^{p, q}(M)
$$
with $(G')^{p, q}$ sitting in total degree $p + q$ as indicated.
Now we are going to construct a $\mathbf{Q}$-linear symmetric monoidal functor
$G$ to the category of graded $F$-vector spaces by setting
$$
G^n(M) = \bigoplus_{n = 3p - q} (G')^{p, q}(M)
$$
We omit the verification that this defines a symmetric monoidal
functor (a technical point is that because we chose odd numbers
$3$ and $-1$ above the functor $G$ is compatible with the
commutativity constraints).
Observe that $G(\mathbf{1}(1))$ is still sitting in degree $-2$!
Hence by Lemma \ref{lemma-from-functor-to-weil-classical}
we obtain a functor $H^*$, cycle classes $\gamma$, and trace maps
satisfying all classical axioms (A), (B), (C), except for possibly
the classical axioms (A)(a) and (A)(d).
However, if $E$ is an elliptic curve over $k$, then we find
$\dim H^{-1}(E) = 1$, i.e., axiom (A)(a) is indeed violated.
\end{example}

\begin{lemma}
\label{lemma-H-0-separable}
Assume given (D1), (D2), (D3) and (D4) satisfying (A), (B), and (C).
Let $X$ be a smooth projective scheme over $k$.
Set $k' = \Gamma(X, \mathcal{O}_X)$. The following are equivalent
\begin{enumerate}
\item there exist finitely many closed points $x_1, \ldots, x_r \in X$
whose residue fields are separable over $k$ such that
$H^0(X) \to H^0(x_1) \oplus \ldots \oplus H^0(x_r)$ is injective,
\item the map $H^0(\Spec(k')) \to H^0(X)$ is an isomorphism.
\end{enumerate}
If $X$ is equidimensional of dimension $d$, these are also equivalent to
\begin{enumerate}
\item[(3)] the classes of closed points generate $H^{2d}(X)(d)$
as a module over $H^0(X)$.
\end{enumerate}
If this is true, then $H^0(X)$ is a finite separable algebra over $F$.
\end{lemma}

\begin{proof}
By decomposing $X$ into its irreducible components we may assume $X$
is irreducible and in particular equidimensional of dimension $d$.
This also implies that $k'$ is a field finite separable over $k$
and that $X$ is geometrically irreducible over $k'$. We omit the details.

\medskip\noindent
By Lemma \ref{lemma-generated-by-separable} we see that the closed
points in (3) may be assumed to have separable residue fields over $k$.
By axioms (A)(a) and (A)(b) we see that conditions (1) and (3) are equivalent.

\medskip\noindent
If (2) holds, then pick any closed point $x \in X$ whose residue field
is finite separable over $k'$. Then
$H^0(\Spec(k')) = H^0(X) \to H^0(x)$ is injective for example by
Lemma \ref{lemma-injective}.

\medskip\noindent
Assume the equivalent conditions (1) and (3) hold. Choose
$x_1, \ldots, x_r \in X$ as in (1). Choose a finite separable
extension $k''/k'$. By Lemma \ref{lemma-otimes} we have
$$
H^0(X) \otimes_{H^0(\Spec(k'))} H^0(\Spec(k'')) =
H^0(X \times_{\Spec(k')} \Spec(k''))
$$
Thus in order to show that
$H^0(\Spec(k')) \to H^0(X)$ is an isomorphism
we may replace $k'$ by $k''$. Thus we may assume $x_1, \ldots, x_r$
are $k'$-rational points (this replaces each $x_i$ with multiple
points, so $r$ is increased in this step). By Lemma \ref{lemma-classes-points}
$\gamma(x_1) = \gamma(x_2) = \ldots = \gamma(x_r)$.
By axiom (A)(b) all the maps $H^0(X) \to H^0(x_i)$
are the same. This means (2) holds.

\medskip\noindent
Finally, Lemma \ref{lemma-dim-0} implies
$H^0(X)$ is a separable $F$-algebra if (1) holds.
\end{proof}

\begin{lemma}
\label{lemma-negative-cohomology}
Assume given (D1), (D2), (D3) and (D4) satisfying (A), (B), and (C).
If there exists a smooth projective scheme $Y$ over $k$ such that
$H^i(Y)$ is nonzero for some $i < 0$, then there exists an equidimensional
smooth projective scheme $X$ over $k$ such that the equivalent conditions
of Lemma \ref{lemma-H-0-separable} fail for $X$.
\end{lemma}

\begin{proof}
By Lemma \ref{lemma-weil-additive} we may assume $Y$ is irreducible
and a fortiori equidimensional. If $i$ is odd, then after replacing
$Y$ by $Y \times Y$ we find an example where $Y$ is equidimensional
and $i = -2l$ for some $l > 0$. Set $X = Y \times (\mathbf{P}^1_k)^l$.
Using axiom (B)(a) we obtain
$$
H^0(X) \supset H^0(Y) \oplus
H^i(Y) \otimes_F H^2(\mathbf{P}^1_k)^{\otimes_F l}
$$
with both summands nonzero. Thus it is clear that $H^0(X)$ cannot be
isomorphic to $H^0$ of the spectrum of
$\Gamma(X, \mathcal{O}_X) = \Gamma(Y, \mathcal{O}_Y)$
as this falls into the first summand.
\end{proof}

\noindent
Thus it makes sense to finally make the following definition.

\begin{definition}
\label{definition-weil-cohomology-theory}
Let $k$ be a field. Let $F$ be a field of characteristic $0$.
A {\it Weil cohomology theory} over $k$ with coefficients in $F$
is given by data (D1), (D2), (D3), and (D4) satisfying
Poincar\'e duality, the K\"unneth formula, and compatibility
with cycle classes, more precisely, satisfying axioms (A), (B), and (C)
of Section \ref{section-axioms}
and in addition such that the equivalent conditions (1) and (2) of
Lemma \ref{lemma-H-0-separable} hold for every smooth projective $X$ over $k$.
\end{definition}

\noindent
By Lemma \ref{lemma-negative-cohomology} this means also that there are no
nonzero negative cohomology groups. In particular, if $k$ is algebraically
closed, then a Weil cohomology theory as above together with an isomorphism
$F \to F(1)$ is the same thing as a classical Weil cohomology theory.

\begin{remark}
\label{remark-betti-numbers-in-some-sense}
Let $H^*$ be a Weil cohomology theory
(Definition \ref{definition-weil-cohomology-theory}).
Let $X$ be a geometrically irreducible smooth projective scheme
of dimension $d$ over $k'$ with $k'/k$ a finite separable extension of fields.
Suppose that
$$
H^0(\Spec(k')) = F_1 \times \ldots \times F_r
$$
for some fields $F_i$. Then we accordingly can write
$$
H^*(X) = \prod\nolimits_{i = 1, \ldots, r}
H^*(X) \otimes_{H^0(\Spec(k'))} F_i
$$
Now, our final assumption in Definition \ref{definition-weil-cohomology-theory}
tells us that $H^0(X)$ is free of rank $1$ over $\prod F_i$.
In other words, each of the factors
$H^0(X) \otimes_{H^0(\Spec(k'))} F_i$ has dimension $1$ over $F_i$.
Poincar\'e duality then tells us that the same is true for
cohomology in degree $2d$.
What isn't clear however is that the same holds in other degrees.
Namely, we don't know that given $0 < n < \dim(X)$ the integers
$$
\dim_{F_i} H^n(X) \otimes_{H^0(\Spec(k'))} F_i
$$
are independent of $i$! This question appears to be closely related
to the question of whether betti numbers of smooth projective
varieties are independent of the choice of a Weil cohomology theory.
\end{remark}






\section{Weil cohomology theories, III}
\label{section-c1}

\noindent
Let $k$ be a field. Let $F$ be a field of characteristic zero.
Suppose we are given the following data
\begin{enumerate}
\item[(D1)] A contravariant functor $H^*(-)$ from the category of smooth
projective schemes over $k$ to the category of graded commutative
$F$-algebras.
\item[(D2)] A $1$-dimensional $F$-vector space $F(1)$.
\item[(D2)] For every smooth projective scheme $X$ a homomorphism
$c_1^H : \Pic(X) \to H^2(X)(1)$ of abelian groups which is compatible
with pullbacks.
\end{enumerate}
We will use the terminology, notation, and conventions regarding
these data as discussed in Section \ref{section-axioms}.

\medskip\noindent
The first axiom is compatibility with disjoint unions
\begin{enumerate}
\item[(A1)] $H^*(-)$ is compatible with finite coproducts
\end{enumerate}
This means precisely that $H^*(\emptyset) = 0$ and that
$(i^*, j^*) : H^*(X \amalg Y) \to H^*(X) \times H^*(Y)$
is an isomorphism where $i$ and $j$ are the coprojections.

\medskip\noindent
The second axiom is the K\"unneth formula
\begin{enumerate}
\item[(A2)] $H^*(-)$ is compatible with finite products
\end{enumerate}
This means precisely that $H^*(\Spec(k)) = F$ and that
$H^*(X) \otimes_F H^*(Y) \to H^*(X \times Y)$,
$a \otimes b \mapsto p^*(a) \cup q^*(b)$ is an isomorphism
where $p$ and $q$ are the projections.

\medskip\noindent
The third axiom is the projective space bundle formula.
Let $X$ be a smooth projective scheme over $k$.
Let $\mathcal{E}$ be a locally free $\mathcal{O}_X$-module of rank $r$.
The projective bundle associated to $\mathcal{E}$ is the morphism
$$
\xymatrix{
P = \mathbf{P}(\mathcal{E}) =
\underline{\text{Proj}}_X(\text{Sym}^*(\mathcal{E}))
\ar[r]^-p
& X
}
$$
Recall that $p_*\mathcal{O}_P(1) = \mathcal{E}$.
Denote $\xi = c_1(\mathcal{O}_P(1)) \cap [P] \in \CH^1(P)$ the first
chern class. Set $c = c_1^H(\xi) \in H^2(P)(1)$.
\begin{enumerate}
\item[(A3)] In the situation above the map
$$
\bigoplus\nolimits_{i = 0, \ldots, r - 1} H^*(X)(-i)
\longrightarrow H^*(P),\quad
(a_0, \ldots, a_{r - 1}) \longmapsto \sum c^i \cup p^*(a_i)
$$
is an isomorphism of $F$-vector spaces.
\end{enumerate}

\medskip\noindent
Given data (D1), (D2), (D3) and axioms (A1), (A2), (A3) we can define
chern classes of virtual vector bundles on $X$ smooth and projective
using the splitting principle (FIXME). Let's denote this
$$
ch^H : K_0(X) \longrightarrow \bigoplus\nolimits_{i \geq 0} H^{2i}(X)(i)
$$
Using the isomorphism
$$
ch : K_0(X) \otimes \mathbf{Q} \longrightarrow \CH_*(X) \otimes \mathbf{Q}
$$
see Chow Homology, Lemma \ref{}
we can define cohomology classes of cycles
$$
\gamma : \CH_*(X) \longrightarrow H^*(X)
$$
by the formula
$$
\gamma(\alpha) = ch^H(ch^{-1}(\alpha))
$$
It follows from the construction that $\gamma$ is compatible with
intersection products and compatible with pullbacks.
It isn't trivial that this preserves gradings. We will prove it does
using Adams operations.

\medskip\noindent
Let $\Psi_2 : K_0(X) \to K_0(X)$ be the second Adams operator; it is given by
$$
[\mathcal{E}] \longmapsto [\text{Sym}^2(\mathcal{E})] - [\wedge^2(\mathcal{E})]
$$
where $\mathcal{E}$ is a finite locally free module on $X$. This is
well defined (FIXME). Using the splitting principle we see (FIXME) that
for $\xi \in K_0((X)$ we have
$$
ch_i(\Psi_2(\xi)) = 2^i ch_i(\xi) \in \CH_{\dim(X) - i}(X) \otimes \mathbf{Q}
$$
and
$$
ch^H_i(\Psi_2(\xi)) = 2^i ch^H_i(\xi) \in H^{2i}(X)
$$
Hence, if $\alpha$ is a $k$-cycle on $X$ then it follows that
$$
\Psi_2(ch^{-1}(\alpha)) = 2^{\dim(X) - k} ch^{-1}(\alpha)
$$
because $ch$ is an isomorphism! Applying $ch^H$ to this we conclude that
$\gamma(\alpha) \in H^{2\dim(X) - 2i}(X)$ as desired.

\medskip\noindent
Class of the diagonal. Let $X$ be equidimensional of dimension $d$.
We may write the class of the diagonal
$$
\gamma([\Delta]) = \sum \eta_i \in
H^*(X \times X)(d) = \bigoplus\nolimits_i H^i(X) \otimes_F H^{2d - i}(X)(d)
$$
The fourth axiom is as follows
\begin{enumerate}
\item[(A4)] For $X$ as above
there exists an $F$-linear map $\lambda : H^{2d}(X)(d) \to F$
such that $(1 \otimes \lambda)(\eta_0) =  1$ in $H^0(X)$.
\end{enumerate}
We will sometimes write this as $(1 \otimes \lambda)(\eta) = 1$
where extend $1 \otimes \lambda$ in the obvious manner.
The map $\lambda$ will turn out to be unique and will be our trace map.

\medskip\noindent
The fifth, sixth, and seventh axioms are as follows
\begin{enumerate}
\item[(A5)] Let $X$ be a smooth projective scheme over $k$ equidimensional
of dimension $d$. Let $i : Y \to X$ be an effective Cartier divisor smooth
over $k$. Set $c = c_1^H(\mathcal{O}_X(Y))$ in $H^2(X)(1)$. Then
\begin{enumerate}
\item for $a \in H^*(X)$ we have $i^*(a) = 0 \Rightarrow a \cup c = 0$, and
\item for $a \in H^{2d - 2}(X)(d - 1)$ we have
$\lambda_Y(i^*(a)) = \lambda_X(a \cup c)$ where
$\lambda_Y$ and $\lambda_X$ are as in axiom (A4) for $X$ and $Y$.
\end{enumerate}
\item[(A6)] If $b : X' \to X$ is the blowing up of a smooth
center in a smooth projective scheme over $k$, then
$b^* : H^*(X) \to H^*(X')$ is injective.
\item[(A7)] If $X$ is a smooth projective scheme over $k$ and
$k' = \Gamma(X, \mathcal{O}_X)$, then the map $H^0(\Spec(k')) \to H^0(X)$
is an isomorphism.
\end{enumerate}
We will show that data (D1), (D2), and (D3) satisfying axioms
(A1) -- (A7) give rise to a Weil cohomology theory.

\begin{lemma}
\label{lemma-good-blowing-up}
Let $b : X' \to X$ be the blowing up of a Noetherian scheme
in a regularly embedded closed subscheme $Z \subset X$ of codimension $r$.
Picture
$$
\xymatrix{
E \ar[r]_j \ar[d]_\pi & X' \ar[d]^b \\
Z \ar[r]^i & X
}
$$
Assume there exists an element of $K_0(X)$ whose restriction to
$Z$ is equal to the class of $\mathcal{C}_{Z/X}$ in $K_0(Z)$.
Then we have
$$
Lb^*\mathcal{O}_Z =
(\mathcal{O}_{X'} - \mathcal{O}_{X'}(-E))
\otimes_{\mathcal{O}_{X'}} \mathcal{V}'
$$
in $K_0(\textit{Coh}(X'))$ for some $\mathcal{V}' \in K_0(X')$
where $E \subset X'$ is the exceptional divisor.
\end{lemma}

\begin{proof}
Consider the short exact sequence
$$
0 \to \mathcal{F} \to \pi^*\mathcal{C}_{Z/X} \to \mathcal{C}_{E/X'} \to 0
$$
of finite locally free $\mathcal{O}_E$-modules defining $\mathcal{F}$.
Observe that $\mathcal{C}_{E/X'} = \mathcal{O}_{X'}(-E)|_E = \mathcal{O}_E(1)$
if we think of $\pi$ as a projective space bundle of $\mathcal{C}_{Z/X}$.
A computation shows that $H^{-i}(Lb^*\mathcal{O}_Z) = \wedge^i\mathcal{F}$
for $i = 0, 1, \ldots, r - 1$ and zero in other degrees. It follows that
$$
\wedge^i\mathcal{F} =
\mathcal{O}_E \otimes_{\mathcal{O}_{X'}}
\wedge^i(b^*\mathcal{V} - \mathcal{O}_{X'}(-E))
$$
in $K_0(\textit{Coh}(X'))$ where $\mathcal{V} \in K_0(X)$
is an element restricting to the class of $\mathcal{C}_{Z/X}$.
This proves the lemma with $\mathcal{V}' =
\sum (-1)^i[\wedge^i(b^*\mathcal{V} - \mathcal{O}_{X'}(-E))]$.
\end{proof}

\begin{lemma}
\label{lemma-divide-pullback-good-blowing-up}
In Lemma \ref{lemma-good-blowing-up} assume $X$ is locally of finite type
over $(S, \delta)$ as in Chow Homology, Situation \ref{chow-situation-setup}.
Then there exists a cycle $\alpha' \in \CH^{r - 1}(X')$ such that
$b^*[Z] = [E] \cdot \alpha'$ and $\pi_*j^*(\alpha') = [Z]$ up to torsion.
\end{lemma}

\begin{proof}
With notation as in the proof of Lemma \ref{lemma-good-blowing-up}.
Recall that $[Z] = ch_r(\mathcal{O}_Z) \cap [X]$, see
Chow Homology, Lemma \ref{chow-lemma-actual-computation}.
Hence
$$
b^*[Z] = b^*ch_r(\mathcal{O}_Z) = ch_r(Lb^*\mathcal{O}_Z)
$$
Using the expression in the proof of Lemma \ref{lemma-good-blowing-up}
we see
$$
ch(Lb^*\mathcal{O}_Z) =
ch(\mathcal{O}_E) \cup
ch(\sum (-1)^i\wedge^i(b^*\mathcal{V} - \mathcal{O}_{X'}(-E)))
$$
we have
$$
ch(\mathcal{O}_E) = ch(\mathcal{O}_{X'}) - ch(\mathcal{O}_{X'}(-E)) =
[E] - (1/2)[E]^2 + (1/6)[E]^3 - \ldots
$$
we can indeed formally ``divide'' the expression above by $[E]$
to get $\alpha'$. The restriction of $\alpha'$ to $E$ is
\begin{align*}
& (1 - (1/2)j^*[E] + (1/6)j^*[E]^2 - \ldots) \cup
ch(\sum (-1)^i\wedge^i(\pi^*\mathcal{V} - \mathcal{C}_{E/X'})) \\
& =
(1 - (1/2)j^*[E] + (1/6)j^*[E]^2 - \ldots) \cup
ch(\sum (-1)^i\wedge^i(\mathcal{F})) \\
& =
(1 - (1/2)j^*[E] + (1/6)j^*[E]^2 - \ldots) \cup
((-1)^{r - 1}c_{r - 1}(\mathcal{F}) + \ldots) \\
& = (-1)^{r - 1}c_{r - 1}(\mathcal{F}) + \ldots
\end{align*}
by a computation similar to the proof of
Chow Homology, Lemma \ref{chow-lemma-compute-koszul}.
To prove that $\pi_*$ of this is equal to $[Z]$ it
suffices to prove that the degree of the codimension $r - 1$ cycle
$(-1)^{r - 1}c_{r - 1}(\mathcal{F})$ on the fibres of $\pi$ is $1$.
This is a computation we omit.
\end{proof}

\begin{lemma}
\label{lemma-A5-A6-imply}
Assume given data (D1), (D2), and (D3) satisfying axioms (A1) -- (A6).
Let $X$ be a smooth projective scheme over $k$. Let $Z \subset X$ be a
smooth closed subscheme. Assume the class of $\mathcal{C}_{Z/X}$ in $K_0(Z)$
is the restriction of an element of $K_0(X)$.
If $a \in H^*(X)$ and $a|_Z = 0$ in $H^*(Z)$, then
$\gamma([Z]) \cup a = 0$.
\end{lemma}

\begin{proof}
Let $b : X' \to X$ be the blowing up. By (A6) it suffices to show
that
$$
b^*(\gamma([Z]) \cup a) = b^*\gamma([Z]) \cup b^*a = 0
$$
Say $Z$ has codimension $r$. By
Lemma \ref{lemma-divide-pullback-good-blowing-up} we have
$$
b^*\gamma([Z]) = \gamma(b^*[Z]) =
\gamma([E] \cup \alpha') =
\gamma([E]) \cup \gamma(\alpha')
$$
Hence because $b^*a$ restricts to zero on $E$ we get what we want from (A5).
\end{proof}

\begin{lemma}
\label{lemma-poincare-duality}
Assume given data (D1), (D2), and (D3) satisfying axioms (A1) -- (A6).
Then axiom (A) holds with $\int_X = \lambda$ as in axiom (A4).
\end{lemma}

\begin{proof}
Let $X$ be a smooth projective scheme over $k$ which is equidimensional
of dimension $d$. We will show that the graded $F$-vector space
$H^*(X)(d)[2d]$ is a left dual to $H^*(X)$. This will prove what we want by
Lemma \ref{lemma-left-dual-graded-vector-spaces}. We are
going to use axiom (A2) which in particular says that
$$
H^*(X \times X)(d) =
\bigoplus\nolimits_i H^i(X) \otimes H^{2d - i}(X)(d) =
\bigoplus\nolimits_j H^{2d - j}(X)(d) \otimes H^j(X)
$$
Define a map
$$
\eta : F \longrightarrow H^{2d}(X \times X)(d)
$$
by multiplying by $\gamma([\Delta]) = \sum \eta_i$ with notation as in
axiom (A4). On the other hand, define a map
$$
\epsilon :
H^{2d}(X \times X)(d) \longrightarrow H^{2d}(X)(d) \xrightarrow{\lambda} F
$$
by first using pullback $\Delta^*$ by the diagonal morphism
$\Delta : X \to X \times X$ and then using the $F$-linear map
$\lambda : H^{2d}(X)(d) \to F$ of axiom (A4).
In order to show that $H^*(X)(d)$ is a left dual to $H^*(X)$
we have to show that the composition of the maps
$$
\eta \otimes 1 :
H^*(X) \longrightarrow H^*(X \times X \times X)(d)
$$
and
$$
1 \otimes \epsilon : H^*(X \times X \times X)(d) \longrightarrow H^*(X)
$$
is the identity. If $a \in H^*(X)$ then we see
that the composition maps $a$ to
$$
(1 \otimes \lambda)(\Delta_{23}^*(q_{12}^*\eta \cup q_3^*a)) =
(1 \otimes \lambda)(\eta \cup p_2^*a)
$$
where $q_i : X \times X \times X \to X$ and
$q_{ij} : X \times X \times X \to X \times X$ are the projections,
$\Delta_{23} : X \times X \to X \times X \times X$ is the diagonal, and
$p_i : X \times X \to X$ are the projections.
Since $\eta \cup p_1^*a = \eta \cup p_2^*a$
(see below) the above simplifies to
$$
(1 \otimes \lambda)(\eta \cup p_1^*a) = a
$$
by our choice of $\lambda$ as desired. The second condition
$(\epsilon \otimes 1) \circ (1 \otimes \eta) = \text{id}$
of Definition \ref{definition-dual} is proved in exactly the
same manner.

\medskip\noindent
Note that $p_1^*a$ and $\text{pr}_2^*a$ restrict to the same
cohomology class on $\Delta$. Moreover we
have $\mathcal{C}_{\Delta/X \times X} = \Omega^1_\Delta$ which
is the restriction of $p_1^*\Omega^1_X$. Hence
Lemma \ref{lemma-A5-A6-imply} implies $\eta \cup p_1^*a = \eta \cup p_2^*a$
and the proof is complete.
\end{proof}

\begin{remark}[Uniqueness of trace maps]
\label{remark-trace}
Let $X$ and $\eta = \sum \eta_i$
be as in the proof of Lemma \ref{lemma-poincare-duality}.
Then we know that $\eta_i \in H^i(X) \otimes H^{2d - i}(X)(d)$
defines a perfect duality between $H^i(X)$ and $H^{2d - i}(X)(d)$
for all $i$, see proof of Lemma \ref{lemma-left-dual-graded-vector-spaces}.
In particular, the linear map $\int_X = \lambda : H^{2d}(X)(d) \to F$ of
axiom (A4) is unique!
\end{remark}

\begin{lemma}
\label{lemma-trace-product}
Assume given data (D1), (D2), and (D3) satisfying axioms (A1) -- (A6).
Then axiom (B) holds.
\end{lemma}

\begin{proof}
Axiom (B)(a) is immediate from axiom (A2).
Let $X$ and $Y$ be smooth projective schemes equidimensional
of dimensions $d$ and $e$. To see that axiom (B)(b)
holds, observe that the diagonal $\Delta_{X \times Y}$ of $X \times Y$
is the intersection product of the pullbacks of the diagonals
$\Delta_X$ of $X$ and $\Delta_Y$ of $Y$ by the projections
$p : X \times Y \times X \times Y \to X \times X$ and
$q : X \times Y \times X \times Y \to Y \times Y$.
Compatibility of $\gamma$ with intersection products then gives
that
$$
\gamma([\Delta_{X \times Y}]) = \eta_{X \times Y} \in
H^{2d + 2e}(X \times Y \times X \times Y)(d + e)
$$
is the cup product of the pullbacks of $\eta_X = \gamma([\Delta_X])$
and $\eta_Y = \gamma([\Delta_Y])$ by $p$ and $q$. In particular we have
$$
\eta_{X \times Y, 0} =
\sum\nolimits_{i \in \mathbf{Z}} p^*\eta_{X, i} \cup q^*\eta_{Y, -i}
$$
(If our cohomology theory vanishes in negative degrees, which will
be true in almost all cases, then only the term for $i = 0$ contributes
and $\eta_{X \times Y, 0}$ lies in
$H^0(X) \otimes H^0(Y) \otimes H^{2d}(X)(d) \otimes H^{2e}(Y)(e)$ as expected,
but we don't need this.) Since $\lambda_X : H^{2d}(X)(d) \to F$ and 
$\lambda_Y : H^{2e}(Y)(e) \to F$ send $\eta_{X, 0}$ and $\eta_{Y, 0}$
to $1$ in $H^0(X)$ and $H^0(Y)$, we see that $\lambda_X \otimes \lambda_Y$
sends $\eta_{X \times Y, 0}$ to $1$ in
$H^0(X) \otimes H^0(Y) \subset H^0(X \times Y)$ and the proof is complete.
\end{proof}

\begin{lemma}
\label{lemma-trace-base}
Assume given data (D1), (D2), and (D3) satisfying axioms (A1) -- (A6).
Then axiom (C)(d) holds.
\end{lemma}

\begin{proof}
We have $\gamma([\Spec(k)]) = 1 \in H^*(\Spec(k))$ by construction.
Since
$$
H^0(\Spec(k)) = F,\quad
H^0(\Spec(k) \times \Spec(k)) = H^0(\Spec(k)) \otimes_F H^0(\Spec(k))
$$
the map $\int_{\Spec(k)} = \lambda$ of axiom (A4) must send $1$ to $1$.
\end{proof}

\noindent
Assume given data (D1), (D2), and (D3) satisfying axioms (A1) -- (A6).
Then we obtain data (D1), (D2), (D3), and (D4) of
Section \ref{section-axioms}
satisfying axioms (A), (B) and (C)(a), (C)(c), and (C)(d), see
Lemmas \ref{lemma-poincare-duality}, \ref{lemma-trace-product}, and
\ref{lemma-trace-base}.
Moreover, we have the pushforwards $f_* : H^*(X) \to H^*(Y)$
as constructed in Section \ref{section-axioms}. The only axiom of
Section \ref{section-axioms}
which isn't clear yet is axiom (C)(b).

\begin{lemma}
\label{lemma-ok-for-projective-bundle}
Assume given data (D1), (D2), and (D3) satisfying axioms (A1) -- (A6).
Let $p : P \to X$ be as in axiom (A3) with $X$ equidimensional.
Then $\gamma$ commutes with pushforward along $p$.
\end{lemma}

\begin{proof}
It suffices to prove this on generators for $\CH_*(P)$.
Thus it suffices to prove this for a cycle class of the
form $\xi^i \cdot p^*\alpha$ where $0 \leq i \leq r - 1$
and $\alpha \in \CH_*(X)$. Note that $p_*(\xi^i \cdot p^*\alpha) = 0$
if $i < r - 1$ and $p_*(\xi^{r - 1} \cdot p^*\alpha = \alpha$.
On the other hand, we have
$\gamma(\xi^i \cdot p^*\alpha) = c^i \cup p^*\gamma(\alpha)$
and by the projection formula (Lemma \ref{lemma-pushforward})
we have
$$
p_*\gamma(\xi^i \cdot p^*\alpha) = p_*(c^i) \cup \gamma(\alpha)
$$
Thus it suffices to show that $p_*c^i = 0$ for $i < r - 1$ and
$p_*c^{r - 1} = 1$. Equivalently, it suffices to prove that
$\lambda_P : H^{2d + 2r - 2}(P)(d + r - 1) \to F$ defined by
the rules
$$
\lambda_P(c^i \cup p^*(a)) =
\left\{
\begin{matrix}
0 & \text{if} & i < r - 1 \\
\int_X(a) & \text{if} & i = r - 1
\end{matrix}
\right.
$$
satisfies the condition of axiom (A4). This follows from the
computation of the class of the diagonal of $P$ in
Lemma \ref{lemma-diagonal-projective-bundle}.
\end{proof}

\begin{lemma}
\label{lemma-enough}
Assume given data (D1), (D2), and (D3) satisfying axioms (A1) -- (A6).
In order to show that $\gamma$ commutes with pushforward it suffices
to show that $i_*(1) = \gamma([Z])$ if $i : Z \to X$ is a closed
immersion of smooth projective equidimensional schemes over $k$.
\end{lemma}

\begin{proof}
Let $f : X \to Y$ be a morphism of equidimensional smooth projective schemes.
We are trying to show $f_*\gamma(\alpha) = \gamma(f_*\alpha)$
for any cycle class $\alpha$ on $X$.
We can write $\alpha$ as a $\mathbf{Q}$-linear combination of products of
chern classes of locally free $\mathcal{O}_X$-modules.
Thus we may assume $\alpha$ is a product of chern classes of
finite locally free $\mathcal{O}_X$-modules
$\mathcal{E}_1, \ldots, \mathcal{E}_r$.
Pick $p : P \to X$ as in the splitting principle
(Chow Homology, Lemma \ref{chow-lemma-splitting-principle}).
Then we see that $N \alpha = p_*(\xi^e \cdot p^*\alpha)$
for some $N > 0$ and $\xi = c_1(\mathcal{L})$ the first
chern class of an ample invertible module (FIXME).
By Lemma \ref{lemma-ok-for-projective-bundle}
we know that we have the desired property for $p_*$.
Thus it suffices to prove the property for the composition
$f \circ p$. Since $p^*\mathcal{E}_1, \ldots, p^*\mathcal{E}_r$
have filtrations whose successive quotients are invertible
modules, this reduces us to the case where $\alpha$ is
of the form $\xi_1 \cap \ldots \cap \xi_t \cap [X]$
for some first chern classes $\xi_i$ if invertible modules $\mathcal{L}_i$.
Since any invertible module is a difference of very ample
invertible modules, this reduces us to the case where
$\mathcal{L}_i$ is very ample.
By Bertini (FIXME), this reduces us to the case
where $\alpha = [Z]$ for some smooth closed subscheme $Z \subset X$.
Choose a closed embedding $X \to \mathbf{P}^n$. We can factor $f$ as
$$
X \to Y \times \mathbf{P}^n \to Y
$$
Since we know the result for the second morphism by
Lemma \ref{lemma-ok-for-projective-bundle}
it suffices to prove the result when
$\alpha = [Z]$ where $i : Z \to X$ is a closed immersion 
and $f$ is a closed immersion.
Then $j = f \circ i$ is a closed embedding too.
Using the hypothesis for $i$ and $j$ we win.
\end{proof}

\begin{lemma}
\label{lemma-grassmanian}
Assume given data (D1), (D2), and (D3) satisfying axioms (A1) -- (A6).
Given integers $0 < l < n$ and an equidimensional
smooth projective scheme $X$ consider the projection morphism
$p : X \times \mathbf{G}(l, n) \to X$.
Then $\gamma$ commutes with pushforward along $p$.
\end{lemma}

\begin{proof}
If $l = 1$ or $l = n - 1$ then $p$ is a projective bundle and
the result follows from Lemma \ref{lemma-ok-for-projective-bundle}.
In general there exists a morphism
$$
h : Y \to X \times \mathbf{G}(l, n)
$$
such that both $h$ and $p \circ h$ are compositions of projective
space bundles. Namely, denote $\mathbf{G}(1, 2, \ldots, l; n)$
the partial flag variety. Then the morphism
$$
\mathbf{G}(1, 2, \ldots, l; n) \to \mathbf{G}(l, n)
$$
is a compostion of projective space bundles and similarly the
structure morphism $\mathbf{G}(1, 2, \ldots, l; n) \to \Spec(k)$
is of this form. Thus we may set $Y = X \times \mathbf{G}(1, 2, \ldots, l; n)$.
Since every cycle on $X \times \mathbf{G}(l, n)$ is the pushforward of
a cycle on $Y$, the result for $Y \to X$ and the result for
$Y \to X \times \mathbf{G}(l, n)$ imply the result for $p$.
\end{proof}

\begin{lemma}
\label{lemma-enough-better}
Assume given data (D1), (D2), and (D3) satisfying axioms (A1) -- (A6).
In order to show that $\gamma$ commutes with pushforward it suffices
to show that $i_*(1) = \gamma([Z])$ if $i : Z \to X$ is a closed
immersion of smooth projective equidimensional schemes over $k$
such that the class of $\mathcal{C}_{Z/X}$ in $K_0(Z)$ is the
pullback of a class in $K_0(X)$.
\end{lemma}

\begin{proof}
By Lemma \ref{lemma-enough} it suffices to show that $i_*(1) = \gamma([Z])$
if $i : Z \to X$ is a closed immersion of smooth projective equidimensional
schemes over $k$. Say $Z$ has codimension $r$ in $X$.
Let $\mathcal{L}$ be a sufficiently ample invertible module on $X$.
Choose $n > 0$ and a surjection
$$
\mathcal{O}_Z^{\oplus n} \to \mathcal{C}_{Z/X} \otimes \mathcal{L}|_Z
$$
This gives a morphism $g : Z \to \mathbf{G}(n - r, n)$
to the Grassmanian over $k$, see
Constructions, Section \ref{constructions-section-grassmannian}.
Consider the composition
$$
Z \to X \times \mathbf{G}(n - r, n) \to X
$$
Pushforward along the second morphism is compatible with classes
of cycles by Lemma \ref{lemma-grassmanian}. The conormal sheaf $\mathcal{C}$
of the closed immersion $Z \to X \times \mathbf{G}(n - r, n)$ sits in
a short exact sequence
$$
0 \to \mathcal{C}_{Z/X} \to \mathcal{C} \to
g^*\Omega_{\mathbf{G}(n - r, n)} \to 0
$$
See More on Morphisms, Lemma
\ref{more-morphisms-lemma-two-unramified-morphisms-formally-smooth}.
Since $\mathcal{C}_{Z/X} \otimes \mathcal{L}|_Z$ is the pull
back of a finite locally free sheaf on $\mathbf{G}(n - r, n)$
we conclude that the class of $\mathcal{C}$ in $K_0(Z)$
is the pullback of a class in $K_0(X \times \mathbf{G}(n - r, n))$.
Hence we have the property for $Z \to X \times \mathbf{G}(n - r, n)$
and we conclude.
\end{proof}

\begin{lemma}
\label{lemma-pushforward-blowup}
Assume given data (D1), (D2), and (D3) satisfying axioms (A1) -- (A7).
Let $b : X' \to X$ be a blowing up of a smooth projective scheme $X$
over $k$ which is equidimensional of dimension $d$ in a smooth center $Z$.
Then $b_*(1) = 1$.
\end{lemma}

\begin{proof}
Set $k' = \Gamma(X, \mathcal{O}_X) = \Gamma(X', \mathcal{O}_{X'})$.
Choose a closed point $x' \in X'$ which isn't contained in the exceptional
divisor and whose residue field $k''$ is separable over $k$.
Denote $x \in X$ the image (whose residue field is equal to $k''$
as well of course). Consider the diagram
$$
\xymatrix{
x' \times X' \ar[r] \ar[d] & X' \times X' \ar[d] \\
x \times X \ar[r] & X \times X
}
$$
The class of the diagonal $\Delta = \Delta_X$ pulls back to the class of the
``diagonal point'' $\delta_x : x \to x \times X$ and similarly for the class of
the diagonal $\Delta'$. On the other hand, the diagonal point $\delta_x$
pulls back to the diagonal point $\delta_{x'}$ by the left vertical arrow.
Denote $\eta = \sum \eta_i \in H^*(X \times X)(d)$ and
$\eta' = \sum \eta'_i \in H^*(X' \times X')$
the class of $\Delta$ and $\Delta'$ as in axiom (A4).
The arguments above show that $\eta_0$ and $\eta'_0$ map to the same
class in
$$
H^0(x') \otimes_F H^{2d}(X')(d)
$$
Since $H^0(\Spec(k')) = H^0(X) = H^0(X')$ by axiom (A7) map injectively
into $H^0(x')$ we conclude that $\eta_0$ maps to $\eta'_0$ by the map
$$
H^0(X) \otimes_F H^{2d}(X)(d)
\longrightarrow
H^0(X') \otimes_F H^{2d}(X')(d)
$$
This means that $\int_X$ is equal to $\int_{X'}$ composed with
the pullback map. This proves the lemma.
\end{proof}

\begin{lemma}
\label{lemma-done}
Assume given data (D1), (D2), and (D3) satisfying axioms (A1) -- (A7).
Then the cycle class map $\gamma$ commutes with pushforward.
\end{lemma}

\begin{proof}
Let $i : Z \to X$ be as in Lemma \ref{lemma-enough-better}. Consider
the diagram
$$
\xymatrix{
E \ar[r]_j \ar[d]_\pi & X' \ar[d]^b \\
Z \ar[r]^i & X
}
$$
Let $\alpha' \in \CH^{r - 1}(X')$ be as in
Lemma \ref{lemma-divide-pullback-good-blowing-up}.
Then $\pi_*j^*\alpha' = [Z]$ in $\CH_*(Z)$ implies that
$\pi_*\gamma(j^*\alpha') = 1$ by Lemma \ref{lemma-ok-for-projective-bundle}
because $\pi$ is a projective space bundle.
Hence we see that
$$
i_*(1) = i_*(\pi_*(\gamma(j^*\alpha'))) =
b_*j_*(j^*\gamma(\alpha')) =
b_*(j_*(1) \cup \gamma(\alpha'))
$$
We have $j_*(1) = \gamma([E])$ by (A5)(b). Thus this is equal to
$$
b_*(\gamma([E]) \cup \gamma(\alpha')) =
b_*(\gamma([E] \cdot \alpha')) =
b_*(\gamma(b^*[Z])) =
b_*b^*\gamma([Z]) = b_*(1) \cup \gamma([Z])
$$
Since $b_*(1) = 1$ by Lemma \ref{lemma-pushforward-blowup} the
proof is complete.
\end{proof}

\begin{proposition}
\label{proposition-get-weil}
Assume given data (D1), (D2), and (D3) satisfying axioms (A1) -- (A7).
Then we have a Weil cohomology theory.
\end{proposition}

\begin{proof}
See discussion above.
\end{proof}












\section{Other chapters}

\begin{multicols}{2}
\begin{enumerate}
\item \hyperref[introduction-section-phantom]{Introduction}
\item \hyperref[conventions-section-phantom]{Conventions}
\item \hyperref[sets-section-phantom]{Set Theory}
\item \hyperref[categories-section-phantom]{Categories}
\item \hyperref[topology-section-phantom]{Topology}
\item \hyperref[sheaves-section-phantom]{Sheaves on Spaces}
\item \hyperref[algebra-section-phantom]{Commutative Algebra}
\item \hyperref[sites-section-phantom]{Sites and Sheaves}
\item \hyperref[homology-section-phantom]{Homological Algebra}
\item \hyperref[derived-section-phantom]{Derived Categories}
\item \hyperref[more-algebra-section-phantom]{More Algebra}
\item \hyperref[simplicial-section-phantom]{Simplicial Methods}
\item \hyperref[modules-section-phantom]{Sheaves of Modules}
\item \hyperref[sites-modules-section-phantom]{Modules on Sites}
\item \hyperref[injectives-section-phantom]{Injectives}
\item \hyperref[cohomology-section-phantom]{Cohomology of Sheaves}
\item \hyperref[sites-cohomology-section-phantom]{Cohomology on Sites}
\item \hyperref[hypercovering-section-phantom]{Hypercoverings}
\item \hyperref[schemes-section-phantom]{Schemes}
\item \hyperref[constructions-section-phantom]{Constructions of Schemes}
\item \hyperref[properties-section-phantom]{Properties of Schemes}
\item \hyperref[morphisms-section-phantom]{Morphisms of Schemes}
\item \hyperref[coherent-section-phantom]{Coherent Cohomology}
\item \hyperref[divisors-section-phantom]{Divisors}
\item \hyperref[limits-section-phantom]{Limits of Schemes}
\item \hyperref[varieties-section-phantom]{Varieties}
\item \hyperref[chow-section-phantom]{Chow Homology}
\item \hyperref[topologies-section-phantom]{Topologies on Schemes}
\item \hyperref[descent-section-phantom]{Descent}
\item \hyperref[more-morphisms-section-phantom]{More on Morphisms}
\item \hyperref[flat-section-phantom]{More on Flatness}
\item \hyperref[groupoids-section-phantom]{Groupoid Schemes}
\item \hyperref[more-groupoids-section-phantom]{More on Groupoid Schemes}
\item \hyperref[etale-section-phantom]{\'Etale Morphisms of Schemes}
\item \hyperref[etale-cohomology-section-phantom]{\'Etale Cohomology}
\item \hyperref[spaces-section-phantom]{Algebraic Spaces}
\item \hyperref[spaces-properties-section-phantom]{Properties of Algebraic Spaces}
\item \hyperref[spaces-morphisms-section-phantom]{Morphisms of Algebraic Spaces}
\item \hyperref[spaces-topologies-section-phantom]{Topologies on Algebraic Spaces}
\item \hyperref[spaces-descent-section-phantom]{Descent and Algebraic Spaces}
\item \hyperref[spaces-more-morphisms-section-phantom]{More on Morphisms of Spaces}
\item \hyperref[quot-section-phantom]{Quot and Hilbert Spaces}
\item \hyperref[stacks-section-phantom]{Stacks}
\item \hyperref[spaces-groupoids-section-phantom]{Groupoids in Algebraic Spaces}
\item \hyperref[spaces-more-groupoids-section-phantom]{More on Groupoids in Spaces}
\item \hyperref[bootstrap-section-phantom]{Bootstrap}
\item \hyperref[examples-stacks-section-phantom]{Examples of Stacks}
\item \hyperref[groupoids-quotients-section-phantom]{Quotients of Groupoids}
\item \hyperref[algebraic-section-phantom]{Algebraic Stacks}
\item \hyperref[criteria-section-phantom]{Criteria for Representability}
\item \hyperref[stacks-properties-section-phantom]{Properties of Algebraic Stacks}
\item \hyperref[stacks-morphisms-section-phantom]{Morphisms of Algebraic Stacks}
\item \hyperref[examples-section-phantom]{Examples}
\item \hyperref[exercises-section-phantom]{Exercises}
\item \hyperref[guide-section-phantom]{Guide to Literature}
\item \hyperref[desirables-section-phantom]{Desirables}
\item \hyperref[coding-section-phantom]{Coding Style}
\item \hyperref[fdl-section-phantom]{GNU Free Documentation License}
\item \hyperref[index-section-phantom]{Auto Generated Index}
\end{enumerate}
\end{multicols}


\bibliography{my}
\bibliographystyle{amsalpha}

\end{document}
