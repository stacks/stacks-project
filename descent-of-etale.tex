\IfFileExists{stacks-project.cls}{%
\documentclass{stacks-project}
}{%
\documentclass{amsart}
}

% The following AMS packages are automatically loaded with
% the amsart documentclass:
%\usepackage{amsmath}
%\usepackage{amssymb}
%\usepackage{amsthm}

% For dealing with references we use the comment environment
\usepackage{verbatim}
\newenvironment{reference}{\comment}{\endcomment}
%\newenvironment{reference}{}{}
\newenvironment{slogan}{\comment}{\endcomment}
\newenvironment{history}{\comment}{\endcomment}

% For commutative diagrams you can use
% \usepackage{amscd}
\usepackage[all]{xy}

% We use 2cell for 2-commutative diagrams.
\xyoption{2cell}
\UseAllTwocells

% To put source file link in headers.
% Change "template.tex" to "this_filename.tex"
% \usepackage{fancyhdr}
% \pagestyle{fancy}
% \lhead{}
% \chead{}
% \rhead{Source file: \url{template.tex}}
% \lfoot{}
% \cfoot{\thepage}
% \rfoot{}
% \renewcommand{\headrulewidth}{0pt}
% \renewcommand{\footrulewidth}{0pt}
% \renewcommand{\headheight}{12pt}

\usepackage{multicol}

% For cross-file-references
\usepackage{xr-hyper}

% Package for hypertext links:
\usepackage{hyperref}

% For any local file, say "hello.tex" you want to link to please
% use \externaldocument[hello-]{hello}
\externaldocument[introduction-]{introduction}
\externaldocument[conventions-]{conventions}
\externaldocument[sets-]{sets}
\externaldocument[categories-]{categories}
\externaldocument[topology-]{topology}
\externaldocument[sheaves-]{sheaves}
\externaldocument[sites-]{sites}
\externaldocument[stacks-]{stacks}
\externaldocument[fields-]{fields}
\externaldocument[algebra-]{algebra}
\externaldocument[brauer-]{brauer}
\externaldocument[homology-]{homology}
\externaldocument[derived-]{derived}
\externaldocument[simplicial-]{simplicial}
\externaldocument[more-algebra-]{more-algebra}
\externaldocument[smoothing-]{smoothing}
\externaldocument[modules-]{modules}
\externaldocument[sites-modules-]{sites-modules}
\externaldocument[injectives-]{injectives}
\externaldocument[cohomology-]{cohomology}
\externaldocument[sites-cohomology-]{sites-cohomology}
\externaldocument[dga-]{dga}
\externaldocument[dpa-]{dpa}
\externaldocument[hypercovering-]{hypercovering}
\externaldocument[schemes-]{schemes}
\externaldocument[constructions-]{constructions}
\externaldocument[properties-]{properties}
\externaldocument[morphisms-]{morphisms}
\externaldocument[coherent-]{coherent}
\externaldocument[divisors-]{divisors}
\externaldocument[limits-]{limits}
\externaldocument[varieties-]{varieties}
\externaldocument[topologies-]{topologies}
\externaldocument[descent-]{descent}
\externaldocument[perfect-]{perfect}
\externaldocument[more-morphisms-]{more-morphisms}
\externaldocument[flat-]{flat}
\externaldocument[groupoids-]{groupoids}
\externaldocument[more-groupoids-]{more-groupoids}
\externaldocument[etale-]{etale}
\externaldocument[chow-]{chow}
\externaldocument[intersection-]{intersection}
\externaldocument[pic-]{pic}
\externaldocument[adequate-]{adequate}
\externaldocument[dualizing-]{dualizing}
\externaldocument[duality-]{duality}
\externaldocument[discriminant-]{discriminant}
\externaldocument[local-cohomology-]{local-cohomology}
\externaldocument[curves-]{curves}
\externaldocument[resolve-]{resolve}
\externaldocument[models-]{models}
\externaldocument[pione-]{pione}
\externaldocument[etale-cohomology-]{etale-cohomology}
\externaldocument[proetale-]{proetale}
\externaldocument[crystalline-]{crystalline}
\externaldocument[spaces-]{spaces}
\externaldocument[spaces-properties-]{spaces-properties}
\externaldocument[spaces-morphisms-]{spaces-morphisms}
\externaldocument[decent-spaces-]{decent-spaces}
\externaldocument[spaces-cohomology-]{spaces-cohomology}
\externaldocument[spaces-limits-]{spaces-limits}
\externaldocument[spaces-divisors-]{spaces-divisors}
\externaldocument[spaces-over-fields-]{spaces-over-fields}
\externaldocument[spaces-topologies-]{spaces-topologies}
\externaldocument[spaces-descent-]{spaces-descent}
\externaldocument[spaces-perfect-]{spaces-perfect}
\externaldocument[spaces-more-morphisms-]{spaces-more-morphisms}
\externaldocument[spaces-flat-]{spaces-flat}
\externaldocument[spaces-groupoids-]{spaces-groupoids}
\externaldocument[spaces-more-groupoids-]{spaces-more-groupoids}
\externaldocument[bootstrap-]{bootstrap}
\externaldocument[spaces-pushouts-]{spaces-pushouts}
\externaldocument[groupoids-quotients-]{groupoids-quotients}
\externaldocument[spaces-more-cohomology-]{spaces-more-cohomology}
\externaldocument[spaces-simplicial-]{spaces-simplicial}
\externaldocument[formal-spaces-]{formal-spaces}
\externaldocument[restricted-]{restricted}
\externaldocument[spaces-resolve-]{spaces-resolve}
\externaldocument[formal-defos-]{formal-defos}
\externaldocument[defos-]{defos}
\externaldocument[cotangent-]{cotangent}
\externaldocument[examples-defos-]{examples-defos}
\externaldocument[algebraic-]{algebraic}
\externaldocument[examples-stacks-]{examples-stacks}
\externaldocument[stacks-sheaves-]{stacks-sheaves}
\externaldocument[criteria-]{criteria}
\externaldocument[artin-]{artin}
\externaldocument[quot-]{quot}
\externaldocument[stacks-properties-]{stacks-properties}
\externaldocument[stacks-morphisms-]{stacks-morphisms}
\externaldocument[stacks-limits-]{stacks-limits}
\externaldocument[stacks-cohomology-]{stacks-cohomology}
\externaldocument[stacks-perfect-]{stacks-perfect}
\externaldocument[stacks-introduction-]{stacks-introduction}
\externaldocument[stacks-more-morphisms-]{stacks-more-morphisms}
\externaldocument[stacks-geometry-]{stacks-geometry}
\externaldocument[moduli-]{moduli}
\externaldocument[moduli-curves-]{moduli-curves}
\externaldocument[examples-]{examples}
\externaldocument[exercises-]{exercises}
\externaldocument[guide-]{guide}
\externaldocument[desirables-]{desirables}
\externaldocument[coding-]{coding}
\externaldocument[obsolete-]{obsolete}
\externaldocument[fdl-]{fdl}
\externaldocument[index-]{index}

% Theorem environments.
%
\theoremstyle{plain}
\newtheorem{theorem}[subsection]{Theorem}
\newtheorem{proposition}[subsection]{Proposition}
\newtheorem{lemma}[subsection]{Lemma}

\theoremstyle{definition}
\newtheorem{definition}[subsection]{Definition}
\newtheorem{example}[subsection]{Example}
\newtheorem{exercise}[subsection]{Exercise}
\newtheorem{situation}[subsection]{Situation}

\theoremstyle{remark}
\newtheorem{remark}[subsection]{Remark}
\newtheorem{remarks}[subsection]{Remarks}

\numberwithin{equation}{subsection}

% Macros
%
\def\lim{\mathop{\rm lim}\nolimits}
\def\colim{\mathop{\rm colim}\nolimits}
\def\Spec{\mathop{\rm Spec}}
\def\Hom{\mathop{\rm Hom}\nolimits}
\def\Ext{\mathop{\rm Ext}\nolimits}
\def\SheafHom{\mathop{\mathcal{H}\!{\it om}}\nolimits}
\def\SheafExt{\mathop{\mathcal{E}\!{\it xt}}\nolimits}
\def\Sch{\textit{Sch}}
\def\Mor{\mathop{\rm Mor}\nolimits}
\def\Ob{\mathop{\rm Ob}\nolimits}
\def\Sh{\mathop{\textit{Sh}}\nolimits}
\def\NL{\mathop{N\!L}\nolimits}
\def\proetale{{pro\text{-}\acute{e}tale}}
\def\etale{{\acute{e}tale}}
\def\QCoh{\textit{QCoh}}
\def\Ker{\mathop{\rm Ker}}
\def\Im{\mathop{\rm Im}}
\def\Coker{\mathop{\rm Coker}}
\def\Coim{\mathop{\rm Coim}}

%
% Macros for moduli stacks/spaces
%
\def\QCohstack{\mathcal{QC}\!{\it oh}}
\def\Cohstack{\mathcal{C}\!{\it oh}}
\def\Spacesstack{\mathcal{S}\!{\it paces}}
\def\Quotfunctor{{\rm Quot}}
\def\Hilbfunctor{{\rm Hilb}}
\def\Curvesstack{\mathcal{C}\!{\it urves}}
\def\Polarizedstack{\mathcal{P}\!{\it olarized}}
\def\Complexesstack{\mathcal{C}\!{\it omplexes}}
% \Pic is the operator that assigns to X its picard group, usage \Pic(X)
% \Picardstack_{X/B} denotes the Picard stack of X over B
% \Picardfunctor_{X/B} denotes the Picard functor of X over B
\def\Pic{\mathop{\rm Pic}\nolimits}
\def\Picardstack{\mathcal{P}\!{\it ic}}
\def\Picardfunctor{{\rm Pic}}
\def\Deformationcategory{\mathcal{D}\!{\it ef}}


% OK, start here.
%
\begin{document}

\title{Descending schemes}

\maketitle

\tableofcontents

\section{Introduction}
\label{section-introduction}

\noindent
In this chapter we discuss when schemes over schemes
have the property that descent data
are effective. See for example \cite{Gr-I}, \cite{Gr-II}, \cite{Gr-III},
\cite{Gr-IV}, \cite{Gr-V}, and \cite{Gr-VI}.
This is also meant to introduce the notions of
descent, descent data, effective descent data, in the less formal
setting of descent questions for schemes.

\section{Notation}
\label{section-notation}

\noindent
In this chapter we will use the convention where
the projection maps $\text{pr}_i : X \times \ldots \times X \to X$
are labeled starting with $i = 0$. Hence we have
$\text{pr}_0, \text{pr}_1 : X \times X  \to X$,
$\text{pr}_0, \text{pr}_1, \text{pr}_2 : X \times X \times X  \to X$,
etc.

\section{Descent data for schemes over schemes}
\label{section-descent-datum}

\noindent
Most of the arguments in this section are formal relying only
on the definition of a descent datum. In Section \ref{section-simplicial}
we will examine the relationship with simplicial schemes which will
somewhat clarify the situation. Hopefully the reader will be convinced
by the end of Section \ref{section-simplicial} that the language of descent
is awkward and the setting of simplicial schemes is natural for the
questions being considered here.

\begin{definition}
\label{definition-descent-datum}
Let $f : X \to S$ be a morphism of schemes.
\begin{enumerate}
\item Let $V \to X$ be a scheme over $X$.
A {\it descent datum for $V/X/S$} is an isomorphism
$\varphi : V \times_S X \to X \times_S V$ of schemes over
$X \times_S X$ such that the diagram
$$
\xymatrix{
V \times_S X \times_S X \ar[rd]^{\varphi_{01}} \ar[rr]_{\varphi_{02}} &
&
X \times_S X \times_S V\\
&
X \times_S V \times_S X \ar[ru]^{\varphi_{12}}
}
$$
commutes (with obvious notation).
\item We also say that the pair $(V/X, \varphi)$ is
{\it a descent datum relative to $X \to S$}.
\item A {\it morphism $f : (V/X, \varphi) \to (V'/X, \varphi')$ of
descent data relative to $X \to S$} is a morphism
$f : V \to V'$ of schemes over $X$ such that 
the diagram
$$
\xymatrix{
V \times_S X \ar[r]_{\varphi} \ar[d]_{f \times \text{id}_X} &
X \times_S V \ar[d]^{\text{id}_X \times f} \\
V' \times_S X \ar[r]^{\varphi'} & X \times_S V'
}
$$
commutes.
\end{enumerate}
\end{definition}

\noindent
There are all kinds of ``miraculous'' identities which arise out of the
definition above. For example the pullback of $\varphi$ via the diagonal
morphism $\Delta : X \to X \times_S X$ can be seen as a morphism
$\Delta^*\varphi : V \to V$.
This because $X \times_{\Delta, X \times_S X} (V \times_S X) = V$
and also $X \times_{\Delta, X \times_S X} (X \times_S V) = V$.
In fact, $\Delta^*\varphi$ is equal to the identity.
This is a good exercise if you are unfamiliar with this material.

\medskip\noindent
Here is the definition in case you have a family of morphisms
with fixed target.

\begin{definition}
\label{definition-descent-datum-for-family-of-morphisms}
Let $S$ be a scheme.
Let $\{X_i \to S\}_{i \in I}$ be a family of morphisms with target $S$.
\begin{enumerate}
\item A {\it descent datum $(V_i, \varphi_{ij})$ relative to the
family $\{X_i \to S\}$} is given by a scheme $V_i$ over $X_i$
for each $i \in I$, an isomorphism
$\varphi_{ij} : V_i\times_S X_j \to X_i \times_S X_j$
of schemes over $X_i \times_S X_j$ for each pair $(i, j) \in I^2$
such that for every triple of indices $(i, j, k) \in I^3$
the diagram
$$
\xymatrix{
V_i \times_S X_j \times_S X_k
\ar[rd]^{\text{pr}_{01}^*\varphi_{ij}}
\ar[rr]_{\text{pr}_{02}^*\varphi_{ik}} &
&
X_i \times_S X_j \times_S V_k\\
&
X_i \times_S V_j \times_S X_k
\ar[ru]^{\text{pr}_{12}^*\varphi_{jk}}
}
$$
of schemes over $X_i \times_S X_j \times_S X_k$ commutes
(with obvious notation).
\item A {\it morphism
$\psi : (V_i, \varphi_{ij}) \to (V'_i, \varphi'_{ij})$
of descent data} is given by a family
$\psi = (\psi_i)_{i \in I}$ of morphisms of
$X_i$-schemes $\psi_i : V_i \to V'_i$ such that all the diagrams
$$
\xymatrix{
V_i \times_S X_j \ar[r]_{\varphi_{ij}} \ar[d]_{\psi_i \times \text{id}} &
X_i \times_S V_j \ar[d]^{\text{id} \times \psi_j} \\
V'_i \times_S X_j \ar[r]^{\varphi'_{ij}} & X_i \times_S V'_j
}
$$
commute.
\end{enumerate}
\end{definition}

\noindent
This is the notion that comes up naturally for example when answering
whether the fibred category of relative curves is a stack in the
fpqc topology (it isn't -- at least not if you stick to schemes).
The reason we will usually stick with the version of a family consisting
of a single morphism is the following lemma.

\begin{lemma}
\label{lemma-family-is-one}
Let $S$ be a scheme.
Let $\{X_i \to S\}_{i \in I}$ be a family of morphisms with target $S$.
Set $X = \coprod_{i \in I} X_i$, and consider it as an $S$-scheme.
There is a canonical equivalence of categories
$$
\begin{matrix}
\text{category of descent data } \\
\text{relative to the family } \{X_i \to S\}_{i \in I}
\end{matrix}
\longrightarrow
\begin{matrix}
\text{ category of descent data} \\
\text{ relative to } X/S
\end{matrix}
$$
which maps $(V_i, \varphi_{ij})$ to $(V, \varphi)$ with
$V = \coprod_{i\in I} V_i$ and $\varphi = \coprod \varphi_{ij}$.
\end{lemma}

\begin{proof}
Observe that $X \times_S X = \coprod_{ij} X_i \times_S X_j$
and similarly for higher fibre products.
Giving a morphism $V \to X$ is exactly the same as
giving a family $V_i \to X_i$. And giving a descent datum
$\varphi$ is exactly the same as giving a family $\varphi_{ij}$.
\end{proof}

\begin{lemma}
\label{lemma-pullback}
Let
$$
\xymatrix{
X' \ar[r]_f \ar[d] & X \ar[d] \\
S' \ar[r] & S
}
$$
be a commutative diagram of morphisms of schemes.
The construction
$$
(V \to X, \varphi) \longmapsto f^*(V \to X, \varphi) = (V' \to X', \varphi')
$$
where $V' = X' \times_X V$ and where
$\varphi'$ is defined as the composition
$$
\xymatrix{
V' \times_{S'} X' \ar@{=}[r] &
(X' \times_X V) \times_{S'} X' \ar@{=}[r] &
(X' \times_{S'} X') \times_{X \times_S X} (V \times_S X)
\ar[d]^{\text{id} \times \varphi} \\
X' \times_{S'} V' \ar@{=}[r] &
X' \times_{S'} (X' \times_X V) &
(X' \times_S X') \times_{X \times_S X} (X \times_S V) \ar@{=}[l]
}
$$
defines a functor from the category of descent data
relative to $X \to S$ to the category of descent data
relative to $X' \to S$.
\end{lemma}

\begin{proof}
Omitted.
\end{proof}

\noindent
Given a scheme $U$ over $S$ we have the {\it trivial descent datum} of $U$
relative to $\text{id} : S \to S$, namely the identity morphism.
For any morphism $X \to S$ we get by Lemma \ref{lemma-pullback} above
a {\it canonical descent datum} on $X \times_S U$
relative to $X \to S$. We denote $(X \times_S U, can)$ this descent datum.

\medskip\noindent
Similarly, suppose we are given a family of morphisms
$\{X_i \to S\}$ with target $S$. Given a scheme $U$ over $S$
we have a {\it canonical descent datum} on the family of
schemes $X_i \times_S U$. Compare with the discussion in
Descent, Section \ref{fpqc-descent-section-equivalence}.
We denote this descent datum $(X_i \times_S U, can)$.

\begin{definition}
\label{definition-effective}
Let $S$ be a scheme.
\begin{enumerate}
\item Let $f : X \to S$ be a morphism of schemes.
A descent datum $(V, \varphi)$ relative to $X/S$ is
is called {\it effective} if there exists a 
scheme $U \to S$ and an isomorphism $\psi : V \to X \times_S U$
of $X$-schemes such that $\varphi$ is equal to the composition
$$
\xymatrix{
V \times_S X \ar[r]^-{\psi \times \text{id}_X} &
X \times_S U \times_S X \ar@{=}[r] &
X \times_S X \times_S U 
\ar[r]^-{\text{id}_X \times \psi^{-1}} &
X \times_S V
}
$$
In other words, $(V, \varphi)$ is effective if it is
isomorphic to the canonical descent datum
$(X \times_S U, can)$ for some scheme $U$ over $S$.
\item Let $\{X_i \to S\}$ be a family of morphisms
with target $S$. A descent datum $(V_i, \varphi_{ij})$
relative to $\{X_i \to S\}$ is called {\it effective}
if there exists a scheme $U$ over $S$ such that
$(V_i, \varphi_{ij})$ is isomorphic to $(X_i \times_S U, can)$.
\end{enumerate}
\end{definition}

\begin{lemma}
\label{lemma-faithful}
In the situation of Lemma \ref{lemma-pullback}
assume that $S' \to S$ and $X' \to X$ are surjective
and flat. Then the pullback functor is faithful.
\end{lemma}

\begin{proof}
Omitted.
\end{proof}

\begin{lemma}
\label{lemma-fully-faithful}
In the situation of Lemma \ref{lemma-pullback}
assume that $S' = S$ that $X' \to X$ is a faithfully flat quasi-compact
morphism of schemes over $S$. Then the pullback
functor is fully faithful.
\end{lemma}

\begin{proof}
Let $(V, \varphi)$ and $(W, \psi)$ be two descent data relative
to $X \to S$. Set $V' = X' \times_X V$ and $W' = X' \times_X W$.
Let $f' : V' \to W'$ be a morphism of descent data for $X' \to S$.
By assumption the diagram
$$
\xymatrix{
V' \times_S X' \ar[r]_{\varphi'} \ar[d]_{f' \times \text{id}} &
X' \times_S V' \ar[d]^{\text{id} \times f'} \\
W' \times_S X' \ar[r]^{\psi'} & X' \times_S W'
}
$$
commutes. We claim the two compositions
$$
\xymatrix{
V' \times_V V' \ar[r]^-{\text{pr}_i} &
V' \ar[r]^{f'} &
W' \ar[r] &
W
}
$$
$i = 0, 1$ are the same. The reader is advised to prove this themselves rather
than read the rest of this paragraph. (Please email if you find a 
nice clean argument.) Let $v_0, v_1$ be points of $V'$ which map to the same
point $v \in V$. Let $x_i \in X'$ be the image of $v_i$, and let
$x$ be the point of $X$ which is the image of $v$ in $X$. In other words,
$v_i = (x_i, v)$ in $V' = X' \times_X V$. Write
$\varphi(v, x) = (x, v')$, which is possible because $\varphi$ is
a morphism over $X \times_S X$. Denote
$v_i' = (x_i, v')$. Then a calculation shows that
$\varphi'(v_i, x_j) = (x_i, v'_j)$. Denote
$w_i = f'(v_i)$ and $w'_i = f'(v_i')$.
Now we may write $w_i = (x_i, u_i)$ for some point $u_i$ of $W$,
and $w_i' = (x_i, u'_i)$ for some point $u_i'$ of $W$.
The claim is equivalent to the assertion: $u_0 = u_1$.
A formal calculation using the definition of $\psi'$
(see Lemma \ref{lemma-pullback}) shows
that the commutativity of the diagram displayed above says that
$$
((x_i, x_j), \psi(u_i, x)) = ((x_i, x_j), (x, u'_j))
$$
as points of $(X' \times_S X') \times_{X \times_S X} (X \times_S V)$
for all $i, j \in \{0, 1\}$. This clearly shows that
$u_0 = u_1$ by taking $\psi^{-1}$ of the second entry
in the above. This proves the claim because we can take
scheme points in the arguments above (in other words, we may
take $(v_0, v_1) = \text{id}_{V' \times_V V'}$).

\medskip\noindent
At this point we can use Descent,
Lemma \ref{fpqc-descent-lemma-fpqc-universal-effective-epimorphisms}.
Namely, since $V' \to V$ is faithfully flat and quasi-compact (as
the base change of such a morphism) we see that the morphism
$V' \to W' \to W$ factors through a unique morphism $f : V \to W$
whose base change is necessarily $f'$.
Finally, we see the diagram
$$
\xymatrix{
V \times_S X \ar[r]_{\varphi} \ar[d]_{f \times \text{id}} &
X \times_S V \ar[d]^{\text{id} \times f} \\
W \times_S X \ar[r]^{\psi} & X \times_S W
}
$$
commutes because its base change to $X' \times_S X'$ commutes.
Hence $f$ is a morphism of descent data $(V, \varphi) \to (W, \psi)$
as desired.
\end{proof}

\begin{lemma}
\label{lemma-pullback-selfmap}
Let $X \to S$ be a morphism of schemes.
Let $f : X \to X$ be a selfmap of $X$ over $S$.
In this case pullback by $f$ is isomorphic to the
identity functor on the category of descent data 
relative to $X \to S$.
\end{lemma}

\begin{proof}
Let $(\pi : V \to X, \varphi)$ be a descent datum relative to $X \to S$.
Consider the morphism
$$
\xymatrix{
V &
V \times_S X \ar[l] &
X \times_S V \ar[l]_-{\varphi^{-1}} &
X \times_{f, X} V = f^*V \ar[l]
}
$$
of schemes over $X$. The morphism $\varphi : V \times_S X \to X \times_S V$
is a morphism over the scheme $X \times_S X$ and we can
pull it back by $\text{id}_X \times f : X \times_S X \to X\times_S X$.
This gives a morphism $\tilde\varphi : V \times_S X \to X \times_S f^*V$.
Consider the morphism
$$
\xymatrix{
V \ar[r] &
V \times_S X \ar[r]^{\tilde\varphi} &
X \times_S f^*V \ar[r] &
f^*V
}
$$
where the first morphism is $v \mapsto (v, \pi(v))$.
We omit the verification that the morphisms
$V \to f^*V$ and $f^*V \to V$ so constructed are 
mutually inverse isomorphisms of descent data relative to
$X \to S$.
\end{proof}

\begin{lemma}
\label{lemma-morphism-with-section-equivalence}
Let $f : X' \to X$ be a morphism of schemes over a base scheme $S$.
If there exists a morphism $g : X \to X'$ over $S$, for example
if $f$ has a section, then the pullback functor
of Lemma \ref{lemma-pullback} defines an equivalence of
categories between the category of descent data relative to
$X/S$ and $X'/S$.
\end{lemma}

\begin{proof}
Let $g : X \to X'$ be a morphism over $S$.
Lemma \ref{lemma-pullback-selfmap} above shows that the functors
$f^* \circ g^* = (g \circ f)^*$ and $g^* \circ f^* = (f \circ g)^*$
are isomorphic
to the respective identity functors as desired.
\end{proof}




\section{Descending affine morphisms}
\label{section-affine}

\begin{lemma}
\label{lemma-affine}
Let $S$ be a scheme.
Let $\{X_i \to S\}_{i\in I}$ be an fpqc covering, see
Topologies, Definition \ref{topologies-definition-fpqc-covering}.
Let $(V_i/X_i, \varphi_{ij})$ be a descent datum
relative to $\{X_i \to S\}$. If each morphism
$V_i \to X_i$ is affine, then the descent datum is
effective.
\end{lemma}

\begin{proof}
Omitted.
\end{proof}



\section{Descending quasi-affine morphisms}
\label{section-quasi-affine}

\begin{lemma}
\label{lemma-quasi-affine}
Let $S$ be a scheme.
Let $\{X_i \to S\}_{i\in I}$ be an fpqc covering, see
Topologies, Definition \ref{topologies-definition-fpqc-covering}.
Let $(V_i/X_i, \varphi_{ij})$ be a descent datum
relative to $\{X_i \to S\}$. If each morphism
$V_i \to X_i$ is quasi-affine, then the descent datum is
effective.
\end{lemma}

\begin{proof}
Omitted.
\end{proof}



\section{Descent in terms of simplicial schemes}
\label{section-simplicial}

\noindent
A {\it simplicial scheme} is a simplicial object in the category of schemes,
see Simplicial, Definition \ref{simplicial-definition-simplicial-object}.
In this chapter we will use the a subscript $\bullet$ to denote simplicial
objects. Recall that a simplicial scheme looks like
$$
\xymatrix{
X_2
\ar@<2ex>[r]
\ar@<0ex>[r]
\ar@<-2ex>[r]
&
X_1 
\ar@<1ex>[r]
\ar@<-1ex>[r]
\ar@<1ex>[l]
\ar@<-1ex>[l]
&
X_0
\ar@<0ex>[l]
}
$$
Here there are two morphisms $d^1_0, d^1_1 : X_1 \to X_0$
and a single morphism $s^0_0 : X_0 \to X_1$, etc.
It is important to remember that $d^n_i : X_n \to X_{n - 1}$
should be thought of as a ``projection forgetting the
$i$th coordinate''.

\begin{definition}
\label{definition-cartesian-morphism}
Let $a : V_\bullet \to X_\bullet$ be a morphism of simplicial schemes.
We say $a$ is {\it cartesian}, or that {\it $V_\bullet$ is cartesian over
$X_\bullet$}, if for every morphism
$\varphi : [n] \to [m]$ of $\Delta$ the corresponding diagram
$$
\xymatrix{
V_m \ar[r]_{a} \ar[d]_{V_\bullet(\varphi)} & X_m \ar[d]^{X_\bullet(\varphi)}\\
V_n \ar[r]^{a} & X_n
}
$$
is a fibre square in the category of schemes.
\end{definition}

\begin{definition}
\label{definition-fibre-products-simplicial-scheme}
Let $f : X \to S$ be a morphism of schemes.
The {\it simplicial scheme associated to $f$}, denoted $(X/S)_\bullet$,
is the functor $\Delta^{opp} \to \textit{Sch}$,
$[n] \mapsto X\times_S \ldots \times_S X$
described in
Simplicial, Example \ref{simplicial-example-fibre-products-simplicial-object}.
\end{definition}

\noindent
Thus $(X/S)_n$ is the $(n + 1)$-fold fibre product of $X$ over $S$.
The morphism $d^1_0 : X \times_S X \to X$ is the map
$(x_0, x_1) \mapsto x_1$ and the morphism $d^1_1$ is the other
projection. The morphism $s^0_0$ is the diagonal morphism
$X \to X \times_S X$.

\begin{lemma}
\label{lemma-cartesian-over}
Let $f : X \to S$ be a morphism of schemes.
Let $\pi : V_\bullet \to (X/S)_\bullet$ be a cartesian morphism.
Set $V = V_0$ considered as a scheme over $X$.
The morphisms $d^1_0, d^1_1 : V_1 \to V_0$ and the morphism
$\pi_1 : V_1 \to X \times_S X$ induce isomorphisms
$$
\xymatrix{
V \times_S X & &
V_1 \ar[ll]_-{(d^1_1, \text{pr}_1 \circ \pi_1)}
\ar[rr]^-{(\text{pr}_0 \circ \pi_1, d^1_0)} & &
X \times_S V.
}
$$
Denote $\varphi : V \times_S X \to X \times_S V$ the
resulting isomorphism.
Then the pair $(V, \varphi)$ is a descent datum relative
to $X \to S$.
\end{lemma}

\begin{proof}
The statement that the displayed morphisms are isomorphisms
is exactly the cartesian property for the maps
$\delta^1_0, \delta^1_1 : [0] \to [1]$. The fact that the diagram
of Definition \ref{definition-descent-datum} (1) commutes
follows from the fact that each of the induced morphisms
$V_2 \to V \times_{X, \text{pr}_i} (X \times_S X \times_S X)$
associated to $[0] \to [2]$, $0 \mapsto i$
is an isomorphism. Details omitted.
\end{proof}

\begin{lemma}
\label{lemma-cartesian-equivalent-descent-datum}
Let $f : X \to S$ be a morphism of schemes. The construction
$$
\begin{matrix}
\text{category of cartesian } \\
\text{schemes over } (X/S)_\bullet 
\end{matrix}
\longrightarrow
\begin{matrix}
\text{ category of descent data} \\
\text{ relative to } X/S
\end{matrix}
$$
of Lemma \ref{lemma-cartesian-over}
is an equivalence of categories.
\end{lemma}

\begin{proof}
Here you have to show that given a descent datum
$(V, \varphi)$ you can canonically construct a
cartesian morphism of simplicial schemes
$V_\bullet \to (X/S)_\bullet$ so that if you apply
the construction of Lemma \ref{lemma-cartesian-over}
then you get back $(V, \varphi)$. This we did carefully
in Descent, Section \ref{fpqc-descent-section-descent-modules}
for the case of descent data for modules over rings
and their associated cosimplicial rings, see especially
Descent, Lemma \ref{fpqc-descent-lemma-descent-datum-cosimplicial}.
We can easily translate this to the current context.
Namely, set
$$
V_n = X \times_S \ldots \times_S X \times_S V.
$$
Given a point $(x_0, \ldots, x_{n - 1}, v)$ of $V_n$
we use the convention that $x_n = \pi(v)$. Using this
convention, given a morphism $\beta : [m] \to [n]$
the associated morphism
$$
V_\bullet(\beta) : V_n \longrightarrow V_m
$$
maps $(x_0, \ldots, x_{n - 1}, v)$ to
$(x_{\beta(0)}, \ldots, x_{\beta(m - 1)}, v')$
where $\varphi^{-1}(x_{\beta(m)}, v) = (v', x_n)$.
(It is a fact that $v' = v$ if $n = \beta(m)$; see discussion
following Definition \ref{definition-descent-datum}.)
We omit the verification that this defines a
simplicial scheme which is cartesian over
$(X/S)_\bullet$.
\end{proof}

\noindent
We may reinterpret the pullback of Lemma \ref{lemma-pullback} as follows.
Suppose given a commutative diagram of morphisms of schemes
$$
\xymatrix{
X' \ar[r]_f \ar[d] & X \ar[d] \\
S' \ar[r] & S.
}
$$
This gives rise to a morphism of simplicial schemes
$$
f_\bullet : (X'/S')_\bullet \longrightarrow (X/S)_\bullet.
$$
It is a pleasant exercise to check that given any morphism
of simplical schemes $f_\bullet : Y_\bullet \to X_\bullet$ and a 
cartesian simplicial scheme $V_\bullet \to X_\bullet$
the fibre product
$$
f_\bullet^*V_\bullet = Y_\bullet \times_{X_\bullet} V_\bullet
$$
is a cartesian simplicial scheme over $Y_\bullet$. We omit
the verification that this applied to the morphism
$(X'/S')_\bullet \to (X/S)_\bullet$ corresponds via
Lemma \ref{lemma-cartesian-equivalent-descent-datum}
with the pullback defined in terms of descent data.



\section{Descent data give equivalence relations}
\label{section-equivalence-relation}

\begin{lemma}
\label{lemma-equivalence-relation}
Let $f : X \to S$ be a morphism of schemes.
Let $\pi : V_\bullet \to (X/S)_\bullet$ be a cartesian morphism.
Then the morphism
$$
j = (d^1_1, d^1_0) : V_1 \to V_0 \times_S V_0
$$
defines an equivalence relation on $V_0$ over $S$,
see Groupoids, Definition \ref{groupoids-definition-equivalence-relation}.
\end{lemma}

\begin{proof}
Note that $j$ is a monomorphism. Namely the
composition $V_1 \to V_0 \times_S V_0 \to V_0 \times_S X$
is an isomorphism as $\pi$ is cartesian.

\medskip\noindent
Consider the morphism
$$
(d^2_2, d^2_0) : V_2 \to V_1 \times_{d^1_0, V_0, d^1_1} V_1.
$$
This works because $d_0 \circ d_2 = d_1 \circ d_0$,
see Simplicial, Remark \ref{simplicial-remark-relations}.
Also, it is a morphism over $(X/S)_2$. It is an isomorphism
because $V_\bullet \to (X/S)_\bullet$ is cartesian.
Note for example that the
right hand side is isomorphic to
$V_0 \times_{\pi_0, X, \text{pr}_1} (X \times_S X \times_S X) =
X \times_S V_0 \times_S X$
because $\pi$ is cartesian. Details omitted.

\medskip\noindent
As usual, see
Groupoids, Definition \ref{groupoids-definition-equivalence-relation}
we denote $t = \text{pr}_0 \circ j = d^1_1$ and
$s = \text{pr}_1 \circ j = d^1_0$.
The isomorphism above, combined with the morphism
$d^2_1 : V_2 \to V_1$ give us a composition morphism
$$
c : V_1 \times_{s, V_0, t} V_1 \longrightarrow V_1
$$
over $V_0 \times_S V_0$. This immediately implies
that for any scheme $T/S$ the relation
$V_1(T) \subset V_0(T) \times V_0(T)$ is transitive.

\medskip\noindent
Reflexivity follows from the fact that the
restriction of the morphism $j$ to the diagonal
$\Delta : X \to X\times_S X$ is an isomorphism
(again use the cartesian property of $\pi$).

\medskip\noindent
To see symmetry we consider the morphism
$$
(d^2_2, d^2_1) : V_2 \to V_1 \times_{d^1_1, V_0, d^1_1} V_1.
$$
This works because $d_1 \circ d_2 = d_1 \circ d_1$,
see Simplicial, Remark \ref{simplicial-remark-relations}.
It is an isomorphism
because $V_\bullet \to (X/S)_\bullet$ is cartesian.
Note for example that the
right hand side is isomorphic to
$V_0 \times_{\pi_0, X, \text{pr}_0} (X \times_S X \times_S X) =
V_0 \times_S X \times_S X$
because $\pi$ is cartesian. Details omitted.

\medskip\noindent
Let $T/S$ be a scheme. Let $a \sim b$ for $a, b \in V_0(T)$
be synonymous with $(a, b) \in V_1(T)$.
The isomorphism $(d^2_2, d^2_1)$ above
implies that if $a \sim b$ and $a \sim c$, then $b \sim c$.
Combined with reflexivity this shows that $\sim$ is
an equivalence relation.
\end{proof}







\section{An example case}
\label{section-example}

\noindent
In this section we show that disjoint unions of spectra
of Artinian rings can be descended along a quasi-compact
surjective flat morphism of schemes.

\begin{lemma}
\label{lemma-equivalence-classes-points}
Let $X \to S$ be a morphism of schemes.
Suppose $V_\bullet \to (X/S)_\bullet$ is cartesian.
For $v \in V_0$ a point define
$$
T_v = \{v' \in V \mid \exists\ v_1 \in V_1:
d^1_1(v_1) = v, d^1_0(v_1) = v'\}
$$
as a subset of $V_0$. Then $v \in T_v$ and
$T_v \cap T_{v'} \not = \emptyset \Rightarrow T_v = T_{v'}$.
\end{lemma}

\begin{proof}
Combine Lemma \ref{lemma-equivalence-relation} and
Groupoids,
Lemma \ref{groupoids-lemma-pre-equivalence-equivalence-relation-points}.
\end{proof}

\begin{lemma}
\label{lemma-quasi-compact}
Let $X \to S$ be a morphism of schemes.
Suppose $V_\bullet \to (X/S)_\bullet$ is cartesian.
Let $v \in V_0$ be a point. If $X \to S$ is quasi-compact, then
$$
T_v = \{v' \in V \mid \exists\ v_1 \in V_1:
d^1_1(v_1) = v, d^1_0(v_1) = v'\}
$$
is a quasi-compact subset of $V_0$.
\end{lemma}

\begin{proof}
Let $F_v$ be the scheme theoretic fibre of $d^1_1 : V_1 \to V_0$
at $v$. Then we see that $T_v$ is the image of the morphism
$$
\xymatrix{
F_v \ar[r] \ar[d] &
V_1 \ar[r]^{d^1_0} \ar[d]^{d^1_1} &
V_0 \\
v \ar[r] &
V_0 &
}
$$
Note that $F_v$ is quasi-compact. This proves the lemma.
\end{proof}

\begin{lemma}
\label{lemma-descent-disjoint-union-Artinian-along-fields}
Let $X \to S$ be a quasi-compact flat surjective morphism.
Let $(V, \varphi)$ be a descent datum relative
to $X \to S$. If $V$ is a disjoint union of
spectra of Artinian rings, then $(V, \varphi)$
is effective.
\end{lemma}

\begin{proof}
We may write $V = \coprod_{i \in I} \text{Spec}(A_i)$
with each $A_i$ local Artinian. Moreover, let
$v_i \in V$ be the unique closed point of $\text{Spec}(A_i)$
for all $i \in I$. Write $i \sim j$ if and only if
$v_i \in T_{v_j}$ with notation as in
Lemma \ref{lemma-equivalence-classes-points} above.
By this lemma and  Lemma \ref{lemma-quasi-compact}
this is an equivalence relation with finite equivalence
classes. Let $\overline{I} = I/\sim$. Then we can write
$V = \coprod_{\overline{i} \in \overline{I}} V_{\overline{i}}$
with
$V_{\overline{i}} = \coprod_{i \in \overline{i}} \text{Spec}(A_i)$.
By construction we see that
$\varphi : V \times_S X \to X \times_S V$ maps 
the open and closed subspaces $V_{\overline{i}} \times_S X$
into the open and closed subspaces $X \times_S V_{\overline{i}}$.
In other words, we get descent data 
$(V_{\overline{i}}, \varphi_{\overline{i}})$, and
$(V, \varphi)$ is the coproduct of them in the category of
descent data.
Since each of the $V_{\overline{i}}$ is a finite union of
spectra of Artinian local rings the morphism $V_{\overline{i}} \to X$
is affine, see Morphisms, Lemma \ref{morphisms-lemma-Artinian-affine}.
Since $\{X \to S\}$ is an fpqc covering we see that all
the descent data $(V_{\overline{i}}, \varphi_{\overline{i}})$ are effective
by Lemma \ref{lemma-affine}.
Hence we win.
\end{proof}

\noindent
To be sure, the lemma above has very limited applicability!

\section{Other chapters}

\begin{multicols}{2}
\begin{enumerate}
\item \hyperref[introduction-section-phantom]{Introduction}
\item \hyperref[conventions-section-phantom]{Conventions}
\item \hyperref[sets-section-phantom]{Set Theory}
\item \hyperref[categories-section-phantom]{Categories}
\item \hyperref[topology-section-phantom]{Topology}
\item \hyperref[sheaves-section-phantom]{Sheaves on Spaces}
\item \hyperref[algebra-section-phantom]{Commutative Algebra}
\item \hyperref[sites-section-phantom]{Sites and Sheaves}
\item \hyperref[homology-section-phantom]{Homological Algebra}
\item \hyperref[derived-section-phantom]{Derived Categories}
\item \hyperref[more-algebra-section-phantom]{More Algebra}
\item \hyperref[simplicial-section-phantom]{Simplicial Methods}
\item \hyperref[modules-section-phantom]{Sheaves of Modules}
\item \hyperref[sites-modules-section-phantom]{Modules on Sites}
\item \hyperref[injectives-section-phantom]{Injectives}
\item \hyperref[cohomology-section-phantom]{Cohomology of Sheaves}
\item \hyperref[sites-cohomology-section-phantom]{Cohomology on Sites}
\item \hyperref[hypercovering-section-phantom]{Hypercoverings}
\item \hyperref[schemes-section-phantom]{Schemes}
\item \hyperref[constructions-section-phantom]{Constructions of Schemes}
\item \hyperref[properties-section-phantom]{Properties of Schemes}
\item \hyperref[morphisms-section-phantom]{Morphisms of Schemes}
\item \hyperref[coherent-section-phantom]{Coherent Cohomology}
\item \hyperref[divisors-section-phantom]{Divisors}
\item \hyperref[limits-section-phantom]{Limits of Schemes}
\item \hyperref[varieties-section-phantom]{Varieties}
\item \hyperref[chow-section-phantom]{Chow Homology}
\item \hyperref[topologies-section-phantom]{Topologies on Schemes}
\item \hyperref[descent-section-phantom]{Descent}
\item \hyperref[more-morphisms-section-phantom]{More on Morphisms}
\item \hyperref[flat-section-phantom]{More on Flatness}
\item \hyperref[groupoids-section-phantom]{Groupoid Schemes}
\item \hyperref[more-groupoids-section-phantom]{More on Groupoid Schemes}
\item \hyperref[etale-section-phantom]{\'Etale Morphisms of Schemes}
\item \hyperref[etale-cohomology-section-phantom]{\'Etale Cohomology}
\item \hyperref[spaces-section-phantom]{Algebraic Spaces}
\item \hyperref[spaces-properties-section-phantom]{Properties of Algebraic Spaces}
\item \hyperref[spaces-morphisms-section-phantom]{Morphisms of Algebraic Spaces}
\item \hyperref[spaces-topologies-section-phantom]{Topologies on Algebraic Spaces}
\item \hyperref[spaces-descent-section-phantom]{Descent and Algebraic Spaces}
\item \hyperref[spaces-more-morphisms-section-phantom]{More on Morphisms of Spaces}
\item \hyperref[quot-section-phantom]{Quot and Hilbert Spaces}
\item \hyperref[stacks-section-phantom]{Stacks}
\item \hyperref[spaces-groupoids-section-phantom]{Groupoids in Algebraic Spaces}
\item \hyperref[spaces-more-groupoids-section-phantom]{More on Groupoids in Spaces}
\item \hyperref[bootstrap-section-phantom]{Bootstrap}
\item \hyperref[examples-stacks-section-phantom]{Examples of Stacks}
\item \hyperref[groupoids-quotients-section-phantom]{Quotients of Groupoids}
\item \hyperref[algebraic-section-phantom]{Algebraic Stacks}
\item \hyperref[criteria-section-phantom]{Criteria for Representability}
\item \hyperref[stacks-properties-section-phantom]{Properties of Algebraic Stacks}
\item \hyperref[stacks-morphisms-section-phantom]{Morphisms of Algebraic Stacks}
\item \hyperref[examples-section-phantom]{Examples}
\item \hyperref[exercises-section-phantom]{Exercises}
\item \hyperref[guide-section-phantom]{Guide to Literature}
\item \hyperref[desirables-section-phantom]{Desirables}
\item \hyperref[coding-section-phantom]{Coding Style}
\item \hyperref[fdl-section-phantom]{GNU Free Documentation License}
\item \hyperref[index-section-phantom]{Auto Generated Index}
\end{enumerate}
\end{multicols}


\bibliography{my}
\bibliographystyle{alpha}


\end{document}
