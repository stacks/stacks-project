\documentclass{amsart}

% The following AMS packages are automatically loaded with amsart 
% documentclass:
%\usepackage{amsmath}
%\usepackage{amssymb}
%\usepackage{amsthm}

%%%%%%%%%%%%%%%%%%%%%%%%%%%%%
% Extra packages and commands used
\usepackage{amsopn}

\newcommand{\br}[1]{\overline{#1}}
\DeclareMathOperator{\PreShv}{PreShv}
\DeclareMathOperator{\Shv}{Shv}
\DeclareMathOperator{\Gpd}{Gpd}
\DeclareMathOperator*{\holim}{holim}
%
%%%%%%%%%%%%%%%%%%%%%%%%%%%%%

% For commutative diagrams you can use
% \usepackage{amscd}
% but Jason prefers xypic
\usepackage[all]{xy}

% To put source file link in headers.
% Change "template.tex" to "this_filename.tex"
\usepackage{fancyhdr}
\pagestyle{fancy}
\lhead{}
\chead{}
\rhead{Source file: \url{src/stacks-groupoids.tex}}
\lfoot{}
\cfoot{\thepage}
\rfoot{}
\renewcommand{\headrulewidth}{0pt}
\renewcommand{\footrulewidth}{0pt}
\renewcommand{\headheight}{12pt}

% For cross-file-references
\usepackage{xr-hyper}

% Package for hypertext links:
\usepackage[colorlinks=true]{hyperref}
% For any local file, say "hello.tex" you want to refer to please use
% \externaldocument[hello-]{hello}
\externaldocument[conventions-]{conventions}
\externaldocument[flat-]{flat}
\externaldocument[schemes-]{schemes}
\externaldocument[desirables-]{desirables}

% The macro \autoref uses the macros \figurename, etc.
% We list the default values and we change some of them
% to start with a captial.
% Figure	\figurename
% Table		\tablename
% Part		\partname
% Appendix	\appendixname
% Equation	\equationname
% item		\Itemname
% \renewcommand{\Itemname}{Item}
\renewcommand{\Itemautorefname}{Item}
% chapter	\Chaptername
% \renewcommand{\Chaptername}{Chapter}
% \renewcommand{\Chapterautorefname}{Chapter}
% section	\sectionname
\renewcommand{\sectionname}{Section}
\renewcommand{\sectionautorefname}{Section}
% subsection	\subsectionname
\renewcommand{\subsectionname}{Subsection}
\renewcommand{\subsectionautorefname}{Subsection}
% subsubsection	\subsubsectionname
\renewcommand{\subsubsectionname}{Subsubsection}
\renewcommand{\subsubsectionautorefname}{Subsubsection}
% paragraph	\paragraphname
\renewcommand{\paragraphname}{Paragraph}
\renewcommand{\paragraphautorefname}{Paragraph}
% footnote	\Hfootnotename
% \renewcommand{\Hfootnotename}{Footnote}
\renewcommand{\Hfootnoteautorefname}{Footnote}
% Equation	\AMSname
% Theorem	\theoremname


% Theorem environments.
%
\newtheorem{theorem}{Theorem}[subsection]
\newtheorem{proposition}[theorem]{Proposition}
\newtheorem{lemma}[theorem]{Lemma}

\theoremstyle{definition}
\newtheorem{definition}[theorem]{Definition}
\newtheorem{example}[theorem]{Example}
\newtheorem{exercise}[theorem]{Exercise}
\newtheorem{situation}[theorem]{Situation}

\theoremstyle{remark}
\newtheorem{remark}[theorem]{Remark}
\newtheorem{remarks}[theorem]{Remarks}

\numberwithin{equation}{subsection}


% OK, start here.
%
\begin{document}

\title{Stacks and groupoids}

%\begin{abstract}
%\end{abstract}

\maketitle
\thispagestyle{fancy}

\tableofcontents

\section{Introduction}
\label{section-introduction}

Dear Johan

\medskip\noindent
I promised to send you a continued explanation of this homotopical version
of stacks.  It turns out that I did not need to say much else --- my
source for this perspective on stacks is, by the way, a paper of 
Sharon Hollander's which you probably know about (see \cite{Hollander}).

\smallskip\noindent
Let $\Gpd$ be the
category of groupoids.  For a groupoid $G$ we will let $G_0$ denote the
objects and $G_1$ denote the morphisms.  
There is a notion of homotopy attached to this
category where the homotopies are given by natural transformations.  There
is a model category structure which reflects this, and for which the weak
equivalences are precisely the equivalences of categories.

\smallskip\noindent
The homotopy category of groupoids is equivalent to the homotopy category
of $1$-coconnective spaces - that is - the homotopy category of spaces
with non-trivial homotopy groups in degrees $0$ and $1$.  The functors
which give this equivalence on the homotopy category are the classifying
space functor, and the fundamental groupoid functor.


\smallskip\noindent
Let $\PreShv_\mathcal{C}(\Gpd)$ be the category of presheaves of groupoids
on $\mathcal{C}$.  Let $F$ be an object of this category.  The presheaf $F$
is a stack if it satisfies homotopy descent.  That is, for every cover $\{
U_i \} \rightarrow U$, the natural map
$$ F(U) \rightarrow \holim \left( \prod F(U_i) \Rightarrow \prod F(U_{ij})
\Rightarrow \prod F(U_{ijk}) \right) $$
is an equivalence of categories.  Let's examine this condition closer.  Let
$L(\{U_i\})$ denote the homotopy limit groupoid displayed above.


\smallskip\noindent
What is an object ($0$-simplex) of $L(\{U_i\})$?  It consists of 
\begin{itemize}
\item a collection of $x_i \in F(U_i)_0$
\item a collection of $f_{ij} \in F(U_{ij})_1$.
\end{itemize}
This data is required to satisfy the cosimplicial identities up to
homotopy.  I'll draw simplex-shaped diagrams to make this translation
clear. 
\begin{itemize}
\item The maps $f_{ij}$ have source $x_i$ and target $x_j$.
$$ 
\xymatrix{
x_i \ar[r]^{f_{ij}} & x_j
}
$$
\item The maps $f_{ij}$ satisfy a cocycle condition.
$$
\xymatrix{
x_i \ar[rr]^{f_{ik}} \ar[rd]_{f_{ij}} & \ar@{}[d]|{Id} & x_k
\\
& x_j \ar[ru]_{f_{jk}} 
}
$$
\end{itemize}
That mysterious $Id$ floating in the middle of the $2$-simplex above is
supposed to be the identity $2$-morphism.  Of course, since $\Gpd$ is a
category of $1$-categories, there are no non-trivial $2$-morphisms, but it
makes the pattern clear.

\smallskip\noindent
The morphisms of $L(\{U_i\})$ from an object $(x_i, f_{ij})$ to an object
$(\br{x}_i, \br{f}_{ij})$ consist of a collection of $g_i \in F(U_i)$ such
that the following diagram commutes
$$
\xymatrix{
x_i\vert_{U_{ij}} \ar[r]^{g_i\vert_{U_{ij}}} \ar[d]_{f_{ij}} & 
\br{x_i}\vert_{U_{ij}} \ar[d]^{\br{f}_{ij}}
\\
x_j\vert_{U_{ij}} \ar[r]_{g_j\vert_{U_{ij}}} &
\br{x}_j\vert_{U_{ij}}
}
$$
Note that the $g_i$ restrict because of the presheaf condition.  The above
diagram is suppose to represent a $1$-simplex in the category whose objects
are morphisms are whose morphisms are commuting squares.


\smallskip\noindent
The functoriality of presheaves of groupoids makes this definition of a
stack rather
compact.  However, this functoriality of pullbacks is not present in either
the categories fibered in groupoids approach or the lax presheaf of
groupoids approach, and these do seem to be what you might be handed in
practice.  Some sort of rectification must take place, and I think Sharon
talks about this.  I imagine you guys have your own rectification
techniques.

\smallskip\noindent
I wrote everything so that you could try to replace the role of groupoid
with that of $n$-groupoid, or $n$-connective simplicial set if you like.
There are probably issues that crop up in the more general approach. (like
fibrancy!  Every groupoid is fibrant.)

\medskip\noindent
Mark

\smallskip\noindent
To continue reading, 
\begin{enumerate}

\item visit the next section: Schemes as stacks and representability,
\autoref{schemes-section-introduction}, or 

\item go back to the
table of contents: \url{index.html#contents}.

\end{enumerate}




\bibliography{my}
\bibliographystyle{alpha}

\end{document}
