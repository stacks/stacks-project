\IfFileExists{stacks-project.cls}{%
\documentclass{stacks-project}
}{%
\documentclass{amsart}
}

% The following AMS packages are automatically loaded with
% the amsart documentclass:
%\usepackage{amsmath}
%\usepackage{amssymb}
%\usepackage{amsthm}

% For dealing with references we use the comment environment
\usepackage{verbatim}
\newenvironment{reference}{\comment}{\endcomment}
%\newenvironment{reference}{}{}
\newenvironment{slogan}{\comment}{\endcomment}
\newenvironment{history}{\comment}{\endcomment}

% For commutative diagrams you can use
% \usepackage{amscd}
\usepackage[all]{xy}

% We use 2cell for 2-commutative diagrams.
\xyoption{2cell}
\UseAllTwocells

% To put source file link in headers.
% Change "template.tex" to "this_filename.tex"
% \usepackage{fancyhdr}
% \pagestyle{fancy}
% \lhead{}
% \chead{}
% \rhead{Source file: \url{template.tex}}
% \lfoot{}
% \cfoot{\thepage}
% \rfoot{}
% \renewcommand{\headrulewidth}{0pt}
% \renewcommand{\footrulewidth}{0pt}
% \renewcommand{\headheight}{12pt}

\usepackage{multicol}

% For cross-file-references
\usepackage{xr-hyper}

% Package for hypertext links:
\usepackage{hyperref}

% For any local file, say "hello.tex" you want to link to please
% use \externaldocument[hello-]{hello}
\externaldocument[introduction-]{introduction}
\externaldocument[conventions-]{conventions}
\externaldocument[sets-]{sets}
\externaldocument[categories-]{categories}
\externaldocument[topology-]{topology}
\externaldocument[sheaves-]{sheaves}
\externaldocument[sites-]{sites}
\externaldocument[stacks-]{stacks}
\externaldocument[fields-]{fields}
\externaldocument[algebra-]{algebra}
\externaldocument[brauer-]{brauer}
\externaldocument[homology-]{homology}
\externaldocument[derived-]{derived}
\externaldocument[simplicial-]{simplicial}
\externaldocument[more-algebra-]{more-algebra}
\externaldocument[smoothing-]{smoothing}
\externaldocument[modules-]{modules}
\externaldocument[sites-modules-]{sites-modules}
\externaldocument[injectives-]{injectives}
\externaldocument[cohomology-]{cohomology}
\externaldocument[sites-cohomology-]{sites-cohomology}
\externaldocument[dga-]{dga}
\externaldocument[dpa-]{dpa}
\externaldocument[hypercovering-]{hypercovering}
\externaldocument[schemes-]{schemes}
\externaldocument[constructions-]{constructions}
\externaldocument[properties-]{properties}
\externaldocument[morphisms-]{morphisms}
\externaldocument[coherent-]{coherent}
\externaldocument[divisors-]{divisors}
\externaldocument[limits-]{limits}
\externaldocument[varieties-]{varieties}
\externaldocument[topologies-]{topologies}
\externaldocument[descent-]{descent}
\externaldocument[perfect-]{perfect}
\externaldocument[more-morphisms-]{more-morphisms}
\externaldocument[flat-]{flat}
\externaldocument[groupoids-]{groupoids}
\externaldocument[more-groupoids-]{more-groupoids}
\externaldocument[etale-]{etale}
\externaldocument[chow-]{chow}
\externaldocument[intersection-]{intersection}
\externaldocument[pic-]{pic}
\externaldocument[adequate-]{adequate}
\externaldocument[dualizing-]{dualizing}
\externaldocument[duality-]{duality}
\externaldocument[discriminant-]{discriminant}
\externaldocument[local-cohomology-]{local-cohomology}
\externaldocument[curves-]{curves}
\externaldocument[resolve-]{resolve}
\externaldocument[models-]{models}
\externaldocument[pione-]{pione}
\externaldocument[etale-cohomology-]{etale-cohomology}
\externaldocument[proetale-]{proetale}
\externaldocument[crystalline-]{crystalline}
\externaldocument[spaces-]{spaces}
\externaldocument[spaces-properties-]{spaces-properties}
\externaldocument[spaces-morphisms-]{spaces-morphisms}
\externaldocument[decent-spaces-]{decent-spaces}
\externaldocument[spaces-cohomology-]{spaces-cohomology}
\externaldocument[spaces-limits-]{spaces-limits}
\externaldocument[spaces-divisors-]{spaces-divisors}
\externaldocument[spaces-over-fields-]{spaces-over-fields}
\externaldocument[spaces-topologies-]{spaces-topologies}
\externaldocument[spaces-descent-]{spaces-descent}
\externaldocument[spaces-perfect-]{spaces-perfect}
\externaldocument[spaces-more-morphisms-]{spaces-more-morphisms}
\externaldocument[spaces-flat-]{spaces-flat}
\externaldocument[spaces-groupoids-]{spaces-groupoids}
\externaldocument[spaces-more-groupoids-]{spaces-more-groupoids}
\externaldocument[bootstrap-]{bootstrap}
\externaldocument[spaces-pushouts-]{spaces-pushouts}
\externaldocument[groupoids-quotients-]{groupoids-quotients}
\externaldocument[spaces-more-cohomology-]{spaces-more-cohomology}
\externaldocument[spaces-simplicial-]{spaces-simplicial}
\externaldocument[formal-spaces-]{formal-spaces}
\externaldocument[restricted-]{restricted}
\externaldocument[spaces-resolve-]{spaces-resolve}
\externaldocument[formal-defos-]{formal-defos}
\externaldocument[defos-]{defos}
\externaldocument[cotangent-]{cotangent}
\externaldocument[examples-defos-]{examples-defos}
\externaldocument[algebraic-]{algebraic}
\externaldocument[examples-stacks-]{examples-stacks}
\externaldocument[stacks-sheaves-]{stacks-sheaves}
\externaldocument[criteria-]{criteria}
\externaldocument[artin-]{artin}
\externaldocument[quot-]{quot}
\externaldocument[stacks-properties-]{stacks-properties}
\externaldocument[stacks-morphisms-]{stacks-morphisms}
\externaldocument[stacks-limits-]{stacks-limits}
\externaldocument[stacks-cohomology-]{stacks-cohomology}
\externaldocument[stacks-perfect-]{stacks-perfect}
\externaldocument[stacks-introduction-]{stacks-introduction}
\externaldocument[stacks-more-morphisms-]{stacks-more-morphisms}
\externaldocument[stacks-geometry-]{stacks-geometry}
\externaldocument[moduli-]{moduli}
\externaldocument[moduli-curves-]{moduli-curves}
\externaldocument[examples-]{examples}
\externaldocument[exercises-]{exercises}
\externaldocument[guide-]{guide}
\externaldocument[desirables-]{desirables}
\externaldocument[coding-]{coding}
\externaldocument[obsolete-]{obsolete}
\externaldocument[fdl-]{fdl}
\externaldocument[index-]{index}

% Theorem environments.
%
\theoremstyle{plain}
\newtheorem{theorem}[subsection]{Theorem}
\newtheorem{proposition}[subsection]{Proposition}
\newtheorem{lemma}[subsection]{Lemma}

\theoremstyle{definition}
\newtheorem{definition}[subsection]{Definition}
\newtheorem{example}[subsection]{Example}
\newtheorem{exercise}[subsection]{Exercise}
\newtheorem{situation}[subsection]{Situation}

\theoremstyle{remark}
\newtheorem{remark}[subsection]{Remark}
\newtheorem{remarks}[subsection]{Remarks}

\numberwithin{equation}{subsection}

% Macros
%
\def\lim{\mathop{\rm lim}\nolimits}
\def\colim{\mathop{\rm colim}\nolimits}
\def\Spec{\mathop{\rm Spec}}
\def\Hom{\mathop{\rm Hom}\nolimits}
\def\Ext{\mathop{\rm Ext}\nolimits}
\def\SheafHom{\mathop{\mathcal{H}\!{\it om}}\nolimits}
\def\SheafExt{\mathop{\mathcal{E}\!{\it xt}}\nolimits}
\def\Sch{\textit{Sch}}
\def\Mor{\mathop{\rm Mor}\nolimits}
\def\Ob{\mathop{\rm Ob}\nolimits}
\def\Sh{\mathop{\textit{Sh}}\nolimits}
\def\NL{\mathop{N\!L}\nolimits}
\def\proetale{{pro\text{-}\acute{e}tale}}
\def\etale{{\acute{e}tale}}
\def\QCoh{\textit{QCoh}}
\def\Ker{\mathop{\rm Ker}}
\def\Im{\mathop{\rm Im}}
\def\Coker{\mathop{\rm Coker}}
\def\Coim{\mathop{\rm Coim}}

%
% Macros for moduli stacks/spaces
%
\def\QCohstack{\mathcal{QC}\!{\it oh}}
\def\Cohstack{\mathcal{C}\!{\it oh}}
\def\Spacesstack{\mathcal{S}\!{\it paces}}
\def\Quotfunctor{{\rm Quot}}
\def\Hilbfunctor{{\rm Hilb}}
\def\Curvesstack{\mathcal{C}\!{\it urves}}
\def\Polarizedstack{\mathcal{P}\!{\it olarized}}
\def\Complexesstack{\mathcal{C}\!{\it omplexes}}
% \Pic is the operator that assigns to X its picard group, usage \Pic(X)
% \Picardstack_{X/B} denotes the Picard stack of X over B
% \Picardfunctor_{X/B} denotes the Picard functor of X over B
\def\Pic{\mathop{\rm Pic}\nolimits}
\def\Picardstack{\mathcal{P}\!{\it ic}}
\def\Picardfunctor{{\rm Pic}}
\def\Deformationcategory{\mathcal{D}\!{\it ef}}


% OK, start here.
%
\begin{document}

\title{Derived Categories of Schemes}


\maketitle

\phantomsection
\label{section-phantom}

\tableofcontents

\section{Introduction}
\label{section-introduction}

\noindent
In this chapter we discuss derived categories of modules on schemes.
Most of the material discussed here can be found in
\cite{TT}, \cite{Bokstedt-Neeman}, \cite{BvdB}, and \cite{LN}.
Of course there are many other references.


\section{Conventions}
\label{section-conventions}

\noindent
If $\mathcal{A}$ is an abelian category and $M$ is an object
of $\mathcal{A}$ then we also denote $M$ the object of $K(\mathcal{A})$
and/or $D(\mathcal{A})$ corresponding to the complex which has
$M$ in degree $0$ and is zero in all other degrees.

\medskip\noindent
If we have a ring $A$, then $K(A)$ denotes the homotopy category
of complexes of $A$-modules and $D(A)$ the associated derived category.
Similarly, if we have a ringed space $(X, \mathcal{O}_X)$ the symbol
$K(\mathcal{O}_X)$ denotes the homotopy category of complexes of
$\mathcal{O}_X$-modules and $D(\mathcal{O}_X)$ the associated derived
category.



\section{Koszul complexes}
\label{section-koszul}

\noindent
Let $A$ be a ring and let $f_1, \ldots, f_r$ be a sequence of elements
of $A$. We have defined the Koszul complex
$K_\bullet(f_1, \ldots, f_r)$ in
More on Algebra, Definition \ref{more-algebra-definition-koszul-complex}.
It is a chain complex sitting in degrees $r, \ldots, 0$.
We define a chain complex $C_\bullet(f_1, \ldots, f_r)$
such that we have a distinguished triangle
$$
C_\bullet(f_1, \ldots, f_r) \to
A \to
K_\bullet(f_1, \ldots, f_r) \to
C_\bullet(f_1, \ldots, f_r)[-1]
$$
in $K(A)$ (the shift is $-1$ as we are doing chain complexes).
In other words, we set
$$
C_i(f_1, \ldots, f_r) =
\left\{
\begin{matrix}
K_{i + 1}(f_1, \ldots, f_r) & \text{if } i \geq 0 \\
0 & \text{else}
\end{matrix}
\right.
$$
and we use the negative of the differential on $K_\bullet(f_1, \ldots, f_r)$.
The maps in the distinguished triangle are the obvious ones. Note that
$C_0(f_1, \ldots, f_r) = A^{\oplus r} \to A$ is given by
multiplication by $f_i$ on the $i$th factor.
Hence $C_\bullet(f_1, \ldots, f_r) \to A$ factors as
$$
C_\bullet(f_1, \ldots, f_r) \to I \to A
$$
where $I = (f_1, \ldots, f_r)$. In fact, there is a short exact sequence
$$
0 \to H_1(K_\bullet(f_1, \ldots, f_s)) \to
H_0(C_\bullet(f_1, \ldots, f_s)) \to I \to 0
$$
and for every $i > 0$ we have
$H_i(K_\bullet(f_1, \ldots, f_r) = H_{i + 1}(C_\bullet(f_1, \ldots, f_r))$.
Observe that given a second sequence $g_1, \ldots, g_r$ of elements of $A$
there are canonical maps
$$
C_\bullet(f_1g_1, \ldots, f_rg_r) \to C_\bullet(f_1, \ldots, f_r)
\quad\text{and}\quad
K_\bullet(f_1g_1, \ldots, f_rg_r) \to K_\bullet(f_1, \ldots, f_r)
$$
compatible with the maps described above. The first of these maps is
given by multiplcation by $g_i$ on the $i$th summand of
$C_0(f_1g_1, \ldots, f_rg_r) = A^{\oplus r}$. In particular, given
$f_1, \ldots, f_r$ we obtain an inverse system of complexes
\begin{equation}
\label{equation-system}
C_\bullet(f_1, \ldots, f_r) \leftarrow
C_\bullet(f_1^2, \ldots, f_r^2) \leftarrow
C_\bullet(f_1^3, \ldots, f_r^3) \leftarrow \ldots
\end{equation}
which will play an important role in that which is to follow.
To easily formulate the following lemmas we fix some notation.

\begin{situation}
\label{situation-complex}
Here $A$ is a ring and $f_1, \ldots, f_r$ is a sequence of elements of $A$.
We set $X = \Spec(A)$ and $U = D(f_1) \cup \ldots \cup D(f_r) \subset X$.
We denote $\mathcal{U} : U = \bigcup_{i = 1, \ldots, r} D(f_i)$ the
given open covering of $U$.
\end{situation}

\noindent
Our first lemma is that the complexes above can be used to compute
the cohomology of quasi-coherent sheaves on $U$. Suppose given a chain
complex $C_\bullet$ of $A$-modules and an $A$-module $M$. Then we
define $\Hom_A(C_\bullet, M)$ to be the cochain complex with $n$th
term $\Hom_A(C_n, M)$ and differentials given as the contragredients
of the differentials on $C_\bullet$.

\begin{lemma}
\label{lemma-alternating-cech-complex}
In Situation \ref{situation-complex}. Let $M$ be an $A$-module and
denote $\mathcal{F}$ the associated $\mathcal{O}_X$-module. Then
there is a canonical isomorphism of complexes
$$
\colim_e \Hom_A(C_\bullet(f_1^e, \ldots, f_r^e), M)
\longrightarrow
\check{\mathcal{C}}_{alt}^\bullet(\mathcal{U}, \mathcal{F})
$$
functorial in $M$.
\end{lemma}

\begin{proof}
Recall that the alternating {\v C}ech complex is the subcomplex
of the usual {\v C}ech complex given by alternating cochains, see
Cohomology, Section \ref{cohomology-section-alternating-cech}.
As usual we view a $p$-cochain in
$\check{\mathcal{C}}_{alt}^\bullet(\mathcal{U}, \mathcal{F})$
as an alternating function $s$ on $\{1, \ldots, r\}^{p + 1}$
whose value $s_{i_0\ldots i_p}$ at $(i_0, \ldots, i_p)$ lies in
$M_{f_{i_0}\ldots f_{i_p}} = \mathcal{F}(U_{i_0\ldots i_p})$.
On the other hand, a $p$-cochain $t$ in
$\Hom_A(C_\bullet(f_1^e, \ldots, f_r^e), M)$
is given by a map $t : \wedge^{p + 1}(A^{\oplus r}) \to M$.
Write $[i] \in A^{\oplus r}$ for the $i$th basis element and
write
$$
[i_0, \ldots, i_p] = [i_0] \wedge \ldots \wedge [i_p]
\in \wedge^{p + 1}(A^{\oplus r})
$$
Then we send $t$ as above to $s$ with
$$
s_{i_0\ldots i_p} = \frac{t([i_0, \ldots, i_p])}{f_{i_0}^e\ldots f_{i_p}^e}
$$
It is clear that $s$ so defined is an alternating cochain.
The construction of this map is compatible with the transition maps
of the system as the transition map
$$
C_\bullet(f_1^e, \ldots, f_r^e) \leftarrow
C_\bullet(f_1^{e + 1}, \ldots, f_r^{e + 1}),
$$
of the (\ref{equation-system}) sends $[i_0, \ldots, i_p]$
to $f_{i_0}\ldots f_{i_p}[i_0, \ldots, i_p]$.
It is clear from the description of the localizations
$M_{f_{i_0}\ldots f_{i_p}}$ in
Algebra, Lemma \ref{algebra-lemma-localization-colimit}
that these maps define an isomorphism of cochain modules in degree $p$
in the limit. To finish the proof we have to show that the map
is compatible with differentials. To see this recall that
\begin{align*}
d(s)_{i_0\ldots i_{p + 1}}
& =
\sum\nolimits_{j = 0}^{p + 1} (-1)^j
s_{i_0\ldots \hat i_j \ldots i_p} \\
& = 
\sum\nolimits_{j = 0}^{p + 1} (-1)^j
\frac{t([i_0, \ldots, \hat i_j, \ldots i_{p + 1}])}
{f_{i_0}^e\ldots \hat f_{i_j}^e \ldots f_{i_{p + 1}}^e}
\end{align*}
On the other hand, we have
\begin{align*}
\frac{d(t)([i_0, \ldots, i_{p + 1}])}{f_{i_0}^e\ldots f_{i_{p + 1}}^e}
& =
\frac{t(d[i_0, \ldots, i_{p + 1}])}{f_{i_0}^e\ldots f_{i_{p + 1}}^e} \\
& =
\frac{\sum_j (-1)^j f_{i_j}^e t([i_0, \ldots, \hat i_j, \ldots i_{p + 1}])}
{f_{i_0}^e \ldots f_{i_{p + 1}}^e}
\end{align*}
The two formulas agree by inspection.
\end{proof}

\medskip\noindent
Suppose given a chain complex $C_\bullet$ of $A$-modules and a
cochain complex of $A$-modules $M^\bullet$. We obtain a double complex
$H^{\bullet, \bullet} = \Hom_A(C_\bullet, M^\bullet)$ where
$H^{p, q} = \Hom_A(C_p, M^q)$. The first differential comes from
the differential on $\Hom_A(C_\bullet, M^q)$ and the second
from the differential on $M^\bullet$. Associated to this double
complex is the total complex with degree $n$ term given by
$$
\bigoplus\nolimits_{p + q = n} \Hom_A(C_p, M^q)
$$
and differential as in
Homology, Definition \ref{homology-definition-associated-simple-complex}.
Below our chain complexes will have finitely many terms.
The conventions for taking the total complex associated to a
{\v C}ech complex of a complex are as in
Cohomology, Section \ref{cohomology-section-cech-cohomology-of-complexes}.

\begin{lemma}
\label{lemma-alternating-cech-complex-complex}
In Situation \ref{situation-complex}. Let $M^\bullet$ be a cochain
complex of $A$-modules and
denote $\mathcal{F}^\bullet$ the associated complex of
$\mathcal{O}_X$-modules. Then
there is a canonical isomorphism of complexes
$$
\colim_e \text{Tot}(\Hom_A(C_\bullet(f_1^e, \ldots, f_r^e), M^\bullet))
\longrightarrow
\text{Tot}(\check{\mathcal{C}}_{alt}^\bullet(\mathcal{U}, \mathcal{F}^\bullet))
$$
functorial in $M^\bullet$.
\end{lemma}

\begin{proof}
Immediate from Lemma \ref{lemma-alternating-cech-complex}.
\end{proof}















\section{Other chapters}

\begin{multicols}{2}
\begin{enumerate}
\item \hyperref[introduction-section-phantom]{Introduction}
\item \hyperref[conventions-section-phantom]{Conventions}
\item \hyperref[sets-section-phantom]{Set Theory}
\item \hyperref[categories-section-phantom]{Categories}
\item \hyperref[topology-section-phantom]{Topology}
\item \hyperref[sheaves-section-phantom]{Sheaves on Spaces}
\item \hyperref[algebra-section-phantom]{Commutative Algebra}
\item \hyperref[sites-section-phantom]{Sites and Sheaves}
\item \hyperref[homology-section-phantom]{Homological Algebra}
\item \hyperref[derived-section-phantom]{Derived Categories}
\item \hyperref[more-algebra-section-phantom]{More Algebra}
\item \hyperref[simplicial-section-phantom]{Simplicial Methods}
\item \hyperref[modules-section-phantom]{Sheaves of Modules}
\item \hyperref[sites-modules-section-phantom]{Modules on Sites}
\item \hyperref[injectives-section-phantom]{Injectives}
\item \hyperref[cohomology-section-phantom]{Cohomology of Sheaves}
\item \hyperref[sites-cohomology-section-phantom]{Cohomology on Sites}
\item \hyperref[hypercovering-section-phantom]{Hypercoverings}
\item \hyperref[schemes-section-phantom]{Schemes}
\item \hyperref[constructions-section-phantom]{Constructions of Schemes}
\item \hyperref[properties-section-phantom]{Properties of Schemes}
\item \hyperref[morphisms-section-phantom]{Morphisms of Schemes}
\item \hyperref[coherent-section-phantom]{Coherent Cohomology}
\item \hyperref[divisors-section-phantom]{Divisors}
\item \hyperref[limits-section-phantom]{Limits of Schemes}
\item \hyperref[varieties-section-phantom]{Varieties}
\item \hyperref[chow-section-phantom]{Chow Homology}
\item \hyperref[topologies-section-phantom]{Topologies on Schemes}
\item \hyperref[descent-section-phantom]{Descent}
\item \hyperref[more-morphisms-section-phantom]{More on Morphisms}
\item \hyperref[flat-section-phantom]{More on Flatness}
\item \hyperref[groupoids-section-phantom]{Groupoid Schemes}
\item \hyperref[more-groupoids-section-phantom]{More on Groupoid Schemes}
\item \hyperref[etale-section-phantom]{\'Etale Morphisms of Schemes}
\item \hyperref[etale-cohomology-section-phantom]{\'Etale Cohomology}
\item \hyperref[spaces-section-phantom]{Algebraic Spaces}
\item \hyperref[spaces-properties-section-phantom]{Properties of Algebraic Spaces}
\item \hyperref[spaces-morphisms-section-phantom]{Morphisms of Algebraic Spaces}
\item \hyperref[spaces-topologies-section-phantom]{Topologies on Algebraic Spaces}
\item \hyperref[spaces-descent-section-phantom]{Descent and Algebraic Spaces}
\item \hyperref[spaces-more-morphisms-section-phantom]{More on Morphisms of Spaces}
\item \hyperref[quot-section-phantom]{Quot and Hilbert Spaces}
\item \hyperref[stacks-section-phantom]{Stacks}
\item \hyperref[spaces-groupoids-section-phantom]{Groupoids in Algebraic Spaces}
\item \hyperref[spaces-more-groupoids-section-phantom]{More on Groupoids in Spaces}
\item \hyperref[bootstrap-section-phantom]{Bootstrap}
\item \hyperref[examples-stacks-section-phantom]{Examples of Stacks}
\item \hyperref[groupoids-quotients-section-phantom]{Quotients of Groupoids}
\item \hyperref[algebraic-section-phantom]{Algebraic Stacks}
\item \hyperref[criteria-section-phantom]{Criteria for Representability}
\item \hyperref[stacks-properties-section-phantom]{Properties of Algebraic Stacks}
\item \hyperref[stacks-morphisms-section-phantom]{Morphisms of Algebraic Stacks}
\item \hyperref[examples-section-phantom]{Examples}
\item \hyperref[exercises-section-phantom]{Exercises}
\item \hyperref[guide-section-phantom]{Guide to Literature}
\item \hyperref[desirables-section-phantom]{Desirables}
\item \hyperref[coding-section-phantom]{Coding Style}
\item \hyperref[fdl-section-phantom]{GNU Free Documentation License}
\item \hyperref[index-section-phantom]{Auto Generated Index}
\end{enumerate}
\end{multicols}


\bibliography{my}
\bibliographystyle{amsalpha}

\end{document}

