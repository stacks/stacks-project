\IfFileExists{stacks-project.cls}{%
\documentclass{stacks-project}
}{%
\documentclass{amsart}
}

% The following AMS packages are automatically loaded with
% the amsart documentclass:
%\usepackage{amsmath}
%\usepackage{amssymb}
%\usepackage{amsthm}

% For dealing with references we use the comment environment
\usepackage{verbatim}
\newenvironment{reference}{\comment}{\endcomment}
%\newenvironment{reference}{}{}
\newenvironment{slogan}{\comment}{\endcomment}
\newenvironment{history}{\comment}{\endcomment}

% For commutative diagrams you can use
% \usepackage{amscd}
\usepackage[all]{xy}

% We use 2cell for 2-commutative diagrams.
\xyoption{2cell}
\UseAllTwocells

% To put source file link in headers.
% Change "template.tex" to "this_filename.tex"
% \usepackage{fancyhdr}
% \pagestyle{fancy}
% \lhead{}
% \chead{}
% \rhead{Source file: \url{template.tex}}
% \lfoot{}
% \cfoot{\thepage}
% \rfoot{}
% \renewcommand{\headrulewidth}{0pt}
% \renewcommand{\footrulewidth}{0pt}
% \renewcommand{\headheight}{12pt}

\usepackage{multicol}

% For cross-file-references
\usepackage{xr-hyper}

% Package for hypertext links:
\usepackage{hyperref}

% For any local file, say "hello.tex" you want to link to please
% use \externaldocument[hello-]{hello}
\externaldocument[introduction-]{introduction}
\externaldocument[conventions-]{conventions}
\externaldocument[sets-]{sets}
\externaldocument[categories-]{categories}
\externaldocument[topology-]{topology}
\externaldocument[sheaves-]{sheaves}
\externaldocument[sites-]{sites}
\externaldocument[stacks-]{stacks}
\externaldocument[fields-]{fields}
\externaldocument[algebra-]{algebra}
\externaldocument[brauer-]{brauer}
\externaldocument[homology-]{homology}
\externaldocument[derived-]{derived}
\externaldocument[simplicial-]{simplicial}
\externaldocument[more-algebra-]{more-algebra}
\externaldocument[smoothing-]{smoothing}
\externaldocument[modules-]{modules}
\externaldocument[sites-modules-]{sites-modules}
\externaldocument[injectives-]{injectives}
\externaldocument[cohomology-]{cohomology}
\externaldocument[sites-cohomology-]{sites-cohomology}
\externaldocument[dga-]{dga}
\externaldocument[dpa-]{dpa}
\externaldocument[hypercovering-]{hypercovering}
\externaldocument[schemes-]{schemes}
\externaldocument[constructions-]{constructions}
\externaldocument[properties-]{properties}
\externaldocument[morphisms-]{morphisms}
\externaldocument[coherent-]{coherent}
\externaldocument[divisors-]{divisors}
\externaldocument[limits-]{limits}
\externaldocument[varieties-]{varieties}
\externaldocument[topologies-]{topologies}
\externaldocument[descent-]{descent}
\externaldocument[perfect-]{perfect}
\externaldocument[more-morphisms-]{more-morphisms}
\externaldocument[flat-]{flat}
\externaldocument[groupoids-]{groupoids}
\externaldocument[more-groupoids-]{more-groupoids}
\externaldocument[etale-]{etale}
\externaldocument[chow-]{chow}
\externaldocument[intersection-]{intersection}
\externaldocument[pic-]{pic}
\externaldocument[adequate-]{adequate}
\externaldocument[dualizing-]{dualizing}
\externaldocument[duality-]{duality}
\externaldocument[discriminant-]{discriminant}
\externaldocument[local-cohomology-]{local-cohomology}
\externaldocument[curves-]{curves}
\externaldocument[resolve-]{resolve}
\externaldocument[models-]{models}
\externaldocument[pione-]{pione}
\externaldocument[etale-cohomology-]{etale-cohomology}
\externaldocument[proetale-]{proetale}
\externaldocument[crystalline-]{crystalline}
\externaldocument[spaces-]{spaces}
\externaldocument[spaces-properties-]{spaces-properties}
\externaldocument[spaces-morphisms-]{spaces-morphisms}
\externaldocument[decent-spaces-]{decent-spaces}
\externaldocument[spaces-cohomology-]{spaces-cohomology}
\externaldocument[spaces-limits-]{spaces-limits}
\externaldocument[spaces-divisors-]{spaces-divisors}
\externaldocument[spaces-over-fields-]{spaces-over-fields}
\externaldocument[spaces-topologies-]{spaces-topologies}
\externaldocument[spaces-descent-]{spaces-descent}
\externaldocument[spaces-perfect-]{spaces-perfect}
\externaldocument[spaces-more-morphisms-]{spaces-more-morphisms}
\externaldocument[spaces-flat-]{spaces-flat}
\externaldocument[spaces-groupoids-]{spaces-groupoids}
\externaldocument[spaces-more-groupoids-]{spaces-more-groupoids}
\externaldocument[bootstrap-]{bootstrap}
\externaldocument[spaces-pushouts-]{spaces-pushouts}
\externaldocument[groupoids-quotients-]{groupoids-quotients}
\externaldocument[spaces-more-cohomology-]{spaces-more-cohomology}
\externaldocument[spaces-simplicial-]{spaces-simplicial}
\externaldocument[formal-spaces-]{formal-spaces}
\externaldocument[restricted-]{restricted}
\externaldocument[spaces-resolve-]{spaces-resolve}
\externaldocument[formal-defos-]{formal-defos}
\externaldocument[defos-]{defos}
\externaldocument[cotangent-]{cotangent}
\externaldocument[examples-defos-]{examples-defos}
\externaldocument[algebraic-]{algebraic}
\externaldocument[examples-stacks-]{examples-stacks}
\externaldocument[stacks-sheaves-]{stacks-sheaves}
\externaldocument[criteria-]{criteria}
\externaldocument[artin-]{artin}
\externaldocument[quot-]{quot}
\externaldocument[stacks-properties-]{stacks-properties}
\externaldocument[stacks-morphisms-]{stacks-morphisms}
\externaldocument[stacks-limits-]{stacks-limits}
\externaldocument[stacks-cohomology-]{stacks-cohomology}
\externaldocument[stacks-perfect-]{stacks-perfect}
\externaldocument[stacks-introduction-]{stacks-introduction}
\externaldocument[stacks-more-morphisms-]{stacks-more-morphisms}
\externaldocument[stacks-geometry-]{stacks-geometry}
\externaldocument[moduli-]{moduli}
\externaldocument[moduli-curves-]{moduli-curves}
\externaldocument[examples-]{examples}
\externaldocument[exercises-]{exercises}
\externaldocument[guide-]{guide}
\externaldocument[desirables-]{desirables}
\externaldocument[coding-]{coding}
\externaldocument[obsolete-]{obsolete}
\externaldocument[fdl-]{fdl}
\externaldocument[index-]{index}

% Theorem environments.
%
\theoremstyle{plain}
\newtheorem{theorem}[subsection]{Theorem}
\newtheorem{proposition}[subsection]{Proposition}
\newtheorem{lemma}[subsection]{Lemma}

\theoremstyle{definition}
\newtheorem{definition}[subsection]{Definition}
\newtheorem{example}[subsection]{Example}
\newtheorem{exercise}[subsection]{Exercise}
\newtheorem{situation}[subsection]{Situation}

\theoremstyle{remark}
\newtheorem{remark}[subsection]{Remark}
\newtheorem{remarks}[subsection]{Remarks}

\numberwithin{equation}{subsection}

% Macros
%
\def\lim{\mathop{\rm lim}\nolimits}
\def\colim{\mathop{\rm colim}\nolimits}
\def\Spec{\mathop{\rm Spec}}
\def\Hom{\mathop{\rm Hom}\nolimits}
\def\Ext{\mathop{\rm Ext}\nolimits}
\def\SheafHom{\mathop{\mathcal{H}\!{\it om}}\nolimits}
\def\SheafExt{\mathop{\mathcal{E}\!{\it xt}}\nolimits}
\def\Sch{\textit{Sch}}
\def\Mor{\mathop{\rm Mor}\nolimits}
\def\Ob{\mathop{\rm Ob}\nolimits}
\def\Sh{\mathop{\textit{Sh}}\nolimits}
\def\NL{\mathop{N\!L}\nolimits}
\def\proetale{{pro\text{-}\acute{e}tale}}
\def\etale{{\acute{e}tale}}
\def\QCoh{\textit{QCoh}}
\def\Ker{\mathop{\rm Ker}}
\def\Im{\mathop{\rm Im}}
\def\Coker{\mathop{\rm Coker}}
\def\Coim{\mathop{\rm Coim}}

%
% Macros for moduli stacks/spaces
%
\def\QCohstack{\mathcal{QC}\!{\it oh}}
\def\Cohstack{\mathcal{C}\!{\it oh}}
\def\Spacesstack{\mathcal{S}\!{\it paces}}
\def\Quotfunctor{{\rm Quot}}
\def\Hilbfunctor{{\rm Hilb}}
\def\Curvesstack{\mathcal{C}\!{\it urves}}
\def\Polarizedstack{\mathcal{P}\!{\it olarized}}
\def\Complexesstack{\mathcal{C}\!{\it omplexes}}
% \Pic is the operator that assigns to X its picard group, usage \Pic(X)
% \Picardstack_{X/B} denotes the Picard stack of X over B
% \Picardfunctor_{X/B} denotes the Picard functor of X over B
\def\Pic{\mathop{\rm Pic}\nolimits}
\def\Picardstack{\mathcal{P}\!{\it ic}}
\def\Picardfunctor{{\rm Pic}}
\def\Deformationcategory{\mathcal{D}\!{\it ef}}


% OK, start here.
%
\begin{document}

\title{Constructions of schemes}

%\begin{abstract}
%\end{abstract}

\maketitle

\tableofcontents

\section{Introduction}
\label{section-introduction}

\noindent
In this chapter we introduce ways of constructing schemes out of others.
A basic reference is \cite{EGA}.








\section{Relative spectrum}
\label{section-spec}



\noindent
Let $S$ be a scheme.
Let $\mathcal{A}$ be a quasi-coherent sheaf of $\mathcal{O}_S$-algebras.
This means that $\mathcal{A}$ is a sheaf of $\mathcal{O}_S$-algebras
which is quasi-coherent as an $\mathcal{O}_S$-module.

\medskip\noindent
For any $f : T \to S$ the pullback
$f^*\mathcal{A}$ is a quasi-coherent sheaf of $\mathcal{O}_T$-algebras.
We are going to consider pairs $(f : T \to S, \varphi)$ where
$f$ is a morphism of schemes and $\varphi : f^*\mathcal{A} \to \mathcal{O}_T$
is a morphism of $\mathcal{O}_T$-algebras. Note that this is the
same as giving a $f^{-1}\mathcal{O}_S$-algebra homomorphism
$\varphi : f^{-1}\mathcal{A} \to \mathcal{O}_T$, see
Sheaves, Lemma \ref{sheaves-lemma-adjointness-tensor-restrict}.
This is also the same as giving a $\mathcal{O}_S$-algebra map
$\varphi : \mathcal{A} \to f_*\mathcal{O}_T$, see
Sheaves, Lemma \ref{sheaves-lemma-adjoint-push-pull-modules}.
We will use all three ways of thinking about $\varphi$,
without further mention.

\medskip\noindent
Given such a
pair $(f : T \to S, \varphi)$ and a morphism $a : T' \to T$ we get
a second pair $(f' = f \circ a, \varphi' = a^*\varphi)$ which we
call the pull back of $(f, \varphi)$. One way to describe
$\varphi' = a^*\varphi$ is as the composition
$\mathcal{A} \to f_*\mathcal{O}_T \to f'_*\mathcal{O}_{T'}$
where the second map is $f_*a^\sharp$ with
$a^\sharp : \mathcal{O}_T \to a_*\mathcal{O}_{T'}$.
In this way we have defined a contravariant functor
\begin{eqnarray}
\label{equation-spec}
F : \textit{Sch} & \longrightarrow & \textit{Sets} \\
T & \longmapsto & F(T) = \{\text{pairs }(f, \varphi) \text{ as above}\}
\nonumber
\end{eqnarray}

\begin{lemma}
\label{lemma-spec-base-change}
Let $S$ be a scheme. Let $\mathcal{A}$ be a quasi-coherent
sheaf of $\mathcal{O}_S$-algebras. Let $F$ be the functor
associated to $(S, \mathcal{A})$ above.
Let $g : S' \to S$ be a morphism of schemes.
Set $\mathcal{A}' = g^*\mathcal{A}$. Let $F'$ be the
functor associated to $(S', \mathcal{A}')$ above.
Then there is a canonical isomorphism
$$
F' \cong h_{S'} \times_{h_S} F
$$
of functors.
\end{lemma}

\begin{proof}
A pair $(f' : T \to S', \varphi' : (f')^*\mathcal{A}' \to \mathcal{O}_T)$
is the same as a pair $(f, \varphi : f^*\mathcal{A} \to \mathcal{O}_T)$
together with a factorization of $f$ as $f = g \circ f'$. Namely with
this notation we have
$(f')^* \mathcal{A}' = (f')^*g^*\mathcal{A} = f^*\mathcal{A}$.
Hence the lemma.
\end{proof}

\begin{lemma}
\label{lemma-spec-affine}
Let $S$ be a scheme. Let $\mathcal{A}$ be a quasi-coherent
sheaf of $\mathcal{O}_S$-algebras. Let $F$ be the functor
associated to $(S, \mathcal{A})$ above.
If $S$ is affine, then $F$ is representable by the
affine scheme $\text{Spec}(\Gamma(S, \mathcal{A}))$.
\end{lemma}

\begin{proof}
Write $S = \text{Spec}(R)$ and $A = \Gamma(S, \mathcal{A})$.
Then $A$ is an $R$-algebra and $\mathcal{A} = \widetilde A$.
The ring map $R \to A$ gives rise to a canonical map
$$
f_{univ} : \text{Spec}(A)
\longrightarrow
S = \text{Spec}(R).
$$
We have
$f_{univ}^*\mathcal{A} =  \widetilde{A \otimes_R A}$
by Schemes, Lemma \ref{schemes-lemma-widetilde-pullback}.
Hence there is a canonical map
$$
\varphi_{univ} :
f_{univ}^*\mathcal{A} = \widetilde{A \otimes_R A}
\longrightarrow
\widetilde A = \mathcal{O}_{\text{Spec}(A)}
$$
coming from the $A$-module map $A \otimes_R A \to A$,
$a \otimes a' \mapsto aa'$. We claim that the pair
$(f_{univ}, \varphi_{univ})$ represents $F$ in this case.
In other words we claim that for any scheme $T$ the map
$$
\text{Mor}(T, \text{Spec}(A)) \longrightarrow
\{\text{pairs } (f, \varphi)\},\ \ 
a \longmapsto (a^*f_{univ}, a^*\varphi)
$$
is bijective.

\medskip\noindent
Let us construct the inverse map.
For any pair $(f : T \to S, \varphi)$ we get the induced
ring map
$$
\xymatrix{
A = \Gamma(S, \mathcal{A}) \ar[r]^{f^*} &
\Gamma(T, f^*\mathcal{A}) \ar[r]^{\varphi} &
\Gamma(T, \mathcal{O}_T)
}
$$
This induces a morphism of schemes $T \to \text{Spec}(A)$
by Schemes, Lemma \ref{schemes-lemma-morphism-into-affine}.

\medskip\noindent
The verification that this map is inverse to the map
displayed above is omitted.
\end{proof}

\begin{lemma}
\label{lemma-spec}
The functor $F$ is representable by a scheme.
\end{lemma}

\begin{proof}
We are going to use Schemes, Lemma \ref{schemes-lemma-glue-functors}.

\medskip\noindent
First we check that $F$ satisfies the sheaf property for the
Zariski topology. Namely, suppose that $T$ is a scheme,
that $T = \bigcup_{i \in I} U_i$ is an open covering,
and that $(f_i, \varphi_i) \in F(U_i)$ such that
$(f_i, \varphi_i)|_{U_i \cap U_j} = (f_j, \varphi_j)|_{U_i \cap U_j}$.
This implies that the morphisms $f_i : U_i \to S$
glue to a morphism of schemes $f : T \to S$ such that
$f|_{I_i} = f_i$, see Schemes, Section \ref{schemes-section-glueing-schemes}.
Thus $f_i^*\mathcal{A} = f^*\mathcal{A}|_{U_i}$ and by assumption the
morphisms $\varphi_i$ agree on $U_i \cap U_j$. Hence by Sheaves,
Section \ref{sheaves-section-glueing-sheaves} these glue to a
morphism of $\mathcal{O}_T$-algebras $f^*\mathcal{A} \to \mathcal{O}_T$.
This proves that $F$ satisfies the sheaf condition with respect to
the Zariski topology.

\medskip\noindent
Let $S = \bigcup_{i \in I} U_i$ be an affine open covering.
Let $F_i \subset F$ be the subfunctor consisting of
those pairs $(f : T \to S, \varphi)$ such that
$f(T) \subset U_i$.

\medskip\noindent
We have to show each $F_i$ is representable.
This is the case because $F_i$ is identified with
the functor associated to $U_i$ equipped with
the quasi-coherent $\mathcal{O}_{U_i}$-algebra $\mathcal{A}|_{U_i})$.
Thus the result follows from Lemma \ref{lemma-spec-affine}.

\medskip\noindent
Next we show that $F_i \subset F$ is representable by open immersions.
Let $(f : T \to S, \varphi) \in F(T)$. Consider $V_i = f^{-1}(U_i)$.
It follows from the definition of $F_i$ that given $a : T' \to T$
we gave $a^*(f, \varphi) \in F_i(T')$ if and only if $a(T') \subset V_i$.
This is what we were required to show.

\medskip\noindent
Finally, we have to show that the collection $(F_i)_{i \in I}$
covers $F$. Let $(f : T \to S, \varphi) \in F(T)$.
Consider $V_i = f^{-1}(U_i)$. Since $S = \bigcup_{i \in I} U_i$
is an open covering of $S$ we see that $T = \bigcup_{i \in I} V_i$
is an open covering of $T$. Moreover $(f, \varphi)|_{V_i} \in F_i(V_i)$.
This finishes the proof of the lemma.
\end{proof}

\begin{definition}
\label{definition-spec}
Let $S$ be a scheme. Let $\mathcal{A}$ be a quasi-coherent
sheaf of $\mathcal{O}_S$-algebras. We denote
$\underline{\text{Spec}}_S(\mathcal{A})$ a scheme
representing the functor $F$ introduced in Equation
\ref{equation-spec} above. This is called the {\it relative
spectrum of $\mathcal{A}$ over $S$}. The universal family
is a pair $f_{univ} : \underline{\text{Spec}}_S(\mathcal{A}) \to S$\ and a morphism of $\mathcal{O}_S$-algebras
$$
\mathcal{A}
\longrightarrow
f_{univ, *}\mathcal{O}_{\underline{\text{Spec}}_S(\mathcal{A})}
$$
\end{definition}

\noindent
The following lemma says among other things that forming the
relative spectrum commutes with base change.

\begin{lemma}
\label{lemma-spec-properties}
Let $S$ be a scheme. Let $\mathcal{A}$ be a quasi-coherent
sheaf of $\mathcal{O}_S$-algebras. Let
$f_{univ} : \underline{\text{Spec}}_S(\mathcal{A}) \to S$
be the relative spectrum of $\mathcal{A}$ over $S$.
\begin{enumerate}
\item For every affine open $U \subset S$ the inverse image
$f^{-1}(U)$ is affine.
\item For every morphism $g : S' \to S$ we have
$S' \times_S \underline{\text{Spec}}_S(\mathcal{A}) =
\underline{\text{Spec}}_{S'}(g^*\mathcal{A})$.
\item
The universal
map
$$
\mathcal{A}
\longrightarrow
f_{univ, *}\mathcal{O}_{\underline{\text{Spec}}_S(\mathcal{A})}
$$
is an isomorphism of $\mathcal{O}_S$-algebras.
\end{enumerate}
\end{lemma}

\begin{proof}
We first note that (2) follows immediately from
Lemma \ref{lemma-spec-base-change}. Then (1) follows because
$f^{-1}(U) = U \times_S \underline{\text{Spec}}_S(\mathcal{A}) =
\underline{\text{Spec}}_U(\mathcal{A}|_U)$ combined with
Lemma \ref{lemma-spec-affine}. Finally, (3) follows because
it is local on $S$ and it is clear in case $S$ is affine
by Lemma \ref{lemma-spec-affine} again.
\end{proof}






\section{Vector bundles}
\label{section-vector-bundle}

\noindent
Let $S$ be a scheme.
Let $\mathcal{E}$ be a quasi-coherent sheaf of $\mathcal{O}_S$-modules.
For every integer $n \geq 0$ we may consider the
$n$th symmetric


\begin{definition}
\label{definition-vector-bundle}

















\section{Other chapters}

\begin{multicols}{2}
\begin{enumerate}
\item \hyperref[introduction-section-phantom]{Introduction}
\item \hyperref[conventions-section-phantom]{Conventions}
\item \hyperref[sets-section-phantom]{Set Theory}
\item \hyperref[categories-section-phantom]{Categories}
\item \hyperref[topology-section-phantom]{Topology}
\item \hyperref[sheaves-section-phantom]{Sheaves on Spaces}
\item \hyperref[algebra-section-phantom]{Commutative Algebra}
\item \hyperref[sites-section-phantom]{Sites and Sheaves}
\item \hyperref[homology-section-phantom]{Homological Algebra}
\item \hyperref[derived-section-phantom]{Derived Categories}
\item \hyperref[more-algebra-section-phantom]{More Algebra}
\item \hyperref[simplicial-section-phantom]{Simplicial Methods}
\item \hyperref[modules-section-phantom]{Sheaves of Modules}
\item \hyperref[sites-modules-section-phantom]{Modules on Sites}
\item \hyperref[injectives-section-phantom]{Injectives}
\item \hyperref[cohomology-section-phantom]{Cohomology of Sheaves}
\item \hyperref[sites-cohomology-section-phantom]{Cohomology on Sites}
\item \hyperref[hypercovering-section-phantom]{Hypercoverings}
\item \hyperref[schemes-section-phantom]{Schemes}
\item \hyperref[constructions-section-phantom]{Constructions of Schemes}
\item \hyperref[properties-section-phantom]{Properties of Schemes}
\item \hyperref[morphisms-section-phantom]{Morphisms of Schemes}
\item \hyperref[coherent-section-phantom]{Coherent Cohomology}
\item \hyperref[divisors-section-phantom]{Divisors}
\item \hyperref[limits-section-phantom]{Limits of Schemes}
\item \hyperref[varieties-section-phantom]{Varieties}
\item \hyperref[chow-section-phantom]{Chow Homology}
\item \hyperref[topologies-section-phantom]{Topologies on Schemes}
\item \hyperref[descent-section-phantom]{Descent}
\item \hyperref[more-morphisms-section-phantom]{More on Morphisms}
\item \hyperref[flat-section-phantom]{More on Flatness}
\item \hyperref[groupoids-section-phantom]{Groupoid Schemes}
\item \hyperref[more-groupoids-section-phantom]{More on Groupoid Schemes}
\item \hyperref[etale-section-phantom]{\'Etale Morphisms of Schemes}
\item \hyperref[etale-cohomology-section-phantom]{\'Etale Cohomology}
\item \hyperref[spaces-section-phantom]{Algebraic Spaces}
\item \hyperref[spaces-properties-section-phantom]{Properties of Algebraic Spaces}
\item \hyperref[spaces-morphisms-section-phantom]{Morphisms of Algebraic Spaces}
\item \hyperref[spaces-topologies-section-phantom]{Topologies on Algebraic Spaces}
\item \hyperref[spaces-descent-section-phantom]{Descent and Algebraic Spaces}
\item \hyperref[spaces-more-morphisms-section-phantom]{More on Morphisms of Spaces}
\item \hyperref[quot-section-phantom]{Quot and Hilbert Spaces}
\item \hyperref[stacks-section-phantom]{Stacks}
\item \hyperref[spaces-groupoids-section-phantom]{Groupoids in Algebraic Spaces}
\item \hyperref[spaces-more-groupoids-section-phantom]{More on Groupoids in Spaces}
\item \hyperref[bootstrap-section-phantom]{Bootstrap}
\item \hyperref[examples-stacks-section-phantom]{Examples of Stacks}
\item \hyperref[groupoids-quotients-section-phantom]{Quotients of Groupoids}
\item \hyperref[algebraic-section-phantom]{Algebraic Stacks}
\item \hyperref[criteria-section-phantom]{Criteria for Representability}
\item \hyperref[stacks-properties-section-phantom]{Properties of Algebraic Stacks}
\item \hyperref[stacks-morphisms-section-phantom]{Morphisms of Algebraic Stacks}
\item \hyperref[examples-section-phantom]{Examples}
\item \hyperref[exercises-section-phantom]{Exercises}
\item \hyperref[guide-section-phantom]{Guide to Literature}
\item \hyperref[desirables-section-phantom]{Desirables}
\item \hyperref[coding-section-phantom]{Coding Style}
\item \hyperref[fdl-section-phantom]{GNU Free Documentation License}
\item \hyperref[index-section-phantom]{Auto Generated Index}
\end{enumerate}
\end{multicols}


\bibliography{my}
\bibliographystyle{alpha}

\end{document}
