\IfFileExists{stacks-project.cls}{%
\documentclass{stacks-project}
}{%
\documentclass{amsart}
}

% The following AMS packages are automatically loaded with
% the amsart documentclass:
%\usepackage{amsmath}
%\usepackage{amssymb}
%\usepackage{amsthm}

% For dealing with references we use the comment environment
\usepackage{verbatim}
\newenvironment{reference}{\comment}{\endcomment}
%\newenvironment{reference}{}{}
\newenvironment{slogan}{\comment}{\endcomment}
\newenvironment{history}{\comment}{\endcomment}

% For commutative diagrams you can use
% \usepackage{amscd}
\usepackage[all]{xy}

% We use 2cell for 2-commutative diagrams.
\xyoption{2cell}
\UseAllTwocells

% To put source file link in headers.
% Change "template.tex" to "this_filename.tex"
% \usepackage{fancyhdr}
% \pagestyle{fancy}
% \lhead{}
% \chead{}
% \rhead{Source file: \url{template.tex}}
% \lfoot{}
% \cfoot{\thepage}
% \rfoot{}
% \renewcommand{\headrulewidth}{0pt}
% \renewcommand{\footrulewidth}{0pt}
% \renewcommand{\headheight}{12pt}

\usepackage{multicol}

% For cross-file-references
\usepackage{xr-hyper}

% Package for hypertext links:
\usepackage{hyperref}

% For any local file, say "hello.tex" you want to link to please
% use \externaldocument[hello-]{hello}
\externaldocument[introduction-]{introduction}
\externaldocument[conventions-]{conventions}
\externaldocument[sets-]{sets}
\externaldocument[categories-]{categories}
\externaldocument[topology-]{topology}
\externaldocument[sheaves-]{sheaves}
\externaldocument[sites-]{sites}
\externaldocument[stacks-]{stacks}
\externaldocument[fields-]{fields}
\externaldocument[algebra-]{algebra}
\externaldocument[brauer-]{brauer}
\externaldocument[homology-]{homology}
\externaldocument[derived-]{derived}
\externaldocument[simplicial-]{simplicial}
\externaldocument[more-algebra-]{more-algebra}
\externaldocument[smoothing-]{smoothing}
\externaldocument[modules-]{modules}
\externaldocument[sites-modules-]{sites-modules}
\externaldocument[injectives-]{injectives}
\externaldocument[cohomology-]{cohomology}
\externaldocument[sites-cohomology-]{sites-cohomology}
\externaldocument[dga-]{dga}
\externaldocument[dpa-]{dpa}
\externaldocument[hypercovering-]{hypercovering}
\externaldocument[schemes-]{schemes}
\externaldocument[constructions-]{constructions}
\externaldocument[properties-]{properties}
\externaldocument[morphisms-]{morphisms}
\externaldocument[coherent-]{coherent}
\externaldocument[divisors-]{divisors}
\externaldocument[limits-]{limits}
\externaldocument[varieties-]{varieties}
\externaldocument[topologies-]{topologies}
\externaldocument[descent-]{descent}
\externaldocument[perfect-]{perfect}
\externaldocument[more-morphisms-]{more-morphisms}
\externaldocument[flat-]{flat}
\externaldocument[groupoids-]{groupoids}
\externaldocument[more-groupoids-]{more-groupoids}
\externaldocument[etale-]{etale}
\externaldocument[chow-]{chow}
\externaldocument[intersection-]{intersection}
\externaldocument[pic-]{pic}
\externaldocument[adequate-]{adequate}
\externaldocument[dualizing-]{dualizing}
\externaldocument[duality-]{duality}
\externaldocument[discriminant-]{discriminant}
\externaldocument[local-cohomology-]{local-cohomology}
\externaldocument[curves-]{curves}
\externaldocument[resolve-]{resolve}
\externaldocument[models-]{models}
\externaldocument[pione-]{pione}
\externaldocument[etale-cohomology-]{etale-cohomology}
\externaldocument[proetale-]{proetale}
\externaldocument[crystalline-]{crystalline}
\externaldocument[spaces-]{spaces}
\externaldocument[spaces-properties-]{spaces-properties}
\externaldocument[spaces-morphisms-]{spaces-morphisms}
\externaldocument[decent-spaces-]{decent-spaces}
\externaldocument[spaces-cohomology-]{spaces-cohomology}
\externaldocument[spaces-limits-]{spaces-limits}
\externaldocument[spaces-divisors-]{spaces-divisors}
\externaldocument[spaces-over-fields-]{spaces-over-fields}
\externaldocument[spaces-topologies-]{spaces-topologies}
\externaldocument[spaces-descent-]{spaces-descent}
\externaldocument[spaces-perfect-]{spaces-perfect}
\externaldocument[spaces-more-morphisms-]{spaces-more-morphisms}
\externaldocument[spaces-flat-]{spaces-flat}
\externaldocument[spaces-groupoids-]{spaces-groupoids}
\externaldocument[spaces-more-groupoids-]{spaces-more-groupoids}
\externaldocument[bootstrap-]{bootstrap}
\externaldocument[spaces-pushouts-]{spaces-pushouts}
\externaldocument[groupoids-quotients-]{groupoids-quotients}
\externaldocument[spaces-more-cohomology-]{spaces-more-cohomology}
\externaldocument[spaces-simplicial-]{spaces-simplicial}
\externaldocument[formal-spaces-]{formal-spaces}
\externaldocument[restricted-]{restricted}
\externaldocument[spaces-resolve-]{spaces-resolve}
\externaldocument[formal-defos-]{formal-defos}
\externaldocument[defos-]{defos}
\externaldocument[cotangent-]{cotangent}
\externaldocument[examples-defos-]{examples-defos}
\externaldocument[algebraic-]{algebraic}
\externaldocument[examples-stacks-]{examples-stacks}
\externaldocument[stacks-sheaves-]{stacks-sheaves}
\externaldocument[criteria-]{criteria}
\externaldocument[artin-]{artin}
\externaldocument[quot-]{quot}
\externaldocument[stacks-properties-]{stacks-properties}
\externaldocument[stacks-morphisms-]{stacks-morphisms}
\externaldocument[stacks-limits-]{stacks-limits}
\externaldocument[stacks-cohomology-]{stacks-cohomology}
\externaldocument[stacks-perfect-]{stacks-perfect}
\externaldocument[stacks-introduction-]{stacks-introduction}
\externaldocument[stacks-more-morphisms-]{stacks-more-morphisms}
\externaldocument[stacks-geometry-]{stacks-geometry}
\externaldocument[moduli-]{moduli}
\externaldocument[moduli-curves-]{moduli-curves}
\externaldocument[examples-]{examples}
\externaldocument[exercises-]{exercises}
\externaldocument[guide-]{guide}
\externaldocument[desirables-]{desirables}
\externaldocument[coding-]{coding}
\externaldocument[obsolete-]{obsolete}
\externaldocument[fdl-]{fdl}
\externaldocument[index-]{index}

% Theorem environments.
%
\theoremstyle{plain}
\newtheorem{theorem}[subsection]{Theorem}
\newtheorem{proposition}[subsection]{Proposition}
\newtheorem{lemma}[subsection]{Lemma}

\theoremstyle{definition}
\newtheorem{definition}[subsection]{Definition}
\newtheorem{example}[subsection]{Example}
\newtheorem{exercise}[subsection]{Exercise}
\newtheorem{situation}[subsection]{Situation}

\theoremstyle{remark}
\newtheorem{remark}[subsection]{Remark}
\newtheorem{remarks}[subsection]{Remarks}

\numberwithin{equation}{subsection}

% Macros
%
\def\lim{\mathop{\rm lim}\nolimits}
\def\colim{\mathop{\rm colim}\nolimits}
\def\Spec{\mathop{\rm Spec}}
\def\Hom{\mathop{\rm Hom}\nolimits}
\def\Ext{\mathop{\rm Ext}\nolimits}
\def\SheafHom{\mathop{\mathcal{H}\!{\it om}}\nolimits}
\def\SheafExt{\mathop{\mathcal{E}\!{\it xt}}\nolimits}
\def\Sch{\textit{Sch}}
\def\Mor{\mathop{\rm Mor}\nolimits}
\def\Ob{\mathop{\rm Ob}\nolimits}
\def\Sh{\mathop{\textit{Sh}}\nolimits}
\def\NL{\mathop{N\!L}\nolimits}
\def\proetale{{pro\text{-}\acute{e}tale}}
\def\etale{{\acute{e}tale}}
\def\QCoh{\textit{QCoh}}
\def\Ker{\mathop{\rm Ker}}
\def\Im{\mathop{\rm Im}}
\def\Coker{\mathop{\rm Coker}}
\def\Coim{\mathop{\rm Coim}}

%
% Macros for moduli stacks/spaces
%
\def\QCohstack{\mathcal{QC}\!{\it oh}}
\def\Cohstack{\mathcal{C}\!{\it oh}}
\def\Spacesstack{\mathcal{S}\!{\it paces}}
\def\Quotfunctor{{\rm Quot}}
\def\Hilbfunctor{{\rm Hilb}}
\def\Curvesstack{\mathcal{C}\!{\it urves}}
\def\Polarizedstack{\mathcal{P}\!{\it olarized}}
\def\Complexesstack{\mathcal{C}\!{\it omplexes}}
% \Pic is the operator that assigns to X its picard group, usage \Pic(X)
% \Picardstack_{X/B} denotes the Picard stack of X over B
% \Picardfunctor_{X/B} denotes the Picard functor of X over B
\def\Pic{\mathop{\rm Pic}\nolimits}
\def\Picardstack{\mathcal{P}\!{\it ic}}
\def\Picardfunctor{{\rm Pic}}
\def\Deformationcategory{\mathcal{D}\!{\it ef}}


% OK, start here.
%
\begin{document}

\title{Etale Cohomology: Prologue}


\maketitle

\phantomsection
\label{section-phantom}

\tableofcontents


\section{Introduction}
\label{section-introduction}

\noindent
These are the notes of a course on \'etale cohomology taught by Johan de Jong
at Columbia University in the Fall of 2009. The original note takers were
Thibaut Pugin, Zachary Maddock and Min Lee. Over time we will add references
to background material in the rest of the stacks project and provide rigorous
proofs of all the statements.




%9.08.09
\section{Prologue}
\label{section-prologue}

\noindent
These lectures are about another cohomology theory. The first thing to remark
is that the Zariski topology is not entirely satisfactory. One of the main
reasons that it fails to give the results that we would want is that if $X$ is
a complex variety and $\mathcal{F}$ is a constant sheaf then
$$
H^i(X, \mathcal{F}) = 0, \quad \text{ for all } i > 0.
$$
The reason for that is the following. In an irreducible scheme (a variety in
particular), any two nonempty open subsets meet, and so the restriction
mappings of a constant sheaf are surjective. We say that the sheaf is
{\it flasque}. In this case, all higher \u Cech cohomology groups vanish, and
so do all higher Zariski cohomology groups. In other words, there are ``not
enough'' open sets in the Zariski topology to detect this higher cohomology.

\medskip\noindent
On the other hand, if $X$ is a smooth projective complex variety, then
$$
H_{Betti}^{2 \dim X}(X (\mathbf{C}), \Lambda) = \Lambda \quad \text{ for }
\Lambda = \mathbf{Z}, \ \mathbf{Z}/n\mathbf{Z},
$$
where $X(\mathbf{C})$ means the set of complex points of $X$. This is a feature
that would be nice to replicate in algebraic geometry. In positive
characteristic in particular.




\section{The \'etale topology}
\label{section-etale-topology}

\noindent
It is very hard to simply ``add'' extra open sets to refine the Zariski
topology. One efficient way to define a topology is to consider not only open
sets, but also some schemes that lie over them. To define the \'etale topology,
one considers all morphisms $\varphi: U \to X$ which are \'etale. If
$X$ is a smooth projective variety over $\mathbf{C}$, then this means
\begin{enumerate}
\item $U$ is a disjoint union of smooth varieties ; and
\item $\varphi$ is (analytically) locally an isomorphism.
\end{enumerate}
The word ``analytically'' refers to the usual (transcendental) topology over
$\mathbf{C}$. So the second condition means that the derivative of $\varphi$
has full rank everywhere (and in particular all the components of $U$
have the same dimension as $X$).

\medskip\noindent
A double cover -- loosely defined as a finite degree $2$ map between varieties
-- for example
$$
\text{Spec}(\mathbf{C}[t])
\longrightarrow
\text{Spec}(\mathbf{C}[t]),
\quad t \longmapsto t^2
$$
will not be an \'etale morphism if it has a fibre consisting of a single point.
In the example this happens when $t = 0$. For a finite map between varieties
over $\mathbf{C}$ to be etale all the fibers should have the same number of
points. Removing the point $t = 0$ from the source of the map in the example
will make the morphism \'etale. But we can remove other points from the source
of the morphism also, and the morphism will still be \'etale.  To consider the
\'etale topology, we have to look at all such morphisms. Unlike the Zariski
topology, these need not be merely be open subsets of $X$, even though their
images always are.

\begin{definition}
\label{definition-etale-covering-initial}
A family of morphisms $\{ \varphi_i : U_i \to X\}_{i \in I}$ is
called an {\it \'etale covering} if each $\varphi_i$ is an \'etale morphism
and their images cover $X$, i.e.,
$X = \bigcup_{i \in I} \varphi_i(U_i)$.
\end{definition}

\noindent
This ``defines'' the \'etale topology. In other words, we can now say what the
sheaves are. An {\it \'etale sheaf} $\mathcal{F}$ of sets
(resp.\ abelian groups, vector spaces, etc) on $X$ is the data:
\begin{enumerate}
\item for each \'etale morphism $\varphi : U \to X$ a set
(resp.\ abelian group, vector space, etc) $\mathcal{F}(U)$,
\item for each pair $U, \ U'$ of \'etale schemes over $X$,
and each morphism $U \to U'$ over $X$ (which is
automatically \'etale) a restriction map $\rho^{U}_{U'}
: \mathcal{F}(U) \to \mathcal{F}(U')$
\end{enumerate}
These data have to satisfy the following {\it sheaf axiom}:
\begin{list}{(*)}{}
\item for every \'etale covering $\{ \varphi_i : U_i \to X\}_{i \in
I}$, the diagram
$$
\xymatrix{
\emptyset \ar[r] &
\mathcal{F} (U) \ar[r] &
\Pi_{i \in I} \mathcal{F} (U_i) \ar@<1ex>[r] \ar@<-1ex>[r] &
\Pi_{i,j \in I} \mathcal{F} (U_i \times_U U_j)
}
$$
is exact in the category of sets (resp.\ abelian groups, vector spaces, etc).
\end{list}

\begin{remark}
\label{remark-i-is-j}
In the last statement, it is essential not to forget the case where $i = j$
which is in general a highly nontrivial condition (unlike in the Zariski
topology). In fact, frequently important coverings have only one element.
\end{remark}

\noindent
Since the identity is an \'etale morphism, we can compute the global sections
of an \'etale sheaf, and cohomology will simply be the corresponding
right-derived functors. In other words, once more theory has been developed and
statements have been made precise, there will be no obstacle to defining
cohomology.




\section{Feats of the \'etale topology}
\label{section-feats}

\noindent
For a natural number $n \in \mathbf{N} = \{1, 2, 3, 4, \dots\}$ it is true that
$$
H_{et}^2 (\mathbf{P}^1_\mathbf{C}, \mathbf{Z}/n\mathbf{Z}) =
\mathbf{Z}/n\mathbf{Z}.
$$
More generally, if $X$ is a complex variety, then its \'etale Betti numbers
with coefficients in a finite field agree with the usual Betti numbers of
$X(\mathbf{C})$, i.e.,
$$
\dim_{\mathbf{F}_q} H_{et}^{2i} (X, \mathbf{F}_q) = \dim_{\mathbf{F}_q}
H_{Betti}^{2i} (X(\mathbf{C}), \mathbf{F}_q).
$$
This is extremely satisfactory. However, these equalities only hold for torsion
coefficients, not in general. For integer coefficients, one has
$$
H_{et}^2 (\mathbf{P}^1_\mathbf{C}, \mathbf{Z}) = 0.
$$
There are ways to get back to nontorsion coefficients from torsion ones by a
limit procedure which we will come to shortly.




\section{A computation}
\label{section-computation}

\noindent
How do we compute the cohomology of $\mathbf{P}^1_\mathbf{C}$ with coefficients
$\Lambda = \mathbf{Z}/n\mathbf{Z}$?
We use \u Cech cohomology. A covering of $\mathbf{P}^1_\mathbf{C}$ is given by
the two standard opens $U_0, U_1$, which are both
isomorphic to $\mathbf{A}^1_\mathbf{C}$, and which intersection is isomorphic
to $\mathbf{A}^1_\mathbf{C} \setminus \{0\} = \mathbf{G}_{m, \mathbf{C}}$.
It turns out that the Mayer-Vietoris sequence holds in \'etale cohomology.
This gives an exact sequence
$$
H_{et}^{i-1}(U_0\cap U_1, \Lambda) \to
H_{et}^i(\mathbf{P}^1_C, \Lambda) \to H_{et}^i(U_0, \Lambda) \oplus
H_{et}^i(U_1, \Lambda) \to H_{et}^i(U_0\cap U_1,
\Lambda).
$$
To get the answer we expect, we would need to show that the direct sum in the
third term vanishes. In fact, it is true that, as for the usual topology,
$$
H_{et}^q (\mathbf{A}^1_\mathbf{C}, \Lambda) = 0 \quad \text{ for } q \geq 1,
$$
and
$$
H_{et}^q (\mathbf{A}^1_\mathbf{C} \setminus \{0\}, \Lambda) = \left\{
\begin{matrix}
\Lambda & \text{ if $q = 1$, and} \\
0 & \text{ for $q \geq 2$.}
\end{matrix}
\right.
$$
These results are already quite hard (what is an elementary proof?). Let us
explain how we would compute this once the machinery of \'etale cohomology is
at our disposal.

\medskip\noindent
{\bf Higher cohomology.} This is taken care of by the following general
fact: if $X$ is an affine curve over $\mathbf{C}$, then
$$
H_{et}^q (X, \mathbf{Z}/n\mathbf{Z}) = 0 \quad \text{ for } q \geq 2.
$$
This is proved by considering the generic point of the curve and doing some
Galois cohomology. So we only have to worry about the cohomology in degree 1.

\medskip\noindent
{\bf Cohomology in degree 1.} We use the following identifications:
\begin{eqnarray*}
H_{et}^1 (X, \mathbf{Z}/n\mathbf{Z}) = \left\{
\begin{matrix}
\text{sheaves of sets $\mathcal{F}$ on the \'etale site $X_{\text{\'et}}$
endowed with an} \\
\text{action $\mathbf{Z}/n\mathbf{Z} \times \mathcal{F} \to \mathcal{F}$ such
that $\mathcal{F}$ is a $\mathbf{Z}/n\mathbf{Z}$-torsor.}
\end{matrix}
\right\}
\Big/ \cong
\\
 = \left\{
\begin{matrix}
\text{morphisms $Y \to X$ which are finite \'etale together} \\
\text{ with a free $\mathbf{Z}/n\mathbf{Z}$ action such that $X = Y
/(\mathbf{Z}/n\mathbf{Z})$.}
\end{matrix}
\right\}
\Big/ \cong.
\end{eqnarray*}
The first identification is very general (it is true for any cohomology theory
on a site) and has nothing to do with the \'etale topology. The second
identification is a consequence of descent theory. The last set describes a
collection of geometric objects on which we can get our hands.

\medskip\noindent
The curve $\mathbf{A}^1_\mathbf{C}$ has no nontrivial finite \'etale covering
and hence $H_{et}^1 (\mathbf{A}^1_\mathbf{C}, \mathbf{Z}/n\mathbf{Z}) = 0$.
This can be seen either topologically or by using the argument in the next
paragraph.

\medskip\noindent
Let us describe the finite \'etale coverings
$\varphi: Y \to \mathbf{A}^1_\mathbf{C} \setminus \{0\}$.
It suffices to consider the case where $Y$ is
connected, which we assume. We are going to find out what $Y$ can be
by applying the Riemann-Hurwitz formula (of course this is a bit silly, and
you can go ahead and skip the the next section if you like).
Say that this morphism is $n$ to 1, and consider a
projective compactification
$$
\xymatrix{
{Y\ } \ar@{^{(}->}[r] \ar^{\varphi}[d] &
{\bar Y} \ar^{\bar\varphi}[d] \\
{\mathbf{A}^1_\mathbf{C} \setminus \{0\}} \ar@{^{(}->}[r] &
{\mathbf{P}^1_\mathbf{C}}
}
$$
Even though $\varphi$ is \'etale and does not ramify, $\bar{\varphi}$ may
ramify at 0 and $\infty$. Say that the preimages of 0 are the points $y_1,
\dots, y_r$ with indices of ramification $e_1, \dots e_r$, and that the
preimages of $\infty$ are the points $y_1', \dots, y_s'$ with indices of
ramification $d_1, \dots d_s$. In particular, $\sum e_i = n = \sum d_j$.
Applying the Riemann-Hurwitz formula, we get
$$
2 g_Y - 2 = -2n + \sum (e_i - 1) + \sum (d_j - 1)
$$
and therefore $g_Y = 0$, $r=s=1$ and $e_1 = d_1 = n$.
Hence $Y \cong {\mathbf{A}^1_\mathbf{C} \setminus \{0\}}$, and it is easy to
see that $\varphi(z) = \lambda z^n$ for some $\lambda \in \mathbf{C}^*$.
After reparametrizing $Y$ we may assume $\lambda = 1$. Thus our
covering is given by taking the $n$th root of the coordinate on
$\mathbf{A}^1_{\mathbf{C}} \setminus \{0\}$.

\medskip\noindent
Remember that we need to classify the coverings of
${\mathbf{A}^1_\mathbf{C} \setminus \{0\}}$ together with free
$\mathbf{Z}/n\mathbf{Z}$-actions on them.
In our case any such action corresponds
to an automorphism of $Y$ sending $z$ to $\zeta_n z$, where $\zeta_n$ is a
primitive $n$th root of unity. There are $\phi(n)$ such actions
(here $\phi(n)$ means the Euler function). Thus there are exactly
$\phi(n)$ connected finite \'etale coverings with a given free
$\mathbf{Z}/n\mathbf{Z}$-action, each corresponding to a primitive
$n$th root of unity. We leave it to the reader to see that the
disconnected finite etale degree $n$ coverings of
$\mathbf{A}^1_{\mathbf{C}} \setminus \{0\}$ with a given free
$\mathbf{Z}/n\mathbf{Z}$-action correspond one-to-one with $n$th
roots of $1$ which are not primitive.
In other words, this computation shows that
$$
H_{et}^1 (\mathbf{A}^1_\mathbf{C} \setminus \{0\}, \mathbf{Z}/n\mathbf{Z})
= \mu_n(\mathbf{C}) \cong \mathbf{Z}/n\mathbf{Z}.
$$
The first identification is canonical, the second isn't. We remark that
since the proof of Riemann-Hurwitz does not use this fact, the above actually
constitutes a proof (provided we fill in the details on vanishing, etc).




\section{Nontorsion coefficients}
\label{section-nontorsion}

\noindent
To study nontorsion coefficients, one makes the following definition:
$$
H_{et}^i (X, \mathbf{Q}_\ell) :=
\left( \lim\nolimits_n H_{et}^i(X, \mathbf{Z}/\ell^n\mathbf{Z}) \right)
\otimes_{\mathbf{Z}_\ell} \mathbf{Q}_\ell.
$$
The symbol $\lim_n$ denote the {\it limit} of the system of
cohomology groups $H_{et}^i(X, \mathbf{Z}/\ell^n\mathbf{Z})$ indexed
by $n$, see
Categories, Section \ref{categories-section-posets-limits}.
Thus we will need to study systems of sheaves satisfying some compatibility
conditions. 

\section{Other chapters}

\begin{multicols}{2}
\begin{enumerate}
\item \hyperref[introduction-section-phantom]{Introduction}
\item \hyperref[conventions-section-phantom]{Conventions}
\item \hyperref[sets-section-phantom]{Set Theory}
\item \hyperref[categories-section-phantom]{Categories}
\item \hyperref[topology-section-phantom]{Topology}
\item \hyperref[sheaves-section-phantom]{Sheaves on Spaces}
\item \hyperref[algebra-section-phantom]{Commutative Algebra}
\item \hyperref[sites-section-phantom]{Sites and Sheaves}
\item \hyperref[homology-section-phantom]{Homological Algebra}
\item \hyperref[derived-section-phantom]{Derived Categories}
\item \hyperref[more-algebra-section-phantom]{More Algebra}
\item \hyperref[simplicial-section-phantom]{Simplicial Methods}
\item \hyperref[modules-section-phantom]{Sheaves of Modules}
\item \hyperref[sites-modules-section-phantom]{Modules on Sites}
\item \hyperref[injectives-section-phantom]{Injectives}
\item \hyperref[cohomology-section-phantom]{Cohomology of Sheaves}
\item \hyperref[sites-cohomology-section-phantom]{Cohomology on Sites}
\item \hyperref[hypercovering-section-phantom]{Hypercoverings}
\item \hyperref[schemes-section-phantom]{Schemes}
\item \hyperref[constructions-section-phantom]{Constructions of Schemes}
\item \hyperref[properties-section-phantom]{Properties of Schemes}
\item \hyperref[morphisms-section-phantom]{Morphisms of Schemes}
\item \hyperref[coherent-section-phantom]{Coherent Cohomology}
\item \hyperref[divisors-section-phantom]{Divisors}
\item \hyperref[limits-section-phantom]{Limits of Schemes}
\item \hyperref[varieties-section-phantom]{Varieties}
\item \hyperref[chow-section-phantom]{Chow Homology}
\item \hyperref[topologies-section-phantom]{Topologies on Schemes}
\item \hyperref[descent-section-phantom]{Descent}
\item \hyperref[more-morphisms-section-phantom]{More on Morphisms}
\item \hyperref[flat-section-phantom]{More on Flatness}
\item \hyperref[groupoids-section-phantom]{Groupoid Schemes}
\item \hyperref[more-groupoids-section-phantom]{More on Groupoid Schemes}
\item \hyperref[etale-section-phantom]{\'Etale Morphisms of Schemes}
\item \hyperref[etale-cohomology-section-phantom]{\'Etale Cohomology}
\item \hyperref[spaces-section-phantom]{Algebraic Spaces}
\item \hyperref[spaces-properties-section-phantom]{Properties of Algebraic Spaces}
\item \hyperref[spaces-morphisms-section-phantom]{Morphisms of Algebraic Spaces}
\item \hyperref[spaces-topologies-section-phantom]{Topologies on Algebraic Spaces}
\item \hyperref[spaces-descent-section-phantom]{Descent and Algebraic Spaces}
\item \hyperref[spaces-more-morphisms-section-phantom]{More on Morphisms of Spaces}
\item \hyperref[quot-section-phantom]{Quot and Hilbert Spaces}
\item \hyperref[stacks-section-phantom]{Stacks}
\item \hyperref[spaces-groupoids-section-phantom]{Groupoids in Algebraic Spaces}
\item \hyperref[spaces-more-groupoids-section-phantom]{More on Groupoids in Spaces}
\item \hyperref[bootstrap-section-phantom]{Bootstrap}
\item \hyperref[examples-stacks-section-phantom]{Examples of Stacks}
\item \hyperref[groupoids-quotients-section-phantom]{Quotients of Groupoids}
\item \hyperref[algebraic-section-phantom]{Algebraic Stacks}
\item \hyperref[criteria-section-phantom]{Criteria for Representability}
\item \hyperref[stacks-properties-section-phantom]{Properties of Algebraic Stacks}
\item \hyperref[stacks-morphisms-section-phantom]{Morphisms of Algebraic Stacks}
\item \hyperref[examples-section-phantom]{Examples}
\item \hyperref[exercises-section-phantom]{Exercises}
\item \hyperref[guide-section-phantom]{Guide to Literature}
\item \hyperref[desirables-section-phantom]{Desirables}
\item \hyperref[coding-section-phantom]{Coding Style}
\item \hyperref[fdl-section-phantom]{GNU Free Documentation License}
\item \hyperref[index-section-phantom]{Auto Generated Index}
\end{enumerate}
\end{multicols}


\bibliography{my}
\bibliographystyle{amsalpha}

\end{document}
