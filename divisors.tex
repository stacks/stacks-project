\IfFileExists{stacks-project.cls}{%
\documentclass{stacks-project}
}{%
\documentclass{amsart}
}

% The following AMS packages are automatically loaded with
% the amsart documentclass:
%\usepackage{amsmath}
%\usepackage{amssymb}
%\usepackage{amsthm}

% For dealing with references we use the comment environment
\usepackage{verbatim}
\newenvironment{reference}{\comment}{\endcomment}
%\newenvironment{reference}{}{}
\newenvironment{slogan}{\comment}{\endcomment}
\newenvironment{history}{\comment}{\endcomment}

% For commutative diagrams you can use
% \usepackage{amscd}
\usepackage[all]{xy}

% We use 2cell for 2-commutative diagrams.
\xyoption{2cell}
\UseAllTwocells

% To put source file link in headers.
% Change "template.tex" to "this_filename.tex"
% \usepackage{fancyhdr}
% \pagestyle{fancy}
% \lhead{}
% \chead{}
% \rhead{Source file: \url{template.tex}}
% \lfoot{}
% \cfoot{\thepage}
% \rfoot{}
% \renewcommand{\headrulewidth}{0pt}
% \renewcommand{\footrulewidth}{0pt}
% \renewcommand{\headheight}{12pt}

\usepackage{multicol}

% For cross-file-references
\usepackage{xr-hyper}

% Package for hypertext links:
\usepackage{hyperref}

% For any local file, say "hello.tex" you want to link to please
% use \externaldocument[hello-]{hello}
\externaldocument[introduction-]{introduction}
\externaldocument[conventions-]{conventions}
\externaldocument[sets-]{sets}
\externaldocument[categories-]{categories}
\externaldocument[topology-]{topology}
\externaldocument[sheaves-]{sheaves}
\externaldocument[sites-]{sites}
\externaldocument[stacks-]{stacks}
\externaldocument[fields-]{fields}
\externaldocument[algebra-]{algebra}
\externaldocument[brauer-]{brauer}
\externaldocument[homology-]{homology}
\externaldocument[derived-]{derived}
\externaldocument[simplicial-]{simplicial}
\externaldocument[more-algebra-]{more-algebra}
\externaldocument[smoothing-]{smoothing}
\externaldocument[modules-]{modules}
\externaldocument[sites-modules-]{sites-modules}
\externaldocument[injectives-]{injectives}
\externaldocument[cohomology-]{cohomology}
\externaldocument[sites-cohomology-]{sites-cohomology}
\externaldocument[dga-]{dga}
\externaldocument[dpa-]{dpa}
\externaldocument[hypercovering-]{hypercovering}
\externaldocument[schemes-]{schemes}
\externaldocument[constructions-]{constructions}
\externaldocument[properties-]{properties}
\externaldocument[morphisms-]{morphisms}
\externaldocument[coherent-]{coherent}
\externaldocument[divisors-]{divisors}
\externaldocument[limits-]{limits}
\externaldocument[varieties-]{varieties}
\externaldocument[topologies-]{topologies}
\externaldocument[descent-]{descent}
\externaldocument[perfect-]{perfect}
\externaldocument[more-morphisms-]{more-morphisms}
\externaldocument[flat-]{flat}
\externaldocument[groupoids-]{groupoids}
\externaldocument[more-groupoids-]{more-groupoids}
\externaldocument[etale-]{etale}
\externaldocument[chow-]{chow}
\externaldocument[intersection-]{intersection}
\externaldocument[pic-]{pic}
\externaldocument[adequate-]{adequate}
\externaldocument[dualizing-]{dualizing}
\externaldocument[duality-]{duality}
\externaldocument[discriminant-]{discriminant}
\externaldocument[local-cohomology-]{local-cohomology}
\externaldocument[curves-]{curves}
\externaldocument[resolve-]{resolve}
\externaldocument[models-]{models}
\externaldocument[pione-]{pione}
\externaldocument[etale-cohomology-]{etale-cohomology}
\externaldocument[proetale-]{proetale}
\externaldocument[crystalline-]{crystalline}
\externaldocument[spaces-]{spaces}
\externaldocument[spaces-properties-]{spaces-properties}
\externaldocument[spaces-morphisms-]{spaces-morphisms}
\externaldocument[decent-spaces-]{decent-spaces}
\externaldocument[spaces-cohomology-]{spaces-cohomology}
\externaldocument[spaces-limits-]{spaces-limits}
\externaldocument[spaces-divisors-]{spaces-divisors}
\externaldocument[spaces-over-fields-]{spaces-over-fields}
\externaldocument[spaces-topologies-]{spaces-topologies}
\externaldocument[spaces-descent-]{spaces-descent}
\externaldocument[spaces-perfect-]{spaces-perfect}
\externaldocument[spaces-more-morphisms-]{spaces-more-morphisms}
\externaldocument[spaces-flat-]{spaces-flat}
\externaldocument[spaces-groupoids-]{spaces-groupoids}
\externaldocument[spaces-more-groupoids-]{spaces-more-groupoids}
\externaldocument[bootstrap-]{bootstrap}
\externaldocument[spaces-pushouts-]{spaces-pushouts}
\externaldocument[groupoids-quotients-]{groupoids-quotients}
\externaldocument[spaces-more-cohomology-]{spaces-more-cohomology}
\externaldocument[spaces-simplicial-]{spaces-simplicial}
\externaldocument[formal-spaces-]{formal-spaces}
\externaldocument[restricted-]{restricted}
\externaldocument[spaces-resolve-]{spaces-resolve}
\externaldocument[formal-defos-]{formal-defos}
\externaldocument[defos-]{defos}
\externaldocument[cotangent-]{cotangent}
\externaldocument[examples-defos-]{examples-defos}
\externaldocument[algebraic-]{algebraic}
\externaldocument[examples-stacks-]{examples-stacks}
\externaldocument[stacks-sheaves-]{stacks-sheaves}
\externaldocument[criteria-]{criteria}
\externaldocument[artin-]{artin}
\externaldocument[quot-]{quot}
\externaldocument[stacks-properties-]{stacks-properties}
\externaldocument[stacks-morphisms-]{stacks-morphisms}
\externaldocument[stacks-limits-]{stacks-limits}
\externaldocument[stacks-cohomology-]{stacks-cohomology}
\externaldocument[stacks-perfect-]{stacks-perfect}
\externaldocument[stacks-introduction-]{stacks-introduction}
\externaldocument[stacks-more-morphisms-]{stacks-more-morphisms}
\externaldocument[stacks-geometry-]{stacks-geometry}
\externaldocument[moduli-]{moduli}
\externaldocument[moduli-curves-]{moduli-curves}
\externaldocument[examples-]{examples}
\externaldocument[exercises-]{exercises}
\externaldocument[guide-]{guide}
\externaldocument[desirables-]{desirables}
\externaldocument[coding-]{coding}
\externaldocument[obsolete-]{obsolete}
\externaldocument[fdl-]{fdl}
\externaldocument[index-]{index}

% Theorem environments.
%
\theoremstyle{plain}
\newtheorem{theorem}[subsection]{Theorem}
\newtheorem{proposition}[subsection]{Proposition}
\newtheorem{lemma}[subsection]{Lemma}

\theoremstyle{definition}
\newtheorem{definition}[subsection]{Definition}
\newtheorem{example}[subsection]{Example}
\newtheorem{exercise}[subsection]{Exercise}
\newtheorem{situation}[subsection]{Situation}

\theoremstyle{remark}
\newtheorem{remark}[subsection]{Remark}
\newtheorem{remarks}[subsection]{Remarks}

\numberwithin{equation}{subsection}

% Macros
%
\def\lim{\mathop{\rm lim}\nolimits}
\def\colim{\mathop{\rm colim}\nolimits}
\def\Spec{\mathop{\rm Spec}}
\def\Hom{\mathop{\rm Hom}\nolimits}
\def\Ext{\mathop{\rm Ext}\nolimits}
\def\SheafHom{\mathop{\mathcal{H}\!{\it om}}\nolimits}
\def\SheafExt{\mathop{\mathcal{E}\!{\it xt}}\nolimits}
\def\Sch{\textit{Sch}}
\def\Mor{\mathop{\rm Mor}\nolimits}
\def\Ob{\mathop{\rm Ob}\nolimits}
\def\Sh{\mathop{\textit{Sh}}\nolimits}
\def\NL{\mathop{N\!L}\nolimits}
\def\proetale{{pro\text{-}\acute{e}tale}}
\def\etale{{\acute{e}tale}}
\def\QCoh{\textit{QCoh}}
\def\Ker{\mathop{\rm Ker}}
\def\Im{\mathop{\rm Im}}
\def\Coker{\mathop{\rm Coker}}
\def\Coim{\mathop{\rm Coim}}

%
% Macros for moduli stacks/spaces
%
\def\QCohstack{\mathcal{QC}\!{\it oh}}
\def\Cohstack{\mathcal{C}\!{\it oh}}
\def\Spacesstack{\mathcal{S}\!{\it paces}}
\def\Quotfunctor{{\rm Quot}}
\def\Hilbfunctor{{\rm Hilb}}
\def\Curvesstack{\mathcal{C}\!{\it urves}}
\def\Polarizedstack{\mathcal{P}\!{\it olarized}}
\def\Complexesstack{\mathcal{C}\!{\it omplexes}}
% \Pic is the operator that assigns to X its picard group, usage \Pic(X)
% \Picardstack_{X/B} denotes the Picard stack of X over B
% \Picardfunctor_{X/B} denotes the Picard functor of X over B
\def\Pic{\mathop{\rm Pic}\nolimits}
\def\Picardstack{\mathcal{P}\!{\it ic}}
\def\Picardfunctor{{\rm Pic}}
\def\Deformationcategory{\mathcal{D}\!{\it ef}}


% OK, start here.
%
\begin{document}

\title{Divisors}


\maketitle

\phantomsection
\label{section-phantom}

\tableofcontents

\section{Introduction}
\label{section-introduction}

\noindent
In this chapter we study some very basic questions related
to defining divisors, etc. A basic reference is \cite{EGA}.








\section{Effective Cartier divisors}
\label{section-effective-Cartier-divisors}

\noindent
For some reason it seem convenient to define the notion of an effective
Cartier divisor before anything else.

\begin{definition}
\label{definition-effective-Cartier-divisor}
Let $S$ be a scheme.
An {\it effective Cartier divisor} on $S$ is a closed subscheme
$D \subset S$ such that the ideal sheaf $\mathcal{I}_D \subset \mathcal{O}_X$
is an invertible $\mathcal{O}_X$-module.
\end{definition}

\noindent
This notion defines closed subschemes which are of pure codimension $1$
in the strongest possible sense. Namely they are locally cut out by
a single element which is not a zero divisor. In particular they
are nowhere dense.

\begin{lemma}
\label{lemma-characterize-effective-Cartier-divisor}
Let $S$ be a scheme.
Let $D \subset S$ be a closed subscheme.
The following are equivalent:
\begin{enumerate}
\item The subscheme $D$ is an effective Cartier divisor on $S$.
\item For every $x \in D$ there exists an affine open neighbourhood
$\text{Spec}(A) = U \subset X$ of $x$ such that
$U \cap D = \text{Spec}(A/(f))$ with $f \in A$ not a zero divisor.
\end{enumerate}
\end{lemma}

\begin{proof}
Assume (1).  For every $x \in D$ there exists an affine open neighbourhood
$\text{Spec}(A) = U \subset X$ of $x$ such that
$\mathcal{I}_D|_U \cong \mathcal{O}_U$. In other words, there exists
a section $f \in \Gamma(U, \mathcal{I}_D)$ which freely generates the
restriction $\mathcal{I}_D|_U$. Hence $f \in A$, and the multiplication
map $f : A \to A$ is injective. Also, since $\mathcal{I}_D$ is
quasi-coherent we see that $D \cap U = \text{Spec}(A/(f))$.

\medskip\noindent
Assume (2). Let $x \in D$. By assumption there exists an affine open
neighbourhood $\text{Spec}(A) = U \subset X$ of $x$ such that
$U \cap D = \text{Spec}(A/(f))$ with $f \in A$ not a zero divisor.
Then $\mathcal{I}_D|_U \cong \mathcal{O}_U$ since it is equal to
$\widetilde{(f)} \cong \widetilde{A} \cong \mathcal{O}_U$.
Of course $\mathcal{I}_D$ restricted to the open subscheme
$S \setminus D$ is isomorphic to $\mathcal{O}_{X \setminus D}$.
Hence $\mathcal{I}_D$ is an invertible $\mathcal{O}_S$-module.
\end{proof}

\begin{definition}
\label{definition-sum-effective-Cartier-divisors}
Let $S$ be a scheme. Given effective Cartier divisors
$D_1$, $D_2$ on $S$ we set $D = D_1 + D_2$ equal to the
closed subscheme of $S$ corresponding to the quasi-coherent
sheaf of ideals
$\mathcal{I}_{D_1}\mathcal{I}_{D_2} \subset \mathcal{O}_S$.
We call this the {\it sum of the effective Cartier divisors
$D_1$ and $D_2$}.
\end{definition}

\noindent
It is clear that we may define the sum $\sum n_iD_i$ given
finitely many effective Cartier divisors $D_i$ on $X$
and nonnegative integers $n_i$.

\begin{lemma}
\label{lemma-sum-effective-Cartier-divisors}
The sum of two effective Cartier divisors is an effective
Cartier divisor.
\end{lemma}

\begin{proof}
Omitted. Locally $f_1, f_2 \in A$ are nonzero divisors, then also
$f_1f_2 \in A$ is a nonzero divisor.
\end{proof}

\begin{lemma}
\label{lemma-difference-effective-Cartier-divisors}
Let $X$ be a scheme.
Let $D, D'$ be two effective Cartier divisors on $X$.
If $D \subset D'$ (as closed subschemes of $X$), then
there exists an effective Cartier divisor $D''$ such
that $D' = D + D''$.
\end{lemma}

\begin{proof}
Omitted.
\end{proof}

\noindent
Recall that we have defined the inverse image of a closed subscheme
under any morphism of schemes in
Schemes, Definition \ref{schemes-definition-inverse-image-closed-subscheme}.

\begin{definition}
\label{definition-pullback-effective-Cartier-divisor}
Let $f : S' \to S$ be a morphism of schemes. Let $D \subset S$
be an effective Cartier divisor. We say the {\it pullback of
$D$ by $f$ is defined} if the closed subscheme $f^{-1}(D) \subset S'$
is an effective Cartier divisor. In this case we denote it either
$f^*D$ or $f^{-1}(D)$ and we call it the
{\it pullback of the effective Cartier divisor}.
\end{definition}

\noindent
The condition that $f^{-1}(D)$ is an effective Cartier divisor
is often satisfied in practice. Here is an example lemma.

\begin{lemma}
\label{lemma-pullback-effective-Cartier-defined}
Let $f : X \to Y$ be a morphism of schemes.
Let $D \subset Y$ be an effective Cartier divisor.
The pullback of $D$ by $f$ is defined in each of the following cases:
\begin{enumerate}
\item $X$, $Y$ integral and $f$ dominant,
\item $X$ reduced, and for any generic point $\xi$ of any
irreducible component of $X$ we have $f(\xi) \not \in D$,
\item $X$ is locally Noetherian and for any associated point
$x$ of $X$ we have $f(x) \not \in D$,
\item $X$ is locally Noetherian, has no embedded points, and
for any generic point $\xi$ of any irreducible component of
$X$ we have $f(\xi) \not \in D$,
\item $f$ is flat, and
\item add more here as needed.
\end{enumerate}
\end{lemma}

\begin{proof}
The question is local on $X$, and hence we reduce to the case
where $X = \text{Spec}(A)$, $Y = \text{Spec}(R)$, $f$ is
given by $\varphi : R \to A$ and
$D = \text{Spec}(R/(t))$ where $t \in R$ is not a zero divisor.
The goal in each case is to show that $\varphi(t) \in A$
is not a zero divisor.

\medskip\noindent
In case (2) this follows as the intersection of all minimal
primes of a ring is the nilradical of the ring, see
Algebra, Lemma \ref{algebra-lemma-Zariski-topology}.

\medskip\noindent
Let us prove (3). By
Coherent, Lemma \ref{coherent-lemma-associated-affine-open}
the associated points of $X$ correspond to the primes
$\mathfrak p \in \text{Ass}(A)$.
By Algebra, Lemma \ref{algebra-lemma-ass-zero-divisors} we have
$\bigcup_{\mathfrak p \in \text{Ass}(A)} \mathfrak p$ is
the set of zero divisors of $A$. The hypothesis of
(3) is that $\varphi(t) \not \in \mathfrak p$ for
all $\mathfrak p \in \text{Ass}(A)$. Hence $\varphi(t)$
is a nonzero divisor as desired.

\medskip\noindent
Part (4) follows from (3) and the definitions.
\end{proof}

\begin{lemma}
\label{lemma-pullback-effective-Cartier-divisors-additive}
Let $f : S' \to S$ be a morphism of schemes.
Let $D_1$, $D_2$ be effective Cartier divisors on $S$.
If the pullbacks of $D_1$ and $D_2$ are defined then the
pullback of $D = D_1 + D_2$ is defined and
$f^*D = f^*D_1 + f^*D_2$.
\end{lemma}

\begin{proof}
Omitted.
\end{proof}

\begin{definition}
\label{definition-invertible-sheaf-effective-Cartier-divisor}
Let $S$ be a scheme and let $D$ be an effective Cartier divisor.
The {\it invertible sheaf $\mathcal{O}_S(D)$ associated to $D$}
is given by
$$
\mathcal{O}_S(D) :=
\textit{Hom}_{\mathcal{O}_S}(\mathcal{I}_D, \mathcal{O}_S) =
\mathcal{I}_D^{\otimes -1}.
$$
The canonical section, usually denoted $1$ or $1_D$, is the
global section of $\mathcal{O}_S(D)$ corresponding to
the inclusion mapping $\mathcal{I}_D \to \mathcal{O}_S$.
\end{definition}

\begin{lemma}
\label{lemma-invertible-sheaf-sum-effective-Cartier-divisors}
Let $S$ be a scheme.
Let $D_1$, $D_2$ be effective Cartier divisors on $S$.
Let $D = D_1 + d_2$.
Then there is a unique isomorphism
$$
\mathcal{O}_S(D_1) \otimes_{\mathcal{O}_S} \mathcal{O}_S(D_2)
\longrightarrow
\mathcal{O}_S(D)
$$
which maps $1_{D_1} \otimes 1_{D_2}$ to $1_D$.
\end{lemma}

\begin{proof}
Omitted.
\end{proof}

\begin{definition}
\label{definition-regular-section}
Let $(X, \mathcal{O}_X)$ be a locally ringed space.
Let $\mathcal{L}$ be an invertible sheaf on $X$.
A global section $s \in \Gamma(X, \mathcal{L})$ is called a
{\it regular section} if the map $\mathcal{O}_X \to \mathcal{L}$,
$f \mapsto fs$ is injective.
\end{definition}

\begin{lemma}
\label{lemma-regular-section-structure-sheaf}
Let $X$ be a locally ringed space. Let $f \in \Gamma(X, \mathcal{O}_X)$.
The following are equivalent:
\begin{enumerate}
\item $f$ is a regular section, and
\item for any $x \in X$ the image $f \in \mathcal{O}_{X, x}$
is not a zero divisor.
\end{enumerate}
If $X$ is a scheme these are also equivalent to
\begin{enumerate}
\item[(3)] for any affine open $\text{Spec}(A) = U \subset X$
the image $f \in A$ is not a zero divisor, and
\item[(4)] there exists an affine open covering
$X = \bigcup \text{Spec}(A_i)$ such that
the image of $f$ in $A_i$ is not a zero divisor for all $i$.
\end{enumerate}
\end{lemma}

\begin{proof}
Omitted.
\end{proof}

\noindent
Note that a global section $s$ of an invertible $\mathcal{O}_X$-module
$\mathcal{L}$ may be seen as an $\mathcal{O}_X$-module map
$s : \mathcal{O}_X \to \mathcal{L}$. Its dual is therefore a
map $s : \mathcal{L}^{\otimes -1} \to \mathcal{O}_X$.
(See Modules, Definition \ref{modules-definition-powers}
for the definition of the dual invertible sheaf.)

\begin{definition}
\label{definition-zero-scheme-s}
Let $X$ be a scheme.
Let $\mathcal{L}$ be an invertible sheaf.
Let $s \in \Gamma(X, \mathcal{L})$.
The {\it zero scheme} of $s$ is the closed subscheme $Z(s) \subset X$
defined by the quasi-coherent sheaf of ideals
$\mathcal{I} \subset \mathcal{O}_X$ which is the image of the
map $s : \mathcal{L}^{\otimes -1} \to \mathcal{O}_X$.
\end{definition}

\begin{lemma}
\label{lemma-zero-scheme}
Let $X$ be a scheme.
Let $\mathcal{L}$ be an invertible sheaf.
Let $s \in \Gamma(X, \mathcal{L})$.
\begin{enumerate}
\item Consider closed immersions $i : Z \to X$ such that
$i^*s \in \Gamma(Z, i^*\mathcal{L}))$ is zero
ordered by inclusion. The zero scheme $Z(s)$ is the
minimal element of this set.
\item For any morphism of schemes $f : Y \to X$ we have
$f^*s = 0$ in $\Gamma(Y, f^*\mathcal{L})$ if and only if
$f$ factors through $Z(s)$.
\item The zero scheme $Z(s)$ is locally cut out by one equation.
\item The zero scheme $Z(s)$ is an effective Cartier divisor
if and only if $s$ is a regular section of $\mathcal{L}$.
\end{enumerate}
\end{lemma}

\begin{proof}
Omitted.
\end{proof}

\begin{lemma}
\label{lemma-characterize-OD}
Let $S$ be a scheme.
\begin{enumerate}
\item If $D \subset S$ is an effective Cartier divisor, then
the canonical section $1_D$ of $\mathcal{O}_S(D)$ is regular.
\item Conversely, if $s$ is a regular section of the invertible
sheaf $\mathcal{L}$, then there exists a unique effective
Cartier divisor $D = Z(s) \subset S$ and a unique isomorphism
$\mathcal{O}_S(D) \to \mathcal{L}$ which maps $1_D$ to $s$.
\end{enumerate}
The constructions
$D \mapsto (\mathcal{O}_X(D), 1_D)$ and $(\mathcal{L}, s) \mapsto Z(s)$
give mutually inverse maps
$$
\left\{
\begin{matrix}
\text{effective Cartier divisors on }X
\end{matrix}
\right\}
\leftrightarrow
\left\{
\begin{matrix}
\text{pairs }(\mathcal{L}, s)\text{ consisting of an invertible}\\
\mathcal{O}_X\text{-module and a regular global section}
\end{matrix}
\right\}
$$
\end{lemma}

\begin{proof}
Omitted.
\end{proof}

\noindent
Here is a way to produce effective Cartier divisors.

\begin{lemma}
\label{lemma-blowing-up-gives-effective-Cartier-divisor}
Let $X$ be a scheme.
Let $Z \subset X$ be a closed subscheme.
The blow up $b : X' \to X$ of $Z$
has the following properties:
\begin{enumerate}
\item $b|_{b^{-1}(X \setminus Z)} : b^{-1}(X \setminus Z) \to X \setminus Z$
is an isomorphism, and
\item $E = b^{-1}(Z)$ is an effective Cartier divisor on $X'$.
\end{enumerate}
\end{lemma}

\begin{proof}
Proof omitted, but here are some hints:
If $X = \text{Spec}(A)$ and $Z = \text{Spec}(A/I)$,
then $X' = \text{Proj}(\bigoplus_{n \geq 0} I^n)$.
Write $S = \bigoplus_{n \geq 0} I^n$ as a graded ring.
Pick an element $f \in I$ and denote $F \in S_1$ the
corresponding element in degree one of $S$. It is clear that
the standard opens $D_{+}(F)$ cover $X'$ in this case.
Each $D_{+}(F)$ is the spectrum of the ring $S_{(F)}$.
Note that $f$ is a nonzero divisor on $S_{(F)}$ since
$f a/F^d = 0$ (some $a \in S_d$)
implies also that $F a/F^{d + 1}$ is zero. Moreover,
$IS_{(F)}$ is generated by the elements
$g = fG/F$ where $G \in S_1$ is the degree $1$ element
of $S$ corresponding to $g$. Hence it is indeed the
case that $IS_{(F)}$ is generated by a single nonzero divisor as desired.
\end{proof}








\section{Meromorphic functions and sections}
\label{section-meromorphic-functions}

\noindent
See \cite{misconceptions} for some possible
pitfalls\footnote{Danger, Will Robinson!}.

\medskip\noindent
Let $(X, \mathcal{O}_X)$ be a locally ringed space.
For any open $U \subset X$ we have defined the set
$\mathcal{S}(U) \subset \mathcal{O}_X(U)$
of regular sections of $\mathcal{O}_X$ over $U$, see
Definition \ref{definition-regular-section}. The restriction
of a regular section to a smaller open is regular. Hence
$\mathcal{S} : U \mapsto \mathcal{S}(U)$ is a subsheaf (of sets)
of $\mathcal{O}_X$. We sometimes denote $\mathcal{S} = \mathcal{S}_X$
if we want to indicate the dependence on $X$.
Moreover, $\mathcal{S}(U)$
is a multiplicative subset of the ring $\mathcal{O}_X(U)$ for
each $U$. Hence we may consider
the presheaf of rings
$$
U \longmapsto \mathcal{S}(U)^{-1} \mathcal{O}_X(U),
$$
see Modules, Lemma \ref{modules-lemma-simple-invert}.

\begin{definition}
\label{definition-sheaf-meromorphic-functions}
Let $(X, \mathcal{O}_X)$ be a locally ringed space.
The {\it sheaf of meromorphic functions on $X$} is
the sheaf {\it $\mathcal{K}_X$} associated to the presheaf
displayed above. A {\it meromorphic function} on $X$
is a global section of $\mathcal{K}_X$.
\end{definition}

\noindent
Since each element of each $\mathcal{S}(U)$ is a nonzero divisor on
$\mathcal{O}_X(U)$ we see that the natural map of sheaves
of rings $\mathcal{O}_X \to \mathcal{K}_X$ is injective.

\begin{example}
\label{example-no-change}
Let $A = \mathbf{C}[x, \{y_\alpha\}_{\alpha \in \mathbf{C}}]/
((x - \alpha)y_\alpha, y_\alpha y_\beta)$. Any element of $A$
can be written uniquely as
$f(x) + \sum \lambda_\alpha y_\alpha$ with $f(x) \in \mathbf{C}[x]$
and $\lambda_\alpha \in \mathbf{C}$.
Let $X = \text{Spec}(A)$.
In this case $\mathcal{O}_X = \mathcal{K}_X$, since on
any affine open $D(f)$ the ring $A_f$ any nonzero divisor is
a unit (proof omitted).
\end{example}

\begin{definition}
\label{definition-pullback-meromorphic-sections}
Let $f : (X, \mathcal{O}_X) \to (Y, \mathcal{O}_Y)$ be a morphism
of locally ringed spaces. We say that {\it pulbacks of meromorphic
functions are defined for $f$} if for every pair of open
$U \subset X$, $V \subset Y$ such that $f(U) \subset V$, and any
section $s \in \Gamma(V, \mathcal{S}_Y)$ the pullback
$f^\sharp(s) \in \Gamma(U, \mathcal{O}_X)$ is an element
of $\Gamma(U, \mathcal{S}_X)$.
\end{definition}

\noindent
In this case there is an induced map
$f^\sharp : f^{-1}\mathcal{K}_Y \to \mathcal{K}_X$,
in other words we obtain a commutative diagram of morphisms
of ringed spaces
$$
\xymatrix{
(X, \mathcal{K}_X) \ar[r] \ar[d]^f &
(X, \mathcal{O}_X) \ar[d]^f \\
(Y, \mathcal{K}_Y) \ar[r] &
(Y, \mathcal{O}_X)
}
$$
We sometimes denote $f^*(s) = f^\sharp(s)$ for a
section $s \in \Gamma(Y, \mathcal{K}_Y)$.

\begin{lemma}
\label{lemma-pullback-meromorphic-sections-defined}
Let $f : X \to Y$ be a morphism of schemes.
In each of the following cases pullbacks of meromorphic
sections are defined.
\begin{enumerate}
\item $X$, $Y$ are integral and $f$ is dominant,
\item $X$ is integral and the generic point of $X$ maps
to a generic point of an irreducible component of $Y$,
\item $X$ is reduced and every generic point of every irreducible
component of $X$ maps to the generic point of an irreducible component
of $Y$,
\item $X$ is locally Noetherian, and any associated point of
$X$ maps to a generic point of an irreducible component of $Y$, and
\item $X$ is locally Noetherian, has no embedded points and
any generic point of an irreducible component of
$X$ maps to the generic point of an irreducible component of $Y$.
\end{enumerate}
\end{lemma}

\begin{proof}
Omitted. Hint: Similar to the proof of
Lemma \ref{lemma-pullback-effective-Cartier-defined}, using
the following fact (on $Y$): if an element $x \in R$ maps to
a nonzero divisor in $R_{\mathfrak p}$ for a minimal prime
$\mathfrak p$ of $R$, then $x \not \in \mathfrak p$.
See Algebra, Lemma \ref{algebra-lemma-minimal-prime-reduced-ring}.
\end{proof}

\noindent
Let $(X, \mathcal{O}_X)$ be a locally ringed space.
Let $\mathcal{F}$ be a sheaf of $\mathcal{O}_X$-modules.
Consider the presheaf $U \mapsto \mathcal{S}(U)^{-1}\mathcal{F}(U)$.
Its sheafification is the sheaf
$\mathcal{F} \otimes_{\mathcal{O}_X} \mathcal{K}_X$, see
Modules, Lemma \ref{modules-lemma-simple-invert-module}.

\begin{definition}
\label{definition-meromorphic-section}
Let $X$ be a locally ringed space.
Let $\mathcal{F}$ be a sheaf of $\mathcal{O}_X$-modules.
\begin{enumerate}
\item We denote
$\mathcal{K}_X(\mathcal{F})$
the sheaf of $\mathcal{K}_X$-modules which is
the sheafification of the presheaf
$U \mapsto \mathcal{S}(U)^{-1}\mathcal{F}(U)$. Equivalently
$\mathcal{K}_X(\mathcal{F}) =
\mathcal{F} \otimes_{\mathcal{O}_X} \mathcal{K}_X$ (see above).
\item A {\it meromorphic section of $\mathcal{F}$}
is a global section of $\mathcal{K}_X(\mathcal{F})$.
\end{enumerate}
\end{definition}

\noindent
In particular we have
$$
\mathcal{K}_X(\mathcal{F})_x
=
\mathcal{F}_x \otimes_{\mathcal{O}_{X, x}} \mathcal{K}_{X, x}
=
\mathcal{S}_x^{-1}\mathcal{F}_x
$$
for any point $x \in X$. However, one has to be careful since it may
not be the case that $\mathcal{S}_x$ is the set of nonzero divisors
in the local ring $\mathcal{O}_{X, x}$.

\begin{lemma}
\label{lemma-locally-Noetherian-K}
Let $X$ be a locally Noetherian scheme.
\begin{enumerate}
\item For any $x \in X$ we have $\mathcal{S}_x \subset \mathcal{O}_{X, x}$
is the set of nonzero divisors. In particular $\mathcal{K}_{X, x}$
is the total quotient ring of $\mathcal{O}_{X, x}$.
\item For any affine open $\text{Spec}(A) = U \subset X$ we have
that $\mathcal{K}_X(U)$ equals the total quotient ring of $A$.
\end{enumerate}
\end{lemma}

\begin{proof}
Omitted.

\medskip\noindent
Hint for (1): Let $A$ be a Noetherian ring.
Let $\mathfrak p \subset A$ be a prime ideal.
Let $f \in A$. Let $I = \{x \in A \mid fx = 0\}$.
Suppose $f$ is a nonzero divisor in $A_{\mathfrak p}$.
Then we see that $I_{\mathfrak p} = 0$ by exactness of
localization. Since $A$ is Noetherian we see that $I$
is finitely generated and hence that $gI = 0$ for some $g \in A$,
$g \not \in \mathfrak p$. Hence $f$ is a nonzero divisor
in $A_g$ (i.e., in a Zariski open neighbourhood of $\mathfrak p$).

\medskip\noindent
Hint for (2): By part (1) the stalks agree.
\end{proof}

\begin{lemma}
\label{lemma-reduced-finite-irreducible}
Let $X$ be a scheme.
Assume $X$ is reduced and any quasi-compact open $U \subset X$
has a finite number of irreducible components.
\begin{enumerate}
\item For any $x \in X$ we have $\mathcal{S}_x \subset \mathcal{O}_{X, x}$
is the set of nonzero divisors. In particular $\mathcal{K}_{X, x}$
is the total quotient ring of $\mathcal{O}_{X, x}$.
\item For any affine open $\text{Spec}(A) = U \subset X$ we have
that $\mathcal{K}_X(U)$ equals the total quotient ring of $A$.
\end{enumerate}
\end{lemma}

\begin{proof}
Omitted.

\medskip\noindent
Hint for (1): Let $A$ be a reduced ring with finitely many minimal
primes $\mathfrak q_1, \ldots, \mathfrak q_t$.
Let $\mathfrak p \subset A$ be a prime ideal.
Let $f \in A$. Let $I = \{x \in A \mid fx = 0\}$.
Then $I = \bigcap_{f \not \in \mathfrak q_i} \mathfrak q_i$.
Suppose $f$ is a nonzero divisor in $A_{\mathfrak p}$.
This means that $\mathfrak p \not \supset \mathfrak q_i$ for
those $i$ such that $f \in \mathfrak q_i$. Then
$\text{Spec}(A) \setminus \bigcup_{f \in \mathfrak q_i} V(\mathfrak q_i)$
is an open neighbourhood of $\mathfrak p$ where $f$ is a nonzero
divisor (i.e., a section of $\mathcal{S}$).

\medskip\noindent
Hint for (2): By part (1) the stalks agree.
\end{proof}

\begin{lemma}
\label{lemma-meromorphic-functions-integral-scheme}
Let $X$ be an integral scheme with generic point $\eta$. We have
\begin{enumerate}
\item the sheaf of meromorphic functions is
isomorphic to the constant sheaf with value the
function field (see
Morphisms, Definition \ref{morphisms-definition-function-field})
of $X$.
\item for any quasi-coherent sheaf $\mathcal{F}$ on $X$ the
sheaf $\mathcal{K}_X(\mathcal{F})$ is isomorphic to the
constant sheaf with value $\mathcal{F}_\eta$.
\end{enumerate}
\end{lemma}

\begin{proof}
Omitted.
\end{proof}

\begin{definition}
\label{definition-regular-meromorphic-section}
Let $X$ be a locally ringed space.
Let $\mathcal{L}$ be an invertible $\mathcal{O}_X$-module.
A meromorphic section $s$ of $\mathcal{L}$ is said to be {\it regular}
if the induced map
$\mathcal{K}_X \to \mathcal{K}_X(\mathcal{L})$
is injective. (In other words, this means that $s$ is a regular
section of the invertible $\mathcal{K}_X$-module
$\mathcal{K}_X(\mathcal{L})$. See
Definiton \ref{definition-regular-section}.)
\end{definition}

\noindent
First we spell out when (regular) meromorphic sections can be pulled back.
After that we discuss the existence of regular meromorphic sections
and consequences.

\begin{lemma}
\label{lemma-meromorphic-sections-pullback}
Let $f : X \to Y$ be a morphism of locally ringed spaces.
Assume that pullbacks of meromorphic functions are defined
for $f$ (see
Definition \ref{definition-pullback-meromorphic-sections}).
\begin{enumerate}
\item Let $\mathcal{F}$ be a sheaf of $\mathcal{O}_Y$-modules.
There is a canonical pullback map
$f^* : \Gamma(Y, \mathcal{K}_Y(\mathcal{F})) \to
\Gamma(X, \mathcal{F}_X(f^*\mathcal{F}))$
for meromorphic sections of $\mathcal{F}$.
\item Let $\mathcal{L}$ be an invertible $\mathcal{O}_X$-module.
A regular meromorphic section $s$ of $\mathcal{L}$ pulls back
to a regular meromorphic section $f^*s$ of $f^*\mathcal{L}$.
\end{enumerate}
\end{lemma}

\begin{proof}
Omitted. 
\end{proof}

\noindent
In some cases we can show regular meromorphic sections exist.

\begin{lemma}
\label{lemma-regular-meromorphic-section-exists}
Let $X$ be a scheme.
Let $\mathcal{L}$ be an invertible $\mathcal{O}_X$-module.
In each of the following cases $\mathcal{L}$ has a regular meromorphic
section:
\begin{enumerate}
\item $X$ is integral,
\item $X$ is reduced and any quasi-compact open has a finite
number of irreducible components, and
\item $X$ is locally Noetherian and has no embedded points.
\end{enumerate}
\end{lemma}

\begin{proof}
In case (1) we have seen in
Lemma \ref{lemma-meromorphic-functions-integral-scheme}
that $\mathcal{K}_X(\mathcal{L})$ is a constant sheaf
with value $\mathcal{L}_\eta$, and hence the result is clear.

\medskip\noindent
Suppose $X$ is a scheme. Let $G \subset X$ be the set of
generic points of irreducible components of $X$. For each $\eta \in G$
denote $j_\eta : \eta \to X$ the canonical morphism
of $\eta = \text{Spec}(\kappa(\eta))$ into $X$
(see Schemes, Lemma \ref{schemes-lemma-characterize-points}). Consider
the sheaf
$$
\mathcal{G}_X(\mathcal{L})
=
\prod\nolimits_{\eta \in G} j_{\eta, *}(\mathcal{L}_\eta).
$$
There is a canonical map
$$
\varphi :
\mathcal{K}_X(\mathcal{L})
\longrightarrow
\mathcal{G}_X(\mathcal{L})
$$
coming from the maps $\mathcal{K}_X(\mathcal{L})_\eta \to \mathcal{L}_\eta$
and adjunction (see
Sheaves, Lemma \ref{sheaves-lemma-stalk-skyscraper-adjoint}).

\medskip\noindent
We claim that in cases (2) and (3) the map $\varphi$ is an isomorphism
for any invertible sheaf $\mathcal{L}$.
Before proving this let us show that cases (2) and (3) follow from this.
Namely, we can choose $s_\eta \in \mathcal{L}_\eta$ which
generate $\mathcal{L}_\eta$, i.e., such that
$\mathcal{L}_\eta = \mathcal{O}_{X, \eta}s_\eta$.
Since the claim applied to $\mathcal{O}_X$ gives
$\mathcal{K}_X = \mathcal{G}_X(\mathcal{O}_X)$ it is
clear that the global section $s = \prod_{\eta \in G} s_\eta$
is regular as desired.

\medskip\noindent
To prove that $\varphi$ is an isomorphism we may work locally on $X$.
For example it suffices to show that sections of
$\mathcal{K}_X(\mathcal{L})$ and $\mathcal{G}_X(\mathcal{L})$
agree over small affine opens $U$. Say $U = \text{Spec}(A)$ and
$\mathcal{L}|_U \cong \mathcal{O}_U$. By
Lemmas \ref{lemma-locally-Noetherian-K} and
\ref{lemma-reduced-finite-irreducible}
we see that $\Gamma(U, \mathcal{K}_X) = Q(A)$ is
the total ring of fractions of $A$. On the other hand,
$\Gamma(U, \mathcal{G}_X(\mathcal{O}_X)) =
\prod_{\mathfrak q \subset A \text{ minimal}} A_{\mathfrak q}$.
In both cases we see that the set of minimal primes of $A$
is finite, say $\mathfrak q_1, \ldots, \mathfrak q_t$,
and that the set of zero divisors of $A$ is equal to
$\mathfrak q_1 \cup \ldots \cup \mathfrak q_t$ (see
Algebra, Lemma \ref{algebra-lemma-ass-zero-divisors}).
Hence the result follows from
Algebra, Lemma \ref{algebra-lemma-total-ring-fractions-no-embedded-points}.
\end{proof}

\begin{lemma}
\label{lemma-regular-meromorphic-ideal-denominators}
Let $X$ be a scheme.
Let $\mathcal{L}$ be an invertible $\mathcal{O}_X$-module.
Let $s$ be a regular meromorphic section of $\mathcal{L}$.
Let us denote $\mathcal{I} \subset \mathcal{O}_X$ the
sheaf of ideals defined by the rule
$$
\mathcal{I}(V)
=
\{f \in \mathcal{O}_Z(V) \mid fs \in \mathcal{L}(V)\}.
$$
The formula makes sense since
$\mathcal{L}(V) \subset \mathcal{K}_X(\mathcal{L})(V)$.
Then $\mathcal{I}$ is a quasi-coherent sheaf of ideals and
we have injective maps
$$
1 : \mathcal{I} \longrightarrow \mathcal{O}_X,
\quad
s : \mathcal{I} \longrightarrow \mathcal{L}
$$
whose cokernels are supported on closed nowhere dense subsets of $X$.
\end{lemma}

\begin{proof}
The question is local on $X$.
Hence we may assume that $X = \text{Spec}(A)$,
and $\mathcal{L} = \mathcal{O}_X$. After shrinking furhter
we may assume that $s = x/y$ with $a, b \in A$ {\it both}
nonzero divisors in $A$. Set $I = \{x \in A \mid x(a/b) \in A\}$.

\medskip\noindent
To show that $\mathcal{I}$ is quasi-coherent we have to show
that $I_f = \{x \in A_f \mid x(a/b) \in A_f\}$ for every
$f \in A$. If $c/f^n \in A_f$, $(c/f^n)(a/b) \in A_f$, then we see
that $f^mc(a/b) \in A$ for some $m$, hence $c/f^n \in I_f$.
Conversely it is easy to see that $I_f$ is contained in
$\{x \in A_f \mid x(a/b) \in A_f\}$. This proves quasi-coherence.

\medskip\noindent
Let us prove the final statement. It is clear that $(b) \subset I$.
Hence $V(I) \subset V(b)$ is a nowhere dense subset as $b$ is
a nonzero divisor. Thus the cokernel of $1$ is supported in a nowhere
dense closed set. The same argument works for the cokerenel
of $s$ since $s(b) = (a) \subset sI \subset A$.
\end{proof}

\begin{definition}
\label{definition-regular-meromorphic-ideal-denominators}
Let $X$ be a scheme.
Let $\mathcal{L}$ be an invertible $\mathcal{O}_X$-module.
Let $s$ be a regular meromorphic section of $\mathcal{L}$.
The sheaf of ideals $\mathcal{I}$ constructed in
Lemma \ref{lemma-regular-meromorphic-ideal-denominators}
is called the {\it ideal sheaf of denominators of $s$}.
\end{definition}

\noindent
Here is a lemma which will be used later.

\begin{lemma}
\label{lemma-make-maps-regular-section}
Suppose given
\begin{enumerate}
\item $X$ a locally Noetherian scheme,
\item $\mathcal{L}$ an invertible $\mathcal{O}_X$-module,
\item $s$ a regular meromorphic section of $\mathcal{L}$, and
\item $\mathcal{F}$ coherent on $X$
without embedded associated points and $\text{Supp}(\mathcal{F}) = X$.
\end{enumerate}
Let $\mathcal{I} \subset \mathcal{O}_X$ be the ideal of
denominators of $s$. Let $T \subset X$ be the union
of the supports of $\mathcal{O}_X/\mathcal{I}$ and
$\mathcal{L}/s(\mathcal{I})$ which is a nowhere dense closed
subset $T \subset X$ according to
Lemma \ref{lemma-regular-meromorphic-ideal-denominators}.
Then there are canonical injective maps
$$
1 : \mathcal{I}\mathcal{F} \to \mathcal{F},\quad
s : \mathcal{I}\mathcal{F} \to \mathcal{F}\otimes_{\mathcal{O}_X}\mathcal{L}
$$
whose cokernels are supported on $T$.
\end{lemma}

\begin{proof}
Reduce to the affine case with $\mathcal{L} \cong \mathcal{O}_X$,
and $s = a/b$ with $a, b \in A$ both nonzero divisors.
Proof of reduction step omitted.
Write $\mathcal{F} = \widetilde{M}$.
Let $I = \{x \in A \mid x(a/b) \in A\}$
so that $\mathcal{I} = \widetilde{I}$ (see
proof of Lemma \ref{lemma-regular-meromorphic-ideal-denominators}).
Note that $T = V(I) \cup V((a/b)I)$.
For any $A$-module $M$ consider the map $1 : IM \to M$; this is the
map that gives rise to the map $1$ of the lemma.
Consider on the other hand the map
$\sigma : IM \to M_b, x \mapsto ax/b$.
Since $b$ is not a zero divisor in $A$, and since
$M$ has support $\text{Spec}(A)$ and no embedded primes we
see that $b$ is a nonzero divisor on $M$ also. Hence $M \subset M_b$.
By definition of $I$ we have $\sigma(IM) \subset M$ as submodules
of $M_b$. Hence we get an $A$-module map $s : IM \to M$ (namely the
unique map such that $s(z)/1 = \sigma(z)$ in $M_b$ for all $z \in IM$).
It is injective because $a$ is a nonzero divisor also (on both $A$ and $M$).
It is clear that $M/IM$ is annihilated by $I$ and that
$M/s(IM)$ is annihilated by $(a/b)I$. Thus the lemma follows.
\end{proof}


















\section{Other chapters}

\begin{multicols}{2}
\begin{enumerate}
\item \hyperref[introduction-section-phantom]{Introduction}
\item \hyperref[conventions-section-phantom]{Conventions}
\item \hyperref[sets-section-phantom]{Set Theory}
\item \hyperref[categories-section-phantom]{Categories}
\item \hyperref[topology-section-phantom]{Topology}
\item \hyperref[sheaves-section-phantom]{Sheaves on Spaces}
\item \hyperref[algebra-section-phantom]{Commutative Algebra}
\item \hyperref[sites-section-phantom]{Sites and Sheaves}
\item \hyperref[homology-section-phantom]{Homological Algebra}
\item \hyperref[derived-section-phantom]{Derived Categories}
\item \hyperref[more-algebra-section-phantom]{More Algebra}
\item \hyperref[simplicial-section-phantom]{Simplicial Methods}
\item \hyperref[modules-section-phantom]{Sheaves of Modules}
\item \hyperref[sites-modules-section-phantom]{Modules on Sites}
\item \hyperref[injectives-section-phantom]{Injectives}
\item \hyperref[cohomology-section-phantom]{Cohomology of Sheaves}
\item \hyperref[sites-cohomology-section-phantom]{Cohomology on Sites}
\item \hyperref[hypercovering-section-phantom]{Hypercoverings}
\item \hyperref[schemes-section-phantom]{Schemes}
\item \hyperref[constructions-section-phantom]{Constructions of Schemes}
\item \hyperref[properties-section-phantom]{Properties of Schemes}
\item \hyperref[morphisms-section-phantom]{Morphisms of Schemes}
\item \hyperref[coherent-section-phantom]{Coherent Cohomology}
\item \hyperref[divisors-section-phantom]{Divisors}
\item \hyperref[limits-section-phantom]{Limits of Schemes}
\item \hyperref[varieties-section-phantom]{Varieties}
\item \hyperref[chow-section-phantom]{Chow Homology}
\item \hyperref[topologies-section-phantom]{Topologies on Schemes}
\item \hyperref[descent-section-phantom]{Descent}
\item \hyperref[more-morphisms-section-phantom]{More on Morphisms}
\item \hyperref[flat-section-phantom]{More on Flatness}
\item \hyperref[groupoids-section-phantom]{Groupoid Schemes}
\item \hyperref[more-groupoids-section-phantom]{More on Groupoid Schemes}
\item \hyperref[etale-section-phantom]{\'Etale Morphisms of Schemes}
\item \hyperref[etale-cohomology-section-phantom]{\'Etale Cohomology}
\item \hyperref[spaces-section-phantom]{Algebraic Spaces}
\item \hyperref[spaces-properties-section-phantom]{Properties of Algebraic Spaces}
\item \hyperref[spaces-morphisms-section-phantom]{Morphisms of Algebraic Spaces}
\item \hyperref[spaces-topologies-section-phantom]{Topologies on Algebraic Spaces}
\item \hyperref[spaces-descent-section-phantom]{Descent and Algebraic Spaces}
\item \hyperref[spaces-more-morphisms-section-phantom]{More on Morphisms of Spaces}
\item \hyperref[quot-section-phantom]{Quot and Hilbert Spaces}
\item \hyperref[stacks-section-phantom]{Stacks}
\item \hyperref[spaces-groupoids-section-phantom]{Groupoids in Algebraic Spaces}
\item \hyperref[spaces-more-groupoids-section-phantom]{More on Groupoids in Spaces}
\item \hyperref[bootstrap-section-phantom]{Bootstrap}
\item \hyperref[examples-stacks-section-phantom]{Examples of Stacks}
\item \hyperref[groupoids-quotients-section-phantom]{Quotients of Groupoids}
\item \hyperref[algebraic-section-phantom]{Algebraic Stacks}
\item \hyperref[criteria-section-phantom]{Criteria for Representability}
\item \hyperref[stacks-properties-section-phantom]{Properties of Algebraic Stacks}
\item \hyperref[stacks-morphisms-section-phantom]{Morphisms of Algebraic Stacks}
\item \hyperref[examples-section-phantom]{Examples}
\item \hyperref[exercises-section-phantom]{Exercises}
\item \hyperref[guide-section-phantom]{Guide to Literature}
\item \hyperref[desirables-section-phantom]{Desirables}
\item \hyperref[coding-section-phantom]{Coding Style}
\item \hyperref[fdl-section-phantom]{GNU Free Documentation License}
\item \hyperref[index-section-phantom]{Auto Generated Index}
\end{enumerate}
\end{multicols}


\bibliography{my}
\bibliographystyle{alpha}

\end{document}
