\IfFileExists{stacks-project.cls}{%
\documentclass{stacks-project}
}{%
\documentclass{amsart}
}

% The following AMS packages are automatically loaded with
% the amsart documentclass:
%\usepackage{amsmath}
%\usepackage{amssymb}
%\usepackage{amsthm}

% For dealing with references we use the comment environment
\usepackage{verbatim}
\newenvironment{reference}{\comment}{\endcomment}
%\newenvironment{reference}{}{}
\newenvironment{slogan}{\comment}{\endcomment}
\newenvironment{history}{\comment}{\endcomment}

% For commutative diagrams you can use
% \usepackage{amscd}
\usepackage[all]{xy}

% We use 2cell for 2-commutative diagrams.
\xyoption{2cell}
\UseAllTwocells

% To put source file link in headers.
% Change "template.tex" to "this_filename.tex"
% \usepackage{fancyhdr}
% \pagestyle{fancy}
% \lhead{}
% \chead{}
% \rhead{Source file: \url{template.tex}}
% \lfoot{}
% \cfoot{\thepage}
% \rfoot{}
% \renewcommand{\headrulewidth}{0pt}
% \renewcommand{\footrulewidth}{0pt}
% \renewcommand{\headheight}{12pt}

\usepackage{multicol}

% For cross-file-references
\usepackage{xr-hyper}

% Package for hypertext links:
\usepackage{hyperref}

% For any local file, say "hello.tex" you want to link to please
% use \externaldocument[hello-]{hello}
\externaldocument[introduction-]{introduction}
\externaldocument[conventions-]{conventions}
\externaldocument[sets-]{sets}
\externaldocument[categories-]{categories}
\externaldocument[topology-]{topology}
\externaldocument[sheaves-]{sheaves}
\externaldocument[sites-]{sites}
\externaldocument[stacks-]{stacks}
\externaldocument[fields-]{fields}
\externaldocument[algebra-]{algebra}
\externaldocument[brauer-]{brauer}
\externaldocument[homology-]{homology}
\externaldocument[derived-]{derived}
\externaldocument[simplicial-]{simplicial}
\externaldocument[more-algebra-]{more-algebra}
\externaldocument[smoothing-]{smoothing}
\externaldocument[modules-]{modules}
\externaldocument[sites-modules-]{sites-modules}
\externaldocument[injectives-]{injectives}
\externaldocument[cohomology-]{cohomology}
\externaldocument[sites-cohomology-]{sites-cohomology}
\externaldocument[dga-]{dga}
\externaldocument[dpa-]{dpa}
\externaldocument[hypercovering-]{hypercovering}
\externaldocument[schemes-]{schemes}
\externaldocument[constructions-]{constructions}
\externaldocument[properties-]{properties}
\externaldocument[morphisms-]{morphisms}
\externaldocument[coherent-]{coherent}
\externaldocument[divisors-]{divisors}
\externaldocument[limits-]{limits}
\externaldocument[varieties-]{varieties}
\externaldocument[topologies-]{topologies}
\externaldocument[descent-]{descent}
\externaldocument[perfect-]{perfect}
\externaldocument[more-morphisms-]{more-morphisms}
\externaldocument[flat-]{flat}
\externaldocument[groupoids-]{groupoids}
\externaldocument[more-groupoids-]{more-groupoids}
\externaldocument[etale-]{etale}
\externaldocument[chow-]{chow}
\externaldocument[intersection-]{intersection}
\externaldocument[pic-]{pic}
\externaldocument[adequate-]{adequate}
\externaldocument[dualizing-]{dualizing}
\externaldocument[duality-]{duality}
\externaldocument[discriminant-]{discriminant}
\externaldocument[local-cohomology-]{local-cohomology}
\externaldocument[curves-]{curves}
\externaldocument[resolve-]{resolve}
\externaldocument[models-]{models}
\externaldocument[pione-]{pione}
\externaldocument[etale-cohomology-]{etale-cohomology}
\externaldocument[proetale-]{proetale}
\externaldocument[crystalline-]{crystalline}
\externaldocument[spaces-]{spaces}
\externaldocument[spaces-properties-]{spaces-properties}
\externaldocument[spaces-morphisms-]{spaces-morphisms}
\externaldocument[decent-spaces-]{decent-spaces}
\externaldocument[spaces-cohomology-]{spaces-cohomology}
\externaldocument[spaces-limits-]{spaces-limits}
\externaldocument[spaces-divisors-]{spaces-divisors}
\externaldocument[spaces-over-fields-]{spaces-over-fields}
\externaldocument[spaces-topologies-]{spaces-topologies}
\externaldocument[spaces-descent-]{spaces-descent}
\externaldocument[spaces-perfect-]{spaces-perfect}
\externaldocument[spaces-more-morphisms-]{spaces-more-morphisms}
\externaldocument[spaces-flat-]{spaces-flat}
\externaldocument[spaces-groupoids-]{spaces-groupoids}
\externaldocument[spaces-more-groupoids-]{spaces-more-groupoids}
\externaldocument[bootstrap-]{bootstrap}
\externaldocument[spaces-pushouts-]{spaces-pushouts}
\externaldocument[groupoids-quotients-]{groupoids-quotients}
\externaldocument[spaces-more-cohomology-]{spaces-more-cohomology}
\externaldocument[spaces-simplicial-]{spaces-simplicial}
\externaldocument[formal-spaces-]{formal-spaces}
\externaldocument[restricted-]{restricted}
\externaldocument[spaces-resolve-]{spaces-resolve}
\externaldocument[formal-defos-]{formal-defos}
\externaldocument[defos-]{defos}
\externaldocument[cotangent-]{cotangent}
\externaldocument[examples-defos-]{examples-defos}
\externaldocument[algebraic-]{algebraic}
\externaldocument[examples-stacks-]{examples-stacks}
\externaldocument[stacks-sheaves-]{stacks-sheaves}
\externaldocument[criteria-]{criteria}
\externaldocument[artin-]{artin}
\externaldocument[quot-]{quot}
\externaldocument[stacks-properties-]{stacks-properties}
\externaldocument[stacks-morphisms-]{stacks-morphisms}
\externaldocument[stacks-limits-]{stacks-limits}
\externaldocument[stacks-cohomology-]{stacks-cohomology}
\externaldocument[stacks-perfect-]{stacks-perfect}
\externaldocument[stacks-introduction-]{stacks-introduction}
\externaldocument[stacks-more-morphisms-]{stacks-more-morphisms}
\externaldocument[stacks-geometry-]{stacks-geometry}
\externaldocument[moduli-]{moduli}
\externaldocument[moduli-curves-]{moduli-curves}
\externaldocument[examples-]{examples}
\externaldocument[exercises-]{exercises}
\externaldocument[guide-]{guide}
\externaldocument[desirables-]{desirables}
\externaldocument[coding-]{coding}
\externaldocument[obsolete-]{obsolete}
\externaldocument[fdl-]{fdl}
\externaldocument[index-]{index}

% Theorem environments.
%
\theoremstyle{plain}
\newtheorem{theorem}[subsection]{Theorem}
\newtheorem{proposition}[subsection]{Proposition}
\newtheorem{lemma}[subsection]{Lemma}

\theoremstyle{definition}
\newtheorem{definition}[subsection]{Definition}
\newtheorem{example}[subsection]{Example}
\newtheorem{exercise}[subsection]{Exercise}
\newtheorem{situation}[subsection]{Situation}

\theoremstyle{remark}
\newtheorem{remark}[subsection]{Remark}
\newtheorem{remarks}[subsection]{Remarks}

\numberwithin{equation}{subsection}

% Macros
%
\def\lim{\mathop{\rm lim}\nolimits}
\def\colim{\mathop{\rm colim}\nolimits}
\def\Spec{\mathop{\rm Spec}}
\def\Hom{\mathop{\rm Hom}\nolimits}
\def\Ext{\mathop{\rm Ext}\nolimits}
\def\SheafHom{\mathop{\mathcal{H}\!{\it om}}\nolimits}
\def\SheafExt{\mathop{\mathcal{E}\!{\it xt}}\nolimits}
\def\Sch{\textit{Sch}}
\def\Mor{\mathop{\rm Mor}\nolimits}
\def\Ob{\mathop{\rm Ob}\nolimits}
\def\Sh{\mathop{\textit{Sh}}\nolimits}
\def\NL{\mathop{N\!L}\nolimits}
\def\proetale{{pro\text{-}\acute{e}tale}}
\def\etale{{\acute{e}tale}}
\def\QCoh{\textit{QCoh}}
\def\Ker{\mathop{\rm Ker}}
\def\Im{\mathop{\rm Im}}
\def\Coker{\mathop{\rm Coker}}
\def\Coim{\mathop{\rm Coim}}

%
% Macros for moduli stacks/spaces
%
\def\QCohstack{\mathcal{QC}\!{\it oh}}
\def\Cohstack{\mathcal{C}\!{\it oh}}
\def\Spacesstack{\mathcal{S}\!{\it paces}}
\def\Quotfunctor{{\rm Quot}}
\def\Hilbfunctor{{\rm Hilb}}
\def\Curvesstack{\mathcal{C}\!{\it urves}}
\def\Polarizedstack{\mathcal{P}\!{\it olarized}}
\def\Complexesstack{\mathcal{C}\!{\it omplexes}}
% \Pic is the operator that assigns to X its picard group, usage \Pic(X)
% \Picardstack_{X/B} denotes the Picard stack of X over B
% \Picardfunctor_{X/B} denotes the Picard functor of X over B
\def\Pic{\mathop{\rm Pic}\nolimits}
\def\Picardstack{\mathcal{P}\!{\it ic}}
\def\Picardfunctor{{\rm Pic}}
\def\Deformationcategory{\mathcal{D}\!{\it ef}}


% OK, start here.
%
\begin{document}

\title{Divisors}


\maketitle

\phantomsection
\label{section-phantom}

\tableofcontents

\section{Introduction}
\label{section-introduction}

\noindent
In this chapter we study some very basic questions related
to defining divisors, etc. A basic reference is \cite{EGA}.








\section{Effective Cartier divisors}
\label{section-effective-Cartier-divisors}

\noindent
For some reason it seem convenient to define the notion of an effective
Cartier divisor before anything else.

\begin{definition}
\label{definition-effective-Cartier-divisor}
Let $S$ be a scheme.
An {\it effective Cartier divisor} on $S$ is a closed subscheme
$D \subset S$ such that the ideal sheaf $\mathcal{I}_D \subset \mathcal{O}_X$
is an invertible $\mathcal{O}_X$-module.
\end{definition}

\noindent
This notion defines closed subschemes which are of pure codimension $1$
in the strongest possible sense. Namely they are locally cut out by
a single element which is not a zero divisor. In particular they
are nowhere dense.

\begin{lemma}
\label{lemma-characterize-effective-Cartier-divisor}
Let $S$ be a scheme.
Let $D \subset S$ be a closed subscheme.
The following are equivalent:
\begin{enumerate}
\item The subscheme $D$ is an effective Cartier divisor on $S$.
\item For every $x \in D$ there exists an affine open neighbourhood
$\text{Spec}(A) = U \subset X$ of $x$ such that
$U \cap D = \text{Spec}(A/(f))$ with $f \in A$ not a zero divisor.
\end{enumerate}
\end{lemma}

\begin{proof}
Assume (1).  For every $x \in D$ there exists an affine open neighbourhood
$\text{Spec}(A) = U \subset X$ of $x$ such that
$\mathcal{I}_D|_U \cong \mathcal{O}_U$. In other words, there exists
a section $f \in \Gamma(U, \mathcal{I}_D)$ which freely generates the
restriction $\mathcal{I}_D|_U$. Hence $f \in A$, and the multiplication
map $f : A \to A$ is injective. Also, since $\mathcal{I}_D$ is
quasi-coherent we see that $D \cap U = \text{Spec}(A/(f))$.

\medskip\noindent
Assume (2). Let $x \in D$. By assumption there exists an affine open
neighbourhood $\text{Spec}(A) = U \subset X$ of $x$ such that
$U \cap D = \text{Spec}(A/(f))$ with $f \in A$ not a zero divisor.
Then $\mathcal{I}_D|_U \cong \mathcal{O}_U$ since it is equal to
$\widetilde{(f)} \cong \widetilde{A} \cong \mathcal{O}_U$.
Of course $\mathcal{I}_D$ restricted to the open subscheme
$S \setminus D$ is isomorphic to $\mathcal{O}_{X \setminus D}$.
Hence $\mathcal{I}_D$ is an invertible $\mathcal{O}_S$-module.
\end{proof}

\begin{definition}
\label{definition-sum-effective-Cartier-divisors}
Let $S$ be a scheme. Given effective Cartier divisors
$D_1$, $D_2$ on $S$ we set $D = D_1 + D_2$ equal to the
closed subscheme of $S$ corresponding to the quasi-coherent
sheaf of ideals
$\mathcal{I}_{D_1}\mathcal{I}_{D_2} \subset \mathcal{O}_S$.
We call this the {\it sum of the effective Cartier divisors
$D_1$ and $D_2$}.
\end{definition}

\begin{lemma}
\label{lemma-sum-effective-Cartier-divisors}
The sum of two effective Cartier divisors is an effective
Cartier divisor.
\end{lemma}

\begin{proof}
Omitted. Locally $f_1, f_2 \in A$ are nonzero divisors, then also
$f_1f_2 \in A$ is a nonzero divisor.
\end{proof}

\noindent
Recall that we have defined the inverse image of a closed subscheme
under any morphism of schemes in
Schemes, Definition \ref{schemes-definition-inverse-image-closed-subscheme}.

\begin{definition}
\label{definition-pullback-effective-Cartier-divisor}
Let $f : S' \to S$ be a morphism of schemes. Let $D \subset S$
be an effective Cartier divisor. We say the {\it pullback of
$D$ by $f$ is defined} if the closed subscheme $f^{-1}(D) \subset S'$
is an effective Cartier divisor. In this case we denote it either
$f^*D$ or $f^{-1}(D)$ and we call it the
{\it pullback of the effective Cartier divisor}.
\end{definition}

\noindent
The condition that $f^{-1}(D)$ is an effective Cartier divisor
is often satisfied in practice. Here is an example lemma.

\begin{lemma}
\label{lemma-pullback-effective-Cartier-defined}
Let $f : X \to Y$ be a morphism of schemes.
Let $D \subset Y$ be an effective Cartier divisor.
The pullback of $D$ by $f$ is defined in each of the following cases:
\begin{enumerate}
\item $X$, $Y$ integral and $f$ dominant,
\item $X$ reduced, and for any generic point $\xi$ of any
irreducible component we have $f(\xi) \not \in D$,
\item $f$ is flat, and
\item add more here as needed.
\end{enumerate}
\end{lemma}

\begin{proof}
Omitted.
\end{proof}

\begin{lemma}
\label{lemma-pullback-effective-Cartier-divisors-additive}
Let $f : S' \to S$ be a morphism of schemes.
Let $D_1$, $D_2$ be effective Cartier divisors on $S$.
If the pullbacks of $D_1$ and $D_2$ are defined then the
pullback of $D = D_1 + D_2$ is defined and
$f^*D = f^*D_1 + f^*D_2$.
\end{lemma}

\begin{proof}
Omitted.
\end{proof}

\begin{definition}
\label{definition-invertible-sheaf-effective-Cartier-divisor}
Let $S$ be a scheme and let $D$ be an effective Cartier divisor.
The {\it invertible sheaf $\mathcal{O}_S(D)$ associated to $D$}
is given by
$$
\mathcal{O}_S(D) :=
\textit{Hom}_{\mathcal{O}_S}(\mathcal{I}_D, \mathcal{O}_S) =
\mathcal{I}_D^{\otimes -1}.
$$
The canonical section, usually denoted $1$ or $1_D$, is the
global section of $\mathcal{O}_S(D)$ corresponding to
the inclusion mapping $\mathcal{I}_D \to \mathcal{O}_S$.
\end{definition}

\begin{definition}
\label{definition-regular-section}
Let $(X, \mathcal{O}_X)$ be a locally ringed space.
Let $\mathcal{L}$ be an invertible sheaf on $X$.
A global section $s \in \Gamma(X, \mathcal{L})$ is called a
{\it regular section} if the map $\mathcal{O}_X \to \mathcal{L}$,
$f \mapsto fs$ is injective.
\end{definition}

\begin{lemma}
\label{lemma-regular-section-structure-sheaf}
Let $X$ be a locally ringed space. Let $f \in \Gamma(X, \mathcal{O}_X)$.
The following are equivalent:
\begin{enumerate}
\item $f$ is a regular section, and
\item for any $x \in X$ the image $f \in \mathcal{O}_{X, x}$
is not a zero divisor.
\end{enumerate}
If $X$ is a scheme these are also equivalent to
\begin{enumerate}
\item[(3)] for any affine open $\text{Spec}(A) = U \subset X$
the image $f \in A$ is not a zero divisor, and
\item[(4)] there exists an affine open covering
$X = \bigcup \text{Spec}(A_i)$ such that
the image of $f$ in $A_i$ is not a zero divisor for all $i$.
\end{enumerate}
\end{lemma}

\begin{proof}
Omitted.
\end{proof}

\begin{lemma}
\label{lemma-characterize-OD}
Let $S$ be a scheme.
\begin{enumerate}
\item If $D \subset S$ is an effective Cartier divisor, then
the canonical section $1_D$ of $\mathcal{O}_S(D)$ is regular.
\item Conversely, if $s$ is a regular section of the invertible
sheaf $\mathcal{L}$, then there exists a unique effective
Cartier divisor $D \subset S$ and a unique isomorphism
$\mathcal{O}_S(D) \to \mathcal{L}$ which maps $1_D$ to $s$.
\end{enumerate}
\end{lemma}

\begin{proof}
Omitted.
\end{proof}








\section{Meromorphic functions and sections}
\label{section-meromorphic-functions}

\noindent
Let $(X, \mathcal{O}_X)$ be a locally ringed space.
For any open $U \subset X$ we have defined the set
$\mathcal{S}(U) \subset \mathcal{O}_X(U)$
of regular sections of $\mathcal{O}_X$ over $U$, see
Definition \ref{definition-regular-section}. The restriction
of a regular section to a smaller open is regular. Hence
$\mathcal{S} : U \mapsto \mathcal{S}(U)$ is a subsheaf (of sets)
of $\mathcal{O}_X$. Moreover, $\mathcal{S}(U)$
is a multiplicative subset of the ring $\mathcal{O}_X(U)$ for
each $U$. Hence we may consider
the presheaf of rings
$$
U \longmapsto \mathcal{S}(U)^{-1} \mathcal{O}_X(U),
$$
see Modules, Lemma \ref{modules-lemma-simple-invert}.

\begin{definition}
\label{definition-sheaf-meromorphic-functions}
Let $(X, \mathcal{O}_X)$ be a locally ringed space.
The {\it sheaf of meromorphic functions on $X$} is
the sheaf {\it $\mathcal{K}_X$} associated to the presheaf
displayed above. A {\it meromorphic function} on $X$
is a global section of $\mathcal{K}_X$.
\end{definition}

\noindent
Since each element of each $\mathcal{S}(U)$ is a nonzero divisor on
$\mathcal{O}_X(U)$ we see that the natural map of sheaves
of rings $\mathcal{O}_X \to \mathcal{K}_X$ is injective.

\begin{example}
\label{example-no-change}
Let $A = \mathbf{C}[x, \{y_\alpha\}_{\alpha \in \mathbf{C}}]/
((x - \alpha)y_\alpha, y_\alpha y_\beta)$. Any element of $A$
can be written uniquely as
$f(x) + \sum \lambda_\alpha y_\alpha$ with $f(x) \in \mathbf{C}[x]$
and $\lambda_\alpha \in \mathbf{C}$.
Let $X = \text{Spec}(A)$.
In this case $\mathcal{O}_X = \mathcal{K}_X$, since on
any affine open $D(f)$ the ring $A_f$ any nonzero divisor is
a unit (proof omitted).
\end{example}

\noindent
Let $(X, \mathcal{O}_X)$ be a locally ringed space.
Let $\mathcal{F}$ be a sheaf of $\mathcal{O}_X$-modules.
Consider the presheaf $U \mapsto \mathcal{S}(U)^{-1}\mathcal{F}(U)$.
Its sheafification is the sheaf
$\mathcal{K}_X(\mathcal{F}) =
\mathcal{F} \otimes_{\mathcal{O}_X} \mathcal{K}_X$, see
Modules, Lemma \ref{modules-lemma-simple-invert-module}.

\begin{definition}
\label{definition-meromorphic-section}
Let $X$ be a locally ringed space.
\begin{enumerate}
\item Let $\mathcal{F}$ be a sheaf of $\mathcal{O}_X$-modules.
A {\it meromorphic section of $\mathcal{F}$}
is a global section of $\mathcal{F} \otimes_{\mathcal{O}_X} \mathcal{K}_X$.
\item Let $\mathcal{L}$ be an invertible $\mathcal{O}_X$-module.
A meromorphic section $s$ of $\mathcal{L}$ is said to be {\it regular}
if the induced map
$\mathcal{K}_X \to \mathcal{L}\otimes_{\mathcal{O}_X}\mathcal{K}_X$
is injective. (In other words, this means that $s$ is a regular
section of the $\mathcal{K}_X$-module
$\mathcal{L}\otimes_{\mathcal{O}_X}\mathcal{K}_X$. See
Definiton \ref{definition-regular-section}.)
\end{enumerate}
\end{definition}

\begin{lemma}
\label{lemma-meromorphic-functions-integral-scheme}
Let $X$ be an integral scheme.
In this case the sheaf of meromorphic functions is
isomorphic to the constant sheaf with value the
function field (see
Morphisms, Definition \ref{morphisms-definition-function-field})
of $X$.
\end{lemma}

\begin{proof}
Omitted.
\end{proof}

























\section{Other chapters}

\begin{multicols}{2}
\begin{enumerate}
\item \hyperref[introduction-section-phantom]{Introduction}
\item \hyperref[conventions-section-phantom]{Conventions}
\item \hyperref[sets-section-phantom]{Set Theory}
\item \hyperref[categories-section-phantom]{Categories}
\item \hyperref[topology-section-phantom]{Topology}
\item \hyperref[sheaves-section-phantom]{Sheaves on Spaces}
\item \hyperref[algebra-section-phantom]{Commutative Algebra}
\item \hyperref[sites-section-phantom]{Sites and Sheaves}
\item \hyperref[homology-section-phantom]{Homological Algebra}
\item \hyperref[derived-section-phantom]{Derived Categories}
\item \hyperref[more-algebra-section-phantom]{More Algebra}
\item \hyperref[simplicial-section-phantom]{Simplicial Methods}
\item \hyperref[modules-section-phantom]{Sheaves of Modules}
\item \hyperref[sites-modules-section-phantom]{Modules on Sites}
\item \hyperref[injectives-section-phantom]{Injectives}
\item \hyperref[cohomology-section-phantom]{Cohomology of Sheaves}
\item \hyperref[sites-cohomology-section-phantom]{Cohomology on Sites}
\item \hyperref[hypercovering-section-phantom]{Hypercoverings}
\item \hyperref[schemes-section-phantom]{Schemes}
\item \hyperref[constructions-section-phantom]{Constructions of Schemes}
\item \hyperref[properties-section-phantom]{Properties of Schemes}
\item \hyperref[morphisms-section-phantom]{Morphisms of Schemes}
\item \hyperref[coherent-section-phantom]{Coherent Cohomology}
\item \hyperref[divisors-section-phantom]{Divisors}
\item \hyperref[limits-section-phantom]{Limits of Schemes}
\item \hyperref[varieties-section-phantom]{Varieties}
\item \hyperref[chow-section-phantom]{Chow Homology}
\item \hyperref[topologies-section-phantom]{Topologies on Schemes}
\item \hyperref[descent-section-phantom]{Descent}
\item \hyperref[more-morphisms-section-phantom]{More on Morphisms}
\item \hyperref[flat-section-phantom]{More on Flatness}
\item \hyperref[groupoids-section-phantom]{Groupoid Schemes}
\item \hyperref[more-groupoids-section-phantom]{More on Groupoid Schemes}
\item \hyperref[etale-section-phantom]{\'Etale Morphisms of Schemes}
\item \hyperref[etale-cohomology-section-phantom]{\'Etale Cohomology}
\item \hyperref[spaces-section-phantom]{Algebraic Spaces}
\item \hyperref[spaces-properties-section-phantom]{Properties of Algebraic Spaces}
\item \hyperref[spaces-morphisms-section-phantom]{Morphisms of Algebraic Spaces}
\item \hyperref[spaces-topologies-section-phantom]{Topologies on Algebraic Spaces}
\item \hyperref[spaces-descent-section-phantom]{Descent and Algebraic Spaces}
\item \hyperref[spaces-more-morphisms-section-phantom]{More on Morphisms of Spaces}
\item \hyperref[quot-section-phantom]{Quot and Hilbert Spaces}
\item \hyperref[stacks-section-phantom]{Stacks}
\item \hyperref[spaces-groupoids-section-phantom]{Groupoids in Algebraic Spaces}
\item \hyperref[spaces-more-groupoids-section-phantom]{More on Groupoids in Spaces}
\item \hyperref[bootstrap-section-phantom]{Bootstrap}
\item \hyperref[examples-stacks-section-phantom]{Examples of Stacks}
\item \hyperref[groupoids-quotients-section-phantom]{Quotients of Groupoids}
\item \hyperref[algebraic-section-phantom]{Algebraic Stacks}
\item \hyperref[criteria-section-phantom]{Criteria for Representability}
\item \hyperref[stacks-properties-section-phantom]{Properties of Algebraic Stacks}
\item \hyperref[stacks-morphisms-section-phantom]{Morphisms of Algebraic Stacks}
\item \hyperref[examples-section-phantom]{Examples}
\item \hyperref[exercises-section-phantom]{Exercises}
\item \hyperref[guide-section-phantom]{Guide to Literature}
\item \hyperref[desirables-section-phantom]{Desirables}
\item \hyperref[coding-section-phantom]{Coding Style}
\item \hyperref[fdl-section-phantom]{GNU Free Documentation License}
\item \hyperref[index-section-phantom]{Auto Generated Index}
\end{enumerate}
\end{multicols}


\bibliography{my}
\bibliographystyle{alpha}

\end{document}
