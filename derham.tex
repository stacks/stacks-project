\IfFileExists{stacks-project.cls}{%
\documentclass{stacks-project}
}{%
\documentclass{amsart}
}

% The following AMS packages are automatically loaded with
% the amsart documentclass:
%\usepackage{amsmath}
%\usepackage{amssymb}
%\usepackage{amsthm}

% For dealing with references we use the comment environment
\usepackage{verbatim}
\newenvironment{reference}{\comment}{\endcomment}
%\newenvironment{reference}{}{}
\newenvironment{slogan}{\comment}{\endcomment}
\newenvironment{history}{\comment}{\endcomment}

% For commutative diagrams you can use
% \usepackage{amscd}
\usepackage[all]{xy}

% We use 2cell for 2-commutative diagrams.
\xyoption{2cell}
\UseAllTwocells

% To put source file link in headers.
% Change "template.tex" to "this_filename.tex"
% \usepackage{fancyhdr}
% \pagestyle{fancy}
% \lhead{}
% \chead{}
% \rhead{Source file: \url{template.tex}}
% \lfoot{}
% \cfoot{\thepage}
% \rfoot{}
% \renewcommand{\headrulewidth}{0pt}
% \renewcommand{\footrulewidth}{0pt}
% \renewcommand{\headheight}{12pt}

\usepackage{multicol}

% For cross-file-references
\usepackage{xr-hyper}

% Package for hypertext links:
\usepackage{hyperref}

% For any local file, say "hello.tex" you want to link to please
% use \externaldocument[hello-]{hello}
\externaldocument[introduction-]{introduction}
\externaldocument[conventions-]{conventions}
\externaldocument[sets-]{sets}
\externaldocument[categories-]{categories}
\externaldocument[topology-]{topology}
\externaldocument[sheaves-]{sheaves}
\externaldocument[sites-]{sites}
\externaldocument[stacks-]{stacks}
\externaldocument[fields-]{fields}
\externaldocument[algebra-]{algebra}
\externaldocument[brauer-]{brauer}
\externaldocument[homology-]{homology}
\externaldocument[derived-]{derived}
\externaldocument[simplicial-]{simplicial}
\externaldocument[more-algebra-]{more-algebra}
\externaldocument[smoothing-]{smoothing}
\externaldocument[modules-]{modules}
\externaldocument[sites-modules-]{sites-modules}
\externaldocument[injectives-]{injectives}
\externaldocument[cohomology-]{cohomology}
\externaldocument[sites-cohomology-]{sites-cohomology}
\externaldocument[dga-]{dga}
\externaldocument[dpa-]{dpa}
\externaldocument[hypercovering-]{hypercovering}
\externaldocument[schemes-]{schemes}
\externaldocument[constructions-]{constructions}
\externaldocument[properties-]{properties}
\externaldocument[morphisms-]{morphisms}
\externaldocument[coherent-]{coherent}
\externaldocument[divisors-]{divisors}
\externaldocument[limits-]{limits}
\externaldocument[varieties-]{varieties}
\externaldocument[topologies-]{topologies}
\externaldocument[descent-]{descent}
\externaldocument[perfect-]{perfect}
\externaldocument[more-morphisms-]{more-morphisms}
\externaldocument[flat-]{flat}
\externaldocument[groupoids-]{groupoids}
\externaldocument[more-groupoids-]{more-groupoids}
\externaldocument[etale-]{etale}
\externaldocument[chow-]{chow}
\externaldocument[intersection-]{intersection}
\externaldocument[pic-]{pic}
\externaldocument[adequate-]{adequate}
\externaldocument[dualizing-]{dualizing}
\externaldocument[duality-]{duality}
\externaldocument[discriminant-]{discriminant}
\externaldocument[local-cohomology-]{local-cohomology}
\externaldocument[curves-]{curves}
\externaldocument[resolve-]{resolve}
\externaldocument[models-]{models}
\externaldocument[pione-]{pione}
\externaldocument[etale-cohomology-]{etale-cohomology}
\externaldocument[proetale-]{proetale}
\externaldocument[crystalline-]{crystalline}
\externaldocument[spaces-]{spaces}
\externaldocument[spaces-properties-]{spaces-properties}
\externaldocument[spaces-morphisms-]{spaces-morphisms}
\externaldocument[decent-spaces-]{decent-spaces}
\externaldocument[spaces-cohomology-]{spaces-cohomology}
\externaldocument[spaces-limits-]{spaces-limits}
\externaldocument[spaces-divisors-]{spaces-divisors}
\externaldocument[spaces-over-fields-]{spaces-over-fields}
\externaldocument[spaces-topologies-]{spaces-topologies}
\externaldocument[spaces-descent-]{spaces-descent}
\externaldocument[spaces-perfect-]{spaces-perfect}
\externaldocument[spaces-more-morphisms-]{spaces-more-morphisms}
\externaldocument[spaces-flat-]{spaces-flat}
\externaldocument[spaces-groupoids-]{spaces-groupoids}
\externaldocument[spaces-more-groupoids-]{spaces-more-groupoids}
\externaldocument[bootstrap-]{bootstrap}
\externaldocument[spaces-pushouts-]{spaces-pushouts}
\externaldocument[groupoids-quotients-]{groupoids-quotients}
\externaldocument[spaces-more-cohomology-]{spaces-more-cohomology}
\externaldocument[spaces-simplicial-]{spaces-simplicial}
\externaldocument[formal-spaces-]{formal-spaces}
\externaldocument[restricted-]{restricted}
\externaldocument[spaces-resolve-]{spaces-resolve}
\externaldocument[formal-defos-]{formal-defos}
\externaldocument[defos-]{defos}
\externaldocument[cotangent-]{cotangent}
\externaldocument[examples-defos-]{examples-defos}
\externaldocument[algebraic-]{algebraic}
\externaldocument[examples-stacks-]{examples-stacks}
\externaldocument[stacks-sheaves-]{stacks-sheaves}
\externaldocument[criteria-]{criteria}
\externaldocument[artin-]{artin}
\externaldocument[quot-]{quot}
\externaldocument[stacks-properties-]{stacks-properties}
\externaldocument[stacks-morphisms-]{stacks-morphisms}
\externaldocument[stacks-limits-]{stacks-limits}
\externaldocument[stacks-cohomology-]{stacks-cohomology}
\externaldocument[stacks-perfect-]{stacks-perfect}
\externaldocument[stacks-introduction-]{stacks-introduction}
\externaldocument[stacks-more-morphisms-]{stacks-more-morphisms}
\externaldocument[stacks-geometry-]{stacks-geometry}
\externaldocument[moduli-]{moduli}
\externaldocument[moduli-curves-]{moduli-curves}
\externaldocument[examples-]{examples}
\externaldocument[exercises-]{exercises}
\externaldocument[guide-]{guide}
\externaldocument[desirables-]{desirables}
\externaldocument[coding-]{coding}
\externaldocument[obsolete-]{obsolete}
\externaldocument[fdl-]{fdl}
\externaldocument[index-]{index}

% Theorem environments.
%
\theoremstyle{plain}
\newtheorem{theorem}[subsection]{Theorem}
\newtheorem{proposition}[subsection]{Proposition}
\newtheorem{lemma}[subsection]{Lemma}

\theoremstyle{definition}
\newtheorem{definition}[subsection]{Definition}
\newtheorem{example}[subsection]{Example}
\newtheorem{exercise}[subsection]{Exercise}
\newtheorem{situation}[subsection]{Situation}

\theoremstyle{remark}
\newtheorem{remark}[subsection]{Remark}
\newtheorem{remarks}[subsection]{Remarks}

\numberwithin{equation}{subsection}

% Macros
%
\def\lim{\mathop{\rm lim}\nolimits}
\def\colim{\mathop{\rm colim}\nolimits}
\def\Spec{\mathop{\rm Spec}}
\def\Hom{\mathop{\rm Hom}\nolimits}
\def\Ext{\mathop{\rm Ext}\nolimits}
\def\SheafHom{\mathop{\mathcal{H}\!{\it om}}\nolimits}
\def\SheafExt{\mathop{\mathcal{E}\!{\it xt}}\nolimits}
\def\Sch{\textit{Sch}}
\def\Mor{\mathop{\rm Mor}\nolimits}
\def\Ob{\mathop{\rm Ob}\nolimits}
\def\Sh{\mathop{\textit{Sh}}\nolimits}
\def\NL{\mathop{N\!L}\nolimits}
\def\proetale{{pro\text{-}\acute{e}tale}}
\def\etale{{\acute{e}tale}}
\def\QCoh{\textit{QCoh}}
\def\Ker{\mathop{\rm Ker}}
\def\Im{\mathop{\rm Im}}
\def\Coker{\mathop{\rm Coker}}
\def\Coim{\mathop{\rm Coim}}

%
% Macros for moduli stacks/spaces
%
\def\QCohstack{\mathcal{QC}\!{\it oh}}
\def\Cohstack{\mathcal{C}\!{\it oh}}
\def\Spacesstack{\mathcal{S}\!{\it paces}}
\def\Quotfunctor{{\rm Quot}}
\def\Hilbfunctor{{\rm Hilb}}
\def\Curvesstack{\mathcal{C}\!{\it urves}}
\def\Polarizedstack{\mathcal{P}\!{\it olarized}}
\def\Complexesstack{\mathcal{C}\!{\it omplexes}}
% \Pic is the operator that assigns to X its picard group, usage \Pic(X)
% \Picardstack_{X/B} denotes the Picard stack of X over B
% \Picardfunctor_{X/B} denotes the Picard functor of X over B
\def\Pic{\mathop{\rm Pic}\nolimits}
\def\Picardstack{\mathcal{P}\!{\it ic}}
\def\Picardfunctor{{\rm Pic}}
\def\Deformationcategory{\mathcal{D}\!{\it ef}}


% OK, start here
%
\begin{document}

\title{de Rham Cohomology; under construction}


\maketitle

\phantomsection
\label{section-phantom}

\tableofcontents

\section{Introduction}
\label{section-introduction}

\noindent
In this chapter we start with a discussion of the de Rham complex
of a ring map and we end with a proof that de Rham cohomology
defines a Weil cohomology theory when the base field has characteristic zero.




\section{The de Rham complex}
\label{section-de-rham-complex}

\noindent
Let $A \to B$ be a ring map. Denote $\text{d} : B \to \Omega_{B/A}$
the module of differentials with its universal $A$-derivation
constructed in Algebra, Section \ref{algebra-section-differentials}.
Let $\Omega_{B/A}^i = \wedge^i_B(\Omega_{B/A})$ for $i \geq 0$ be the
$i$th exterior popwer as in
Algebra, Section \ref{algebra-section-tensor-algebra}.
The {\it de Rham complex of $B$ over $A$} is the complex
$$
\Omega_{B/A}^0 \to \Omega_{B/A}^1 \to \Omega_{B/A}^2 \to \ldots
$$
constructed below.

\medskip\noindent
Given just a ring $R$ we set $\Omega_R = \Omega_{R/\mathbf{Z}}$.
This is sometimes called the absolute module of differentials of $R$;
this makes sense: if $\Omega_R$ is the module of differentials
where we only assume the Leibniz rule and not the vanishing of $\text{d}1$,
then the Leibniz rule gives $\text{d}1 = \text{d}(1 \cdot 1) =
1 \text{d}1 + 1 \text{d}1 = 2 \text{d}1$ and hence
$\text{d}1 = 0$ in $\Omega_R$. In this case the
{\it absolute de Rham complex of $R$} is the corresponding complex
$$
\Omega_R^0 \to \Omega_R^1 \to \Omega_R^2 \to \ldots
$$
where we set $\Omega^i_R = \Omega^i_{R/\mathbf{Z}}$ and so on.

\medskip\noindent
We define the maps $\text{d} : \Omega_{B/A}^p \to \Omega_{B/A}^{p + 1}$
by the rule
$$
\text{d}\left(b_0\text{d}b_1 \wedge \ldots \wedge \text{d}b_p\right) =
\text{d}b_0 \wedge \text{d}b_1 \wedge \ldots \wedge \text{d}b_p
$$
which we will show is well defined; note that
$\text{d} \circ \text{d} = 0$ so we get a complex.
Recall that $\Omega_{B/A}$ is the $B$-module generated by
elements $\text{d}b$ subject to the relations
$\text{d}a = 0$ for $a \in A$ and
$\text{d}(b + b') = \text{d}b + \text{d}b'$ and
$\text{d}(bb') = b\text{d}b' + b'\text{d}b$
for $b, b' \in B$. To prove that our map is well defined for $p = 1$
we have to show that the elements
$$
\text{d}a,
\quad\text{and}\quad
b\text{d}(b' + b'') - b\text{d}b' - b\text{d}b''
\quad\text{and}\quad
b\text{d}(b'b'') - bb'\text{d}b'' - bb''\text{d}b'
$$
for $a \in A$ and  $b, b', b'' \in B$ are mapped to zero by our rule.
This is clear by direct computation (using the Leibniz rule).
Thus we get a map
$$
\Omega^1_{B/A} \otimes_A \ldots \otimes_A \Omega^1_{B/A}
\longrightarrow
\Omega_{B/A}^{p + 1}
$$
defined by the formula
$$
\omega_1 \otimes \ldots \otimes \omega_p
\longmapsto
\sum (-1)^{i + 1}
\omega_1 \wedge \ldots \wedge \text{d}(\omega_i) \wedge \ldots \wedge \omega_p
$$
which matches our rule above on elements of the form
$b_0\text{d}b_1 \otimes \text{d}b_2 \otimes \ldots \otimes \text{d}b_p$.
It is clear that this map is alternating. To finish we have to show
that
$$
\omega_1 \otimes \ldots \otimes f\omega_i \otimes \ldots \otimes \omega_p
\quad\text{and}\quad
\omega_1 \otimes \ldots \otimes f\omega_j \otimes \ldots \otimes \omega_p
$$
are mapped to the same element for $f \in B$ and
$\omega_i$ in $\Omega^1_{B/A}$. By $A$-linearity and
the alternating property, it is enough to show this for $p = 2$, $i = 1$,
$j = 2$, $\omega_1 = b \text{d}b'$ and $\omega_2 = c \text{d} c'$
for $b, b', c, c' \in B$.
Thus we need to show that
\begin{align*}
& \text{d}(fb) \wedge \text{d}b' \wedge c \text{d}c'
- fb \text{d}b' \wedge \text{d}c \wedge \text{d}c' \\
& =
\text{d}b \wedge \text{d}b' \wedge fc\text{d}c'
- b \text{d}b' \wedge \text{d}(fc) \wedge \text{d}c'
\end{align*}
in other words that
$$
(c \text{d}(fb) + fb \text{d}c - fc \text{d}b - b \text{d}(fc))
\wedge \text{d}b' \wedge \text{d}c' = 0.
$$
This follows from the Leibniz rule.

\begin{lemma}
\label{lemma-de-rham-complex}
Let $A \to B$ be a ring map. Let $\pi : \Omega_{B/A} \to \Omega$
be a surjective $B$-module map. Denote $\text{d} : B \to \Omega$
the composition of $\pi$ with the universal derivation
$\text{d}_{B/A} : B \to \Omega_{B/A}$. Set $\Omega^i = \wedge_B^i(\Omega)$.
Assume that the kernel of $\pi$ is generated, as a $B$-module,
by elements $\omega \in \Omega_{B/A}$ such that
$\text{d}_{B/A}(\omega) \in \Omega_{B/A}^2$ maps to zero in $\Omega^2$.
Then there is a de Rham complex
$$
\Omega^0 \to \Omega^1 \to \Omega^2 \to \ldots
$$
whose differential is defined by the rule
$$
\text{d} : \Omega^p \to \Omega^{p + 1},\quad
\text{d}\left(f_0\text{d}f_1 \wedge \ldots \wedge \text{d}f_p\right) =
\text{d}f_0 \wedge \text{d}f_1 \wedge \ldots \wedge \text{d}f_p
$$
\end{lemma}

\begin{proof}
We will show that there exists a commutative diagram
$$
\xymatrix{
\Omega_{B/A}^0 \ar[d] \ar[r]_{\text{d}_{B/A}} &
\Omega_{B/A}^1 \ar[d]_\pi \ar[r]_{\text{d}_{B/A}} &
\Omega_{B/A}^2 \ar[d]_{\wedge^2\pi} \ar[r]_{\text{d}_{B/A}} &
\ldots \\
\Omega^0 \ar[r]^{\text{d}} &
\Omega^1 \ar[r]^{\text{d}} &
\Omega^2 \ar[r]^{\text{d}} &
\ldots
}
$$
the description of the map $\text{d}$ will follow from the construction
of the differentials
$\text{d}_{B/A} : \Omega^p_{B/A} \to \Omega^{p + 1}_{B/A}$ of the
de Rham complex of $B$ over $A$ given above.
Since the left most vertical arrow is an isomorphism we have
the first square. Because $\pi$ is surjective, to get the second
square it suffices to show that $\text{d}_{B/A}$ maps the kernel
of $\pi$ into the kernel of $\wedge^2\pi$. We are given that any element
of the kernel of $\pi$ is of the form
$\sum b_i\omega_i$ with $\pi(\omega_i) = 0$ and
$\wedge^2\pi(\text{d}_{B/A}(\omega_i)) = 0$.
By the Leibniz rule for $\text{d}_{B/A}$ we have
$\text{d}_{B/A}(\sum b_i\omega_i) = \sum b_i \text{d}_{B/A}(\omega_i) +
\sum \text{d}_{B/A}(b_i) \wedge \omega_i$. Hence this maps to zero
under $\wedge^2\pi$.

\medskip\noindent
For $i > 1$ we note that $\wedge^i \pi$ is surjective with
kernel the image of $\Ker(\pi) \wedge \Omega^{i - 1}_{B/A}
\to \Omega_{B/A}^i$. For $\omega_1 \in \Ker(\pi)$ and
$\omega_2 \in \Omega^{i - 1}_{B/A}$ we have
$$
\text{d}_{B/A}(\omega_1 \wedge \omega_2) =
\text{d}_{B/A}(\omega_1) \wedge \omega_2 -
\omega_1 \wedge \text{d}_{B/A}(\omega_2)
$$
which is in the kernel of $\wedge^{i + 1}\pi$ by what we just proved above.
Hence we get the $(i + 1)$st square in the diagram above.
This concludes the proof.
\end{proof}





\section{Trace maps on de Rham complexes}
\label{section-trace}

\noindent
A reference for some of the material in this section is \cite{Garel}.

\begin{example}
\label{example-no-trace}
Even in mild situations one doesn't have a trace map on de Rham complexes.
For example, consider the $\mathbf{C}$-algebra $B = \mathbf{C}[x, y]$ with
action of $G = \{\pm 1\}$ given by $x \mapsto -x$ and $y \mapsto -y$.
The invariants $A = B^G$ form a normal domain of finite type over $\mathbf{C}$
generated by $x^2, xy, y^2$. We claim that for the inclusion $A \subset B$
there is no reasonable trace map
$\Omega_{B/\mathbf{C}} \to \Omega_{A/\mathbf{C}}$
on $1$-forms. Namely, consider the element
$\omega = x \text{d} y \in \Omega_{B/\mathbf{C}}$.
Since $\omega$ is invariant under the action of $G$ if a ``reasonable''
trace map exists, then $2\omega$ should be in the image of
$\Omega_{A/\mathbf{C}} \to \Omega_{B/\mathbf{C}}$. This is
not the case: there is no way to write $2\omega$ as a linear
combination of $\text{d}(x^2)$, $\text{d}(xy)$, and $\text{d}(y^2)$
even with coefficients in $B$.
This example contradicts the main theorem in
\cite{Zannier}.
\end{example}








\section{Other chapters}

\begin{multicols}{2}
\begin{enumerate}
\item \hyperref[introduction-section-phantom]{Introduction}
\item \hyperref[conventions-section-phantom]{Conventions}
\item \hyperref[sets-section-phantom]{Set Theory}
\item \hyperref[categories-section-phantom]{Categories}
\item \hyperref[topology-section-phantom]{Topology}
\item \hyperref[sheaves-section-phantom]{Sheaves on Spaces}
\item \hyperref[algebra-section-phantom]{Commutative Algebra}
\item \hyperref[sites-section-phantom]{Sites and Sheaves}
\item \hyperref[homology-section-phantom]{Homological Algebra}
\item \hyperref[derived-section-phantom]{Derived Categories}
\item \hyperref[more-algebra-section-phantom]{More Algebra}
\item \hyperref[simplicial-section-phantom]{Simplicial Methods}
\item \hyperref[modules-section-phantom]{Sheaves of Modules}
\item \hyperref[sites-modules-section-phantom]{Modules on Sites}
\item \hyperref[injectives-section-phantom]{Injectives}
\item \hyperref[cohomology-section-phantom]{Cohomology of Sheaves}
\item \hyperref[sites-cohomology-section-phantom]{Cohomology on Sites}
\item \hyperref[hypercovering-section-phantom]{Hypercoverings}
\item \hyperref[schemes-section-phantom]{Schemes}
\item \hyperref[constructions-section-phantom]{Constructions of Schemes}
\item \hyperref[properties-section-phantom]{Properties of Schemes}
\item \hyperref[morphisms-section-phantom]{Morphisms of Schemes}
\item \hyperref[coherent-section-phantom]{Coherent Cohomology}
\item \hyperref[divisors-section-phantom]{Divisors}
\item \hyperref[limits-section-phantom]{Limits of Schemes}
\item \hyperref[varieties-section-phantom]{Varieties}
\item \hyperref[chow-section-phantom]{Chow Homology}
\item \hyperref[topologies-section-phantom]{Topologies on Schemes}
\item \hyperref[descent-section-phantom]{Descent}
\item \hyperref[more-morphisms-section-phantom]{More on Morphisms}
\item \hyperref[flat-section-phantom]{More on Flatness}
\item \hyperref[groupoids-section-phantom]{Groupoid Schemes}
\item \hyperref[more-groupoids-section-phantom]{More on Groupoid Schemes}
\item \hyperref[etale-section-phantom]{\'Etale Morphisms of Schemes}
\item \hyperref[etale-cohomology-section-phantom]{\'Etale Cohomology}
\item \hyperref[spaces-section-phantom]{Algebraic Spaces}
\item \hyperref[spaces-properties-section-phantom]{Properties of Algebraic Spaces}
\item \hyperref[spaces-morphisms-section-phantom]{Morphisms of Algebraic Spaces}
\item \hyperref[spaces-topologies-section-phantom]{Topologies on Algebraic Spaces}
\item \hyperref[spaces-descent-section-phantom]{Descent and Algebraic Spaces}
\item \hyperref[spaces-more-morphisms-section-phantom]{More on Morphisms of Spaces}
\item \hyperref[quot-section-phantom]{Quot and Hilbert Spaces}
\item \hyperref[stacks-section-phantom]{Stacks}
\item \hyperref[spaces-groupoids-section-phantom]{Groupoids in Algebraic Spaces}
\item \hyperref[spaces-more-groupoids-section-phantom]{More on Groupoids in Spaces}
\item \hyperref[bootstrap-section-phantom]{Bootstrap}
\item \hyperref[examples-stacks-section-phantom]{Examples of Stacks}
\item \hyperref[groupoids-quotients-section-phantom]{Quotients of Groupoids}
\item \hyperref[algebraic-section-phantom]{Algebraic Stacks}
\item \hyperref[criteria-section-phantom]{Criteria for Representability}
\item \hyperref[stacks-properties-section-phantom]{Properties of Algebraic Stacks}
\item \hyperref[stacks-morphisms-section-phantom]{Morphisms of Algebraic Stacks}
\item \hyperref[examples-section-phantom]{Examples}
\item \hyperref[exercises-section-phantom]{Exercises}
\item \hyperref[guide-section-phantom]{Guide to Literature}
\item \hyperref[desirables-section-phantom]{Desirables}
\item \hyperref[coding-section-phantom]{Coding Style}
\item \hyperref[fdl-section-phantom]{GNU Free Documentation License}
\item \hyperref[index-section-phantom]{Auto Generated Index}
\end{enumerate}
\end{multicols}


\bibliography{my}
\bibliographystyle{amsalpha}

\end{document}
