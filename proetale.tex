\IfFileExists{stacks-project.cls}{%
\documentclass{stacks-project}
}{%
\documentclass{amsart}
}

% The following AMS packages are automatically loaded with
% the amsart documentclass:
%\usepackage{amsmath}
%\usepackage{amssymb}
%\usepackage{amsthm}

% For dealing with references we use the comment environment
\usepackage{verbatim}
\newenvironment{reference}{\comment}{\endcomment}
%\newenvironment{reference}{}{}
\newenvironment{slogan}{\comment}{\endcomment}
\newenvironment{history}{\comment}{\endcomment}

% For commutative diagrams you can use
% \usepackage{amscd}
\usepackage[all]{xy}

% We use 2cell for 2-commutative diagrams.
\xyoption{2cell}
\UseAllTwocells

% To put source file link in headers.
% Change "template.tex" to "this_filename.tex"
% \usepackage{fancyhdr}
% \pagestyle{fancy}
% \lhead{}
% \chead{}
% \rhead{Source file: \url{template.tex}}
% \lfoot{}
% \cfoot{\thepage}
% \rfoot{}
% \renewcommand{\headrulewidth}{0pt}
% \renewcommand{\footrulewidth}{0pt}
% \renewcommand{\headheight}{12pt}

\usepackage{multicol}

% For cross-file-references
\usepackage{xr-hyper}

% Package for hypertext links:
\usepackage{hyperref}

% For any local file, say "hello.tex" you want to link to please
% use \externaldocument[hello-]{hello}
\externaldocument[introduction-]{introduction}
\externaldocument[conventions-]{conventions}
\externaldocument[sets-]{sets}
\externaldocument[categories-]{categories}
\externaldocument[topology-]{topology}
\externaldocument[sheaves-]{sheaves}
\externaldocument[sites-]{sites}
\externaldocument[stacks-]{stacks}
\externaldocument[fields-]{fields}
\externaldocument[algebra-]{algebra}
\externaldocument[brauer-]{brauer}
\externaldocument[homology-]{homology}
\externaldocument[derived-]{derived}
\externaldocument[simplicial-]{simplicial}
\externaldocument[more-algebra-]{more-algebra}
\externaldocument[smoothing-]{smoothing}
\externaldocument[modules-]{modules}
\externaldocument[sites-modules-]{sites-modules}
\externaldocument[injectives-]{injectives}
\externaldocument[cohomology-]{cohomology}
\externaldocument[sites-cohomology-]{sites-cohomology}
\externaldocument[dga-]{dga}
\externaldocument[dpa-]{dpa}
\externaldocument[hypercovering-]{hypercovering}
\externaldocument[schemes-]{schemes}
\externaldocument[constructions-]{constructions}
\externaldocument[properties-]{properties}
\externaldocument[morphisms-]{morphisms}
\externaldocument[coherent-]{coherent}
\externaldocument[divisors-]{divisors}
\externaldocument[limits-]{limits}
\externaldocument[varieties-]{varieties}
\externaldocument[topologies-]{topologies}
\externaldocument[descent-]{descent}
\externaldocument[perfect-]{perfect}
\externaldocument[more-morphisms-]{more-morphisms}
\externaldocument[flat-]{flat}
\externaldocument[groupoids-]{groupoids}
\externaldocument[more-groupoids-]{more-groupoids}
\externaldocument[etale-]{etale}
\externaldocument[chow-]{chow}
\externaldocument[intersection-]{intersection}
\externaldocument[pic-]{pic}
\externaldocument[adequate-]{adequate}
\externaldocument[dualizing-]{dualizing}
\externaldocument[duality-]{duality}
\externaldocument[discriminant-]{discriminant}
\externaldocument[local-cohomology-]{local-cohomology}
\externaldocument[curves-]{curves}
\externaldocument[resolve-]{resolve}
\externaldocument[models-]{models}
\externaldocument[pione-]{pione}
\externaldocument[etale-cohomology-]{etale-cohomology}
\externaldocument[proetale-]{proetale}
\externaldocument[crystalline-]{crystalline}
\externaldocument[spaces-]{spaces}
\externaldocument[spaces-properties-]{spaces-properties}
\externaldocument[spaces-morphisms-]{spaces-morphisms}
\externaldocument[decent-spaces-]{decent-spaces}
\externaldocument[spaces-cohomology-]{spaces-cohomology}
\externaldocument[spaces-limits-]{spaces-limits}
\externaldocument[spaces-divisors-]{spaces-divisors}
\externaldocument[spaces-over-fields-]{spaces-over-fields}
\externaldocument[spaces-topologies-]{spaces-topologies}
\externaldocument[spaces-descent-]{spaces-descent}
\externaldocument[spaces-perfect-]{spaces-perfect}
\externaldocument[spaces-more-morphisms-]{spaces-more-morphisms}
\externaldocument[spaces-flat-]{spaces-flat}
\externaldocument[spaces-groupoids-]{spaces-groupoids}
\externaldocument[spaces-more-groupoids-]{spaces-more-groupoids}
\externaldocument[bootstrap-]{bootstrap}
\externaldocument[spaces-pushouts-]{spaces-pushouts}
\externaldocument[groupoids-quotients-]{groupoids-quotients}
\externaldocument[spaces-more-cohomology-]{spaces-more-cohomology}
\externaldocument[spaces-simplicial-]{spaces-simplicial}
\externaldocument[formal-spaces-]{formal-spaces}
\externaldocument[restricted-]{restricted}
\externaldocument[spaces-resolve-]{spaces-resolve}
\externaldocument[formal-defos-]{formal-defos}
\externaldocument[defos-]{defos}
\externaldocument[cotangent-]{cotangent}
\externaldocument[examples-defos-]{examples-defos}
\externaldocument[algebraic-]{algebraic}
\externaldocument[examples-stacks-]{examples-stacks}
\externaldocument[stacks-sheaves-]{stacks-sheaves}
\externaldocument[criteria-]{criteria}
\externaldocument[artin-]{artin}
\externaldocument[quot-]{quot}
\externaldocument[stacks-properties-]{stacks-properties}
\externaldocument[stacks-morphisms-]{stacks-morphisms}
\externaldocument[stacks-limits-]{stacks-limits}
\externaldocument[stacks-cohomology-]{stacks-cohomology}
\externaldocument[stacks-perfect-]{stacks-perfect}
\externaldocument[stacks-introduction-]{stacks-introduction}
\externaldocument[stacks-more-morphisms-]{stacks-more-morphisms}
\externaldocument[stacks-geometry-]{stacks-geometry}
\externaldocument[moduli-]{moduli}
\externaldocument[moduli-curves-]{moduli-curves}
\externaldocument[examples-]{examples}
\externaldocument[exercises-]{exercises}
\externaldocument[guide-]{guide}
\externaldocument[desirables-]{desirables}
\externaldocument[coding-]{coding}
\externaldocument[obsolete-]{obsolete}
\externaldocument[fdl-]{fdl}
\externaldocument[index-]{index}

% Theorem environments.
%
\theoremstyle{plain}
\newtheorem{theorem}[subsection]{Theorem}
\newtheorem{proposition}[subsection]{Proposition}
\newtheorem{lemma}[subsection]{Lemma}

\theoremstyle{definition}
\newtheorem{definition}[subsection]{Definition}
\newtheorem{example}[subsection]{Example}
\newtheorem{exercise}[subsection]{Exercise}
\newtheorem{situation}[subsection]{Situation}

\theoremstyle{remark}
\newtheorem{remark}[subsection]{Remark}
\newtheorem{remarks}[subsection]{Remarks}

\numberwithin{equation}{subsection}

% Macros
%
\def\lim{\mathop{\rm lim}\nolimits}
\def\colim{\mathop{\rm colim}\nolimits}
\def\Spec{\mathop{\rm Spec}}
\def\Hom{\mathop{\rm Hom}\nolimits}
\def\Ext{\mathop{\rm Ext}\nolimits}
\def\SheafHom{\mathop{\mathcal{H}\!{\it om}}\nolimits}
\def\SheafExt{\mathop{\mathcal{E}\!{\it xt}}\nolimits}
\def\Sch{\textit{Sch}}
\def\Mor{\mathop{\rm Mor}\nolimits}
\def\Ob{\mathop{\rm Ob}\nolimits}
\def\Sh{\mathop{\textit{Sh}}\nolimits}
\def\NL{\mathop{N\!L}\nolimits}
\def\proetale{{pro\text{-}\acute{e}tale}}
\def\etale{{\acute{e}tale}}
\def\QCoh{\textit{QCoh}}
\def\Ker{\mathop{\rm Ker}}
\def\Im{\mathop{\rm Im}}
\def\Coker{\mathop{\rm Coker}}
\def\Coim{\mathop{\rm Coim}}

%
% Macros for moduli stacks/spaces
%
\def\QCohstack{\mathcal{QC}\!{\it oh}}
\def\Cohstack{\mathcal{C}\!{\it oh}}
\def\Spacesstack{\mathcal{S}\!{\it paces}}
\def\Quotfunctor{{\rm Quot}}
\def\Hilbfunctor{{\rm Hilb}}
\def\Curvesstack{\mathcal{C}\!{\it urves}}
\def\Polarizedstack{\mathcal{P}\!{\it olarized}}
\def\Complexesstack{\mathcal{C}\!{\it omplexes}}
% \Pic is the operator that assigns to X its picard group, usage \Pic(X)
% \Picardstack_{X/B} denotes the Picard stack of X over B
% \Picardfunctor_{X/B} denotes the Picard functor of X over B
\def\Pic{\mathop{\rm Pic}\nolimits}
\def\Picardstack{\mathcal{P}\!{\it ic}}
\def\Picardfunctor{{\rm Pic}}
\def\Deformationcategory{\mathcal{D}\!{\it ef}}


% OK, start here.
%
\begin{document}

\title{Pro-\'etale Cohomology}


\maketitle

\phantomsection
\label{section-phantom}

\tableofcontents

\section{Introduction}
\label{section-introduction}

\noindent
Most of the material in this chapter and more can be found in the preprint
\cite{BS}.

\medskip\noindent
The goal of this chapter is to introduce the pro-\'etale topology and
show how it simplifies the introduction of $\ell$-adic cohomology in
algebraic geometry.



\section{Local isomorphisms}
\label{section-local-isomorphism}

\noindent
We start with a defintion.

\begin{definition}
\label{definition-local-isomorphism}
A ring map $A \to B$ is a {\it local isomorphism} if for every prime
$\mathfrak q \subset B$ there exists a $g \in B$, $g \not \in \mathfrak q$
such that $A \to B_g$ induces an open immersion $\Spec(B_g) \to \Spec(A)$.
\end{definition}

\noindent
We list some elementary properties.

\begin{lemma}
\label{lemma-local-isomorphism}
Properties of local isomorphisms.
\begin{enumerate}
\item The composition of two local isomorphisms is a local isomorphism.
\item The base change of a local isomorphism is a local isomorphism.
\item If $A \to B$ and $A \to C$ are local isomorphisms, then any $A$-algebra
map $B \to C$ is a local isomorphism.
\end{enumerate}
\end{lemma}

\begin{proof}
Omitted.
\end{proof}

\begin{lemma}
\label{lemma-structure-local-isomorphism}
Let $A \to B$ be a local isomorphism. Then there exist $n \geq 0$,
$g_1, \ldots, g_n \in B$, $f_1, \ldots, f_n \in A$ such that
$(g_1, \ldots, g_n) = B$ and $A_{f_i} \cong B_{g_i}$.
\end{lemma}

\begin{proof}
Omitted.
\end{proof}

\begin{lemma}
\label{lemma-local-isomorphism-fully-faithful}
Let $A$ be a ring. Set $X = \Spec(A)$. The functor
$$
B \longmapsto \Spec(B)
$$
from the category of local isomorphisms
$A \to B$ to the category of topological spaces over $X$
is fully faithful.
\end{lemma}

\begin{proof}
This is clear because if $\varphi : Y \to X$ is in the essential image of
the functor of the lemma, then the structure sheaf $\mathcal{O}_Y$
is equal to $\varphi^{-1}\mathcal{O}_X$.
\end{proof}





\section{Ind-Zariski algebra}
\label{section-ind-zariski}

\noindent
We start with a defintion.

\begin{definition}
\label{definition-ind-zariski}
A ring map $A \to B$ is said to be {\it ind-Zariski} if $B$ can be written
as a filtered colimit $B = \colim B_i$ with each $A \to B_i$ a local
isomorphism.
\end{definition}

\noindent
An example of an Ind-Zariski map is a localization $A \to S^{-1}A$, see
Algebra, Lemma \ref{algebra-lemma-localization-colimit}.
The category of ind-Zariski algebras is closed under several natural
operations.

\begin{lemma}
\label{lemma-ind-zariski}
Properties of ind-Zariski maps.
\begin{enumerate}
\item The composition of two ind-Zariski maps is ind-Zariski.
\item The base change of an ind-Zariski map is another.
\item If $A \to B$ and $A \to C$ are ind-Zariski, then $A \to B \times C$
is ind-Zariski.
\end{enumerate}
\end{lemma}

\begin{proof}
Omitted.
\end{proof}

\begin{lemma}
\label{lemma-ind-ind-zariski}
A filtered colimit of ind-Zariski $A$-algebras is ind-Zariski over $A$.
\end{lemma}

\begin{proof}
Omitted.
\end{proof}

\begin{lemma}
\label{lemma-ind-permanence-zariski}
Let $A$ be a ring. Let $B \to C$ be an $A$-algebra map of ind-Zariski
$A$-algebras. Then $C$ is an ind-Zariski $B$-algebra.
\end{lemma}

\begin{proof}
Omitted.
\end{proof}

\begin{lemma}
\label{lemma-ind-zariski-fully-faithful}
Let $A$ be a ring. Set $X = \Spec(A)$. The functor
$$
B \longmapsto \Spec(B)
$$
from the category of ind-Zariski $A$-algebras to the category of topological
spaces over $X$ is fully faithful.
\end{lemma}

\begin{proof}
This is clear because if $\varphi : Y \to X$ is in the essential image of
the functor of the lemma, then the structure sheaf $\mathcal{O}_Y$
is equal to $\varphi^{-1}\mathcal{O}_X$. Details omitted.
\end{proof}

\begin{lemma}
\label{lemma-localization}
Let $A$ be a ring. Set $X = \Spec(A)$. Let $Z \subset X$ be a locally closed
subscheme which of the form $D(f) \cap V(I)$ for some $f \in A$ and
ideal $I \subset A$. Then
\begin{enumerate}
\item there exists a multiplicative subset $S \subset A$ such that
$\Spec(S^{-1}A)$ maps isomorphically to the set of points of $X$
specializing to $Z$,
\item the $A$-algebra $A_Z^\sim = S^{-1}A$ depends only on
the underlying locally closed subset $Z \subset X$,
\item $Z$ is a closed subscheme of $\Spec(A_Z^\sim)$,
\item if $Z' \subset X$ is a second locally closed subscheme of the same
form and if $Z' \subset Z$ then there is a unique $A$-algebra map
$A_Z^\sim \to A_{Z'}^\sim$.
\end{enumerate}
\end{lemma}

\begin{proof}
Let $S \subset A$ be the multiplicative set of elements which map
to invertible elements of $\Gamma(Z, \mathcal{O}_Z) = (A/I)_f$.
If $\mathfrak p$ is a prime of $A$ which does not specialize to $Z$,
then $\mathfrak p$ generates the unit ideal in $(A/I)_f$. Hence
we can write $f^n =  g + h$ for some $n \geq 0$, $g \in \mathfrak p$,
$h \in I$. Then $g \in S$ and we see that $\mathfrak p$ is not in
the spectrum of $S^{-1}A$. Conversely, if $\mathfrak p$ does specialize
to $Z$, say $\mathfrak p \subset \mathfrak q \supset I$ with
$f \not \in \mathfrak q$, then we see that $S^{-1}A$ maps to
$A_\mathfrak q$ and hence $\mathfrak p$ is in the spectrum of $S^{-1}A$.
This proves (1).

\medskip\noindent
The isomorphism class of the localization $S^{-1}A$ depends only
on the corresponding subset $\Spec(S^{-1}A) \subset \Spec(A)$, whence
(2) holds. By construction $S^{-1}A$ maps surjectively onto
$(A/I)_f$, hence (3). Part (4) follows as the multiplicative subset
$S' \subset A$ corresponding to $Z'$ contains the multiplicative subset $S$.
\end{proof}

\noindent
Let $A$ be a ring. Let $E \subset A$ be a finite subset. Given $E$
we get a stratification of $X = \Spec(A)$ into locally closed subschemes
as in Lemma \ref{lemma-localization} by looking at the vanishing behaviour
of the elements of $E$. More precisely, given a disjoint union
decomposition $E = E' \amalg E''$ we set
$$
Z(E', E'') =
\bigcap\nolimits_{f \in E'} D(f) \cap \bigcap\nolimits_{f \in E''} V(f) =
D(\prod\nolimits_{f \in E'} f) \cap V( \sum\nolimits_{f \in E''} fA)
$$
The points of $Z(E', E'')$ are exactly those $x \in X$ such that
$f \in E'$ maps to a nonzero element in $\kappa(x)$ and $f \in E''$
maps to zero in $\kappa(x)$. Thus it is clear that
\begin{equation}
\label{equation-stratify}
X = \coprod\nolimits_{E = E' \amalg E''} Z(E', E'')
\end{equation}
set theoretically. Observe that each stratum is constructible.

\begin{lemma}
\label{lemma-refine}
Let $X = \Spec(A)$ as above. Given any finite stratification
$X = \coprod T_i$ by constructible subsets, there exists a finite
subset $E \subset A$ such that the stratification (\ref{equation-stratify})
refines $X = \coprod T_i$.
\end{lemma}

\begin{proof}
We may write $T_i = \bigcup_j U_{i, j} \cap V_{i, j}^c$ as a finite union
for some $U_{i, j}$ and $V_{i, j}$ quasi-compact open in $X$.
Then we may write $U_{i, j} = \bigcup D(f_{i, j, k})$ and
$V_{i, j} = \bigcup D(g_{i, j, l})$. Then we set
$E = \{f_{i, j, k}\} \cup \{g_{i, j, l}\}$. This does the job, because
the stratification (\ref{equation-stratify}) is the one whose strata are
labeled by the vanishing pattern of the elements of $E$ which
clearly refines the given stratification.
\end{proof}

\noindent
We continue the discussion.
Given a finite subset $E \subset A$ we set
$$
B_E = \prod\nolimits_{E = E' \amalg E''} A_{Z(E', E'')}^\sim
$$
with notation as in Lemma \ref{lemma-localization}. We introduce a bit
more notation
$$
X_E = \Spec(B_E) = \coprod\nolimits_{E = E' \amalg E''} X_{E', E''}
$$
with $X_{E', E''} = \Spec(A_{Z(E', E'')}^\sim)$. Note that
$$
Z_E = \coprod\nolimits_{E = E' \amalg E''} Z(E', E'')
\longrightarrow
X_E
$$
is a closed subscheme. By construction the closed subscheme $Z_E$
contains all the closed points of the affine scheme $X_E$ as every point
of $X_{E', E''}$ specializes to a point of $Z(E', E'')$.

\medskip\noindent
Let $I(A)$ be the partially ordered set of all finite subsets of $A$.
This is a directed partially ordered set. For $E_1 \subset E_2$ there
is a canonical transition map $B_{E_1} \to B_{E_2}$ of $A$-algebras.
Namely, given a decomposition $E_2 = E'_2 \amalg E''_2$ we set
$E'_1 = E_1 \cap E'_2$ and $E''_1 = E_1 \cap E''_2$. Then observe that
$Z(E'_1, E''_1) \subset Z(E'_2, E''_2)$ hence a unique $A$-algebra map
$A_{Z(E'_1, E''_1)}^\sim \to A_{Z(E'_2, E''_2)}^\sim$ by
Lemma \ref{lemma-localization}. Using these maps collectively we obtain
the desired ring map $B_{E_1} \to B_{E_2}$. Observe that the corresponding
map of affine schemes
$$
X_{E_2} \longrightarrow X_{E_1}
$$
maps $Z_{E_2}$ into $Z_{E_1}$. By uniqueness we obtain a system of
$A$-algebras over $I(A)$ and we set
$$
B = \colim_{E \in I(A)} B_E
$$
This $A$-algebra is ind-Zariski and faithfully flat over $A$.
Set $Y = \Spec(B)$, in other words
$$
Y = \lim_{E \in I(A)} X_E
$$
and denote $Z = \lim_{E \in I(A)} Z_E$ which is a closed subscheme of $Y$.

\begin{lemma}
\label{lemma-closed-points-T}
In the situation above
\begin{enumerate}
\item The morphism $Y \to X$ induces a bijection $Z \to X$.
\item The set of closed points of $Y$ is $Z$.
\item $Z$ is a reduced scheme.
\item Every quasi-compact open of $Z$ is closed.
\end{enumerate}
\end{lemma}

\begin{proof}
For every $E$ the morphism $Z_E \to X$ is a bijection which proves (1).

\medskip\noindent
Suppose that $y \in Y$, $y \not \in Z$. Then there
exists an $E$ such that the image of $y$ in $X_E$ is not contained in
$Z_E$. Then for all $E \subset E'$ also $y$ maps to an element of $X_{E'}$
not contained in $Z_{E'}$. Let $T_{E'} \subset X_{E'}$ be the reduced
closed subscheme which is the closure of the image of $y$. It is
clear that $T = \lim_{E \subset E'} T_{E'}$ is the closure of $y$ in $Y$.
For every $E \subset E'$ the scheme $T_{E'} \cap Z_{E'}$ is nonempty
by construction of $X_{E'}$. Hence $\lim T_{E'} \cap Z_{E'}$ is nonempty
and we conclude that $T \cap Z$ is nonempty. Thus $y$ is not a closed point.

\medskip\noindent
Suppose that $z, z' \in Z$ and $z \leadsto z'$ is a specialization.
Let $x, x' \in X$ be their images. If $x \not = x'$, then there exists
an $f \in A$ such that $x \in D(f)$ and $x' \in V(f)$. Set $E = \{f\}$
so that
$$
X_E = \Spec(A_f) \amalg \Spec(A_{V(f)}^\sim)
$$
Then we see that $z$ and $z'$ map to different parts of the given
decomposition of $X_E$ above which is a contradiction. This proves (2).

\medskip\noindent
Recall that given a finite subset $E \subset A$ we have $Z_E$
is a disjoint union of the locally closed subschemes $Z(E', E'')$
each isomorphic to the spectrum of $(A/I)_f$ where $I$ is the ideal
generated by $E''$ and $f$ the product of the elements of $E'$.
Any nilpotent element $b$ of $(A/I)_f$ is the class of $g/f^n$
for some $g \in A$. Then setting $E' = E \cup \{g\}$ the reader
verifies that $b$ is pulls back to zero under the transition map
$Z_{E'} \to Z_E$ of the system. This proves (3).

\medskip\noindent
If $V \subset Z$ is quasi-compact open, then $V$ is the inverse
image of some quasi-compact open $V_E \subset Z_E$. Recall
that $Z_E = \coprod_{E = E' \amalg E''} Z(E', E'')$ maps
to a stratification of $X$ by locally closed constructible strata.
Then
$$
X =
\coprod\nolimits_{E = E' \amalg E''} Z(E', E'') \cap V_E
\ \amalg\ \coprod\nolimits_{E = E' \amalg E''} Z(E', E'') \cap V_E^c
$$
is also a stratification of $X$ by locally closed constructible strata.
By Lemma \ref{lemma-refine} we can find $E \subset E_1 \subset A$
such that the stratification (\ref{equation-stratify})
refines the displayed stratification above. It follows that under
the transition morphism $\tau : Z_{E_1} \to Z_E$ the inverse
image of $V_E$ is a disjoint union of open and closed subspaces of
$Z_{E_1}$, in particularly closed. This proves (4).
\end{proof}







\section{Connected components}
\label{section-connected-components}

\noindent
Let $A$ be a ring. Let $X = \Spec(A)$. The space of connected
components $\pi_0(X)$ is a profinite space by
Topology, Lemmas \ref{topology-lemma-pi0-profinite} and
\ref{topology-lemma-connected-component-intersection}.

\begin{lemma}
\label{lemma-construct}
Let $A$ be a ring. Let $X = \Spec(A)$. Let $T \subset \pi_0(X)$ be a
closed subset. There exists a surjective ind-Zariski ring map $A \to B$
such that $\Spec(B) \to \Spec(A)$ induces a homeomorphism of $\Spec(B)$
with the inverse image of $T$ in $X$.
\end{lemma}

\begin{proof}
Let $Z \subset X$ be the inverse image of $T$. Then $Z$ is the intersection
$Z = \bigcap Z_\alpha$ of the open and closed subsets of $X$ containing $Z$,
see Topology, Lemma \ref{topology-lemma-closed-union-connected-components}.
For each $\alpha$ we have $Z_\alpha = \Spec(A_\alpha)$ where
$A \to A_\alpha$ is a local isomorphism (a localization at an idempotent).
Setting $B = \colim A_\alpha$ proves the lemma.
\end{proof}

\begin{lemma}
\label{lemma-construct-profinite}
Let $A$ be a ring and let $X = \Spec(A)$. Let $T$ be a profinite space and
let $T \to \pi_0(X)$ be a continuous map. There exists an
ind-Zariski ring map $A \to B$ such that with $Y = \Spec(B)$ the diagram
$$
\xymatrix{
Y \ar[r] \ar[d] & \pi_0(Y) \ar[d] \\
X \ar[r] & \pi_0(X)
}
$$
is cartesian in the category of topological spaces and such that
$\pi_0(Y) = T$ as spaces over $\pi_0(X)$.
\end{lemma}

\begin{proof}
Namely, write $T = \lim T_i$ as the limit of an inverse system finite
discrete spaces over a directed partially ordered set (see
Topology, Lemma \ref{topology-lemma-profinite}). For each $i$ let
$Z_i = \text{Im}(T \to \pi_0(X) \times T_i)$. This is a closed subset.
Observe that $X \times T_i$ is the spectrum of $A_i = \prod_{t \in T_i} A$
and that $A \to A_i$ is a local isomorphism. By Lemma \ref{lemma-construct}
we see that $Z_i \subset \pi_0(X \times T_i) = \pi_0(X) \times T_i$
corresponds to a surjection $A_i \to B_i$ which is ind-Zariski
such that $\Spec(B_i) = X \times_{\pi_0(X)} Z_i$ as subsets of
$X \times T_i$. The transition maps $T_i \to T_{i'}$ induce maps
$Z_i \to Z_{i'}$ and $X \times_{\pi_0(X)} Z_i \to Z \times_{\pi_0(X)} Z_{i'}$.
Hence ring maps $B_{i'} \to B_i$
(Lemma \ref{lemma-ind-zariski-fully-faithful}).
Set $B = \colim B_i$. Then $Y = \Spec(B) = \lim \Spec(B_i)$
has all the desired properties.
\end{proof}





\section{Ind-\'etale algebra}
\label{section-ind-etale}

\noindent
We start with a defintion.

\begin{definition}
\label{definition-ind-etale}
A ring map $A \to B$ is said to be {\it ind-\'etale} if $B$ can be written
as a filtered colimit of \'etale $A$-algebras.
\end{definition}

\noindent
The category of ind-\'etale algebras is closed under taking filtered colimits.

\begin{lemma}
\label{lemma-ind-ind}
A filtered colimit of ind-\'etale $A$-algebras is ind-\'etale over $A$.
\end{lemma}

\begin{proof}
Omitted.
\end{proof}

\begin{lemma}
\label{lemma-ind-permanence}
Let $A$ be a ring. Let $B \to C$ be an $A$-algebra map of ind-\'etale
$A$-algebras. Then $C$ is an ind-\'etale $B$-algebra.
\end{lemma}

\begin{proof}
Write $B = \colim B_i$ and $C = \colim C_j$ as filtered colimits
of \'etale $A$-algebras. Then
$$
C = B \otimes_B C = \colim_{(i, j)} B \otimes_{B_i} C_j
$$
where the colimit is over the partially ordered set of pairs $(i, j)$
such that $B_i \to B \to C$ factors through $C_j \to C$. Note that
the factorization $B_i \to C_j$ is \'etale by
Algebra, Lemma \ref{algebra-lemma-map-between-etale}.
Some details omitted.
\end{proof}

\noindent
Let $A$ be a ring. Recall that any \'etale ring map $A \to B$ is isomorphic
to a standard smooth ring map of relative dimension $0$. Such a ring map
is of the form
$$
A \longrightarrow A[x_1, \ldots, x_n]/(f_1, \ldots, f_n)
$$
where the determinant of the $n \times n$-matrix with entries
$\partial f_i/\partial x_j$ is invertible in the quotient ring. See
Algebra, Lemma \ref{algebra-lemma-etale-standard-smooth}.

\medskip\noindent
Let $S(A)$ be the set of all {\it faithfully flat}\footnote{In the presence
of flatness, e.g., for smooth or \'etale ring maps,
this just means that the induced map on spectra is surjective. See
Algebra, Lemma \ref{algebra-lemma-ff-rings}.}
standard smooth $A$-algebras of relative dimension $0$.
Let $I(A)$ be the partially ordered (by inclusion) set of finite
subsets $E$ of $S(A)$. Note that $I(A)$ is a directed partially
ordered set. For $E = \{A \to B_1, \ldots, A \to B_n\}$ set
$$
B_E = B_1 \otimes_A \ldots \otimes_A B_n
$$
Observe that $B_E$ is a faithfully flat \'etale $A$-algebra.
For $E \subset E'$, there is a canonical transition map $B_E \to B_{E'}$
of \'etale $A$-algebras. Namely, say $E = \{A \to B_1, \ldots, A \to B_n\}$
and $E' = \{A \to B_1, \ldots, A \to B_{n + m}\}$ then
$B_E \to B_{E'}$ sends $b_1 \otimes \ldots \otimes b_n$ to the
element $b_1 \otimes \ldots \otimes b_n \otimes 1 \otimes \ldots \otimes 1$
of $B_{E'}$. This construction defines a system of faithfully flat
\'etale $A$-algebras over $I(A)$ and we set
$$
T(A) = \colim_{E \in I(A)} B_E
$$
Observe that $T(A)$ is a faitfully flat ind-\'etale $A$-algebra
(Algebra, Lemma \ref{algebra-lemma-colimit-faithfully-flat}). By construction
given any faithfully flat \'etale $A$-algebra $B$ there is a (non-unique)
$A$-algebra map $B \to T(A)$. Namely, pick some $(A \to B_0) \in S(A)$ such
and an isomorphism $B \cong B_0$. Then the canonical coprojection
$$
B \to B_0 \to 
T(A) = \colim_{E \in I(A)} B_E
$$
is the desired map.

\begin{lemma}
\label{lemma-first-construction}
Given a ring $A$ there exists a faithfully flat ind-\'etale $A$-algebra $C$
such that every faithfully flat \'etale ring map $C \to B$ has a section.
\end{lemma}

\begin{proof}
Set $T^1(A) = T(A)$ and $T^{n + 1}(A) = T(T^n(A))$. Let
$$
C = \colim T^n(A)
$$
This algebra is faithfully flat over each $T^n(A)$ and in particular
over $A$, see
Algebra, Lemma \ref{algebra-lemma-colimit-faithfully-flat}.
Moreover, $C$ is ind-\'etale over $A$ by Lemma \ref{lemma-ind-ind}.
If $C \to B$ is \'etale, then there exists an $n$ and an \'etale
ring map $T^n(A) \to B'$ such that $B = C \otimes_{T^n(A)} B'$, see
Algebra, Lemma \ref{algebra-lemma-etale}.
If $C \to B$ is faithfully flat, then $\Spec(B) \to \Spec(C) \to \Spec(T^n(A))$
is surjective, hence $\Spec(B') \to \Spec(T^n(A))$ is surjective.
In other words, $T^n(A) \to B'$ is faithfully flat.
By our construction, there is a $T^n(A)$-algebra map
$B' \to T^{n + 1}(A)$. This induces a $C$-algebra map $B \to C$
which finishes the proof.
\end{proof}

\begin{remark}
\label{remark-size-T}
Let $A$ be a ring. Let $\kappa$ be an infinite cardinal bigger or
equal than the cardinality of $A$. Then the cardinality of $T(A)$
is at most $\kappa$. Namely, each $B_E$ has cardinality at most
$\kappa$ and the index set $I(A)$ has cardinality at most $\kappa$
as well. Thus the result follows as $\kappa \otimes \kappa = \kappa$, see
Sets, Section \ref{sets-section-cardinals}. It follows that the
ring constructed in the proof of Lemma \ref{lemma-first-construction}
has cardinality at most $\kappa$ as well.
\end{remark}

\begin{remark}
\label{remark-first-construction-functorial}
The construction $A \mapsto T(A)$ is functorial in the following sense:
If $A \to A'$ is a ring map, then we can construct a commutative diagram
$$
\xymatrix{
A \ar[r] \ar[d] & T(A) \ar[d] \\
A' \ar[r] & T(A')
}
$$
Namely, given $(A \to A[x_1, \ldots, x_n]/(f_1, \ldots, f_n))$ in
$S(A)$ we can use the ring map $\varphi : A \to A'$ to obtain a corresponding
element $(A' \to A'[x_1, \ldots, x_n]/(f^\varphi_1, \ldots, f^\varphi_n))$
of $S(A')$ where $f^\varphi$ means the polynomial obtained by applying
$\varphi$ to the coefficients of the polynomial $f$.
Moreover, there is a commutative diagram
$$
\xymatrix{
A \ar[r] \ar[d] & A[x_1, \ldots, x_n]/(f_1, \ldots, f_n) \ar[d] \\
A' \ar[r] & A'[x_1, \ldots, x_n]/(f^\varphi_1, \ldots, f^\varphi_n)
}
$$
which is a in the category of rings. For $E \subset S(A)$ finite, set
$E' = \varphi(E)$ and define $B_E \to B_{E'}$ in the obvious manner.
Taking the colimit gives the desired map $T(A) \to T(A')$, see
Categories, Lemma \ref{categories-lemma-functorial-colimit}.
\end{remark}

\begin{lemma}
\label{lemma-have-sections-quotient}
Let $A$ be a ring such that every faithfully flat \'etale ring map
$A \to B$ has a section. Then the same is true for every quotient ring
$A/I$.
\end{lemma}

\begin{proof}
Omitted.
\end{proof}




\section{Other chapters}

\begin{multicols}{2}
\begin{enumerate}
\item \hyperref[introduction-section-phantom]{Introduction}
\item \hyperref[conventions-section-phantom]{Conventions}
\item \hyperref[sets-section-phantom]{Set Theory}
\item \hyperref[categories-section-phantom]{Categories}
\item \hyperref[topology-section-phantom]{Topology}
\item \hyperref[sheaves-section-phantom]{Sheaves on Spaces}
\item \hyperref[algebra-section-phantom]{Commutative Algebra}
\item \hyperref[sites-section-phantom]{Sites and Sheaves}
\item \hyperref[homology-section-phantom]{Homological Algebra}
\item \hyperref[derived-section-phantom]{Derived Categories}
\item \hyperref[more-algebra-section-phantom]{More Algebra}
\item \hyperref[simplicial-section-phantom]{Simplicial Methods}
\item \hyperref[modules-section-phantom]{Sheaves of Modules}
\item \hyperref[sites-modules-section-phantom]{Modules on Sites}
\item \hyperref[injectives-section-phantom]{Injectives}
\item \hyperref[cohomology-section-phantom]{Cohomology of Sheaves}
\item \hyperref[sites-cohomology-section-phantom]{Cohomology on Sites}
\item \hyperref[hypercovering-section-phantom]{Hypercoverings}
\item \hyperref[schemes-section-phantom]{Schemes}
\item \hyperref[constructions-section-phantom]{Constructions of Schemes}
\item \hyperref[properties-section-phantom]{Properties of Schemes}
\item \hyperref[morphisms-section-phantom]{Morphisms of Schemes}
\item \hyperref[coherent-section-phantom]{Coherent Cohomology}
\item \hyperref[divisors-section-phantom]{Divisors}
\item \hyperref[limits-section-phantom]{Limits of Schemes}
\item \hyperref[varieties-section-phantom]{Varieties}
\item \hyperref[chow-section-phantom]{Chow Homology}
\item \hyperref[topologies-section-phantom]{Topologies on Schemes}
\item \hyperref[descent-section-phantom]{Descent}
\item \hyperref[more-morphisms-section-phantom]{More on Morphisms}
\item \hyperref[flat-section-phantom]{More on Flatness}
\item \hyperref[groupoids-section-phantom]{Groupoid Schemes}
\item \hyperref[more-groupoids-section-phantom]{More on Groupoid Schemes}
\item \hyperref[etale-section-phantom]{\'Etale Morphisms of Schemes}
\item \hyperref[etale-cohomology-section-phantom]{\'Etale Cohomology}
\item \hyperref[spaces-section-phantom]{Algebraic Spaces}
\item \hyperref[spaces-properties-section-phantom]{Properties of Algebraic Spaces}
\item \hyperref[spaces-morphisms-section-phantom]{Morphisms of Algebraic Spaces}
\item \hyperref[spaces-topologies-section-phantom]{Topologies on Algebraic Spaces}
\item \hyperref[spaces-descent-section-phantom]{Descent and Algebraic Spaces}
\item \hyperref[spaces-more-morphisms-section-phantom]{More on Morphisms of Spaces}
\item \hyperref[quot-section-phantom]{Quot and Hilbert Spaces}
\item \hyperref[stacks-section-phantom]{Stacks}
\item \hyperref[spaces-groupoids-section-phantom]{Groupoids in Algebraic Spaces}
\item \hyperref[spaces-more-groupoids-section-phantom]{More on Groupoids in Spaces}
\item \hyperref[bootstrap-section-phantom]{Bootstrap}
\item \hyperref[examples-stacks-section-phantom]{Examples of Stacks}
\item \hyperref[groupoids-quotients-section-phantom]{Quotients of Groupoids}
\item \hyperref[algebraic-section-phantom]{Algebraic Stacks}
\item \hyperref[criteria-section-phantom]{Criteria for Representability}
\item \hyperref[stacks-properties-section-phantom]{Properties of Algebraic Stacks}
\item \hyperref[stacks-morphisms-section-phantom]{Morphisms of Algebraic Stacks}
\item \hyperref[examples-section-phantom]{Examples}
\item \hyperref[exercises-section-phantom]{Exercises}
\item \hyperref[guide-section-phantom]{Guide to Literature}
\item \hyperref[desirables-section-phantom]{Desirables}
\item \hyperref[coding-section-phantom]{Coding Style}
\item \hyperref[fdl-section-phantom]{GNU Free Documentation License}
\item \hyperref[index-section-phantom]{Auto Generated Index}
\end{enumerate}
\end{multicols}


\bibliography{my}
\bibliographystyle{amsalpha}

\end{document}
