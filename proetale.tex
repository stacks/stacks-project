\IfFileExists{stacks-project.cls}{%
\documentclass{stacks-project}
}{%
\documentclass{amsart}
}

% The following AMS packages are automatically loaded with
% the amsart documentclass:
%\usepackage{amsmath}
%\usepackage{amssymb}
%\usepackage{amsthm}

% For dealing with references we use the comment environment
\usepackage{verbatim}
\newenvironment{reference}{\comment}{\endcomment}
%\newenvironment{reference}{}{}
\newenvironment{slogan}{\comment}{\endcomment}
\newenvironment{history}{\comment}{\endcomment}

% For commutative diagrams you can use
% \usepackage{amscd}
\usepackage[all]{xy}

% We use 2cell for 2-commutative diagrams.
\xyoption{2cell}
\UseAllTwocells

% To put source file link in headers.
% Change "template.tex" to "this_filename.tex"
% \usepackage{fancyhdr}
% \pagestyle{fancy}
% \lhead{}
% \chead{}
% \rhead{Source file: \url{template.tex}}
% \lfoot{}
% \cfoot{\thepage}
% \rfoot{}
% \renewcommand{\headrulewidth}{0pt}
% \renewcommand{\footrulewidth}{0pt}
% \renewcommand{\headheight}{12pt}

\usepackage{multicol}

% For cross-file-references
\usepackage{xr-hyper}

% Package for hypertext links:
\usepackage{hyperref}

% For any local file, say "hello.tex" you want to link to please
% use \externaldocument[hello-]{hello}
\externaldocument[introduction-]{introduction}
\externaldocument[conventions-]{conventions}
\externaldocument[sets-]{sets}
\externaldocument[categories-]{categories}
\externaldocument[topology-]{topology}
\externaldocument[sheaves-]{sheaves}
\externaldocument[sites-]{sites}
\externaldocument[stacks-]{stacks}
\externaldocument[fields-]{fields}
\externaldocument[algebra-]{algebra}
\externaldocument[brauer-]{brauer}
\externaldocument[homology-]{homology}
\externaldocument[derived-]{derived}
\externaldocument[simplicial-]{simplicial}
\externaldocument[more-algebra-]{more-algebra}
\externaldocument[smoothing-]{smoothing}
\externaldocument[modules-]{modules}
\externaldocument[sites-modules-]{sites-modules}
\externaldocument[injectives-]{injectives}
\externaldocument[cohomology-]{cohomology}
\externaldocument[sites-cohomology-]{sites-cohomology}
\externaldocument[dga-]{dga}
\externaldocument[dpa-]{dpa}
\externaldocument[hypercovering-]{hypercovering}
\externaldocument[schemes-]{schemes}
\externaldocument[constructions-]{constructions}
\externaldocument[properties-]{properties}
\externaldocument[morphisms-]{morphisms}
\externaldocument[coherent-]{coherent}
\externaldocument[divisors-]{divisors}
\externaldocument[limits-]{limits}
\externaldocument[varieties-]{varieties}
\externaldocument[topologies-]{topologies}
\externaldocument[descent-]{descent}
\externaldocument[perfect-]{perfect}
\externaldocument[more-morphisms-]{more-morphisms}
\externaldocument[flat-]{flat}
\externaldocument[groupoids-]{groupoids}
\externaldocument[more-groupoids-]{more-groupoids}
\externaldocument[etale-]{etale}
\externaldocument[chow-]{chow}
\externaldocument[intersection-]{intersection}
\externaldocument[pic-]{pic}
\externaldocument[adequate-]{adequate}
\externaldocument[dualizing-]{dualizing}
\externaldocument[duality-]{duality}
\externaldocument[discriminant-]{discriminant}
\externaldocument[local-cohomology-]{local-cohomology}
\externaldocument[curves-]{curves}
\externaldocument[resolve-]{resolve}
\externaldocument[models-]{models}
\externaldocument[pione-]{pione}
\externaldocument[etale-cohomology-]{etale-cohomology}
\externaldocument[proetale-]{proetale}
\externaldocument[crystalline-]{crystalline}
\externaldocument[spaces-]{spaces}
\externaldocument[spaces-properties-]{spaces-properties}
\externaldocument[spaces-morphisms-]{spaces-morphisms}
\externaldocument[decent-spaces-]{decent-spaces}
\externaldocument[spaces-cohomology-]{spaces-cohomology}
\externaldocument[spaces-limits-]{spaces-limits}
\externaldocument[spaces-divisors-]{spaces-divisors}
\externaldocument[spaces-over-fields-]{spaces-over-fields}
\externaldocument[spaces-topologies-]{spaces-topologies}
\externaldocument[spaces-descent-]{spaces-descent}
\externaldocument[spaces-perfect-]{spaces-perfect}
\externaldocument[spaces-more-morphisms-]{spaces-more-morphisms}
\externaldocument[spaces-flat-]{spaces-flat}
\externaldocument[spaces-groupoids-]{spaces-groupoids}
\externaldocument[spaces-more-groupoids-]{spaces-more-groupoids}
\externaldocument[bootstrap-]{bootstrap}
\externaldocument[spaces-pushouts-]{spaces-pushouts}
\externaldocument[groupoids-quotients-]{groupoids-quotients}
\externaldocument[spaces-more-cohomology-]{spaces-more-cohomology}
\externaldocument[spaces-simplicial-]{spaces-simplicial}
\externaldocument[formal-spaces-]{formal-spaces}
\externaldocument[restricted-]{restricted}
\externaldocument[spaces-resolve-]{spaces-resolve}
\externaldocument[formal-defos-]{formal-defos}
\externaldocument[defos-]{defos}
\externaldocument[cotangent-]{cotangent}
\externaldocument[examples-defos-]{examples-defos}
\externaldocument[algebraic-]{algebraic}
\externaldocument[examples-stacks-]{examples-stacks}
\externaldocument[stacks-sheaves-]{stacks-sheaves}
\externaldocument[criteria-]{criteria}
\externaldocument[artin-]{artin}
\externaldocument[quot-]{quot}
\externaldocument[stacks-properties-]{stacks-properties}
\externaldocument[stacks-morphisms-]{stacks-morphisms}
\externaldocument[stacks-limits-]{stacks-limits}
\externaldocument[stacks-cohomology-]{stacks-cohomology}
\externaldocument[stacks-perfect-]{stacks-perfect}
\externaldocument[stacks-introduction-]{stacks-introduction}
\externaldocument[stacks-more-morphisms-]{stacks-more-morphisms}
\externaldocument[stacks-geometry-]{stacks-geometry}
\externaldocument[moduli-]{moduli}
\externaldocument[moduli-curves-]{moduli-curves}
\externaldocument[examples-]{examples}
\externaldocument[exercises-]{exercises}
\externaldocument[guide-]{guide}
\externaldocument[desirables-]{desirables}
\externaldocument[coding-]{coding}
\externaldocument[obsolete-]{obsolete}
\externaldocument[fdl-]{fdl}
\externaldocument[index-]{index}

% Theorem environments.
%
\theoremstyle{plain}
\newtheorem{theorem}[subsection]{Theorem}
\newtheorem{proposition}[subsection]{Proposition}
\newtheorem{lemma}[subsection]{Lemma}

\theoremstyle{definition}
\newtheorem{definition}[subsection]{Definition}
\newtheorem{example}[subsection]{Example}
\newtheorem{exercise}[subsection]{Exercise}
\newtheorem{situation}[subsection]{Situation}

\theoremstyle{remark}
\newtheorem{remark}[subsection]{Remark}
\newtheorem{remarks}[subsection]{Remarks}

\numberwithin{equation}{subsection}

% Macros
%
\def\lim{\mathop{\rm lim}\nolimits}
\def\colim{\mathop{\rm colim}\nolimits}
\def\Spec{\mathop{\rm Spec}}
\def\Hom{\mathop{\rm Hom}\nolimits}
\def\Ext{\mathop{\rm Ext}\nolimits}
\def\SheafHom{\mathop{\mathcal{H}\!{\it om}}\nolimits}
\def\SheafExt{\mathop{\mathcal{E}\!{\it xt}}\nolimits}
\def\Sch{\textit{Sch}}
\def\Mor{\mathop{\rm Mor}\nolimits}
\def\Ob{\mathop{\rm Ob}\nolimits}
\def\Sh{\mathop{\textit{Sh}}\nolimits}
\def\NL{\mathop{N\!L}\nolimits}
\def\proetale{{pro\text{-}\acute{e}tale}}
\def\etale{{\acute{e}tale}}
\def\QCoh{\textit{QCoh}}
\def\Ker{\mathop{\rm Ker}}
\def\Im{\mathop{\rm Im}}
\def\Coker{\mathop{\rm Coker}}
\def\Coim{\mathop{\rm Coim}}

%
% Macros for moduli stacks/spaces
%
\def\QCohstack{\mathcal{QC}\!{\it oh}}
\def\Cohstack{\mathcal{C}\!{\it oh}}
\def\Spacesstack{\mathcal{S}\!{\it paces}}
\def\Quotfunctor{{\rm Quot}}
\def\Hilbfunctor{{\rm Hilb}}
\def\Curvesstack{\mathcal{C}\!{\it urves}}
\def\Polarizedstack{\mathcal{P}\!{\it olarized}}
\def\Complexesstack{\mathcal{C}\!{\it omplexes}}
% \Pic is the operator that assigns to X its picard group, usage \Pic(X)
% \Picardstack_{X/B} denotes the Picard stack of X over B
% \Picardfunctor_{X/B} denotes the Picard functor of X over B
\def\Pic{\mathop{\rm Pic}\nolimits}
\def\Picardstack{\mathcal{P}\!{\it ic}}
\def\Picardfunctor{{\rm Pic}}
\def\Deformationcategory{\mathcal{D}\!{\it ef}}


% OK, start here.
%
\begin{document}

\title{Pro-\'etale Cohomology}


\maketitle

\phantomsection
\label{section-phantom}

\tableofcontents

\section{Introduction}
\label{section-introduction}

\noindent
Most of the material in this chapter and more can be found in the preprint
\cite{BS}.

\medskip\noindent
The goal of this chapter is to introduce the pro-\'etale topology and
show how it simplifies the introduction of $\ell$-adic cohomology in
algebraic geometry.




\section{Ind-\'etale algebra}
\label{section-ind-etale}

\noindent
We start with a defintion.

\begin{definition}
\label{definition-ind-etale}
A ring map $A \to B$ is said to be {\it ind-\'etale} if $B$ can be written
as a filtered colimit of \'etale $A$-algebras.
\end{definition}

\noindent
The category of ind-\'etale algebras is closed under taking filtered colimits.

\begin{lemma}
\label{lemma-ind-ind}
A filtered colimit of ind-\'etale $A$-algebras is ind-\'etale over $A$.
\end{lemma}

\begin{proof}
Omitted.
\end{proof}

\begin{lemma}
\label{lemma-ind-permanence}
Let $A$ be a ring. Let $B \to C$ be an $A$-algebra map of ind-\'etale
$A$-algebras. Then $C$ is an ind-\'etale $B$-algebra.
\end{lemma}

\begin{proof}
Write $B = \colim B_i$ and $C = \colim C_j$ as filtered colimits
of \'etale $A$-algebras. Then
$$
C = B \otimes_B C = \colim_{(i, j)} B \otimes_{B_i} C_j
$$
where the colimit is over the partially ordered set of pairs $(i, j)$
such that $B_i \to B \to C$ factors through $C_j \to C$. Note that
the factorization $B_i \to C_j$ is \'etale by
Algebra, Lemma \ref{algebra-lemma-map-between-etale}.
Some details omitted.
\end{proof}

\noindent
Let $A$ be a ring. Recall that any \'etale ring map $A \to B$ is isomorphic
to a standard smooth ring map of relative dimension $0$. Such a ring map
is of the form
$$
A \longrightarrow A[x_1, \ldots, x_n]/(f_1, \ldots, f_n)
$$
where the determinant of the $n \times n$-matrix with entries
$\partial f_i/\partial x_j$ is invertible in the quotient ring. See
Algebra, Lemma \ref{algebra-lemma-etale-standard-smooth}.

\medskip\noindent
Let $S(A)$ be the set of all {\it faithfully flat}\footnote{In the presence
of flatness, e.g., for smooth or \'etale ring maps,
this just means that the induced map on spectra is surjective. See
Algebra, Lemma \ref{algebra-lemma-ff-rings}.}
standard smooth $A$-algebras of relative dimension $0$.
Let $I(A)$ be the partially ordered (by inclusion) set of finite
subsets $E$ of $S(A)$. Note that $I(A)$ is a directed partially
ordered set. For $E = \{A \to B_1, \ldots, A \to B_n\}$ set
$$
B_E = B_1 \otimes_A \ldots \otimes_A B_n
$$
Observe that $B_E$ is a faithfully flat \'etale $A$-algebra.
For $E \subset E'$, there is a canonical transition map $B_E \to B_{E'}$
of \'etale $A$-algebras. Namely, say $E = \{A \to B_1, \ldots, A \to B_n\}$
and $E' = \{A \to B_1, \ldots, A \to B_{n + m}\}$ then
$B_E \to B_{E'}$ sends $b_1 \otimes \ldots \otimes b_n$ to the
element $b_1 \otimes \ldots \otimes b_n \otimes 1 \otimes \ldots \otimes 1$
of $B_{E'}$. This construction defines a system of faithfully flat
\'etale $A$-algebras over $I(A)$ and we set
$$
T(A) = \colim_{E \in I(A)} B_E
$$
Observe that $T(A)$ is a faitfully flat ind-\'etale $A$-algebra
(Algebra, Lemma \ref{algebra-lemma-colimit-faithfully-flat}). By construction
given any faithfully flat \'etale $A$-algebra $B$ there is a (non-unique)
$A$-algebra map $B \to T(A)$. Namely, pick some $(A \to B_0) \in S(A)$ such
and an isomorphism $B \cong B_0$. Then the canonical coprojection
$$
B \to B_0 \to 
T(A) = \colim_{E \in I(A)} B_E
$$
is the desired map.

\begin{lemma}
\label{lemma-first-construction}
Given a ring $A$ there exists a faithfully flat ind-\'etale $A$-algebra $C$
such that every faithfully flat \'etale ring map $C \to B$ has a section.
\end{lemma}

\begin{proof}
Set $T^1(A) = T(A)$ and $T^{n + 1}(A) = T(T^n(A))$. Let
$$
C = \colim T^n(A)
$$
This algebra is faithfully flat over each $T^n(A)$ and in particular
over $A$, see
Algebra, Lemma \ref{algebra-lemma-colimit-faithfully-flat}.
Moreover, $C$ is ind-\'etale over $A$ by Lemma \ref{lemma-ind-ind}.
If $C \to B$ is \'etale, then there exists an $n$ and an \'etale
ring map $T^n(A) \to B'$ such that $B = C \otimes_{T^n(A)} B'$, see
Algebra, Lemma \ref{algebra-lemma-etale}.
If $C \to B$ is faithfully flat, then $\Spec(B) \to \Spec(C) \to \Spec(T^n(A))$
is surjective, hence $\Spec(B') \to \Spec(T^n(A))$ is surjective.
In other words, $T^n(A) \to B'$ is faithfully flat.
By our construction, there is a $T^n(A)$-algebra map
$B' \to T^{n + 1}(A)$. This induces a $C$-algebra map $B \to C$
which finishes the proof.
\end{proof}

\begin{remark}
\label{remark-size-T}
Let $A$ be a ring. Let $\kappa$ be an infinite cardinal bigger or
equal than the cardinality of $A$. Then the cardinality of $T(A)$
is at most $\kappa$. Namely, each $B_E$ has cardinality at most
$\kappa$ and the index set $I(A)$ has cardinality at most $\kappa$
as well. Thus the result follows as $\kappa \otimes \kappa = \kappa$, see
Sets, Section \ref{sets-section-cardinals}. It follows that the
ring constructed in the proof of Lemma \ref{lemma-first-construction}
has cardinality at most $\kappa$ as well.
\end{remark}

\begin{remark}
\label{remark-first-construction-functorial}
The construction $A \mapsto T(A)$ is functorial in the following sense:
If $A \to A'$ is a ring map, then we can construct a commutative diagram
$$
\xymatrix{
A \ar[r] \ar[d] & T(A) \ar[d] \\
A' \ar[r] & T(A')
}
$$
Namely, given $(A \to A[x_1, \ldots, x_n]/(f_1, \ldots, f_n))$ in
$S(A)$ we can use the ring map $\varphi : A \to A'$ to obtain a corresponding
element $(A' \to A'[x_1, \ldots, x_n]/(f^\varphi_1, \ldots, f^\varphi_n))$
of $S(A')$ where $f^\varphi$ means the polynomial obtained by applying
$\varphi$ to the coefficients of the polynomial $f$.
Moreover, there is a commutative diagram
$$
\xymatrix{
A \ar[r] \ar[d] & A[x_1, \ldots, x_n]/(f_1, \ldots, f_n) \ar[d] \\
A' \ar[r] & A'[x_1, \ldots, x_n]/(f^\varphi_1, \ldots, f^\varphi_n)
}
$$
which is a in the category of rings. For $E \subset S(A)$ finite, set
$E' = \varphi(E)$ and define $B_E \to B_{E'}$ in the obvious manner.
Taking the colimit gives the desired map $T(A) \to T(A')$, see
Categories, Lemma \ref{categories-lemma-functorial-colimit}.
\end{remark}

\begin{lemma}
\label{lemma-have-sections-quotient}
Let $A$ be a ring such that every faithfully flat \'etale ring map
$A \to B$ has a section. Then the same is true for every quotient ring
$A/I$.
\end{lemma}

\begin{proof}
Omitted.
\end{proof}




\section{Other chapters}

\begin{multicols}{2}
\begin{enumerate}
\item \hyperref[introduction-section-phantom]{Introduction}
\item \hyperref[conventions-section-phantom]{Conventions}
\item \hyperref[sets-section-phantom]{Set Theory}
\item \hyperref[categories-section-phantom]{Categories}
\item \hyperref[topology-section-phantom]{Topology}
\item \hyperref[sheaves-section-phantom]{Sheaves on Spaces}
\item \hyperref[algebra-section-phantom]{Commutative Algebra}
\item \hyperref[sites-section-phantom]{Sites and Sheaves}
\item \hyperref[homology-section-phantom]{Homological Algebra}
\item \hyperref[derived-section-phantom]{Derived Categories}
\item \hyperref[more-algebra-section-phantom]{More Algebra}
\item \hyperref[simplicial-section-phantom]{Simplicial Methods}
\item \hyperref[modules-section-phantom]{Sheaves of Modules}
\item \hyperref[sites-modules-section-phantom]{Modules on Sites}
\item \hyperref[injectives-section-phantom]{Injectives}
\item \hyperref[cohomology-section-phantom]{Cohomology of Sheaves}
\item \hyperref[sites-cohomology-section-phantom]{Cohomology on Sites}
\item \hyperref[hypercovering-section-phantom]{Hypercoverings}
\item \hyperref[schemes-section-phantom]{Schemes}
\item \hyperref[constructions-section-phantom]{Constructions of Schemes}
\item \hyperref[properties-section-phantom]{Properties of Schemes}
\item \hyperref[morphisms-section-phantom]{Morphisms of Schemes}
\item \hyperref[coherent-section-phantom]{Coherent Cohomology}
\item \hyperref[divisors-section-phantom]{Divisors}
\item \hyperref[limits-section-phantom]{Limits of Schemes}
\item \hyperref[varieties-section-phantom]{Varieties}
\item \hyperref[chow-section-phantom]{Chow Homology}
\item \hyperref[topologies-section-phantom]{Topologies on Schemes}
\item \hyperref[descent-section-phantom]{Descent}
\item \hyperref[more-morphisms-section-phantom]{More on Morphisms}
\item \hyperref[flat-section-phantom]{More on Flatness}
\item \hyperref[groupoids-section-phantom]{Groupoid Schemes}
\item \hyperref[more-groupoids-section-phantom]{More on Groupoid Schemes}
\item \hyperref[etale-section-phantom]{\'Etale Morphisms of Schemes}
\item \hyperref[etale-cohomology-section-phantom]{\'Etale Cohomology}
\item \hyperref[spaces-section-phantom]{Algebraic Spaces}
\item \hyperref[spaces-properties-section-phantom]{Properties of Algebraic Spaces}
\item \hyperref[spaces-morphisms-section-phantom]{Morphisms of Algebraic Spaces}
\item \hyperref[spaces-topologies-section-phantom]{Topologies on Algebraic Spaces}
\item \hyperref[spaces-descent-section-phantom]{Descent and Algebraic Spaces}
\item \hyperref[spaces-more-morphisms-section-phantom]{More on Morphisms of Spaces}
\item \hyperref[quot-section-phantom]{Quot and Hilbert Spaces}
\item \hyperref[stacks-section-phantom]{Stacks}
\item \hyperref[spaces-groupoids-section-phantom]{Groupoids in Algebraic Spaces}
\item \hyperref[spaces-more-groupoids-section-phantom]{More on Groupoids in Spaces}
\item \hyperref[bootstrap-section-phantom]{Bootstrap}
\item \hyperref[examples-stacks-section-phantom]{Examples of Stacks}
\item \hyperref[groupoids-quotients-section-phantom]{Quotients of Groupoids}
\item \hyperref[algebraic-section-phantom]{Algebraic Stacks}
\item \hyperref[criteria-section-phantom]{Criteria for Representability}
\item \hyperref[stacks-properties-section-phantom]{Properties of Algebraic Stacks}
\item \hyperref[stacks-morphisms-section-phantom]{Morphisms of Algebraic Stacks}
\item \hyperref[examples-section-phantom]{Examples}
\item \hyperref[exercises-section-phantom]{Exercises}
\item \hyperref[guide-section-phantom]{Guide to Literature}
\item \hyperref[desirables-section-phantom]{Desirables}
\item \hyperref[coding-section-phantom]{Coding Style}
\item \hyperref[fdl-section-phantom]{GNU Free Documentation License}
\item \hyperref[index-section-phantom]{Auto Generated Index}
\end{enumerate}
\end{multicols}


\bibliography{my}
\bibliographystyle{amsalpha}

\end{document}
