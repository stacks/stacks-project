\IfFileExists{stacks-project.cls}{%
\documentclass{stacks-project}
}{%
\documentclass{amsart}
}

% The following AMS packages are automatically loaded with
% the amsart documentclass:
%\usepackage{amsmath}
%\usepackage{amssymb}
%\usepackage{amsthm}

% For dealing with references we use the comment environment
\usepackage{verbatim}
\newenvironment{reference}{\comment}{\endcomment}
%\newenvironment{reference}{}{}
\newenvironment{slogan}{\comment}{\endcomment}
\newenvironment{history}{\comment}{\endcomment}

% For commutative diagrams you can use
% \usepackage{amscd}
\usepackage[all]{xy}

% We use 2cell for 2-commutative diagrams.
\xyoption{2cell}
\UseAllTwocells

% To put source file link in headers.
% Change "template.tex" to "this_filename.tex"
% \usepackage{fancyhdr}
% \pagestyle{fancy}
% \lhead{}
% \chead{}
% \rhead{Source file: \url{template.tex}}
% \lfoot{}
% \cfoot{\thepage}
% \rfoot{}
% \renewcommand{\headrulewidth}{0pt}
% \renewcommand{\footrulewidth}{0pt}
% \renewcommand{\headheight}{12pt}

\usepackage{multicol}

% For cross-file-references
\usepackage{xr-hyper}

% Package for hypertext links:
\usepackage{hyperref}

% For any local file, say "hello.tex" you want to link to please
% use \externaldocument[hello-]{hello}
\externaldocument[introduction-]{introduction}
\externaldocument[conventions-]{conventions}
\externaldocument[sets-]{sets}
\externaldocument[categories-]{categories}
\externaldocument[topology-]{topology}
\externaldocument[sheaves-]{sheaves}
\externaldocument[sites-]{sites}
\externaldocument[stacks-]{stacks}
\externaldocument[fields-]{fields}
\externaldocument[algebra-]{algebra}
\externaldocument[brauer-]{brauer}
\externaldocument[homology-]{homology}
\externaldocument[derived-]{derived}
\externaldocument[simplicial-]{simplicial}
\externaldocument[more-algebra-]{more-algebra}
\externaldocument[smoothing-]{smoothing}
\externaldocument[modules-]{modules}
\externaldocument[sites-modules-]{sites-modules}
\externaldocument[injectives-]{injectives}
\externaldocument[cohomology-]{cohomology}
\externaldocument[sites-cohomology-]{sites-cohomology}
\externaldocument[dga-]{dga}
\externaldocument[dpa-]{dpa}
\externaldocument[hypercovering-]{hypercovering}
\externaldocument[schemes-]{schemes}
\externaldocument[constructions-]{constructions}
\externaldocument[properties-]{properties}
\externaldocument[morphisms-]{morphisms}
\externaldocument[coherent-]{coherent}
\externaldocument[divisors-]{divisors}
\externaldocument[limits-]{limits}
\externaldocument[varieties-]{varieties}
\externaldocument[topologies-]{topologies}
\externaldocument[descent-]{descent}
\externaldocument[perfect-]{perfect}
\externaldocument[more-morphisms-]{more-morphisms}
\externaldocument[flat-]{flat}
\externaldocument[groupoids-]{groupoids}
\externaldocument[more-groupoids-]{more-groupoids}
\externaldocument[etale-]{etale}
\externaldocument[chow-]{chow}
\externaldocument[intersection-]{intersection}
\externaldocument[pic-]{pic}
\externaldocument[adequate-]{adequate}
\externaldocument[dualizing-]{dualizing}
\externaldocument[duality-]{duality}
\externaldocument[discriminant-]{discriminant}
\externaldocument[local-cohomology-]{local-cohomology}
\externaldocument[curves-]{curves}
\externaldocument[resolve-]{resolve}
\externaldocument[models-]{models}
\externaldocument[pione-]{pione}
\externaldocument[etale-cohomology-]{etale-cohomology}
\externaldocument[proetale-]{proetale}
\externaldocument[crystalline-]{crystalline}
\externaldocument[spaces-]{spaces}
\externaldocument[spaces-properties-]{spaces-properties}
\externaldocument[spaces-morphisms-]{spaces-morphisms}
\externaldocument[decent-spaces-]{decent-spaces}
\externaldocument[spaces-cohomology-]{spaces-cohomology}
\externaldocument[spaces-limits-]{spaces-limits}
\externaldocument[spaces-divisors-]{spaces-divisors}
\externaldocument[spaces-over-fields-]{spaces-over-fields}
\externaldocument[spaces-topologies-]{spaces-topologies}
\externaldocument[spaces-descent-]{spaces-descent}
\externaldocument[spaces-perfect-]{spaces-perfect}
\externaldocument[spaces-more-morphisms-]{spaces-more-morphisms}
\externaldocument[spaces-flat-]{spaces-flat}
\externaldocument[spaces-groupoids-]{spaces-groupoids}
\externaldocument[spaces-more-groupoids-]{spaces-more-groupoids}
\externaldocument[bootstrap-]{bootstrap}
\externaldocument[spaces-pushouts-]{spaces-pushouts}
\externaldocument[groupoids-quotients-]{groupoids-quotients}
\externaldocument[spaces-more-cohomology-]{spaces-more-cohomology}
\externaldocument[spaces-simplicial-]{spaces-simplicial}
\externaldocument[formal-spaces-]{formal-spaces}
\externaldocument[restricted-]{restricted}
\externaldocument[spaces-resolve-]{spaces-resolve}
\externaldocument[formal-defos-]{formal-defos}
\externaldocument[defos-]{defos}
\externaldocument[cotangent-]{cotangent}
\externaldocument[examples-defos-]{examples-defos}
\externaldocument[algebraic-]{algebraic}
\externaldocument[examples-stacks-]{examples-stacks}
\externaldocument[stacks-sheaves-]{stacks-sheaves}
\externaldocument[criteria-]{criteria}
\externaldocument[artin-]{artin}
\externaldocument[quot-]{quot}
\externaldocument[stacks-properties-]{stacks-properties}
\externaldocument[stacks-morphisms-]{stacks-morphisms}
\externaldocument[stacks-limits-]{stacks-limits}
\externaldocument[stacks-cohomology-]{stacks-cohomology}
\externaldocument[stacks-perfect-]{stacks-perfect}
\externaldocument[stacks-introduction-]{stacks-introduction}
\externaldocument[stacks-more-morphisms-]{stacks-more-morphisms}
\externaldocument[stacks-geometry-]{stacks-geometry}
\externaldocument[moduli-]{moduli}
\externaldocument[moduli-curves-]{moduli-curves}
\externaldocument[examples-]{examples}
\externaldocument[exercises-]{exercises}
\externaldocument[guide-]{guide}
\externaldocument[desirables-]{desirables}
\externaldocument[coding-]{coding}
\externaldocument[obsolete-]{obsolete}
\externaldocument[fdl-]{fdl}
\externaldocument[index-]{index}

% Theorem environments.
%
\theoremstyle{plain}
\newtheorem{theorem}[subsection]{Theorem}
\newtheorem{proposition}[subsection]{Proposition}
\newtheorem{lemma}[subsection]{Lemma}

\theoremstyle{definition}
\newtheorem{definition}[subsection]{Definition}
\newtheorem{example}[subsection]{Example}
\newtheorem{exercise}[subsection]{Exercise}
\newtheorem{situation}[subsection]{Situation}

\theoremstyle{remark}
\newtheorem{remark}[subsection]{Remark}
\newtheorem{remarks}[subsection]{Remarks}

\numberwithin{equation}{subsection}

% Macros
%
\def\lim{\mathop{\rm lim}\nolimits}
\def\colim{\mathop{\rm colim}\nolimits}
\def\Spec{\mathop{\rm Spec}}
\def\Hom{\mathop{\rm Hom}\nolimits}
\def\Ext{\mathop{\rm Ext}\nolimits}
\def\SheafHom{\mathop{\mathcal{H}\!{\it om}}\nolimits}
\def\SheafExt{\mathop{\mathcal{E}\!{\it xt}}\nolimits}
\def\Sch{\textit{Sch}}
\def\Mor{\mathop{\rm Mor}\nolimits}
\def\Ob{\mathop{\rm Ob}\nolimits}
\def\Sh{\mathop{\textit{Sh}}\nolimits}
\def\NL{\mathop{N\!L}\nolimits}
\def\proetale{{pro\text{-}\acute{e}tale}}
\def\etale{{\acute{e}tale}}
\def\QCoh{\textit{QCoh}}
\def\Ker{\mathop{\rm Ker}}
\def\Im{\mathop{\rm Im}}
\def\Coker{\mathop{\rm Coker}}
\def\Coim{\mathop{\rm Coim}}

%
% Macros for moduli stacks/spaces
%
\def\QCohstack{\mathcal{QC}\!{\it oh}}
\def\Cohstack{\mathcal{C}\!{\it oh}}
\def\Spacesstack{\mathcal{S}\!{\it paces}}
\def\Quotfunctor{{\rm Quot}}
\def\Hilbfunctor{{\rm Hilb}}
\def\Curvesstack{\mathcal{C}\!{\it urves}}
\def\Polarizedstack{\mathcal{P}\!{\it olarized}}
\def\Complexesstack{\mathcal{C}\!{\it omplexes}}
% \Pic is the operator that assigns to X its picard group, usage \Pic(X)
% \Picardstack_{X/B} denotes the Picard stack of X over B
% \Picardfunctor_{X/B} denotes the Picard functor of X over B
\def\Pic{\mathop{\rm Pic}\nolimits}
\def\Picardstack{\mathcal{P}\!{\it ic}}
\def\Picardfunctor{{\rm Pic}}
\def\Deformationcategory{\mathcal{D}\!{\it ef}}


% OK, start here.
%
\begin{document}

\title{Properties of Algebraic Spaces}


\maketitle

\phantomsection
\label{section-phantom}

\tableofcontents

\section{Introduction}
\label{section-introduction}

\noindent
Please see Spaces, Section \ref{spaces-section-introduction}
for a brief introduction to algebraic spaces, and please read
some of that chapter for our basic definitions and conventions
concerning algebraic spaces. In this chapter we start introducing
some basic notions and properties of algebraic spaces. A fundamental
reference for the case of quasi-separated algebraic spaces is
\cite{Kn}.

\medskip\noindent
The discussion is somewhat awkward at times since we made the design
descision to first talk about properties of algebraic spaces by
themselves, and only later about properties of morphisms of algebraic
spaces.



\section{Conventions}
\label{section-conventions}

\noindent
The standing assumption is that all schemes are contained in
a big fppf site $\textit{Sch}_{fppf}$. And all rings $A$ considered
have the property that $\text{Spec}(A)$ is (isomorphic) to an
object of this big site.


\section{Separation axioms}
\label{section-separation}

\noindent
In this section we collect all the ``absolute'' separation conditions
of algebraic spaces. Here is a list.

\begin{definition}
\label{definition-separated}
(See Spaces, Definition \ref{spaces-definition-separated}.)
Let $S$ be a scheme contained in $\textit{Sch}_{fppf}$.
Let $F$ be an algebraic space over $S$.
Let $\Delta : F \to F \times F$ be the diagonal morphism.
\begin{enumerate}
\item We say $F$ is {\it separated} if $\Delta$ is a closed immersion.
\item We say $F$ is {\it weakly locally separated}\footnote{This is probably
nonstandard notation.} if $\Delta$ is an
immersion.
\item We say $F$ is {\it locally separated} if $\Delta$ is a
quasi-compact immersion.
\item We say $F$ is {\it quasi-separated} if $\Delta$ is quasi-compact.
\item We say $F$ is {\it Zariski locally quasi-separated}\footnote{
This notion was suggested by B.\ Conrad.} if there
exists a Zariski covering $F = \bigcup_{i \in I} F_i$ (see Spaces,
Definition \ref{spaces-definition-Zariski-open-covering}) such that
each $F_i$ is quasi-separated.
\end{enumerate}
\end{definition}

\begin{lemma}
\label{lemma-trivial-implications}
Let $S$ be a scheme.
Let $X$ be an algebraic space over $S$.
We have the following implications among the separation axioms
of Definition \ref{definition-separated}:
\begin{enumerate}
\item separated implies all the others,
\item locally separated implies quasi-separated and weakly locally separated,
\item quasi-separated implies Zariski locally quasi-separated.
\end{enumerate}
\end{lemma}

\begin{proof}
Omitted.
\end{proof}














\section{Points of algebraic spaces}
\label{section-points}

\noindent
As is clear from Spaces, Example \ref{spaces-example-affine-line-translation}
a point of an algebraic stack should not be defined as a monomorphism
from the spectrum of a field.
Instead we define them as equivalence classes of morphisms of specra
of fields as equivalence classes of morphisms from spectra of fields.
exactly as explained in Schemes, Section \ref{schemes-section-points}.

\medskip\noindent
Let $S$ be a scheme.
Let $F$ be a presheaf on $(\textit{Sch}/S)_{fppf}$.
Let $K$ is a field. Consider a morphism
$$
\text{Spec}(K) \longrightarrow F.
$$
By the Yoneda Lemma this is given by an
element $p \in F(\text{Spec}(K))$. We say that two such
pairs $(\text{Spec}(K), p)$ and $(\text{Spec}(L), q)$
are {\it equivalent} if there exists
a third field $\Omega$ and a commutative diagram
$$
\xymatrix{
\text{Spec}(\Omega) \ar[r] \ar[d] &
\text{Spec}(L) \ar[d]^q \\
\text{Spec}(K) \ar[r]^p &
F.
}
$$
In other words, there are field extensions
$K \to \Omega$ and $L \to \Omega$ such that
$p$ and $q$ map to the same element
of $F(\text{Spec}(\Omega))$. We omit the verification that this
defines an equivalence relation.

\begin{definition}
\label{definition-points}
Let $S$ be a scheme. Let $X$ be an algebraic space over $S$.
A {\it point} of $X$ is an equivalence class of morphisms
from spectra of fields into $X$.
The set of points of $X$ is denoted $|X|$.
\end{definition}

\noindent
Note that if $f : X \to Y$ is a morphism of algebraic spaces
over $S$, then there is an induced map $|f| : |X| \to |Y|$ which
maps a representative $x : \text{Spec}(K) \to X$ to the representative
$f \circ x : \text{Spec}(K) \to Y$.

\begin{lemma}
\label{lemma-scheme-points}
Let $S$ be a scheme. Let $X$ be a scheme over $S$.
The points of $X$ as a scheme are in canonical 1-1 correspondence
with the points of $X$ as an algebraic space.
\end{lemma}

\begin{proof}
This is Schemes, Lemma \ref{schemes-lemma-characterize-points}.
\end{proof}

\begin{lemma}
\label{lemma-points-presentation}
Let $S$ be a scheme.
Let $X$ be an algebraic space over $S$.
Let $X = U/R$ be a presentation of $X$, see
Spaces, Definition \ref{spaces-definition-presentation}.
Then $|R| \to |U| \times |U|$ is an equivalence relation
and $|X| = |U|/|R|$.
\end{lemma}

\begin{proof}
We will use Lemma \ref{lemma-scheme-points} without further mention.
The assumption means that $U$ is a scheme, $p : U \to X$ is a
surjective, etale morphism, $R = U \times_X U$ is a scheme
and defines an etale equivalence relation on $U$ such that
$X = U/R$ as sheaves. Let $x : \text{Spec}(K) \to X$ be a morphism
from the spectrum of a field into $X$. By assumption the scheme
$\text{Spec}(K) \times_X U$ is nonempty. Hence there exists a
field extension $K \subset K'$ and a morphism $\text{Spec}(K') \to U$
such that the diagram
$$
\xymatrix{
\text{Spec}(K') \ar[r] \ar[d] & U \ar[d]^p \\
\text{Spec}(K) \ar[r]^-x & X
}
$$
commutes. Hence we see that $|p| : |U| \to |X|$ is surjective.

\medskip\noindent
Since $j = (t, s) : R \to U \times_S U$ is a monomorphism we see that
$|R| \to |U| \times |U|$ is injective. Moreover, it is an equivalence
relation since $R \to U \times_S U$ is an equivalence relation.
Namely, if $u_i \in |U|$, $i = 1, 2, 3$ and $(u_1, u_2)$ is
the image of $r \in |R|$, and $(u_2, u_3)$ is the image of
$r' \in |R|$, then we see that there exists some point of
$R \times_{s, U, t} R$ contains a point $\tilde r$ mapping to
$(r, r')$. Set $r'' = \text{pr}_{02}(\tilde r) \in |R|$. Then we
see that $\tilde r$ maps to $(u_1, u_2)$ in $|U| \times |U|$.
Also, since $U \to X$ equalizes $s, t : R \to U$ we see that
the map $|U| \to |X|$ factors through $|U|/|R|$. Finally, we have
to show that if $u_1, u_2 \in |U|$ map to the same point of $X$
then they come from an element of $|R|$. Let us write this out in
a little more detail. Namely, choose representatives
$u_i : \text{Spec}(K_i) \to U$ of $u_i$. The assumption that they map
to the same point of $X$ means that there exists a commutive diagram
$$
\xymatrix{
\text{Spec}(L) \ar[r] \ar[d] &
\text{Spec}(K_2) \ar[d]^{p \circ u_2} \\
\text{Spec}(K_1) \ar[r]^{p \circ u_1} &
X.
}
$$
Clearly this means that $\text{Spec}(L)$ maps into $R = U \times_X U$
giving a poitn $r \in |R|$ which maps to $(u_1, u_2) \in |U| \times |U|$.
\end{proof}

\begin{lemma}
\label{lemma-topology-points}
Let $S$ be a scheme. There exists a unique topology on the set of points
of algebraic spaces over $S$ with the following properties:
\begin{enumerate}
\item for every morphism of algebraic spaces $X \to Y$ over $S$
the map $|X| \to |Y|$ is continuous, and
\item for every etale morphism $U \to X$ with $U$ a scheme
the map of topological spaces $|U| \to |X|$ is continuous and open.
\end{enumerate}
\end{lemma}

\begin{proof}
Let $X$ be an algebraic space over $S$. Let $X = U/R$ be a presentation,
where $j = (t, s) : R \to U$ is an etale equivalence relation.
Let $p : U \to X$ denote the etale surjective morphism.
We define $W \subset |X|$ is open if and only if $|p|^{-1}(W)$
is an open subset of $|U|$. This is a topology on $|X|$ by
the result of Lemma \ref{lemma-points-presentation}.

\medskip\noindent
Let us first prove that $|p| : |U| \to |X|$ is open (of course it is
already continuous). Namely, if $V \subset U$ is an open subscheme,
then $V' = t(s^{-1}(V)) \subset U$ is an open subscheme as well
as the etale morphism $t$ is open (see
Morphisms, Lemma \ref{morphisms-lemma-fppf-open}).
Moreover, $|V'| = |p|^{-1}(W)$ with $W = |p|(V)$. In other words
$|p|(V)$ is open as desired.

\medskip\noindent
Next, let us prove that the topology is independent of the choice of
the presentation. To do this it suffices to show that if $U'$ is a scheme,
and $U' \to X$ is an etale morphism, then the map $|U'| \to |X|$
is open and continuous. In this case, let $U'' = U \times_X U'$, so that
we have the commutative diagram
$$
\xymatrix{
U'' \ar[r] \ar[d] & U' \ar[d] \\
U \ar[r] & X
}
$$
As $U \to X$ and $U' \to X$ are etale we see that
both $U'' \to U$ and $U'' \to U'$ are etale. Moreover, $U'' \to U'$
is surjective. Hence
we get a commutative diagram of maps of sets
$$
\xymatrix{
|U''| \ar[r] \ar[d] & |U'| \ar[d] \\
|U| \ar[r] & |X|
}
$$
By the above we know that:
the horizontal arrows are surjective, continuous and open,
the left vertical arrow is continuous and open.
It follows that also the right vertical arrow
is continuous and open as desired.

\medskip\noindent
Let $a : X \to Y$ be a morphism of algebraic spaces. According to
Spaces, Lemma \ref{spaces-lemma-lift-morphism-presentations}
we can find a diagram
$$
\xymatrix{
U \ar[d]_p \ar[r]_\alpha & V \ar[d]^q \\
X \ar[r]^a & Y
}
$$
where $U$ and $V$ are schemes, and $p$ and $q$ are surjective and etale.
This gives rise to the diagram
$$
\xymatrix{
|U| \ar[d]_p \ar[r]_\alpha & |V| \ar[d]^q \\
|X| \ar[r]^a & |Y|
}
$$
where all but the lower horizontal arrows are known to be continuous and
the two vertical arrows are surjective and open. It follows that the
lower horizontal arrow is continuous.

\medskip\noindent
This proves the first part of the lemma. Part (2) was proved in the
third paragraph of the proof.
\end{proof}

\begin{definition}
\label{definition-topological-space}
Let $S$ be a scheme. Let $X$ be an algebraic space over $S$.
The underlying {\it topological space} of $X$ is the set of points
$|X|$ endowed with the topology constructed in
Lemma \ref{lemma-topology-points}.
\end{definition}

\noindent
It turns out that this topological space carries the same information
as the small Zariski site $X_{Zar}$ of
Spaces, Definition \ref{spaces-definition-small-Zariski-site}.

\begin{lemma}
\label{lemma-open-subspaces}
Let $S$ be a scheme.
Let $X$ be an algebraic space over $S$.
\begin{enumerate}
\item The rule $X' \mapsto |X'|$ defines an inclusion preserving
bijection between open subspaces $X'$ (see
Spaces, Definition \ref{spaces-definition-immersion})
of $X$, and opens of the topological space $|X|$.
\item A family $\{X_i \subset X\}_{i \in I}$ of open subspaces of $X$
is a Zariski covering (see
Spaces, Definition \ref{spaces-definition-Zariski-open-covering})
if and only if $|X| = \bigcup |X_i|$.
\end{enumerate}
In other words, the small Zariski site $X_{Zar}$ of $X$ is canonically
identified with a site associated to the topological space $|X|$ (see
Sites, Example \ref{sites-example-site-topological}).
\end{lemma}

\begin{proof}
In order to prove (1) let us construct the inverse of the rule.
Namely, suppose that $W \subset |X|$ is open. Choose a presentation
$X = U/R$ corresponding to the surjective etale map
$p : U \to X$ and etale maps $s, t : R \to U$.
By construction we see that $|p|^{-1}(W)$ is an
open of $U$. Denote $W' \subset U$ the corresponding open subscheme.
It is clear that $R' = s^{-1}(W') = t^{-1}(W')$ is a Zariski open
of $R$ which defines an etale equivalence relation on $W'$.
By Spaces, Lemma \ref{spaces-lemma-finding-opens} the morphism
$X' = W'/R' \to X$ is an open immersion. Hence $X'$ is an algebraic space
by Spaces, Lemma \ref{spaces-lemma-representable-over-space}.
By construction $|X'| = W$, i.e., $X'$ is a subspace of $X$
corresponding to $W$. Thus (1) is proved.

\medskip\noindent
To prove (2), note that if $\{X_i \subset X\}_{i \in I}$ is a collection
of open subspaces, then it is a Zariski covering if and only if the
$U = \bigcup U \times_X X_i$ is an open covering. This follows from
the definition of a Zariski covering and the fact that the morphism
$U \to X$ is surjective as a map of presheaves on $(\textit{Sch}/S)_{fppf}$.
On the other hand, we see that $|X| = \bigcup |X_i|$ if and only if
$U = \bigcup U \times_X X_i$ by Lemma \ref{lemma-points-presentation}
(and the fact that the projections $U \times_X X_i \to X_i$ are surjective
and etale). Thus the equivalence of (2) follows.
\end{proof}

\begin{lemma}
\label{lemma-point-like-spaces}
Let $S$ be a scheme. Let $k$ be a field.
Let $X$ be an algebraic space over $S$ and assume that there exists
a surjective etale morphism $\text{Spec}(k) \to X$.
If $X$ is quasi-separated, then $X \cong \text{Spec}(k')$
where $k' \subset k$ is a finite separable extension.
\end{lemma}

\begin{proof}
Set $R = \text{Spec}(k) \times_X \text{Spec}(k)$, so that we have a
fibre product diagram
$$
\xymatrix{
R \ar[r]_-s \ar[d]_-t & \text{Spec}(k) \ar[d] \\
\text{Spec}(k) \ar[r] & X
}
$$
By Spaces, Lemma \ref{spaces-lemma-space-presentation}
we know $X = \text{Spec}(k)/R$ is the quotient sheaf.
Because $\text{Spec}(k) \to X$ is etale we see that
$R = \coprod_{i \in I} \text{Spec}(k_i)$ is a disjoint
union of spectra of fields, and both $s$ and $t$
induce finite separable field extensions $s, t : k \subset k_i$,
see Morphisms, Lemma \ref{morphisms-lemma-etale-over-field}. Because
$$
R = \text{Spec}(k) \times_X \text{Spec}(k)
= (\text{Spec}(k) \times_S \text{Spec}(k)) \times_{X \times X, \Delta} X
$$
and since $\Delta$ is quasi-compact by assumption we conclude that
$R \to \text{Spec}(k) \times_S \text{Spec}(k)$ is quasi-compact.
Hence $R$ is quasi-compact, and we see that $I$ is finite. This implies
that $s$ and $t$ are finite locally free morphisms. Hence by
Groupoids, Proposition \ref{groupoids-proposition-finite-flat-equivalence}
we conclude that $\text{Spec}(k)/R$ is
represented by $\text{Spec}(k')$, with $k' \subset k$ finite locally free
where
$$
k' = \{x \in k \mid s_i(x) = t_i(x)\text{ for all }i \in I\}
$$
It is easy to see that $k'$ is a field.
\end{proof}

\begin{remark}
\label{remark-cannot-decide-yet}
It is possible that Lemma \ref{lemma-point-like-spaces}
also holds when we only assume that
$X$ is weakly locally separated (and if so maybe we should rename
``weakly locally separated'' to ``locally separated'').
To prove this one would have to show that the index set $I$ in the proof of
Lemma \ref{lemma-point-like-spaces} is
finite, if we only assume that $R \to \text{Spec}(k) \times_S \text{Spec}(k)$
is an immersion (and an etale equivalence relation of course).
\end{remark}

\begin{lemma}
\label{lemma-points-monomorphism}
Let $S$ be a scheme. Let $X$ be an algebraic space over $S$.
Consider the map
$$
\{\text{Spec}(k) \to X \text{ monomorphism}\}
\longrightarrow
|X|
$$
This map is always injective, and if $X$ is Zariski locally
quasi-separated, then this map is a bijection.
\end{lemma}

\begin{proof}
Suppose that $\varphi_i : \text{Spec}(k_i) \to X$ are monomorphisms
for $i = 1, 2$. If $\varphi_1$ and $\varphi_2$ define the same point
of $|X|$, then we see that the scheme
$$
Y = \text{Spec}(k_1) \times_{\varphi_1, X, \varphi_2} \text{Spec}(k_2)
$$
is nonempty. Since the base change of a monomorphism is a monomorphism
this means that the projection morphisms $Y \to \text{Spec}(k_i)$
are monomorphisms. Hence $\text{Spec}(k_1) = Y = \text{Spec}(k_2)$
as schemes over $X$, see
Schemes, Lemma \ref{schemes-lemma-mono-towards-spec-field}.
We conclude that $\varphi_1 = \varphi_2$, which proves the first statement
of the lemma.

\medskip\noindent
Assume that $X$ is Zariski locally quasi-separated.
Then there exists a Zariski open covering $X = \bigcup X_i$
such that each $X_i$ is quasi-separated. Note that this means
$|X| = \bigcup |X_i|$ by Lemma \ref{lemma-open-subspaces}.
By definition each of the maps
$X_i \to X$ is a monomorphism. Hence to prove the surjectivity
of the map for $X$ it sufffices to prove the surjectivity of the
map for each $X_i$. This reduces the last statement to the case
where $X$ is quasi-separated.

\medskip\noindent
Assume $X$ is quasi-separated. Pick $x \in |X|$. We have to show that
$x$ is in the image of the map displayed in the lemma.
Let $U \to X$ be an etale surjective map, with $U$ a scheme.
Set $R = U \times_X U$. We have seen that $|U| \to |X|$ is surjective,
see Lemma \ref{lemma-points-presentation}.
Let $u \in U$ be a point, and write $k =\kappa(u)$
for its residue field. Consider the restriction $R'$
of the equivalence relation $R \to U \times_S U$ via the canonical map
$g : \text{Spec}(k) \to U$.
Here is the diagram (see
Groupoids, Lemma \ref{groupoids-lemma-restrict-groupoid}):
$$
\xymatrix{
R' \ar[d] \ar[r] \ar@/_4pc/[dd]_{t'} \ar@/^1pc/[rr]^{s'} &
R \times_{s, U} \text{Spec}(k) \ar[r] \ar[d] &
\text{Spec}(k) \ar[d]^g \\
\text{Spec}(k) \times_{U, t} R \ar[d] \ar[r] &
R \ar[r]^s \ar[d]_t &
U \\
\text{Spec}(k) \ar[r]^g &
U
}
$$
Since $s$ and $t$ are etale we see that
$R \times_{s, U} \text{Spec}(k)$ and
$\text{Spec}(k) \times_{U,t} R$ are disjoint unions of spectra
of finite separable extensions of $k$, see
Morphisms, Lemma \ref{morphisms-lemma-etale-over-field}.
Since $g$ is a monomorphism (see
Schemes, Lemma \ref{schemes-lemma-injective-points-surjective-stalks})
we see that the morphisms $R' \to R \times_{s, U} \text{Spec}(k)$
and $R' \to \text{Spec}(k) \times_{U,t} R$ are monomorphisms
also (see Schemes, Lemma \ref{schemes-lemma-base-change-monomorphism}).
Hence by Schemes, Lemma \ref{schemes-lemma-mono-towards-spec-field}
$R'$ is also a disjoint union of spectra of fields finite separable
over $k$ (in two ways),
and we conclude that $s', t' : R' \to \text{Spec}(k)$ are etale!
By the proof of
Groupoids, Lemma \ref{groupoids-lemma-restrict-groupoid-relation}
we also have that $R' \to \text{Spec}(k) \times_S \text{Spec}(k)$ is a base
change of $R \to U \times_S U$, namely by the morphism
$\text{Spec}(k) \times_S \text{Spec}(k) \to U \times_S U$.
Hence since $R \to U \times_S U$ is quasi-compact, also
$R' \to \text{Spec}(k) \times_S \text{Spec}(k)$ is quasi-compact.
By Spaces, Theorem \ref{spaces-theorem-presentation}
we know that the quotient sheaf
$X' = \text{Spec}(k)/R$ is an algebraic space, and by
Groupoids, Lemma \ref{groupoids-lemma-quotient-groupoid-restrict}
$X' \to X$ is a monomorphism.
Finally, by Lemma \ref{lemma-point-like-spaces}
we see that $X' = \text{Spec}(k')$. Hence we get a commutative diagram
$$
\xymatrix{
\text{Spec}(k) \ar[r]_u \ar[d] & U \ar[d] \\
\text{Spec}(k') \ar[r] & X
}
$$
which shows that $\text{Spec}(k')$ is a monomorphism mapping
to $x \in |X|$.
\end{proof}




































\section{Other chapters}

\begin{multicols}{2}
\begin{enumerate}
\item \hyperref[introduction-section-phantom]{Introduction}
\item \hyperref[conventions-section-phantom]{Conventions}
\item \hyperref[sets-section-phantom]{Set Theory}
\item \hyperref[categories-section-phantom]{Categories}
\item \hyperref[topology-section-phantom]{Topology}
\item \hyperref[sheaves-section-phantom]{Sheaves on Spaces}
\item \hyperref[algebra-section-phantom]{Commutative Algebra}
\item \hyperref[sites-section-phantom]{Sites and Sheaves}
\item \hyperref[homology-section-phantom]{Homological Algebra}
\item \hyperref[derived-section-phantom]{Derived Categories}
\item \hyperref[more-algebra-section-phantom]{More Algebra}
\item \hyperref[simplicial-section-phantom]{Simplicial Methods}
\item \hyperref[modules-section-phantom]{Sheaves of Modules}
\item \hyperref[sites-modules-section-phantom]{Modules on Sites}
\item \hyperref[injectives-section-phantom]{Injectives}
\item \hyperref[cohomology-section-phantom]{Cohomology of Sheaves}
\item \hyperref[sites-cohomology-section-phantom]{Cohomology on Sites}
\item \hyperref[hypercovering-section-phantom]{Hypercoverings}
\item \hyperref[schemes-section-phantom]{Schemes}
\item \hyperref[constructions-section-phantom]{Constructions of Schemes}
\item \hyperref[properties-section-phantom]{Properties of Schemes}
\item \hyperref[morphisms-section-phantom]{Morphisms of Schemes}
\item \hyperref[coherent-section-phantom]{Coherent Cohomology}
\item \hyperref[divisors-section-phantom]{Divisors}
\item \hyperref[limits-section-phantom]{Limits of Schemes}
\item \hyperref[varieties-section-phantom]{Varieties}
\item \hyperref[chow-section-phantom]{Chow Homology}
\item \hyperref[topologies-section-phantom]{Topologies on Schemes}
\item \hyperref[descent-section-phantom]{Descent}
\item \hyperref[more-morphisms-section-phantom]{More on Morphisms}
\item \hyperref[flat-section-phantom]{More on Flatness}
\item \hyperref[groupoids-section-phantom]{Groupoid Schemes}
\item \hyperref[more-groupoids-section-phantom]{More on Groupoid Schemes}
\item \hyperref[etale-section-phantom]{\'Etale Morphisms of Schemes}
\item \hyperref[etale-cohomology-section-phantom]{\'Etale Cohomology}
\item \hyperref[spaces-section-phantom]{Algebraic Spaces}
\item \hyperref[spaces-properties-section-phantom]{Properties of Algebraic Spaces}
\item \hyperref[spaces-morphisms-section-phantom]{Morphisms of Algebraic Spaces}
\item \hyperref[spaces-topologies-section-phantom]{Topologies on Algebraic Spaces}
\item \hyperref[spaces-descent-section-phantom]{Descent and Algebraic Spaces}
\item \hyperref[spaces-more-morphisms-section-phantom]{More on Morphisms of Spaces}
\item \hyperref[quot-section-phantom]{Quot and Hilbert Spaces}
\item \hyperref[stacks-section-phantom]{Stacks}
\item \hyperref[spaces-groupoids-section-phantom]{Groupoids in Algebraic Spaces}
\item \hyperref[spaces-more-groupoids-section-phantom]{More on Groupoids in Spaces}
\item \hyperref[bootstrap-section-phantom]{Bootstrap}
\item \hyperref[examples-stacks-section-phantom]{Examples of Stacks}
\item \hyperref[groupoids-quotients-section-phantom]{Quotients of Groupoids}
\item \hyperref[algebraic-section-phantom]{Algebraic Stacks}
\item \hyperref[criteria-section-phantom]{Criteria for Representability}
\item \hyperref[stacks-properties-section-phantom]{Properties of Algebraic Stacks}
\item \hyperref[stacks-morphisms-section-phantom]{Morphisms of Algebraic Stacks}
\item \hyperref[examples-section-phantom]{Examples}
\item \hyperref[exercises-section-phantom]{Exercises}
\item \hyperref[guide-section-phantom]{Guide to Literature}
\item \hyperref[desirables-section-phantom]{Desirables}
\item \hyperref[coding-section-phantom]{Coding Style}
\item \hyperref[fdl-section-phantom]{GNU Free Documentation License}
\item \hyperref[index-section-phantom]{Auto Generated Index}
\end{enumerate}
\end{multicols}


\bibliography{my}
\bibliographystyle{amsalpha}

\end{document}
