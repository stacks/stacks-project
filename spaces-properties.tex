\IfFileExists{stacks-project.cls}{%
\documentclass{stacks-project}
}{%
\documentclass{amsart}
}

% The following AMS packages are automatically loaded with
% the amsart documentclass:
%\usepackage{amsmath}
%\usepackage{amssymb}
%\usepackage{amsthm}

% For dealing with references we use the comment environment
\usepackage{verbatim}
\newenvironment{reference}{\comment}{\endcomment}
%\newenvironment{reference}{}{}
\newenvironment{slogan}{\comment}{\endcomment}
\newenvironment{history}{\comment}{\endcomment}

% For commutative diagrams you can use
% \usepackage{amscd}
\usepackage[all]{xy}

% We use 2cell for 2-commutative diagrams.
\xyoption{2cell}
\UseAllTwocells

% To put source file link in headers.
% Change "template.tex" to "this_filename.tex"
% \usepackage{fancyhdr}
% \pagestyle{fancy}
% \lhead{}
% \chead{}
% \rhead{Source file: \url{template.tex}}
% \lfoot{}
% \cfoot{\thepage}
% \rfoot{}
% \renewcommand{\headrulewidth}{0pt}
% \renewcommand{\footrulewidth}{0pt}
% \renewcommand{\headheight}{12pt}

\usepackage{multicol}

% For cross-file-references
\usepackage{xr-hyper}

% Package for hypertext links:
\usepackage{hyperref}

% For any local file, say "hello.tex" you want to link to please
% use \externaldocument[hello-]{hello}
\externaldocument[introduction-]{introduction}
\externaldocument[conventions-]{conventions}
\externaldocument[sets-]{sets}
\externaldocument[categories-]{categories}
\externaldocument[topology-]{topology}
\externaldocument[sheaves-]{sheaves}
\externaldocument[sites-]{sites}
\externaldocument[stacks-]{stacks}
\externaldocument[fields-]{fields}
\externaldocument[algebra-]{algebra}
\externaldocument[brauer-]{brauer}
\externaldocument[homology-]{homology}
\externaldocument[derived-]{derived}
\externaldocument[simplicial-]{simplicial}
\externaldocument[more-algebra-]{more-algebra}
\externaldocument[smoothing-]{smoothing}
\externaldocument[modules-]{modules}
\externaldocument[sites-modules-]{sites-modules}
\externaldocument[injectives-]{injectives}
\externaldocument[cohomology-]{cohomology}
\externaldocument[sites-cohomology-]{sites-cohomology}
\externaldocument[dga-]{dga}
\externaldocument[dpa-]{dpa}
\externaldocument[hypercovering-]{hypercovering}
\externaldocument[schemes-]{schemes}
\externaldocument[constructions-]{constructions}
\externaldocument[properties-]{properties}
\externaldocument[morphisms-]{morphisms}
\externaldocument[coherent-]{coherent}
\externaldocument[divisors-]{divisors}
\externaldocument[limits-]{limits}
\externaldocument[varieties-]{varieties}
\externaldocument[topologies-]{topologies}
\externaldocument[descent-]{descent}
\externaldocument[perfect-]{perfect}
\externaldocument[more-morphisms-]{more-morphisms}
\externaldocument[flat-]{flat}
\externaldocument[groupoids-]{groupoids}
\externaldocument[more-groupoids-]{more-groupoids}
\externaldocument[etale-]{etale}
\externaldocument[chow-]{chow}
\externaldocument[intersection-]{intersection}
\externaldocument[pic-]{pic}
\externaldocument[adequate-]{adequate}
\externaldocument[dualizing-]{dualizing}
\externaldocument[duality-]{duality}
\externaldocument[discriminant-]{discriminant}
\externaldocument[local-cohomology-]{local-cohomology}
\externaldocument[curves-]{curves}
\externaldocument[resolve-]{resolve}
\externaldocument[models-]{models}
\externaldocument[pione-]{pione}
\externaldocument[etale-cohomology-]{etale-cohomology}
\externaldocument[proetale-]{proetale}
\externaldocument[crystalline-]{crystalline}
\externaldocument[spaces-]{spaces}
\externaldocument[spaces-properties-]{spaces-properties}
\externaldocument[spaces-morphisms-]{spaces-morphisms}
\externaldocument[decent-spaces-]{decent-spaces}
\externaldocument[spaces-cohomology-]{spaces-cohomology}
\externaldocument[spaces-limits-]{spaces-limits}
\externaldocument[spaces-divisors-]{spaces-divisors}
\externaldocument[spaces-over-fields-]{spaces-over-fields}
\externaldocument[spaces-topologies-]{spaces-topologies}
\externaldocument[spaces-descent-]{spaces-descent}
\externaldocument[spaces-perfect-]{spaces-perfect}
\externaldocument[spaces-more-morphisms-]{spaces-more-morphisms}
\externaldocument[spaces-flat-]{spaces-flat}
\externaldocument[spaces-groupoids-]{spaces-groupoids}
\externaldocument[spaces-more-groupoids-]{spaces-more-groupoids}
\externaldocument[bootstrap-]{bootstrap}
\externaldocument[spaces-pushouts-]{spaces-pushouts}
\externaldocument[groupoids-quotients-]{groupoids-quotients}
\externaldocument[spaces-more-cohomology-]{spaces-more-cohomology}
\externaldocument[spaces-simplicial-]{spaces-simplicial}
\externaldocument[formal-spaces-]{formal-spaces}
\externaldocument[restricted-]{restricted}
\externaldocument[spaces-resolve-]{spaces-resolve}
\externaldocument[formal-defos-]{formal-defos}
\externaldocument[defos-]{defos}
\externaldocument[cotangent-]{cotangent}
\externaldocument[examples-defos-]{examples-defos}
\externaldocument[algebraic-]{algebraic}
\externaldocument[examples-stacks-]{examples-stacks}
\externaldocument[stacks-sheaves-]{stacks-sheaves}
\externaldocument[criteria-]{criteria}
\externaldocument[artin-]{artin}
\externaldocument[quot-]{quot}
\externaldocument[stacks-properties-]{stacks-properties}
\externaldocument[stacks-morphisms-]{stacks-morphisms}
\externaldocument[stacks-limits-]{stacks-limits}
\externaldocument[stacks-cohomology-]{stacks-cohomology}
\externaldocument[stacks-perfect-]{stacks-perfect}
\externaldocument[stacks-introduction-]{stacks-introduction}
\externaldocument[stacks-more-morphisms-]{stacks-more-morphisms}
\externaldocument[stacks-geometry-]{stacks-geometry}
\externaldocument[moduli-]{moduli}
\externaldocument[moduli-curves-]{moduli-curves}
\externaldocument[examples-]{examples}
\externaldocument[exercises-]{exercises}
\externaldocument[guide-]{guide}
\externaldocument[desirables-]{desirables}
\externaldocument[coding-]{coding}
\externaldocument[obsolete-]{obsolete}
\externaldocument[fdl-]{fdl}
\externaldocument[index-]{index}

% Theorem environments.
%
\theoremstyle{plain}
\newtheorem{theorem}[subsection]{Theorem}
\newtheorem{proposition}[subsection]{Proposition}
\newtheorem{lemma}[subsection]{Lemma}

\theoremstyle{definition}
\newtheorem{definition}[subsection]{Definition}
\newtheorem{example}[subsection]{Example}
\newtheorem{exercise}[subsection]{Exercise}
\newtheorem{situation}[subsection]{Situation}

\theoremstyle{remark}
\newtheorem{remark}[subsection]{Remark}
\newtheorem{remarks}[subsection]{Remarks}

\numberwithin{equation}{subsection}

% Macros
%
\def\lim{\mathop{\rm lim}\nolimits}
\def\colim{\mathop{\rm colim}\nolimits}
\def\Spec{\mathop{\rm Spec}}
\def\Hom{\mathop{\rm Hom}\nolimits}
\def\Ext{\mathop{\rm Ext}\nolimits}
\def\SheafHom{\mathop{\mathcal{H}\!{\it om}}\nolimits}
\def\SheafExt{\mathop{\mathcal{E}\!{\it xt}}\nolimits}
\def\Sch{\textit{Sch}}
\def\Mor{\mathop{\rm Mor}\nolimits}
\def\Ob{\mathop{\rm Ob}\nolimits}
\def\Sh{\mathop{\textit{Sh}}\nolimits}
\def\NL{\mathop{N\!L}\nolimits}
\def\proetale{{pro\text{-}\acute{e}tale}}
\def\etale{{\acute{e}tale}}
\def\QCoh{\textit{QCoh}}
\def\Ker{\mathop{\rm Ker}}
\def\Im{\mathop{\rm Im}}
\def\Coker{\mathop{\rm Coker}}
\def\Coim{\mathop{\rm Coim}}

%
% Macros for moduli stacks/spaces
%
\def\QCohstack{\mathcal{QC}\!{\it oh}}
\def\Cohstack{\mathcal{C}\!{\it oh}}
\def\Spacesstack{\mathcal{S}\!{\it paces}}
\def\Quotfunctor{{\rm Quot}}
\def\Hilbfunctor{{\rm Hilb}}
\def\Curvesstack{\mathcal{C}\!{\it urves}}
\def\Polarizedstack{\mathcal{P}\!{\it olarized}}
\def\Complexesstack{\mathcal{C}\!{\it omplexes}}
% \Pic is the operator that assigns to X its picard group, usage \Pic(X)
% \Picardstack_{X/B} denotes the Picard stack of X over B
% \Picardfunctor_{X/B} denotes the Picard functor of X over B
\def\Pic{\mathop{\rm Pic}\nolimits}
\def\Picardstack{\mathcal{P}\!{\it ic}}
\def\Picardfunctor{{\rm Pic}}
\def\Deformationcategory{\mathcal{D}\!{\it ef}}


% OK, start here.
%
\begin{document}

\title{Properties of Algebraic Spaces}


\maketitle

\phantomsection
\label{section-phantom}

\tableofcontents

\section{Introduction}
\label{section-introduction}

\noindent
Please see Spaces, Section \ref{spaces-section-introduction}
for a brief introduction to algebraic spaces, and please read
some of that chapter for our basic definitions and conventions
concerning algebraic spaces. In this chapter we start introducing
some basic notions and properties of algebraic spaces. A fundamental
reference for the case of quasi-separated algebraic spaces is
\cite{Kn}.

\medskip\noindent
The discussion is somewhat awkward at times since we made the design
descision to first talk about properties of algebraic spaces by
themselves, and only later about properties of morphisms of algebraic
spaces.



\section{Conventions}
\label{section-conventions}

\noindent
The standing assumption is that all schemes are contained in
a big fppf site $\textit{Sch}_{fppf}$. And all rings $A$ considered
have the property that $\text{Spec}(A)$ is (isomorphic) to an
object of this big site.


\section{Separation axioms}
\label{section-separation}

\noindent
In this section we collect all the ``absolute'' separation conditions
of algebraic spaces. Here is a list.

\begin{definition}
\label{definition-separated}
(See Spaces, Definition \ref{spaces-definition-separated}.)
Let $S$ be a scheme contained in $\textit{Sch}_{fppf}$.
Let $F$ be an algebraic space over $S$.
Let $\Delta : F \to F \times F$ be the diagonal morphism.
\begin{enumerate}
\item We say $F$ is {\it separated} if $\Delta$ is a closed immersion.
\item We say $F$ is {\it weakly locally separated}\footnote{This is probably
nonstandard notation.} if $\Delta$ is an
immersion.
\item We say $F$ is {\it locally separated} if $\Delta$ is a
quasi-compact immersion.
\item We say $F$ is {\it quasi-separated} if $\Delta$ is quasi-compact.
\item We say $F$ is {\it Zariski locally quasi-separated}\footnote{
This notion was suggested by B.\ Conrad.} if there
exists a Zariski covering $F = \bigcup_{i \in I} F_i$ (see Spaces,
Definition \ref{spaces-definition-Zariski-open-covering}) such that
each $F_i$ is quasi-separated.
\end{enumerate}
\end{definition}

\begin{lemma}
\label{lemma-trivial-implications}
Let $S$ be a scheme.
Let $X$ be an algebraic space over $S$.
We have the following implications among the separation axioms
of Definition \ref{definition-separated}:
\begin{enumerate}
\item separated implies all the others,
\item locally separated implies quasi-separated and weakly locally separated,
\item quasi-separated implies Zariski locally quasi-separated.
\end{enumerate}
\end{lemma}

\begin{proof}
Omitted.
\end{proof}














\section{Points of algebraic spaces}
\label{section-points}

\noindent
As is clear from Spaces, Example \ref{spaces-example-affine-line-translation}
a point of an algebraic stack should not be defined as a monomorphism
from the spectrum of a field.
Instead we define them as equivalence classes of morphisms of specra
of fields as equivalence classes of morphisms from spectra of fields.
exactly as explained in Schemes, Section \ref{schemes-section-points}.

\medskip\noindent
Let $S$ be a scheme.
Let $F$ be a presheaf on $(\textit{Sch}/S)_{fppf}$.
Let $K$ is a field. Consider a morphism
$$
\text{Spec}(K) \longrightarrow F.
$$
By the Yoneda Lemma this is given by an
element $p \in F(\text{Spec}(K))$. We say that two such
pairs $(\text{Spec}(K), p)$ and $(\text{Spec}(L), q)$
are {\it equivalent} if there exists
a third field $\Omega$ and a commutative diagram
$$
\xymatrix{
\text{Spec}(\Omega) \ar[r] \ar[d] &
\text{Spec}(L) \ar[d]^q \\
\text{Spec}(K) \ar[r]^p &
F.
}
$$
In other words, there are field extensions
$K \to \Omega$ and $L \to \Omega$ such that
$p$ and $q$ map to the same element
of $F(\text{Spec}(\Omega))$. We omit the verification that this
defines an equivalence relation.

\begin{definition}
\label{definition-points}
Let $S$ be a scheme. Let $X$ be an algebraic space over $S$.
A {\it point} of $X$ is an equivalence class of morphisms
from spectra of fields into $X$.
The set of points of $X$ is denoted $|X|$.
\end{definition}

\noindent
Note that if $f : X \to Y$ is a morphism of algebraic spaces
over $S$, then there is an induced map $|f| : |X| \to |Y|$ which
maps a representative $x : \text{Spec}(K) \to X$ to the representative
$f \circ x : \text{Spec}(K) \to Y$.

\begin{lemma}
\label{lemma-scheme-points}
Let $S$ be a scheme. Let $X$ be a scheme over $S$.
The points of $X$ as a scheme are in canonical 1-1 correspondence
with the points of $X$ as an algebraic space.
\end{lemma}

\begin{proof}
This is Schemes, Lemma \ref{schemes-lemma-characterize-points}.
\end{proof}

\begin{lemma}
\label{lemma-points-cartesian}
Let $S$ be a scheme. Let
$$
\xymatrix{
Z \times_Y X \ar[r] \ar[d] & X \ar[d] \\
Z \ar[r] & Y
}
$$
be a cartesian diagram of algebraic spaces. Then the map of sets
of points
$$
|Z \times_Y X|
\longrightarrow
|Z| \times_{|Y|} |X|
$$
is surjective.
\end{lemma}

\begin{proof}
Namely, suppose given fields $K$, $L$ and morphisms
$\text{Spec}(K) \to X$, $\text{Spec}(L) \to Y$, then the
assumption that they agree as elements of $|Y|$ means that
there is a common extension $K \subset M$ and $L \subset M$
such that
$\text{Spec}(M) \to \text{Spec}(K) \to X \to Y$ and
$\text{Spec}(M) \to \text{Spec}(L) \to Z \to Y$ agree.
And this is exactly the condition which says you get a
morphism $\text{Spec}(M) \to Z \times_Y X$.
\end{proof}

\begin{lemma}
\label{lemma-points-presentation}
Let $S$ be a scheme.
Let $X$ be an algebraic space over $S$.
Let $X = U/R$ be a presentation of $X$, see
Spaces, Definition \ref{spaces-definition-presentation}.
Then $|R| \to |U| \times |U|$ is an equivalence relation
and $|X| = |U|/|R|$.
\end{lemma}

\begin{proof}
We will use Lemmas \ref{lemma-scheme-points} and \ref{lemma-points-cartesian}
without further mention.
The assumption means that $U$ is a scheme, $p : U \to X$ is a
surjective, etale morphism, $R = U \times_X U$ is a scheme
and defines an etale equivalence relation on $U$ such that
$X = U/R$ as sheaves. Let $x : \text{Spec}(K) \to X$ be a morphism
from the spectrum of a field into $X$. By assumption the scheme
$\text{Spec}(K) \times_X U$ is nonempty. Hence there exists a
field extension $K \subset K'$ and a morphism $\text{Spec}(K') \to U$
such that the diagram
$$
\xymatrix{
\text{Spec}(K') \ar[r] \ar[d] & U \ar[d]^p \\
\text{Spec}(K) \ar[r]^-x & X
}
$$
commutes. Hence we see that $|p| : |U| \to |X|$ is surjective.

\medskip\noindent
Since $j = (t, s) : R \to U \times_S U$ is a monomorphism we see that
$|R| \to |U| \times |U|$ is injective. Moreover, it is an equivalence
relation since $R \to U \times_S U$ is an equivalence relation.
Namely, if $u_i \in |U|$, $i = 1, 2, 3$ and $(u_1, u_2)$ is
the image of $r \in |R|$, and $(u_2, u_3)$ is the image of
$r' \in |R|$, then we see that there exists some point of
$R \times_{s, U, t} R$ contains a point $\tilde r$ mapping to
$(r, r')$. Set $r'' = \text{pr}_{02}(\tilde r) \in |R|$. Then we
see that $\tilde r$ maps to $(u_1, u_2)$ in $|U| \times |U|$.
Also, since $U \to X$ equalizes $s, t : R \to U$ we see that
the map $|U| \to |X|$ factors through $|U|/|R|$. Finally, we have
to show that if $u_1, u_2 \in |U|$ map to the same point of $X$
then they come from an element of $|R|$.
This again follows from Lemma \ref{lemma-points-cartesian}.
\end{proof}

\begin{lemma}
\label{lemma-topology-points}
Let $S$ be a scheme. There exists a unique topology on the set of points
of algebraic spaces over $S$ with the following properties:
\begin{enumerate}
\item for every morphism of algebraic spaces $X \to Y$ over $S$
the map $|X| \to |Y|$ is continuous, and
\item for every etale morphism $U \to X$ with $U$ a scheme
the map of topological spaces $|U| \to |X|$ is continuous and open.
\end{enumerate}
\end{lemma}

\begin{proof}
Let $X$ be an algebraic space over $S$. Let $X = U/R$ be a presentation,
where $j = (t, s) : R \to U$ is an etale equivalence relation.
Let $p : U \to X$ denote the etale surjective morphism.
We define $W \subset |X|$ is open if and only if $|p|^{-1}(W)$
is an open subset of $|U|$. This is a topology on $|X|$ by
the result of Lemma \ref{lemma-points-presentation}.

\medskip\noindent
Let us first prove that $|p| : |U| \to |X|$ is open (of course it is
already continuous). Namely, if $V \subset U$ is an open subscheme,
then $V' = t(s^{-1}(V)) \subset U$ is an open subscheme as well
as the etale morphism $t$ is open (see
Morphisms, Lemma \ref{morphisms-lemma-fppf-open}).
Moreover, $|V'| = |p|^{-1}(W)$ with $W = |p|(V)$. In other words
$|p|(V)$ is open as desired.

\medskip\noindent
Next, let us prove that the topology is independent of the choice of
the presentation. To do this it suffices to show that if $U'$ is a scheme,
and $U' \to X$ is an etale morphism, then the map $|U'| \to |X|$
is open and continuous. In this case, let $U'' = U \times_X U'$, so that
we have the commutative diagram
$$
\xymatrix{
U'' \ar[r] \ar[d] & U' \ar[d] \\
U \ar[r] & X
}
$$
As $U \to X$ and $U' \to X$ are etale we see that
both $U'' \to U$ and $U'' \to U'$ are etale. Moreover, $U'' \to U'$
is surjective. Hence
we get a commutative diagram of maps of sets
$$
\xymatrix{
|U''| \ar[r] \ar[d] & |U'| \ar[d] \\
|U| \ar[r] & |X|
}
$$
By the above we know that:
the horizontal arrows are surjective, continuous and open,
the left vertical arrow is continuous and open.
It follows that also the right vertical arrow
is continuous and open as desired.

\medskip\noindent
Let $a : X \to Y$ be a morphism of algebraic spaces. According to
Spaces, Lemma \ref{spaces-lemma-lift-morphism-presentations}
we can find a diagram
$$
\xymatrix{
U \ar[d]_p \ar[r]_\alpha & V \ar[d]^q \\
X \ar[r]^a & Y
}
$$
where $U$ and $V$ are schemes, and $p$ and $q$ are surjective and etale.
This gives rise to the diagram
$$
\xymatrix{
|U| \ar[d]_p \ar[r]_\alpha & |V| \ar[d]^q \\
|X| \ar[r]^a & |Y|
}
$$
where all but the lower horizontal arrows are known to be continuous and
the two vertical arrows are surjective and open. It follows that the
lower horizontal arrow is continuous.

\medskip\noindent
This proves the first part of the lemma. Part (2) was proved in the
third paragraph of the proof.
\end{proof}

\begin{definition}
\label{definition-topological-space}
Let $S$ be a scheme. Let $X$ be an algebraic space over $S$.
The underlying {\it topological space} of $X$ is the set of points
$|X|$ endowed with the topology constructed in
Lemma \ref{lemma-topology-points}.
\end{definition}

\noindent
It turns out that this topological space carries the same information
as the small Zariski site $X_{Zar}$ of
Spaces, Definition \ref{spaces-definition-small-Zariski-site}.

\begin{lemma}
\label{lemma-open-subspaces}
Let $S$ be a scheme.
Let $X$ be an algebraic space over $S$.
\begin{enumerate}
\item The rule $X' \mapsto |X'|$ defines an inclusion preserving
bijection between open subspaces $X'$ (see
Spaces, Definition \ref{spaces-definition-immersion})
of $X$, and opens of the topological space $|X|$.
\item A family $\{X_i \subset X\}_{i \in I}$ of open subspaces of $X$
is a Zariski covering (see
Spaces, Definition \ref{spaces-definition-Zariski-open-covering})
if and only if $|X| = \bigcup |X_i|$.
\end{enumerate}
In other words, the small Zariski site $X_{Zar}$ of $X$ is canonically
identified with a site associated to the topological space $|X|$ (see
Sites, Example \ref{sites-example-site-topological}).
\end{lemma}

\begin{proof}
In order to prove (1) let us construct the inverse of the rule.
Namely, suppose that $W \subset |X|$ is open. Choose a presentation
$X = U/R$ corresponding to the surjective etale map
$p : U \to X$ and etale maps $s, t : R \to U$.
By construction we see that $|p|^{-1}(W)$ is an
open of $U$. Denote $W' \subset U$ the corresponding open subscheme.
It is clear that $R' = s^{-1}(W') = t^{-1}(W')$ is a Zariski open
of $R$ which defines an etale equivalence relation on $W'$.
By Spaces, Lemma \ref{spaces-lemma-finding-opens} the morphism
$X' = W'/R' \to X$ is an open immersion. Hence $X'$ is an algebraic space
by Spaces, Lemma \ref{spaces-lemma-representable-over-space}.
By construction $|X'| = W$, i.e., $X'$ is a subspace of $X$
corresponding to $W$. Thus (1) is proved.

\medskip\noindent
To prove (2), note that if $\{X_i \subset X\}_{i \in I}$ is a collection
of open subspaces, then it is a Zariski covering if and only if the
$U = \bigcup U \times_X X_i$ is an open covering. This follows from
the definition of a Zariski covering and the fact that the morphism
$U \to X$ is surjective as a map of presheaves on $(\textit{Sch}/S)_{fppf}$.
On the other hand, we see that $|X| = \bigcup |X_i|$ if and only if
$U = \bigcup U \times_X X_i$ by Lemma \ref{lemma-points-presentation}
(and the fact that the projections $U \times_X X_i \to X_i$ are surjective
and etale). Thus the equivalence of (2) follows.
\end{proof}

\begin{lemma}
\label{lemma-characterize-surjective}
Let $S$ be a scheme.
Let $X$ be an algebraic space over $S$.
Let $f : T \to X$ be a morphism from a scheme to $X$.
The following are equivalent
\begin{enumerate}
\item $f : T \to X$ is surjective (according to
Spaces, Definition \ref{spaces-definition-relative-representable-property}),
and
\item $|f| : |T| \to |X|$ is surjective.
\end{enumerate}
\end{lemma}

\begin{proof}
Assume (1). Let $U \to X$ be a surjective etale morphism, where $U$ is
a scheme. By assumption (1) we see that $T \times_X U \to U$ is a surjective
morphism of schemes. Hence for any $u \in |U|$ there exists a 
$\tilde t \in |T \times_X U|$ which maps to $u$. The image $t$ of $\tilde t$
in $|T|$ is a point of $|T|$ which maps to the image of $u$ in $|X|$.
Since $|X|$ is a quotient of $|U|$ we conclude that $|f|$ is surjective.

\medskip\noindent
Assume (2). Let $U \to X$ be a surjective etale morphism, where $U$ is
a scheme. Set $R = U \times_X U$. Note that $T \times_X U \to T$ is a
surjective etale morphism of schemes. Take $u \in |U|$. By assumption (2)
we can choose a point $t \in |T|$ which maps to the image of $u$ in $|X|$.
By Lemma \ref{lemma-points-cartesian} there exists a point
$\tilde t \in |T \times_X U|$ which maps to $t$ and $u$.
Hence $T \times_X U \to U$ is surjective (as a morphism of schemes,
see Lemma \ref{lemma-scheme-points}). By
Spaces, Lemma \ref{spaces-lemma-representable-morphisms-spaces-property}
this implies that $T \to X$ is surjective.
\end{proof}

\begin{lemma}
\label{lemma-point-like-spaces}
Let $S$ be a scheme. Let $k$ be a field.
Let $X$ be an algebraic space over $S$ and assume that there exists
a surjective etale morphism $\text{Spec}(k) \to X$.
If $X$ is quasi-separated, then $X \cong \text{Spec}(k')$
where $k' \subset k$ is a finite separable extension.
\end{lemma}

\begin{proof}
Set $R = \text{Spec}(k) \times_X \text{Spec}(k)$, so that we have a
fibre product diagram
$$
\xymatrix{
R \ar[r]_-s \ar[d]_-t & \text{Spec}(k) \ar[d] \\
\text{Spec}(k) \ar[r] & X
}
$$
By Spaces, Lemma \ref{spaces-lemma-space-presentation}
we know $X = \text{Spec}(k)/R$ is the quotient sheaf.
Because $\text{Spec}(k) \to X$ is etale we see that
$R = \coprod_{i \in I} \text{Spec}(k_i)$ is a disjoint
union of spectra of fields, and both $s$ and $t$
induce finite separable field extensions $s, t : k \subset k_i$,
see Morphisms, Lemma \ref{morphisms-lemma-etale-over-field}. Because
$$
R = \text{Spec}(k) \times_X \text{Spec}(k)
= (\text{Spec}(k) \times_S \text{Spec}(k)) \times_{X \times X, \Delta} X
$$
and since $\Delta$ is quasi-compact by assumption we conclude that
$R \to \text{Spec}(k) \times_S \text{Spec}(k)$ is quasi-compact.
Hence $R$ is quasi-compact, and we see that $I$ is finite. This implies
that $s$ and $t$ are finite locally free morphisms. Hence by
Groupoids, Proposition \ref{groupoids-proposition-finite-flat-equivalence}
we conclude that $\text{Spec}(k)/R$ is
represented by $\text{Spec}(k')$, with $k' \subset k$ finite locally free
where
$$
k' = \{x \in k \mid s_i(x) = t_i(x)\text{ for all }i \in I\}
$$
It is easy to see that $k'$ is a field.
\end{proof}

\begin{remark}
\label{remark-cannot-decide-yet}
It is possible that Lemma \ref{lemma-point-like-spaces}
also holds when we only assume that
$X$ is weakly locally separated (and if so maybe we should rename
``weakly locally separated'' to ``locally separated'').
To prove this one would have to show that the index set $I$ in the proof of
Lemma \ref{lemma-point-like-spaces} is
finite, if we only assume that $R \to \text{Spec}(k) \times_S \text{Spec}(k)$
is an immersion (and an etale equivalence relation of course).
\end{remark}

\begin{lemma}
\label{lemma-points-monomorphism}
Let $S$ be a scheme. Let $X$ be an algebraic space over $S$.
Consider the map
$$
\{\text{Spec}(k) \to X \text{ monomorphism}\}
\longrightarrow
|X|
$$
This map is always injective, and if $X$ is Zariski locally
quasi-separated, then this map is a bijection.
\end{lemma}

\begin{proof}
Suppose that $\varphi_i : \text{Spec}(k_i) \to X$ are monomorphisms
for $i = 1, 2$. If $\varphi_1$ and $\varphi_2$ define the same point
of $|X|$, then we see that the scheme
$$
Y = \text{Spec}(k_1) \times_{\varphi_1, X, \varphi_2} \text{Spec}(k_2)
$$
is nonempty. Since the base change of a monomorphism is a monomorphism
this means that the projection morphisms $Y \to \text{Spec}(k_i)$
are monomorphisms. Hence $\text{Spec}(k_1) = Y = \text{Spec}(k_2)$
as schemes over $X$, see
Schemes, Lemma \ref{schemes-lemma-mono-towards-spec-field}.
We conclude that $\varphi_1 = \varphi_2$, which proves the first statement
of the lemma.

\medskip\noindent
Assume that $X$ is Zariski locally quasi-separated.
Then there exists a Zariski open covering $X = \bigcup X_i$
such that each $X_i$ is quasi-separated. Note that this means
$|X| = \bigcup |X_i|$ by Lemma \ref{lemma-open-subspaces}.
By definition each of the maps
$X_i \to X$ is a monomorphism. Hence to prove the surjectivity
of the map for $X$ it sufffices to prove the surjectivity of the
map for each $X_i$. This reduces the last statement to the case
where $X$ is quasi-separated.

\medskip\noindent
Assume $X$ is quasi-separated. Pick $x \in |X|$. We have to show that
$x$ is in the image of the map displayed in the lemma.
Let $U \to X$ be an etale surjective map, with $U$ a scheme.
Set $R = U \times_X U$. We have seen that $|U| \to |X|$ is surjective,
see Lemma \ref{lemma-points-presentation}.
Let $u \in U$ be a point, and write $k =\kappa(u)$
for its residue field. Consider the restriction $R'$
of the equivalence relation $R \to U \times_S U$ via the canonical map
$g : \text{Spec}(k) \to U$.
Here is the diagram (see
Groupoids, Lemma \ref{groupoids-lemma-restrict-groupoid}):
$$
\xymatrix{
R' \ar[d] \ar[r] \ar@/_4pc/[dd]_{t'} \ar@/^1pc/[rr]^{s'} &
R \times_{s, U} \text{Spec}(k) \ar[r] \ar[d] &
\text{Spec}(k) \ar[d]^g \\
\text{Spec}(k) \times_{U, t} R \ar[d] \ar[r] &
R \ar[r]^s \ar[d]_t &
U \\
\text{Spec}(k) \ar[r]^g &
U
}
$$
Since $s$ and $t$ are etale we see that
$R \times_{s, U} \text{Spec}(k)$ and
$\text{Spec}(k) \times_{U,t} R$ are disjoint unions of spectra
of finite separable extensions of $k$, see
Morphisms, Lemma \ref{morphisms-lemma-etale-over-field}.
Since $g$ is a monomorphism (see
Schemes, Lemma \ref{schemes-lemma-injective-points-surjective-stalks})
we see that the morphisms $R' \to R \times_{s, U} \text{Spec}(k)$
and $R' \to \text{Spec}(k) \times_{U,t} R$ are monomorphisms
also (see Schemes, Lemma \ref{schemes-lemma-base-change-monomorphism}).
Hence by Schemes, Lemma \ref{schemes-lemma-mono-towards-spec-field}
$R'$ is also a disjoint union of spectra of fields finite separable
over $k$ (in two ways),
and we conclude that $s', t' : R' \to \text{Spec}(k)$ are etale!
By the proof of
Groupoids, Lemma \ref{groupoids-lemma-restrict-groupoid-relation}
we also have that $R' \to \text{Spec}(k) \times_S \text{Spec}(k)$ is a base
change of $R \to U \times_S U$, namely by the morphism
$\text{Spec}(k) \times_S \text{Spec}(k) \to U \times_S U$.
Hence since $R \to U \times_S U$ is quasi-compact, also
$R' \to \text{Spec}(k) \times_S \text{Spec}(k)$ is quasi-compact.
By Spaces, Theorem \ref{spaces-theorem-presentation}
we know that the quotient sheaf
$X' = \text{Spec}(k)/R$ is an algebraic space, and by
Groupoids, Lemma \ref{groupoids-lemma-quotient-groupoid-restrict}
$X' \to X$ is a monomorphism.
Finally, by Lemma \ref{lemma-point-like-spaces}
we see that $X' = \text{Spec}(k')$. Hence we get a commutative diagram
$$
\xymatrix{
\text{Spec}(k) \ar[r]_-u \ar[d] & U \ar[d] \\
\text{Spec}(k') \ar[r] & X
}
$$
which shows that $\text{Spec}(k')$ is a monomorphism mapping
to $x \in |X|$.
\end{proof}











\section{Quasi-compact spaces}
\label{section-quasi-compact}

\begin{definition}
\label{definition-quasi-compact}
Let $S$ be a scheme.
Let $X$ be an algebraic space over $S$.
We say $X$ is {\it quasi-compact} if there exists a surjective
etale morphism $U \to X$ with $U$ quasi-compact.
\end{definition}

\begin{lemma}
\label{lemma-quasi-compact-space}
Let $S$ be a scheme.
Let $X$ be an algebraic space over $S$.
Then $X$ is quasi-compact if and only if $|X|$ is quasi-compact.
\end{lemma}

\begin{proof}
Choose a scheme $U$ and an etale surjective morphism $U \to X$.
We will use Lemma \ref{lemma-characterize-surjective}.
If $U$ is quasi-compact, then since $|U| \to |X|$ is surjective
we conclude that $|X|$ is quasi-compact.
If $|X|$ is quasi-compact, then since $|U| \to |X|$ is open
we see that there exists a quasi-compact open $U' \subset U$
such that $|U'| \to |X|$ is surjective (and still etale).
Hence we win.
\end{proof}




\section{Properties of spaces defined by properties of schemes}
\label{section-types-properties}

\begin{definition}
\label{definition-type-property}
Let $\mathcal{P}$ be a property of schemes which is 
local in the etale topology, see
Descent, Definition \ref{descent-definition-property-local}.
Let $S$ be a scheme.
Let $X$ be an algebraic space over $S$.
We say an algebraic space $X$ {\it has property $\mathcal{P}$}
if there exists an etale surjective morphism $U \to X$ such
that $U$ has property $\mathcal{P}$.
\end{definition}

\begin{remark}
\label{remark-list-properties-local-etale-topology}
Here is a list of properties which are local for the etale topology
(keep in mind that the fpqc, fppf, syntomic, and smooth topologies are
stronger than the etale topology):
\begin{enumerate}
\item locally Noetherian, see
Descent, Lemma \ref{descent-lemma-Noetherian-local-fppf},
\item Jacobson, see
Descent, Lemma \ref{descent-lemma-Jacobson-local-fppf},
\item locally Noetherian and $(S_k)$, see
Descent, Lemma \ref{descent-lemma-Sk-local-syntomic},
\item Cohen-Macaulay, see
Descent, Lemma \ref{descent-lemma-CM-local-syntomic},
\item reduced, see
Descent, Lemma \ref{descent-lemma-reduced-local-smooth},
\item normal, see
Descent, Lemma \ref{descent-lemma-normal-local-smooth},
\item locally Noetherian and $(R_k)$, see
Descent, Lemma \ref{descent-lemma-Rk-local-smooth},
\item regular, see
Descent, Lemma \ref{descent-lemma-regular-local-smooth},
\item Nagata, see
Descent, Lemma \ref{descent-lemma-Nagata-local-smooth}.
\end{enumerate}
\end{remark}

\begin{lemma}
\label{lemma-type-property}
Let $\mathcal{P}$ be a property of schemes which is 
local in the etale topology.
Let $S$ be a scheme.
Let $X$ be an algebraic space over $S$.
The following are equivalent
\begin{enumerate}
\item $X$ has property $\mathcal{P}$,
\item for every scheme $U$ and every etale morphism $U \to X$
the scheme $U$ has property $\mathcal{P}$.
\end{enumerate}
\end{lemma}

\begin{proof}
Omitted.
\end{proof}






\section{Reduced spaces}
\label{section-reduced}

\noindent
We have already defined reduced algebraic spaces in
Section \ref{section-types-properties}.
Here we just prove some simple lemmas regarding reduced algebraic
spaces.

\begin{lemma}
\label{lemma-reduced-closed-subspace}
Let $S$ be a scheme.
Let $X$ be an algebraic space over $S$.
Let $T \subset |X|$ be a closed subset.
There exists a unique closed subspace $Z \subset X$ with
the following properties: (a) we have $|Z| = T$, and (b) $Z$ is reduced.
\end{lemma}

\begin{proof}
Let $U \to X$ be a surjective etale morphism, where $U$ is a scheme.
Set $R = U \times_X U$, so that $X = U/R$, see
Spaces, Lemma \ref{spaces-lemma-space-presentation}.
As usual we denote $s, t : R \to U$ the two projection morphisms.
By Lemma \ref{lemma-points-presentation}
we see that $T$ corresponds to a closed subset $T' \subset |U|$ such
that $s^{-1}(T') = t^{-1}(T')$.
Let $Z' \subset U$ be the reduced induced scheme structure on $T'$.
In this case the fibre products
$Z' \times_{U, t} R$ and $Z' \times_{U, s} R$ are closed subschemes
of $R$
(Schemes, Lemma \ref{schemes-lemma-base-change-immersion})
which are etale over $Z'$
(Morphisms, Lemma \ref{morphisms-lemma-base-change-etale}),
and hence reduced
(because being reduced is local in the etale topology, see
Remark \ref{remark-list-properties-local-etale-topology}).
Since they have the same underlying topological space (see above)
we conclude that $Z' \times_{U, t} R = Z' \times_{U, s} R$.
Hence the common value $R'$ is the restriction of $R$ to $Z'$, see
Groupoids, Definition \ref{groupoids-definition-restrict-groupoid}. By
Spaces, Theorem \ref{spaces-theorem-presentation} we see that
$Z = Z'/R'$ is an algebraic space. By
Groupoids, Lemma \ref{groupoids-lemma-quotient-groupoid-restrict}
we see that $Z \to X$ is a monomorphism. By construction we have
$U \times_X Z = Z'$, so $U \times_X Z \to Z$ is a closed immersion.
This means all the hypotheses of
Spaces,
Lemma \ref{spaces-lemma-morphism-sheaves-with-P-effective-descent-etale}
are satisfied
for the transformation $Z \to X$, $\mathcal{P}=$``closed immersion'' (closed
immersions satisfy descent for etale coverings, see
Descent, Lemma \ref{descent-lemma-closed-immersion}),
and the etale surjective morphism $U \to X$. We conclude that $Z \to X$
is representable, a monomorphism and a closed immersion, which is the
definition of a closed subspace (see
Spaces, Definition \ref{spaces-definition-immersion}). By construction
$|Z| = T$ and $Z$ is reduced. This proves existence. We omit the proof
of uniqueness.
\end{proof}






\section{Noetherian spaces}
\label{section-noetherian}

\begin{definition}
\label{definition-noetherian}
Let $S$ be a scheme. Let $X$ be an algebraic space over $S$.
We say $X$ is {\it Noetherian}
if $X$ is quasi-compact and locally Noetherian.
\end{definition}

\noindent
Note that if we do not assume that $X$ is quasi-separated, then
$X$ can still be very different from a scheme
even if it is Noetherian. For example $X = [\mathbf{A}^1_k/\mathbf{Z}]$,
as in Spaces, Example \ref{spaces-example-affine-line-translation} is
Noetherian according to the definition above.





\section{Etale morphisms of algebraic spaces}
\label{section-etale-morphisms}

\noindent
This section really belongs in the chapter on morphisms of algebraic
spaces, but we need the notion of an algebraic space etale over another
in order to define the small etale site of an algebraic space.
Thus we need to do some preliminary work on etale morphisms from schemes to
algebraic spaces, and etale morphisms between algebraic spaces.
For more about etale morphisms of algebraic spaces, see
(insert future reference here).

\begin{lemma}
\label{lemma-etale-over-space}
Let $S$ be a scheme.
Let $X$ be an algebraic space over $S$.
Let $U$, $U'$ be schemes over $S$.
\begin{enumerate}
\item If $U \to U'$ is an etale morphism of schemes, and
if $U' \to X$ is an etale morphism from $U'$ to $X$, then the
composition $U \to X$ is an etale morphism from $U$ to $X$.
\item If $\varphi : U \to X$ and $\varphi' : U' \to X$ are
etale morphisms towards $X$, and if $\chi : U \to U'$ is a
morphism of schemes such that $\varphi = \varphi' \circ \chi$,
then $\chi$ is an etale morphism of schemes.
\end{enumerate}
\end{lemma}

\begin{proof}
Recall that our definition of an etale morphism from a scheme into an
algebraic space comes from
Spaces, Definition \ref{spaces-definition-relative-representable-property}
via the fact that any morphism from a scheme into an algebraic space
is representable. Part (1) of the lemma follows from this, the fact that
etale morphisms are preserved under composition
(Morphisms, Lemma \ref{morphisms-lemma-composition-etale})
and
Spaces, Lemmas
\ref{spaces-lemma-morphism-schemes-gives-representable-transformation-property}
and
\ref{spaces-lemma-composition-representable-transformations-property}
(which are formal).
To prove part (2) choose a scheme $W$ over $S$ and a
surjective etale morphism $W \to X$. Consider the base change
$\chi_W : W \times_X U \to W \times_X U'$ of $\chi$.
As $W \times_X U$ and $W \times_X U'$ are etale over $W$, we conclude that
$\chi_W$ is etale, by
Morphisms, Lemma \ref{morphisms-lemma-etale-permanence-two}.
On the other hand, in the commutative diagram
$$
\xymatrix{
W \times_X U \ar[r] \ar[d] & W \times_X U' \ar[d] \\
U \ar[r] & U'
}
$$
the two vertical arrows are etale and surjective.
Hence by
Descent, Lemma \ref{descent-lemma-syntomic-smooth-etale-permanence}
we conclude that $U \to U'$ is etale.
\end{proof}

\begin{definition}
\label{definition-etale}
Let $S$ be a scheme.
A morphism $f : X \to Y$ between algebraic spaces over $S$ is
called {\it etale} if and only if for every etale morphism
$\varphi : U \to X$ where $U$ is a scheme, the composition
$\varphi \circ f$ is etale also.
\end{definition}

\noindent
If $X$ and $Y$ are schemes, then this agree with the usual notion of an
etale morphism of schemes. In fact, whenever $X \to Y$ is a representable
morphism of algebraic spaces, then this agrees with the notion defined via
Spaces, Definition \ref{spaces-definition-relative-representable-property}.
This follows by combining Lemma \ref{lemma-etale-local} below and
Spaces, Lemma \ref{spaces-lemma-representable-morphisms-spaces-property}.

\begin{lemma}
\label{lemma-etale-local}
Let $S$ be a scheme.
Let $f : X \to Y$ be a morphism of algebraic spaces over $S$.
The following are equivalent:
\begin{enumerate}
\item $f$ is etale,
\item there exists a surjective etale morphism $\varphi : U \to X$,
where $U$ is a scheme, such that the composition $f \circ \varphi$ is
etale (as a morphism of algebraic spaces),
\item there exists a surjective etale morphism $\psi : V \to Y$,
where $V$ is a scheme, such that the base change $V \times_X Y \to V$
is etale (as a morphism of algebraic spaces),
\item there exists a commutative diagram
$$
\xymatrix{
U \ar[d] \ar[r] & V \ar[d] \\
X \ar[r] & Y
}
$$
where $U$, $V$ are schemes and the vertical arrows are surjective etale
such that the horizontal arrow is etale.
\end{enumerate}
\end{lemma}

\begin{proof}
Let us prove that (4) implies (1). Assume a diagram as in (4) given.
Let $W \to X$ be an etale morphism with $W$ a scheme. Then we see
that $W \times_X U \to U$ is etale. Hence $W \times_X U \to V$ is etale,
and also $W \times_X U \to Y$ is etale by
Lemma \ref{lemma-etale-over-space} (1). Since also
the projection $W \times_X U \to W$ is surjective and etale, we conclude
from Lemma \ref{lemma-etale-over-space} (2) that $W \to Y$ is etale.

\medskip\noindent
Let us prove that (1) implies (4). Assume (1). Choose a commutative diagram
$$
\xymatrix{
U \ar[d] \ar[r] & V \ar[d] \\
X \ar[r] & Y
}
$$
where $U \to X$ and $V \to Y$ are surjective and etale, see
Spaces, Lemma \ref{spaces-lemma-lift-morphism-presentations}.
By assumption the morphism $U \to Y$ is etale,
and hence $U \to V$ is etale by Lemma \ref{lemma-etale-over-space} (2).

\medskip\noindent
We omit the proof that (2) and (3) are also equivalent to (1).
\end{proof}

\begin{lemma}
\label{lemma-composition-etale}
The composition of two etale morphisms of algebraic spaces
is etale.
\end{lemma}

\begin{proof}
This is immediate from the definition.
\end{proof}

\begin{lemma}
\label{lemma-base-change-etale}
The base change of an etale morphism of algebraic spaces
by any morphism of algebraic spaces is etale.
\end{lemma}

\begin{proof}
Let $X \to Y$ be an etale morphism of algebraic spaces over $S$.
Let $Z \to Y$ be a morphism of algebraic spaces.
Choose a scheme $U$ and a surjective etale morphism $U \to X$.
Choose a scheme $W$ and a surjective etale morphism $W \to Z$.
Then $U \to Y$ is etale, hence in the diagram
$$
\xymatrix{
W \times_Y U \ar[d] \ar[r] & W \ar[d] \\
Z \times_Y X \ar[r] & Z
}
$$
the top horizontal arrow is etale.
Moreover, the left vertical arrow is surjective
and etale (verification omitted). Hence we conclude that the lower
horizontal arrow is etale by Lemma \ref{lemma-etale-local}.
\end{proof}

\begin{lemma}
\label{lemma-etale-permanence}
Let $S$ be a scheme. Let $X, Y, Z$ be algebraic spaces.
Let $g : X \to Z$, $h : Y \to Z$ be etale morphisms and let
$f : X \to Y$ be a morphism such that $h \circ f = g$.
Then $f$ is etale.
\end{lemma}

\begin{proof}
Choose a commutative diagram
$$
\xymatrix{
U \ar[d] \ar[r]_\chi & V \ar[d] \\
X \ar[r] & Y
}
$$
where $U \to X$ and $V \to Y$ are surjective and etale, see
Spaces, Lemma \ref{spaces-lemma-lift-morphism-presentations}.
By assumption the morphisms $\varphi : U \to X \to Z$ and
$\psi : V \to Y \to Z$ are etale. Moreover, $\psi \circ \chi = \varphi$
by our assumption on $f, g, h$.
Hence $U \to V$ is etale by Lemma \ref{lemma-etale-over-space}
part (2).
\end{proof}

\begin{lemma}
\label{lemma-etale-open}
Let $S$ be a scheme.
If $X \to Y$ is an etale morphism of algebraic spaces over $S$,
then the associated map $|X| \to |Y|$ of topological spaces
is open.
\end{lemma}

\begin{proof}
This is clear from the diagram in
Lemma \ref{lemma-etale-local} and Lemma \ref{lemma-topology-points}.
\end{proof}



\section{Special coverings}
\label{section-special-coverings}

\noindent
In this section we collect some straightforward lemmas on the existence
of etale surjective coverings of algebraic spaces.

\begin{lemma}
\label{lemma-cover-by-union-affines}
Let $S$ be a scheme.
Let $X$ be an algebraic space over $S$.
There exists a surjective etale morphism $U \to X$ where
$U$ is a disjoint union of affine schemes.
\end{lemma}

\begin{proof}
Let $V \to X$ be a surjective etale morphism.
Let $V = \bigcup_{i \in I}$ be a Zariksi open covering.
Then set $U = \coprod_{i \in I} V_i$ with induced morphism
$U \to V \to X$. This is etale according to
Lemma \ref{lemma-etale-over-space} and it is clearly surjective.
\end{proof}

\begin{lemma}
\label{lemma-union-of-quasi-compact}
Let $S$ be a scheme.
Let $X$ be an algebraic space over $S$.
There exists a Zariski covering $X = \bigcup X_i$
such that each algebraic space $X_i$ has a surjective
etale covering by an affine scheme.
\end{lemma}

\begin{proof}
By Lemma \ref{lemma-cover-by-union-affines} we can find a surjective
etale morphism $U = \coprod U_i \to X$, with $U_i$ affine.
According to
Spaces, Lemma \ref{spaces-lemma-space-presentation},
$R$ is an etale equivalence relation on $U$ and we have $X = U/R$. Set
$R_i = U_i \times_X U_i$; this is also the restriction $R|_{U_i}$ of $R$
to $U_i$. By
Spaces, Lemma \ref{spaces-lemma-finding-opens}
we see that $X_i = U_i/R_i$ is an open subspace of $X$. Since clearly
$X = \bigcup X_i$ is a Zariski covering, we win.
\end{proof}

\begin{lemma}
\label{lemma-quasi-compact-affine-cover}
Let $S$ be a scheme.
Let $X$ be an algebraic space over $S$.
Then $S$ is quasi-compact if and only if
there exists an etale surjective morphism $U \to X$
with $U$ an affine scheme.
\end{lemma}

\begin{proof}
If there exists an etale surjective morphism $U \to X$ with $U$
affine then $X$ is quasi-compact by Definition \ref{definition-quasi-compact}.
Conversely, if $X$ is quasi-compact, then $|X|$ is quasi-compact.
Let $U = \coprod_{i \in I} U_i$ be a disjoint union of affine schemes
with an etale and surjective map $\varphi : U \to X$
(Lemma \ref{lemma-cover-by-union-affines}).
Then $|X| = \bigcup \varphi(|U_i|)$ and
by quasi-compactness there is a finite subset $i_1, \ldots, i_n$
such that $|X| = \bigcup \varphi(|U_{i_j}|)$. Hence
$U_{i_1} \cup \ldots \cup U_{i_n}$ is an affine scheme with a 
finite surjective morphism towards $X$.
\end{proof}

\begin{lemma}
\label{lemma-algebraic-space-affine-cover}
Let $S$ be a scheme.
Let $X$ be an algebraic space over $S$.
Let $U$ be an affine scheme, or a disjoint union of affine schemes,
and $U \to X$ surjective etale.
Then $R = U \times_X U$ is a separated scheme.
\end{lemma}

\begin{proof}
See Spaces, Lemma \ref{spaces-lemma-properties-diagonal}.
\end{proof}



\section{The etale site of an algebraic space}
\label{section-etale-site}

\noindent
In this section we define the small etale site of an algebraic space.
This is the analogue of the small etale site $S_{etale}$ of a scheme.
Lemma \ref{lemma-etale-over-space} implies that in the definition below
any morphism between objects of the etale site of $X$ is etale, and that
any scheme etale over an object of $X_{etale}$ is also an object of
$X_{etale}$.

\begin{definition}
\label{definition-etale-site}
Let $S$ be a scheme.
Let $\textit{Sch}_{fppf}$ be a big fppf site containing $S$,
and let $\textit{Sch}_{etale}$ be the corresponding big etale site
(i.e., having the same underlying category).
Let $X$ be an algebraic space over $S$.
The {\it small etale site $X_{etale}$} of $X$ is defined as follows:
\begin{enumerate}
\item An object of $X_{etale}$ is a morphism $\varphi : U \to X$
where $U \in \text{Ob}((\textit{Sch}/S)_{etale})$ is a scheme and
$\varphi$ is an etale morphism,
\item a morphism $(\varphi : U \to X) \to (\varphi' : U' \to X)$
is given by a morphism of schemes $\chi : U \to U'$ such that
$\varphi = \varphi' \circ \chi$, and
\item a family of morphisms $\{(U_i \to X) \to (U \to X)\}_{i \in I}$
of $X_{etale}$ is a covering if and only if $\{U_i \to U\}_{i \in I}$
is a covering of $(\textit{Sch}/S)_{etale}$.
\end{enumerate}
\end{definition}

\noindent
A consequence of our choice is that the etale site of an algebraic space
in general does not have a final object!

\medskip\noindent
There are several other choices we could have made here. For example
we could have considered all {\it algebraic spaces} $U$ which are etale
over $X$. We decided not to do so, since we like to think of schemes as
the fundamental objects of algebraic geometry. On the other hand, we
do need this notion also, since the small etale site of an algebraic
space is not sufficiently flexible, especially when discussing functoriality
of the small etale site, see Lemma \ref{lemma-functoriality-etale-site}
below.

\begin{definition}
\label{definition-spaces-etale-site}
Let $S$ be a scheme.
Let $\textit{Sch}_{fppf}$ be a big fppf site containing $S$,
and let $\textit{Sch}_{etale}$ be the corresponding big etale site
(i.e., having the same underlying category).
Let $X$ be an algebraic space over $S$.
The site {\it $X_{spaces, etale}$} of $X$ is defined as follows:
\begin{enumerate}
\item An object of $X_{etale}$ is a morphism $\varphi : U \to X$
where $U$ is an algebraic space over $S$ and
$\varphi$ is an etale morphism of algebraic spaces over $S$,
\item a morphism $(\varphi : U \to X) \to (\varphi' : U' \to X)$
is given by a morphism of algebraic spaces $\chi : U \to U'$ such that
$\varphi = \varphi' \circ \chi$, and
\item a family of morphisms
$\{\varphi_i : (U_i \to X) \to (U \to X)\}_{i \in I}$
of $X_{etale}$ is a covering if and only if
$U = \bigcup \varphi_i(U_i)$.
\end{enumerate}
(As usual we choose a set of coverings of this type, including at least
the coverings in $X_{etale}$, as in Sets, Lemma \ref{sets-lemma-coverings-site}
to turn $X_{spaces, etale}$ into a site.)
\end{definition}

\noindent
Let us show right away that the corresponding topos equals the
small etale topos of $X$.

\begin{lemma}
\label{lemma-compare-etale-sites}
The functor
$$
X_{etale} \longrightarrow X_{spaces, etale}, \quad
U/X \longmapsto U/X
$$
is a special cocontinuous functor
(Sites, Definition \ref{sites-definition-special-cocontinuous-functor})
and hence induces an equivalence of topoi
$\textit{Sh}(X_{etale}) \to \textit{Sh}(X_{spaces, etale})$.
\end{lemma}

\begin{proof}
We have to show that the functor satisfies the assumptions (1) -- (5) of
Sites, Lemma \ref{sites-lemma-equivalence}.
It is clear that the functor is continuous and cocontinuous, which
proves assumptions (1) and (2).
Assumptions (3) and (4) hold simply because the functor is fully faithful.
Assumption (5) holds, because an algebraic space by definition has
a covering by a scheme.
\end{proof}

\begin{remark}
\label{remark-explain-equivalence}
Let us explain the meaning of Lemma \ref{lemma-compare-etale-sites}.
Let $S$ be a scheme, and let $X$ be an algebraic space over $S$.
Let $\mathcal{F}$ be a sheaf on the small etale site $X_{etale}$ of $X$.
The lemma says that there exists a unique sheaf $\mathcal{F}'$ on
$X_{spaces, etale}$ which restricts back to $\mathcal{F}$ on the
subcategory $X_{etale}$. If $U \to X$ is an etale morphism of algebraic
spaces, then how do we compute $\mathcal{F}'(U)$? Well, by definition
of an algebraic space there exists a scheme $U'$ and a surjective
etale morphism $U' \to U$. Then $\{U' \to U\}$ is a covering in
$X_{spaces, etale}$ and hence we get an equalizer diagram
$$
\xymatrix{
\mathcal{F}'(U) \ar[r] &
\mathcal{F}(U') \ar@<1ex>[r] \ar@<-1ex>[r] &
\mathcal{F}(U' \times_U U').
}
$$
Note that $U' \times_U U'$ is a scheme, and hence we may
write $\mathcal{F}$ and not $\mathcal{F}'$.
Thus we see how to compute $\mathcal{F}'$
when given the sheaf $\mathcal{F}$.
\end{remark}

\begin{lemma}
\label{lemma-functoriality-etale-site}
Let $S$ be a scheme.
Let $f : X \to Y$ be a morphism of algebraic spaces over $S$.
\begin{enumerate}
\item The continuous functor
$$
Y_{spaces, etale} \longrightarrow X_{spaces, etale}, \quad
V \longmapsto X \times_Y V
$$
induces a morphism of sites
$$
f_{spaces, etale} : X_{spaces, etale} \to Y_{spaces, etale}.
$$
\item The rule $f \mapsto f_{spaces, etale}$ is compatible with
compositions, in other words $(f \circ g)_{spaces, etale}
= f_{spaces, etale} \circ g_{spaces, etale}$ (see
Sites, Definition \ref{sites-definition-composition-morphisms-sites}).
\item There is an associated morphism of topoi
$f_{small} : \textit{Sh}(X_{etale}) \to \textit{Sh}(Y_{etale})$
whose construction is compatible with compositions.
\item If $f$ is a representable morphism of algebraic spaces,
then $f_{small}$ comes from a morphism of sites $X_{etale} \to Y_{etale}$,
corresponding to the continuous functor $V \mapsto X \times_Y V$.
\end{enumerate}
\end{lemma}

\begin{proof}
Let us show that the functor described in (1) satisfies the assumptions
of Sites, Proposition \ref{sites-proposition-get-morphism}.
Thus we have to show that
$Y_{spaces, etale}$ has a final object (namely $Y$) and that
the functor transforms this into a final object in $X_{spaces, etale}$
(namely $X$). This is clear as $X \times_Y Y = X$ in any category.
Next, we have to show that $Y_{spaces, etale}$ has fibre products.
This is true since the category of algebraic spaces has fibre products,
and since $V \times_Y V'$ is etale over $Y$ if $V$ and $V'$ are etale
over $Y$ (see Lemmas \ref{lemma-composition-etale} and
\ref{lemma-base-change-etale} above).
OK, so the proposition applies and we see that we get a morphism
of sites as described in (1).

\medskip\noindent
Part (2) you get by unwinding the definitions.
Part (3) is clear by using the equivalences for $X$ and $Y$
from Lemma \ref{lemma-compare-etale-sites} above.
Part (4) follows, because if $f$ is representable, then the
functors above fit into a commutative diagram
$$
\xymatrix{
X_{etale} \ar[r] &
X_{spaces, etale} \\
Y_{etale} \ar[r] \ar[u] &
Y_{spaces, etale} \ar[u]
}
$$
of categories.
\end{proof}

\noindent
We can do a little bit better than the lemma above in describing
the relationship between sheaves on $X$ and sheaves on $Y$.
Namely, we can formulate this in turns of $f$-maps, compare
Sheaves, Definition \ref{sheaves-definition-f-map}, as follows.

\begin{definition}
\label{definition-f-map}
Let $S$ be a scheme.
Let $f : X \to Y$ be a morphism of algebraic spaces over $S$.
Let $\mathcal{F}$ be a sheaf of sets on $X_{etale}$ and
let $\mathcal{G}$ be a sheaf of sets on $Y_{etale}$.
An {\it $f$-map $\varphi : \mathcal{G} \to \mathcal{F}$}
is a collection of maps
$\varphi_{(U,V,g)} : \mathcal{G}(V) \to \mathcal{F}(U)$
indexed by commutative diagrams
$$
\xymatrix{
U \ar[d]_g \ar[r] & X \ar[d]^f \\
V \ar[r] & Y
}
$$
where $U \in X_{etale}$, $V \in Y_{etale}$ such that whenever given
an extended diagram
$$
\xymatrix{
U' \ar[r] \ar[d]_{g'} & U \ar[d]_g \ar[r] & X \ar[d]^f \\
V' \ar[r] & V \ar[r] & Y
}
$$
with $V' \to V$ and $U' \to U$ etale morphisms of schemes the diagram
$$
\xymatrix{
\mathcal{G}(V)
\ar[rr]_{\varphi_{(U, V, g)}}
\ar[d]_{\text{restriction of }\mathcal{G}} & &
\mathcal{F}(U)
\ar[d]^{\text{restriction of }\mathcal{F}} \\
\mathcal{G}(V')
\ar[rr]^{\varphi_{(U', V', g')}} & &
\mathcal{F}(U')
}
$$
commutes.
\end{definition}

\begin{lemma}
\label{lemma-f-map}
Let $S$ be a scheme.
Let $f : X \to Y$ be a morphism of algebraic spaces over $S$.
Let $\mathcal{F}$ be a sheaf of sets on $X_{etale}$ and
let $\mathcal{G}$ be a sheaf of sets on $Y_{etale}$.
There are canonical bijections between the following three sets:
\begin{enumerate}
\item The set of maps $\mathcal{G} \to f_{small, *}\mathcal{F}$.
\item The set of maps $f_{small}^{-1}\mathcal{G} \to \mathcal{F}$.
\item The set of $f$-maps $\varphi : \mathcal{G} \to \mathcal{F}$.
\end{enumerate}
\end{lemma}

\begin{proof}
Note that (1) and (2) are the same because the functors $f_{small, *}$
and $f_{small}^{-1}$ are a pair of adjoint functors.
Suppose that $\alpha : f_{small}^{-1}\mathcal{G} \to \mathcal{F}$
is a map of sheaves on $Y_{etale}$. Let a diagram
$$
\xymatrix{
U \ar[d]_g \ar[r]_{j_U} & X \ar[d]^f \\
V \ar[r]^{j_V} & Y
}
$$
as in Definition \ref{definition-f-map} be given.
By the commutativity of the diagram we also get a map
$g_{small}^{-1}(j_V)^{-1}\mathcal{G} \to (j_U)^{-1}\mathcal{F}$
(compare Sites, Section \ref{sites-section-localization} for the
description of the localization functors). Hence we certainly
get a map
$\varphi_{(V, U, g)} :
\mathcal{G}(V) = (j_V)^{-1}\mathcal{G}(V) 
\to
(j_U)^{-1}\mathcal{F}(U) = \mathcal{F}(U)$.
We omit the verification that this rule is compatible with
further restrictions and defines an $f$-map from $\mathcal{G}$ to
$\mathcal{F}$.

\medskip\noindent
Conversely, suppose that we are given an $f$-map
$\varphi = (\varphi_{(U, V, g)})$.
Let $\mathcal{G}'$ (resp.\ $\mathcal{F}'$) denote the extension of
$\mathcal{G}$ (resp.\ $\mathcal{F}$) to $Y_{spaces, etale}$
(resp.\ $X_{spaces, etale}$), see Lemma \ref{lemma-compare-etale-sites}.
Then we have to construct a map of sheaves
$$
\mathcal{G}' \longrightarrow (f_{spaces, etale})_*\mathcal{F}'
$$
To do this, let $V \to Y$ be an etale morphism of algebraic spaces.
We have to construct a map of sets
$$
\mathcal{G}'(V) \to \mathcal{F}'(X \times_Y V)
$$
Choose an etale surjective morphism $V' \to V$ with $V'$ a scheme,
and after that choose an etale surjective morphsm
$U' \to X \times_U V'$ with $U'$ a scheme. We get a morphism of
schemes $g' : U' \to V'$ and also a morphism of schemes
$$
g'' : U' \times_{X \times_Y V} U' \longrightarrow V' \times_V V'
$$
Consider the following diagram
$$
\xymatrix{
\mathcal{F}'(X \times_Y V) \ar[r] &
\mathcal{F}(U') \ar@<1ex>[r] \ar@<-1ex>[r] &
\mathcal{F}(U' \times_{X \times_Y V} U') \\
\mathcal{G}'(X \times_Y V) \ar[r] \ar@{..>}[u] &
\mathcal{G}(V') \ar@<1ex>[r] \ar@<-1ex>[r] \ar[u]_{\varphi_{(U', V', g')}} &
\mathcal{G}(V' \times_V V') \ar[u]_{\varphi_{(U'', V'', g'')}}
}
$$
The compatibility of the maps $\varphi_{...}$
with restriction shows that the two left squares commute.
The definition of coverings in $X_{spaces, etale}$ shows that
the horizontal rows are equalizer diagrams. Hence we get
the dotted arrow. We leave it to the reader to show that these
arrows are compatible with the restriction mappings.
\end{proof}







\section{Quasi-coherent sheaves on algebraic spaces}
\label{section-quasi-coherent}

\noindent
By the results of
Descent, Section \ref{descent-section-quasi-coherent-sheaves}
we may think of a quasi-coherent sheaf $\mathcal{F}$ on a scheme
$X$ as a quasi-coherent sheaf of $\mathcal{O}$-modules on the small
etale site $X_{etale}$. This is our motivation for the definition of
a quasi-coherent sheaf on an algebraic space.

\begin{lemma}
\label{lemma-sheaf-condition-holds}
Let $S$ be a scheme. Let $X$ be an algebraic space over $S$.
The rule $U \mapsto \Gamma(U, \mathcal{O}_U)$ defines
a sheaf of rings on $X_{etale}$.
\end{lemma}

\begin{proof}
Immediate from the definition of a covering and
Descent, Lemma \ref{descent-lemma-sheaf-condition-holds}.
\end{proof}

\begin{definition}
\label{definition-structure-sheaf}
Let $S$ be a scheme.
Let $X$ be an algebraic space over $S$.
The {\it structure sheaf} of $X$
is the sheaf of rings $\mathcal{O}_X$
on the small etale site $X_{etale}$ described in
Lemma \ref{lemma-sheaf-condition-holds}.
\end{definition}

\begin{lemma}
\label{lemma-morphism-ringed-topoi}
Let $S$ be a scheme.
Let $f : X \to Y$ be a morphism of algebraic spaces over $S$.
Then there is a canonical map $f^\sharp$ such that
$(f_{small}, f^\sharp) :
(X_{etale}, \mathcal{O}_X)
\to
(Y_{etale}, \mathcal{O}_Y)$
is a morphism of ringed topoi, and such that the construction
$f \mapsto (f_{small}, f^\sharp)$ is compatible with compositions.
\end{lemma}

\begin{proof}
By Lemma \ref{lemma-f-map} it suffices to give an $f$-map from
$\mathcal{O}_Y$ to $\mathcal{O}_X$. In other words, for every
commutative diagram
$$
\xymatrix{
U \ar[d]_g \ar[r] & X \ar[d]^f \\
V \ar[r] & Y
}
$$
where $U \in X_{etale}$, $V \in Y_{etale}$ we have to give a map of
rings
$
(f^\sharp)_{(U, V, g)} :
\Gamma(V, \mathcal{O}_V)
\to
\Gamma(U, \mathcal{O}_U).
$
Of course we just take $(f^\sharp)_{(U, V, g)} = g^\sharp$.
It is clear that this is compatible with restriction mappings
and hence indeed gives and $f$-map.
The proof showing compatibility with compositions is omitted.
\end{proof}

\noindent
Thus we see that given a morphism $f : X \to Y$ of algebraic spaces
we get well defined pullback and direct image functors
$$
f^* :
\textit{Mod}(\mathcal{O}_Y)
\longrightarrow
\textit{Mod}(\mathcal{O}_X), \quad
f_* :
\textit{Mod}(\mathcal{O}_X)
\longrightarrow
\textit{Mod}(\mathcal{O}_Y)
$$
which are adjoint in the usual way.

\begin{definition}
\label{definition-quasi-coherent}
Let $S$ be a scheme.
Let $X$ be an algebraic space over $S$.
A {\it quasi-coherent} $\mathcal{O}_X$-module 
is a quasi-coherent module on the ringed site $(X_{etale}, \mathcal{O}_X)$
in the sense of
Modules on Sites,
Definition \ref{sites-modules-definition-site-local}.
\end{definition}

\noindent
As usual, quasi-coherent sheaves behave well with respect to pullback.

\begin{lemma}
\label{lemma-pullback-quasi-coherent}
Let $S$ be a scheme.
Let $f : X \to Y$ be a morphism of algebraic spaces over $S$.
The pullback functor
$f^* : \textit{Mod}(\mathcal{O}_Y) \to \textit{Mod}(\mathcal{O}_X)$
preserves quasi-coherent sheaves.
\end{lemma}

\begin{proof}
This is a general fact, see
Modules on Sites, Lemma \ref{sites-modules-lemma-local-pullback}.
\end{proof}













\section{Other chapters}

\begin{multicols}{2}
\begin{enumerate}
\item \hyperref[introduction-section-phantom]{Introduction}
\item \hyperref[conventions-section-phantom]{Conventions}
\item \hyperref[sets-section-phantom]{Set Theory}
\item \hyperref[categories-section-phantom]{Categories}
\item \hyperref[topology-section-phantom]{Topology}
\item \hyperref[sheaves-section-phantom]{Sheaves on Spaces}
\item \hyperref[algebra-section-phantom]{Commutative Algebra}
\item \hyperref[sites-section-phantom]{Sites and Sheaves}
\item \hyperref[homology-section-phantom]{Homological Algebra}
\item \hyperref[derived-section-phantom]{Derived Categories}
\item \hyperref[more-algebra-section-phantom]{More Algebra}
\item \hyperref[simplicial-section-phantom]{Simplicial Methods}
\item \hyperref[modules-section-phantom]{Sheaves of Modules}
\item \hyperref[sites-modules-section-phantom]{Modules on Sites}
\item \hyperref[injectives-section-phantom]{Injectives}
\item \hyperref[cohomology-section-phantom]{Cohomology of Sheaves}
\item \hyperref[sites-cohomology-section-phantom]{Cohomology on Sites}
\item \hyperref[hypercovering-section-phantom]{Hypercoverings}
\item \hyperref[schemes-section-phantom]{Schemes}
\item \hyperref[constructions-section-phantom]{Constructions of Schemes}
\item \hyperref[properties-section-phantom]{Properties of Schemes}
\item \hyperref[morphisms-section-phantom]{Morphisms of Schemes}
\item \hyperref[coherent-section-phantom]{Coherent Cohomology}
\item \hyperref[divisors-section-phantom]{Divisors}
\item \hyperref[limits-section-phantom]{Limits of Schemes}
\item \hyperref[varieties-section-phantom]{Varieties}
\item \hyperref[chow-section-phantom]{Chow Homology}
\item \hyperref[topologies-section-phantom]{Topologies on Schemes}
\item \hyperref[descent-section-phantom]{Descent}
\item \hyperref[more-morphisms-section-phantom]{More on Morphisms}
\item \hyperref[flat-section-phantom]{More on Flatness}
\item \hyperref[groupoids-section-phantom]{Groupoid Schemes}
\item \hyperref[more-groupoids-section-phantom]{More on Groupoid Schemes}
\item \hyperref[etale-section-phantom]{\'Etale Morphisms of Schemes}
\item \hyperref[etale-cohomology-section-phantom]{\'Etale Cohomology}
\item \hyperref[spaces-section-phantom]{Algebraic Spaces}
\item \hyperref[spaces-properties-section-phantom]{Properties of Algebraic Spaces}
\item \hyperref[spaces-morphisms-section-phantom]{Morphisms of Algebraic Spaces}
\item \hyperref[spaces-topologies-section-phantom]{Topologies on Algebraic Spaces}
\item \hyperref[spaces-descent-section-phantom]{Descent and Algebraic Spaces}
\item \hyperref[spaces-more-morphisms-section-phantom]{More on Morphisms of Spaces}
\item \hyperref[quot-section-phantom]{Quot and Hilbert Spaces}
\item \hyperref[stacks-section-phantom]{Stacks}
\item \hyperref[spaces-groupoids-section-phantom]{Groupoids in Algebraic Spaces}
\item \hyperref[spaces-more-groupoids-section-phantom]{More on Groupoids in Spaces}
\item \hyperref[bootstrap-section-phantom]{Bootstrap}
\item \hyperref[examples-stacks-section-phantom]{Examples of Stacks}
\item \hyperref[groupoids-quotients-section-phantom]{Quotients of Groupoids}
\item \hyperref[algebraic-section-phantom]{Algebraic Stacks}
\item \hyperref[criteria-section-phantom]{Criteria for Representability}
\item \hyperref[stacks-properties-section-phantom]{Properties of Algebraic Stacks}
\item \hyperref[stacks-morphisms-section-phantom]{Morphisms of Algebraic Stacks}
\item \hyperref[examples-section-phantom]{Examples}
\item \hyperref[exercises-section-phantom]{Exercises}
\item \hyperref[guide-section-phantom]{Guide to Literature}
\item \hyperref[desirables-section-phantom]{Desirables}
\item \hyperref[coding-section-phantom]{Coding Style}
\item \hyperref[fdl-section-phantom]{GNU Free Documentation License}
\item \hyperref[index-section-phantom]{Auto Generated Index}
\end{enumerate}
\end{multicols}


\bibliography{my}
\bibliographystyle{amsalpha}

\end{document}
