\IfFileExists{stacks-project.cls}{%
\documentclass{stacks-project}
}{%
\documentclass{amsart}
}

% The following AMS packages are automatically loaded with
% the amsart documentclass:
%\usepackage{amsmath}
%\usepackage{amssymb}
%\usepackage{amsthm}

% For dealing with references we use the comment environment
\usepackage{verbatim}
\newenvironment{reference}{\comment}{\endcomment}
%\newenvironment{reference}{}{}
\newenvironment{slogan}{\comment}{\endcomment}
\newenvironment{history}{\comment}{\endcomment}

% For commutative diagrams you can use
% \usepackage{amscd}
\usepackage[all]{xy}

% We use 2cell for 2-commutative diagrams.
\xyoption{2cell}
\UseAllTwocells

% To put source file link in headers.
% Change "template.tex" to "this_filename.tex"
% \usepackage{fancyhdr}
% \pagestyle{fancy}
% \lhead{}
% \chead{}
% \rhead{Source file: \url{template.tex}}
% \lfoot{}
% \cfoot{\thepage}
% \rfoot{}
% \renewcommand{\headrulewidth}{0pt}
% \renewcommand{\footrulewidth}{0pt}
% \renewcommand{\headheight}{12pt}

\usepackage{multicol}

% For cross-file-references
\usepackage{xr-hyper}

% Package for hypertext links:
\usepackage{hyperref}

% For any local file, say "hello.tex" you want to link to please
% use \externaldocument[hello-]{hello}
\externaldocument[introduction-]{introduction}
\externaldocument[conventions-]{conventions}
\externaldocument[sets-]{sets}
\externaldocument[categories-]{categories}
\externaldocument[topology-]{topology}
\externaldocument[sheaves-]{sheaves}
\externaldocument[sites-]{sites}
\externaldocument[stacks-]{stacks}
\externaldocument[fields-]{fields}
\externaldocument[algebra-]{algebra}
\externaldocument[brauer-]{brauer}
\externaldocument[homology-]{homology}
\externaldocument[derived-]{derived}
\externaldocument[simplicial-]{simplicial}
\externaldocument[more-algebra-]{more-algebra}
\externaldocument[smoothing-]{smoothing}
\externaldocument[modules-]{modules}
\externaldocument[sites-modules-]{sites-modules}
\externaldocument[injectives-]{injectives}
\externaldocument[cohomology-]{cohomology}
\externaldocument[sites-cohomology-]{sites-cohomology}
\externaldocument[dga-]{dga}
\externaldocument[dpa-]{dpa}
\externaldocument[hypercovering-]{hypercovering}
\externaldocument[schemes-]{schemes}
\externaldocument[constructions-]{constructions}
\externaldocument[properties-]{properties}
\externaldocument[morphisms-]{morphisms}
\externaldocument[coherent-]{coherent}
\externaldocument[divisors-]{divisors}
\externaldocument[limits-]{limits}
\externaldocument[varieties-]{varieties}
\externaldocument[topologies-]{topologies}
\externaldocument[descent-]{descent}
\externaldocument[perfect-]{perfect}
\externaldocument[more-morphisms-]{more-morphisms}
\externaldocument[flat-]{flat}
\externaldocument[groupoids-]{groupoids}
\externaldocument[more-groupoids-]{more-groupoids}
\externaldocument[etale-]{etale}
\externaldocument[chow-]{chow}
\externaldocument[intersection-]{intersection}
\externaldocument[pic-]{pic}
\externaldocument[adequate-]{adequate}
\externaldocument[dualizing-]{dualizing}
\externaldocument[duality-]{duality}
\externaldocument[discriminant-]{discriminant}
\externaldocument[local-cohomology-]{local-cohomology}
\externaldocument[curves-]{curves}
\externaldocument[resolve-]{resolve}
\externaldocument[models-]{models}
\externaldocument[pione-]{pione}
\externaldocument[etale-cohomology-]{etale-cohomology}
\externaldocument[proetale-]{proetale}
\externaldocument[crystalline-]{crystalline}
\externaldocument[spaces-]{spaces}
\externaldocument[spaces-properties-]{spaces-properties}
\externaldocument[spaces-morphisms-]{spaces-morphisms}
\externaldocument[decent-spaces-]{decent-spaces}
\externaldocument[spaces-cohomology-]{spaces-cohomology}
\externaldocument[spaces-limits-]{spaces-limits}
\externaldocument[spaces-divisors-]{spaces-divisors}
\externaldocument[spaces-over-fields-]{spaces-over-fields}
\externaldocument[spaces-topologies-]{spaces-topologies}
\externaldocument[spaces-descent-]{spaces-descent}
\externaldocument[spaces-perfect-]{spaces-perfect}
\externaldocument[spaces-more-morphisms-]{spaces-more-morphisms}
\externaldocument[spaces-flat-]{spaces-flat}
\externaldocument[spaces-groupoids-]{spaces-groupoids}
\externaldocument[spaces-more-groupoids-]{spaces-more-groupoids}
\externaldocument[bootstrap-]{bootstrap}
\externaldocument[spaces-pushouts-]{spaces-pushouts}
\externaldocument[groupoids-quotients-]{groupoids-quotients}
\externaldocument[spaces-more-cohomology-]{spaces-more-cohomology}
\externaldocument[spaces-simplicial-]{spaces-simplicial}
\externaldocument[formal-spaces-]{formal-spaces}
\externaldocument[restricted-]{restricted}
\externaldocument[spaces-resolve-]{spaces-resolve}
\externaldocument[formal-defos-]{formal-defos}
\externaldocument[defos-]{defos}
\externaldocument[cotangent-]{cotangent}
\externaldocument[examples-defos-]{examples-defos}
\externaldocument[algebraic-]{algebraic}
\externaldocument[examples-stacks-]{examples-stacks}
\externaldocument[stacks-sheaves-]{stacks-sheaves}
\externaldocument[criteria-]{criteria}
\externaldocument[artin-]{artin}
\externaldocument[quot-]{quot}
\externaldocument[stacks-properties-]{stacks-properties}
\externaldocument[stacks-morphisms-]{stacks-morphisms}
\externaldocument[stacks-limits-]{stacks-limits}
\externaldocument[stacks-cohomology-]{stacks-cohomology}
\externaldocument[stacks-perfect-]{stacks-perfect}
\externaldocument[stacks-introduction-]{stacks-introduction}
\externaldocument[stacks-more-morphisms-]{stacks-more-morphisms}
\externaldocument[stacks-geometry-]{stacks-geometry}
\externaldocument[moduli-]{moduli}
\externaldocument[moduli-curves-]{moduli-curves}
\externaldocument[examples-]{examples}
\externaldocument[exercises-]{exercises}
\externaldocument[guide-]{guide}
\externaldocument[desirables-]{desirables}
\externaldocument[coding-]{coding}
\externaldocument[obsolete-]{obsolete}
\externaldocument[fdl-]{fdl}
\externaldocument[index-]{index}

% Theorem environments.
%
\theoremstyle{plain}
\newtheorem{theorem}[subsection]{Theorem}
\newtheorem{proposition}[subsection]{Proposition}
\newtheorem{lemma}[subsection]{Lemma}

\theoremstyle{definition}
\newtheorem{definition}[subsection]{Definition}
\newtheorem{example}[subsection]{Example}
\newtheorem{exercise}[subsection]{Exercise}
\newtheorem{situation}[subsection]{Situation}

\theoremstyle{remark}
\newtheorem{remark}[subsection]{Remark}
\newtheorem{remarks}[subsection]{Remarks}

\numberwithin{equation}{subsection}

% Macros
%
\def\lim{\mathop{\rm lim}\nolimits}
\def\colim{\mathop{\rm colim}\nolimits}
\def\Spec{\mathop{\rm Spec}}
\def\Hom{\mathop{\rm Hom}\nolimits}
\def\Ext{\mathop{\rm Ext}\nolimits}
\def\SheafHom{\mathop{\mathcal{H}\!{\it om}}\nolimits}
\def\SheafExt{\mathop{\mathcal{E}\!{\it xt}}\nolimits}
\def\Sch{\textit{Sch}}
\def\Mor{\mathop{\rm Mor}\nolimits}
\def\Ob{\mathop{\rm Ob}\nolimits}
\def\Sh{\mathop{\textit{Sh}}\nolimits}
\def\NL{\mathop{N\!L}\nolimits}
\def\proetale{{pro\text{-}\acute{e}tale}}
\def\etale{{\acute{e}tale}}
\def\QCoh{\textit{QCoh}}
\def\Ker{\mathop{\rm Ker}}
\def\Im{\mathop{\rm Im}}
\def\Coker{\mathop{\rm Coker}}
\def\Coim{\mathop{\rm Coim}}

%
% Macros for moduli stacks/spaces
%
\def\QCohstack{\mathcal{QC}\!{\it oh}}
\def\Cohstack{\mathcal{C}\!{\it oh}}
\def\Spacesstack{\mathcal{S}\!{\it paces}}
\def\Quotfunctor{{\rm Quot}}
\def\Hilbfunctor{{\rm Hilb}}
\def\Curvesstack{\mathcal{C}\!{\it urves}}
\def\Polarizedstack{\mathcal{P}\!{\it olarized}}
\def\Complexesstack{\mathcal{C}\!{\it omplexes}}
% \Pic is the operator that assigns to X its picard group, usage \Pic(X)
% \Picardstack_{X/B} denotes the Picard stack of X over B
% \Picardfunctor_{X/B} denotes the Picard functor of X over B
\def\Pic{\mathop{\rm Pic}\nolimits}
\def\Picardstack{\mathcal{P}\!{\it ic}}
\def\Picardfunctor{{\rm Pic}}
\def\Deformationcategory{\mathcal{D}\!{\it ef}}


% OK, start here.
%
\begin{document}

\title{Properties of Algebraic Spaces}


\maketitle

\phantomsection
\label{section-phantom}

\tableofcontents

\section{Introduction}
\label{section-introduction}

\noindent
Please see Spaces, Section \ref{spaces-section-introduction}
for a brief introduction to algebraic spaces, and please read
some of that chapter for our basic definitions and conventions
concerning algebraic spaces. In this chapter we start introducing
some basic notions and properties of algebraic spaces. A fundamental
reference for the case of quasi-separated algebraic spaces is
\cite{Kn}.

\medskip\noindent
The discussion is somewhat awkward at times since we made the design
descision to first talk about properties of algebraic spaces by
themselves, and only later about properties of morphisms of algebraic
spaces. We make an exception for this rule regarding
{\it \'etale morphisms} of algebraic spaces, which we introduce in
Section \ref{section-etale-morphisms}. But until that section whenever
we say a morphism has a certain property, it automatically means the
source of the morphism is a scheme (or perhaps the morphism is representable).



\section{Conventions}
\label{section-conventions}

\noindent
The standing assumption is that all schemes are contained in
a big fppf site $\textit{Sch}_{fppf}$. And all rings $A$ considered
have the property that $\text{Spec}(A)$ is (isomorphic) to an
object of this big site.

\medskip\noindent
Let $S$ be a scheme and let $X$ be an algebraic space over $S$.
In this chapter and the following we will write $X \times_S X$
for the product of $X$ with itself (in the category of algebraic
spaces over $S$), instead of $X \times X$. The reason is that we
want to avoid confusion when changing base schemes, as in
Spaces, Section \ref{spaces-section-change-base-scheme}.


\section{Separation axioms}
\label{section-separation}

\noindent
In this section we collect all the ``absolute'' separation conditions
of algebraic spaces. Since in our language any algebraic space is an
algebraic space over some definite base scheme, any absolute property
of $X$ over $S$ corresponds to a conditions imposed on $X$ viewed
as an algebraic space over $\text{Spec}(\mathbf{Z})$. Here is the
precise formulation.

\begin{definition}
\label{definition-separated}
(Compare Spaces, Definition \ref{spaces-definition-separated}.)
Consider a big fppf site
$\textit{Sch}_{fppf} = (\textit{Sch}/\text{Spec}(\mathbf{Z}))_{fppf}$.
Let $X$ be an algebraic space over
$\text{Spec}(\mathbf{Z})$. Let $\Delta : X \to X \times X$
be the diagonal morphism.
\begin{enumerate}
\item We say $X$ is {\it separated} if $\Delta$ is a closed immersion.
\item We say $X$ is {\it locally separated}\footnote{In the
literature this often refers to quasi-separated and locally
separated algebraic spaces.} if $\Delta$ is an
immersion.
\item We say $X$ is {\it quasi-separated} if $\Delta$ is quasi-compact.
\item We say $X$ is {\it Zariski locally quasi-separated}\footnote{
This notion was suggested by B.\ Conrad.} if there
exists a Zariski covering $X = \bigcup_{i \in I} X_i$ (see Spaces,
Definition \ref{spaces-definition-Zariski-open-covering}) such that
each $X_i$ is quasi-separated.
\end{enumerate}
Let $S$ is a scheme contained in $\textit{Sch}_{fppf}$, and let
$X$ be an algebraic space over $S$. Then we say $X$ is {\it separated},
{\it locally separated}, {\it quasi-separated}, or
{\it Zariski locally quasi-separated}
if $X$ viewed as an algebraic space over $\text{Spec}(\mathbf{Z})$ (see
Spaces, Definition \ref{spaces-definition-base-change})
has the corresponding property.
\end{definition}

\noindent
It is true that an algebraic space $X$ over $S$ which is separated
(in the absolute sense above) is separated over $S$ (and similarly
for the other absolute separation properties above). This will be discussed
in great detail in
Morphisms of Spaces, Section \ref{spaces-morphisms-section-separation-axioms}.

\begin{lemma}
\label{lemma-trivial-implications}
Let $S$ be a scheme.
Let $X$ be an algebraic space over $S$.
We have the following implications among the separation axioms
of Definition \ref{definition-separated}:
\begin{enumerate}
\item separated implies all the others,
\item quasi-separated implies Zariski locally quasi-separated.
\end{enumerate}
\end{lemma}

\begin{proof}
Omitted.
\end{proof}











\section{Points of algebraic spaces}
\label{section-points}

\noindent
As is clear from
Spaces, Example \ref{spaces-example-affine-line-translation}
a point of an algebraic space should not be defined as a monomorphism
from the spectrum of a field.
Instead we define them as equivalence classes of morphisms of specra
of fields as equivalence classes of morphisms from spectra of fields.
exactly as explained in
Schemes, Section \ref{schemes-section-points}.

\medskip\noindent
Let $S$ be a scheme.
Let $F$ be a presheaf on $(\textit{Sch}/S)_{fppf}$.
Let $K$ is a field. Consider a morphism
$$
\text{Spec}(K) \longrightarrow F.
$$
By the Yoneda Lemma this is given by an
element $p \in F(\text{Spec}(K))$. We say that two such
pairs $(\text{Spec}(K), p)$ and $(\text{Spec}(L), q)$
are {\it equivalent} if there exists
a third field $\Omega$ and a commutative diagram
$$
\xymatrix{
\text{Spec}(\Omega) \ar[r] \ar[d] &
\text{Spec}(L) \ar[d]^q \\
\text{Spec}(K) \ar[r]^p &
F.
}
$$
In other words, there are field extensions
$K \to \Omega$ and $L \to \Omega$ such that
$p$ and $q$ map to the same element
of $F(\text{Spec}(\Omega))$. We omit the verification that this
defines an equivalence relation.

\begin{definition}
\label{definition-points}
Let $S$ be a scheme. Let $X$ be an algebraic space over $S$.
A {\it point} of $X$ is an equivalence class of morphisms
from spectra of fields into $X$.
The set of points of $X$ is denoted $|X|$.
\end{definition}

\noindent
Note that if $f : X \to Y$ is a morphism of algebraic spaces
over $S$, then there is an induced map $|f| : |X| \to |Y|$ which
maps a representative $x : \text{Spec}(K) \to X$ to the representative
$f \circ x : \text{Spec}(K) \to Y$.

\begin{lemma}
\label{lemma-scheme-points}
Let $S$ be a scheme. Let $X$ be a scheme over $S$.
The points of $X$ as a scheme are in canonical 1-1 correspondence
with the points of $X$ as an algebraic space.
\end{lemma}

\begin{proof}
This is Schemes, Lemma \ref{schemes-lemma-characterize-points}.
\end{proof}

\begin{lemma}
\label{lemma-points-cartesian}
Let $S$ be a scheme. Let
$$
\xymatrix{
Z \times_Y X \ar[r] \ar[d] & X \ar[d] \\
Z \ar[r] & Y
}
$$
be a cartesian diagram of algebraic spaces. Then the map of sets
of points
$$
|Z \times_Y X|
\longrightarrow
|Z| \times_{|Y|} |X|
$$
is surjective.
\end{lemma}

\begin{proof}
Namely, suppose given fields $K$, $L$ and morphisms
$\text{Spec}(K) \to X$, $\text{Spec}(L) \to Z$, then the
assumption that they agree as elements of $|Y|$ means that
there is a common extension $K \subset M$ and $L \subset M$
such that
$\text{Spec}(M) \to \text{Spec}(K) \to X \to Y$ and
$\text{Spec}(M) \to \text{Spec}(L) \to Z \to Y$ agree.
And this is exactly the condition which says you get a
morphism $\text{Spec}(M) \to Z \times_Y X$.
\end{proof}

\begin{lemma}
\label{lemma-characterize-surjective}
Let $S$ be a scheme.
Let $X$ be an algebraic space over $S$.
Let $f : T \to X$ be a morphism from a scheme to $X$.
The following are equivalent
\begin{enumerate}
\item $f : T \to X$ is surjective (according to
Spaces, Definition \ref{spaces-definition-relative-representable-property}),
and
\item $|f| : |T| \to |X|$ is surjective.
\end{enumerate}
\end{lemma}

\begin{proof}
Assume (1). Let $x : \text{Spec}(K) \to X$ be a morphism
from the spectrum of a field into $X$. By assumption the morphism of
schemes $\text{Spec}(K) \times_X T \to \text{Spec}(K)$ is surjective.
Hence there exists a field extension $K \subset K'$ and a morphism
$\text{Spec}(K') \to \text{Spec}(K) \times_X T$ such that the left
square in the diagram
$$
\xymatrix{
\text{Spec}(K') \ar[r] \ar[d] &
\text{Spec}(K) \times_X T \ar[d] \ar[r] &
T \ar[d] \\
\text{Spec}(K) \ar@{=}[r] &
\text{Spec}(K) \ar[r]^-x & X
}
$$
is commutative. This shows that $|f| : |T| \to |X|$ is surjective.

\medskip\noindent
Assume (2). Let $Z \to X$ be a morphism where $Z$ is
a scheme. We have to show that the morphism of schemes $Z \times_X T \to T$
is surjective, i.e., that $|Z \times_X T| \to |Z|$ is surjective.
This follows from (2) and
Lemma \ref{lemma-points-cartesian}.
\end{proof}

\begin{lemma}
\label{lemma-points-presentation}
Let $S$ be a scheme.
Let $X$ be an algebraic space over $S$.
Let $X = U/R$ be a presentation of $X$, see
Spaces, Definition \ref{spaces-definition-presentation}.
Then the image of $|R| \to |U| \times |U|$ is an equivalence relation
and $|X|$ is the quotient of $|U|$ by this equivalence relation.
\end{lemma}

\begin{proof}
The assumption means that $U$ is a scheme, $p : U \to X$ is a surjective,
\'etale morphism, $R = U \times_X U$ is a scheme and defines an \'etale
equivalence relation on $U$ such that $X = U/R$ as sheaves. By
Lemma \ref{lemma-characterize-surjective}
we see that $|U| \to |X|$ is surjective. By
Lemma \ref{lemma-points-cartesian}
the map
$$
|R| \longrightarrow |U| \times_{|X|} |U|
$$
is surjective. Hence the image of $|R| \to |U| \times |U|$ is
exactly the set of pairs $(u_1, u_2) \in |U| \times |U|$
such that $u_1$ and $u_2$ have the same image in $|X|$.
Combining these two statements we get the result of the lemma.
\end{proof}

\begin{lemma}
\label{lemma-topology-points}
Let $S$ be a scheme. There exists a unique topology on the set of points
of algebraic spaces over $S$ with the following properties:
\begin{enumerate}
\item for every morphism of algebraic spaces $X \to Y$ over $S$
the map $|X| \to |Y|$ is continuous, and
\item for every \'etale morphism $U \to X$ with $U$ a scheme
the map of topological spaces $|U| \to |X|$ is continuous and open.
\end{enumerate}
\end{lemma}

\begin{proof}
Let $X$ be an algebraic space over $S$. Let $p : U \to X$ be a
surjective \'etale morphism where $U$ is a scheme over $S$.
We define $W \subset |X|$ is open if and only if $|p|^{-1}(W)$
is an open subset of $|U|$. This is a topology on $|X|$.

\medskip\noindent
Let us prove that the topology is independent of the choice of
the presentation. To do this it suffices to show that if $U'$ is a scheme,
and $U' \to X$ is an \'etale morphism, then the map $|U'| \to |X|$
(with topology on $|X|$ defined using $U \to X$ as above)
is open and continuous; which in addition will prove that (2) holds.
Set $U'' = U \times_X U'$, so that we have the commutative diagram
$$
\xymatrix{
U'' \ar[r] \ar[d] & U' \ar[d] \\
U \ar[r] & X
}
$$
As $U \to X$ and $U' \to X$ are \'etale we see that
both $U'' \to U$ and $U'' \to U'$ are \'etale morphisms of schemes.
Moreover, $U'' \to U'$ is surjective. Hence
we get a commutative diagram of maps of sets
$$
\xymatrix{
|U''| \ar[r] \ar[d] & |U'| \ar[d] \\
|U| \ar[r] & |X|
}
$$
The lower horizontal arrow is surjective (see
Lemma \ref{lemma-characterize-surjective}
or
Lemma \ref{lemma-points-presentation})
and continuous by definition of the topology on $|X|$.
The top horizontal arrow is surjective, continuous, and open by
Morphisms, Lemma \ref{morphisms-lemma-etale-open}.
The left vertical arrow is continuous and open (by
Morphisms, Lemma \ref{morphisms-lemma-etale-open}
again.) Hence it follows formally that the right vertical
arrow is continuous and open.

\medskip\noindent
To finish the proof we prove (1).
Let $a : X \to Y$ be a morphism of algebraic spaces. According to
Spaces, Lemma \ref{spaces-lemma-lift-morphism-presentations}
we can find a diagram
$$
\xymatrix{
U \ar[d]_p \ar[r]_\alpha & V \ar[d]^q \\
X \ar[r]^a & Y
}
$$
where $U$ and $V$ are schemes, and $p$ and $q$ are surjective and \'etale.
This gives rise to the diagram
$$
\xymatrix{
|U| \ar[d]_p \ar[r]_\alpha & |V| \ar[d]^q \\
|X| \ar[r]^a & |Y|
}
$$
where all but the lower horizontal arrows are known to be continuous and
the two vertical arrows are surjective and open. It follows that the
lower horizontal arrow is continuous as desired.
\end{proof}

\begin{definition}
\label{definition-topological-space}
Let $S$ be a scheme. Let $X$ be an algebraic space over $S$.
The underlying {\it topological space} of $X$ is the set of points
$|X|$ endowed with the topology constructed in
Lemma \ref{lemma-topology-points}.
\end{definition}

\noindent
It turns out that this topological space carries the same information
as the small Zariski site $X_{Zar}$ of
Spaces, Definition \ref{spaces-definition-small-Zariski-site}.

\begin{lemma}
\label{lemma-open-subspaces}
Let $S$ be a scheme.
Let $X$ be an algebraic space over $S$.
\begin{enumerate}
\item The rule $X' \mapsto |X'|$ defines an inclusion preserving
bijection between open subspaces $X'$ (see
Spaces, Definition \ref{spaces-definition-immersion})
of $X$, and opens of the topological space $|X|$.
\item A family $\{X_i \subset X\}_{i \in I}$ of open subspaces of $X$
is a Zariski covering (see
Spaces, Definition \ref{spaces-definition-Zariski-open-covering})
if and only if $|X| = \bigcup |X_i|$.
\end{enumerate}
In other words, the small Zariski site $X_{Zar}$ of $X$ is canonically
identified with a site associated to the topological space $|X|$ (see
Sites, Example \ref{sites-example-site-topological}).
\end{lemma}

\begin{proof}
In order to prove (1) let us construct the inverse of the rule.
Namely, suppose that $W \subset |X|$ is open. Choose a presentation
$X = U/R$ corresponding to the surjective \'etale map
$p : U \to X$ and \'etale maps $s, t : R \to U$.
By construction we see that $|p|^{-1}(W)$ is an
open of $U$. Denote $W' \subset U$ the corresponding open subscheme.
It is clear that $R' = s^{-1}(W') = t^{-1}(W')$ is a Zariski open
of $R$ which defines an \'etale equivalence relation on $W'$.
By Spaces, Lemma \ref{spaces-lemma-finding-opens} the morphism
$X' = W'/R' \to X$ is an open immersion. Hence $X'$ is an algebraic space
by Spaces, Lemma \ref{spaces-lemma-representable-over-space}.
By construction $|X'| = W$, i.e., $X'$ is a subspace of $X$
corresponding to $W$. Thus (1) is proved.

\medskip\noindent
To prove (2), note that if $\{X_i \subset X\}_{i \in I}$ is a collection
of open subspaces, then it is a Zariski covering if and only if the
$U = \bigcup U \times_X X_i$ is an open covering. This follows from
the definition of a Zariski covering and the fact that the morphism
$U \to X$ is surjective as a map of presheaves on $(\textit{Sch}/S)_{fppf}$.
On the other hand, we see that $|X| = \bigcup |X_i|$ if and only if
$U = \bigcup U \times_X X_i$ by Lemma \ref{lemma-points-presentation}
(and the fact that the projections $U \times_X X_i \to X_i$ are surjective
and \'etale). Thus the equivalence of (2) follows.
\end{proof}

\begin{lemma}
\label{lemma-factor-through-open-subspace}
Let $S$ be a scheme.
Let $X$, $Y$ be algebraic spaces over $S$.
Let $X' \subset X$ be an open subspace.
Let $f : Y \to X$ be a morphism of algebraic spaces over $S$.
Then $f$ factors through $X'$ if and only if $|f| : |Y| \to |X|$
factors through $|X'| \subset |X|$.
\end{lemma}

\begin{proof}
By Spaces, Lemma \ref{spaces-lemma-base-change-immersions}
we see that $Y' = Y \times_X X' \to Y$ is an open immersion.
If $|f|(|Y|) \subset |X'|$, then clearly $|Y'| = |Y|$. Hence $Y' = Y$ by
Lemma \ref{lemma-open-subspaces}.
\end{proof}

\begin{lemma}
\label{lemma-points-monomorphism}
Let $S$ be a scheme. Let $X$ be an algebraic space over $S$.
Consider the map
$$
\{\text{Spec}(k) \to X \text{ monomorphism}\}
\longrightarrow
|X|
$$
This map is injective.
\end{lemma}

\begin{proof}
Suppose that $\varphi_i : \text{Spec}(k_i) \to X$ are monomorphisms
for $i = 1, 2$. If $\varphi_1$ and $\varphi_2$ define the same point
of $|X|$, then we see that the scheme
$$
Y = \text{Spec}(k_1) \times_{\varphi_1, X, \varphi_2} \text{Spec}(k_2)
$$
is nonempty. Since the base change of a monomorphism is a monomorphism
this means that the projection morphisms $Y \to \text{Spec}(k_i)$
are monomorphisms. Hence $\text{Spec}(k_1) = Y = \text{Spec}(k_2)$
as schemes over $X$, see
Schemes, Lemma \ref{schemes-lemma-mono-towards-spec-field}.
We conclude that $\varphi_1 = \varphi_2$, which proves the lemma.
\end{proof}

\noindent
We will see in
Lemma \ref{lemma-very-reasonable-points-monomorphism}
that this map is a bijection when $X$ is very reasonable.

















\section{Quasi-compact spaces}
\label{section-quasi-compact}

\begin{definition}
\label{definition-quasi-compact}
Let $S$ be a scheme.
Let $X$ be an algebraic space over $S$.
We say $X$ is {\it quasi-compact} if there exists a surjective
\'etale morphism $U \to X$ with $U$ quasi-compact.
\end{definition}

\begin{lemma}
\label{lemma-quasi-compact-space}
Let $S$ be a scheme.
Let $X$ be an algebraic space over $S$.
Then $X$ is quasi-compact if and only if $|X|$ is quasi-compact.
\end{lemma}

\begin{proof}
Choose a scheme $U$ and an \'etale surjective morphism $U \to X$.
We will use Lemma \ref{lemma-characterize-surjective}.
If $U$ is quasi-compact, then since $|U| \to |X|$ is surjective
we conclude that $|X|$ is quasi-compact.
If $|X|$ is quasi-compact, then since $|U| \to |X|$ is open
we see that there exists a quasi-compact open $U' \subset U$
such that $|U'| \to |X|$ is surjective (and still \'etale).
Hence we win.
\end{proof}

\begin{lemma}
\label{lemma-finite-disjoint-quasi-compact}
A finite disjoint union of quasi-compact algebraic spaces is
a quasi-compact algebraic space.
\end{lemma}

\begin{proof}
This is clear from
Lemma \ref{lemma-quasi-compact-space}
and the corresponding topological fact.
\end{proof}

\begin{example}
\label{example-quasi-compact-not-very-reasonable}
The space $[\mathbf{A}^1_{\mathbf{Q}}/\mathbf{Z}]$ is
quasi-compact but not very reasonable.
\end{example}

\begin{lemma}
\label{lemma-space-locally-quasi-compact}
Let $S$ be a scheme.
Let $X$ be an algebraic space over $S$.
Every point of $|X|$ has a fundamental system of open
quasi-compact neighbourhoods.
In particular $|X|$ is locally quasi-compact in the sense of
Topology, Definition \ref{topology-definition-locally-quasi-compact}.
\end{lemma}

\begin{proof}
This follows formally from the fact that there exists a scheme
$U$ and a surjective, open, continuous map
$U \to |X|$ of topological spaces. To be a bit more precise, if
$u \in U$ maps to $x \in |X|$, then the images of the affine
neighbourhoods of $u$ will give a fundamental system of quasi-compact
open neighbourhoods of $x$.
\end{proof}






\section{Properties of spaces defined by properties of schemes}
\label{section-types-properties}

\noindent
Any \'etale local property of schemes gives rise to a corresponding
property of algebraic spaces via the following lemma.

\begin{lemma}
\label{lemma-type-property}
Let $S$ be a scheme.
Let $X$ be an algebraic space over $S$.
Let $\mathcal{P}$ be a property of schemes which is local in the \'etale
topology, see
Descent, Definition \ref{descent-definition-property-local}.
The following are equivalent
\begin{enumerate}
\item for some scheme $U$ and surjective \'etale morphism $U \to X$
the scheme $U$ has property $\mathcal{P}$, and
\item for every scheme $U$ and every \'etale morphism $U \to X$
the scheme $U$ has property $\mathcal{P}$.
\end{enumerate}
If $X$ is representable this is equivalent to $\mathcal{P}(X)$.
\end{lemma}

\begin{proof}
Omitted.
\end{proof}

\begin{definition}
\label{definition-type-property}
Let $\mathcal{P}$ be a property of schemes which is 
local in the \'etale topology.
Let $S$ be a scheme.
Let $X$ be an algebraic space over $S$.
We say $X$ {\it has property $\mathcal{P}$}
if any of the equivalent conditions of
Lemma \ref{lemma-type-property}
hold.
\end{definition}

\begin{remark}
\label{remark-list-properties-local-etale-topology}
Here is a list of properties which are local for the \'etale topology
(keep in mind that the fpqc, fppf, syntomic, and smooth topologies are
stronger than the \'etale topology):
\begin{enumerate}
\item locally Noetherian, see
Descent, Lemma \ref{descent-lemma-Noetherian-local-fppf},
\item Jacobson, see
Descent, Lemma \ref{descent-lemma-Jacobson-local-fppf},
\item locally Noetherian and $(S_k)$, see
Descent, Lemma \ref{descent-lemma-Sk-local-syntomic},
\item Cohen-Macaulay, see
Descent, Lemma \ref{descent-lemma-CM-local-syntomic},
\item reduced, see
Descent, Lemma \ref{descent-lemma-reduced-local-smooth},
\item normal, see
Descent, Lemma \ref{descent-lemma-normal-local-smooth},
\item locally Noetherian and $(R_k)$, see
Descent, Lemma \ref{descent-lemma-Rk-local-smooth},
\item regular, see
Descent, Lemma \ref{descent-lemma-regular-local-smooth},
\item Nagata, see
Descent, Lemma \ref{descent-lemma-Nagata-local-smooth}.
\end{enumerate}
\end{remark}

\noindent
Any \'etale local property of germs of schemes gives rise to a corresponding
property of algebraic spaces. Here is the obligatory lemma.

\begin{lemma}
\label{lemma-local-source-target-at-point}
Let $\mathcal{P}$ be a property of germs of schemes which is \'etale local, see
Descent, Definition \ref{descent-definition-local-at-point}.
Let $S$ be a scheme.
Let $X$ be an algebraic space over $S$.
Let $x \in |X|$ be a point of $X$.
Consider \'etale morphisms $a : U \to X$ where $U$ is a scheme.
The following are equivalent
\begin{enumerate}
\item for any $U \to X$ as above and $u \in U$ with $a(u) = x$ we have
$\mathcal{P}(U, u)$, and
\item for some $U \to X$ as above and $u \in U$ with $a(u) = x$ we have
$\mathcal{P}(U, u)$.
\end{enumerate}
If $X$ is representable, then this is equivalent to $\mathcal{P}(X, x)$.
\end{lemma}

\begin{proof}
Omitted.
\end{proof}

\begin{definition}
\label{definition-property-at-point}
Let $S$ be a scheme. Let $X$ be an algebraic space over $S$.
Let $x \in |X|$. Let $\mathcal{P}$ be a property of germs of schemes which is 
\'etale local.
We say $X$ {\it has property $\mathcal{P}$ at $x$} if any of the
equivalent conditions of
Lemma \ref{lemma-local-source-target-at-point}
hold.
\end{definition}





\section{Dimension at a point}
\label{section-dimension}

\noindent
We can use
Descent, Lemma \ref{descent-lemma-dimension-at-point-local}
to define the dimension of an algebraic
space $X$ at a point $x$. This will give us a different notion than the
topological one (i.e., the dimension of $|X|$ at $x$).

\begin{definition}
\label{definition-dimension-at-point}
Let $S$ be a scheme.
Let $X$ be an algebraic space over $S$.
Let $x \in |X|$ be a point of $X$.
We define the {\it dimension of $X$ at $x$} to be
the element $\dim_x(X) \in \{0, 1, 2, \ldots, \infty\}$
such that $\dim_x(X) = \dim_u(U)$ for any (equivalently some)
pair $(a : U \to X, u)$ consisting of an \'etale morphism $a : U \to X$
from a scheme to $X$ and a point $u \in U$ with $a(u) = x$.
See
Definition \ref{definition-property-at-point},
Lemma \ref{lemma-local-source-target-at-point}, and
Descent, Lemma \ref{descent-lemma-dimension-at-point-local}.
\end{definition}

\noindent
Warning: It is {\bf not} the case that $\dim_x(X) = \dim_x(|X|)$
in general. A counter example is the algebraic space $X$ of
Spaces, Example \ref{spaces-example-infinite-product}.
Namely, in this example we have $\dim_x(X) = 0$ and
$\dim_x(|X|) = 1$ (this holds for any $x \in |X|$).
In particular, it also means that the dimension of $X$ (as defined
below) is different from the dimension of $|X|$.

\begin{definition}
\label{definition-dimension}
Let $S$ be a scheme. Let $X$ be an algebraic space over $S$.
The {\it dimension} $\dim(X)$ of $X$ is defined by the rule
$$
\dim(X) = \sup\nolimits_{x \in |X|} \dim_x(X)
$$
\end{definition}

\noindent
By
Properties, Lemma \ref{properties-lemma-dimension}
we see that this is the usual notion if $X$ is a scheme.
There is another integer that measures the dimension of a scheme
at a point, namely the dimension of the local ring. This invariant
is compatible with \'etale morphisms also, see
Section \ref{section-dimension-local-ring}.





\section{Reduced spaces}
\label{section-reduced}

\noindent
We have already defined reduced algebraic spaces in
Section \ref{section-types-properties}.
Here we just prove some simple lemmas regarding reduced algebraic
spaces.

\begin{lemma}
\label{lemma-reduced-closed-subspace}
Let $S$ be a scheme.
Let $X$ be an algebraic space over $S$.
Let $T \subset |X|$ be a closed subset.
There exists a unique closed subspace $Z \subset X$ with
the following properties: (a) we have $|Z| = T$, and (b) $Z$ is reduced.
\end{lemma}

\begin{proof}
Let $U \to X$ be a surjective \'etale morphism, where $U$ is a scheme.
Set $R = U \times_X U$, so that $X = U/R$, see
Spaces, Lemma \ref{spaces-lemma-space-presentation}.
As usual we denote $s, t : R \to U$ the two projection morphisms.
By Lemma \ref{lemma-points-presentation}
we see that $T$ corresponds to a closed subset $T' \subset |U|$ such
that $s^{-1}(T') = t^{-1}(T')$.
Let $Z' \subset U$ be the reduced induced scheme structure on $T'$.
In this case the fibre products
$Z' \times_{U, t} R$ and $Z' \times_{U, s} R$ are closed subschemes
of $R$
(Schemes, Lemma \ref{schemes-lemma-base-change-immersion})
which are \'etale over $Z'$
(Morphisms, Lemma \ref{morphisms-lemma-base-change-etale}),
and hence reduced
(because being reduced is local in the \'etale topology, see
Remark \ref{remark-list-properties-local-etale-topology}).
Since they have the same underlying topological space (see above)
we conclude that $Z' \times_{U, t} R = Z' \times_{U, s} R$.
Hence the common value $R'$ is the restriction of $R$ to $Z'$, see
Groupoids, Definition \ref{groupoids-definition-restrict-groupoid}. By
Spaces, Theorem \ref{spaces-theorem-presentation} we see that
$Z = Z'/R'$ is an algebraic space. By
Groupoids, Lemma \ref{groupoids-lemma-quotient-groupoid-restrict}
we see that $Z \to X$ is a monomorphism. By construction we have
$U \times_X Z = Z'$, so $U \times_X Z \to Z$ is a closed immersion.
This means all the hypotheses of
Spaces,
Lemma \ref{spaces-lemma-morphism-sheaves-with-P-effective-descent-etale}
are satisfied
for the transformation $Z \to X$, $\mathcal{P}=$``closed immersion'' (closed
immersions satisfy descent for \'etale coverings, see
Descent, Lemma \ref{descent-lemma-closed-immersion}),
and the \'etale surjective morphism $U \to X$. We conclude that $Z \to X$
is representable, a monomorphism and a closed immersion, which is the
definition of a closed subspace (see
Spaces, Definition \ref{spaces-definition-immersion}). By construction
$|Z| = T$ and $Z$ is reduced. This proves existence. We omit the proof
of uniqueness.
\end{proof}

\begin{lemma}
\label{lemma-map-into-reduction}
Let $S$ be a scheme.
Let $X$, $Y$ be algebraic spaces over $S$.
Let $Z \subset X$ be a closed subspace.
Assume $Y$ is reduced.
A morphism $f : Y \to X$ factors through $Z$ if and only if
$f(|Y|) \subset |Z|$.
\end{lemma}

\begin{proof}
Assume $f(|Y|) \subset |Z|$. Choose a diagram
$$
\xymatrix{
V \ar[d]_b \ar[r]_h & U \ar[d]^a \\
Y \ar[r]^f & X
}
$$
where $U$, $V$ are schemes, and the vertical arrows are surjective and
\'etale. The scheme $V$ is reduced, see
Lemma \ref{lemma-type-property}.
Hence $h$ factors through $a^{-1}(Z)$ by
Schemes, Lemma \ref{schemes-lemma-map-into-reduction}.
So $a \circ h$ factors through $Z$.
As $Z \subset X$ is a subsheaf, and $V \to Y$ is a surjection of sheaves
on $(\textit{Sch}/S)_{fppf}$ we conclude that $X \to Y$ factors
through $Z$.
\end{proof}

\begin{definition}
\label{definition-reduced-induced-space}
Let $S$ be a scheme, and let $X$ be an algebraic space over $S$.
Let $Z \subset |X|$ be a closed subset.
An {\it algebraic space structure on $Z$} is given by a closed subspace
$Z'$ of $X$ with $|Z'|$ equal to $Z$.
The {\it reduced induced algebraic space structure}
on $Z$ is the one constructed in
Lemma \ref{lemma-reduced-closed-subspace}.
The {\it reduction $X_{red}$ of $X$} is the reduced induced algebraic
space structure on $|X|$.
\end{definition}






\section{Universally bounded fibres}
\label{section-universally-bounded}

\noindent
We briefly discuss what it means for a morphism from a scheme to an
algebraic space to have universally bounded fibres. Please refer to
Morphisms, Section \ref{morphisms-section-universally-bounded}
for similar definitions and results on morphisms of schemes.

\begin{definition}
\label{definition-universally-bounded}
Let $S$ be a scheme. Let $X$ be an algebraic space over $S$, and
let $U$ be a scheme over $S$. Let $f : U \to X$ be a morphism over $S$.
We say the {\it fibres of $f$ are universally bounded}\footnote{This is
probably nonstandard notation.}
if there exists an integer $n$ such that for all fields
$k$ and all morphisms $\text{Spec}(k) \to X$ the fibre
product $\text{Spec}(k) \times_X U$ is a finite scheme over $k$
whose degree over $k$ is $\leq n$.
\end{definition}

\noindent
This definition makes sense because the fibre product
$\text{Spec}(k) \times_Y X$ is a scheme. Moreover, if $Y$ is a scheme
we recover the notion of
Morphisms, Definition \ref{morphisms-definition-universally-bounded}
by virtue of
Morphisms, Lemma \ref{morphisms-lemma-characterize-universally-bounded}.

\begin{lemma}
\label{lemma-composition-universally-bounded}
Let $S$ be a scheme. Let $X$ be an algebraic space over $S$.
Let $V \to U$ be a morphism of schemes over $S$, and let
$U \to X$ be a morphism from $U$ to $X$. If the fibres of
$V \to U$ and $U \to X$ are universally bounded, then so
are the fibres of $V \to X$.
\end{lemma}

\begin{proof}
Let $n$ be an integer which works for $V \to U$, and let $m$ be
an integer which works for $U \to X$ in
Defintion \ref{definition-universally-bounded}.
Let $\text{Spec}(k) \to X$ be a morphism, where $k$ is a field.
Consider the morphisms
$$
\text{Spec}(k) \times_X V
\longrightarrow
\text{Spec}(k) \times_X U
\longrightarrow
\text{Spec}(k).
$$
By assumption the scheme $\text{Spec}(k) \times_X U$
is finite of degree at most $m$ over $k$, and $n$ is an integer which
bounds the degree of the fibres of the first morphism. Hence by
Morphisms, Lemma \ref{morphisms-lemma-composition-universally-bounded}
we conclude that $\text{Spec}(k) \times_X V$ is finite over $k$
of degree at most $nm$.
\end{proof}

\begin{lemma}
\label{lemma-base-change-universally-bounded}
Let $S$ be a scheme.
Let $Y \to X$ be a representable morphism of algebraic spaces over $S$.
Let $U \to X$ be a morphism from a scheme to $X$.
If the fibres of $U \to X$ are universally bounded, then the fibres
of $U \times_X Y \to Y$ are universally bounded.
\end{lemma}

\begin{proof}
This is clear from the definition, and properties of fibre products.
(Note that $U \times_X Y$ is a scheme
as we assumed $Y \to X$ representable, so the definition applies.)
\end{proof}

\begin{lemma}
\label{lemma-descent-universally-bounded}
Let $S$ be a scheme. Let $g : Y \to X$ be a representable morphism of
algebraic spaces over $S$. Let $f : U \to X$ be a morphism from a scheme
towards $X$. Let $f' : U \times_X Y \to Y$ be the base change of $f$.
If
$$
\text{Im}(|f| : |U| \to |X|) \subset \text{Im}(|g| : |Y| \to |X|)
$$
and $f'$ has universally bounded fibres, then $f$ has universally
bounded fibres.
\end{lemma}

\begin{proof}
Let $n \geq 0$ be an integer bounding the degrees of the fibre
products $\text{Spec}(k) \times_Y (U \times_X Y)$ as in
Definition \ref{definition-universally-bounded} for the morphism $f'$.
We claim that $n$ works for $f$ also. Namely, suppose that
$x : \text{Spec}(k) \to X$ is a morphism from the spectrum of
a field. Then either $\text{Spec}(k) \times_X U$ is empty (and there
is nothing to prove), or $x$ is in the image of $|f|$. By
Lemma \ref{lemma-points-cartesian} and the assumption of the lemma we see
that this means there exists a field extension $k \subset k'$ and a
commutative diagram
$$
\xymatrix{
\text{Spec}(k') \ar[r] \ar[d] & Y \ar[d] \\
\text{Spec}(k) \ar[r] & X
}
$$
Hence we see that
$$
\text{Spec}(k') \times_Y (U \times_X Y) =
\text{Spec}(k') \times_{\text{Spec}(k)} (\text{Spec}(k) \times_X U)
$$
Since the scheme $\text{Spec}(k') \times_Y (U \times_X Y)$ is assumed finite
of degree $\leq n$ over $k'$ it follows that also $\text{Spec}(k) \times_X U$
is finite of degree $\leq n$ over $k$ as desired. (Some details omitted.)
\end{proof}

\begin{lemma}
\label{lemma-universally-bounded-permanence}
Let $S$ be a scheme. Let $X$ be an algebraic space over $S$.
Consider a commutative diagram
$$
\xymatrix{
U \ar[rd]_g \ar[rr]_{f} & & V \ar[ld]^h \\
& X &
}
$$
where $U$ and $V$ are schemes. If $g$ has universally bounded fibres,
and $f$ is surjective and flat, then also $h$ has universally bounded fibres.
\end{lemma}

\begin{proof}
Assume $g$ has universally bounded fibres, and $f$ is surjective and flat.
Say $n \geq 0$. is an integer which bounds the degrees of the schemes
$\text{Spec}(k) \times_X U$ as in
Definition \ref{definition-universally-bounded}.
We claim $n$ also works for $h$.
Let $\text{Spec}(k) \to X$ be a morphism from the spectrum of a
field to $X$. Consider the morphism of schemes
$$
\text{Spec}(k) \times_X V \longrightarrow \text{Spec}(k) \times_X U
$$
It is flat and surjective. By assumption the scheme
on the left is finite of degree $\leq n$ over $\text{Spec}(k)$.
It follows from
Morphisms, Lemma \ref{morphisms-lemma-universally-bounded-permanence}
that the degree of the scheme on the right is also bounded by $n$
as desired.
\end{proof}

\begin{lemma}
\label{lemma-universally-bounded-finite-fibres}
Let $S$ be a scheme.
Let $X$ be an algebraic space over $S$, and let $U$ be a scheme over $S$.
Let $\varphi : U \to X$ be a morphism over $S$.
If the fibres of $\varphi$ are universally bounded, then there exists an
integer $n$ such that each fibre of $|U| \to |X|$ has at most
$n$ elements.
\end{lemma}

\begin{proof}
The integer $n$ of Definition \ref{definition-universally-bounded} works.
Namely, pick $x \in |X|$. Represent $x$ by a morphism
$x : \text{Spec}(k) \to X$. Then we get a commutative diagram
$$
\xymatrix{
\text{Spec}(k) \times_X U \ar[r] \ar[d] & U \ar[d] \\
\text{Spec}(k) \ar[r]^x & X
}
$$
which shows (via Lemma \ref{lemma-points-cartesian})
that the inverse image of $x$ in $|U|$ is the image of
the top horizontal arrow. Since $\text{Spec}(k) \times_X U$ is finite
of degree $\leq n$ over $k$ it has at most $n$ points.
\end{proof}








\section{Finiteness conditions and points}
\label{section-points-monomorphisms}

\noindent
In this section we elaborate on the question of when points can be represented
by monomorphisms from spectra of fields into the space.

\begin{remark}
\label{remark-recall}
Before we give the proof of the next lemma let us recall some facts
about \'etale morphisms of schemes:
\begin{enumerate}
\item An \'etale morphism is flat and hence generalizations lift along
an \'etale morphism
(Morphisms, Lemmas \ref{morphisms-lemma-etale-flat}
and \ref{morphisms-lemma-generalizations-lift-flat}).
\item An \'etale morphism is unramified, an unramified morphism is locally
quasi-finite, hence fibres are discrete
(Morphisms, Lemmas \ref{morphisms-lemma-flat-unramified-etale},
\ref{morphisms-lemma-unramified-quasi-finite}, and
\ref{morphisms-lemma-quasi-finite-at-point-characterize}).
\item A quasi-compact \'etale morphism is quasi-finite and in particular
has finite fibres
(Morphisms, Lemmas \ref{morphisms-lemma-quasi-finite-locally-quasi-compact} and
\ref{morphisms-lemma-quasi-finite}).
\item An \'etale scheme over a field $k$ is a disjoint union of spectra
of finite separable field extension of $k$
(Morphisms, Lemma \ref{morphisms-lemma-etale-over-field}).
\end{enumerate}
For a general discussion of \'etale morphisms, please see
\'Etale Morphisms of Schemes, Section \ref{etale-section-etale-morphisms}.
\end{remark}

\noindent
The following lemma is the key lemma which we will use to prove that
certain algebraic spaces are isomorphic to the spectrum of a field.

\begin{lemma}
\label{lemma-point-like-spaces}
Let $S$ be a scheme. Let $k$ be a field.
Let $X$ be an algebraic space over $S$ and assume that there exists
a surjective \'etale morphism $\text{Spec}(k) \to X$.
If $X$ is quasi-separated, then $X \cong \text{Spec}(k')$
where $k' \subset k$ is a finite separable extension.
\end{lemma}

\begin{proof}
Set $R = \text{Spec}(k) \times_X \text{Spec}(k)$, so that we have a
fibre product diagram
$$
\xymatrix{
R \ar[r]_-s \ar[d]_-t & \text{Spec}(k) \ar[d] \\
\text{Spec}(k) \ar[r] & X
}
$$
By Spaces, Lemma \ref{spaces-lemma-space-presentation}
we know $X = \text{Spec}(k)/R$ is the quotient sheaf.
Because $\text{Spec}(k) \to X$ is \'etale, the morphisms $s$ and $t$ are
\'etale. Hence $R = \coprod_{i \in I} \text{Spec}(k_i)$ is a disjoint
union of spectra of fields, and both $s$ and $t$
induce finite separable field extensions $s, t : k \subset k_i$,
see Remark \ref{remark-recall}. Because
$$
R = \text{Spec}(k) \times_X \text{Spec}(k)
= (\text{Spec}(k) \times_S \text{Spec}(k)) \times_{X \times_S X, \Delta} X
$$
and since $\Delta$ is quasi-compact by assumption we conclude that
$R \to \text{Spec}(k) \times_S \text{Spec}(k)$ is quasi-compact.
Hence $R$ is quasi-compact as $\text{Spec}(k) \times_S \text{Spec}(k)$ is
affine. We conclude that $I$ is finite. This implies
that $s$ and $t$ are finite locally free morphisms. Hence by
Groupoids, Proposition \ref{groupoids-proposition-finite-flat-equivalence}
we conclude that $\text{Spec}(k)/R$ is
represented by $\text{Spec}(k')$, with $k' \subset k$ finite locally free
where
$$
k' = \{x \in k \mid s_i(x) = t_i(x)\text{ for all }i \in I\}
$$
It is easy to see that $k'$ is a field.
\end{proof}

\begin{remark}
\label{remark-cannot-decide-yet}
It is possible that
Lemma \ref{lemma-point-like-spaces}
also holds when $X$ is locally separated. To prove this one would
have to show that the index set $I$ in the proof of
Lemma \ref{lemma-point-like-spaces}
is finite, if we only assume that
$R \to \text{Spec}(k) \times_S \text{Spec}(k)$ is an immersion (and an \'etale
equivalence relation of course).
\end{remark}

\begin{lemma}
\label{lemma-U-finite-above-x}
Let $S$ be a scheme. Let $X$ be an algebraic space over $S$.
Let $x \in |X|$. The following are equivalent:
\begin{enumerate}
\item there exists a family of schemes $U_i$ and
\'etale morphisms $\varphi_i : U_i \to X$ such that
$\coprod \varphi_i : \coprod U_i \to X$ is surjective,
and such that for each $i$ the fibre of
$|U_i| \to |X|$ over $x$ is finite, and
\item for every affine scheme $U$ and \'etale morphism $\varphi : U \to X$
the fibre of $|U| \to |X|$ over $x$ is finite.
\end{enumerate}
\end{lemma}

\begin{proof}
The implication (2) $\Rightarrow$ (1) is trivial.
Let $\varphi_i : U_i \to X$ be a family of \'etale morphisms as in (1).
Let $\varphi : U \to X$ be an \'etale morphism from a scheme
towards $X$. Consider the fibre product diagrams
$$
\xymatrix{
U \times_X U_i \ar[r]_-{p_i} \ar[d]_{q_i} & U_i \ar[d]^{\varphi_i} \\
U \ar[r]^\varphi & X
}
\quad \quad
\xymatrix{
\coprod U \times_X U_i \ar[r]_-{\coprod p_i} \ar[d]_{\coprod q_i} &
\coprod U_i \ar[d]^{\coprod \varphi_i} \\
U \ar[r]^\varphi & X
}
$$
Since $q_i$ is \'etale it is open (see Remark \ref{remark-recall}).
Moreover, the morphism $\coprod q_i$ is surjective.
Hence there exist finitely many indices $i_1, \ldots, i_n$ and
a quasi-compact opens $W_{i_j} \subset U \times_X U_{i_j}$
which surject onto $U$.
The morphism $p_i$ is \'etale, hence locally quasi-finite (see remark on
\'etale morphisms above). Thus we may apply
Morphisms, Lemma
\ref{morphisms-lemma-locally-quasi-finite-qc-source-universally-bounded}
to see the fibres of $p_{i_j}|_{W_{i_j}} : W_{i_j} \to U_i$ are finite.
Hence by
Lemma \ref{lemma-points-cartesian}
and the assumption on $\varphi_i$ we conclude that the fibre 
of $\varphi$ over $x$ is finite. In other words (2) holds.
\end{proof}

\begin{lemma}
\label{lemma-U-universally-bounded}
Let $S$ be a scheme. Let $X$ be an algebraic space over $S$.
The following are equivalent:
\begin{enumerate}
\item there exist schemes $U_i$ and \'etale morphisms
$U_i \to X$ such that $\coprod U_i \to X$ is surjective and
each $U_i \to X$ has universally bounded fibres, and
\item for every affine scheme $U$ and \'etale morphism $\varphi : U \to X$
the fibres of $U \to X$ are universally bounded.
\end{enumerate}
\end{lemma}

\begin{proof}
The implication (2) $\Rightarrow$ (1) is trivial.
Assume (1). Let $(\varphi_i : U_i \to X)_{i \in I}$ be a collection of
\'etale morphisms from schemes towards $X$, covering $X$, such that
each $\varphi_i$ has universally bounded fibres.
Let $\psi : U \to X$ be an \'etale morphism from an affine scheme towards $X$.
For each $i$ consider the fibre product diagram
$$
\xymatrix{
U \times_X U_i \ar[r]_{p_i} \ar[d]_{q_i} & U_i \ar[d]^{\varphi_i} \\
U \ar[r]^\psi & X
}
$$
Since $q_i$ is \'etale it is open (see Remark \ref{remark-recall}).
Moreover, we have $U = \bigcup \text{Im}(q_i)$, since the family
$(\varphi_i)_{i \in I}$ is surjective. Since $U$ is affine, hence quasi-compact
we can finite finitely many $i_1, \ldots, i_n \in I$ and quasi-compact
opens $W_j \subset U \times_X U_{i_j}$ such that
$U = \bigcup p_{i_j}(W_j)$.
The morphism $p_{i_j}$ is \'etale, hence locally quasi-finite
(see remark on \'etale morphisms above). Thus we may apply
Morphisms, Lemma
\ref{morphisms-lemma-locally-quasi-finite-qc-source-universally-bounded}
to see the fibres of $p_{i_j}|_{W_j} : W_j \to U_{i_j}$ are universally
bounded. Hence by
Lemma \ref{lemma-composition-universally-bounded}
we see that the fibres of $W_j \to X$ are universally bounded.
Thus also $\coprod_{j = 1, \ldots, n} W_j \to X$ has universally
bounded fibres. Since $\coprod_{j = 1, \ldots, n} W_j \to X$ factors
through the surjective \'etale map
$\coprod q_{i_j}|_{W_j} : \coprod_{j = 1, \ldots, n} W_j \to U$ we
see that the fibres of $U \to X$ are universally bounded by
Lemma \ref{lemma-universally-bounded-permanence}.
In other words (2) holds.
\end{proof}

\begin{lemma}
\label{lemma-finite-fibres-presentation}
Let $S$ be a scheme.
Let $X$ be an algebraic space over $S$.
Let $U$ be a scheme. Let $\varphi : U \to X$ be an \'etale morphism such that
the projections $R = U \times_X U \to U$ are quasi-compact; for example if
$\varphi$ is quasi-compact. Then the fibres of
$$
|U| \to |X|
\quad\text{and}\quad
|R| \to |X|
$$
are finite.
\end{lemma}

\begin{proof}
Denote $R = U \times_X U$, and $s, t : R \to U$ the projections.
Let $u \in U$ be a point, and let $x \in |X|$ be its image.
The fibre of $|U| \to |X|$ over $x$ is equal to
$s(t^{-1}(\{u\}))$ by Lemma \ref{lemma-points-cartesian}, and
the fibre of $|R| \to |X|$ over $x$ is $t^{-1}(s(t^{-1}(\{u\})))$.
Since $t : R \to U$ is \'etale and quasi-compact, it has finite fibres
(as its fibres are disjoint unions of spectra of fields by
Morphisms, Lemma \ref{morphisms-lemma-etale-over-field}
and quasi-compact; see also Remark \ref{remark-recall}). Hence we win.
\end{proof}

\begin{lemma}
\label{lemma-R-finite-above-x}
Let $S$ be a scheme. Let $X$ be an algebraic space over $S$.
Let $x \in |X|$. The following are equivalent:
\begin{enumerate}
\item there exists a scheme $U$, an \'etale morphism
$\varphi : U \to X$, and points $u, u' \in U$ mapping to
$x$ such that setting $R = U \times_X U$ the fibre of
$$
|R| \to |U| \times_{|X|} |U|
$$
over $(u, u')$ is finite,
\item for every scheme $U$, \'etale morphism $\varphi : U \to X$ and
any points $u, u' \in U$ mapping to
$x$ setting $R = U \times_X U$ the fibre of
$$
|R| \to |U| \times_{|X|} |U|
$$
over $(u, u')$ is finite,
\item there exists a morphism $\text{Spec}(k) \to X$ with $k$ a field
in the equivalence class of $x$ such that the projections
$\text{Spec}(k) \times_X \text{Spec}(k) \to \text{Spec}(k)$ are
\'etale and quasi-compact, and
\item there exists a monomorphism $\text{Spec}(k) \to X$ with $k$ a field
in the equivalence class of $x$.
\end{enumerate}
\end{lemma}

\begin{proof}
Assume (1), i.e., let $\varphi : U \to X$ be an \'etale morphism from a scheme
towards $X$, and let $u, u'$ be points of $U$ lying over $x$
such that the fibre of $|R| \to |U| \times_{|X|} |U|$ over $(u, u')$
is a finite set. In this proof we think of a point $u = \text{Spec}(\kappa(u))$
as a scheme. Note that $u \to U$, $u' \to U$ are monomorphisms (see
Schemes, Lemma \ref{schemes-lemma-injective-points-surjective-stalks}),
hence $u \times_X u' \to R = U \times_X U$ is a monomorphism.
In this language the assumption really means that
$u \times_X u'$ is a scheme whose underlying topological space has
finitely many points.
Let $\psi : W \to X$ be an \'etale morphism from a scheme towards $X$.
Let $w, w' \in W$ be points of $W$ mapping to $x$.
We have to show that $w \times_X w'$ is a scheme whose underlying topological
space has finitely many points.
Consider the fibre product diagram
$$
\xymatrix{
W \times_X U \ar[r]_p \ar[d]_q & U \ar[d]^\varphi \\
W \ar[r]^\psi & X
}
$$
As $x$ is the image of $u$ and $u'$ we may pick points
$\tilde w, \tilde w'$ in $W \times_X U$ with $q(\tilde w) = w$,
$q(\tilde w') = w'$, $u = p(\tilde w)$ and $u' = p(\tilde w')$, see
Lemma \ref{lemma-points-cartesian}. As $p$, $q$ are \'etale the field extensions
$\kappa(w) \subset \kappa(\tilde w) \supset \kappa(u)$ and
$\kappa(w') \subset \kappa(\tilde w') \supset \kappa(u')$ are
finite separable, see Remark \ref{remark-recall}.
Then we get a commutative diagram
$$
\xymatrix{
w \times_X w' \ar[d] &
\tilde w \times_X \tilde w' \ar[l] \ar[d] \ar[r] &
u \times_X u' \ar[d] \\
w \times_X w' &
\tilde w \times_S \tilde w' \ar[l] \ar[r] &
u \times_S u'
}
$$
where the squares are fibre product squares. The lower horizontal
morpisms are \'etale and quasi-compact, as any scheme of the form
$\text{Spec}(k) \times_S \text{Spec}(k')$ is affine, and by our
observations about the field extensions above.
Thus we see that the top horizontal arrows are \'etale and quasi-compact
and hence have finite fibres.
We have seen above that $|u \times_X u'|$ is finite, so we conclude that
$|w \times_X w'|$ is finite. In other words, (2) holds.

\medskip\noindent
Assume (2). Let $U \to X$ be an \'etale morphism from a scheme $U$
such that $x$ is in the image of $|U| \to |X|$. Let $u \in U$ be
a point mapping to $x$. Then we have seen in the previous
paragraph that $u = \text{Spec}(\kappa(u)) \to X$ has the property that
$u \times_X u$ has a finite underlying topological space. On the other
hand, the projection maps $u \times_X u \to u$ are the composition
$$
u \times_X u \longrightarrow
u \times_X U \longrightarrow
u \times_X X = u,
$$
i.e., the composition of a monomorphism (the base change of the monomorphism
$u \to U$) by an \'etale morphism (the base change of the \'etale morphism
$U \to X$). Hence $u \times_X U$ is a disjoint union of spectra of fields
finite separable over $\kappa(u)$ (see
Remark \ref{remark-recall}). Since $u \times_X u$ is finite the image
of it in $u \times_X U$ is a finite disjoint union of spectra of fields
finite separable over $\kappa(u)$. By
Schemes, Lemma \ref{schemes-lemma-mono-towards-spec-field}
we conclude that $u \times_X u$ is a finite disjoint union of spectra
of fields finite separable over $\kappa(u)$. In other words, we see that
$u \times_X u \to u$ is quasi-compact and \'etale. This means that (3) holds.

\medskip\noindent
Let us prove that (3) implies (4). Let $\text{Spec}(k) \to X$ be a morphism
from the spectrum of a field into $X$, in the equivalence class of $x$
such that the two projections
$t, s : R = \text{Spec}(k) \times_X \text{Spec}(k)  \to \text{Spec}(k)$
are quasi-compact and \'etale.
This means in particular
that $R$ is an \'etale equivalence relation on $\text{Spec}(k)$.
By Spaces, Theorem \ref{spaces-theorem-presentation}
we know that the quotient sheaf
$X' = \text{Spec}(k)/R$ is an algebraic space. By
Groupoids, Lemma \ref{groupoids-lemma-quotient-groupoid-restrict}
the map $X' \to X$ is a monomorphism.
Since $s, t$ are quasi-compact, we see that $R$ is quasi-compact and hence
Lemma \ref{lemma-point-like-spaces} applies to $X'$, and we see that
$X' = \text{Spec}(k')$ for some field $k'$. Hence we get a factorization
$$
\text{Spec}(k) \longrightarrow
\text{Spec}(k') \longrightarrow X
$$
which shows that $\text{Spec}(k') \to X$ is a monomorphism mapping
to $x \in |X|$. In other words (4) holds.

\medskip\noindent
Finally, we prove that (4) implies (1). Let $\text{Spec}(k) \to X$
be a monomorphism with $k$ a field in the equivalence class of $x$.
Let $U \to X$ be a surjectve \'etale morphism from a scheme $U$ to $X$.
Let $u \in U$ be a point over $x$. Since $\text{Spec}(k) \times_X u$
is nonempty, and since $\text{Spec}(k) \times_X u \to u$ is a monomorphism
we conclude that $\text{Spec}(k) \times_X u = u$ (see
Schemes, Lemma \ref{schemes-lemma-mono-towards-spec-field}).
Hence $u \to U \to X$ factors through $\text{Spec}(k) \to X$, here is
a picture
$$
\xymatrix{
u \ar[r] \ar[d] & U \ar[d] \\
\text{Spec}(k) \ar[r] & X
}
$$
Since the right vertical arrow is \'etale this implies that
$k \subset \kappa(u)$ is a finite separable extension. Hence we conclude that
$$
u \times_X u = u \times_{\text{Spec}(k)} u
$$
is a finite scheme, and we win by the discussion of the meaning of property
(1) in the first paragraph of this proof.
\end{proof}

\begin{lemma}
\label{lemma-weak-UR-finite-above-x}
Let $S$ be a scheme. Let $X$ be an algebraic space over $S$.
Let $x \in |X|$.
Let $U$ be a scheme and let $\varphi : U \to X$ be an \'etale morphism.
The following are equivalent:
\begin{enumerate}
\item $x$ is in the image of $|U| \to |X|$, and
setting $R = U \times_X U$ the fibres of both
$$
|U| \longrightarrow |X|
\quad\text{and}\quad
|R| \longrightarrow |X|
$$
over $x$ are finite,
\item there exists a monomorphism $\text{Spec}(k) \to X$ with $k$ a field
in the equivalence class of $x$, and
the fibre product $\text{Spec}(k) \times_X U$ is
a finite nonempty scheme over $k$.
\end{enumerate}
\end{lemma}

\begin{proof}
Assume (1). This clearly implies the first condition of
Lemma \ref{lemma-R-finite-above-x} and hence we obtain a monomorphism
$\text{Spec}(k) \to X$ in the class of $x$. Taking the fibre product
we see that $\text{Spec}(k) \times_X U \to \text{Spec}(k)$ is a scheme
\'etale over $\text{Spec}(k)$ with finitely many points, hence a finite
nonempty scheme over $k$, i.e., (2) holds.

\medskip\noindent
Assume (2). By assumption $x$ is in the image of
$|U| \to |X|$. The finiteness of the fibre of
$|U| \to |X|$ over $x$ is clear since this fibre is equal to
$|\text{Spec}(k) \times_X U|$ by Lemma \ref{lemma-points-cartesian}.
The finiteness of the fibre of $|R| \to |X|$ above $x$ is also clear
since it is equal to the set underlying the scheme
$$
(\text{Spec}(k) \times_X U) \times_{\text{Spec}(k)} (\text{Spec}(k) \times_X U)
$$
which is finite over $k$. Thus (1) holds.
\end{proof}

\begin{lemma}
\label{lemma-UR-finite-above-x}
Let $S$ be a scheme. Let $X$ be an algebraic space over $S$.
Let $x \in |X|$. The following are equivalent:
\begin{enumerate}
\item for every affine scheme $U$, any \'etale morphism
$\varphi : U \to X$ setting $R = U \times_X U$ the fibres of both
$$
|U| \longrightarrow |X|
\quad\text{and}\quad
|R| \longrightarrow |X|
$$
over $x$ are finite,
\item there exists a monomorphism $\text{Spec}(k) \to X$ with $k$ a field
in the equivalence class of $x$, and for any affine scheme $U$ and \'etale
morphism $U \to X$ the fibre product $\text{Spec}(k) \times_X U$ is
a finite scheme over $k$, and
\item there exist schemes $U_i$ and \'etale morphisms
$U_i \to X$ such that $\coprod U_i \to X$ is surjective and for each
$i$, setting $R_i = U_i \times_X U_i$ the fibres of both
$$
|U_i| \longrightarrow |X|
\quad\text{and}\quad
|R_i| \longrightarrow |X|
$$
over $x$ are finite.
\end{enumerate}
\end{lemma}

\begin{proof}
The equivalence of (1) and (2) follows on applying
Lemma \ref{lemma-weak-UR-finite-above-x} to every \'etale morphism
$U \to X$ with $U$ affine. It is clear that (2) implies (3).
Assume $U_i \to X$ and $R_i$ are as in (3). We conclude from
Lemma \ref{lemma-U-finite-above-x}
that for any affine scheme $U$ and \'etale morphism $U \to X$
the fibre of $|U| \to |X|$ over $x$ is finite.
Say this fibre is $\{u_1, \ldots, u_n\}$.
Then, as
Lemma \ref{lemma-R-finite-above-x} (1)
applies to $U_i \to X$ for some $i$ such that $x$ is in the image of
$|U_i| \to |X|$, we see that the fibre of
$|R = U \times_X U| \to |U| \times_{|X|} |U|$
is finite over $(u_a, u_b)$, $a, b \in \{1, \ldots, n\}$.
Hence the fibre of $|R| \to |X|$ over $x$ is finite.
In this way we see that (1) holds.
\end{proof}







\section{Reasonable and decent algebraic spaces}
\label{section-reasonable-decent}

\noindent
The conditions in the following definition
are not exactly conditions on the diagonal of $X$, but they are in some
sense separation conditions on $X$.

\begin{definition}
\label{definition-very-reasonable}
Let $S$ be a scheme.
Let $X$ be an algebraic space over $S$.
\begin{enumerate}
\item We say $X$ is {\it decent} if for every point $x \in X$ the equivalent
conditions of Lemma \ref{lemma-UR-finite-above-x} hold, in other words
property $(\gamma)$ of
Lemma \ref{lemma-bounded-fibres}
holds.
\item We say $X$ is {\it reasonable} if the equivalent conditions of
Lemma \ref{lemma-U-universally-bounded}
hold, in other words property $(\delta)$ of
Lemma \ref{lemma-bounded-fibres}
holds.
\item We say $X$ is {\it very reasonable} if there exists a set of schemes
$U_i$ and morphisms $U_i \to X$ such that
\begin{enumerate}
\item each $U_i \to X$ is \'etale,
\item both projections $U_i \times_X U_i \to U_i$ are
quasi-compact, and
\item the morphism $\coprod U_i \to X$ is surjective (and \'etale).
\end{enumerate}
This property is denoted $(\epsilon)$ in
Lemma \ref{lemma-bounded-fibres}.
\end{enumerate}
\end{definition}

\noindent
The notion of a very reasonable algebraic space was introduced because
the assumption was sufficient to prove some of the results below, especially
Proposition \ref{proposition-very-reasonable-open-dense-scheme} and
\ref{proposition-very-reasonable-sober}.
We hope (in the future) to strengthen these results to
the case where the space $X$ is reasonable or even just decent.
Condition (3)(b) means that $U_i \to X$ is quasi-compact onto its image;
see Lemma \ref{lemma-characterize-very-reasonable} and its proof.
In particular, if there exists a scheme $U$ and a surjective, quasi-compact
morphism $U \to X$, then $X$ is very reasonable. Namely, in this case both
projections $U \times_X U \to U$ are quasi-compact.

\begin{lemma}
\label{lemma-characterize-very-reasonable}
Let $S$ be a scheme.
Let $X$ be an algebraic space over $S$.
The following are equivalent:
\begin{enumerate}
\item $X$ is very reasonable, and
\item there exists a Zariski covering $X = \bigcup X_i$ and for
each $i$ a scheme $U_i$ and a quasi-compact surjective \'etale
morphism $U_i \to X_i$.
\end{enumerate}
\end{lemma}

\begin{proof}
If (2) holds then the morphisms $U_i \to X_i \to X$ are \'etale (combine
Morphisms, Lemma \ref{morphisms-lemma-composition-etale}
and
Spaces, Lemmas
\ref{spaces-lemma-morphism-schemes-gives-representable-transformation-property}
and
\ref{spaces-lemma-composition-representable-transformations-property}).
Moreover, as $U_i \times_X U_i = U_i \times_{X_i} U_i$,
both projections $U_i \times_X U_i \to U_i$ are quasi-compact.

\medskip\noindent
If $X$ is very reasonable then there exists a surjective \'etale morphism
$\coprod U_i \to X$, where each $U_i$ is a scheme, such that
the projections $U_i \times_X U_i \to U_i$ are quasi-compact.
Let $X_i \subset X$ be the open subspace corresponding to the image
of the open map $|U_i| \to |X|$ (use
Lemmas \ref{lemma-open-subspaces} and \ref{lemma-topology-points}).
By Lemma \ref{lemma-factor-through-open-subspace}
we get morphisms $U_i \to X_i$ which are
surjective by Lemma \ref{lemma-characterize-surjective}.
Hence $U_i \to X_i$ is surjective \'etale, and the projections
$U_i \times_{X_i} U_i \to U_i$ are quasi-compact, again because
$U_i \times_{X_i} U_i = U_i \times_X U_i$. Thus by
Spaces, Lemma \ref{spaces-lemma-representable-morphisms-spaces-property}
the morphisms $U_i \to X_i$ are quasi-compact.
\end{proof}

\begin{lemma}
\label{lemma-scheme-very-reasonable}
A scheme is very reasonable.
\end{lemma}

\begin{proof}
This is true because the identity map is a quasi-compact, surjective
\'etale morphism.
\end{proof}

\begin{lemma}
\label{lemma-very-reasonable-Zariski-local}
Let $S$ be a scheme.
Let $X$ be an algebraic space over $S$.
If there exists a Zariski open covering $X = \bigcup X_i$ such that
each $X_i$ is very reasonable, then $X$ is very reasonable.
\end{lemma}

\begin{proof}
Assume there exists a Zariski open covering
$X = \bigcup X_i$, where each $X_i$ is very reasonable.
Then we can find sets $J_i$ and morphisms
$\varphi_{ij} : U_{ij} \to X_i$ such that each $\varphi_{ij}$
is \'etale, both projections $U_{ij} \times_{X_i} U_{ij} \to U_{ij}$
are quasi-compact, and $\coprod_{j \in J_i} U_{ij} \to X_i$ is surjective.
In this case the compositions $U_{ij} \to X_i \to X$ are \'etale
(combine
Morphisms, Lemmas
\ref{morphisms-lemma-composition-etale}
and
\ref{morphisms-lemma-open-immersion-etale}
and
Spaces, Lemmas
\ref{spaces-lemma-morphism-schemes-gives-representable-transformation-property}
and
\ref{spaces-lemma-composition-representable-transformations-property}).
Since $X_i \subset X$ is a subspace we see that
$U_{ij} \times_{X_i} U_{ij} = U_{ij} \times_X U_{ij}$, and hence the
condition on fibre products is preserved. And clearly
$\coprod_{i, j} U_{ij} \to X$ is surjective. Hence $X$ is very reasonable.
\end{proof}

\begin{lemma}
\label{lemma-quasi-separated-very-reasonable}
An algebraic space which is Zariski locally quasi-separated is very reasonable.
In particular any quasi-separated algebraic space is very reasonable.
\end{lemma}

\begin{proof}
By Lemma \ref{lemma-very-reasonable-Zariski-local}
it suffices to show that a quasi-separated algebraic space is very reasonable.

\medskip\noindent
Let $S$ be a scheme, and let $X$ be a quasi-separated algebraic space
over $S$. Let $U$ be a scheme and let $U \to X$ be a surjective \'etale
morphism. Let $U = \bigcup U_i$ be an affine open covering of $U$.
Each of the morphisms $U_i \to X$ is \'etale (combine
Morphisms, Lemma \ref{morphisms-lemma-composition-etale}
and
Spaces, Lemmas
\ref{spaces-lemma-morphism-schemes-gives-representable-transformation-property}
and
\ref{spaces-lemma-composition-representable-transformations-property}).
Hence $\coprod U_i \to X$ is surjective \'etale.
To finish the proof we show that $U_i \to X$ is quasi-compact, which
in particular implies that both projections $U_i \times_X U_i \to U_i$ are
quasi-compact. To do this we may by
Spaces, Lemma \ref{spaces-lemma-viewed-as-properties}
view $X$ as an algebraic space over $\text{Spec}(\mathbf{Z})$.
In other words, in the rest of the proof we may assume that
$S = \text{Spec}(\mathbf{Z})$.

\medskip\noindent
Let $T \to X$ be a morphism from a scheme to $X$. Then
$$
T \times_X U_i
=
(T \times_{\text{Spec}(\mathbf{Z})} U_i)
\times_{X, \Delta}
(X \times_{\text{Spec}(\mathbf{Z})} X)
$$
and hence $T \times_X U_i \to T$ is the composition
$$
(T \times_{\text{Spec}(\mathbf{Z})} U_i)
\times_{X, \Delta}
(X \times_{\text{Spec}(\mathbf{Z})} X)
\longrightarrow
T \times_{\text{Spec}(\mathbf{Z})} U_i
\longrightarrow T
$$
The first arrow is quasi-compact by our assumption that $X$ is
(absolutely) quasi-separated, and the second is quasi-compact because
it is affine (since $U_i$ was chosen to be affine, and
Morphisms, Lemma \ref{morphisms-lemma-base-change-affine}).
\end{proof}

\begin{lemma}
\label{lemma-representable-very-reasonable}
Let $S$ be a scheme.
Let $X$, $Y$ be algebraic spaces over $S$.
Let $Y \to X$ be a representable morphism.
If $X$ is very reasonable, so is $Y$.
\end{lemma}

\begin{proof}
We will repeatedlty use
Spaces, Lemma
\ref{spaces-lemma-base-change-representable-transformations-property}.
Let $U_i \to X$ be as in Definition \ref{definition-very-reasonable}.
Set $V_i = Y \times_X U_i$. The morphisms $V_i \to Y$ are \'etale,
and $\coprod V_i \to Y$ is surjective. Because
$V_i \times_Y V_i = Y \times_X (U_i \times_X U_i)$ we see
that the projections $V_i \times_Y V_i \to V_i$ are
base changes of the projections $U_i \times_X U_i \to U_i$, and so
quasi-compact as well. Hence $Y$ is very reasonable.
\end{proof}

\begin{remark}
\label{remark-very-reasonable-Zariski-locally-quasi-separated}
Very reasonable algebraic spaces form a stricly larger collection than
Zariski locally quasi-separated algebraic spaces. Consider
an algebraic space of the form $X = [U/G]$ (see
Spaces, Definition \ref{spaces-definition-quotient})
where $G$ is a finite group acting without fixed points on a
non-quasi-separated scheme $U$. Namely, in this case
$U \times_X U = U \times G$ and clearly both projections to $U$ are
quasi-compact, hence $X$ is very reasonable. On the other hand, the diagonal
$U \times_X U \to U \times U$ is not quasi-compact, hence this
algebraic space is not quasi-separated. Now, take $U$ the infinite
affine space over a field $k$ of characteristic $\not = 2$ with
zero doubled, see
Schemes, Example \ref{schemes-example-not-quasi-separated}.
Let $0_1, 0_2$ be the two zeros of $U$. Let $G = \{+1, -1\}$, and
let $-1$ act by $-1$ on all coordinates, and by switching
$0_1$ and $0_2$. Then $[U/G]$ is very reasonable but not Zariski locally
quasi-separated (details omitted).
\end{remark}

\begin{example}
\label{example-not-very-reasonable}
The algebraic space $[\mathbf{A}^1_{\mathbf{Q}}/\mathbf{Z}]$ constructed in
Spaces, Example \ref{spaces-example-affine-line-translation}
is not very reasonable.
\end{example}

\begin{remark}
\label{remark-reasonable}
Let $S$ be a scheme and let $X$ be an algebraic space over $S$.
Suppose that for any affine scheme $U$ and \'etale morphism
$\varphi : U \to X$ the fibres of $\varphi$ are universally bounded. In
Definition \ref{definition-very-reasonable}
we called such an algebraic space {\it reasonable}. Reasonable spaces are
technically easier to work with than very reasonable algebraic spaces
(this has to do with descent of the property; see
Morphisms of Spaces, Remark \ref{spaces-morphisms-remark-very-reasonable}).
On the other hand, we do not know whether a reasonable algebraic
space has an open dense subspace which is a scheme, and we also do not know
whether its underlying topological space is sober, whereas we do know that
very reasonable spaces have those properties (see
Propositions \ref{proposition-very-reasonable-open-dense-scheme}, and
\ref{proposition-very-reasonable-sober}).
\end{remark}

\begin{remark}
\label{remark-fun-property-reasonable}
This is a continuation of Remark \ref{remark-reasonable}.
Observation: A reasonable space is a colimit of quasi-separated
algebraic spaces. We sketch the proof in the case $X = U/R$ with $U$ affine.
In this case, reasonable means $U \to X$ is universally bounded.
Hence there exists an integer $N$ such that the ``fibres'' of $U \to X$
have degree at most $N$, see
Definition \ref{definition-universally-bounded}.
Denote $s, t : R \to U$ and $c : R \times_{s, U, t} R \to R$ the
groupoid structural maps.
We claim that for every quasi-compact open $A \subset R$ there exists
an open $R' \subset R$ such that
\begin{enumerate}
\item $A \subset R'$,
\item $R'$ is quasi-compact, and
\item $(U, R', s|_{R'}, t|_{R'}, c|_{R' \times_{s, U, t} R'})$ is
a groupoid scheme.
\end{enumerate}
Note that $e : U \to R$ is open as it is a section of the \'etale morphism
$s : R \to U$, see
\'Etale Morphisms of Schemes,
Proposition \ref{etale-proposition-properties-sections}. Moreover
$U$ is affine hence quasi-compact. Hence we may replace $A$ by
$A \cup e(U) \subset R$, and assume that $A$ contains $e(U)$. Next, we
define inductively $A^1 = A$, and
$$
A^n = c(A^{n - 1} \times_{s, U, t} A) \subset R
$$
for $n \geq 2$. Arguing inductively, we see that $A^n$ is quasi-compact for
all $n \geq 2$, as the image of the quasi-compact fibre product
$A^{n - 1} \times_{s, U, t} A$. If $k$ is an algebraically
closed field over $S$, and we consider $k$-points then
$$
A^n(k) = \left\{(u, u') \in U(k)
:
\begin{matrix}
\text{there exist } u = u_1, u_2, \ldots, u_n \in U(k)\text{ with} \\
(u_i , u_{i + 1}) \in A \text{ for all }i = 1, \ldots, n - 1.
\end{matrix}
\right\}
$$
But as the fibres of $U(k) \to X(k)$ have size at most $N$ we see that if
$n > N$ then we get a repeat in the sequence above, and we can shorten it
proving $A^N = A^n$ for all $n \geq N$.
This implies that $R' = A^N$ gives a groupoid scheme
$(U, R', s|_{R'}, t|_{R'}, c|_{R' \times_{s, U, t} R'})$, proving the claim
above. Consider the map of sheaves on $(\textit{Sch}/S)_{fppf}$
$$
\text{colim}_{R' \subset R}\ U/R' \longrightarrow U/R
$$
where $R' \subset R$ runs over the quasi-compact open subschemes
of $R$ which give \'etale equivalence relations as above. Each of the
quotients $U/R'$ is an algebraic space
(see Spaces, Theorem \ref{spaces-theorem-presentation}).
Since $R'$ is quasi-compact, and $U$ affine the morphism
$R' \to U \times_{\text{Spec}(\mathbf{Z})} U$ is quasi-compact,
and hence $U/R'$ is quasi-compact. Finally, if $T$ is a quasi-compact
scheme, then
$$
\text{colim}_{R' \subset R}\ U(T)/R'(T) \longrightarrow U(T)/R(T)
$$
is a bijection, since every morphism from $T$ into $R$ ends up in one
of the open subrelations $R'$ by the claim above. This clearly implies
that the colimit of the sheaves $U/R'$ is $U/R$. In other words
the algebraic space $X = U/R$ is the colimit of the quasi-separated
algebraic spaces $U/R'$.
\end{remark}







\section{The schematic locus}
\label{section-schematic}

\noindent
Every algebraic space has a largest open subspace which is a
scheme; this is more or less clear but we also write out the proof below.
Of course this subspace may be empty, for example if
$X = [\mathbf{A}^1_{\mathbf{Q}}/\mathbf{Z}]$ (the universal
counter example). On the other hand, if $X$ is very reasonable, then
this largest open subscheme is actually dense in $X$!

\begin{lemma}
\label{lemma-subscheme}
Let $S$ be a scheme.
Let $X$ be an algebraic space over $S$.
There exists a largest open subspace $X' \subset X$ which is a scheme.
\end{lemma}

\begin{proof}
Let $U \to X$ be an \'etale surjective morphism, where $U$ is a scheme.
Let $R = U \times_X U$. The open subspaces of $X$ correspond $1 - 1$
with open subschemes of $U$ which are $R$-invariant. Hence there is a
set of them. Let $X_i$, $i \in I$ be the set of open subspaces
of $X$ which are schemes, i.e., are representable. Consider the
open subspace $X' \subset X$ whose underlying set of points is
the open $\bigcup |X_i|$ of $|X|$. By
Lemma \ref{lemma-characterize-surjective}
we see that
$$
\coprod X_i \longrightarrow X'
$$
is a surjective map of sheaves on $(\textit{Sch}/S)_{fppf}$.
But since each $X_i \to X'$ is representable by open immersions
we see that in fact the map is surjective in the Zariski
topology. (Because if $T \to X'$ is a morphism from a scheme
into $X'$, then $X_i \times_X' T$ is an open subscheme of $T$.)
Hence we can apply
Schemes, Lemma \ref{schemes-lemma-glue-functors}
to see that $X'$ is a scheme.
\end{proof}

\noindent
In the following proposition and its proof we say that an open subspace
$X'$ of an algebraic space $X$ is {\it dense} if the corresponding
open subset $|X'| \subset |X|$ is dense.

\begin{proposition}
\label{proposition-very-reasonable-open-dense-scheme}
Let $S$ be a scheme.
Let $X$ be an algebraic space over $S$.
If $X$ is very reasonable, then there exists a dense open subspace
of $X$ which is a scheme.
\end{proposition}

\begin{proof}
By Lemmas \ref{lemma-subscheme} and \ref{lemma-characterize-very-reasonable}
we may assume that there exists a surjective quasi-compact, \'etale morphism
$U \to X$. Set $R = U \times_X U$, and denote $s, t : R \to U$ the projections
as usual. Note that $s, t$ are surjective, quasi-compact and \'etale, hence
also quasi-finite (see
\'Etale Morphisms of Schemes, Section \ref{etale-section-etale-morphisms}).
By
More on Morphisms,
Lemma \ref{more-morphisms-lemma-quasi-finite-finite-over-dense-open}
there exists a dense open subscheme $W \subset U$ such that
$s^{-1}(W) \to W$ is finite. By
Descent, Lemma \ref{descent-lemma-descending-property-finite}
being finite is fpqc (and in particular \'etale) local on the target.
Hence we may apply
More on Groupoids, Lemma \ref{more-groupoids-lemma-property-invariant}
which says that the largest open $W \subset U$ over which $s$ is
finite is $R$-invariant. It is still dense of course.
The restriction $R_W$ of $R$ to $W$ equals $R_W = s^{-1}(W) = t^{-1}(W)$
(see Groupoids, Definition \ref{groupoids-definition-invariant-open}
and discussion following it).
By construction $s_W, t_W : R_W \to W$ are finite \'etale.
If we can show the open subspace $W/R_W \subset X$ (see
Spaces, Lemma \ref{spaces-lemma-finding-opens})
contains a dense open subspace which is a scheme, then the
proposition follows for $X$. This reduces us to the case discussed
in the next paragraph.

\medskip\noindent
Assume $X$ is an algebraic space, $U$ a scheme, and $U \to X$ is a finite
\'etale surjective morphism. Write $R = U \times_X U$ and denote
$s, t : R \to U$ the projections as usual. Note that $s, t$ are surjective,
finite and \'etale in this case. Claim:
The union of the $R$-invariant affine opens of $U$ is topologically
dense in $U$.

\medskip\noindent
Proof of the claim\footnote{The claim is easier to prove if
$U$ is assumed quasi-separated, since in that case
Properties, Lemma \ref{properties-lemma-maximal-points-affine}
may be applied immediately to the $R$-equivalence class of any
generic point of $U$.}. Let $W \subset U$ be an affine open.
Set $W' = t(s^{-1}(W)) \subset U$. Since $s^{-1}(W)$ is affine
(hence quasi-compact) we see that $W' \subset U$ is a quasi-compact open. By
Properties, Lemma \ref{properties-lemma-quasi-compact-dense-open-separated}
there exists a dense open $W'' \subset W'$ which is a separated scheme.
Set $\Delta' = W' \setminus W''$. This is a nowhere dense closed subset of
$W''$. Since $t|_{s^{-1}(W)} : s^{-1}(W) \to W'$ is open (because it is \'etale)
we see that the inverse image
$(t|_{s^{-1}(W)})^{-1}(\Delta') \subset s^{-1}(W)$
is a nowhere dense closed subset (see
Topology, Lemma \ref{topology-lemma-open-inverse-image-closed-nowhere-dense}).
Hence, by
Morphisms, Lemma \ref{morphisms-lemma-image-nowhere-dense-finite} 
we see that
$$
\Delta = s\left((t|_{s^{-1}(W)})^{-1}(\Delta')\right)
$$
is a nowhere dense closed subset of $W$. Pick any point $\eta \in W$,
$\eta \not \in \Delta$ which is a generic point of an irreducible
component of $W$ (and hence of $U$). By our choices above the finite set
$t(s^{-1}(\{\eta\})) = \{\eta_1, \ldots, \eta_n\}$
is contained in the separated scheme $W''$.
Note that the fibres of $s$ is are finite discrete spaces, and that
generalizations lift along the \'etale morphism $t$, see
Morphisms, Lemmas \ref{morphisms-lemma-etale-flat}
and \ref{morphisms-lemma-generalizations-lift-flat}.
In this way we see that each $\eta_i$ is a generic point of an
irreducible component of $W''$. Thus, by
Properties, Lemma \ref{properties-lemma-maximal-points-affine}
we can find an affine open $V \subset W''$ such that
$\{\eta_1, \ldots, \eta_n\} \subset V$.
By
Groupoids, Lemma \ref{groupoids-lemma-find-invariant-affine}
this implies that $\eta$ is contained in an $R$-invariant affine
open subscheme of $U$. The claim follows as $W$ was chosen as an
arbitrary affine open of $U$ and because the set of generic points
of irreducible components of $W \setminus \Delta$ is dense in $W$.

\medskip\noindent
Using the claim we can finish the proof. Namely, if $W \subset U$ is
an $R$-invariant affine open, then the restriction $R_W$ of $R$ to $W$
equals $R_W = s^{-1}(W) = t^{-1}(W)$ (see
Groupoids, Definition \ref{groupoids-definition-invariant-open}
and discussion following it). In particular the maps $R_W \to W$ are
finite \'etale also. It follows in particular that $R_W$ is affine.
Thus we see that $W/R_W$ is a scheme, by
Groupoids, Proposition \ref{groupoids-proposition-finite-flat-equivalence}.
On the other hand, $W/R_W$ is an open subspace of $X$ by
Spaces, Lemma \ref{spaces-lemma-finding-opens}.
Hence having a dense collection of points contained in $R$-invariant
affine open of $U$ certainly implies that the schematic locus of $X$
(see Lemma \ref{lemma-subscheme})
is open dense in $X$.
\end{proof}











\section{Conditions on algebraic spaces}
\label{section-conditions}

\noindent
In this section we discuss the relationship between various natural
conditions on algebraic spaces we have seen above.

\begin{lemma}
\label{lemma-bounded-fibres}
Let $S$ be a scheme. Let $X$ be an algebraic space over $S$.
Consider the following conditions on $X$:
\begin{enumerate}
\item[$(\alpha)$] For every $x \in |X|$, the equivalent conditions of
Lemma \ref{lemma-U-finite-above-x} hold.
\item[$(\beta)$] The map
$$
\{\text{Spec}(k) \to X \text{ monomorphism}\}
\longrightarrow
|X|
$$
is bijective, i.e., for every $x \in |X|$ the equivalent conditions of
Lemma \ref{lemma-R-finite-above-x} hold.
\item[$(\gamma)$] For every $x \in |X|$, the equivalent conditions of
Lemma \ref{lemma-UR-finite-above-x} hold, in other words $X$ is decent.
\item[$(\delta)$] The equivalent conditions of
Lemma \ref{lemma-U-universally-bounded}
hold, in other words $X$ is reasonable.
\item[$(\epsilon)$] The space $X$ is very reasonable.
\item[$(\zeta)$] The space $X$ is quasi-separated.
\item[$(\eta)$] The space $X$ is representable, i.e., $X$ is a scheme.
\item[$(\theta)$] The space $X$ is a quasi-separated scheme.
\end{enumerate}
We have
$$
\xymatrix{
& (\eta) \ar@{=>}[rd] & & & &  \\
(\theta) \ar@{=>}[ru] \ar@{=>}[rd] & & 
(\epsilon) \ar@{=>}[r] &
(\delta) \ar@{=>}[r] &
(\gamma) \ar@{<=>}[r] & (\alpha) + (\beta) \\
& (\zeta) \ar@{=>}[ru] & & & & 
}
$$
\end{lemma}

\begin{proof}
The implication $(\gamma) \Leftrightarrow (\alpha) + (\beta)$ is immediate.
The implications in the diamond on the left we have seen in
Section \ref{section-reasonable-decent}.

\medskip\noindent
Assume $(\delta)$. Let $U$ be an affine scheme, and let $U \to X$ be an \'etale
morphism. By assumption the fibres of the morphism $U \to X$ are universally
bounded. Thus also the fibres of both projections $R = U \times_X U \to U$
are universally bounded, see
Lemma \ref{lemma-base-change-universally-bounded}.
And by
Lemma \ref{lemma-composition-universally-bounded}
also the fibres of $R \to X$ are universally bounded.
Hence for any $x \in X$ the fibres of $|U| \to |X|$ and $|R| \to |X|$
over $x$ are finite, see
Lemma \ref{lemma-universally-bounded-finite-fibres}.
In other words, the equivalent conditions of
Lemma \ref{lemma-UR-finite-above-x}
hold. This proves that $(\delta) \Rightarrow (\gamma)$.

\medskip\noindent
Let us show that $(\epsilon)$ implies $(\delta)$.
Assume $(\epsilon)$. By
Lemma \ref{lemma-characterize-very-reasonable}
there exists
a Zariski open covering $X = \bigcup X_i$ such that for each $i$
there exists a scheme $U_i$ and a quasi-compact surjective \'etale morphism
$U_i \to X_i$. Choose an $i$ and an affine open subscheme $W \subset U_i$.
It suffices to show that $W \to X$ has universally bounded fibres, since then
the family of all these morphisms $W \to X$ covers $X$.
To do this we consider the diagram
$$
\xymatrix{
W \times_X U_i \ar[r]_-p \ar[d]_q & U_i \ar[d] \\
W \ar[r] & X
}
$$
Since $W \to X$ factors through $X_i$ we see that
$W \times_X U_i = W \times_{X_i} U_i$, and hence $q$ is quasi-compact.
Since $W$ is affine this implies that the scheme $W \times_X U_i$
is quasi-compact. Thus we may apply
Morphisms, Lemma
\ref{morphisms-lemma-locally-quasi-finite-qc-source-universally-bounded}
and we conclude that $p$ has universally bounded fibres.
We may apply
Lemma \ref{lemma-descent-universally-bounded}
to conclude that $W \to X$ has universally bounded fibres as well.
\end{proof}

\begin{lemma}
\label{lemma-properties-local}
Let $S$ be a scheme.
Let $\mathcal{P}$ be one of the properties
$(\alpha)$, $(\beta)$, $(\gamma)$, $(\delta)$, $(\epsilon)$, or
$(\eta)$ of algebraic spaces
listed in Lemma \ref{lemma-bounded-fibres}.
Then if $X$ is an algebraic space over $S$, and $X = \bigcup X_i$ is a
Zariski open covering such that each $X_i$ has $\mathcal{P}$,
then $X$ has $\mathcal{P}$.
\end{lemma}

\begin{proof}
Let $X$ be an algebraic space over $S$, and let $X = \bigcup X_i$ is a
Zariski open covering such that each $X_i$ has $\mathcal{P}$.
The condition $(\alpha)$ for $X_i$ can be formulated as the condition
that for every $x \in |X_i|$ and every affine scheme $U$, and \'etale morphism
$\varphi : U \to X_i$ the fibre of $\varphi : |U| \to |X_i|$
over $x$ are finite. Consider $x \in X$, an affine scheme $U$ and
an \'etale morphism $U \to X$. Since $X = \bigcup X_i$ is a
Zariski open covering there exits a finite affine open covering
$U = U_1 \cup \ldots \cup U_n$ such that each $U_j \to X$ factors through
some $X_{i_j}$. By assumption the fibres of $|U_j | \to |X_{i_j}|$
over $x$ are finite for $j = 1, \ldots, n$. Clearly this means that
the fibre of $|U| \to |X|$ over $x$ is finite.
This proves the result for $(\alpha)$.

\medskip\noindent
Note that $(\gamma) = (\alpha) + (\beta)$ by
Lemma \ref{lemma-bounded-fibres}
hence if we prove the lemma for $(\beta)$ then the lemma follows for
$(\gamma)$.

\medskip\noindent
The lemma for $(\beta)$ is immediate from the definition as $X_i \to X$ is
a monomorphism. The lemma for property $(\delta)$ is clear also since given
schemes $U_{ij}$ and \'etale morphisms $U_{ij} \to X_i$ with universally
bounded fibres which cover $X_i$, then these schemes also given an
\'etale surjective morphism $\coprod U_{ij} \to X$ and $U_{ij} \to X$
still has universally bounded fibres. For $(\epsilon)$, see
Lemma \ref{lemma-very-reasonable-Zariski-local}.
For $(\eta)$, see Lemma \ref{lemma-subscheme}.
\end{proof}

\begin{lemma}
\label{lemma-representable-properties}
Let $S$ be a scheme.
Let $\mathcal{P}$ be one of the properties
$(\beta)$, $(\gamma)$, $(\delta)$, $(\epsilon)$, or $(\eta)$
of algebraic spaces listed in Lemma \ref{lemma-bounded-fibres}.
Let $X$, $Y$ be algebraic spaces over $S$.
Let $X \to Y$ be a representable morphism.
If $Y$ has property $\mathcal{P}$, so does $X$.
\end{lemma}

\begin{proof}
Assume $f : X \to Y$ is a representable morphism of algebraic spaces,
and assume that $Y$ has $\mathcal{P}$. Let $x \in |X|$, and set
$y = f(x) \in |Y|$.

\medskip\noindent
If $\mathcal{P}$ is $(\beta)$, then there exists a monomorphism
$\text{Spec}(k) \to Y$ representing $y$. The fibre product
$X_y = \text{Spec}(k) \times_Y X$ is a scheme, and $x$ corresponds
to a point of $X_y$, i.e., to a monomorphism $\text{Spec}(k') \to X_y$.
As $X_y \to X$ is a monomorphism also we see that $x$ is represented
by the monomorphism $\text{Spec}(k') \to X_y \to X$. In other words
$(\beta)$ holds for $X$.

\medskip\noindent
Suppose $\mathcal{P}$ is $(\gamma)$. Since $(\gamma) \Rightarrow (\beta)$
we have seen in the preceding paragraph that $y$ and $x$ can be represented
by monomorphisms as in the following diagram
$$
\xymatrix{
\text{Spec}(k') \ar[r]_-x \ar[d] & X \ar[d] \\
\text{Spec}(k) \ar[r]^-y & Y
}
$$
Also, by definition of property $(\gamma)$ via
Lemma \ref{lemma-UR-finite-above-x} (3)
there exist schemes
$V_i$ and \'etale morphisms $V_i \to Y$ such that $\coprod V_i \to Y$
is surjective and for each $i$, setting $R_i = V_i \times_Y V_i$
the fibres of both
$$
|V_i| \longrightarrow |Y|
\quad\text{and}\quad
|R_i| \longrightarrow |Y|
$$
over $y$ are finite. This means that the schemes
$(V_i)_y$ and $(R_i)_y$ are finite schemes over $y = \text{Spec}(k)$.
As $X \to Y$ is representable, the fibre products $U_i = V_i \times_Y X$
are schemes. The morphisms $U_i \to X$ are \'etale, and
$\coprod U_i \to X$ is surjective. Finally, for each $i$ we have
$$
(U_i)_x =
(V_i \times_Y X)_x =
(V_i)_y \times_{\text{Spec}(k)} \text{Spec}(k')
$$
and
$$
(U_i \times_X U_i)_x =
\left((V_i \times_Y X) \times_X (V_i \times_Y X)\right)_x =
(R_i)_y \times_{\text{Spec}(k)} \text{Spec}(k')
$$
hence these are finite over $k'$ as base changes of the finite
schemes $(V_i)_y$ and $(R_i)_y$. This implies that $(\gamma)$ holds for $X$,
again via the third condition of
Lemma \ref{lemma-UR-finite-above-x}.

\medskip\noindent
Suppose $\mathcal{P}$ is $(\delta)$. Let $V \to Y$ be an \'etale morphism with
$V$ an affine scheme. Since $Y$ has property $(\delta)$ this morphism has
universally bounded fibres. By
Lemma \ref{lemma-base-change-universally-bounded}
the base change $V \times_Y X \to X$ also has universally bounded fibres.
Hence the first part of
Lemma \ref{lemma-U-universally-bounded}
applies and we see that $Y$ also has property $(\delta)$.

\medskip\noindent
In case $\mathcal{P} = (\epsilon)$ the result is
Lemma \ref{lemma-representable-very-reasonable}.
In case $\mathcal{P} = (\eta)$ the result is
Categories, Lemma \ref{categories-lemma-representable-over-representable}.
\end{proof}






\section{Points and specializations}
\label{section-specializations}

\begin{lemma}
\label{lemma-no-specializations-map-to-same-point}
Let $S$ be a scheme.
Let $X$ be an algebraic space over $S$.
Let $U \to X$ be an \'etale morphism from a scheme to $X$.
If $u, u' \in |U|$ map to the same point $x$ of $|X|$, and
$u' \leadsto u$. If the pair $(X, x)$ satisfies the
equivalent conditions for Lemma \ref{lemma-U-finite-above-x}
then $u = u'$.
\end{lemma}

\begin{proof}
Assume the pair $(X, x)$ satisfies the
equivalent conditions for Lemma \ref{lemma-U-finite-above-x}.
Let $U$ be a scheme, $U \to X$ \'etale, and
let $u, u' \in |U|$ map to $x$ of $|X|$, and
$u' \leadsto u$. We may and do replace $U$ by an affine
neighbourhood of $u$. Let $t, s : R = U \times_X U \to U$
be the \'etale projection maps.

\medskip\noindent
We finish the proof as follows.
Pick a point $r \in R$ with $t(r) = u$ and $s(r) = u'$.
This is possible by
Lemma \ref{lemma-points-presentation}.
Because generalizations lift along the \'etale morphism $t$
(Remark \ref{remark-recall}) we can find a specialization $r' \leadsto r$ with
$t(r') = u'$. Set $u'' = s(r')$. Then $u'' \leadsto u'$.
Thus we may repeat and find $r'' \leadsto r'$ with
$t(r'') = u''$. Set $u''' = s(r'')$, and so on.
Here is a picture:
$$
\xymatrix{
& r'' \ar[rd]^s \ar[ld]_t \ar@{~>}[d] & \\
u'' \ar@{~>}[d] & r' \ar[rd]^s \ar[ld]_t \ar@{~>}[d] & u''' \ar@{~>}[d] \\
u' \ar@{~>}[d] & r \ar[rd]^s \ar[ld]_t & u'' \ar@{~>}[d] \\
u & & u'
}
$$
In Remark \ref{remark-recall} we have seen that there are no specializations
among points in the fibres of the \'etale morphism $s$. Hence if
$u^{(n + 1)} = u^{(n)}$ for some $n$, then also $r^{(n)} = r^{(n - 1)}$ and
hence also (by taking $t$) $u^{(n)} = u^{(n - 1)}$. This then forces the
whole tower to collapse, in particular $u = u'$. Thus we see that if
$u \not = u'$, then all the specializations are strict and
$\{u, u', u'', \ldots\}$ is an infinite set of points in $U$ which map to the
point $x$ in $|X|$. As we chose $U$ affine this contradicts the second part of
Lemma \ref{lemma-U-finite-above-x}, as desired.
\end{proof}

\begin{lemma}
\label{lemma-specialization}
Let $S$ be an algebraic space.
Let $X$ be an algebraic space over $S$.
Let $x, x' \in |X|$ and assume $x' \leadsto x$, i.e., $x$ is a
specialization of $x'$.
Assume the pair $(X, x')$ satisfies the equivalent conditions
of Lemma \ref{lemma-UR-finite-above-x}. Then
for every \'etale morphism $\varphi : U \to X$ from a scheme $U$ and any
$u \in U$ with $\varphi(u) = x$, exists a point $u'\in U$,
$u' \leadsto u$ with $\varphi(u') = x'$.
\end{lemma}

\begin{proof}
We may replace $U$ by an affine open neighbourhood of $u$.
Hence we may assume that $U$ is affine. As $x$ is in the
image of the open map $|U| \to |X|$, so is $x'$. Thus we may
replace $X$ by the Zariski open subspace corresponding to
the image of $|U| \to |X|$, see
Lemma \ref{lemma-open-subspaces}.
In other words we may assume that
$U \to X$ is surjective and \'etale.
Let $s, t : R = U \times_X U \to U$ be the projections.
By our assumption that $(X, x')$ satisfies the equivalent conditions
of Lemma \ref{lemma-UR-finite-above-x} we see that the fibres
of $|U| \to |X|$ and $|R| \to |X|$
over $x'$ are finite. Say $\{u'_1, \ldots, u'_n\} \subset U$ and
$\{r'_1, \ldots, r'_m\} \subset R$ form the complete inverse image
of $\{x'\}$.
Consider the closed sets
$$
T = \overline{\{u'_1\}} \cup \ldots \cup \overline{\{u'_n\}} \subset |U|,
\quad
T' = \overline{\{r'_1\}} \cup \ldots \cup \overline{\{r'_m\}} \subset |R|.
$$
Trivially we have $s(T') \subset T$. Because $R$ is an equivalence
relation we also have $t(T') = s(T')$ as the set $\{r_j'\}$
is invariant under the inverse of $R$ by construction. Let $w \in T$
be any point. Then $u'_i \leadsto w$ for some $i$. Choose $r \in R$
with $s(r) = w$. Since generalizations lift along $s : R \to U$, see
Remark \ref{remark-recall}, we can find $r' \leadsto r$ with
$s(r') = u_i'$. Then $r' = r'_j$ for some $j$ and we conclude that
$w \in s(T')$. Hence $T = s(T') = t(T')$ is an $|R|$-invariant closed
set in $|U|$. This means $T$ is the inverse image of a closed (!)
subset $T'' = \varphi(T)$ of $|X|$, see
Lemmas \ref{lemma-points-presentation} and \ref{lemma-topology-points}.
Hence $T'' = \overline{\{x'\}}$.
Thus $T$ contains some point $u_1$ mapping to $x$ as $x \in T''$.
I.e., we see that for some $i$ there exists a specialization
$u'_i \leadsto u_1$ which maps to the given specialization
$x' \leadsto x$.

\medskip\noindent
To finish the proof, choose a point $r \in R$ such that
$s(r) = u$ and $t(r) = u_1$ (using Lemma \ref{lemma-points-cartesian}).
As generalizations lift along $t$, and $u'_i \leadsto u_1$
we can find a specialization $r' \leadsto r$ such that $t(r') = u'_i$.
Set $u' = s(r')$. Then $u' \leadsto u$ and $\varphi(u') = x'$ as
desired.
\end{proof}

\begin{lemma}
\label{lemma-kolmogorov}
Let $S$ be a scheme.
Let $X$ be an algebraic space over $S$.
Assume that for every $x \in |X|$ the equivalent conditions
of Lemma \ref{lemma-UR-finite-above-x} hold, i.e., $X$ is decent.
Then $|X|$ is Kolmogorov (see
Topology, Definition \ref{topology-definition-generic-point}).
\end{lemma}

\begin{proof}
Let $x_1, x_2 \in |X|$ with $x_1 \leadsto x_2$ and $x_2 \leadsto x_1$.
We have to show that $x_1 = x_2$. Pick a scheme $U$ and an \'etale morphism
$U \to X$ such that $x_1, x_2$ are both in the image of $|U| \to |X|$.
By Lemma \ref{lemma-specialization} we can find a specialization
$u_1 \leadsto u_2$ in $U$ mapping to $x_1 \leadsto x_2$.
By Lemma \ref{lemma-specialization} we can find
$u_2' \leadsto u_1$ mapping to $x_2 \leadsto x_1$. This means that
$u_2' \leadsto u_2$ is a specialization between points of $U$ mapping to
the same point of $X$, namely $x_2$. This is not possible, unless
$u_2' = u_2$, see
Lemma \ref{lemma-no-specializations-map-to-same-point}. Hence
also $u_1 = u_2$ as desired.
\end{proof}
















\section{Points on very reasonable spaces}
\label{section-points-very-reasonable}

\noindent
In this section we prove some properties of points on
very reasonable algebraic spaces.

\begin{lemma}
\label{lemma-very-reasonable-points-monomorphism}
Let $S$ be a scheme. Let $X$ be an algebraic space over $S$.
Consider the map
$$
\{\text{Spec}(k) \to X \text{ monomorphism}\}
\longrightarrow
|X|
$$
This map is always injective. If $X$ is very reasonable then this map
is a bijection.
\end{lemma}

\begin{proof}
We have seen in Lemma \ref{lemma-points-monomorphism}
that the map is an injection in general.
By Lemma \ref{lemma-bounded-fibres} it is surjective when $X$ is
very reasonable.
\end{proof}

\noindent
The following lemma is a tiny bit stronger than
Lemma \ref{lemma-point-like-spaces}.
We will improve this lemma in Lemma \ref{lemma-when-field}.

\begin{lemma}
\label{lemma-very-reasonable-point-like-spaces}
Let $S$ be a scheme. Let $k$ be a field.
Let $X$ be an algebraic space over $S$ and assume that there exists
a surjective \'etale morphism $\text{Spec}(k) \to X$.
If $X$ is very reasonable, then $X \cong \text{Spec}(k')$
where $k' \subset k$ is a finite separable extension.
\end{lemma}

\begin{proof}
This can be proved directly by adding a few words to the proof of
Lemma \ref{lemma-point-like-spaces},
but we think it is fun to deduce it from the results obtained so far.
By Lemma \ref{lemma-very-reasonable-points-monomorphism}
we see that $\text{Spec}(k) \to X$ factors as
$\text{Spec}(k) \to \text{Spec}(k') \to X$ where
$\text{Spec}(k') \to X$ is a monomorphism. But since $\text{Spec}(k) \to X$
is a surjection of sheaves on $(\textit{Sch}/S)_{fppf}$, we see that
also $\text{Spec}(k') \to X$ is surjective (as a map of sheaves). But a
map of sheaves which is both injective and surjective is an isomorphism.
Finally, the fact that $\text{Spec}(k) \to X$ is \'etale means that
$k \otimes_{k'} k$ is \'etale over $k$, which implies easily that
$k' \subset k$ is a finite separable extension.
\end{proof}

\noindent
The following lemma shows that specialization of points behaves in a
reasonable manner on very reasonable algebraic spaces.
Spaces, Example \ref{spaces-example-infinite-product}
shows that this is {\bf not} true in general.

\begin{lemma}
\label{lemma-very-reasonable-no-specializations-map-to-same-point}
Let $S$ be a scheme.
Let $X$ be a very reasonable algebraic space over $S$.
Let $U \to X$ be an \'etale morphism from a scheme to $X$.
If $u, u' \in |U|$ map to the same point of $|X|$, and
$u' \leadsto u$, then $u = u'$.
\end{lemma}

\begin{proof}
Combine Lemmas \ref{lemma-bounded-fibres} and
\ref{lemma-no-specializations-map-to-same-point}.
\end{proof}

\begin{lemma}
\label{lemma-very-reasonable-specialization}
Let $S$ be an algebraic space.
Let $X$ be an algebraic space over $S$.
Let $x, x' \in |X|$ and assume $x' \leadsto x$, i.e., $x$ is a
specialization of $x'$.
Assume $X$ is very reasonable. Then for every \'etale morphism
$\varphi : U \to X$ from a scheme $U$ and any $u \in U$ with
$\varphi(u) = x$, exists a point $u'\in U$, $u' \leadsto u$ with
$\varphi(u') = x'$.
\end{lemma}

\begin{proof}
Combine Lemmas \ref{lemma-bounded-fibres} and
\ref{lemma-specialization}.
\end{proof}

\begin{lemma}
\label{lemma-very-reasonable-kolmogorov}
Let $S$ be a scheme.
Let $X$ be a very reasonable algebraic space over $S$.
Then $|X|$ is Kolmogorov (see
Topology, Definition \ref{topology-definition-generic-point}).
\end{lemma}

\begin{proof}
Combine Lemmas \ref{lemma-bounded-fibres} and
\ref{lemma-kolmogorov}.
\end{proof}

\begin{proposition}
\label{proposition-very-reasonable-sober}
Let $S$ be a scheme.
Let $X$ be a very reasonable algebraic space over $S$.
Then the topological space $|X|$ is sober (see
Topology, Definition \ref{topology-definition-generic-point}).
\end{proposition}

\begin{proof}
We have seen in
Lemma \ref{lemma-very-reasonable-kolmogorov}
that $|X|$ is Kolmogorov.
Hence it remains to show that every irreducible closed subset
$T \subset |X|$ has a generic point. By
Lemma \ref{lemma-reduced-closed-subspace}
there exists a closed subspace $Z \subset X$ with $|Z| = |T|$.
By definition this means that $Z \to X$ is a representable morphism
of algebraic spaces. Hence $Z$ is a very reasonable algebraic space
by Lemma \ref{lemma-representable-very-reasonable}. By
Proposition \ref{proposition-very-reasonable-open-dense-scheme}
we see that there exists an open dense subspace $Z' \subset Z$
which is a scheme. This means that $|Z'| \subset T$ is open dense.
Hence the topological space $|Z'|$ is irreducible, which means that
$Z'$ is an irreducible scheme. By
Schemes, Lemma \ref{schemes-lemma-scheme-sober}
we conclude that $|Z'|$ is the closure of a single point $\eta \in T$
and hence also $T = \overline{\{\eta\}}$, and we win.
\end{proof}










\section{Noetherian spaces}
\label{section-noetherian}

\noindent
We have already defined locally Noetherian algebraic spaces in
Section \ref{section-types-properties}.

\begin{definition}
\label{definition-noetherian}
Let $S$ be a scheme. Let $X$ be an algebraic space over $S$.
We say $X$ is {\it Noetherian} if $X$ is quasi-compact, quasi-separated
and locally Noetherian.
\end{definition}

\noindent
Note that a Noetherian algebraic space $X$ is not just quasi-compact
and locally Noetherian, but also quasi-separated. This does not conflict
with the definition of a Noetherian scheme, as a locally Noetherian
scheme is quasi-separated, see
Properties, Lemma \ref{properties-lemma-locally-Noetherian-quasi-separated}.
This does not hold for algebraic spaces. Namely,
$X = [\mathbf{A}^1_k/\mathbf{Z}]$, see
Spaces, Example \ref{spaces-example-affine-line-translation}
is locally Noetherian and quasi-compact but not quasi-separated
(hence not Noetherian according to our definitions).

\medskip\noindent
A consequence of the choice made above is that an algebraic space
of finite type over a Noetherian algebraic space is not automatically
Noetherian, i.e., the analogue of
Morphisms, Lemma \ref{morphisms-lemma-finite-type-noetherian}
does not hold. The correct statement is that an algebraic space of
finite presentation over a Noetherian algebraic space is Noetherian
(see
Morphisms of Spaces,
Lemma \ref{spaces-morphisms-lemma-finite-presentation-noetherian}).

\medskip\noindent
A Noetherian algebraic space $X$ is very close to being a scheme.
In the rest of this section we collect some lemmas to illustrate this.

\begin{lemma}
\label{lemma-Noetherian-topology}
Let $S$ be a scheme. Let $X$ be an algebraic space over $S$.
\begin{enumerate}
\item If $X$ is locally Noetherian then $|X|$ is a locally Noetherian
topological space.
\item If $X$ is quasi-compact and locally Noetherian, then $|X|$
is a Noetherian toplogical space.
\end{enumerate}
\end{lemma}

\begin{proof}
Assume $X$ is locally Noetherian.
Choose a scheme $U$ and a surjective \'etale morphism
$U \to X$. As $X$ is locally Noetherian we see that $U$ is locally
Noetherian. By
Properties, Lemma \ref{properties-lemma-Noetherian-topology}
this means that $|U|$ is a locally Noetherian topological space.
Since $|U| \to |X|$ is open and surjective we conclude that
$|X|$ is locally Noetherian by
Topology, Lemma \ref{topology-lemma-image-Noetherian}.
This proves (1). If $X$ is quasi-compact and locally Noetherian,
then $|X|$ is quasi-compact and locally Noetherian. Hence $|X|$
is Noetherian by
Topology,
Lemma \ref{topology-lemma-quasi-compact-locally-Noetherian-Noetherian}.
\end{proof}

\begin{lemma}
\label{lemma-Noetherian-sober}
Let $S$ be a scheme. Let $X$ be an algebraic space over $S$.
If $X$ is Noetherian, then $|X|$ is a sober Noetherian topological space.
\end{lemma}

\begin{proof}
A quasi-separated algebraic space is very reasonable, see
Lemma \ref{lemma-quasi-separated-very-reasonable}.
Hence $|X|$ is sober by
Proposition \ref{proposition-very-reasonable-sober}.
It is Noetherian by
Lemma \ref{lemma-Noetherian-topology}.
\end{proof}




\section{\'Etale morphisms of algebraic spaces}
\label{section-etale-morphisms}

\noindent
This section really belongs in the chapter on morphisms of algebraic
spaces, but we need the notion of an algebraic space \'etale over another
in order to define the small \'etale site of an algebraic space.
Thus we need to do some preliminary work on \'etale morphisms from schemes to
algebraic spaces, and \'etale morphisms between algebraic spaces.
For more about \'etale morphisms of algebraic spaces, see
Morphisms of Spaces, Section \ref{spaces-morphisms-section-etale}.

\begin{lemma}
\label{lemma-etale-over-space}
Let $S$ be a scheme.
Let $X$ be an algebraic space over $S$.
Let $U$, $U'$ be schemes over $S$.
\begin{enumerate}
\item If $U \to U'$ is an \'etale morphism of schemes, and
if $U' \to X$ is an \'etale morphism from $U'$ to $X$, then the
composition $U \to X$ is an \'etale morphism from $U$ to $X$.
\item If $\varphi : U \to X$ and $\varphi' : U' \to X$ are
\'etale morphisms towards $X$, and if $\chi : U \to U'$ is a
morphism of schemes such that $\varphi = \varphi' \circ \chi$,
then $\chi$ is an \'etale morphism of schemes.
\end{enumerate}
\end{lemma}

\begin{proof}
Recall that our definition of an \'etale morphism from a scheme into an
algebraic space comes from
Spaces, Definition \ref{spaces-definition-relative-representable-property}
via the fact that any morphism from a scheme into an algebraic space
is representable. Part (1) of the lemma follows from this, the fact that
\'etale morphisms are preserved under composition
(Morphisms, Lemma \ref{morphisms-lemma-composition-etale})
and
Spaces, Lemmas
\ref{spaces-lemma-morphism-schemes-gives-representable-transformation-property}
and
\ref{spaces-lemma-composition-representable-transformations-property}
(which are formal).
To prove part (2) choose a scheme $W$ over $S$ and a
surjective \'etale morphism $W \to X$. Consider the base change
$\chi_W : W \times_X U \to W \times_X U'$ of $\chi$.
As $W \times_X U$ and $W \times_X U'$ are \'etale over $W$, we conclude that
$\chi_W$ is \'etale, by
Morphisms, Lemma \ref{morphisms-lemma-etale-permanence-two}.
On the other hand, in the commutative diagram
$$
\xymatrix{
W \times_X U \ar[r] \ar[d] & W \times_X U' \ar[d] \\
U \ar[r] & U'
}
$$
the two vertical arrows are \'etale and surjective.
Hence by
Descent, Lemma \ref{descent-lemma-syntomic-smooth-etale-permanence}
we conclude that $U \to U'$ is \'etale.
\end{proof}

\begin{definition}
\label{definition-etale}
Let $S$ be a scheme.
A morphism $f : X \to Y$ between algebraic spaces over $S$ is
called {\it \'etale} if and only if for every \'etale morphism
$\varphi : U \to X$ where $U$ is a scheme, the composition
$\varphi \circ f$ is \'etale also.
\end{definition}

\noindent
If $X$ and $Y$ are schemes, then this agree with the usual notion of an
\'etale morphism of schemes. In fact, whenever $X \to Y$ is a representable
morphism of algebraic spaces, then this agrees with the notion defined via
Spaces, Definition \ref{spaces-definition-relative-representable-property}.
This follows by combining Lemma \ref{lemma-etale-local} below and
Spaces, Lemma \ref{spaces-lemma-representable-morphisms-spaces-property}.

\begin{lemma}
\label{lemma-etale-local}
Let $S$ be a scheme.
Let $f : X \to Y$ be a morphism of algebraic spaces over $S$.
The following are equivalent:
\begin{enumerate}
\item $f$ is \'etale,
\item there exists a surjective \'etale morphism $\varphi : U \to X$,
where $U$ is a scheme, such that the composition $f \circ \varphi$ is
\'etale (as a morphism of algebraic spaces),
\item there exists a surjective \'etale morphism $\psi : V \to Y$,
where $V$ is a scheme, such that the base change $V \times_X Y \to V$
is \'etale (as a morphism of algebraic spaces),
\item there exists a commutative diagram
$$
\xymatrix{
U \ar[d] \ar[r] & V \ar[d] \\
X \ar[r] & Y
}
$$
where $U$, $V$ are schemes, the vertical arrows are \'etale, and the
left vertical arrow is surjective such that the horizontal arrow is \'etale.
\end{enumerate}
\end{lemma}

\begin{proof}
Let us prove that (4) implies (1). Assume a diagram as in (4) given.
Let $W \to X$ be an \'etale morphism with $W$ a scheme. Then we see
that $W \times_X U \to U$ is \'etale. Hence $W \times_X U \to V$ is \'etale,
and also $W \times_X U \to Y$ is \'etale by
Lemma \ref{lemma-etale-over-space} (1). Since also
the projection $W \times_X U \to W$ is surjective and \'etale, we conclude
from Lemma \ref{lemma-etale-over-space} (2) that $W \to Y$ is \'etale.

\medskip\noindent
Let us prove that (1) implies (4). Assume (1). Choose a commutative diagram
$$
\xymatrix{
U \ar[d] \ar[r] & V \ar[d] \\
X \ar[r] & Y
}
$$
where $U \to X$ and $V \to Y$ are surjective and \'etale, see
Spaces, Lemma \ref{spaces-lemma-lift-morphism-presentations}.
By assumption the morphism $U \to Y$ is \'etale,
and hence $U \to V$ is \'etale by Lemma \ref{lemma-etale-over-space} (2).

\medskip\noindent
We omit the proof that (2) and (3) are also equivalent to (1).
\end{proof}

\begin{lemma}
\label{lemma-composition-etale}
The composition of two \'etale morphisms of algebraic spaces
is \'etale.
\end{lemma}

\begin{proof}
This is immediate from the definition.
\end{proof}

\begin{lemma}
\label{lemma-base-change-etale}
The base change of an \'etale morphism of algebraic spaces
by any morphism of algebraic spaces is \'etale.
\end{lemma}

\begin{proof}
Let $X \to Y$ be an \'etale morphism of algebraic spaces over $S$.
Let $Z \to Y$ be a morphism of algebraic spaces.
Choose a scheme $U$ and a surjective \'etale morphism $U \to X$.
Choose a scheme $W$ and a surjective \'etale morphism $W \to Z$.
Then $U \to Y$ is \'etale, hence in the diagram
$$
\xymatrix{
W \times_Y U \ar[d] \ar[r] & W \ar[d] \\
Z \times_Y X \ar[r] & Z
}
$$
the top horizontal arrow is \'etale.
Moreover, the left vertical arrow is surjective
and \'etale (verification omitted). Hence we conclude that the lower
horizontal arrow is \'etale by Lemma \ref{lemma-etale-local}.
\end{proof}

\begin{lemma}
\label{lemma-etale-permanence}
Let $S$ be a scheme. Let $X, Y, Z$ be algebraic spaces.
Let $g : X \to Z$, $h : Y \to Z$ be \'etale morphisms and let
$f : X \to Y$ be a morphism such that $h \circ f = g$.
Then $f$ is \'etale.
\end{lemma}

\begin{proof}
Choose a commutative diagram
$$
\xymatrix{
U \ar[d] \ar[r]_\chi & V \ar[d] \\
X \ar[r] & Y
}
$$
where $U \to X$ and $V \to Y$ are surjective and \'etale, see
Spaces, Lemma \ref{spaces-lemma-lift-morphism-presentations}.
By assumption the morphisms $\varphi : U \to X \to Z$ and
$\psi : V \to Y \to Z$ are \'etale. Moreover, $\psi \circ \chi = \varphi$
by our assumption on $f, g, h$.
Hence $U \to V$ is \'etale by Lemma \ref{lemma-etale-over-space}
part (2).
\end{proof}

\begin{lemma}
\label{lemma-etale-open}
Let $S$ be a scheme.
If $X \to Y$ is an \'etale morphism of algebraic spaces over $S$,
then the associated map $|X| \to |Y|$ of topological spaces
is open.
\end{lemma}

\begin{proof}
This is clear from the diagram in
Lemma \ref{lemma-etale-local} and Lemma \ref{lemma-topology-points}.
\end{proof}

\noindent
Finally, here is a fun lemma. It is not true that an algebraic space
with an \'etale morphism towards a scheme is a scheme, see
Spaces, Example \ref{spaces-example-non-representable-descent}.
But it is true if the target is the spectrum of a field.

\begin{lemma}
\label{lemma-etale-over-field-scheme}
Let $S$ be a scheme. Let $X \to \text{Spec}(k)$
be \'etale morphism over $S$, where $k$ is a field.
Then $X$ is a scheme.
\end{lemma}

\begin{proof}
Let $U$ be an affine scheme, and let $U \to X$ be an \'etale morphism. By
Definition \ref{definition-etale}
we see that $U \to \text{Spec}(k)$ is an \'etale
morphism. Hence $U = \coprod_{i = 1, \ldots, n} \text{Spec}(k_i)$
is a finite disjoint union of spectra of finite separable extensions
$k_i$ of $k$, see
Remark \ref{remark-recall}.
The $R = U \times_X U \to U \times_{\text{Spec}(k)} U$ is a monomorphism
and $U \times_{\text{Spec}(k)} U$ is also a finite disjoint union of
spectra of finite separable extensions of $k$. Hence by
Schemes, Lemma \ref{schemes-lemma-mono-towards-spec-field}
we see that $R$ is similarly a finite disjoint union of
spectra of finite separable extensions of $k$.
This $U$ and $R$ are affine and
both projections $R \to U$ are finite locally free.
Hence $U/R$ is a scheme by
Groupoids, Lemma \ref{groupoids-proposition-finite-flat-equivalence}.
By
Spaces, Lemma \ref{spaces-lemma-finding-opens}
it is also an open subspace of $X$. By
Lemma \ref{lemma-subscheme}
we conclude that $X$ is a scheme.
\end{proof}



\section{Special coverings}
\label{section-special-coverings}

\noindent
In this section we collect some straightforward lemmas on the existence
of \'etale surjective coverings of algebraic spaces.

\begin{lemma}
\label{lemma-cover-by-union-affines}
Let $S$ be a scheme.
Let $X$ be an algebraic space over $S$.
There exists a surjective \'etale morphism $U \to X$ where
$U$ is a disjoint union of affine schemes.
We may in addition assume each of these affines
maps into an affine open of $S$.
\end{lemma}

\begin{proof}
Let $V \to X$ be a surjective \'etale morphism.
Let $V = \bigcup_{i \in I} V_i$ be a Zariski open covering
such that each $V_i$ maps into an affine open of $S$.
Then set $U = \coprod_{i \in I} V_i$ with induced morphism
$U \to V \to X$. This is \'etale according to
Lemma \ref{lemma-etale-over-space} and it is clearly surjective.
\end{proof}

\begin{lemma}
\label{lemma-union-of-quasi-compact}
Let $S$ be a scheme.
Let $X$ be an algebraic space over $S$.
There exists a Zariski covering $X = \bigcup X_i$
such that each algebraic space $X_i$ has a surjective
\'etale covering by an affine scheme. We may in addition assume
each $X_i$ maps into an affine open of $S$.
\end{lemma}

\begin{proof}
By Lemma \ref{lemma-cover-by-union-affines} we can find a surjective
\'etale morphism $U = \coprod U_i \to X$, with $U_i$ affine and mapping
into an affine open of $S$. According to
Spaces, Lemma \ref{spaces-lemma-space-presentation},
$R$ is an \'etale equivalence relation on $U$ and we have $X = U/R$. Set
$R_i = U_i \times_X U_i$; this is also the restriction $R|_{U_i}$ of $R$
to $U_i$. By
Spaces, Lemma \ref{spaces-lemma-finding-opens}
we see that $X_i = U_i/R_i$ is an open subspace of $X$. Since
$X = \bigcup X_i$ is a Zariski covering, we get our covering.
As $U_i \to X_i$ is surjective it follows that $X_i \to S$ maps into
an affine open of $S$.
\end{proof}

\begin{lemma}
\label{lemma-quasi-compact-affine-cover}
Let $S$ be a scheme.
Let $X$ be an algebraic space over $S$.
Then $X$ is quasi-compact if and only if
there exists an \'etale surjective morphism $U \to X$
with $U$ an affine scheme.
\end{lemma}

\begin{proof}
If there exists an \'etale surjective morphism $U \to X$ with $U$
affine then $X$ is quasi-compact by Definition \ref{definition-quasi-compact}.
Conversely, if $X$ is quasi-compact, then $|X|$ is quasi-compact.
Let $U = \coprod_{i \in I} U_i$ be a disjoint union of affine schemes
with an \'etale and surjective map $\varphi : U \to X$
(Lemma \ref{lemma-cover-by-union-affines}).
Then $|X| = \bigcup \varphi(|U_i|)$ and
by quasi-compactness there is a finite subset $i_1, \ldots, i_n$
such that $|X| = \bigcup \varphi(|U_{i_j}|)$. Hence
$U_{i_1} \cup \ldots \cup U_{i_n}$ is an affine scheme with a 
finite surjective morphism towards $X$.
\end{proof}

\begin{lemma}
\label{lemma-algebraic-space-affine-cover}
Let $S$ be a scheme.
Let $X$ be an algebraic space over $S$.
Let $U$ be an affine scheme, or a disjoint union of affine schemes,
and $U \to X$ surjective \'etale.
Then $U \to X$ is separated, and $R = U \times_X U$ is a separated scheme.
\end{lemma}

\begin{proof}
Consider the commutative diagram
$$
\xymatrix{
R = U \times_X U \ar[r] \ar[d] & U \ar[d] \\
U \ar[r] & X
}
$$
In the proof of Spaces, Lemma \ref{spaces-lemma-properties-diagonal}
we have seen that $j : R \to U \times_S U$ is separated.
The morphism of schemes $U \to S$ is separated as $U$ is a separated
scheme, see 
Schemes, Lemma \ref{schemes-lemma-compose-after-separated}.
Hence $U \times_S U \to U$ is separated as a base change, see
Schemes, Lemma \ref{schemes-lemma-separated-permanence}.
Hence the scheme $U \times_S U$ is separated (by the same lemma).
Since $j$ is separated we see in the same way that $R$ is separated.
Hence $R \to U$ is a separated morphism (by
Schemes, Lemma \ref{schemes-lemma-compose-after-separated}
again). Thus by
Spaces, Lemma \ref{spaces-lemma-representable-morphisms-spaces-property}
and the diagram above we conclude that $U \to X$ is separated.
\end{proof}

\begin{lemma}
\label{lemma-very-reasonable-quasi-compact-pieces}
Let $S$ be a scheme.
Let $X$ be a very reasonable algebraic space over $S$.
There exists a set of schemes
$U_i$ and morphisms $U_i \to X$ such that
\begin{enumerate}
\item each $U_i$ is a quasi-compact scheme,
\item each $U_i \to X$ is \'etale,
\item both projections $U_i \times_X U_i \to U_i$ are quasi-compact, and
\item the morphism $\coprod U_i \to X$ is surjective (and \'etale).
\end{enumerate}
\end{lemma}

\begin{proof}
Definition \ref{definition-very-reasonable}
says that there exist $U_i \to X$ such that (2), (3) and (4) hold.
Fix $i$, and set $R_i = U_i \times_X U_i$, and denote $s, t : R_i \to U_i$
the projections.
For any affine open $W \subset U_i$ the open $W' = t(s^{-1}(W)) \subset U_i$
is a quasi-compact $R_i$-invariant open (see
Groupoids, Lemma \ref{groupoids-lemma-constructing-invariant-opens}).
Hence $W'$ is a quasi-compact scheme, $W' \to X$ is \'etale, and
$W' \times_X W' = s^{-1}(W') = t^{-1}(W')$ so both projections
$W' \times_X W' \to W'$ are quasi-compact. This means the family of
$W' \to X$, where $W \subset U_i$ runs through the members of affine
open coverings of the $U_i$ gives what we want.
\end{proof}

\begin{lemma}
\label{lemma-quasi-separated-quasi-compact-pieces}
Let $S$ be a scheme.
Let $X$ be an algebraic space over $S$.
Assume $X$ is Zariski locally quasi-separated over $S$.
There exists a Zariski open covering $X = \bigcup X_i$ such
that for each $i$ there exists an affine scheme
$U_i$ and a separated quasi-compact surjective \'etale
morphism $U_i \to X_i$.
\end{lemma}

\begin{proof}
There is an immediate reduction to the case where $X$ is quasi-separated. By
Lemma \ref{lemma-union-of-quasi-compact}
we can find a Zariski open covering $X = \bigcup X_i$ such that each
$X_i$ maps into an affine open of $S$, and such that there exist affine
schemes $U_i$ and surjective \'etale morphisms $U_i \to X_i$. By
Lemma \ref{lemma-algebraic-space-affine-cover}
the morphisms $U_i \to X_i$ are separated. Since $U_i \to S$ maps into
an affine open of $S$ we see that $U_i \times_S U_i$ is affine, see
Schemes, Section \ref{schemes-section-fibre-products}.
As $X$ is quasi-separated over $S$, the morphisms
$$
R_i = U_i \times_{X_i} U_i = U_i \times_X U_i
\longrightarrow
U_i \times_S U_i
$$
as base changes of $\Delta_{X/S}$ are quasi-compact. Hence we conclude
that $R_i$ is a quasi-compact scheme. This in turn implies that each
projection $R_i \to U_i$ is quasi-compact. Hence, applying
Spaces, Lemma \ref{spaces-lemma-representable-morphisms-spaces-property}
to the covering $U_i \to X$ and the morphism $U_i \to X$
we conclude that the morphisms $U_i \to X_i$ are quasi-compact as desired.
\end{proof}









\section{Spaces and fpqc coverings}
\label{section-fpqc}

\noindent
Let $S$ be a scheme.
An algebraic space over $S$ is defined as a sheaf in the fppf topology with
additional properties. Hence it is not clear that it satisfies the sheaf
property for the fpqc topology (see
Topologies, Definition \ref{topologies-lemma-sheaf-property-fpqc}).
In this section we discuss this question. However, when we say that
the algebraic space $X$ satisfies the sheaf property for the fpqc topology
we really only consider fpqc coverings $\{f_i : T_i \to T\}_{i \in I}$ such
that $T, T_i$ are objects of the big site $(\textit{Sch}/S)_{fppf}$ (as per
our conventions, see Section \ref{section-conventions}). We first address the
question as to whether an algebraic space is separated as a presheaf for the
fpqc topology.

\begin{lemma}
\label{lemma-separated-fpqc}
Let $S$ be a scheme.
Let $X$ be an algebraic space over $S$.
Let $\{f_i : T_i \to T\}_{i \in I}$ be a fpqc covering of schemes over $S$.
Then the map
$$
\text{Mor}_S(T, X)
\longrightarrow
\prod\nolimits_{i \in I} \text{Mor}_S(T_i, X)
$$
is injective.
\end{lemma}

\begin{proof}
Let $a, b : T \to X$ be two morphisms such that $a \circ f_i = b \circ f_i$
for all $i$. Consider the fibre product
$$
T' = X \times_{\Delta_{X/S}, X \times_S X, (a, b)} T.
$$
By
Spaces, Lemma \ref{spaces-lemma-properties-diagonal}
the morphism $\Delta_{X/S}$ is a representable monomorphism which is
locally of finite type. Hence $T' \to T$ is a monomorphism of schemes which
is locally of finite type. Our assumption that $a \circ f_i = b \circ f_i$
implies that each $T_i \to T$ factors (uniquely) through $T'$. Consider
the commutative diagram
$$
\xymatrix{
T_i \times_T T' \ar[r] \ar[d] & T' \ar[d] \\
T_i \ar[r] \ar@/^5ex/[u] \ar[ru] & T
}
$$
Since the projection $T_i \times_T T' \to T_i$ is a monomorphism
with a section we conclude it is an isomorphism. Hence we conclude that
$T' \to T$ is an isomorphism by
Descent, Lemma \ref{descent-lemma-descending-property-isomorphism}.
This means $a = b$ as desired.
\end{proof}

\begin{lemma}
\label{lemma-sheaf-fpqc-open-covering}
Let $S$ be a scheme.
Let $X$ be an algebraic space over $S$.
Let $X = \bigcup_{j \in J} X_j$ be a Zariski covering, see
Spaces, Definition \ref{spaces-definition-Zariski-open-covering}.
If each $X_j$ satisfies the sheaf property for the fpqc topology
then $X$ satisfies the sheaf property for the fpqc topology.
\end{lemma}

\begin{proof}
Assume each $X_j$ satisfies the sheaf property for the fpqc topology.
Let $\{f_i : T_i \to T\}_{i \in I}$ be a fpqc covering of schemes over $S$.
Let $a_i \in \text{Mor}_S(T_i, X)$ be such that
$a_i \circ \text{pr}_0 = a_{i'} \circ \text{pr}_1$ as morphisms
$T_i \times_T T_{i'} \to X$. We have to prove that these morphisms
come from a morphism $a : T \to X$. Consider the open subsets
$T_{i, j} = a_i^{-1}(X_j)$. Then it is clear that
$\text{pr}_0^{-1}(T_{i, j}) = \text{pr}_1^{-1}(T_{i', j})$ as open
subsets of $T_i \times_T T_{i'}$. Hence there exist open subsets
$U_j \subset T$ such that $T_{i, j} = f_i^{-1}(U_j)$, see
Descent, Lemma \ref{descent-lemma-open-fpqc-covering}.
In particular, $\{T_{i, j} \to U_j\}_{i \in I}$ is a fpqc covering
of $U_j$, and the morphisms $a_{i, j} = a_i|_{T_{i, j}}$ are morphisms
into $X_j$. By assumption there exist morphisms $\alpha_j : U_j \to X_j$
such that $T_{i, j} \to U_j \to X_j$ agrees with $a_{i, j}$.
By Lemma \ref{lemma-separated-fpqc} we conclude that
$\alpha_j|_{U_j \cap U_{j'}}$ agrees with $\alpha_{j'}|_{U_j \cap U_{j'}}$.
Hence, since $X$ is a sheaf for the Zariski topology we conclude that
the $\alpha_j$ glue to a morphism $a : T \to X$.
By construction we have $a \circ f_i|_{T_{i, j}} = a_{i, j} = a_i|_{T_{i, j}}$.
Using the sheaf condition for the Zariski topology one more time we
conclude that $a \circ f_i = a_i$ as desired.
\end{proof}

\begin{lemma}
\label{lemma-sheaf-fpqc-quasi-separated}
Let $S$ be a scheme.
Let $X$ be an algebraic space over $S$.
If $X$ is Zariski locally quasi-separated over $S$, then $X$ satisfies
the sheaf condition for the fpqc topology.
\end{lemma}

\begin{proof}
By
Lemmas \ref{lemma-quasi-separated-quasi-compact-pieces}
and \ref{lemma-sheaf-fpqc-open-covering}
we may assume there exists an affine scheme $U$ and a
surjective \'etale separated quasi-compact morphism $\varphi : U \to X$.

\medskip\noindent
Let $\{f_i : T_i \to T\}_{i \in I}$ be a fpqc covering of schemes over $S$.
Let $a_i \in \text{Mor}_S(T_i, X)$ be such that
$a_i \circ \text{pr}_0 = a_j \circ \text{pr}_1$ as morphisms
$T_i \times_T T_j \to X$. We have to prove that these morphisms
come from a morphism $a : T \to X$. Consider the schemes
$$
W_i = T_i \times_{a_i, X, \varphi} U.
$$
The strucure morphisms $W_i \to T_i$ are surjective, separated, quasi-compact
and \'etale, in particular also quasi-finite (see Remark \ref{remark-recall}).
Hence each $W_i \to T_i$ is quasi-affine, see
More on Morphisms,
Lemma \ref{more-morphisms-lemma-quasi-finite-separated-quasi-affine}.
For each pair of indices $i, j \in I$ we have canonical isomorphisms
\begin{align*}
W_i \times_T T_j & =
(T_i \times_{a_i, X, \varphi} U) \times_T T_j \\
& =
(T_i \times_T T_j) \times_{a_i \circ \text{pr}_0, X, \varphi} U \\
& =
(T_i \times_T T_j) \times_{a_j \circ \text{pr}_1, X, \varphi} U \\
& =
T_i \times_T (T_j \times_{a_j, X, \varphi} U) \\
& =
T_i \times_T W_j.
\end{align*}
These isomorphisms satsify the cocycle condition of
Descent,
Definition \ref{descent-definition-descent-datum-for-family-of-morphisms}
as $a_i$, $a_j$ and $a_k$ agree over $T_i \times_T T_j \times_T T_k$
(some details omitted). By
Descent, Lemma \ref{descent-lemma-quasi-affine}
this descent datum is effective and
we conclude that there exists a scheme $W \to T$ and isomorphisms
$T_i \times_T W = W_i$ compatible with the canonical descent datum
and the one given above. In particular we see that
$\{W_i \to W\}_{i \in I}$ is the base change of a fpqc covering,
and hence a fpqc covering. Note that by construction the
morphisms $b_i : W_i = T_i \times_X U \to U$ have the property
$b_i \circ \text{pr}_0 = b_j \circ \text{pr}_1$ as morphisms
$W_i \times_W W_j \to U$. Hence by
Descent, Lemma \ref{descent-lemma-fpqc-universal-effective-epimorphisms}
we see that there exists a morphism of schemes $b : W \to U$
which restricts to $b_i$ on $W_i$ for all $i$.

\medskip\noindent
By
Descent, Lemmas \ref{descent-lemma-descending-property-surjective}
and \ref{descent-lemma-descending-property-etale}
the morphism $W \to T$ is surjective and \'etale. Hence, in order to see
that $b$ gives rise to the morphism $a : T \to X$ we are looking for,
it suffices to show that $b \circ \text{pr}_0 = b \circ \text{pr}_1$
as morphisms $W \times_T W \to X$. For this we note that we do know
that $b_i \circ \text{pr}_0 = b_j \circ \text{pr}_1$ as morphisms
$W_i \times_T W_j \to X$, because
$$
W_i \times_T W_j = (T_i \times_T T_j)
\times_{(a_i, a_j), X \times_S X} U \times_S U
$$
and $(a_i, a_j) : T_i \times_T T_j \to X \times_S X$ factors
through $\Delta_{X/S}$ by assumption. In other words we conclude that
over the members of the fpqc covering $\{W_i \times_T W_j \to W \times_T W\}$
the morphisms $b \circ \text{pr}_0$ and $b \circ \text{pr}_1$ agree,
and hence by
Lemma \ref{lemma-separated-fpqc}
they agree. As $X$ is a sheaf for the fppf topology we obtain a unique
morphism $a : T \to X$ whose composition with $W \to T$ agrees with $b$.
We omit the final verification that $a|_{T_i} = a_i$.
\end{proof}

\begin{remark}
\label{remark-proof-works-when}
The proof of
Lemma \ref{lemma-sheaf-fpqc-quasi-separated}
works for any algebraic space which has a
Zariski covering $X = \bigcup X_i$ such that for each $i$ there exists
a surjective \'etale separated quasi-compact morphism $U_i \to X_i$
where $U_i$ is a scheme. This condition is slightly stronger than the
condition of being very reasonable, and the current proof does not work
even for very reasonable spaces. There are results in the literature, see
David Rydh's paper \cite{rydh_descent} and its references, to remedy this.
On the other hand, it seems that the question for general algebraic spaces
as defined in the stacks project is still open. If this is no longer the
case, please email
\href{mailto:stacks.project@gmail.com}{stacks.project@gmail.com}
so we can update this remark.
\end{remark}








\section{The \'etale site of an algebraic space}
\label{section-etale-site}

\noindent
In this section we define the small \'etale site of an algebraic space.
This is the analogue of the small \'etale site $S_{\acute{e}tale}$ of a scheme.
Lemma \ref{lemma-etale-over-space} implies that in the definition below
any morphism between objects of the \'etale site of $X$ is \'etale, and that
any scheme \'etale over an object of $X_{\acute{e}tale}$ is also an object of
$X_{\acute{e}tale}$.

\begin{definition}
\label{definition-etale-site}
Let $S$ be a scheme.
Let $\textit{Sch}_{fppf}$ be a big fppf site containing $S$,
and let $\textit{Sch}_{\acute{e}tale}$ be the corresponding big \'etale site
(i.e., having the same underlying category).
Let $X$ be an algebraic space over $S$.
The {\it small \'etale site $X_{\acute{e}tale}$} of $X$ is defined as follows:
\begin{enumerate}
\item An object of $X_{\acute{e}tale}$ is a morphism $\varphi : U \to X$
where $U \in \text{Ob}((\textit{Sch}/S)_{\acute{e}tale})$ is a scheme and
$\varphi$ is an \'etale morphism,
\item a morphism $(\varphi : U \to X) \to (\varphi' : U' \to X)$
is given by a morphism of schemes $\chi : U \to U'$ such that
$\varphi = \varphi' \circ \chi$, and
\item a family of morphisms $\{(U_i \to X) \to (U \to X)\}_{i \in I}$
of $X_{\acute{e}tale}$ is a covering if and only if $\{U_i \to U\}_{i \in I}$
is a covering of $(\textit{Sch}/S)_{\acute{e}tale}$.
\end{enumerate}
\end{definition}

\noindent
A consequence of our choice is that the \'etale site of an algebraic space
in general does not have a final object! On the other hand, if $X$ happens
to be a scheme, then the definition above agrees with
Topologies on Schemes,
Definition \ref{topologies-definition-big-small-etale}.

\medskip\noindent
There are several other choices we could have made here. For example
we could have considered all {\it algebraic spaces} $U$ which are \'etale
over $X$, or we could have considered all {\it affine schemes} $U$ which
are \'etale over $X$. We decided not to do so, since we like to think of
plain old schemes as the fundamental objects of algebraic geometry. On
the other hand, we do need these notions also, since the small \'etale site
of an algebraic space is not sufficiently flexible, especially when
discussing functoriality of the small \'etale site, see
Lemma \ref{lemma-functoriality-etale-site}
below.

\begin{definition}
\label{definition-spaces-etale-site}
Let $S$ be a scheme.
Let $\textit{Sch}_{fppf}$ be a big fppf site containing $S$,
and let $\textit{Sch}_{\acute{e}tale}$ be the corresponding big \'etale site
(i.e., having the same underlying category).
Let $X$ be an algebraic space over $S$.
The site {\it $X_{spaces, \acute{e}tale}$} of $X$ is defined as follows:
\begin{enumerate}
\item An object of $X_{spaces, \acute{e}tale}$ is a morphism
$\varphi : U \to X$ where $U$ is an algebraic space over $S$ and
$\varphi$ is an \'etale morphism of algebraic spaces over $S$,
\item a morphism $(\varphi : U \to X) \to (\varphi' : U' \to X)$ of
$X_{spaces, \acute{e}tale}$ is given by a morphism of algebraic spaces
$\chi : U \to U'$ such that $\varphi = \varphi' \circ \chi$, and
\item a family of morphisms
$\{\varphi_i : (U_i \to X) \to (U \to X)\}_{i \in I}$
of $X_{spaces, \acute{e}tale}$ is a covering if and only if
$|U| = \bigcup \varphi_i(|U_i|)$.
\end{enumerate}
(As usual we choose a set of coverings of this type, including at least
the coverings in $X_{\acute{e}tale}$, as in
Sets, Lemma \ref{sets-lemma-coverings-site}
to turn $X_{spaces, \acute{e}tale}$ into a site.)
\end{definition}

\noindent
Since the identity morphism of $X$ is \'etale it is clear that
$X_{spaces, \acute{e}tale}$ does have a final object.
Let us show right away that the corresponding topos equals the
small \'etale topos of $X$.

\begin{lemma}
\label{lemma-compare-etale-sites}
The functor
$$
X_{\acute{e}tale} \longrightarrow X_{spaces, \acute{e}tale}, \quad
U/X \longmapsto U/X
$$
is a special cocontinuous functor
(Sites, Definition \ref{sites-definition-special-cocontinuous-functor})
and hence induces an equivalence of topoi
$\textit{Sh}(X_{\acute{e}tale}) \to \textit{Sh}(X_{spaces, \acute{e}tale})$.
\end{lemma}

\begin{proof}
We have to show that the functor satisfies the assumptions (1) -- (5) of
Sites, Lemma \ref{sites-lemma-equivalence}.
It is clear that the functor is continuous and cocontinuous, which
proves assumptions (1) and (2).
Assumptions (3) and (4) hold simply because the functor is fully faithful.
Assumption (5) holds, because an algebraic space by definition has
a covering by a scheme.
\end{proof}

\begin{remark}
\label{remark-explain-equivalence}
Let us explain the meaning of Lemma \ref{lemma-compare-etale-sites}.
Let $S$ be a scheme, and let $X$ be an algebraic space over $S$.
Let $\mathcal{F}$ be a sheaf on the small \'etale site $X_{\acute{e}tale}$ of
$X$. The lemma says that there exists a unique sheaf $\mathcal{F}'$ on
$X_{spaces, \acute{e}tale}$ which restricts back to $\mathcal{F}$ on the
subcategory $X_{\acute{e}tale}$. If $U \to X$ is an \'etale morphism of
algebraic spaces, then how do we compute $\mathcal{F}'(U)$? Well, by definition
of an algebraic space there exists a scheme $U'$ and a surjective
\'etale morphism $U' \to U$. Then $\{U' \to U\}$ is a covering in
$X_{spaces, \acute{e}tale}$ and hence we get an equalizer diagram
$$
\xymatrix{
\mathcal{F}'(U) \ar[r] &
\mathcal{F}(U') \ar@<1ex>[r] \ar@<-1ex>[r] &
\mathcal{F}(U' \times_U U').
}
$$
Note that $U' \times_U U'$ is a scheme, and hence we may
write $\mathcal{F}$ and not $\mathcal{F}'$.
Thus we see how to compute $\mathcal{F}'$
when given the sheaf $\mathcal{F}$.
\end{remark}

\begin{lemma}
\label{lemma-alternative}
Let $S$ be a scheme.
Let $X$ be an algebraic space over $S$.
Let $X_{affine, \acute{e}tale}$ denote the full subcategory of
$X_{\acute{e}tale}$ whose objects are those
$U/X \in \text{Ob}(X_{\acute{e}tale})$ with $U$ affine.
A covering of $X_{affine, \acute{e}tale}$ will be a
standard \'etale covering, see
Topologies, Definition \ref{topologies-definition-standard-etale}.
Then restriction
$$
\mathcal{F} \longmapsto \mathcal{F}|_{X_{affine, \acute{e}tale}}
$$
defines an equivalence of topoi
$\textit{Sh}(S_{\acute{e}tale}) \cong \textit{Sh}(S_{affine, \acute{e}tale})$.
\end{lemma}

\begin{proof}
This you can show directly from the definitions, and is a good exercise.
But it also follows immediately from
Sites, Lemma \ref{sites-lemma-equivalence}
by checking that the inclusion functor
$X_{affine, \acute{e}tale} \to X_{\acute{e}tale}$ is a special cocontinuous
functor as in
Sites, Definition \ref{sites-definition-special-cocontinuous-functor}.
\end{proof}

\begin{definition}
\label{definition-etale-topos}
Let $S$ be a scheme. Let $X$ be an algebraic space over $S$.
The {\it \'etale topos} of $X$, or more precisely the
{\it small \'etale topos} of $X$ is the category
$\textit{Sh}(X_{\acute{e}tale})$
of sheaves of sets on $X_{\acute{e}tale}$.
\end{definition}

\noindent
By
Lemma \ref{lemma-compare-etale-sites}
we have
$\textit{Sh}(X_{\acute{e}tale}) = \textit{Sh}(X_{spaces, \acute{e}tale})$,
so we can also think of this as the category of sheaves of sets on
$X_{spaces, \acute{e}tale}$. Similarly, by
Lemma \ref{lemma-alternative}
we see that
$\textit{Sh}(X_{\acute{e}tale}) = \textit{Sh}(X_{affine, \acute{e}tale})$.
It turns out that the topos is functorial with respect to morphisms
of algebraic spaces. Here is a precise statement.

\begin{lemma}
\label{lemma-functoriality-etale-site}
Let $S$ be a scheme.
Let $f : X \to Y$ be a morphism of algebraic spaces over $S$.
\begin{enumerate}
\item The continuous functor
$$
Y_{spaces, \acute{e}tale} \longrightarrow X_{spaces, \acute{e}tale}, \quad
V \longmapsto X \times_Y V
$$
induces a morphism of sites
$$
f_{spaces, \acute{e}tale} :
X_{spaces, \acute{e}tale}
\to
Y_{spaces, \acute{e}tale}.
$$
\item The rule $f \mapsto f_{spaces, \acute{e}tale}$ is compatible with
compositions, in other words $(f \circ g)_{spaces, \acute{e}tale}
= f_{spaces, \acute{e}tale} \circ g_{spaces, \acute{e}tale}$ (see
Sites, Definition \ref{sites-definition-composition-morphisms-sites}).
\item The morphism of topoi associated to $f_{spaces, \acute{e}tale}$
induces, via Lemma \ref{lemma-compare-etale-sites}, a morphism of topoi
$f_{small} : \textit{Sh}(X_{\acute{e}tale}) \to \textit{Sh}(Y_{\acute{e}tale})$
whose construction is compatible with compositions.
\item If $f$ is a representable morphism of algebraic spaces,
then $f_{small}$ comes from a morphism of sites
$X_{\acute{e}tale} \to Y_{\acute{e}tale}$,
corresponding to the continuous functor $V \mapsto X \times_Y V$.
\end{enumerate}
\end{lemma}

\begin{proof}
Let us show that the functor described in (1) satisfies the assumptions
of Sites, Proposition \ref{sites-proposition-get-morphism}.
Thus we have to show that
$Y_{spaces, \acute{e}tale}$ has a final object (namely $Y$) and that
the functor transforms this into a final object in $X_{spaces, \acute{e}tale}$
(namely $X$). This is clear as $X \times_Y Y = X$ in any category.
Next, we have to show that $Y_{spaces, \acute{e}tale}$ has fibre products.
This is true since the category of algebraic spaces has fibre products,
and since $V \times_Y V'$ is \'etale over $Y$ if $V$ and $V'$ are \'etale
over $Y$ (see Lemmas \ref{lemma-composition-etale} and
\ref{lemma-base-change-etale} above).
OK, so the proposition applies and we see that we get a morphism
of sites as described in (1).

\medskip\noindent
Part (2) you get by unwinding the definitions.
Part (3) is clear by using the equivalences for $X$ and $Y$
from Lemma \ref{lemma-compare-etale-sites} above.
Part (4) follows, because if $f$ is representable, then the
functors above fit into a commutative diagram
$$
\xymatrix{
X_{\acute{e}tale} \ar[r] &
X_{spaces, \acute{e}tale} \\
Y_{\acute{e}tale} \ar[r] \ar[u] &
Y_{spaces, \acute{e}tale} \ar[u]
}
$$
of categories.
\end{proof}

\noindent
We can do a little bit better than the lemma above in describing
the relationship between sheaves on $X$ and sheaves on $Y$.
Namely, we can formulate this in turns of $f$-maps, compare
Sheaves, Definition \ref{sheaves-definition-f-map}, as follows.

\begin{definition}
\label{definition-f-map}
Let $S$ be a scheme.
Let $f : X \to Y$ be a morphism of algebraic spaces over $S$.
Let $\mathcal{F}$ be a sheaf of sets on $X_{\acute{e}tale}$ and
let $\mathcal{G}$ be a sheaf of sets on $Y_{\acute{e}tale}$.
An {\it $f$-map $\varphi : \mathcal{G} \to \mathcal{F}$}
is a collection of maps
$\varphi_{(U,V,g)} : \mathcal{G}(V) \to \mathcal{F}(U)$
indexed by commutative diagrams
$$
\xymatrix{
U \ar[d]_g \ar[r] & X \ar[d]^f \\
V \ar[r] & Y
}
$$
where $U \in X_{\acute{e}tale}$, $V \in Y_{\acute{e}tale}$ such that whenever
given an extended diagram
$$
\xymatrix{
U' \ar[r] \ar[d]_{g'} & U \ar[d]_g \ar[r] & X \ar[d]^f \\
V' \ar[r] & V \ar[r] & Y
}
$$
with $V' \to V$ and $U' \to U$ \'etale morphisms of schemes the diagram
$$
\xymatrix{
\mathcal{G}(V)
\ar[rr]_{\varphi_{(U, V, g)}}
\ar[d]_{\text{restriction of }\mathcal{G}} & &
\mathcal{F}(U)
\ar[d]^{\text{restriction of }\mathcal{F}} \\
\mathcal{G}(V')
\ar[rr]^{\varphi_{(U', V', g')}} & &
\mathcal{F}(U')
}
$$
commutes.
\end{definition}

\begin{lemma}
\label{lemma-f-map}
Let $S$ be a scheme.
Let $f : X \to Y$ be a morphism of algebraic spaces over $S$.
Let $\mathcal{F}$ be a sheaf of sets on $X_{\acute{e}tale}$ and
let $\mathcal{G}$ be a sheaf of sets on $Y_{\acute{e}tale}$.
There are canonical bijections between the following three sets:
\begin{enumerate}
\item The set of maps $\mathcal{G} \to f_{small, *}\mathcal{F}$.
\item The set of maps $f_{small}^{-1}\mathcal{G} \to \mathcal{F}$.
\item The set of $f$-maps $\varphi : \mathcal{G} \to \mathcal{F}$.
\end{enumerate}
\end{lemma}

\begin{proof}
Note that (1) and (2) are the same because the functors $f_{small, *}$
and $f_{small}^{-1}$ are a pair of adjoint functors.
Suppose that $\alpha : f_{small}^{-1}\mathcal{G} \to \mathcal{F}$
is a map of sheaves on $Y_{\acute{e}tale}$. Let a diagram
$$
\xymatrix{
U \ar[d]_g \ar[r]_{j_U} & X \ar[d]^f \\
V \ar[r]^{j_V} & Y
}
$$
as in Definition \ref{definition-f-map} be given.
By the commutativity of the diagram we also get a map
$g_{small}^{-1}(j_V)^{-1}\mathcal{G} \to (j_U)^{-1}\mathcal{F}$
(compare Sites, Section \ref{sites-section-localize} for the
description of the localization functors). Hence we certainly
get a map
$\varphi_{(V, U, g)} :
\mathcal{G}(V) = (j_V)^{-1}\mathcal{G}(V) 
\to
(j_U)^{-1}\mathcal{F}(U) = \mathcal{F}(U)$.
We omit the verification that this rule is compatible with
further restrictions and defines an $f$-map from $\mathcal{G}$ to
$\mathcal{F}$.

\medskip\noindent
Conversely, suppose that we are given an $f$-map
$\varphi = (\varphi_{(U, V, g)})$.
Let $\mathcal{G}'$ (resp.\ $\mathcal{F}'$) denote the extension of
$\mathcal{G}$ (resp.\ $\mathcal{F}$) to $Y_{spaces, \acute{e}tale}$
(resp.\ $X_{spaces, \acute{e}tale}$), see
Lemma \ref{lemma-compare-etale-sites}.
Then we have to construct a map of sheaves
$$
\mathcal{G}' \longrightarrow (f_{spaces, \acute{e}tale})_*\mathcal{F}'
$$
To do this, let $V \to Y$ be an \'etale morphism of algebraic spaces.
We have to construct a map of sets
$$
\mathcal{G}'(V) \to \mathcal{F}'(X \times_Y V)
$$
Choose an \'etale surjective morphism $V' \to V$ with $V'$ a scheme,
and after that choose an \'etale surjective morphsm
$U' \to X \times_U V'$ with $U'$ a scheme. We get a morphism of
schemes $g' : U' \to V'$ and also a morphism of schemes
$$
g'' : U' \times_{X \times_Y V} U' \longrightarrow V' \times_V V'
$$
Consider the following diagram
$$
\xymatrix{
\mathcal{F}'(X \times_Y V) \ar[r] &
\mathcal{F}(U') \ar@<1ex>[r] \ar@<-1ex>[r] &
\mathcal{F}(U' \times_{X \times_Y V} U') \\
\mathcal{G}'(X \times_Y V) \ar[r] \ar@{..>}[u] &
\mathcal{G}(V') \ar@<1ex>[r] \ar@<-1ex>[r] \ar[u]_{\varphi_{(U', V', g')}} &
\mathcal{G}(V' \times_V V') \ar[u]_{\varphi_{(U'', V'', g'')}}
}
$$
The compatibility of the maps $\varphi_{...}$
with restriction shows that the two right squares commute.
The definition of coverings in $X_{spaces, \acute{e}tale}$ shows that
the horizontal rows are equalizer diagrams. Hence we get
the dotted arrow. We leave it to the reader to show that these
arrows are compatible with the restriction mappings.
\end{proof}

\noindent
If the morphism of algebraic spaces $X \to Y$ is \'etale, then the morphism
of topoi $\textit{Sh}(X_{\acute{e}tale}) \to \textit{Sh}(Y_{\acute{e}tale})$
is a localization. Here is a statement.

\begin{lemma}
\label{lemma-etale-morphism-topoi}
Let $S$ be a scheme, and let $f : X \to Y$ be a morphism of algebraic spaces
over $S$. Assume $f$ is \'etale. In this case there is a functor
$$
j : X_{\acute{e}tale} \to Y_{\acute{e}tale},\quad
(\varphi : U \to X) \mapsto (f \circ \varphi : U \to Y)
$$
which is cocontinuous. The morphism of topoi $f_{small}$ is the
morphism of topoi associated to $j$, see
Sites, Lemma \ref{sites-lemma-cocontinuous-morphism-topoi}.
Moreover, $j$ is continuous as well, hence
Sites, Lemma \ref{sites-lemma-when-shriek}
applies. In particular $f_{small}^{-1}\mathcal{G}(U) = \mathcal{G}(jU)$
for all sheaves $\mathcal{G}$ on $Y_{\acute{e}tale}$.
\end{lemma}

\begin{proof}
Note that by our very definition of an \'etale morphism of algebraic spaces
(Definition \ref{definition-etale}) it is
indeed the case that the rule given defines a functor $j$ as indicated.
It is clear that $j$ is cocontinuous and continuous, simply because a covering
$\{U_i \to U\}$ of $j(\varphi : U \to X)$ in $Y_{\acute{e}tale}$ is the
same thing as a covering of $(\varphi : U \to X)$ in $X_{\acute{e}tale}$. It
remains to show that $j$ induces the same morphism of topoi as $f_{small}$.
To see this we consider the diagram
$$
\xymatrix{
X_{\acute{e}tale} \ar[r] \ar[d]^j &
X_{spaces, \acute{e}tale} \ar@/_/[d]_{j_{spaces}} \\
Y_{\acute{e}tale} \ar[r] &
Y_{spaces, \acute{e}tale} \ar@/_/[u]_{v : V \mapsto X \times_Y V}
}
$$
of categories. Here the functor $j_{spaces}$ is the obvious extension of $j$
to the category $X_{spaces, \acute{e}tale}$. Thus the inner square is
commutative. In fact $j_{spaces}$ can be identified with the
localization functor
$j_X : Y_{spaces, \acute{e}tale}/X \to Y_{spaces, \acute{e}tale}$
discussed in
Sites, Section \ref{sites-section-localize}.
Hence, by
Sites, Lemma \ref{sites-lemma-localize-given-products}
the cocontinuous functor $j_{spaces}$ and the functor $v$ of the diagram
induce the same morphism of topoi. By
Sites, Lemma \ref{sites-lemma-composition-cocontinuous}
the commutativity of the inner square (consisting of cocontinuous functors
between sites) gives a commutative diagram of associated morphisms of topoi.
Hence, by the construction of $f_{small}$ in
Lemma \ref{lemma-functoriality-etale-site} we win.
\end{proof}

\noindent
The lemma above says that the pullback of $\mathcal{G}$ via an \'etale morphism
$f : X \to Y$ of algebraic spaces is simply the restriction of $\mathcal{G}$
to the category $X_{\acute{e}tale}$. We will often use the short hand
\begin{equation}
\label{equation-restrict}
\mathcal{G}|_{X_{\acute{e}tale}} = f_{small}^{-1}\mathcal{G}
\end{equation}
to indicate this. Note that the functor
$j : X_{\acute{e}tale} \to Y_{\acute{e}tale}$
of the lemma in this situation is faithful, but not fully faithful in
general. We will discuss this in a more technical fashion in
Section \ref{section-localize}.

\begin{lemma}
\label{lemma-pushforward-etale-base-change}
Let $S$ be a scheme. Let
$$
\xymatrix{
X' \ar[r] \ar[d]_{f'} & X \ar[d]^f \\
Y' \ar[r]^g & Y
}
$$
be a cartesian square of algebraic spaces over $S$. Let
$\mathcal{F}$ be a sheaf on $X_{\acute{e}tale}$. If $g$ is \'etale, then
$f'_{small, *}(\mathcal{F}|_{X'}) = (f_{small, *}\mathcal{F})|_{Y'}$ in
$\textit{Sh}(Y'_{\acute{e}tale})$\footnote{Also
$(f')_{small}^{-1}(\mathcal{G}|_{Y'}) = (f_{small}^{-1}\mathcal{G})|_{X'}$
because of commutativity of the diagram and (\ref{equation-restrict})}.
\end{lemma}

\begin{proof}
Consider the following diagram of functors
$$
\xymatrix{
X'_{spaces, \acute{e}tale} \ar[r]_j &
X_{spaces, \acute{e}tale} \\
Y'_{spaces, \acute{e}tale} \ar[r]^j \ar[u]^{V' \mapsto V' \times_{Y'} X'} &
Y_{spaces, \acute{e}tale} \ar[u]_{V \mapsto V \times_Y X}
}
$$
The horizontal arrows are localizations and the vertical arrows induce
morphisms of sites. Hence the last statement of
Sites, Lemma \ref{sites-lemma-localize-morphism}
gives what we want.
\end{proof}

\noindent
The following lemma says that you can think of a sheaf on the small
\'etale site of an algebraic space as a compatible collection of sheaves
on the small \'etale sites of schemes \'etale over the space. Please note
that all the comparison mappings $c_f$ in the lemma are isomorphisms,
which is compatible with
Topologies on Schemes,
Lemma \ref{topologies-lemma-characterize-sheaf-big-etale}
and the fact that all morphisms between objects of $X_{\acute{e}tale}$
are \'etale.

\begin{lemma}
\label{lemma-characterize-sheaf-small-etale}
Let $S$ be a scheme. Let $X$ be an algebraic space over $S$.
A sheaf $\mathcal{F}$ on $X_{\acute{e}tale}$ is given by the following data:
\begin{enumerate}
\item for every $U \in \text{Ob}(X_{\acute{e}tale})$ a sheaf
$\mathcal{F}_U$ on $U_{\acute{e}tale}$,
\item for every $f : U' \to U$ in $X_{\acute{e}tale}$ an isomorphism
$c_f : f_{small}^{-1}\mathcal{F}_U \to \mathcal{F}_{U'}$.
\end{enumerate}
These data are subject to the condition that given any $f : U' \to U$
and $g : U'' \to U'$ in $X_{\acute{e}tale}$ the composition
$g_{small}^{-1}c_f \circ c_g$ is equal to $c_{f \circ g}$.
\end{lemma}

\begin{proof}
Given a sheaf $\mathcal{F}$ on $X_{\acute{e}tale}$ and an object
$\varphi : U \to X$ of
$X_{\acute{e}tale}$ we set $\mathcal{F}_U = \varphi_{small}^{-1}\mathcal{F}$.
If $\varphi' : U' \to X$ is a second object of $X_{\acute{e}tale}$, and
$f : U' \to U$ is a morphism between them, then
the isomorphism $c_f$ comes from the fact that
$f_{small}^{-1} \circ \varphi_{small}^{-1} = (\varphi')^{-1}_{small}$, see
Lemma \ref{lemma-functoriality-etale-site}. The condition on the
transitivity of the isomorphisms $c_f$ follows from the functoriality
of the small \'etale sites also; verification omitted.

\medskip\noindent
Conversely, suppose we are given a collection of data $(\mathcal{F}_U, c_f)$
as in the lemma. In this case we simply define $\mathcal{F}$ by the rule
$U \mapsto \mathcal{F}_U(U)$. Details omitted.
\end{proof}

\noindent
Let $S$ be a scheme. Let $X$ be an algebraic space over $S$.
Let $X = U/R$ be a presentation of $X$ coming from any surjective
\'etale morphism $\varphi : U \to X$, see
Spaces, Definition \ref{spaces-definition-presentation}.
In particular, we obtain a groupoid $(U, R, s, t, c, e, i)$ such that
$j = (t, s) : R \to U \times_S U$, see
Groupoids, Lemma \ref{groupoids-lemma-equivalence-groupoid}.

\begin{lemma}
\label{lemma-descent-sheaf}
With $S$, $\varphi : U \to X$, and $(U, R, s, t, c, e, i)$ as above.
For any sheaf $\mathcal{F}$ on $X_{\acute{e}tale}$ the
sheaf\footnote{In this lemma
and its proof we write simply $\varphi^{-1}$ instead of $\varphi_{small}^{-1}$
and similarly for all the other pullbacks.}
$\mathcal{G} = \varphi^{-1}\mathcal{F}$ comes equipped with a canonical
isomorphism
$$
\alpha :
t^{-1}\mathcal{G}
\longrightarrow
s^{-1}\mathcal{G}
$$
such that the diagram
$$
\xymatrix{
& \text{pr}_1^{-1}t^{-1}\mathcal{G} \ar[r]_-{\text{pr}_1^{-1}\alpha} &
\text{pr}_1^{-1}s^{-1}\mathcal{G} \ar@{=}[rd] & \\
\text{pr}_0^{-1}s^{-1}\mathcal{G} \ar@{=}[ru] & & &
c^{-1}s^{-1}\mathcal{G} \\
&
\text{pr}_0^{-1}t^{-1}\mathcal{G} \ar[lu]^{\text{pr}_0^{-1}\alpha} \ar@{=}[r] &
c^{-1}t^{-1}\mathcal{G} \ar[ru]_{c^{-1}\alpha}
}
$$
is a commutative. The functor $\mathcal{F} \mapsto (\mathcal{G}, \alpha)$
defines an equivalence of categories between sheaves on
$X_{\acute{e}tale}$ and pairs $(\mathcal{G}, \alpha)$ as above.
\end{lemma}

\begin{proof}[First proof of Lemma \ref{lemma-descent-sheaf}]
Let $\mathcal{C} = X_{spaces, \acute{e}tale}$. By
Lemma \ref{lemma-etale-morphism-topoi}
and its proof we have $U_{spaces, \acute{e}tale} = \mathcal{C}/U$
and the pullback functor $\varphi^{-1}$ is just the restriction functor. 
Moreover, $\{U \to X\}$ is a covering of the site $\mathcal{C}$ and
$R = U \times_X U$. The isomorphism $\alpha$ is just the canonical
identification
$$
\left(\mathcal{F}|_{\mathcal{C}/U}\right)|_{\mathcal{C}/U \times_X U}
=
\left(\mathcal{F}|_{\mathcal{C}/U}\right)|_{\mathcal{C}/U \times_X U}
$$
and the commutativity of the diagram is the cocylce condition for glueing
data. Hence this lemma is a special case of glueing of sheaves, see
Sites, Section \ref{sites-section-glueing-sheaves}.
\end{proof}

\begin{proof}[Second proof of Lemma \ref{lemma-descent-sheaf}]
The existence of $\alpha$ comes from the fact that
$\varphi \circ t = \varphi \circ s$ and that pullback is
functorial in the morphism, see 
Lemma \ref{lemma-functoriality-etale-site}.
In exacty the same way, i.e., by functoriality of pullback, we see
that the isomorphism $\alpha$ fits into the commutative diagram.
The construction $\mathcal{F} \mapsto (\varphi^{-1}\mathcal{F}, \alpha)$
is clearly functorial in the sheaf $\mathcal{F}$.
Hence we obtain the functor.

\medskip\noindent
Conversely, suppose that $(\mathcal{G}, \alpha)$ is a pair.
Let $V \to X$ be an object of $X_{\acute{e}tale}$.
In this case the morphism $V' = U \times_X V \to V$ is a surjective \'etale
morphism of schemes, and hence $\{V' \to V\}$ is an \'etale
covering of $V$. Set $\mathcal{G}' = (V' \to V)^{-1}\mathcal{G}$.
Since $R = U \times_X U$ with $t = \text{pr}_0$
and $s = \text{pr}_0$ we see that $V' \times_V V' = R \times_X V$
with projection maps $s', t' : V' \times_V V' \to V'$ equal to the pullbacks
of $t$ and $s$. Hence $\alpha$ pulls back to an isomorphism
$\alpha' : (t')^{-1}\mathcal{G}' \to (s')^{-1}\mathcal{G}'$. Having said this
we simply define
$$
\xymatrix{
\mathcal{F}(V) \ar@{=}[r] &
\text{Equalizer}(\mathcal{G}(V') \ar@<1ex>[r] \ar@<-1ex>[r] &
\mathcal{G}(V' \times_V V').
}
$$
We omit the verification that this defines a sheaf. To see that
$\mathcal{G}(V) = \mathcal{F}(V)$ if there exists a morphism $V \to U$
note that in this case the equalizer is
$H^0(\{V' \to V\}, \mathcal{G}) = \mathcal{G}(V)$.
\end{proof}







\section{Points of the small \'etale site}
\label{section-points-small-etale-site}

\noindent
This section is the analogue of
E\'tale Cohomology, Section \ref{etale-cohomology-section-stalks}.

\begin{definition}
\label{definition-geometric-point}
Let $S$ be a scheme. Let $X$ be an algebraic space over $S$.
\begin{enumerate}
\item A {\it geometric point} of $X$ is a morphism
$\overline{x} : \text{Spec}(k) \to X$, where $k$ is an algebraically
closed field. We often abuse notation and
write $\overline{x} = \text{Spec}(k)$.
\item For every geometric point $\overline{x}$ we have the corresponding
``image'' point $x \in |X|$. We say that $\overline{x}$ is a
{\it geometric point lying over $x$}.
\end{enumerate}
\end{definition}

\noindent
It turns out that we can take stalks of sheaves on $X_{\acute{e}tale}$
at geometric point exactly in the same way as was done in the
case of the small \'etale site of a scheme. In order to do this we
define the notion of an \'etale neighbourhood as follows.

\begin{definition}
\label{definition-etale-neighbourhood}
Let $S$ be a scheme. Let $X$ be an algebraic space over $S$.
Let $\overline{x}$ be a geometric point of $X$.
\begin{enumerate}
\item An {\it \'etale neighborhood} of $\overline{x}$
of $X$ is a commutative diagram
$$
\xymatrix{
& U \ar^{\varphi}[d] \\
{\bar x} \ar^{\bar x}[r] \ar^{\bar u}[ur] & X
}
$$
where $\varphi$ is an \'etale morphism of algebraic spaces over $S$.
We will use the notation $\varphi : (U, \overline{u}) \to (X, \overline{x})$
to indicate this situation.
\item A {\it morphism of \'etale neighborhoods}
$(U, \overline{u}) \to (U', \overline{u}')$
is an $X$-morphism $h : U \to U'$
such that $\overline{u}' = h \circ \overline{u}$.
\end{enumerate}
\end{definition}

\noindent
Note that we allow $U$ to be an algebraic space. When we take stalks
of a sheaf on $X_{\acute{e}tale}$ we have to restrict to those $U$ which
are in $X_{\acute{e}tale}$, and so in this case we will only consider the case
where $U$ is a scheme. Alternately we can work with the site
$X_{space, \acute{e}tale}$ and consider all \'etale neighbourhoods. And there
won't be any difference because of the last assertion in the following lemma.

\begin{lemma}
\label{lemma-cofinal-etale}
Let $S$ be a scheme. Let $X$ be an algebraic space over $S$.
Let $\overline{x}$ be a geometric point of $X$.
The category of \'etale neighborhoods is cofiltered. More precisely:
\begin{enumerate}
\item Let $(U_i, \overline{u}_i)_{i = 1, 2}$ be two \'etale neighborhoods of
$\overline{x}$ in $X$. Then there exists a third \'etale neighborhood
$(U, \overline{u})$ and morphisms
$(U, \overline{u}) \to (U_i, \overline{u}_i)$, $i = 1, 2$.
\item Let $h_1, h_2: (U, \overline{u}) \to (U', \overline{u}')$ be two
morphisms between \'etale neighborhoods of $\overline{s}$. Then there exist an
\'etale neighborhood $(U'', \overline{u}'')$ and a morphism
$h : (U'', \overline{u}'') \to (U, \overline{u})$
which equalizes $h_1$ and $h_2$, i.e., such that
$h_1 \circ h = h_2 \circ h$.		
\end{enumerate}
Moreover, given any \'etale neighbourhood
$(U, \overline{u}) \to (X, \overline{x})$
there exists a morphism of \'etale neighbourhoods
$(U', \overline{u}') \to (U, \overline{u})$
where $U'$ is a scheme.
\end{lemma}

\begin{proof}
For part (1), consider the fibre product $U = U_1 \times_X U_2$.
It is \'etale over both $U_1$ and $U_2$ because \'etale morphisms are
preserved under base change and composition, see
Lemmas \ref{lemma-base-change-etale} and \ref{lemma-composition-etale}.
The map $\overline{u} \to U$ defined by $(\overline{u}_1, \overline{u}_2)$
gives it the structure of an \'etale neighborhood mapping to both
$U_1$ and $U_2$.

\medskip\noindent
For part (2), define $U''$ as the fibre product
$$
\xymatrix{
U'' \ar[r] \ar[d] & U \ar^{(h_1,h_2)}[d] \\
U' \ar^-\Delta[r] & U' \times_X U'.
}
$$
Since $\overline{u}$ and $\overline{u}'$ agree over $X$ with $\overline{x}$,
we see that $\overline{u}'' = (\overline{u}, \overline{u}')$ is a geometric
point of $U''$. In particular $U'' \not = \emptyset$.
Moreover, since $U'$ is \'etale over $X$, so is the fibre product
$U'\times_X U'$ (as seen above in the case of $U_1 \times_X U_2$).
Hence the vertical arrow $(h_1, h_2)$ is \'etale by
Lemma \ref{lemma-etale-permanence}.
Therefore $U''$ is \'etale over $U'$ by base change, and hence also
\'etale over $X$ (because compositions of \'etale morphisms are \'etale).
Thus $(U'', \overline{u}'')$ is a solution to the problem posed by (2).

\medskip\noindent
To see the final assertion, choose any surjective \'etale morphism
$U' \to U$ where $U'$ is a scheme. Then
$U' \times_U \overline{u}$ is a scheme surjective and \'etale over
$\overline{u} = \text{Spec}(k)$ with $k$ algebraically closed.
It follows (see
Remark \ref{remark-recall})
that $U' \times_U \overline{u} \to \overline{u}$ has a section
which gives us the desired $\overline{u}'$.
\end{proof}

\begin{lemma}
\label{lemma-geometric-lift-to-usual}
Let $S$ be a scheme. Let $X$ be an algebraic space over $S$.
Let $\overline{x} : \text{Spec}(k) \to X$ be a geometric point of $X$
lying over $x \in |X|$. Let $\varphi : U \to X$ be an \'etale morphism
of algebraic spaces and let $u \in |U|$ with $\varphi(u) = x$.
Then there exists a geometric point
$\overline{u} : \text{Spec}(k) \to U$ lying over $u$ with
$\overline{x} = f \circ \overline{u}$.
\end{lemma}

\begin{proof}
Choose an affine scheme $U'$ with $u' \in U'$ and an \'etale morphism
$U' \to U$ which maps $u'$ to $u$. If we can prove the lemma for
$(U', u') \to (X, x)$ then the lemma follows. Hence we may assume that
$U$ is a scheme, in particular that $U \to X$ is representable.
Then look at the cartesian diagram
$$
\xymatrix{
\text{Spec}(k) \times_{\overline{x}, X, \varphi} U
\ar[d]_{\text{pr}_1} \ar[r]_-{\text{pr}_2} & U
\ar[d]^\varphi \\
\text{Spec}(k) \ar[r]^-{\overline{x}} & X
}
$$
The projection $\text{pr}_1$ is the base change of an \'etale morphisms
so it is \'etale, see
Lemma \ref{lemma-base-change-etale}.
Therefore, the scheme $\text{Spec}(k) \times_{\overline{x}, X, \varphi} U$
is a disjoint union of finite separable extensions of $k$, see
Remark \ref{remark-recall}.
But $k$ is algebraically closed, so all these extensions are trivial,
so $\text{Spec}(k) \times_{\overline{x}, X, \varphi} U$
is a disjoint union of copies of $\text{Spec}(k)$ and each of
these corresponds to a geometric point $\overline{u}$ with
$f \circ \overline{u} = \overline{x}$. By
Lemma \ref{lemma-points-cartesian}
the map
$$
|\text{Spec}(k) \times_{\overline{x}, X, \varphi} U|
\longrightarrow
|\text{Spec}(k)| \times_{|X|} |U|
$$
is surjective, hence we can pick $\overline{u}$ to lie over $u$.
\end{proof}

\begin{lemma}
\label{lemma-geometric-lift-to-cover}
Let $S$ be a scheme. Let $X$ be an algebraic space over $S$.
Let $\overline{x}$ be a geometric point of $X$.
Let $(U, \overline{u})$ an \'etale neighborhood of $\overline{x}$.
Let $\{\varphi_i : U_i \to U\}_{i \in I}$ be an \'etale covering in
$X_{spaces, \acute{e}tale}$.
Then there exist $i \in I$ and $\overline{u}_i : \overline{x} \to U_i$
such that $\varphi_i : (U_i, \overline{u}_i) \to (U, \overline{u})$
is a morphism of \'etale neighborhoods.
\end{lemma}

\begin{proof}
Let $u \in |U|$ be the image of $\overline{u}$.
As $|U| = \bigcup_{i \in I} \varphi_i(|U_i|)$ there exists an
$i$ and a point $u_i \in U_i$ mapping to $x$. Apply
Lemma \ref{lemma-geometric-lift-to-usual}
to $(U_i, u_i) \to (U, u)$ and $\overline{u}$ to
get the desired geometric point.
\end{proof}

\begin{definition}
\label{definition-stalk}
Let $S$ be a scheme. Let $X$ be an algebraic space over $S$.
Let $\mathcal{F}$ be a presheaf on $X_{\acute{e}tale}$.
Let $\overline{x}$ be a geometric point of $X$.
The {\it stalk} of $\mathcal{F}$ at $\overline{x}$ is
$$
\mathcal{F}_{\bar x}
=
\text{colim}_{(U, \overline{u})}\ \mathcal{F}(U)
$$
where $(U, \overline{u})$ runs over all \'etale neighborhoods
of $\overline{x}$ in $X$ with $U \in \text{Ob}(X_{\acute{e}tale})$.
\end{definition}

\noindent
By
Lemma \ref{lemma-cofinal-etale},
this colimit is over a filtered
index category, namely the oppsite of the category of \'etale neighborhoods
in $X_{\acute{e}tale}$. More precisely
Lemma \ref{lemma-cofinal-etale}
says the opposite of the category of all \'etale neighbourhoods is filtered,
and the full subcategory of those which are in $X_{\acute{e}tale}$ is a cofinal
subcategory hence also filtered.

\medskip\noindent
This means an element of $\mathcal{F}_{\overline{x}}$ can be
thought of as a triple $(U, \overline{u}, \sigma)$ where
$U \in \text{Ob}(X_{\acute{e}tale})$ and $\sigma \in \mathcal{F}(U)$.
Two triples $(U, \overline{u}, \sigma)$, $(U', \overline{u}', \sigma')$
define the same element of the stalk if there exists a third
\'etale neighbourhood
$(U'', \overline{u}'')$, $U'' \in \text{Ob}(X_{\acute{e}tale})$
and morphisms of \'etale neighbourhoods
$h : (U'', \overline{u}'') \to (U, \overline{u})$,
$h' : (U'', \overline{u}'') \to (U', \overline{u}')$ such that
$h^*\sigma = (h')^*\sigma'$ in $\mathcal{F}(U'')$. See
Categories, Section \ref{categories-section-directed-colimits}.

\medskip\noindent
This also implies that if $\mathcal{F}'$ is the sheaf on
$X_{spaces, \acute{e}tale}$ corresponding to $\mathcal{F}$ on
$X_{\acute{e}tale}$, then
\begin{equation}
\label{equation-stalk-spaces-etale}
\mathcal{F}_{\overline{x}} = \text{colim}_{(U, \overline{u})}\ \mathcal{F}'(U)
\end{equation}
where now the colimit is over all the \'etale neighbourhoods of $\overline{x}$.
We will often jump between the point of view of using $X_{\acute{e}tale}$
and $X_{spaces, \acute{e}tale}$ without further mention.

\medskip\noindent
In particular this means that if $\mathcal{F}$ is a presheaf of
abelian groups, rings, etc then $\mathcal{F}_{\overline{x}}$ is
an abelian group, ring, etc simply by the usual way of defining the
group structure on a directed colimit of abelian groups, rings, etc.

\begin{lemma}
\label{lemma-stalk-gives-point}
Let $S$ be a scheme.
Let $X$ be an algebraic space over $S$.
Let $\overline{x}$ be a geometric point of $X$.
Consider the functor
$$
u : X_{\acute{e}tale} \longrightarrow \textit{Sets},\quad
U \longmapsto |U_{\overline{x}}|
$$
Then $u$ defines a point $p$ of the site $X_{\acute{e}tale}$
(Sites, Definition \ref{sites-definition-point})
and its associated stalk functor $\mathcal{F} \mapsto \mathcal{F}_p$
(Sites, Equation \ref{sites-equation-stalk})
is the functor $\mathcal{F} \mapsto \mathcal{F}_{\overline{x}}$
defined above.
\end{lemma}

\begin{proof}
In the proof of
Lemma \ref{lemma-geometric-lift-to-cover}
we have seen that the scheme $U_{\overline{x}}$ is a disjoint union of
schemes isomorphic to $\overline{x}$. Thus we can also think of
$|U_{\overline{x}}|$ as the set of geometric points of $U$ lying over
$\overline{x}$, i.e., as the collection of morphisms
$\overline{u} : \overline{x} \to U$ fitting into the diagram of
Definition \ref{definition-geometric-point}.
From this it follows that $u(X)$ is a singleton, and that
$u(U \times_V W) = u(U) \times_{u(V)} u(W)$
whenever $U \to V$ and $W \to V$ are morphisms in $X_{\acute{e}tale}$.
And, given a covering $\{U_i \to U\}_{i \in I}$ in $X_{\acute{e}tale}$ we see
that $\coprod u(U_i) \to u(U)$ is surjective by
Lemma \ref{lemma-geometric-lift-to-cover}.
Hence
Sites, Remark \ref{sites-remark-improve-proposition-points-limits}
applies, so $p$ is a point of the site $X_{\acute{e}tale}$.
Finally, the our functor $\mathcal{F} \mapsto \mathcal{F}_{\overline{s}}$
is given by exactly the same colimit as the functor
$\mathcal{F} \mapsto \mathcal{F}_p$ associated to $p$ in
Sites, Equation \ref{sites-equation-stalk}
which proves the final assertion.
\end{proof}

\begin{lemma}
\label{lemma-stalk-exact}
Let $S$ be a scheme.
Let $X$ be an algebraic space over $S$.
Let $\overline{x}$ be a geometric point of $X$.
\begin{enumerate}
\item The stalk functor
$\textit{PAb}(X_{\acute{e}tale}) \to \textit{Ab}$,
$\mathcal{F}  \mapsto  \mathcal{F}_{\overline{x}}$
is exact.
\item We have $(\mathcal{F}^\#)_{\overline{x}} = \mathcal{F}_{\overline{x}}$
for any presheaf of sets $\mathcal{F}$ on $X_{\acute{e}tale}$.
\item The functor
$\textit{Ab}(X_{\acute{e}tale}) \to \textit{Ab}$,
$\mathcal{F} \mapsto \mathcal{F}_{\overline{x}}$ is exact.
\item Similarly the functors
$\textit{PSh}(X_{\acute{e}tale}) \to \textit{Sets}$ and
$\textit{Sh}(X_{\acute{e}tale}) \to \textit{Sets}$ given by the stalk functor
$\mathcal{F} \mapsto \mathcal{F}_{\overline{x}}$ are exact (see
Categories, Definition \ref{categories-definition-exact})
and commute with arbitrary colimits.
\end{enumerate}
\end{lemma}

\begin{proof}
This result follows from the general material in
Modules on Sites, Section \ref{sites-modules-section-stalks}.
This is true because $\mathcal{F} \mapsto \mathcal{F}_{\overline{x}}$
comes from a point of the small \'etale site of $X$, see
Lemma \ref{lemma-stalk-gives-point}. See the proof of
\'Etale Cohomology, Lemma \ref{etale-cohomology-lemma-stalk-exact}
for a direct proof of some of these statements in the setting of
the small \'etale site of a scheme.
\end{proof}

\noindent
We will see below that the stalk functor
$\mathcal{F} \mapsto \mathcal{F}_{\overline{x}}$
is really the pullback along the morphism $\overline{x}$. In that sense
the following lemma is a generalization of the lemma above.

\begin{lemma}
\label{lemma-stalk-pullback}
Let $S$ be a scheme.
Let $f : X \to Y$ be a morphism of algebraic spaces over $S$.
\begin{enumerate}
\item The functor
$f_{small}^{-1} :
\textit{Ab}(Y_{\acute{e}tale})
\to
\textit{Ab}(X_{\acute{e}tale})$
is exact.
\item The functor
$f_{small}^{-1} :
\textit{Sh}(Y_{\acute{e}tale})
\to
\textit{Sh}(X_{\acute{e}tale})$
is exact, i.e., it commutes with finite limits and colimits, see
Categories, Definition \ref{categories-definition-exact}.
\item For any \'etale morphism $V \to Y$ of algebraic spaces
we have $f_{small}^{-1}h_V = h_{X \times_Y V}$.
\item Let $\overline{x} \to X$ be a geometric point.
Let $\mathcal{G}$ be a sheaf on $Y_{\acute{e}tale}$.
Then there is a canonical identification
$$
(f_{small}^{-1}\mathcal{G})_{\overline{x}} = \mathcal{G}_{\overline{y}}.
$$
where $\overline{y} = f \circ \overline{x}$.
\end{enumerate}
\end{lemma}

\begin{proof}
Recall that $f_{small}$ is defined via $f_{spaces, small}$ in
Lemma \ref{lemma-functoriality-etale-site}.
Parts (1), (2) and (3) are general consequences of the fact that
$f_{spaces, \acute{e}tale} :
X_{spaces, \acute{e}tale}
\to
Y_{spaces, \acute{e}tale}$
is a morphism of sites, see
Sites, Definition \ref{sites-definition-morphism-sites}
for (2),
Modules on Sites, Lemma \ref{sites-modules-lemma-flat-pullback-exact}
for (1), and
Sites, Lemma \ref{sites-lemma-pullback-representable-sheaf}
for (3).

\medskip\noindent
Proof of (4). This statement is a special case of
Sites, Lemma \ref{sites-lemma-point-morphism-sites}
via
Lemma \ref{lemma-stalk-gives-point}.
We also provide a direct proof. Note that by
Lemma \ref{lemma-stalk-exact}.
taking stalks commutes with sheafification.
Let $\mathcal{G}'$ be the sheaf on $Y_{spaces, \acute{e}tale}$ whose restriction
to $Y_{\acute{e}tale}$ is $\mathcal{G}$.
Recall that $f_{spaces, \acute{e}tale}^{-1}\mathcal{G}'$ is the sheaf
associated to the presheaf
$$
U \longrightarrow \text{colim}_{U \to X \times_Y V}\ \mathcal{G}'(V),
$$
see
Sites, Sections \ref{sites-section-continuous-functors} and
\ref{sites-section-functoriality-PSh}.
Thus we have
\begin{align*}
(f_{spaces, \acute{e}tale}^{-1}\mathcal{G}')_{\overline{x}}
& =
\text{colim}_{(U, \overline{u})}\ f_{spaces, \acute{e}tale}^{-1}\mathcal{G}'(U)
\\
& = \text{colim}_{(U, \overline{u})}
\ \text{colim}_{a : U \to X \times_Y V}\ \mathcal{G}'(V) \\
& = \text{colim}_{(V, \overline{v})}\ \mathcal{G}'(V) \\
& = \mathcal{G}'_{\overline{y}}
\end{align*}
in the third equality the pair $(U, \overline{u})$ and the map
$a : U \to X \times_Y V$ corresponds to the pair $(V, a \circ \overline{u})$.
Since the stalk of $\mathcal{G}'$
(resp.\ $f_{spaces, \acute{e}tale}^{-1}\mathcal{G}'$)
agrees with the stalk of $\mathcal{G}$ (resp.\ $f_{small}^{-1}\mathcal{G}$),
see
Equation (\ref{equation-stalk-spaces-etale})
the result follows.
\end{proof}

\begin{remark}
\label{remark-stalk-pullback}
This remark is the analogue of
\'Etale Cohomology, Remark \ref{etale-cohomology-remark-stalk-pullback}.
Let $S$ be a scheme.
Let $X$ be an algebraic space over $S$.
Let $\overline{x} : \text{Spec}(k) \to X$ be a geometric point of $X$.
By
\'Etale Cohomology,
Theorem \ref{etale-cohomology-theorem-equivalence-sheaves-point}
the category of sheaves on $\text{Spec}(k)_{\acute{e}tale}$ is
equivalent to the category of sets (by taking a sheaf to its global sections).
Hence it follows from
Lemma \ref{lemma-stalk-pullback} part (4)
applied to the morphism $\overline{x}$ that the functor
$$
\textit{Sh}(X_{\acute{e}tale}) \longrightarrow \textit{Sets},\quad
\mathcal{F} \longmapsto \mathcal{F}_{\overline{x}}
$$
is isomorphic to the functor
$$
\textit{Sh}(X_{\acute{e}tale})
\longrightarrow
\textit{Sh}(\text{Spec}(k)_{\acute{e}tale}) = \textit{Sets},
\quad
\mathcal{F} \longmapsto \overline{x}^*\mathcal{F}
$$
Hence we may view the stalk functors as pullback functors along
geometric morphisms (and not just some abstract morphisms of topoi
as in the result of
Lemma \ref{lemma-stalk-gives-point}).
\end{remark}

\begin{remark}
\label{remark-map-stalks}
Let $S$ be a scheme.
Let $X$ be an algebraic space over $S$.
Let $x \in |X|$.
We claim that for any pair of geometric points $\overline{x}$ and
$\overline{x}'$ lying over $x$ the stalk functors are isomorphic.
By definition of $|X|$ we can find a third geometric point
$\overline{x}''$ so that there exists a commutative diagram
$$
\xymatrix{
\overline{x}'' \ar[r] \ar[d] \ar[rd]^{\overline{x}''} &
\overline{x}' \ar[d]^{\overline{x}'} \\
\overline{x} \ar[r]^{\overline{x}} & X.
}
$$
Since the stalk functor $\mathcal{F} \mapsto \mathcal{F}_{\overline{x}}$
is given by pullback along the morphism $\overline{x}$ (and similarly for
the others) we conclude by functoriality of pullbacks.
\end{remark}




\noindent
The following theorem says that the small \'etale site of an algebraic
space has enough points.

\begin{theorem}
\label{theorem-exactness-stalks}
Let $S$ be a scheme. Let $X$ be an algebraic space over $S$.
A map $a : \mathcal{F} \to \mathcal{G}$ of sheaves of sets is injective
(resp.\ surjective) if and only if the map on stalks
$a_{\overline{x}} : \mathcal{F}_{\overline{x}} \to \mathcal{G}_{\overline{x}}$
is injective (resp.\ surjective) for all geometric points of $X$.
A sequence of abelian sheaves on $X_{\acute{e}tale}$ is exact
if and only if it is exact on all stalks at geometric points of $S$.
\end{theorem}

\begin{proof}
We know the theorem is true if $X$ is a scheme, see
\'Etale Cohomology, Theorem \ref{etale-cohomology-theorem-exactness-stalks}.
Choose a surjective \'etale morphism $f : U \to X$ where $U$ is a scheme.
Since $\{U \to X\}$ is a covering (in $X_{spaces, \acute{e}tale}$) we can check
whether a map of sheaves is injective, or surjective by restricting
to $U$. Now if $\overline{u} : \text{Spec}(k) \to U$ is a geometric
point of $U$, then
$(\mathcal{F}|_U)_{\overline{u}} = \mathcal{F}_{\overline{x}}$
where $\overline{x} = f \circ \overline{u}$. (This is clear from the
colimits defining the stalks at $\overline{u}$ and $\overline{x}$, but
it also follows from
Lemma \ref{lemma-stalk-pullback}.)
Hence the result for $U$ implies the result for $X$ and we win.
\end{proof}

\noindent
The following lemma should be skipped on a first reading.

\begin{lemma}
\label{lemma-points-small-etale-site}
Let $S$ be a scheme.
Let $X$ be an algebraic space over $S$.
Let $p : \textit{Sh}(pt) \to \textit{Sh}(X_{\acute{e}tale})$
be a point of the small \'etale topos of $X$.
Then there exists a geometric point $\overline{x}$ of $X$
such that the stalk functor $\mathcal{F} \mapsto \mathcal{F}_p$
is isomorphic to the stalk functor
$\mathcal{F} \mapsto \mathcal{F}_{\overline{x}}$.
\end{lemma}

\begin{proof}
By
Sites, Lemma \ref{sites-lemma-point-site-topos}
there is a one to one correspondence between points of the site and points
of the associated topos. Hence we may assume that $p$ is given by
a functor $u : X_{\acute{e}tale} \to \textit{Sets}$ which defines a point
of the site $X_{\acute{e}tale}$.
Let $U \in \text{Ob}(X_{\acute{e}tale})$ be an object whose structure morphism
$j : U \to X$ is surjective. Note that $h_U$ is a sheaf
which surjects onto the final sheaf. Since taking stalks is exact
we see that $(h_U)_p = u(U)$ is not empty (use
Sites, Lemma \ref{sites-lemma-points-recover}).
Pick $x \in u(U)$. By
Sites, Lemma \ref{sites-lemma-point-localize}
we obtain a point $q : \textit{Sh}(pt) \to \textit{Sh}(U_{\acute{e}tale})$
such that $p = j_{small} \circ q$, so that
$\mathcal{F}_p = (\mathcal{F}|_U)_q$ functorially.
By
\'Etale Cohomology, Lemma \ref{etale-cohomology-lemma-points-small-etale-site}
there is a geometric point $\overline{u}$ of $U$ and a functorial
isomorphism $\mathcal{G}_q = \mathcal{G}_{\overline{u}}$
for $\mathcal{G} \in \textit{Sh}(U_{\acute{e}tale})$. Set
$\overline{x} = j \circ \overline{u}$. Then we see that
$\mathcal{F}_{\overline{x}} \cong (\mathcal{F}|_U)_{\overline{u}}$
functorially in $\mathcal{F}$ on $X_{\acute{e}tale}$ by
Lemma \ref{lemma-stalk-pullback}
and we win.
\end{proof}





\section{Supports of abelian sheaves}
\label{section-support}

\noindent
First we talk about supports of local sections.

\begin{lemma}
\label{lemma-support-subsheaf-final}
Let $S$ be a scheme. Let $X$ be an algebraic space over $S$.
Let $\mathcal{F}$ be a subsheaf of the final object of the \'etale
topos of $X$ (see
Sites, Example \ref{sites-example-singleton-sheaf}).
Then there exists a unique open
$W \subset X$ such that $\mathcal{F} = h_W$.
\end{lemma}

\begin{proof}
The condition means that $\mathcal{F}(U)$ is a singleton or
empty for all $\varphi : U \to X$ in $\text{Ob}(X_{spaces, \acute{e}tale})$.
In particular local sections always glue. If
$\mathcal{F}(U) \not = \emptyset$, then
$\mathcal{F}(\varphi(U)) \not = \emptyset$ because
$\varphi(U) \subset X$ is an open subspace 
(Lemma \ref{lemma-etale-open})
and
$\{\varphi : U \to \varphi(U)\}$ is a covering in $X_{spaces, \acute{e}tale}$.
Take
$W = \bigcup_{\varphi : U \to S, \mathcal{F}(U) \not = \emptyset} \varphi(U)$
to conclude.
\end{proof}

\begin{lemma}
\label{lemma-zero-over-image}
Let $S$ be a scheme.
Let $X$ be an algebraic space over $S$.
Let $\mathcal{F}$ be an abelian sheaf on $X_{spaces, \acute{e}tale}$.
Let $\sigma \in \mathcal{F}(U)$ be a local section.
There exists an open subspace $W \subset U$ such that
\begin{enumerate}
\item $W \subset U$ is the largest open subspace of $U$ such
that $\sigma|_W = 0$,
\item for every $\varphi : V \to U$ in $X_{\acute{e}tale}$ we have
$$
\sigma|_V = 0 \Leftrightarrow \varphi(V) \subset W,
$$
\item for every geometric point $\overline{u}$ of $U$ we have
$$
(U, \overline{u}, \sigma) = 0\text{ in }\mathcal{F}_{\overline{s}}
\Leftrightarrow
\overline{u} \in W
$$
where $\overline{s} = (U \to S) \circ \overline{u}$.
\end{enumerate}
\end{lemma}

\begin{proof}
Since $\mathcal{F}$ is a sheaf in the \'etale topology the restriction of
$\mathcal{F}$ to $U_{Zar}$ is a sheaf on $U$ in the Zariski topology.
Hence there exists a Zariski open $W$ having property (1), see
Modules, Lemma \ref{modules-lemma-support-section-closed}. Let
$\varphi : V \to U$ be an arrow of $X_{\acute{e}tale}$. Note that
$\varphi(V) \subset U$ is an open subspace
(Lemma \ref{lemma-etale-open})
and that $\{V \to \varphi(V)\}$ is an \'etale covering. Hence if
$\sigma|_V = 0$, then by the sheaf condition for $\mathcal{F}$ we
see that $\sigma|_{\varphi(V)} = 0$. This proves (2).
To prove (3) we have to show that if $(U, \overline{u}, \sigma)$
defines the zero element of $\mathcal{F}_{\overline{s}}$, then
$\overline{u} \in W$. This is true because the assumption means
there exists a morphism of \'etale neighbourhoods
$(V, \overline{v}) \to (U, \overline{u})$ such that
$\sigma|_V = 0$. Hence by (2) we see that $V \to U$ maps into $W$, and
hence $\overline{u} \in W$.
\end{proof}

\noindent
Let $S$ be a scheme.
Let $X$ be an algebraic space over $S$.
Let $x \in |X|$.
Let $\mathcal{F}$ be a sheaf on $X_{\acute{e}tale}$. By
Remark \ref{remark-map-stalks}
the isomorphism class of the stalk of the sheaf $\mathcal{F}$
at a geometric points lying over $x$ is well defined.

\begin{definition}
\label{definition-support}
Let $S$ be a scheme.
Let $X$ be an algebraic space over $S$.
Let $\mathcal{F}$ be an abelian sheaf on $X_{\acute{e}tale}$.
\begin{enumerate}
\item The {\it support of $\mathcal{F}$} is the set of
points $x \in |X|$ such that $\mathcal{F}_{\overline{x}} \not = 0$
for any (some) geometric point $\overline{x}$ lying over $x$.
\item Let $\sigma \in \mathcal{F}(U)$ be a section.
The {\it support of $\sigma$} is the closed subset $U \setminus W$, where
$W \subset U$ is the largest open subset of $U$ on which $\sigma$
restricts to zero (see
Lemma \ref{lemma-zero-over-image}).
\end{enumerate}
\end{definition}

\begin{lemma}
\label{lemma-support-section-closed}
Let $S$ be a scheme.
Let $X$ be an algebraic space over $S$.
Let $\mathcal{F}$ be an abelian sheaf on $X_{\acute{e}tale}$.
Let $U \in \text{Ob}(X_{\acute{e}tale})$ and $\sigma \in \mathcal{F}(U)$.
\begin{enumerate}
\item The support of $\sigma$ is closed in $|X|$.
\item The support of $\sigma + \sigma'$ is contained in the union of
the supports of $\sigma, \sigma' \in \mathcal{F}(X)$.
\item If $\varphi : \mathcal{F} \to \mathcal{G}$ is a map of
abelian sheaves on $X_{\acute{e}tale}$, then the support of $\varphi(\sigma)$ is
contained in the support of $\sigma \in \mathcal{F}(U)$.
\item The support of $\mathcal{F}$ is the union of the images of the
supports of all local sections of $\mathcal{F}$.
\item If $\mathcal{F} \to \mathcal{G}$ is surjective then the support
of $\mathcal{G}$ is a subset of the support of $\mathcal{F}$.
\item If $\mathcal{F} \to \mathcal{G}$ is injective then the support
of $\mathcal{F}$ is a subset of the support of $\mathcal{G}$.
\end{enumerate}
\end{lemma}

\begin{proof}
Part (1) holds by definition.
Parts (2) and (3) hold because they holds for the restriction of
$\mathcal{F}$ and $\mathcal{G}$ to $U_{Zar}$, see
Modules, Lemma \ref{modules-lemma-support-section-closed}.
Part (4) is a direct consequence of
Lemma \ref{lemma-zero-over-image} part (3).
Parts (5) and (6) follow from the other parts.
\end{proof}

\begin{lemma}
\label{lemma-support-sheaf-rings-closed}
The support of a sheaf of rings on the small \'etale site of an
algebraic space is closed.
\end{lemma}

\begin{proof}
This is true because (according to our conventions)
a ring is $0$ if and only if
$1 = 0$, and hence the support of a sheaf of rings
is the support of the unit section.
\end{proof}






\section{The structure sheaf of an algebraic space}
\label{section-structure sheaf}

\noindent
The structure sheaf of an algebraic space is the sheaf of rings of the
following lemma.

\begin{lemma}
\label{lemma-sheaf-condition-holds}
Let $S$ be a scheme. Let $X$ be an algebraic space over $S$.
The rule $U \mapsto \Gamma(U, \mathcal{O}_U)$ defines
a sheaf of rings on $X_{\acute{e}tale}$.
\end{lemma}

\begin{proof}
Immediate from the definition of a covering and
Descent, Lemma \ref{descent-lemma-sheaf-condition-holds}.
\end{proof}

\begin{definition}
\label{definition-structure-sheaf}
Let $S$ be a scheme.
Let $X$ be an algebraic space over $S$.
The {\it structure sheaf} of $X$
is the sheaf of rings $\mathcal{O}_X$
on the small \'etale site $X_{\acute{e}tale}$ described in
Lemma \ref{lemma-sheaf-condition-holds}.
\end{definition}

\noindent
According to Lemma \ref{lemma-characterize-sheaf-small-etale} the sheaf
$\mathcal{O}_X$ corresponds to a system of \'etale sheaves $(\mathcal{O}_X)_U$
for $U$ ranging through the objects of $X_{\acute{e}tale}$. It is clear from
the proof of that lemma and our definition that we have simply
$(\mathcal{O}_X)_U = \mathcal{O}_U$ where $\mathcal{O}_U$ is the structure
sheaf of $U_{\acute{e}tale}$ as introduced in
Descent, Definition \ref{descent-definition-structure-sheaf}.
In particular, if $X$ is a scheme we recover the sheaf $\mathcal{O}_X$
on the small \'etale site of $X$.

\medskip\noindent
Via the equivalence
$\textit{Sh}(X_{\acute{e}tale}) = \textit{Sh}(X_{spaces, \acute{e}tale})$
of Lemma \ref{lemma-compare-etale-sites} we may also think of $\mathcal{O}_X$
as a sheaf of rings on $X_{spaces, \acute{e}tale}$. It is explained in
Remark \ref{remark-explain-equivalence}
how to compute $\mathcal{O}_X(Y)$, and in particular $\mathcal{O}_X(X)$, when
$Y \to X$ is an object of $X_{spaces, \acute{e}tale}$.

\begin{lemma}
\label{lemma-morphism-ringed-topoi}
Let $S$ be a scheme.
Let $f : X \to Y$ be a morphism of algebraic spaces over $S$.
Then there is a canonical map
$f^\sharp : f_{small}^{-1}\mathcal{O}_Y \to \mathcal{O}_X$ such that
$$
(f_{small}, f^\sharp) :
(X_{\acute{e}tale}, \mathcal{O}_X)
\to
(Y_{\acute{e}tale}, \mathcal{O}_Y)
$$
is a morphism of ringed topoi. Furthermore,
\begin{enumerate}
\item The construction $f \mapsto (f_{small}, f^\sharp)$ is compatible with
compositions.
\item If $f$ is a morphism of schemes, then $f^\sharp$ is the map described in
Descent, Remark \ref{descent-remark-change-topologies-ringed-sites}.
\end{enumerate}
\end{lemma}

\begin{proof}
By Lemma \ref{lemma-f-map} it suffices to give an $f$-map from
$\mathcal{O}_Y$ to $\mathcal{O}_X$. In other words, for every
commutative diagram
$$
\xymatrix{
U \ar[d]_g \ar[r] & X \ar[d]^f \\
V \ar[r] & Y
}
$$
where $U \in X_{\acute{e}tale}$, $V \in Y_{\acute{e}tale}$ we have to give a
map of rings
$
(f^\sharp)_{(U, V, g)} :
\Gamma(V, \mathcal{O}_V)
\to
\Gamma(U, \mathcal{O}_U).
$
Of course we just take $(f^\sharp)_{(U, V, g)} = g^\sharp$.
It is clear that this is compatible with restriction mappings
and hence indeed gives an $f$-map.
We omit checking compatibility with compositions and agreement with the
construction in
Descent, Remark \ref{descent-remark-change-topologies-ringed-sites}.
\end{proof}







\section{Stalks of the structure sheaf}
\label{section-stalks-structure-sheaf}

\noindent
This section is the analogue of
\'Etale Cohomology, Section \ref{section-stalks-structure-sheaf}.

\begin{lemma}
\label{lemma-describe-etale-local-ring}
Let $S$ be a scheme.
Let $X$ be an algebraic space over $S$.
Let $\overline{x}$ be a geometric point of $X$.
Let $(U, \overline{u})$ be an \'etale neighbourhood of $\overline{x}$
where $U$ is a scheme. Then we have
$$
\mathcal{O}_{X, \overline{x}} =
\mathcal{O}_{U, \overline{u}} =
\mathcal{O}_{U, u}^{sh}
$$
where the left hand side is the stalk of the structure sheaf of $X$,
and the right hand side is the strict henselization of the local ring
of $U$ at the point $u$ at which $\overline{u}$ is centered.
\end{lemma}

\begin{proof}
We know that the structure sheaf $\mathcal{O}_U$ on
$U_{\acute{e}tale}$ is the restriction of the structure sheaf of $X$.
Hence the first equality follows from
Lemma \ref{lemma-stalk-pullback} part (4).
The second equality is explained in
\'Etale Cohomology,
Lemma \ref{etale-cohomology-lemma-describe-etale-local-ring}.
\end{proof}

\begin{definition}
\label{definition-etale-local-rings}
Let $S$ be a scheme.
Let $X$ be an algebraic space over $S$.
Let $\overline{x}$ be a geometric point of $X$ lying over the point
$x \in |X|$.
\begin{enumerate}
\item The {\it \'etale local ring of $X$ at $\overline{x}$}
is the stalk of the structure sheaf $\mathcal{O}_X$ on $X_{\acute{e}tale}$
at $\overline{x}$.
Notation: $\mathcal{O}_{X, \overline{x}}$.
\item The {\it strict henselization of $X$ at $\overline{x}$}
is the scheme $\text{Spec}(\mathcal{O}_{X, \overline{x}})$.
\end{enumerate}
\end{definition}

\noindent
The isomorphism type of the strict henselization of $X$ at $\overline{x}$
(as a scheme over $X$) depends only on the point $x \in |X|$ and not on
the choice of the geometric point lying over $x$, see
Remark \ref{remark-map-stalks}.

\begin{lemma}
\label{lemma-etale-site-locally-ringed}
Let $S$ be a scheme.
Let $X$ be an algebraic space over $S$.
The small \'etale site $X_{\acute{e}tale}$ endowed with its
structure sheaf $\mathcal{O}_X$ is a locally ringed site, see
Modules on Sites, Definition \ref{sites-modules-definition-locally-ringed}.
\end{lemma}

\begin{proof}
This follows because the stalks
$\mathcal{O}_{X, \overline{x}}$ are
local, and because $S_{\acute{e}tale}$ has enough points, see
Lemmas \ref{lemma-describe-etale-local-ring} and
Theorem \ref{theorem-exactness-stalks}.
See
Modules on Sites, Lemma \ref{sites-modules-lemma-locally-ringed-stalk} and
\ref{sites-modules-lemma-ringed-stalk-not-zero}
for the fact that this implies the small \'etale site is locally ringed.
\end{proof}




\section{Dimension of local rings}
\label{section-dimension-local-ring}

\noindent
It turns out the dimension of the local ring of an algebraic space is a
well defined concept.

\begin{lemma}
\label{lemma-dimension-local-ring}
Let $S$ be a scheme.
Let $X$ be an algebraic space over $S$.
Let $x \in |X|$ be a point.
Let $d \in \{0, 1, 2, \ldots, \infty\}$.
The following are equivalent
\begin{enumerate}
\item for some scheme $U$ and \'etale morphism $a : U \to X$ and point
$u \in U$ with $a(u) = x$ we have $\dim(\mathcal{O}_{U, u}) = d$,
\item for any scheme $U$, any \'etale morphism $a : U \to X$, and any point
$u \in U$ with $a(u) = x$ we have $\dim(\mathcal{O}_{U, u}) = d$,
\item $\dim(\mathcal{O}_{X, \overline{x}}) = d$ for some geometric
point $\overline{x}$ lying over $x$, and
\item $\dim(\mathcal{O}_{X, \overline{x}}) = d$ for any geometric
point $\overline{x}$ lying over $x$.
\end{enumerate}
\end{lemma}

\begin{proof}
The equivalence of (1) and (2) follows from a combination of
Lemma \ref{lemma-local-source-target-at-point} and
Descent, Lemma \ref{descent-lemma-dimension-local-ring-local}.
The equivalence of (3) and (4) follows from the fact that the
isomorphism type of $\mathcal{O}_{X, \overline{x}}$ only depends
on $x \in |X|$, see
Remark \ref{remark-map-stalks}.

\medskip\noindent
Using
Lemma \ref{lemma-describe-etale-local-ring}
the equivalence of (1)$+$(2) and (3)$+$(4) comes down to the
following statement: Given any local ring $R$ we have
$\dim(R) = \dim(R^{sh})$. This is
\'Etale Cohomology, Lemma \ref{etale-cohomology-lemma-henselization-dimension}.
\end{proof}

\begin{definition}
\label{definition-dimension-local-ring}
Let $S$ be a scheme. Let $X$ be an algebraic space over $S$.
Let $x \in |X|$ be a point.
The {\it dimension of the local ring of $X$ at $x$} is
the element $d \in \{0, 1, 2, \ldots, \infty\}$
satisfying the equivalent conditions of
Lemma \ref{lemma-dimension-local-ring}.
\end{definition}








\section{Local irreducibility}
\label{section-irreducible-local-ring}

\noindent
A point on an algebraic space has a well defined \'etale local ring, which
corresponds to the strict henselization of the local ring in the case of a
scheme. In general it is impossible to read of from the \'etale local ring
the irreducible components of the algebraic stack passing through the given
point. Here is something we can do.

\begin{lemma}
\label{lemma-irreducible-local-ring}
Let $S$ be a scheme.
Let $X$ be an algebraic space over $S$.
Let $x \in |X|$ be a point.
The following are equivalent
\begin{enumerate}
\item for any scheme $U$ and \'etale morphism $a : U \to X$ and
$u \in U$ with $a(u) = x$ the local ring $\mathcal{O}_{U, u}$ has a
unique minimal prime,
\item for any scheme $U$ and \'etale morphism $a : U \to X$ and
$u \in U$ with $a(u) = x$ there is a unique irreducible component of $U$
through $u$, and
\item $\mathcal{O}_{X, \overline{x}}$ has a unique minimal prime
for any geometric point $\overline{x}$ lying over $x$.
\end{enumerate}
\end{lemma}

\begin{proof}
The equivalence of (1) and (2) follows from the fact that irreducible
components of $U$ passing through $u$ are in $1-1$ correspondence with
minimal primes of the local ring of $U$ at $u$. Let $a : U \to X$ and
$u \in U$ be as in (1). Then
$\mathcal{O}_{U, u} \to \mathcal{O}_{X, \overline{x}}$
is flat in particular injective. Hence if $f, g \in \mathcal{O}_{U, u}$
are non-nilpotent elements such that $fg = 0$, then the same is true in
$\mathcal{O}_{X, \overline{x}}$. Conversely, suppose that
$f, g \in \mathcal{O}_{X, \overline{x}}$ are non-nilpotent such that
$fg = 0$. Since $\mathcal{O}_{X, \overline{x}}$ is the filtered colimit
of the rings $\mathcal{O}_{U, u}$ we see that $f, g$ are the images
of elements of $\mathcal{O}_{U, u}$ for some choice of $a : U \to X$.
Hence we see that $\mathcal{O}_{U, u}$ doesn't have a unique minimal prime.
In this way we see the equivalence of (1) and (3).
\end{proof}

\begin{definition}
\label{definition-unibranch}
Let $S$ be a scheme. Let $X$ be an algebraic space over $S$.
Let $x \in |X|$. We say that $X$ is {\it geometrically unibranch
at $x$} if the equivalent conditions of
Lemma \ref{lemma-irreducible-local-ring}
hold. We say that $X$ is {\it geometrically unibranch} if $X$ is
geometrically unibranch at every $x \in |X|$.
\end{definition}

\noindent
To prove this is consistent with the definition of \cite{EGA}
for schemes we offer the following lemma (see
\cite[Lemma 2.2]{Etale-coverings}).

\begin{lemma}
\label{lemma-geometrically-unibranch}
Let $A$ be a local ring. Let $A^{sh}$ be a strict henselization of $A$.
The following are equivalent
\begin{enumerate}
\item $A^{sh}$ has a unique minimal prime, and
\item $A$ has a unique minimal prime $\mathfrak p$ and
the integral closure $A'$ of $A/\mathfrak p$ in its fraction field is a
local ring whose residue field is purely inseparable over the residue
field of $A$.
\end{enumerate}
\end{lemma}

\begin{proof}
Denote $\mathfrak m$ the maximal ideal of the ring $A$.
Denote $\kappa$, $\kappa^{sh}$ the residue field of $A$, $A^{sh}$.

\medskip\noindent
Assume (1). Let $\mathfrak p^{sh}$ be the unique minimal prime of
$A^{sh}$. The flatness of $A \to A^{sh}$ implies that
$\mathfrak p = A \cap \mathfrak p^{sh}$ is the unique minimal
prime of $A$ (by going down, see
Algebra, Lemma \ref{algebra-lemma-flat-going-down}).
Also, since $A^{sh}/\mathfrak pA^{sh} = (A/\mathfrak p)^{sh}$ (see
Algebra, Lemma \ref{algebra-lemma-quotient-strict-henselization})
is reduced by
Etale Cohomology, Lemma \ref{etale-cohomology-lemma-henselization-reduced}
we see that $\mathfrak p^{sh} = \mathfrak pA^{sh}$.
Since $A \to A'$ is integral, every maximal ideal of $A'$ lies over
$\mathfrak m$ (by going up for integral ring maps, see
Algebra, Lemma \ref{algebra-lemma-integral-going-up}).
If $A'$ is not local, then we can find distinct maximal ideals
$\mathfrak m_1$, $\mathfrak m_2$. Choosing elements $f_1, f_2 \in A'$
with $f_i \in \mathfrak m_i, f_i \not \in \mathfrak m_{3 - i}$ we find
a finite subalgebra $B = A[f_1, f_2] \subset A'$ with distinct maximal
ideals $B \cap \mathfrak m_i$, $i = 1, 2$. If $A'$ is local with maximal
ideal $\mathfrak m'$, but $A/\mathfrak m \subset A'/\mathfrak m'$
is not purely inseparable, then we can find a $f \in A'$ whose image in
$A'/\mathfrak m'$ generates finite, not purely inseparable extension
of $A/\mathfrak m$ and we find a finite local subalgebra $B = A[f] \subset A'$
whose residue field is not a purely inseparable extension of $A/\mathfrak m$.
Note that the inclusions
$$
A/\mathfrak p \subset B \subset \kappa(\mathfrak p)
$$
give, on tensoring with the flat ring map $A \to A^{sh}$ the inclusions
$$
A^{sh}/\mathfrak p^{sh} \subset
B \otimes_A A^{sh} \subset
\kappa(\mathfrak p) \otimes_A A^{sh} \subset
\kappa(\mathfrak p^{sh})
$$
the last inclusion because
$\kappa(\mathfrak p) \otimes_A A^{sh} =
\kappa(\mathfrak p) \otimes_{A/\mathfrak p} A^{sh}/\mathfrak p^{sh}$
is a localization of the domain $A^{sh}/\mathfrak p^{sh}$.
Note that $B \otimes_A \kappa^{sh}$ has at least two maximal ideals
because $B/\mathfrak mB$ either has two maximal ideals or one whose
residue field is not purely inseparable over $\kappa$, and because
$\kappa^{sh}$ is separably algebraically closed. Hence, as
$A^{sh}$ is strictly henselian we see that
$B \otimes_A A^{sh}$ is a product of $\geq 2$ local rings, see
Algebra, Lemma \ref{algebra-lemma-mop-up-strictly-henselian}.
But we've just seen that $B \otimes_A A^{sh}$ is a subring of a domain
and we get a contradiction.

\medskip\noindent
Assume (2). Let $A \to B$ be a local map of local rings which is a
localization of an \'etale $A$-algebra. In particular $\mathfrak m_B$
is the unique prime containing $\mathfrak m_AB$. Then $B' = A' \otimes_A B$
is integral over $B$ and the assumption that $A \to A'$ is local
with purely inseparable residue field extension implies that $B'$
is local. On the other hand, $A' \to B'$ is the localization
of an \'etale ring map, hence $B'$ is normal, see
Algebra, Lemma \ref{algebra-lemma-normal-goes-up}.
Thus $B'$ is a (local) normal domain. Finally, we have
$$
B/\mathfrak pB \subset B \otimes_A \kappa(\mathfrak p)
= B' \otimes_{A'} f.f.(A') \subset f.f.(B')
$$
Hence $B/\mathfrak pB$ is a domain, which implies that $B$ has a unique
minimal prime (since by flatness of $A \to B$ these all have to lie
over $\mathfrak p$). Hence, by
Lemma \ref{lemma-irreducible-local-ring}
we see that $A^{sh}$ has a unique minimal prime.
\end{proof}








\section{Regular algebraic spaces}
\label{section-regular}

\noindent
We have already defined regular algebraic spaces in
Section \ref{section-types-properties}.

\begin{lemma}
\label{lemma-regular}
Let $S$ be a scheme.
Let $X$ be a locally Noetherian algebraic space over $S$.
The following are equivalent
\begin{enumerate}
\item $X$ is regular, and
\item every \'etale local ring $\mathcal{O}_{X, \overline{x}}$ is
regular.
\end{enumerate}
\end{lemma}

\begin{proof}
Let $U$ be a scheme and let $U \to X$ be a surjective \'etale morphism.
By assumption $U$ is locally Noetherian. Moreover, every \'etale local
ring $\mathcal{O}_{X, \overline{x}}$ is the strict henselization of
a local ring on $U$ and conversely, see
Lemma \ref{lemma-describe-etale-local-ring}.
Thus by
\'Etale Cohomology, Lemma \ref{etale-cohomology-lemma-henselization-regular}
we see that (2) is equivalent to every local ring of $U$ being
regular, i.e., $U$ being a regular scheme (see
Properties, Lemma \ref{properties-lemma-characterize-regular}).
This equivalent to (1) by
Definition \ref{definition-type-property}.
\end{proof}







\section{Sheaves of modules on algebraic spaces}
\label{section-modules}

\noindent
If $X$ is an algebraic space, then a sheaf of modules on $X$ is
a sheaf of $\mathcal{O}_X$-modules on the small \'etale site of $X$
where $\mathcal{O}_X$ is the structure sheaf of $X$. The category
of sheaves of modules is denoted $\textit{Mod}(\mathcal{O}_X)$.

\medskip\noindent
Given a morphism $f : X \to Y$ of algebraic spaces, by
Lemma \ref{lemma-morphism-ringed-topoi}
we get a morphism of ringed topoi and hence by
Modules on Sites, Definition \ref{sites-modules-definition-pushforward}
we get well defined pullback and direct image functors
\begin{equation}
\label{equation-push-pull}
f^* :
\textit{Mod}(\mathcal{O}_Y)
\longrightarrow
\textit{Mod}(\mathcal{O}_X), \quad
f_* :
\textit{Mod}(\mathcal{O}_X)
\longrightarrow
\textit{Mod}(\mathcal{O}_Y)
\end{equation}
which are adjoint in the usual way. If $g : Y \to Z$ is another morphism
of algebraic spaces over $S$, then we have
$(g \circ f)^* = f^* \circ g^*$ and $(g \circ f)_* = g_* \circ f_*$
simply because the morphisms of ringed topoi compose in the corresponding
way (by the lemma).

\begin{lemma}
\label{lemma-etale-exact-pullback}
Let $S$ be a scheme.
Let $f : X \to Y$ be an \'etale morphism of algebraic spaces over $S$.
Then $f^{-1}\mathcal{O}_Y = \mathcal{O}_X$, and
$f^*\mathcal{G} = f_{small}^{-1}\mathcal{G}$ for any sheaf of
$\mathcal{O}_Y$-modules $\mathcal{G}$. In particular,
$f^* : \textit{Mod}(\mathcal{O}_X) \to \textit{Mod}(\mathcal{O}_Y)$
is exact.
\end{lemma}

\begin{proof}
By the description of inverse image in Lemma \ref{lemma-etale-morphism-topoi}
and the definition of the structure sheaves it is clear that
$f_{small}^{-1}\mathcal{O}_Y = \mathcal{O}_X$. Since the pullback
$$
f^*\mathcal{G} =
f_{small}^{-1}\mathcal{G} \otimes_{f_{small}^{-1}\mathcal{O}_Y}
\mathcal{O}_X
$$
by definition we conclude that $f^*\mathcal{G} = f_{small}^{-1}\mathcal{G}$.
The exactness is clear because $f_{small}^{-1}$ is exact, as $f_{small}$
is a morphism of topoi.
\end{proof}

\noindent
We continue our abuse of notation introduced in
Equation (\ref{equation-restrict})
by writing
\begin{equation}
\label{equation-restrict-modules}
\mathcal{G}|_{X_{\acute{e}tale}}
= f^*\mathcal{G}
= f_{small}^{-1}\mathcal{G}
\end{equation}
in the situation of the lemma above. We will discuss this in a more
technical fashion in
Section \ref{section-localize}.

\begin{lemma}
\label{lemma-pushforward-etale-base-change-modules}
Let $S$ be a scheme. Let
$$
\xymatrix{
X' \ar[r] \ar[d]_{f'} & X \ar[d]^f \\
Y' \ar[r]^g & Y
}
$$
be a cartesian square of algebraic spaces over $S$. Let
$\mathcal{F} \in \text{Mod}(\mathcal{O}_X)$. If $g$ is \'etale, then
$f'_*(\mathcal{F}|_{X'}) = (f_*\mathcal{F})|_{Y'}$ in
$\text{Mod}(\mathcal{O}_{Y'})$.
\end{lemma}

\begin{proof}
This is a reformulation of
Lemma \ref{lemma-pushforward-etale-base-change}
in the case of modules.
\end{proof}

\begin{lemma}
\label{lemma-characterize-module-small-etale}
Let $S$ be a scheme. Let $X$ be an algebraic space over $S$.
A sheaf $\mathcal{F}$ of $\mathcal{O}_X$-modules is given by the following
data:
\begin{enumerate}
\item for every $U \in \text{Ob}(X_{\acute{e}tale})$ a sheaf
$\mathcal{F}_U$ of $\mathcal{O}_U$-modules on $U_{\acute{e}tale}$,
\item for every $f : U' \to U$ in $X_{\acute{e}tale}$ an isomorphism
$c_f : f_{small}^*\mathcal{F}_U \to \mathcal{F}_{U'}$.
\end{enumerate}
These data are subject to the condition that given any $f : U' \to U$
and $g : U'' \to U'$ in $X_{\acute{e}tale}$ the composition
$g_{small}^{-1}c_f \circ c_g$ is equal to $c_{f \circ g}$.
\end{lemma}

\begin{proof}
Combine Lemmas \ref{lemma-etale-exact-pullback}
and \ref{lemma-characterize-sheaf-small-etale}, and use the fact that
any morphism between objects of $X_{\acute{e}tale}$ is an \'etale morphism
of schemes.
\end{proof}







\section{\'Etale localization}
\label{section-localize}

\noindent
Reading this section should be avoided at all cost.

\medskip\noindent
Let $X \to Y$ be an \'etale morphism of algebraic spaces.
Then $X$ is an object of $Y_{spaces, \acute{e}tale}$ and it is
immediate from the definitions, see also the proof of
Lemma \ref{lemma-etale-morphism-topoi},
that
\begin{equation}
\label{equation-localize}
X_{spaces, \acute{e}tale} = Y_{spaces, \acute{e}tale}/X
\end{equation}
where the right hand side is the localization of the site
$Y_{spaces, \acute{e}tale}$ at the object $X$, see
Sites, Definition \ref{sites-definition-localize}.
Moreover, this identification is compatible with the structure sheaves by
Lemma \ref{lemma-etale-exact-pullback}.
Hence the ringed site $(X_{spaces, \acute{e}tale}, \mathcal{O}_X)$
is identified with the localization of the ringed site
$(Y_{spaces, \acute{e}tale}, \mathcal{O}_Y)$ at the object $X$:
\begin{equation}
\label{equation-localize-ringed}
(X_{spaces, \acute{e}tale}, \mathcal{O}_X) =
(Y_{spaces, \acute{e}tale}/X, \mathcal{O}_Y|_{Y_{spaces, \acute{e}tale}/X})
\end{equation}
The localization of a ringed site used on the right hand side is defined in
Modules on Sites,
Definition \ref{sites-modules-definition-localize-ringed-site}.

\medskip\noindent
Assume now $X \to Y$ is an \'etale morphism of algebraic spaces and $X$ is
a scheme. Then $X$ is an object of $Y_{\acute{e}tale}$ and it follows that
\begin{equation}
\label{equation-localize-at-scheme}
X_{\acute{e}tale} = Y_{\acute{e}tale}/X
\end{equation}
and
\begin{equation}
\label{equation-localize-at-scheme-ringed}
(X_{\acute{e}tale}, \mathcal{O}_X) =
(Y_{\acute{e}tale}/X, \mathcal{O}_Y|_{Y_{\acute{e}tale}/X})
\end{equation}
as above.

\medskip\noindent
Finally, if $X \to Y$ is an \'etale morphism of algebraic spaces and $X$ is
an affine scheme, then $X$ is an object of $Y_{affine, \acute{e}tale}$ and
\begin{equation}
\label{equation-localize-at-affine}
X_{affine, \acute{e}tale} = Y_{affine, \acute{e}tale}/X
\end{equation}
and
\begin{equation}
\label{equation-localize-at-affine-ringed}
(X_{affine, \acute{e}tale}, \mathcal{O}_X) =
(Y_{affine, \acute{e}tale}/X, \mathcal{O}_Y|_{Y_{affine, \acute{e}tale}/X})
\end{equation}
as above.

\medskip\noindent
Next, we show that these localizations are compatible with morphisms.

\begin{lemma}
\label{lemma-relocalize-morphism}
Let $S$ be a scheme. Let
$$
\xymatrix{
U \ar[d]_p \ar[r]_g & V \ar[d]^q \\
X \ar[r]^f & Y
}
$$
be a commutative diagram of algebraic spaces over $S$ with $p$ and $q$ \'etale.
Via the identifications
(\ref{equation-localize-ringed}) for $U \to X$ and $V \to Y$
the morphism of ringed topoi
$$
(g_{spaces, \acute{e}tale}, g^\sharp) :
(\textit{Sh}(U_{spaces, \acute{e}tale}), \mathcal{O}_U)
\longrightarrow
(\textit{Sh}(V_{spaces, \acute{e}tale}), \mathcal{O}_V)
$$
is $2$-isomorphic to the morphism $(f_{spaces, \acute{e}tale, c}, f_c^\sharp)$
constructed in
Modules on Sites,
Lemma \ref{sites-modules-lemma-relocalize-morphism-ringed-sites}
starting with the morphism of ringed sites
$(f_{spaces, \acute{e}tale}, f^\sharp)$ and
the map $c : U \to V \times_Y X$ corresponding to $g$.
\end{lemma}

\begin{proof}
The morphism $(f_{spaces, \acute{e}tale, c}, f_c^\sharp)$ is defined as a
composition $f' \circ j$
of a localization and a base change map. Similarly $g$ is a composition
$U \to V \times_Y X \to V$. Hence it suffices to prove
the lemma in the following two cases: (1) $f = \text{id}$, and
(2) $U = X \times_Y V$. In case (1) the morphism $g : U \to V$ is
\'etale, see
Lemma \ref{lemma-etale-permanence}.
Hence $(g_{spaces, \acute{e}tale}, g^\sharp)$ is a localization morphism
by the discussion surrounding
Equations (\ref{equation-localize}) and
(\ref{equation-localize-ringed})
which is exactly the content of the lemma in this case.
In case (2) the morphism $g_{spaces, \acute{e}tale}$
comes from the morphism of ringed sites given by the functor
$V_{spaces, \acute{e}tale} \to U_{spaces, \acute{e}tale}$,
$V'/V \mapsto V' \times_V U/U$
which is also what the morphism $f'$ is defined by, see
Sites, Lemma \ref{sites-lemma-localize-morphism}.
We omit the verification that $(f')^\sharp = g^\sharp$
in this case (both are the restriction of $f^\sharp$
to $U_{spaces, \acute{e}tale}$).
\end{proof}

\begin{lemma}
\label{lemma-relocalize-morphism-at-schemes}
Same notation and assumptions as in
Lemma \ref{lemma-relocalize-morphism}
except that we also assume $U$ and $V$ are schemes.
Via the identifications
(\ref{equation-localize-at-scheme-ringed})
for $U \to X$ and $V \to Y$ the morphism of ringed topoi
$$
(g_{small}, g^\sharp) :
(\textit{Sh}(U_{\acute{e}tale}), \mathcal{O}_U)
\longrightarrow
(\textit{Sh}(V_{\acute{e}tale}), \mathcal{O}_V)
$$
is $2$-isomorphic to the morphism $(f_{small, s}, f_s^\sharp)$
constructed in
Modules on Sites,
Lemma \ref{sites-modules-lemma-relocalize-morphism-ringed-topoi}
starting with $(f_{small}, f^\sharp)$ and
the map $s : h_U \to f_{small}^{-1}h_V$ corresponding to $g$.
\end{lemma}

\begin{proof}
Note that $(g_{small}, g^\sharp)$ is $2$-isomorphic as a
morphism of ringed topoi to the morphism of ringed topoi
associated to the morphism of ringed sites
$(g_{spaces, \acute{e}tale}, g^\sharp)$. Hence we conclude by
Lemma \ref{lemma-relocalize-morphism}
and
Modules on Sites,
Lemma \ref{sites-modules-lemma-relocalize-morphism-compare}.
\end{proof}












\section{Recovering morphisms}
\label{section-morphisms}

\noindent
In this section we prove that the rule which associates to an algebraic space
its locally ringed small \'etale topos is fully faithful in a suitable
sense, see
Theorem \ref{theorem-fully-faithful}.

\begin{lemma}
\label{lemma-morphism-locally-ringed}
Let $S$ be a scheme.
Let $f : X \to Y$ be a morphism of algebraic spaces over $S$.
The morphism of ringed topoi $(f_{small}, f^\sharp)$
associated to $f$ is a morphism of locally ringed topoi, see
Modules on Sites,
Definition \ref{sites-modules-definition-morphism-locally-ringed-topoi}.
\end{lemma}

\begin{proof}
Note that the assertion makes sense since we have seen that
$(X_{\acute{e}tale}, \mathcal{O}_{X_{\acute{e}tale}})$ and
$(Y_{\acute{e}tale}, \mathcal{O}_{Y_{\acute{e}tale}})$
are locally ringed sites, see
Lemma \ref{lemma-etale-site-locally-ringed}.
Moreover, we know that $X_{\acute{e}tale}$ has enough points, see
Theorem \ref{theorem-exactness-stalks}.
Hence it suffices to prove that $(f_{small}, f^\sharp)$
satisfies condition (3) of
Modules on Sites,
Lemma \ref{sites-modules-lemma-locally-ringed-morphism}.
To see this take a point $p$ of $X_{\acute{e}tale}$. By
Lemma \ref{lemma-points-small-etale-site}
$p$ corresponds to a geometric point $\overline{x}$ of $X$.
By
Lemma \ref{lemma-stalk-pullback}
the point $q = f_{small} \circ p$ corresponds to the
geometric point $\overline{y} = f \circ \overline{x}$ of $Y$.
Hence the assertion we have to prove is that the induced map
of \'etale local rings
$$
\mathcal{O}_{Y, \overline{y}} \longrightarrow \mathcal{O}_{X, \overline{x}}
$$
is a local ring map. You can prove this directly, but instead we deduce it
from the corresponding result for schemes. To do this choose a commutative
diagram
$$
\xymatrix{
U \ar[d] \ar[r]_\psi & V \ar[d] \\
X \ar[r] & Y
}
$$
where $U$ and $V$ are schemes, and the vertical arrows are surjective
\'etale (see
Spaces, Lemma \ref{spaces-lemma-lift-morphism-presentations}).
Choose a lift $\overline{u} : \overline{x} \to U$ (possible by
Lemma \ref{lemma-geometric-lift-to-cover}).
Set $\overline{v} = \psi \circ \overline{u}$. We obtain a commutative
diagram of \'etale local rings
$$
\xymatrix{
\mathcal{O}_{U, \overline{u}} &
\mathcal{O}_{V, \overline{v}} \ar[l] \\
\mathcal{O}_{X, \overline{x}} \ar[u] &
\mathcal{O}_{Y, \overline{y}}. \ar[l] \ar[u]
}
$$
By
\'Etale Cohomology, Lemma \ref{etale-cohomology-lemma-morphism-locally-ringed}
the top horizontal arrow is a local ring map. Finally by
Lemma \ref{lemma-describe-etale-local-ring}
the vertical arrows are isomorphisms. Hence we win.
\end{proof}

\begin{lemma}
\label{lemma-2-morphism}
Let $S$ be a scheme.
Let $X$, $Y$ be algebraic spaces over $S$.
Let $f : X \to Y$ be a morphism of algebraic spaces over $S$.
Let $t$ be a $2$-morphism from $(f_{small}, f^\sharp)$ to itself, see
Modules on Sites,
Definition \ref{sites-modules-definition-2-morphism-ringed-topoi}.
Then $t = \text{id}$.
\end{lemma}

\begin{proof}
Let $X'$, resp.\ $Y'$ be $X$ viewed as an algebraic space over
$\text{Spec}(\mathbf{Z})$, see
Spaces, Definition \ref{spaces-definition-base-change}.
It is clear from the construction that $(X_{small}, \mathcal{O})$
is equal to $(X'_{small}, \mathcal{O})$ and similarly for $Y$.
Hence we may work with $X'$ and $Y'$. In other words we may
assume that $S = \text{Spec}(\mathbf{Z})$.

\medskip\noindent
Assume $S = \text{Spec}(\mathbf{Z})$, $f : X \to Y$ and $t$ are as in
the lemma. This means that $t : f^{-1}_{small} \to f^{-1}_{small}$
is a transformation of functors such that the diagram
$$
\xymatrix{
f_{small}^{-1}\mathcal{O}_Y
\ar[rd]_{f^\sharp}  & &
f_{small}^{-1}\mathcal{O}_Y \ar[ll]^t \ar[ld]^{f^\sharp} \\
& \mathcal{O}_X
}
$$
is commutative. Suppose $V \to Y$ is \'etale with $V$ affine.
Write $V = \text{Spec}(B)$. Choose generators $b_j \in B$, $j \in J$
for $B$ as a $\mathbf{Z}$-algebra. Set
$T = \text{Spec}(\mathbf{Z}[\{x_j\}_{j \in J}])$.
In the following we will use that
$\text{Mor}_{\textit{Sch}}(U, T) = \prod_{j \in J} \Gamma(U, \mathcal{O}_U)$
for any scheme $U$ without further mention.
The surjective ring map $\mathbf{Z}[x_j] \to B$, $x_j \mapsto b_j$
corresponds to a closed immersion $V \to T$.
We obtain a monomorphism
$$
i : V \longrightarrow T_Y = T \times Y
$$
of algebraic spaces over $Y$. In terms of sheaves on $Y_{\acute{e}tale}$
the morphism $i$ induces an injection
$h_i : h_V \to \prod_{j \in J} \mathcal{O}_Y$ of sheaves.
The base change $i' : X \times_Y V \to T_X$ of $i$ to $X$
is a monomorphism too
(Spaces,
Lemma \ref{spaces-lemma-base-change-representable-transformations-property}).
Hence $i' : X \times_Y V \to T_X$ is a monomorphism, which
in turn means that
$h_{i'} : h_{X \times_Y V} \to \prod_{j \in J} \mathcal{O}_X$
is an injection of sheaves.
Via the identification $f_{small}^{-1}h_V = h_{X \times_Y V}$ of
Lemma \ref{lemma-stalk-pullback}
the map $h_{i'}$ is equal to
$$
\xymatrix{
f_{small}^{-1}h_V \ar[r]^-{f^{-1}h_i} &
\prod_{j \in J} f_{small}^{-1}\mathcal{O}_Y
\ar[r]^{\prod f^\sharp} &
\prod_{j \in J} \mathcal{O}_X
}
$$
(verification omitted). This means that the map
$t : f_{small}^{-1}h_V \to f_{small}^{-1}h_V$
fits into the commutative diagram
$$
\xymatrix{
f_{small}^{-1}h_V \ar[r]^-{f^{-1}h_i} \ar[d]^t &
\prod_{j \in J} f_{small}^{-1}\mathcal{O}_Y
\ar[r]^-{\prod f^\sharp} \ar[d]^{\prod t} &
\prod_{j \in J} \mathcal{O}_X \ar[d]^{\text{id}}\\
f_{small}^{-1}h_V \ar[r]^-{f^{-1}h_i} &
\prod_{j \in J} f_{small}^{-1}\mathcal{O}_Y
\ar[r]^-{\prod f^\sharp} &
\prod_{j \in J} \mathcal{O}_X
}
$$
The commutativity of the right square holds by our assumption on $t$
explained above.
Since the composition of the horizontal arrows is injective
by the discussion above we conclude that the left vertical arrow
is the identity map as well. Any sheaf of sets on
$Y_{\acute{e}tale}$ admits a surjection from a (huge) coproduct of sheaves
of the form $h_V$ with $V$ affine (combine
Lemma \ref{lemma-alternative}
with
Sites, Lemma \ref{sites-lemma-sheaf-coequalizer-representable}).
Thus we conclude that $t : f_{small}^{-1} \to f_{small}^{-1}$
is the identity transformation as desired.
\end{proof}

\begin{lemma}
\label{lemma-faithful}
Let $S$ be a scheme.
Let $X$, $Y$ be algebraic spaces over $S$.
Any two morphisms $a, b : X \to Y$ of algebraic spaces over $S$
for which there exists a $2$-isomorphism
$(a_{small}, a^\sharp) \cong (b_{small}, b^\sharp)$
in the $2$-category of ringed topoi are equal.
\end{lemma}

\begin{proof}
Let $t : a_{small}^{-1} \to b_{small}^{-1}$ be the $2$-isomorphism.
We may equivalently think of $t$ as a transformation
$t : a_{spaces, \acute{e}tale}^{-1} \to b_{spaces, \acute{e}tale}^{-1}$
since there is not difference between sheaves on $X_{\acute{e}tale}$
and sheaves on $X_{spaces, \acute{e}tale}$.
Choose a commutative diagram
$$
\xymatrix{
U \ar[d]_p \ar[r]_\alpha  & V \ar[d]^q \\
X \ar[r]^a & Y
}
$$
where $U$ and $V$ are schemes, and $p$ and $q$ are surjective \'etale.
Consider the diagram
$$
\xymatrix{
h_U \ar[r]_-\alpha \ar@{=}[d] & a_{spaces, \acute{e}tale}^{-1}h_V \ar[d]^t \\
h_U \ar@{..>}[r] & b_{spaces, \acute{e}tale}^{-1}h_V
}
$$
Since the sheaf $b_{spaces, \acute{e}tale}^{-1}h_V$ is isomorphic to
$h_{V \times_{Y, b} X}$ we see that the dotted arrow comes from a
morphism of schemes
$\beta : U \to V$ fitting into a commutative diagram
$$
\xymatrix{
U \ar[d]_p \ar[r]_\beta  & V \ar[d]^q \\
X \ar[r]^b & Y
}
$$
We claim that there exists a sequence of $2$-isomorphisms
\begin{align*}
(\alpha_{small}, \alpha^\sharp)
& \cong 
(\alpha_{spaces, \acute{e}tale}, \alpha^\sharp) \\
& \cong
(a_{spaces, \acute{e}tale, c}, a_c^\sharp) \\
& \cong
(b_{spaces, \acute{e}tale, d}, b_d^\sharp) \\
& \cong
(\beta_{spaces, \acute{e}tale}, \beta^\sharp) \\
& \cong
(\beta_{small}, \beta^\sharp)
\end{align*}
The first and the last $2$-isomorphisms come from the identifications
between sheaves on $U_{spaces, \acute{e}tale}$ and sheaves on
$U_{\acute{e}tale}$ and similarly for $V$. The second and fourth
$2$-isomorphisms are those of
Lemma \ref{lemma-relocalize-morphism}
with $c : U \to X \times_{a, Y} V$ induced by $\alpha$ and
$d : U \to X \times_{b, Y} V$ induced by $\beta$.
The middle $2$-isomorphism comes from the transformation $t$.
Namely, the functor $a_{spaces, \acute{e}tale, c}^{-1}$ corresponds
to the functor
$$
(\mathcal{H} \to h_V) \longmapsto
(a_{spaces, \acute{e}tale}^{-1}\mathcal{H}
\times_{a_{spaces, \acute{e}tale}^{-1}h_V, \alpha}
h_U \to
h_U)
$$
and similarly for $b_{spaces, \acute{e}tale, d}^{-1}$, see
Sites, Lemma \ref{sites-lemma-relocalize-morphism}.
This uses the identification of sheaves on $Y_{spaces, \acute{e}tale}/V$
as arrows $(\mathcal{H} \to h_V)$ in $\textit{Sh}(Y_{spaces, \acute{e}tale})$
and similarly for $U/X$, see
Sites, Lemma \ref{sites-lemma-essential-image-j-shriek}.
Via this identification the structure sheaf $\mathcal{O}_V$ corresponds to the
pair $(\mathcal{O}_Y \times h_V \to h_V)$ and similarly
for $\mathcal{O}_U$, see
Modules on Sites,
Lemma \ref{sites-modules-lemma-localize-compare}.
Since $t$ switches $\alpha$ and $\beta$ we see that $t$ induces an isomorphism
$$
t :
a_{spaces, \acute{e}tale}^{-1}\mathcal{H}
\times_{a_{spaces, \acute{e}tale}^{-1}h_V, \alpha}
h_U
\longrightarrow
b_{spaces, \acute{e}tale}^{-1}\mathcal{H}
\times_{b_{spaces, \acute{e}tale}^{-1}h_V, \beta}
h_U
$$
over $h_U$ functorially in $(\mathcal{H} \to h_V)$. Also, $t$ is compatible
with $a_c^\sharp$ and $b_d^\sharp$ as $t$ is
compatible with $a^\sharp$ and $b^\sharp$ by our description
of the structure sheaves $\mathcal{O}_U$ and $\mathcal{O}_V$
above. Hence, the morphisms of ringed topoi
$(\alpha_{small}, \alpha^\sharp)$ and $(\beta_{small}, \beta^\sharp)$
are $2$-isomorphic. By
\'Etale Cohomology, Lemma \ref{etale-cohomology-lemma-faithful}
we conclude $\alpha = \beta$! Since $p : U \to X$ is a surjection
of sheaves it follows that $a = b$.
\end{proof}

\noindent
Here is the main result of this section.

\begin{theorem}
\label{theorem-fully-faithful}
Let $X$, $Y$ be algebraic spaces over $\text{Spec}(\mathbf{Z})$.
Let
$$
(g, g^\sharp) :
(\textit{Sh}(X_{\acute{e}tale}), \mathcal{O}_X)
\longrightarrow
(\textit{Sh}(Y_{\acute{e}tale}), \mathcal{O}_Y)
$$
be a morphism of locally ringed topoi. Then there exists a
unique morphism of algebraic spaces $f : X \to Y$ such that
$(g, g^\sharp)$ is isomorphic to $(f_{small}, f^\sharp)$.
In other words, the construction
$$
\textit{Spaces}/\text{Spec}(\mathbf{Z})
\longrightarrow \textit{Locally ringed topoi},
\quad
X \longrightarrow (X_{\acute{e}tale}, \mathcal{O}_X)
$$
is fully faithful (morphisms up to $2$-isomorphisms on the right hand side).
\end{theorem}

\begin{proof}
The uniqueness we have seen in
Lemma \ref{lemma-faithful}.
Thus it suffices to prove existence.
In this proof we will freely use the identifications of
Equation (\ref{equation-localize-at-scheme-ringed})
as well as the result of
Lemma \ref{lemma-relocalize-morphism-at-schemes}.

\medskip\noindent
Let $U \in \text{Ob}(X_{\acute{e}tale})$, let
$V \in \text{Ob}(Y_{\acute{e}tale})$
and let $s \in g^{-1}h_V(U)$ be a section. We may think of
$s$ as a map of sheaves $s : h_U \to g^{-1}h_V$. By
Modules on Sites,
Lemma \ref{sites-modules-lemma-relocalize-morphism-ringed-topoi}
we obtain a commutative diagram of morphisms of ringed topoi
$$
\xymatrix{
(\textit{Sh}(X_{\acute{e}tale}/U), \mathcal{O}_U)
\ar[rr]_-{(j, j^\sharp)} \ar[d]_{(g_s, g_s^\sharp)} & &
(\textit{Sh}(X_{\acute{e}tale}), \mathcal{O}_X) \ar[d]^{(g, g^\sharp)} \\
(\textit{Sh}(V_{\acute{e}tale}), \mathcal{O}_V) \ar[rr] & &
(\textit{Sh}(Y_{\acute{e}tale}), \mathcal{O}_Y).
}
$$
By
\'Etale Cohomology, Theorem \ref{etale-cohomology-theorem-fully-faithful}
we obtain a unique morphism of schemes $f_s : U \to V$ such that
$(g_s, g_s^\sharp)$ is $2$-isomorphic to $(f_{s, small}, f_s^\sharp)$.
The construction $(U, V, s) \leadsto f_s$ just explained satisfies
the following functoriality property: Suppose given morphisms
$a : U' \to U$ in $X_{\acute{e}tale}$ and $b : V' \to V$ in $Y_{\acute{e}tale}$
and a map $s' : h_{U'} \to g^{-1}h_{V'}$ such that the diagram
$$
\xymatrix{
h_{U'} \ar[d]_a \ar[r]_{s'} & g^{-1}h_{V'} \ar[d]^{g^{-1}b} \\
h_U \ar[r]^s & g^{-1}h_V
}
$$
commutes. Then the diagram
$$
\xymatrix{
U' \ar[r]_-{f_{s'}} \ar[d]_a & u(V') \ar[d]^{u(b)} \\
U \ar[r]^-{f_s} & u(V)
}
$$
of schemes commutes. The reason this is true is that the same condition
holds for the morphisms $(g_s, g_s^\sharp)$ constructed in
Modules on Sites,
Lemma \ref{sites-modules-lemma-relocalize-morphism-ringed-topoi}
and the uniqueness in
\'Etale Cohomology, Theorem \ref{etale-cohomology-theorem-fully-faithful}.

\medskip\noindent
The problem is to glue the morphisms $f_s$ to a morphism of algebraic
spaces. To do this first choose a scheme $V$ and a surjective \'etale
morphism $V \to Y$. This means that $h_V \to *$ is surjective and hence
$g^{-1}h_V \to *$ is surjective too. This means there exists a scheme $U$
and a surjective \'etale morphism $U \to X$ and a morphism
$s : h_U \to g^{-1}h_V$. Next, set $R = V \times_Y V$ and
$R' = U \times_X U$. Then we get
$g^{-1}h_R = g^{-1}h_V \times g^{-1}h_V$ as $g^{-1}$ is exact.
Thus $s$ induces a morphism $s \times s : h_{R'} \to g^{-1}h_R$.
Applying the constructions above we see that we get a
commutative diagram of morphisms of schemes
$$
\xymatrix{
R' \ar@<1ex>[d] \ar@<-1ex>[d] \ar[rr]_{f_{s \times s}} & &
R \ar@<1ex>[d] \ar@<-1ex>[d] \\
U \ar[rr]^{f_s} & &
V
}
$$
Since we have $X = U/R'$ and $Y = V/R$ (see
Spaces, Lemma \ref{spaces-lemma-space-presentation})
we conclude that this diagram
defines a morphism of algebraic spaces $f : X \to Y$ fitting
into an obvious commutative diagram.
Now we still have to show that $(f_{small}, f^\sharp)$ is
$2$-isomorphic to $(g, g^\sharp)$.
Let $t_V : f_{s, small}^{-1} \to g_s^{-1}$ and
$t_R : f_{s \times s, small}^{-1} \to g_{s \times s}^{-1}$ be
the $2$-isomorphisms which are given to us by the construction above.
Let $\mathcal{G}$ be a sheaf on $Y_{\acute{e}tale}$. Then we see that
$t_V$ defines an isomorphism
$$
f_{small}^{-1}\mathcal{G}|_{U_{\acute{e}tale}}
=
f_{s, small}^{-1}\mathcal{G}|_{V_{\acute{e}tale}}
\xrightarrow{t_V}
g_s^{-1}\mathcal{G}|_{V_{\acute{e}tale}}
=
g^{-1}\mathcal{G}|_{U_{\acute{e}tale}}.
$$
Moreover, this isomorphism pulled back to $R'$ via either projection
$R' \to U$ is the isomorphism
$$
f_{small}^{-1}\mathcal{G}|_{R'_{\acute{e}tale}}
=
f_{s \times s, small}^{-1}\mathcal{G}|_{R_{\acute{e}tale}}
\xrightarrow{t_R}
g_{s \times s}^{-1}\mathcal{G}|_{R_{\acute{e}tale}}
=
g^{-1}\mathcal{G}|_{R'_{\acute{e}tale}}.
$$
Since $\{U \to X\}$ is a covering in the site $X_{spaces, \acute{e}tale}$
this means the first displayed isomorphism descends to an isomorphism
$t : f_{small}^{-1}\mathcal{G} \to g^{-1}\mathcal{G}$
of sheaves (small detail omitted). The isomorphism is functorial
in $\mathcal{G}$ since $t_V$ and $t_R$ are transformations of functors.
Finally, $t$ is compatible with $f^\sharp$ and $g^\sharp$ as
$t_V$ and $t_R$ are (some details omitted).
This finishes the proof of the theorem.
\end{proof}

\begin{lemma}
\label{lemma-isomorphism-ringed-topoi}
Let $X$, $Y$ be algebraic spaces over $\mathbf{Z}$. If
$$
(g, g^\sharp) :
(\textit{Sh}(X_{\acute{e}tale}), \mathcal{O}_X)
\longrightarrow
(\textit{Sh}(Y_{\acute{e}tale}), \mathcal{O}_Y)
$$
is an isomorphism of ringed topoi, then there exists a unique
morphism $f : X \to Y$ of algebraic spaces such that
$(g, g^\sharp)$ is isomorphic to $(f_{small}, f^\sharp)$
and moreover $f$ is an isomorphism of algebraic spaces.
\end{lemma}

\begin{proof}
By
Theorem \ref{theorem-fully-faithful}
it suffices to show that $(g, g^\sharp)$ is a morphism of
locally ringed topoi. By
Modules on Sites, Lemma \ref{sites-modules-lemma-locally-ringed-morphism}
(and since the site $X_{\acute{e}tale}$ has enough points)
it suffices to check that the map
$\mathcal{O}_{Y, q} \to \mathcal{O}_{X, p}$ induced by $g^\sharp$
is a local ring map where $q = f \circ p$ and $p$ is any point of
$X_{\acute{e}tale}$. As it is an isomorphism this is clear.
\end{proof}














\section{Quasi-coherent sheaves on algebraic spaces}
\label{section-quasi-coherent}

\noindent
In
Descent, Section \ref{descent-section-quasi-coherent-sheaves}
we have seen that for a scheme $U$, there is no difference between a
quasi-coherent $\mathcal{O}_U$-module on $U$, or a quasi-coherent
$\mathcal{O}$-module on the small \'etale site of $U$. Hence the following
definition is compatible with our original notion of a quasi-coherent sheaf
on a scheme
(Schemes, Section \ref{schemes-section-quasi-coherent}),
when applied to a representable algebraic space.

\begin{definition}
\label{definition-quasi-coherent}
Let $S$ be a scheme. Let $X$ be an algebraic space over $S$.
A {\it quasi-coherent} $\mathcal{O}_X$-module 
is a quasi-coherent module on the ringed site
$(X_{\acute{e}tale}, \mathcal{O}_X)$ in the sense of
Modules on Sites,
Definition \ref{sites-modules-definition-site-local}.
\end{definition}

\noindent
Note that as being quasi-coherent is an intrinsic notion (see
Modules on Sites, Lemma \ref{sites-modules-lemma-special-locally-free})
this is equivalent to saying that the corresponding $\mathcal{O}_X$-module
on $X_{spaces, \acute{e}tale}$ is quasi-coherent.

\medskip\noindent
As usual, quasi-coherent sheaves behave well with respect to pullback.

\begin{lemma}
\label{lemma-pullback-quasi-coherent}
Let $S$ be a scheme.
Let $f : X \to Y$ be a morphism of algebraic spaces over $S$.
The pullback functor
$f^* : \textit{Mod}(\mathcal{O}_Y) \to \textit{Mod}(\mathcal{O}_X)$
preserves quasi-coherent sheaves.
\end{lemma}

\begin{proof}
This is a general fact, see
Modules on Sites, Lemma \ref{sites-modules-lemma-local-pullback}.
\end{proof}

\noindent
Note that this pullback functor agrees with the usual pullback functor
between quasi-coherent sheaves of modules if $X$ and $Y$ happen to be
schemes, see
Descent, Proposition
\ref{descent-proposition-equivalence-quasi-coherent-functorial}.
Here is the obligatory lemma comparing this with quasi-coherent sheaves
on the objects of the small \'etale site of $X$.

\begin{lemma}
\label{lemma-characterize-quasi-coherent-small-etale}
Let $S$ be a scheme. Let $X$ be an algebraic space over $S$.
A quasi-coherent $\mathcal{O}_X$-module $\mathcal{F}$
is given by the following data:
\begin{enumerate}
\item for every $U \in \text{Ob}(X_{\acute{e}tale})$ a quasi-coherent
$\mathcal{O}_U$-module $\mathcal{F}_U$ on $U_{\acute{e}tale}$,
\item for every $f : U' \to U$ in $X_{\acute{e}tale}$ an isomorphism
$c_f : f_{small}^*\mathcal{F}_U \to \mathcal{F}_{U'}$.
\end{enumerate}
These data are subject to the condition that given any $f : U' \to U$
and $g : U'' \to U'$ in $X_{\acute{e}tale}$ the composition
$g_{small}^{-1}c_f \circ c_g$ is equal to $c_{f \circ g}$.
\end{lemma}

\begin{proof}
Combine Lemmas \ref{lemma-pullback-quasi-coherent} and
\ref{lemma-characterize-module-small-etale}.
\end{proof}

\begin{lemma}
\label{lemma-stalk-quasi-coherent}
Let $S$ be a scheme.
Let $X$ be an algebraic space over $S$.
Let $\mathcal{F}$ be a quasi-coherent $\mathcal{O}_X$-module.
Let $x \in |X|$ be a point and let $\overline{x}$ be a geometric
point lying over $x$. Finally, let
$\varphi : (U, \overline{u}) \to (X, \overline{x})$
be an \'etale neighbourhood where $U$ is a scheme.
Then
$$
(\varphi^*\mathcal{F})_u \otimes_{\mathcal{O}_{U, u}}
\mathcal{O}_{X, \overline{x}} =
\mathcal{F}_{\overline{x}}
$$
where $u \in U$ is the image of $\overline{u}$.
\end{lemma}

\begin{proof}
Note that $\mathcal{O}_{X, \overline{x}} = \mathcal{O}_{U, u}^{sh}$ by
Lemma \ref{lemma-describe-etale-local-ring}
hence the tensor product makes sense. Moreover, from
Definition \ref{definition-stalk}
it is clear that
$$
\mathcal{F}_{\overline{u}} = \text{colim}\ (\varphi^*\mathcal{F})_u
$$
where the colimit is over $\varphi : (U, \overline{u}) \to (X, \overline{x})$
as in the lemma. Hence there is a canonical map from left to right in
the statement of the lemma. We have a similar colimit description for
$\mathcal{O}_{X, \overline{x}}$
and by
Lemma \ref{lemma-characterize-quasi-coherent-small-etale}
we have
$$
((\varphi')^*\mathcal{F})_{u'} =
(\varphi^*\mathcal{F})_u \otimes_{\mathcal{O}_{U, u}} \mathcal{O}_{U', u'}
$$
whenever $(U', \overline{u}') \to (U, \overline{u})$ is a morphism of
\'etale neighbourhoods. To complete the proof we use that
$\otimes$ commutes with colimits.
\end{proof}

\begin{lemma}
\label{lemma-stalk-pullback-quasi-coherent}
Let $S$ be a scheme. Let $f : X \to Y$ be a morphism of algebraic spaces
over $S$. Let $\mathcal{G}$ be a quasi-coherent $\mathcal{O}_Y$-module.
Let $\overline{x}$ be a geometric point of $X$ and let
$\overline{y} = f \circ \overline{x}$ be the image in $Y$.
Then there is a canonical isomorphism
$$
(f^*\mathcal{G})_{\overline{x}} =
\mathcal{G}_{\overline{y}} \otimes_{\mathcal{O}_{Y, \overline{y}}}
\mathcal{O}_{X, \overline{x}}
$$
of the stalk of the pullback with the tensor product of the stalk
with the local ring of $X$ at $\overline{x}$.
\end{lemma}

\begin{proof}
Since $f^*\mathcal{G} =
f_{small}^{-1}\mathcal{G} \otimes_{f_{small}^{-1}\mathcal{O}_Y} \mathcal{O}_X$
this follows from the description of stalks of pullbacks in
Lemma \ref{lemma-stalk-pullback}
and the fact that taking stalks commutes with tensor products.
A more direct way to see this is as follows.
Choose a commutative diagram
$$
\xymatrix{
U \ar[d]_p \ar[r]_\alpha  & V \ar[d]^q \\
X \ar[r]^a & Y
}
$$
where $U$ and $V$ are schemes, and $p$ and $q$ are surjective \'etale.
By
Lemma \ref{lemma-geometric-lift-to-usual}
we can choose a geometric point $\overline{u}$ of $U$ such that
$\overline{x} = p \circ \overline{u}$. Set
$\overline{v} = \alpha \circ \overline{u}$.
Then we see that
\begin{align*}
(f^*\mathcal{G})_{\overline{x}} & =
(p^*f^*\mathcal{G})_u \otimes_{\mathcal{O}_{U, u}}
\mathcal{O}_{X, \overline{x}} \\
& = (\alpha^*q^*\mathcal{G})_u \otimes_{\mathcal{O}_{U, u}}
\mathcal{O}_{X, \overline{x}} \\
& = (q^*\mathcal{G})_v \otimes_{\mathcal{O}_{V, v}}
\mathcal{O}_{U, u} \otimes_{\mathcal{O}_{U, u}}
\mathcal{O}_{X, \overline{x}} \\
& = (q^*\mathcal{G})_v \otimes_{\mathcal{O}_{V, v}}
\mathcal{O}_{X, \overline{x}} \\
& = (q^*\mathcal{G})_v \otimes_{\mathcal{O}_{V, v}}
\mathcal{O}_{Y, \overline{y}} \otimes_{\mathcal{O}_{Y, \overline{y}}}
\mathcal{O}_{X, \overline{x}} \\
& = \mathcal{G}_{\overline{y}} \otimes_{\mathcal{O}_{Y, \overline{y}}}
\mathcal{O}_{X, \overline{x}}
\end{align*}
Here we have used
Lemma \ref{lemma-stalk-quasi-coherent} (twice)
and the corresponding result for pullbacks of quasi-coherent sheaves
on schemes, see
Sheaves, Lemma \ref{sheaves-lemma-stalk-pullback-modules}.
\end{proof}

\begin{lemma}
\label{lemma-characterize-quasi-coherent}
Let $S$ be a scheme. Let $X$ be an algebraic space over $S$.
Let $\mathcal{F}$ be a sheaf of $\mathcal{O}_X$-modules.
The following are equivalent
\begin{enumerate}
\item $\mathcal{F}$ is a quasi-coherent $\mathcal{O}_X$-module,
\item there exists an \'etale morphism $f : Y \to X$ of
algebraic spaces over $S$ with $|f| : |Y| \to |X|$ surjective
such that $f^*\mathcal{F}$ is quasi-coherent on $Y$,
\item there exists a scheme $U$ and a surjective \'etale morphism
$\varphi : U \to X$ such that $\varphi^*\mathcal{F}$ is a quasi-coherent
$\mathcal{O}_U$-module, and
\item for every affine scheme $U$ and \'etale morphism $\varphi : U \to X$ the
restriction $\varphi^*\mathcal{F}$ is a quasi-coherent $\mathcal{O}_U$-module.
\end{enumerate}
\end{lemma}

\begin{proof}
It is clear that (1) implies (2) by considering $\text{id}_X$.
Assume $f : Y \to X$ is as in (2), and let $V \to Y$ be a surjective
\'etale morphism from a scheme towards $Y$. Then the composition $V \to X$ is
surjective \'etale as well
and by Lemma \ref{lemma-pullback-quasi-coherent} the pullback of $\mathcal{F}$
to $V$ is quasi-coherent as well. Hence we see that (2) implies (3).

\medskip\noindent
Let $U \to X$ be as in (3). Let us use the abuse of notation introduced
in Equation (\ref{equation-restrict-modules}).
As $\mathcal{F}|_{U_{\acute{e}tale}}$ is quasi-coherent there exists an
\'etale covering $\{U_i \to U\}$ such that
$\mathcal{F}|_{U_{i, \acute{e}tale}}$ has a global presentation, see
Modules on Sites, Definition \ref{sites-modules-definition-global} and
Lemma \ref{sites-modules-lemma-local-final-object}.
Let $V \to X$ be an object of $X_{\acute{e}tale}$. Since $U \to X$ is
surjective and \'etale, the family of maps $\{U_i \times_X V \to V\}$ is an
\'etale covering
of $V$. Via the morphisms $U_i \times_X V \to U_i$ we can restrict the
global presentations of $\mathcal{F}|_{U_{i, \acute{e}tale}}$ to get a global
presentation of $\mathcal{F}|_{(U_i \times_X V)_{\acute{e}tale}}$
Hence the sheaf $\mathcal{F}$ on $X_{\acute{e}tale}$ satisfies the condition of
Modules on Sites, Definition \ref{sites-modules-definition-site-local}
and hence is quasi-coherent.

\medskip\noindent
The equivalence of (3) and (4) comes from the fact that any scheme has
an affine open covering.
\end{proof}

\begin{lemma}
\label{lemma-properties-quasi-coherent}
Let $S$ be a scheme. Let $X$ be an algebraic space over $S$.
The category $\text{QCoh}(X)$ of quasi-coherent sheaves on $X$ has
the following properties:
\begin{enumerate}
\item Any direct sum of quasi-coherent sheaves is quasi-coherent.
\item Any colimit of quasi-coherent sheaves is quasi-coherent.
\item The kernel and cokernel of a morphism of quasi-coherent sheaves
is quasi-coherent.
\item Given a short exact sequence of $\mathcal{O}_X$-modules
$0 \to \mathcal{F}_1 \to \mathcal{F}_2 \to \mathcal{F}_3 \to 0$
if two out of three are quasi-coherent so is the third.
\item Given two quasi-coherent $\mathcal{O}_X$-modules
the tensor product is quasi-coherent.
\item Given two quasi-coherent $\mathcal{O}_X$-modules
$\mathcal{F}$, $\mathcal{G}$ such that $\mathcal{F}$
is of finite presentation (see
Section \ref{section-properties-modules}),
then the internal hom
$\textit{Hom}_{\mathcal{O}_X}(\mathcal{F}, \mathcal{G})$
is quasi-coherent.
\end{enumerate}
\end{lemma}

\begin{proof}
Choose a scheme $U$ and a surjective \'etale morphism $\varphi : U \to X$.
We have a commutative diagram
$$
\xymatrix{
\text{QCoh}(X) \ar[r] \ar[d] & \text{QCoh}(U) \ar[d] \\
\text{Mod}(\mathcal{O}_X) \ar[r] & \text{Mod}(\mathcal{O}_U)
}
$$
where the bottom horizontal arrow is the restriction functor
(\ref{equation-restrict-modules})
$\mathcal{G} \mapsto \mathcal{G}|_{U_{\acute{e}tale}}$ which commutes
with all limits and colimits (because it has both a left adjoint and
a right adjoint, see
Modules on Sites, Section \ref{sites-modules-section-localize}).
Moreover, we know that an object of $\text{Mod}(\mathcal{O}_X)$ is in
$\text{QCoh}(X)$ if and only if its restriction to $U$ is in
$\text{QCoh}(U)$. Hence the result follows from the corresponding
result on quasi-coherent modules on schemes, see
Schemes, Section \ref{schemes-section-quasi-coherent}.
\end{proof}

\noindent
It is in general not the case that the pushforward of a quasi-coherent sheaf
along a morphism of algebraic spaces is quasi-coherent. We will return to this
issue in
Morphisms of Spaces, Section \ref{spaces-morphisms-section-pushforward}.




\section{Properties of modules}
\label{section-properties-modules}

\noindent
In
Modules on Sites, Sections \ref{sites-modules-section-global},
\ref{sites-modules-section-local}, and
\ref{sites-modules-definition-flat}
we have defined a number of intrinsic properties of modules of
$\mathcal{O}$-module on any ringed topos. If $X$ is an algebraic
space, we will apply these notions freely to modules on the ringed
site $(X_{\acute{e}tale}, \mathcal{O}_X)$, or equivalently on the ringed site
$(X_{spaces, \acute{e}tale}, \mathcal{O}_X)$.

\medskip\noindent
Global properties $\mathcal{P}$:
\begin{enumerate}
\item {\it free},
\item {\it finite free},
\item {\it generated by global sections},
\item {\it generated by finitely many global sections},
\item having a {\it global presentation}, and
\item having a {\it global finite presentation}.
\end{enumerate}
Local properties $\mathcal{P}$:
\begin{enumerate}
\item {\it locally free},
\item {\it finite locally free},
\item {\it locally generated by sections},
\item {\it finite type},
\item {\it quasi-coherent} (see Section \ref{section-quasi-coherent}),
\item {\it of finite presentation},
\item {\it coherent}, and
\item {\it flat}.
\end{enumerate}
In each case, except for $\mathcal{P}=$``coherent'', the property is preserved
under pullback, see
Modules on Sites, Lemma \ref{sites-modules-lemma-global-pullback},
Modules on Sites, Lemma \ref{sites-modules-lemma-local-pullback}, and
Modules on Sites, Lemma \ref{sites-modules-lemma-pullback-flat}.
In particular, if $\mathcal{F}$ is an $\mathcal{O}_X$-module on
$X_{\acute{e}tale}$ satisfying one of the properties $\mathcal{P}$ above
and $\varphi : U \to X$ is a surjective \'etale
morphism with $U$ a scheme, then the pullback $\varphi^*\mathcal{F}$
has property $\mathcal{P}$ as a sheaf of modules on $U_{\acute{e}tale}$.
Moreover, for each of the local properties $\mathcal{P}$, the fact that
$\varphi^*\mathcal{G}$ has $\mathcal{P}$ implies
that $\mathcal{G}$ has $\mathcal{P}$. This follows as $\{U \to X\}$
is a covering in $X_{spaces, \acute{e}tale}$ and
Modules on Sites, Lemma \ref{sites-modules-lemma-local-final-object}.
Finally, if $\mathcal{G}$ is assumed quasi-coherent and for any
$\mathcal{P}$ except $\mathcal{P}=$``coherent'' or ``locally free'',
then $\mathcal{P}$ for $\varphi^*\mathcal{G}$ on $U_{\acute{e}tale}$ is
equivalent to the corresponding property for
$\varphi^*\mathcal{G}|_{U_{Zar}}$, i.e., it corresponds to $\mathcal{P}$
for $\varphi^*\mathcal{G}$ when we think of it as a quasi-coherent sheaf
on the scheme $U$. See
Descent, Lemma \ref{descent-lemma-equivalence-quasi-coherent-properties}.



\section{Locally projective modules}
\label{section-locally-projective}

\noindent
Recall that in
Properties, Section \ref{properties-section-locally-projective}
we defined the notion of a locally projective
quasi-coherent module.

\begin{lemma}
\label{lemma-locally-projective}
Let $S$ be a scheme. Let $X$ be an algebraic space over $S$.
Let $\mathcal{F}$ be a quasi-coherent $\mathcal{O}_X$-module.
The following are equivalent
\begin{enumerate}
\item for some scheme $U$ and surjective \'etale morphism
$U \to X$ the restriction $\mathcal{F}|_U$ is locally projective
on $U$, and
\item for any scheme $U$ and any \'etale morphism
$U \to X$ the restriction $\mathcal{F}|_U$ is locally projective
on $U$.
\end{enumerate}
\end{lemma}

\begin{proof}
Let $U \to X$ be as in (1) and let $V \to X$ be \'etale where
$V$ is a scheme. Then $\{U \times_X V \to V\}$ is an fppf covering
of schemes. Hence if $\mathcal{F}|_U$ is locally projective, then
$\mathcal{F}|_{U \times_X V}$ is locally projective (see
Properties, Lemma \ref{properties-lemma-locally-projective-pullback})
and hence $\mathcal{F}|_V$ is locally projective, see
Descent, Lemma \ref{descent-lemma-locally-projective-descends}.
\end{proof}

\begin{definition}
\label{definition-locally-projective}
Let $S$ be a scheme. Let $X$ be an algebraic space over $S$.
Let $\mathcal{F}$ be a quasi-coherent $\mathcal{O}_X$-module.
We say $\mathcal{F}$ is {\it locally projective}
if the equivalent conditions of
Lemma \ref{lemma-locally-projective}
are satisfied.
\end{definition}

\begin{lemma}
\label{lemma-locally-projective-pullback}
Let $S$ be a scheme.
Let $f : X \to Y$ be a morphism of algebraic spaces over $S$.
Let $\mathcal{G}$ be a quasi-coherent $\mathcal{O}_Y$-module.
If $\mathcal{G}$ is locally projective on $Y$, then $f^*\mathcal{G}$
is locally projective on $X$.
\end{lemma}

\begin{proof}
Choose a surjective \'etale morphism $V \to Y$ with $V$ a scheme.
Choose a surjective \'etale morphism $U \to V \times_Y X$ with
$U$ a scheme. Denote $\psi : U \to V$ the induced morphism.
Then
$$
f^*\mathcal{G}|_U = \psi^*(\mathcal{G}|_V)
$$
Hence the lemma follows from the definition and the result in the
case of schemes, see
Properties, Lemma \ref{properties-lemma-locally-projective-pullback}.
\end{proof}





\section{Quasi-coherent sheaves and presentations}
\label{section-quasi-coherent-presentation}

\noindent
Let $S$ be a scheme. Let $X$ be an algebraic space over $S$.
Let $X = U/R$ be a presentation of $X$ coming from any surjective
\'etale morphism $\varphi : U \to X$, see
Spaces, Definition \ref{spaces-definition-presentation}.
In particular, we obtain a groupoid $(U, R, s, t, c)$, such that
$j = (t, s) : R \to U \times_S U$, see
Groupoids, Lemma \ref{groupoids-lemma-equivalence-groupoid}.
In
Groupoids, Definition \ref{groupoids-definition-groupoid-module}
we have the defined the notion of a quasi-coherent sheaf
on an arbitrary groupoid. With these notions in place we have
the following observation.

\begin{proposition}
\label{proposition-quasi-coherent}
With $S$, $\varphi : U \to X$, and $(U, R, s, t, c)$ as above.
For any quasi-coherent $\mathcal{O}_X$-module $\mathcal{F}$ the
sheaf $\varphi^*\mathcal{F}$ comes equipped with a canonical
isomorphism
$$
\alpha : t^*\varphi^*\mathcal{F} \longrightarrow s^*\varphi^*\mathcal{F}
$$
which satisfies the conditions of 
Groupoids, Definition \ref{groupoids-definition-groupoid-module}
and therefore defines a quasi-coherent sheaf on $(U, R, s, t, c)$.
The functor $\mathcal{F} \mapsto (\varphi^*\mathcal{F}, \alpha)$
defines an equivalence of categories
$$
\begin{matrix}
\text{Quasi-coherent} \\
\mathcal{O}_X\text{-modules}
\end{matrix}
\longleftrightarrow
\begin{matrix}
\text{Quasi-coherent modules}\\
\text{on }(U, R, s, t, c)
\end{matrix}
$$
\end{proposition}

\begin{proof}
In the statement of the proposition, and in this proof we think of a
quasi-coherent sheaf on a scheme as a quasi-coherent sheaf on the small
\'etale site of that scheme. This is permissible by the results of
Descent, Section \ref{descent-section-quasi-coherent-sheaves}.

\medskip\noindent
The existence of $\alpha$ comes from the fact that
$\varphi \circ t = \varphi \circ s$ and that pullback is
functorial in the morphism, see discussion surrounding
Equation (\ref{equation-push-pull}). In exacty the same way, i.e., by
functoriality of pullback, we see that the isomorphism $\alpha$ satisfies
condition (1) of
Groupoids, Definition \ref{groupoids-definition-groupoid-module}.
To see condition (2) of the definition it suffices to see that $\alpha$
is an isomorphism which is clear. The construction
$\mathcal{F} \mapsto (\varphi^*\mathcal{F}, \alpha)$
is clearly functorial in the quasi-coherent sheaf $\mathcal{F}$.
Hence we obtain the functor from left to right in the displayed
formula of the lemma.

\medskip\noindent
Conversely, suppose that $(\mathcal{F}, \alpha)$ is a quasi-coherent
sheaf on $(U, R, s, t, c)$. Let $V \to X$ be an object of $X_{\acute{e}tale}$.
In this case the morphism $V' = U \times_X V \to V$ is a surjective \'etale
morphism of schemes, and hence $\{V' \to V\}$ is an \'etale
covering of $V$. Moreover, the quasi-coherent sheaf $\mathcal{F}$
pulls back to a quasi-coherent sheaf $\mathcal{F}'$ on $V'$.
Since $R = U \times_X U$ with $t = \text{pr}_0$ and $s = \text{pr}_0$
we see that $V' \times_V V' = R \times_X V$ with projection maps
$V' \times_V V' \to V'$ equal to the pullbacks of $t$ and $s$. Hence
$\alpha$ pulls back to an isomorphism
$\alpha' : \text{pr}_0^*\mathcal{F}' \to \text{pr}_1^*\mathcal{F}'$, and
the pair $(\mathcal{F}', \alpha')$ is a descend datum for quasi-coherent
sheaves with respect to $\{V' \to V\}$. By
Descent, Lemma \ref{descent-proposition-fpqc-descent-quasi-coherent}
this descent datum is effective, and we obtain a quasi-coherent
$\mathcal{O}_V$-module $\mathcal{F}_V$ on $V_{\acute{e}tale}$.
To see that this gives a quasi-coherent sheaf on $X_{\acute{e}tale}$ we have
to show (by
Lemma \ref{lemma-characterize-quasi-coherent-small-etale})
that for any morphism $f : V_1 \to V_2$ in $X_{\acute{e}tale}$
there is a canonical isomorphism
$c_f : \mathcal{F}_{V_1} \to \mathcal{F}_{V_2}$
compatible with compositions of morphisms. We omit the verification.
We also omit the verification that this defines a functor from the
category on the right to the category on the left which is inverse
to the functor described above.
\end{proof}




\section{Decent spaces}
\label{section-decent}

\noindent
In this section we collect some useful facts on decent spaces.

\begin{lemma}
\label{lemma-when-field}
Let $S$ be a scheme.
Let $X$ be a decent reduced algebraic space over $S$.
Assume that $|X|$ is a singleton.
Then $X \cong \text{Spec}(k)$ for some field $k$.
\end{lemma}

\begin{proof}
As $|X|$ is a singleton $X$ is quasi-compact, see
Lemma \ref{lemma-quasi-compact-space}.
Let $U \to X$ be surjective \'etale with $U$ an affine scheme, see
Lemma \ref{lemma-quasi-compact-affine-cover}.
Since $X$ is reduced we see that $U$ is reduced, see
Section \ref{section-types-properties}.
As $X$ is decent there exists a monomorphism $\text{Spec}(k) \to X$,
and $V = \text{Spec}(k) \times_X U$ which is a scheme finite \'etale
over $\text{Spec}(k)$. Namely, this
follows from the definition of decent, see
Definition \ref{definition-very-reasonable},
which says that the equivalent conditions of
Lemma \ref{lemma-UR-finite-above-x}
hold at the unique point of $X$. Hence $V$ is a finite disjoint union of
spectra of finite separable field extensions of $k$, see
Morphisms, Lemma \ref{morphisms-lemma-etale-over-field}.
On the other hand $V \to U$ is a monomorphism (as $\text{Spec}(k) \to X$
is a monomorphism) and surjective (as $\text{Spec}(k) \to X$ is surjective by
Lemma \ref{lemma-characterize-surjective}).
In particular $U$ has finitely many points.
By Lemma \ref{lemma-no-specializations-map-to-same-point}
there are no specializations among the points of $U$ (note that decent
implies the condition of that lemma are satisfied in view of
Lemma \ref{lemma-bounded-fibres}).
It follows that $U$ is a finite discrete topological space.
As $U$ is also reduced it follows that $U$ is a disjoint union
of spectra of fields. By
Schemes, Lemma \ref{schemes-lemma-mono-towards-spec-field}
we conclude that $V \to U$ is an isomorphism. Hence we see that
$U \to X$ factors through $\text{Spec}(k)$ which implies that
$\text{Spec}(k) \to X$ is also a surjection of sheaves, whence
an isomorphism as desired.
\end{proof}

\begin{remark}
\label{remark-one-point-decent-scheme}
We will see later (insert future reference here) that an algebraic space whose
reduction is a scheme is a scheme. Hence it follows from 
Lemma \ref{lemma-when-field}
that a decent algebraic space with one point is a scheme.
\end{remark}




\section{Morphisms towards schemes}
\label{section-morphisms-to-schemes}

\noindent
Here is the analogue of
Schemes, Lemma \ref{schemes-lemma-morphism-into-affine}.

\begin{lemma}
\label{lemma-morphism-to-affine-scheme}
Let $X$ be an algebraic space over $\mathbf{Z}$.
Let $T$ be an affine scheme.
The map
$$
\text{Mor}(X, T)
\longrightarrow
\text{Hom}(\Gamma(T, \mathcal{O}_T), \Gamma(X, \mathcal{O}_X))
$$
which maps $f$ to $f^\sharp$ (on global sections) is bijective.
\end{lemma}

\begin{proof}
We construct the inverse of the map.
Let $\varphi : \Gamma(T, \mathcal{O}_T) \to \Gamma(X, \mathcal{O}_X)$
be a ring map. Choose a presentation $X = U/R$, see
Spaces, Definition \ref{spaces-definition-presentation}.
By
Schemes, Lemma \ref{schemes-lemma-morphism-into-affine}
the composition
$$
\Gamma(T, \mathcal{O}_T) \to \Gamma(X, \mathcal{O}_X) \to
\Gamma(U, \mathcal{O}_U)
$$
corresponds to a unique morphism of schemes $g : U \to T$. By the same lemma
the two compositions $R \to U \to T$ are equal. Hence we obtain a morphism
$f : X = U/R \to T$ such that $U \to X \to T$ equals $g$. By construction
the diagram
$$
\xymatrix{
\Gamma(U, \mathcal{O}_U) & \Gamma(X, \mathcal{O}_X) \ar[l]^{f^\sharp} \\
& \Gamma(T, \mathcal{O}_T) \ar[lu]^{g^\sharp} \ar[u]^\varphi
}
$$
commutes. Hence $f^\sharp$ equals $\varphi$ because $U \to X$ is an
\'etale covering and $\mathcal{O}_X$ is a sheaf on $X_{\acute{e}tale}$.
The uniqueness of $f$ follows from the uniqueness of $g$.
\end{proof}






\section{Other chapters}

\begin{multicols}{2}
\begin{enumerate}
\item \hyperref[introduction-section-phantom]{Introduction}
\item \hyperref[conventions-section-phantom]{Conventions}
\item \hyperref[sets-section-phantom]{Set Theory}
\item \hyperref[categories-section-phantom]{Categories}
\item \hyperref[topology-section-phantom]{Topology}
\item \hyperref[sheaves-section-phantom]{Sheaves on Spaces}
\item \hyperref[algebra-section-phantom]{Commutative Algebra}
\item \hyperref[sites-section-phantom]{Sites and Sheaves}
\item \hyperref[homology-section-phantom]{Homological Algebra}
\item \hyperref[derived-section-phantom]{Derived Categories}
\item \hyperref[more-algebra-section-phantom]{More Algebra}
\item \hyperref[simplicial-section-phantom]{Simplicial Methods}
\item \hyperref[modules-section-phantom]{Sheaves of Modules}
\item \hyperref[sites-modules-section-phantom]{Modules on Sites}
\item \hyperref[injectives-section-phantom]{Injectives}
\item \hyperref[cohomology-section-phantom]{Cohomology of Sheaves}
\item \hyperref[sites-cohomology-section-phantom]{Cohomology on Sites}
\item \hyperref[hypercovering-section-phantom]{Hypercoverings}
\item \hyperref[schemes-section-phantom]{Schemes}
\item \hyperref[constructions-section-phantom]{Constructions of Schemes}
\item \hyperref[properties-section-phantom]{Properties of Schemes}
\item \hyperref[morphisms-section-phantom]{Morphisms of Schemes}
\item \hyperref[coherent-section-phantom]{Coherent Cohomology}
\item \hyperref[divisors-section-phantom]{Divisors}
\item \hyperref[limits-section-phantom]{Limits of Schemes}
\item \hyperref[varieties-section-phantom]{Varieties}
\item \hyperref[chow-section-phantom]{Chow Homology}
\item \hyperref[topologies-section-phantom]{Topologies on Schemes}
\item \hyperref[descent-section-phantom]{Descent}
\item \hyperref[more-morphisms-section-phantom]{More on Morphisms}
\item \hyperref[flat-section-phantom]{More on Flatness}
\item \hyperref[groupoids-section-phantom]{Groupoid Schemes}
\item \hyperref[more-groupoids-section-phantom]{More on Groupoid Schemes}
\item \hyperref[etale-section-phantom]{\'Etale Morphisms of Schemes}
\item \hyperref[etale-cohomology-section-phantom]{\'Etale Cohomology}
\item \hyperref[spaces-section-phantom]{Algebraic Spaces}
\item \hyperref[spaces-properties-section-phantom]{Properties of Algebraic Spaces}
\item \hyperref[spaces-morphisms-section-phantom]{Morphisms of Algebraic Spaces}
\item \hyperref[spaces-topologies-section-phantom]{Topologies on Algebraic Spaces}
\item \hyperref[spaces-descent-section-phantom]{Descent and Algebraic Spaces}
\item \hyperref[spaces-more-morphisms-section-phantom]{More on Morphisms of Spaces}
\item \hyperref[quot-section-phantom]{Quot and Hilbert Spaces}
\item \hyperref[stacks-section-phantom]{Stacks}
\item \hyperref[spaces-groupoids-section-phantom]{Groupoids in Algebraic Spaces}
\item \hyperref[spaces-more-groupoids-section-phantom]{More on Groupoids in Spaces}
\item \hyperref[bootstrap-section-phantom]{Bootstrap}
\item \hyperref[examples-stacks-section-phantom]{Examples of Stacks}
\item \hyperref[groupoids-quotients-section-phantom]{Quotients of Groupoids}
\item \hyperref[algebraic-section-phantom]{Algebraic Stacks}
\item \hyperref[criteria-section-phantom]{Criteria for Representability}
\item \hyperref[stacks-properties-section-phantom]{Properties of Algebraic Stacks}
\item \hyperref[stacks-morphisms-section-phantom]{Morphisms of Algebraic Stacks}
\item \hyperref[examples-section-phantom]{Examples}
\item \hyperref[exercises-section-phantom]{Exercises}
\item \hyperref[guide-section-phantom]{Guide to Literature}
\item \hyperref[desirables-section-phantom]{Desirables}
\item \hyperref[coding-section-phantom]{Coding Style}
\item \hyperref[fdl-section-phantom]{GNU Free Documentation License}
\item \hyperref[index-section-phantom]{Auto Generated Index}
\end{enumerate}
\end{multicols}


\bibliography{my}
\bibliographystyle{amsalpha}

\end{document}
