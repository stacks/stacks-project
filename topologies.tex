\IfFileExists{stacks-project.cls}{%
\documentclass{stacks-project}
}{%
\documentclass{amsart}
}

% The following AMS packages are automatically loaded with
% the amsart documentclass:
%\usepackage{amsmath}
%\usepackage{amssymb}
%\usepackage{amsthm}

% For dealing with references we use the comment environment
\usepackage{verbatim}
\newenvironment{reference}{\comment}{\endcomment}
%\newenvironment{reference}{}{}
\newenvironment{slogan}{\comment}{\endcomment}
\newenvironment{history}{\comment}{\endcomment}

% For commutative diagrams you can use
% \usepackage{amscd}
\usepackage[all]{xy}

% We use 2cell for 2-commutative diagrams.
\xyoption{2cell}
\UseAllTwocells

% To put source file link in headers.
% Change "template.tex" to "this_filename.tex"
% \usepackage{fancyhdr}
% \pagestyle{fancy}
% \lhead{}
% \chead{}
% \rhead{Source file: \url{template.tex}}
% \lfoot{}
% \cfoot{\thepage}
% \rfoot{}
% \renewcommand{\headrulewidth}{0pt}
% \renewcommand{\footrulewidth}{0pt}
% \renewcommand{\headheight}{12pt}

\usepackage{multicol}

% For cross-file-references
\usepackage{xr-hyper}

% Package for hypertext links:
\usepackage{hyperref}

% For any local file, say "hello.tex" you want to link to please
% use \externaldocument[hello-]{hello}
\externaldocument[introduction-]{introduction}
\externaldocument[conventions-]{conventions}
\externaldocument[sets-]{sets}
\externaldocument[categories-]{categories}
\externaldocument[topology-]{topology}
\externaldocument[sheaves-]{sheaves}
\externaldocument[sites-]{sites}
\externaldocument[stacks-]{stacks}
\externaldocument[fields-]{fields}
\externaldocument[algebra-]{algebra}
\externaldocument[brauer-]{brauer}
\externaldocument[homology-]{homology}
\externaldocument[derived-]{derived}
\externaldocument[simplicial-]{simplicial}
\externaldocument[more-algebra-]{more-algebra}
\externaldocument[smoothing-]{smoothing}
\externaldocument[modules-]{modules}
\externaldocument[sites-modules-]{sites-modules}
\externaldocument[injectives-]{injectives}
\externaldocument[cohomology-]{cohomology}
\externaldocument[sites-cohomology-]{sites-cohomology}
\externaldocument[dga-]{dga}
\externaldocument[dpa-]{dpa}
\externaldocument[hypercovering-]{hypercovering}
\externaldocument[schemes-]{schemes}
\externaldocument[constructions-]{constructions}
\externaldocument[properties-]{properties}
\externaldocument[morphisms-]{morphisms}
\externaldocument[coherent-]{coherent}
\externaldocument[divisors-]{divisors}
\externaldocument[limits-]{limits}
\externaldocument[varieties-]{varieties}
\externaldocument[topologies-]{topologies}
\externaldocument[descent-]{descent}
\externaldocument[perfect-]{perfect}
\externaldocument[more-morphisms-]{more-morphisms}
\externaldocument[flat-]{flat}
\externaldocument[groupoids-]{groupoids}
\externaldocument[more-groupoids-]{more-groupoids}
\externaldocument[etale-]{etale}
\externaldocument[chow-]{chow}
\externaldocument[intersection-]{intersection}
\externaldocument[pic-]{pic}
\externaldocument[adequate-]{adequate}
\externaldocument[dualizing-]{dualizing}
\externaldocument[duality-]{duality}
\externaldocument[discriminant-]{discriminant}
\externaldocument[local-cohomology-]{local-cohomology}
\externaldocument[curves-]{curves}
\externaldocument[resolve-]{resolve}
\externaldocument[models-]{models}
\externaldocument[pione-]{pione}
\externaldocument[etale-cohomology-]{etale-cohomology}
\externaldocument[proetale-]{proetale}
\externaldocument[crystalline-]{crystalline}
\externaldocument[spaces-]{spaces}
\externaldocument[spaces-properties-]{spaces-properties}
\externaldocument[spaces-morphisms-]{spaces-morphisms}
\externaldocument[decent-spaces-]{decent-spaces}
\externaldocument[spaces-cohomology-]{spaces-cohomology}
\externaldocument[spaces-limits-]{spaces-limits}
\externaldocument[spaces-divisors-]{spaces-divisors}
\externaldocument[spaces-over-fields-]{spaces-over-fields}
\externaldocument[spaces-topologies-]{spaces-topologies}
\externaldocument[spaces-descent-]{spaces-descent}
\externaldocument[spaces-perfect-]{spaces-perfect}
\externaldocument[spaces-more-morphisms-]{spaces-more-morphisms}
\externaldocument[spaces-flat-]{spaces-flat}
\externaldocument[spaces-groupoids-]{spaces-groupoids}
\externaldocument[spaces-more-groupoids-]{spaces-more-groupoids}
\externaldocument[bootstrap-]{bootstrap}
\externaldocument[spaces-pushouts-]{spaces-pushouts}
\externaldocument[groupoids-quotients-]{groupoids-quotients}
\externaldocument[spaces-more-cohomology-]{spaces-more-cohomology}
\externaldocument[spaces-simplicial-]{spaces-simplicial}
\externaldocument[formal-spaces-]{formal-spaces}
\externaldocument[restricted-]{restricted}
\externaldocument[spaces-resolve-]{spaces-resolve}
\externaldocument[formal-defos-]{formal-defos}
\externaldocument[defos-]{defos}
\externaldocument[cotangent-]{cotangent}
\externaldocument[examples-defos-]{examples-defos}
\externaldocument[algebraic-]{algebraic}
\externaldocument[examples-stacks-]{examples-stacks}
\externaldocument[stacks-sheaves-]{stacks-sheaves}
\externaldocument[criteria-]{criteria}
\externaldocument[artin-]{artin}
\externaldocument[quot-]{quot}
\externaldocument[stacks-properties-]{stacks-properties}
\externaldocument[stacks-morphisms-]{stacks-morphisms}
\externaldocument[stacks-limits-]{stacks-limits}
\externaldocument[stacks-cohomology-]{stacks-cohomology}
\externaldocument[stacks-perfect-]{stacks-perfect}
\externaldocument[stacks-introduction-]{stacks-introduction}
\externaldocument[stacks-more-morphisms-]{stacks-more-morphisms}
\externaldocument[stacks-geometry-]{stacks-geometry}
\externaldocument[moduli-]{moduli}
\externaldocument[moduli-curves-]{moduli-curves}
\externaldocument[examples-]{examples}
\externaldocument[exercises-]{exercises}
\externaldocument[guide-]{guide}
\externaldocument[desirables-]{desirables}
\externaldocument[coding-]{coding}
\externaldocument[obsolete-]{obsolete}
\externaldocument[fdl-]{fdl}
\externaldocument[index-]{index}

% Theorem environments.
%
\theoremstyle{plain}
\newtheorem{theorem}[subsection]{Theorem}
\newtheorem{proposition}[subsection]{Proposition}
\newtheorem{lemma}[subsection]{Lemma}

\theoremstyle{definition}
\newtheorem{definition}[subsection]{Definition}
\newtheorem{example}[subsection]{Example}
\newtheorem{exercise}[subsection]{Exercise}
\newtheorem{situation}[subsection]{Situation}

\theoremstyle{remark}
\newtheorem{remark}[subsection]{Remark}
\newtheorem{remarks}[subsection]{Remarks}

\numberwithin{equation}{subsection}

% Macros
%
\def\lim{\mathop{\rm lim}\nolimits}
\def\colim{\mathop{\rm colim}\nolimits}
\def\Spec{\mathop{\rm Spec}}
\def\Hom{\mathop{\rm Hom}\nolimits}
\def\Ext{\mathop{\rm Ext}\nolimits}
\def\SheafHom{\mathop{\mathcal{H}\!{\it om}}\nolimits}
\def\SheafExt{\mathop{\mathcal{E}\!{\it xt}}\nolimits}
\def\Sch{\textit{Sch}}
\def\Mor{\mathop{\rm Mor}\nolimits}
\def\Ob{\mathop{\rm Ob}\nolimits}
\def\Sh{\mathop{\textit{Sh}}\nolimits}
\def\NL{\mathop{N\!L}\nolimits}
\def\proetale{{pro\text{-}\acute{e}tale}}
\def\etale{{\acute{e}tale}}
\def\QCoh{\textit{QCoh}}
\def\Ker{\mathop{\rm Ker}}
\def\Im{\mathop{\rm Im}}
\def\Coker{\mathop{\rm Coker}}
\def\Coim{\mathop{\rm Coim}}

%
% Macros for moduli stacks/spaces
%
\def\QCohstack{\mathcal{QC}\!{\it oh}}
\def\Cohstack{\mathcal{C}\!{\it oh}}
\def\Spacesstack{\mathcal{S}\!{\it paces}}
\def\Quotfunctor{{\rm Quot}}
\def\Hilbfunctor{{\rm Hilb}}
\def\Curvesstack{\mathcal{C}\!{\it urves}}
\def\Polarizedstack{\mathcal{P}\!{\it olarized}}
\def\Complexesstack{\mathcal{C}\!{\it omplexes}}
% \Pic is the operator that assigns to X its picard group, usage \Pic(X)
% \Picardstack_{X/B} denotes the Picard stack of X over B
% \Picardfunctor_{X/B} denotes the Picard functor of X over B
\def\Pic{\mathop{\rm Pic}\nolimits}
\def\Picardstack{\mathcal{P}\!{\it ic}}
\def\Picardfunctor{{\rm Pic}}
\def\Deformationcategory{\mathcal{D}\!{\it ef}}


% OK, start here.
%
\begin{document}

\title{Topologies on Schemes}

\maketitle

\tableofcontents



\section{Introduction}
\label{section-introduction}

\noindent
In this document we explain what the different topologies on the
category of schemes are. Some references are \cite{SGA1} and \cite{Ner}.
Before doing so we would like to point out that there are many
different choices of sites (as defined in
Sites, Definition \ref{sites-definition-site}) which give rise to
the same notion of sheaf on the underlying category. Hence
our choices may be slightly different from those in the references
but ultimately lead to the same cohomology groups, etc.

\section{The general procedure}
\label{section-procedure}

\noindent
In this section we explain a general procedure for producing the
sites we will be working with. Suppose we want to study sheaves
over schemes with respect to some topology $\tau$. In order to
get a site, as in Sites, Definition \ref{sites-definition-site},
of schemes with that topology we have to do some work. Namely,
we cannot simply say ``consider all schemes with the Zariski topology''
since that would give a ``big'' category. Instead, in each section of
this chapter we will proceed as follows:
\begin{enumerate}
\item We define a class $\text{Cov}_\tau$ of coverings of schemes
satisfying the axioms of Sites, Definition \ref{sites-definition-site}.
It will always be the case that a Zariski open covering of
a scheme is a covering for $\tau$.
\item We single out a notion of standard
$\tau$-covering within the category of affine schemes.
\item We define what is an ``absolute'' big $\tau$-site $\textit{Sch}_\tau$.
These are the sites one gets by appropriately choosing a set of schemes
and a set of coverings.
\item For any object $S$ of $\textit{Sch}_\tau$
we define the big $\tau$-site $(\textit{Sch}/S)_\tau$ and for suitable
$\tau$ the small\footnote{The words big and
small here do not relate to bigness/smallness of the corresponding
categories.} $\tau$-site $S_\tau$.
\item In addition there is a site $(\textit{Aff}/S)_\tau$ using the
notion of standard $\tau$-covering of affines whose category of sheaves
is equivalent to the category of sheaves on $(\textit{Sch}/S)_\tau$.
\end{enumerate}
The above is a little clumsy in that we do not end up with a canonical
choice for the big $\tau$-site of a scheme, or even the small
$\tau$-site of a scheme. If you are willing to ignore set theoretic
difficulties, then you can work with classes and end up with
canonical big and small sites...







\section{The Zariski topology}
\label{section-zariski}

\begin{definition}
\label{definition-zariski-covering}
Let $T$ be a scheme. A {\it Zariski covering of $T$} is a family
of morphisms $\{f_i : T_i \to T\}_{i \in I}$ of schemes
such that each $f_i$ is an open immersion and such
that $T = \bigcup f_i(T_i)$.
\end{definition}

\noindent
This defines a (proper) class of coverings.
Next, we show that this notion satisfies the conditions of
Sites, Definition \ref{sites-definition-site}.

\begin{lemma}
\label{lemma-zariski}
Let $T$ be a scheme.
\begin{enumerate}
\item If $T' \to T$ is an isomorphism then $\{T' \to T\}$
is a Zariski covering of $T$.
\item If $\{T_i \to T\}_{i\in I}$ is a Zariski covering and for each
$i$ we have a Zariski covering $\{T_{ij} \to T_i\}_{j\in J_i}$, then
$\{T_{ij} \to T\}_{i \in I, j\in J_i}$ is a Zariski covering.
\item If $\{T_i \to T\}_{i\in I}$ is a Zariski covering
and $T' \to T$ is a morphism of schemes then
$\{T' \times_T T_i \to T'\}_{i\in I}$ is a Zariski covering.
\end{enumerate}
\end{lemma}

\begin{proof}
Omitted.
\end{proof}

\begin{lemma}
\label{lemma-zariski-affine}
Let $T$ be an affine scheme. Let $\{T_i \to T\}_{i \in I}$ be a
Zariski covering of $T$. Then there exists a Zariski covering
$\{U_j \to T\}_{j = 1, \ldots, m}$ which is a refinement
of $\{T_i \to T\}_{i \in I}$ such that each $U_j$ is a standard
open of $T$, see
Schemes, Definition \ref{schemes-definition-standard-covering}.
Moreover, we may choose each $U_j$ to be an open of one of the $T_i$.
\end{lemma}

\begin{proof}
Follows as $T$ is quasi-compact and standard opens form a basis
for its topology.
\end{proof}

\noindent
Thus we define the corresponding standard coverings of affines as follows.

\begin{definition}
\label{definition-standard-Zariski}
Compare Schemes, Definition \ref{schemes-definition-standard-covering}.
Let $T$ be an affine scheme. A {\it standard Zariski covering}
of $T$ is a a Zariski covering $\{U_j \to T\}_{j = 1, \ldots, m}$
with each $U_j \subset T$ standard affine open.
\end{definition}

\begin{definition}
\label{definition-big-zariski-site}
A {\it big Zariski site} is any site $\textit{Sch}_{Zar}$ as in
Sites, Definition \ref{sites-definition-site} constructed as follows:
\begin{enumerate}
\item Choose any set of schemes $S_0$, and any set of Zariski coverings
$\text{Cov}_0$ among these schemes.
\item As underlying category of $\textit{Sch}_{Zar}$
take any category $\textit{Sch}_\alpha$ constructed as in
Sets, Lemma \ref{sets-lemma-construct-category} starting with the set $S_0$.
\item As coverings of $\textit{Sch}_{Zar}$ choose any set of coverings as in
Sets, Lemma \ref{sets-lemma-coverings-site} starting with the
category $\textit{Sch}_\alpha$ and the class of Zariski coverings,
and the set $\text{Cov}_0$ chosen above.
\end{enumerate}
\end{definition}

\noindent
It is shown in Sites, Lemma \ref{sites-lemma-choice-set-coverings-immaterial}
that, after having chosen the category $\textit{Sch}_\alpha$, the
category of sheaves on $\textit{Sch}_\alpha$ does not depend on the
choice of coverings chosen in (3) above. In other words, the topos
$\textit{Sh}(\textit{Sch}_{Zar})$ only depends on the choice of
the category $\textit{Sch}_\alpha$. It is shown in
Sets, Lemma \ref{sets-lemma-what-is-in-it} that these categories
are closed under many constructions of algebraic geometry, e.g.,
fibre products and taking open and closed subschemes. We can also show
that the exact choice of $\text{Sch}_\alpha$ does not matter
too much, see Section \ref{section-change-alpha}.

\medskip\noindent
Another approach is to assume the existence of a
strongly inaccessible cardinal and to define $\textit{Sch}_{Zar}$
to be the category of schemes contained in a universe with
set of coverings the Zariski coverings contained in that same
universe.

\begin{definition}
\label{definition-big-small-Zariski}
Let $S$ be a scheme. Let $\textit{Sch}_{Zar}$ be a big Zariski
site containing $S$.
\begin{enumerate}
\item The {\it big Zariski site of $S$}, denoted
$(\textit{Sch}/S)_{Zar}$, is the site $\textit{Sch}_{Zar}/S$
introduced in Sites, Section \ref{sites-section-localization}.
\item The {\it small Zariski site of $S$}, denoted
$S_{Zar}$, is the full subcategory of $(\textit{Sch}/S)_{Zar}$
whose objects are those $U/S$ such that $U \to S$ is an open immersion.
A covering of $S_{Zar}$ is any covering $\{U_i \to U\}$ of
$(\textit{Sch}/S)_{Zar}$ with $U \in \text{Ob}(S_{Zar})$.
\item The {\it big affine Zariski site of $S$}, denoted
$(\textit{Aff}/S)_{Zar}$, is the full subcategory of
$(\textit{Sch}/S)_{Zar}$ whose objects are affine $U/S$.
A covering of $(\textit{Aff}/S)_{Zar}$ is any covering
$\{U_i \to U\}$ of $(\textit{Sch}/S)_{Zar}$ which is a
standard Zariski covering.
\end{enumerate}
\end{definition}

\noindent
It is not completely clear that the small Zariski site and
the big affine Zariski site are sites. We check this now
and we make sure that the definition above corresponds to
the expected notion.

\begin{lemma}
\label{lemma-verify-site-Zariski}
Let $S$ be a scheme. Let $\textit{Sch}_{Zar}$ be a big Zariski
site containing $S$.
\begin{enumerate}
\item Both $S_{Zar}$ and $(\textit{Aff}/S)_{Zar}$ are sites.
\item The category of sheaves on $S_{Zar}$ is equivalent to the
category of sheaves on the underlying topological space of $S$.
\item The functor $(\textit{Aff}/S)_{Zar} \to (\textit{Sch}/S)_{Zar}$
is continuous and cocontinuous and induces an equivalence of topoi
between
$\textit{Sh}((\textit{Aff}/S)_{Zar})$ and
$\textit{Sh}((\textit{Sch}/S)_{Zar})$.
\end{enumerate}
\end{lemma}

\begin{proof}
Omitted.
\end{proof}













\section{The etale topology}
\label{section-etale}

\noindent
Let $S$ be a scheme. We would like to define the etale-topology on
the category of schemes over $S$. According to our general principle
we first introduce the notion of an etale covering.

\begin{definition}
\label{definition-etale-covering}
Let $T$ be a scheme. An {\it etale covering of $T$} is a family
of morphisms $\{f_i : T_i \to T\}_{i \in I}$ of schemes
such that each $f_i$ is etale and such that $T = \bigcup f_i(T_i)$.
\end{definition}

\noindent
Next, we show that this notion satisfies the conditions of
Sites, Definition \ref{sites-definition-site}.

\begin{lemma}
\label{lemma-etale}
Let $T$ be a scheme.
\begin{enumerate}
\item If $T' \to T$ is an isomorphism then $\{T' \to T\}$
is an etale covering of $T$.
\item If $\{T_i \to T\}_{i\in I}$ is an etale covering and for each
$i$ we have an etale covering $\{T_{ij} \to T_i\}_{j\in J_i}$, then
$\{T_{ij} \to T\}_{i \in I, j\in J_i}$ is an etale covering.
\item If $\{T_i \to T\}_{i\in I}$ is an etale covering
and $T' \to T$ is a morphism of schemes then
$\{T' \times_T T_i \to T'\}_{i\in I}$ is an etale covering.
\end{enumerate}
\end{lemma}

\begin{proof}
Omitted.
\end{proof}

\begin{lemma}
\label{lemma-etale-affine}
Let $T$ be an affine scheme.
Let $\{T_i \to T\}_{i \in I}$ be an etale covering of $T$.
Then there exists an etale covering
$\{U_j \to T\}_{j = 1, \ldots, m}$ which is a refinement
of $\{T_i \to T\}_{i \in I}$ such that each $U_j$ is an affine
scheme. Moreover, we may choose each $U_j$ to be open affine
in one of the $T_i$.
\end{lemma}

\begin{proof}
Omitted.
\end{proof}

\noindent
Thus we define the corresponding standard coverings of affines as follows.

\begin{definition}
\label{definition-standard-etale}
Let $T$ be an affine scheme. A {\it standard etale covering}
of $T$ is a family $\{f_j : U_j \to T\}_{j = 1, \ldots, m}$
with each $U_j$ is affine and etale over $T$ and
$T = \bigcup f_j(U_j)$.
\end{definition}

\begin{definition}
\label{definition-big-etale-site}
A {\it big etale site} is any site $\textit{Sch}_{etale}$ as in
Sites, Definition \ref{sites-definition-site} constructed as follows:
\begin{enumerate}
\item Choose any set of schemes $S_0$, and any set of etale coverings
$\text{Cov}_0$ among these schemes.
\item As underlying category take any category $\textit{Sch}_\alpha$
constructed as in Sets, Lemma \ref{sets-lemma-construct-category}
starting with the set $S_0$.
\item Choose any set of coverings as in
Sets, Lemma \ref{sets-lemma-coverings-site} starting with the
category $\textit{Sch}_\alpha$ and the class of etale coverings,
and the set $\text{Cov}_0$ chosen above.
\end{enumerate}
\end{definition}

\noindent
See the remarks following Definition \ref{definition-big-zariski-site}
for motivation and explanation regarding the definition of big sites.

\begin{definition}
\label{definition-big-small-etale}
Let $S$ be a scheme. Let $\textit{Sch}_{etale}$ be a big etale
site containing $S$.
\begin{enumerate}
\item The {\it big etale site of $S$}, denoted
$(\textit{Sch}/S)_{etale}$, is the site $\textit{Sch}_{etale}/S$
introduced in Sites, Section \ref{sites-section-localization}.
\item The {\it small etale site of $S$}, denoted
$S_{etale}$, is the full subcategory of $(\textit{Sch}/S)_{etale}$
whose objects are those $U/S$ such that $U \to S$ is etale.
A covering of $S_{etale}$ is any covering $\{U_i \to U\}$ of
$(\textit{Sch}/S)_{etale}$ with $U \in \text{Ob}(S_{etale})$.
\item The {\it big affine etale site of $S$}, denoted
$(\textit{Aff}/S)_{etale}$, is the full subcategory of
$(\textit{Sch}/S)_{etale}$ whose objects are affine $U/S$.
A covering of $(\textit{Aff}/S)_{etale}$ is any covering
$\{U_i \to U\}$ of $(\textit{Sch}/S)_{etale}$ which is a
standard etale covering.
\end{enumerate}
\end{definition}

\noindent
It is not completely clear that
the big affine etale site or the small etale site are sites.
We check this now and we make sure that the definition above corresponds to
the expected notion.

\begin{lemma}
\label{lemma-verify-site-etale}
Let $S$ be a scheme. Let $\textit{Sch}_{etale}$ be a big etale
site containing $S$.
\begin{enumerate}
\item The categories $S_{etale}$ and $(\textit{Aff}/S)_{etale}$ with coverings
as defined in Definition \ref{definition-big-small-etale} are sites.
\item The functor $(\textit{Aff}/S)_{etale} \to (\textit{Sch}/S)_{etale}$
is continuous and cocontinuous and induces an equivalence of topoi of
$\textit{Sh}((\textit{Aff}/S)_{etale})$ with
$\textit{Sh}((\textit{Sch}/S)_{etale})$.
\end{enumerate}
\end{lemma}

\begin{proof}
Omitted.
\end{proof}




















\section{The fppf topology}
\label{section-fppf}

\noindent
Let $S$ be a scheme. We would like to define the fppf-topology\footnote{
The letters fppf stand for ``fid\`element plat de pr\'esentation finie''.} on
the category of schemes over $S$. According to our general principle
we first introduce the notion of an fppf-covering.

\begin{definition}
\label{definition-fppf-covering}
Let $T$ be a scheme. An {\it fppf covering of $T$} is a family
of morphisms $\{f_i : T_i \to T\}_{i \in I}$ of schemes
such that each $f_i$ is flat, locally of finite presentation and such
that $T = \bigcup f_i(T_i)$.
\end{definition}

\noindent
Next, we show that this notion satisfies the conditions of
Sites, Definition \ref{sites-definition-site}.

\begin{lemma}
\label{lemma-fppf}
Let $T$ be a scheme.
\begin{enumerate}
\item If $T' \to T$ is an isomorphism then $\{T' \to T\}$
is an fppf covering of $T$.
\item If $\{T_i \to T\}_{i\in I}$ is an fppf covering and for each
$i$ we have an fppf covering $\{T_{ij} \to T_i\}_{j\in J_i}$, then
$\{T_{ij} \to T\}_{i \in I, j\in J_i}$ is an fppf covering.
\item If $\{T_i \to T\}_{i\in I}$ is an fppf covering
and $T' \to T$ is a morphism of schemes then
$\{T' \times_T T_i \to T'\}_{i\in I}$ is an fppf covering.
\end{enumerate}
\end{lemma}

\begin{proof}
The first assertion is clear.
The second follows as the composition of flat morphisms is flat
(see Morphisms, Lemma \ref{morphisms-lemma-composition-flat})
and the composition of morphisms of finite presentation is
of finite presentation
(see Morphisms, Lemma \ref{morphisms-lemma-composition-finite-presentation}).
The third follows as the base change of a flat morphism is flat
(see Morphisms, Lemma \ref{morphisms-lemma-base-change-flat})
and the base change of a morphism of finite presentation is
of finite presentation
(see Morphisms, Lemma \ref{morphisms-lemma-base-change-finite-presentation}).
Moreover, the base change of a surjective family of morphisms is surjective
(proof omitted).
\end{proof}

\begin{lemma}
\label{lemma-fppf-affine}
Let $T$ be an affine scheme.
Let $\{T_i \to T\}_{i \in I}$ be an fppf covering of $T$.
Then there exists an fppf covering
$\{U_j \to T\}_{j = 1, \ldots, m}$ which is a refinement
of $\{T_i \to T\}_{i \in I}$ such that each $U_j$ is an affine
scheme. Moreover, we may choose each $U_j$ to be open affine
in one of the $T_i$.
\end{lemma}

\begin{proof}
This follows directly from the definitions using that a
morphism which is flat and locally of finite presentation is open,
see Morphisms, Lemma \ref{morphisms-lemma-fppf-open}.
\end{proof}

\noindent
Thus we define the corresponding standard coverings of affines as follows.

\begin{definition}
\label{definition-standard-fppf}
Let $T$ be an affine scheme. A {\it standard fppf covering}
of $T$ is a family $\{f_j : U_j \to T\}_{j = 1, \ldots, m}$
with each $U_j$ is affine, flat and of finite presentation over $T$
and $T = \bigcup f_j(U_j)$.
\end{definition}

\begin{definition}
\label{definition-big-fppf-site}
A {\it big fppf site} is any site $\textit{Sch}_{fppf}$ as in
Sites, Definition \ref{sites-definition-site} constructed as follows:
\begin{enumerate}
\item Choose any set of schemes $S_0$, and any set of fppf coverings
$\text{Cov}_0$ among these schemes.
\item As underlying category take any category $\textit{Sch}_\alpha$
constructed as in Sets, Lemma \ref{sets-lemma-construct-category}
starting with the set $S_0$.
\item Choose any set of coverings as in
Sets, Lemma \ref{sets-lemma-coverings-site} starting with the
category $\textit{Sch}_\alpha$ and the class of fppf coverings,
and the set $\text{Cov}_0$ chosen above.
\end{enumerate}
\end{definition}

\noindent
See the remarks following Definition \ref{definition-big-zariski-site}
for motivation and explanation regarding the definition of big sites.

\begin{definition}
\label{definition-big-small-fppf}
Let $S$ be a scheme. Let $\textit{Sch}_{fppf}$ be a big fppf
site containing $S$.
\begin{enumerate}
\item The {\it big fppf site of $S$}, denoted
$(\textit{Sch}/S)_{fppf}$, is the site $\textit{Sch}_{fppf}/S$
introduced in Sites, Section \ref{sites-section-localization}.
\item The {\it big affine fppf site of $S$}, denoted
$(\textit{Aff}/S)_{fppf}$, is the full subcategory of
$(\textit{Sch}/S)_{fppf}$ whose objects are affine $U/S$.
A covering of $(\textit{Aff}/S)_{fppf}$ is any covering
$\{U_i \to U\}$ of $(\textit{Sch}/S)_{fppf}$ which is a
standard fppf covering.
\end{enumerate}
\end{definition}

\noindent
It is not completely clear that
the big affine fppf site is a site. We check this now
and we make sure that the definition above corresponds to
the expected notion.

\begin{lemma}
\label{lemma-verify-site-fppf}
Let $S$ be a scheme. Let $\textit{Sch}_{fppf}$ be a big fppf
site containing $S$.
\begin{enumerate}
\item The category $(\textit{Aff}/S)_{fppf}$ with coverings
as defined in Definition \ref{definition-big-small-fppf} is a site.
\item The functor $(\textit{Aff}/S)_{fppf} \to (\textit{Sch}/S)_{fppf}$
is continuous and cocontinuous and induces an equivalence of topoi
of $\textit{Sh}((\textit{Aff}/S)_{fppf})$ with
$\textit{Sh}((\textit{Sch}/S)_{fppf})$.
\end{enumerate}
\end{lemma}

\begin{proof}
Omitted.
\end{proof}











\section{The fpqc topology}
\label{section-fpqc}

\begin{definition}
\label{definition-fpqc-covering}
Let $T$ be a scheme. An {\it fpqc
covering of $T$} is a family
of morphisms $\{f_i : T_i \to T\}_{i \in I}$ of schemes
such that each $f_i$ is flat and such that for every affine open
$U \subset T$ there exists $n \geq 0$, a map
$a : \{1, \ldots, n\} \to I$ and affine opens
$V_j \subset T_{a(j)}$, $j = 1, \ldots, n$
with $\bigcup_{j = 1}^n f_{a(j)}(V_j) = U$.
\end{definition}

\noindent
The fpqc\footnote{The letters fpqc stand for
``fid\`element plat quasi-compacte''.}
topology cannot be treated in the same way as the fppf topology.
Namely, suppose that $R$ is a nonzero ring. For any faithfully flat
ring map $R \to R'$ the morphism $\text{Spec}(R') \to \text{Spec}(R)$
is an fpqc-covering. We claim that there does not exist a set $A$ of
fpqc-coverings of $\text{Spec}(R)$ such that every fpqc-covering can
be refined by an element of $A$. For example, if $R = k$ is a field,
then for any set $I$ we can consider the purely transcendental field extension
$k \subset k(\{t_i\}_{i \in I})$. We leave it to the reader to show
that there does not exist a set of morphisms of schemes
$\{S_j \to \text{Spec}(k)\}_{j \in J}$ such that every morphism
$\text{Spec}(k(\{t_i\}_{i \in I}))$ is dominated by one of
the schemes $S_j$.

\medskip\noindent
A mildly interesting option is to consider only those faithfully flat ring
extensions $R \to R'$ where the cardinality of $R'$ is suitably bounded.
(And if you consider all schemes in a fixed universe as in SGA4 then you
are bounding the cardinality by a strongly inaccessible cardinal.)
However, it is not so clear what happens if you change the cardinal
to a bigger one.

\medskip\noindent
For these reasons we do not introduce fpqc sites and we will not consider
cohomology with respect to the fpqc-topology.

\medskip\noindent
On the other hand, given a contravariant functor
$F : \textit{Sch}^{opp} \to \textit{Sets}$
it does make sense to ask whether $F$ satisfies the sheaf property
for the fpqc topology. Moreover, we can wonder about descent of object
in the fpqc topology, etc.

\begin{lemma}
\label{lemma-fpqc}
Let $T$ be a scheme.
\begin{enumerate}
\item If $T' \to T$ is an isomorphism then $\{T' \to T\}$
is an fpqc covering of $T$.
\item If $\{T_i \to T\}_{i\in I}$ is an fpqc covering and for each
$i$ we have an fpqc covering $\{T_{ij} \to T_i\}_{j\in J_i}$, then
$\{T_{ij} \to T\}_{i \in I, j\in J_i}$ is an fpqc covering.
\item If $\{T_i \to T\}_{i\in I}$ is an fppf covering
and $T' \to T$ is a morphism of schemes then
$\{T' \times_T T_i \to T'\}_{i\in I}$ is an fpqc covering.
\end{enumerate}
\end{lemma}

\begin{proof}
Omitted.
\end{proof}

\begin{lemma}
\label{lemma-fpqc-affine}
Let $T$ be an affine scheme.
Let $\{T_i \to T\}_{i \in I}$ be an fpqc covering of $T$.
Then there exists an fpqc covering
$\{U_j \to T\}_{j = 1, \ldots, n}$ which is a refinement
of $\{T_i \to T\}_{i \in I}$ such that each $U_j$ is an affine
scheme. Moreover, we may choose each $U_j$ to be open affine
in one of the $T_i$.
\end{lemma}

\begin{proof}
This follows directly from the definition.
\end{proof}

\begin{definition}
\label{definition-standard-fpqc}
Let $T$ be an affine scheme. A {\it standard fpqc covering}
of $T$ is a family $\{f_j : U_j \to T\}_{j = 1, \ldots, n}$
with each $U_j$ is affine, flat over $T$ and $T = \bigcup f_j(U_j)$.
\end{definition}




























\section{Change of big sites}
\label{section-change-alpha}

\noindent
In this section we explain what happens on changing the big
Zariski/fppf/etale sites.

\medskip\noindent
Given two big Zariski sites $\textit{Sch}_{Zar}$ and
$\textit{Sch}_{Zar}'$ we say that {\it $\textit{Sch}_{Zar}$ is contained in
$\textit{Sch}_{Zar}'$} if
$\text{Ob}(\textit{Sch}_{Zar}) \subset \text{Ob}(\textit{Sch}_{Zar}')$
and
$\text{Cov}(\textit{Sch}_{Zar}) \subset \text{Cov}(\textit{Sch}_{Zar}')$.
The same definition applies to big fppf and etale sites. It also makes
sense to ask whether a given big Zariski site is contained in a
given big etale site and whether a given big etale site is contained in a
given big fppf site (but of course no fppf site can be contained in an
etale site with these defintions).

\begin{lemma}
\label{lemma-contained-in}
Any set of big Zariski sites is contained in a common big Zariski site.
The same is true, mutatis mutandis, for big fppf and big etale sites.
\end{lemma}

\begin{proof}
This is true because the union of a set of sets is a set, and the
constructions in the chapter on sets.
\end{proof}

\begin{lemma}
\label{lemma-change-alpha-Zariski}
Suppose given big Zariski sites $\textit{Sch}_{Zar}$ and
$\textit{Sch}_{Zar}'$. Assume that $\textit{Sch}_{Zar}$
is contained in $\textit{Sch}_{Zar}'$. There are morphisms of
topoi
\begin{eqnarray*}
g : \textit{Sh}(\textit{Sch}_{Zar}) &
\longrightarrow &
\textit{Sh}(\textit{Sch}_{Zar}') \\
f : \textit{Sh}(\textit{Sch}_{Zar}') &
\longrightarrow &
\textit{Sh}(\textit{Sch}_{Zar})
\end{eqnarray*}
such that $f \circ g \cong \text{id}$.
\end{lemma}

\begin{proof}
Follows from Sites, Lemma \ref{sites-lemma-bigger-site} and the properties of
the big Zariski sites.
\end{proof}




\section{Other chapters}

\begin{multicols}{2}
\begin{enumerate}
\item \hyperref[introduction-section-phantom]{Introduction}
\item \hyperref[conventions-section-phantom]{Conventions}
\item \hyperref[sets-section-phantom]{Set Theory}
\item \hyperref[categories-section-phantom]{Categories}
\item \hyperref[topology-section-phantom]{Topology}
\item \hyperref[sheaves-section-phantom]{Sheaves on Spaces}
\item \hyperref[algebra-section-phantom]{Commutative Algebra}
\item \hyperref[sites-section-phantom]{Sites and Sheaves}
\item \hyperref[homology-section-phantom]{Homological Algebra}
\item \hyperref[derived-section-phantom]{Derived Categories}
\item \hyperref[more-algebra-section-phantom]{More Algebra}
\item \hyperref[simplicial-section-phantom]{Simplicial Methods}
\item \hyperref[modules-section-phantom]{Sheaves of Modules}
\item \hyperref[sites-modules-section-phantom]{Modules on Sites}
\item \hyperref[injectives-section-phantom]{Injectives}
\item \hyperref[cohomology-section-phantom]{Cohomology of Sheaves}
\item \hyperref[sites-cohomology-section-phantom]{Cohomology on Sites}
\item \hyperref[hypercovering-section-phantom]{Hypercoverings}
\item \hyperref[schemes-section-phantom]{Schemes}
\item \hyperref[constructions-section-phantom]{Constructions of Schemes}
\item \hyperref[properties-section-phantom]{Properties of Schemes}
\item \hyperref[morphisms-section-phantom]{Morphisms of Schemes}
\item \hyperref[coherent-section-phantom]{Coherent Cohomology}
\item \hyperref[divisors-section-phantom]{Divisors}
\item \hyperref[limits-section-phantom]{Limits of Schemes}
\item \hyperref[varieties-section-phantom]{Varieties}
\item \hyperref[chow-section-phantom]{Chow Homology}
\item \hyperref[topologies-section-phantom]{Topologies on Schemes}
\item \hyperref[descent-section-phantom]{Descent}
\item \hyperref[more-morphisms-section-phantom]{More on Morphisms}
\item \hyperref[flat-section-phantom]{More on Flatness}
\item \hyperref[groupoids-section-phantom]{Groupoid Schemes}
\item \hyperref[more-groupoids-section-phantom]{More on Groupoid Schemes}
\item \hyperref[etale-section-phantom]{\'Etale Morphisms of Schemes}
\item \hyperref[etale-cohomology-section-phantom]{\'Etale Cohomology}
\item \hyperref[spaces-section-phantom]{Algebraic Spaces}
\item \hyperref[spaces-properties-section-phantom]{Properties of Algebraic Spaces}
\item \hyperref[spaces-morphisms-section-phantom]{Morphisms of Algebraic Spaces}
\item \hyperref[spaces-topologies-section-phantom]{Topologies on Algebraic Spaces}
\item \hyperref[spaces-descent-section-phantom]{Descent and Algebraic Spaces}
\item \hyperref[spaces-more-morphisms-section-phantom]{More on Morphisms of Spaces}
\item \hyperref[quot-section-phantom]{Quot and Hilbert Spaces}
\item \hyperref[stacks-section-phantom]{Stacks}
\item \hyperref[spaces-groupoids-section-phantom]{Groupoids in Algebraic Spaces}
\item \hyperref[spaces-more-groupoids-section-phantom]{More on Groupoids in Spaces}
\item \hyperref[bootstrap-section-phantom]{Bootstrap}
\item \hyperref[examples-stacks-section-phantom]{Examples of Stacks}
\item \hyperref[groupoids-quotients-section-phantom]{Quotients of Groupoids}
\item \hyperref[algebraic-section-phantom]{Algebraic Stacks}
\item \hyperref[criteria-section-phantom]{Criteria for Representability}
\item \hyperref[stacks-properties-section-phantom]{Properties of Algebraic Stacks}
\item \hyperref[stacks-morphisms-section-phantom]{Morphisms of Algebraic Stacks}
\item \hyperref[examples-section-phantom]{Examples}
\item \hyperref[exercises-section-phantom]{Exercises}
\item \hyperref[guide-section-phantom]{Guide to Literature}
\item \hyperref[desirables-section-phantom]{Desirables}
\item \hyperref[coding-section-phantom]{Coding Style}
\item \hyperref[fdl-section-phantom]{GNU Free Documentation License}
\item \hyperref[index-section-phantom]{Auto Generated Index}
\end{enumerate}
\end{multicols}



\bibliography{my}
\bibliographystyle{alpha}

\end{document}
