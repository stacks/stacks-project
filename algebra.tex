\IfFileExists{stacks-project.cls}{%
\documentclass{stacks-project}
}{%
\documentclass{amsart}
}

% The following AMS packages are automatically loaded with
% the amsart documentclass:
%\usepackage{amsmath}
%\usepackage{amssymb}
%\usepackage{amsthm}

% For dealing with references we use the comment environment
\usepackage{verbatim}
\newenvironment{reference}{\comment}{\endcomment}
%\newenvironment{reference}{}{}
\newenvironment{slogan}{\comment}{\endcomment}
\newenvironment{history}{\comment}{\endcomment}

% For commutative diagrams you can use
% \usepackage{amscd}
\usepackage[all]{xy}

% We use 2cell for 2-commutative diagrams.
\xyoption{2cell}
\UseAllTwocells

% To put source file link in headers.
% Change "template.tex" to "this_filename.tex"
% \usepackage{fancyhdr}
% \pagestyle{fancy}
% \lhead{}
% \chead{}
% \rhead{Source file: \url{template.tex}}
% \lfoot{}
% \cfoot{\thepage}
% \rfoot{}
% \renewcommand{\headrulewidth}{0pt}
% \renewcommand{\footrulewidth}{0pt}
% \renewcommand{\headheight}{12pt}

\usepackage{multicol}

% For cross-file-references
\usepackage{xr-hyper}

% Package for hypertext links:
\usepackage{hyperref}

% For any local file, say "hello.tex" you want to link to please
% use \externaldocument[hello-]{hello}
\externaldocument[introduction-]{introduction}
\externaldocument[conventions-]{conventions}
\externaldocument[sets-]{sets}
\externaldocument[categories-]{categories}
\externaldocument[topology-]{topology}
\externaldocument[sheaves-]{sheaves}
\externaldocument[sites-]{sites}
\externaldocument[stacks-]{stacks}
\externaldocument[fields-]{fields}
\externaldocument[algebra-]{algebra}
\externaldocument[brauer-]{brauer}
\externaldocument[homology-]{homology}
\externaldocument[derived-]{derived}
\externaldocument[simplicial-]{simplicial}
\externaldocument[more-algebra-]{more-algebra}
\externaldocument[smoothing-]{smoothing}
\externaldocument[modules-]{modules}
\externaldocument[sites-modules-]{sites-modules}
\externaldocument[injectives-]{injectives}
\externaldocument[cohomology-]{cohomology}
\externaldocument[sites-cohomology-]{sites-cohomology}
\externaldocument[dga-]{dga}
\externaldocument[dpa-]{dpa}
\externaldocument[hypercovering-]{hypercovering}
\externaldocument[schemes-]{schemes}
\externaldocument[constructions-]{constructions}
\externaldocument[properties-]{properties}
\externaldocument[morphisms-]{morphisms}
\externaldocument[coherent-]{coherent}
\externaldocument[divisors-]{divisors}
\externaldocument[limits-]{limits}
\externaldocument[varieties-]{varieties}
\externaldocument[topologies-]{topologies}
\externaldocument[descent-]{descent}
\externaldocument[perfect-]{perfect}
\externaldocument[more-morphisms-]{more-morphisms}
\externaldocument[flat-]{flat}
\externaldocument[groupoids-]{groupoids}
\externaldocument[more-groupoids-]{more-groupoids}
\externaldocument[etale-]{etale}
\externaldocument[chow-]{chow}
\externaldocument[intersection-]{intersection}
\externaldocument[pic-]{pic}
\externaldocument[adequate-]{adequate}
\externaldocument[dualizing-]{dualizing}
\externaldocument[duality-]{duality}
\externaldocument[discriminant-]{discriminant}
\externaldocument[local-cohomology-]{local-cohomology}
\externaldocument[curves-]{curves}
\externaldocument[resolve-]{resolve}
\externaldocument[models-]{models}
\externaldocument[pione-]{pione}
\externaldocument[etale-cohomology-]{etale-cohomology}
\externaldocument[proetale-]{proetale}
\externaldocument[crystalline-]{crystalline}
\externaldocument[spaces-]{spaces}
\externaldocument[spaces-properties-]{spaces-properties}
\externaldocument[spaces-morphisms-]{spaces-morphisms}
\externaldocument[decent-spaces-]{decent-spaces}
\externaldocument[spaces-cohomology-]{spaces-cohomology}
\externaldocument[spaces-limits-]{spaces-limits}
\externaldocument[spaces-divisors-]{spaces-divisors}
\externaldocument[spaces-over-fields-]{spaces-over-fields}
\externaldocument[spaces-topologies-]{spaces-topologies}
\externaldocument[spaces-descent-]{spaces-descent}
\externaldocument[spaces-perfect-]{spaces-perfect}
\externaldocument[spaces-more-morphisms-]{spaces-more-morphisms}
\externaldocument[spaces-flat-]{spaces-flat}
\externaldocument[spaces-groupoids-]{spaces-groupoids}
\externaldocument[spaces-more-groupoids-]{spaces-more-groupoids}
\externaldocument[bootstrap-]{bootstrap}
\externaldocument[spaces-pushouts-]{spaces-pushouts}
\externaldocument[groupoids-quotients-]{groupoids-quotients}
\externaldocument[spaces-more-cohomology-]{spaces-more-cohomology}
\externaldocument[spaces-simplicial-]{spaces-simplicial}
\externaldocument[formal-spaces-]{formal-spaces}
\externaldocument[restricted-]{restricted}
\externaldocument[spaces-resolve-]{spaces-resolve}
\externaldocument[formal-defos-]{formal-defos}
\externaldocument[defos-]{defos}
\externaldocument[cotangent-]{cotangent}
\externaldocument[examples-defos-]{examples-defos}
\externaldocument[algebraic-]{algebraic}
\externaldocument[examples-stacks-]{examples-stacks}
\externaldocument[stacks-sheaves-]{stacks-sheaves}
\externaldocument[criteria-]{criteria}
\externaldocument[artin-]{artin}
\externaldocument[quot-]{quot}
\externaldocument[stacks-properties-]{stacks-properties}
\externaldocument[stacks-morphisms-]{stacks-morphisms}
\externaldocument[stacks-limits-]{stacks-limits}
\externaldocument[stacks-cohomology-]{stacks-cohomology}
\externaldocument[stacks-perfect-]{stacks-perfect}
\externaldocument[stacks-introduction-]{stacks-introduction}
\externaldocument[stacks-more-morphisms-]{stacks-more-morphisms}
\externaldocument[stacks-geometry-]{stacks-geometry}
\externaldocument[moduli-]{moduli}
\externaldocument[moduli-curves-]{moduli-curves}
\externaldocument[examples-]{examples}
\externaldocument[exercises-]{exercises}
\externaldocument[guide-]{guide}
\externaldocument[desirables-]{desirables}
\externaldocument[coding-]{coding}
\externaldocument[obsolete-]{obsolete}
\externaldocument[fdl-]{fdl}
\externaldocument[index-]{index}

% Theorem environments.
%
\theoremstyle{plain}
\newtheorem{theorem}[subsection]{Theorem}
\newtheorem{proposition}[subsection]{Proposition}
\newtheorem{lemma}[subsection]{Lemma}

\theoremstyle{definition}
\newtheorem{definition}[subsection]{Definition}
\newtheorem{example}[subsection]{Example}
\newtheorem{exercise}[subsection]{Exercise}
\newtheorem{situation}[subsection]{Situation}

\theoremstyle{remark}
\newtheorem{remark}[subsection]{Remark}
\newtheorem{remarks}[subsection]{Remarks}

\numberwithin{equation}{subsection}

% Macros
%
\def\lim{\mathop{\rm lim}\nolimits}
\def\colim{\mathop{\rm colim}\nolimits}
\def\Spec{\mathop{\rm Spec}}
\def\Hom{\mathop{\rm Hom}\nolimits}
\def\Ext{\mathop{\rm Ext}\nolimits}
\def\SheafHom{\mathop{\mathcal{H}\!{\it om}}\nolimits}
\def\SheafExt{\mathop{\mathcal{E}\!{\it xt}}\nolimits}
\def\Sch{\textit{Sch}}
\def\Mor{\mathop{\rm Mor}\nolimits}
\def\Ob{\mathop{\rm Ob}\nolimits}
\def\Sh{\mathop{\textit{Sh}}\nolimits}
\def\NL{\mathop{N\!L}\nolimits}
\def\proetale{{pro\text{-}\acute{e}tale}}
\def\etale{{\acute{e}tale}}
\def\QCoh{\textit{QCoh}}
\def\Ker{\mathop{\rm Ker}}
\def\Im{\mathop{\rm Im}}
\def\Coker{\mathop{\rm Coker}}
\def\Coim{\mathop{\rm Coim}}

%
% Macros for moduli stacks/spaces
%
\def\QCohstack{\mathcal{QC}\!{\it oh}}
\def\Cohstack{\mathcal{C}\!{\it oh}}
\def\Spacesstack{\mathcal{S}\!{\it paces}}
\def\Quotfunctor{{\rm Quot}}
\def\Hilbfunctor{{\rm Hilb}}
\def\Curvesstack{\mathcal{C}\!{\it urves}}
\def\Polarizedstack{\mathcal{P}\!{\it olarized}}
\def\Complexesstack{\mathcal{C}\!{\it omplexes}}
% \Pic is the operator that assigns to X its picard group, usage \Pic(X)
% \Picardstack_{X/B} denotes the Picard stack of X over B
% \Picardfunctor_{X/B} denotes the Picard functor of X over B
\def\Pic{\mathop{\rm Pic}\nolimits}
\def\Picardstack{\mathcal{P}\!{\it ic}}
\def\Picardfunctor{{\rm Pic}}
\def\Deformationcategory{\mathcal{D}\!{\it ef}}


% OK, start here.
%
\begin{document}

\title{Commutative Algebra}

%\begin{abstract}
%\end{abstract}

\maketitle

\tableofcontents

\section{Introduction}
\label{section-introduction}

\noindent
Basic commutative algebra will be explained in this document.
A reference is \cite{MatCA}.

\section{Conventions}
\label{section-conventions}

\noindent
A ring is commutative with $1$. The zero ring is a ring. In fact it is
the only ring that does not have a prime ideal.

\section{Basic notions}
\label{section-rings-basic}

\noindent
The following notions are considered basic and will not be defined,
and or proved. This does not mean they are all necessarily easy or 
well known.

\begin{enumerate}
\item $R$ is a {\it ring},
\label{ring}
\item $x\in R$ is {\it nilpotent},
\label{ring-element-nilpotent}
\item $x\in R$ is a {\it zero-divisor},
\label{ring-element-zerodivisor}
\item $x\in R$ is a {\it unit},
\label{ring-element-unit}
\item $\varphi : R_1 \to R_2$ is a {\it ring homomorphism},
\label{ring-homomorphism}
\item $R$ is a {\it (integral) domain},
\label{ring-domain}
\item $R$ is {\it reduced},
\label{ring-reduced}
\item $R$ is {\it Noetherian},
\label{ring-Noetherian}
\item $K$ is a {\it field},
\label{field}
\item $I \subset R$ is an {\it ideal},
\label{ideal}
\item if $I$ is an ideal then we have its {\it radical} $\sqrt{I}$,
\label{radical-ideal}
\item $\mathfrak p \subset R$ is a {\it prime ideal},
\label{prime-ideal}
\item $\mathfrak m \subset R$ is a {\it maximal ideal},
\label{maximal-ideal}
\item any nonzero ring has a maximal ideal,
\label{exists-maximal-ideal}
\item the ideal $(T)$ {\it generated} by a subset $T \subset R$,
\label{ideal-generated-by}
\item the {\it quotient ring} $R/I$,
\label{quotient-ring}
\item if $\varphi : R_1 \to R_2$ is a ring homomorphism, and if
$I \subset R_2$ is an ideal, then $\varphi^{-1}(I)$ is an
ideal of $R_1$,
\label{inverse-image-ideal}
\item if $\varphi : R_1 \to R_2$ is a ring homomorphism, and if
$I \subset R_1$ is an ideal, then $\varphi(I) \cdot R_2$ (sometimes
denoted $I \cdot R_2$, or $IR_2$) is the ideal of $R_2$ generated
by $\varphi(I)$,
\label{image-ideal}
\item if $\varphi : R_1 \to R_2$ is a ring homomorphism, and if
$\mathfrak p \subset R_2$ is a prime ideal, then
$\varphi^{-1}(\mathfrak p)$ is a prime ideal of $R_1$,
\label{inverse-image-prime}
\item $M$ is an {\it $R$-module},
\label{module}
\item $N \subset M$ is an {\it $R$-submodule},
\label{submodule}
\item $M$ is an {\it Noetherian $R$-module},
\label{Noetherian-module}
\item $M$ is a {\it finite $R$-module},
\label{finite-module}
\item $M$ is a {\it finitely generated $R$-module},
\label{finitely-generated-module}
\item $M$ is a {\it finitely presented $R$-module},
\label{finitely-presented-module}
\item $M$ is a {\it free $R$-module},
\label{free-module}
\item if $N \subset M \subset L$ are $R$-modules,
then $L/M = (L/N)/(M/N)$,
\label{isomorphism-theorem}
\item $S$ is a {\it multiplicative subset of $R$},
\label{multiplicative-subset}
\item the {\it localization} $R \to S^{-1}R$ of $R$,
\label{localization-ring}
\item if $R$ is a ring and $S$ is a multiplicative subset
of $R$ then $S^{-1}R$ is the zero ring if and only if $S$ contains
$0$,
\label{localization-zero}
\item if $R$ is a ring and if the multiplicative subset $S$
consists completely of nonzero divisors, then $R \to S^{-1}R$
is injective,
\label{localize-nonzerodivisors}
\item if $\varphi : R_1 \to R_2$ is a ring homomorphism, and
$S$ is a multiplicative subsets of $R_1$, then $\varphi(S)$ is
a multiplicative subset of $R_2$,
\item if $S$, $S'$ are multiplicative subsets of $R$,
and if $SS'$ denotes the set of products $SS' =
\{r \in R \mid \exists s\in S, \exists s' \in S', r = ss'\}$
then $SS'$ is a multiplicative subset of $R$,
\label{products-multiplicative-subsets}
\item if $S$, $S'$ are multiplicative subsets of $R$,
and if $\overline{S}$ denotes the image of $S$ in $(S')^{-1}R$,
then $(SS')^{-1}R = \overline{S}^{-1}((S')^{-1}R)$,
\label{localization-localization}
\item the {\it localization} $S^{-1}M$ of the $R$-module $M$,
\label{localization-module}
\item the functor $M \mapsto S^{-1}M$ preserves injective maps,
surjective maps, and exactness,
\label{localization-exact}
\item if $S$, $S'$ are multiplicative subsets of $R$,
and $M$ and $R$-module, then $(SS')^{-1}M =
S^{-1}((S')^{-1}M)$,
\label{localization-localization-module}
\item if $R$ is a ring, $I$ and ideal of $R$ and $S$ a multiplicative
subset of $R$, then $S^{-1}I$ is an ideal of $S^{-1}R$, and we have
$S^{-1}R/S^{-1}I = \overline{S}^{-1}(R/I)$, where $\overline{S}$
is the image of $S$ in $R/I$,
\label{localize-ideal}
\item if $R$ is a ring, and $S$ a multiplicative
subset of $R$, then any ideal $I'$ of $S^{-1}R$ is
of the form $S^{-1}I$, where one can take $I$ to be
the inverse image of $I'$ in $R$,
\label{ideal-in-localization}
\item if $R$ is a ring, $M$ an $R$-module, and $S$ a multiplicative
subset of $R$, then any submodule $N'$ of $S^{-1}M$ is of the form
$S^{-1}N$ for some submodule $N \subset M$, where
one can take $N$ to be the inverse image of $N'$ in $M$,
\label{submodule-in-localization}
\item if $S = \{1, f, f^2,\ldots\}$ then $R_f = S^{-1}R$, and
$M_f = S^{-1}M$,
\label{localiza-f}
\item if $S = R \setminus \mathfrak p$, then $R_{\mathfrak p} = S^{-1}R$
and $M_{\mathfrak p} = S^{-1}M$,
\label{localize-p}
\item etc.
\end{enumerate}

\section{The spectrum of a ring}
\label{section-spectrum-ring}

\noindent
We arbitrarily decide that the spectrum of a ring as a topological space
is part of the algebra chapter, whereas an affine scheme is part of the
chapter on schemes.

\begin{definition}
\label{definition-spectrum-ring}
Let $R$ be a ring.
\begin{enumerate}
\item The {\it spectrum} of $R$ is the set of prime ideals of $R$.
It is usually denoted $\text{Spec}(R)$.
\item Given a subset $T \subset R$ we let $V(T) \subset \text{Spec}(R)$
be the set of primes containing $T$, i.e., $V(T) = \{ \mathfrak p \in
\text{Spec}(R) \mid \forall f\in T, f\in \mathfrak p\}$.
\item Given an element $f \in R$ we let $D(f) \subset \text{Spec}(R)$
be the set of primes not containing $f$.
\end{enumerate}
\end{definition}

\begin{lemma}
\label{lemma-Zariski-topology}
Let $R$ be a ring.
\begin{enumerate}
\item The spectrum of a ring $R$ is empty if and only if $R$
is the zero ring.
\item If $T \subset R$, and if $(T)$ is the ideal generated by
$T$ in $R$, then $V((T)) = V(T)$.
\item If $I$ is an ideal and $\sqrt{I}$ is its radical,
see basic notion (\ref{radical-ideal}), then $V(I) = V(\sqrt{I})$.
\item If $I$ is an ideal then $V(I) = \emptyset$ if and only
if $I$ is the unit ideal.
\item If $I$, $J$ are ideals of $R$ then $V(I) \cup V(J) =
V(I \cap J)$.
\item If $(I_a)_{a\in A}$ is a set of ideals of $R$ then
$\cap_{a\in A} V(I_a) = V(\cup_{a\in A} I_a)$.
\item If $f \in R$, then $D(f) \sqcup V(f) = \text{Spec}(R)$.
\item If $f = u f'$ for some unit $u \in R$, then $D(f) = D(f')$.
\item If $I \subset R$ is an ideal, and $\mathfrak p$ is a prime of
$R$ with $\mathfrak p \not\in V(I)$, then there exists an $f \in R$
such that $\mathfrak p \in D(f)$, and $D(f) \cap V(I) = \emptyset$.
\item If $f,g \in R$, then $D(fg) = D(f) \cap D(g)$.
\end{enumerate}
\end{lemma}

\begin{proof}
FIXME.
\end{proof}

\noindent
The lemma implies that the subsets $V(T)$ from
Definition \ref{definition-spectrum-ring} form the closed
subsets of a topology on $\text{Spec}(R)$. And it also shows that
the sets $D(f)$ are open and form a basis for this
topology.

\begin{definition}
\label{definition-Zariski-topology}
Let $R$ be a ring.
The topology on $\text{Spec}(R)$ whose closed sets are the
sets $V(T)$ is called the {\it Zariski} topology. The open
subsets $D(f)$ are called the {\it standard opens} of $\text{Spec}(R)$.
\end{definition}

\noindent
It should be clear from context whether we consider $\text{Spec}(R)$
just as a set or as a topological space.

\begin{lemma}
\label{lemma-spec-functorial}
Suppose that $\varphi : R \to R'$ is a ring homomorphism.
The induced map
$$
\text{Spec}(\varphi) :
\text{Spec}(R')
\longrightarrow
\text{Spec}(R),\ \ 
\mathfrak p'
\longmapsto
\varphi^{-1}(\mathfrak p')
$$
is continuous for the Zariski toplogies. In fact, for
$f \in R$ we have
$\text{Spec}(\varphi)^{-1}(D(f)) = D(\varphi(f))$.
\end{lemma}

\begin{proof}
It is basic notion (\ref{inverse-image-prime}) that
$\mathfrak p := \varphi^{-1}(\mathfrak p')$
is indeed a prime ideal of $R$. The last assertion
of the lemma follows directly from the definitions,
and implies the first.
\end{proof}

\noindent
If $\varphi' : R' \to R''$ is a second ring homomorphism
then the composition
$$
\text{Spec}(R')
\longrightarrow
\text{Spec}(R')
\longrightarrow
\text{Spec}(R'')
$$
equals $\text{Spec}(\varphi' \circ \varphi)$. In other
words, $\text{Spec}$ is a contravariant functor from the
category of rings to the category of topological spaces.

\begin{lemma}
\label{lemma-spec-localization}
Let $R$ be a ring. Let $S \subset R$ be a multiplicative subset.
The map $R \to S^{-1}R$ induces via the functoriality of $\text{Spec}$
a homeomorphism 
$$
\text{Spec}(S^{-1}R)
\longrightarrow 
\{\mathfrak p \in \text{Spec}(R) \mid S \cap \mathfrak p = \emptyset \}
$$
where the topology on the right hand side is that induced from the
Zariski topology on $\text{Spec}(R)$. The inverse map is given
by $\mathfrak p \mapsto S^{-1}\mathfrak p$.
\end{lemma}

\begin{proof}
Denote the left hand side of the arrow of the lemma by $D$.
Choose a prime $\mathfrak p' \subset R_f$ and let $\mathfrak p$
the inverse image of $\mathfrak p'$ in $R$. Since $\mathfrak p'$
does not contain $1$ we see that $\mathfrak p$ does not contain
any element of $S$. Hence $\mathfrak p \in D$ and we see that
the image is contained in $D$. Let $\mathfrak p \in D$. By assumption
the set $S$ maps injectively into $R/\mathfrak p$, in other
words the image $\overline{S}$ does not contain $0$.
By basic notion (\ref{localization-zero})
$\overline{S}^{-1}(R/\mathfrak p)$ is not the zero ring.
By basic notion (\ref{localize-ideal}) we see
$S^{-1}R / S^{-1}\mathfrak p = \overline{S}^{-1}(R/\mathfrak p)$
is a domain, and hence $S^{-1}\mathfrak p$ is a prime.
The equality of rings also shows that the inverse image of
$S^{-1}\mathfrak p$ in $R$ is equal to $\mathfrak p$,
because $R/\mathfrak p \to \overline{S}^{-1}(R/\mathfrak p)$
is injective by basic notion (\ref{localize-nonzerodivisors}).
This proves that the map $\text{Spec}(S^{-1}R) \to \text{Spec}(R)$
is bijective onto $D$ with inverse as given.
It is continuous by Lemma \ref{lemma-spec-functorial}.
Finally, let $D(g) \subset \text{Spec}(S^{-1}R)$ be a standard
open. Write $g = h/s$ for some $h\in R$ and $s\in S$.
Since $g$ and $h/1$ differ by a unit we have $D(g) = 
D(h/1)$ in $\text{Spec}(S^{-1}R)$.
Hence by Lemma \ref{lemma-spec-functorial} and the bijectivity
above the image of $D(g) = D(h/1)$ is $D \cap D(h)$.
This proves the map is open as well.
\end{proof}

\begin{lemma}
\label{lemma-standard-open}
Let $R$ be a ring. Let $f \in R$.
The map $R \to R_f$ induces via the functoriality of
$\text{Spec}$ a homeomorphism
$$
\text{Spec}(R_f) \longrightarrow D(f) \subset \text{Spec}(R).
$$
The inverse is given by $\mathfrak p \mapsto \mathfrak p \cdot R_f$.
\end{lemma}

\begin{proof}
This is a special case of Lemma \ref{lemma-spec-localization}
above.
\end{proof}

\begin{lemma}
\label{lemma-spec-closed}
Let $R$ be a ring. Let $I \subset R$ be an ideal.
The map $R \to R/I$ induces via the functoriality of
$\text{Spec}$ a homeorphism
$$
\text{Spec}(R/I) \longrightarrow V(I) \subset \text{Spec}(R).
$$
The inverse is given by $\mathfrak p \mapsto \mathfrak p / I$.
\end{lemma}

\begin{proof}
It is immediate that the image is contained in $V(I)$.
On the other hand, if $\mathfrak p \in V(I)$
then $\mathfrak p \supset I$ and we may consider
the ideal $\mathfrak p /I \subset R/I$. Using
basic notion (\ref{isomorphism-theorem}) we see that
$(R/I)/(\mathfrak p/I) = R/\mathfrak p$ is a domain
and hence $\mathfrak p/I$ is a prime ideal. From this
it is immediately clear that the image of $D(f + I)$
is $D(f) \cap V(I)$, and hence the map is a homeomorphism.
\end{proof}

\begin{lemma}
\label{lemma-quasicompact}
Let $R$ be a ring.
The topology on $\text{Spec}(R)$ is quasicompact.
\end{lemma}

\begin{proof}
It suffices to prove that any covering of $\text{Spec}(R)$
by standard opens can be refined by a finite covering.
Thus suppose that $\text{Spec}(R) = \cup D(f_i)$
for a set of elements $f_i$ of $R$. This means that
$\cap V(f_i) = \emptyset$. According to Lemma
\ref{lemma-Zariski-topology} this means that
$V(\{f_i \}) = \emptyset$. According to the
same lemma this means that the ideal generated
by the $f_i$ is the unit ideal of $R$. This means
that we can write $1$ as a {\it finite} sum like so
$1 = \sum_{i \in finite\ list} r_i f_i$.
And then it follows that $\text{Spec}(R) 
= \cup_{i \in finite\ list} D(f_i)$.
\end{proof}


\section{Other chapters}

\begin{multicols}{2}
\begin{enumerate}
\item \hyperref[introduction-section-phantom]{Introduction}
\item \hyperref[conventions-section-phantom]{Conventions}
\item \hyperref[sets-section-phantom]{Set Theory}
\item \hyperref[categories-section-phantom]{Categories}
\item \hyperref[topology-section-phantom]{Topology}
\item \hyperref[sheaves-section-phantom]{Sheaves on Spaces}
\item \hyperref[algebra-section-phantom]{Commutative Algebra}
\item \hyperref[sites-section-phantom]{Sites and Sheaves}
\item \hyperref[homology-section-phantom]{Homological Algebra}
\item \hyperref[derived-section-phantom]{Derived Categories}
\item \hyperref[more-algebra-section-phantom]{More Algebra}
\item \hyperref[simplicial-section-phantom]{Simplicial Methods}
\item \hyperref[modules-section-phantom]{Sheaves of Modules}
\item \hyperref[sites-modules-section-phantom]{Modules on Sites}
\item \hyperref[injectives-section-phantom]{Injectives}
\item \hyperref[cohomology-section-phantom]{Cohomology of Sheaves}
\item \hyperref[sites-cohomology-section-phantom]{Cohomology on Sites}
\item \hyperref[hypercovering-section-phantom]{Hypercoverings}
\item \hyperref[schemes-section-phantom]{Schemes}
\item \hyperref[constructions-section-phantom]{Constructions of Schemes}
\item \hyperref[properties-section-phantom]{Properties of Schemes}
\item \hyperref[morphisms-section-phantom]{Morphisms of Schemes}
\item \hyperref[coherent-section-phantom]{Coherent Cohomology}
\item \hyperref[divisors-section-phantom]{Divisors}
\item \hyperref[limits-section-phantom]{Limits of Schemes}
\item \hyperref[varieties-section-phantom]{Varieties}
\item \hyperref[chow-section-phantom]{Chow Homology}
\item \hyperref[topologies-section-phantom]{Topologies on Schemes}
\item \hyperref[descent-section-phantom]{Descent}
\item \hyperref[more-morphisms-section-phantom]{More on Morphisms}
\item \hyperref[flat-section-phantom]{More on Flatness}
\item \hyperref[groupoids-section-phantom]{Groupoid Schemes}
\item \hyperref[more-groupoids-section-phantom]{More on Groupoid Schemes}
\item \hyperref[etale-section-phantom]{\'Etale Morphisms of Schemes}
\item \hyperref[etale-cohomology-section-phantom]{\'Etale Cohomology}
\item \hyperref[spaces-section-phantom]{Algebraic Spaces}
\item \hyperref[spaces-properties-section-phantom]{Properties of Algebraic Spaces}
\item \hyperref[spaces-morphisms-section-phantom]{Morphisms of Algebraic Spaces}
\item \hyperref[spaces-topologies-section-phantom]{Topologies on Algebraic Spaces}
\item \hyperref[spaces-descent-section-phantom]{Descent and Algebraic Spaces}
\item \hyperref[spaces-more-morphisms-section-phantom]{More on Morphisms of Spaces}
\item \hyperref[quot-section-phantom]{Quot and Hilbert Spaces}
\item \hyperref[stacks-section-phantom]{Stacks}
\item \hyperref[spaces-groupoids-section-phantom]{Groupoids in Algebraic Spaces}
\item \hyperref[spaces-more-groupoids-section-phantom]{More on Groupoids in Spaces}
\item \hyperref[bootstrap-section-phantom]{Bootstrap}
\item \hyperref[examples-stacks-section-phantom]{Examples of Stacks}
\item \hyperref[groupoids-quotients-section-phantom]{Quotients of Groupoids}
\item \hyperref[algebraic-section-phantom]{Algebraic Stacks}
\item \hyperref[criteria-section-phantom]{Criteria for Representability}
\item \hyperref[stacks-properties-section-phantom]{Properties of Algebraic Stacks}
\item \hyperref[stacks-morphisms-section-phantom]{Morphisms of Algebraic Stacks}
\item \hyperref[examples-section-phantom]{Examples}
\item \hyperref[exercises-section-phantom]{Exercises}
\item \hyperref[guide-section-phantom]{Guide to Literature}
\item \hyperref[desirables-section-phantom]{Desirables}
\item \hyperref[coding-section-phantom]{Coding Style}
\item \hyperref[fdl-section-phantom]{GNU Free Documentation License}
\item \hyperref[index-section-phantom]{Auto Generated Index}
\end{enumerate}
\end{multicols}


\bibliography{my}
\bibliographystyle{alpha}

\end{document}
