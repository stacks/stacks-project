\IfFileExists{stacks-project.cls}{%
\documentclass{stacks-project}
}{%
\documentclass{amsart}
}

% The following AMS packages are automatically loaded with
% the amsart documentclass:
%\usepackage{amsmath}
%\usepackage{amssymb}
%\usepackage{amsthm}

% For dealing with references we use the comment environment
\usepackage{verbatim}
\newenvironment{reference}{\comment}{\endcomment}
%\newenvironment{reference}{}{}
\newenvironment{slogan}{\comment}{\endcomment}
\newenvironment{history}{\comment}{\endcomment}

% For commutative diagrams you can use
% \usepackage{amscd}
\usepackage[all]{xy}

% We use 2cell for 2-commutative diagrams.
\xyoption{2cell}
\UseAllTwocells

% To put source file link in headers.
% Change "template.tex" to "this_filename.tex"
% \usepackage{fancyhdr}
% \pagestyle{fancy}
% \lhead{}
% \chead{}
% \rhead{Source file: \url{template.tex}}
% \lfoot{}
% \cfoot{\thepage}
% \rfoot{}
% \renewcommand{\headrulewidth}{0pt}
% \renewcommand{\footrulewidth}{0pt}
% \renewcommand{\headheight}{12pt}

\usepackage{multicol}

% For cross-file-references
\usepackage{xr-hyper}

% Package for hypertext links:
\usepackage{hyperref}

% For any local file, say "hello.tex" you want to link to please
% use \externaldocument[hello-]{hello}
\externaldocument[introduction-]{introduction}
\externaldocument[conventions-]{conventions}
\externaldocument[sets-]{sets}
\externaldocument[categories-]{categories}
\externaldocument[topology-]{topology}
\externaldocument[sheaves-]{sheaves}
\externaldocument[sites-]{sites}
\externaldocument[stacks-]{stacks}
\externaldocument[fields-]{fields}
\externaldocument[algebra-]{algebra}
\externaldocument[brauer-]{brauer}
\externaldocument[homology-]{homology}
\externaldocument[derived-]{derived}
\externaldocument[simplicial-]{simplicial}
\externaldocument[more-algebra-]{more-algebra}
\externaldocument[smoothing-]{smoothing}
\externaldocument[modules-]{modules}
\externaldocument[sites-modules-]{sites-modules}
\externaldocument[injectives-]{injectives}
\externaldocument[cohomology-]{cohomology}
\externaldocument[sites-cohomology-]{sites-cohomology}
\externaldocument[dga-]{dga}
\externaldocument[dpa-]{dpa}
\externaldocument[hypercovering-]{hypercovering}
\externaldocument[schemes-]{schemes}
\externaldocument[constructions-]{constructions}
\externaldocument[properties-]{properties}
\externaldocument[morphisms-]{morphisms}
\externaldocument[coherent-]{coherent}
\externaldocument[divisors-]{divisors}
\externaldocument[limits-]{limits}
\externaldocument[varieties-]{varieties}
\externaldocument[topologies-]{topologies}
\externaldocument[descent-]{descent}
\externaldocument[perfect-]{perfect}
\externaldocument[more-morphisms-]{more-morphisms}
\externaldocument[flat-]{flat}
\externaldocument[groupoids-]{groupoids}
\externaldocument[more-groupoids-]{more-groupoids}
\externaldocument[etale-]{etale}
\externaldocument[chow-]{chow}
\externaldocument[intersection-]{intersection}
\externaldocument[pic-]{pic}
\externaldocument[adequate-]{adequate}
\externaldocument[dualizing-]{dualizing}
\externaldocument[duality-]{duality}
\externaldocument[discriminant-]{discriminant}
\externaldocument[local-cohomology-]{local-cohomology}
\externaldocument[curves-]{curves}
\externaldocument[resolve-]{resolve}
\externaldocument[models-]{models}
\externaldocument[pione-]{pione}
\externaldocument[etale-cohomology-]{etale-cohomology}
\externaldocument[proetale-]{proetale}
\externaldocument[crystalline-]{crystalline}
\externaldocument[spaces-]{spaces}
\externaldocument[spaces-properties-]{spaces-properties}
\externaldocument[spaces-morphisms-]{spaces-morphisms}
\externaldocument[decent-spaces-]{decent-spaces}
\externaldocument[spaces-cohomology-]{spaces-cohomology}
\externaldocument[spaces-limits-]{spaces-limits}
\externaldocument[spaces-divisors-]{spaces-divisors}
\externaldocument[spaces-over-fields-]{spaces-over-fields}
\externaldocument[spaces-topologies-]{spaces-topologies}
\externaldocument[spaces-descent-]{spaces-descent}
\externaldocument[spaces-perfect-]{spaces-perfect}
\externaldocument[spaces-more-morphisms-]{spaces-more-morphisms}
\externaldocument[spaces-flat-]{spaces-flat}
\externaldocument[spaces-groupoids-]{spaces-groupoids}
\externaldocument[spaces-more-groupoids-]{spaces-more-groupoids}
\externaldocument[bootstrap-]{bootstrap}
\externaldocument[spaces-pushouts-]{spaces-pushouts}
\externaldocument[groupoids-quotients-]{groupoids-quotients}
\externaldocument[spaces-more-cohomology-]{spaces-more-cohomology}
\externaldocument[spaces-simplicial-]{spaces-simplicial}
\externaldocument[formal-spaces-]{formal-spaces}
\externaldocument[restricted-]{restricted}
\externaldocument[spaces-resolve-]{spaces-resolve}
\externaldocument[formal-defos-]{formal-defos}
\externaldocument[defos-]{defos}
\externaldocument[cotangent-]{cotangent}
\externaldocument[examples-defos-]{examples-defos}
\externaldocument[algebraic-]{algebraic}
\externaldocument[examples-stacks-]{examples-stacks}
\externaldocument[stacks-sheaves-]{stacks-sheaves}
\externaldocument[criteria-]{criteria}
\externaldocument[artin-]{artin}
\externaldocument[quot-]{quot}
\externaldocument[stacks-properties-]{stacks-properties}
\externaldocument[stacks-morphisms-]{stacks-morphisms}
\externaldocument[stacks-limits-]{stacks-limits}
\externaldocument[stacks-cohomology-]{stacks-cohomology}
\externaldocument[stacks-perfect-]{stacks-perfect}
\externaldocument[stacks-introduction-]{stacks-introduction}
\externaldocument[stacks-more-morphisms-]{stacks-more-morphisms}
\externaldocument[stacks-geometry-]{stacks-geometry}
\externaldocument[moduli-]{moduli}
\externaldocument[moduli-curves-]{moduli-curves}
\externaldocument[examples-]{examples}
\externaldocument[exercises-]{exercises}
\externaldocument[guide-]{guide}
\externaldocument[desirables-]{desirables}
\externaldocument[coding-]{coding}
\externaldocument[obsolete-]{obsolete}
\externaldocument[fdl-]{fdl}
\externaldocument[index-]{index}

% Theorem environments.
%
\theoremstyle{plain}
\newtheorem{theorem}[subsection]{Theorem}
\newtheorem{proposition}[subsection]{Proposition}
\newtheorem{lemma}[subsection]{Lemma}

\theoremstyle{definition}
\newtheorem{definition}[subsection]{Definition}
\newtheorem{example}[subsection]{Example}
\newtheorem{exercise}[subsection]{Exercise}
\newtheorem{situation}[subsection]{Situation}

\theoremstyle{remark}
\newtheorem{remark}[subsection]{Remark}
\newtheorem{remarks}[subsection]{Remarks}

\numberwithin{equation}{subsection}

% Macros
%
\def\lim{\mathop{\rm lim}\nolimits}
\def\colim{\mathop{\rm colim}\nolimits}
\def\Spec{\mathop{\rm Spec}}
\def\Hom{\mathop{\rm Hom}\nolimits}
\def\Ext{\mathop{\rm Ext}\nolimits}
\def\SheafHom{\mathop{\mathcal{H}\!{\it om}}\nolimits}
\def\SheafExt{\mathop{\mathcal{E}\!{\it xt}}\nolimits}
\def\Sch{\textit{Sch}}
\def\Mor{\mathop{\rm Mor}\nolimits}
\def\Ob{\mathop{\rm Ob}\nolimits}
\def\Sh{\mathop{\textit{Sh}}\nolimits}
\def\NL{\mathop{N\!L}\nolimits}
\def\proetale{{pro\text{-}\acute{e}tale}}
\def\etale{{\acute{e}tale}}
\def\QCoh{\textit{QCoh}}
\def\Ker{\mathop{\rm Ker}}
\def\Im{\mathop{\rm Im}}
\def\Coker{\mathop{\rm Coker}}
\def\Coim{\mathop{\rm Coim}}

%
% Macros for moduli stacks/spaces
%
\def\QCohstack{\mathcal{QC}\!{\it oh}}
\def\Cohstack{\mathcal{C}\!{\it oh}}
\def\Spacesstack{\mathcal{S}\!{\it paces}}
\def\Quotfunctor{{\rm Quot}}
\def\Hilbfunctor{{\rm Hilb}}
\def\Curvesstack{\mathcal{C}\!{\it urves}}
\def\Polarizedstack{\mathcal{P}\!{\it olarized}}
\def\Complexesstack{\mathcal{C}\!{\it omplexes}}
% \Pic is the operator that assigns to X its picard group, usage \Pic(X)
% \Picardstack_{X/B} denotes the Picard stack of X over B
% \Picardfunctor_{X/B} denotes the Picard functor of X over B
\def\Pic{\mathop{\rm Pic}\nolimits}
\def\Picardstack{\mathcal{P}\!{\it ic}}
\def\Picardfunctor{{\rm Pic}}
\def\Deformationcategory{\mathcal{D}\!{\it ef}}


% OK, start here.
%
\begin{document}

\title{Algebraic Curves}


\maketitle

\phantomsection
\label{section-phantom}

\tableofcontents

\section{Introduction}
\label{section-introduction}

\noindent
In this chapter we develop some of the theory of algebraic curves.
A reference covering algebraic curves over the complex numbers is
the book \cite{ACGH}.











\section{Riemann-Roch and duality}
\label{section-Riemann-Roch}

\noindent
Let $k$ be a field. Let $X$ be a proper scheme of dimension $\leq 1$
over $k$. In Varieties, Section \ref{varieties-section-divisors-curves}
we have defined the degree of a locally free $\mathcal{O}_X$-module
$\mathcal{E}$ of constant rank by the formula
\begin{equation}
\label{equation-degree}
\deg(\mathcal{E}) =
\chi(X, \mathcal{E}) - \text{rank}(\mathcal{E})\chi(X, \mathcal{O}_X)
\end{equation}
see Varieties, Definition \ref{varieties-definition-degree-invertible-sheaf}.
In the chapter on Chow Homology we defined the first chern class of
$\mathcal{E}$ as an operation on cycles
(Chow Homology, Section
\ref{chow-section-intersecting-chern-classes}) and we proved that
\begin{equation}
\label{equation-degree-c1}
\deg(\mathcal{E}) = \deg(c_1(\mathcal{E}) \cap [X]_1)
\end{equation}
see Chow Homology, Lemma \ref{chow-lemma-degree-vector-bundle}.
Combining (\ref{equation-degree}) and (\ref{equation-degree-c1})
we obtain our first version of the Riemann-Roch formula
\begin{equation}
\label{equation-rr}
\chi(X, \mathcal{E}) =
\deg(c_1(\mathcal{E}) \cap [X]_1) +
\text{rank}(\mathcal{E})\chi(X, \mathcal{O}_X)
\end{equation}
If $\mathcal{L}$ is an invertible $\mathcal{O}_X$-module, then
we can also consider the numerical intersection
$(\mathcal{L} \cdot X)$ as defined in
Varieties, Definition \ref{varieties-definition-intersection-number}.
However, this does not give anything new as
\begin{equation}
\label{equation-numerical-degree}
(\mathcal{L} \cdot X) = \deg(\mathcal{L})
\end{equation}
by Varieties, Lemma
\ref{varieties-lemma-intersection-numbers-and-degrees-on-curves}. If
$\mathcal{L}$ is ample, then this integer is positive and is
called the degree
\begin{equation}
\label{equation-degree-X}
\deg_\mathcal{L}(X) = (\mathcal{L} \cdot X) = \deg(\mathcal{L})
\end{equation}
of $X$ with respect to $\mathcal{L}$, see
Varieties, Definition \ref{varieties-definition-degree}.

\medskip\noindent
To obtain a true Riemann-Roch theorem we would like to write
$\chi(X, \mathcal{O}_X)$ as the degree of a canonical zero cycle on $X$.
We refer to \cite{F} for a fully general version of this. We will use
duality to get a formula in the case where $X$ is Gorenstein; however,
in some sense this is a cheat (for example because this method cannot
work in higher dimension).

\begin{lemma}
\label{lemma-duality-dim-1}
Let $X$ be a proper scheme of dimension $\leq 1$ over a field $k$.
There exists a dualizing complex $\omega_X^\bullet$ with the
following properties
\begin{enumerate}
\item $H^i(\omega_X^\bullet)$ is nonzero only for $i = -1, 0$,
\item $\omega_X = H^{-1}(\omega_X^\bullet)$
is a coherent Cohen-Macaulay module whose support is the
irreducible components of dimension $1$,
\item for $x \in X$ closed, the module $H^0(\omega_{X, x}^\bullet)$
is nonzero if and only if either
\begin{enumerate}
\item $\dim(\mathcal{O}_{X, x}) = 0$ or
\item $\dim(\mathcal{O}_{X, x}) = 1$
and $\mathcal{O}_{X, x}$ is not Cohen-Macaulay,
\end{enumerate}
\item there are functorial isomorphisms
$\text{Ext}^i_X(K, \omega_X^\bullet) = \Hom_k(H^{-i}(X, K), k)$
compatible with shifts for $K \in D_\QCoh(X)$,
\item there are functorial isomorphisms
$\Hom(\mathcal{F}, \omega_X) = \Hom_k(H^1(X, \mathcal{F}), k)$
for $\mathcal{F}$ quasi-coherent on $X$.
\end{enumerate}
\end{lemma}

\begin{proof}
We start with the relative dualizing complex
$\omega_X^\bullet = \omega_{X/k}^\bullet$
as described in Dualizing Complexes, 
Remark \ref{dualizing-remark-relative-dualizing-complex}.
Then property (4) holds by construction.
Observe that $\omega_X^\bullet$ is also the dualizing complex
normalized relative to
$\omega_{\Spec(k)}^\bullet = \mathcal{O}_{\Spec(k)}$, i.e.,
it is the dualizing complex $\omega_X^\bullet$
as in Dualizing Complexes, Example \ref{dualizing-example-proper-over-local}
with $A = k$ and $\omega_A = k[0]$.
Parts (1) and (2) follow from
Dualizing Complexes, Lemma \ref{dualizing-lemma-vanishing-good-dualizing}.
For a closed point $x \in X$ we see that $\omega_{X, x}^\bullet$ is a
normalized dualizing complex over $\mathcal{O}_{X, x}$, see
Dualizing Complexes, Lemma \ref{dualizing-lemma-good-dualizing-normalized}.
Assertion (3) then follows from
Dualizing Complexes, Lemma \ref{dualizing-lemma-apply-CM}.
Finally, assertion (5) follows from
Dualizing Complexes, Lemma \ref{dualizing-lemma-dualizing-module-proper-over-A}
for coherent $\mathcal{F}$ and in general by unwinding
(4) for $K = \mathcal{F}[0]$ and $i = -1$.
\end{proof}

\begin{lemma}
\label{lemma-duality-dim-1-CM}
Let $X$ be a proper scheme over a field $k$ which is Cohen-Macaulay
and equidimensional of dimension $1$. There exists a dualizing module
$\omega_X$ with the following properties
\begin{enumerate}
\item $\omega_X$ is a coherent Cohen-Macaulay module whose support is $X$,
\item there are functorial isomorphisms
$\text{Ext}^i_X(K, \omega_X[1]) = \Hom_k(H^{-i}(X, K), k)$
compatible with shifts for $K \in D_\QCoh(X)$,
\item there are functorial isomorphisms
$\text{Ext}^{1 + i}(\mathcal{F}, \omega_X) = \Hom_k(H^{-i}(X, \mathcal{F}), k)$
for $\mathcal{F}$ quasi-coherent on $X$.
\end{enumerate}
\end{lemma}

\begin{proof}
Let us take $\omega_X$ normalized as in
as in Dualizing Complexes, Example
\ref{dualizing-example-equidimensional-over-field}.
Then the statements follow from
Lemma \ref{lemma-duality-dim-1}
and the fact that $\omega_X^\bullet = \omega_X[1]$
as $X$ is Cohen-Macualay (Dualizing Complexes, Lemma
\ref{dualizing-lemma-dualizing-module-CM-scheme}).
\end{proof}

\begin{remark}
\label{remark-rework-duality-locally-free}
Let $X$ be a proper scheme of dimension $\leq 1$ over a field $k$.
Let $\omega_X^\bullet$ be as in Lemma \ref{lemma-duality-dim-1}.
If $\mathcal{E}$ is a finite locally free $\mathcal{O}_X$-module
with dual $\mathcal{E}^\wedge$ then we have canonical isomorphisms
$$
\Hom_k(H^{-i}(X, \mathcal{E}), k) =
H^i(X, \mathcal{E}^\wedge \otimes_{\mathcal{O}_X}^\mathbf{L} \omega_X^\bullet)
$$
This follows from the lemma and
Cohomology, Lemma \ref{cohomology-lemma-dual-perfect-complex}.
If $X$ is Cohen-Macaulay and equidimensional of dimension $1$, then
we have canonical isomorphisms
$$
\Hom_k(H^{-i}(X, \mathcal{E}), k) =
H^{1 - i}(X, \mathcal{E}^\wedge \otimes_{\mathcal{O}_X} \omega_X)
$$
where $\omega_X$ is as in Lemma \ref{lemma-duality-dim-1-CM}.
\end{remark}

\noindent
We can use Lemmas \ref{lemma-duality-dim-1} and \ref{lemma-duality-dim-1-CM}
to get a relation between the euler
characteristic of $\mathcal{O}_X$ and the euler characteristic
of the dualizing complex or the dualizing module.

\begin{lemma}
\label{lemma-euler}
Let $X$ be a proper scheme of dimension $\leq 1$ over a field $k$.
With $\omega_X^\bullet$ as in Lemma \ref{lemma-duality-dim-1} we have
$$
\chi(X, \mathcal{O}_X) = \chi(X, \omega_X^\bullet)
$$
If $X$ is Cohen-Macaulay and equidimensional of dimension $1$, then
$$
\chi(X, \mathcal{O}_X) = - \chi(X, \omega_X)
$$
with $\omega_X$ as in Lemma \ref{lemma-duality-dim-1-CM}.
\end{lemma}

\begin{proof}
We define the right hand side of the first formula as follows:
$$
\chi(X, \omega_X^\bullet) =
\sum\nolimits_{i \in \mathbf{Z}} (-1)^i\dim_k H^i(X, \omega_X^\bullet)
$$
This is well defined because $\omega_X^\bullet$ is in
$D^b_{\textit{Coh}}(\mathcal{O}_X)$, but also because
$$
H^i(X, \omega_X^\bullet) =
\text{Ext}^i(\mathcal{O}_X, \omega_X^\bullet) =
H^{-i}(X, \mathcal{O}_X)
$$
which is always finite dimensional and nonzero only if $i = 0, -1$.
This of course also proves the first formula. The second is a consequence
of the first because $\omega_X^\bullet = \omega_X[1]$ in the CM case.
\end{proof}

\noindent
We will use Lemma \ref{lemma-euler} to get the desired formula for
$\chi(X, \mathcal{O}_X)$ in the case that $\omega_X$ is
invertible, i.e., that $X$ is Gorenstein.
The statement is that $-1/2$ of the first chern class of $\omega_X$
capped with the cycle $[X]_1$ associated to $X$ is a natural zero
cycle on $X$ with half-integer coefficients whose degree is
$\chi(X, \mathcal{O}_X)$.
The occurence of fractions in the statement of Riemann-Roch cannot
be avoided.

\begin{lemma}[Rieman-Roch]
\label{lemma-rr}
Let $X$ be a proper scheme over a field $k$ which is Gorenstein and
equidimensional of dimension $1$. Let $\omega_X$ be as in
Lemma \ref{lemma-duality-dim-1-CM}. Then
\begin{enumerate}
\item $\omega_X$ is an invertible $\mathcal{O}_X$-module,
\item $\deg(\omega_X) = -2\chi(X, \mathcal{O}_X)$,
\item for a locally free $\mathcal{O}_X$-module $\mathcal{E}$
of constant rank we have
$$
\chi(X, \mathcal{E}) = \deg(\mathcal{E}) -
\textstyle{\frac{1}{2}} \text{rank}(\mathcal{E}) \deg(\omega_X)
$$
and $\dim_k(H^i(X, \mathcal{E})) =
\dim_k(H^{1 - i}(X, \mathcal{E}^\wedge \otimes_{\mathcal{O}_X} \omega_X)$
for all $i \in \mathbf{Z}$.
\end{enumerate}
\end{lemma}

\begin{proof}
It follows more or less from the definition of the Gorenstein property
that the dualizing sheaf is invertible, see
Dualizing Complexes, Section \ref{dualizing-section-gorenstein}.
By (\ref{equation-rr}) applied to $\omega_X$ we have
$$
\chi(X, \omega_X) = 
\deg(c_1(\omega_X) \cap [X]_1) + \chi(X, \mathcal{O}_X)
$$
Combined with Lemma \ref{lemma-euler} this gives
$$
2\chi(X, \mathcal{O}_X) = - \deg(c_1(\omega_X) \cap [X]_1) = - \deg(\omega_X)
$$
the second equality by (\ref{equation-degree-c1}). Putting this back into
(\ref{equation-rr}) for $\mathcal{E}$ gives the displayed formula of the lemma.
The symmetry in dimensions is a consequence of duality for $X$, see
Remark \ref{remark-rework-duality-locally-free}.
\end{proof}




\section{Some vanishing results}
\label{section-vanishing}

\noindent
In this section we work in the following situation.

\begin{situation}
\label{situation-Cohen-Macaulay-curve}
Here $k$ is a field and $X$ is a proper scheme over $k$ which
is Cohen-Macaulay, equidimensional of dimension $1$, and
has $H^0(X, \mathcal{O}_X) = k$. Let $\omega_X$ be the dualizing
sheaf of $X$ as in Dualizing Complexes, Example
\ref{dualizing-example-equidimensional-over-field}.
\end{situation}

\noindent
From the discussion in Section \ref{section-Riemann-Roch} we see that the
dualizing sheaf $\omega_X$ on $X$ has nonvanishing $H^1$. It turns out
that anything slightly more ``positive'' than $\omega_X$ has vanishing $H^1$.

\begin{lemma}
\label{lemma-vanishing}
In Situation \ref{situation-Cohen-Macaulay-curve}. Given an exact sequence
$$
\omega_X \to \mathcal{F} \to \mathcal{Q} \to 0
$$
of coherent $\mathcal{O}_X$-modules with $H^1(X, \mathcal{Q}) = 0$
(for example if $\dim(\text{Supp}(\mathcal{Q})) = 0$), then
either $H^1(X, \mathcal{F}) = 0$ or
$\mathcal{F} = \omega_X \oplus \mathcal{Q}$.
\end{lemma}

\begin{proof}
(The parenthetical statement follows from
Cohomology of Schemes, Lemma \ref{coherent-lemma-coherent-support-dimension-0}.)
Since $H^0(X, \mathcal{O}_X) = k$ is dual to $H^1(X, \omega_X)$
(see Section \ref{section-Riemann-Roch})
we see that $\dim H^1(X, \omega_X) = 1$. The sheaf $\omega_X$
represents the functor
$\mathcal{F} \mapsto \Hom_k(H^1(X, \mathcal{F}), k)$
on the category of coherent $\mathcal{O}_X$-modules
(Dualizing Complexes, Lemma
\ref{dualizing-lemma-dualizing-module-proper-over-A}).
Consider an exact sequence as in the statement of the lemma
and assume that $H^1(X, \mathcal{F}) \not = 0$. Since
$H^1(X, \mathcal{Q}) = 0$ we see that
$H^1(X, \omega_X) \to H^1(X, \mathcal{F})$ is an isomorphism.
By the universal property of $\omega_X$ stated above, we conclude there
is a map $\mathcal{F} \to \omega_X$ whose action on $H^1$ is the inverse
of this isomorphism. The composition $\omega_X \to \mathcal{F} \to \omega_X$
is the identity (by the universal property) and the lemma is proved.
\end{proof}

\begin{lemma}
\label{lemma-vanishing-twist}
In Situation \ref{situation-Cohen-Macaulay-curve}. Let
$\mathcal{L}$ be an invertible $\mathcal{O}_X$-module which is
globally generated and not isomorphic to $\mathcal{O}_X$. Then
$H^1(X, \omega_X \otimes \mathcal{L}) = 0$.
\end{lemma}

\begin{proof}
By duality as discussed in Section \ref{section-Riemann-Roch} we have to
show that $H^0(X, \mathcal{L}^{\otimes - 1}) = 0$. If not, then we can
choose a global section $t$ of $\mathcal{L}^{\otimes - 1}$
and a global section $s$ of $\mathcal{L}$ such that $st \not = 0$.
However, then $st$ is a constant multiple of $1$, by our assumption
that $H^0(X, \mathcal{O}_X) = k$. It follows that
$\mathcal{L} \cong \mathcal{O}_X$, which is a contradiction.
\end{proof}

\begin{lemma}
\label{lemma-globally-generated}
In Situation \ref{situation-Cohen-Macaulay-curve}. Given an exact sequence
$$
\omega_X \to \mathcal{F} \to \mathcal{Q} \to 0
$$
of coherent $\mathcal{O}_X$-modules with $\dim(\text{Supp}(\mathcal{Q})) = 0$
and $\dim_k H^0(X, \mathcal{Q}) \geq 2$ and such that there is no nonzero
submodule $\mathcal{Q}' \subset \mathcal{F}$ such that
$\mathcal{Q}' \to \mathcal{Q}$ is injective.
Then the submodule of $\mathcal{F}$ generated by global
sections surjects onto $\mathcal{Q}$.
\end{lemma}

\begin{proof}
Let $\mathcal{F}' \subset \mathcal{F}$ be the submodule generated by
global sections and the image of $\omega_X \to \mathcal{F}$. Since
$\dim_k H^0(X, \mathcal{Q}) \geq 2$ and
$\dim_k H^1(X, \omega_X) = \dim_k H^0(X, \mathcal{O}_X) = 1$,
we see that $\mathcal{F}' \to \mathcal{Q}$ is not zero and
$\omega_X \to \mathcal{F}'$ is not an isomorphism.
Hence $H^1(X, \mathcal{F}') = 0$ by Lemma \ref{lemma-vanishing}
and our assumption on $\mathcal{F}$.
Consider the short exact sequence
$$
0 \to \mathcal{F}' \to \mathcal{F} \to
\mathcal{Q}/\Im(\mathcal{F}' \to \mathcal{Q}) \to 0
$$
If the quotient on the right is nonzero, then we obtain a contradiction
because then $H^0(X, \mathcal{F})$ is bigger than $H^0(X, \mathcal{F}')$.
\end{proof}

\noindent
Here is an example global generation statement.

\begin{lemma}
\label{lemma-globally-generated-curve}
In Situation \ref{situation-Cohen-Macaulay-curve} assume that
$X$ is integral. Let $0 \to \omega_X \to \mathcal{F} \to \mathcal{Q} \to 0$
be a short exact sequence of coherent $\mathcal{O}_X$-modules with
$\mathcal{F}$ torsion free, $\dim(\text{Supp}(\mathcal{Q})) = 0$,
and $\dim_k H^0(X, \mathcal{Q}) \geq 2$. Then $\mathcal{F}$
is globally generated.
\end{lemma}

\begin{proof}
Consider the submodule $\mathcal{F}'$ generated by the global sections. By
Lemma \ref{lemma-globally-generated} we see that $\mathcal{F}' \to \mathcal{Q}$
is surjective, in particular $\mathcal{F}' \not = 0$. Since $X$ is a curve, we
see that $\mathcal{F}' \subset \mathcal{F}$ is an inclusion of rank $1$
sheaves, hence $\mathcal{Q}' = \mathcal{F}/\mathcal{F}'$ is supported in
finitely many points. To get a contradiction, assume that
$\mathcal{Q}'$ is nonzero. Then we see that $H^1(X, \mathcal{F}') \not = 0$.
Then we get a nonzero map $\mathcal{F}' \to \omega_X$ by the universal
property (Dualizing Complexes, Lemma
\ref{dualizing-lemma-dualizing-module-proper-over-A}).
The image of the composition $\mathcal{F}' \to \omega_X \to \mathcal{F}$
is generated by global sections, hence is inside of $\mathcal{F}'$.
Thus we get a nonzero self map $\mathcal{F}' \to \mathcal{F}'$.
Since $\mathcal{F}'$ is torsion free of rank $1$ on a proper curve
this has to be an automorphism (details omitted). But then this implies that
$\mathcal{F}'$ is contained in $\omega_X \subset \mathcal{F}$
contradicting the surjectivity of $\mathcal{F}' \to \mathcal{Q}$.
\end{proof}

\begin{lemma}
\label{lemma-tensor-omega-with-globally-generated-invertible}
In Situation \ref{situation-Cohen-Macaulay-curve}. Let $\mathcal{L}$
be a very ample invertible $\mathcal{O}_X$-module with
$\deg(\mathcal{L}) \geq 2$. Then
$\omega_X \otimes_{\mathcal{O}_X} \mathcal{L}$ is globally generated.
\end{lemma}

\begin{proof}
Assume $k$ is algebraically closed. Let $x \in X$ be a closed point.
Let $C_i \subset X$ be the irreducible components and for each $i$
let $x_i \in C_i$ be the generic point. By
Varieties, Lemma \ref{varieties-lemma-very-ample-vanish-at-point}
we can choose a section $s \in H^0(X, \mathcal{L})$ such that $s$
vanishes at $x$ but not at $x_i$ for all $i$. The corresponding
module map $s : \mathcal{O}_X \to \mathcal{L}$ is injective with
cokernel $\mathcal{Q}$ supported in finitely many points and
with $H^0(X, \mathcal{Q}) \geq 2$. Consider the corresponding
exact sequence
$$
0 \to \omega_X \to \omega_X \otimes \mathcal{L} \to
\omega_X \otimes \mathcal{Q} \to 0
$$
By Lemma \ref{lemma-globally-generated} we see that the module generated
by global sections surjects onto $\omega_X \otimes \mathcal{Q}$.
Since $x$ was arbitrary this proves the lemma. Some details omitted.

\medskip\noindent
We will reduce the case where $k$ is not algebraically closed, to
the algebraically closed field case. We suggest the reader skip
the rest of the proof. Choose an algebraic closure $\overline{k}$
of $k$ and consider the base change $X_{\overline{k}}$. Let us
check that $X_{\overline{k}} \to \Spec(\overline{k})$ is an example
of Situation \ref{situation-Cohen-Macaulay-curve}. By flat base change
(Cohomology of Schemes, Lemma \ref{coherent-lemma-flat-base-change-cohomology})
we see that $H^0(X_{\overline{k}}, \mathcal{O}) = \overline{k}$.
By Varieties, Lemma \ref{varieties-lemma-CM-base-change}
we see that $X_{\overline{k}}$ is Cohen-Macaulay. The scheme
$X_{\overline{k}}$ is proper over $\overline{k}$ (Morphisms,
Lemma \ref{morphisms-lemma-base-change-proper}) and
equidimensional of dimension $1$
(Morphisms, Lemma \ref{morphisms-lemma-dimension-fibre-after-base-change}).
The pullback of $\omega_X$ to $X_{\overline{k}}$ is the dualizing
module of $X_{\overline{k}}$ by
Dualizing Complexes, Lemma \ref{dualizing-lemma-more-base-change}.
The pullback of $\mathcal{L}$ to $X_{\overline{k}}$ is very ample
(Morphisms, Lemma \ref{morphisms-lemma-very-ample-base-change}).
The degree of the pullback of $\mathcal{L}$ to $X_{\overline{k}}$
is equal to the degree of $\mathcal{L}$ on $X$ (Varieties, Lemma
\ref{varieties-lemma-degree-base-change}). Finally, we see that
$\omega_X \otimes \mathcal{L}$ is globally generated if and only
if its base change is so
(Varieties, Lemma \ref{varieties-lemma-globally-generated-base-change}).
In this way we see that the result follows from the result in the
case of an algebraically closed ground field.
\end{proof}




















\section{Other chapters}

\begin{multicols}{2}
\begin{enumerate}
\item \hyperref[introduction-section-phantom]{Introduction}
\item \hyperref[conventions-section-phantom]{Conventions}
\item \hyperref[sets-section-phantom]{Set Theory}
\item \hyperref[categories-section-phantom]{Categories}
\item \hyperref[topology-section-phantom]{Topology}
\item \hyperref[sheaves-section-phantom]{Sheaves on Spaces}
\item \hyperref[algebra-section-phantom]{Commutative Algebra}
\item \hyperref[sites-section-phantom]{Sites and Sheaves}
\item \hyperref[homology-section-phantom]{Homological Algebra}
\item \hyperref[derived-section-phantom]{Derived Categories}
\item \hyperref[more-algebra-section-phantom]{More Algebra}
\item \hyperref[simplicial-section-phantom]{Simplicial Methods}
\item \hyperref[modules-section-phantom]{Sheaves of Modules}
\item \hyperref[sites-modules-section-phantom]{Modules on Sites}
\item \hyperref[injectives-section-phantom]{Injectives}
\item \hyperref[cohomology-section-phantom]{Cohomology of Sheaves}
\item \hyperref[sites-cohomology-section-phantom]{Cohomology on Sites}
\item \hyperref[hypercovering-section-phantom]{Hypercoverings}
\item \hyperref[schemes-section-phantom]{Schemes}
\item \hyperref[constructions-section-phantom]{Constructions of Schemes}
\item \hyperref[properties-section-phantom]{Properties of Schemes}
\item \hyperref[morphisms-section-phantom]{Morphisms of Schemes}
\item \hyperref[coherent-section-phantom]{Coherent Cohomology}
\item \hyperref[divisors-section-phantom]{Divisors}
\item \hyperref[limits-section-phantom]{Limits of Schemes}
\item \hyperref[varieties-section-phantom]{Varieties}
\item \hyperref[chow-section-phantom]{Chow Homology}
\item \hyperref[topologies-section-phantom]{Topologies on Schemes}
\item \hyperref[descent-section-phantom]{Descent}
\item \hyperref[more-morphisms-section-phantom]{More on Morphisms}
\item \hyperref[flat-section-phantom]{More on Flatness}
\item \hyperref[groupoids-section-phantom]{Groupoid Schemes}
\item \hyperref[more-groupoids-section-phantom]{More on Groupoid Schemes}
\item \hyperref[etale-section-phantom]{\'Etale Morphisms of Schemes}
\item \hyperref[etale-cohomology-section-phantom]{\'Etale Cohomology}
\item \hyperref[spaces-section-phantom]{Algebraic Spaces}
\item \hyperref[spaces-properties-section-phantom]{Properties of Algebraic Spaces}
\item \hyperref[spaces-morphisms-section-phantom]{Morphisms of Algebraic Spaces}
\item \hyperref[spaces-topologies-section-phantom]{Topologies on Algebraic Spaces}
\item \hyperref[spaces-descent-section-phantom]{Descent and Algebraic Spaces}
\item \hyperref[spaces-more-morphisms-section-phantom]{More on Morphisms of Spaces}
\item \hyperref[quot-section-phantom]{Quot and Hilbert Spaces}
\item \hyperref[stacks-section-phantom]{Stacks}
\item \hyperref[spaces-groupoids-section-phantom]{Groupoids in Algebraic Spaces}
\item \hyperref[spaces-more-groupoids-section-phantom]{More on Groupoids in Spaces}
\item \hyperref[bootstrap-section-phantom]{Bootstrap}
\item \hyperref[examples-stacks-section-phantom]{Examples of Stacks}
\item \hyperref[groupoids-quotients-section-phantom]{Quotients of Groupoids}
\item \hyperref[algebraic-section-phantom]{Algebraic Stacks}
\item \hyperref[criteria-section-phantom]{Criteria for Representability}
\item \hyperref[stacks-properties-section-phantom]{Properties of Algebraic Stacks}
\item \hyperref[stacks-morphisms-section-phantom]{Morphisms of Algebraic Stacks}
\item \hyperref[examples-section-phantom]{Examples}
\item \hyperref[exercises-section-phantom]{Exercises}
\item \hyperref[guide-section-phantom]{Guide to Literature}
\item \hyperref[desirables-section-phantom]{Desirables}
\item \hyperref[coding-section-phantom]{Coding Style}
\item \hyperref[fdl-section-phantom]{GNU Free Documentation License}
\item \hyperref[index-section-phantom]{Auto Generated Index}
\end{enumerate}
\end{multicols}


\bibliography{my}
\bibliographystyle{amsalpha}

\end{document}
