\IfFileExists{stacks-project.cls}{%
\documentclass{stacks-project}
}{%
\documentclass{amsart}
}

% The following AMS packages are automatically loaded with
% the amsart documentclass:
%\usepackage{amsmath}
%\usepackage{amssymb}
%\usepackage{amsthm}

% For dealing with references we use the comment environment
\usepackage{verbatim}
\newenvironment{reference}{\comment}{\endcomment}
%\newenvironment{reference}{}{}
\newenvironment{slogan}{\comment}{\endcomment}
\newenvironment{history}{\comment}{\endcomment}

% For commutative diagrams you can use
% \usepackage{amscd}
\usepackage[all]{xy}

% We use 2cell for 2-commutative diagrams.
\xyoption{2cell}
\UseAllTwocells

% To put source file link in headers.
% Change "template.tex" to "this_filename.tex"
% \usepackage{fancyhdr}
% \pagestyle{fancy}
% \lhead{}
% \chead{}
% \rhead{Source file: \url{template.tex}}
% \lfoot{}
% \cfoot{\thepage}
% \rfoot{}
% \renewcommand{\headrulewidth}{0pt}
% \renewcommand{\footrulewidth}{0pt}
% \renewcommand{\headheight}{12pt}

\usepackage{multicol}

% For cross-file-references
\usepackage{xr-hyper}

% Package for hypertext links:
\usepackage{hyperref}

% For any local file, say "hello.tex" you want to link to please
% use \externaldocument[hello-]{hello}
\externaldocument[introduction-]{introduction}
\externaldocument[conventions-]{conventions}
\externaldocument[sets-]{sets}
\externaldocument[categories-]{categories}
\externaldocument[topology-]{topology}
\externaldocument[sheaves-]{sheaves}
\externaldocument[sites-]{sites}
\externaldocument[stacks-]{stacks}
\externaldocument[fields-]{fields}
\externaldocument[algebra-]{algebra}
\externaldocument[brauer-]{brauer}
\externaldocument[homology-]{homology}
\externaldocument[derived-]{derived}
\externaldocument[simplicial-]{simplicial}
\externaldocument[more-algebra-]{more-algebra}
\externaldocument[smoothing-]{smoothing}
\externaldocument[modules-]{modules}
\externaldocument[sites-modules-]{sites-modules}
\externaldocument[injectives-]{injectives}
\externaldocument[cohomology-]{cohomology}
\externaldocument[sites-cohomology-]{sites-cohomology}
\externaldocument[dga-]{dga}
\externaldocument[dpa-]{dpa}
\externaldocument[hypercovering-]{hypercovering}
\externaldocument[schemes-]{schemes}
\externaldocument[constructions-]{constructions}
\externaldocument[properties-]{properties}
\externaldocument[morphisms-]{morphisms}
\externaldocument[coherent-]{coherent}
\externaldocument[divisors-]{divisors}
\externaldocument[limits-]{limits}
\externaldocument[varieties-]{varieties}
\externaldocument[topologies-]{topologies}
\externaldocument[descent-]{descent}
\externaldocument[perfect-]{perfect}
\externaldocument[more-morphisms-]{more-morphisms}
\externaldocument[flat-]{flat}
\externaldocument[groupoids-]{groupoids}
\externaldocument[more-groupoids-]{more-groupoids}
\externaldocument[etale-]{etale}
\externaldocument[chow-]{chow}
\externaldocument[intersection-]{intersection}
\externaldocument[pic-]{pic}
\externaldocument[adequate-]{adequate}
\externaldocument[dualizing-]{dualizing}
\externaldocument[duality-]{duality}
\externaldocument[discriminant-]{discriminant}
\externaldocument[local-cohomology-]{local-cohomology}
\externaldocument[curves-]{curves}
\externaldocument[resolve-]{resolve}
\externaldocument[models-]{models}
\externaldocument[pione-]{pione}
\externaldocument[etale-cohomology-]{etale-cohomology}
\externaldocument[proetale-]{proetale}
\externaldocument[crystalline-]{crystalline}
\externaldocument[spaces-]{spaces}
\externaldocument[spaces-properties-]{spaces-properties}
\externaldocument[spaces-morphisms-]{spaces-morphisms}
\externaldocument[decent-spaces-]{decent-spaces}
\externaldocument[spaces-cohomology-]{spaces-cohomology}
\externaldocument[spaces-limits-]{spaces-limits}
\externaldocument[spaces-divisors-]{spaces-divisors}
\externaldocument[spaces-over-fields-]{spaces-over-fields}
\externaldocument[spaces-topologies-]{spaces-topologies}
\externaldocument[spaces-descent-]{spaces-descent}
\externaldocument[spaces-perfect-]{spaces-perfect}
\externaldocument[spaces-more-morphisms-]{spaces-more-morphisms}
\externaldocument[spaces-flat-]{spaces-flat}
\externaldocument[spaces-groupoids-]{spaces-groupoids}
\externaldocument[spaces-more-groupoids-]{spaces-more-groupoids}
\externaldocument[bootstrap-]{bootstrap}
\externaldocument[spaces-pushouts-]{spaces-pushouts}
\externaldocument[groupoids-quotients-]{groupoids-quotients}
\externaldocument[spaces-more-cohomology-]{spaces-more-cohomology}
\externaldocument[spaces-simplicial-]{spaces-simplicial}
\externaldocument[formal-spaces-]{formal-spaces}
\externaldocument[restricted-]{restricted}
\externaldocument[spaces-resolve-]{spaces-resolve}
\externaldocument[formal-defos-]{formal-defos}
\externaldocument[defos-]{defos}
\externaldocument[cotangent-]{cotangent}
\externaldocument[examples-defos-]{examples-defos}
\externaldocument[algebraic-]{algebraic}
\externaldocument[examples-stacks-]{examples-stacks}
\externaldocument[stacks-sheaves-]{stacks-sheaves}
\externaldocument[criteria-]{criteria}
\externaldocument[artin-]{artin}
\externaldocument[quot-]{quot}
\externaldocument[stacks-properties-]{stacks-properties}
\externaldocument[stacks-morphisms-]{stacks-morphisms}
\externaldocument[stacks-limits-]{stacks-limits}
\externaldocument[stacks-cohomology-]{stacks-cohomology}
\externaldocument[stacks-perfect-]{stacks-perfect}
\externaldocument[stacks-introduction-]{stacks-introduction}
\externaldocument[stacks-more-morphisms-]{stacks-more-morphisms}
\externaldocument[stacks-geometry-]{stacks-geometry}
\externaldocument[moduli-]{moduli}
\externaldocument[moduli-curves-]{moduli-curves}
\externaldocument[examples-]{examples}
\externaldocument[exercises-]{exercises}
\externaldocument[guide-]{guide}
\externaldocument[desirables-]{desirables}
\externaldocument[coding-]{coding}
\externaldocument[obsolete-]{obsolete}
\externaldocument[fdl-]{fdl}
\externaldocument[index-]{index}

% Theorem environments.
%
\theoremstyle{plain}
\newtheorem{theorem}[subsection]{Theorem}
\newtheorem{proposition}[subsection]{Proposition}
\newtheorem{lemma}[subsection]{Lemma}

\theoremstyle{definition}
\newtheorem{definition}[subsection]{Definition}
\newtheorem{example}[subsection]{Example}
\newtheorem{exercise}[subsection]{Exercise}
\newtheorem{situation}[subsection]{Situation}

\theoremstyle{remark}
\newtheorem{remark}[subsection]{Remark}
\newtheorem{remarks}[subsection]{Remarks}

\numberwithin{equation}{subsection}

% Macros
%
\def\lim{\mathop{\rm lim}\nolimits}
\def\colim{\mathop{\rm colim}\nolimits}
\def\Spec{\mathop{\rm Spec}}
\def\Hom{\mathop{\rm Hom}\nolimits}
\def\Ext{\mathop{\rm Ext}\nolimits}
\def\SheafHom{\mathop{\mathcal{H}\!{\it om}}\nolimits}
\def\SheafExt{\mathop{\mathcal{E}\!{\it xt}}\nolimits}
\def\Sch{\textit{Sch}}
\def\Mor{\mathop{\rm Mor}\nolimits}
\def\Ob{\mathop{\rm Ob}\nolimits}
\def\Sh{\mathop{\textit{Sh}}\nolimits}
\def\NL{\mathop{N\!L}\nolimits}
\def\proetale{{pro\text{-}\acute{e}tale}}
\def\etale{{\acute{e}tale}}
\def\QCoh{\textit{QCoh}}
\def\Ker{\mathop{\rm Ker}}
\def\Im{\mathop{\rm Im}}
\def\Coker{\mathop{\rm Coker}}
\def\Coim{\mathop{\rm Coim}}

%
% Macros for moduli stacks/spaces
%
\def\QCohstack{\mathcal{QC}\!{\it oh}}
\def\Cohstack{\mathcal{C}\!{\it oh}}
\def\Spacesstack{\mathcal{S}\!{\it paces}}
\def\Quotfunctor{{\rm Quot}}
\def\Hilbfunctor{{\rm Hilb}}
\def\Curvesstack{\mathcal{C}\!{\it urves}}
\def\Polarizedstack{\mathcal{P}\!{\it olarized}}
\def\Complexesstack{\mathcal{C}\!{\it omplexes}}
% \Pic is the operator that assigns to X its picard group, usage \Pic(X)
% \Picardstack_{X/B} denotes the Picard stack of X over B
% \Picardfunctor_{X/B} denotes the Picard functor of X over B
\def\Pic{\mathop{\rm Pic}\nolimits}
\def\Picardstack{\mathcal{P}\!{\it ic}}
\def\Picardfunctor{{\rm Pic}}
\def\Deformationcategory{\mathcal{D}\!{\it ef}}


% OK, start here.
%
\begin{document}

\title{Examples of Algebraic Stacks}


\maketitle

\phantomsection
\label{section-phantom}

\tableofcontents

\section{Introduction}
\label{section-introduction}

\noindent
This is a list of examples of algebraic stacks.
The reason for being of this list is to serve also as a list of
topics for lectures by students in the Graduate Student Algebraic Geometry
Seminar held at Columbia University in the Fall of 2009.

\medskip\noindent
As a reference we use \cite[Section 4]{DM}.


\section{Quotient stacks}
\label{section-quotient-stacks}

\noindent
Introduce $[X/G]$ and give lots of examples, such as
$B(G)$ where $G$ is either an abstract group or a group
scheme. Another example could be $\mathcal{M}_{1, 1}$.
Another example $\mathcal{P}(a_0, \ldots, a_n)$
(weighted projective space stack -- also discuss what happens if
all $a_i$ are $1$).


\section{Stacky curves}
\label{section-stacky-curves}

\noindent
They are always gerbs over orbi-curves. Explain this.
Universal coverings? When is it a quotient stack?
What is the genus of a stacky curve? What is the canonical sheaf?
Is there a dualizing sheaf? Etc, etc.
What is a morphism from a stacky (generically non-gerby) curve
into $B(S_n)$?

\section{Actions of group(schemes) on algebraic stacks}
\label{section-actions}

\noindent
Ask Michael Thaddeus what should come out when $H$ acts on $B(G)$.
Use this to come up with a definition. Do examples: $H$ acting on
$B(G)$, when can $\mathbf{G}_m$ act nontrivially on a stacky curve?
Define toric stacks, and give examples.

\section{Picard stacks}
\label{section-picard}

\noindent
This includes the notion of a Picard stack of an algebraic stack
(just because it is selfreferential). Discuss inertia stack.
When is the Picard stack supposed to be algebraic? This may be a good
lecture to discuss Artin's axioms and a little deformation theory.

\section{Vector bundles on algebraic stacks}
\label{section-vectorbundles}

\noindent
Discuss a tiny little bit the notion of a quasi-coherent sheaf on
an algebraic stack. Especially: Vector bundles on quotient stacks.
Vector bundles on $B(G)$ where $G$ is a group or an algebraic group.
Classify vector bundles on $[\mathbf{P}^1/\mathbf{G}_m]$
(ask Michael Thaddeus).


\section{Coarse module spaces (schemes)}
\label{section-coarse}

\noindent
Elementary remarks and introduce the theorem of Keel-Mori.
Applications: Existence without GIT. Discuss how this leads to
coarse moduli schemes in certain cases (for example
$\overline{\mathcal{M}}_g$).


\section{GIT-stacks?}
\label{section-GIT}

\begin{definition}
\label{definition-GIT-stack}
Let $k$ be a field.
A {\it GIT-stack}\footnote{This is likely nonstandard notation.}
$\mathcal{X}$ over $k$ is a quotient stack of the form
$$
\mathcal{X} = [X^{ss}/G]
$$
where $G$ is a reductive linear algebraic group over $k$,
and $X^{ss}$ is the semi-stable locus for an action of $G$ on
a is a projective scheme $(X, \mathcal{O}_X(1))$ over $k$.
\end{definition}

\noindent
A GIT-stack $\mathcal{X}$ does not come equipped with the data
of $(G, X, \mathcal{O}_X(1))$. Show GIT-stacks have coarse moduli
schemes. Show GIT-stacks saitsfy the existence part of the valuative
criterion of properness. Are GIT-stacks proper?
Ask Jarod Alper about some of the good properties of the
morphism $\mathcal{X} \to X$ from a GIT-stack to its coarse
moduli scheme. Examples:
\begin{enumerate}
\item Let $C$ be a projective, smooth,
geometrically connected curve over $k$. Let
$$
\mathcal{M} = \textit{Sh}(C, r, \mathcal{O}_C)
$$
be the stack parametrizing locally free sheaves of rank $r$ on
$C$ with a given trivialization of the determinant. Show that
$\mathcal{M}$ contains an open dense substack which is a GIT-stack.
\item Another similar example is to take $\textit{Curves}$ the algebraic
stack parametrizing (all) flat, proper families of curves
whose geometric fibres are $1$-dimensional proper schemes
having arithmetic genus $1$. Then
$\overline{\mathcal{M}}_g \subset \textit{Curves}$ is an
open substack which is a GIT-stack.
\item Weighted projective space stacks.
\end{enumerate}


\section{Irreducible, connected, normal, etc algebraic stacks}
\label{section-elementary-properties}

\noindent
In other words, discuss some elementary properties of algebraic stacks.
Also: dimension, closed substacks, etc.
If $X$ is a connected smooth algebraic stack over $\mathbf{C}$
why is dimension the dimension at any point the same?


\section{Cohomology groups of stacks}
\label{section-cohomology}

\noindent
What is the problem with defining cohomology? (Different sites
associated to an algebraic stack.)
What should be the answer? What should be the answer for $[X/G]$?
Give a definition that just works.
Compute examples, such as $\mathcal{M}_{1, 1, \mathbf{C}}$ for example.
Lefschetz trace formula for fixed points of Frobenius
for a stack over a finite field (focus on what should be true, not on
what can be proved).

\section{Intersection theory on stacks}
\label{section-intersection-theory}

\noindent
Something about how on quotient stacks.
Compute the following Chow rings:
\begin{enumerate}
\item $A_*(B(\mathbf{G}_m))$,
\item $A_*(B(G))$ where $G$ is a finite cyclic group,
\item $A_*(\mathcal{P}(a_0, \ldots, a_n))$ where
$$
\mathcal{P}(a_0, \ldots, a_n)
=
[\mathbf{A}^{n + 1} \setminus \{0\}/\mathbf{G}_m]
$$
with $\mathbf{G}_m$ acting with weights $a_0, \ldots, a_n$ on
$\mathbf{A}^{n + 1}$,
\item $A_*(B(\text{GL}_n))$, and
\item $A_*([\mathbf{A}^n/\text{GL}_n])$ (this is actually trivial
if you know the answer to the previous one).
\end{enumerate}
GIT-stacks + fundamental class problem.

\section{Examples of algebraic spaces}
\label{section-spaces}

\noindent
Examples of algebraic spaces which are not schemes, and which are still
completely natural from the point of view of moduli theory:
\begin{enumerate}
\item Example of noneffective descent from \cite[Section 6.7]{Ner}.
The descent is effective in the category of algerbaic spaces.
Think of this as giving a flat
family of curves over a base scheme whose total space is
an algebraic space and all of whose fibres are projective curves.
\item Similarly: Take a (suitably general) $1$-parameter
family of smooth surfaces
of degree $d \geq 4$ in $\mathbf{P}^3$ degenerating to a surface
having an ordinary double point. After possibly making a double cover
of the base here exists an algebraic space all of whose fibres are
smooth surfaces, which is the same as the original family except
for the central fibre, and with smooth central fibre. But the total
space is not a scheme.
\item Related to previous example: Small resolutions of
threefolds with ordinary double points. There are $2^{\#\text{of nodes}}$
of them. Explain ``nonseparatedness'' of moduli stack of surfaces.
\item Hironaka example mod $\mathbf{Z}/2\mathbf{Z}$ if you like.
\end{enumerate}




\section{Other chapters}

\begin{multicols}{2}
\begin{enumerate}
\item \hyperref[introduction-section-phantom]{Introduction}
\item \hyperref[conventions-section-phantom]{Conventions}
\item \hyperref[sets-section-phantom]{Set Theory}
\item \hyperref[categories-section-phantom]{Categories}
\item \hyperref[topology-section-phantom]{Topology}
\item \hyperref[sheaves-section-phantom]{Sheaves on Spaces}
\item \hyperref[algebra-section-phantom]{Commutative Algebra}
\item \hyperref[sites-section-phantom]{Sites and Sheaves}
\item \hyperref[homology-section-phantom]{Homological Algebra}
\item \hyperref[derived-section-phantom]{Derived Categories}
\item \hyperref[more-algebra-section-phantom]{More Algebra}
\item \hyperref[simplicial-section-phantom]{Simplicial Methods}
\item \hyperref[modules-section-phantom]{Sheaves of Modules}
\item \hyperref[sites-modules-section-phantom]{Modules on Sites}
\item \hyperref[injectives-section-phantom]{Injectives}
\item \hyperref[cohomology-section-phantom]{Cohomology of Sheaves}
\item \hyperref[sites-cohomology-section-phantom]{Cohomology on Sites}
\item \hyperref[hypercovering-section-phantom]{Hypercoverings}
\item \hyperref[schemes-section-phantom]{Schemes}
\item \hyperref[constructions-section-phantom]{Constructions of Schemes}
\item \hyperref[properties-section-phantom]{Properties of Schemes}
\item \hyperref[morphisms-section-phantom]{Morphisms of Schemes}
\item \hyperref[coherent-section-phantom]{Coherent Cohomology}
\item \hyperref[divisors-section-phantom]{Divisors}
\item \hyperref[limits-section-phantom]{Limits of Schemes}
\item \hyperref[varieties-section-phantom]{Varieties}
\item \hyperref[chow-section-phantom]{Chow Homology}
\item \hyperref[topologies-section-phantom]{Topologies on Schemes}
\item \hyperref[descent-section-phantom]{Descent}
\item \hyperref[more-morphisms-section-phantom]{More on Morphisms}
\item \hyperref[flat-section-phantom]{More on Flatness}
\item \hyperref[groupoids-section-phantom]{Groupoid Schemes}
\item \hyperref[more-groupoids-section-phantom]{More on Groupoid Schemes}
\item \hyperref[etale-section-phantom]{\'Etale Morphisms of Schemes}
\item \hyperref[etale-cohomology-section-phantom]{\'Etale Cohomology}
\item \hyperref[spaces-section-phantom]{Algebraic Spaces}
\item \hyperref[spaces-properties-section-phantom]{Properties of Algebraic Spaces}
\item \hyperref[spaces-morphisms-section-phantom]{Morphisms of Algebraic Spaces}
\item \hyperref[spaces-topologies-section-phantom]{Topologies on Algebraic Spaces}
\item \hyperref[spaces-descent-section-phantom]{Descent and Algebraic Spaces}
\item \hyperref[spaces-more-morphisms-section-phantom]{More on Morphisms of Spaces}
\item \hyperref[quot-section-phantom]{Quot and Hilbert Spaces}
\item \hyperref[stacks-section-phantom]{Stacks}
\item \hyperref[spaces-groupoids-section-phantom]{Groupoids in Algebraic Spaces}
\item \hyperref[spaces-more-groupoids-section-phantom]{More on Groupoids in Spaces}
\item \hyperref[bootstrap-section-phantom]{Bootstrap}
\item \hyperref[examples-stacks-section-phantom]{Examples of Stacks}
\item \hyperref[groupoids-quotients-section-phantom]{Quotients of Groupoids}
\item \hyperref[algebraic-section-phantom]{Algebraic Stacks}
\item \hyperref[criteria-section-phantom]{Criteria for Representability}
\item \hyperref[stacks-properties-section-phantom]{Properties of Algebraic Stacks}
\item \hyperref[stacks-morphisms-section-phantom]{Morphisms of Algebraic Stacks}
\item \hyperref[examples-section-phantom]{Examples}
\item \hyperref[exercises-section-phantom]{Exercises}
\item \hyperref[guide-section-phantom]{Guide to Literature}
\item \hyperref[desirables-section-phantom]{Desirables}
\item \hyperref[coding-section-phantom]{Coding Style}
\item \hyperref[fdl-section-phantom]{GNU Free Documentation License}
\item \hyperref[index-section-phantom]{Auto Generated Index}
\end{enumerate}
\end{multicols}


\bibliography{my}
\bibliographystyle{alpha}

\end{document}
