\IfFileExists{stacks-project.cls}{%
\documentclass{stacks-project}
}{%
\documentclass{amsart}
}

% The following AMS packages are automatically loaded with
% the amsart documentclass:
%\usepackage{amsmath}
%\usepackage{amssymb}
%\usepackage{amsthm}

% For dealing with references we use the comment environment
\usepackage{verbatim}
\newenvironment{reference}{\comment}{\endcomment}
%\newenvironment{reference}{}{}
\newenvironment{slogan}{\comment}{\endcomment}
\newenvironment{history}{\comment}{\endcomment}

% For commutative diagrams you can use
% \usepackage{amscd}
\usepackage[all]{xy}

% We use 2cell for 2-commutative diagrams.
\xyoption{2cell}
\UseAllTwocells

% To put source file link in headers.
% Change "template.tex" to "this_filename.tex"
% \usepackage{fancyhdr}
% \pagestyle{fancy}
% \lhead{}
% \chead{}
% \rhead{Source file: \url{template.tex}}
% \lfoot{}
% \cfoot{\thepage}
% \rfoot{}
% \renewcommand{\headrulewidth}{0pt}
% \renewcommand{\footrulewidth}{0pt}
% \renewcommand{\headheight}{12pt}

\usepackage{multicol}

% For cross-file-references
\usepackage{xr-hyper}

% Package for hypertext links:
\usepackage{hyperref}

% For any local file, say "hello.tex" you want to link to please
% use \externaldocument[hello-]{hello}
\externaldocument[introduction-]{introduction}
\externaldocument[conventions-]{conventions}
\externaldocument[sets-]{sets}
\externaldocument[categories-]{categories}
\externaldocument[topology-]{topology}
\externaldocument[sheaves-]{sheaves}
\externaldocument[sites-]{sites}
\externaldocument[stacks-]{stacks}
\externaldocument[fields-]{fields}
\externaldocument[algebra-]{algebra}
\externaldocument[brauer-]{brauer}
\externaldocument[homology-]{homology}
\externaldocument[derived-]{derived}
\externaldocument[simplicial-]{simplicial}
\externaldocument[more-algebra-]{more-algebra}
\externaldocument[smoothing-]{smoothing}
\externaldocument[modules-]{modules}
\externaldocument[sites-modules-]{sites-modules}
\externaldocument[injectives-]{injectives}
\externaldocument[cohomology-]{cohomology}
\externaldocument[sites-cohomology-]{sites-cohomology}
\externaldocument[dga-]{dga}
\externaldocument[dpa-]{dpa}
\externaldocument[hypercovering-]{hypercovering}
\externaldocument[schemes-]{schemes}
\externaldocument[constructions-]{constructions}
\externaldocument[properties-]{properties}
\externaldocument[morphisms-]{morphisms}
\externaldocument[coherent-]{coherent}
\externaldocument[divisors-]{divisors}
\externaldocument[limits-]{limits}
\externaldocument[varieties-]{varieties}
\externaldocument[topologies-]{topologies}
\externaldocument[descent-]{descent}
\externaldocument[perfect-]{perfect}
\externaldocument[more-morphisms-]{more-morphisms}
\externaldocument[flat-]{flat}
\externaldocument[groupoids-]{groupoids}
\externaldocument[more-groupoids-]{more-groupoids}
\externaldocument[etale-]{etale}
\externaldocument[chow-]{chow}
\externaldocument[intersection-]{intersection}
\externaldocument[pic-]{pic}
\externaldocument[adequate-]{adequate}
\externaldocument[dualizing-]{dualizing}
\externaldocument[duality-]{duality}
\externaldocument[discriminant-]{discriminant}
\externaldocument[local-cohomology-]{local-cohomology}
\externaldocument[curves-]{curves}
\externaldocument[resolve-]{resolve}
\externaldocument[models-]{models}
\externaldocument[pione-]{pione}
\externaldocument[etale-cohomology-]{etale-cohomology}
\externaldocument[proetale-]{proetale}
\externaldocument[crystalline-]{crystalline}
\externaldocument[spaces-]{spaces}
\externaldocument[spaces-properties-]{spaces-properties}
\externaldocument[spaces-morphisms-]{spaces-morphisms}
\externaldocument[decent-spaces-]{decent-spaces}
\externaldocument[spaces-cohomology-]{spaces-cohomology}
\externaldocument[spaces-limits-]{spaces-limits}
\externaldocument[spaces-divisors-]{spaces-divisors}
\externaldocument[spaces-over-fields-]{spaces-over-fields}
\externaldocument[spaces-topologies-]{spaces-topologies}
\externaldocument[spaces-descent-]{spaces-descent}
\externaldocument[spaces-perfect-]{spaces-perfect}
\externaldocument[spaces-more-morphisms-]{spaces-more-morphisms}
\externaldocument[spaces-flat-]{spaces-flat}
\externaldocument[spaces-groupoids-]{spaces-groupoids}
\externaldocument[spaces-more-groupoids-]{spaces-more-groupoids}
\externaldocument[bootstrap-]{bootstrap}
\externaldocument[spaces-pushouts-]{spaces-pushouts}
\externaldocument[groupoids-quotients-]{groupoids-quotients}
\externaldocument[spaces-more-cohomology-]{spaces-more-cohomology}
\externaldocument[spaces-simplicial-]{spaces-simplicial}
\externaldocument[formal-spaces-]{formal-spaces}
\externaldocument[restricted-]{restricted}
\externaldocument[spaces-resolve-]{spaces-resolve}
\externaldocument[formal-defos-]{formal-defos}
\externaldocument[defos-]{defos}
\externaldocument[cotangent-]{cotangent}
\externaldocument[examples-defos-]{examples-defos}
\externaldocument[algebraic-]{algebraic}
\externaldocument[examples-stacks-]{examples-stacks}
\externaldocument[stacks-sheaves-]{stacks-sheaves}
\externaldocument[criteria-]{criteria}
\externaldocument[artin-]{artin}
\externaldocument[quot-]{quot}
\externaldocument[stacks-properties-]{stacks-properties}
\externaldocument[stacks-morphisms-]{stacks-morphisms}
\externaldocument[stacks-limits-]{stacks-limits}
\externaldocument[stacks-cohomology-]{stacks-cohomology}
\externaldocument[stacks-perfect-]{stacks-perfect}
\externaldocument[stacks-introduction-]{stacks-introduction}
\externaldocument[stacks-more-morphisms-]{stacks-more-morphisms}
\externaldocument[stacks-geometry-]{stacks-geometry}
\externaldocument[moduli-]{moduli}
\externaldocument[moduli-curves-]{moduli-curves}
\externaldocument[examples-]{examples}
\externaldocument[exercises-]{exercises}
\externaldocument[guide-]{guide}
\externaldocument[desirables-]{desirables}
\externaldocument[coding-]{coding}
\externaldocument[obsolete-]{obsolete}
\externaldocument[fdl-]{fdl}
\externaldocument[index-]{index}

% Theorem environments.
%
\theoremstyle{plain}
\newtheorem{theorem}[subsection]{Theorem}
\newtheorem{proposition}[subsection]{Proposition}
\newtheorem{lemma}[subsection]{Lemma}

\theoremstyle{definition}
\newtheorem{definition}[subsection]{Definition}
\newtheorem{example}[subsection]{Example}
\newtheorem{exercise}[subsection]{Exercise}
\newtheorem{situation}[subsection]{Situation}

\theoremstyle{remark}
\newtheorem{remark}[subsection]{Remark}
\newtheorem{remarks}[subsection]{Remarks}

\numberwithin{equation}{subsection}

% Macros
%
\def\lim{\mathop{\rm lim}\nolimits}
\def\colim{\mathop{\rm colim}\nolimits}
\def\Spec{\mathop{\rm Spec}}
\def\Hom{\mathop{\rm Hom}\nolimits}
\def\Ext{\mathop{\rm Ext}\nolimits}
\def\SheafHom{\mathop{\mathcal{H}\!{\it om}}\nolimits}
\def\SheafExt{\mathop{\mathcal{E}\!{\it xt}}\nolimits}
\def\Sch{\textit{Sch}}
\def\Mor{\mathop{\rm Mor}\nolimits}
\def\Ob{\mathop{\rm Ob}\nolimits}
\def\Sh{\mathop{\textit{Sh}}\nolimits}
\def\NL{\mathop{N\!L}\nolimits}
\def\proetale{{pro\text{-}\acute{e}tale}}
\def\etale{{\acute{e}tale}}
\def\QCoh{\textit{QCoh}}
\def\Ker{\mathop{\rm Ker}}
\def\Im{\mathop{\rm Im}}
\def\Coker{\mathop{\rm Coker}}
\def\Coim{\mathop{\rm Coim}}

%
% Macros for moduli stacks/spaces
%
\def\QCohstack{\mathcal{QC}\!{\it oh}}
\def\Cohstack{\mathcal{C}\!{\it oh}}
\def\Spacesstack{\mathcal{S}\!{\it paces}}
\def\Quotfunctor{{\rm Quot}}
\def\Hilbfunctor{{\rm Hilb}}
\def\Curvesstack{\mathcal{C}\!{\it urves}}
\def\Polarizedstack{\mathcal{P}\!{\it olarized}}
\def\Complexesstack{\mathcal{C}\!{\it omplexes}}
% \Pic is the operator that assigns to X its picard group, usage \Pic(X)
% \Picardstack_{X/B} denotes the Picard stack of X over B
% \Picardfunctor_{X/B} denotes the Picard functor of X over B
\def\Pic{\mathop{\rm Pic}\nolimits}
\def\Picardstack{\mathcal{P}\!{\it ic}}
\def\Picardfunctor{{\rm Pic}}
\def\Deformationcategory{\mathcal{D}\!{\it ef}}


% OK, start here.
%
\begin{document}

\title{Examples of Stacks}


\maketitle

\phantomsection
\label{section-phantom}

\tableofcontents

\section{Introduction}
\label{section-introduction}

\noindent
This is a discussion of examples of stacks in algebraic geometry.
Some of them are algebraic stacks, some are not.
We will discuss which are algebraic stacks in a later chapter.
This means that in this chapter we mainly worry about the descent
conditions. See \cite{Vis2} for example.




\section{Notation}
\label{section-notation}

\noindent
In this chapter we fix a suitable big fppf site $\textit{Sch}_{fppf}$
as in Topologies, Definition \ref{topologies-definition-big-fppf-site}.
So, if not explicitly stated otherwise all schemes will be objects
of $\textit{Sch}_{fppf}$.
We will always work relative to a base $S$ contained in $\textit{Sch}_{fppf}$.
And we will then work with the big fppf site $(\textit{Sch}/S)_{fppf}$,
see Topologies, Definition \ref{topologies-definition-big-small-fppf}.
The absolute case can be recovered by taking
$S = \text{Spec}(\mathbf{Z})$.
 









\section{Quasi-coherent sheaves}
\label{section-stack-of-quasi-coherent-sheaves}

\noindent
We define a category $\textit{QCoh}$ as follows:
\begin{enumerate}
\item An object of $\textit{QCoh}$ is a pair $(X, \mathcal{F})$,
where $X/S$ is an object of $(\textit{Sch}/S)_{fppf}$, and $\mathcal{F}$
is a quasi-coherent $\mathcal{O}_X$-module, and
\item a morphism $(f, \varphi) : (Y, \mathcal{G}) \to (X, \mathcal{F})$
is a pair consisting of a morphism $f : Y \to X$ of schemes over $S$
and an $f$-map (see
Sheaves, Section \ref{sheaves-section-ringed-spaces-functoriality-modules})
$\varphi : \mathcal{F} \to \mathcal{G}$.
\item The composition of morphisms
$$
(Z, \mathcal{H}) \xrightarrow{(g, \psi)}
(Y, \mathcal{G}) \xrightarrow{(f, \phi)} (X, \mathcal{F})
$$
is $(f \circ g, \psi \circ \phi)$ where $\psi \circ \phi$ is
the composition of $f$-maps.
\end{enumerate}
Thus $\textit{QCoh}$ is a category and
$$
p : \textit{QCoh} \to (\textit{Sch}/S)_{fppf},
\quad
(X, \mathcal{F}) \mapsto X
$$
is a functor. Note that the fibre category of $\textit{QCoh}$ over
a scheme $X$ is just the category $\textit{QCoh}(X)$
of quasi-coherent $\mathcal{O}_X$-modules.
We remark for later use that given
$(X, \mathcal{F}), (Y, \mathcal{G}) \in \text{Ob}(\textit{QCoh})$
we have
\begin{equation}
\label{equation-morphisms-qcoh}
\text{Mor}_{\textit{QCoh}}((Y, \mathcal{G}), (X, \mathcal{F}))
=
\coprod\nolimits_{f \in \text{Mor}_S(Y, X)}
\text{Mor}_{\textit{QCoh}(Y)}(f^*\mathcal{F}, \mathcal{G})
\end{equation}
See the discussion on $f$-maps of modules in
Sheaves, Section \ref{sheaves-section-ringed-spaces-functoriality-modules}.

\medskip\noindent
The category $\textit{QCoh}$ is not a stack over $(\textit{Sch}/S)_{fppf}$
because its collection of objects is a proper class. On the other hand
we will see that it does satisfy all the axioms of a stack. We will
get around the set theoretical issue in
Section \ref{section-stack-of-finitely-generated-quasi-coherent-sheaves}.

\begin{lemma}
\label{lemma-quasi-coherent-strongly-cartesian}
A morphism $(f, \varphi) : (Y, \mathcal{G}) \to (X, \mathcal{F})$
of $\textit{QCoh}$ is strongly cartesian if and only if the
map $\varphi$ induces an isomorphism $f^*\mathcal{F} \to \mathcal{G}$.
\end{lemma}

\begin{proof}
Let $(X, \mathcal{F}) \in \text{Ob}(\textit{QCoh})$.
Let $f : Y \to X$ be a morphism of $(\textit{Sch}/S)_{fppf}$.
Note that there is a canonical $f$-map $c : \mathcal{F} \to f^*\mathcal{F}$
and hence we get a morphism
$(f, c) : (Y, f^*\mathcal{F}) \to (X, \mathcal{F})$.
We claim that $(f, c)$ is strongly cartesian.
Namely, for any object $(Z, \mathcal{H})$ of $\textit{QCoh}$ we have
\begin{align*}
\text{Mor}_{\textit{QCoh}}((Z, \mathcal{H}), (Y, f^*\mathcal{F}))
& =
\coprod\nolimits_{g \in \text{Mor}_S(Z, Y)}
\text{Mor}_{\textit{QCoh}(Z)}(g^*f^*\mathcal{F}, \mathcal{H}) \\
& =
\coprod\nolimits_{g \in \text{Mor}_S(Z, Y)}
\text{Mor}_{\textit{QCoh}(Z)}((f \circ g)^*\mathcal{F}, \mathcal{H}) \\
& =
\text{Mor}_{\textit{QCoh}}((Z, \mathcal{H}), (X, \mathcal{F}))
\times_{\text{Mor}_S(Z, X)} \text{Mor}_S(Z, Y)
\end{align*}
where we have used Equation (\ref{equation-morphisms-qcoh}) twice.
This proves that the condition of
Categories, Definition \ref{categories-definition-cartesian-over-C}
holds for $(f, c)$, and hence our claim is true. Now by
Categories, Lemma \ref{categories-lemma-composition-cartesian}
we see that isomorphisms are strongly cartesian and
compositions of strongly cartesian morphisms are strongly cartesian
which proves the ``if'' part of the lemma. For the converse, note
that given $(X, \mathcal{F})$ and $f : Y \to X$, if there exists a
strongly cartesian morphism lifting $f$ with target $(X, \mathcal{F})$
then it has to be isomorphic to $(f, c)$ (see discussion following
Categories, Definition \ref{categories-definition-cartesian-over-C}).
Hence the "only if" part of the lemma holds.
\end{proof}

\begin{lemma}
\label{lemma-stack-of-quasi-coherent-sheaves}
The functor $p : \textit{QCoh} \to (\textit{Sch}/S)_{fppf}$
satisfies conditions (1), (2) and (3) of
Stacks, Definition \ref{stacks-definition-stack}.
\end{lemma}

\begin{proof}
It is clear from
Lemma \ref{lemma-quasi-coherent-strongly-cartesian}
that $\textit{QCoh}$ is a fibred category over $(\textit{Sch}/S)_{fppf}$.
Given covering $\mathcal{U} = \{X_i \to X\}_{i \in I}$ of
$(\textit{Sch}/S)_{fppf}$ the functor
$$
\textit{QCoh}(T) \longrightarrow DD(\mathcal{U})
$$
is fully faithful and essentially surjective, see
Descent, Proposition \ref{descent-proposition-fpqc-descent-quasi-coherent}.
Hence
Stacks, Lemma \ref{stacks-lemma-stack-equivalences}
applies to show that $\textit{QCoh}$ satisfies all the
axioms of a stack.
\end{proof}





\section{The stack of finitely generated quasi-coherent sheaves}
\label{section-stack-of-finitely-generated-quasi-coherent-sheaves}

\noindent
It turns out that we can get a stack of quasi-coherent sheaves
if we only consider finite type quasi-coherent modules.
Let us denote
$$
p_{fg} : \textit{QCoh}_{fg} \to (\textit{Sch}/S)_{fppf}
$$
the full subcategory of $\textit{QCoh}$ over $(\textit{Sch}/S)_{fppf}$ 
consisting of pairs $(T, \mathcal{F})$ such that $\mathcal{F}$
is a quasi-coherent $\mathcal{O}_T$-module of finite type.

\begin{lemma}
\label{lemma-stack-of-finite-type-quasi-coherent-sheaves}
The functor $p_{fg} : \textit{QCoh}_{fg} \to (\textit{Sch}/S)_{fppf}$
satisfies conditions (1), (2) and (3) of
Stacks, Definition \ref{stacks-definition-stack}.
\end{lemma}

\begin{proof}
We will verify assumptions (1), (2), (3) of
Stacks, Lemma \ref{stacks-lemma-substack}
to prove this. By
Lemma \ref{lemma-quasi-coherent-strongly-cartesian}
a morphism $(Y, \mathcal{G}) \to (X, \mathcal{F})$ is
strongly cartesian if and only if it induces an isomorphism
$f^*\mathcal{F} \to \mathcal{G}$. By
Modules, Lemma \ref{modules-lemma-pullback-finite-type}
the pullback of a finite type $\mathcal{O}_X$-module is of finite
type. Hence assumption (1) of
Stacks, Lemma \ref{stacks-lemma-substack}
holds. Assumption (2) holds trivially.
Finally, to prove assumption (3) we have to show:
If $\mathcal{F}$ is a quasi-coherent $\mathcal{O}_X$-module
and $\{f_i : X_i \to X\}$ is an fppf covering such that each
$f_i^*\mathcal{F}$ is of finite type, then $\mathcal{F}$ is of
finite type. Considering the restriction of $\mathcal{F}$ to
an affine open of $X$ this reduces to the following algebra statement:
Suppose that $R \to S$ is a finitely presented, faithfully flat ring map
and $M$ an $R$-module. If $M \otimes_R S$ is a finitely generated
$S$-module, then $M$ is a finitely generated $R$-module.
A stronger form of the algebra fact can be found in
Algebra, Lemma \ref{algebra-lemma-descend-properties-modules}.
\end{proof}

\begin{lemma}
\label{lemma-finite-type}
Let $(X, \mathcal{O}_X)$ be a ringed space.
\begin{enumerate}
\item The category of finite type $\mathcal{O}_X$-modules has a
set of isomorphism classes.
\item The category of finite type quasi-coherent
$\mathcal{O}_X$-modules has a set of isomorphism classes.
\end{enumerate}
\end{lemma}

\begin{proof}
Part (2) follows from part (1) as the category in (2) is a full subcategory
of the category in (1). Consider any open covering
$\mathcal{U} : X = \bigcup_{i \in I} U_i$. Denote $j_i : U_i \to X$ the inclusion
maps. Consider any map $r : I \to \mathbf{N}$.
If $\mathcal{F}$ is an $\mathcal{O}_X$-module whose restriction to
$U_i$ is generated by at most $r(i)$ sections from $\mathcal{F}(U_i)$,
then $\mathcal{F}$ is a quotient of the sheaf
$$
\mathcal{H}_{\mathcal{U}, r} =
\bigoplus\nolimits_{i \in I} j_{i, !}\mathcal{O}_{U_i}^{\oplus r(i)}
$$
By definition, if $\mathcal{F}$ is of finite type, then there exists
some open covering with $\mathcal{U}$ whose index set is $I = X$
such that this condition is true. Hence it suffices to show that
there is a set of possible choices for $\mathcal{U}$ (obvious),
a set of possible choices for $r : I \to \mathbf{N}$ (obvious), and
a set of possible quotient modules of $\mathcal{H}_{\mathcal{U}, r}$
for each $\mathcal{U}$ and $r$. In other words, it suffices to show
that given an $\mathcal{O}_X$-module $\mathcal{H}$ there is at most
a set of isomorphism classes of quotients.
This last assertion becomes obvious
by thinking of the kernels of a quotient map
$\mathcal{H} \to \mathcal{F}$
as being parametrized by a subset of the power set of
$\prod_{U \subset X\text{ open}} \mathcal{H}(U)$.
\end{proof}

\begin{lemma}
\label{lemma-stack-fg-quasi-coherent}
There exists a subcategory
$\textit{QCoh}_{fg, small} \subset \textit{QCoh}_{fg}$
with the following properties:
\begin{enumerate}
\item the inclusion functor
$\textit{QCoh}_{fg, small} \to \textit{QCoh}_{fg}$ is
fully faithful and essentially surjective, and
\item the functor
$p_{fg, small} : \textit{QCoh}_{fg, small} \to (\textit{Sch}/S)_{fppf}$
turns $\textit{QCoh}_{fg, small}$ into a stack over $(\textit{Sch}/S)_{fppf}$.
\end{enumerate}
\end{lemma}

\begin{proof}
We have seen in
Lemmas \ref{lemma-stack-of-finite-type-quasi-coherent-sheaves} and
\ref{lemma-finite-type}
that $p_{fg} : \textit{QCoh}_{fg} \to (\textit{Sch}/S)_{fppf}$
satisfies (1), (2) and (3) of
Stacks, Definition \ref{stacks-definition-stack}
as well as the additional condition (4) of
Stacks, Remark \ref{stacks-remark-stack-make-small}.
Hence we obtain $\textit{QCoh}_{fg, small}$ from the discussion
in that remark.
\end{proof}

\noindent
We will often perform the replacement
$$
\textit{QCoh}_{fg} \leadsto \textit{QCoh}_{fg, small}
$$
without further remarking on it, and by abuse of notation we will
simply denote $\textit{QCoh}_{fg}$ this replacement.

\begin{remark}
\label{remark-higher-rank}
Note that the whole discussion in this section works
if we want to consider those
quasi-coherent sheaves which are locally generated by at most $\kappa$
sections, for some infinite cardinal $\kappa$, e.g., $\kappa = \aleph_0$.
\end{remark}




\section{The stack of algebraic spaces}
\label{section-stack-of-spaces}

\noindent
We define a category $\textit{Spaces}$ as follows:
\begin{enumerate}
\item An object of $\textit{Spaces}$ is a morphism $X \to U$
of algebraic spaces over $S$, where $U$ is representable by an object of
$(\textit{Sch}/S)_{fppf}$, and
\item a morphism $(f, g) : (X \to U) \to (Y \to V)$
is a commutative diagram
$$
\xymatrix{
X \ar[d] \ar[r]_f & Y \ar[d] \\
U \ar[r]^g & V
}
$$
of morphisms of algebraic spaces over $S$.
\end{enumerate}
Thus $\textit{Spaces}$ is a category and
$$
p : \textit{Spaces} \to (\textit{Sch}/S)_{fppf},
\quad
(X \to U) \mapsto U
$$
is a functor. Note that the fibre category of $\textit{Spaces}$ over
a scheme $U$ is just the category $\textit{Spaces}/U$ of
algebraic spaces over $U$ (see
Topologies on Spaces, Section \ref{spaces-topologies-section-procedure}).
Hence we sometimes think of an object of $\textit{Spaces}$ as a
pair $X/U$ consisting of a scheme $U$ and an algebraic space $X$ over $U$.
We remark for later use that given
$(X/U), (Y/V) \in \text{Ob}(\textit{Spaces})$
we have
\begin{equation}
\label{equation-morphisms-spaces}
\text{Mor}_{\textit{Spaces}}(X/U, Y/V)
=
\coprod\nolimits_{g \in \text{Mor}_S(U, V)}
\text{Mor}_{\textit{Spaces}/U}(X, U \times_{g, V} Y)
\end{equation}
The category $\textit{Spaces}$ is almost, but not quite a stack
over $(\textit{Sch}/S)_{fppf}$. The problem is a set theoretical
issue as we will explain below.

\begin{lemma}
\label{lemma-spaces-strongly-cartesian}
A morphism $(f, g) : (X \to U) \to (Y \to V)$
of $\textit{Spaces}$ is strongly cartesian if and only if the
map $f$ induces an isomorphism $X \to U \times_{g, V} Y$.
\end{lemma}

\begin{proof}
Let $Y/V \in \text{Ob}(\textit{Spaces})$.
Let $g : U \to V$ be a morphism of $(\textit{Sch}/S)_{fppf}$.
Note that the projection $p : U \times_{g, V} Y \to Y$
gives rise a morphism
$(p, g) : U \times_{g, V} Y/U \to Y/V$ of $\textit{Spaces}$.
We claim that $(p, g)$ is strongly cartesian.
Namely, for any object $Z/W$ of $\textit{Spaces}$ we have
\begin{align*}
\text{Mor}_{\textit{Spaces}}(Z/W, U \times_{g, V} Y/U)
& =
\coprod\nolimits_{h \in \text{Mor}_S(W, U)}
\text{Mor}_{\textit{Spaces}/W}(Z, W \times_{h, U} U \times_{g, V} Y) \\
& =
\coprod\nolimits_{h \in \text{Mor}_S(W, U)}
\text{Mor}_{\textit{Spaces}/W}(Z, W \times_{g \circ h, V} Y) \\
& =
\text{Mor}_{\textit{Spaces}}(Z/W, Y/V)
\times_{\text{Mor}_S(W, V)} \text{Mor}_S(W, U)
\end{align*}
where we have used Equation (\ref{equation-morphisms-spaces}) twice.
This proves that the condition of
Categories, Definition \ref{categories-definition-cartesian-over-C}
holds for $(p, g)$, and hence our claim is true. Now by
Categories, Lemma \ref{categories-lemma-composition-cartesian}
we see that isomorphisms are strongly cartesian and
compositions of strongly cartesian morphisms are strongly cartesian
which proves the ``if'' part of the lemma. For the converse, note
that given $Y/V$ and $g : U \to V$, if there exists a
strongly cartesian morphism lifting $g$ with target $Y/V$
then it has to be isomorphic to $(p, g)$ (see discussion following
Categories, Definition \ref{categories-definition-cartesian-over-C}).
Hence the "only if" part of the lemma holds.
\end{proof}

\begin{lemma}
\label{lemma-pre-stack-of-spaces}
The functor $p : \textit{Spaces} \to (\textit{Sch}/S)_{fppf}$
satisfies conditions (1) and (2) of
Stacks, Definition \ref{stacks-definition-stack}.
\end{lemma}

\begin{proof}
It is follows from
Lemma \ref{lemma-spaces-strongly-cartesian}
that $\textit{Spaces}$ is a fibred category over $(\textit{Sch}/S)_{fppf}$
which proves (1).
Suppose that $\{U_i \to U\}_{i \in I}$ is a covering of
$(\textit{Sch}/S)_{fppf}$. Suppose that $X, Y$ are algebraic spaces over
$U$. Finally, suppose that $\varphi_i : X_{U_i} \to Y_{U_i}$ are morphisms
of $\textit{Spaces}/U_i$ such that $\varphi_i$ and $\varphi_j$ restrict
to the same morphisms $X_{U_i \times_U U_j} \to Y_{U_i \times_U U_j}$
of algebraic spaces over $U_i \times_U U_j$. 
To prove (2) we have to show that there exists a unique morphism
$\varphi  : X \to Y$ over $U$ whose base change to $U_i$ is
equal to $\varphi_i$. As a morphism from $X$ to $Y$ is the same thing
as a map of sheaves this follows directly from
Sites, Lemma \ref{sites-lemma-glue-maps}.
\end{proof}











\section{Examples of stacks in groupoids}
\label{section-examples-stacks}

\noindent
The examples above are examples of stacks which are not stacks in
groupoids. In the rest of this chapter we give
algebraic geometric examples of stacks in groupoids.



\section{The stack associated to a sheaf}
\label{section-stack-associated-to-sheaf}

\noindent
Let $F : (\textit{Sch}/S)_{fppf}^{opp} \to \textit{Sets}$ be a presheaf.
In this case the category fibred in sets
$p_F : \mathcal{S}_F \to (\textit{Sch}/S)_{fppf}$
is a stack in sets if and only if $F$ is a sheaf, see
Stacks, Lemma \ref{stacks-lemma-stack-in-setoids-characterize}.



\section{The stack in groupoids of finitely generated quasi-coherent sheaves}
\label{section-stack-in-groupoids-of-quasi-coherent-sheaves}

\noindent
Let $p : \textit{QCoh}_{fg} \to (\textit{Sch}/S)_{fppf}$ be the stack
introduced in
Section \ref{section-stack-of-finitely-generated-quasi-coherent-sheaves}
(using the abuse of notation introduced there).
We can turn this into a stack in groupoids
$p' : \textit{QCoh}_{fg}' \to (\textit{Sch}/S)_{fppf}$ by
the procedure of
Categories, Lemma \ref{categories-lemma-fibred-gives-fibred-groupoids},
see
Stacks, Lemma \ref{stacks-lemma-stack-gives-stack-groupoids}.
In this particular case this simply means $\textit{QCoh}_{fg}'$ has
the same objects as $\textit{QCoh}_{fg}$ but the morphsms are
morphisms $(f, \varphi) : (T, \mathcal{F}) \to (T', \mathcal{F}')$
such that $\varphi$ induces an isomorphism $f^*\mathcal{F}' \to \mathcal{F}$.






\section{Stacks in groupoids classifying torsors}
\label{section-torsors}

\medskip\noindent
In the following example we use $G$-torsors; these are defined in
Groupoids, Definition \ref{groupoids-definition-principal-homogeneous-space}.

\begin{example}
\label{example-X-mod-G-groupscheme}
Let $S$ be a scheme.
Let $G \to S$ be a group scheme over $S$.
Assume that $G \to S$ is affine.
Let $X \to S$ be a scheme over $S$.
Let $a : G \times_S X \to X$
be an action of $G$ on $X$ over $S$.
The {\it quotient stack} $[X/G]$ is defined as follows.
\begin{enumerate}
\item An object of $[X/G]$ consists of a triple
$(T \to S, P \to T, \varphi : P \to X)$ where
\begin{enumerate}
\item $T \to S$ is a scheme over $S$,
\item $P \to T$ is a $G_T$-torsor in fppf topology\footnote{We can also
consider here instead all $G_T$-torsors, i.e., those pseudo torsors which
become trivial over the members of an fpqc covering of $T$. This would lead
to a second stack in groupoids $[X/G]'$ over $(\textit{Sch}/S)_{fppf}$.
In general the two stacks obtained do not agree, see
Examples, Section \ref{examples-section-torsor-not-fppf}.
But if $G \to S$ is flat and of finite presentation,
then they do (insert future reference here). If $G$ is not
of finite type over $S$, then neither $[X/G]'$ nor $[X/G]$ will be
an algebraic stack in general, see
Examples, Section \ref{examples-section-not-algebraic-stack}.
The upshot is that in many interesting cases the difference is irrelevant.}
over $T$, and
\item $\varphi : P \to X$ is a $G$-equivariant morphism of schemes.
\end{enumerate}
\item A morphism
$(h, f) : (T/S, P/T, \varphi) \to (T'/S, P'/T', \varphi')$
is given by a morphism of schemes $f : T \to T'$ and a $G$-equivariant
morphism $h : P \to P'$ over $f$ which induces an isomorphism
$P \cong T \times_{T'} P'$, and such that $\varphi = \varphi' \circ h$.
\item the functor $[X/G] \to (\textit{Sch}/S)_{fppf}$ is the forgetful
functor $(T/S, P/T, \varphi) \mapsto T/S$.
\end{enumerate}
It is not so hard to verify that this is a fibred category. (Omitted.)
This is a stack in groupoids for the fppf topology. We omit the proof.
The hardest part is to show descent for objects. To do this one shows that
a torsor for an affine group scheme is affine over its base (insert future
reference here), and we have descent for affine schemes over schemes (see
Descent, Lemma \ref{descent-lemma-affine}).
\end{example}

\noindent
The following example is almost, but not quite the same.
The difference is that in the example above $G$ is an affine group scheme
over $S$, and in the example below $G$ is an abstract group.
The key to the examples is that in both cases all
$\underline{G}$-torsors on $(\textit{Sch}/S)_{fppf}$ are representable.

\begin{example}
\label{example-X-mod-G-group}
Let $S$ be a scheme.
Let $G$ be a group.
Let $X \to S$ be a scheme over $S$.
Let $a : G \times X \to X$ be an action of $G$ on $X$ over $S$.
The stack $[X/G]$ is defined as follows.
\begin{enumerate}
\item An object of $[X/G]$ consists of a triple
$(T \to S, P \to T, \varphi : P \to X)$ where
\begin{enumerate}
\item $T \to S$ is a scheme over $S$,
\item $P \to T$ is a $G_T$-torsor over $T$ in the fppf topology, and
\item $\varphi : P \to X$ is a $G$-equivariant morphism of schemes.
\end{enumerate}
\item A morphism
$(h, f) : (T/S, P/T, \varphi) \to (T'/S, P'/T', \varphi')$
is given by a morphism of schemes $f : T \to T'$ and a $G$-equivariant
morphism $h : P \to P'$ over $f$ which induces an isomorphism
$P \cong T \times_{T'} P'$, and such that $\varphi = \varphi' \circ h$.
\item the functor $[X/G] \to (\textit{Sch}/S)_{fppf}$ is the forgetful
functor $(T/S, P/T, \varphi) \mapsto T/S$.
\end{enumerate}
It is not so hard to verify that this is a fibred category. (Omitted.)
This is a stack in groupoids for the fppf topology. We omit the proof.
The hardest part is to show descent for objects.
To do this, note that a torsor for an abstract group
is a separated and etale scheme over its base, and by
More on Morphisms, Lemma
\ref{more-morphisms-lemma-separated-locally-quasi-finite-morphisms-fppf-descend}
we have descent for separated and locally quasi-finite schemes over schemes.
\end{example}



\section{Picard stacks}
\label{section-picard-stack}

\noindent
Let $f : X \to S$ be a morphism of schemes.
The {\it Picard stack} $\textit{Pic}_{X/S}$ is defined as follows
\begin{enumerate}
\item An object is a pair $(T/S, \mathcal{L})$, where $T/S$ is a
scheme over $S$ and $\mathcal{L}$ is in invertible sheaf on
the base change $X_T$,
\item a morphism $(f, \alpha) : (T/S, \mathcal{L}) \to (T'/S, \mathcal{L}')$
is given by a morphism of schemes $f : T \to T'$ over $S$ and an
isomorphism $\alpha : f^*\mathcal{L}' \to \mathcal{L}$, and
\item the structure functor $\textit{Pic}_{X/S} \to (\textit{Sch}/S)_{fppf}$
is given by the forgetful functor $(T/S, \mathcal{L}) \mapsto T/S$.
\end{enumerate}
It is not hard to verify that this is a fibred category (omitted).
In fact it is a stack in groupoids for the fppf topology.
As usual, the hardest part is to show descent for objects.
But, if $\{T_i \to T\}$ is an fppf covering for $T$, then the
pullbacks $\{X_{T_i} \to X_T\}$ form an fppf covering as well.
Hence given invertible sheaves $\mathcal{L}_i$ on the schemes
$X_{T_i}$ and isomorphisms between their pullbacks over the schemes
$$
X_{T_i \times_T T_j} = X_{T_i} \times_{X_T} X_{T_i}
$$
satisfying a suitable cocycle condition we can descend the
$\mathcal{L}_i$ to an invertible sheaf $\mathcal{L}$ over $X_T$ using
Descent, Proposition \ref{descent-proposition-fpqc-descent-quasi-coherent}. 
Details omitted.


\section{Examples of inertia stacks}
\label{section-examples-inertia}

\noindent
Here are some examples of inertia stacks.

\begin{example}
\label{example-inertia-stack-of-X-mod-G}
Let $S$ be a scheme. Let $G$ be a commutative group.
Let $X \to S$ be a scheme over $S$.
Let $a : G \times X \to X$ be an action of $G$ on $X$.
For $g \in G$ we denote $g : X \to X$ the corresponding automorphism.
In this case the inertia stack of $[X/G]$
(see Example \ref{example-X-mod-G-group})
is given by
$$	
I_{[X/G]} = \coprod\nolimits_{g\in G} [X^g/G],
$$	
where, given an element $g$ of $G$, the symbol $X^g$ denotes the
scheme $X^g = \{x \in X \mid g(x) = x\}$. In a formula
$X^g$ is really the fibre
product
$$
X^g =  X \times_{(1, 1), X \times_S X, (g, 1)} X.
$$ 
Indeed, for any $S$-scheme $T$, a
$T$-point on the inertia stack of $[X/G]$ consists of a
triple $(P/T, \phi, \alpha)$ consisting of a $G$-torsor
$P\to T$ together with a $G$-equivariant isomorphism
$\phi : P \to X$, together
with an automorphism $\alpha$ of $P\to T$ over $T$ such that
$\phi \circ \alpha = \phi$.
Since $G$ is a sheaf of \emph{commutative} groups,
$\alpha$ is, locally in the fppf topology over $T$,
given by multiplication by some element $g$ of $G$.
The condition that $\phi \circ \alpha = \phi$ means that $\phi$
factors through the inclusion of $X^g$
in $X$, i.e., $\phi$ is obtained by composing that inclusion with a
morphism $P \to X^\gamma$.
The above discussion allows us to define a morphism of fibred categories
$I_{[X/G]} \to \coprod_{g\in G} [X^g/G]$ given on $T$-points by the discussion
above. We omit showing that this is an equivalence.
\end{example}

\begin{example}
\label{example-inertia-stack-of-picard}
Let $X\to S$ be a morphism of schemes.
Assume that for any $T \to S$ the base change $f_T : X_T \to T$
has the property that the map $\mathcal{O}_T \to f_{T, *}\mathcal{O}_{X_T}$
is an isomorphism. (This implies that $f$ is
{\it cohomologically flat in dimension $0$} (insert future reference here)
but is stronger.) Consider the Picard stack $\textit{Pic}_{X/S}$, see
Section \ref{section-picard-stack}.
The points of its inertia stack over an
$S$-scheme $T$ consist of pairs $(\mathcal{L}, \alpha)$
where $\mathcal{L}$ is a line bundle
on $X_T$ and $\alpha$ is an automorphism of that line bundle.
I.e., we can think of $\alpha$ as an element of
$H^0(X_T, \mathcal{O}_{X_T})^\times = H^0(T, \mathcal{O}_T^*)$
by our condition. Note that $H^0(T,\mathcal{O}_T^*) = \mathbf{G}_{m,S}(T)$,
see Groupoids, Example \ref{groupoids-example-multiplicative-group}.
Hence the inertia stack of $\textit{Pic}_{X/S}$ is
$$
I_{\textit{Pic}_{X/S}} = \mathbf{G}_{m,S} \times_S \textit{Pic}_{X/S}.
$$
as a stack over $(\textit{Sch}/S)_{fppf}$.
\end{example}

















\section{Other chapters}

\begin{multicols}{2}
\begin{enumerate}
\item \hyperref[introduction-section-phantom]{Introduction}
\item \hyperref[conventions-section-phantom]{Conventions}
\item \hyperref[sets-section-phantom]{Set Theory}
\item \hyperref[categories-section-phantom]{Categories}
\item \hyperref[topology-section-phantom]{Topology}
\item \hyperref[sheaves-section-phantom]{Sheaves on Spaces}
\item \hyperref[algebra-section-phantom]{Commutative Algebra}
\item \hyperref[sites-section-phantom]{Sites and Sheaves}
\item \hyperref[homology-section-phantom]{Homological Algebra}
\item \hyperref[derived-section-phantom]{Derived Categories}
\item \hyperref[more-algebra-section-phantom]{More Algebra}
\item \hyperref[simplicial-section-phantom]{Simplicial Methods}
\item \hyperref[modules-section-phantom]{Sheaves of Modules}
\item \hyperref[sites-modules-section-phantom]{Modules on Sites}
\item \hyperref[injectives-section-phantom]{Injectives}
\item \hyperref[cohomology-section-phantom]{Cohomology of Sheaves}
\item \hyperref[sites-cohomology-section-phantom]{Cohomology on Sites}
\item \hyperref[hypercovering-section-phantom]{Hypercoverings}
\item \hyperref[schemes-section-phantom]{Schemes}
\item \hyperref[constructions-section-phantom]{Constructions of Schemes}
\item \hyperref[properties-section-phantom]{Properties of Schemes}
\item \hyperref[morphisms-section-phantom]{Morphisms of Schemes}
\item \hyperref[coherent-section-phantom]{Coherent Cohomology}
\item \hyperref[divisors-section-phantom]{Divisors}
\item \hyperref[limits-section-phantom]{Limits of Schemes}
\item \hyperref[varieties-section-phantom]{Varieties}
\item \hyperref[chow-section-phantom]{Chow Homology}
\item \hyperref[topologies-section-phantom]{Topologies on Schemes}
\item \hyperref[descent-section-phantom]{Descent}
\item \hyperref[more-morphisms-section-phantom]{More on Morphisms}
\item \hyperref[flat-section-phantom]{More on Flatness}
\item \hyperref[groupoids-section-phantom]{Groupoid Schemes}
\item \hyperref[more-groupoids-section-phantom]{More on Groupoid Schemes}
\item \hyperref[etale-section-phantom]{\'Etale Morphisms of Schemes}
\item \hyperref[etale-cohomology-section-phantom]{\'Etale Cohomology}
\item \hyperref[spaces-section-phantom]{Algebraic Spaces}
\item \hyperref[spaces-properties-section-phantom]{Properties of Algebraic Spaces}
\item \hyperref[spaces-morphisms-section-phantom]{Morphisms of Algebraic Spaces}
\item \hyperref[spaces-topologies-section-phantom]{Topologies on Algebraic Spaces}
\item \hyperref[spaces-descent-section-phantom]{Descent and Algebraic Spaces}
\item \hyperref[spaces-more-morphisms-section-phantom]{More on Morphisms of Spaces}
\item \hyperref[quot-section-phantom]{Quot and Hilbert Spaces}
\item \hyperref[stacks-section-phantom]{Stacks}
\item \hyperref[spaces-groupoids-section-phantom]{Groupoids in Algebraic Spaces}
\item \hyperref[spaces-more-groupoids-section-phantom]{More on Groupoids in Spaces}
\item \hyperref[bootstrap-section-phantom]{Bootstrap}
\item \hyperref[examples-stacks-section-phantom]{Examples of Stacks}
\item \hyperref[groupoids-quotients-section-phantom]{Quotients of Groupoids}
\item \hyperref[algebraic-section-phantom]{Algebraic Stacks}
\item \hyperref[criteria-section-phantom]{Criteria for Representability}
\item \hyperref[stacks-properties-section-phantom]{Properties of Algebraic Stacks}
\item \hyperref[stacks-morphisms-section-phantom]{Morphisms of Algebraic Stacks}
\item \hyperref[examples-section-phantom]{Examples}
\item \hyperref[exercises-section-phantom]{Exercises}
\item \hyperref[guide-section-phantom]{Guide to Literature}
\item \hyperref[desirables-section-phantom]{Desirables}
\item \hyperref[coding-section-phantom]{Coding Style}
\item \hyperref[fdl-section-phantom]{GNU Free Documentation License}
\item \hyperref[index-section-phantom]{Auto Generated Index}
\end{enumerate}
\end{multicols}


\bibliography{my}
\bibliographystyle{amsalpha}

\end{document}
