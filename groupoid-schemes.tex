\IfFileExists{stacks-project.cls}{%
\documentclass{stacks-project}
}{%
\documentclass{amsart}
}

% The following AMS packages are automatically loaded with
% the amsart documentclass:
%\usepackage{amsmath}
%\usepackage{amssymb}
%\usepackage{amsthm}

% For dealing with references we use the comment environment
\usepackage{verbatim}
\newenvironment{reference}{\comment}{\endcomment}
%\newenvironment{reference}{}{}
\newenvironment{slogan}{\comment}{\endcomment}
\newenvironment{history}{\comment}{\endcomment}

% For commutative diagrams you can use
% \usepackage{amscd}
\usepackage[all]{xy}

% We use 2cell for 2-commutative diagrams.
\xyoption{2cell}
\UseAllTwocells

% To put source file link in headers.
% Change "template.tex" to "this_filename.tex"
% \usepackage{fancyhdr}
% \pagestyle{fancy}
% \lhead{}
% \chead{}
% \rhead{Source file: \url{template.tex}}
% \lfoot{}
% \cfoot{\thepage}
% \rfoot{}
% \renewcommand{\headrulewidth}{0pt}
% \renewcommand{\footrulewidth}{0pt}
% \renewcommand{\headheight}{12pt}

\usepackage{multicol}

% For cross-file-references
\usepackage{xr-hyper}

% Package for hypertext links:
\usepackage{hyperref}

% For any local file, say "hello.tex" you want to link to please
% use \externaldocument[hello-]{hello}
\externaldocument[introduction-]{introduction}
\externaldocument[conventions-]{conventions}
\externaldocument[sets-]{sets}
\externaldocument[categories-]{categories}
\externaldocument[topology-]{topology}
\externaldocument[sheaves-]{sheaves}
\externaldocument[sites-]{sites}
\externaldocument[stacks-]{stacks}
\externaldocument[fields-]{fields}
\externaldocument[algebra-]{algebra}
\externaldocument[brauer-]{brauer}
\externaldocument[homology-]{homology}
\externaldocument[derived-]{derived}
\externaldocument[simplicial-]{simplicial}
\externaldocument[more-algebra-]{more-algebra}
\externaldocument[smoothing-]{smoothing}
\externaldocument[modules-]{modules}
\externaldocument[sites-modules-]{sites-modules}
\externaldocument[injectives-]{injectives}
\externaldocument[cohomology-]{cohomology}
\externaldocument[sites-cohomology-]{sites-cohomology}
\externaldocument[dga-]{dga}
\externaldocument[dpa-]{dpa}
\externaldocument[hypercovering-]{hypercovering}
\externaldocument[schemes-]{schemes}
\externaldocument[constructions-]{constructions}
\externaldocument[properties-]{properties}
\externaldocument[morphisms-]{morphisms}
\externaldocument[coherent-]{coherent}
\externaldocument[divisors-]{divisors}
\externaldocument[limits-]{limits}
\externaldocument[varieties-]{varieties}
\externaldocument[topologies-]{topologies}
\externaldocument[descent-]{descent}
\externaldocument[perfect-]{perfect}
\externaldocument[more-morphisms-]{more-morphisms}
\externaldocument[flat-]{flat}
\externaldocument[groupoids-]{groupoids}
\externaldocument[more-groupoids-]{more-groupoids}
\externaldocument[etale-]{etale}
\externaldocument[chow-]{chow}
\externaldocument[intersection-]{intersection}
\externaldocument[pic-]{pic}
\externaldocument[adequate-]{adequate}
\externaldocument[dualizing-]{dualizing}
\externaldocument[duality-]{duality}
\externaldocument[discriminant-]{discriminant}
\externaldocument[local-cohomology-]{local-cohomology}
\externaldocument[curves-]{curves}
\externaldocument[resolve-]{resolve}
\externaldocument[models-]{models}
\externaldocument[pione-]{pione}
\externaldocument[etale-cohomology-]{etale-cohomology}
\externaldocument[proetale-]{proetale}
\externaldocument[crystalline-]{crystalline}
\externaldocument[spaces-]{spaces}
\externaldocument[spaces-properties-]{spaces-properties}
\externaldocument[spaces-morphisms-]{spaces-morphisms}
\externaldocument[decent-spaces-]{decent-spaces}
\externaldocument[spaces-cohomology-]{spaces-cohomology}
\externaldocument[spaces-limits-]{spaces-limits}
\externaldocument[spaces-divisors-]{spaces-divisors}
\externaldocument[spaces-over-fields-]{spaces-over-fields}
\externaldocument[spaces-topologies-]{spaces-topologies}
\externaldocument[spaces-descent-]{spaces-descent}
\externaldocument[spaces-perfect-]{spaces-perfect}
\externaldocument[spaces-more-morphisms-]{spaces-more-morphisms}
\externaldocument[spaces-flat-]{spaces-flat}
\externaldocument[spaces-groupoids-]{spaces-groupoids}
\externaldocument[spaces-more-groupoids-]{spaces-more-groupoids}
\externaldocument[bootstrap-]{bootstrap}
\externaldocument[spaces-pushouts-]{spaces-pushouts}
\externaldocument[groupoids-quotients-]{groupoids-quotients}
\externaldocument[spaces-more-cohomology-]{spaces-more-cohomology}
\externaldocument[spaces-simplicial-]{spaces-simplicial}
\externaldocument[formal-spaces-]{formal-spaces}
\externaldocument[restricted-]{restricted}
\externaldocument[spaces-resolve-]{spaces-resolve}
\externaldocument[formal-defos-]{formal-defos}
\externaldocument[defos-]{defos}
\externaldocument[cotangent-]{cotangent}
\externaldocument[examples-defos-]{examples-defos}
\externaldocument[algebraic-]{algebraic}
\externaldocument[examples-stacks-]{examples-stacks}
\externaldocument[stacks-sheaves-]{stacks-sheaves}
\externaldocument[criteria-]{criteria}
\externaldocument[artin-]{artin}
\externaldocument[quot-]{quot}
\externaldocument[stacks-properties-]{stacks-properties}
\externaldocument[stacks-morphisms-]{stacks-morphisms}
\externaldocument[stacks-limits-]{stacks-limits}
\externaldocument[stacks-cohomology-]{stacks-cohomology}
\externaldocument[stacks-perfect-]{stacks-perfect}
\externaldocument[stacks-introduction-]{stacks-introduction}
\externaldocument[stacks-more-morphisms-]{stacks-more-morphisms}
\externaldocument[stacks-geometry-]{stacks-geometry}
\externaldocument[moduli-]{moduli}
\externaldocument[moduli-curves-]{moduli-curves}
\externaldocument[examples-]{examples}
\externaldocument[exercises-]{exercises}
\externaldocument[guide-]{guide}
\externaldocument[desirables-]{desirables}
\externaldocument[coding-]{coding}
\externaldocument[obsolete-]{obsolete}
\externaldocument[fdl-]{fdl}
\externaldocument[index-]{index}

% Theorem environments.
%
\theoremstyle{plain}
\newtheorem{theorem}[subsection]{Theorem}
\newtheorem{proposition}[subsection]{Proposition}
\newtheorem{lemma}[subsection]{Lemma}

\theoremstyle{definition}
\newtheorem{definition}[subsection]{Definition}
\newtheorem{example}[subsection]{Example}
\newtheorem{exercise}[subsection]{Exercise}
\newtheorem{situation}[subsection]{Situation}

\theoremstyle{remark}
\newtheorem{remark}[subsection]{Remark}
\newtheorem{remarks}[subsection]{Remarks}

\numberwithin{equation}{subsection}

% Macros
%
\def\lim{\mathop{\rm lim}\nolimits}
\def\colim{\mathop{\rm colim}\nolimits}
\def\Spec{\mathop{\rm Spec}}
\def\Hom{\mathop{\rm Hom}\nolimits}
\def\Ext{\mathop{\rm Ext}\nolimits}
\def\SheafHom{\mathop{\mathcal{H}\!{\it om}}\nolimits}
\def\SheafExt{\mathop{\mathcal{E}\!{\it xt}}\nolimits}
\def\Sch{\textit{Sch}}
\def\Mor{\mathop{\rm Mor}\nolimits}
\def\Ob{\mathop{\rm Ob}\nolimits}
\def\Sh{\mathop{\textit{Sh}}\nolimits}
\def\NL{\mathop{N\!L}\nolimits}
\def\proetale{{pro\text{-}\acute{e}tale}}
\def\etale{{\acute{e}tale}}
\def\QCoh{\textit{QCoh}}
\def\Ker{\mathop{\rm Ker}}
\def\Im{\mathop{\rm Im}}
\def\Coker{\mathop{\rm Coker}}
\def\Coim{\mathop{\rm Coim}}

%
% Macros for moduli stacks/spaces
%
\def\QCohstack{\mathcal{QC}\!{\it oh}}
\def\Cohstack{\mathcal{C}\!{\it oh}}
\def\Spacesstack{\mathcal{S}\!{\it paces}}
\def\Quotfunctor{{\rm Quot}}
\def\Hilbfunctor{{\rm Hilb}}
\def\Curvesstack{\mathcal{C}\!{\it urves}}
\def\Polarizedstack{\mathcal{P}\!{\it olarized}}
\def\Complexesstack{\mathcal{C}\!{\it omplexes}}
% \Pic is the operator that assigns to X its picard group, usage \Pic(X)
% \Picardstack_{X/B} denotes the Picard stack of X over B
% \Picardfunctor_{X/B} denotes the Picard functor of X over B
\def\Pic{\mathop{\rm Pic}\nolimits}
\def\Picardstack{\mathcal{P}\!{\it ic}}
\def\Picardfunctor{{\rm Pic}}
\def\Deformationcategory{\mathcal{D}\!{\it ef}}


% OK, start here.
%
\begin{document}

\title{Groupoid schemes}


\maketitle

\tableofcontents

\section{Introduction}
\label{section-introduction}

\noindent
This chapter is devoted to generalities concering groupoid schemes.
See for example the beautiful paper \cite{K-M} by Keel and Mori.








\section{Equivalence relations}
\label{section-equivalence-relations}

\noindent
Given a(n equivalence) relation $R$ on a set $A$ we denote
$A/R$ the set of equivalence classes.

\begin{definition}
\label{definition-equivalence-relation}
Let $S$ be a scheme. Let $U$ be a scheme over $S$.
\begin{enumerate}
\item A {\it relation} on $U$ over $S$ is any morphism
of schemes $j : R \to U \times_S U$. In this case we set
$t = \text{pr}_0 \circ j$ and $s = \text{pr}_1 \circ j$.
\item A {\it pre-equivalence relation} is a relation
$j : R \to U\times_SU$ such that the image of
$j(T) : R(T) \to U(T) \times U(T)$ is an equivalence relation for
all $T/S$.
\item We say a morphism $R \to U \times_S U$ is
an {\it equivalence relation on $U$ over $S$}
if and only if for every $T/S$ the $T$-valued
points of $R$ define an equivalence relation
on the set of $T$-valued points of $U$.
(So it is a pre-equivalence relation such that $j(T)$
is injective for all $T$.)
\end{enumerate}
\end{definition}

\begin{lemma}
\label{lemma-pre-equivalence-equivalence-relation-points}
Let $j : R \to U\times_S U$ be a relation.
Consider the relation on points of the scheme $U$ defined by
the rule
$$
x \sim y
\Leftrightarrow
\exists\ r \in R :
t(r) = x,
s(r) = y.
$$
If $j$ is a pre-equivalence relation then this is an
equivalence relation.
\end{lemma}

\begin{proof}
Suppose that $x \sim y$ and $y \sim z$.
Pick $r \in R$ with $t(r) = x$, $s(r) = y$ and
pick $r' \in R$ with $t(r') = y$, $s(r') = z$.
Pick a field $K$ fitting into the following commutative
diagram
$$
\xymatrix{
\kappa(r) \ar[r] & K \\
\kappa(y) \ar[u] \ar[r] & \kappa(r') \ar[u]
}
$$
Denote $x_K, y_K, z_K : \text{Spec}(K) \to U$
the morphisms
$$
\begin{matrix}
\text{Spec}(K) \to \text{Spec}(\kappa(r))
\to 
\text{Spec}(\kappa(x)) \to U \\
\text{Spec}(K) \to \text{Spec}(\kappa(r))
\to
\text{Spec}(\kappa(y)) \to U \\
\text{Spec}(K) \to \text{Spec}(\kappa(r'))
\to
\text{Spec}(\kappa(z)) \to U
\end{matrix}
$$
By construction $(x_K, y_K) \in j(R(K))$ and
$(y_K, z_K) \in j(R(K))$. Since $j$ is a pre-equivalence relation
we see that also $(x_K, z_K) \in j(R(K))$.
This clearly implies that $x \sim z$.

\medskip\noindent
The proof that $\sim$ is reflexive and symmetric is omitted.
\end{proof}











\section{Groupoids}
\label{section-groupoids}

\noindent
Recall that a groupoid is a category in which every morphism
is an isomorphism, see
Categories, Definition \ref{categories-definition-groupoid}.
Hence a groupoid has a set of objects $\text{Ob}$,
a set of arrows $\text{Arrows}$, a {\it source} and {\it target}
map $s, t : \text{Arrows} \to \text{Ob}$,
an {\it identity} $e : \text{Ob} \to \text{Arrows}$,
and  a {\it composition law}
$c : \text{Arrows} \times_{s, \text{Ob}, t} \text{Arrows}
\to \text{Arrows}$.
These maps satisfy exactly the following axioms
\begin{enumerate}
\item $s \circ e = t \circ e = \text{id}$ as maps $\text{Ob} \to \text{Ob}$,
\item $c \circ (1, c) = c \circ (c, 1)$ as maps
$\text{Arrows} \times_{s, \text{Ob}, t}
\text{Arrows} \times_{s, \text{Ob}, t}
\text{Arrows} \to \text{Arrows}$,
\item $c \circ (1, e \circ s) = c \circ (e \circ t, 1) = 1$
as maps $\text{Arrows} \to \text{Arrows}$,
\item there exists a map $i : \text{Arrows} \to \text{Arrows}$
called the {\it inverse} such that $s \circ i = t$, $t \circ i = s$ and
$c \circ (1, i) = e \circ t$ and
$c \circ (i, 1) = e \circ s$.
\end{enumerate}
If this is the case the map $i$ is uniquely determined and
is a bijection.

\begin{definition}
\label{definition-groupoid}
Let $S$ be a scheme.
A {\it groupoid scheme over $S$}, or simply a
{\it groupoid over $S$} is a
sixtuple $(U, R, s, t, c, e)$ where
$U$ and $R$ are schemes over $S$,
$s, t : R \to U$, $c : R \times_{s, U, t} R \to R$,
$e : U \to R$ are morphisms
of schemes over $S$ such that for any scheme
$T$ over $S$ the sixtuple
$$
(U(T), R(T), s, t, c, e)
$$
is a groupoid category in the sense described above.
\end{definition}

\noindent
Note that this implies in particular, via the Yoneda lemma,
that there is a unique morphism of schemes
$i : R \to R$ over $S$ such that for every scheme $T$ over $S$
the induced map $i : R(T) \to R(T)$ is the inverse of
the groupoid category. Note that $i$ is an isomorphism.
Moreover, given a groupoid scheme over $S$ we denote
$$
j = (t, s) : R \longrightarrow U \times_S U
$$
which is compatible with our conventions in Section
\ref{section-equivalence-relations} above.

\begin{lemma}
\label{lemma-groupoid-pre-equivalence}
Given a groupoid scheme $(U, R, s, t, c, e)$ over $S$
the morphism $j : R \to U\times_S U$ is a pre-equivalence
relation.
\end{lemma}

\begin{proof}
Omitted.
This is a nice exercise in the definitions.
\end{proof}

\begin{lemma}
\label{lemma-equivalence-groupoid}
Given an equivalence relation $j : R \to U$ over $S$
there is a unique way to extend it to a groupoid
$(U, R, s, t, c, e)$ over $S$.
\end{lemma}

\begin{proof}
Omitted.
This is a nice exercise in the definitions.
\end{proof}
























\section{Other chapters}

\begin{multicols}{2}
\begin{enumerate}
\item \hyperref[introduction-section-phantom]{Introduction}
\item \hyperref[conventions-section-phantom]{Conventions}
\item \hyperref[sets-section-phantom]{Set Theory}
\item \hyperref[categories-section-phantom]{Categories}
\item \hyperref[topology-section-phantom]{Topology}
\item \hyperref[sheaves-section-phantom]{Sheaves on Spaces}
\item \hyperref[algebra-section-phantom]{Commutative Algebra}
\item \hyperref[sites-section-phantom]{Sites and Sheaves}
\item \hyperref[homology-section-phantom]{Homological Algebra}
\item \hyperref[derived-section-phantom]{Derived Categories}
\item \hyperref[more-algebra-section-phantom]{More Algebra}
\item \hyperref[simplicial-section-phantom]{Simplicial Methods}
\item \hyperref[modules-section-phantom]{Sheaves of Modules}
\item \hyperref[sites-modules-section-phantom]{Modules on Sites}
\item \hyperref[injectives-section-phantom]{Injectives}
\item \hyperref[cohomology-section-phantom]{Cohomology of Sheaves}
\item \hyperref[sites-cohomology-section-phantom]{Cohomology on Sites}
\item \hyperref[hypercovering-section-phantom]{Hypercoverings}
\item \hyperref[schemes-section-phantom]{Schemes}
\item \hyperref[constructions-section-phantom]{Constructions of Schemes}
\item \hyperref[properties-section-phantom]{Properties of Schemes}
\item \hyperref[morphisms-section-phantom]{Morphisms of Schemes}
\item \hyperref[coherent-section-phantom]{Coherent Cohomology}
\item \hyperref[divisors-section-phantom]{Divisors}
\item \hyperref[limits-section-phantom]{Limits of Schemes}
\item \hyperref[varieties-section-phantom]{Varieties}
\item \hyperref[chow-section-phantom]{Chow Homology}
\item \hyperref[topologies-section-phantom]{Topologies on Schemes}
\item \hyperref[descent-section-phantom]{Descent}
\item \hyperref[more-morphisms-section-phantom]{More on Morphisms}
\item \hyperref[flat-section-phantom]{More on Flatness}
\item \hyperref[groupoids-section-phantom]{Groupoid Schemes}
\item \hyperref[more-groupoids-section-phantom]{More on Groupoid Schemes}
\item \hyperref[etale-section-phantom]{\'Etale Morphisms of Schemes}
\item \hyperref[etale-cohomology-section-phantom]{\'Etale Cohomology}
\item \hyperref[spaces-section-phantom]{Algebraic Spaces}
\item \hyperref[spaces-properties-section-phantom]{Properties of Algebraic Spaces}
\item \hyperref[spaces-morphisms-section-phantom]{Morphisms of Algebraic Spaces}
\item \hyperref[spaces-topologies-section-phantom]{Topologies on Algebraic Spaces}
\item \hyperref[spaces-descent-section-phantom]{Descent and Algebraic Spaces}
\item \hyperref[spaces-more-morphisms-section-phantom]{More on Morphisms of Spaces}
\item \hyperref[quot-section-phantom]{Quot and Hilbert Spaces}
\item \hyperref[stacks-section-phantom]{Stacks}
\item \hyperref[spaces-groupoids-section-phantom]{Groupoids in Algebraic Spaces}
\item \hyperref[spaces-more-groupoids-section-phantom]{More on Groupoids in Spaces}
\item \hyperref[bootstrap-section-phantom]{Bootstrap}
\item \hyperref[examples-stacks-section-phantom]{Examples of Stacks}
\item \hyperref[groupoids-quotients-section-phantom]{Quotients of Groupoids}
\item \hyperref[algebraic-section-phantom]{Algebraic Stacks}
\item \hyperref[criteria-section-phantom]{Criteria for Representability}
\item \hyperref[stacks-properties-section-phantom]{Properties of Algebraic Stacks}
\item \hyperref[stacks-morphisms-section-phantom]{Morphisms of Algebraic Stacks}
\item \hyperref[examples-section-phantom]{Examples}
\item \hyperref[exercises-section-phantom]{Exercises}
\item \hyperref[guide-section-phantom]{Guide to Literature}
\item \hyperref[desirables-section-phantom]{Desirables}
\item \hyperref[coding-section-phantom]{Coding Style}
\item \hyperref[fdl-section-phantom]{GNU Free Documentation License}
\item \hyperref[index-section-phantom]{Auto Generated Index}
\end{enumerate}
\end{multicols}


\bibliography{my}
\bibliographystyle{alpha}

\end{document}
