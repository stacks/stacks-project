\IfFileExists{stacks-project.cls}{%
\documentclass{stacks-project}
}{%
\documentclass{amsart}
}

% The following AMS packages are automatically loaded with
% the amsart documentclass:
%\usepackage{amsmath}
%\usepackage{amssymb}
%\usepackage{amsthm}

% For dealing with references we use the comment environment
\usepackage{verbatim}
\newenvironment{reference}{\comment}{\endcomment}
%\newenvironment{reference}{}{}
\newenvironment{slogan}{\comment}{\endcomment}
\newenvironment{history}{\comment}{\endcomment}

% For commutative diagrams you can use
% \usepackage{amscd}
\usepackage[all]{xy}

% We use 2cell for 2-commutative diagrams.
\xyoption{2cell}
\UseAllTwocells

% To put source file link in headers.
% Change "template.tex" to "this_filename.tex"
% \usepackage{fancyhdr}
% \pagestyle{fancy}
% \lhead{}
% \chead{}
% \rhead{Source file: \url{template.tex}}
% \lfoot{}
% \cfoot{\thepage}
% \rfoot{}
% \renewcommand{\headrulewidth}{0pt}
% \renewcommand{\footrulewidth}{0pt}
% \renewcommand{\headheight}{12pt}

\usepackage{multicol}

% For cross-file-references
\usepackage{xr-hyper}

% Package for hypertext links:
\usepackage{hyperref}

% For any local file, say "hello.tex" you want to link to please
% use \externaldocument[hello-]{hello}
\externaldocument[introduction-]{introduction}
\externaldocument[conventions-]{conventions}
\externaldocument[sets-]{sets}
\externaldocument[categories-]{categories}
\externaldocument[topology-]{topology}
\externaldocument[sheaves-]{sheaves}
\externaldocument[sites-]{sites}
\externaldocument[stacks-]{stacks}
\externaldocument[fields-]{fields}
\externaldocument[algebra-]{algebra}
\externaldocument[brauer-]{brauer}
\externaldocument[homology-]{homology}
\externaldocument[derived-]{derived}
\externaldocument[simplicial-]{simplicial}
\externaldocument[more-algebra-]{more-algebra}
\externaldocument[smoothing-]{smoothing}
\externaldocument[modules-]{modules}
\externaldocument[sites-modules-]{sites-modules}
\externaldocument[injectives-]{injectives}
\externaldocument[cohomology-]{cohomology}
\externaldocument[sites-cohomology-]{sites-cohomology}
\externaldocument[dga-]{dga}
\externaldocument[dpa-]{dpa}
\externaldocument[hypercovering-]{hypercovering}
\externaldocument[schemes-]{schemes}
\externaldocument[constructions-]{constructions}
\externaldocument[properties-]{properties}
\externaldocument[morphisms-]{morphisms}
\externaldocument[coherent-]{coherent}
\externaldocument[divisors-]{divisors}
\externaldocument[limits-]{limits}
\externaldocument[varieties-]{varieties}
\externaldocument[topologies-]{topologies}
\externaldocument[descent-]{descent}
\externaldocument[perfect-]{perfect}
\externaldocument[more-morphisms-]{more-morphisms}
\externaldocument[flat-]{flat}
\externaldocument[groupoids-]{groupoids}
\externaldocument[more-groupoids-]{more-groupoids}
\externaldocument[etale-]{etale}
\externaldocument[chow-]{chow}
\externaldocument[intersection-]{intersection}
\externaldocument[pic-]{pic}
\externaldocument[adequate-]{adequate}
\externaldocument[dualizing-]{dualizing}
\externaldocument[duality-]{duality}
\externaldocument[discriminant-]{discriminant}
\externaldocument[local-cohomology-]{local-cohomology}
\externaldocument[curves-]{curves}
\externaldocument[resolve-]{resolve}
\externaldocument[models-]{models}
\externaldocument[pione-]{pione}
\externaldocument[etale-cohomology-]{etale-cohomology}
\externaldocument[proetale-]{proetale}
\externaldocument[crystalline-]{crystalline}
\externaldocument[spaces-]{spaces}
\externaldocument[spaces-properties-]{spaces-properties}
\externaldocument[spaces-morphisms-]{spaces-morphisms}
\externaldocument[decent-spaces-]{decent-spaces}
\externaldocument[spaces-cohomology-]{spaces-cohomology}
\externaldocument[spaces-limits-]{spaces-limits}
\externaldocument[spaces-divisors-]{spaces-divisors}
\externaldocument[spaces-over-fields-]{spaces-over-fields}
\externaldocument[spaces-topologies-]{spaces-topologies}
\externaldocument[spaces-descent-]{spaces-descent}
\externaldocument[spaces-perfect-]{spaces-perfect}
\externaldocument[spaces-more-morphisms-]{spaces-more-morphisms}
\externaldocument[spaces-flat-]{spaces-flat}
\externaldocument[spaces-groupoids-]{spaces-groupoids}
\externaldocument[spaces-more-groupoids-]{spaces-more-groupoids}
\externaldocument[bootstrap-]{bootstrap}
\externaldocument[spaces-pushouts-]{spaces-pushouts}
\externaldocument[groupoids-quotients-]{groupoids-quotients}
\externaldocument[spaces-more-cohomology-]{spaces-more-cohomology}
\externaldocument[spaces-simplicial-]{spaces-simplicial}
\externaldocument[formal-spaces-]{formal-spaces}
\externaldocument[restricted-]{restricted}
\externaldocument[spaces-resolve-]{spaces-resolve}
\externaldocument[formal-defos-]{formal-defos}
\externaldocument[defos-]{defos}
\externaldocument[cotangent-]{cotangent}
\externaldocument[examples-defos-]{examples-defos}
\externaldocument[algebraic-]{algebraic}
\externaldocument[examples-stacks-]{examples-stacks}
\externaldocument[stacks-sheaves-]{stacks-sheaves}
\externaldocument[criteria-]{criteria}
\externaldocument[artin-]{artin}
\externaldocument[quot-]{quot}
\externaldocument[stacks-properties-]{stacks-properties}
\externaldocument[stacks-morphisms-]{stacks-morphisms}
\externaldocument[stacks-limits-]{stacks-limits}
\externaldocument[stacks-cohomology-]{stacks-cohomology}
\externaldocument[stacks-perfect-]{stacks-perfect}
\externaldocument[stacks-introduction-]{stacks-introduction}
\externaldocument[stacks-more-morphisms-]{stacks-more-morphisms}
\externaldocument[stacks-geometry-]{stacks-geometry}
\externaldocument[moduli-]{moduli}
\externaldocument[moduli-curves-]{moduli-curves}
\externaldocument[examples-]{examples}
\externaldocument[exercises-]{exercises}
\externaldocument[guide-]{guide}
\externaldocument[desirables-]{desirables}
\externaldocument[coding-]{coding}
\externaldocument[obsolete-]{obsolete}
\externaldocument[fdl-]{fdl}
\externaldocument[index-]{index}

% Theorem environments.
%
\theoremstyle{plain}
\newtheorem{theorem}[subsection]{Theorem}
\newtheorem{proposition}[subsection]{Proposition}
\newtheorem{lemma}[subsection]{Lemma}

\theoremstyle{definition}
\newtheorem{definition}[subsection]{Definition}
\newtheorem{example}[subsection]{Example}
\newtheorem{exercise}[subsection]{Exercise}
\newtheorem{situation}[subsection]{Situation}

\theoremstyle{remark}
\newtheorem{remark}[subsection]{Remark}
\newtheorem{remarks}[subsection]{Remarks}

\numberwithin{equation}{subsection}

% Macros
%
\def\lim{\mathop{\rm lim}\nolimits}
\def\colim{\mathop{\rm colim}\nolimits}
\def\Spec{\mathop{\rm Spec}}
\def\Hom{\mathop{\rm Hom}\nolimits}
\def\Ext{\mathop{\rm Ext}\nolimits}
\def\SheafHom{\mathop{\mathcal{H}\!{\it om}}\nolimits}
\def\SheafExt{\mathop{\mathcal{E}\!{\it xt}}\nolimits}
\def\Sch{\textit{Sch}}
\def\Mor{\mathop{\rm Mor}\nolimits}
\def\Ob{\mathop{\rm Ob}\nolimits}
\def\Sh{\mathop{\textit{Sh}}\nolimits}
\def\NL{\mathop{N\!L}\nolimits}
\def\proetale{{pro\text{-}\acute{e}tale}}
\def\etale{{\acute{e}tale}}
\def\QCoh{\textit{QCoh}}
\def\Ker{\mathop{\rm Ker}}
\def\Im{\mathop{\rm Im}}
\def\Coker{\mathop{\rm Coker}}
\def\Coim{\mathop{\rm Coim}}

%
% Macros for moduli stacks/spaces
%
\def\QCohstack{\mathcal{QC}\!{\it oh}}
\def\Cohstack{\mathcal{C}\!{\it oh}}
\def\Spacesstack{\mathcal{S}\!{\it paces}}
\def\Quotfunctor{{\rm Quot}}
\def\Hilbfunctor{{\rm Hilb}}
\def\Curvesstack{\mathcal{C}\!{\it urves}}
\def\Polarizedstack{\mathcal{P}\!{\it olarized}}
\def\Complexesstack{\mathcal{C}\!{\it omplexes}}
% \Pic is the operator that assigns to X its picard group, usage \Pic(X)
% \Picardstack_{X/B} denotes the Picard stack of X over B
% \Picardfunctor_{X/B} denotes the Picard functor of X over B
\def\Pic{\mathop{\rm Pic}\nolimits}
\def\Picardstack{\mathcal{P}\!{\it ic}}
\def\Picardfunctor{{\rm Pic}}
\def\Deformationcategory{\mathcal{D}\!{\it ef}}


% OK, start here.
%
\begin{document}

\title{Differential Graded Algebra}


\maketitle

\phantomsection
\label{section-phantom}

\tableofcontents

\section{Introduction}
\label{section-introduction}

\noindent
In this chapter we talk about differential graded algebras, modules,
categories, etc. A basic reference is \cite{Keller-Deriving}.
A survey paper is \cite{Keller-survey}.

\medskip\noindent
Since we do not worry about length of exposition in the Stacks project
we first develop the material in the setting of categories of differential
graded modules. After that we redo the constructions in the setting of
differential graded modules over differential graded categories.



\section{Conventions}
\label{section-conventions}

\noindent
In this chapter we hold on to the convention that {\it ring} means
commutative ring with $1$. If $R$ is a ring, then an {\it $R$-algebra $A$}
will be an $R$-module $A$ endowed with an $R$-bilinear map $A \times A \to A$
(multiplication) such that multiplication is associative and has a unit.
In other words, these are unital associative $R$-algebras
such that the structure map $R \to A$ maps into the center of $A$.




\section{Differential graded algebras}
\label{section-dga}


\noindent
Just the definitions.

\begin{definition}
\label{definition-dga}
Let $R$ be a commutative ring. A {\it differential graded algebra over $R$}
is either
\begin{enumerate}
\item a chain complex $A_\bullet$ of $R$-modules endowed with
$R$-bilinear maps $A_n \times A_m \to A_{n + m}$,
$(a, b) \mapsto ab$ such that
$$
\text{d}_{n + m}(ab) = \text{d}_n(a)b + (-1)^n a\text{d}_m(b)
$$
and such that $\bigoplus A_n$ becomes an associative and unital
$R$-algebra, or
\item a cochain complex $A^\bullet$ of $R$-modules endowed with
$R$-bilinear maps $A^n \times A^m \to A^{n + m}$, $(a, b) \mapsto ab$
such that
$$
\text{d}^{n + m}(ab) = \text{d}^n(a)b + (-1)^n a\text{d}^m(b)
$$
and such that $\bigoplus A^n$ becomes an associative and unital $R$-algebra.
\end{enumerate}
\end{definition}

\noindent
We often just write $A = \bigoplus A_n$ or $A = \bigoplus A^n$ and
think of this as an associative unital $R$-algebra endowed with a
$\mathbf{Z}$-grading and an $R$-linear operator $\text{d}$ whose square
is zero and which satisfies the Leibniz rule as explained above. In this case
we often say ``Let $(A, \text{d})$ be a differential graded algebra''.

\begin{definition}
\label{definition-homomorphism-dga}
A {\it homomorphism of differential graded algebras}
$f : (A, \text{d}) \to (B, \text{d})$ is an algebra map $f : A \to B$
compatible with the gradings and $\text{d}$.
\end{definition}

\begin{definition}
\label{definition-opposite-dga}
Let $R$ be a ring. Let $(A, \text{d})$ be a differential graded algebra
over $R$. The {\it opposite differential graded algebra} is the differential
graded algebra $(A^{opp}, \text{d})$ over $R$ where $A^{opp} = A$
as an $R$-module, $\text{d} = \text{d}$, and multiplication is
given by
$$
a \cdot_{opp} b = (-1)^{\deg(a)\deg(b)} b a
$$
for homogeneous elements $a, b \in A$.
\end{definition}

\noindent
This makes sense because
\begin{align*}
\text{d}(a \cdot_{opp} b)
& =
(-1)^{\deg(a)\deg(b)} \text{d}(b a) \\
& =
(-1)^{\deg(a)\deg(b)} \text{d}(b) a +
(-1)^{\deg(a)\deg(b) + \deg(b)}b\text{d}(a) \\
& =
(-1)^{\deg(a)}a \cdot_{opp} \text{d}(b) + \text{d}(a) \cdot_{opp} b
\end{align*}
as desired.

\begin{definition}
\label{definition-cdga}
A differential graded algebra $(A, \text{d})$ is {\it commutative} if
$ab = (-1)^{nm}ba$ for $a$ in degree $n$ and $b$ in degree $m$.
We say $A$ is {\it strictly commutative} if in addition $a^2 = 0$
for $\deg(a)$ odd.
\end{definition}

\noindent
The following definition makes sense in general but is perhaps
``correct'' only when tensoring commutative differential graded
algebras.

\begin{definition}
\label{definition-tensor-product}
Let $R$ be a ring.
Let $(A, \text{d})$, $(B, \text{d})$ be differential graded algebras over $R$.
The {\it tensor product differential graded algebra} of $A$ and $B$
is the algebra $A \otimes_R B$ with multiplication defined by
$$
(a \otimes b)(a' \otimes b') = (-1)^{\deg(a')\deg(b)} aa' \otimes bb'
$$
endowed with differential $\text{d}$ defined by the rule
$\text{d}(a \otimes b) = \text{d}(a) \otimes b + (-1)^m a \otimes \text{d}(b)$
where $m = \deg(b)$.
\end{definition}

\begin{lemma}
\label{lemma-total-complex-tensor-product}
Let $R$ be a ring.
Let $(A, \text{d})$, $(B, \text{d})$ be differential graded algebras over $R$.
Denote $A^\bullet$, $B^\bullet$ the underlying cochain complexes.
As cochain complexes of $R$-modules we have
$$
(A \otimes_R B)^\bullet = \text{Tot}(A^\bullet \otimes_A B^\bullet).
$$
\end{lemma}

\begin{proof}
Recall that the differential of the total complex is given by
$\text{d}_1^{p, q} + (-1)^p \text{d}_2^{p, q}$ on $A^p \otimes_R B^q$.
And this is exactly the same as the rule for the differential
on $A \otimes_R B$ in
Definition \ref{definition-tensor-product}.
\end{proof}






\section{Differential graded modules}
\label{section-modules}

\noindent
Just the definitions.

\begin{definition}
\label{definition-dgm}
Let $R$ be a ring.
Let $(A, \text{d})$ be a differential graded algebra over $R$.
A (right) {\it differential graded module} $M$ over $A$ is a right $A$-module
$M$ which has a grading $M = \bigoplus M^n$ and a differential $\text{d}$
such that $M^n A^m \subset M^{n + m}$, such that
$\text{d}(M^n) \subset M^{n + 1}$, and such that
$$
\text{d}(ma) = \text{d}(m) + (-1)^n m\text{d}(a)
$$
for $a \in A$ and $m \in M^n$. A
{\it homomorphism of differential graded modules} $f : M \to N$
is an $A$-module map compatible with gradings and differentials.
The category of (right) differential graded $A$-modules is denoted
$\text{Mod}_{(A, \text{d})}$.
\end{definition}

\noindent
Note that we can think of $M$ as a cochain complex $M^\bullet$
of (right) $R$-modules. Namely, for $r \in R$ we have $\text{d}(r) = 0$
and $r$ maps to a degree $0$ element of $A$, hence
$\text{d}(mr) = \text{d}(m)r$.

\medskip\noindent
We can define {\it left differential graded $A$-modules} in exactly the same
manner. If $M$ is a left $A$-module, then we can think of $M$ as a
right $A^{opp}$-module with mulitplication $\cdot_{opp}$ defined by
the rule
$$
m \cdot_{opp} a = (-1)^{\deg(a)\deg(m)} a m
$$
for $a$ and $m$ homogeneous. The category of left differential graded
$A$-modules is equivalent to the category of right differential
graded $A^{opp}$-modules. We prefer to work with right modules
(essentially because of what happens in Example \ref{example-dgm-dg-cat}), but
the reader is free to switch to left modules if (s)he so desires.

\begin{lemma}
\label{lemma-dgm-abelian}
Let $(A, d)$ be a differential graded algebra. The category
$\text{Mod}_{(A, \text{d})}$ is abelian and has arbitrary colimits.
\end{lemma}

\begin{proof}
Omitted. Hint: kernels and cokernels commute with taking
homogeneous components. Similarly for colimits.
\end{proof}

\noindent
Thus, if $(A, \text{d})$ is a differential graded
algebra over $R$, then there is an exact functor
$$
\text{Mod}_{(A, \text{d})} \longrightarrow \text{Comp}(R)
$$
of abelian categories. For a differential graded module $M$ the
cohomology groups $H^n(M)$ are defined as the cohomology of the
corresponding complex of $R$-modules. Therefore, a short exact
sequence $0 \to K \to L \to M \to 0$ of differential graded modules
gives rise to a long exact sequence
\begin{equation}
\label{equation-les}
H^n(K) \to H^n(L) \to H^n(M) \to H^{n + 1}(K)
\end{equation}
of cohomology modules, see
Homology, Lemma \ref{homology-lemma-long-exact-sequence-cochain}.

\medskip\noindent
Moreover, from now on we borrow all the terminology used for
complexes of modules. For example, we say that a differential
graded $A$-module $M$ is {\it acyclic} if $H^k(M) = 0$ for
all $k \in \mathbf{Z}$. We say that a homomorphism $M \to N$
of differential graded $A$-modules is a {\it quasi-isomorphism}
if it induces isomorphisms $H^k(M) \to H^k(N)$ for all $k \in \mathbf{Z}$.
And so on and so forth.

\begin{definition}
\label{definition-shift}
Let $(A, \text{d})$ be a differential graded algebra.
Let $M$ be a differential graded module.
For any $k \in \mathbf{Z}$ we define the {\it $k$-shifted module}
$M[k]$ as follows
\begin{enumerate}
\item as $A$-module $M[k] = M$,
\item $M[k]^n = M^{n + k}$,
\item $\text{d}_{M[k]} = (-1)^k\text{d}_M$.
\end{enumerate}
For a morphism $f : M \to N$ of differential graded $A$-modules
we let $f[k] : M[k] \to N[k]$ be the map equal to $f$ on underlying
$A$-modules. This defines a functor
$[k] : \text{Mod}_{(A, \text{d})} \to \text{Mod}_{(A, \text{d})}$.
\end{definition}

\noindent
The remarks in Homology, Section \ref{homology-section-homotopy-shift} apply.
In particular, we will identify the cohomology groups of all shifts
$M[k]$ without the intervention of signs.

\medskip\noindent
At this point we have enough structure to talk about {\it triangles},
see Derived, Definition \ref{derived-definition-triangle}.
In fact, our next goal is to develop enough theory to be able to
state and prove that the homotopy category of differential graded
modules is a triangulated category. First we define the homotopy category.






\section{The homotopy category}
\label{section-homotopy}

\noindent
Our homotopies take into account the $A$-module structure and the
grading, but not the differential (of course).

\begin{definition}
\label{definition-homotopy}
Let $(A, \text{d})$ be a differential graded algebra. Let
$f, g : M \to N$ be homomorphisms of differential graded $A$-modules.
A {\it homotopy between $f$ and $g$} is an $A$-module map $h : M \to N$
such that
\begin{enumerate}
\item $h(M^n) \subset N^{n - 1}$ for all $n$, and
\item $f(x) - g(x) = \text{d}_N(h(x)) + h(\text{d}_M(x))$ for
all $x \in M$.
\end{enumerate}
If a homotopy exists, then we say $f$ and $g$ are {\it homotopic}.
\end{definition}

\noindent
Thus $h$ is compatible with the $A$-module structure and the grading
but not with the differential. If $f = g$ and $h$ is a homotopy
as in the definition, then $h$ defines a morphism $h : M \to N[1]$
in $\text{Mod}_{(A, \text{d})}$.

\begin{lemma}
\label{lemma-compose-homotopy}
Let $(A, \text{d})$ be a differential graded algebra.
Let $f, g : L \to M$ be homomorphisms of differential graded $A$-modules.
Suppose given further homomorphisms $a : K \to L$, and $c : M \to N$.
If $h : M \to N$ defines a homotopy between $f$ and $g$, then
$c \circ h \circ a$ defines a homotopy between $c \circ f \circ a$ and
$c \circ g \circ a$.
\end{lemma}

\begin{proof}
Immediate from Homology, Lemma \ref{homology-lemma-compose-homotopy-cochain}.
\end{proof}

\noindent
This lemma allows us to define the homotopy category as follows.

\begin{definition}
\label{definition-complexes-notation}
Let $(A, \text{d})$ be a differential graded algebra.
The {\it homotopy category}, denoted $K(\text{Mod}_{(A, \text{d})})$, is
the the category whose objects are the objects of
$\text{Mod}_{(A, \text{d})}$ and whose morphisms are homotopy classes
of homomorphisms of differential graded $A$-modules.
\end{definition}

\noindent
The notation $K(\text{Mod}_{(A, \text{d})})$ is not standard but at least is
consistent with the use of $K(-)$ in other places of the Stacks project.

\begin{lemma}
\label{lemma-homotopy-direct-sums}
Let $(A, \text{d})$ be a differential graded algebra.
The homotopy category $K(\text{Mod}_{(A, \text{d})})$
has arbitrary direct sums.
\end{lemma}

\begin{proof}
Omitted.
\end{proof}


















\section{Admissible short exact sequences}
\label{section-admissible}

\noindent
An admissible short exact sequence is the analogue of termwise split exact
sequences in the setting of differential graded modules.

\begin{definition}
\label{definition-admissible-ses}
Let $(A, \text{d})$ be a differential graded algebra.
\begin{enumerate}
\item A homomorphism $K \to L$ of differential graded $A$-modules
is an {\it admissible monomorphism} if there exists a graded $A$-module
map $L \to K$ which is left inverse to $K \to L$.
\item A homomorphism $L \to M$ of differential graded $A$-modules
is an {\it admissible epimorphism} if there exists a graded $A$-module
map $M \to L$ which is right inverse to $K \to L$.
\item A short exact sequence $0 \to K \to L \to M \to 0$ of differential
graded $A$-modules is an {\it admissible short exact sequence}
if it is split as a sequence of graded $A$-modules.
\end{enumerate}
\end{definition}

\noindent
Thus the splittings are compatible with all the data except for
the differentials. Given an admissible short exact sequence we
obtain a triangle; this is the reason that we require our splittings
to be compatible with the $A$-module structure.

\begin{lemma}
\label{lemma-admissible-ses}
Let $(A, \text{d})$ be a differential graded algebra.
Let $0 \to K \to L \to M \to 0$ be an admissible short exact sequence
of differential graded $A$-modules. Let $s : M \to L$ and $\pi : L \to K$
be splittings such that $\text{ker}(\pi) = \text{Im}(s)$.
Then we obtain a morphism
$$
\delta = \pi \circ \text{d}_L \circ s : M \to K[1]
$$
of $\text{Mod}_{(A, \text{d})}$ which induces the boundary maps
in the long exact sequence of cohomology (\ref{equation-les}).
\end{lemma}

\begin{proof}
The map $\pi \circ \text{d}_L \circ s$ is compatible with the $A$-module
structure and the gradings by construction. It is compatible with
differentials by Homology, Lemmas
\ref{homology-lemma-ses-termwise-split-cochain}.
Let $R$ be the ring that $A$ is a differential graded algebra over.
The equality of maps is a statement about $R$-modules. Hence this
follows from Homology, Lemmas
\ref{homology-lemma-ses-termwise-split-cochain} and
\ref{homology-lemma-ses-termwise-split-long-cochain}.
\end{proof}

\begin{lemma}
\label{lemma-make-commute-map}
Let $(A, \text{d})$ be a differential graded algebra. Let
$$
\xymatrix{
K \ar[r]_f \ar[d]_a & L \ar[d]^b \\
M \ar[r]^g & N
}
$$
be a diagram of homomorphisms of differential graded $A$-modules
commuting up to homotopy.
\begin{enumerate}
\item If $f$ is an admissible monomorphism, then $b$ is homotopic to a
homomorphism which makes the diagram commute.
\item If $g$ is an admissible epimorphism, then $a$ is homotopic to a
morphism which makes the diagram commute.
\end{enumerate}
\end{lemma}

\begin{proof}
Let $h : K \to N$ be a homotopy between $bf$ and $ga$, i.e.,
$bf - ga = \text{d}h + h\text{d}$. Suppose that $\pi : L \to K$
is a graded $A$-module map left inverse to $f$. Take
$b' = b + \text{d}h\pi + h\pi \text{d}$.
Suppose $s : N \to M$ is a graded $A$-module map right inverse to $g$.
Take $a' = a + \text{d}sh + sh\text{d}$.
Computations omitted.
\end{proof}

\begin{lemma}
\label{lemma-make-injective}
Let $(A, \text{d})$ be a differential graded algebra.
Let $\alpha : K \to L$ be a homomorphism of differential graded
$A$-modules. There exists a factorization
$$
\xymatrix{
K \ar[r]^{\tilde \alpha} \ar@/_1pc/[rr]_\alpha &
\tilde L \ar[r]^\pi & L
}
$$
in $\text{Mod}_{(A, \text{d})}$ such that
\begin{enumerate}
\item $\tilde \alpha$ is an admissible monomorphism (see
Definition \ref{definition-admissible-ses}),
\item there is a morphism $s : L \to \tilde L$
such that $\pi \circ s = \text{id}_L$ and such that
$s \circ \pi$ is homotopic to $\text{id}_{\tilde L}$.
\end{enumerate}
\end{lemma}

\begin{proof}
The proof is identical to the proof of
Derived Categories, Lemma \ref{derived-lemma-make-injective}.
\end{proof}

\begin{lemma}
\label{lemma-sequence-maps-split}
Let $(A, \text{d})$ be a differential graded algebra.
Let $L_1 \to L_2 \to \ldots \to L_n$
be a sequence of composable homomorphisms of
differential graded $A$-modules.
There exists a commutative diagram
$$
\xymatrix{
L_1 \ar[r] &
L_2 \ar[r] &
\ldots \ar[r] &
L_n \\
M_1 \ar[r] \ar[u] &
M_2 \ar[r] \ar[u] &
\ldots \ar[r] &
M_n \ar[u]
}
$$
in $\text{Mod}_{(A, \text{d})}$ such that each $M_i \to M_{i + 1}$
is an admissible monomorphism and each $M_i \to L_i$
is a homotopy equivalence.
\end{lemma}

\begin{proof}
The case $n = 1$ is without content.
Lemma \ref{lemma-make-injective} is the case $n = 2$.
Suppose we have constructed the diagram
except for $M_n$. Apply Lemma \ref{lemma-make-injective} to
the composition $M_{n - 1} \to L_{n - 1} \to L_n$.
The result is a factorization $M_{n - 1} \to M_n \to L_n$
as desired.
\end{proof}



\begin{lemma}
\label{lemma-nilpotent}
Let $(A, \text{d})$ be a differential graded algebra.
Let $0 \to K_i \to L_i \to M_i \to 0$, $i = 1, 2, 3$
be admissible short exact sequence of differential graded $A$-modules.
Let $b : L_1 \to L_2$ and $b' : L_2 \to L_3$
be homomorphisms of differential graded modules such that
$$
\vcenter{
\xymatrix{
K_1 \ar[d]_0 \ar[r] &
L_1 \ar[r] \ar[d]_b &
M_1 \ar[d]_0 \\
K_2 \ar[r] & L_2 \ar[r] & M_2
}
}
\quad\text{and}\quad
\vcenter{
\xymatrix{
K_2 \ar[d]^0 \ar[r] &
L_2 \ar[r] \ar[d]^{b'} &
M_2 \ar[d]^0 \\
K_3 \ar[r] & L_3 \ar[r] & M_3
}
}
$$
commute up to homotopy. Then $b' \circ b$ is homotopic to $0$.
\end{lemma}

\begin{proof}
By Lemma \ref{lemma-make-commute-map} we can replace $b$ and $b'$ by
homotopic maps such that the right square of the left diagram commutes
and the left square of the right diagram commutes. In other words, we have
$\text{Im}(b) \subset \text{Im}(K_2 \to L_2)$ and
$\text{ker}((b')^n) \supset \text{Im}(K_2 \to L_2)$.
Then $b \circ b' = 0$ as a map of modules.
\end{proof}

\noindent
Let $(A, \text{d})$ be a differential graded algebra. We can think of $A$
as a right differential graded $A$-module over itself. Moreover, we have
the shifts $A[k]$ for all $k \in \mathbf{Z}$. Direct summands of direct sums
of shifts will play an important role in the following.

\begin{lemma}
\label{lemma-target-graded-projective}
Let $(A, \text{d})$ be a differential graded algebra.
Let $M \to P$ be a surjective homomorphism of differential graded
$A$-modules. If $P$ is a direct summand of a direct sum of
shifts of $A$, then $M \to P$ is an admissible epimorphism.
\end{lemma}

\begin{proof}
If $P = A$, then we pick any $m \in M^0$ mapping to $1 \in A$.
The map $A \to M$, $a \mapsto ma$ will be a left inverse to the
map $M \to A$. This property extennds in the usual manner to
direct sums and direct summands of such.
\end{proof}












\section{Distinguished triangles}
\label{section-distinguished}

\noindent
The following lemma produces our distinguished triangles.

\begin{lemma}
\label{lemma-triangle-independent-splittings}
Let $(A, \text{d})$ be a differential graded algebra. Let
$0 \to K \to L \to M \to 0$ be an admissible short exact sequence
of differential graded $A$-modules. The triangle
\begin{equation}
\label{equation-triangle-associated-to-admissible-ses}
K \to L \to M \xrightarrow{\delta} K[1]
\end{equation}
with $\delta$ as in Lemma \ref{lemma-admissible-ses} is, up to canonical
isomorphism in $K(\text{Mod}_{(A, \text{d})})$, indepedent of the choices
made in Lemma \ref{lemma-admissible-ses}.
\end{lemma}

\begin{proof}
Namely, let $(s', \pi')$ be a second choice of splittings as in
Lemma \ref{lemma-admissible-ses}. Then we claim that $\delta$ and $\delta'$
are homotopic. Namely, write $s' = s + \alpha \circ h$ and
$\pi' = \pi + g \circ \beta$ for some unique homomorphisms
of $A$-modules $h : M \to K$ and $g : M \to K$ of degree $-1$.
Then $g = -h$ and $g$ is a homotopy between $\delta$ and $\delta'$.
The computations are done in the proof of
Homology, Lemma \ref{homology-lemma-ses-termwise-split-homotopy-cochain}.
\end{proof}

\begin{definition}
\label{definition-distinguished-triangle}
Let $(A, \text{d})$ be a differential graded algebra.
\begin{enumerate}
\item If $0 \to K \to L \to M \to 0$ is an admissible short exact sequence
of differential graded $A$-modules, then the {\it triangle associated
to $0 \to K \to L \to M \to 0$} is the triangle 
(\ref{equation-triangle-associated-to-admissible-ses})
of $K(\text{Mod}_{(A, \text{d})})$.
\item A triangle of $K(\text{Mod}_{(A, \text{d})})$ is called a
{\it distinguished triangle} if it is isomorphic to a triangle
associated to an admissible short exact sequence
of differential graded $A$-modules.
\end{enumerate}
\end{definition}







\section{Cones}
\label{section-cones}

\noindent
We quickly develop a theory of cones for the category of differential
graded modules.

\begin{definition}
\label{definition-cone}
Let $(A, \text{d})$ be a differential graded algebra.
Let $f : K \to L$ be a homomorphism of differential graded $A$-modules.
The {\it cone} of $f$ is the differential graded $A$-module
$C(f)$ given by $C(f) = L \oplus K$ with grading
$C(f)^n = L^n \oplus K^{n + 1}$ and
differential
$$
d_{C(f)} =
\left(
\begin{matrix}
\text{d}_L & f \\
0 & -\text{d}_K
\end{matrix}
\right)
$$
It comes equipped with canonical morphisms of complexes $i : L \to C(f)$
and $p : C(f) \to K[1]$ induced by the obvious maps $L \to C(f)$
and $C(f) \to K$.
\end{definition}

\noindent
In other words $(K, L, C(f), f, i, p)$ forms a triangle:
$$
K \to L \to C(f) \to K[1]
$$
in $\text{Mod}_{(A, \text{d})}$ and hence in $K(\text{Mod}_{(A, \text{d})})$.
Cones are {\bf not} distinguished triangles in general, but the difference
is a sign or a rotation (your choice). Here are two precise statements.

\begin{lemma}
\label{lemma-rotate-cone}
Let $(A, \text{d})$ be a differential graded algebra.
Let $f : K \to L$ be a homomorphism of differential graded modules.
The triangle $(L, C(f), K[1], i, p, f[1])$ is
the triangle associated to the admissible short exact sequence
$$
0 \to L \to C(f) \to K[1] \to 0
$$
coming from the definition of the cone of $f$.
\end{lemma}

\begin{proof}
Immediate from the definitions.
\end{proof}

\begin{lemma}
\label{lemma-rotate-triangle}
Let $(A, \text{d})$ be a differential graded algebra.
Let $\alpha : K \to L$ and $\beta : L \to M$
define an admissible short exact sequence
$$
0 \to K \to L \to M \to 0
$$
of differential graded $A$-modules.
Let $(K, L, M, \alpha, \beta, \delta)$
be the associated triangle. Then the triangles
$$
(M[-1], K, L, \delta[-1], \alpha, \beta)
\quad\text{and}\quad
(M[-1], K, C(\delta[-1]), \delta[-1], i, p)
$$
are isomorphic.
\end{lemma}

\begin{proof}
Using a choice of splittings we write $L = K \oplus M$ and we identify
$\alpha$ and $\beta$ with the natural inclusion and projection maps.
By construction of $\delta$ we have
$$
d_B =
\left(
\begin{matrix}
d_K & \delta \\
0 & d_M
\end{matrix}
\right)
$$
On the other hand the cone of $\delta[-1] : M[-1] \to K$
is given as $C(\delta[-1]) = K \oplus M$ with differential identical
with the matrix above! Whence the lemma.
\end{proof}

\noindent
The formation of the cone triangle is functorial in the following sense.

\begin{lemma}
\label{lemma-functorial-cone}
Let $(A, \text{d})$ be a differential graded algebra.
Suppose that
$$
\xymatrix{
K_1 \ar[r]_{f_1} \ar[d]_a & L_1 \ar[d]^b \\
K_2 \ar[r]^{f_2} & L_2
}
$$
is a diagram of homomorphisms of diferential graded $A$-modules which is
commutative up to homotopy.
Then there exists a morphism $c : C(f_1) \to C(f_2)$ which gives rise to
a morphism of triangles
$$
(a, b, c) : (K_1, L_1, C(f_1), f_1, i_1, p_1) \to
(K_1, L_1, C(f_1), f_2, i_2, p_2)
$$
in $K(\text{Mod}_{(A, \text{d})})$.
\end{lemma}

\begin{proof}
Let $h : K_1 \to L_2$ be a homotopy between $f_2 \circ a$ and $b \circ f_1$.
Define $c$ by the matrix
$$
c =
\left(
\begin{matrix}
a & h \\
0 & b
\end{matrix}
\right) :
L_1 \oplus K_1 \to L_2 \oplus K_2
$$
A matrix computation show that $c$ is a morphism of differential
graded modules. It is trivial that $c \circ i_1 = i_2 \circ b$, and it is
trivial also to check that $p_2 \circ c = a \circ p_1$.
\end{proof}

\begin{lemma}
\label{lemma-third-isomorphism}
Let $(A, \text{d})$ be a differential graded algebra.
Let $f_1 : K_1 \to L_1$ and $f_2 : K_2 \to L_2$ be homomorphisms of
differential graded $A$-modules. Let
$$
(a, b, c) :
(K_1, L_1, C(f_1), f_1, i_1, p_1)
\longrightarrow
(K_1, L_1, C(f_1), f_2, i_2, p_2)
$$
be any morphism of triangles of $K(\text{Mod}_{(A, \text{d})})$.
If $a$ and $b$ are homotopy equivalences then so is $c$.
\end{lemma}

\begin{proof}
Let $a^{-1} : K_2 \to K_1$ be a homomorphism of differential graded $A$-modules
which is inverse to $a$ in $K(\text{Mod}_{(A, \text{d})})$.
Let $b^{-1} : L_2 \to L_1$ be a homomorphism of differential graded $A$-modules
which is inverse to $b$ in $K(\text{Mod}_{(A, \text{d})})$.
Let $c' : C(f_2) \to C(f_1)$ be the morphism from
Lemma \ref{lemma-functorial-cone} applied to
$f_1 \circ a^{-1} = b^{-1} \circ f_2$.
If we can show that $c \circ c'$ and $c' \circ c$ are isomorphisms in
$K(\text{Mod}_{(A, \text{d})})$
then we win. Hence it suffices to prove the following: Given
a morphism of triangles
$(1, 1, c) : (K, L, C(f), f, i, p)$
in $K(\text{Mod}_{(A, \text{d})})$ the morphism $c$ is an isomorphism in $K(\text{Mod}_{(A, \text{d})})$.
By assumption the two squares in the diagram
$$
\xymatrix{
L \ar[r] \ar[d]_1 &
C(f) \ar[r] \ar[d]_c &
K[1] \ar[d]_1 \\
L \ar[r] &
C(f) \ar[r] &
K[1]
}
$$
commute up to homotopy. By construction of $C(f)$ the rows
form admissible short exact sequences. Thus we see that
$(c - 1)^2 = 0$ in $K(\text{Mod}_{(A, \text{d})})$ by
Lemma \ref{lemma-nilpotent}.
Hence $c$ is an isomorphism in $K(\text{Mod}_{(A, \text{d})})$
with inverse $2 - c$.
\end{proof}

\noindent
The following lemma shows that the collection of triangles of the homotopy
category given by cones and the distinguished triangles are the same
up to isomorphisms, at least up to sign!

\begin{lemma}
\label{lemma-the-same-up-to-isomorphisms}
Let $(A, \text{d})$ be a differential graded algebra.
\begin{enumerate}
\item Given an admissible short exact sequence
$0 \to K \xrightarrow{\alpha} L \to M \to 0$
of differential graded $A$-modules there exists a homotopy equivalence
$C(\alpha) \to M$ such that the diagram
$$
\xymatrix{
K \ar[r] \ar[d] & L \ar[d] \ar[r] &
C(\alpha) \ar[r]_{-p} \ar[d] & K[1] \ar[d] \\
K \ar[r]^\alpha & L \ar[r]^\beta &
M \ar[r]^\delta & K[1]
}
$$
defines an isomorphism of triangles in $K(\text{Mod}_{(A, \text{d})})$.
\item Given a morphism of complexes $f : K \to L$
there exists an isomorphism of triangles
$$
\xymatrix{
K \ar[r] \ar[d] & \tilde L \ar[d] \ar[r] &
M \ar[r]_{\delta} \ar[d] & K[1] \ar[d] \\
K \ar[r] & L \ar[r] &
C(f) \ar[r]^{-p} & K[1]
}
$$
where the upper triangle is the triangle associated to a
termwise split exact sequence $K \to \tilde L \to M$.
\end{enumerate}
\end{lemma}

\begin{proof}
Proof of (1). We have $C(\alpha) = L \oplus K$ and we simply define
$C(\alpha) \to M$ via the projection onto $L$ followed by $\beta$.
This defines a morphism of differential graded modules because the
compositions $K^{n + 1} \to L^{n + 1} \to L^n \to M^n$ are zero.
Choose splittings $s : M \to L$ and $\pi : L \to K$ with
$\text{Ker}(\pi) = \text{Im}(s)$ and set
$\delta = \pi \circ \text{d}_L \circ s$ as usual.
To get a homotopy inverse we take
$M \to C(\alpha)$ given by $(s , -\delta)$. This is compatible with
differentials because $\delta^n$ can be characterized as the
unique map $M^n \to K^{n + 1}$ such that
$\text{d} \circ s^n - s^{n + 1} \circ \text{d} = \alpha \circ \delta^n$,
see proof of
Homology, Lemma \ref{homology-lemma-ses-termwise-split-cochain}.
The composition $M \to C(f) \to M$ is the identity.
The composition $C(f) \to M \to C(f)$ is equal to the morphism
$$
\left(
\begin{matrix}
s \circ \beta & 0 \\
-\delta \circ \beta & 0
\end{matrix}
\right)
$$
To see that this is homotopic to the identity map
use the homotopy $h : C(\alpha) \to C(\alpha)$
given by the matrix
$$
\left(
\begin{matrix}
0 & 0 \\
\pi & 0
\end{matrix}
\right) :
C(\alpha) = L \oplus K
\to
L \oplus K = C(\alpha)
$$
It is trivial to verify that
$$
\left(
\begin{matrix}
1 & 0 \\
0 & 1
\end{matrix}
\right)
-
\left(
\begin{matrix}
s &
-\delta
\end{matrix}
\right)
\left(
\begin{matrix}
\beta \\
0
\end{matrix}
\right)
=
\left(
\begin{matrix}
\text{d} & \alpha \\
0 & -\text{d}
\end{matrix}
\right)
\left(
\begin{matrix}
0 & 0 \\
\pi & 0
\end{matrix}
\right)
+
\left(
\begin{matrix}
0 & 0 \\
\pi & 0
\end{matrix}
\right)
\left(
\begin{matrix}
\text{d} & \alpha \\
0 & -\text{d}
\end{matrix}
\right)
$$
To finish the proof of (1) we have to show that the morphisms
$-p : C(\alpha) \to K[1]$ (see
Definition \ref{definition-cone})
and $C(\alpha) \to M \to K[1]$ agree up
to homotopy. This is clear from the above. Namely, we can use the homotopy
inverse $(s, -\delta) : M \to C(\alpha)$
and check instead that the two maps
$M \to K[1]$ agree. And note that
$p \circ (s, -\delta) = -\delta$ as desired.

\medskip\noindent
Proof of (2). We let $\tilde f : K \to \tilde L$,
$s : L \to \tilde L$
and $\pi : L \to L$ be as in
Lemma \ref{lemma-make-injective}. By
Lemmas \ref{lemma-functorial-cone} and \ref{lemma-third-isomorphism}
the triangles $(K, L, C(f), i, p)$ and
$(K, \tilde L, C(\tilde f), \tilde i, \tilde p)$
are isomorphic. Note that we can compose isomorphisms of
triangles. Thus we may replace $L$ by
$\tilde L$ and $f$ by $\tilde f$. In other words
we may assume that $f$ is a termwise split injection.
In this case the result follows from part (1).
\end{proof}







\section{The homotopy category is triangulated}
\label{section-homotopy-triangulated}

\noindent
We first prove that it is pre-triangulated.

\begin{lemma}
\label{lemma-homotopy-category-pre-triangulated}
Let $(A, \text{d})$ be a differential graded algebra.
The homotopy category $K(\text{Mod}_{(A, \text{d})})$
with its natural translation functors and distinguished triangles
is a pre-triangulated category.
\end{lemma}

\begin{proof}
Proof of TR1. By definition every triangle isomorphic to a distinguished
one is distinguished. Also, any triangle $(K, K, 0, 1, 0, 0)$
is distinguished since $0 \to K \to K \to 0 \to 0$ is
an admissible short exact sequence. Finally, given any homomorphism
$f : K \to L$ of differential graded $A$-modules the triangle
$(K, L, C(f), f, i, -p)$ is distinguished by
Lemma \ref{lemma-the-same-up-to-isomorphisms}.

\medskip\noindent
Proof of TR2. Let $(X, Y, Z, f, g, h)$ be a triangle.
Assume $(Y, Z, X[1], g, h, -f[1])$ is distinguished.
Then there exists an admissible short exact sequence
$0 \to K \to L \to M \to 0$ such that the associated
triangle $(K, L, M, \alpha, \beta, \delta)$
is isomorphic to $(Y, Z, X[1], g, h, -f[1])$. Rotating back we see
that $(X, Y, Z, f, g, h)$ is isomorphic to
$(M[-1], K, L, -\delta[-1], \alpha, \beta)$.
It follows from Lemma \ref{lemma-rotate-triangle} that the triangle
$(M[-1], K, L, \delta[-1], \alpha, \beta)$
is isomorphic to
$(M[-1], K, C(\delta[-1]), \delta[-1], i, p)$.
Precomposing the previous isomorphism of triangles with $-1$ on $Y$
it follows that $(X, Y, Z, f, g, h)$ is isomorphic to
$(M[-1], K, C(\delta[-1]), \delta[-1], i, -p)$.
Hence it is distinguished by
Lemma \ref{lemma-the-same-up-to-isomorphisms}.
On the other hand, suppose that $(X, Y, Z, f, g, h)$ is distinguished.
By Lemma \ref{lemma-the-same-up-to-isomorphisms} this means that it is
isomorphic to a triangle of the form
$(K, L, C(f), f, i, -p)$ for some morphism $f$ of
$\text{Mod}_{(A, \text{d})}$. Then the rotated triangle
$(Y, Z, X[1], g, h, -f[1])$ is
isomorphic to $(L, C(f), K[1], i, -p, -f[1])$ which is
isomorphic to the triangle
$(L, C(f), K[1], i, p, f[1])$.
By Lemma \ref{lemma-rotate-cone} this triangle is distinguished.
Hence $(Y, Z, X[1], g, h, -f[1])$ is distinguished as desired.

\medskip\noindent
Proof of TR3. Let $(X, Y, Z, f, g, h)$ and $(X', Y', Z', f', g', h')$
be distinguished triangles of $K(\mathcal{A})$ and let $a : X \to X'$
and $b : Y \to Y'$ be morphisms such that $f' \circ a = b \circ f$. By
Lemma \ref{lemma-the-same-up-to-isomorphisms} we may assume that
$(X, Y, Z, f, g, h) = (X, Y, C(f), f, i, -p)$ and
$(X', Y', Z', f', g', h') = (X', Y', C(f'), f', i', -p')$.
At this point we simply apply Lemma \ref{lemma-functorial-cone}
to the commutative diagram given by $f, f', a, b$.
\end{proof}

\noindent
Before we prove TR4 in general we prove it in a special case.

\begin{lemma}
\label{lemma-two-split-injections}
Let $(A, \text{d})$ be a differential graded algebra. Suppose that
$\alpha : K \to L$ and $\beta : L \to M$ are admissible monomorphisms
of differential graded $A$-modules. Then there exist distinguished triangles
$(K, L, Q_1, \alpha, p_1, d_1)$, $(K, M, Q_2, \beta \circ \alpha, p_2, d_2)$
and $(L, M, Q_3, \beta, p_3, d_3)$ for which TR4 holds.
\end{lemma}

\begin{proof}
Say $\pi_1 : L \to K$ and $\pi_3 : M \to L$ are homomorphisms
of graded $A$-modules which are left inverse to $\alpha$ and $\beta$.
Then also $K \to M$ is an admissible monomorphism with left
inverse $\pi_2 = \pi_1 \circ \pi_3$. Let us write $Q_1$, $Q_2$
and $Q_3$ for the cokernels of $K \to L$, $K \to M$, and $L \to M$.
Then we obtain identifications (as graded $A$-modules)
$Q_1 = \text{Ker}(\pi_1)$, $Q_3 = \text{Ker}(\pi_3)$ and
$Q_2 = \text{Ker}(\pi_2)$. Then $L = K \oplus Q_1$ and
$M_n = L \oplus Q_3$ as graded $A$-modules. This implies
$M = K \oplus Q_1 \oplus Q_3$. Note that $\pi_2 = \pi_1 \circ \pi_3$
is zero on both $Q_1$ and $Q_3$. Hence $Q_2 = Q_1 \oplus Q_3$.
Consider the commutative diagram
$$
\begin{matrix}
0 & \to & K & \to & L & \to & Q_1 & \to & 0 \\
  &     & \downarrow&     & \downarrow&     & \downarrow  & \\
0 & \to & K & \to & M & \to & Q_2 & \to & 0 \\
  &     & \downarrow&     & \downarrow&     & \downarrow  & \\
0 & \to & L & \to & M & \to & Q_3 & \to & 0
\end{matrix}
$$
The rows of this diagram are admissible short exact sequences, and
hence determine distinguished triangles by definition. Moreover
downward arrows in the diagram above are compatible with the chosen
splittings and hence define morphisms of triangles
$$
(K \to L \to Q_1 \to K[1])
\longrightarrow
(K \to M \to Q_2 \to K[1])
$$
and
$$
(K \to M \to Q_2 \to K[1])
\longrightarrow
(L \to M \to Q_3 \to L[1]).
$$
Note that the splittings $Q_3 \to M$ of the bottom sequence in the
diagram provides a splitting for the split sequence
$0 \to Q_1 \to Q_2 \to Q_3 \to 0$ upon composing with $M \to Q_2$.
It follows easily from this that the morphism $\delta : Q_3 \to Q_1[1]$
in the corresponding distinguished triangle
$$
(Q_1 \to Q_2 \to Q_3 \to Q_1[1])
$$
is equal to the composition $Q_3 \to L[1] \to Q_1[1]$.
Hence we get a structure as in the conclusion of axiom TR4.
\end{proof}

\noindent
Here is the final result.

\begin{proposition}
\label{proposition-homotopy-category-triangulated}
Let $(A, \text{d})$ be a differential graded algebra. The homotopy category
$K(\text{Mod}_{(A, \text{d})})$ of differential graded $A$-modules with its
natural translation functors and distinguished triangles is a triangulated
category.
\end{proposition}

\begin{proof}
We know that $K(\text{Mod}_{(A, \text{d})})$ is a pre-triangulated category.
Hence it suffices to prove TR4 and to prove it we can use
Derived Categories, Lemma \ref{derived-lemma-easier-axiom-four}.
Let $K \to L$ and $L \to M$ be composable morphisms of
$K(\text{Mod}_{(A, \text{d})})$. By
Lemma \ref{lemma-sequence-maps-split} we may assume that
$K \to L$ and $L \to M$ are admissible monomorphisms.
In this case the result follows from
Lemma \ref{lemma-two-split-injections}.
\end{proof}







\section{P-resolutions}
\label{section-P-resolutions}

\noindent
This section is the analogue of
Derived Categories, Section \ref{derived-section-unbounded}.

\medskip\noindent
Let $(A, \text{d})$ be a differential graded algebra.
Let $P$ be a differential graded $A$-module. We say $P$
{\it has property (F)} if it there exists a filtration
$$
0 = F_{-1}P \subset F_0P \subset F_1P \subset \ldots \subset P
$$
by differential graded submodules such that
\begin{enumerate}
\item $P = \bigcup F_pP$,
\item the inclusions $F_iP \to F_{i + 1}P$ are admissible
monomorphisms,
\item the quotiens $F_{i + 1}P/F_iP$ are isomorphic to a
direct summand of a direct sum of shifts of $A$.
\end{enumerate}
In fact, condition (2) is a consequence of condition (3), see
Lemma \ref{lemma-target-graded-projective}.

\begin{lemma}
\label{lemma-property-F-sequence}
Let $(A, \text{d})$ be a differential graded algebra.
Let $P$ be a differential graded $A$-module. If $F_\bullet$
is a filtration as in property (F), then we obtain an
admissible short exact sequence
$$
0 \to
\bigoplus\nolimits F_iP \to
\bigoplus\nolimits F_iP \to P \to 0
$$
of differential graded $A$-modules.
\end{lemma}

\begin{proof}
Omitted.
\end{proof}

\noindent
The following lemma shows that differential graded modules with
property (F) are the dual notion to K-injective modules
(i.e., they are K-projective in some sense). See
Derived Categories, Definition \ref{derived-definition-K-injective}.

\begin{lemma}
\label{lemma-property-F-K-projective}
Let $(A, \text{d})$ be a differential graded algebra.
Let $P$ be a differential graded $A$-module with property (F).
Then
$$
\Hom_{K(\text{Mod}_{(A, \text{d})})}(P, N) = 0
$$
for all acyclic differential graded $A$-modules $N$.
\end{lemma}

\begin{proof}
We will use that $K(\text{Mod}_{(A, \text{d})})$ is a triangulated
category (Proposition \ref{proposition-homotopy-category-triangulated}).
Let $F_\bullet$ be a filtration on $P$ as in property (F).
The short exact sequence of Lemma \ref{lemma-property-F-sequence}
produces a distinguished triangle. Hence by
Derived Categories, Lemma \ref{derived-lemma-representable-homological}
it suffices to show that
$$
\Hom_{K(\text{Mod}_{(A, \text{d})})}(F_iP, N) = 0
$$
for all acyclic differential graded $A$-modules $N$ and all $i$.
Each of the differential graded modules $F_iP$ has a finite filtration
by admissible monomorphisms, whose graded pieces are direct summands
of direct sums of shifts of $A$. Thus it suffices to prove that
$$
\Hom_{K(\text{Mod}_{(A, \text{d})})}(A[k], N) = 0
$$
for all acyclic differential graded $A$-modules $N$ and all $k$.
Since
$$
\Hom_{K(\text{Mod}_{(A, \text{d})})}(A[k], M) = H^{-k}(M)
$$
for any differential graded $A$-module $M$ we conclude. We'd like to
point out that we use that $A$ is unital here.
\end{proof}

\begin{lemma}
\label{lemma-good-quotient}
Let $(A, \text{d})$ be a differential graded algebra.
Let $M$ be a differential graded $A$-module. There exists a homomorphism
$P \to M$ of differential graded $A$-modules with the following
properties
\begin{enumerate}
\item $P \to M$ is surjective,
\item $\text{Ker}(\text{d}_P) \to \text{Ker}(\text{d}_M)$ is surjective, and
\item $P$ sits in an admissible short exact sequence
$0 \to P' \to P \to P'' \to 0$ where $P'$, $P''$ are direct sums
of shifts of $A$.
\end{enumerate}
\end{lemma}

\begin{proof}
Let $P_k$ be the free $A$-module with generators $x, y$ in degrees
$k$ and $k + 1$. Define the structure of a differential graded
$A$-module on $P_k$ by setting $\text{d}(x) = y$ and $\text{d}(y) = 0$.
For every element $m \in M^k$ there is a homomorphism
$P_k \to M$ sending $x$ to $m$ and $y$ to $\text{d}(m)$.
Thus we see that there is a surjection from a direct sum
of copies of $P_k$ to $M$. This clearly produces $P \to M$
having properties (1) and (3). To obtain property (2) note
that if $m \in \text{Ker}(\text{d}_M)$ has degree $k$, then there is a map
$A[k] \to M$ mapping $1$ to $m$. Hence we can achieve (2) by adding
a direct sum of copies of shifts of $A$.
\end{proof}

\begin{lemma}
\label{lemma-resolve}
Let $(A, \text{d})$ be a differential graded algebra.
Let $M$ be a differential graded $A$-module. There exists a homomorphism
$P \to M$ of differential graded $A$-modules such that
\begin{enumerate}
\item $P \to M$ is a quasi-isomorphism, and
\item $P$ has property (F).
\end{enumerate}
\end{lemma}

\begin{proof}
Set $M = M_0$. We inductively choose short exact sequences
$$
0 \to M_{i + 1} \to P_i \to M_i \to 0
$$
where the maps $P_i \to M_i$ are chosen as in Lemma \ref{lemma-good-quotient}.
This gives a ``resolution''
$$
\ldots \to P_2 \xrightarrow{f_2} P_1 \xrightarrow{f_1} P_0 \to M \to 0
$$
Then we set
$$
P = \bigoplus\nolimits_{i \geq 0} P_i
$$
as an $A$-module with grading given by
$P^n = \bigoplus_{a + b = n} P_{-a}^b$ and
differential (as in the construction of the total complex associated
to a double complex) by
$$
\text{d}_P(x) = f_{-a}(x) + (-1)^a \text{d}_{P_{-a}}
$$
for $x \in P_{-a}^b$. With these conventions $P$ is indeed a differential
graded $A$-module. Recalling that each $P_i$ has a two step filtration
$0 \to P_i' \to P_i \to P_i'' \to 0$ we set
$$
F_{2i}P = \bigoplus\nolimits_{i \geq j \geq 0} P_j
\subset
\bigoplus\nolimits_{i \geq 0} P_i = P
$$
and we add $P'_{i + 1}$ to $F_{2i}P$ to get $F_{2i + 1}$.
These are differential graded submodules and the successive quotients
are direct sums of shifts of $A$. By
Lemma \ref{lemma-target-graded-projective} we see that
the inclusions $F_iP \to F_{i + 1}P$ are admissible monomorphisms.
Finally, we have to show that the map $P \to M$ (given by the
augmentation $P_0 \to M$) is a quasi-isomorphism. This is a special
case of
Homology, Lemma \ref{homology-lemma-good-resolution-gives-qis}.
\end{proof}







\section{The derived category}
\label{section-derived}

\noindent
Recall that the notions of acyclic differential graded modules
and quasi-isomorphism of differential graded modules make sense
(see Section \ref{section-modules}).

\begin{lemma}
\label{lemma-acyclic}
Let $(A, \text{d})$ be a differential graded algebra.
The full subcategory $\text{Ac}$ of $K(\text{Mod}_{(A, \text{d})})$
consisting of acyclic modules is a strictly full saturated triangulated
subcategory of $K(\text{Mod}_{(A, \text{d})})$.
The corresponding saturated multiplicative system
(see Derived Categories, Lemma \ref{derived-lemma-operations})
of $K(\text{Mod}_{(A, \text{d})})$ is the class $\text{Qis}$
of quasi-isomorphisms. In particular, the kernel of the localization
functor
$$
Q : K(\text{Mod}_{(A, \text{d})}) \to
\text{Qis}^{-1}K(\text{Mod}_{(A, \text{d})})
$$
is $\text{Ac}$. Moreover, the functor $H^0$ factors through $Q$.
\end{lemma}

\begin{proof}
We know that $H^0$ is a homological functor by the long exact
sequence of homology (\ref{equation-les}).
The kernel of $H^0$ is the subcategory of acyclic objects and
the arrows with induce isomorphisms on all $H^i$ are the
quasi-isomorphisms. Thus this lemma is a special case of
Derived Categories, Lemma \ref{derived-lemma-acyclic-general}.

\medskip\noindent
Set theoretical remark. The construction of the localization in
Derived Categories, Proposition
\ref{derived-proposition-construct-localization}
assumes the given triangulated category is ``small'', i.e., that the
underlying collection of objects forms a set. Let $V_\alpha$ be a
partial universe (as in Sets, Section \ref{sets-section-sets-hierarchy})
containing $(A, \text{d})$ and where the cofinality of $\alpha$
is bigger than $\aleph_0$
(see Sets, Proposition \ref{sets-proposition-exist-ordinals-large-cofinality}).
Then we can consider the category $\text{Mod}_{(A, \text{d}), \alpha}$
of differential graded $A$-modules contained in $V_\alpha$.
A straightforward check shows that all the constructions used in
the proof of Propostion \ref{proposition-homotopy-category-triangulated}
work inside of $\text{Mod}_{(A, \text{d}), \alpha}$
(because at worst we take finite direct sums of differential graded modules).
Thus we obtain a triangulated category
$\text{Qis}_\alpha^{-1}K(\text{Mod}_{(A, \text{d}), \alpha})$.
We will see below that if $\beta > \alpha$, then the transition functors
$$
\text{Qis}_\alpha^{-1}K(\text{Mod}_{(A, \text{d}), \alpha})
\longrightarrow
\text{Qis}_\beta^{-1}K(\text{Mod}_{(A, \text{d}), \beta})
$$
are fully faithful as the morphism sets in the quotient categories
are computed by maps in the homotopy categories from P-resolutions
(the construction of a P-resolution in the proof of Lemma \ref{lemma-resolve}
takes countable direct sums as well as direct sums indexed over subsets
of the given module). The reader should therefore think of the category
of the lemma as the union of these subcategories.
\end{proof}

\noindent
Taking into account the set theoretical remark at the end of the
proof of the preceding lemma we define the derived category as follows.

\begin{definition}
\label{definition-unbounded-derived-category}
Let $(A, \text{d})$ be a differential graded algebra.
Let $\text{Ac}$ and $\text{Qis}$ be as in Lemma \ref{lemma-acyclic}.
The {\it derived category of $(A, \text{d})$} is the triangulated
category
$$
D(A, \text{d}) =
K(\text{Mod}_{(A, \text{d})})/\text{Ac} =
\text{Qis}^{-1}K(\text{Mod}_{(A, \text{d})}).
$$
We denote $H^0 : D(A, \text{d}) \to \text{Mod}_R$ the unique functor
whose composition with the quotient functor gives back the functor
$H^0$ defined above.
\end{definition}

\noindent
Here is the promised lemma computing the morphisms in the
derived category.

\begin{lemma}
\label{lemma-hom-derived}
Let $(A, \text{d})$ be a differential graded algebra.
Let $M$ and $N$ be differential graded $A$-modules.
Let $P \to M$ be a P-resolution as in
Lemma \ref{lemma-resolve}. Then
$$
\Hom_{D(A, \text{d})}(M, N) =
\Hom_{K(\text{Mod}_{(A, \text{d})})}(P, N)
$$
\end{lemma}

\begin{proof}
Since $P \to M$ is a quasi-isomorphism we see that
$$
\Hom_{D(A, \text{d})}(P, N) = \Hom_{D(A, \text{d})}(M, N)
$$
by definition of the derived category. A morphism
$f : P \to N$ in $D(A, \text{d})$ is equal to
$s^{-1}f'$ where $f' : P \to N'$ is a morphism and
$s : N \to N'$ is a quasi-isomorphism. Choose a distringuished triangle
$$
N \to N' \to Q \to N[1]
$$
As $s$ is a quasi-isomorphism, we see that $Q$ is acyclic. Thus
$\Hom_{K(\text{Mod}_{(A, \text{d})})}(P, Q[k]) = 0$ for all $k$ by
Lemma \ref{lemma-property-F-K-projective}. Since
$\Hom_{K(\text{Mod}_{(A, \text{d})})}(P, -)$
is cohomological, we conclude that we can lift $f' : P \to N'$
uniquely to a morphism $f : P \to N$. This finishes the proof.
\end{proof}









\section{The canonical delta-functor}
\label{section-canonical-delta-functor}

\noindent
Let $(A, \text{d})$ be a differential graded algebra.
Consider the functor
$\text{Mod}(\mathcal{A}) \to K(\text{Mod}_{(A, \text{d})})$.
This functor is {\bf not} a $\delta$-functor in general.
However, it turns out that the functor
$\text{Mod}_{(A, \text{d})} \to D(A, \text{d})$ is a
$\delta$-functor. In order to see this we have to define
the morphisms $\delta$ associated to a short exact sequence
$$
0 \to K \xrightarrow{a} L \xrightarrow{b} M \to 0
$$
in the abelian category $\text{Mod}_{(A, \text{d})}$.
Consider the cone $C(a)$ of the morphism $a$. We have $C(a) = L \oplus K$
and we define $q : C(a) \to M$ via the projection to $L$ followed
by $b$. Hence a homomorphism of differential graded $A$-modules
$$
q : C(a) \longrightarrow M.
$$
It is clear that $q \circ i = b$ where $i$ is as in
Definition \ref{definition-cone}.
Note that, as $a$ is injective, the kernel of $q$ is identified with the
cone of $\text{id}_K$ which is acyclic. Hence we see that
$q$ is a quasi-isomorphism. According to
Lemma \ref{lemma-the-same-up-to-isomorphisms}
the triangle
$$
(K, L, C(a), a, i, -p)
$$
is a distinguished triangle in $K(\text{Mod}_{(A, \text{d})})$.
As the localization functor
$K(\text{Mod}_{(A, \text{d})}) \to D(A, \text{d})$ is
exact we see that $(K, L, C(a), a, i, -p)$ is a distinguished
triangle in $D(A, \text{d})$. Since $q$ is a quasi-isomorphism
we see that $q$ is an isomorphism in $D(A, \text{d})$.
Hence we deduce that
$$
(K, L, M, a, b, -p \circ q^{-1})
$$
is a distinguished triangle of $D(A, \text{d})$.
This suggests the following lemma.

\begin{lemma}
\label{lemma-derived-canonical-delta-functor}
Let $(A, \text{d})$ be a differential graded algebra. The functor
$\text{Mod}_{(A, \text{d})} \to D(A, \text{d})$
defined has the natural structure of a $\delta$-functor, with
$$
\delta_{K \to L \to M} = - p \circ q^{-1}
$$
with $p$ and $q$ as explained above.
\end{lemma}

\begin{proof}
We have already seen that this choice leads to a distinguished
triangle whenever given a short exact sequence of complexes.
We have to show functoriality of this construction, see
Derived Categories, Definition \ref{derived-definition-delta-functor}.
This follows from Lemma \ref{lemma-functorial-cone} with a bit of
work. Compare with
Derived Categories, Lemma \ref{derived-lemma-derived-canonical-delta-functor}.
\end{proof}








\section{Differential graded categories}
\label{section-dga-categories}

\noindent
Note that if $R$ is a ring, then $R$ is a differential graded algebra
over itself (with $R = R^0$ of course). In this case a differential
graded $R$-module is the same thing as a complex of $R$-modules.
In particular, given two differential graded $R$-modules $M$ and $N$
we denote $M \otimes_R N$ the differential graded $R$-module
corresponding to the total complex associated to the double
complex obtained by the tensor product of the complexes of $R$-modules
associated to $M$ and $N$.

\begin{definition}
\label{definition-dga-category}
Let $R$ be a ring. A {\it differential graded category $\mathcal{A}$
over $R$} is a category where every morphism set is given the structure
of a differential graded $R$-module and where for
$x, y, z \in \Ob(\mathcal{A})$ composition is $R$-bilinear and induces
a homomorphism
$$
\Hom_\mathcal{A}(y, z) \otimes_R \Hom_\mathcal{A}(x, y)
\longrightarrow
\Hom_\mathcal{A}(x, z)
$$
of differential graded $R$-modules.
\end{definition}

\noindent
The final condition of the definition signifies the following:
if $f \in \Hom_\mathcal{A}^n(x, y)$ and
$g \in \Hom_\mathcal{A}^m(y, z)$ are homogeneous
of degrees $n$ and $m$, then
$$
\text{d}(g \circ f) = \text{d}(g) \circ f + (-1)^mg \circ \text{d}(f)
$$
in $\Hom_\mathcal{A}^{n + m + 1}(x, z)$. This follows from the sign
rule for the differential on the total complex of a double complex, see
Homology, Definition \ref{homology-definition-associated-simple-complex}.
An important example is given by categories of complexes.

\begin{example}[Differential graded category of complexes]
\label{example-category-complexes}
Let $\mathcal{B}$ be an additive category. We will construct
a differential graded category $\text{Comp}^{dg}(\mathcal{B})$
over $R = \mathbf{Z}$. As objects of $\text{Comp}^{dg}(\mathcal{B})$
we take complexes of $\mathcal{B}$. Then, given complexes
$A^\bullet$ and $B^\bullet$ of $\mathcal{B}$ we set
$$
\Hom_{\text{Comp}^{dg}(\mathcal{B})}(A^\bullet, B^\bullet) =
\bigoplus\nolimits_{n \in \mathbf{Z}}
\Hom^n(A^\bullet, B^\bullet)
$$
where the graded piece of degree $n$ is defined by the rule
$$
\Hom^n(A^\bullet, B^\bullet) =
\prod\nolimits_{p + q = n} \Hom_\mathcal{B}(A^{-q}, B^p)
$$
(observe reversal of indices). For an element
$f = (f_{p, q}) \in \Hom^n(A^\bullet, B^\bullet)$
with $f_{p, q} \in \Hom_\mathcal{B}(A^{-q}, B^p)$ we set
$$
\text{d}(f) = \prod\nolimits_{p + q = n} \text{d}(f_{p, q})
$$
where we define each $\text{d}(f_{p, q})$ by the formula\footnote{
An ``explanation'' for the convention as in
(\ref{equation-differential-hom-complex}) is as follows.
The differential of the ``dual'' $\Hom(A^\bullet, -)$ of $A^\bullet$
is $(-1)^q$ times the contragredient on the term
$\Hom(A^{-q}, - ) \to \Hom(A^{-q - 1}, -)$.
Then we think of $\Hom_{\text{Comp}^{dg}(\mathcal{B})}(A^\bullet, B^\bullet)$
as $B^\bullet \otimes \Hom(A^\bullet, -)$ and we use
the convention of the differential on the total complex of a double complex
as in
Homology, Definition \ref{homology-definition-associated-simple-complex}.}
\begin{equation}
\label{equation-differential-hom-complex}
\text{d}(f_{p, q}) =
\text{d}_B \circ f_{p, q} + (-1)^{p + q} f_{p, q} \circ \text{d}_A 
\end{equation}
Note that the first term of this expression is in
$\Hom_\mathcal{B}(A^{-q}, B^{p + 1})$ and the second term is in
$\Hom_\mathcal{B}(A^{-q - 1}, B^p)$ hence the infinite product
above makes sense. In other words, given $p + q = n + 1$ we have
$$
\text{d}(f)_{p, q} =
\text{d}_B \circ f_{p - 1, q} - (-1)^n f_{p, q - 1} \circ \text{d}_A
$$
with obvious notation. We often simply write
$\text{d}(f) = \text{d}_B \circ f - (-1)^n f \circ \text{d}_A$.
The reader checks that
\begin{enumerate}
\item $\text{d}$ has square zero,
\item an element $f = (f_{p, -p})$ in $\Hom^0(A^\bullet, B^\bullet)$
has $\text{d}(f) = 0$ if and only if the morphisms $f_{-p, p} : A^p \to B^p$
fit together into a morphism of complexes,
\item the morphism of complexes defined by $f$ as in (2)
is homotopy equivalent to zero if and only if $f = \text{d}(g)$
for some $g \in \Hom^{-1}(A^\bullet, B^\bullet)$.
\item in other words, we obtain a canonical isomorphism
$$
\Hom_{K(\mathcal{B})}(A^\bullet, B^\bullet)
\longrightarrow
H^0(\Hom_{\text{Comp}^{dg}(\mathcal{B})}(A^\bullet, B^\bullet))
$$
\end{enumerate}
Let $A^\bullet$, $B^\bullet$, $C^\bullet$ be complexes.
Composition is defined via the maps
$$
\Hom^m(B^\bullet, C^\bullet) \times \Hom^n(A^\bullet, B^\bullet)
\longrightarrow
\Hom^{n + m}(A^\bullet, C^\bullet)
$$
$(g, f) \mapsto g \circ f$ with $(g \circ f)_{p, r} = g_{p, q} \circ f_{q, r}$
where $q$ is such that $p + q = m$ and $q + r = n$. This defines a map
of differential graded modules as in
Definition \ref{definition-dga-category}
because
\begin{align*}
\text{d}(g \circ f) & =
\text{d}_C \circ g \circ f - (-1)^{n + m} g \circ f \circ \text{d}_A \\
& =
\left(\text{d}_C \circ g - (-1)^m g \circ \text{d}_B\right) \circ f +
(-1)^m g \circ \left(\text{d}_B \circ f - (-1)^n f \circ \text{d}_A\right) \\
& =
\text{d}(g) \circ f + (-1)^m g \circ \text{d}(f)
\end{align*}
as desired.
\end{example}

\begin{example}[Differential graded category of differential graded modules]
\label{example-dgm-dg-cat}
Let $(A, \text{d})$ be a differential graded algebra over a ring $R$.
We will construct a differential graded category
$\text{Mod}^{dg}_{(A, \text{d})}$ over $R$ which is closely related to
$\text{Mod}_{(A, \text{d})}$. As objects of $\text{Mod}^{dg}_{(A, \text{d})}$
we take the differential graded $A$-modules. Given differential
graded $A$-modules $L$ and $M$ we set
$$
\Hom_{\text{Mod}^{dg}_{(A, \text{d})}}(L, M) =
\bigoplus\nolimits_{n \in \mathbf{Z}} \Hom^n(L, M)
$$
where the graded piece of degree $n$ is the submodule
$$
\Hom^n(L, M) \subset \prod\nolimits_{p + q = n} \Hom_R(L^{-q}, M^p)
$$
(observe reversal of indices) consisting of those
$f = (f_{p, q})$ such that
$$
f_{p, q}(x a) = f_{p - i, q + i}(m)a
$$
for $a \in A^i$ and $m \in L^{-q - i}$. In other words, $\Hom^n(L, M)$
is the set of (right) $A$-module maps $L \to M$ which are homogeneous of
degree $n$. For an element $f \in \Hom^n(L, M)$ we set
$$
\text{d}(f) = \prod\nolimits_{p + q = n} \text{d}(f_{p, q})
$$
where we define each $\text{d}(f_{p, q})$ by the formula
$$
\text{d}(f_{p, q}) =
\text{d}_M \circ f_{p, q} + (-1)^{p + q} f_{p, q} \circ \text{d}_L
$$
Note that the first term of this expression is in
$\Hom_R(L^{-q}, M^{p + 1})$ and the second term is in
$\Hom_R(L^{-q - 1}, M^p)$ hence the infinite product
above makes sense. In other words, given $p + q = n + 1$ we have
$$
\text{d}(f)_{p, q} =
\text{d}_M \circ f_{p - 1, q} - (-1)^n f_{p, q - 1} \circ \text{d}_L
$$
with obvious notation. We often simply write
$\text{d}(f) = \text{d}_M \circ f - (-1)^n f \circ \text{d}_L$.
The reader checks that
\begin{enumerate}
\item $\text{d}$ has square zero,
\item an element $f$ in $\Hom^0(L, M)$ has $\text{d}(f) = 0$ if and only if
$f : L \to M$ is a homomorphism of differential graded $A$-modules,
\item the homomorphism defined by $f$ as in (2)
is homotopy equivalent to zero if and only if $f = \text{d}(g)$
for some $g \in \Hom^{-1}(L, M)$.
\item in other words, we obtain a canonical isomorphism
$$
\Hom_{K(\text{Mod}_{(A, \text{d})})}(L, M)
\longrightarrow
H^0(\Hom_{\text{Mod}^{dg}_{(A, \text{d})}}(L, M))
$$
\end{enumerate}
Let $K$, $L$, $M$ be complexes. Composition is defined via the maps
$$
\Hom^m(L, M) \times \Hom^n(K, L) \longrightarrow \Hom^{n + m}(K, M)
$$
by simple composition of right $A$-module maps: $(g, f) \mapsto g \circ f$.
This defines a map of differential graded modules as in
Definition \ref{definition-dga-category}
because
\begin{align*}
\text{d}(g \circ f) & =
\text{d}_M \circ g \circ f - (-1)^{n + m} g \circ f \circ \text{d}_K \\
& =
\left(\text{d}_M \circ g - (-1)^m g \circ \text{d}_L\right) \circ f +
(-1)^m g \circ \left(\text{d}_L \circ f - (-1)^n f \circ \text{d}_K\right) \\
& =
\text{d}(g) \circ f + (-1)^m g \circ \text{d}(f)
\end{align*}
as desired.
\end{example}










\section{Other chapters}

\begin{multicols}{2}
\begin{enumerate}
\item \hyperref[introduction-section-phantom]{Introduction}
\item \hyperref[conventions-section-phantom]{Conventions}
\item \hyperref[sets-section-phantom]{Set Theory}
\item \hyperref[categories-section-phantom]{Categories}
\item \hyperref[topology-section-phantom]{Topology}
\item \hyperref[sheaves-section-phantom]{Sheaves on Spaces}
\item \hyperref[algebra-section-phantom]{Commutative Algebra}
\item \hyperref[sites-section-phantom]{Sites and Sheaves}
\item \hyperref[homology-section-phantom]{Homological Algebra}
\item \hyperref[derived-section-phantom]{Derived Categories}
\item \hyperref[more-algebra-section-phantom]{More Algebra}
\item \hyperref[simplicial-section-phantom]{Simplicial Methods}
\item \hyperref[modules-section-phantom]{Sheaves of Modules}
\item \hyperref[sites-modules-section-phantom]{Modules on Sites}
\item \hyperref[injectives-section-phantom]{Injectives}
\item \hyperref[cohomology-section-phantom]{Cohomology of Sheaves}
\item \hyperref[sites-cohomology-section-phantom]{Cohomology on Sites}
\item \hyperref[hypercovering-section-phantom]{Hypercoverings}
\item \hyperref[schemes-section-phantom]{Schemes}
\item \hyperref[constructions-section-phantom]{Constructions of Schemes}
\item \hyperref[properties-section-phantom]{Properties of Schemes}
\item \hyperref[morphisms-section-phantom]{Morphisms of Schemes}
\item \hyperref[coherent-section-phantom]{Coherent Cohomology}
\item \hyperref[divisors-section-phantom]{Divisors}
\item \hyperref[limits-section-phantom]{Limits of Schemes}
\item \hyperref[varieties-section-phantom]{Varieties}
\item \hyperref[chow-section-phantom]{Chow Homology}
\item \hyperref[topologies-section-phantom]{Topologies on Schemes}
\item \hyperref[descent-section-phantom]{Descent}
\item \hyperref[more-morphisms-section-phantom]{More on Morphisms}
\item \hyperref[flat-section-phantom]{More on Flatness}
\item \hyperref[groupoids-section-phantom]{Groupoid Schemes}
\item \hyperref[more-groupoids-section-phantom]{More on Groupoid Schemes}
\item \hyperref[etale-section-phantom]{\'Etale Morphisms of Schemes}
\item \hyperref[etale-cohomology-section-phantom]{\'Etale Cohomology}
\item \hyperref[spaces-section-phantom]{Algebraic Spaces}
\item \hyperref[spaces-properties-section-phantom]{Properties of Algebraic Spaces}
\item \hyperref[spaces-morphisms-section-phantom]{Morphisms of Algebraic Spaces}
\item \hyperref[spaces-topologies-section-phantom]{Topologies on Algebraic Spaces}
\item \hyperref[spaces-descent-section-phantom]{Descent and Algebraic Spaces}
\item \hyperref[spaces-more-morphisms-section-phantom]{More on Morphisms of Spaces}
\item \hyperref[quot-section-phantom]{Quot and Hilbert Spaces}
\item \hyperref[stacks-section-phantom]{Stacks}
\item \hyperref[spaces-groupoids-section-phantom]{Groupoids in Algebraic Spaces}
\item \hyperref[spaces-more-groupoids-section-phantom]{More on Groupoids in Spaces}
\item \hyperref[bootstrap-section-phantom]{Bootstrap}
\item \hyperref[examples-stacks-section-phantom]{Examples of Stacks}
\item \hyperref[groupoids-quotients-section-phantom]{Quotients of Groupoids}
\item \hyperref[algebraic-section-phantom]{Algebraic Stacks}
\item \hyperref[criteria-section-phantom]{Criteria for Representability}
\item \hyperref[stacks-properties-section-phantom]{Properties of Algebraic Stacks}
\item \hyperref[stacks-morphisms-section-phantom]{Morphisms of Algebraic Stacks}
\item \hyperref[examples-section-phantom]{Examples}
\item \hyperref[exercises-section-phantom]{Exercises}
\item \hyperref[guide-section-phantom]{Guide to Literature}
\item \hyperref[desirables-section-phantom]{Desirables}
\item \hyperref[coding-section-phantom]{Coding Style}
\item \hyperref[fdl-section-phantom]{GNU Free Documentation License}
\item \hyperref[index-section-phantom]{Auto Generated Index}
\end{enumerate}
\end{multicols}


\bibliography{my}
\bibliographystyle{amsalpha}

\end{document}
