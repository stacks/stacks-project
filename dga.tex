\IfFileExists{stacks-project.cls}{%
\documentclass{stacks-project}
}{%
\documentclass{amsart}
}

% The following AMS packages are automatically loaded with
% the amsart documentclass:
%\usepackage{amsmath}
%\usepackage{amssymb}
%\usepackage{amsthm}

% For dealing with references we use the comment environment
\usepackage{verbatim}
\newenvironment{reference}{\comment}{\endcomment}
%\newenvironment{reference}{}{}
\newenvironment{slogan}{\comment}{\endcomment}
\newenvironment{history}{\comment}{\endcomment}

% For commutative diagrams you can use
% \usepackage{amscd}
\usepackage[all]{xy}

% We use 2cell for 2-commutative diagrams.
\xyoption{2cell}
\UseAllTwocells

% To put source file link in headers.
% Change "template.tex" to "this_filename.tex"
% \usepackage{fancyhdr}
% \pagestyle{fancy}
% \lhead{}
% \chead{}
% \rhead{Source file: \url{template.tex}}
% \lfoot{}
% \cfoot{\thepage}
% \rfoot{}
% \renewcommand{\headrulewidth}{0pt}
% \renewcommand{\footrulewidth}{0pt}
% \renewcommand{\headheight}{12pt}

\usepackage{multicol}

% For cross-file-references
\usepackage{xr-hyper}

% Package for hypertext links:
\usepackage{hyperref}

% For any local file, say "hello.tex" you want to link to please
% use \externaldocument[hello-]{hello}
\externaldocument[introduction-]{introduction}
\externaldocument[conventions-]{conventions}
\externaldocument[sets-]{sets}
\externaldocument[categories-]{categories}
\externaldocument[topology-]{topology}
\externaldocument[sheaves-]{sheaves}
\externaldocument[sites-]{sites}
\externaldocument[stacks-]{stacks}
\externaldocument[fields-]{fields}
\externaldocument[algebra-]{algebra}
\externaldocument[brauer-]{brauer}
\externaldocument[homology-]{homology}
\externaldocument[derived-]{derived}
\externaldocument[simplicial-]{simplicial}
\externaldocument[more-algebra-]{more-algebra}
\externaldocument[smoothing-]{smoothing}
\externaldocument[modules-]{modules}
\externaldocument[sites-modules-]{sites-modules}
\externaldocument[injectives-]{injectives}
\externaldocument[cohomology-]{cohomology}
\externaldocument[sites-cohomology-]{sites-cohomology}
\externaldocument[dga-]{dga}
\externaldocument[dpa-]{dpa}
\externaldocument[hypercovering-]{hypercovering}
\externaldocument[schemes-]{schemes}
\externaldocument[constructions-]{constructions}
\externaldocument[properties-]{properties}
\externaldocument[morphisms-]{morphisms}
\externaldocument[coherent-]{coherent}
\externaldocument[divisors-]{divisors}
\externaldocument[limits-]{limits}
\externaldocument[varieties-]{varieties}
\externaldocument[topologies-]{topologies}
\externaldocument[descent-]{descent}
\externaldocument[perfect-]{perfect}
\externaldocument[more-morphisms-]{more-morphisms}
\externaldocument[flat-]{flat}
\externaldocument[groupoids-]{groupoids}
\externaldocument[more-groupoids-]{more-groupoids}
\externaldocument[etale-]{etale}
\externaldocument[chow-]{chow}
\externaldocument[intersection-]{intersection}
\externaldocument[pic-]{pic}
\externaldocument[adequate-]{adequate}
\externaldocument[dualizing-]{dualizing}
\externaldocument[duality-]{duality}
\externaldocument[discriminant-]{discriminant}
\externaldocument[local-cohomology-]{local-cohomology}
\externaldocument[curves-]{curves}
\externaldocument[resolve-]{resolve}
\externaldocument[models-]{models}
\externaldocument[pione-]{pione}
\externaldocument[etale-cohomology-]{etale-cohomology}
\externaldocument[proetale-]{proetale}
\externaldocument[crystalline-]{crystalline}
\externaldocument[spaces-]{spaces}
\externaldocument[spaces-properties-]{spaces-properties}
\externaldocument[spaces-morphisms-]{spaces-morphisms}
\externaldocument[decent-spaces-]{decent-spaces}
\externaldocument[spaces-cohomology-]{spaces-cohomology}
\externaldocument[spaces-limits-]{spaces-limits}
\externaldocument[spaces-divisors-]{spaces-divisors}
\externaldocument[spaces-over-fields-]{spaces-over-fields}
\externaldocument[spaces-topologies-]{spaces-topologies}
\externaldocument[spaces-descent-]{spaces-descent}
\externaldocument[spaces-perfect-]{spaces-perfect}
\externaldocument[spaces-more-morphisms-]{spaces-more-morphisms}
\externaldocument[spaces-flat-]{spaces-flat}
\externaldocument[spaces-groupoids-]{spaces-groupoids}
\externaldocument[spaces-more-groupoids-]{spaces-more-groupoids}
\externaldocument[bootstrap-]{bootstrap}
\externaldocument[spaces-pushouts-]{spaces-pushouts}
\externaldocument[groupoids-quotients-]{groupoids-quotients}
\externaldocument[spaces-more-cohomology-]{spaces-more-cohomology}
\externaldocument[spaces-simplicial-]{spaces-simplicial}
\externaldocument[formal-spaces-]{formal-spaces}
\externaldocument[restricted-]{restricted}
\externaldocument[spaces-resolve-]{spaces-resolve}
\externaldocument[formal-defos-]{formal-defos}
\externaldocument[defos-]{defos}
\externaldocument[cotangent-]{cotangent}
\externaldocument[examples-defos-]{examples-defos}
\externaldocument[algebraic-]{algebraic}
\externaldocument[examples-stacks-]{examples-stacks}
\externaldocument[stacks-sheaves-]{stacks-sheaves}
\externaldocument[criteria-]{criteria}
\externaldocument[artin-]{artin}
\externaldocument[quot-]{quot}
\externaldocument[stacks-properties-]{stacks-properties}
\externaldocument[stacks-morphisms-]{stacks-morphisms}
\externaldocument[stacks-limits-]{stacks-limits}
\externaldocument[stacks-cohomology-]{stacks-cohomology}
\externaldocument[stacks-perfect-]{stacks-perfect}
\externaldocument[stacks-introduction-]{stacks-introduction}
\externaldocument[stacks-more-morphisms-]{stacks-more-morphisms}
\externaldocument[stacks-geometry-]{stacks-geometry}
\externaldocument[moduli-]{moduli}
\externaldocument[moduli-curves-]{moduli-curves}
\externaldocument[examples-]{examples}
\externaldocument[exercises-]{exercises}
\externaldocument[guide-]{guide}
\externaldocument[desirables-]{desirables}
\externaldocument[coding-]{coding}
\externaldocument[obsolete-]{obsolete}
\externaldocument[fdl-]{fdl}
\externaldocument[index-]{index}

% Theorem environments.
%
\theoremstyle{plain}
\newtheorem{theorem}[subsection]{Theorem}
\newtheorem{proposition}[subsection]{Proposition}
\newtheorem{lemma}[subsection]{Lemma}

\theoremstyle{definition}
\newtheorem{definition}[subsection]{Definition}
\newtheorem{example}[subsection]{Example}
\newtheorem{exercise}[subsection]{Exercise}
\newtheorem{situation}[subsection]{Situation}

\theoremstyle{remark}
\newtheorem{remark}[subsection]{Remark}
\newtheorem{remarks}[subsection]{Remarks}

\numberwithin{equation}{subsection}

% Macros
%
\def\lim{\mathop{\rm lim}\nolimits}
\def\colim{\mathop{\rm colim}\nolimits}
\def\Spec{\mathop{\rm Spec}}
\def\Hom{\mathop{\rm Hom}\nolimits}
\def\Ext{\mathop{\rm Ext}\nolimits}
\def\SheafHom{\mathop{\mathcal{H}\!{\it om}}\nolimits}
\def\SheafExt{\mathop{\mathcal{E}\!{\it xt}}\nolimits}
\def\Sch{\textit{Sch}}
\def\Mor{\mathop{\rm Mor}\nolimits}
\def\Ob{\mathop{\rm Ob}\nolimits}
\def\Sh{\mathop{\textit{Sh}}\nolimits}
\def\NL{\mathop{N\!L}\nolimits}
\def\proetale{{pro\text{-}\acute{e}tale}}
\def\etale{{\acute{e}tale}}
\def\QCoh{\textit{QCoh}}
\def\Ker{\mathop{\rm Ker}}
\def\Im{\mathop{\rm Im}}
\def\Coker{\mathop{\rm Coker}}
\def\Coim{\mathop{\rm Coim}}

%
% Macros for moduli stacks/spaces
%
\def\QCohstack{\mathcal{QC}\!{\it oh}}
\def\Cohstack{\mathcal{C}\!{\it oh}}
\def\Spacesstack{\mathcal{S}\!{\it paces}}
\def\Quotfunctor{{\rm Quot}}
\def\Hilbfunctor{{\rm Hilb}}
\def\Curvesstack{\mathcal{C}\!{\it urves}}
\def\Polarizedstack{\mathcal{P}\!{\it olarized}}
\def\Complexesstack{\mathcal{C}\!{\it omplexes}}
% \Pic is the operator that assigns to X its picard group, usage \Pic(X)
% \Picardstack_{X/B} denotes the Picard stack of X over B
% \Picardfunctor_{X/B} denotes the Picard functor of X over B
\def\Pic{\mathop{\rm Pic}\nolimits}
\def\Picardstack{\mathcal{P}\!{\it ic}}
\def\Picardfunctor{{\rm Pic}}
\def\Deformationcategory{\mathcal{D}\!{\it ef}}


% OK, start here.
%
\begin{document}

\title{Differential Graded Algebra}


\maketitle

\phantomsection
\label{section-phantom}

\tableofcontents

\section{Introduction}
\label{section-introduction}

\noindent
In this chapter we talk about differential graded algebras, modules,
categories, etc. A basic reference is \cite{Keller-Deriving}.
A survey paper is \cite{Keller-survey}.




\section{Conventions}
\label{section-conventions}

\noindent
In this chapter we hold on to the convention that {\it ring} means
commutative ring with $1$. If $R$ is a ring, then an {\it $R$-algebra $A$}
will be an $R$-module $A$ endowed with an $R$-bilinear map $A \times A \to A$
(multiplication) such that multiplication is associative and has a unit.
In other words, these are unital associative $R$-algebras
such that the structure map $R \to A$ maps into the center of $A$.




\section{Differential graded algebras}
\label{section-dga}


\noindent
Just the definitions.

\begin{definition}
\label{definition-dga}
Let $R$ be a commutative ring. A {\it differential graded algebra over $R$}
is either
\begin{enumerate}
\item a chain complex $A_\bullet$ of $R$-modules endowed with
$R$-bilinear maps $A_n \times A_m \to A_{n + m}$,
$(a, b) \mapsto ab$ such that
$$
\text{d}_{n + m}(ab) = \text{d}_n(a)b + (-1)^n a\text{d}_m(b)
$$
and such that $\bigoplus A_n$ becomes an associative and unital
$R$-algebra, or
\item a cochain complex $A^\bullet$ of $R$-modules endowed with
$R$-bilinear maps $A^n \times A^m \to A^{n + m}$, $(a, b) \mapsto ab$
such that
$$
\text{d}^{n + m}(ab) = \text{d}^n(a)b + (-1)^n a\text{d}^m(b)
$$
and such that $\bigoplus A^n$ becomes an associative and unital $R$-algebra.
\end{enumerate}
\end{definition}

\noindent
We often just write $A = \bigoplus A_n$ or $A = \bigoplus A^n$ and
think of this as an associative unital $R$-algebra endowed with a
$\mathbf{Z}$-grading and an $R$-linear operator $\text{d}$ whose square
is zero and which satisfies the Leibniz rule as explained above. In this case
we often say ``Let $(A, \text{d})$ be a differential graded algebra''.

\begin{definition}
\label{definition-homomorphism-dga}
A {\it homomorphism of differential graded algebras}
$f : (A, \text{d}) \to (B, \text{d})$ is an algebra map $f : A \to B$
compatible with the gradings and $\text{d}$.
\end{definition}

\begin{definition}
\label{definition-cdga}
A differential graded algebra $(A, \text{d})$ is {\it commutative} if
$ab = (-1)^{nm}ba$ for $a$ in degree $n$ and $b$ in degree $m$.
We say $A$ is {\it strictly commutative} if in addition $a^2 = 0$
for $\deg(a)$ odd.
\end{definition}

\noindent
The following definition makes sense in general but is perhaps
``correct'' only when tensoring commutative differential graded
algebras.

\begin{definition}
\label{definition-tensor-product}
Let $R$ be a ring.
Let $(A, \text{d})$, $(B, \text{d})$ be differential graded algebras over $R$.
The {\it tensor product differential graded algebra} of $A$ and $B$
is the algebra $A \otimes_R B$ with multiplication defined by
$$
(a \otimes b)(a' \otimes b') = (-1)^{\deg(a')\deg(b)} aa' \otimes bb'
$$
endowed with differential $\text{d}$ defined by the rule
$\text{d}(a \otimes b) = \text{d}(a) \otimes b + (-1)^m a \otimes \text{d}(b)$
where $m = \deg(b)$.
\end{definition}

\begin{lemma}
\label{lemma-total-complex-tensor-product}
Let $R$ be a ring.
Let $(A, \text{d})$, $(B, \text{d})$ be differential graded algebras over $R$.
Denote $A^\bullet$, $B^\bullet$ the underlying cochain complexes.
As cochain complexes of $R$-modules we have
$$
(A \otimes_R B)^\bullet = \text{Tot}(A^\bullet \otimes_A B^\bullet).
$$
\end{lemma}

\begin{proof}
Recall that the differential of the total complex is given by
$\text{d}_1^{p, q} + (-1)^p \text{d}_2^{p, q}$ on $A^p \otimes_R B^q$.
And this is exactly the same as the rule for the differential
on $A \otimes_R B$ in
Definition \ref{definition-tensor-product}.
\end{proof}






\section{Differential graded modules}
\label{section-modules}

\noindent
Just the definitions.

\begin{definition}
\label{definition-dgm}
Let $(A, \text{d})$ be a differential graded algebra.
A (left) {\it differential graded module} $M$ over $A$ is a left $A$-module
$M$ which has a grading $M = \bigoplus M^n$ and a differential $\text{d}$
such that $A^n M^m \subset M^{n + m}$, such that
$\text{d}(M^n) \subset M^{n + 1}$, and such that
$$
\text{d}(am) = \text{d}(a)m + (-1)^na\text{d}m
$$
for $a \in A^n$ and $m \in M$. A
{\it homomorphism of differential graded modules} $f : M \to N$
is an $A$-module map compatible with gradings and differentials.
The category of differential graded $A$-modules is denoted
$\text{Mod}_{(A, \text{d})}$.
\end{definition}

\noindent
Note that we can think of $M$ as a cochain complex $M^\bullet$
of $R$-modules. Namely, for $r \in R$ we have $\text{d}(r) = 0$
and $r$ maps to a degree $0$ element of $A$, hence
$\text{d}(rm) = r\text{d}(m)$

\begin{lemma}
\label{lemma-dgm-abelian}
Let $(A, d)$ be a differential graded algebra. Then
$\text{Mod}_{(A, \text{d})}$ is abelian.
\end{lemma}

\begin{proof}
Omitted. Hint: kernels and cokernels commute with taking
homogeneous components.
\end{proof}

\noindent
Thus, if $(A, \text{d})$ is a differential graded
algebra over $R$, then there is an exact functor
$$
\text{Mod}_{(A, \text{d})} \longrightarrow \text{Comp}(R)
$$
of abelian categories. For a differential graded module $M$ the
cohomology groups $H^n(M)$ are defined as the cohomology of the
corresponding complex of $R$-modules. Therefore, a short exact
sequence $0 \to K \to L \to M \to 0$ of differential graded modules
gives rise to a long exact sequence
\begin{equation}
\label{equation-les}
H^n(K) \to H^n(L) \to H^n(M) \to H^{n + 1}(K)
\end{equation}
of cohomology modules, see
Homology, Lemma \ref{homology-lemma-long-exact-sequence-cochain}.

\begin{definition}
\label{definition-shift}
Let $(A, \text{d})$ be a differential graded algebra.
Let $M$ be a differential graded module.
For any $k \in \mathbf{Z}$ we define the {\it $k$-shifted module}
$M[k]$ as follows
\begin{enumerate}
\item as $A$-module $M[k] = M$,
\item $M[k]^n = M^{n + k}$,
\item $\text{d}_{M[k]} = (-1)^k\text{d}_M$.
\end{enumerate}
For a morphism $f : M \to N$ of differential graded $A$-modules
we let $f[k] : M[k] \to N[k]$ be the map equal to $f$ on underlying
$A$-modules. This defines a functor
$[k] : \text{Mod}_{(A, \text{d})} \to \text{Mod}_{(A, \text{d})}$.
\end{definition}

\noindent
The remarks in Homology, Section \ref{homology-section-homotopy-shift} apply.
In particular, we will identify the cohomology groups of all shifts
$M[k]$ without the intervention of signs.

\medskip\noindent
At this point we have enough structure to talk about {\it triangles},
see Derived, Definition \ref{derived-definition-triangle}.
In fact, our next goal is to develop enough theory to be able to
state and prove that the homotopy category of differential graded
modules is a triangulated category. First we define the homotopy category.






\section{The homotopy category}
\label{section-homotopy}

\noindent
Our homotopies take into account the $A$-module structure and the
grading, but not the differential (of course).

\begin{definition}
\label{definition-homotopy}
Let $(A, \text{d})$ be a differential graded algebra. Let
$f, g : M \to N$ be homomorphisms of differential graded $A$-modules.
A {\it homotopy between $f$ and $g$} is an $A$-module map $h : M \to N$
such that
\begin{enumerate}
\item $h(M^n) \subset N^{n - 1}$ for all $n$, and
\item $f(x) - g(x) = \text{d}_N(h(x)) + h(\text{d}_M(x))$ for
all $x \in M$.
\end{enumerate}
If a homotopy exists, then we say $f$ and $g$ are {\it homotopic}.
\end{definition}

\noindent
Thus $h$ is compatible with the $A$-module structure and the grading
but not with the differential. If $f = g$ and $h$ is a homotopy
as in the definition, then $h$ defines a morphism $h : M \to N[1]$
in $\text{Mod}_{(A, \text{d})}$.

\begin{lemma}
\label{lemma-compose-homotopy}
Let $(A, \text{d})$ be a differential graded algebra.
Let $f, g : L \to M$ be homomorphisms of differential graded $A$-modules.
Suppose given further homomorphisms $a : K \to L$, and $c : M \to N$.
If $h : M \to N$ defines a homotopy between $f$ and $g$, then
$c \circ h \circ a$ defines a homotopy between $c \circ f \circ a$ and
$c \circ g \circ a$.
\end{lemma}

\begin{proof}
Immediate from Homology, Lemma \ref{homology-lemma-compose-homotopy-cochain}.
\end{proof}

\noindent
This lemma allows us to define the homotopy category as follows.

\begin{definition}
\label{definition-complexes-notation}
Let $(A, \text{d})$ be a differential graded algebra.
The {\it homotopy category}, denoted $K(\text{Mod}_{(A, \text{d})})$, is
the the category whose objects are the objects of
$\text{Mod}_{(A, \text{d})}$ and whose morphisms are homotopy classes
of homomorphisms of differential graded $A$-modules.
\end{definition}

\noindent
The notation $K(\text{Mod}_{(A, \text{d})})$ is not standard but at least is
consistent with the use of $K(-)$ in other places of the Stacks project.


















\section{Admissible short exact sequences}
\label{section-admissible}

\noindent
An admissible short exact sequence is the analogue of termwise split exact
sequences in the setting of differential graded modules.

\begin{definition}
\label{definition-admissible-ses}
Let $(A, \text{d})$ be a differential graded algebra.
\begin{enumerate}
\item A homomorphism $K \to L$ of differential graded $A$-modules
is an {\it admissible monomorphism} if there exists a graded $A$-module
map $L \to K$ which is left inverse to $K \to L$.
\item A homomorphism $L \to M$ of differential graded $A$-modules
is an {\it admissible epimorphism} if there exists a graded $A$-module
map $M \to L$ which is right inverse to $K \to L$.
\item A short exact sequence $0 \to K \to L \to M \to 0$ of differential
graded $A$-modules is an {\it admissible short exact sequence}
if it is split as a sequence of graded $A$-modules.
\end{enumerate}
\end{definition}

\noindent
Thus the splittings are compatible with all the data except for
the differentials. Given an admissible short exact sequence we
obtain a triangle; this is the reason that we require our splittings
to be compatible with the $A$-module structure.

\begin{lemma}
\label{lemma-admissible-ses}
Let $(A, \text{d})$ be a differential graded algebra.
Let $0 \to K \to L \to M \to 0$ be an admissible short exact sequence
of differential graded $A$-modules. Let $s : M \to L$ and $\pi : L \to K$
be splittings such that $\text{ker}(\pi) = \text{Im}(s)$.
Then we obtain a morphism
$$
\delta = \pi \circ \text{d}_L \circ s : M \to K[1]
$$
of $\text{Mod}_{(A, \text{d})}$ which induces the boundary maps
in the long exact sequence of cohomology (\ref{equation-les}).
\end{lemma}

\begin{proof}
The map $\pi \circ \text{d}_L \circ s$ is compatible with the $A$-module
structure and the gradings by construction. It is compatible with
differentials by Homology, Lemmas
\ref{homology-lemma-ses-termwise-split-cochain}.
Let $R$ be the ring that $A$ is a differential graded algebra over.
The equality of maps is a statement about $R$-modules. Hence this
follows from Homology, Lemmas
\ref{homology-lemma-ses-termwise-split-cochain} and
\ref{homology-lemma-ses-termwise-split-long-cochain}.
\end{proof}

\begin{lemma}
\label{lemma-make-commute-map}
Let $(A, \text{d})$ be a differential graded algebra. Let
$$
\xymatrix{
K \ar[r]_f \ar[d]_a & L \ar[d]^b \\
M \ar[r]^g & N
}
$$
be a diagram of homomorphisms of differential graded $A$-modules
commuting up to homotopy.
\begin{enumerate}
\item If $f$ is an admissible monomorphism, then $b$ is homotopic to a
homomorphism which makes the diagram commute.
\item If $g$ is an admissible epimorphism, then $a$ is homotopic to a
morphism which makes the diagram commute.
\end{enumerate}
\end{lemma}

\begin{proof}
Let $h : A \to D$ be a homotopy between $bf$ and $ga$, i.e.,
$bf - ga = \text{d}h + h\text{d}$. Suppose that $\pi : B \to A$
is a graded $A$-module map left inverse to $f$. Take
$b' = b + \text{d}h\pi + h\pi \text{d}$.
Suppose $s : D \to C$ is a graded $A$-module map right inverse to
$g : C \to D$. Take $a' = a + \text{d}sh + sh\text{d}$.
Computations omitted.
\end{proof}

\begin{lemma}
\label{lemma-make-injective}
Let $(A, \text{d})$ be a differential graded algebra.
Let $\alpha : K \to L$ be a homomorphism of differential graded
$A$-modules. There exists a factorization
$$
\xymatrix{
K \ar[r]^{\tilde \alpha} \ar@/_1pc/[rr]_\alpha &
\tilde L \ar[r]^\pi & L
}
$$
in $\text{Mod}_{(A, \text{d})}$ such that
\begin{enumerate}
\item $\tilde \alpha$ is an admissible monomorphism (see
Definition \ref{definition-admissible-ses}),
\item there is a morphism $s : L \to \tilde L$
such that $\pi \circ s = \text{id}_L$ and such that
$s \circ \pi$ is homotopic to $\text{id}_{\tilde L}$.
\end{enumerate}
\end{lemma}

\begin{proof}
The proof is identical to the proof of
Derived Categories, Lemma \ref{derived-lemma-make-injective}.
\end{proof}

\begin{lemma}
\label{lemma-sequence-maps-split}
Let $(A, \text{d})$ be a differential graded algebra.
Let $L_1 \to L_2 \to \ldots \to L_n$
be a sequence of composable homomorphisms of
differential graded $A$-modules.
There exists a commutative diagram
$$
\xymatrix{
L_1 \ar[r] &
L_2 \ar[r] &
\ldots \ar[r] &
L_n \\
M_1 \ar[r] \ar[u] &
M_2 \ar[r] \ar[u] &
\ldots \ar[r] &
M_n \ar[u]
}
$$
in $\text{Mod}_{(A, \text{d})}$ such that each $M_i \to M_{i + 1}$
is an admissible monomorphism and each $M_i \to L_i$
is a homotopy equivalence.
\end{lemma}

\begin{proof}
The case $n = 1$ is without content.
Lemma \ref{lemma-make-injective} is the case $n = 2$.
Suppose we have constructed the diagram
except for $M_n$. Apply Lemma \ref{lemma-make-injective} to
the composition $M_{n - 1} \to L_{n - 1} \to L_n$.
The result is a factorization $M_{n - 1} \to M_n \to L_n$
as desired.
\end{proof}



\begin{lemma}
\label{lemma-nilpotent}
Let $(A, \text{d})$ be a differential graded algebra.
Let $0 \to K_i \to L_i \to M_i \to 0$, $i = 1, 2, 3$
be admissible short exact sequence of differential graded $A$-modules.
Let $b : L_1 \to L_2$ and $b' : L_2 \to L_3$
be homomorphisms of differential graded modules such that
$$
\vcenter{
\xymatrix{
K_1 \ar[d]_0 \ar[r] &
L_1 \ar[r] \ar[d]_b &
M_1 \ar[d]_0 \\
K_2 \ar[r] & L_2 \ar[r] & M_2
}
}
\quad\text{and}\quad
\vcenter{
\xymatrix{
K_2 \ar[d]^0 \ar[r] &
L_2 \ar[r] \ar[d]^{b'} &
M_2 \ar[d]^0 \\
K_3 \ar[r] & L_3 \ar[r] & M_3
}
}
$$
commute up to homotopy. Then $b' \circ b$ is homotopic to $0$.
\end{lemma}

\begin{proof}
By Lemma \ref{lemma-make-commute-map} we can replace $b$ and $b'$ by
homotopic maps such that the right square of the left diagram commutes
and the left square of the right diagram commutes. In other words, we have
$\text{Im}(b) \subset \text{Im}(K_2 \to L_2)$ and
$\text{ker}((b')^n) \supset \text{Im}(K_2 \to L_2)$.
Then $b \circ b' = 0$ as a map of modules.
\end{proof}












\section{Distinguished triangles}
\label{section-distinguished}

\noindent
The following lemma produces our distinguished triangles.

\begin{lemma}
\label{lemma-triangle-independent-splittings}
Let $(A, \text{d})$ be a differential graded algebra. Let
$0 \to K \to L \to M \to 0$ be an admissible short exact sequence
of differential graded $A$-modules. The triangle
\begin{equation}
\label{equation-triangle-associated-to-admissible-ses}
K \to L \to M \xrightarrow{\delta} K[1]
\end{equation}
with $\delta$ as in Lemma \ref{lemma-admissible-ses} is, up to canonical
isomorphism in $K(\text{Mod}_{(A, \text{d})})$, indepedent of the choices
made in Lemma \ref{lemma-admissible-ses}.
\end{lemma}

\begin{proof}
Namely, let $(s', \pi')$ be a second choice of splittings as in
Lemma \ref{lemma-admissible-ses}. Then we claim that $\delta$ and $\delta'$
are homotopic. Namely, write $s' = s + \alpha \circ h$ and
$\pi' = \pi + g \circ \beta$ for some unique homomorphisms
of $A$-modules $h : M \to K$ and $g : M \to K$ of degree $-1$.
Then $g = -h$ and $g$ is a homotopy between $\delta$ and $\delta'$.
The computations are done in the proof of
Homology, Lemma \ref{homology-lemma-ses-termwise-split-homotopy-cochain}.
\end{proof}

\begin{definition}
\label{definition-distinguished-triangle}
Let $(A, \text{d})$ be a differential graded algebra.
\begin{enumerate}
\item If $0 \to K \to L \to M \to 0$ is an admissible short exact sequence
of differential graded $A$-modules, then the {\it triangle associated
to $0 \to K \to L \to M \to 0$} is the triangle 
(\ref{equation-triangle-associated-to-admissible-ses})
of $K(\text{Mod}_{(A, \text{d})}$.
\item A triangle of $K(\text{Mod}_{(A, \text{d})}$ is called a
{\it distinguished triangle} if it is isomorphic to a triangle
associated to an admissible short exact sequence
of differential graded $A$-modules.
\end{enumerate}
\end{definition}







\section{Cones}
\label{section-cones}

\noindent
We quickly develop a theory of cones for the category of differential
graded modules.

\begin{definition}
\label{definition-cone}
Let $(A, \text{d})$ be a differential graded algebra.
Let $f : K \to L$ be a homomorphism of differential graded $A$-modules.
The {\it cone} of $f$ is the differential graded $A$-module
$C(f)$ given by $C(f) = L \oplus K$ with grading
$C(f)^n = L^n \oplus K^{n + 1}$ and
differential
$$
d_{C(f)} =
\left(
\begin{matrix}
\text{d}_L & f \\
0 & -\text{d}_K
\end{matrix}
\right)
$$
It comes equipped with canonical morphisms of complexes $i : L \to C(f)$
and $p : C(f) \to K[1]$ induced by the obvious maps $L \to C(f)$
and $C(f) \to K$.
\end{definition}

\noindent
In other words $(K, L, C(f), f, i, p)$ forms a triangle:
$$
K \to L \to C(f) \to K[1]
$$
in $\text{Mod}_{(A, \text{d})}$ and hence in $K(\text{Mod}_{(A, \text{d})})$.
Cones are {\bf not} distinguished triangles in general, but the difference
is a sign or a rotation (your choice). Here are two precise statements.

\begin{lemma}
\label{lemma-rotate-cone}
Let $(A, \text{d})$ be a differential graded algebra.
Let $f : K \to L$ be a homomorphism of differential graded modules.
The triangle $(L, C(f), K[1], i, p, f[1])$ is
the triangle associated to the admissible short exact sequence
$$
0 \to L \to C(f) \to K[1] \to 0
$$
coming from the definition of the cone of $f$.
\end{lemma}

\begin{proof}
Immediate from the definitions.
\end{proof}

\begin{lemma}
\label{lemma-rotate-triangle}
Let $(A, \text{d})$ be a differential graded algebra.
Let $\alpha : K \to L$ and $\beta : L \to M$
define an admissible short exact sequence
$$
0 \to K \to L \to M \to 0
$$
of differential graded $A$-modules.
Let $(K, L, M, \alpha, \beta, \delta)$
be the associated triangle. Then the triangles
$$
(M[-1], K, L, \delta[-1], \alpha, \beta)
\quad\text{and}\quad
(M[-1], K, C(\delta[-1]), \delta[-1], i, p)
$$
are isomorphic.
\end{lemma}

\begin{proof}
Using a choice of splittings we write $L = K \oplus M$ and we identify
$\alpha$ and $\beta$ with the natural inclusion and projection maps.
By construction of $\delta$ we have
$$
d_B =
\left(
\begin{matrix}
d_K & \delta \\
0 & d_M
\end{matrix}
\right)
$$
On the other hand the cone of $\delta[-1] : M[-1] \to K$
is given as $C(\delta[-1]) = K \oplus M$ with differential identical
with the matrix above! Whence the lemma.
\end{proof}

\noindent
The formation of the cone triangle is functorial in the following sense.

\begin{lemma}
\label{lemma-functorial-cone}
Let $(A, \text{d})$ be a differential graded algebra.
Suppose that
$$
\xymatrix{
K_1 \ar[r]_{f_1} \ar[d]_a & L_1 \ar[d]^b \\
K_2 \ar[r]^{f_2} & L_2
}
$$
is a diagram of homomorphisms of diferential graded $A$-modules which is
commutative up to homotopy.
Then there exists a morphism $c : C(f_1) \to C(f_2)$ which gives rise to
a morphism of triangles
$$
(a, b, c) : (K_1, L_1, C(f_1), f_1, i_1, p_1) \to
(K_1, L_1, C(f_1), f_2, i_2, p_2)
$$
in $K(\text{Mod}_{(A, \text{d})})$.
\end{lemma}

\begin{proof}
Let $h : K_1 \to L_2$ be a homotopy between $f_2 \circ a$ and $b \circ f_1$.
Define $c$ by the matrix
$$
c =
\left(
\begin{matrix}
a & h \\
0 & b
\end{matrix}
\right) :
L_1 \oplus K_1 \to L_2 \oplus K_2
$$
A matrix computation show that $c$ is a morphism of differential
graded modules. It is trivial that $c \circ i_1 = i_2 \circ b$, and it is
trivial also to check that $p_2 \circ c = a \circ p_1$.
\end{proof}

\begin{lemma}
\label{lemma-third-isomorphism}
Let $(A, \text{d})$ be a differential graded algebra.
Let $f_1 : K_1 \to L_1$ and $f_2 : K_2 \to L_2$ be homomorphisms of
differential graded $A$-modules. Let
$$
(a, b, c) :
(K_1, L_1, C(f_1), f_1, i_1, p_1)
\longrightarrow
(K_1, L_1, C(f_1), f_2, i_2, p_2)
$$
be any morphism of triangles of $K(\text{Mod}_{(A, \text{d})})$.
If $a$ and $b$ are homotopy equivalences then so is $c$.
\end{lemma}

\begin{proof}
Let $a^{-1} : K_2 \to K_1$ be a homomorphism of differential graded $A$-modules
which is inverse to $a$ in $K(\text{Mod}_{(A, \text{d})})$.
Let $b^{-1} : L_2 \to L_1$ be a homomorphism of differential graded $A$-modules
which is inverse to $b$ in $K(\text{Mod}_{(A, \text{d})})$.
Let $c' : C(f_2) \to C(f_1)$ be the morphism from
Lemma \ref{lemma-functorial-cone} applied to
$f_1 \circ a^{-1} = b^{-1} \circ f_2$.
If we can show that $c \circ c'$ and $c' \circ c$ are isomorphisms in
$K(\text{Mod}_{(A, \text{d})})$
then we win. Hence it suffices to prove the following: Given
a morphism of triangles
$(1, 1, c) : (K, L, C(f), f, i, p)$
in $K(\text{Mod}_{(A, \text{d})})$ the morphism $c$ is an isomorphism in $K(\text{Mod}_{(A, \text{d})})$.
By assumption the two squares in the diagram
$$
\xymatrix{
L \ar[r] \ar[d]_1 &
C(f) \ar[r] \ar[d]_c &
K[1] \ar[d]_1 \\
L \ar[r] &
C(f) \ar[r] &
K[1]
}
$$
commute up to homotopy. By construction of $C(f)$ the rows
form admissible short exact sequences. Thus we see that
$(c - 1)^2 = 0$ in $K(\text{Mod}_{(A, \text{d})})$ by
Lemma \ref{lemma-nilpotent}.
Hence $c$ is an isomorphism in $K(\text{Mod}_{(A, \text{d})})$
with inverse $2 - c$.
\end{proof}

\noindent
The following lemma shows that the collection of triangles of the homotopy
category given by cones and the distinguished triangles are the same
up to isomorphisms, at least up to sign!

\begin{lemma}
\label{lemma-the-same-up-to-isomorphisms}
Let $(A, \text{d})$ be a differential graded algebra.
\begin{enumerate}
\item Given an admissible short exact sequence $0 \to K \to L \to M \to 0$
of differential graded $A$-modules there exists a homotopy equivalence
$C(\alpha) \to M$ such that the diagram
$$
\xymatrix{
K \ar[r] \ar[d] & L \ar[d] \ar[r] &
C(\alpha) \ar[r]_{-p} \ar[d] & K[1] \ar[d] \\
K \ar[r]^\alpha & L \ar[r]^\beta &
M \ar[r]^\delta & K[1]
}
$$
defines an isomorphism of triangles in $K(\text{Mod}_{(A, \text{d})})$.
\item Given a morphism of complexes $f : K \to L$
there exists an isomorphism of triangles
$$
\xymatrix{
K \ar[r] \ar[d] & \tilde L \ar[d] \ar[r] &
M \ar[r]_{\delta} \ar[d] & K[1] \ar[d] \\
K \ar[r] & L \ar[r] &
C(f) \ar[r]^{-p} & K[1]
}
$$
where the upper triangle is the triangle associated to a
termwise split exact sequence $K \to \tilde L \to M$.
\end{enumerate}
\end{lemma}

\begin{proof}
Proof of (1). We have $C(\alpha) = L \oplus K$ and we simply define
$C(\alpha) \to M$ via the projection onto $L$ followed by $\beta$.
This defines a morphism of differential graded modules because the
compositions $K^{n + 1} \to L^{n + 1} \to L^n \to M^n$ are zero.
Choose splittings $s : M \to L$ and $\pi : L \to K$ with
$\text{Ker}(\pi) = \text{Im}(s)$ and set
$\delta = \pi \circ \text{d}_L \circ s$ as usual.
To get a homotopy inverse we take
$M \to C(\alpha)$ given by $(s , -\delta)$. This is compatible with
differentials because $\delta^n$ can be characterized as the
unique map $M^n \to K^{n + 1}$ such that
$d \circ s^n - s^{n + 1} \circ d = \alpha \circ \delta^n$,
see proof of
Homology, Lemma \ref{homology-lemma-ses-termwise-split-cochain}.
The composition $M \to C(f) \to M$ is the identity.
The composition $C(f) \to M \to C(f)$ is equal to the morphism
$$
\left(
\begin{matrix}
s \circ \beta & 0 \\
-\delta \circ \beta & 0
\end{matrix}
\right)
$$
To see that this is homotopic to the identity map
use the homotopy $h : C(\alpha) \to C(\alpha)$
given by the matrix
$$
\left(
\begin{matrix}
0 & 0 \\
\pi & 0
\end{matrix}
\right) :
C(\alpha) = L \oplus K
\to
L \oplus K = C(\alpha)
$$
It is trivial to verify that
$$
\left(
\begin{matrix}
1 & 0 \\
0 & 1
\end{matrix}
\right)
-
\left(
\begin{matrix}
s &
-\delta
\end{matrix}
\right)
\left(
\begin{matrix}
\beta \\
0
\end{matrix}
\right)
=
\left(
\begin{matrix}
d & \alpha \\
0 & -d
\end{matrix}
\right)
\left(
\begin{matrix}
0 & 0 \\
\pi & 0
\end{matrix}
\right)
+
\left(
\begin{matrix}
0 & 0 \\
\pi & 0
\end{matrix}
\right)
\left(
\begin{matrix}
d & \alpha \\
0 & -d
\end{matrix}
\right)
$$
To finish the proof of (1) we have to show that the morphisms
$-p : C(\alpha) \to K[1]$ (see
Definition \ref{definition-cone})
and $C(\alpha) \to M \to K[1]$ agree up
to homotopy. This is clear from the above. Namely, we can use the homotopy
inverse $(s, -\delta) : M \to C(\alpha)$
and check instead that the two maps
$M \to K[1]$ agree. And note that
$p \circ (s, -\delta) = -\delta$ as desired.

\medskip\noindent
Proof of (2). We let $\tilde f : K \to \tilde L$,
$s : L \to \tilde L$
and $\pi : L \to L$ be as in
Lemma \ref{lemma-make-injective}. By
Lemmas \ref{lemma-functorial-cone} and \ref{lemma-third-isomorphism}
the triangles $(K, L, C(f), i, p)$ and
$(K, \tilde L, C(\tilde f), \tilde i, \tilde p)$
are isomorphic. Note that we can compose isomorphisms of
triangles. Thus we may replace $L$ by
$\tilde L$ and $f$ by $\tilde f$. In other words
we may assume that $f$ is a termwise split injection.
In this case the result follows from part (1).
\end{proof}








\section{Other chapters}

\begin{multicols}{2}
\begin{enumerate}
\item \hyperref[introduction-section-phantom]{Introduction}
\item \hyperref[conventions-section-phantom]{Conventions}
\item \hyperref[sets-section-phantom]{Set Theory}
\item \hyperref[categories-section-phantom]{Categories}
\item \hyperref[topology-section-phantom]{Topology}
\item \hyperref[sheaves-section-phantom]{Sheaves on Spaces}
\item \hyperref[algebra-section-phantom]{Commutative Algebra}
\item \hyperref[sites-section-phantom]{Sites and Sheaves}
\item \hyperref[homology-section-phantom]{Homological Algebra}
\item \hyperref[derived-section-phantom]{Derived Categories}
\item \hyperref[more-algebra-section-phantom]{More Algebra}
\item \hyperref[simplicial-section-phantom]{Simplicial Methods}
\item \hyperref[modules-section-phantom]{Sheaves of Modules}
\item \hyperref[sites-modules-section-phantom]{Modules on Sites}
\item \hyperref[injectives-section-phantom]{Injectives}
\item \hyperref[cohomology-section-phantom]{Cohomology of Sheaves}
\item \hyperref[sites-cohomology-section-phantom]{Cohomology on Sites}
\item \hyperref[hypercovering-section-phantom]{Hypercoverings}
\item \hyperref[schemes-section-phantom]{Schemes}
\item \hyperref[constructions-section-phantom]{Constructions of Schemes}
\item \hyperref[properties-section-phantom]{Properties of Schemes}
\item \hyperref[morphisms-section-phantom]{Morphisms of Schemes}
\item \hyperref[coherent-section-phantom]{Coherent Cohomology}
\item \hyperref[divisors-section-phantom]{Divisors}
\item \hyperref[limits-section-phantom]{Limits of Schemes}
\item \hyperref[varieties-section-phantom]{Varieties}
\item \hyperref[chow-section-phantom]{Chow Homology}
\item \hyperref[topologies-section-phantom]{Topologies on Schemes}
\item \hyperref[descent-section-phantom]{Descent}
\item \hyperref[more-morphisms-section-phantom]{More on Morphisms}
\item \hyperref[flat-section-phantom]{More on Flatness}
\item \hyperref[groupoids-section-phantom]{Groupoid Schemes}
\item \hyperref[more-groupoids-section-phantom]{More on Groupoid Schemes}
\item \hyperref[etale-section-phantom]{\'Etale Morphisms of Schemes}
\item \hyperref[etale-cohomology-section-phantom]{\'Etale Cohomology}
\item \hyperref[spaces-section-phantom]{Algebraic Spaces}
\item \hyperref[spaces-properties-section-phantom]{Properties of Algebraic Spaces}
\item \hyperref[spaces-morphisms-section-phantom]{Morphisms of Algebraic Spaces}
\item \hyperref[spaces-topologies-section-phantom]{Topologies on Algebraic Spaces}
\item \hyperref[spaces-descent-section-phantom]{Descent and Algebraic Spaces}
\item \hyperref[spaces-more-morphisms-section-phantom]{More on Morphisms of Spaces}
\item \hyperref[quot-section-phantom]{Quot and Hilbert Spaces}
\item \hyperref[stacks-section-phantom]{Stacks}
\item \hyperref[spaces-groupoids-section-phantom]{Groupoids in Algebraic Spaces}
\item \hyperref[spaces-more-groupoids-section-phantom]{More on Groupoids in Spaces}
\item \hyperref[bootstrap-section-phantom]{Bootstrap}
\item \hyperref[examples-stacks-section-phantom]{Examples of Stacks}
\item \hyperref[groupoids-quotients-section-phantom]{Quotients of Groupoids}
\item \hyperref[algebraic-section-phantom]{Algebraic Stacks}
\item \hyperref[criteria-section-phantom]{Criteria for Representability}
\item \hyperref[stacks-properties-section-phantom]{Properties of Algebraic Stacks}
\item \hyperref[stacks-morphisms-section-phantom]{Morphisms of Algebraic Stacks}
\item \hyperref[examples-section-phantom]{Examples}
\item \hyperref[exercises-section-phantom]{Exercises}
\item \hyperref[guide-section-phantom]{Guide to Literature}
\item \hyperref[desirables-section-phantom]{Desirables}
\item \hyperref[coding-section-phantom]{Coding Style}
\item \hyperref[fdl-section-phantom]{GNU Free Documentation License}
\item \hyperref[index-section-phantom]{Auto Generated Index}
\end{enumerate}
\end{multicols}


\bibliography{my}
\bibliographystyle{amsalpha}

\end{document}
