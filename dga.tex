\IfFileExists{stacks-project.cls}{%
\documentclass{stacks-project}
}{%
\documentclass{amsart}
}

% The following AMS packages are automatically loaded with
% the amsart documentclass:
%\usepackage{amsmath}
%\usepackage{amssymb}
%\usepackage{amsthm}

% For dealing with references we use the comment environment
\usepackage{verbatim}
\newenvironment{reference}{\comment}{\endcomment}
%\newenvironment{reference}{}{}
\newenvironment{slogan}{\comment}{\endcomment}
\newenvironment{history}{\comment}{\endcomment}

% For commutative diagrams you can use
% \usepackage{amscd}
\usepackage[all]{xy}

% We use 2cell for 2-commutative diagrams.
\xyoption{2cell}
\UseAllTwocells

% To put source file link in headers.
% Change "template.tex" to "this_filename.tex"
% \usepackage{fancyhdr}
% \pagestyle{fancy}
% \lhead{}
% \chead{}
% \rhead{Source file: \url{template.tex}}
% \lfoot{}
% \cfoot{\thepage}
% \rfoot{}
% \renewcommand{\headrulewidth}{0pt}
% \renewcommand{\footrulewidth}{0pt}
% \renewcommand{\headheight}{12pt}

\usepackage{multicol}

% For cross-file-references
\usepackage{xr-hyper}

% Package for hypertext links:
\usepackage{hyperref}

% For any local file, say "hello.tex" you want to link to please
% use \externaldocument[hello-]{hello}
\externaldocument[introduction-]{introduction}
\externaldocument[conventions-]{conventions}
\externaldocument[sets-]{sets}
\externaldocument[categories-]{categories}
\externaldocument[topology-]{topology}
\externaldocument[sheaves-]{sheaves}
\externaldocument[sites-]{sites}
\externaldocument[stacks-]{stacks}
\externaldocument[fields-]{fields}
\externaldocument[algebra-]{algebra}
\externaldocument[brauer-]{brauer}
\externaldocument[homology-]{homology}
\externaldocument[derived-]{derived}
\externaldocument[simplicial-]{simplicial}
\externaldocument[more-algebra-]{more-algebra}
\externaldocument[smoothing-]{smoothing}
\externaldocument[modules-]{modules}
\externaldocument[sites-modules-]{sites-modules}
\externaldocument[injectives-]{injectives}
\externaldocument[cohomology-]{cohomology}
\externaldocument[sites-cohomology-]{sites-cohomology}
\externaldocument[dga-]{dga}
\externaldocument[dpa-]{dpa}
\externaldocument[hypercovering-]{hypercovering}
\externaldocument[schemes-]{schemes}
\externaldocument[constructions-]{constructions}
\externaldocument[properties-]{properties}
\externaldocument[morphisms-]{morphisms}
\externaldocument[coherent-]{coherent}
\externaldocument[divisors-]{divisors}
\externaldocument[limits-]{limits}
\externaldocument[varieties-]{varieties}
\externaldocument[topologies-]{topologies}
\externaldocument[descent-]{descent}
\externaldocument[perfect-]{perfect}
\externaldocument[more-morphisms-]{more-morphisms}
\externaldocument[flat-]{flat}
\externaldocument[groupoids-]{groupoids}
\externaldocument[more-groupoids-]{more-groupoids}
\externaldocument[etale-]{etale}
\externaldocument[chow-]{chow}
\externaldocument[intersection-]{intersection}
\externaldocument[pic-]{pic}
\externaldocument[adequate-]{adequate}
\externaldocument[dualizing-]{dualizing}
\externaldocument[duality-]{duality}
\externaldocument[discriminant-]{discriminant}
\externaldocument[local-cohomology-]{local-cohomology}
\externaldocument[curves-]{curves}
\externaldocument[resolve-]{resolve}
\externaldocument[models-]{models}
\externaldocument[pione-]{pione}
\externaldocument[etale-cohomology-]{etale-cohomology}
\externaldocument[proetale-]{proetale}
\externaldocument[crystalline-]{crystalline}
\externaldocument[spaces-]{spaces}
\externaldocument[spaces-properties-]{spaces-properties}
\externaldocument[spaces-morphisms-]{spaces-morphisms}
\externaldocument[decent-spaces-]{decent-spaces}
\externaldocument[spaces-cohomology-]{spaces-cohomology}
\externaldocument[spaces-limits-]{spaces-limits}
\externaldocument[spaces-divisors-]{spaces-divisors}
\externaldocument[spaces-over-fields-]{spaces-over-fields}
\externaldocument[spaces-topologies-]{spaces-topologies}
\externaldocument[spaces-descent-]{spaces-descent}
\externaldocument[spaces-perfect-]{spaces-perfect}
\externaldocument[spaces-more-morphisms-]{spaces-more-morphisms}
\externaldocument[spaces-flat-]{spaces-flat}
\externaldocument[spaces-groupoids-]{spaces-groupoids}
\externaldocument[spaces-more-groupoids-]{spaces-more-groupoids}
\externaldocument[bootstrap-]{bootstrap}
\externaldocument[spaces-pushouts-]{spaces-pushouts}
\externaldocument[groupoids-quotients-]{groupoids-quotients}
\externaldocument[spaces-more-cohomology-]{spaces-more-cohomology}
\externaldocument[spaces-simplicial-]{spaces-simplicial}
\externaldocument[formal-spaces-]{formal-spaces}
\externaldocument[restricted-]{restricted}
\externaldocument[spaces-resolve-]{spaces-resolve}
\externaldocument[formal-defos-]{formal-defos}
\externaldocument[defos-]{defos}
\externaldocument[cotangent-]{cotangent}
\externaldocument[examples-defos-]{examples-defos}
\externaldocument[algebraic-]{algebraic}
\externaldocument[examples-stacks-]{examples-stacks}
\externaldocument[stacks-sheaves-]{stacks-sheaves}
\externaldocument[criteria-]{criteria}
\externaldocument[artin-]{artin}
\externaldocument[quot-]{quot}
\externaldocument[stacks-properties-]{stacks-properties}
\externaldocument[stacks-morphisms-]{stacks-morphisms}
\externaldocument[stacks-limits-]{stacks-limits}
\externaldocument[stacks-cohomology-]{stacks-cohomology}
\externaldocument[stacks-perfect-]{stacks-perfect}
\externaldocument[stacks-introduction-]{stacks-introduction}
\externaldocument[stacks-more-morphisms-]{stacks-more-morphisms}
\externaldocument[stacks-geometry-]{stacks-geometry}
\externaldocument[moduli-]{moduli}
\externaldocument[moduli-curves-]{moduli-curves}
\externaldocument[examples-]{examples}
\externaldocument[exercises-]{exercises}
\externaldocument[guide-]{guide}
\externaldocument[desirables-]{desirables}
\externaldocument[coding-]{coding}
\externaldocument[obsolete-]{obsolete}
\externaldocument[fdl-]{fdl}
\externaldocument[index-]{index}

% Theorem environments.
%
\theoremstyle{plain}
\newtheorem{theorem}[subsection]{Theorem}
\newtheorem{proposition}[subsection]{Proposition}
\newtheorem{lemma}[subsection]{Lemma}

\theoremstyle{definition}
\newtheorem{definition}[subsection]{Definition}
\newtheorem{example}[subsection]{Example}
\newtheorem{exercise}[subsection]{Exercise}
\newtheorem{situation}[subsection]{Situation}

\theoremstyle{remark}
\newtheorem{remark}[subsection]{Remark}
\newtheorem{remarks}[subsection]{Remarks}

\numberwithin{equation}{subsection}

% Macros
%
\def\lim{\mathop{\rm lim}\nolimits}
\def\colim{\mathop{\rm colim}\nolimits}
\def\Spec{\mathop{\rm Spec}}
\def\Hom{\mathop{\rm Hom}\nolimits}
\def\Ext{\mathop{\rm Ext}\nolimits}
\def\SheafHom{\mathop{\mathcal{H}\!{\it om}}\nolimits}
\def\SheafExt{\mathop{\mathcal{E}\!{\it xt}}\nolimits}
\def\Sch{\textit{Sch}}
\def\Mor{\mathop{\rm Mor}\nolimits}
\def\Ob{\mathop{\rm Ob}\nolimits}
\def\Sh{\mathop{\textit{Sh}}\nolimits}
\def\NL{\mathop{N\!L}\nolimits}
\def\proetale{{pro\text{-}\acute{e}tale}}
\def\etale{{\acute{e}tale}}
\def\QCoh{\textit{QCoh}}
\def\Ker{\mathop{\rm Ker}}
\def\Im{\mathop{\rm Im}}
\def\Coker{\mathop{\rm Coker}}
\def\Coim{\mathop{\rm Coim}}

%
% Macros for moduli stacks/spaces
%
\def\QCohstack{\mathcal{QC}\!{\it oh}}
\def\Cohstack{\mathcal{C}\!{\it oh}}
\def\Spacesstack{\mathcal{S}\!{\it paces}}
\def\Quotfunctor{{\rm Quot}}
\def\Hilbfunctor{{\rm Hilb}}
\def\Curvesstack{\mathcal{C}\!{\it urves}}
\def\Polarizedstack{\mathcal{P}\!{\it olarized}}
\def\Complexesstack{\mathcal{C}\!{\it omplexes}}
% \Pic is the operator that assigns to X its picard group, usage \Pic(X)
% \Picardstack_{X/B} denotes the Picard stack of X over B
% \Picardfunctor_{X/B} denotes the Picard functor of X over B
\def\Pic{\mathop{\rm Pic}\nolimits}
\def\Picardstack{\mathcal{P}\!{\it ic}}
\def\Picardfunctor{{\rm Pic}}
\def\Deformationcategory{\mathcal{D}\!{\it ef}}


% OK, start here.
%
\begin{document}

\title{Differential Graded Algebra}


\maketitle

\phantomsection
\label{section-phantom}

\tableofcontents

\section{Introduction}
\label{section-introduction}

\noindent
In this chapter we talk about differential graded algebras, modules,
categories, etc. A basic reference is \cite{Keller-Deriving}.
A survey paper is \cite{Keller-survey}.

\medskip\noindent
Since we do not worry about length of exposition in the Stacks project
we first develop the material in the setting of categories of differential
graded modules. After that we redo the constructions in the setting of
differential graded modules over differential graded categories.



\section{Conventions}
\label{section-conventions}

\noindent
In this chapter we hold on to the convention that {\it ring} means
commutative ring with $1$. If $R$ is a ring, then an {\it $R$-algebra $A$}
will be an $R$-module $A$ endowed with an $R$-bilinear map $A \times A \to A$
(multiplication) such that multiplication is associative and has a unit.
In other words, these are unital associative $R$-algebras
such that the structure map $R \to A$ maps into the center of $A$.




\section{Differential graded algebras}
\label{section-dga}


\noindent
Just the definitions.

\begin{definition}
\label{definition-dga}
Let $R$ be a commutative ring. A {\it differential graded algebra over $R$}
is either
\begin{enumerate}
\item a chain complex $A_\bullet$ of $R$-modules endowed with
$R$-bilinear maps $A_n \times A_m \to A_{n + m}$,
$(a, b) \mapsto ab$ such that
$$
\text{d}_{n + m}(ab) = \text{d}_n(a)b + (-1)^n a\text{d}_m(b)
$$
and such that $\bigoplus A_n$ becomes an associative and unital
$R$-algebra, or
\item a cochain complex $A^\bullet$ of $R$-modules endowed with
$R$-bilinear maps $A^n \times A^m \to A^{n + m}$, $(a, b) \mapsto ab$
such that
$$
\text{d}^{n + m}(ab) = \text{d}^n(a)b + (-1)^n a\text{d}^m(b)
$$
and such that $\bigoplus A^n$ becomes an associative and unital $R$-algebra.
\end{enumerate}
\end{definition}

\noindent
We often just write $A = \bigoplus A_n$ or $A = \bigoplus A^n$ and
think of this as an associative unital $R$-algebra endowed with a
$\mathbf{Z}$-grading and an $R$-linear operator $\text{d}$ whose square
is zero and which satisfies the Leibniz rule as explained above. In this case
we often say ``Let $(A, \text{d})$ be a differential graded algebra''.

\begin{definition}
\label{definition-homomorphism-dga}
A {\it homomorphism of differential graded algebras}
$f : (A, \text{d}) \to (B, \text{d})$ is an algebra map $f : A \to B$
compatible with the gradings and $\text{d}$.
\end{definition}

\begin{definition}
\label{definition-opposite-dga}
Let $R$ be a ring. Let $(A, \text{d})$ be a differential graded algebra
over $R$. The {\it opposite differential graded algebra} is the differential
graded algebra $(A^{opp}, \text{d})$ over $R$ where $A^{opp} = A$
as an $R$-module, $\text{d} = \text{d}$, and multiplication is
given by
$$
a \cdot_{opp} b = (-1)^{\deg(a)\deg(b)} b a
$$
for homogeneous elements $a, b \in A$.
\end{definition}

\noindent
This makes sense because
\begin{align*}
\text{d}(a \cdot_{opp} b)
& =
(-1)^{\deg(a)\deg(b)} \text{d}(b a) \\
& =
(-1)^{\deg(a)\deg(b)} \text{d}(b) a +
(-1)^{\deg(a)\deg(b) + \deg(b)}b\text{d}(a) \\
& =
(-1)^{\deg(a)}a \cdot_{opp} \text{d}(b) + \text{d}(a) \cdot_{opp} b
\end{align*}
as desired.

\begin{definition}
\label{definition-cdga}
A differential graded algebra $(A, \text{d})$ is {\it commutative} if
$ab = (-1)^{nm}ba$ for $a$ in degree $n$ and $b$ in degree $m$.
We say $A$ is {\it strictly commutative} if in addition $a^2 = 0$
for $\deg(a)$ odd.
\end{definition}

\noindent
The following definition makes sense in general but is perhaps
``correct'' only when tensoring commutative differential graded
algebras.

\begin{definition}
\label{definition-tensor-product}
Let $R$ be a ring.
Let $(A, \text{d})$, $(B, \text{d})$ be differential graded algebras over $R$.
The {\it tensor product differential graded algebra} of $A$ and $B$
is the algebra $A \otimes_R B$ with multiplication defined by
$$
(a \otimes b)(a' \otimes b') = (-1)^{\deg(a')\deg(b)} aa' \otimes bb'
$$
endowed with differential $\text{d}$ defined by the rule
$\text{d}(a \otimes b) = \text{d}(a) \otimes b + (-1)^m a \otimes \text{d}(b)$
where $m = \deg(b)$.
\end{definition}

\begin{lemma}
\label{lemma-total-complex-tensor-product}
Let $R$ be a ring.
Let $(A, \text{d})$, $(B, \text{d})$ be differential graded algebras over $R$.
Denote $A^\bullet$, $B^\bullet$ the underlying cochain complexes.
As cochain complexes of $R$-modules we have
$$
(A \otimes_R B)^\bullet = \text{Tot}(A^\bullet \otimes_R B^\bullet).
$$
\end{lemma}

\begin{proof}
Recall that the differential of the total complex is given by
$\text{d}_1^{p, q} + (-1)^p \text{d}_2^{p, q}$ on $A^p \otimes_R B^q$.
And this is exactly the same as the rule for the differential
on $A \otimes_R B$ in
Definition \ref{definition-tensor-product}.
\end{proof}






\section{Differential graded modules}
\label{section-modules}

\noindent
Just the definitions.

\begin{definition}
\label{definition-dgm}
Let $R$ be a ring.
Let $(A, \text{d})$ be a differential graded algebra over $R$.
A (right) {\it differential graded module} $M$ over $A$ is a right $A$-module
$M$ which has a grading $M = \bigoplus M^n$ and a differential $\text{d}$
such that $M^n A^m \subset M^{n + m}$, such that
$\text{d}(M^n) \subset M^{n + 1}$, and such that
$$
\text{d}(ma) = \text{d}(m)a + (-1)^n m\text{d}(a)
$$
for $a \in A$ and $m \in M^n$. A
{\it homomorphism of differential graded modules} $f : M \to N$
is an $A$-module map compatible with gradings and differentials.
The category of (right) differential graded $A$-modules is denoted
$\text{Mod}_{(A, \text{d})}$.
\end{definition}

\noindent
Note that we can think of $M$ as a cochain complex $M^\bullet$
of (right) $R$-modules. Namely, for $r \in R$ we have $\text{d}(r) = 0$
and $r$ maps to a degree $0$ element of $A$, hence
$\text{d}(mr) = \text{d}(m)r$.

\medskip\noindent
We can define {\it left differential graded $A$-modules} in exactly the same
manner. If $M$ is a left $A$-module, then we can think of $M$ as a
right $A^{opp}$-module with mulitplication $\cdot_{opp}$ defined by
the rule
$$
m \cdot_{opp} a = (-1)^{\deg(a)\deg(m)} a m
$$
for $a$ and $m$ homogeneous. The category of left differential graded
$A$-modules is equivalent to the category of right differential
graded $A^{opp}$-modules. We prefer to work with right modules
(essentially because of what happens in Example \ref{example-dgm-dg-cat}), but
the reader is free to switch to left modules if (s)he so desires.

\begin{lemma}
\label{lemma-dgm-abelian}
Let $(A, d)$ be a differential graded algebra. The category
$\text{Mod}_{(A, \text{d})}$ is abelian and has arbitrary limits and colimits.
\end{lemma}

\begin{proof}
Kernels and cokernels commute with taking homogeneous components. Similarly
for colimits. In other words, these operations in $\text{Mod}_{(A, \text{d})}$
commute with the forgetful functor to the category of (graded) $A$-modules.
If $N_i$, $i \in I$ is a family of differential graded $A$-modules,
then the product $\prod N_i$ in $\text{Mod}_{(A, \text{d})}$ is
given by setting $(\prod N_i)^n = \prod N_i^n$ and then
$\prod N_i = \bigoplus_n (\prod N_i)^n$. A category with products
and equalizers has limits, see
Categories, Lemma \ref{categories-lemma-limits-products-equalizers}.
\end{proof}

\noindent
Thus, if $(A, \text{d})$ is a differential graded
algebra over $R$, then there is an exact functor
$$
\text{Mod}_{(A, \text{d})} \longrightarrow \text{Comp}(R)
$$
of abelian categories. For a differential graded module $M$ the
cohomology groups $H^n(M)$ are defined as the cohomology of the
corresponding complex of $R$-modules. Therefore, a short exact
sequence $0 \to K \to L \to M \to 0$ of differential graded modules
gives rise to a long exact sequence
\begin{equation}
\label{equation-les}
H^n(K) \to H^n(L) \to H^n(M) \to H^{n + 1}(K)
\end{equation}
of cohomology modules, see
Homology, Lemma \ref{homology-lemma-long-exact-sequence-cochain}.

\medskip\noindent
Moreover, from now on we borrow all the terminology used for
complexes of modules. For example, we say that a differential
graded $A$-module $M$ is {\it acyclic} if $H^k(M) = 0$ for
all $k \in \mathbf{Z}$. We say that a homomorphism $M \to N$
of differential graded $A$-modules is a {\it quasi-isomorphism}
if it induces isomorphisms $H^k(M) \to H^k(N)$ for all $k \in \mathbf{Z}$.
And so on and so forth.

\begin{definition}
\label{definition-shift}
Let $(A, \text{d})$ be a differential graded algebra.
Let $M$ be a differential graded module.
For any $k \in \mathbf{Z}$ we define the {\it $k$-shifted module}
$M[k]$ as follows
\begin{enumerate}
\item as $A$-module $M[k] = M$,
\item $M[k]^n = M^{n + k}$,
\item $\text{d}_{M[k]} = (-1)^k\text{d}_M$.
\end{enumerate}
For a morphism $f : M \to N$ of differential graded $A$-modules
we let $f[k] : M[k] \to N[k]$ be the map equal to $f$ on underlying
$A$-modules. This defines a functor
$[k] : \text{Mod}_{(A, \text{d})} \to \text{Mod}_{(A, \text{d})}$.
\end{definition}

\noindent
The remarks in Homology, Section \ref{homology-section-homotopy-shift} apply.
In particular, we will identify the cohomology groups of all shifts
$M[k]$ without the intervention of signs.

\medskip\noindent
At this point we have enough structure to talk about {\it triangles},
see Derived Categories, Definition \ref{derived-definition-triangle}.
In fact, our next goal is to develop enough theory to be able to
state and prove that the homotopy category of differential graded
modules is a triangulated category. First we define the homotopy category.






\section{The homotopy category}
\label{section-homotopy}

\noindent
Our homotopies take into account the $A$-module structure and the
grading, but not the differential (of course).

\begin{definition}
\label{definition-homotopy}
Let $(A, \text{d})$ be a differential graded algebra. Let
$f, g : M \to N$ be homomorphisms of differential graded $A$-modules.
A {\it homotopy between $f$ and $g$} is an $A$-module map $h : M \to N$
such that
\begin{enumerate}
\item $h(M^n) \subset N^{n - 1}$ for all $n$, and
\item $f(x) - g(x) = \text{d}_N(h(x)) + h(\text{d}_M(x))$ for
all $x \in M$.
\end{enumerate}
If a homotopy exists, then we say $f$ and $g$ are {\it homotopic}.
\end{definition}

\noindent
Thus $h$ is compatible with the $A$-module structure and the grading
but not with the differential. If $f = g$ and $h$ is a homotopy
as in the definition, then $h$ defines a morphism $h : M \to N[1]$
in $\text{Mod}_{(A, \text{d})}$.

\begin{lemma}
\label{lemma-compose-homotopy}
Let $(A, \text{d})$ be a differential graded algebra.
Let $f, g : L \to M$ be homomorphisms of differential graded $A$-modules.
Suppose given further homomorphisms $a : K \to L$, and $c : M \to N$.
If $h : M \to N$ defines a homotopy between $f$ and $g$, then
$c \circ h \circ a$ defines a homotopy between $c \circ f \circ a$ and
$c \circ g \circ a$.
\end{lemma}

\begin{proof}
Immediate from Homology, Lemma \ref{homology-lemma-compose-homotopy-cochain}.
\end{proof}

\noindent
This lemma allows us to define the homotopy category as follows.

\begin{definition}
\label{definition-complexes-notation}
Let $(A, \text{d})$ be a differential graded algebra.
The {\it homotopy category}, denoted $K(\text{Mod}_{(A, \text{d})})$, is
the category whose objects are the objects of
$\text{Mod}_{(A, \text{d})}$ and whose morphisms are homotopy classes
of homomorphisms of differential graded $A$-modules.
\end{definition}

\noindent
The notation $K(\text{Mod}_{(A, \text{d})})$ is not standard but at least is
consistent with the use of $K(-)$ in other places of the Stacks project.

\begin{lemma}
\label{lemma-homotopy-direct-sums}
Let $(A, \text{d})$ be a differential graded algebra.
The homotopy category $K(\text{Mod}_{(A, \text{d})})$
has arbitrary direct sums.
\end{lemma}

\begin{proof}
Omitted.
\end{proof}







\section{Cones}
\label{section-cones}

\noindent
We introduce cones for the category of differential graded modules.

\begin{definition}
\label{definition-cone}
Let $(A, \text{d})$ be a differential graded algebra.
Let $f : K \to L$ be a homomorphism of differential graded $A$-modules.
The {\it cone} of $f$ is the differential graded $A$-module
$C(f)$ given by $C(f) = L \oplus K$ with grading
$C(f)^n = L^n \oplus K^{n + 1}$ and
differential
$$
d_{C(f)} =
\left(
\begin{matrix}
\text{d}_L & f \\
0 & -\text{d}_K
\end{matrix}
\right)
$$
It comes equipped with canonical morphisms of complexes $i : L \to C(f)$
and $p : C(f) \to K[1]$ induced by the obvious maps $L \to C(f)$
and $C(f) \to K$.
\end{definition}

\noindent
The formation of the cone triangle is functorial in the following sense.

\begin{lemma}
\label{lemma-functorial-cone}
Let $(A, \text{d})$ be a differential graded algebra.
Suppose that
$$
\xymatrix{
K_1 \ar[r]_{f_1} \ar[d]_a & L_1 \ar[d]^b \\
K_2 \ar[r]^{f_2} & L_2
}
$$
is a diagram of homomorphisms of diferential graded $A$-modules which is
commutative up to homotopy.
Then there exists a morphism $c : C(f_1) \to C(f_2)$ which gives rise to
a morphism of triangles
$$
(a, b, c) : (K_1, L_1, C(f_1), f_1, i_1, p_1) \to
(K_1, L_1, C(f_1), f_2, i_2, p_2)
$$
in $K(\text{Mod}_{(A, \text{d})})$.
\end{lemma}

\begin{proof}
Let $h : K_1 \to L_2$ be a homotopy between $f_2 \circ a$ and $b \circ f_1$.
Define $c$ by the matrix
$$
c =
\left(
\begin{matrix}
b & h \\
0 & a
\end{matrix}
\right) :
L_1 \oplus K_1 \to L_2 \oplus K_2
$$
A matrix computation show that $c$ is a morphism of differential
graded modules. It is trivial that $c \circ i_1 = i_2 \circ b$, and it is
trivial also to check that $p_2 \circ c = a \circ p_1$.
\end{proof}











\section{Admissible short exact sequences}
\label{section-admissible}

\noindent
An admissible short exact sequence is the analogue of termwise split exact
sequences in the setting of differential graded modules.

\begin{definition}
\label{definition-admissible-ses}
Let $(A, \text{d})$ be a differential graded algebra.
\begin{enumerate}
\item A homomorphism $K \to L$ of differential graded $A$-modules
is an {\it admissible monomorphism} if there exists a graded $A$-module
map $L \to K$ which is left inverse to $K \to L$.
\item A homomorphism $L \to M$ of differential graded $A$-modules
is an {\it admissible epimorphism} if there exists a graded $A$-module
map $M \to L$ which is right inverse to $L \to M$.
\item A short exact sequence $0 \to K \to L \to M \to 0$ of differential
graded $A$-modules is an {\it admissible short exact sequence}
if it is split as a sequence of graded $A$-modules.
\end{enumerate}
\end{definition}

\noindent
Thus the splittings are compatible with all the data except for
the differentials. Given an admissible short exact sequence we
obtain a triangle; this is the reason that we require our splittings
to be compatible with the $A$-module structure.

\begin{lemma}
\label{lemma-admissible-ses}
Let $(A, \text{d})$ be a differential graded algebra.
Let $0 \to K \to L \to M \to 0$ be an admissible short exact sequence
of differential graded $A$-modules. Let $s : M \to L$ and $\pi : L \to K$
be splittings such that $\text{ker}(\pi) = \text{Im}(s)$.
Then we obtain a morphism
$$
\delta = \pi \circ \text{d}_L \circ s : M \to K[1]
$$
of $\text{Mod}_{(A, \text{d})}$ which induces the boundary maps
in the long exact sequence of cohomology (\ref{equation-les}).
\end{lemma}

\begin{proof}
The map $\pi \circ \text{d}_L \circ s$ is compatible with the $A$-module
structure and the gradings by construction. It is compatible with
differentials by Homology, Lemmas
\ref{homology-lemma-ses-termwise-split-cochain}.
Let $R$ be the ring that $A$ is a differential graded algebra over.
The equality of maps is a statement about $R$-modules. Hence this
follows from Homology, Lemmas
\ref{homology-lemma-ses-termwise-split-cochain} and
\ref{homology-lemma-ses-termwise-split-long-cochain}.
\end{proof}

\begin{lemma}
\label{lemma-make-commute-map}
Let $(A, \text{d})$ be a differential graded algebra. Let
$$
\xymatrix{
K \ar[r]_f \ar[d]_a & L \ar[d]^b \\
M \ar[r]^g & N
}
$$
be a diagram of homomorphisms of differential graded $A$-modules
commuting up to homotopy.
\begin{enumerate}
\item If $f$ is an admissible monomorphism, then $b$ is homotopic to a
homomorphism which makes the diagram commute.
\item If $g$ is an admissible epimorphism, then $a$ is homotopic to a
morphism which makes the diagram commute.
\end{enumerate}
\end{lemma}

\begin{proof}
Let $h : K \to N$ be a homotopy between $bf$ and $ga$, i.e.,
$bf - ga = \text{d}h + h\text{d}$. Suppose that $\pi : L \to K$
is a graded $A$-module map left inverse to $f$. Take
$b' = b - \text{d}h\pi - h\pi \text{d}$.
Suppose $s : N \to M$ is a graded $A$-module map right inverse to $g$.
Take $a' = a + \text{d}sh + sh\text{d}$.
Computations omitted.
\end{proof}

\begin{lemma}
\label{lemma-make-injective}
Let $(A, \text{d})$ be a differential graded algebra.
Let $\alpha : K \to L$ be a homomorphism of differential graded
$A$-modules. There exists a factorization
$$
\xymatrix{
K \ar[r]^{\tilde \alpha} \ar@/_1pc/[rr]_\alpha &
\tilde L \ar[r]^\pi & L
}
$$
in $\text{Mod}_{(A, \text{d})}$ such that
\begin{enumerate}
\item $\tilde \alpha$ is an admissible monomorphism (see
Definition \ref{definition-admissible-ses}),
\item there is a morphism $s : L \to \tilde L$
such that $\pi \circ s = \text{id}_L$ and such that
$s \circ \pi$ is homotopic to $\text{id}_{\tilde L}$.
\end{enumerate}
\end{lemma}

\begin{proof}
The proof is identical to the proof of
Derived Categories, Lemma \ref{derived-lemma-make-injective}.
Namely, we set $\tilde L = L \oplus C(1_K)$ and we use elementary
properties of the cone construction.
\end{proof}

\begin{lemma}
\label{lemma-sequence-maps-split}
Let $(A, \text{d})$ be a differential graded algebra.
Let $L_1 \to L_2 \to \ldots \to L_n$
be a sequence of composable homomorphisms of
differential graded $A$-modules.
There exists a commutative diagram
$$
\xymatrix{
L_1 \ar[r] &
L_2 \ar[r] &
\ldots \ar[r] &
L_n \\
M_1 \ar[r] \ar[u] &
M_2 \ar[r] \ar[u] &
\ldots \ar[r] &
M_n \ar[u]
}
$$
in $\text{Mod}_{(A, \text{d})}$ such that each $M_i \to M_{i + 1}$
is an admissible monomorphism and each $M_i \to L_i$
is a homotopy equivalence.
\end{lemma}

\begin{proof}
The case $n = 1$ is without content.
Lemma \ref{lemma-make-injective} is the case $n = 2$.
Suppose we have constructed the diagram
except for $M_n$. Apply Lemma \ref{lemma-make-injective} to
the composition $M_{n - 1} \to L_{n - 1} \to L_n$.
The result is a factorization $M_{n - 1} \to M_n \to L_n$
as desired.
\end{proof}



\begin{lemma}
\label{lemma-nilpotent}
Let $(A, \text{d})$ be a differential graded algebra.
Let $0 \to K_i \to L_i \to M_i \to 0$, $i = 1, 2, 3$
be admissible short exact sequence of differential graded $A$-modules.
Let $b : L_1 \to L_2$ and $b' : L_2 \to L_3$
be homomorphisms of differential graded modules such that
$$
\vcenter{
\xymatrix{
K_1 \ar[d]_0 \ar[r] &
L_1 \ar[r] \ar[d]_b &
M_1 \ar[d]_0 \\
K_2 \ar[r] & L_2 \ar[r] & M_2
}
}
\quad\text{and}\quad
\vcenter{
\xymatrix{
K_2 \ar[d]^0 \ar[r] &
L_2 \ar[r] \ar[d]^{b'} &
M_2 \ar[d]^0 \\
K_3 \ar[r] & L_3 \ar[r] & M_3
}
}
$$
commute up to homotopy. Then $b' \circ b$ is homotopic to $0$.
\end{lemma}

\begin{proof}
By Lemma \ref{lemma-make-commute-map} we can replace $b$ and $b'$ by
homotopic maps such that the right square of the left diagram commutes
and the left square of the right diagram commutes. In other words, we have
$\text{Im}(b) \subset \text{Im}(K_2 \to L_2)$ and
$\text{ker}((b')^n) \supset \text{Im}(K_2 \to L_2)$.
Then $b \circ b' = 0$ as a map of modules.
\end{proof}
















\section{Distinguished triangles}
\label{section-distinguished}

\noindent
The following lemma produces our distinguished triangles.

\begin{lemma}
\label{lemma-triangle-independent-splittings}
Let $(A, \text{d})$ be a differential graded algebra. Let
$0 \to K \to L \to M \to 0$ be an admissible short exact sequence
of differential graded $A$-modules. The triangle
\begin{equation}
\label{equation-triangle-associated-to-admissible-ses}
K \to L \to M \xrightarrow{\delta} K[1]
\end{equation}
with $\delta$ as in Lemma \ref{lemma-admissible-ses} is, up to canonical
isomorphism in $K(\text{Mod}_{(A, \text{d})})$, independent of the choices
made in Lemma \ref{lemma-admissible-ses}.
\end{lemma}

\begin{proof}
Namely, let $(s', \pi')$ be a second choice of splittings as in
Lemma \ref{lemma-admissible-ses}. Then we claim that $\delta$ and $\delta'$
are homotopic. Namely, write $s' = s + \alpha \circ h$ and
$\pi' = \pi + g \circ \beta$ for some unique homomorphisms
of $A$-modules $h : M \to K$ and $g : M \to K$ of degree $-1$.
Then $g = -h$ and $g$ is a homotopy between $\delta$ and $\delta'$.
The computations are done in the proof of
Homology, Lemma \ref{homology-lemma-ses-termwise-split-homotopy-cochain}.
\end{proof}

\begin{definition}
\label{definition-distinguished-triangle}
Let $(A, \text{d})$ be a differential graded algebra.
\begin{enumerate}
\item If $0 \to K \to L \to M \to 0$ is an admissible short exact sequence
of differential graded $A$-modules, then the {\it triangle associated
to $0 \to K \to L \to M \to 0$} is the triangle 
(\ref{equation-triangle-associated-to-admissible-ses})
of $K(\text{Mod}_{(A, \text{d})})$.
\item A triangle of $K(\text{Mod}_{(A, \text{d})})$ is called a
{\it distinguished triangle} if it is isomorphic to a triangle
associated to an admissible short exact sequence
of differential graded $A$-modules.
\end{enumerate}
\end{definition}









\section{Cones and distinguished triangles}
\label{section-cones-and-triangles}

\noindent
Let $(A, \text{d})$ be a differential graded algebra.
Let $f : K \to L$ be a homomorphism of differential graded $A$-modules.
Then $(K, L, C(f), f, i, p)$ forms a triangle:
$$
K \to L \to C(f) \to K[1]
$$
in $\text{Mod}_{(A, \text{d})}$ and hence in $K(\text{Mod}_{(A, \text{d})})$.
Cones are {\bf not} distinguished triangles in general, but the difference
is a sign or a rotation (your choice). Here are two precise statements.

\begin{lemma}
\label{lemma-rotate-cone}
Let $(A, \text{d})$ be a differential graded algebra.
Let $f : K \to L$ be a homomorphism of differential graded modules.
The triangle $(L, C(f), K[1], i, p, f[1])$ is
the triangle associated to the admissible short exact sequence
$$
0 \to L \to C(f) \to K[1] \to 0
$$
coming from the definition of the cone of $f$.
\end{lemma}

\begin{proof}
Immediate from the definitions.
\end{proof}

\begin{lemma}
\label{lemma-rotate-triangle}
Let $(A, \text{d})$ be a differential graded algebra.
Let $\alpha : K \to L$ and $\beta : L \to M$
define an admissible short exact sequence
$$
0 \to K \to L \to M \to 0
$$
of differential graded $A$-modules.
Let $(K, L, M, \alpha, \beta, \delta)$
be the associated triangle. Then the triangles
$$
(M[-1], K, L, \delta[-1], \alpha, \beta)
\quad\text{and}\quad
(M[-1], K, C(\delta[-1]), \delta[-1], i, p)
$$
are isomorphic.
\end{lemma}

\begin{proof}
Using a choice of splittings we write $L = K \oplus M$ and we identify
$\alpha$ and $\beta$ with the natural inclusion and projection maps.
By construction of $\delta$ we have
$$
d_B =
\left(
\begin{matrix}
d_K & \delta \\
0 & d_M
\end{matrix}
\right)
$$
On the other hand the cone of $\delta[-1] : M[-1] \to K$
is given as $C(\delta[-1]) = K \oplus M$ with differential identical
with the matrix above! Whence the lemma.
\end{proof}

\begin{lemma}
\label{lemma-third-isomorphism}
Let $(A, \text{d})$ be a differential graded algebra.
Let $f_1 : K_1 \to L_1$ and $f_2 : K_2 \to L_2$ be homomorphisms of
differential graded $A$-modules. Let
$$
(a, b, c) :
(K_1, L_1, C(f_1), f_1, i_1, p_1)
\longrightarrow
(K_1, L_1, C(f_1), f_2, i_2, p_2)
$$
be any morphism of triangles of $K(\text{Mod}_{(A, \text{d})})$.
If $a$ and $b$ are homotopy equivalences then so is $c$.
\end{lemma}

\begin{proof}
Let $a^{-1} : K_2 \to K_1$ be a homomorphism of differential graded $A$-modules
which is inverse to $a$ in $K(\text{Mod}_{(A, \text{d})})$.
Let $b^{-1} : L_2 \to L_1$ be a homomorphism of differential graded $A$-modules
which is inverse to $b$ in $K(\text{Mod}_{(A, \text{d})})$.
Let $c' : C(f_2) \to C(f_1)$ be the morphism from
Lemma \ref{lemma-functorial-cone} applied to
$f_1 \circ a^{-1} = b^{-1} \circ f_2$.
If we can show that $c \circ c'$ and $c' \circ c$ are isomorphisms in
$K(\text{Mod}_{(A, \text{d})})$
then we win. Hence it suffices to prove the following: Given
a morphism of triangles
$(1, 1, c) : (K, L, C(f), f, i, p)$
in $K(\text{Mod}_{(A, \text{d})})$ the morphism $c$ is an isomorphism
in $K(\text{Mod}_{(A, \text{d})})$.
By assumption the two squares in the diagram
$$
\xymatrix{
L \ar[r] \ar[d]_1 &
C(f) \ar[r] \ar[d]_c &
K[1] \ar[d]_1 \\
L \ar[r] &
C(f) \ar[r] &
K[1]
}
$$
commute up to homotopy. By construction of $C(f)$ the rows
form admissible short exact sequences. Thus we see that
$(c - 1)^2 = 0$ in $K(\text{Mod}_{(A, \text{d})})$ by
Lemma \ref{lemma-nilpotent}.
Hence $c$ is an isomorphism in $K(\text{Mod}_{(A, \text{d})})$
with inverse $2 - c$.
\end{proof}

\noindent
The following lemma shows that the collection of triangles of the homotopy
category given by cones and the distinguished triangles are the same
up to isomorphisms, at least up to sign!

\begin{lemma}
\label{lemma-the-same-up-to-isomorphisms}
Let $(A, \text{d})$ be a differential graded algebra.
\begin{enumerate}
\item Given an admissible short exact sequence
$0 \to K \xrightarrow{\alpha} L \to M \to 0$
of differential graded $A$-modules there exists a homotopy equivalence
$C(\alpha) \to M$ such that the diagram
$$
\xymatrix{
K \ar[r] \ar[d] & L \ar[d] \ar[r] &
C(\alpha) \ar[r]_{-p} \ar[d] & K[1] \ar[d] \\
K \ar[r]^\alpha & L \ar[r]^\beta &
M \ar[r]^\delta & K[1]
}
$$
defines an isomorphism of triangles in $K(\text{Mod}_{(A, \text{d})})$.
\item Given a morphism of complexes $f : K \to L$
there exists an isomorphism of triangles
$$
\xymatrix{
K \ar[r] \ar[d] & \tilde L \ar[d] \ar[r] &
M \ar[r]_{\delta} \ar[d] & K[1] \ar[d] \\
K \ar[r] & L \ar[r] &
C(f) \ar[r]^{-p} & K[1]
}
$$
where the upper triangle is the triangle associated to a
admissible short exact sequence $K \to \tilde L \to M$.
\end{enumerate}
\end{lemma}

\begin{proof}
Proof of (1). We have $C(\alpha) = L \oplus K$ and we simply define
$C(\alpha) \to M$ via the projection onto $L$ followed by $\beta$.
This defines a morphism of differential graded modules because the
compositions $K^{n + 1} \to L^{n + 1} \to M^{n + 1}$ are zero.
Choose splittings $s : M \to L$ and $\pi : L \to K$ with
$\text{Ker}(\pi) = \text{Im}(s)$ and set
$\delta = \pi \circ \text{d}_L \circ s$ as usual.
To get a homotopy inverse we take
$M \to C(\alpha)$ given by $(s , -\delta)$. This is compatible with
differentials because $\delta^n$ can be characterized as the
unique map $M^n \to K^{n + 1}$ such that
$\text{d} \circ s^n - s^{n + 1} \circ \text{d} = \alpha \circ \delta^n$,
see proof of
Homology, Lemma \ref{homology-lemma-ses-termwise-split-cochain}.
The composition $M \to C(f) \to M$ is the identity.
The composition $C(f) \to M \to C(f)$ is equal to the morphism
$$
\left(
\begin{matrix}
s \circ \beta & 0 \\
-\delta \circ \beta & 0
\end{matrix}
\right)
$$
To see that this is homotopic to the identity map
use the homotopy $h : C(\alpha) \to C(\alpha)$
given by the matrix
$$
\left(
\begin{matrix}
0 & 0 \\
\pi & 0
\end{matrix}
\right) :
C(\alpha) = L \oplus K
\to
L \oplus K = C(\alpha)
$$
It is trivial to verify that
$$
\left(
\begin{matrix}
1 & 0 \\
0 & 1
\end{matrix}
\right)
-
\left(
\begin{matrix}
s \\
-\delta
\end{matrix}
\right)
\left(
\begin{matrix}
\beta & 0
\end{matrix}
\right)
=
\left(
\begin{matrix}
\text{d} & \alpha \\
0 & -\text{d}
\end{matrix}
\right)
\left(
\begin{matrix}
0 & 0 \\
\pi & 0
\end{matrix}
\right)
+
\left(
\begin{matrix}
0 & 0 \\
\pi & 0
\end{matrix}
\right)
\left(
\begin{matrix}
\text{d} & \alpha \\
0 & -\text{d}
\end{matrix}
\right)
$$
To finish the proof of (1) we have to show that the morphisms
$-p : C(\alpha) \to K[1]$ (see
Definition \ref{definition-cone})
and $C(\alpha) \to M \to K[1]$ agree up
to homotopy. This is clear from the above. Namely, we can use the homotopy
inverse $(s, -\delta) : M \to C(\alpha)$
and check instead that the two maps
$M \to K[1]$ agree. And note that
$p \circ (s, -\delta) = -\delta$ as desired.

\medskip\noindent
Proof of (2). We let $\tilde f : K \to \tilde L$,
$s : L \to \tilde L$
and $\pi : L \to L$ be as in
Lemma \ref{lemma-make-injective}. By
Lemmas \ref{lemma-functorial-cone} and \ref{lemma-third-isomorphism}
the triangles $(K, L, C(f), i, p)$ and
$(K, \tilde L, C(\tilde f), \tilde i, \tilde p)$
are isomorphic. Note that we can compose isomorphisms of
triangles. Thus we may replace $L$ by
$\tilde L$ and $f$ by $\tilde f$. In other words
we may assume that $f$ is an admissible monomorphism.
In this case the result follows from part (1).
\end{proof}







\section{The homotopy category is triangulated}
\label{section-homotopy-triangulated}

\noindent
We first prove that it is pre-triangulated.

\begin{lemma}
\label{lemma-homotopy-category-pre-triangulated}
Let $(A, \text{d})$ be a differential graded algebra.
The homotopy category $K(\text{Mod}_{(A, \text{d})})$
with its natural translation functors and distinguished triangles
is a pre-triangulated category.
\end{lemma}

\begin{proof}
Proof of TR1. By definition every triangle isomorphic to a distinguished
one is distinguished. Also, any triangle $(K, K, 0, 1, 0, 0)$
is distinguished since $0 \to K \to K \to 0 \to 0$ is
an admissible short exact sequence. Finally, given any homomorphism
$f : K \to L$ of differential graded $A$-modules the triangle
$(K, L, C(f), f, i, -p)$ is distinguished by
Lemma \ref{lemma-the-same-up-to-isomorphisms}.

\medskip\noindent
Proof of TR2. Let $(X, Y, Z, f, g, h)$ be a triangle.
Assume $(Y, Z, X[1], g, h, -f[1])$ is distinguished.
Then there exists an admissible short exact sequence
$0 \to K \to L \to M \to 0$ such that the associated
triangle $(K, L, M, \alpha, \beta, \delta)$
is isomorphic to $(Y, Z, X[1], g, h, -f[1])$. Rotating back we see
that $(X, Y, Z, f, g, h)$ is isomorphic to
$(M[-1], K, L, -\delta[-1], \alpha, \beta)$.
It follows from Lemma \ref{lemma-rotate-triangle} that the triangle
$(M[-1], K, L, \delta[-1], \alpha, \beta)$
is isomorphic to
$(M[-1], K, C(\delta[-1]), \delta[-1], i, p)$.
Precomposing the previous isomorphism of triangles with $-1$ on $Y$
it follows that $(X, Y, Z, f, g, h)$ is isomorphic to
$(M[-1], K, C(\delta[-1]), \delta[-1], i, -p)$.
Hence it is distinguished by
Lemma \ref{lemma-the-same-up-to-isomorphisms}.
On the other hand, suppose that $(X, Y, Z, f, g, h)$ is distinguished.
By Lemma \ref{lemma-the-same-up-to-isomorphisms} this means that it is
isomorphic to a triangle of the form
$(K, L, C(f), f, i, -p)$ for some morphism $f$ of
$\text{Mod}_{(A, \text{d})}$. Then the rotated triangle
$(Y, Z, X[1], g, h, -f[1])$ is
isomorphic to $(L, C(f), K[1], i, -p, -f[1])$ which is
isomorphic to the triangle
$(L, C(f), K[1], i, p, f[1])$.
By Lemma \ref{lemma-rotate-cone} this triangle is distinguished.
Hence $(Y, Z, X[1], g, h, -f[1])$ is distinguished as desired.

\medskip\noindent
Proof of TR3. Let $(X, Y, Z, f, g, h)$ and $(X', Y', Z', f', g', h')$
be distinguished triangles of $K(\mathcal{A})$ and let $a : X \to X'$
and $b : Y \to Y'$ be morphisms such that $f' \circ a = b \circ f$. By
Lemma \ref{lemma-the-same-up-to-isomorphisms} we may assume that
$(X, Y, Z, f, g, h) = (X, Y, C(f), f, i, -p)$ and
$(X', Y', Z', f', g', h') = (X', Y', C(f'), f', i', -p')$.
At this point we simply apply Lemma \ref{lemma-functorial-cone}
to the commutative diagram given by $f, f', a, b$.
\end{proof}

\noindent
Before we prove TR4 in general we prove it in a special case.

\begin{lemma}
\label{lemma-two-split-injections}
Let $(A, \text{d})$ be a differential graded algebra. Suppose that
$\alpha : K \to L$ and $\beta : L \to M$ are admissible monomorphisms
of differential graded $A$-modules. Then there exist distinguished triangles
$(K, L, Q_1, \alpha, p_1, d_1)$, $(K, M, Q_2, \beta \circ \alpha, p_2, d_2)$
and $(L, M, Q_3, \beta, p_3, d_3)$ for which TR4 holds.
\end{lemma}

\begin{proof}
Say $\pi_1 : L \to K$ and $\pi_3 : M \to L$ are homomorphisms
of graded $A$-modules which are left inverse to $\alpha$ and $\beta$.
Then also $K \to M$ is an admissible monomorphism with left
inverse $\pi_2 = \pi_1 \circ \pi_3$. Let us write $Q_1$, $Q_2$
and $Q_3$ for the cokernels of $K \to L$, $K \to M$, and $L \to M$.
Then we obtain identifications (as graded $A$-modules)
$Q_1 = \text{Ker}(\pi_1)$, $Q_3 = \text{Ker}(\pi_3)$ and
$Q_2 = \text{Ker}(\pi_2)$. Then $L = K \oplus Q_1$ and
$M = L \oplus Q_3$ as graded $A$-modules. This implies
$M = K \oplus Q_1 \oplus Q_3$. Note that $\pi_2 = \pi_1 \circ \pi_3$
is zero on both $Q_1$ and $Q_3$. Hence $Q_2 = Q_1 \oplus Q_3$.
Consider the commutative diagram
$$
\begin{matrix}
0 & \to & K & \to & L & \to & Q_1 & \to & 0 \\
  &     & \downarrow&     & \downarrow&     & \downarrow  & \\
0 & \to & K & \to & M & \to & Q_2 & \to & 0 \\
  &     & \downarrow&     & \downarrow&     & \downarrow  & \\
0 & \to & L & \to & M & \to & Q_3 & \to & 0
\end{matrix}
$$
The rows of this diagram are admissible short exact sequences, and
hence determine distinguished triangles by definition. Moreover
downward arrows in the diagram above are compatible with the chosen
splittings and hence define morphisms of triangles
$$
(K \to L \to Q_1 \to K[1])
\longrightarrow
(K \to M \to Q_2 \to K[1])
$$
and
$$
(K \to M \to Q_2 \to K[1])
\longrightarrow
(L \to M \to Q_3 \to L[1]).
$$
Note that the splittings $Q_3 \to M$ of the bottom sequence in the
diagram provides a splitting for the split sequence
$0 \to Q_1 \to Q_2 \to Q_3 \to 0$ upon composing with $M \to Q_2$.
It follows easily from this that the morphism $\delta : Q_3 \to Q_1[1]$
in the corresponding distinguished triangle
$$
(Q_1 \to Q_2 \to Q_3 \to Q_1[1])
$$
is equal to the composition $Q_3 \to L[1] \to Q_1[1]$.
Hence we get a structure as in the conclusion of axiom TR4.
\end{proof}

\noindent
Here is the final result.

\begin{proposition}
\label{proposition-homotopy-category-triangulated}
Let $(A, \text{d})$ be a differential graded algebra. The homotopy category
$K(\text{Mod}_{(A, \text{d})})$ of differential graded $A$-modules with its
natural translation functors and distinguished triangles is a triangulated
category.
\end{proposition}

\begin{proof}
We know that $K(\text{Mod}_{(A, \text{d})})$ is a pre-triangulated category.
Hence it suffices to prove TR4 and to prove it we can use
Derived Categories, Lemma \ref{derived-lemma-easier-axiom-four}.
Let $K \to L$ and $L \to M$ be composable morphisms of
$K(\text{Mod}_{(A, \text{d})})$. By
Lemma \ref{lemma-sequence-maps-split} we may assume that
$K \to L$ and $L \to M$ are admissible monomorphisms.
In this case the result follows from
Lemma \ref{lemma-two-split-injections}.
\end{proof}













\section{Projective modules over algebras}
\label{section-projectives-over-algebras}

\noindent
In this section we discuss projective modules over algebras
and over graded algebras. Thus it is the analogue of
Algebra, Section \ref{algebra-section-projective}
in the setting of this chapter.

\medskip\noindent
{\bf Algebras and modules.} Let $R$ be a ring and let $A$ be an
$R$-algebra, see Section \ref{section-conventions} for our conventions.
It is clear that $A$ is a projective right $A$-module since
$\Hom_A(A, M) = M$ for any right $A$-module $M$ (and thus $\Hom_A(A, -)$
is exact). Conversely, let $P$ be a projective right $A$-module. Then
we can choose a surjection
$\bigoplus_{i \in I} A \to M$ by choosing a set $\{m_i\}_{i \in I}$
of generators of $P$ over $A$. Since $P$ is projective there is a
left inverse to the surjection, and we find that $P$ is isomorphic
to a direct summand of a free module, exactly as in the commutative case
(Algebra, Lemma \ref{algebra-lemma-characterize-projective}).

\medskip\noindent
{\bf Graded algebras and modules.}
Let $R$ be a ring. Let $A$ be a graded algebra over $R$.
Let $\text{Mod}_A$ denote the category of graded right $A$-modules.
For an integer $k$ let $A[k]$ denote the shift of $A$.
For an graded right $A$-module we have
$$
\Hom_{\text{Mod}_A}(A[k], M) = M^{-k}
$$
As the functor $M \mapsto M^{-k}$ is exact on $\text{Mod}_A$ we
conclude that $A[k]$ is a projective object of $\text{Mod}_A$.
Conversely, suppose that $P$ is a projective object of $\text{Mod}_A$.
By choosing a set of homogeneous generators of $P$ as an $A$-module,
we can find a surjection
$$
\bigoplus\nolimits_{i \in I} A[k_i] \longrightarrow P
$$
Thus we conclude that a projective object of $\text{Mod}_A$ is
a direct summand of a direct sum of the shifts $A[k]$.

\medskip\noindent
If $(A, \text{d})$ is a differential graded algebra and $P$ is
an object of $\text{Mod}_{(A, \text{d})}$ then we say
{\it $P$ is projective as a graded $A$-module} or sometimes
{\it $P$ is graded projective} to mean
that $P$ is a projective object of the abelian category $\text{Mod}_A$
of graded $A$-modules.

\begin{lemma}
\label{lemma-target-graded-projective}
Let $(A, \text{d})$ be a differential graded algebra.
Let $M \to P$ be a surjective homomorphism of differential graded
$A$-modules. If $P$ is projective as a graded $A$-module, then
$M \to P$ is an admissible epimorphism.
\end{lemma}

\begin{proof}
This is immediate from the definitions.
\end{proof}

\begin{lemma}
\label{lemma-hom-from-shift-free}
Let $(A, d)$ be a differential graded algebra. Then we have
$$
\Hom_{\text{Mod}_{(A, \text{d})}}(A[k], M) =
\text{Ker}(\text{d} : M^{-k} \to M^{-k + 1})
$$
and
$$
\Hom_{K(\text{Mod}_{(A, \text{d})})}(A[k], M) = H^{-k}(M)
$$
for any differential graded $A$-module $M$.
\end{lemma}

\begin{proof}
This is clear from the discussion above.
\end{proof}







\section{Injective modules over algebras}
\label{section-modules-noncommutative}

\noindent
In this section we discuss injective modules over algebras
and over graded algebras. Thus it is the analogue of
More on Algebra, Section \ref{more-algebra-section-injectives-modules}
in the setting of this chapter.

\medskip\noindent
{\bf Algebras and modules.} Let $R$ be a ring and let $A$ be an
$R$-algebra, see Section \ref{section-conventions} for our conventions.
For a right $A$-module $M$ we set
$$
M^\vee = \Hom_\mathbf{Z}(M, \mathbf{Q}/\mathbf{Z})
$$
which we think of as a left $A$-module by the multiplication
$(a f)(x) = f(xa)$. Namely, $((ab)f)(x) = f(xab) = (bf)(xa) = (a(bf))(x)$.
Conversely, if $M$ is a left $A$-module, then $M^\vee$ is a right
$A$-module. Since $\mathbf{Q}/\mathbf{Z}$ is an injective abelian
group (More on Algebra, Lemma \ref{more-algebra-lemma-injective-abelian}), the
functor $M \mapsto M^\vee$ is exact
(More on Algebra, Lemma \ref{more-algebra-lemma-vee-exact}).
Moreover, the evaluation map $M \to (M^\vee)^\vee$ is
injective for all modules $M$
(More on Algebra, Lemma \ref{more-algebra-lemma-ev-injective}).

\medskip\noindent
We claim that $A^\vee$ is an injective right $A$-module. Namely, given
a right $A$-module $N$ we have
$$
\Hom_A(N, A^\vee) =
\Hom_A(N, \Hom_\mathbf{Z}(A, \mathbf{Q}/\mathbf{Z})) = N^\vee
$$
and we conclude because the functor $N \mapsto N^\vee$ is exact.
The second equality holds because
$$
\Hom_\mathbf{Z}(N, \Hom_\mathbf{Z}(A, \mathbf{Q}/\mathbf{Z})) =
\Hom_\mathbf{Z}(N \otimes_\mathbf{Z} A, \mathbf{Q}/\mathbf{Z})
$$
by Algebra, Lemma \ref{algebra-lemma-hom-from-tensor-product}.
Inside this module $A$-linearity exactly picks out the bilinear maps
$\varphi : N \times A \to \mathbf{Q}/\mathbf{Z}$ which
have the same value on $x \otimes a$ and $xa \otimes 1$, i.e.,
come from elements of $N^\vee$.

\medskip\noindent
Finally, for every right $A$-module $M$ we can choose a surjection
$\bigoplus_{i \in I} A \to M^\vee$ to get an injection
$M \to (M^\vee)^\vee \to \prod_{i \in I} A^\vee$.

\medskip\noindent
We conclude
\begin{enumerate}
\item the category of $A$-modules has enough injectives,
\item $A^\vee$ is an injective $A$-module, and
\item every $A$-module injects into a product of copies of $A^\vee$.
\end{enumerate}

\medskip\noindent
{\bf Graded algebras and modules.}
Let $R$ be a ring. Let $A$ be a graded algebra over $R$.
If $M$ is a graded $A$-module we set
$$
M^\vee =
\bigoplus\nolimits_{n \in \mathbf{Z}}
\Hom_\mathbf{Z}(M^{-n}, \mathbf{Q}/\mathbf{Z}) =
\bigoplus\nolimits_{n \in \mathbf{Z}} (M^{-n})^\vee
$$
as a graded $R$-module with the $A$-module structure defined as above
(for homogeneous elements). This again switches left and right modules.
On the category of graded $A$-modules the functor $M \mapsto M^\vee$
is exact (check on graded pieces). Moreover, the evaluation map
$M \to (M^\vee)^\vee$ is injective as before (because we can check
this on the graded pieces).

\medskip\noindent
We claim that $A^\vee$ is an injective object of the category
$\text{Mod}_A$ of graded right $A$-modules. Namely, given a graded
right $A$-module $N$ we have
$$
\Hom_{\text{Mod}_A}(N, A^\vee) =
\Hom_{\text{Mod}_A}(
N, \bigoplus \Hom_\mathbf{Z}(A^{-n}, \mathbf{Q}/\mathbf{Z})) = (N^0)^\vee
$$
and we conclude because the functor $N \mapsto (N^0)^\vee = (N^\vee)^0$
is exact. To see that the second equality holds we use the equalities
$$
\Hom_\mathbf{Z}(N^n, \Hom_\mathbf{Z}(A^{-n}, \mathbf{Q}/\mathbf{Z})) =
\Hom_\mathbf{Z}(N^n \otimes_\mathbf{Z} A^{-n}, \mathbf{Q}/\mathbf{Z})
$$
of Algebra, Lemma \ref{algebra-lemma-hom-from-tensor-product}.
Thus an element of $\Hom_{\text{Mod}_A}(N, A^\vee)$ corresponds
to a family of $\mathbf{Z}$-bilinear maps
$\psi_n : N^n \times A^{-n} \to \mathbf{Q}/\mathbf{Z}$ such that
$\psi_n(x, a) = \psi_0(xa, 1)$ for all $x \in N^n$ and $a \in A^{-n}$.
Moreover, $\psi_0(x, a) = \psi_0(xa, 1)$ for all $x \in N^0$, $a \in A^0$.
It follows that the maps $\psi_n$ are determined by $\psi_0$ and that
$\psi_0(x, a) = \varphi(xa)$ for a unique element $\varphi \in (N^0)^\vee$.

\medskip\noindent
Finally, for every graded right $A$-module $M$ we can choose a surjection
(of graded left $A$-modules)
$$
\bigoplus\nolimits_{i \in I} A[k_i] \to M^\vee
$$
where $A[k_i]$ denotes the shift of $A$ by $k_i \in \mathbf{Z}$.
(We do this by choosing homogeneous generators for $M^\vee$.)
In this way we get an injection
$$
M \to (M^\vee)^\vee \to \prod A[k_i]^\vee = \prod A^\vee[-k_i]
$$
Observe that the products in the formula above are products in the
category of graded modules (in other words, take products in each degree
and then take the direct sum of the pieces).

\medskip\noindent
We conclude that
\begin{enumerate}
\item the category of graded $A$-modules has enough injectives,
\item for every $k \in \mathbf{Z}$ the module $A^\vee[k]$ is injective, and
\item every $A$-module injects into a product in the category of graded
modules of copies of shifts $A^\vee[k]$.
\end{enumerate}

\medskip\noindent
If $(A, \text{d})$ is a differential graded algebra and $I$ is
an object of $\text{Mod}_{(A, \text{d})}$ then we say
{\it $I$ is injective as a graded $A$-module} to mean
that $I$ is a injective object of the abelian category $\text{Mod}_A$
of graded $A$-modules.

\begin{lemma}
\label{lemma-source-graded-injective}
Let $(A, \text{d})$ be a differential graded algebra.
Let $I \to M$ be an injective homomorphism of differential graded
$A$-modules. If $I$ is an injective object of the category
of graded $A$-modules, then $I \to M$ is an admissible monomorphism.
\end{lemma}

\begin{proof}
This is immediate from the definitions.
\end{proof}

\noindent
Let $(A, \text{d})$ be a differential graded algebra. If $M$ is a
(right) differential graded $A$-module, then $M^\vee$ (as defined above
for graded modules) is a differential graded module as well with
differential given by the contragredient of $\text{d}_M$ up to sign.
The sign rule (see Example \ref{example-dgm-dg-cat}) is to set
$$
\text{d}_{M^\vee}(f) = - (-1)^n f \circ \text{d}_M^{-n - 1}
\quad\text{in }(M^\vee)^{n + 1}
$$
where $f \in (M^\vee)^n = \Hom_\mathbf{Z}(M^{-n}, \mathbf{Q}/\mathbf{Z})$
and $\text{d}_M^{-n - 1} : M^{-n - 1} \to M^{-n}$ is the differential
of $M$. This does indeed work, namely, if $a \in A^m$, $x \in M^{-n - m - 1}$
and $f$ as above, then
\begin{align*}
\text{d}_{M^\vee}(a f)(x) & =
- (-1)^{n + m} (a f)(\text{d}(x)) \\
& =
- (-1)^{n + m} f(\text{d}(x)a) \\
& =
- (-1)^{n + m} f(\text{d}(xa) - (-1)^{n + m}x\text{d}(a)) \\
& =
(\text{d}(a) f)(x) + (-1)^m (a \text{d}_{M^\vee}(f))(x)
\end{align*}
in other words we have $\text{d}(af) = \text{d}(a)f + (-1)^m a \text{d}(f)$
as desired.

\medskip\noindent
Similarly, if $M$ is a left differential graded module, then
$M^\vee$ becomes a right differential graded module with the same
sign rule for $\text{d}_{M^\vee}$. Computation omitted.

\begin{lemma}
\label{lemma-map-into-dual}
Let $(A, \text{d})$ be a differential graded algebra. If
$M$ is a left differential graded $A$-module and $N$ is a
right differential graded $A$-module, then
$$
\Hom_{\text{Mod}_{(A, \text{d})}}(N, M^\vee)
$$
is isomorphic to the set of sequences $(\psi_n)$ of $\mathbf{Z}$-bilinear
pairings
$$
\psi_n : N^n \times M^{-n} \longrightarrow \mathbf{Q}/\mathbf{Z}
$$
such that $\psi_{n + m}(y, ax) = \psi_{n + m}(ya, x)$ for all
$y \in N^n$, $x \in M^{-m}$, and $a \in A^{m - n}$ and such that
$\psi_{n + 1}(\text{d}(y), x) + (-1)^n \psi_n(y, \text{d}(x)) = 0$
for all $y \in N^n$ and $x \in M^{-n - 1}$.
\end{lemma}

\begin{proof}
If $f \in \Hom_{\text{Mod}_{(A, \text{d})}}(N, M^\vee)$, then we
map this to the sequence of pairings defined by
$\psi_n(y, x) = f(y)(x)$. It is a computation (omitted) to see
that these pairings satisfy the conditions as in the lemma.
For the converse, use
Algebra, Lemma \ref{algebra-lemma-hom-from-tensor-product} to turn a sequence
of pairings into a map $f : N \to M^\vee$.
\end{proof}

\begin{lemma}
\label{lemma-hom-into-shift-dual-free}
Let $(A, \text{d})$ be a differential graded algebra. Then we have
$$
\Hom_{\text{Mod}_{(A, \text{d})}}(M, A^\vee[k]) =
\text{Ker}(\text{d} : (M^\vee)^k \to (M^\vee)^{k + 1})
$$
and
$$
\Hom_{K(\text{Mod}_{(A, \text{d})})}(A[k], M) = H^k(M^\vee)
$$
for any differential graded $A$-module $M$.
\end{lemma}

\begin{proof}
This is clear from the discussion above.
\end{proof}















\section{P-resolutions}
\label{section-P-resolutions}

\noindent
This section is the analogue of
Derived Categories, Section \ref{derived-section-unbounded}.

\medskip\noindent
Let $(A, \text{d})$ be a differential graded algebra.
Let $P$ be a differential graded $A$-module. We say $P$
{\it has property (P)} if it there exists a filtration
$$
0 = F_{-1}P \subset F_0P \subset F_1P \subset \ldots \subset P
$$
by differential graded submodules such that
\begin{enumerate}
\item $P = \bigcup F_pP$,
\item the inclusions $F_iP \to F_{i + 1}P$ are admissible
monomorphisms,
\item the quotients $F_{i + 1}P/F_iP$ are isomorphic to direct
sums of shifts $A[k]$.
\end{enumerate}
In fact, condition (2) is a consequence of condition (3), see
Lemma \ref{lemma-target-graded-projective}.

\begin{lemma}
\label{lemma-property-P-sequence}
Let $(A, \text{d})$ be a differential graded algebra.
Let $P$ be a differential graded $A$-module. If $F_\bullet$
is a filtration as in property (P), then we obtain an
admissible short exact sequence
$$
0 \to
\bigoplus\nolimits F_iP \to
\bigoplus\nolimits F_iP \to P \to 0
$$
of differential graded $A$-modules.
\end{lemma}

\begin{proof}
The second map is the direct sum of the inclusion maps.
The first map on the summand $F_iP$ of the source is the sum
of the identity $F_iP \to F_iP$ and the negative of the inclusion
map $F_iP \to P_{i + 1}P$. Choose homomorphisms $s_i : F_{i + 1}P \to F_iP$
of graded $A$-modules which are left inverse to the inclusion maps.
Composing gives maps $s_{j, i} : F_jP \to F_iP$ for all $j > i$.
Then a left inverse of the first arrow maps $x \in F_jP$ to
$(s_{j, 0}(x), s_{j, 1}(x), \ldots, s_{j, j - 1}(x), 0, \ldots)$
in $\bigoplus F_iP$.
\end{proof}

\noindent
The following lemma shows that differential graded modules with
property (P) are the dual notion to K-injective modules
(i.e., they are K-projective in some sense). See
Derived Categories, Definition \ref{derived-definition-K-injective}.

\begin{lemma}
\label{lemma-property-P-K-projective}
Let $(A, \text{d})$ be a differential graded algebra.
Let $P$ be a differential graded $A$-module with property (P).
Then
$$
\Hom_{K(\text{Mod}_{(A, \text{d})})}(P, N) = 0
$$
for all acyclic differential graded $A$-modules $N$.
\end{lemma}

\begin{proof}
We will use that $K(\text{Mod}_{(A, \text{d})})$ is a triangulated
category (Proposition \ref{proposition-homotopy-category-triangulated}).
Let $F_\bullet$ be a filtration on $P$ as in property (P).
The short exact sequence of Lemma \ref{lemma-property-P-sequence}
produces a distinguished triangle. Hence by
Derived Categories, Lemma \ref{derived-lemma-representable-homological}
it suffices to show that
$$
\Hom_{K(\text{Mod}_{(A, \text{d})})}(F_iP, N) = 0
$$
for all acyclic differential graded $A$-modules $N$ and all $i$.
Each of the differential graded modules $F_iP$ has a finite filtration
by admissible monomorphisms, whose graded pieces are direct sums
of shifts $A[k]$. Thus it suffices to prove that
$$
\Hom_{K(\text{Mod}_{(A, \text{d})})}(A[k], N) = 0
$$
for all acyclic differential graded $A$-modules $N$ and all $k$.
This follows from Lemma \ref{lemma-hom-from-shift-free}.
\end{proof}

\begin{lemma}
\label{lemma-good-quotient}
Let $(A, \text{d})$ be a differential graded algebra.
Let $M$ be a differential graded $A$-module. There exists a homomorphism
$P \to M$ of differential graded $A$-modules with the following
properties
\begin{enumerate}
\item $P \to M$ is surjective,
\item $\text{Ker}(\text{d}_P) \to \text{Ker}(\text{d}_M)$ is surjective, and
\item $P$ sits in an admissible short exact sequence
$0 \to P' \to P \to P'' \to 0$ where $P'$, $P''$ are direct sums
of shifts of $A$.
\end{enumerate}
\end{lemma}

\begin{proof}
Let $P_k$ be the free $A$-module with generators $x, y$ in degrees
$k$ and $k + 1$. Define the structure of a differential graded
$A$-module on $P_k$ by setting $\text{d}(x) = y$ and $\text{d}(y) = 0$.
For every element $m \in M^k$ there is a homomorphism
$P_k \to M$ sending $x$ to $m$ and $y$ to $\text{d}(m)$.
Thus we see that there is a surjection from a direct sum
of copies of $P_k$ to $M$. This clearly produces $P \to M$
having properties (1) and (3). To obtain property (2) note
that if $m \in \text{Ker}(\text{d}_M)$ has degree $k$, then there is a map
$A[k] \to M$ mapping $1$ to $m$. Hence we can achieve (2) by adding
a direct sum of copies of shifts of $A$.
\end{proof}

\begin{lemma}
\label{lemma-resolve}
Let $(A, \text{d})$ be a differential graded algebra.
Let $M$ be a differential graded $A$-module. There exists a homomorphism
$P \to M$ of differential graded $A$-modules such that
\begin{enumerate}
\item $P \to M$ is a quasi-isomorphism, and
\item $P$ has property (P).
\end{enumerate}
\end{lemma}

\begin{proof}
Set $M = M_0$. We inductively choose short exact sequences
$$
0 \to M_{i + 1} \to P_i \to M_i \to 0
$$
where the maps $P_i \to M_i$ are chosen as in Lemma \ref{lemma-good-quotient}.
This gives a ``resolution''
$$
\ldots \to P_2 \xrightarrow{f_2} P_1 \xrightarrow{f_1} P_0 \to M \to 0
$$
Then we set
$$
P = \bigoplus\nolimits_{i \geq 0} P_i
$$
as an $A$-module with grading given by
$P^n = \bigoplus_{a + b = n} P_{-a}^b$ and
differential (as in the construction of the total complex associated
to a double complex) by
$$
\text{d}_P(x) = f_{-a}(x) + (-1)^a \text{d}_{P_{-a}}(x)
$$
for $x \in P_{-a}^b$. With these conventions $P$ is indeed a differential
graded $A$-module. Recalling that each $P_i$ has a two step filtration
$0 \to P_i' \to P_i \to P_i'' \to 0$ we set
$$
F_{2i}P = \bigoplus\nolimits_{i \geq j \geq 0} P_j
\subset
\bigoplus\nolimits_{i \geq 0} P_i = P
$$
and we add $P'_{i + 1}$ to $F_{2i}P$ to get $F_{2i + 1}$.
These are differential graded submodules and the successive quotients
are direct sums of shifts of $A$. By
Lemma \ref{lemma-target-graded-projective} we see that
the inclusions $F_iP \to F_{i + 1}P$ are admissible monomorphisms.
Finally, we have to show that the map $P \to M$ (given by the
augmentation $P_0 \to M$) is a quasi-isomorphism. This follows from
Homology, Lemma \ref{homology-lemma-good-resolution-gives-qis}.
\end{proof}





\section{I-resolutions}
\label{section-I-resolutions}

\noindent
This section is the dual of the section on P-resolutions.

\medskip\noindent
Let $(A, \text{d})$ be a differential graded algebra.
Let $I$ be a differential graded $A$-module. We say $I$
{\it has property (I)} if it there exists a filtration
$$
I = F_0I \supset F_1I \supset F_2I \supset \ldots \supset 0
$$
by differential graded submodules such that
\begin{enumerate}
\item $I = \lim I/F_pI$,
\item the inclusions $F_iI \to F_{i - 1}I$ are admissible
monomorphisms,
\item the quotients $F_{i - 1}I/F_iI$ are direct sums of the\
modules $A^\vee[k]$.
\end{enumerate}
In fact, condition (2) is a consequence of condition (3), see
Lemma \ref{lemma-source-graded-injective}.

\begin{lemma}
\label{lemma-property-I-sequence}
Let $(A, \text{d})$ be a differential graded algebra.
Let $I$ be a differential graded $A$-module. If $F_\bullet$
is a filtration as in property (I), then we obtain an
admissible short exact sequence
$$
0 \to I \to
\prod\nolimits I/F_iI \to
\prod\nolimits I/F_iI \to 0
$$
of differential graded $A$-modules.
\end{lemma}

\begin{proof}
Omitted. Hint: This is dual to Lemma \ref{lemma-property-P-sequence}.
\end{proof}

\noindent
The following lemma shows that differential graded modules with
property (I) are the analogue of K-injective modules. See
Derived Categories, Definition \ref{derived-definition-K-injective}.

\begin{lemma}
\label{lemma-property-I-K-injective}
Let $(A, \text{d})$ be a differential graded algebra.
Let $I$ be a differential graded $A$-module with property (I).
Then
$$
\Hom_{K(\text{Mod}_{(A, \text{d})})}(N, I) = 0
$$
for all acyclic differential graded $A$-modules $N$.
\end{lemma}

\begin{proof}
We will use that $K(\text{Mod}_{(A, \text{d})})$ is a triangulated
category (Proposition \ref{proposition-homotopy-category-triangulated}).
Let $F_\bullet$ be a filtration on $I$ as in property (I).
The short exact sequence of Lemma \ref{lemma-property-I-sequence}
produces a distinguished triangle. Hence by
Derived Categories, Lemma \ref{derived-lemma-representable-homological}
it suffices to show that
$$
\Hom_{K(\text{Mod}_{(A, \text{d})})}(N, I/F_iI) = 0
$$
for all acyclic differential graded $A$-modules $N$ and all $i$.
Each of the differential graded modules $I/F_iI$ has a finite filtration
by admissible monomorphisms, whose graded pieces are
direct sums of $A^\vee[k]$. Thus it suffices to prove that
$$
\Hom_{K(\text{Mod}_{(A, \text{d})})}(N, A^\vee[k]) = 0
$$
for all acyclic differential graded $A$-modules $N$ and all $k$.
This follows from Lemma \ref{lemma-hom-into-shift-dual-free}
and the fact that $(-)^\vee$ is an exact functor.
\end{proof}

\begin{lemma}
\label{lemma-good-sub}
Let $(A, \text{d})$ be a differential graded algebra.
Let $M$ be a differential graded $A$-module. There exists a homomorphism
$M \to I$ of differential graded $A$-modules with the following
properties
\begin{enumerate}
\item $M \to I$ is injective,
\item $\text{Coker}(\text{d}_M) \to \text{Coker}(\text{d}_I)$ is injective,
and
\item $I$ sits in an admissible short exact sequence
$0 \to I' \to I \to I'' \to 0$ where $I'$, $I''$ are products
of shifts of $A^\vee$.
\end{enumerate}
\end{lemma}

\begin{proof}
For every $k \in \mathbf{Z}$ let $Q_k$ be the free left $A$-module with
generators $x, y$ in degrees $k$ and $k + 1$. Define the structure of a
left differential graded $A$-module on $Q_k$ by setting $\text{d}(x) = y$
and $\text{d}(y) = 0$. Let $I_k = Q_{-k}^\vee$ be the ``dual'' right
differential graded $A$-module, see
Section \ref{section-modules-noncommutative}.
The next paragraph shows that we can embed $M$ into a product
of copies of $I_k$ (for varying $k$). The dual statement (that any
differential graded module is a quotient of a direct sum of of $P_k$'s)
is easy to prove (see proof of Lemma \ref{lemma-good-quotient})
and using double duals there should be a noncomputational
way to deduce what we want. Thus we suggest skipping the next paragraph.

\medskip\noindent
Given a $\mathbf{Z}$-linear map $\lambda : M^k \to \mathbf{Q}/\mathbf{Z}$
we construct pairings
$$
\psi_n : M^n \times Q_k^{-n} \longrightarrow \mathbf{Q}/\mathbf{Z}
$$
by setting
$$
\psi_n(m, ax + by) = \lambda(ma + (-1)^{k + 1}\text{d}(mb))
$$
for $m \in M^n$, $a \in A^{-n - k}$, and $b \in A^{-n - k - 1}$. We compute
\begin{align*}
\psi_{n + 1}(\text{d}(m), ax + by)
& =
\lambda\left(\text{d}(m)a + (-1)^{k + 1}\text{d}(\text{d}(m)b)\right) \\
& =
\lambda\left(\text{d}(m)a + (-1)^{k + n}\text{d}(m)\text{d}(b)\right)
\end{align*}
and because
$\text{d}(ax + by) = \text{d}(a)x + (-1)^{-n - k}ay + \text{d}(b)y$ we have
\begin{align*}
\psi_n(m, \text{d}(ax + by))
& =
\lambda\left(
m\text{d}(a) + (-1)^{k + 1}\text{d}(m((-1)^{-n - k}a + \text{d}(b)))
\right) \\
& =
\lambda\left(
m\text{d}(a) + (-1)^{-n + 1}\text{d}(ma) + (-1)^{k + 1}\text{d}(m)\text{d}(b)))
\right)
\end{align*}
and we see that
$$
\psi_{n + 1}(\text{d}(m), ax + by) + (-1)^n\psi_n(m, \text{d}(ax + by)) = 0
$$
Thus these pairings define a homomorphism
$f_\lambda : M \to I_k$ by Lemma \ref{lemma-map-into-dual}
such that the composition
$$
M^k \xrightarrow{f^k_\lambda} I_k^k = (Q_k^k)^\vee
\xrightarrow{\text{evaluation at }x} \mathbf{Q}/\mathbf{Z}
$$
is the given map $\lambda$. It is clear that we can find an embedding
into a product of copies of $I_k$'s by using a map of the form
$\prod f_\lambda$ for a suitable choice of the maps $\lambda$.

\medskip\noindent
The result of the previous paragraph produces $M \to I$
having properties (1) and (3). To obtain property (2), suppose
$\overline{m} \in \text{Coker}(\text{d}_M)$ is a nonzero element of
degree $k$. Pick a map $\lambda : M^k \to \mathbf{Q}/\mathbf{Z}$
which vanishes on $\text{Im}(M^{k - 1} \to M^k)$ but not on $m$. By
Lemma \ref{lemma-hom-into-shift-dual-free}
this corresponds to a homomorphism $M \to A^\vee[k]$ of
differential graded $A$-modules which does not vanish on $m$.
Hence we can achieve (2) by adding
a product of copies of shifts of $A^\vee$.
\end{proof}

\begin{lemma}
\label{lemma-right-resolution}
Let $(A, \text{d})$ be a differential graded algebra.
Let $M$ be a differential graded $A$-module. There exists a homomorphism
$M \to I$ of differential graded $A$-modules such that
\begin{enumerate}
\item $M \to I$ is a quasi-isomorphism, and
\item $I$ has property (I).
\end{enumerate}
\end{lemma}

\begin{proof}
Set $M = M_0$. We inductively choose short exact sequences
$$
0 \to M_i \to I_i \to M_{i + 1} \to 0
$$
where the maps $M_i \to I_i$ are chosen as in Lemma \ref{lemma-good-sub}.
This gives a ``resolution''
$$
0 \to M \to I_0 \xrightarrow{f_0} I_1 \xrightarrow{f_1} I_1 \to \ldots
$$
Then we set
$$
I = \prod\nolimits_{i \geq 0} I_i
$$
where we take the product in the category of graded $A$-modules
and differential defined by
$$
\text{d}_I(x) = f_a(x) + (-1)^a \text{d}_{I_a}(x)
$$
for $x \in I_a^b$. With these conventions $I$ is indeed a differential
graded $A$-module. Recalling that each $I_i$ has a two step filtration
$0 \to I_i' \to I_i \to I_i'' \to 0$ we set
$$
F_{2i}P = \prod\nolimits_{j \geq i} I_j
\subset
\prod\nolimits_{i \geq 0} I_i = I
$$
and we add a factor $I'_{i + 1}$ to $F_{2i}I$ to get $F_{2i + 1}I$.
These are differential graded submodules and the successive quotients
are products of shifts of $A^\vee$. By
Lemma \ref{lemma-source-graded-injective} we see that
the inclusions $F_{i + 1}I \to F_iI$ are admissible monomorphisms.
Finally, we have to show that the map $M \to I$ (given by the
augmentation $M \to I_0$) is a quasi-isomorphism. This follows from
Homology, Lemma \ref{homology-lemma-good-right-resolution-gives-qis}.
\end{proof}






\section{The derived category}
\label{section-derived}

\noindent
Recall that the notions of acyclic differential graded modules
and quasi-isomorphism of differential graded modules make sense
(see Section \ref{section-modules}).

\begin{lemma}
\label{lemma-acyclic}
Let $(A, \text{d})$ be a differential graded algebra.
The full subcategory $\text{Ac}$ of $K(\text{Mod}_{(A, \text{d})})$
consisting of acyclic modules is a strictly full saturated triangulated
subcategory of $K(\text{Mod}_{(A, \text{d})})$.
The corresponding saturated multiplicative system
(see Derived Categories, Lemma \ref{derived-lemma-operations})
of $K(\text{Mod}_{(A, \text{d})})$ is the class $\text{Qis}$
of quasi-isomorphisms. In particular, the kernel of the localization
functor
$$
Q : K(\text{Mod}_{(A, \text{d})}) \to
\text{Qis}^{-1}K(\text{Mod}_{(A, \text{d})})
$$
is $\text{Ac}$. Moreover, the functor $H^0$ factors through $Q$.
\end{lemma}

\begin{proof}
We know that $H^0$ is a homological functor by the long exact
sequence of homology (\ref{equation-les}).
The kernel of $H^0$ is the subcategory of acyclic objects and
the arrows with induce isomorphisms on all $H^i$ are the
quasi-isomorphisms. Thus this lemma is a special case of
Derived Categories, Lemma \ref{derived-lemma-acyclic-general}.

\medskip\noindent
Set theoretical remark. The construction of the localization in
Derived Categories, Proposition
\ref{derived-proposition-construct-localization}
assumes the given triangulated category is ``small'', i.e., that the
underlying collection of objects forms a set. Let $V_\alpha$ be a
partial universe (as in Sets, Section \ref{sets-section-sets-hierarchy})
containing $(A, \text{d})$ and where the cofinality of $\alpha$
is bigger than $\aleph_0$
(see Sets, Proposition \ref{sets-proposition-exist-ordinals-large-cofinality}).
Then we can consider the category $\text{Mod}_{(A, \text{d}), \alpha}$
of differential graded $A$-modules contained in $V_\alpha$.
A straightforward check shows that all the constructions used in
the proof of Proposition \ref{proposition-homotopy-category-triangulated}
work inside of $\text{Mod}_{(A, \text{d}), \alpha}$
(because at worst we take finite direct sums of differential graded modules).
Thus we obtain a triangulated category
$\text{Qis}_\alpha^{-1}K(\text{Mod}_{(A, \text{d}), \alpha})$.
We will see below that if $\beta > \alpha$, then the transition functors
$$
\text{Qis}_\alpha^{-1}K(\text{Mod}_{(A, \text{d}), \alpha})
\longrightarrow
\text{Qis}_\beta^{-1}K(\text{Mod}_{(A, \text{d}), \beta})
$$
are fully faithful as the morphism sets in the quotient categories
are computed by maps in the homotopy categories from P-resolutions
(the construction of a P-resolution in the proof of Lemma \ref{lemma-resolve}
takes countable direct sums as well as direct sums indexed over subsets
of the given module). The reader should therefore think of the category
of the lemma as the union of these subcategories.
\end{proof}

\noindent
Taking into account the set theoretical remark at the end of the
proof of the preceding lemma we define the derived category as follows.

\begin{definition}
\label{definition-unbounded-derived-category}
Let $(A, \text{d})$ be a differential graded algebra.
Let $\text{Ac}$ and $\text{Qis}$ be as in Lemma \ref{lemma-acyclic}.
The {\it derived category of $(A, \text{d})$} is the triangulated
category
$$
D(A, \text{d}) =
K(\text{Mod}_{(A, \text{d})})/\text{Ac} =
\text{Qis}^{-1}K(\text{Mod}_{(A, \text{d})}).
$$
We denote $H^0 : D(A, \text{d}) \to \text{Mod}_R$ the unique functor
whose composition with the quotient functor gives back the functor
$H^0$ defined above.
\end{definition}

\noindent
Here is the promised lemma computing morphism sets in the
derived category.

\begin{lemma}
\label{lemma-hom-derived}
Let $(A, \text{d})$ be a differential graded algebra.
Let $M$ and $N$ be differential graded $A$-modules.
\begin{enumerate}
\item Let $P \to M$ be a P-resolution as in
Lemma \ref{lemma-resolve}. Then
$$
\Hom_{D(A, \text{d})}(M, N) =
\Hom_{K(\text{Mod}_{(A, \text{d})})}(P, N)
$$
\item Let $N \to I$ be an I-resolution as in
Lemma \ref{lemma-right-resolution}. Then
$$
\Hom_{D(A, \text{d})}(M, N) =
\Hom_{K(\text{Mod}_{(A, \text{d})})}(M, I)
$$
\end{enumerate}
\end{lemma}

\begin{proof}
Let $P \to M$ be as in (1). Since $P \to M$ is a quasi-isomorphism we see that
$$
\Hom_{D(A, \text{d})}(P, N) = \Hom_{D(A, \text{d})}(M, N)
$$
by definition of the derived category. A morphism
$f : P \to N$ in $D(A, \text{d})$ is equal to
$s^{-1}f'$ where $f' : P \to N'$ is a morphism and
$s : N \to N'$ is a quasi-isomorphism. Choose a distringuished triangle
$$
N \to N' \to Q \to N[1]
$$
As $s$ is a quasi-isomorphism, we see that $Q$ is acyclic. Thus
$\Hom_{K(\text{Mod}_{(A, \text{d})})}(P, Q[k]) = 0$ for all $k$ by
Lemma \ref{lemma-property-P-K-projective}. Since
$\Hom_{K(\text{Mod}_{(A, \text{d})})}(P, -)$
is cohomological, we conclude that we can lift $f' : P \to N'$
uniquely to a morphism $f : P \to N$. This finishes the proof.

\medskip\noindent
The proof of (2) is dual to that of (1) using
Lemma \ref{lemma-property-I-K-injective} in stead of
Lemma \ref{lemma-property-P-K-projective}.
\end{proof}








\section{The canonical delta-functor}
\label{section-canonical-delta-functor}

\noindent
Let $(A, \text{d})$ be a differential graded algebra.
Consider the functor
$\text{Mod}(\mathcal{A}) \to K(\text{Mod}_{(A, \text{d})})$.
This functor is {\bf not} a $\delta$-functor in general.
However, it turns out that the functor
$\text{Mod}_{(A, \text{d})} \to D(A, \text{d})$ is a
$\delta$-functor. In order to see this we have to define
the morphisms $\delta$ associated to a short exact sequence
$$
0 \to K \xrightarrow{a} L \xrightarrow{b} M \to 0
$$
in the abelian category $\text{Mod}_{(A, \text{d})}$.
Consider the cone $C(a)$ of the morphism $a$. We have $C(a) = L \oplus K$
and we define $q : C(a) \to M$ via the projection to $L$ followed
by $b$. Hence a homomorphism of differential graded $A$-modules
$$
q : C(a) \longrightarrow M.
$$
It is clear that $q \circ i = b$ where $i$ is as in
Definition \ref{definition-cone}.
Note that, as $a$ is injective, the kernel of $q$ is identified with the
cone of $\text{id}_K$ which is acyclic. Hence we see that
$q$ is a quasi-isomorphism. According to
Lemma \ref{lemma-the-same-up-to-isomorphisms}
the triangle
$$
(K, L, C(a), a, i, -p)
$$
is a distinguished triangle in $K(\text{Mod}_{(A, \text{d})})$.
As the localization functor
$K(\text{Mod}_{(A, \text{d})}) \to D(A, \text{d})$ is
exact we see that $(K, L, C(a), a, i, -p)$ is a distinguished
triangle in $D(A, \text{d})$. Since $q$ is a quasi-isomorphism
we see that $q$ is an isomorphism in $D(A, \text{d})$.
Hence we deduce that
$$
(K, L, M, a, b, -p \circ q^{-1})
$$
is a distinguished triangle of $D(A, \text{d})$.
This suggests the following lemma.

\begin{lemma}
\label{lemma-derived-canonical-delta-functor}
Let $(A, \text{d})$ be a differential graded algebra. The functor
$\text{Mod}_{(A, \text{d})} \to D(A, \text{d})$
defined has the natural structure of a $\delta$-functor, with
$$
\delta_{K \to L \to M} = - p \circ q^{-1}
$$
with $p$ and $q$ as explained above.
\end{lemma}

\begin{proof}
We have already seen that this choice leads to a distinguished
triangle whenever given a short exact sequence of complexes.
We have to show functoriality of this construction, see
Derived Categories, Definition \ref{derived-definition-delta-functor}.
This follows from Lemma \ref{lemma-functorial-cone} with a bit of
work. Compare with
Derived Categories, Lemma \ref{derived-lemma-derived-canonical-delta-functor}.
\end{proof}





\section{Linear categories}
\label{section-linear}

\noindent
Just the definitions.

\begin{definition}
\label{definition-linear-category}
Let $R$ be a ring. An {\it $R$-linear category $\mathcal{A}$} is a category
where every morphism set is given the structure of an $R$-module
and where for $x, y, z \in \Ob(\mathcal{A})$ composition law
$$
\Hom_\mathcal{A}(y, z) \times \Hom_\mathcal{A}(x, y)
\longrightarrow
\Hom_\mathcal{A}(x, z)
$$
is $R$-bilinear.
\end{definition}

\noindent
Thus composition determines an $R$-linear map
$$
\Hom_\mathcal{A}(y, z) \otimes_R \Hom_\mathcal{A}(x, y)
\longrightarrow
\Hom_\mathcal{A}(x, z)
$$
of $R$-modules. Note that we do not assume $R$-linear categories to be
additive.

\begin{definition}
\label{definition-functor-linear-categories}
Let $R$ be a ring. A {\it functor of $R$-linear categories}, or an
{\it $R$-linear} is a functor $F : \mathcal{A} \to \mathcal{B}$
where for all objects $x, y$ of $\mathcal{A}$ the map
$F : \Hom_\mathcal{A}(x, y) \to \Hom_\mathcal{A}(F(x), F(y))$
is a homomorphism of $R$-modules.
\end{definition}







\section{Graded categories}
\label{section-graded}

\noindent
Just some definitions.

\begin{definition}
\label{definition-graded-category}
Let $R$ be a ring. A {\it graded category $\mathcal{A}$
over $R$} is a category where every morphism set is given the structure
of a graded $R$-module and where for
$x, y, z \in \Ob(\mathcal{A})$ composition is $R$-bilinear and induces
a homomorphism
$$
\Hom_\mathcal{A}(y, z) \otimes_R \Hom_\mathcal{A}(x, y)
\longrightarrow
\Hom_\mathcal{A}(x, z)
$$
of graded $R$-modules (i.e., preserving degrees).
\end{definition}

\noindent
In this situation we denote $\Hom_\mathcal{A}^i(x, y)$ the degree $i$
part of the graded object $\Hom_\mathcal{A}(x, y)$, so that
$$
\Hom_\mathcal{A}(x, y) =
\bigoplus\nolimits_{i \in \mathbf{Z}} \Hom_\mathcal{A}^i(x, y)
$$
is the direct sum decomposition into graded parts.

\begin{definition}
\label{definition-functor-graded-categories}
Let $R$ be a ring. A {\it functor of graded categories over $R$}, or a
{\it graded functor}
is a functor $F : \mathcal{A} \to \mathcal{B}$ where for all objects
$x, y$ of $\mathcal{A}$ the map
$F : \Hom_\mathcal{A}(x, y) \to \Hom_\mathcal{A}(F(x), F(y))$
is a homomorphism of graded $R$-modules.
\end{definition}

\noindent
Given a graded category we are often interested in the
corresponding ``usual'' category of maps of degree $0$.
Here is a formal definition.

\begin{definition}
\label{definition-H0-of-graded-category}
Let $R$ be a ring. Let $\mathcal{A}$ be a differential graded category
over $R$. We let {\it $\mathcal{A}^0$} be the category with the
same objects as $\mathcal{A}$ and with
$$
\Hom_{\mathcal{A}^0}(x, y) = \Hom^0_\mathcal{A}(x, y)
$$
the degree $0$ graded piece of the graded module of morphisms of
$\mathcal{A}$.
\end{definition}

\begin{definition}
\label{definition-graded-direct-sum}
Let $R$ be a ring. Let $\mathcal{A}$ be a graded category over $R$.
A direct sum $(x, y, z, i, j, p, q)$ in $\mathcal{A}$ (notation as in
Homology, Remark \ref{homology-remark-direct-sum})
is a {\it graded direct sum} if $i, j, p, q$ are homogeneous
of degree $0$.
\end{definition}

\begin{example}[Graded category of graded objects]
\label{example-graded-category-graded-objects}
Let $\mathcal{B}$ be an additive category. Recall that we have defined
the category $\text{Gr}(\mathcal{B})$ of graded objects of $\mathcal{B}$ in
Homology, Definition \ref{homology-definition-graded}.
In this example, we will construct a graded category
$\text{Gr}^{gr}(\mathcal{B})$ over $R = \mathbf{Z}$
whose associated category $\text{Gr}^{gr}(\mathcal{B})^0$
recovers $\text{Gr}(\mathcal{B})$.
As objects of $\text{Comp}^{gr}(\mathcal{B})$
we take graded objects of $\mathcal{B}$. Then, given graded objects
$A = (A^i)$ and $B = (B^i)$ of $\mathcal{B}$ we set
$$
\Hom_{\text{Gr}^{gr}(\mathcal{B})}(A, B) =
\bigoplus\nolimits_{n \in \mathbf{Z}} \Hom^n(A, B)
$$
where the graded piece of degree $n$ is the abelian group of homogeneous
maps of degree $n$ from $A$ to $B$
defined by the rule
$$
\Hom^n(A, B) = \Hom_{\text{Gr}(\mathcal{A})}(A, B[n]) =
\Hom_{\text{Gr}(\mathcal{A})}(A[-n], B)
$$
see Homology, Equation (\ref{homology-equation-hom-into-shift}).
Explicitly we have
$$
\Hom^n(A, B) = \prod\nolimits_{p + q = n} \Hom_\mathcal{B}(A^{-q}, B^p)
$$
(observe reversal of indices and observe that we have a product here and
not a direct sum). In other words, a degree $n$ morphism $f$
from $A$ to $B$ can be seen as a system $f = (f_{p, q})$ where
$p, q \in \mathbf{Z}$, $p + q = n$ with
$f_{p, q} : A^{-q} \to B^p$ a morphism of $\mathcal{B}$.
Given graded objects $A$, $B$, $C$ of $\mathcal{B}$
composition of morphisms in $\text{Gr}^{gr}(\mathcal{B})$ is defined
via the maps
$$
\Hom^m(B, C) \times \Hom^n(A, B) \longrightarrow \Hom^{n + m}(A, C)
$$
by simple composition $(g, f) \mapsto g \circ f$ of homogeneous
maps of graded objects. In terms of components we have
$$
(g \circ f)_{p, r} = g_{p, q} \circ f_{-q, r}
$$
where $q$ is such that $p + q = m$ and $-q + r = n$.
\end{example}

\begin{example}[Graded category of graded modules]
\label{example-gm-gr-cat}
Let $A$ be a $\mathbf{Z}$-graded algebra over a ring $R$. We will construct
a graded category $\text{Mod}^{gr}_A$ over $R$ whose associated category
$(\text{Mod}^{gr}_A)^0$ is the category of graded $A$-modules. As objects
of $\text{Mod}^{gr}_A$ we take right graded $A$-modules (see
Section \ref{section-projectives-over-algebras}). Given graded
$A$-modules $L$ and $M$ we set
$$
\Hom_{\text{Mod}^{gr}_A}(L, M) =
\bigoplus\nolimits_{n \in \mathbf{Z}} \Hom^n(L, M)
$$
where $\Hom^n(L, M)$ is the set of right $A$-module maps $L \to M$ which
are homogeneous of degree $n$, i.e., $f(L^i) \subset M^{i + n}$ for
all $i \in \mathbf{Z}$. In terms of components, we have that
$$
\Hom^n(L, M) \subset \prod\nolimits_{p + q = n} \Hom_R(L^{-q}, M^p)
$$
(observe reversal of indices) is the subset consisting of those
$f = (f_{p, q})$ such that
$$
f_{p, q}(m a) = f_{p - i, q + i}(m)a
$$
for $a \in A^i$ and $m \in L^{-q - i}$. For graded $A$-modules
$K$, $L$, $M$ we define composition in $\text{Mod}^{gr}_A$ via
the maps
$$
\Hom^m(L, M) \times \Hom^n(K, L) \longrightarrow \Hom^{n + m}(K, M)
$$
by simple composition of right $A$-module maps: $(g, f) \mapsto g \circ f$.
\end{example}

\begin{remark}
\label{remark-graded-shift-functors}
Let $R$ be a ring. Let $\mathcal{D}$ be an $R$-linear category endowed with a
collection of $R$-linear functors $[n] : \mathcal{D} \to \mathcal{D}$,
$x \mapsto x[n]$ indexed by $n \in \mathbf{Z}$ such that
$[n] \circ [m] = [n + m]$ and $[0] = \text{id}_\mathcal{D}$ (equality as
functors). This allows us to construct a graded category $\mathcal{D}^{gr}$
over $R$ with the same objects of $\mathcal{D}$ setting
$$
\Hom_{\mathcal{D}^{gr}}(x, y) =
\bigoplus\nolimits_{n \in \mathbf{Z}} \Hom_\mathcal{D}(x, y[n])
$$
for $x, y$ in $\mathcal{D}$. Observe that $(\mathcal{D}^{gr})^0 = \mathcal{D}$
(see Definition \ref{definition-H0-of-graded-category}). Moreover, the graded
category $\mathcal{D}^{gr}$ inherits $R$-linear graded functors $[n]$
satisfying $[n] \circ [m] = [n + m]$ and $[0] = \text{id}_{\mathcal{D}^{gr}}$
with the property that
$$
\Hom_{\mathcal{D}^{gr}}(x, y[n]) = \Hom_{\mathcal{D}^{gr}}(x, y)[n]
$$
as graded $R$-modules compatible with composition of morphisms.

\medskip\noindent
Conversely, suppose given a graded category $\mathcal{A}$ over $R$ endowed
with a collection of $R$-linear graded functors $[n]$
satisfying $[n] \circ [m] = [n + m]$ and $[0] = \text{id}_\mathcal{A}$
which are moreover equipped with isomorphisms
$$
\Hom_\mathcal{A}(x, y[n]) = \Hom_\mathcal{A}(x, y)[n]
$$
as graded $R$-modules compatible with composition of morphisms. Then
the reader easily shows that $\mathcal{A} = (\mathcal{A}^0)^{gr}$.

\medskip\noindent
Here are two examples of the relationship
$\mathcal{D} \leftrightarrow \mathcal{A}$ we established above:
\begin{enumerate}
\item Let $\mathcal{B}$ be an additive category. If
$\mathcal{D} = \text{Gr}(\mathcal{B})$, then
$\mathcal{A} = \text{Gr}^{gr}(\mathcal{B})$ as in
Example \ref{example-graded-category-graded-objects}.
\item If $A$ is a graded ring and $\mathcal{D} = \text{Mod}_A$
is the category of graded right $A$-modules, then
$\mathcal{A} = \text{Mod}^{gr}_A$, see Example \ref{example-gm-gr-cat}.
\end{enumerate}
\end{remark}






\section{Differential graded categories}
\label{section-dga-categories}

\noindent
Note that if $R$ is a ring, then $R$ is a differential graded algebra
over itself (with $R = R^0$ of course). In this case a differential
graded $R$-module is the same thing as a complex of $R$-modules.
In particular, given two differential graded $R$-modules $M$ and $N$
we denote $M \otimes_R N$ the differential graded $R$-module
corresponding to the total complex associated to the double
complex obtained by the tensor product of the complexes of $R$-modules
associated to $M$ and $N$.

\begin{definition}
\label{definition-dga-category}
Let $R$ be a ring. A {\it differential graded category $\mathcal{A}$
over $R$} is a category where every morphism set is given the structure
of a differential graded $R$-module and where for
$x, y, z \in \Ob(\mathcal{A})$ composition is $R$-bilinear and induces
a homomorphism
$$
\Hom_\mathcal{A}(y, z) \otimes_R \Hom_\mathcal{A}(x, y)
\longrightarrow
\Hom_\mathcal{A}(x, z)
$$
of differential graded $R$-modules.
\end{definition}

\noindent
The final condition of the definition signifies the following:
if $f \in \Hom_\mathcal{A}^n(x, y)$ and
$g \in \Hom_\mathcal{A}^m(y, z)$ are homogeneous
of degrees $n$ and $m$, then
$$
\text{d}(g \circ f) = \text{d}(g) \circ f + (-1)^mg \circ \text{d}(f)
$$
in $\Hom_\mathcal{A}^{n + m + 1}(x, z)$. This follows from the sign
rule for the differential on the total complex of a double complex, see
Homology, Definition \ref{homology-definition-associated-simple-complex}.

\begin{definition}
\label{definition-functor-dga-categories}
Let $R$ be a ring. A {\it functor of differential graded categories over $R$}
is a functor $F : \mathcal{A} \to \mathcal{B}$ where for all objects
$x, y$ of $\mathcal{A}$ the map
$F : \Hom_\mathcal{A}(x, y) \to \Hom_\mathcal{A}(F(x), F(y))$
is a homomorphism of differential graded $R$-modules.
\end{definition}

\noindent
Given a diffferential graded category we are often interested in the
corresponding categories of complexes and homotopy category.
Here is a formal definition.

\begin{definition}
\label{definition-homotopy-category-of-dga-category}
Let $R$ be a ring. Let $\mathcal{A}$ be a differential graded category
over $R$. Then we let
\begin{enumerate}
\item the {\it category of complexes of $\mathcal{A}$}\footnote{This may
be nonstandard terminology.} be the category
$\text{Comp}(\mathcal{A})$ whose objects are the same as the objects
of $\mathcal{A}$ and with
$$
\Hom_{\text{Comp}(\mathcal{A})}(x, y) =
\text{Ker}(d : \Hom_\mathcal{A}(x, y) \to \Hom_\mathcal{A}(x, y))
$$
\item the {\it homotopy category of $\mathcal{A}$} be the category
$K(\mathcal{A})$ whose objects are the same as the objects
of $\mathcal{A}$ and with
$$
\Hom_{\text{Comp}(\mathcal{A})}(x, y) = H^0(\Hom_\mathcal{A}(x, y))
$$
\end{enumerate}
\end{definition}

\noindent
Our use of the symbol $K(\mathcal{A})$ is nonstandard, but at least
is compatible with the use of $K(-)$ in other chapters of the Stacks project.

\begin{definition}
\label{definition-dg-direct-sum}
Let $R$ be a ring. Let $\mathcal{A}$ be a differential graded category over
$R$. A direct sum $(x, y, z, i, j, p, q)$ in $\mathcal{A}$ (notation as in
Homology, Remark \ref{homology-remark-direct-sum})
is a {\it differential graded direct sum} if $i, j, p, q$ are homogeneous
of degree $0$ and closed, i.e., $\text{d}(i) = 0$, etc.
\end{definition}

\begin{lemma}
\label{lemma-functorial}
Let $R$ be a ring. A functor
$F : \mathcal{A} \to \mathcal{B}$
of differential graded categories over $R$
induces functors $\text{Comp}(\mathcal{A}) \to \text{Comp}(\mathcal{B})$
and $K(\mathcal{A}) \to K(\mathcal{B})$.
\end{lemma}

\begin{proof}
Omitted.
\end{proof}

\begin{example}[Differential graded category of complexes]
\label{example-category-complexes}
Let $\mathcal{B}$ be an additive category. We will construct
a differential graded category $\text{Comp}^{dg}(\mathcal{B})$
over $R = \mathbf{Z}$ whose associated category of complexes
is $\text{Comp}(\mathcal{B})$ and whose associated homotopy
category is $K(\mathcal{B})$. As objects of $\text{Comp}^{dg}(\mathcal{B})$
we take complexes of $\mathcal{B}$. Given complexes
$A^\bullet$ and $B^\bullet$ of $\mathcal{B}$, we sometimes also
denote $A^\bullet$ and $B^\bullet$ the corresponding graded objects
of $\mathcal{B}$ (i.e., forget about the differential).
Using this abuse of notation, we set
$$
\Hom_{\text{Comp}^{dg}(\mathcal{B})}(A^\bullet, B^\bullet) =
\Hom_{\text{Gr}^{gr}(\mathcal{B})}(A^\bullet, B^\bullet)
$$
as a graded $\mathbf{Z}$-module where the right hand side is defined
in Example \ref{example-graded-category-graded-objects}.
In other words, the $n$th graded piece is
the abelian group of homogeneous morphism of degree $n$ of graded objects
$$
\Hom^n(A^\bullet, B^\bullet) =
\Hom_{\text{Gr}(\mathcal{B})}(A^\bullet, B^\bullet[n]) =
\prod\nolimits_{p + q = n} \Hom_\mathcal{B}(A^{-q}, B^p)
$$
(observe reversal of indices and observe we have a direct product
and not a direct sum). For an element
$f \in \Hom^n(A^\bullet, B^\bullet)$ of degree $n$ we set
$$
\text{d}(f) = \text{d}_B \circ f - (-1)^n f \circ \text{d}_A
$$
To make sense of this we think of $\text{d}_B$ and $\text{d}_A$
as maps of graded objects of $\mathcal{B}$ homogeneous of degree $1$
and we use composition in the category $\text{Gr}^{gr}(\mathcal{B})$
on the right hand side. In terms of components, if $f = (f_{p, q})$ with
$f_{p, q} : A^{-q} \to B^p$ we have
\begin{equation}
\label{equation-differential-hom-complex}
\text{d}(f_{p, q}) =
\text{d}_B \circ f_{p, q} + (-1)^{p + q + 1} f_{p, q} \circ \text{d}_A 
\end{equation}
Note that the first term of this expression is in
$\Hom_\mathcal{B}(A^{-q}, B^{p + 1})$ and the second term is in
$\Hom_\mathcal{B}(A^{-q - 1}, B^p)$.
In other words, given $p + q = n + 1$ we have
$$
\text{d}(f)_{p, q} =
\text{d}_B \circ f_{p - 1, q} - (-1)^n f_{p, q - 1} \circ \text{d}_A
$$
with obvious notation. The reader checks\footnote{What may be useful
here is to think of the double complex $H^{\bullet, \bullet}$ with terms
$H^{p, q} = \Hom_\mathcal{B}(A^{-q}, B^p)$ and differentials
$d_1$ of degree $(1, 0)$ given by $\text{d}_B$ and $d_2$ of degree
$(0, 1)$ given by the contragredient of $d_A$. Up to sign and up to
replacing the direct sum by a direct product, the differential graded
$\mathbf{Z}$-module
$\Hom_{\text{Comp}^{dg}(\mathcal{B})}(A^\bullet, B^\bullet)$
is the total complex associated to $H^{\bullet, \bullet}$, see
Homology, Definition \ref{homology-definition-associated-simple-complex}.
To get the sign correct, change $d_2^{p, q} : H^{p, q} \to H^{p, q + 1}$ by
$(-1)^{q + 1}$ (after this change we still have a double complex).} that
\begin{enumerate}
\item $\text{d}$ has square zero,
\item an element $f$ in $\Hom^n(A^\bullet, B^\bullet)$
has $\text{d}(f) = 0$ if and only if the morphism
$f : A^\bullet \to B^\bullet[n]$ of graded objects of $\mathcal{B}$
is actually a map of complexes,
\item in particular, the category of complexes of
$\text{Comp}^{dg}(\mathcal{B})$ is equal to $\text{Comp}(\mathcal{B})$,
\item the morphism of complexes defined by $f$ as in (2)
is homotopy equivalent to zero if and only if $f = \text{d}(g)$
for some $g \in \Hom^{n - 1}(A^\bullet, B^\bullet)$.
\item in particular, we obtain a canonical isomorphism
$$
\Hom_{K(\mathcal{B})}(A^\bullet, B^\bullet)
\longrightarrow
H^0(\Hom_{\text{Comp}^{dg}(\mathcal{B})}(A^\bullet, B^\bullet))
$$
and the homotopy category of $\text{Comp}^{dg}(\mathcal{B})$ is equal to
$K(\mathcal{B})$.
\end{enumerate}
Given complexes $A^\bullet$, $B^\bullet$, $C^\bullet$ we define
composition
$$
\Hom^m(B^\bullet, C^\bullet) \times \Hom^n(A^\bullet, B^\bullet)
\longrightarrow
\Hom^{n + m}(A^\bullet, C^\bullet)
$$
by composition $(g, f) \mapsto g \circ f$ in the graded category
$\text{Gr}^{gr}(\mathcal{B})$, see
Example \ref{example-graded-category-graded-objects}.
This defines a map of differential graded modules as in
Definition \ref{definition-dga-category}
because
\begin{align*}
\text{d}(g \circ f) & =
\text{d}_C \circ g \circ f - (-1)^{n + m} g \circ f \circ \text{d}_A \\
& =
\left(\text{d}_C \circ g - (-1)^m g \circ \text{d}_B\right) \circ f +
(-1)^m g \circ \left(\text{d}_B \circ f - (-1)^n f \circ \text{d}_A\right) \\
& =
\text{d}(g) \circ f + (-1)^m g \circ \text{d}(f)
\end{align*}
as desired.
\end{example}

\begin{lemma}
\label{lemma-additive-functor-induces-dga-functor}
Let $F : \mathcal{B} \to \mathcal{B}'$ be an additive functor between
additive categories. Then $F$ induces a functor of differential
graded categories
$$
F : \text{Comp}^{dg}(\mathcal{B}) \to \text{Comp}^{dg}(\mathcal{B}')
$$
of Example \ref{example-category-complexes}
inducing the usual functors on the category of complexes and the
homotopy categories.
\end{lemma}

\begin{proof}
Omitted.
\end{proof}

\begin{example}[Differential graded category of differential graded modules]
\label{example-dgm-dg-cat}
Let $(A, \text{d})$ be a differential graded algebra over a ring $R$. We will
construct a differential graded category $\text{Mod}^{dg}_{(A, \text{d})}$
over $R$ whose category of complexes is $\text{Mod}_{(A, \text{d})}$ and
whose homotopy category is $K(\text{Mod}_{(A, \text{d})})$.
As objects of $\text{Mod}^{dg}_{(A, \text{d})}$
we take the differential graded $A$-modules. Given differential
graded $A$-modules $L$ and $M$ we set
$$
\Hom_{\text{Mod}^{dg}_{(A, \text{d})}}(L, M) =
\Hom_{\text{Mod}^{gr}_A}(L, M) = \bigoplus \Hom^n(L, M)
$$
as a graded $R$-module where the right hand side is defined as in
Example \ref{example-gm-gr-cat}. In other words, the $n$th graded piece
$\Hom^n(L, M)$ is the $R$-module of right $A$-module maps homogeneous
of degree $n$. For an element $f \in \Hom^n(L, M)$ we set
$$
\text{d}(f) = \text{d}_M \circ f - (-1)^n f \circ \text{d}_L
$$
To make sense of this we think of $\text{d}_M$ and $\text{d}_L$
as graded $R$-module maps and we use composition of graded
$R$-module maps. It is clear that $\text{d}(f)$ is homogeneous of
degree $n + 1$ as a graded $R$-module map, and it is linear
because
\begin{align*}
\text{d}(f)(xa)
& =
\text{d}_M(f(x) a) - (-1)^n f (\text{d}_L(xa)) \\
& =
\text{d}_M(f(x)) a + (-1)^{\deg(x) + n} f(x) \text{d}(a) 
- (-1)^n f(\text{d}_L(x)) a - (-1)^{n + \deg(x)} f(x) \text{d}(a) \\
& = \text{d}(f)(x) a
\end{align*}
as desired (observe that this calculation would not work without the
sign in the definition of our differential on $\Hom$). Similar formulae
to those of Example \ref{example-category-complexes} hold for the
differential of $f$ in terms of components.
The reader checks (in the same way as in
Example \ref{example-category-complexes}) that
\begin{enumerate}
\item $\text{d}$ has square zero,
\item an element $f$ in $\Hom^n(L, M)$ has $\text{d}(f) = 0$ if and only if
$f : L \to M[n]$ is a homomorphism of differential graded $A$-modules,
\item in particular, the category of complexes of
$\text{Mod}^{dg}_{(A, \text{d})}$ is $\text{Mod}_{(A, \text{d})}$,
\item the homomorphism defined by $f$ as in (2) is homotopy equivalent
to zero if and only if $f = \text{d}(g)$ for some
$g \in \Hom^{n - 1}(L, M)$.
\item in particular, we obtain a canonical isomorphism
$$
\Hom_{K(\text{Mod}_{(A, \text{d})})}(L, M)
\longrightarrow
H^0(\Hom_{\text{Mod}^{dg}_{(A, \text{d})}}(L, M))
$$
and the homotopy category of $\text{Mod}^{dg}_{(A, \text{d})}$ is
$K(\text{Mod}_{(A, \text{d})})$.
\end{enumerate}
Given differential graded $A$-modules $K$, $L$, $M$ we define
composition
$$
\Hom^m(L, M) \times \Hom^n(K, L) \longrightarrow \Hom^{n + m}(K, M)
$$
by composition of homogeneous right $A$-module maps $(g, f) \mapsto g \circ f$.
This defines a map of differential graded modules as in
Definition \ref{definition-dga-category}
because
\begin{align*}
\text{d}(g \circ f) & =
\text{d}_M \circ g \circ f - (-1)^{n + m} g \circ f \circ \text{d}_K \\
& =
\left(\text{d}_M \circ g - (-1)^m g \circ \text{d}_L\right) \circ f +
(-1)^m g \circ \left(\text{d}_L \circ f - (-1)^n f \circ \text{d}_K\right) \\
& =
\text{d}(g) \circ f + (-1)^m g \circ \text{d}(f)
\end{align*}
as desired.
\end{example}

\begin{lemma}
\label{lemma-homomorphism-induces-dga-functor}
Let $\varphi : (A, \text{d}) \to (E, \text{d})$ be a homomorphism of
differential graded algebras. Then $\varphi$ induces a functor of differential
graded categories
$$
F :
\text{Mod}^{dg}_{(E, \text{d})}
\longrightarrow
\text{Mod}^{dg}_{(A, \text{d})}
$$
of Example \ref{example-dgm-dg-cat} inducing obvious restriction functors
on the categories of differential graded modules and homotopy categories.
\end{lemma}

\begin{proof}
Omitted.
\end{proof}

\begin{lemma}
\label{lemma-construction}
Let $R$ be a ring. Let $\mathcal{A}$ be a differential graded category
over $R$. Let $x$ be an object of $\mathcal{A}$. Let
$$
(E, \text{d}) = \Hom_\mathcal{A}(x, x)
$$
be the differential graded $R$-algebra of endomorphisms of $x$.
We obtain a functor
$$
\mathcal{A} \longrightarrow \text{Mod}^{dg}_{(E, \text{d})},\quad
y \longmapsto \Hom_\mathcal{A}(x, y)
$$
of differential graded categories by letting $E$ act on
$\Hom_\mathcal{A}(x, y)$ via composition in $\mathcal{A}$.
This functor induces functors
$$
\text{Comp}(\mathcal{A}) \to \text{Mod}_{(A, \text{d})}
\quad\text{and}\quad
K(\mathcal{A}) \to K(\text{Mod}_{(A, \text{d})})
$$
by an application of Lemma \ref{lemma-functorial}.
\end{lemma}

\begin{proof}
This lemma proves itself.
\end{proof}









\section{Obtaining triangulated categories}
\label{section-review}

\noindent
In this section we discuss the most general setup to which the arguments
proving Derived Categories, Proposition
\ref{proposition-homotopy-category-triangulated} and
Proposition \ref{proposition-homotopy-category-triangulated} apply.

\medskip\noindent
Let $\mathcal{A}$ be a differential graded category. To make our argument
work, we impose some axioms on $\mathcal{A}$:
\begin{enumerate}
\item[(A)] $\mathcal{A}$ has a zero object and differential
graded direct sums of two objects
(as in Definition \ref{definition-dg-direct-sum}).
\item[(B)] there are functors $[n] : \mathcal{A} \longrightarrow \mathcal{A}$
of differential graded categories such that
$[0] = \text{id}_\mathcal{A}$ and $[n + m] = [n] \circ [m]$
and given isomorphisms
$$
\Hom_\mathcal{A}(x, y[n]) = \Hom_\mathcal{A}(x, y)[n]
$$
of differential graded $R$-modules compatible with composition.
\end{enumerate}

\noindent
Given our differential graded category $\mathcal{A}$ we say
\begin{enumerate}
\item a morphism $x \to y$ of $\text{Comp}(\mathcal{A})$
is an {\it admissible monomorphism} if there exists a degree $0$ morphism
$y \to x$ of $\mathcal{A}$ which is left inverse to $x \to y$,
\item a morphism $y \to z$ of $\text{Comp}(\mathcal{A})$
is an {\it admissible epimorphism} if there exists a degree $0$ morphism
$z \to y$ of $\mathcal{A}$ which is right inverse to $y \to z$,
\item a sequence $x \to y \to z$ of morphisms of $\text{Comp}(\mathcal{A})$
is an {\it admissible short exact sequence} if there exists
an isomorphism $y \cong x \oplus z$ in the underlying graded category
such that $x \to z$ and $y \to z$ are (co)projections.
\end{enumerate}
The next lemma tells us an admissible short exact sequence gives a
triangle, provided we have axioms (A) and (B).

\begin{lemma}
\label{lemma-get-triangle}
Let $\mathcal{A}$ be a differential graded category satisfying
axioms (A) and (B). Given an admissible short exact sequence
$x \to y \to z$ we obtain (see proof) a triangle
$$
x \to y \to z \to x[1]
$$
in $\text{Comp}(\mathcal{A})$ with the property that any two compositions
in $z[-1] \to x \to y \to z \to x[1]$ are zero in $K(\mathcal{A})$.
\end{lemma}

\begin{proof}
Choose a diagram
$$
\xymatrix{
x \ar[rr]_1 \ar[rd]_a & & x \\
& y \ar[ru]_\pi \ar[rd]^b & \\
z \ar[rr]^1 \ar[ru]^s & & z
}
$$
giving the isomorphism of graded objects $y \cong x \oplus z$ as in the
defintion of an admissible short exact sequence. Here are some equations
that hold in this situation
\begin{enumerate}
\item $1 = \pi a$ and hence $\text{d}(\pi) a = 0$,
\item $1 = b s$ and hence $b \text{d}(s) = 0$,
\item $1 = a \pi + s b$ and hence $a \text{d}(\pi) + \text{d}(s) b = 0$,
\item $\pi s = 0$ and hence $\text{d}(\pi)s + \pi \text{d}(s) = 0$,
\item $\text{d}(s) = a \pi \text{d}(s)$ because
$\text{d}(s) = (a \pi + s b)\text{d}(s)$ and $b\text{d}(s) = 0$,
\item $\text{d}(\pi) = \text{d}(\pi) s b$ because
$\text{d}(\pi) = \text{d}(\pi)(a \pi + s b)$ and $\text{d}(\pi)a = 0$,
\item $\text{d}(\pi \text{d}(s)) = 0$ because if we postcompose it
with the monomorphism $a$ we get
$\text{d}(a\pi \text{d}(s)) = \text{d}(\text{d}(s)) = 0$, and
\item $\text{d}(\text{d}(\pi)s) = 0$ as by (4) it is the negative
of $\text{d}(\pi\text{d}(s))$ which is $0$ by (7).
\end{enumerate}
We've used repeatedly that $\text{d}(a) = 0$, $\text{d}(b) = 0$,
and that $\text{d}(1) = 0$. By (7) we see that
$$
\delta = \pi \text{d}(s) = - \text{d}(\pi) s : z \to x[1]
$$
is a morphism in $\text{Comp}(\mathcal{A})$. By (5) we see that
the composition $a \delta = a \pi \text{d}(s) = \text{d}(s)$
is homotopic to zero. By (6) we see that the composition
$\delta b = - \text{d}(\pi)sb = \text{d}(-\pi)$ is homotopic to zero.
\end{proof}

\noindent
Next, we need an axiom concerning the existence of cones
\begin{enumerate}
\item[(C)] given an arrow $f : x \to y$ of degree $0$ with
$\text{d}(f) = 0$ there exists an admissible short exact sequence
$y \to c(f) \to x[1]$ in $\text{Comp}(\mathcal{A})$ such that the map
$x[1] \to y[1]$ of Lemma \ref{lemma-get-triangle} is equal to $f[1]$.
\end{enumerate}
Cones are functorial in a weak sense.

\begin{lemma}
\label{lemma-cone}
Let $\mathcal{A}$ be a differential graded category with axioms (A), (B), and
(C). Suppose that
$$
\xymatrix{
x_1 \ar[r]_{f_1} \ar[d]_a & y_1 \ar[d]^b \\
x_2 \ar[r]^{f_2} & y_2
}
$$
is a diagram of $\text{Comp}(\mathcal{A})$ commutative up to homotopy.
Then there exists a morphism $c : c(f_1) \to c(f_2)$ which gives rise to
a morphism of triangles
$$
(a, b, c) : (x_1, y_1, c(f_1)) \to (x_1, y_1, c(f_1))
$$
in $K(\mathcal{A})$.
\end{lemma}

\begin{proof}
The assumption means there exists a morphism $h : x_1 \to y_2$ of degree
$-1$ such that $\text{d}(h) = b f_1 - f_2 a$. Choose isomorphisms
$c(f_i) = y_i \oplus x_i[1]$ of graded objects compatible with the
morphisms $y_i \to c(f_i) \to x_i[1]$. Let's denote
$a_i : y_i \to c(f_i)$, $b_i : c(f_i) \to x_i[1]$, $s_i : x_i[1] \to c(f_i)$,
and $\pi_i : c(f_i) \to y_i$ the given morphisms. Recall that
$x_i[1] \to y_i[1]$ is given by $\pi_i \text{d}(s_i)$. By axiom (C)
this means that
$$
f_i = \pi_i \text{d}(s_i) = - \text{d}(\pi_i) s_i
$$
(we identify $\Hom(x_i, y_i)$ with $\Hom(x_i[1], y_i[1])$
using the shift functor $[1]$).
Set $c = a_2 b \pi_1 + s_2 a b_1 + a_2hb$. Then, using the
equalities found in the proof of Lemma \ref{lemma-get-triangle}
we obtain
\begin{align*}
\text{d}(c)
& =
a_2 b \text{d}(\pi_1) + \text{d}(s_2) a b_1 + a_2 \text{d}(h) b_1 \\
& =
- a_2 b f_1 b_1 + a_2 f_2 a b_1 + a_2 (b f_1 - f_2 a) b_1 \\
& = 0
\end{align*}
(where we have used in particular that
$\text{d}(\pi_1) = \text{d}(\pi_1) s_1 b_1 = f_1 b_1$ and
$\text{d}(s_2) = a_2 \pi_2 \text{d}(s_2) = a_2 f_2$).
Thus $c$ is a degree $0$ morphism $c : c(f_1) \to c(f_2)$ of $\mathcal{A}$
compatible with the given morphisms $y_i \to c(f_i) \to x_i[1]$.
\end{proof}

\noindent
Let $R$ be a ring. Let $\mathcal{A}$ be a differential graded category
over $R$ with axioms (A), (B), and (C). We say that a triangle
$(x, y, z, f, g, h)$ in $K(\mathcal{A})$ is a
{\it distinguished triangle} if there exists an admissible
short exact sequence $x' \to y' \to z'$ such that
$(x, y, z, f, g, h)$ is isomorphic as a triangle in $K(\mathcal{A})$
to the triangle $(x', y', z', x' \to y', y' \to z', \delta)$
constructed in Lemma \ref{lemma-get-triangle}. We claim that
$$
\fbox{$K(\mathcal{A})$ is a triangulated category}
$$
To prove this in the same way as in the examples mentioned at
the beginning of this section, we would prove the analogues of
\begin{itemize}
\item Lemma \ref{lemma-make-commute-map},
\item Lemma \ref{lemma-make-injective}, 
\item Lemma \ref{lemma-sequence-maps-split},
\item Lemma \ref{lemma-nilpotent},
\item Lemma \ref{lemma-triangle-independent-splittings},
\item Lemma \ref{lemma-rotate-cone},
\item Lemma \ref{lemma-rotate-triangle},
\item Lemma \ref{lemma-third-isomorphism},
\item Lemma \ref{lemma-the-same-up-to-isomorphisms},
\item Lemma \ref{lemma-homotopy-category-pre-triangulated},
\item Lemma \ref{lemma-two-split-injections},
\end{itemize}
and finally deduce the result itself (as in the proof of
Proposition \ref{proposition-homotopy-category-triangulated}).
If we ever need this result, we will add these lemmas here
and prove every one of them (most of them are formal from
the axioms and the two lemmas we've proved so far).

\medskip\noindent
This result, although not as general as one might think, applies to a
number of natural generalizations of the cases covered so far in the
Stacks project. Here are some examples:
\begin{enumerate}
\item Let $(X, \mathcal{O}_X)$ be a ringed space. Let $(A, d)$ be a
sheaf of differential graded $\mathcal{O}_X$-algebras. Let
$\mathcal{A}$ be the differential graded category of differential
graded $A$-modules. Then $K(\mathcal{A})$ is a triangulated category.
\item Let $(\mathcal{C}, \mathcal{O})$ be a ringed site. Let $(A, d)$ be a
sheaf of differential graded $\mathcal{O}$-algebras. Let
$\mathcal{A}$ be the differential graded category of differential
graded $A$-modules. Then $K(\mathcal{A})$ is a triangulated category.
\item Add more here.
\end{enumerate}









\section{Derived Hom}
\label{section-restriction}

\noindent
Let $R$ be a ring. Let $(B, \text{d})$ be a differential graded algebra
over $R$. Denote $\mathcal{B} = \text{Mod}^{dg}_{(B, \text{d})}$
the differential graded category of differential graded $B$-modules, see
Example \ref{example-dgm-dg-cat}. Let $N$ be a differential graded $B$-module.
Then the endomorphisms of $N$ in $\mathcal{B}$
$$
\Hom_\mathcal{B}(N, N)
$$
is differential graded algebra over $R$. Now let $N'$ be a second differential
graded $B$-module. Then
$$
\Hom_\mathcal{B}(N, N')
$$
becomes a right differential graded $\Hom_\mathcal{B}(N, N)$-module by the
composition
$$
\Hom_\mathcal{B}(N, N') \times \Hom_\mathcal{B}(N, N)
\longrightarrow
\Hom_\mathcal{B}(N, N')
$$
We need one more piece of data, in order to be able to formulate the
results in the correct generality. Namely, let $(A, \text{d})$ be a
differential graded $R$-algebra and let $A \to \Hom_\mathcal{B}(N, N)$
be a homomorphism of differential graded $R$-algebras\footnote{A very
interesting case is when $A = \Hom_\mathcal{B}(N, N)$.}. Using this
homomorphism we obtain a functor
\begin{equation}
\label{equation-restriction}
\text{Mod}_{(B, \text{d})}
\longrightarrow
\text{Mod}_{(A, \text{d})},\quad
N' \longmapsto \Hom_\mathcal{B}(N, N')
\end{equation}
where $A$ acts on $\Hom_\mathcal{B}(N, N')$ via the given homomorphism
and the action of $\Hom_\mathcal{B}(N, N)$ given above.

\begin{lemma}
\label{lemma-restriction-homotopy}
The functor (\ref{equation-restriction}) defines an exact functor
of triangulated categories
$K(\text{Mod}_{(B, \text{d})}) \to K(\text{Mod}_{(A, \text{d})})$.
\end{lemma}

\begin{proof}
Combining
Lemmas \ref{lemma-homomorphism-induces-dga-functor},
\ref{lemma-construction}, and
\ref{lemma-functorial}
we obtain the functor of the statement.
We have to show that (\ref{equation-restriction}) transforms distinguished
triangles into distinguished triangles. To see this
suppose that $0 \to N_1 \to N_2 \to N_3 \to 0$ is an admissible short
exact sequence of differential graded $B$-modules. Let $s : N_3 \to N_2$
be a graded $B$-module homomorphism which is left inverse to $N_2 \to N_3$.
Then $s$ defines a graded $A$-module homomorphism
$\Hom_\mathcal{B}(N, N_3) \to \Hom_\mathcal{B}(N, N_2)$
which is left inverse to
$\Hom_\mathcal{B}(N, N_2) \to \Hom_\mathcal{B}(N, N_3)$.
This finishes the proof.
\end{proof}

\noindent
At this point we can consider the diagram
$$
\xymatrix{
K(\text{Mod}_{(B, \text{d})}) \ar[d] \ar[rr]_{\Hom_\mathcal{B}(N, -)}
\ar[rrd]_F & &
K(\text{Mod}_{(A, \text{d})}) \ar[d] \\
D(B, \text{d}) \ar@{..>}[rr] & &
D(A, \text{d})
}
$$
We would like to construct a dotted arrow as the
{\it right derived functor} of the composition $F$.
({\it Warning}: the diagram will not commute.)
Namely, in the general setting of
Derived Categories, Section \ref{derived-section-derived-functors}
we want to compute the
right derived functor of $F$ with respect to the multplicative system of
quasi-isomorphisms in $K(\text{Mod}_{(A, \text{d})})$.

\begin{lemma}
\label{lemma-derived-restriction}
In the situation above, the right derived functor of $F$ exists.
We denote it $R\Hom(N, -) : D(B, \text{d}) \to D(A, \text{d})$.
\end{lemma}

\begin{proof}
We will use
Derived Categories, Lemma \ref{derived-lemma-find-existence-computes}
to prove this. As our collection $\mathcal{I}$
of objects we will use the objects with property (I).
Property (1) was shown in Lemma \ref{lemma-right-resolution}.
Property (2) holds because if $s : I \to I'$ is a quasi-isomorphism
of modules with property (I), then $s$ is a homotopy equivalence
by Lemma \ref{lemma-hom-derived}.
\end{proof}





\section{Variant of derived Hom}
\label{section-variant}

\noindent
Let $\mathcal{A}$ be an abelian category. Consider the differential graded
category $\text{Comp}^{dg}(\mathcal{A})$ of complexes of $\mathcal{A}$, see
Example \ref{example-category-complexes}.
Let $K^\bullet$ be a complex of $\mathcal{A}$. Set
$$
(E, \text{d}) = \Hom_{\text{Comp}^{dg}(\mathcal{A})}(K^\bullet, K^\bullet)
$$
and consider the functor of differential graded categories
$$
\text{Comp}^{dg}(\mathcal{A}) \longrightarrow \text{Mod}^{dg}_{(E, \text{d})},
\quad
X^\bullet
\longmapsto
\Hom_{\text{Comp}^{dg}(\mathcal{A})}(K^\bullet, X^\bullet)
$$
of Lemma \ref{lemma-construction}.

\begin{lemma}
\label{lemma-existence-of-derived}
In the situation above. If the right derived functor $R\Hom(K^\bullet, -)$
of $\Hom(K^\bullet, -) : K(\mathcal{A}) \to D(\textit{Ab})$
is everywhere defined on $D(\mathcal{A})$, then we obtain a canonical exact
functor
$$
R\Hom(K^\bullet, -) : D(\mathcal{A}) \longrightarrow D(E, \text{d})
$$
of triangulated categories which reduces to the usual one  on taking
associated complexes of abelian groups.
\end{lemma}

\begin{proof}
Note that we have an associated functor
$K(\mathcal{A}) \to K(\text{Mod}_{(E, \text{d})})$ by
Lemma \ref{lemma-construction}.
We claim this functor is an exact functor of triangulated categories.
Namely, let $f : A^\bullet \to B^\bullet$ be a map of complexes
of $\mathcal{A}$. Then a computation shows that
$$
\Hom_{\text{Comp}^{dg}(\mathcal{A})}(K^\bullet, C(f)^\bullet)
=
C\left(
\Hom_{\text{Comp}^{dg}(\mathcal{A})}(K^\bullet, A^\bullet) \to
\Hom_{\text{Comp}^{dg}(\mathcal{A})}(K^\bullet, B^\bullet)
\right)
$$
where the right hand side is the cone in $\text{Mod}_{(E, \text{d})}$
defined earlier in this chapter.
This shows that our functor is compatible with cones, hence with
distinguished triangles. Let $X^\bullet$ be an object of $K(\mathcal{A})$.
Consider the category of quasi-isomorphisms $s : X^\bullet \to Y^\bullet$.
We are given that the functor
$(s : X^\bullet \to Y^\bullet) \mapsto \Hom_\mathcal{A}(K^\bullet, Y^\bullet)$
is essentially constant when viewed in $D(\textit{Ab})$.
But since the forgetful functor $D(E, \text{d}) \to D(\textit{Ab})$
is compatible with taking cohomology, the same thing is true in
$D(E, \text{d})$. This proves the lemma.
\end{proof}

\noindent
{\bf Warning:} Although the lemma holds as stated and may be useful
as stated, the differential algebra $E$ isn't the ``correct'' one unless
$H^n(E) = \text{Ext}^n_{D(\mathcal{A})}(K^\bullet, K^\bullet)$
for all $n \in \mathbf{Z}$.




\section{Tensor product}
\label{section-tensor-product}

\noindent
This section should be moved somewhere else. Let $R$ be a ring.
Let $A$ be an $R$-algebra (see Section \ref{section-conventions}).
Given a right $A$-module $M$ and a left $A$-module $N$ there is
a {\it tensor product}
$$
M \otimes_A N
$$
This tensor product is a module over $R$. In fact, it is the receptacle
of the universal $A$-bilinear map $M \times N \to M \otimes_A N$,
$(m, n) \mapsto m \otimes n$.

\medskip\noindent
We list some properties of the tensor product
\begin{enumerate}
\item In each variable the tensor product is right exact, in fact commutes
with direct sums and arbitrary colimits.
\item If $A$, $M$, $N$ are graded and the module structures are compatible
with gradings then $M \otimes_A N$ is graded as well.
Then $n$th graded piece $(M \otimes_A N)^n$ of $M \otimes_A N$
is the quotient of $\bigoplus_{p + q = n} M^p \otimes_{A^0} N^q$
by the submodule generated by $m \otimes an - ma \otimes n$ where
$m \in M^p$, $n \in N^q$, and $a \in A^{n - p - q}$.
\item If $(A, \text{d})$ is a differential graded algebra, and
$M$ and $N$ are (left and right) differential graded $A$-modules, then
$M \otimes_A N$ is a differential graded $R$-module with differential
$$
\text{d}(m \otimes n) = \text{d}(m) \otimes n + (-1)^im \otimes \text{d}(n)
$$
for $m \in M^i$ and $n \in N$.
\item If $N$ is a $(A, B)$-bimodule then $M \otimes_A N$ is a right
$B$-module.
\item If $A$ and $B$ are graded algebras,
$M$ is a graded $A$-module, and $N$ is an $(A, B)$-bimodule
which comes with a grading such that it is both
a left graded $A$-module and a right graded
$B$-module, then $M \otimes_A N$ is a graded $B$-module.
\item If $(A, \text{d})$ and $(B, \text{d})$ are differential graded algebras,
$M$ is a differential graded $A$-module, and $N$ is an $(A, B)$-bimodule
which comes with a grading and a differential such that it is both
a left differential graded $A$-module and a right differential graded
$B$-module, then $M \otimes_A N$ is a differential graded $B$-module.
\end{enumerate}
We state the following as a lemma.

\begin{lemma}
\label{lemma-tensor}
Let $(A, \text{d})$ and $(B, \text{d})$ be differential graded algebras,
and let $N$ be an $(A, B)$-bimodule which comes with a grading and a
differential such that it is both a left differential graded $A$-module
and a right differential graded $B$-module. Then
$M \mapsto M \otimes_A N$ defines a functor
$$
- \otimes_A N :
\text{Mod}^{dg}_{(A, \text{d})}
\longrightarrow
\text{Mod}^{dg}_{(B, \text{d})}
$$
of differential graded categories. This functor induces functors
$$
\text{Mod}_{(A, \text{d})} \to \text{Mod}_{(B, \text{d})}
\quad\text{and}\quad
K(\text{Mod}_{(A, \text{d})}) \to K(\text{Mod}_{(B, \text{d})})
$$
by an application of Lemma \ref{lemma-functorial}.
\end{lemma}

\begin{proof}
This follows from the discussion above.
\end{proof}

\noindent
If $A$ is an algebra and $M$, $M'$ are right $A$-modules, then we
define
$$
\Hom_A(M, M') = \{f : M \to M'\mid f \text{ is }A\text{-linear}\}
$$
as usual. If $A$ is graded and $M$ and $M'$ are graded $A$-modules,
then we set
$$
\Hom_A(M, M') =
\bigoplus\nolimits_{n \in \mathbf{Z}}
\left\{
f : M \to M'
\middle|
\begin{matrix}
f \text{ is }A\text{-linear and}\\
f(M^m) \subset (M')^{m + n}\text{ for }m \in \mathbf{Z}
\end{matrix}
\right\}
$$
We apologize for using the same symbol to mean two different things.
These definitions have the following consequence.

\begin{lemma}
\label{lemma-tensor-hom-adjunction}
Let $A$ and $B$ be algebras. Let $M$ be a right $A$-module, $N$ an
$(A, B)$-bimodule, and $N'$ a right $B$-module. Then we have
$$
\Hom_B(M \otimes_A N, N') = \Hom_A(M, \Hom_B(N, N'))
$$
If $A$, $B$, $M$, $N$, $N'$ are compatibly graded, then the same thing
is true for the graded versions.
\end{lemma}

\begin{proof}
This follows by interpreting both sides as $A$-bilinear maps
$\psi : M \times N \to N'$ which are $B$-linear on the right.
\end{proof}





\section{Derived tensor product}
\label{section-base-change}

\noindent
This section is analogous to More on Algebra, Section
\ref{more-algebra-section-derived-base-change}.

\medskip\noindent
Let $R$ be a ring. Let $(B, \text{d})$ be a differential graded algebra
over $R$. Denote $\mathcal{B} = \text{Mod}^{dg}_{(B, \text{d})}$
the differential graded category of differential graded $B$-modules, see
Example \ref{example-dgm-dg-cat}. Let $N$ be a differential graded $B$-module.
Then the endomorphisms of $N$ in $\mathcal{B}$
$$
\Hom_\mathcal{B}(N, N)
$$
is differential graded algebra over $R$. The pairing
$$
\Hom_\mathcal{B}(N, N) \times N \longrightarrow N,\quad
(f, x) \longmapsto f(x)
$$
defines a left differential graded $\Hom_\mathcal{B}(N, N)$-module
structure on $N$. The multiplication with $f \in \Hom_\mathcal{B}(N, N)$
commutes with the right action of $B$ by our
construction of the category $\mathcal{B}$. Hence $N$ is an
$(\Hom_\mathcal{B}(N, N), B)$-bimodule.

\medskip\noindent
We need one more piece of data, in order to be able to formulate the
results in the correct generality. Namely, let $(A, \text{d})$ be a
differential graded $R$-algebra and let $A \to \Hom_\mathcal{B}(N, N)$
be a homomorphism of differential graded $R$-algebras\footnote{A very
interesting case is when $A = \Hom_\mathcal{B}(N, N)$.}. Using this
homomorphism and the remarks above we see that $N$ becomes an
$(A, B)$-bimodule. Thus we obtain a functor
\begin{equation}
\label{equation-bc}
\text{Mod}_{(A, \text{d})}
\longrightarrow
\text{Mod}_{(B, \text{d})},\quad
M \longmapsto M \otimes_A N
\end{equation}
see Section \ref{section-tensor-product}.

\begin{lemma}
\label{lemma-bc-homotopy}
The functor (\ref{equation-bc}) defines an exact functor
of triangulated categories
$K(\text{Mod}_{(A, \text{d})}) \to K(\text{Mod}_{(B, \text{d})})$.
\end{lemma}

\begin{proof}
The functor was constructed in Lemma \ref{lemma-tensor}.
We have to show that $- \otimes_A N$ transforms distinguished triangles
into distinguished triangles.
Suppose that $0 \to K \to L \to M \to 0$ is an admissible short
exact sequence of differential graded $A$-modules. Let $s : M \to L$ be
a graded $A$-module homomorphism which is left inverse to $L \to M$.
Then $s$ defines a graded $B$-module homomorphism
$M \otimes_A N \to L \otimes_A N$ which is left inverse to
$L \otimes_A N \to M \otimes_A N$.
\end{proof}

\noindent
At this point we can consider the diagram
$$
\xymatrix{
K(\text{Mod}_{(A, \text{d})}) \ar[d] \ar[rr]_{- \otimes_A N} \ar[rrd]_F & &
K(\text{Mod}_{(B, \text{d})}) \ar[d] \\
D(A, \text{d}) \ar@{..>}[rr] & &
D(B, \text{d})
}
$$
The dotted arrow that we will construct below will be the
{\it left derived functor} of the composition $F$.
({\it Warning}: the diagram will not commute.)
Namely, in the general setting of
Derived Categories, Section \ref{derived-section-derived-functors}
we want to compute the
left derived functor of $F$ with respect to the multplicative system of
quasi-isomorphisms in $K(\text{Mod}_{(A, \text{d})})$.

\begin{lemma}
\label{lemma-derived-bc}
In the situation above, the left derived functor of $F$ exists.
We denote it
$- \otimes_A^\mathbf{L} N : D(A, \text{d}) \to D(B, \text{d})$.
\end{lemma}

\begin{proof}
We will use
Derived Categories, Lemma \ref{derived-lemma-find-existence-computes}
to prove this. As our collection $\mathcal{P}$
of objects we will use the objects with property (P).
Property (1) was shown in Lemma \ref{lemma-resolve}.
Property (2) holds because if $s : P \to P'$ is a quasi-isomorphism
of modules with property (P), then $s$ is a homotopy equivalence
by Lemma \ref{lemma-hom-derived}.
\end{proof}

\begin{lemma}
\label{lemma-tensor-hom-adjoint}
Let $(A, \text{d})$ and $(B, \text{d})$ be differential graded algebras.
Let $N$ be an $(A, B)$-bimodule which comes with a grading and a differential
such that it is a differential graded module for both $A$ and $B$.
Then the functors
$$
- \otimes_A^\mathbf{L} N : D(A, \text{d}) \longrightarrow D(B, \text{d})
$$
of Lemma \ref{lemma-derived-bc} and
$$
R\Hom(N, -) : D(B, \text{d}) \longrightarrow D(A, \text{d})
$$
of Lemma \ref{lemma-derived-restriction} are adjoint.
\end{lemma}

\begin{proof}
The statement means that we have
$$
\Hom_{D(A, \text{d})}(M, R\Hom(N, N')) =
\Hom_{D(B, \text{d})}(M \otimes^\mathbf{L}_A N, N')
$$
bifunctorially in $M$ and $N'$. To see this we may assume that $M$
is a differential graded $A$-module with property (P) and that $N'$
is a differential graded $B$-module with property (I). The computation
of the derived functors given in the lemmas referenced in the statement
combined with Lemma \ref{lemma-hom-derived} translates the above into
$$
\Hom_{K(\text{Mod}_{(A, \text{d})})}(M, \Hom_\mathcal{B}(N, N')) =
\Hom_{K(\text{Mod}_{(B, \text{d})})}(M \otimes_A N, N')
$$
where $\mathcal{B} = \text{Mod}^{dg}_{(B, \text{d})}$. Thus it is certainly
sufficient to show that
$$
\Hom_\mathcal{A}(M, \Hom_\mathcal{B}(N, N')) =
\Hom_\mathcal{B}(M \otimes_A N, N')
$$
as differential graded $\mathbf{Z}$-modules where
$\mathcal{A} = \text{Mod}^{dg}_{(A, \text{d})}$.
This follows from the fact that the isomorphism
(Lemma \ref{lemma-tensor-hom-adjunction})
$$
\Hom_A(M, \Hom_B(N, N')) = \Hom_B(M \otimes_A N, N')
$$
of internal homs of graded modules respects the differentials.
\end{proof}




\section{Variant of derived tensor product}
\label{section-variant-base-change}

\noindent
Let $(\mathcal{C}, \mathcal{O})$ be a ringed site. Then we have the functors
$$
\text{Comp}(\mathcal{O}) \to K(\mathcal{O}) \to D(\mathcal{O})
$$
and as we've seen above we have differential graded enhancement
$\text{Comp}^{dg}(\mathcal{O})$. Namely, this is the differential
graded category of Example \ref{example-category-complexes} associated
to the abelian category $\textit{Mod}(\mathcal{O})$.
Let $K^\bullet$ be a complex of $\mathcal{O}$-modules in other
words, an object of $\text{Comp}^{dg}(\mathcal{O})$. Set
$$
(E, \text{d}) =
\Hom_{\text{Comp}^{dg}(\mathcal{O})}(K^\bullet, K^\bullet)
$$
This is a differential graded $\mathbf{Z}$-algebra. We claim there is
an analogue of the derived base change in this situation.

\begin{lemma}
\label{lemma-tensor-with-complex}
In the situation above there is a functor
$$
- \otimes_E K^\bullet :
\text{Mod}^{dg}_{(E, \text{d})}
\longrightarrow
\text{Comp}^{dg}(\mathcal{O})
$$
of differential graded categories. This functor sends $E$ to $K^\bullet$
and commutes with direct sums.
\end{lemma}

\begin{proof}
Let $M$ be a differential graded $E$-module. For every object $U$ of
$\mathcal{C}$ the complex $K^\bullet(U)$ is a left differential
graded $E$-module as well as a right $\mathcal{O}(U)$-module.
The actions commute, so we have a bimodule.
Thus, by the constructions in Section \ref{section-tensor-product}
we can form the tensor product
$$
M \otimes_E K^\bullet(U)
$$
which is a differential graded $\mathcal{O}(U)$-module, i.e., a complex
of $\mathcal{O}(U)$-modules. This construction is functorial with respect
to $U$, hence we can sheafify to get a complex of $\mathcal{O}$-modules
which we denote
$$
M \otimes_E K^\bullet
$$
Moreover, for each $U$ the construction determines a functor
$\text{Mod}^{dg}_{(E, \text{d})} \to \text{Comp}^{dg}(\mathcal{O}(U))$
of differential graded categories by Lemma \ref{lemma-tensor}.
It is therefore clear that we obtain a functor as stated in the lemma.
\end{proof}

\begin{lemma}
\label{lemma-tensor-with-complex-homotopy}
The functor of Lemma \ref{lemma-tensor-with-complex} defines an exact functor
of triangulated categories
$K(\text{Mod}_{(E \text{d})}) \to K(\mathcal{O})$.
\end{lemma}

\begin{proof}
The functor induces a functor between homotopy categories by
Lemma \ref{lemma-functorial}.
We have to show that $- \otimes_E K^\bullet$ transforms distinguished
triangles into distinguished triangles.
Suppose that $0 \to K \to L \to M \to 0$ is an admissible short
exact sequence of differential graded $E$-modules. Let $s : M \to L$ be
a graded $E$-module homomorphism which is left inverse to $L \to M$.
Then $s$ defines a map $M \otimes_E K^\bullet \to L \otimes_E K^\bullet$
of graded $\mathcal{O}$-modules (i.e., respecting $\mathcal{O}$-module
structure and grading, but not differentials)
which is left inverse to $L \otimes_E K^\bullet \to M \otimes_E K^\bullet$.
Thus we see that
$$
0 \to K \otimes_E K^\bullet \to L \otimes_E K^\bullet \to
M \otimes_E K^\bullet \to 0
$$
is a termwise split short exact sequences of complexes, i.e., a
defines a distinguished triangle in $K(\mathcal{O})$.
\end{proof}

\begin{lemma}
\label{lemma-tensor-with-complex-derived}
The functor $K(\text{Mod}_{(E, \text{d})}) \to K(\mathcal{O})$
of Lemma \ref{lemma-tensor-with-complex-homotopy} has a left derived
version defined on all of $D(E, \text{d})$. We denote it
$- \otimes_E^\mathbf{L} K^\bullet : D(E, \text{d}) \to D(\mathcal{O})$.
\end{lemma}

\begin{proof}
We will use
Derived Categories, Lemma \ref{derived-lemma-find-existence-computes}
to prove this. As our collection $\mathcal{P}$
of objects we will use the objects with property (P).
Property (1) was shown in Lemma \ref{lemma-resolve}.
Property (2) holds because if $s : P \to P'$ is a quasi-isomorphism
of modules with property (P), then $s$ is a homotopy equivalence
by Lemma \ref{lemma-hom-derived}.
\end{proof}

\begin{lemma}
\label{lemma-tensor-with-complex-hom-adjoint}
Let $(\mathcal{C}, \mathcal{O})$ be a ringed site.
Let $K^\bullet$ be a complex of $\mathcal{O}$-modules.
Then the functors
$$
- \otimes_E^\mathbf{L} K^\bullet :
D(E, \text{d})
\longrightarrow
D(\mathcal{O})
$$
of Lemma \ref{lemma-tensor-with-complex-derived} and
$$
R\Hom(K^\bullet, -) : D(\mathcal{O}) \longrightarrow D(E, \text{d})
$$
of Lemma \ref{lemma-existence-of-derived} are adjoint.
\end{lemma}

\begin{proof}
The statement means that we have
$$
\Hom_{D(E, \text{d})}(M, R\Hom(K^\bullet, L^\bullet)) =
\Hom_{D(\mathcal{O})}(M \otimes^\mathbf{L}_E K^\bullet, L^\bullet)
$$
bifunctorially in $M$ and $L^\bullet$. To see this we may replace $M$
by a differential graded $E$-module $P$ with property (P).
We also may replace $L^\bullet$ by a K-injective complex of
$\mathcal{O}$-modules $I^\bullet$. The computation
of the derived functors given in the lemmas referenced in the statement
combined with Lemma \ref{lemma-hom-derived} translates the above into
$$
\Hom_{K(\text{Mod}_{(E, \text{d})})}
(P, \Hom_\mathcal{B}(K^\bullet, I^\bullet)) =
\Hom_{K(\mathcal{O})}(P \otimes_E K^\bullet, I^\bullet)
$$
where $\mathcal{B} = \text{Comp}^{dg}(\mathcal{O})$. 
There is an evalution map from right to left functorial
in $P$ and $I^\bullet$ (details omitted).
Choose a filtration $F_\bullet$ on $P$ as in the definition of property (P).
By Lemma \ref{lemma-property-P-sequence} and the fact that
both sides of the equation are homological functors in $P$
on $K(\text{Mod}_{(E, \text{d})})$
we reduce to the case where $P$ is replaced by
the differential graded $E$-module $\bigoplus F_iP$.
Since both sides turn direct sums in the variable $P$
into direct products we reduce to the case where $P$ is one of the
differential graded $E$-modules $F_iP$.
Since each $F_iP$ has a finite filtration (given by admissible
monomorpisms) whose graded pieces are graded projective $E$-modules
we reduce to the case where $P$ is a graded projective $E$-module.
In this case we clearly have
$$
\Hom_{\text{Mod}^{dg}_{(E, \text{d})}}
(P, \Hom_\mathcal{B}(K^\bullet, I^\bullet)) =
\Hom_{\text{Comp}^{dg}(\mathcal{O})}(P \otimes_E K^\bullet, I^\bullet)
$$
as graded $\mathbf{Z}$-modules (because this statement reduces to the case
$P = E[k]$ where it is obvious). As the isomorphism is compatible with
differentials we conclude.
\end{proof}

\begin{lemma}
\label{lemma-fully-faithful-in-compact-case}
Let $(\mathcal{C}, \mathcal{O})$ be a ringed site.
Let $K^\bullet$ be a complex of $\mathcal{O}$-modules
which represents a compact object of $D(\mathcal{O})$
such that $E = \Hom_{\text{Comp}^{dg}(\mathcal{O})}(K^\bullet, K^\bullet)$
computes the ext groups of $K^\bullet$ in $D(\mathcal{O})$.
Then the functor
$$
- \otimes_E^\mathbf{L} K^\bullet :
D(E, \text{d})
\longrightarrow
D(\mathcal{O})
$$
of Lemma \ref{lemma-tensor-with-complex-derived} is fully faithful.
\end{lemma}

\begin{proof}
Because our functor has a left adjoint given by
$R\Hom(K^\bullet, -)$ by Lemma \ref{lemma-tensor-with-complex-hom-adjoint}
it suffices to show for a differential graded $E$-module $M$ that the map
$$
H^0(M) \longrightarrow
\Hom_{D(\mathcal{O})}(K^\bullet, M \otimes_E^\mathbf{L} K^\bullet)
$$
is an isomorphism. We may assume that $M = P$ is a differential graded
$E$-module which has property (P). Since $K^\bullet$ defines a
compact object, we reduce using
Lemma \ref{lemma-property-P-sequence}
to the case where $P$ has a finite filtration whose graded pieces
are direct sums of $E[k]$. Again using compactness we reduce
to the case $P = E[k]$. The assumption on $K^\bullet$ is that
the result holds for these.
\end{proof}










\section{Tate resolutions}
\label{section-tate}

\noindent
In this section we briefly discuss the free resolutions
constructed in \cite{Tate-homology}.
The material in this section is of a different flavour as the material
in the preceding sections (perhaps it should be moved elsewhere).
Moreover, in this section we will use {\it homological notation}
and our differential graded algebras are sitting in nonnegative homological
degrees. Thus our differential graded algebras $(A, \text{d})$ will be given
as chain complexes
$$
\ldots \to A_2 \to A_1 \to A_0 \to 0 \to \ldots
$$
endowed with a multiplication.

\medskip\noindent
Let $R$ be a ring. In this section we will often consider graded
$R$-algebras $A = \bigoplus_{d \geq 0} A_d$ whose components are
zero in negative degrees. We will set $A_+ = \bigoplus_{d > 0} A_d$.
We will write $A_{even} = \bigoplus_{d \geq 0} A_{2d}$ and
$A_{odd} = \bigoplus_{d \geq 0} A_{2d + 1}$.
Recall that $A$ is graded commutative if
$x y = (-1)^{\deg(x)\deg(y)} y x$ for homogeneous elements $x, y$.
Recall that $A$ is strictly graded commutative if in addition
$x^2 = 0$ for homogeneous elements $x$ of odd degree. Finally, to understand
the following definition, keep in mind that $\gamma_n(x) = x^n/n!$
if $A$ is a $\mathbf{Q}$-algebra.

\begin{definition}
\label{definition-divided-powers}
Let $R$ be a ring. Let $A = \bigoplus_{d \geq 0} A_d$ be a graded
$R$-algebra which is strictly graded commutative. A collection of maps
$\gamma_n : A_{even, +} \to A_{even, +}$ defined for all $n > 0$ is called
a {\it divided power structure} on $A$ if we have
\begin{enumerate}
\item $\gamma_n(x) \in A_{2nd}$ if $x \in A_{2d}$,
\item $\gamma_1(x) = x$ for any $x$, we also set $\gamma_0(x) = 1$,
\item $\gamma_n(x)\gamma_m(x) = \frac{(n + m)!}{n! m!} \gamma_{n + m}(x)$,
\item $\gamma_n(xy) = x^n \gamma_n(y)$ for all $x \in A_{even}$ and
$y \in A_{even, +}$,
\item $\gamma_n(xy) = 0$ if $x, y \in A_{odd}$ homogeneous and $n > 1$
\item if $x, y \in A_{even, +}$ then
$\gamma_n(x + y) = \sum_{i = 0, \ldots, n} \gamma_i(x)\gamma_{n - i}(y)$,
\item $\gamma_n(\gamma_m(x)) =
\frac{(nm)!}{n! (m!)^n} \gamma_{nm}(x)$ for $x \in A_{even, +}$.
\end{enumerate}
\end{definition}

\noindent
Observe that conditions (2), (3), (4), (6), and (7) imply that
$\gamma$ is a ``usual'' divided power structure on the ideal
$A_{even, +}$ of the (commutative) ring $A_{even}$, see
Crystalline Cohomology, Sections \ref{crystalline-section-divided-powers},
\ref{crystalline-section-divided-power-rings},
\ref{crystalline-section-extend}, and
\ref{crystalline-section-divided-power-polynomial-ring}.
In particular, we have $\gamma_n(x) = n! x^n$ for all $x \in A_{even, +}$.
Condition (1) states that $\gamma$ is compatible with grading and condition
(5) tells us $\gamma_n$ for $n > 1$ vanishes on products
of homogeneous elements of odd degree. But note that it may happen
that
$$
\gamma_2(z_1 z_2 + z_3 z_4) = z_1z_2z_3z_4
$$
is nonzero if $z_1, z_2, z_3, z_4$ are homogeneous elements of odd degree.

\begin{example}[Adjoining odd variable]
\label{example-adjoining-odd}
Let $R$ be a ring. Let $(A, \gamma)$ be a strictly graded commutative
graded $R$-algebra endowed with a divided power structure as in the
definition above. Let $d > 0$ be an odd integer.
In this setting we can adjoin a variable $T$ of degree $d$ to $A$.
Namely, set
$$
A\langle T \rangle = A \oplus AT
$$
with grading given by $A\langle T \rangle_m = A_m \oplus A_{m - d}T$.
We claim there is a unique divided power structure on
$A\langle T \rangle$ compatible with the given divided power
structure on $A$. Namely, we set
$$
\gamma_n(x + yT) = \gamma_n(x) + \gamma_{n - 1}(x)yT
$$
for $x \in A_{even, +}$ and $y \in A_{odd}$.
\end{example}

\begin{example}[Adjoining even variable]
\label{example-adjoining-even}
Let $R$ be a ring. Let $(A, \gamma)$ be a strictly graded commutative
graded $R$-algebra endowed with a divided power structure as in the
definition above. Let $d > 0$ be an even integer.
In this setting we can adjoin a variable $T$ of degree $d$ to $A$.
Namely, set
$$
A\langle T \rangle = A \oplus AT \oplus AT^{(2)} \oplus AT^{(3)} \oplus \ldots
$$
with multiplication given by
$$
T^{(n)} T^{(m)} = \frac{(n + m)!}{n!m!} T^{(n + m)}
$$
and with grading given by
$$
A\langle T \rangle_m =
A_m \oplus A_{m - d}T \oplus A_{m - 2d}T^{(2)} \oplus \ldots
$$
We claim there is a unique divided power structure on
$A\langle T \rangle$ compatible with the given divided power
structure on $A$ such that $\gamma_n(T^{(i)}) = T^{(ni)}$.
To define the divided power structure we first set
$$
\gamma_n\left(\sum\nolimits_{i > 0} x_i T^{(i)}\right) =
\sum \prod\nolimits_{n = \sum e_i} x_i^{e_i} T^{(ie_i)}
$$
if $x_i$ is in $A_{even}$. If $x_0 \in A_{even, +}$
then we take
$$
\gamma_n\left(\sum\nolimits_{i \geq 0} x_i T^{(i)}\right) =
\sum\nolimits_{a + b = n}
\gamma_a(x_0)\gamma_b\left(\sum\nolimits_{i > 0} x_iT^{(i)}\right)
$$
where $\gamma_b$ is as defined above.
\end{example}

\noindent
At this point we tie in the definition of divided power structures
with differentials. To understand the definition note that
$\text{d}(x^n/n!) = \text{d}(x) x^{n - 1}/(n - 1)!$ if $A$
is a $\mathbf{Q}$-algebra and $x \in A_{even, +}$.

\begin{definition}
\label{definition-divided-powers-dga}
Let $R$ be a ring. Let $A = \bigoplus_{d \geq 0} A_d$ be a
differential graded $R$-algebra which is strictly graded commutative.
A divided power structure $\gamma$ on $A$ is {\it compatible with
the differential graded structure} if
$\text{d}(\gamma_n(x)) = \text{d}(x) \gamma_{n - 1}(x)$ for
all $x \in A_{even, +}$.
\end{definition}

\noindent
Warning: Let $(A, \text{d}, \gamma)$ be as in
Definition \ref{definition-divided-powers-dga}.
It may not be true that $\gamma_n(x)$ is a boundary, if
$x$ is a boundary. Thus $\gamma$ in general does not induce
a divided power structure on the homology algebra $H(A)$.
In some papers the authors put an additional compatibility
condition in order to insure this is the case, but we elect
not to do so.

\begin{lemma}
\label{lemma-dpdga-good}
Let $(A, \text{d}, \gamma)$ and $(B, \text{d}, \gamma)$ be as in
Definition \ref{definition-divided-powers-dga}. Let $f : A \to B$
be a map of differential graded algebras compatible with divided
power structures. Assume
\begin{enumerate}
\item $H_k(A) = 0$ for $k > 0$, and
\item $f$ is surjective.
\end{enumerate}
Then $\gamma$ induces a divided power structure on the graded
$R$-algebra $H(B)$.
\end{lemma}

\begin{proof}
Suppose that $x$ and $x'$ are homogeneous of the same degree $2d$
and define the same cohomology class in $H(B)$. Say $x' - x = \text{d}(w)$.
Choose a lift $y \in A_{2d}$ of $x$ and a lift $z \in A_{2d + 1}$
of $w$. Then $y' = y + \text{d}(z)$ is a lift of $x'$.
Hence
$$
\gamma_n(y') = \sum \gamma_i(y) \gamma_{n - i}(\text{d}(z))
= \gamma_n(y) +
\sum\nolimits_{i < n} \gamma_i(y) \gamma_{n - i}(\text{d}(z))
$$
Since $A$ is acyclic in positive degrees and since
$\text{d}(\gamma_j(\text{d}(z))) = 0$ for all $j$ we can write
this as
$$
\gamma_n(y') = \gamma_n(y) +
\sum\nolimits_{i < n} \gamma_i(y) \text{d}(z_i)
$$
for some $z_i$ in $A$. Moreover, for $0 < i < n$ we have
$$
\text{d}(\gamma_i(y) z_i) =
\text{d}(\gamma_i(y))z_i + \gamma_i(y)\text{d}(z_i) =
\text{d}(y) \gamma_{i - 1}(y) z_i + \gamma_i(y)\text{d}(z_i)
$$
and the first term maps to zero in $B$ as $\text{d}(y)$ maps to zero in $B$.
Hence $\gamma_n(x')$ and $\gamma_n(x)$ map to the same element of $H(B)$.
Thus we obtain a well defined map $\gamma_n : H_{2d}(B) \to H_{2nd}(B)$
for all $d > 0$ and $n > 0$. We omit the verification that this
defines a divided power structure on $H(B)$.
\end{proof}

\begin{lemma}
\label{lemma-base-change-div}
Let $(A, \text{d}, \gamma)$ is as in
Definition \ref{definition-divided-powers-dga}.
Let $R \to R'$ be a ring map.
Then $\text{d}$ and $\gamma$ induce similar structures on
$A' = A \otimes_R R'$ such that $(A, \text{d}, \gamma)$ is as in
Definition \ref{definition-divided-powers-dga}.
\end{lemma}

\begin{proof}
Observe that $A'_{even} = A_{even} \otimes_R R'$ and
$A'_{even, +} = A_{even, +} \otimes_R R'$. Hence we are trying to
show that the divided powers $\gamma$ extend to $A'_{even}$
(terminology as in
Crystalline Cohomology, Definition \ref{crystalline-definition-extends}).
Once we have shown $\gamma$ extends it follows easily that this
extension has all the desired properties.

\medskip\noindent
Choose a polynomial $R$-algebra $P$ and a surjection of $R$-algebras
$P \to R'$. The ring map $A_{even} \to A_{even} \otimes_R P$ is flat,
hence the divided powers $\gamma$ extend to $A_{even} \otimes_R P$
uniquely by
Crystalline Cohomology, Lemma \ref{crystalline-lemma-gamma-extends}.
Let $J = \text{Ker}(P \to R')$. To show that $\gamma$ extends
to $A \otimes_R R'$ it suffices to show that
$I' = \text{Ker}(A_{even, +} \otimes_R P \to A_{even, +} \otimes_R R')$
is generated by elements $z$ such that $\gamma_n(z) \in I'$
for all $n > 0$. This is clear as $I'$ is generated by elements
of the form $x \otimes f$ with
$x \in A_{even, +}$ and $f \in \text{Ker}(P \to R')$.
\end{proof}

\begin{lemma}
\label{lemma-extend-differential}
Let $(A, \text{d}, \gamma)$ is as in
Definition \ref{definition-divided-powers-dga}.
Let $d \geq 1$ be an integer.
Let $A\langle T \rangle$ be the graded divided power polynomial algebra
on $T$ with $\deg(T) = d$
constructed in Example \ref{example-adjoining-odd} or
\ref{example-adjoining-even}.
Let $f \in A_{d - 1}$ be an element with $\text{d}(f) = 0$.
There exists a unique differential $\text{d}$
on $A\langle T\rangle$ such that $\text{d}(T) = f$ and
such that $\text{d}$ is compatible with the divided power
structure on $A\langle T \rangle$.
\end{lemma}

\begin{proof}
Omitted.
\end{proof}

\noindent
Next, consider a ring map $R \to S$. For simplicity assume
that $R$ is Noetherian and the ring map of finite type
(if this is not the case, infinite polynomial algebras
will occur below). Following Tate will construct a factorization
$$
R \to A \to S
$$
with the following properties
\begin{enumerate}
\item $(A, \text{d}, \gamma)$ is as in
Definition \ref{definition-divided-powers-dga},
\item $A \to S$ is a quasi-isomorphism (if we endow $S$ with
the zero differential),
\item $A$ is a graded divided power polynomial algebra over $R$
with finitely many variables in each degree.
\end{enumerate}
The last condition means that $A$ is constructed out of $R$ by
successively adjoining variables $T$ as in
Examples \ref{example-adjoining-odd} and \ref{example-adjoining-even}.

\medskip\noindent
Start of the construction. Let $A(0) = R[x_1, \ldots, x_n]$ be
a (usual) polynomial ring and let $A(0) \to S$ be a surjection.
As grading we take $A(0)_0 = A(0)$ and $A(0)_d = 0$ for $d \not = 0$.
Thus $\text{d} = 0$ and $\gamma_n$, $n > 0$ is zero as well.

\medskip\noindent
Choose generators $f_1, \ldots, f_m \in R[x_1, \ldots, x_m]$
for the kernel of the given map $A(0) = R[x_1, \ldots, x_m] \to S$.
We apply Examples \ref{example-adjoining-odd} $m$ times to get
$$
A(1) = A(0)\langle T_1, \ldots, T_m\rangle
$$
with $\deg(T_i) = 1$ as a graded divided power polynomial algebra.
We set $\text{d}(T_i) = f_i$. Since $A(1)$ is a divided power polynomial
algebra over $A(0)$ and since $\text{d}(f_i) = 0$
this extends uniquely to a differential on $A(1)$ by
Lemma \ref{lemma-extend-differential}.

\medskip\noindent
Induction hypothesis: Assume we are given factorizations
$$
R \to A(0) \to A(1) \to \ldots \to A(m) \to S
$$
where $A(0)$ and $A(1)$ are as above and each $R \to A(m') \to S$
for $2 \leq m' \leq m$ satisfies properties (1) and (3)
as above and (2) replaced by the condition that
$H_i(A(m')) \to H_i(S)$ is an isomorphism for
$m' > i \geq 0$. The base case is $m = 1$.

\medskip\noindent
Induction step. Assume we have $R \to A(m) \to S$
as in the induction assumption. Consider the
group $H_m(A(m))$. This is a module over $H_0(A(m)) = S$.
In fact, it is a subquotient of $A(m)_m$ which is a finite
type module over $A(m)_0 = R[x_1, \ldots, x_n]$.
Thus we can pick finitely many elements
$$
e_1, \ldots, e_t \in \text{Ker}(\text{d} : A(m)_m \to A(m)_{m - 1})
$$
which map to generators of this module. Applying
Example \ref{example-adjoining-even} or
\ref{example-adjoining-odd} $t$ times
$$
A(m + 1) = A(m)\langle T_1, \ldots, T_t\rangle
$$
with $\deg(T_i) = m + 1$ as a graded divided power algebra. We set
$\text{d}(T_i) = e_i$. Since $A(1)$ is a divided power polynomial
algebra over $A(0)$ and since $\text{d}(e_i) = 0$
this extends uniquely to a differential on $A(m + 1)$
compatible with the divided power structure.
Since we've added only material in degree $m + 1$ and higher we see
that $H_i(A(m + 1)) = H_i(A(m))$ for $i < m$. Moreover, it is
clear that $H_m(A(m + 1)) = 0$ by construction.

\medskip\noindent
To finish the proof we observe that we have shown there exists
a tower
$$
R \to A(0) \to A(1) \to \ldots \to A(m) \to A(m + 1) \to \ldots \to S
$$
and we can simply take
$$
A = \colim A(m)
$$

\begin{lemma}
\label{lemma-divided-powers-on-tor}
Let $R$ be a Noetherian ring. Let $R \to S$ and $R \to T$ be
finite type ring maps. There exists a canonical structure
of a divided power graded $R$-algebra on
$$
\text{Tor}_*^R(S, T)
$$
\end{lemma}

\begin{proof}
Choose a factorization $R \to A \to S$ as above. Since $A \to S$
is a quasi-isomorphism and since $A_d$ is a free $R$-module,
we see that the differential graded algebra $B = A \otimes_R T$ computes
the tor groups displayed in the lemma. Choose a surjection
$R[y_1, \ldots, y_k] \to T$. Then we see that
$B$ is a quotient of the differential graded algebra
$A[y_1, \ldots, y_k]$ whose homology sits in degree $0$ (it is equal
to $S[y_1, \ldots, y_k]$).
By Lemma \ref{lemma-base-change-div} the differential graded algebras $B$ and
$A[y_1, \ldots, y_k]$ have divided power structures compatible
with the differentials. Hence we obtain our divided
power structure on $H(B)$ by Lemma \ref{lemma-dpdga-good}.
\end{proof}










\section{Other chapters}

\begin{multicols}{2}
\begin{enumerate}
\item \hyperref[introduction-section-phantom]{Introduction}
\item \hyperref[conventions-section-phantom]{Conventions}
\item \hyperref[sets-section-phantom]{Set Theory}
\item \hyperref[categories-section-phantom]{Categories}
\item \hyperref[topology-section-phantom]{Topology}
\item \hyperref[sheaves-section-phantom]{Sheaves on Spaces}
\item \hyperref[algebra-section-phantom]{Commutative Algebra}
\item \hyperref[sites-section-phantom]{Sites and Sheaves}
\item \hyperref[homology-section-phantom]{Homological Algebra}
\item \hyperref[derived-section-phantom]{Derived Categories}
\item \hyperref[more-algebra-section-phantom]{More Algebra}
\item \hyperref[simplicial-section-phantom]{Simplicial Methods}
\item \hyperref[modules-section-phantom]{Sheaves of Modules}
\item \hyperref[sites-modules-section-phantom]{Modules on Sites}
\item \hyperref[injectives-section-phantom]{Injectives}
\item \hyperref[cohomology-section-phantom]{Cohomology of Sheaves}
\item \hyperref[sites-cohomology-section-phantom]{Cohomology on Sites}
\item \hyperref[hypercovering-section-phantom]{Hypercoverings}
\item \hyperref[schemes-section-phantom]{Schemes}
\item \hyperref[constructions-section-phantom]{Constructions of Schemes}
\item \hyperref[properties-section-phantom]{Properties of Schemes}
\item \hyperref[morphisms-section-phantom]{Morphisms of Schemes}
\item \hyperref[coherent-section-phantom]{Coherent Cohomology}
\item \hyperref[divisors-section-phantom]{Divisors}
\item \hyperref[limits-section-phantom]{Limits of Schemes}
\item \hyperref[varieties-section-phantom]{Varieties}
\item \hyperref[chow-section-phantom]{Chow Homology}
\item \hyperref[topologies-section-phantom]{Topologies on Schemes}
\item \hyperref[descent-section-phantom]{Descent}
\item \hyperref[more-morphisms-section-phantom]{More on Morphisms}
\item \hyperref[flat-section-phantom]{More on Flatness}
\item \hyperref[groupoids-section-phantom]{Groupoid Schemes}
\item \hyperref[more-groupoids-section-phantom]{More on Groupoid Schemes}
\item \hyperref[etale-section-phantom]{\'Etale Morphisms of Schemes}
\item \hyperref[etale-cohomology-section-phantom]{\'Etale Cohomology}
\item \hyperref[spaces-section-phantom]{Algebraic Spaces}
\item \hyperref[spaces-properties-section-phantom]{Properties of Algebraic Spaces}
\item \hyperref[spaces-morphisms-section-phantom]{Morphisms of Algebraic Spaces}
\item \hyperref[spaces-topologies-section-phantom]{Topologies on Algebraic Spaces}
\item \hyperref[spaces-descent-section-phantom]{Descent and Algebraic Spaces}
\item \hyperref[spaces-more-morphisms-section-phantom]{More on Morphisms of Spaces}
\item \hyperref[quot-section-phantom]{Quot and Hilbert Spaces}
\item \hyperref[stacks-section-phantom]{Stacks}
\item \hyperref[spaces-groupoids-section-phantom]{Groupoids in Algebraic Spaces}
\item \hyperref[spaces-more-groupoids-section-phantom]{More on Groupoids in Spaces}
\item \hyperref[bootstrap-section-phantom]{Bootstrap}
\item \hyperref[examples-stacks-section-phantom]{Examples of Stacks}
\item \hyperref[groupoids-quotients-section-phantom]{Quotients of Groupoids}
\item \hyperref[algebraic-section-phantom]{Algebraic Stacks}
\item \hyperref[criteria-section-phantom]{Criteria for Representability}
\item \hyperref[stacks-properties-section-phantom]{Properties of Algebraic Stacks}
\item \hyperref[stacks-morphisms-section-phantom]{Morphisms of Algebraic Stacks}
\item \hyperref[examples-section-phantom]{Examples}
\item \hyperref[exercises-section-phantom]{Exercises}
\item \hyperref[guide-section-phantom]{Guide to Literature}
\item \hyperref[desirables-section-phantom]{Desirables}
\item \hyperref[coding-section-phantom]{Coding Style}
\item \hyperref[fdl-section-phantom]{GNU Free Documentation License}
\item \hyperref[index-section-phantom]{Auto Generated Index}
\end{enumerate}
\end{multicols}


\bibliography{my}
\bibliographystyle{amsalpha}

\end{document}
