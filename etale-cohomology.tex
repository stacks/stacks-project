\IfFileExists{stacks-project.cls}{%
\documentclass{stacks-project}
}{%
\documentclass{amsart}
}

% The following AMS packages are automatically loaded with
% the amsart documentclass:
%\usepackage{amsmath}
%\usepackage{amssymb}
%\usepackage{amsthm}

% For dealing with references we use the comment environment
\usepackage{verbatim}
\newenvironment{reference}{\comment}{\endcomment}
%\newenvironment{reference}{}{}
\newenvironment{slogan}{\comment}{\endcomment}
\newenvironment{history}{\comment}{\endcomment}

% For commutative diagrams you can use
% \usepackage{amscd}
\usepackage[all]{xy}

% We use 2cell for 2-commutative diagrams.
\xyoption{2cell}
\UseAllTwocells

% To put source file link in headers.
% Change "template.tex" to "this_filename.tex"
% \usepackage{fancyhdr}
% \pagestyle{fancy}
% \lhead{}
% \chead{}
% \rhead{Source file: \url{template.tex}}
% \lfoot{}
% \cfoot{\thepage}
% \rfoot{}
% \renewcommand{\headrulewidth}{0pt}
% \renewcommand{\footrulewidth}{0pt}
% \renewcommand{\headheight}{12pt}

\usepackage{multicol}

% For cross-file-references
\usepackage{xr-hyper}

% Package for hypertext links:
\usepackage{hyperref}

% For any local file, say "hello.tex" you want to link to please
% use \externaldocument[hello-]{hello}
\externaldocument[introduction-]{introduction}
\externaldocument[conventions-]{conventions}
\externaldocument[sets-]{sets}
\externaldocument[categories-]{categories}
\externaldocument[topology-]{topology}
\externaldocument[sheaves-]{sheaves}
\externaldocument[sites-]{sites}
\externaldocument[stacks-]{stacks}
\externaldocument[fields-]{fields}
\externaldocument[algebra-]{algebra}
\externaldocument[brauer-]{brauer}
\externaldocument[homology-]{homology}
\externaldocument[derived-]{derived}
\externaldocument[simplicial-]{simplicial}
\externaldocument[more-algebra-]{more-algebra}
\externaldocument[smoothing-]{smoothing}
\externaldocument[modules-]{modules}
\externaldocument[sites-modules-]{sites-modules}
\externaldocument[injectives-]{injectives}
\externaldocument[cohomology-]{cohomology}
\externaldocument[sites-cohomology-]{sites-cohomology}
\externaldocument[dga-]{dga}
\externaldocument[dpa-]{dpa}
\externaldocument[hypercovering-]{hypercovering}
\externaldocument[schemes-]{schemes}
\externaldocument[constructions-]{constructions}
\externaldocument[properties-]{properties}
\externaldocument[morphisms-]{morphisms}
\externaldocument[coherent-]{coherent}
\externaldocument[divisors-]{divisors}
\externaldocument[limits-]{limits}
\externaldocument[varieties-]{varieties}
\externaldocument[topologies-]{topologies}
\externaldocument[descent-]{descent}
\externaldocument[perfect-]{perfect}
\externaldocument[more-morphisms-]{more-morphisms}
\externaldocument[flat-]{flat}
\externaldocument[groupoids-]{groupoids}
\externaldocument[more-groupoids-]{more-groupoids}
\externaldocument[etale-]{etale}
\externaldocument[chow-]{chow}
\externaldocument[intersection-]{intersection}
\externaldocument[pic-]{pic}
\externaldocument[adequate-]{adequate}
\externaldocument[dualizing-]{dualizing}
\externaldocument[duality-]{duality}
\externaldocument[discriminant-]{discriminant}
\externaldocument[local-cohomology-]{local-cohomology}
\externaldocument[curves-]{curves}
\externaldocument[resolve-]{resolve}
\externaldocument[models-]{models}
\externaldocument[pione-]{pione}
\externaldocument[etale-cohomology-]{etale-cohomology}
\externaldocument[proetale-]{proetale}
\externaldocument[crystalline-]{crystalline}
\externaldocument[spaces-]{spaces}
\externaldocument[spaces-properties-]{spaces-properties}
\externaldocument[spaces-morphisms-]{spaces-morphisms}
\externaldocument[decent-spaces-]{decent-spaces}
\externaldocument[spaces-cohomology-]{spaces-cohomology}
\externaldocument[spaces-limits-]{spaces-limits}
\externaldocument[spaces-divisors-]{spaces-divisors}
\externaldocument[spaces-over-fields-]{spaces-over-fields}
\externaldocument[spaces-topologies-]{spaces-topologies}
\externaldocument[spaces-descent-]{spaces-descent}
\externaldocument[spaces-perfect-]{spaces-perfect}
\externaldocument[spaces-more-morphisms-]{spaces-more-morphisms}
\externaldocument[spaces-flat-]{spaces-flat}
\externaldocument[spaces-groupoids-]{spaces-groupoids}
\externaldocument[spaces-more-groupoids-]{spaces-more-groupoids}
\externaldocument[bootstrap-]{bootstrap}
\externaldocument[spaces-pushouts-]{spaces-pushouts}
\externaldocument[groupoids-quotients-]{groupoids-quotients}
\externaldocument[spaces-more-cohomology-]{spaces-more-cohomology}
\externaldocument[spaces-simplicial-]{spaces-simplicial}
\externaldocument[formal-spaces-]{formal-spaces}
\externaldocument[restricted-]{restricted}
\externaldocument[spaces-resolve-]{spaces-resolve}
\externaldocument[formal-defos-]{formal-defos}
\externaldocument[defos-]{defos}
\externaldocument[cotangent-]{cotangent}
\externaldocument[examples-defos-]{examples-defos}
\externaldocument[algebraic-]{algebraic}
\externaldocument[examples-stacks-]{examples-stacks}
\externaldocument[stacks-sheaves-]{stacks-sheaves}
\externaldocument[criteria-]{criteria}
\externaldocument[artin-]{artin}
\externaldocument[quot-]{quot}
\externaldocument[stacks-properties-]{stacks-properties}
\externaldocument[stacks-morphisms-]{stacks-morphisms}
\externaldocument[stacks-limits-]{stacks-limits}
\externaldocument[stacks-cohomology-]{stacks-cohomology}
\externaldocument[stacks-perfect-]{stacks-perfect}
\externaldocument[stacks-introduction-]{stacks-introduction}
\externaldocument[stacks-more-morphisms-]{stacks-more-morphisms}
\externaldocument[stacks-geometry-]{stacks-geometry}
\externaldocument[moduli-]{moduli}
\externaldocument[moduli-curves-]{moduli-curves}
\externaldocument[examples-]{examples}
\externaldocument[exercises-]{exercises}
\externaldocument[guide-]{guide}
\externaldocument[desirables-]{desirables}
\externaldocument[coding-]{coding}
\externaldocument[obsolete-]{obsolete}
\externaldocument[fdl-]{fdl}
\externaldocument[index-]{index}

% Theorem environments.
%
\theoremstyle{plain}
\newtheorem{theorem}[subsection]{Theorem}
\newtheorem{proposition}[subsection]{Proposition}
\newtheorem{lemma}[subsection]{Lemma}

\theoremstyle{definition}
\newtheorem{definition}[subsection]{Definition}
\newtheorem{example}[subsection]{Example}
\newtheorem{exercise}[subsection]{Exercise}
\newtheorem{situation}[subsection]{Situation}

\theoremstyle{remark}
\newtheorem{remark}[subsection]{Remark}
\newtheorem{remarks}[subsection]{Remarks}

\numberwithin{equation}{subsection}

% Macros
%
\def\lim{\mathop{\rm lim}\nolimits}
\def\colim{\mathop{\rm colim}\nolimits}
\def\Spec{\mathop{\rm Spec}}
\def\Hom{\mathop{\rm Hom}\nolimits}
\def\Ext{\mathop{\rm Ext}\nolimits}
\def\SheafHom{\mathop{\mathcal{H}\!{\it om}}\nolimits}
\def\SheafExt{\mathop{\mathcal{E}\!{\it xt}}\nolimits}
\def\Sch{\textit{Sch}}
\def\Mor{\mathop{\rm Mor}\nolimits}
\def\Ob{\mathop{\rm Ob}\nolimits}
\def\Sh{\mathop{\textit{Sh}}\nolimits}
\def\NL{\mathop{N\!L}\nolimits}
\def\proetale{{pro\text{-}\acute{e}tale}}
\def\etale{{\acute{e}tale}}
\def\QCoh{\textit{QCoh}}
\def\Ker{\mathop{\rm Ker}}
\def\Im{\mathop{\rm Im}}
\def\Coker{\mathop{\rm Coker}}
\def\Coim{\mathop{\rm Coim}}

%
% Macros for moduli stacks/spaces
%
\def\QCohstack{\mathcal{QC}\!{\it oh}}
\def\Cohstack{\mathcal{C}\!{\it oh}}
\def\Spacesstack{\mathcal{S}\!{\it paces}}
\def\Quotfunctor{{\rm Quot}}
\def\Hilbfunctor{{\rm Hilb}}
\def\Curvesstack{\mathcal{C}\!{\it urves}}
\def\Polarizedstack{\mathcal{P}\!{\it olarized}}
\def\Complexesstack{\mathcal{C}\!{\it omplexes}}
% \Pic is the operator that assigns to X its picard group, usage \Pic(X)
% \Picardstack_{X/B} denotes the Picard stack of X over B
% \Picardfunctor_{X/B} denotes the Picard functor of X over B
\def\Pic{\mathop{\rm Pic}\nolimits}
\def\Picardstack{\mathcal{P}\!{\it ic}}
\def\Picardfunctor{{\rm Pic}}
\def\Deformationcategory{\mathcal{D}\!{\it ef}}


% OK, start here.
%
\begin{document}

\title{Etale Cohomology}


\maketitle

\phantomsection
\label{section-phantom}

\tableofcontents


\section*{Introduction}% (9.08.09)
\addcontentsline{toc}{section}{Introduction}
\subsection{Prologue}

This class is about another cohomology theory. So the first thing to remark is that the Zariski topology is not entirely satisfactory. One of the main reasons that it fails to give the results that we would want is that if $X$ is a complex variety and $\mathcal{F}$ is a constant sheaf then
$$
H_{Zar}^i ( X, \mathcal{F}) = 0 \quad \text{ for all } i>0.
$$
The reason for that is the following. In an irreducible scheme (a variety in particular), any two nonempty open subsets meet, and so the restriction mappings of a constant sheaf are surjective. We say that the sheaf is \emph{flasque}. In this case, all higher \u Cech cohomology groups vanish, and so do all higher Zariski cohomology groups. In other words, there are ``not enough'' open sets in the Zariski topology to detect this higher cohomology.
\\
On the other hand, if $X$ is a smooth projective complex variety, then
$$
H_{Betti}^{2 \dim X} (X (\mathbf{C}), \Lambda) = \Lambda \quad \text{ for } \Lambda = \mathbf{Z}, \ \mathbf{Z}/n\mathbf{Z},
$$
where $X(\mathbf{C})$ means the set of complex points of $X$. This is a feature that would be nice to replicate in algebraic geometry. In positive characteristic in particular.

\subsection{The \'Etale Topology}

It is very hard to simply ``add'' extra open sets to refine the Zariski topology. One efficient way to define a topology is to consider not only open sets, but also some schemes that lie over them. To define the \'etale topology, one considers all morphisms $\varphi: \mathcal{U} \to X$ which are \'etale.  If $X$ is a smooth projective variety over $\mathbf{C}$, then this means 
\begin{enumerate}
\item $\mathcal{U}$ is a disjoint union of smooth varieties ; and
\item $\varphi$ is (analytically) locally an isomorphism.
\end{enumerate}
The word ``analytically'' refers to the usual (transcendental) topology over $\mathbf{C}$. So the second condition means that the derivative of $\varphi$ has full rank everywhere (and in particular all the components of $\mathcal{U}$ have the same dimension as $X$).

A double cover may not be an \'etale morphism if it has a double point. All the fibers should have the same number of points. Removing that point will make the morphism \'etale. To consider the \'etale topology, we have to look at all such morphisms. Unlike the Zariski topology, these need not be merely subsets of $X$, even though their images always are.

\begin{definition}
A family of morphisms $\{ \varphi_i : \mathcal{U}_i \to X\}_{i \in I}$ is called an \emph{\'etale covering} if each $\varphi_i$ is an \'etale morphism and their images cover $X$, {\it i.e.} $X = \cup_{i \in I} \varphi_i (\mathcal{U}_i)$.
\end{definition}
This ``defines'' the \'etale topology. In other words, we can now say what the sheaves are. An \emph{\'etale sheaf} $\mathcal{F}$ on $X$ of sets (respectively abelian groups, vector spaces, rings, etc) is the data
\begin{enumerate}
\item for each \'etale morphism $\varphi : \mathcal{U} \to X$, of a set (resp. abelian group, etc) $\mathcal{F} (\mathcal{U})$ ;
\item for each pair $\mathcal{U}, \ \mathcal{U}'$ of \'etale schemes over $X$, and each morphism $\mathcal{U} \to \mathcal{U}'$ over $X$ (which is automatically \'etale), of a restriction map $\rho^{\mathcal{U}}_{\mathcal{U}'} : \mathcal{F}(\mathcal{U}) \to \mathcal{F}(\mathcal{U}')$ ; such that
\item for every \'etale covering $\{ \varphi_i : \mathcal{U}_i \to X\}_{i \in I}$, the diagram 
$$
\xymatrix{
\emptyset \ar[r] &
\mathcal{F} (\mathcal{U}) \ar[r] &
\Pi_{i \in I} \mathcal{F} (\mathcal{U}_i) \ar@<1ex>[r] \ar@<-1ex>[r] &
\Pi_{i,j \in I} \mathcal{F} (\mathcal{U}_i \times_\mathcal{U} \mathcal{U}_j)
}
$$
is exact (in the category of sets, and similarly in the appropriate category of abelian groups, vector spaces etc). 
\end{enumerate} 
\begin{remark}
In the last statement, it is essential not to forget the case where $i = j$ which is in general a highly nontrivial condition (unlike in the Zariski topology). In fact, the most important coverings have only one element.
\end{remark}

Since the identity is an \'etale morphism, we can compute the global sections of an \'etale sheaf, and cohomology will simply be the corresponding right-derived functors. In other words, once more theory has been developed and statements have been made precise, there will be no obstacle to defining cohomology.

\subsection{Feats of the \'Etale Topology}

For a natural number $n \in \mathbf{N} = \{1, 2, 3, 4, \dots\}$ it is true that
$$
H_{et}^2 (\mathbf{P}^1_\mathbf{C}, \mathbf{Z}/n\mathbf{Z}) = \mathbf{Z}/n\mathbf{Z}.
$$
More generally, if $X$ is a complex variety, then its \'etale Betti numbers with coefficients in a finite field agree with the usual Betti numbers of $X(\mathbf{C})$, {\it i.e.}
$$
\dim_{\mathbf{F}_q} H_{et}^{2i} (X, \mathbf{F}_q) = \dim_{\mathbf{F}_q} H_{Betti}^{2i} (X(\mathbf{C}), \mathbf{F}_q).
$$ 
This is extremely satisfactory. However, these equalities only hold for torsion coefficients, not in general. For integer coefficients, one has
$$
H_{et}^2 (\mathbf{P}^1_\mathbf{C}, \mathbf{Z}) = 0.
$$
There are ways to get back to nontorsion coefficients from torsion ones.

\subsubsection{A Computation}

How do we compute the cohomology of $\mathbf{P}^1_\mathbf{C}$ with coefficients $\Lambda = \mathbf{Z}/n\mathbf{Z}$? 
We use \u Cech cohomology. A covering of $\mathbf{P}^1_\mathbf{C}$ is given by the two standard opens $\mathcal{U}_0, \mathcal{U}_1$, which are both isomorphic to $\mathbf{A}^1_\mathbf{C}$, and  which intersection is isomorphic to $\mathbf{A}^1_\mathbf{C} -  \{0\} = \mathbf{G}_m$. It turns out that the Mayer-Vietoris sequence holds in the \'etale topology, therefore there is an exact sequence
$$
H_{et}^{i-1}(\mathcal{U}_0\cap \mathcal{U}_1, \Lambda) \to H_{et}^i(\mathbf{P}^1_C, \Lambda) \to H_{et}^i(\mathcal{U}_0, \Lambda) \oplus H_{et}^i(\mathcal{U}_1, \Lambda) \to H_{et}^i(\mathcal{U}_0\cap \mathcal{U}_1, \Lambda).
$$
To get the answer we expect, we would need to show that the direct sum in the third term vanishes. In fact, it is true that, as for the usual topology,
$$
H_{et}^q (\mathbf{A}^1_\mathbf{C}, \Lambda) = 0 \quad \text{ for } q \geq 1,
$$ 
and
$$
H_{et}^q (\mathbf{A}^1_\mathbf{C}-\{0\}, \Lambda) = \left\{
\begin{array}{ll}
\Lambda & \text{ if $q = 1$, and} \\
0 & \text{ for $q \geq 2$.}
\end{array}
\right. 
$$
These results are already quite hard (what is an elementary proof?). Let us explain how we would compute this once the machinery of \'etale cohomology is at our disposal.

\paragraph{Higher cohomology.} This is taken care of by the following general fact: if $X$ is an affine curve over $\mathbf{C}$, then 
$$
H_{et}^q (X, \mathbf{Z}/n\mathbf{Z}) = 0 \quad \text{ for } q \geq 2.
$$
This is proved by considering the generic point of the curve and doing some Galois cohomology. So we only have to worry about the cohomology in degree 1. 

\paragraph{Cohomology in degree 1.} We use the following identifications:
\begin{eqnarray*}
H_{et}^1 (X, \mathbf{Z}/n\mathbf{Z}) & = & \left\{
\begin{array}{c}
\text{sheaves of sets $\mathcal{F}$ on the \'etale site $X_{\text{\'et}}$ endowed with an} \\
\text{action $\mathbf{Z}/n\mathbf{Z} \times \mathcal{F} \to \mathcal{F}$ such that $\mathcal{F}$ is a $\mathbf{Z}/n\mathbf{Z}$-torsor.}
\end{array}
\right\}
\Big/ \cong 
\\
& = & \left\{
\begin{array}{c}
\text{morphisms $Y \to X$ which are finite \'etale together} \\
\text{ with a free $\mathbf{Z}/n\mathbf{Z}$ action such that $X = Y /(\mathbf{Z}/n\mathbf{Z})$.}
\end{array}
\right\}
\Big/ \cong.
\end{eqnarray*}
The first identification is very general (it is true for any cohomology theory on a site) and has nothing to do with the \'etale topology. The second identification is a consequence of descent theory. The last set describes a collection of geometric objects on which we can get our hands.

\begin{itemize}
\item 
Since $\mathbf{A}^1_\mathbf{C}$ has no nontrivial finite \'etale covering, $H_{et}^1 (\mathbf{A}^1_\mathbf{C}, \mathbf{Z}/n\mathbf{Z}) = 0$.
\item
We need to study the finite \'etale coverings $\varphi: Y \to \mathbf{A}^1_\mathbf{C}-\{0\}$. It suffices to consider the case where $Y$ is connected, which we assume. Say that this morphism is $n$ to 1, and consider a projective compactification
$$
\xymatrix{
{Y\ } \ar@{^{(}->}[r] \ar^{\varphi}[d] & {\bar Y} \ar^{\bar\varphi}[d] \\
{\mathbf{A}^1_\mathbf{C}-\{0\}} \ar@{^{(}->}[r]  &{\mathbf{P}^1_\mathbf{C}}
}
$$
Even though $\varphi$ is \'etale and does not ramify, $\bar{\varphi}$ may ramify at 0 and $\infty$. Say that the preimages of 0 are the points $y_1, \dots, y_r$ with indices of ramification $e_1, \dots e_r$, and that the preimages of $\infty$ are the points $y_1', \dots, y_s'$ with indices of ramification $d_1, \dots d_s$. (In particular, $\sum e_i = n = \sum d_j$.) Applying the Riemann-Hurwitz formula, we get
$$
2 g_Y - 2 = -2n + \sum (e_i - 1) + \sum (d_j - 1)
$$
and therefore $g_Y = 0$, $r=s=1$ and $e_1 = d_1 = n$. One such covering is given by taking the $n$th root of the coordinates. Here $Y = {\mathbf{A}^1_\mathbf{C}-\{0\}}$ and $\varphi (z) = z^n$. 

Remember that we need not only classify the coverings of ${\mathbf{A}^1_\mathbf{C}-\{0\}}$ but also the free $\mathbf{Z}/n\mathbf{Z}$-actions on it. In our case any such action corresponds to an automorphism of $Y$ sending $z$ to $\zeta_n z$, where $\zeta_n$ is an $n$th root of unity. There are $\varphi(n)$ such actions (here $\varphi(n)$ means the Euler function). 
In other words, this computation shows that
$$
H_{et}^1 (\mathbf{A}^1_\mathbf{C}-\{0\}, \mathbf{Z}/n\mathbf{Z}) = \mu_n (\mathbf{C}) \cong \mathbf{Z}/n\mathbf{Z}.
$$
The first identification is canonical, the second isn't.
\end{itemize}

\subsubsection{Nontorsion Coefficients}

To study nontorsion coefficients, one makes the following definition:
$$
H_{et}^i (X, \mathbf{Q}_\ell) := \left( \varprojlim_n H_{et}^i (X; \mathbf{Z}/\ell^n\mathbf{Z}) \right) \otimes_{\mathbf{Z}_\ell} \mathbf{Q}_\ell.
$$
Thus we will need to study systems of sheaves satisfying some compatibility conditions.\section{Sheaf Theory} %(9.10.09)

\subsection{Presheaves}

\begin{definition}
Let $\mathcal{C}$ be a category. A \emph{presheaf of sets} (respectively, an \emph{abelian presheaf}) on $\mathcal{C}$ is a functor $\mathcal{C}^{opp} \to \textit{Sets}$ (resp. $\text{Ab}$).
\end{definition}

\begin{remark}
\label{remark-terminology}
Terminology.
If $\mathcal{U} \in \text{Ob}(\mathcal{C})$, then elements of $\mathcal{F}(\mathcal{U})$ are called \emph{sections} of $\mathcal{F}$ on $\mathcal{U}$; for $\varphi: \mathcal{V} \to \mathcal{U}$ in $\mathcal{C}$, the map $\mathcal{F}(\varphi) : \mathcal{F}(\mathcal{V}) \to \mathcal{F}(\mathcal{U})$ is denoted $s \mapsto \mathcal{F}(\varphi) (s) = \varphi^* (s) = s |_\mathcal{V} $ and called \emph{restriction mapping}. This last notation is ambiguous since the restriction map depends on $\varphi$, but it's standard. We also use the notation $\Gamma(\mathcal{U}, \mathcal{F}) = \mathcal{F}(\mathcal{U})$.
\end{remark}

Saying that $\mathcal{F}$ is a functor merely means that if $\mathcal{W} \to \mathcal{V} \to \mathcal{U}$ are morphisms in $\mathcal{C}$ and $s \in \Gamma(\mathcal{U},\mathcal{F})$ then $(s|_\mathcal{V})|_\mathcal{W} = s |_\mathcal{W}$, with the abuse of notation just seen.

The category of presheaves of sets (respectively of abelian presheaves) on $\mathcal{C}$ is denoted $\textit{PSh} (\mathcal{C})$ (resp. $\text{PAb} (\mathcal{C})$). It is the category of functors from $\mathcal{C}^{opp}$ to $\textit{Sets}$ (resp. $\text{Ab}$), which is to say that the morphisms of presheaves are natural transformations of functors. 

\begin{example}
Given an object $X \in \text{Ob}(\mathcal{C})$, we consider the functor
$$
\begin{array}{rccl}
h_X : & \mathcal{C}^{opp} & \to & \textit{Sets} \\
& \mathcal{U} & \mapsto & h_X(\mathcal{U}) = \text{Hom}_\mathcal{C} (\mathcal{U},X) \\
& \mathcal{V}\xrightarrow{\varphi} \mathcal{U} & \mapsto & \varphi \circ  - : h_X(\mathcal{U}) \to h_X(\mathcal{V}).
\end{array}
$$
It is a presheaf, called the \emph{representable presheaf associated to $X$.} It is not true that representable presheaves are sheaves in every topology on every site.
\end{example}

\begin{lemma}[Yoneda] Let $\mathcal{C}$ be a category, and $X,Y \in \text{Ob}(\mathcal{C})$. There is a natural bijection 
$$
\begin{array}{rcl}
\text{Hom}_\mathcal{C}(X,Y) & \cong & \text{Hom}_{\textit{PSh}(\mathcal{C})} (h_X,h_Y) \\
\psi & \longmapsto & h_\psi =  \psi \circ - : h_X \to h_Y.
\end{array}
$$
\end{lemma}

\subsection{Sites}

\begin{definition}
Let $\mathcal{C}$ be a category. A \emph{family of morphisms with fixed target} $\{\varphi_i : \mathcal{U}_i \to \mathcal{U} \}_{i\in I}$ is the data of
\begin{itemize} 
\item an object $\mathcal{U} \in \mathcal{C}$ ; 
\item a set $I$ (possibly empty) ; and 
\item for all $i\in I$, a morphism $\varphi_i : \mathcal{U}_i \to \mathcal{U}$. 
\end{itemize}
A \emph{site} consists of a category $\mathcal{C}$ and a set $\text{Cov}(\mathcal{C})$ consising of families of morphisms with fixed target called \emph{coverings}, such that
\begin{description}
\item[\it(isom)]
if $\varphi : \mathcal{V} \to \mathcal{U}$ is an isomorphism in $\mathcal{C}$, then $\{\varphi : \mathcal{V} \to \mathcal{U}\}$ is a covering ;
\item[\it(locality)]
if $\{\varphi_i : \mathcal{U}_i \to \mathcal{U} \}_{i\in I}$ and for all $i \in I$, $\{\psi_{ij} : \mathcal{U}_{ij} \to \mathcal{U}_i \}_{j\in I_i}$ are all coverings, then $$\{\varphi_i \circ \psi_{ij} : \mathcal{U}_{ij} \to \mathcal{U} \}_{(i,j)\in \text{pr}od_{i\in I} \{i\} \times I_i}$$ is also a covering ;
\item[\it(base change)]
if $\{\mathcal{U}_i \to \mathcal{U} \}_{i\in I}$ is a covering and $\mathcal{V} \to \mathcal{U}$ is a morphism in $\mathcal{C}$, then
\begin{itemize}
\item for all $i \in I$, $\mathcal{U}_i \times_\mathcal{U} \mathcal{V}$ exists in $\mathcal{C}$ ; and
\item  $\{\mathcal{U}_i \times_\mathcal{U} \mathcal{V} \to \mathcal{V} \}_{i\in I}$ is a covering.
\end{itemize}
\end{description}
\end{definition}

\begin{example}
If $X$ is a topological space, then it has an associated site $\mathcal{T}_X$ defined as follows: the objects of $\mathcal{T}_X$ are the open subsets of $X$, the morphisms between these are the inclusion mappings, and the coverings are the usual topological (surjective) coverings. Observe that if $\mathcal{U}, \mathcal{V} \subseteq \mathcal{W} \subseteq X$ are open subsets then $\mathcal{U} \times_\mathcal{W} \mathcal{V} = \mathcal{U} \cap \mathcal{V}$ exists: this category has fiber products. All the verifications are trivial and everything works as can be expected. 
\end{example}

\subsection{Sheaves}

\begin{definition}
A presheaf $\mathcal{F}$ of sets (resp. abelian presheaf) on a site $\mathcal{C}$ is called a \emph{sheaf} if for all coverings $\{\varphi_i : \mathcal{U}_i \to \mathcal{U} \}_{i\in I} \in \text{Cov} (\mathcal{C})$, the diagram
\begin{equation}
\label{sheafaxiom}
\xymatrix{
\mathcal{F}(\mathcal{U}) \ar[r] &
\text{pr}od_{i\in I} \mathcal{F}(\mathcal{U}_i) \ar@<1ex>[r] \ar@<-1ex>[r] &
\text{pr}od_{i,j \in I} \mathcal{F}(\mathcal{U}_i \times_\mathcal{U} \mathcal{U}_j),
}
\end{equation}
where the first map is $s \mapsto (s|_{\mathcal{U}_i})_{i\in I}$ and the two maps on the right are $(s_i)_{i\in I} \mapsto (s_i |_{\mathcal{U}_i \times_\mathcal{U} \mathcal{U}_j})$ and $(s_i)_{i\in I} \mapsto (s_j |_{\mathcal{U}_i \times_\mathcal{U} \mathcal{U}_j})$, is an equalizer diagram (in the appropriate category of sets or abelian groups). 
\end{definition}

\begin{remark}
For the empty covering (where $I = \emptyset$), this implies that $\mathcal{F}(\emptyset)$ is an empty product, which is a final object in the corresponding category (a singleton, for $\textit{Sets}$ and $\text{Ab}$).
\end{remark}

\begin{example}
Working this out for $\mathcal{T}_X$ gives the usual notion of sheaves.
\end{example}

\begin{definition}
We denote $\textit{Sh}(\mathcal{C})$ (respectively $\text{Ab}(\mathcal{C})$) the full subcategory of $\textit{PSh}(\mathcal{C})$ (resp. $\text{PAb}(\mathcal{C})$) which objects are sheaves. This is the \emph{category of sheaves of sets} (resp. \emph{abelian sheaves}) on $\mathcal{C}$.
\end{definition}

\subsubsection{The example of $G\textit{-Sets}$}
\label{subsubsection:GSets}

Let $G$ be a group and define a site $\mathcal{T}_G$ as follows: the underlying category is the category of $G$-sets, {\it i.e.} its objects are sets endowed with a left $G$-action and the morphisms are equivariant maps; and the coverings of $\mathcal{T}_G$ are the families $\{\varphi_i : \mathcal{U}_i \to \mathcal{U} \}_{i\in I}$ satisfying $\mathcal{U} = \cup_{i\in I} \varphi_i (\mathcal{U}_i)$. 
\\
There is a special object in the site $\mathcal{T}_G$, namely the $G$-set $G$ endowed with its natural action by left translations. We denote it ${}_G G$. Observe that there is a natural group isomorphism
$$
\begin{array}{rccl}
\rho: & G^{opp} & \cong & \text{Aut}_{G\textit{-Sets}}({}_G G) \\
& g & \longmapsto & (h \mapsto hg).
\end{array}
$$
In particular, for any presheaf $\mathcal{F}$, the set $\mathcal{F}({}_G G)$  inherits a $G$-action \emph{via} $\rho$. (Note that by contravariance of $\mathcal{F}$, the set $\mathcal{F}({}_G G)$ is again a left $G$-set.) In fact, the functor 
$$
\begin{array}{rcl}
\textit{Sh}(\mathcal{T}_G) & \longrightarrow & G\textit{-Sets} \\
\mathcal{F} & \longmapsto & \mathcal{F}({}_G G)
\end{array}
$$
is an equivalence of categories. Its quasi-inverse is the functor $X \mapsto h_X$. Without giving an actual proof, let's try to explain why some of this is true.
\begin{enumerate}
\item 
If $S$ is a $G$-set, we can decompose it into orbits $S = \coprod_{i\in I} O_i$. The sheaf axiom for the covering $\{O_i \to S\}_{i\in I}$ says that 
$$
\xymatrix{
\mathcal{F}(S) \ar[r] &
\text{pr}od_{i\in I} \mathcal{F}(O_i) \ar@<1ex>[r] \ar@<-1ex>[r] &
\text{pr}od_{i,j \in I} \mathcal{F}(O_i \times_S O_j)
}
$$
is an equalizer. Observing that fibered products in $G\textit{-Sets}$ are induced from fibered products in $\textit{Sets}$, and using the fact that $\mathcal{F}(\emptyset)$ is a $G$-singleton, we get that 
$$
\text{pr}od_{i,j \in I} \mathcal{F}(O_i \times_S O_j) = \text{pr}od_{i \in I} \mathcal{F}(O_i)
$$
and the two maps above are in fact the same. Therefore the sheaf axiom merely says that $\mathcal{F}(S) = \text{pr}od_{i\in I} \mathcal{F}(O_i)$.
\item
If $S$ is the $G$-set $S= G/H$ and $\mathcal{F}$ is a sheaf on $\mathcal{T}_G$, then we claim that
$$
\mathcal{F}(G/H) = \mathcal{F}({}_G G)^H
$$
and in particular $\mathcal{F}(\{*\}) = \mathcal{F}({}_G G)^G$. To see this, let's use the sheaf axiom for the covering $\{ {}_G G \to G/H \}$ of $S$. We have 
\begin{eqnarray*}
{}_G G  \times_{G/H} {}_G G & \cong & G \times H \\
(g_1, g_2) & \longmapsto & (g_1, g_1 g_2^{-1})
\end{eqnarray*}
is a disjoint union of copies of ${}_G G$ (as a $G$-set). Hence the sheaf axiom reads
$$
\xymatrix{
\mathcal{F} (G/H) \ar[r] &
\mathcal{F}({}_G G) \ar@<1ex>[r] \ar@<-1ex>[r] &
\text{pr}od_{h\in H} \mathcal{F}({}_G G)
}
$$
where the two maps on the right are $s \mapsto  (s)_{h \in H}$ and $s \mapsto  (hs)_{h \in H}$. Therefore $\mathcal{F}(G/H) = \mathcal{F}({}_G G)^H$ as claimed.
\end{enumerate}
This doesn't quite prove the claimed equivalence of categories, but it shows at least that a sheaf $\mathcal{F}$ is entirely determined by its sections over ${}_G G$.

\subsection{Sheafification}

\begin{definition}
Let $\mathcal{F}$ be a presheaf on the site $\mathcal{C}$ and $\mathcal{U} = \{ \mathcal{U}_i \to \mathcal{U}\} \in \text{Cov} (\mathcal{C})$. We define the first \u Cech cohomology group by
$$
\check H^0 (\mathcal{U}, \mathcal{F}) = \left\{\left.  (s_i)_{i\in I} \in \text{pr}od_{i\in I }\mathcal{F}(\mathcal{U}_i) \ \right| \ s_i |_{\mathcal{U}_i \times_\mathcal{U} \mathcal{U}_j} = s_j |_{\mathcal{U}_i \times_\mathcal{U} \mathcal{U}_j} \right\}.
$$
There is a canonical map $ \mathcal{F}(\mathcal{U}) \to \check H^0 (\mathcal{U}, \mathcal{F})$, $s \mapsto (s |_{\mathcal{U}_i})_{i\in I}.$
We say that a covering $\mathcal{V} = \{ \mathcal{V}_j \to \mathcal{U}\}_{j \in J}$ \emph{refines} $\mathcal{U}$ if there exists $\alpha : J \to I$ and for all $j \in J$, a commutative diagram
$$
\xymatrix{
{\mathcal{V}_j\ } \ar^{\chi_j}[rr] \ar[dr] & &{\mathcal{U}_{\alpha(j)}} \ar[dl] \\
& {\mathcal{U}.}
}
$$ 
\end{definition}

Given the data $\alpha, \{\chi_j\}_{i\in J}$ as above, define
\begin{eqnarray*}
\check H^0 (\mathcal{U}, \mathcal{F}) & \to & \check H^0 (\mathcal{V}, \mathcal{F}) \\
(s_i)_{i\in I} & \mapsto & \left(\chi_j^*\left(s _{\alpha(j)}\right)\right)_{j\in J}.
\end{eqnarray*}
We then claim that 
\begin{enumerate}
\item the map is well-defined (easy verification) ; and
\item this map is independent of the choice of $\alpha, \{\chi_j\}_{i\in J}$. To see this, consider also $\beta, \{\psi_j : \mathcal{V}_j \to \mathcal{U}_{\beta(j)}\}_{j \in J}$ and the commutative diagram
$$
\xymatrix{
{\mathcal{V}_j}  \ar^{(\chi_j, \psi_j)\qquad}[rr] & & {\mathcal{U}_{\alpha(j)}\times_\mathcal{U} \mathcal{U}_{\beta(j)}} \ar[dl] \ar[dr] \\
& {\mathcal{U}_{\alpha(j)}} \ar[dr] && {\mathcal{U}_{\beta(j)}} \ar[dl] \\
& &{\mathcal{U}.}
}
$$
Given a section $s \in \mathcal{F}(\mathcal{U})$, its image in $\mathcal{F}(\mathcal{V}_j)$ under the map given by $\alpha, \{\chi_j\}_{i\in J}$ is $s_{\alpha(j)} | _{{\mathcal{U}_{\alpha(j)}\times_\mathcal{U} \mathcal{U}_{\beta(j)}}}$, and its image under the map given by $\beta, \{\psi_j\}_{i\in J}$ is $s_{\beta(j)} | _{{\mathcal{U}_{\alpha(j)}\times_\mathcal{U} \mathcal{U}_{\beta(j)}}}$. These two are equal since by assumption $s \in \check H(\mathcal{U}, \mathcal{F})$. By the presheaf axiom, these two are the same when restricted to $\mathcal{V}_j$.
\end{enumerate}

\begin{theorem}
Let $\mathcal{C}$ be a site and $\mathcal{F}$ a presheaf on $\mathcal{C}$.
\begin{enumerate}
\item The rule
$$
\mathcal{U} \mapsto \mathcal{F}^+ \mathcal{U} := \text{colim}_{\mathcal{U} \in \text{Cov}(\mathcal{U})} \check H^0(\mathcal{U}, \mathcal{F})
$$
is a presheaf. (And the colimit is a directed one.)
\item There is a canonical map of presheaves $\mathcal{F} \to \mathcal{F}^+$.
\item If $\mathcal{F}$ is a separated presheaf then $\mathcal{F}^+$ is a sheaf and the map in {\it ii} is injective.
\item $\mathcal{F}^+$ is a separated presheaf.
\item $\mathcal{F}^\sharp = (\mathcal{F}^+)^+$ is a sheaf, and the canonical map induces a functorial isomorphism
$$ 
\text{Hom}_{\textit{PSh}(\mathcal{C})}(\mathcal{F}, \mathcal{G}) = \text{Hom}_{\textit{Sh}(\mathcal{C})}(\mathcal{F}^\sharp,\mathcal{G})
$$ 
for any $\mathcal{G} \in \textit{Sh}(\mathcal{C})$.
\end{enumerate}
\end{theorem}

In other words, this means that the natural inclusion map $\mathcal{F} \hookrightarrow \mathcal{F}^\sharp$ is a left adjoint to the forgetful functor $\textit{Sh}(\mathcal{C}) \to \textit{PSh}(\mathcal{C})$.

\subsection{Cohomology}

\begin{theorem}
The category of abelian sheaves on a site has enough injectives.
\end{theorem}

\begin{proof}
Omitted.
\end{proof}

So we can define cohomology as the right-derived functors of the global sections functor: if $X \in \text{Ob}(\mathcal{C})$ and $\mathcal{F} \in \text{Ab}(\mathcal{C})$, 
$$
H^p(X,\mathcal{F}) := R^p \Gamma (X, \mathcal{F}) = H^p (\Gamma(X, \mathcal{I}^\bullet))
$$
where $\mathcal{F} \to \mathcal{I}^\bullet$ is an injective resolution. To do this, we should check that the functor $\Gamma (X, -)$ is left exact. This is true indeed. \section{The fpqc Site} % (9.15.09)

\begin{definition}
Let $T$ be a scheme. An \emph{$fpqc$-covering} of $T$ is a family $\{ \varphi_i : T_i \to T\}_{i \in I}$ such that
\begin{enumerate}
\item 
each $\varphi_i$ is a flat morphism and $\cup_{i\in I} \varphi_i (T_i) = T$ ; and
\item
for each affine open $\mathcal{U} \subseteq T$  there exists a finite set $K$, a map $\mathbf{i} : K \to I$ and affine opens $\mathcal{U}_{\mathbf{i}(k)} \subseteq T_{\mathbf{i}(k)}$ such that $\mathcal{U} = \cup_{k \in K} \varphi_{\mathbf{i}(k)}(\mathcal{U}_{\mathbf{i}(k)})$.
\end{enumerate}
\end{definition}

\begin{remark}
The first condition corresponds to fp, which stands for \emph{fid\`element plat}, faithfully flat in french, and the second to qc, \emph{quasi-compact}. The second part of the first condition is unnecessary when the second condition holds.
\end{remark}

\begin{example}$ $
\begin{itemize}
\item Any Zariski open covering of $T$ is an $fpqc$-covering.
\item If $f: X \to Y$ is flat, surjective and quasi-compact, then $\{ f: X\to Y\}$ is an $fpqc$-covering.
\item The morphism $\varphi: \coprod_{x \in \mathbf{A}^n_k} \text{Spec} (\mathcal{O}_{\mathbf{A}^n_k,x}) \to \mathbf{A}^n_k $, where $k$ is an infinite field, is flat and surjective, but not quasi-compact, hence the family $\{ \varphi \}$ is not an $fpqc$-covering.
\item Write $\mathbf{A}^2_k = \text{Spec} k[x,y]$, $i_x : D(x) \hookrightarrow \mathbf{A}^2_k$ and $i_y : D(y) \hookrightarrow \mathbf{A}^2_k$ the standard opens. Then the families $\{i_x, i_y, \text{Spec} k[[ x,y ]] \to \mathbf{A}^2_k \}$ and $\{i_x, i_y, \text{Spec} \mathcal{O}_{\mathbf{A}^2_k,0} \to \mathbf{A}^2_k \}$ are $fpqc$-coverings.
\end{itemize}
\end{example}

\begin{lemma}
The collection of $fpqc$-coverings defines a site on the category of schemes.
\end{lemma}

The proof is left as an exercise.

\begin{definition}
The site defined by the $fpqc$-coverings on the category of schemes (respectively, over a fixed scheme $S$) is denoted $\textit{Sch}_fpqc$ (resp. $\textit{Sch}Sfpqc$)  and called the \emph{big $fpqc$ site} (resp. \emph{over $S$}). Note that $\textit{Sch} = \textit{Sch}_{/\text{Spec}\mathbf{Z}}$ so we may deal with the relative case only, without loss of generality. 
\end{definition}

\begin{lemma}\label{fpqc-sheaves}
Let $\mathcal{F}$ be a presheaf on $\textit{Sch}S$. Then $\mathcal{F}$ is a sheaf on $\textit{Sch}Sfpqc$ if and only if
\begin{enumerate}
\item $\mathcal{F}$ is a sheaf with respect to the Zariski topology, and
\item for every faithfully flat morphism $\text{Spec} B \to \text{Spec} A$ of affine schemes over $S$, the sheaf axiom holds for the covering $\{\text{Spec} B \to \text{Spec} A\}$. Namely, this means that 
$$
\xymatrix{
\mathcal{F}(\text{Spec} A) \ar[r] &
\mathcal{F}(\text{Spec} B) \ar@<1ex>[r] \ar@<-1ex>[r] &
\mathcal{F}(\text{Spec} B \otimes_A B)
}
$$
is an equalizer diagram.
\end{enumerate}
\end{lemma}

An alternative way to think of an $fpqc$ sheaf $\mathcal{F}$ on $\textit{Sch}Sfpqc$ is as the following data:
\begin{enumerate}
\item for each $T_{/S}$, a usual ({\it i.e.} Zariski) sheaf $\mathcal{F}_T$ on $T_{\mathrm{Zar}}$ ;
\item for every map $f : T' \to T$ over $S$, a restriction mapping $f^* \mathcal{F}_T \to \mathcal{F}_{T'} $; such that
\item the restriction mappings are functorial. These three conditions give the data of a Zariski sheaf on $\textit{Sch}S$. The above lemma says that for $\mathcal{F}$ to be an $fpqc$ sheaf, one also needs that
\item for every faithfully flat morphism $\text{Spec} B \to \text{Spec} A$ over $S$, the diagram 
$$
\xymatrix{
\mathcal{F}_{\text{Spec} A}(\text{Spec} A) \ar[r] &
\mathcal{F}_{\text{Spec} B}(\text{Spec} B) \ar@<1ex>[r] \ar@<-1ex>[r] &
\mathcal{F}_{\text{Spec} B \otimes_A B}(\text{Spec} B \otimes_A B)
}
$$
is an equalizer.
\end{enumerate}

\begin{example}
Consider the presheaf 
$$
\begin{array}{rccl}
\mathcal{F} : & \textit{Sch}S^{opp} & \to & \text{Ab} \\
& T_{/S} & \mapsto & \Gamma(T, \Omega^1_{T/S}).
\end{array}
$$
The compatibility of differentials with localization implies that $\mathcal{F}$ is a sheaf on the Zariski site (conditions (a)-(c) hold). However, it is not a sheaf on the $fpqc$ site: consider the case $S = \text{Spec} \mathbf{F}_p$ and the morphism 
$$
\varphi: \mathcal{V} = \text{Spec} \mathbf{F}_p[v] \to \mathcal{U} = \text{Spec} \mathbf{F}_p[u]
$$
given by mapping $u$ to $v^p$. The family $\{\varphi \}$ is an $fpqc$-covering, yet the restriction mapping $\mathcal{F}(\mathcal{U}) \to \mathcal{F}(\mathcal{V})$ send the generator $\mathrm{d} u$ to $\mathrm{d}(v^p) = 0$, so it is the zero map, and the diagram
$$
\xymatrix{
\mathcal{F}(\mathcal{U}) \ar[r]^{0} &
\mathcal{F}(\mathcal{V}) \ar@<1ex>[r] \ar@<-1ex>[r] &
\mathcal{F}(\mathcal{V} \times_\mathcal{U} \mathcal{V})
}
$$
is not an equalizer. (We will see that $\mathcal{F}$ is in fact a sheaf on the \'etale and smooth sites.)
\end{example}

\begin{lemma}
Any representable presheaf on $\textit{Sch}S$ is a sheaf on the $fpqc$ site.
\end{lemma}

We will prove this further. We say that the $fpqc$ site is \emph{subcanonical}.

\begin{remark}
The $fpqc$ is the finest topology that we will see. Hence any sheaf on the $fpqc$ site will also be a sheaf in the subsequent sites (\'etale, smooth, etc).
\end{remark}

\begin{example}
For the additive group scheme $\mathbf{G}_{a,S} = \mathbf{A}^1_S$, one has $h_{\mathbf{G}_{a,S}} (T) = \text{Hom}_S (T,\mathbf{G}_{a,S}) = \Gamma(T,\mathcal{O}_T)$, hence the structure sheaf
$$
\begin{array}{rccl}
\mathcal{O} : & \textit{Sch}S^{opp} & \to & \textit{Rings} \\
& T_{/S} & \mapsto & \Gamma(T, \mathcal{O}_{T})
\end{array}
$$
is a sheaf on the $fpqc$ site. So for instance there is a notion of $\mathcal{O}$-modules on this site.
\end{example}

\subsection{Faithfully Flat Descent}

\begin{definition}
Let $\mathcal{U} = \{ t_i : T_i \to T\}_{i \in I}$ be a family with fixed target. A \emph{descent datum} for quasi-coherent sheaves with respect to $\mathcal{U}$ is a family $(\mathcal{F}_i, \varphi_{ij})_{i,j\in I}$ where
\begin{enumerate}
\item for all $i$, $\mathcal{F}_i$ is a quasi-coherent sheaf on $T_i$ ; and
\item for all $i, j \in I$, $\varphi_{ij} : \text{pr}_0^* \mathcal{F}_i \cong \text{pr}_1^* \mathcal{F}_j$ is an isomorphism on $T_i \times_T T_j$ such that the diagrams
$$
\xymatrix{
{\text{pr}_0^* \mathcal{F}_i} \ar_{\text{pr}_{02}^*\varphi_{ik}}[dr] \ar^{\text{pr}_{01}^*\varphi_{ij}}[rr] & & {\text{pr}_1^* \mathcal{F}_j}\ar^{\text{pr}_{12}^*\varphi_{jk}}[dl] \\
& {\text{pr}_2^* \mathcal{F}_k}  
}
$$ 
commutes on $T_i \times_T T_j \times_T T_k$.
\end{enumerate}

This descent datum is called \emph{effective} if there exist a quasi-coherent sheaf $\mathcal{F}$ over $T$ and $\mathcal{O}_{T_i}$-module isomorphisms $\varphi_i : t_i^* \mathcal{F} \cong \mathcal{F}_i$ satisfying the cocycle condition, namely
$$
\varphi_{ij} = \text{pr}_1^* (\varphi_j) \circ \text{pr}_0^* (\varphi_i)^{-1}.
$$ 
\end{definition}

\begin{theorem} \label{thm:DescentIsEffectiveForQCoh}
If $\mathcal{V} = \{T_i \to T\}_{i\in I}$ is an $fpqc$-covering, then all descent data with respect to $\mathcal{V}$ is effective.
\end{theorem}

In other words, the fibered category of quasi-coherent sheaves is a stack on the $fpqc$ site.
The proof of the theorem is in two steps. The first one is to realize that for the Zariski site this is easy (or well-known) using standard glueing of sheaves and the locality of quasi-coherence. The second step is the case of an $fpqc$-covering of the form $\{ \text{Spec} B \to \text{Spec} A\}$ where $A \to B$ is a faithfully flat ring map. This is a lemma in algebra, which we now present.

\paragraph{Descent}
If $A \to B$ is a ring map, we consider the complex
$$
(B/A)_\bullet : \qquad B \to B\otimes_A B \to B\otimes_A B \otimes_A B \to \cdots
$$
where $B$ is in degree 0, $B\otimes_A B$ in degree 1, etc, and the maps are given by 
\begin{eqnarray*}
b & \mapsto  & 1 \otimes b - b \otimes 1, \\
b_0 \otimes b_1 &  \mapsto & 1 \otimes b_0 \otimes b_1 - b_0 \otimes 1 \otimes b_1 + b_0 \otimes b_1 \otimes 1, \\
& \text{etc.}
\end{eqnarray*}

\begin{lemma}
If $A \to B$ is faithfully flat, then the complex $(B/A)_\bullet$ is exact in positive degrees, and $H^0((B/A)_\bullet) = A$.
\end{lemma}

Grothendieck proves this in three steps. Firstly, he assumes that the map $A \to B$ has a section, and constructs an explicit homotopy to the complex where $A$ is the only nonzero term, in degree 0. Secondly, he observes that to prove the result, it suffices to do so after a faithfully flat base change $A \to A'$,  replacing $B$ with $B' = B \otimes_A A'$. Thirdly, he applies the faithfully flat base change $A \to A' =B$ and remarks that the map $A' = B \to B' = B\otimes_A B$ has a natural section.

\begin{lemma} \label{lem:descentForModules}
If $A \to B$ is faithfully flat and $M$ is an $A$-module, then the
complex $(B/A)_\bullet \otimes_A M$ is exact in positive degrees, and $H^0((B/A)_\bullet \otimes_A M) = M$.
\end{lemma}

The same strategy of proof works.

\begin{definition}
Let $A \to B$ be a ring map and $N$ a $B$-module. A \emph{descent datum} for $N$ with respect to $A \to B$ is an isomorphism $\varphi: N\otimes_A B \cong B\otimes_A N$ of $B\otimes_A B$-modules such that the diagram of $B\otimes_A B \otimes_A B$-modules
$$
\xymatrix{
{N \otimes_A  B \otimes_A B} \ar_{\varphi_{01}}[dr] \ar^{\varphi_{02}}[rr] & & {B \otimes_A  N \otimes_A B}\ar^{\varphi_{12}}[dl] \\
& {B \otimes_A  B \otimes_A N}  
}
$$ 
commutes.
\end{definition}

If $N' = B \otimes_A M$ for some $A$-module M, then it has a canonical descent datum given by the map
$$
\begin{array}{rrcl}
 \varphi_\text{can}: & N' \otimes_A B & \to & B \otimes_A N' \\
& b_0 \otimes m \otimes b_1 & \mapsto & b_0 \otimes b_1 \otimes m.
\end{array}
$$

\begin{definition}
A descent datum $(N,\varphi)$ is called \emph{effective} if there exists an $A$-module $M$ such that $(N,\varphi) \cong (B\otimes_A M, \varphi_\text{can})$, with the obvious notion of isomorphism of descent data.
\end{definition}

The previous lemma is then equivalent to the following result.

\begin{theorem}
If $A \to B$ is faithfully flat then all descent data with respect to $A\to B$ is effective. 
\end{theorem}

\begin{remark}$ $
\begin{itemize}
\item
This fact gives the exactness of the \u Cech complex in positive degrees for the covering $\{ \text{Spec} B \to \text{Spec} A\}$ where $A \to B$ is faithfully flat. In particular, this will give some vanishing of cohomology.
\item
If $(N,\varphi)$ is a descent datum with respect to a faithfully flat map $A\to B$, then the corresponding $A$-module is given by 
$$
M = \ker \left(
\begin{array}{rcl}
N & \longrightarrow & B\otimes_A N \\
 n & \longmapsto & 1 \otimes n - \varphi(n\otimes 1)
 \end{array}
\right).
$$ 
\end{itemize}
\end{remark}

%(9.17.09)
\subsection{Quasi-coherent Sheaves}

\begin{proposition} For any quasi-coherent sheaf $\mathcal{F}$ on $S$ the presheaf
$$
\begin{array}{rccl}
\mathcal{F}_fpqc : & \textit{Sch}Sfpqc & \to & \text{Ab}\\ 
& (f: T \to S) &\mapsto & \Gamma(T, f^*\mathcal{F})
\end{array}
$$
  is a sheaf of $\mathcal{O}$-modules (on the $fpqc$ site). 
\end{proposition}
\begin{proof}
  As established in a previous lemma, it is enough to check the sheaf property on Zariski coverings and faithfully flat morphisms of affine schemes.  The sheaf property for Zariski coverings is standard scheme theory, since $\Gamma(\mathcal{U}, i^\ast \mathcal{F}) = \mathcal{F}(\mathcal{U})$ when $i: \mathcal{U} \hookrightarrow S$ is an open immersion. 
  \\
  For $\left\{\text{Spec}(B)\to \text{Spec}(A)\right\}$ with $A\to B$ faithfully
  flat and 
  $\mathcal{F}|_{\text{Spec}(A)} = \widetilde{M}$ 
  this corresponds to the fact that
  $M=H^0\left((B/A)_\bullet\otimes_AM \right)$, {\it i.e.} that
  \begin{align*}
0 \to M \to B\otimes_A M \to B\otimes_A B \otimes_A M
\end{align*}
is exact.
\end{proof}

  \begin{definition}
    Let $\mathcal{C}$ be a \emph{ringed site}, {\it i.e.} a site endowed with a sheaf of rings $\mathcal{O}$. A sheaf of $\mathcal{O}$-modules $\mathcal{F}$ on $\mathcal{C}$ is called \emph{quasi-coherent} if for all $\mathcal{U}\in \text{Ob}(\mathcal{C})$ there exists a covering $\left\{\mathcal{U}_i \to \mathcal{U}\right\}_{i\in I}$ of $\mathcal{C}$ such that  $\mathcal{F}|_{\mathcal{C}_{/\mathcal{U}_i}}$ is isomorphic to the cokernel of an $\mathcal{O}$-linear map of free $\mathcal{O}$-modules $$\mathcal{O}|_{\mathcal{C}_{/\mathcal{U}_i}}^{(K)} \to \mathcal{O}|_{\mathcal{C}_{/\mathcal{U}_i}}^{(L)}.$$ Here, $\mathcal{C}_{/\mathcal{U}}$ is the category of objects of $\mathcal{C}$ over $\mathcal{U}$, whose objects are morphisms $\mathcal{V} \to \mathcal{U}$ and whose morphisms are over $\mathcal{U}$, and the sheaf $\mathcal{O}^{(K)}$ is the sheaf associated to the presheaf $\bigoplus_{f \in K} \mathcal{O}$. 
  \end{definition}
  
  \begin{remark}
  In the case where $\mathcal{C}$ has a final object ({\it e.g.} $S$) it suffices to take $\mathcal{U}$ to be the final object in the above statement.
  \end{remark}
  
  \begin{theorem}
Any quasi-coherent $\mathcal{O}$-module on $\textit{Sch}Sfpqc$ is of the form $\mathcal{F}_fpqc$ for some quasi-coherent sheaf $\mathcal{F}$ on $S$. 
  \end{theorem}
  
In other words, there is no difference between quasi-coherent $\mathcal{O}$-modules on $S$ (or on $\textit{Sch}SZar$) and on the site $\textit{Sch}Sfpqc$.
  
\begin{proof} 
After some formal arguments this is exactly the descent theorem \ref{thm:DescentIsEffectiveForQCoh}.
\end{proof}
  
\subsection{\u Cech Cohomology}

Our next goal is to use descent theory to show that $H_{fpqc}^i(S, \mathcal{F}) = H_{Zar}^i(S, \mathcal{F})$ for all quasi-coherent  sheaves $\mathcal{F}$ on $S$, allowing us to compute cohomology of some $fpqc$-sheaves.  To this end, we introduce \u Cech cohomology on sites. See \cite{Artin} for more details.
 
\begin{definition} 
Let $\mathcal{C}$ be a site, $\mathcal{U}=\left\{\mathcal{U}_i\to \mathcal{U}\right\}_{i\in I}$ a covering of $\mathcal{C}$ and $\mathcal{F}\in \text{PAb}(\mathcal{C})$ an abelian presheaf. We define the \emph{\u Cech complex} $\check{\mathcal{C}}^\bullet(\mathcal{U}, \mathcal{F})$ by 
$$ 
 \text{pr}od_{i_0\in I}\mathcal{F}(\mathcal{U}_{i_0}) \to \text{pr}od_{i_0, i_1\in I}\mathcal{F}\left(\mathcal{U}_{i_0}\times_{\mathcal{U}} \mathcal{U}_{i_1}\right) \to \text{pr}od_{i_0, i_1, i_2 \in I} \mathcal{F}\left(U_{i_0}\times_U U_{i_1} \times_U U_{i_2}\right) \to \cdots
$$
where the first term is in degree 0, and the maps are the usual ones. Again, it is essential to allow the case $i_0 = i_1$ etc. The \emph{\u Cech cohomology groups} are defined by
$$
\check{H}^p(\mathcal{U}, \mathcal{F})= H^p(\check{\mathcal{C}}^\bullet\left(\mathcal{U}, \mathcal{F}\right)).
$$
\end{definition}

\begin{lemma} 
The functor $\check{\mathcal{C}}^\bullet(\mathcal{U}, -)$ is exact on the category $\text{PAb}(\mathcal{C})$. 
\end{lemma}	
  
In other words, if $0\to \mathcal{F}_1\to \mathcal{F}_2\to \mathcal{F}_3\to 0$ is a short exact sequence of presheaves of abelian groups, then  
$$
0 \to \check{\mathcal{C}}^\bullet\left(\mathcal{U}, \mathcal{F}_1\right) \to\check{\mathcal{C}}^\bullet(\mathcal{U}, \mathcal{F}_2) \to \check{\mathcal{C}}^\bullet(\mathcal{U}, \mathcal{F}_3)\to 0 
$$
is a short exact sequence of complexes.
  
\begin{proof}
This follows at once from the definition of a short exact sequence of presheaves. Since the category of abelian presheaves is the category of functors on some category with values in $\text{Ab}$, it is automatically an abelian category: a sequence $\mathcal{F}_1\to \mathcal{F}_2\to \mathcal{F}_3$ is exact in $\text{PAb}$ if and only if for all $\mathcal{U}\in \text{Ob}(\mathcal{C})$, the sequence $\mathcal{F}_1(\mathcal{U})\to \mathcal{F}_2(\mathcal{U})\to \mathcal{F}_3(\mathcal{U})$ is exact in $\text{Ab}$. So the complex above is merely a product of short exact sequences in each degree.
  \end{proof}
  
This shows that $\check{H}^\bullet(\mathcal{U}, -)$ is a $\delta$-functor. 
We now proceed to show that it is a universal $\delta$-functor. We thus need to show that it is an \emph{effaceable} functor. We start by recalling the Yoneda lemma.
  
  \begin{lemma}[Yoneda Lemma]
    For any presheaf $\mathcal{F}$ on a site $\mathcal{C}$ there is a functorial isomophism
      $$
        \text{Hom}_{\textit{PSh}(\mathcal{C})} (h_{\mathcal{U}}, \mathcal{F}) = \mathcal{F}(\mathcal{U}).
$$
  \end{lemma}
  
  \begin{definition}
  Given a presheaf of sets $\mathcal{G}$ , we define the \emph{free abelian presheaf on $\mathcal{G}$}, denoted $\mathbf{Z}_{\mathcal{G}}$, by
$$
    \mathbf{Z}_{\mathcal{G}}(\mathcal{U}) := \text{ free abelian group on }\mathcal{G}(\mathcal{U}) = \bigoplus_{g\in \mathcal{G}(\mathcal{U})}\mathbf{Z}  := \mathbf{Z}\left[\mathcal{G}(\mathcal{U})\right]
  $$
  with restriction maps induced by the original restriction maps of sets. In the special case $\mathcal{G} = h_\mathcal{U}$ we write simply $\mathbf{Z}_\mathcal{U} = \mathbf{Z}_{h_\mathcal{U}}$.
  \end{definition}
  
  The functor $\mathcal{G} \mapsto \mathbf{Z}_\mathcal{G}$ is left adjoint to the forgetful
  functor $\text{PAb}(\mathcal{C}) \to \textit{PSh}(\mathcal{C})$.  Thus, for any presheaf $\mathcal{F}$, there is a canonical isomorphism
 $$
    \text{Hom}_{\text{PAb}(\mathcal{C})}\left(\mathbf{Z}_\mathcal{U}, \mathcal{F}\right)=
    \text{Hom}_{\textit{PSh}(\mathcal{C})}(h_\mathcal{U}, \mathcal{F}) = \mathcal{F}(\mathcal{U}).
$$
  In particular, we have the following result.
  
  \begin{lemma}
  The \u Cech complex $\check{\mathcal{C}}^\bullet(\mathcal{U}, \mathcal{F})$ can be described explicitly as follows
\begin{eqnarray*}
   \check{\mathcal{C}}^\bullet(\mathcal{U}, \mathcal{F})  
   & = &
   \left(\text{pr}od_{i_0\in I}\text{Hom}_{\text{PAb}(\mathcal{C})}(\mathbf{Z}_{\mathcal{U}_0}, \mathcal{F})\to \text{pr}od_{i_0, i_1 \in I} \text{Hom}_{\text{PAb}(\mathcal{C})}\left(\mathbf{Z}_{\mathcal{U}_0\times_\mathcal{U} \mathcal{U}_1}, \mathcal{F}\right) \to \cdots\right)\\ 
   & = & \text{Hom}_{\text{PAb}(\mathcal{C})} \left(\left(\displaystyle \bigoplus_{i_0\in I} \mathbf{Z}_{\mathcal{U}_{i_0}}\leftarrow \bigoplus_{i_0, i_1\in I} \mathbf{Z}_{\mathcal{U}_{i_0}\times_\mathcal{U} \mathcal{U}_{i_1}} \leftarrow \cdots\right), \mathcal{F}\right) \\ 
   & = & \text{Hom}_{\text{PAb}(\mathcal{C})}\left(\left(\mathbf{Z}_{(\coprod_{i_0\in I} \mathcal{U}_{i_0})}\leftarrow \mathbf{Z}_{(\coprod_{i_0, i_1\in I} \mathcal{U}_{i_0}\times_\mathcal{U} \mathcal{U}_{i_1})}\leftarrow \cdots\right),\mathcal{F}\right) 
\end{eqnarray*}
  \end{lemma}
  
  This reduces us to studying only the complex in the first argument of the last $\text{Hom}$.
  
 \begin{lemma}
 The complex of abelian presheaves
 \begin{align*}
 \mathbf{Z}_{\mathcal{U}}^\bullet :  \quad \mathbf{Z}_{(\coprod_{i_0} \mathcal{U}_{i_0})} \leftarrow \mathbf{Z}_{(\coprod_{i_0, i_1\in I} \mathcal{U}_{i_0}\times_\mathcal{U} \mathcal{U}_{i_1})}\leftarrow \cdots 
 \end{align*}
 is exact in negative degrees (in $\text{PAb}(\mathcal{C})$).
  \end{lemma}
  
\begin{proof}
For any $\mathcal{V}\in \text{Ob}(\mathcal{C})$ the complex $\mathbf{Z}_{\mathcal{U}}^\bullet(\mathcal{V})$ is
$$
\begin{array}{cl}
&
\displaystyle
\mathbf{Z}\left[\coprod_{i_0\in I} \text{Hom}_{\mathcal{C}}(\mathcal{V}, \mathcal{U}_{i_0}) \right]\leftarrow \mathbf{Z}\left[\coprod_{i_0, i_1 \in I}\text{Hom}_{\mathcal{C}}(\mathcal{V}, \mathcal{U}_{i_0}\times_\mathcal{U} \mathcal{U}_{i_1})\right]\leftarrow \cdots\\ 
 = &
 \displaystyle \bigoplus_{\varphi: \mathcal{V} \to \mathcal{U}} \left( \mathbf{Z}\left[\coprod_{i_0 \in I}\text{Hom}_\varphi(\mathcal{V},  \mathcal{U}_{i_0})\right]\leftarrow \mathbf{Z}\left[\coprod_{i_0, i_1\in I} \text{Hom}_{\varphi}(\mathcal{V}, \mathcal{U}_{i_0})\times \text{Hom}_{\varphi}(\mathcal{V}, \mathcal{U}_{i_1})\right]\leftarrow \cdots\right) 
\end{array}
$$
where 
$\text{Hom}_{\varphi}(\mathcal{V}, \mathcal{U}_i)= \{\mathcal{V}\to \mathcal{U}_i \ | \ \mathcal{V}\to \mathcal{U}_i \to \mathcal{U} = \varphi \}$. Set $S_\varphi=\coprod_{i\in I} \text{Hom}_{\varphi}(\mathcal{V}, \mathcal{U}_i)$, so that 
$$
\mathbf{Z}_\mathcal{U}^\bullet [\mathcal{V}] = \bigoplus_{\varphi: \mathcal{V} \to \mathcal{U}} \left( \mathbf{Z}[S_\varphi] \leftarrow \mathbf{Z}[S_\varphi\times S_\varphi] \leftarrow \mathbf{Z}[S_\varphi\times S_\varphi\times S_\varphi] \leftarrow\cdots \right).
$$
Thus it suffices to show that for each $S = S_\varphi$, the complex  
\begin{align*}
\mathbf{Z}[S] \leftarrow \mathbf{Z}[S\times S] \leftarrow \mathbf{Z}[S\times S\times S] \leftarrow\cdots
\end{align*}
is exact in negative degrees.  To see this, we can give an explicit homotopy. Fix $s\in S$ and define $K:    n_{(s_0, \ldots, s_p)} \mapsto n_{(s, s_0, \ldots, s_p)}.$ One easily checks that $K$ is a nullhomotopy for the operator
    $$\delta: \eta_{(s_0,\ldots,s_p)} \mapsto 
    \sum_{i=0}^p (-1)^p \eta_{(s_0,\ldots, \hat s_i,\ldots, s_p)}.$$
  \end{proof}
  
  \begin{lemma} \label{lem:HomInjSheafIsExact}
 Let $\mathcal{C}$ be a category. If $\mathcal{I}$ is an injective object of $\text{PAb}(\mathcal{C})$ and $\mathcal{U}$ is a family of morphisms with fixed target in $\mathcal{C}$, then $\check H^p(\mathcal{U}, \mathcal{I}) = 0$ for all $p>0$. 
  \end{lemma}
  
  \begin{proof}
  The \u Cech complex is the result of applying the functor $\text{Hom}_{\text{PAb}(\mathcal{C})}(-, \mathcal{I}) $ to the complex $ \mathbf{Z}^\bullet_\mathcal{U} $, {\it i.e.}
    $$
    \check H^p(\mathcal{U} ; \mathcal{I}) = H^p (\text{Hom}_{\text{PAb}(\mathcal{C})} (\mathbf{Z}^\bullet_\mathcal{U}, \mathcal{I})).
    $$
 But we have just seen that $\mathbf{Z}^\bullet_\mathcal{U}$ is exact in negative degrees, and the functor $\text{Hom}_{\text{PAb}(\mathcal{C})}(-, \mathcal{I})$ is exact, hence $\text{Hom}_{\text{PAb}(\mathcal{C})} (\mathbf{Z}^\bullet_\mathcal{U}, \mathcal{I})$ is exact in positive degrees.
  \end{proof}
  
  \begin{theorem} 
  On $\text{PAb}(\mathcal{C})$ the functors $\check{H}^p(\mathcal{U}, -)$ are the right derived functors of $\check{H}^0(\mathcal{U}, -)$.  
  \end{theorem}
  
  \begin{proof}
 By the lemma \ref{lem:HomInjSheafIsExact}, the functors $\check H^p(\mathcal{U}, -)$ are universal $\delta$-functors since they are effaceable.  So are the right derived functors of $\check H^0(\mathcal{U}, -)$.  Since they agree in degree $0$, they agree by the universal property of universal $\delta$-functors.
  \end{proof}

\begin{remark}
Observe that all of the preceding statements are about presheaves so we haven't made use of the topology yet. 
\end{remark}

\subsection{The \u Cech-to-cohomology Spectral Sequence}

\begin{theorem} \label{thm:SpectralSequenceCechToCohom}
Let $\mathcal{C}$ be a site. For any covering $\mathcal{U}= \left\{\mathcal{U}_i\to \mathcal{U}\right\}_{i\in I}$ of  $\mathcal{U}\in \text{Ob}(\mathcal{C})$ and any abelian sheaf $\mathcal{F}$ on $\mathcal{C}$ there is a spectral sequence 
  $$
 E_2^{p, q}
  =
 \check H^p(\mathcal{U},\underline{H}^q(\mathcal{F})) \Rightarrow H^{p+q}(\mathcal{U}, \mathcal{F}),
$$
    where $\underline{H}^q(\mathcal{F})$ is the abelian presheaf  $\mathcal{U}\mapsto H^q(\mathcal{U}, \mathcal{F})$.  
  \end{theorem}
  
  \begin{proof}
    Choose an injective resolution $\mathcal{F}\to \mathcal{I}^\bullet$ in $\text{Ab}(\mathcal{C})$, and consider the double complex
$$
\xymatrix{
{\Gamma(\mathcal{U},  I^\bullet)}  \ar[r] & {\check{\mathcal{C}}^\bullet(\mathcal{U}, \mathcal{I}^\bullet)} \\ 
& {\check{\mathcal{C}}^\bullet(\mathcal{U}, \mathcal{F})} \ar[u]
}
$$
where the horizontal map is the natural map $\Gamma(\mathcal{U},  I^\bullet) \to \check{\mathcal{C}}^0(\mathcal{U}, \mathcal{I}^\bullet)$ to the left column, and the vertical map is induced by  $\mathcal{F}\to \mathcal{I}^0$ and lands in the bottom row. By assumption, $\mathcal{I}^\bullet$ is injective in $\text{Ab}(\mathcal{C})$, hence by lemma \ref{lem:ForgetInjectivesAreInjectives} below, it is injective in $\text{PAb}(\mathcal{C})$. Thus, the rows of the double complex are exact in positive degrees, and the kernel of the horizontal map is equal to $\Gamma(\mathcal{U}, \mathcal{I}^\bullet)$, since $\mathcal{I}$ is a complex of sheaves. In particular, the cohomology of the total complex is the standard cohomology of the global sections functor $H^0(\mathcal{U}, \mathcal{F})$. 
\\
For the vertical direction, the $q$th cohomology group of the $p$th column is 
$$ 
\text{pr}od_{i_0,\ldots, i_p}  H^q(\mathcal{U}_{i_0}\times_\mathcal{U} \cdots \times_\mathcal{U} \mathcal{U}_{i_p} ; \mathcal{F}) =\text{pr}od_{i_0, \dots, i_p}\underline{H}^q(\mathcal{U}_{i_0}\times_\mathcal{U} \cdots \times \mathcal{U}_{i_p})
$$
  in the entry $E_1^{p,q}$. So this is a standard double complex spectral sequence, and the $E_2$-page is as prescribed.
  \end{proof}
  
  \begin{remark}
  This is a Grothendieck spectral sequence for the functors $\text{Ab}(\mathcal{C}) \to \text{PAb}(\mathcal{C})$ and $\check H^0$.
  \end{remark}
  
  \begin{lemma} \label{lem:ForgetInjectivesAreInjectives}
The forgetful functor $\text{Ab}(\mathcal{C})\to \text{PAb}(\mathcal{C})$ transforms injectives into injectives.  
  \end{lemma}
  
\begin{proof}
This is formal using the fact that the forgetful functor has a left adjoint, namely sheafification, which is an exact functor. 
\end{proof}

\subsection{Cohomology of Quasi-coherent Sheaves} % (9.22.09)

\label{section:CechCohomology}

\begin{lemma} \label{lem:CechComplexes}
Let $\mathcal{U}$ be a scheme and $\mathcal{U} = \{ \mathcal{U}_i \to \mathcal{U} \}_{i \in I}$ an $fpqc$-covering of $\mathcal{U}$. Let $\mathcal{V} = \coprod_{i \in I} \mathcal{U}_i$. Then
\begin{enumerate}
\item 
$\mathcal{V} = \{ \mathcal{V} \to \mathcal{U} \}$ is an $fpqc$-covering ; and
\item
the \u Cech complexes $\check{\mathcal{C}}^\bullet (\mathcal{U}, \mathcal{F})$ and $\check{\mathcal{C}}^\bullet (\mathcal{V}, \mathcal{F})$ agree whenever $\mathcal{F}$ is an abelian sheaf.
\end{enumerate} 
\end{lemma}

The proof is straightforward by unwinding the definitions and observing that if $\mathcal{F}$ is a sheaf and $\{ T_j \}_{j \in J}$ is a family of schemes, then the family of morphisms with fixed target $\{ T_i \to \coprod_{j \in J} T_j \}_{i \in J}$ is an $fpqc$-covering, and so
$$
\mathcal{F} \left( \coprod_{j \in J} T_j \right) = \text{pr}od_{j \in J} \mathcal{F} (T_j).
$$
Note that this equality is false for a presheaf. It does not hold on any site, but it does for all the usual ones (at least all the ones we will study).

\begin{remark}
In the statement of the lemma, $\mathcal{U}$ is a refinement of $\mathcal{V}$, so this does not mean that it suffices to look at coverings with a single morphism to compute \u Cech cohomology.
\end{remark}

\begin{lemma}[Locality of cohomology] \label{lem:LocOfCohomology}
Let $\mathcal{C}$ be a site, $\mathcal{F}$ an abelian sheaf on $\mathcal{C}$, $\mathcal{U}$ an object of $\mathcal{C}$, $p >0$ an integer and $\xi \in H^p(\mathcal{U}, \mathcal{F})$. Then there exists a covering $\mathcal{U} = \{ \mathcal{U}_i \to \mathcal{U} \}_{i \in I}$ of $\mathcal{U}$ in $\mathcal{C}$ such that $\xi |_{\mathcal{U}_i} = 0$ for all $i \in I$.
\end{lemma}

\begin{proof}
Choose an injective resolution $\mathcal{F} \to \mathcal{I}^\bullet$. Then $\xi$ is represented by a cocycle $\tilde{\xi} \in \mathcal{I}^p(\mathcal{U})$ with $d^p(\tilde{\xi}) = 0$. By assumption, the sequence $\mathcal{I}^{p-1} \to \mathcal{I}^{p} \to \mathcal{I}^{p+1}$ in exact in $\text{Ab}(\mathcal{C})$, which means that there exists a covering $\mathcal{U} = \{ \mathcal{U}_i \to \mathcal{U} \}_{i \in I}$ such that $\tilde{\xi}|_{\mathcal{U}_i} = d^{p-1}(\xi_i)$ for some $\xi_i \in \mathcal{I}^{p-1}(\mathcal{U}_i)$. Since the cohomology class $\xi|_{\mathcal{U}_i}$ is represented by the cocycle $\tilde{\xi}|_{\mathcal{U}_i}$ which is a coboundary, it vanishes.
\end{proof}

\begin{theorem} \label{thm:ZarIsFpqcForQCoh}
Let $S$ be a scheme and $\mathcal{F}$ a quasi-coherent $\mathcal{O}_S$-module. Then 
$$
H_{Zar}^p(S, \mathcal{F}) = H^p(S, \mathcal{F}_fpqc) \quad \text{ for all } p\geq 0.
$$
\end{theorem}

\begin{remark}
Since $S$ is a final object in the category $\textit{Sch}S$, the cohomology groups on the right-hand side are merely the right derived functors of the global sections functor.
\end{remark}

\begin{proof}
The result is true for $p=0$ by the sheaf property. We only prove the result for $S$ separated, and we start with the case of an affine scheme.
\begin{enumerate}
\item
Assume that $S$ is affine and that $\mathcal{F}$ is a quasi-coherent sheaf on $S$. Then we want to prove that $H^p(S, \mathcal{F}_fpqc) = 0$ for all $p>0$. We use induction on $p$.
\begin{description}
\item[$p=1$.]
Pick $\xi \in H^1(S, \mathcal{F}_fpqc)$. By lemma \ref{lem:LocOfCohomology}, there exists an $fpqc$ covering $\mathcal{U} = \{ \mathcal{U}_i \to S \}_{i \in I}$ such that $\xi|_{\mathcal{U}_i} = 0$ for all $i \in I$. Up to refining $\mathcal{U}$, we may assume that each $\mathcal{U}_i$ is affine and $I$ is finite. Applying the spectral sequence \ref{thm:SpectralSequenceCechToCohom}, we see that $\xi$ comes from a cohomology class $\check{\xi} \in \check H^1(\mathcal{U}, \mathcal{F}_fpqc)$. Consider the covering $\mathcal{V} = \{ \coprod_{i\in I} \mathcal{U}_i \to S\} = \{ \text{Spec} B \to \text{Spec} A \}$, then by lemma \ref{lem:CechComplexes}, $\check H^\bullet(\mathcal{U}, \mathcal{F}_fpqc) = \check H^\bullet(\mathcal{V}, \mathcal{F}_fpqc)$. On the other hand, the \u Cech complex $\check{\mathcal{C}}^\bullet (\mathcal{V}, \mathcal{F})$ is none other than the complex $(B/A)_\bullet \otimes_A M$ where $\mathcal{F} = \widetilde{M}$. Now by lemma \ref{lem:descentForModules}, $H^p((B/A)_\bullet \otimes_A M) = 0$ for $p>0$, hence $\check{\xi} = 0$ and so $\xi = 0$.
\item[$p>1$.]
Observe that the intersections $\mathcal{U}_{i_0} \times_S \cdots \times_S \mathcal{U}_{i_p}$ are affine, so that by induction hypothesis the cohomology groups 
$$
E_2^{p,q} = \check H^p(\mathcal{U}, \underline{H}^q (\mathcal{F}_fpqc))
$$ 
vanish for all $0 < q < p$. Now the same argument as above works: if $\xi \in H^p(S, \mathcal{F}_fpqc)$, we can find a finite covering $\mathcal{U}$ of $S$ by affines for which $\xi$ is locally trivial. Using the spectral sequence \ref{thm:SpectralSequenceCechToCohom} and the induction hypothesis, we see that $\xi$ must come from a $\check{\xi} \in \check H^p(\mathcal{U}, \mathcal{F}_fpqc)$. Replacing $\mathcal{U}$ with the covering $\mathcal{V}$ containing only one morphism and using lemma \ref{lem:descentForModules} again, we see that the \u Cech cohomology class $\check{\xi}$ must be zero, hence $\xi = 0$.
\end{description}
\item
Assume that $S$ is separated. Choose an affine open covering $S = \cup_{i \in I} \mathcal{U}_i$ of $S$. The family $\mathcal{U} = \{ \mathcal{U}_i \to S \}_{i \in I}$ is then an $fpqc$-covering, and all the intersections $\mathcal{U}_{i_0} \times_S \cdots \times_S \mathcal{U}_{i_p}$ are affine since $S$ is separated. So all rows of the spectral sequence \ref{thm:SpectralSequenceCechToCohom} are zero, except the zeroth row. Therefore
$$
H^p(S, \mathcal{F}_fpqc) = \check H^p(\mathcal{U}, \mathcal{F}_fpqc) = \check H^p(\mathcal{U}, \mathcal{F}) = H^p(S, \mathcal{F})
$$
where the last equality results from standard scheme theory.
\end{enumerate}
The general case is technical and requires a discussion about maps of spectral sequences, so we won't treat it.
\end{proof}

\begin{definition}
The sheaf $T \mapsto \Gamma(T, \mathcal{O}_T)$ is denoted $\mathbf{G}_a$, regardless of the site on which it is considered. Its usual restriction on a scheme $S$ (in other words, $\mathcal{O}_S$) is sometimes denoted $\mathbf{G}_{a,S}$. Similarly, the sheaf $T \mapsto \Gamma(T, \mathcal{O}^*_T)$ is denoted $\mathbf{G}_m$.
The \emph{constant sheaf} $\underline{\mathbf{Z}/n\mathbf{Z}}$ on any site is the sheafification of the constant presheaf $\mathcal{U} \mapsto \mathbf{Z}/n\mathbf{Z}$ with restriction maps the identity.
\end{definition}

\begin{remark}
If $\mathcal{C}$ is the small \'etale site over $S$ (defined later) and $\mathcal{U} \to S$ is \'etale, then $\Gamma(\mathcal{U}, \underline{\mathbf{Z}/n\mathbf{Z}})$ is the set of Zariski locally constant functions from $\mathcal{U}$ to $\mathbf{Z}/n\mathbf{Z}$.
\end{remark}

\begin{remark}
A special case of theorem \ref{thm:ZarIsFpqcForQCoh} is $H_{Zar}^p (X, \mathcal{O}_X) = H_{fpqc}^p(X, \mathbf{G}_a)$ for all $p \geq 0$.
\end{remark}

\subsection{Picard Groups}

\begin{theorem}
Let $X$ be a scheme. Then $H_{fpqc}^1(X ; \mathbf{G}_m) = \text{Pic} (X) = H_{Zar}^1(X; \mathcal{O}_X^*)$.
\end{theorem}

\begin{proof}[Sketch of proof.]
Arguing as above, one shows that
$$
H_{fpqc}^1(X ; \mathbf{G}_m) = \text{colim}_{\mathcal{U} \in \text{Cov}(\textit{Sch}_{/X,fpqc})} \check H^1(\mathcal{U}, \mathbf{G}_m).
$$
Given an $fpqc$-covering $\{ \mathcal{U}_i \to \mathcal{U} \}_{i \in I}$ and a \u Cech cocycle $(f_{ij})_{i,j \in I}$, $f_{ij} \in \Gamma(\mathcal{U}_i \times_X \mathcal{U}_j; \mathbf{G}_m)$, we get a descent datum
$$
\left(\mathcal{O}_{\mathcal{U}_i}, f_{ij} |_{\mathcal{U}_i \times_X \mathcal{U}_j}: \text{pr}_0^*(\mathcal{O}_{\mathcal{U}_i}) \to \text{pr}_1^*(\mathcal{O}_{\mathcal{U}_j})\right)
$$
for coherent sheaves, which is effective by theorem \ref{thm:DescentIsEffectiveForQCoh}, hence we can descend line bundles.
\end{proof}

\section{The \'Etale Site}
\subsection{\'Etale Morphisms}

For more details, see the section on \'etale morphisms in \cite{Stacks}.

\begin{definition}
A morhism of schemes is \emph{\'etale} if it is smooth of relative dimension 0. Recall that a morphism of algebras over an algebraically closed field is \emph{smooth} if it is of finite type and the sheaf of differentials is locally free of rank equal to the dimension. A morphism of schemes is \emph{smooth} if it is flat, finitely presented, and the geometric fibers are smooth. A ring map $A \to B$ is said to be \emph{\'etale at a prime $\frak q$} of $B$ if there exists $h \in B$, $h \notin \frak q$ such that $A \to B_{\frak q}$ is \'etale.
\end{definition}

\begin{proposition} $ $ \label{prop:ofEtaleMorphisms}
\begin{enumerate}
\item
Let $k$ be a field. A morphism of schemes $\mathcal{U} \to \text{Spec} k$ is \'etale if and only if $\mathcal{U} = \coprod_{(i)} \text{Spec} k_i$ where for each $i$, $k_i$ is a finite separable extension of $k$.
\item
Let $\varphi : \mathcal{U} \to S$ be a morphism of schemes. The following conditions are equivalent:  
\begin{itemize}
\item $\varphi$ is \'etale,
\item $\varphi$ is locally finitely presented, flat, and all its fibres are \'etale,
\item $\varphi$ is flat and unramified.
\end{itemize}
\item
A ring map $A \to B$ is \'etale if and only if $B \cong_A A[x_1, \dots,x_n]/(f_1,\dots,f_n)$ such that $\Delta = \det \left( \frac{\partial f_i}{\partial x_j} \right) $ is invertible in $B$.
\item
The base change of an \'etale morphism is \'etale.
\item
An \'etale morphism has relative dimension 0.
\item
Let $Y \to X$ be an \'etale morphism. If  $X$ is reduced (respectively smooth) then so is $Y$.
\item 
Etale morphims are open.
\item
If $X\to S$ and $Y\to S$ are \'etale, then any $S$-morphism $X \to Y$ is also \'etale.
\end{enumerate}
\end{proposition}

\begin{definition}
A ring map $A \to B$ is called \emph{standard \'etale} if $B \cong_A \left( A[t]/(f(t)) \right)_{g(t)}$ with $f(t)$ monic and $\frac{\mathrm{d}f}{\mathrm{d}t}$ invertible in $B$.
\end{definition}

\begin{theorem}
A ring map $A \to B$ is \'etale at a prime $\frak q$ if and only if there exists $g \in B$, $g \notin \frak q$ such that $B_{\frak g}$ is standard \'etale over $A$.
\end{theorem}

\subsection{\'Etale Coverings}

\begin{definition}
An \emph{\'etale covering} of a scheme $\mathcal{U}$ is a family of morphisms of schemes with fixed target $\{ \varphi_i : \mathcal{U}_i \to \mathcal{U} \}_{i \in I}$ such that 
\begin{enumerate}
\item each $\varphi_i$ is an \'etale morphism ;
\item the $\mathcal{U}_i$ cover $\mathcal{U}$, {\it i.e.} $\mathcal{U} = \cup_{i\in I}\varphi_i(\mathcal{U}_i)$.
\end{enumerate}
\end{definition}

\begin{lemma}
Any \'etale covering is an $fpqc$-covering. 
\end{lemma}

\begin{proof}
Since an \'etale morphism is flat, and the elements of the covering should cover its target, the property fp (faithfully flat) is satisfied. To check qc (quasi-compact), let $\mathcal{V} \subseteq \mathcal{U}$ be an affine open, and write $\varphi_i^{-1} = \cup_{j \in J_i} \mathcal{V}_{ij}$ for some affine opens $\mathcal{V}_{ij} \subseteq \mathcal{U}_i$. Since $\varphi_i$ is open (\'etale morphisms are open), we see that $\mathcal{V} = \cup_{i\in I}\cup_{j \in J_i} \varphi_i(\mathcal{V}_{ij})$ is an open covering of $\mathcal{U}$. Further, since $\mathcal{V}$ is quasi-compact, this covering has a finite refinement.
\end{proof}

So any statement which is true for $fpqc$-coverings (respectively sheaves etc) remains true {\it a fortiori} for \'etale coverings (resp. sheaves etc). For instance, the \'etale site is subcanonical.

\begin{definition}
Let $S$ be a scheme. The \emph{big \'etale site over $S$} is the site $\textit{Sch}Set$ defined by the \'etale coverings on the category $\textit{Sch}S$ of schemes over $S$. The \emph{small \'etale site over $S$} is the site $S_{et}$ defined by the \'etale coverings on the full subcategory of $\textit{Sch}S$ which objects are morphisms $\mathcal{U} \to S$. We define similarly the \emph{big} and \emph{small Zariski sites} on $S$, $\textit{Sch}S$ and $S_\mathrm{Zar}$. Note that there is no notion of a small $fpqc$ site.
\end{definition}

The small \'etale site has fewer objects than the big \'etale site, it contains only the ``opens'' of the \'etale topology on $S$. It is a full subsite of the big \'etale site, and it is true that the restriction functor from the big \'etale site to the small one is exact and maps injectives to injectives. This has the following consequence.

\begin{proposition}
Let $S$ be a scheme and $\mathcal{F}$ an abelian sheaf on $\textit{Sch}Set$. Then $\mathcal{F}|_{S_{et}}$ is a sheaf on $S_{et}$ and 
$$
H^p(S_{et}, \mathcal{F}|_{S_{et}}) = H^p(S, \mathcal{F}) 
$$
for all $p \geq 0$. 
\end{proposition}

We write $H_{et}^p(S,\mathcal{F})$ for the above cohomology group.

% (9.24.09)
\subsection{Kummer Theory}
Let $n \in \mathbf{N}$ and consider the presheaf $\mu_n$ defined by
$$
\begin{array}{ccl}
\textit{Sch}^{opp} & \longrightarrow  & \text{Ab} \\
S & \longmapsto &  \mu_n(T) =  \left\{t \in \Gamma(S, \mathcal{O}_S^*) \  | \ t^n = 1 \right\}.
\end{array}
$$
This presheaf is an $fpqc$ sheaf (in particular, it is also an \'etale sheaf, Zariski sheaf, etc.) and it is representable by the group scheme $\text{Spec} (\mathbf{Z}[t]/(t^n-1))$. 

\begin{lemma} \label{lem:KummerSequence}
If $n\in \mathcal{O}_S^*$ then 
$$
0\to \mu_{n, S} \to \mathbf{G}_{m, S} \xrightarrow{(\cdot)^n} \mathbf{G}_{m, S}\to 0
$$
is a short exact sequence of sheaves on the small \'etale site of  $S$. 
\end{lemma}

\begin{remark}
This lemma is false when $S_{et}$ is replaced with $S_\mathrm{Zar}$.
\end{remark}

\begin{proof}
The only nontrivial step is to show that the last map is surjective. Let $\mathcal{U} \to S$ be an \'etale map and $f \in \mathbf{G}_m(\mathcal{U}) = \Gamma(\mathcal{U}, \mathcal{O}_\mathcal{U}^*)$. We need to show that we can find a cover of $\mathcal{U}$ in $S_{et}$ over which the restriction of $f$ is an $n$th power. Set
$$
\mathcal{U}' = \underline{\text{Spec}} (\mathcal{O}_\mathcal{U}[T] / (T^n-f)) \xrightarrow{\pi} \mathcal{U}.
$$ 
Locally,  the map $\pi$ has the form $A \to B = A[T] / (T^n-a)$ for some $a \in A^*$ and $n \in A^*$. Since it is an injective integral ring map, $\pi$ is surjective. In addition, $n$ and $T^{n-1}$ are invertible in $B$, so $nT^{n-1} \in B^*$ and the ring map $A \to B$ is \'etale. Hence $\mathcal{U} = \{\pi : \mathcal{U}' \to \mathcal{U}\}$ is an \'etale covering. Moreover, $f|_{U'} = (f')^n$ where $f'$ is the class of $T$ in $\Gamma(\mathcal{U}', \mathcal{O}_{\mathcal{U}'}^*)$, so $\mathcal{U}$ has the desired property.  
\end{proof}

Lemma \ref{lem:KummerSequence} gives the long exact cohomology sequence
$$
\xymatrix{
0  \ar[r] & H_{et}^0(S, \mu_n) \ar[r] & \Gamma(S, \mathcal{O}_S^*) \ar^{(\cdot)^n}[r] & \Gamma(S, \mathcal{O}_S^*) 
\ar@(rd,ul)[rdllllr]
\\ 
& H_{et}^1(S, \mu_n) \ar[r] & \text{Pic}(S) \ar^{(\cdot)^n}[r] & \text{Pic} (S) \ar@(rd,ul)[rdllllr] \\
& H_{et}^2(S, \mu_n) \ar[r] & \cdots 
}
$$
It is also true that
$$
H_{et}^1(S, \mu_n) = \left\{(\mathcal{L}, \alpha)\; 
\left| \; 
\begin{array}{c} 
\mathcal{L}\text{ invertible sheaf on } S  \\
\text{and }\alpha: \mathcal{L}^{\otimes n} \cong \mathcal{O}_S
\end{array}
\right.
\right\}
\Big/\cong.
$$

\subsection{Neighborhoods, Stalks and Points}

\begin{definition}
Let $S$ be a scheme. A \emph{geometric point} of $S$ is a morphism $\text{Spec} k = \bar s \to S$ (usually denoted $\bar s$ again)  where $k$ is separably closed. An \emph{\'etale neighborhood} of $\bar s$ is a commutative diagram
$$
\xymatrix{
& U \ar^{\varphi}[d] \\
{\bar s} \ar^{\bar s}[r] \ar^{\bar u}[ur] & S 
}
$$
where $\varphi$ is \'etale. We write $(\mathcal{U}, \bar u)\to (S, \bar s)$. A \emph{morphism} of \'etale neighborhoods $(\mathcal{U}, \bar u)\to (\mathcal{U}',\bar u')$ is an $S$-morphism $h: \mathcal{U}\to \mathcal{U}'$ such that $\bar u'=h\circ\bar u$. 
\end{definition}

\begin{remark}
Since $\mathcal{U}$ and $\mathcal{U}'$ are \'etale over $S$, any $S$-morphism between them is also \'etale. In particular all morphisms of \'etale neighborhoods are \'etale.
\end{remark}

\begin{lemma} $ $ \label{lem:CofinalityOfEtaleNbhds}
\begin{enumerate}
\item Let $(\mathcal{U}_i, \bar u_i)_{i=1, 2}$ be two \'etale neighborhoods of $\bar s$ in $S$. Then there exists a third \'etale neighborhood $(\mathcal{U}, \bar u)$ and morphims $(\mathcal{U}, \bar u) \to (\mathcal{U}_i, \bar u_i)_{i=1,2}$.
\item Let $h_1, h_2: (\mathcal{U}, \bar u) \to (\mathcal{U}', \bar u')$ be two morphisms between \'etale neighborhoods of $\bar s$. Then there exist an \'etale neighborhood and a morphism $h : (\mathcal{U}'', \bar u'')\to (\mathcal{U}, \bar u)$ which equalizes $h_1$ and $h_2$, {\it i.e.}  such that $h_1\circ h = h_2\circ h$.		
\end{enumerate}
\end{lemma}

In other words, the category of \'etale neighborhoods is codirected.

\begin{proof} $ $
For part {\it i}, consider the fibre product $\mathcal{U} = \mathcal{U}_1 \times_S \mathcal{U}_2$. It is \'etale over both $\mathcal{U}_1$ and $\mathcal{U}_2$ and the map $\bar s \to U$ defined by $(\bar u_1, \bar u_2)$ gives it the structure of an \'etale neighborhood mapping to both $\mathcal{U}_1$ and $\mathcal{U}_2$. For part {\it ii}, define $\mathcal{U}''$ as the fibre product 
$$
\xymatrix{
{\mathcal{U}''} \ar[r] \ar[d] & {\mathcal{U}} \ar^{(h_1,h_2)}[d] \\
{\mathcal{U}'} \ar^{\Delta \quad}[r] & {\mathcal{U}' \times_S \mathcal{U}'}
}
$$
and let $\bar u'' = (\bar u, \bar u')$. Since $\bar u$ and $\bar u'$ agree over $S$ with $\bar s$, we see that $\mathcal{U}''\neq\emptyset$. Moreover, since $\mathcal{U}$ and $\mathcal{U}'$ are \'etale over $S$, so is the fibre product $\mathcal{U}'\times_S \mathcal{U}'$ and therefore $\mathcal{U}''$ by base change.
\end{proof}

\begin{lemma} \label{lem:geomPointsAreInSomeOpen}
Let $\bar s$ be a geometric point of $S$, $(\mathcal{U}, \bar u)$ an \'etale neighborhood of $\bar s$, and $\mathcal{U} = \left\{\varphi_i : \mathcal{U}_i \to \mathcal{U} \right\}_{i\in I}$ an \'etale covering. Then there exist $i\in I$ and $\bar u_i: \bar s  \to \mathcal{U}_i$ such that $\varphi_i: (\mathcal{U}_i, \bar u_i) \to (\mathcal{U}, \bar u)$ is a morphism of \'etale neighborhoods.  
\end{lemma}

\begin{proof}
As $\mathcal{U} = \bigcup_{i\in I}\varphi_i(\mathcal{U}_i)$, the fibre product $\bar s \times_{\bar u, \ \mathcal{U}, \varphi_i} \mathcal{U}_i$ is not empty for some $i$. Then look at the cartesian diagram
$$
\xymatrix{ 
{\bar s \times_{\bar u, \ \mathcal{U}, \varphi_i} \mathcal{U}_i} \ar^{\text{pr}_1}[d] \ar^{\qquad \text{pr}_2}[r] & {\mathcal{U}_i} \ar^{\varphi_i}[d] \\ 
{\text{Spec} k = \bar s} \ar@/^1pc/^{\sigma}[u] \ar^{\qquad \bar u}[r] & {\mathcal{U}.} 
}
$$
The projection $\text{pr}_1$ is the base change of an \'etale morphisms so it is \'etale. Therefore, $\bar s \times_{\bar u , \mathcal{U}, \varphi_i} \mathcal{U}_i$ is a disjoint union of finite separable extensions of $k$, where $\bar s = \text{Spec} k$. But $k$ is separably closed, so all these extensions are trivial, and there exists a section $\sigma$ of $\text{pr}_1$. The composition 
$\text{pr}_2 \circ \sigma$ gives a map compatible with $\bar u$.
\end{proof}

\begin{definition} \label{defi:EtaleLocalRings}
Let $S$ be a scheme, $\mathcal{F}$ a presheaf (or a sheaf) on $S_{et}$ and $\bar s$ a geometric point of $S$. The \emph{stalk} of $\mathcal{F}$ at $\bar s$ is  
$$
\mathcal{F}_{\bar s} = \text{colim}_{(\mathcal{U}, \bar u) \to (S,\bar s)} \mathcal{F}(\mathcal{U}) 
$$
where $(\mathcal{U}, \bar u) \to (S,\bar s)$ runs over all \'etale neighborhoods of $\bar s$ in $S$. By lemma \ref{lem:CofinalityOfEtaleNbhds}, this colimit is directed. We also define the \emph{\'etale local ring at $\bar s$} to be the stalk of the sheaf $\mathbf{G}_a$ at $\bar s$, that is
$$
\mathcal{O}_{S, \bar{s}}^\text{sh} := \text{colim}_{(\mathcal{U}, \bar u) \to (S,\bar s)} \Gamma(\mathcal{U}, \mathcal{O}_\mathcal{U}).
$$
\end{definition}

\begin{lemma}
The stalk functor 
$$
\begin{array}{rcl}
\text{PAb}(S_{et}) & \longrightarrow & \text{Ab}\\ 
\mathcal{F} & \longmapsto & \mathcal{F}_{\bar s}
\end{array}
$$ 
is exact. Furthermore, $\left(\mathcal{F}^\sharp\right)_{\bar s} = \mathcal{F}_{\bar s}$ and hence it induces an exact functor $\text{Ab}(S_{et})\to \text{Ab}$.
\end{lemma}

\begin{proof}
Exactness as a functor from $\text{PAb}(S_{et})$ is formal from the fact that directed colimits commute with all colimits and with finite limits. The identification of the stalks is {\it via} the map
$$
\kappa : \mathcal{F}_{\bar s}\longrightarrow \left(\mathcal{F}^\sharp\right)_{\bar s}
$$
induced by the natural morphism $\mathcal{F}\to \mathcal{F}^\sharp$.  We claim that this map is an isomorphism of abelian groups.  We will show injectivity and omit surjectivity. 
\\
Let $\sigma\in \mathcal{F}_{\bar s}$. There exists an \'etale neighborhood $(\mathcal{U}, \bar u)\to (S, \bar s)$ such that $\sigma$ is the image of some section $s \in \mathcal{F}(U)$. If $\kappa(\sigma) = 0$ in $(\mathcal{F}^\sharp)_{\bar s}$ then there exists a morphism of \'etale neighborhoods $(\mathcal{U}', \bar u')\to (\mathcal{U}, \bar u)$ such that $s|_{\mathcal{U}'}$ is zero in $\mathcal{F}^\sharp(\mathcal{U}')$. Following, there exists an \'etale covering $\left\{\mathcal{U}_i'\to \mathcal{U}'\right\}_{i\in I}$ such that $s|_{\mathcal{U}_i'}=0$ in $\mathcal{F}(\mathcal{U}_i')$ for all $i$. By lemma \ref{lem:geomPointsAreInSomeOpen} there exist  $i \in I$ and a morphism $\bar u_i': \bar s \to \mathcal{U}_i'$ such that  $(\mathcal{U}_i', \bar u_i')\to (\mathcal{U}', \bar u')\to (\mathcal{U}, \bar u)$ are morphisms of \'etale neighborhoods. Hence for some $\bar u_i'$, we have $s|_{\mathcal{U}'_i}=0$, which implies $\sigma = 0$.  Therefore, $s|_{\mathcal{U}'_i} = 0 \in \mathcal{F}_{\bar s}$ and $\kappa$ is injective.
\\
To show that the functor $\text{Ab}(S_{et}) \to \text{Ab}$ is exact, consider any short exact sequence in $\text{Ab}(S_{et})$:
$
0\to \mathcal{F}\to \mathcal{G}\to \mathcal H \to 0.
$
This gives us the  exact sequence of presheaves
$$
0 \to \mathcal{F}\to \mathcal{G} \to \mathcal H\to \mathcal H/^p\mathcal{G} \to 0,
$$
where $/^p$ denotes the quotient in $\text{PAb}(S_{et})$. Taking stalks at $\bar s$, we see that $(\mathcal H /^p\mathcal{G})_{\bar{s}} = (\mathcal H /\mathcal{G})_{\bar{s}} = 0$, since the sheafification of $\mathcal H/^p\mathcal{G}$ is $0$. 
Therefore, 
$$
0\to \mathcal{F}_{\bar s	} \to \mathcal{G}_{\bar s} \to \mathcal{H}_{\bar s} \to 0 = (\mathcal H/^p\mathcal{G})_{\bar s}
$$
is exact, since taking stalks is exact as a functor from presheaves.
\end{proof}

\begin{theorem}
A sequence of abelian sheaves on $S_{et}$ is exact if and only if it is exact on stalks.  
\end{theorem}

\begin{proof}
The necessity of exactness on stalks was proven in the previous lemma. For the converse, it suffices to show that a map of sheaves is surjective (respectively injective) if and only if it is surjective (resp. injective) on all stalks.  We only treat the case of surjectivity.
\\
Let $\alpha : \mathcal{F} \to \mathcal{G}$ be a map of abelian sheaves such that $\mathcal{F}_{\bar s} \to \mathcal{G}_{\bar s}$ is surjective for all geometric points.  Fix $\mathcal{U}\in \text{Ob}(S_{et})$ and $s \in \mathcal{G}(\mathcal{U})$. For every $u\in \mathcal{U}$ choose some $\bar u\to \mathcal{U}$ lying over $u$ and an \'etale neighborhood $(\mathcal{V}_u , \bar v_u)\to (\mathcal{U}, \bar u)$ such that $s|_{\mathcal{V}_u}=\alpha(s_{\mathcal{V}_{u}})$ for some $s_{\mathcal{V}_u}\in \mathcal{F}(\mathcal{V}_u)$. This is possible since $\alpha$ is surjective on stalks. Then $\left\{\mathcal{V}_u\to \mathcal{U}\right\}_{u\in \mathcal{U}}$ is an \'etale covering on which the restrictions of $s$ are in the image of the map $\alpha$.  Thus, $\alpha$ is surjective.
\end{proof}

\subsection{Direct Images}

\begin{definition}
Let $f: X\to Y$ be a morphism of schemes, $\mathcal{F} $ a presheaf on $X_{et}$. The \emph{direct image} of $\mathcal{F}$ (under $f$) is 
$$
\begin{array}{rrcl}
f_*\mathcal{F} : &Y_{et}^{opp} & \longrightarrow & \textit{Sets} \\
& \left(V\to Y\right) & \longmapsto & \mathcal{F}\left(X\times_Y V\to X\right).
\end{array}
$$
This is a well-defined \'etale presheaf since the base change of an \'etale morphism is again \'etale, and defines a functor $f_* : \textit{PSh}(X_{et}) \to \textit{PSh}(Y_{et})$ since base change is a functor. 
\end{definition}

\begin{remark}
If $\left\{\mathcal{V}_i\to \mathcal{V}\right\}_{i\in I}$ is an \'etale covering in $Y_{et}$ then $\left\{X\times_Y \mathcal{V}_i\to X\times_Y \mathcal{V}\right\}$ is an \'etale covering in $X_{et}$. Hence the sheaf condition for $\mathcal{F}$ with respect to the latter is equivalent to the sheaf condition for $f_*\mathcal{F}$ with respect to the former. Thus if $\mathcal{F}$ is a sheaf, so is $ f_*\mathcal{F}$. 
\end{remark}

\begin{definition}
The previous functor therefore induces $f_*:\text{Ab}(X_{et})\to \text{Ab}(Y_{et})$, called \emph{direct image} again. It is left exact, and its right derived functors $\{R^pf_*\}_{p \geq 1}$ are called \emph{higher direct images}.  
\end{definition}
	
\subsection{Inverse Image}

\begin{definition}
Let $f: X\to Y$ be a morphism of schemes. The \emph{inverse image} functor  $f^{-1} : \textit{Sh}(Y_{et})\to \textit{Sh}(X_{et})$ (respectively $f^{-1}: \text{Ab}(Y_{et}) \to \text{Ab}(X_{et})$) is the left adjoint to $f_*$. It is thus characterized by the fact that 
$$
\text{Hom}_{{\textit{Sh}(X_{et})}} (f^{-1}\mathcal{G}, \mathcal{F}) = \text{Hom}_{\textit{Sh}(Y_{et})} (\mathcal{G}, f_*\mathcal{F}) 
$$
functorially, for any $\mathcal{F} \in \textit{Sh}(X_{et})$ (resp. $\text{Ab}(X_{et})$), $\mathcal{G} \in \textit{Sh}(Y_{et})$ (resp. $\text{Ab}(Y_{et})$).
\end{definition}

\begin{lemma} 
Let $f : X \to Y$ be a morphism of schemes, $\bar x \to X$ a geometric point and $\mathcal{G}$ a presheaf on $Y_{et}$. Then there is a canonical identification 
$$
\left(f^{-1}\mathcal{G}\right)_{\bar x} = \mathcal{G}_{f\circ \bar x}.
$$
Moreover, $f^{-1}$ is exact.
\end{lemma}

\begin{proof}
The exactness of $f^{-1}$ is a formal consequence of the first statement, the proof of which is omitted.
\end{proof}
	
\begin{remark}
More generally, let $\mathcal{C}_1, \mathcal{C}_2$ be sites, and assume they have final objects and fibre products.  Let  $u: \mathcal{C}_2 \to \mathcal{C}_1$ be a functor satisfying:
\begin{enumerate}
\item $u (\text{Cov}(\mathcal{C}_2)) \subseteq \text{Cov}(\mathcal{C}_1)$ (we say that $u$ is \emph{continuous}) ; and
\item $u$ commutes with finite limits ({\it i.e.} $u$ is left exact, {\it i.e.} $u$ preserves fibre products and final objects).
\end{enumerate}
Then one can define $f_*: \textit{Sh}(\mathcal{C}_1) \to \textit{Sh}(\mathcal{C}_2)$ by $ f_* \mathcal{F}(\mathcal{V}) = \mathcal{F}(u(\mathcal{V}))$. Moreover, there exists a functor $f^{-1}$ which is left adjoint to $f_*$ and is exact.
\\
To recover our definition, notice that a morphism of schemes $ f: X  \to Y$ induces a continous functor of sites $u_f: Y_{et} \to X_{et}$ {\it via} $u_f (V \to Y) = V \times_Y X \to X$.
\end{remark} % (9.29.09)
\section{\'Etale Cohomology}
\subsection{Colimits}

Let us start by recalling that if $\left\{\mathcal{F}_i\right\}_{i\in I}$ is a system of sheaves on $X$, its colimit (in the category of sheaves) is the sheafification of the presheaf $\mathcal{U} \mapsto \text{colim}_{i\in I} \mathcal{F}_i(\mathcal{U})$. In the case where $X$ is noetherian, the sheafification is superfluous. See \cite{Hartshorne}.  

\begin{theorem}
Let $X$ be a quasi-compact and quasi-separated scheme. Let $\left(\mathcal{F}_i, \varphi_{ij}\right)$ be a system of abelian sheaves on $X_{et}$ over the partially ordered set $I$. If $I$ is directed then
$$
\text{colim}_{i\in I} H_{et}^p(X, \mathcal{F}_i) = H_{et}^p(X, \text{colim}_{i\in I} \mathcal{F}_i).
$$
\end{theorem}

\begin{proof}[Sketch of proof.] 
This is proven for all $X$ at the same time, by induction on $p$. 
\begin{enumerate}
\item 
For any quasi-compact and quasi-separated scheme $X$ and any \'etale covering $\mathcal{U}$ of $X$, show that there exists a refinement $\mathcal{V} =\left\{\mathcal{V}_j \to X\right\}_{j\in J}$ with $J$ finite and each $V_j$ quasi-compact and quasi-separated such that all the $\mathcal{V}_{j_0} \times_X \cdots \times_X \mathcal{V}_{j_p}$ are also quasi-compact and quasi-separated. 
\item 
Using the previous step and the definition of colimits in the category of sheaves, show that the theorem holds for $p=0$, all $X$. (Exercise.)
\item 
Using the locality of cohomology (lemma \ref{lem:LocOfCohomology}), the \u Cech-to-cohomology spectral sequence (theorem \ref{thm:SpectralSequenceCechToCohom}) and the fact that the induction hypothesis applies to all $\mathcal{V}_{j_0}\times_X \cdots \times_X \mathcal{V}_{j_p}$ in the above situation, prove the induction step $p\to p+1$. 
\end{enumerate}
\end{proof}

\begin{theorem} \label{thm:directedColimitsAndCohomology}
Let $A$ be a ring, $(I, \leq)$ a directed poset and $(B_i, \varphi_{ij})$ a system of $A$-algebras. Set $B=\text{colim}_{i\in I} B_i$. Let $X \to \text{Spec} A$ be a quasi-compact and quasi-separated morphism of schemes and $\mathcal{F}$ an abelian sheaf on $X_{et}$. Denote $X_i = X\times_{\text{Spec} A} \text{Spec} B_i$,  $Y= X \times_{\text{Spec} A}\text{Spec} B$, $\mathcal{F}_i = (X_i\to X)^{-1}\mathcal{F}$ and $\mathcal{G} = (Y\to X)^{-1}\mathcal{F}$. Then
$$
H_{et}^p(Y, \mathcal{G}) = \text{colim}_{i\in I} H_{et}^p ((X_i), \mathcal{F}_i).
$$
\end{theorem}

%margin: 
%$Y=\text{lim}_{i\in I} X_i$ 
%$$
%H^p(\text{lim}_{\leftarrow}X_i, \mathbf{Z}/n\mathbf{Z})=\text{lim}_{\to}H^p(X_i, \mathbf{Z}/n\mathbf{Z})
%$$
%$$
%\begin{diagram}
%\node{X}\arrow{s}\node{\cdots}\node{X_i}\arrow{w}\arrow{s}\node{X_{i'}}\arrow{w}\arrow{s}\node{\cdots}\node{Y}\arrow{s}\\
%\node{\text{Spec}(A)}\node{\cdots}\node{\text{Spec}(B_i)}\arrow{w}\node{\text{Spec}(B_{i'})}\arrow{w}\node{\cdots}\node{\text{Spec}(B)}
%\end{diagram}
%$$

\begin{proof}[Sketch of proof.]$ $
\begin{enumerate}
\item 
Given $\mathcal{V}\to Y$ \'etale with $\mathcal{V}$ quasi-compact and quasi-separated, there exist $i\in I$ and $\mathcal{U}_i \to X_i$ such that $\mathcal{V} = \mathcal{U}_i \times_{X_i} Y$. 
\begin{remark}
If all the schemes considered were affine, this would correspond to the following algebra statement: if $B=\text{colim} B_i$ and $B\to C$ is \'etale, then there exist $i\in I$ and $B_i\to C_i$ \'etale such that $C \cong B \otimes_{B_i} C_i$. 
\\
This is proven as follows: write $C \cong B\left[x_1,\ldots, x_n\right]/(f_1, \ldots, f_n)$ with $\det (f_j(x_k)) \in C^*$ and pick $i\in I$ large enough so that all the coefficients of the $f_j$s lie in $B_i$, and let $C_i = B_i\left[x_1, \ldots, x_n\right]/(f_1, \dots, f_n)$. This makes sense by the assumption. After further increasing $i$, $\det (f_j(x_k))$ will be invertible in $C_i$, and $C_i$ will be \'etale over $B_i$. 
\end{remark}
\item 
By Step 1, we see that for every \'etale covering $\mathcal{V} = \left\{\mathcal{V}_j\to Y\right\}_{j\in J}$ with $J$ finite and the $\mathcal{V}_j$s quasi-compact and quasi-separated, there exists $i\in I$ and an \'etale covering $\mathcal{V}_i = \left\{\mathcal{V}_{ij} \to X_i \right\}_{j\in J}$ such that $\mathcal{V} \cong \mathcal{V}_i\times_{X_i} Y$. 
\item 
Show that Step 2 implies 
$$
\check H^*(\mathcal{V}, \mathcal{G})=\text{colim}_{i\in I}\check H^*(\mathcal{V}_i, \mathcal{F}_i).
$$ 
This is not clear, as we have not explained how to deal with $\mathcal{F}_i$ and $\mathcal{G}$, in particular with the dual.
\item 
Use the \u Cech-to-cohomology spectral sequence (theorem \ref{thm:SpectralSequenceCechToCohom}). 
\end{enumerate}
\end{proof}

\subsection{Stalks of Higher Direct Images}

\begin{lemma} \label{lem:higherDirectImagesPresheaf}
Let $f: X\to Y$ be a morphism of schemes and $\mathcal{F}\in \text{Ab}(X_{et})$. Then $R^pf_*\mathcal{F}$ is the sheaf associated to the presheaf
$$
(V\to Y)\longmapsto H_{et}^0 \left(X\times_Y V, \mathcal{F}|_{X\times_YV}\right).
$$
\end{lemma}

This lemma is valid for topological spaces, and the proof in this case is the same.

\begin{theorem} \label{thm:stalkOfHigherDirectImages}
Let $f: X\to S$ be a quasi-compact and quasi-separated morphism of schemes, $\mathcal{F}$ an abelian sheaf on $X_{et}$, and $\bar s$ a geometric point of $S$. Then
$$
\left(R^pf_* \mathcal{F}\right)_{\bar s} = H_{et}^p\left( X\times_S \text{Spec}(\mathcal{O}_{S, \bar s}^\mathrm{sh}), \text{pr}^{-1}\mathcal{F}\right)
$$
where $\text{pr}$ is the projection $X\times_S \text{Spec}(\mathcal{O}_{S, \bar{s}}^\mathrm{sh}) \to X$.
\end{theorem}

\begin{proof}
Let $\mathcal{I}$ be the category opposite to the category of \'etale neighborhoods of $\bar s$ on $S$. By lemma \ref{lem:higherDirectImagesPresheaf} we have
$$
\left(R^pf_*\mathcal{F}\right)_{\bar{s}} = \text{colim}_{(\mathcal{V}, \bar{v})\in \mathcal{I}} H^p(X\times_S\mathcal{V}, \mathcal{F}|_{X\times_S\mathcal{V}}).
$$
On the other hand, 
$$
\mathcal{O}_{S, \bar{s}}^\mathrm{sh} = \text{colim}_{(\mathcal{V}, \bar v)\in \mathcal{I}} \Gamma(\mathcal{V}, \mathcal{O}_\mathcal{V}).
$$
Replacing $\mathcal{I}$ with its cofinal subset $\mathcal{I}^\mathrm{aff}$ consisting of affine \'etale neighborhoods $\mathcal{V}_i= \text{Spec} B_i$ of $\bar s$ mapping into some fixed affine open $\text{Spec} A \subseteq S$, we get
$$
\mathcal{O}_{S, \bar{s}}^\mathrm{sh} = \text{colim}_{i\in \mathcal{I}^\mathrm{aff}} B_i,
$$
and the result follows from theorem \ref{thm:directedColimitsAndCohomology}.
\end{proof}

\subsection{The Leray Spectral Sequence}

\begin{lemma}
Let $f: X\to Y$ be a morphism and $\mathcal{I}$ an injective sheaf in $\text{Ab}(X_{et})$. Then 
\begin{enumerate}
\item 
for any $\mathcal{V}\in\text{Ob}(Y_{et})$ and any \'etale covering $\mathcal{V}=\left\{\mathcal{V}_j\to \mathcal{V}\right\}_{j\in J}$  we have $\check H^p(\mathcal{V}, f_*\mathcal{I}) = 0$ for all $p>0$ ;
\item 
$f_*\mathcal{I}$ is acyclic for the functors $\Gamma(Y, -)$ and $\Gamma(\mathcal{V},-)$ ; and 
\item 
if $g: Y\to Z$, then $f_*\mathcal{I}$ is acyclic for $g_*$.
\end{enumerate}
\end{lemma}

\begin{proof}
Observe that $\check{\mathcal{C}}^\bullet(\mathcal{V}, f_*\mathcal{I}) = \check{\mathcal{C}}^\bullet(\mathcal{V}\times_Y X, \mathcal{I})$ which has no cohomology by lemma \ref{lem:HomInjSheafIsExact}, which proves {\it i}. The second statement is a great exercise in using the \u Cech-to-cohomology spectral sequence. See \cite{Stacks} for more details. Part {\it iii} is a consequence of {\it ii} and the description of $R^pg_*$ from theorem \ref{thm:stalkOfHigherDirectImages}. 
\end{proof}

Using the formalism of Grothendieck spectral sequences, this gives the following.

\begin{proposition}[Leray spectral sequence]
Let $f: X \to Y$ be a morphism of schemes and $\mathcal{F}$ an \'etale sheaf on $X$. Then there is a spectral sequence
$$
E_2^{p,q} = H_{et}^p(Y, R^qf_*\mathcal{F}) \Rightarrow H_{et}^{p+q}(X, \mathcal{F}).
$$
\end{proposition}

\subsection{Henselian Rings}

\begin{theorem} 
Let $A\to B$ be finite type ring map and $\frak p \subseteq A$ a prime ideal. Then there exist an \'etale ring map $A \to A'$ and a prime $\frak p' \subseteq A'$ lying over $\frak p$ such that 
\begin{enumerate}
\item 
$\kappa(\frak p) = \kappa(\frak p')$ ;
\item 
$ B \otimes_A A' = B_1\times \cdots \times B_r \times C$ ;
\item 
$ A'\to B_i$ is finite and there exists a unique prime $q_i\subseteq B_i$ lying over $\frak p'$ ; and
\item 
all irreducible components of the fibre $\text{Spec}(C\otimes_{A'} \kappa(\frak p'))$ of $C$ over $\frak p'$ have dimension at least 1.
\end{enumerate}
\end{theorem}

\begin{proof}
Omitted (see EGA IV, th\'eor\`eme 18.12.1).
\end{proof}

%Hensel's lemma: $f\in \mathbf{Z}_p[T]$ monic, $\bar{f}\pmod{p}$ factors as $\bar g\bar h$ with $gcd(\bar{g}, \bar{h})=1$ then $f$ factors as $f=gh$ with $\bar{g}=\bar{g}, \; \bar{h}=\bar{h}$. 
%$f\in \mathbf{Z}_p[T]$, monic $\alpha\in \mathbf{F}_p$, $\bar f(\alpha) =0$ but $\bar f'(\alpha)\neq 0$ then $\exists $ root of $f$ in $\mathbf{Z}_p$ which residue to $\alpha$. 

%(10.01.09)

\begin{definition}
A local ring $(R, \mathfrak m, \kappa)$ is called \emph{henselian} if for all $f\in R[T]$ monic, for all $\alpha\in \kappa$ such that $\bar f(\alpha)=0$ and $\bar f'(\alpha)\neq 0$, there exists $\tilde\alpha\in R$ such that $f(\tilde\alpha) = 0$ and $\tilde\alpha\mod\mathfrak m = \alpha$. 
\end{definition} 

Recall that a complete local ring is a local ring $(R, \frak m)$ such that  $R\cong \text{lim}_n R/\mathfrak m^n$, {\it i.e.} it is complete and separated for the $\frak m$-adic topology.

\begin{theorem}
Complete local rings are henselian.
\end{theorem}

\begin{proof}
Newton's method.
\end{proof}

\begin{theorem} \label{thm:eqDefHenselian}
Let $(R, \mathfrak m, \kappa)$ be a local ring. The following are equivalent:
\begin{enumerate}
\item $R$ is henselian ;
\item for any $f\in R[T]$ and any factorization $\bar f = g_0 h_0$ in $\kappa[T]$ with $\gcd(g_0,h_0)=1$, there exists a factorization $f=gh$ in $R[T]$ with $\bar g = g_0$ and $\bar h=h_0$ ;
\item any finite $R$-module is isomorphic to a product of (finite) local rings ;
\item any finite type $R$-algebra $A$ is isomorphic to a product $A \cong A' \times C$ where $A' \cong A_1 \times \cdots \times A_r$ is a product of finite local rings and all the irreducible components of $C\otimes_R\kappa$ have dimension at least 1 ;
\item if $A$ is an \'etale $R$-algebra and $\mathfrak n$ is a maximal ideal of $A$ lying over $\mathfrak m$ such that $\kappa \cong A/\frak n$, then there exists an isomorphism $\varphi: A \cong R\times A'$ such that $\varphi(\frak n) = \frak m \times A'\subseteq R\times A'$.
\end{enumerate}
\end{theorem}

%\begin{proof} of v using the previous ones.
%We have seen that
%		$$A_{\mathfrak m'}\cong \left(R[T]/f\right)_{\mathfrak m'}
%		$$ 
%		for some $f\in R[T]$ monic and 
%		$$
%		\frac{df}{dT}\text{ inv. in } \left(R[T]/f\right)_{\mathfrak m'}
%		$$
%		The fact that 
%		$\kappa(\mathfrak m') = \kappa$ means we have class of $T$ in 
%			$$A/\mathfrak m' \leftrightarrow \alpha\in \kappa
%			$$
%			which is a root of $\overline f$ and 
%			$$\frac{d\overline f}{dT}(\alpha)\neq 0
%			$$
%			Then applying definition of henselian right to get $\widetilde\alpha \in R$ such that $f(\widetilde \alpha) = 0$, $\widetilde \alpha\mod \mathfrak m = \alpha$. 
%			\vskip 3pt
%			Check: 
%				$$
%				\begin{array}{ccccc}
%				A_{ \mathfrak m'}\cong & \left(R[T]/f\right)_{\mathfrak m'}&\to^{\cong}& R
%				\\& T&\mapsto &\widetilde{\alpha}
%				\end{array}
%				$$
%				... shows that $R$ splits off from $A$. 
%\end{proof}

\begin{example}
In the case $R = \mathbf{C} [[ t ]]$, the finite henselian $R$-algebras are the trivial one $R \to R$ and the extensions $ \mathbf{C} [[ t ]] \to \mathbf{C} [[ t ]] [X, X^{-1}]/(X^n-t)$. The latter ones always miss the origin, so any \'etale covering contains the identity and thus has the trivial covering as refinement. We will see below that this is in fact a somewhat general fact and this will give us the vanishing of higher direct images for a finite morphism.
\end{example}

\begin{lemma} \label{cor:FiniteOverHenselianIs Henselian}
If $R$ is henselian and $A$ is a finite $R$-module, then $A$ is a finite product of henselian local rings. 
\end{lemma}

\begin{definition} 
A local ring $R$ is called \emph{strictly henselian} if it is henselian and its residue field is separably closed. 
\end{definition}

\begin{theorem}
Let $(R, \mathfrak m, \kappa)$ be a local ring and $\kappa\subseteq\kappa^{sep}$ a separable closure. There exist canonical local ring maps $ R\to R^\text{h} \to R^\text{sh}$ where
\begin{enumerate}
\item $R^\text{h}$, $R^\text{sh}$ are colimits of \'etale $R$-algebras ;
\item $R^\text{h}$, $R^\text{sh}$ are henselian ;
\item $\mathfrak m R^\text{h}$ (respectively $\mathfrak m R^\text{sh}$) is the maximal ideal of $R^\text{h}$ (resp. $R^\text{sh}$) ; and 
\item the first residue field extension is trivial $\kappa=R^\text{h}/\mathfrak m R^\text{h}$, and the second is the separable closure $\kappa^{sep} = R^\text{sh}/\mathfrak m R^\text{sh}$. 
\end{enumerate} 
Moreover, $R^\text{sh}\cong \mathcal{O}^\text{sh}_{\text{Spec}(R), \; \text{Spec}(\kappa^\text{sh})}$ as defined in \ref{defi:EtaleLocalRings}.
\end{theorem}

\begin{remark} 
If $R$ is noetherian then $R^\text{h}$ is also noetherian and they have the same completion: $\hat R\cong \widehat{R^\text{h}}$. In particular, $R\subseteq R^\text{h} \subseteq \hat R$. The henselization of $R$ is in general much smaller than its completion and inherits many of its properties ({\it e.g} if $R$ is reduced, then so is $R^\text{h}$, but not $\hat R$ in general).
\end{remark}

\subsection{Vanishing of Finite Higher Direct Images} 

The next goal is to prove that the higher direct images of a finite morphism of schemes vanish.

\begin{lemma} \label{lem:vanishingOfCohomologyOverHenselianRings}
Let $R$ be a strictly henselian ring and $S=\text{Spec}(R)$. Then the global sections functor $\Gamma(S, -): \text{Ab}(S_{et})\to \text{Ab}$ is exact. In particular
$$
\forall p\geq 1, \quad H_{et}^p(S, \mathcal{F})=0
$$
for all $\mathcal{F}\in \text{Ab}(S_{et})$. 
\end{lemma}

\begin{proof}
Let $\mathcal{U} = \left\{f_i : \mathcal{U}_i \to S \right\}_{i\in I}$ be an \'etale covering, and denote $s$ the closed point of $S$. Then $s = f_i (u_i)$ for some $i\in I$ and some $u_i \in U_i$ by lemma \ref{lem:geomPointsAreInSomeOpen}. Pick an affine open neighborhood $\text{Spec} A$ of $u_i$ in $\mathcal{U}_i$. Then there is a commutative diagram
$$
\xymatrix{
R \ar[r] \ar[d] & A \ar[d] \\
{\kappa(s)} \ar[r] & {\kappa(u_i)}
}
$$
where $\kappa(s)$ is separably closed, and the residue extension is finite separable. Therefore, $\kappa(s) \cong \kappa(u_i)$, and using part {\it v} of theorem \ref{thm:eqDefHenselian}, we see that $A \cong R\times A'$ and we get a section
$$
\xymatrix{
{\text{Spec} A\ }\ar[dr] \ar@{^{(}->}[r] & {\mathcal{U}_i} \ar[d]\\
& {S.} \ar@/^1pc/[ul]
}
$$
In particular, the covering $\left\{\text{id} : S\to S\right\}$ refines $\mathcal{U}$. This implies that if 
$$
0 \to \mathcal{F}_1\to \mathcal{F}_2 \xrightarrow{\alpha} \mathcal{F}_3\to 0
$$
is a short exact sequence in $\text{Ab}(S_{et})$, then the sequence
$$
0 \to \Gamma(S_{et}, \mathcal{F}_1) \to \Gamma(S_{et}, \mathcal{F}_2) \to \Gamma(S_{et}, \mathcal{F}_3)\to 0
$$
is also exact. Indeed, exactness is clear except possibly at the last step. But given a section $s \in \Gamma(S_{et}, \mathcal{F}_3)$, we know that there exist a covering $\mathcal{U}$ and local sections $s_i$ such that $\alpha (s_i) = s|_{\mathcal{U}_i}$. But since this covering can be refined by the identity, the $s_i$ must agree locally with $s$, hence they glue to a global section of $\mathcal{F}_2$.
\end{proof}

\begin{proposition} \label{prop:FiniteHigherDirectImagesVanish}
Let $f: X\to Y$ be a finite morphism of schemes. Then for all $q\geq 1$ and all $\mathcal{F}\in \text{Ab}(X_{et})$, $R^q f_*\mathcal{F}=0$. 
\end{proposition}

\begin{proof}
Let $X_{\bar y}^\text{sh}$ denote the fiber product $X\times_Y \text{Spec}(\mathcal{O}_{Y, \bar y}^\text{sh})$. It suffices to show that for all $q\geq 1$, $H_{et}^q(X_{\bar y}^\text{sh}, \mathcal{G})=0$. Since $f$ is finite, $X_{\bar y}^\text{sh}$ is finite over $\text{Spec}(\mathcal{O}_{Y, \bar y}^\text{sh})$, thus $X_{\bar y}^\text{sh} = \text{Spec} A$ for some ring $A$ finite over $\mathcal{O}_{Y, \bar y}^\text{sh}$. Since the latter is strictly henselian, corollary \ref{cor:FiniteOverHenselianIs Henselian} implies that $A$ is henselian and therefore splits as a product of henselian local rings $A_1 \times \cdots \times A_r$. Furthermore, $\kappa(\mathcal{O}_{Y, \bar y}^\text{sh})$ is separably closed and for each $i$, the residue field extension $\kappa(\mathcal{O}_{Y, \bar y}^\text{sh}) \subseteq \kappa(A_i)$ is finite, hence $\kappa(A_i)$ is separably closed  and $A_i$ is strictly henselian. This implies that $\text{Spec} A = \coprod_{i=1}^r \text{Spec} A_i$, and we can apply lemma \ref{lem:vanishingOfCohomologyOverHenselianRings} to get the result.
\end{proof}

\subsection{Cohomology of a Point}

\begin{lemma}
Let $K$ be a field and $K^{sep}$ a separable closure of $K$. Consider
$$
\mathcal{G} := \text{Aut}_{\text{Spec}(K)}(\text{Spec}(K^{sep}))^{opp}  = \text{Gal}(K^{sep} | K)
$$
as a topological group, and denote $\mathcal{G}\textit{-Sets}$ (respectively $\mathcal{C}^0(\mathcal{G}\text{-Sets})$) the category of (resp. continuous) left $\mathcal{G}$-sets. Then the functor
$$
\begin{array}{ccl}
\text{Spec} K _{et} &  \longrightarrow & \mathcal{G}\textit{-Sets} \\
(X\to\text{Spec} K) & \longmapsto & \text{Hom}_{\text{Spec} K}\left(\text{Spec}(K^{sep}), X\right)
\end{array}
$$
is an equivalence of categories, with essential image $\mathcal{C}^0(\mathcal{G}\text{-Sets})$.
\end{lemma}

Recall that a $\mathcal{G}$-set is \emph{continuous} if each of its elements has an open stabilizer. This means that the action is continuous when $\mathcal{G}$ is endowed with its profinite topology and the $\mathcal{G}$-sets have the discrete topology.

\begin{proof}
Recall that $X$ is \'etale over $K$ if and only if  $X=\coprod_{i\in I} \text{Spec} K_i$ with $K_i | K$ finite and separable. Then use standard Galois theory.
\end{proof}

\begin{remark}
Under the correspondence of the lemma, the coverings in $\text{Spec}(K)_{et}$ correspond to surjective families of maps in $\mathcal{C}^0(\mathcal{G}\text{-Sets})$. 
\end{remark}

\begin{lemma} \label{lem:EqOfCatContGSets}
The stalk functor
$$
\begin{array}{ccl}
\textit{Sh}(\text{Spec} K_{et}) & \longrightarrow & \mathcal{C}^0(\mathcal{G}\text{-Sets}) \\
\mathcal{F} & \longmapsto & \mathcal{F}_{\text{Spec} K^{sep}}
\end{array}
$$
is an equivalence of categories. In other words, every sheaf on $\text{Spec}(K)_{et}$ is representable.
\end{lemma}

\begin{proof}
The category $\text{Spec} K_{et}$ has some extra structure (maybe pushouts or something -- figure it out) which makes it automatic that all sheaves are representable. In the language of paragraph \ref{subsubsection:GSets}, we have identifications $\textit{Sh}(\mathcal{T}_G) = \textit{Sh}(\mathcal{G}\textit{-Sets}) = \mathcal{G}\textit{-Sets}$.
\end{proof}

\begin{lemma}
Let $\mathcal{F}$ be an abelian sheaf on $\text{Spec}(K)_{et}$. Then
\begin{enumerate}
\item the $\mathcal{G}$-module $M = \mathcal{F}_{\text{Spec}(K^{sep})}$ is continuous ;
\item $H_{et}^0(\text{Spec}(K), \mathcal{F})=M^{\mathcal{G}}$ ; and
\item $H_{et}^q(\text{Spec}(K), \mathcal{F}) = H_{\mathcal{C}^0}^q(\mathcal{G}, M)$.
\end{enumerate}
\end{lemma}

\begin{proof}
Part {\it i} is clear (use that the stalk functor is exact). For {\it ii}, we have 
\begin{eqnarray*}
\Gamma(\text{Spec}(K)_{et}, \mathcal{F}) & = & \text{Hom}_{\textit{Sh}(\text{Spec}(K)_{et})}(h_{\text{Spec}(K)}, \mathcal{F}) \\
& = & \text{Hom}_{\mathcal{C}^0(\mathcal{G}\text{-Sets})}(\{*\}, M) \\
& = &  M^{\mathcal{G}}
\end{eqnarray*}
where the first identification is of Yoneda type, the second results from lemma \ref{lem:EqOfCatContGSets} and the third is clear. Part {\it iii} is also a straightforward consequence of lemma \ref{lem:EqOfCatContGSets}.
\end{proof}

\begin{example}[Sheaves on $\text{Spec}(K)_{et}$.] $ $
\begin{enumerate}
\item The constant sheaf $\underline{\mathbf{Z}/n\mathbf{Z}}$ corresponds to the module $\mathbf{Z}/n\mathbf{Z}$ with trivial action ; 
\item the sheaf $\mathbf{G}_m|_{\text{Spec}(K)_{et}}$ corresponds to $(K^{sep})^*$ with the canonical (left) $\mathcal{G}$-action ;
\item the sheaf $\mathbf{G}_a|_{\text{Spec}(K^{sep})}$ corresponds to $(K^{sep}, +)$ with the canonical (left) $\mathcal{G}$-action ;
\item the sheaf $\mu_n|_{\text{Spec}(K^{sep})}$ corresponds to $\mu_n(K^{sep})$ with the canonical $\mathcal{G}$-action.
\end{enumerate}

The same arguments as in the $fpqc$ case (see section \ref{section:CechCohomology}) give the following identifications for cohomology groups:
$$
\begin{array}{ccl}
H_{et}^0(S_{et}, \mathbf{G}_m) & = & \Gamma(S, \mathcal{O}_S^*) \; ; \\
H_{et}^1(S_{et}, \mathbf{G}_m) & = & H^1(S, \mathcal{O}_S^*) \, = \,  H^1(S, \mathcal{O}_S^*) \, = \, \text{Pic}(S) \; ;\\
H_{et}^i(S_{et}, \mathbf{G}_a) & = & H_{Zar}^i(S, \mathcal{O}_S).
\end{array}
$$
Also, for any quasi-coherent sheaf $\mathcal{F}$ on $S_{et}$, $H^i(S_{et}, \mathcal{F}) = H_{Zar}^i(S, \mathcal{F})$.
In particular, this gives the following sequence of equalities
\begin{eqnarray*}
0 & = & \text{Pic}(\text{Spec}(K)) \\
& = & H_{et}^1(\text{Spec}(K)_{et}, \mathbf{G}_m) \\
& = & H^1_{\mathcal{C}^0}(\mathcal{G}, (K^{sep})^*) 
\end{eqnarray*}
which is none other than Hilbert's 90 theorem. Similarly, for $i \geq 1$,
\begin{eqnarray*}
0 & = & H^i(\text{Spec}(K), \mathcal{O}) \\
&  = & H_{et}^i(\text{Spec}(K)_{et}, \mathbf{G}_a) \\
& = & H^i(\mathcal{G}, (K^{sep}, +)).
\end{eqnarray*}
\end{example}

%(10.06.09)

*** MERGE THESE TWO SECTIONS ***

\subsection{Galois Action on Stalks}
\label{section:GaloisActionOnStalks}

\begin{definition}
Let $S$ be a scheme. A geometric point $\bar s$ of $S$ is called \emph{algebraic} if $\kappa(s) \subseteq \kappa(\bar s)$ is algebraic, {\it i.e.} if $\kappa(\bar s)$ is a separable algebraic closure of $\kappa(s)$.
\end{definition}

\begin{example}
The geometric point $\text{Spec} \mathbf{C} \to \text{Spec} \mathbf{Q}$ is not algebraic.
\end{example}

\paragraph{Stalks as sets.}
Let $S$ be a scheme, $\mathcal{F}$ an \'etale sheaf on $X$, and $\bar s$ a geometric point of $S$. Then 
$$
\mathcal{F}_{\bar s} = \left\{
(\mathcal{U},\bar u, t) \ \big| \ \mathcal{U} \to S \text{ is \'etale, } \bar u : \bar s \to \mathcal{U} \text{ is an $S$-morphism and } t \in \mathcal{F}(\mathcal{U})
\right\}
\big/\sim
$$
where $(\mathcal{U},\bar u, t) \sim (\mathcal{U}',\bar u', t')$ if there exist an \'etale neighborhood $(\mathcal{U}'',\bar u'')$ of $\bar s$ and a commutative diagram
$$
\xymatrix{
\bar s \ar^{\bar u''}[rd] \ar^{\bar u'}@/^1pc/[rrd] \ar^{\bar u}@/_1pc/[rdd] \\
& \mathcal{U}'' \ar^{\beta}[r] \ar^{\alpha}[d] & \mathcal{U}' \ar[d] \\
& \mathcal{U} \ar[r] & S
}
$$
such that $\alpha^*(t) = \beta^*(t')$ in $\mathcal{F}(\mathcal{U}'')$.

\paragraph{Galois action on stalks.}
Given an algebraic geometric point $\bar s$ of $S$, set $\mathcal{G} = \text{Gal}(\kappa(\bar s) | \kappa(s))$ and define an action of $\mathcal{G}$ on $\mathcal{F}_{\bar s}$ as follows
$$
\begin{array}{ccccl}
\mathcal{G} & \times & \mathcal{F}_{\bar s} & \longrightarrow & \mathcal{F}_{\bar s} \\
(\sigma & , & (\mathcal{U},\bar u, t)) & \longmapsto & (\mathcal{U},\bar u \circ \text{Spec} \sigma, t).
\end{array}
$$
It is easy to check that this is a well-defined left action on the stalk $\mathcal{F}_{\bar s}$. We can thus restate the theorem of last time as follows.

\begin{theorem}
The action of $\mathcal{G}$ on the stalks $\mathcal{F}_{\bar s}$ is continuous (for the discrete topology on $\mathcal{F}_{\bar s}$. Moreover, if $\bar s= \text{Spec} K$ then it induces an equivalence of categories
$$
\begin{array}{rcl}
\textit{Sh}(S_{et}) & \longrightarrow & \{ \text{discrete $\mathcal{G}$-sets with continuous action} \} \\
\mathcal{F} & \longmapsto & \mathcal{F}_{\bar s}.
\end{array}
$$
\end{theorem}

In particular, the category $\text{Ab}(S_{et})$ corresponds to the full subcategory of discrete $\mathcal{G}$-modules, and we have the identification $H_{et}^0(S,\mathcal{F}) = (\mathcal{F}_{\bar s})^\mathcal{G}$, and more generally $H_{et}^q(S,\mathcal{F}) = H_{cont}^q (\mathcal{G}, \mathcal{F}_{\bar s})$ for $q \geq 1$.
\section{Cohomology of Curves}

The next task at hand is to compute the \'etale cohomology of a smooth curve with torsion coefficients, and in particular show that it vanishes in degree at least 3. To prove this, we will compute cohomology at the generic point, which amounts to some Galois cohomology. We now review without proofs. the relevant facts about Brauer groups. For references, see \cite{SerreAlgebra} or \cite{WeilNumberTheory}.

\subsection{Brauer Groups}

\begin{theorem} \label{defthm:CSA}
Let $K$ be a field. A unital, associative (not necessarily commutative) $K$-algebra $A$ is a \emph{central simple algebra} over $K$ if the following equivalent conditions hold
\begin{enumerate}
\item
$A$ is finite dimensional over $K$, $K$ is the center of $A$ and $A$ has no nontrivial two-sided ideal ;
\item
there exists $d \geq 1$ such that $A \otimes_K \bar K \cong_{\bar K} \text{Mat}_d(\bar K)$ ;
\item
there exists $d \geq 1$ such that $A \otimes_K K^{sep} \cong_{K^{sep}} \text{Mat}_d(K^{sep})$ ;
\item
there exist $d \geq 1$ and a finite Galois extension $K \subseteq K'$ such that $A \otimes_{K'} K' \cong_{K'} \text{Mat}_d(K')$; 
\item
there exist $f \geq 1$ and a finite dimensional division algebra $D$ with center $K$ such that $A \cong_{K'} \text{Mat}_f(D)$.
\end{enumerate}
The integer $d$ is called the \emph{degree} of $A$. 
\end{theorem}

\begin{definition}
Two central simple algebras $A_1$ and $A_2$ over $K$ are called \emph{equivalent} if there exist $m, n \geq 1$ such that $\text{Mat}_n(A_1) \cong \text{Mat}_m(A_2)$. We write $A_1 \sim A_2$.
\end{definition}

\begin{lemma}
Let $A$ be a central simple algebra over $K$. Then
$$
\begin{array}{lcl}
A \otimes_K A^{opp} & \longrightarrow & \text{End}_{K-\text{Mod}}(A) \\
\ a \otimes a' & \longmapsto & (x \mapsto a x a')
\end{array}
$$
is an isomorphism of central simple algebras over $K$.
\end{lemma}

\begin{definition}
Let $K$ be a field. The \emph{Brauer group} of $K$ is the set $\text{Br} (K)$ of central simple algebras over $K$ modulo equivalence, endowed with the group law induced by tensor product (over $K$). The class of $A$ in $\text{Br} (K)$ is denoted by $[A]$. The neutral element is $[K] = [\text{Mat}_d(K)]$ for any $d \geq 1$.
\end{definition}

The previous lemma thus mean that inverses exist, and that $-[A] = [A^{opp}]$. The Brauer group is always torsion, but not finitely generated in general. It is also true (exercise) that $A^{\otimes \deg A} \sim K$ for any central simple algebra $A$.

\begin{lemma}
Let $K$ be a field and $\mathcal{G} = \text{Gal} (K^{sep}|K))$. Then the set of isomorphism classes of central simple algebras of degree $d$ over $K$ is in bijection with the anabelian cohomology $H_{cont}^1 (\mathcal{G}, \text{PGL}_d(K^{sep}))$.
\end{lemma}

\begin{proof}[Sketch of proof.]
The Skolem-Noether theorem implies that for any field $L$, $\text{Aut}_{L\text{-Algebras}}(\text{Mat}_d(L))$ $ = \text{PGL}_d(L)$. By \ref{defthm:CSA}, we see that central simple algebras of degree $d$ correspond to forms of the $K$-algebra $\text{Mat}_d(K)$, which in turn correspond to $H_{cont}^1 (\mathcal{G}, \text{PGL}_d(K^{sep}))$. For more details, see \cite{Silverman:EllipticCurves}.
\end{proof}

If $A$ is a central simple algebra over $K$, we denote $\xi_A$ the corresponding cohomology class in $H_{cont}^1 (\mathcal{G}, \text{PGL}_{\deg A}(K^{sep}))$. Consider now the short exact sequence 
$$
1 \to (K^{sep})^* \to \text{GL}_d(K^{sep}) \to \text{PGL}_d(K^{sep}) \to 1,
$$
 which gives rise to a long exact cohomology sequence (up to degree 2) with coboundary map 
 $$
 \delta_d : H_{cont} ^1(\mathcal{G}, \text{PGL}_d(K^{sep})) \to H^2 (\mathcal{G}, (K^{sep})^*).
 $$ 
Explicitly, this is given as follows: if $\xi$ is a cohomology class represented by the 1-cocyle $(g_\sigma)$, then $\delta_d(\xi)$ is the class of the 2-cocycle $((g_\sigma^\tau)^{-1} g_{\sigma \tau} g_\tau^{-1})$. 

\begin{theorem} \label{thm:BrauerDelta}
The map
$$
\begin{array}{rrcl}
\delta : & \text{Br}(K) &  \longrightarrow & H^2(\mathcal{G}, (K^{sep})^*) \\
& [A] & \longmapsto & \delta_{\deg A} (\xi_A)
\end{array}
$$
is a group isomorphism.
\end{theorem}

We omit the proof of this theorem. Note, however, that in the abelian case, one has the identification
$$
H^1 (\mathcal{G}, \text{GL}_d(K^{sep})) = H_{et}^1 (\text{Spec} K, \text{GL}_d(\mathcal{O}))
$$
the latter of which is trivial by $fpqc$ descent. If this were true in the anabelian case, this would readily imply injectivity of $\delta$. (See \cite{SGA4.5}.) Rather, to prove this, one can reinterpret $\delta([A])$ as the obstruction to the existence of a $K$-vector space $V$ with a left $A$-module structure and such that $\dim_K V = \deg A$. In the case where $V$ exists, one has $A \cong \text{End}_{K-\text{Mod}} (V)$. For surjectivity, pick a cohomology class $\xi \in H^2(\mathcal{G}, (K^{sep})^*)$, then there exists a finite Galois extension $K \subseteq K' \subseteq K^{sep}$ such that $\xi$ is the image of some $\xi' \in H_{cont}^2(\text{Gal}(K'|K), (K')^*)$. Then write down an explicit central simple algebra over $K$ using the data $K', \xi'$.

\paragraph{The Brauer group of a scheme.}
Let $S$ be a scheme. An $\mathcal{O}_S$-algebra $\mathcal{A}$ is called \emph{Azumaya} if it is \'etale locally a matrix algebra, {\it i.e.} if there exists an \'etale covering $\mathcal{U} = \{ \varphi_i : \mathcal{U}_i \to S \}_{i \in I}$ such that $\varphi_i^*\mathcal{A} \cong \text{Mat}_{d_i}(\mathcal{O}_{\mathcal{U}_i})$ for some $d_i \geq 1$. Two such $\mathcal{A}$ and $\mathcal{B}$ are called \emph{equivalent} if there exist finite locally free sheaves $\mathcal{F}$ and $\mathcal{G}$ on $S$ such that $\mathcal{A} \otimes_{\mathcal{O}_S} \text{End}(\mathcal{F}) \cong \mathcal{B} \otimes_{\mathcal{O}_S} \text{End}(\mathcal{G})$. The \emph{Brauer group} of $S$ is the set $\text{Br}(S)$ of equivalence classes of Azumaya $\mathcal{O}_S$-algebras with the operation induced by tensor product (over $\mathcal{O}_S$). 
\\
In this setting, the analogue of the isomorphism $\delta$ of theorem \ref{thm:BrauerDelta} is a map 
$$
\delta_S: \text{Br}(S) \to H_{et}^2(S,\mathbf{G}_m).
$$ 
It is true that $\delta_S$ is injective (the previous argument still works). If $S$ is quasi-compact or connected, then $\text{Br}(S)$ is a torsion group, so in this case the image of $\delta_S$ is contained in the \emph{cohomological Brauer group} of $S$ 
$$
\text{Br}'(S) := H_{et}^2(S,\mathbf{G}_m)_\text{torsion}.
$$ 
So if $S$ is quasi-compact or connected, there is an inclusion $\text{Br}(S) \subseteq \text{Br}'(S)$. This is not always an equality: there exists a nonseparated singular surface $S$ for which $\text{Br}(S) \subset \text{Br}'(S)$ is a strict inclusion. If $S$ is quasi-projective, then $\text{Br}(S) = \text{Br}'(S)$. However, it is not known whether this holds for a smooth proper variety over $\mathbf{C}$, say.  


\begin{proposition} \label{prop:SerreGalois}
Let $K$ be a field, $\mathcal{G} = \text{Gal}(K^{sep}|K)$ and suppose that for any finite extension $K'$ of $K$, $\text{Br}(K') = 0$. Then
\begin{enumerate}
\item
for all $ q \geq 1$, $H^q (\mathcal{G},(K^{sep})^*) = 0$ ; and
\item
for any torsion $\mathcal{G}$-module $M$ and any $q \geq 2$, $H_{cont}^q (\mathcal{G},M) = 0$.
\end{enumerate}
\end{proposition}

See \cite{Serre:GaloisCohomology} for proofs.

% (10.08.09)

\begin{definition} 
A field $K$ is called $C_r$ if for every $0 < d^r < n$ and every $f \in K[T_1, \ldots, T_n]$ homogeneous of degree $d$, there exist $\alpha = (\alpha_1, \ldots, \alpha_n)$, $\alpha_i \in K$ not all zero, such that $f(\alpha) = 0$. Such an $\alpha$ is called a \emph{nontrivial solution} of $f$.
\end{definition}

\begin{example} 
An algebraically closed field is $C_r$. 
\end{example}

In fact, we have the following simple lemma.

\begin{lemma} 
Let $k$ be an algebraically closed field. Let $f_1, \ldots, f_s \in k[T_1, \ldots, T_n]$ be homogeneous polynomials of degree $d_1, \ldots, d_s$ with $d_i > 0$. If $s < n$, then $f_1 = \ldots = f_s = 0$ have a common nontrivial solution.
\end{lemma}

\begin{proof} Omitted. \end{proof}

The following result computes the Brauer group of $C_1$ fields.

\begin{theorem} 
Let $K$ be a $C_1$ field. Then $\text{Br}(K) = 0$.
\end{theorem}

\begin{proof} 
Let $D$ be a finite dimensional division algebra over $K$ with center $K$. We have seen that
$$
D \otimes_K K^{sep} \cong \text{Mat}_d(K^{sep})
$$
uniquely up to inner isomorphism. Hence the determinant $\det : \text{Mat}_d(K^{sep}) \to K^{sep}$ is Galois invariant and descends to a homogeneous degree $d$ map
$$
\det = N_\text{red} : D \longrightarrow K
$$
called the \emph{reduced norm}. Since $K$ is $C_1$, if $d > 1$, then there exists a nonzero $x \in D$ with $N_\text{red}(x) = 0$. This clearly implies that $x$ is not invertible, which is a contradiction. Hence $\text{Br}(K) = 0$.
\end{proof}

\begin{theorem}[Tsen]
The function field of a variety of dimension $r$ over an algebraically closed field $k$ is $C_r$.
\end{theorem}

\begin{proof}
\begin{enumerate}
\item
Projective space. The field $k(x_1, \ldots, x_r)$ is $C_r$ (exercise).
\item
General case. Without loss of generality, we may assume $X$ to be projective. Let $f \in K[T_1, \dots, T_n]_d$ with $0 < d^r <n$. Say the coefficients of $f$ are in $\Gamma(X,\mathcal{O}_X(H))$ for some ample $H \subseteq X$. Let $\mathbf{\alpha} = (\alpha_1, \dots, \alpha_n)$ with $\alpha_i \in \Gamma(X, \mathcal{O}_X(eH))$. Then $f(\mathbf{\alpha}) \in \Gamma(X, \mathcal{O}_X((de+1)H))$. Consider the system of equations $f(\mathbf{\alpha}) =0$. Then by asymptotic Riemann-Roch, 
\begin{itemize}
\item
the number of variables is $n\dim_K \Gamma(X,\mathcal{O}_X(eH)) \sim n\frac{e^r}{r!} (H^r)$ ; and
\item
the number of equations is $\dim_K \Gamma(X,\mathcal{O}_X((de+1)H)) \sim \frac{(de+1)^r}{r!} (H^r).$
\end{itemize}
Since $n> d^r$, there are more variables than equations, and since there is a trivial solution, there are also nontrivial solutions.
\end{enumerate}
\end{proof}

\begin{definition}
We call \emph{variety} a separated, geometrically irreducible and geometrically reduced scheme of finite type over a field, and  \emph{curve} a variety of dimension 1. 
\end{definition}

\begin{lemma}
Let $C$ be a curve over an algebraically closed field $k$. Then $\text{Br}(k(C)) = 0$.
\end{lemma}

This is clear from the theorem.

\begin{lemma} \label{cor:HqGmKvanishes}
Let $k$ be an algebraically closed field and $k \subseteq K$ a field extension of transcendence degree 1. Then for all $q \geq 1$, $H_{et}^q(\text{Spec} K, \mathbf{G}_m) = 0$.
\end{lemma}

\begin{proof}
It suffices to show that if $K \subseteq K'$ is a finite field extension, then $\text{Br}(K') = 0$. Now observe that $K' = \text{colim} K''$, where $K''$ runs over the finitely generated subextensions of $k$ contained in $K'$ of transcendence degree 1. By some result in \cite{Hartshorne}, each $K''$ is the function field of a curve, hence has trivial Brauer group by the previous  corollary. It now suffices to observe that $\text{Br}(K') = \text{colim} \text{Br}(K'')$.
\end{proof}

\subsection{Higher Vanishing for $\mathbf{G}_m$}

In this section, we fix an algebraically closed field $k$ and a smooth curve $X$ over $k$. We denote $i_x : x \hookrightarrow X$ the inclusion of a closed point of $X$ and $j : \eta \hookrightarrow X$ the inclusion of the generic point. We also denote $X^0$ the set of closed points of $X$. 

\begin{theorem}[The Fundamental Exact Sequence]
There is a short exact sequence of \'etale sheaves on $X$
$$
0 \longrightarrow \mathbf{G}_{m,X} \longrightarrow j_* \mathbf{G}_{m,\eta} \xrightarrow{\ \div\ } \bigoplus_{x \in X^0} {i_x}_* \underline{\mathbf{Z}} \longrightarrow 0.
$$
\end{theorem}

\begin{proof}
Let $\varphi : \mathcal{U} \to X$ be an \'etale morphism. Then by properties {\it v} and {\it vi} of \'etale morphisms (proposition \ref{prop:ofEtaleMorphisms}), $\mathcal{U} = \coprod_i \mathcal{U}_i$ where each $\mathcal{U}_i$ is a smooth curve mapping to $X$. The above sequence for $X$ is a product of the corresponding sequences for each $\mathcal{U}_i$, so it suffices to treat the case where $\mathcal{U}$ is connected, hence irreducible. In this case, there is a well known exact sequence (see \cite{Hartshorne})
$$
1 \longrightarrow \Gamma(\mathcal{U},\mathcal{O}_\mathcal{U}^*) \longrightarrow k(\mathcal{U})^* \xrightarrow{\ \div\ } \bigoplus_{y \in \mathcal{U}^0} \mathbf{Z}_y.
$$
This amounts to a sequence
$$
\Gamma(\mathcal{U},\mathcal{O}_\mathcal{U}^*) \longrightarrow \Gamma(\eta\times_X\mathcal{U},\mathcal{O}_{\eta\times_X\mathcal{U}}^*) \xrightarrow{\ \div\ } \bigoplus_{x \in X^0} \Gamma(x\times_X\mathcal{U},\underline{\mathbf{Z}}) 
$$
which, unfolding definitions, is nothing but a sequence
$$
\mathbf{G}_m(\mathcal{U}) \longrightarrow j_* \mathbf{G}_{m,\eta}(\mathcal{U}) \xrightarrow{\ \div\ } \bigoplus_{x \in X^0} {i_x}_* \underline{\mathbf{Z}} (\mathcal{U}).
$$
This defines the maps in the Fundamental Exact Sequence and shows it is exact except possibly at the last step. To see surjectivity, let us recall (from \cite{Hartshorne} again) that if $C$ is a nonsingular curve and $D$ is a divisor on $C$, then there exists a Zariski open covering $\{ \mathcal{V}_j \to C \}$ of $C$ such that $D |_{\mathcal{V}_j} = \div(f_j)$ for some $f_j \in k(C)^*$.
\end{proof}

\begin{lemma}
For any $q \geq 1$, $R^q j_*\mathbf{G}_{m,\eta} = 0$.
\end{lemma}

\begin{proof}
We need to show that $(R^q j_*\mathbf{G}_{m,\eta})_{\bar x} = 0$ for every geometric point $\bar x$ of $X$. 
\begin{enumerate}
\item
Assume that $\bar x$ lies over a closed point $x$ of $X$. Let $\text{Spec} A$ be an open neighborhood of $x$ in $X$, and $K$ the fraction field of $A$, so that
$$
\text{Spec}(\mathcal{O}^\text{sh}_{X,\bar x}) \times_X \eta = \text{Spec}(\mathcal{O}^\text{sh}_{X,\bar x} \otimes_A K).
$$
The ring $\mathcal{O}^\text{sh}_{X,\bar x} \otimes_A K$ is a localization of the discrete valuation ring $\mathcal{O}^\text{sh}_{X,\bar x}$, so it is either $\mathcal{O}^\text{sh}_{X,\bar x}$ again, or its fraction field $K^\text{sh}_{\bar x}$. But since some local uniformizer gets inverted, it must be the latter. Hence
$$
(R^q j_*\mathbf{G}_{m,\eta})_{(X, \bar x)} = H_{et}^q(\text{Spec} K^\text{sh}_{\bar x}, \mathbf{G}_m).
$$
Now recall that $\mathcal{O}^\text{sh}_{X, \bar x} = \text{colim}_{(\mathcal{U},\bar u) \to \bar x} \mathcal{O} (\mathcal{U}) = \text{colim}_{A \subseteq B} B$ where $A \to B$ is \'etale, hence $K^\text{sh}_{\bar x}$ is an algebraic extension of $k(X)$, and we may apply corollary \ref{cor:HqGmKvanishes} to get the vanishing.
\item
Assume that $\bar x = \bar \eta$ lies over the generic point $\eta$ of $X$ (in fact, this case is superfluous). Then $\mathcal{O}_{X,\bar \eta} = \kappa(\eta)^{sep}$ and thus
\begin{eqnarray*}
(R^q j_*\mathbf{G}_{m,\eta})_{\bar \eta}
& = &
H_{et}^q(\text{Spec} \kappa(\eta)^{sep} \times_X \eta, \mathbf{G}_m) \\
& = & H_{et}^q (\text{Spec} \kappa(\eta)^{sep}, \mathbf{G}_m)  \\
& = & 0 \ \ \text{ for } q \geq 1
\end{eqnarray*}
since the corresponding Galois group is trivial.
\end{enumerate}
\end{proof}

\begin{lemma}
For all $p \geq 1$, $H_{et}^p(X, j_*\mathbf{G}_{m,\eta}) = 0$.
\end{lemma}

\begin{proof}
The Leray spectral sequence reads
$$
E_2^{p,q} = H_{et}^p(X, R^qj_*\mathbf{G}_{m,\eta}) \Rightarrow H_{et}^{p+q}(\eta, \mathbf{G}_{m,\eta}),
$$
which vanishes for $p+q \geq 1$ by corollary \ref{cor:HqGmKvanishes}. Taking $q = 0$, we get the desired vanishing.
\end{proof}

\begin{lemma}
For all $q \geq 1$, $H_{et}^q(X, \bigoplus_{x \in X^0} {i_x}_* \underline{\mathbf{Z}}) = 0$.
\end{lemma}

\begin{proof}
For $X$ quasi-compact and quasi-separated, cohomology commutes with colimits, so it suffices to show the vanishing of $H_{et}^q(X, {i_x}_* \underline{\mathbf{Z}})$. But then the inclusion $i_x$ of a closed point is finite so $R^p {i_x}_* \underline{\mathbf{Z}} = 0$ for all $p \geq 1$ by proposition \ref{prop:FiniteHigherDirectImagesVanish}. Applying the Leray spectral sequence, we see that $H_{et}^q(X, {i_x}_* \underline{\mathbf{Z}}) = H_{et}^q(x, \underline{\mathbf{Z}})$. Finally, since $x$ is the spectrum of an algebraically closed field, all higher cohomology on $x$ vanishes.
\end{proof}

Concluding this series of lemmata, we get the following result.

\begin{theorem}
Let $X$ be a smooth curve over an algebraically closed field. Then 
$$
H_{et}^q(X, \mathbf{G}_m) = 0 \ \ \text{ for all } q \geq 2.
$$
\end{theorem}

We also get the cohomology long exact sequence
$$
0 \to H_{et}^0(X,\mathbf{G}_m) \to H_{et}^0(X,j_*\mathbf{G}_{m\eta}) \xrightarrow{\div} H_{et}^0(X,\bigoplus {i_x}_*\underline{\mathbf{Z}}) \to H_{et}^1(X,\mathbf{G}_m) \to 0
$$
although this is the familiar
$$
0 \to H_{Zar}^0(X,\mathcal{O}_X^*) \to k(X)^* \xrightarrow{\div} \text{Div}(X) \to \text{Pic}(X) \to 0.
$$

We would like to use the Kummer sequence to deduce some information about the cohomology group of a curve with finite coefficients. In order to get vanishing in the long exact sequence, we review some facts about Picard groups.

\subsection{Picards Groups of Curves}

Let $X$ be a smooth projective curve over an algebraically closed field $k$. There exists a short exact sequence
$$
0\to \text{Pic}^0(X) \to  \text{Pic}(X)\xrightarrow{\deg} \mathbf{Z} \to 0.
$$
The abelian group $\text{Pic}^0(X)$ can be identified with $\text{Pic}^0(X) = \underline{\text{Pic}}^0_{X/k}(k)$, {\it i.e.} the $k$-valued points of an abelian variety $\underline{\text{Pic}}^0_{X/k}$ of dimension $g=g(X)$ over $k$. 

\begin{definition} 
An \emph{abelian variety} over $k$ is a proper smooth connected group scheme over $k$ ({\it i.e.} a proper group variety over $k$).
\end{definition}	

\begin{proposition} \label{prop:ReviewAbelianVarieties}
Let $A$ be an abelian variety over an algebraically closed field $k$. Then
\begin{enumerate}
\item 
$A$ is projective over $k$;
\item  
$A$ is a commutative group scheme;
\item 
the morphism $[n]: A\to A$ is surjective for all $n\geq 1$, in other words $A(k)$ is a divisible abelian group;
\item 
$A[n] = \text{Ker}(A\xrightarrow{[n]} A)$ is a finite flat group scheme of rank $n^{2\dim A}$ over $k$. It is reduced if and only if $n\in k^*$;
\item 
if $n\in k^*$ then $A(k)[n] = A[n](k)\cong(\mathbf{Z}/n\mathbf{Z})^{2\dim(A)}$.
\end{enumerate}
\end{proposition}

Consequently, if $n\in k^*$ then $\text{Pic}^0(X)[n] \cong \left(\mathbf{Z}/n\mathbf{Z}\right)^{2g}$ as abelian groups. 

\begin{lemma} \label{cor:CohomologyOfASmoothProjectiveCurve}
Let $X$ be a smooth projective of genus $g$ over an algebraically closed field $k$ and $n\geq 1$, $n\in k^*$. Then there are canonical identifications
$$
H_{et}^q(X, \mu_n) = 
\left\{ \begin{array}{cl}
 \mu_n(k) & \text{ if $q=0$ ;} \\
\text{Pic}^0(X)[n] & \text{ if $q=1$ ;} \\
\mathbf{Z}/n\mathbf{Z} & \text{ if $q=2$ ;}\\
0 & \text{ if $q \geq 3$.}
\end{array}
\right.
$$
Since $\mu_n \cong \underline{\mathbf{Z}/n\mathbf{Z}}$, this gives (noncanonical) identifications
$$
H_{et}^q(X, \underline{\mathbf{Z}/n\mathbf{Z}}) \cong 
\left\{ \begin{array}{cl}
 \mathbf{Z}/n\mathbf{Z} & \text{ if $q=0$ ;} \\
(\mathbf{Z}/n\mathbf{Z})^{2g} & \text{ if $q=1$ ;} \\
\mathbf{Z}/n\mathbf{Z} & \text{ if $q=2$ ;}\\
0 & \text{ if $q \geq 3$.}
\end{array}
\right.
$$
\end{lemma}	

\begin{proof} 
The Kummer sequence $0\to \mu_{n, X} \to \mathbf{G}_{m, X} \xrightarrow{(\cdot)^n} \mathbf{G}_{m, X}\to 0$ give the long exact cohomology sequence
$$
\xymatrix{
0  \ar[r] & \mu_n(k) \ar[r] & k^* \ar^{(\cdot)^n}[r] & k^* \ar@(rd,ul)[rdllllr]
\\ 
& H_{et}^1(X, \mu_n) \ar[r] & \text{Pic}(X) \ar^{(\cdot)^n}[r] & \text{Pic} (X) \ar@(rd,ul)[rdllllr] \\
& H_{et}^2(X, \mu_n) \ar[r] & 0 \ar[r] & 0 \cdots
}
$$
The $n$ power map $k^* \to k^*$ is surjective since $k$ is algebraically closed. So we need to compute the kernel and cokernel of the map $\text{Pic}(X) \xrightarrow{(\cdot)^n} \text{Pic}(X)$. Consider the commutative diagram with exact rows
$$
\xymatrix{
0 \ar[r] & {\text{Pic}^0(X)} \ar[r] \ar^{(\cdot)^n}@{>>}[d] & {\text{Pic}(X)} \ar^{\ \ \deg}[r] \ar^{(\cdot)^n}[d] & {\mathbf{Z}} \ar[r] \ar^{n}@{^{(}->}[d] & 0\\
0 \ar[r] & {\text{Pic}^0(X)} \ar[r] & {\text{Pic}(X)} \ar^{\ \ \deg}[r] & {\mathbf{Z}} \ar[r] & 0 
}
$$
where the left vertical map is surjective by proposition \ref{prop:ReviewAbelianVarieties}, {\it iii}. Applying the snake lemma gives the desired identifications.
\end{proof}

\begin{lemma} \label{cor:VanishingOfMuForASmoothCurve}
Let $X$ be an affine smooth curve over an algebraically closed field $k$ and $n\in k^*$. Then
\begin{enumerate}
\item 
$H_{et}^0(X, \mu_n) = \mu_n(k)$;
\item
$H_{et}^1(X, \mu_n) \cong \left(\mathbf{Z}/n\mathbf{Z}\right)^{2g+r-1}$, where $r$ is the number of points in $\bar X - X$ for some smooth projective compactification $\bar X$ of $X$ ; and
\item
for all $q\geq 2$, $H_{et}^q(X, \mu_n) = 0$.
\end{enumerate}
\end{lemma}

\begin{proof}
Write $X = \bar X - \{ x_1, \dots, x_r\}$. Then $\text{Pic}(X) = \text{Pic}(\bar X)/ R$, where $R$ is the subgroup generated by $\mathcal{O}_{\bar X}(x_i)$, $1 \leq i \leq r$. Since $r \geq 1$, we see that $\text{Pic}^0(X) \to \text{Pic}(X)$ is surjective, hence $\text{Pic}(X)$ is divisible. Applying the Kummer sequence, we get {\it i} and {\it iii}. For {\it ii}, recall that
\begin{eqnarray*}
H_{et}^1(X, \mu_n) & = &
\left\{
{(\mathcal L, \alpha) {\bigg |} {{\mathcal L \in \text{Pic}(X) }
\atop 
{\alpha: \mathcal L^{\otimes n} \cong \mathcal{O}_X}}}
\right\}
{\bigg /} {\cong} \\
& = & 
\left\{(\bar{\mathcal L}, \ D, \ \bar \alpha) \right\} {\big /} \tilde{R}
\end{eqnarray*}
where $\bar{\mathcal L} \in \text{Pic}^0(\bar X)$, $D$ is a divisor on $\bar X$ supported on $\left\{x_1, \cdots, x_r\right\}$ and $ \bar{\alpha}: \bar{\mathcal L}^{\otimes n} \cong \mathcal{O}_{\bar{X}}(D)$ is an isomorphism. Note that $D$ must have degree 0. Further $\tilde{R}$ is the subgroup of triples of the form $(\mathcal{O}_{\bar X}(D'), n D', 1^{\otimes n})$ where $D'$ is supported on $\left\{x_1, \cdots, x_r\right\}$ and has degree 0. Thus, we get an exact sequence
$$
0 \longrightarrow
H_{et}^1(\bar X, \mu_n) \longrightarrow
H_{et}^1(X, \mu_n)  \longrightarrow
\bigoplus_{i=1}^r \mathbf{Z}/n\mathbf{Z} 
\xrightarrow{\ \sum\ }
\mathbf{Z}/n\mathbf{Z} \longrightarrow 0
$$
where the middle map sends the class of a triple $(\bar{ \mathcal L}, D, \bar \alpha)$ with $D = \sum_{i=1}^r a_i (x_i)$ to the $r$-tuple $(a_i)_{i=1}^r$. It now suffices to use corollary \ref{cor:CohomologyOfASmoothProjectiveCurve} to count ranks.
\end{proof}

\begin{remark}
The ``natural'' way to prove the previous corollary is to excise $X$ from $\bar X$. This is possible, we just haven't developed that theory.
\end{remark}

Our main goal is to prove the following result.

\begin{theorem} \label{thm:vanishingForCurves}
Let $X$ be a separated, finite type, dimension $1$ scheme over an algebraically closed field $k$ and  $\mathcal{F}$ a torsion sheaf on $X_{et}$. Then 
$$
H_{et}^q(X, \mathcal{F}) = 0, \quad \forall q\geq 3.
$$ 
If $X$ affine then also $H_{et}^2(X, \mathcal{F}) = 0$. 
\end{theorem}	

Recall that an abelian sheaf is called a \emph{torsion sheaf} if all of its stalks are torsion groups. We have computed the cohomology of constant sheaves. We now generalize the latter notion to get all the way to torsion sheaves.

\subsection{Constructible Sheaves}

\begin{definition}
Let $X$ be a scheme and $\mathcal{F}$ an abelian sheaf on $X_{et}$. We say that $\mathcal{F}$ is \emph{finite locally constant} if it is represented by a finite \'etale morphism to $X$. 
\end{definition}

\begin{lemma} \label{lem:CharacterizationOfFiniteLocallyConstant}
Let $X$ be a scheme and $\mathcal{F}$ an abelian sheaf on $X_{et}$. Then the following are equivalent
\begin{enumerate}
\item 
$\mathcal{F}$ is finite locally constant ;
\item 
there exists an \'etale covering $\left\{ \mathcal{U}_i \to X\right\}_{i\in I}$ such that $\mathcal{F}|_{\mathcal{U}_i} \cong \underline{A_i}$ for some finite abelian group $A_i$.
\end{enumerate}
\end{lemma}

For a proof, see \cite{SGA4.5}.

\begin{definition}
Let $X$ be a quasi-compact and quasi-separated scheme. A sheaf $\mathcal{F}$ on $X_{et}$ is \emph{constructible} if there exists a finite decomposition of $X$ into locally closed subsets $X=\coprod_i X_i$ such that $\mathcal{F}|_{X_i}$ is finite locally constant for all $i$.
\end{definition}

\begin{lemma}  \label{lem:KerOfFiniteLocallyConstant}
The kernel and cokernel of a map of finite locally constant sheaves are finite locally constant. 
\end{lemma}

\begin{proof}
Let $\mathcal{U}$ be a connected scheme, $A$ and $B$ finite abelian groups. Then 
$$
\text{Hom}_{\text{Ab}(\mathcal{U}_{et})} \left(\underline A_\mathcal{U}, \underline B_\mathcal{U}\right) = \text{Hom}_{\text{Ab}}(A, B),
$$ 
so $\text{Ker}\left(\underline A_\mathcal{U} \xrightarrow{\varphi} \underline B_\mathcal{U}\right) = \underline{\text{Ker}(\varphi)}_\mathcal{U}$ and similarly for the cokernel. 
\end{proof}

\begin{remark} 
If $X$ is noetherian, then any constructible sheaf on $X_{et}$ is a torsion sheaf. 
\end{remark}

\begin{lemma}
Let $X$ be a noetherian scheme. Then: 
\begin{enumerate}
\item 
the category of constructible sheaves is abelian ;
\item 
it is a full exact subcategory of $\text{Ab}(X_{et})$ ;
\item 
any extension of constructible sheaves is constructible ; and
\item 
the image of a constructible sheaf is constructible.
\end{enumerate}
\end{lemma} 

\begin{proof}[Proof of i.]
Let $\varphi: \mathcal{F} \to \mathcal{G}$ be a map of constructible sheaves. By assumption, there exists a stratification $X = \coprod X_i$ such that $\mathcal{F}|_{X_i}$ and $\mathcal{G}|_{X_i}$ are finite locally constant. Since pullback if exact, we thus have $\text{Ker} \varphi|_{X_i} = \text{Ker} (\mathcal{F}|_{X_i}\xrightarrow{\varphi} \mathcal{G}|_{X_i})$ which is finite locally constant by lemma \ref{lem:KerOfFiniteLocallyConstant}. Statement {\it iv} means that if $\varphi :\mathcal{F}\to\mathcal{G}$ is a map in $\text{Ab}(X_{et})$ and $\mathcal{F}$ is constructible then $\Im(\varphi)$ is constructible. It is proven in \cite{SGA4.5}.
\end{proof}

\begin{lemma} \label{lem:EtaleRefinesToFiniteEtale}
Let $\varphi: \mathcal{U} \to X$ be an \'etale morphism of noetherian schemes. Then there exists a stratification $X=\coprod_i X_i$ such that for all $i$, $X_i\times_X \mathcal{U} \to X_i$ is finite \'etale. 
\end{lemma}

\begin{proof} 
By noetherian induction it suffices to find some nonempty open $\mathcal{V}\subseteq X$ such that $\varphi^{-1}(\mathcal{V})\to \mathcal{V}$ is finite. This follows from the following very general lemma.
\begin{lemma}[02NW in \cite{Stacks}]
Let $f: X\to Y$ be a quasi-compact and quasi-separated morphism of schemes and $\eta$ a generic point of $Y$ such that $f^{-1}(\eta)$ is finite. Then there exists an open $\mathcal{V} \subseteq Y$ containing $\eta$ such that $f^{-1}(\mathcal{V})\to \mathcal{V}$ is finite.
\end{lemma}
\end{proof}

\subsection{Extension by Zero}

\begin{definition}
Let $j: \mathcal{U} \to X$ be an \'etale morphism of schemes. The restriction functor $j^{-1}$ is right exact, so it has a left adjoint, denoted $j_! : \text{Ab}(\mathcal{U}_{et})\to \text{Ab}(X_{et})$ and called \emph{extension by zero}. Thus it is characterized by the functorial isomorphism
$$\text{Hom}_X(j_!\mathcal{F}, \mathcal{G}) = \text{Hom}_\mathcal{U}(\mathcal{F}, j^{-1}\mathcal{G})$$
for all $\mathcal{F} \in \text{Ab}(\mathcal{U}_{et})$ and $\mathcal{G} \in \text{Ab}(X_{et})$.
\end{definition}

To describe it more explicitly, recall that $j^{-1}$ is just the restriction functor $\mathcal{U}_{et}\to X_{et}$, that is, 
$$
j^{-1}\mathcal{G}(\mathcal{U}'\to \mathcal{U}) = \mathcal{G} \left(\mathcal{U}'\to \mathcal{U} \xrightarrow{j} X\right).$$
For $\mathcal{F} \in \text{Ab}(\mathcal{U}_{et})$ we consider the presheaf
$$
\begin{array}{rrcl}
j_!^{\textit{PSh}}\mathcal{F}: & X_{et} &\longrightarrow & \text{Ab}\\
& (\mathcal{V}\to X) & \longmapsto & \displaystyle \bigoplus_{\mathcal{V}\xrightarrow{\varphi} \mathcal{U}\text{ over }X} \mathcal{F}(\mathcal{V}\xrightarrow{\varphi}\mathcal{U}),
\end{array}
$$
then $j_!\mathcal{F}$ is the sheafification $\left(j_!^{\textit{PSh}}\mathcal{F}\right)^\sharp$.

\begin{exercise}
Prove directly that $j_!$ is left adjoint to $j^{-1}$ and that $j_*$ is right adjoint to $j^{-1}$.
\end{exercise}

\begin{proposition}
Let $j : \mathcal{U} \to X$ be an \'etale morphism of schemes. Then
\begin{enumerate}
\item the functors $j^{-1}$ and $j_!$ are exact ;
\item $j^{-1}$ transforms injectives into injectives ;
\item $H_{et}^p(\mathcal{U}, \mathcal{G})= H_{et}^p(\mathcal{U}, j^{-1}\mathcal{G})$ for any $\mathcal{G} \in \text{Ab}(X_{et})$
\item if $\bar x$ is a geometric point of $X$, then $\left(j_!\mathcal{F}\right)_{\bar x} =\displaystyle \bigoplus_{(\mathcal{U}, \bar u) \to (X, x)} \mathcal{F}_{\bar{u}}$.
\end{enumerate}
\end{proposition}

\begin{proof}
The functor $j^{-1}$ has both a right and a left adjoint, so it is exact. The functor $j_!$ has a right adjoint, so it is right exact. To see that it is left exact, use the description above and the fact that sheafification is exact. Property {\it ii} is standard general nonsense. In part {\it iii}, the left-hand side refers (as it should) to the right derived functors of $\mathcal{G}\mapsto \mathcal{G}(\mathcal{U})$ on $\text{Ab}(X_{et})$, and the right-hand side refers to global cohomology on $\text{Ab}(\mathcal{U}_{et})$. It is a formal consequence of {\it ii}. Part {\it iv} is again a consequence of the above description.
\end{proof}

\begin{lemma} \label{lem:shriekCommutesWithBaseChange}
Extension by zero commutes with base change. More precisely, let $f: Y \to X$ be a morphism of schemes, $j: \mathcal{V} \to X$ be an \'etale morphism and $\mathcal{F}$ a sheaf on $\mathcal{V}_{et}$. Consider the cartesian diagram 
$$
\xymatrix{
{\mathcal{V}'=Y\times_X \mathcal{V}} \ar^{f'}[d] \ar^{\qquad j'}[r] & {Y} \ar^{f}[d] \\
{\mathcal{V}} \ar^{j}[r] & {X}
}
$$
then $j'_! f'^{-1}\mathcal{F} = f^{-1}j_!\mathcal{F}$.
\end{lemma}

\begin{proof}[Sketch of proof. ]
By general nonsense, there exists a map $j'_! \circ f'^{-1} \to f^{-1}\circ j_!$. We merely verify that they agree on stalks. We have
$$
\left(j_!'f'^{-1}\mathcal{F}\right)_{\bar y}  = 
\bigoplus_{\bar v' \to \bar y} (f'^{-1}\mathcal{F})_{\bar v'} =
\bigoplus_{\bar v \to f(\bar y)} \mathcal{F}_{\bar v} = 
(j_!\mathcal{F})_{f(\bar y)} =  
(f^{-1}j_!\mathcal{F})_{\bar y}.
$$
\end{proof}

\begin{lemma} \label{lem:ShriekEqualsStarForFiniteEtale}
Let $j: \mathcal{V}\to X$ be finite and \'etale. Then $j_! = j_*$.
\end{lemma}

\begin{proof}[Sketch of proof] 
In this situation, one can again construct a map $j_! \to j_*$ although in this case it is not just by general nonsense and uses the assumptions on $j$. Again, we only check that the stalks agree. We have on the one hand
$$
(j_!\mathcal{F})_{\bar x} = 
\bigoplus_{\bar v \to \bar x} \mathcal{F}_{\bar v},
$$
and on the other hand
$$
\left(j_* \mathcal{F} \right)_{\bar x} = H_{et}^0(\text{Spec}(\mathcal{O}_{X, \bar x}^\text{sh})\times_X \mathcal{V}, \mathcal{F}).
$$
But $j$ is finite and $\mathcal{O}_{X, \bar x}$ is strictly henselian, hence $\text{Spec}(\mathcal{O}_{X, \bar x}^\text{sh})\times_X \mathcal{V}$ splits completely into spectra of strictly henselian  local rings
$$
\text{Spec}(\mathcal{O}_{X, \bar x}^\text{sh})\times_X \mathcal{V} = \coprod_{\bar v \to \bar x} \text{Spec}(\mathcal{O}_{X, \bar x}^\text{sh})
$$
and so $\left(j_* \mathcal{F} \right)_{\bar x} = \text{pr}od_{\bar v \to \bar x} \mathcal{F}_{\bar v}$ by lemma \ref{lem:shriekCommutesWithBaseChange}. Since finite products and finite coproducts agree, we get the result. Note that this last step fails if we take infinite colimits, and indeed the result is not true anymore for ind-morphisms, say.
\end{proof}

\begin{lemma} 
Let $X$ be a noetherian scheme and $j: \mathcal{U} \to X$ an \'etale, quasi-compact morphism. Then $j_!\underline{\mathbf{Z}/n\mathbf{Z}}$ is constructible on $X$. 
\end{lemma}

\begin{proof} 
By lemma \ref{lem:EtaleRefinesToFiniteEtale}, $X$ has a stratification $\coprod_i X_i$ such that $\pi_i: j^{-1}(X_i)\to X_i$ is finite \'etale, hence
$$
j_!(\underline{\mathbf{Z}/n\mathbf{Z}})|_{X_i} = 
\pi_{i!}(\underline{\mathbf{Z}/n\mathbf{Z}}) = 
\pi_{i*}(\underline{\mathbf{Z}/n\mathbf{Z}})
$$
by lemma \ref{lem:ShriekEqualsStarForFiniteEtale}. Thus it suffices to show that for $\pi: Y\to X$ finite \'etale, $\pi_*(\underline{\mathbf{Z}/n\mathbf{Z}})$ is finite locally constant. This is clear because it is the sheaf represented by $Y\times \mathbf{Z}/n\mathbf{Z}$.
\end{proof}

\begin{remark}
Using the alternative definition of finite locally constant (as in \ref{lem:CharacterizationOfFiniteLocallyConstant}), the last step is replaced by considering a Galois closure of $Y$.
\end{remark}

\begin{lemma} \label{lem:TosionSheavesAreColimitsOfConstructibleOnes}
Let $X$ be a noetherian scheme and $\mathcal{F}$ a torsion sheaf on $X_{et}$. Then $\mathcal{F}$ is a directed (filtered) colimit of constructible sheaves.
\end{lemma}

\begin{proof}[Sketch of proof]
Let $j: \mathcal{U} \to X$ in $X_{et}$ and $s\in \mathcal{F}(\mathcal{U})$ for some $\mathcal{U}$ noetherian. Then $ns = 0$ for some $n>0$. Hence we get a map $\underline{\mathbf{Z}/n\mathbf{Z}}_\mathcal{U}\to \mathcal{F}|_\mathcal{U}$, by sending $\bar 1$ to $s$. By adjointness, this gives a map $\varphi: j_!(\underline{\mathbf{Z}/n\mathbf{Z}}) \to \mathcal{F}$ whose image contains $s$. There is an element $1_{\text{id}_\mathcal{U}} \in \Gamma(\mathcal{U}, j_!\underline{\mathbf{Z}/n\mathbf{Z}})$ which maps to $s$. Thus, $\Im(\varphi) \subseteq \mathcal{F}$ is a constructible subsheaf and $s\in \Im(\varphi)(\mathcal{U})$. A similar argument applies for a finite collection of section, and the result follows by taking colimits.
\end{proof}

\subsection{Higher Vanishing for Torsion Sheaves}
\label{subsection:HigherVanishingForTorsionSheaves}

The goal of this section is to prove the result that follows now.

\begin{theorem} \label{thm:VanishingForAffineCurves}
Let $X$ be an affine curve over an algebraically closed field $k$ and $\mathcal{F}$ a torsion sheaf on $X_{et}$. Then $H_{et}^q(X, \mathcal{F}) = 0$ for all $q\geq 2$. 
\end{theorem}

We begin by reducing the proof to a more simpler statement.
\begin{enumerate}
\item
{\it If suffices to prove the vanishing when $\mathcal{F}$ is a constructible sheaf.}
\\
Using the compatibility of \'etale cohomology with colimits and lemma \ref{lem:TosionSheavesAreColimitsOfConstructibleOnes}, we have $\text{colim} H_{et}^q(X, \mathcal{F}) = H_{et}^q(X, \text{colim} \mathcal{F}_i)$ for some constructible sheaves $\mathcal{F}_i$, whence the result.
\item
{\it It suffices to assume that $\mathcal{F} = j_!\mathcal{G}$ where $\mathcal{U}\subseteq X$ is open, $\mathcal{G}$ is finite locally constant on $\mathcal{U}$ smooth.}
\\
Choose a nonempty open $\mathcal{U}\subseteq X$ such that $\mathcal{F}|_\mathcal{U}$ is finite locally constant, and consider the exact sequence
$$
0\to j_!(\mathcal{F}|_\mathcal{U})\to \mathcal{F}\to Q\to 0.
$$
By looking at stalks we get $Q_{\bar x}=0$ unless $\bar x\in X-U$. It follows that $\displaystyle Q = \bigoplus_{x\in X-U} i_{x*} (Q_x)$
which has no higher cohomology.
\item
{\it It suffices to assume that $X$ is smooth and affine (over $k$), $\mathcal{G}$ is a finite locally constant sheaf on a open $\mathcal{U}$ of $X$ and $\mathcal{F} = j_!\mathcal{G}$.}
\\
Let $\mathcal{U}$, $X$ and $\mathcal{G}$ be as in the step 2, and consider the commutative diagram
$$\xymatrix{
& {X^\nu} \ar^{\nu}[d]\\
{\mathcal{U}} \ar^{j}[r] \ar^{j^\nu}[ur] & {X}
} 
$$
where $\nu: X^\nu \to X$ is the normalization of $X$. Since $\nu$ is finite, $H_{et}^*(X, j_!\mathcal{G}) = H_{et}^*(X^\nu, j^\nu_!\mathcal{G})$, which implies that $\nu_*((j^\nu)_!\mathcal{G}) = j_!\mathcal{G}$ by looking at stalks. 
\end{enumerate}

We are thus reduced to proving the following lemma.

\begin{lemma} \label{lem:VanishingForSmoothAffineCurvesAndSimpleSheaves}
Let $X$ be a smooth affine curve over an algebraically closed field $k$, $j: \mathcal{U} \hookrightarrow X$ an open immersion and $\mathcal{F}$ a finite locally constant sheaf on $\mathcal{U}_{et}$. Then for all $q \geq 2$, $H_{et}^q(X, j_! \mathcal{F}) = 0$.
\end{lemma}

%%%%%%%%%%%%%%%%%%%%
\begin{proposition}[Topological invariance of \'etale cohomology WHERE DO I GO?] 
Let $X$ be a scheme and $X_0\hookrightarrow X$ a closed immersion defined by a nilpotent sheaf of ideals. Then the \'etale sites $X_{et}$ and $(X_0)_{et}$ are isomorphic. In particular, for any sheaf $\mathcal{F}$ on $X_{et}$, $H^q(X, \mathcal{F}) = H^q(X_0, \mathcal{F}|_{X_0})$ for all $q$.
\end{proposition}
%%%%%%%%%%%%%%%%%%%%

%10.20.09

The proof of this follows the ``m\'ethode de la trace'' as explained in SGA 4, expos\'e IX, \S5. 
\begin{definition}
Let $f : Y \to X$ be a finite \'etale morphism. There are pairs of adjoint functors $(f_!,f^{-1})$ and $(f^{-1},f_*)$ on $\text{Ab}(X_{et})$. The adjunction map $\text{id} \to f_* f^{-1}$ is called \emph{restriction}. Since $f$ is finite, $f_! = f_*$ and the adjunction map $f_* f^{-1} = f_! f^{-1} \to \text{id}$ is called the \emph{trace}.  
\end{definition}

The trace map is characterized by the following two properties:
\begin{enumerate}
\item
it commutes with \'etale localization ; and
\item
if $f: Y = \coprod_{i=1}^d X \to X$ then the trace map is just the sum map $f_* f^{-1} \mathcal{F} = \mathcal{F}^{\oplus d} \to \mathcal{F}$.
\end{enumerate}

It follows that if $f$ has constant degree $d$, then the composition $\mathcal{F} \xrightarrow{res} f_* f^{-1} \mathcal{F} \xrightarrow{trace} \mathcal{F}$ is multiplication by $d$. The ``m\'ethode'' then essentially consits in the following observation: if $\mathcal{F}$ is an abelian sheaf on $X_{et}$ such that multiplication by $d$ is an isomorphism $\mathcal{F} \cong \mathcal{F}$, and if furthermore $H_{et}^q(Y,f^{-1}\mathcal{F}) = 0$ then $H_{et}^q(X,\mathcal{F}) = 0$ as well. Indeed, multiplication by $d$ induces an isomorphism on $H_{et}^q(X, \mathcal{F})$ which factors through $H_{et}^q(Y,f^{-1}\mathcal{F})= 0$.

Using this method, we further reduce the proof of lemma \ref{lem:VanishingForSmoothAffineCurvesAndSimpleSheaves}] to a yet simpler statement.
\begin{enumerate}
\item
{\it We may assume that $\mathcal{F}$ is killed by a prime $\ell$.}
\\
Writing $\mathcal{F} = \mathcal{F}_1 \oplus \cdots \oplus \mathcal{F}_r$ where $\mathcal{F}_i$ is $\ell_i$-primary for some prime $\ell_i$, we may assume that $\ell^n$ kills $\mathcal{F}$ for some prime $\ell$. Now consider the exact sequence 
$$
0 \to \mathcal{F}[\ell] \to \mathcal{F} \to \mathcal{F}/\mathcal{F}[\ell] \to 0.
$$
Applying the exact functor $j_!$ and looking at the long exact cohomology sequence, we see that it suffices to assume that $\mathcal{F}$ is $\ell$-torsion, which we do.
\item
{\it There exists a finite \'etale morphism $f: \mathcal{V} \to \mathcal{U}$ of degree prime to $\ell$ such that $f^{-1} \mathcal{F}$ has a filtration
$$
0 \subset \mathcal{G}_1 \subset \mathcal{G}_2 \subset \cdots \subset \mathcal{G}_s = f^{-1} \mathcal{F}
$$
with $\mathcal{G}_i /\mathcal{G}_{i-1} \cong \underline{\mathbf{Z}/n\mathbf{Z}}_\mathcal{V}$ for all $i \leq s$.}
\\
Since $\mathcal{F}$ is finite locally constant, there exists a finite \'etale Galois cover $h : \mathcal{U}' \to \mathcal{U}$ such that $h^{-1} \mathcal{F} \cong \underline{A}_{\mathcal{U}'}$ for some finite abelian group $A$. Note that $A \cong (\mathbf{Z}/\ell\mathbf{Z})^{\oplus m}$ for some $m$. Saying that the cover is \emph{Galois} means that the finite group $G = \text{Aut}(\mathcal{U}' | \mathcal{U})$ has (maximal) cardinality $\# G = \deg h$. Now let $H \subseteq G$ be the $\ell$-Sylow, and set
$$
\mathcal{U}' \xrightarrow{\ \ \pi \ \ } \mathcal{V} = \mathcal{U}'/H \xrightarrow{\ \ f \ \ } \mathcal{U}.
$$
The quotient exists by taking invariants (schemes are affine). By construction, $\deg f = \#G/\#H$ is prime to $\ell$. The sheaf $\mathcal{G} = f^{-1} \mathcal{F}$ is then a finite locally constant sheaf on $\mathcal{V}$ and 
$$
\pi^{-1} \mathcal{G} = h^{-1}\mathcal{F} \cong \underline{(\mathbf{Z}/\ell\mathbf{Z})}^{\oplus m}_{\mathcal{U}'}.
$$
Moreover, 
$$
H_{et}^0(\mathcal{V}, \mathcal{G}) = H_{et}^0(\mathcal{U}', \pi^{-1}\mathcal{G})^H = \left((\mathbf{Z}/\ell\mathbf{Z})^{\oplus m}\right)^H \neq 0,
$$
where the first equality follows from writing out the sheaf condition for $\mathcal{G}$ (again, schemes are affine), and the last inequality is an exercise in linear algebra over $\mathbf{F}_\ell$. Following, we have found a subsheaf $\underline{\mathbf{Z}/\ell\mathbf{Z}}_\mathcal{V} \hookrightarrow \mathcal{G}$. Repeating the argument for the quotient $\mathcal{G}/ \underline{\mathbf{Z}/\ell\mathbf{Z}}_\mathcal{V}$ if necessary, we eventually get a subsheaf of $\mathcal{G}$ with quotient $\underline{\mathbf{Z}/\ell\mathbf{Z}}_\mathcal{V}$. This is the first step of the filtration.

\begin{exercise}
Let $f: X \to Y$ be a finite \'etale morphism with $Y$ noetherian, and $X, Y$ irreducible. Then there exists a finite \'etale Galois morphism $X' \to Y$ which dominates $X$ over $Y$.
\end{exercise}
\item
{\it We consider the normalization $Y$ of $X$ in $\mathcal{V}$, that is, we have the commutative diagram
$$
\xymatrix{
\mathcal{V} \ar^{f}[d] \ar^{j'}@{^{(}->}[r] & Y \ar^{f'}[d] \\
\mathcal{U} \ar^{j}@{^{(}->}[r] & X.
}
$$
Then there is an injection $H_{et}^q(X, j_!\mathcal{F}) \hookrightarrow H_{et}^q(Y, j'_! f^{-1} \mathcal{F})$ for all $q$.}
\\
We have seen that the composition $\mathcal{F} \xrightarrow{res} f_* f^{-1} \mathcal{F} \xrightarrow{trace} \mathcal{F}$ is multiplication by the degree of $f$, which is prime to $\ell$. On the other hand, 
$$
j_! f_* f^{-1} \mathcal{F} = j_! f_! f^{-1} \mathcal{F} = f'_* j'_! f^{-1}\mathcal{F}
$$ 
since $f$ and $f'$ are both finite and the above diagram is commutative. Hence applying $j_!$ to the previous sequence gives a sequence
$$
j_! \mathcal{F} \longrightarrow f'^* j'_! f^{-1} \mathcal{F} \longrightarrow j_! \mathcal{F}.
$$
Taking cohomology,  we see that  $H_{et}^q(X, j_!\mathcal{F})$ injects into $H_{et}^q( X , f'^* j'_! f^{-1} \mathcal{F})$. But since $f'$ is finite, this is merely $H_{et}^q( Y,  j'_! f^{-1} \mathcal{F})$, as desired.
\item
{\it It suffices to prove $H_{et}^q (Y, j'_! \underline{\mathbf{Z}/\ell\mathbf{Z}}) = 0$.}
\\
By Step 3, it suffices to show vanishing of $H_{et}^q( Y,  j'_! f^{-1} \mathcal{F})$. But then by Step 2, we may assume that $f^{-1}\mathcal{F}$ has a finite filtration with quotients isomorphic to $\underline{\mathbf{Z}/n\mathbf{Z}}$, whence the claim.
\end{enumerate}

Finally, we are reduced to proving the following lemma.

\begin{lemma} 
Let $X$ be a smooth affine curve over an algebraically closed field, $j: \mathcal{U} \hookrightarrow X$ an open immersion and $\ell$ a prime number. Then for all $q \geq 2$, $H_{et}^q(X, j_! \underline{\mathbf{Z}/\ell\mathbf{Z}}) = 0$.
\end{lemma}

\begin{proof}
Consider the short exact sequence
$$
0 \longrightarrow j_!\underline{\mathbf{Z}/\ell\mathbf{Z}}_\mathcal{U} \longrightarrow \underline{\mathbf{Z}/\ell\mathbf{Z}}_X \longrightarrow \bigoplus_{x \in X-\mathcal{U}} {i_x}_*(\underline{\mathbf{Z}/\ell\mathbf{Z}}) \longrightarrow 0.
$$
We know that the cohomology of the middle sheaf vanishes in degree at least 2 by corollary \ref{cor:VanishingOfMuForASmoothCurve} and that of the skyscraper sheaf on the right vanishes in degree at least 1. Thus applying the long exact cohomology sequence, we get the vanishing of $j_!\underline{\mathbf{Z}/\ell\mathbf{Z}}_\mathcal{U}$ in degree at least 2. This finishes the proof of the lemma, hence of lemma \ref{lem:VanishingForSmoothAffineCurvesAndSimpleSheaves}, hence of theorem \ref{thm:VanishingForAffineCurves}.
\end{proof}

\begin{remark} $ $
\begin{itemize}
\item 
This m\'ethode is very g\'en\'erale. For instance, it applies in Galois cohomology, and this is essentially how proposition \ref{prop:SerreGalois} is proved.
\item
In fact, we have overlooked the case where $\ell$ is the characteristic of the field $k$, since the Kummer sequence is not exact then and we cannot use corollary \ref{cor:VanishingOfMuForASmoothCurve} anymore. The result is still true, as shown by considering the \emph{Artin-Schreier} exact sequence for a scheme $S$ of characteristic $p >0$, namely
$$
0 \longrightarrow \underline{\mathbf{Z}/p\mathbf{Z}}_S \longrightarrow \mathbf{G}_{a,S} \xrightarrow{F-1} \mathbf{G}_{a,S} \longrightarrow 0
$$
where $F-1$ is the map $x \mapsto x^p - x$. Using this, it can be shown that is $S$ is affine then $H_{et}^q(S,\underline{\mathbf{Z}/p\mathbf{Z}}) = 0$ for all $q \geq 2$. In fact, if $X$ is projective over $k$, then $H_{et}^q(X,\underline{\mathbf{Z}/p\mathbf{Z}}) = 0$ for all $q \geq \dim X+2$.
\item
If $X$ is a projective curve over an algebraically closed field then $H_{et}^q(X,\mathcal{F}) = 0$ for all $q \geq 3$ and all torsion sheaves $\mathcal{F}$ on $X_{et}$. This can be shown using Serre's Mayer Vietoris argument, thereby proving theorem \ref{thm:vanishingForCurves}.
\end{itemize}
\end{remark}\section{The Trace Formula}

A typical course in \'etale cohomology would normally state and prove the proper and smooth base change theorems, purity and Poincar\'e duality. All of these can be found in \cite[Arcata]{SGA4.5}. Instead, we are going to study the trace formula for the frobenius, following the account of Deligne in \cite[Rapport]{SGA4.5}. We will only look at dimension 1, but using proper base change this is enough for the general case. Since all the cohomology groups considered will be \'etale, we drop the subscript $_{et}$. Let us now describe the formula we are after. Let $X$ be a finite type scheme of dimension 1 over a finite field $k$, $\ell$ a prime number and $\mathcal{F}$ a constructible, flat $\mathbf{Z}/\ell^n\mathbf{Z}$ sheaf. Then
\begin{equation*} \tag{$*$} \label{eq:TraceFormula}
\sum_{x \in X(k)} \text{Tr}(\text{Frob} | \mathcal{F}_{\bar x}) = \sum_{i=0}^2 (-1)^i \text{Tr}(\pi_X^* | H^i_c(X\otimes_k \bar k, \mathcal{F}))
\end{equation*}
as elements of $\mathbf{Z}/\ell^n\mathbf{Z}$. As we will see, this formulation is slightly wrong as stated. Let us nevertheless describe the symbols that occur therein.

\subsection{Frobenii}
\label{subsection:Frobenii}

Throughout this section, $X$ will denote a scheme of finite type over a finite field $k$ with $q = p^f$ elements. We let $\alpha : X \to \text{Spec} k$ denote the structural morphism, $\bar k$ a fixed algebraic closure of $k$ and $G_k = \text{Gal}(\bar k | k)$ the absolute Galois group of $k$.

\begin{definition}
The \emph{absolute frobenius} of $X$ is the morphism $F = F_X : X \to X$ which is the identity on the induced topological space, and which takes a section to its $p$th power. That is, $F^\sharp : \mathcal{O}_X \to \mathcal{O}_X$ is given by $g \mapsto g^p$. It is clear that this induces the identity on the topological space indeed.
\end{definition}

\begin{theorem}[The Baffling Theorem] \label{thm:TheBafflingTheorem}
Let $X$ be a scheme in characteristic $p>0$. Then the absolute frobenius induces (by pullback) the trivial map on cohomology, {\it i.e.} for all integers $j\geq 0$,
$$
F_X^* : H^j (X, \underline{\mathbf{Z}/n\mathbf{Z}}) \longrightarrow  H^j (X, \underline{\mathbf{Z}/n\mathbf{Z}})
$$
is the identity.
\end{theorem}

This theorem is purely formal. It is a good idea, however, to review how to compute the pullback of a cohomology class. Let us simply say that in the case where cohomology agrees with \u Cech cohomology, it suffices to pull back (using the fiber products on a site) the \u Cech cocycles. The general case is quite technical and can be found in \cite[somewhere]{Stacks}. A topological analogue of the baffling theorem is the following.

\begin{exercise}
Let $X$ be a topological space and $g : X \to X$ a continuous map such that $g^{-1}(U) = U$ for all opens $U$ of $X$. Then $g$ induces the identity on cohomology on $X$ (for any coefficients).
\end{exercise}

We now turn to the statement for the \'etale site.

\begin{lemma} \label{lem:FormalStuffForBafflingThm}
Let $X$ be a scheme and $g : X \to X$ a morphism. Assume that for all $\varphi: \mathcal{U} \to X$ \'etale, there is a functorial isomorphism 
$$
\xymatrix{
\mathcal{U} \ar_{\varphi}[rd] \ar^{\sim \quad}[rr] &&  {\mathcal{U} \times_{\varphi, X,g} X} \ar^{\text{pr}_2}[ld]\\
& X,
}
$$
then $g$ induces the identity on cohomology (for any sheaf). 
\end{lemma}

The proof is formal and without difficulty. To prove the theorem, we merely verify that the assumption of the lemma holds for the frobenius.

\begin{proof}[Proof of theorem \ref{thm:TheBafflingTheorem}]
We need to verify the existence of a functorial isomorphism as above. For an \'etale morphism $\varphi: \mathcal{U} \to S$, consider the diagram
$$
\xymatrix{
\mathcal{U} \ar@{-->}[rd] \ar^{F_\mathcal{U}}@/^1pc/[rrd] \ar_{\varphi}@/_1pc/[rdd] \\
& {\mathcal{U} \times_{\varphi, X,F_X} X} \ar^{\qquad \text{pr}_1 \ \ }[r] \ar^{\text{pr}_2}[d] & \mathcal{U} \ar^{\varphi}[d] \\
& X \ar^{F_X}[r] & X.
}
$$
The dotted arrow is an \'etale morphism which induces an isomorphism on the underlying topological spaces, so it is an isomorphism.
\end{proof}

%10.22.09

\begin{definition}
The \emph{geometric frobenius} of $X$ is the morphism $\pi_X : X \to X$ over $\text{Spec} k$ which equals $F_X^f$. We can base change it to any scheme over $k$, and in particular to $X_{\bar k} = \text{Spec}  \bar k \times_{\text{Spec} k} X$ to get the morphism $\text{id}_{\text{Spec} \bar k } \times \pi_X : X_{\bar k} \to X_{\bar k}$ which we denote $\pi_X$ again. This should not be ambiguous, as $X_{\bar k}$ does not have a geometric frobenius of its own.
\end{definition}

\begin{lemma}
Let $\mathcal{F}$ be a sheaf on $X_{et}$. Then there are canonical isomorphisms $\pi_X^{-1} \mathcal{F} \cong \mathcal{F}$ and $\mathcal{F} \cong {\pi_X}_* \mathcal{F}$.
\end{lemma}

This is false for, say, the flat site.

\begin{proof}
Let $\varphi: \mathcal{U} \to X$ be \'etale. Recall that ${\pi_X}_* \mathcal{F} (\mathcal{U}) = \mathcal{F} (\mathcal{U} \times{\varphi, X, \pi_X} X)$. Since $\pi_X = F_X^f$, by lemma \ref{lem:FormalStuffForBafflingThm} that there is a functorial isomorphism
$$
\xymatrix{
\mathcal{U} \ar_{\varphi}[rd] \ar^{\sim \quad}_{\gamma_\mathcal{U} \quad}[rr] &&  {\mathcal{U} \times_{\varphi, X, \pi_X} X} \ar^{\text{pr}_2}[ld]\\
& X
}
$$
where $\gamma_\mathcal{U} = (\varphi, F_\mathcal{U}^f)$. Now we define an isomorphism
$$
\mathcal{F} (\mathcal{U}) \longrightarrow {\pi_X}_* \mathcal{F} (\mathcal{U}) = \mathcal{F} (\mathcal{U} \times_{\varphi, X, \pi_X} X)
$$
by taking the restriction map of $\mathcal{F}$ along $\gamma_\mathcal{U}^{-1}$. The other isomorphism is analogous. 
\end{proof}

\begin{remark}
It may or may not be the case that $F^f_\mathcal{U}$ equals $\pi_\mathcal{U}$.
\end{remark}

Let $\mathcal{F}$ be an abelian sheaf on $X_{et}$. Consider the cohomology group $H^j (X_{\bar k}, \mathcal{F}|_{X_{\bar k}})$ as a left $G_k$-module as follows: if $\sigma \in G_k$, the diagram 
$$
\xymatrix{
X_{\bar k} \ar[rd] \ar^{\text{Spec} \sigma \times \text{id}_X}[rr] && X_{\bar k} \ar[ld]\\
& X
}
$$
commutes. Thus we can set, for $\xi \in H^j (X_{\bar k}, \mathcal{F}|_{X_{\bar k}})$
$$
\sigma \cdot \xi := (\text{Spec} \sigma \times \text{id}_X)^*\xi \in H^j(X_{\bar k}, (\text{Spec} \sigma \times \text{id}_X)^{-1} \mathcal{F}|{X_{\bar k}})
= H^j (X_{\bar k}, \mathcal{F}|_{X_{\bar k}}),
$$
where the last equality follows from the commutativity of the previous diagram. This endows the latter group with the structure of a $G_k$-module.

\begin{lemma}
Let $\mathcal{F}$ be an abelian sheaf on $X_{et}$. Consider $(R^j\alpha_*\mathcal{F})_{\text{Spec} \bar k}$ endowed with its natural Galois action as in paragraph \ref{section:GaloisActionOnStalks}. Then the identification
$$
(R^j\alpha_*\mathcal{F})_{\text{Spec} \bar k} \cong H^j (X_{\bar k}, \mathcal{F}|_{X_{\bar k}})
$$
from theorem \ref{thm:stalkOfHigherDirectImages} is an isomorphism of $G_k$-modules.
\end{lemma}

A similar result holds comparing $(R^j\alpha_!\mathcal{F})_{\text{Spec} \bar k}$ with $H^j_c (X_{\bar k}, \mathcal{F}|_{X_{\bar k}})$. We omit the proof.

\begin{definition}
The \emph{arithmetic frobenius} is the map $\text{frob}_k : \bar k \to \bar k$, $x \mapsto x^q$  of $G_k$.
\end{definition}

\begin{theorem}
Let $\mathcal{F}$ be an abelian sheaf on $X_{et}$. Then for all $j\geq 0$, $\text{frob}_k$ acts on the cohomology group $H^j(X_{\bar k}, \mathcal{F}|_{X_{\bar k}})$ as the inverse of the map $\pi_X^*$.
\end{theorem}

The map $\pi_X^*$ is defined by the composition
$$
H^j(X_{\bar k}, \mathcal{F}|_{X_{\bar k}}) \xrightarrow{{\pi_X}_{\bar k}^*}
H^j(X_{\bar k}, (\pi_X^{-1} \mathcal{F})|_{X_{\bar k}}) \cong
H^j(X_{\bar k}, \mathcal{F}|_{X_{\bar k}}).
$$

\begin{proof}
The composition $X_{\bar k} \xrightarrow{\text{Spec}(\text{frob}_k)} X_{\bar k} \xrightarrow{\pi_X} X_{\bar k}$ is equal to $F_{X_{\bar k}}^f$, hence the result follows from the baffling theorem suitably generalized to nontrivial coefficients. Note that the previous composition commutes in the sense that $F_{X_{\bar k}}^f = \pi_X \circ \text{Spec}(\text{frob}_k) = \text{Spec}(\text{frob}_k) \circ \pi_X$.
\end{proof}

\begin{definition}
If $x \in X(k)$ is a rational point and $\bar x : \text{Spec} \bar k \to X$ the geometric point lying over $x$, we let $\pi_X : \mathcal{F}_{\bar x} \to \mathcal{F}_{\bar x}$ denote the action by $\text{frob}_k^{-1}$ and call it the \emph{geometric frobenius}. This notation is not standard (this is denoted $F_x$ in \cite{SGA4.5}).
\end{definition}

We can now make a more precise statement (albeit a false one) of the trace formula (\ref{eq:TraceFormula}). Let $X$ be a finite type scheme of dimension 1 over a finite field $k$, $\ell$ a prime number and $\mathcal{F}$ a constructible, flat $\mathbf{Z}/\ell^n\mathbf{Z}$ sheaf. Then
$$
\sum_{x \in X(k)} \text{Tr}(\pi_X | \mathcal{F}_{\bar x}) = \sum_{i=0}^2 (-1)^i \text{Tr}(\pi_X^* | H^i_c(X_{\bar k}, \mathcal{F}))
$$
as elements of $\mathbf{Z}/\ell^n\mathbf{Z}$. The reason this equation is wrong is that the trace in the right-hand side does not make sense for the kind of sheaves considered. Before addressing this issue, we try to motivate the appearance of the geometric frobenius (apart from the fact that it is a natural morphism!). 

Let us consider the case where $X = \mathbf{P}^1_k$ and $\mathcal{F} = \underline{\mathbf{Z}/\ell\mathbf{Z}}$. For any point, the Galois module $\mathcal{F}_{\bar x}$ is trivial, hence for any morphism $\varphi$ acting on $\mathcal{F}_{\bar x}$, the left-hand side is
$$
\sum_{x \in X(k)} \text{Tr}(\varphi | \mathcal{F}_{\bar x}) = \#\mathbf{P}^1_k(k) = q+1.
$$
Now $\mathbf{P}^1_k$ is proper, so compactly supported cohomology equals standard cohomology, and so for a morphism $\pi : \mathbf{P}^1_k \to \mathbf{P}^1_k$, the right-hand side equals
$$
\text{Tr}(\pi^* | H^0 (\mathbf{P}^1_{\bar k}, \underline{\mathbf{Z}/\ell\mathbf{Z}})) + \text{Tr}(\pi^* | H^2 (\mathbf{P}^1_{\bar k}, \underline{\mathbf{Z}/\ell\mathbf{Z}})).
$$
The Galois module $H^0 (\mathbf{P}^1_{\bar k}, \underline{\mathbf{Z}/\ell\mathbf{Z}}) = \mathbf{Z}/\ell\mathbf{Z}$ is trivial, since the pullback of the identity is the identity. Hence the first trace is 1, regardless of $\pi$. For the second trace, we need to compute the pullback of a map $\pi: \mathbf{P}^1_{\bar k} \to \mathbf{P}^1_{\bar k}$ on $H^2 (\mathbf{P}^1_{\bar k}, \underline{\mathbf{Z}/\ell\mathbf{Z}}))$. This is a good exercise and the answer is multiplication by the degree of $\pi$. In other words, this works as in the familiar situation of complex cohomology. In particular, if $\pi$ is the geometric frobenius we get 
$$
\text{Tr}(\pi_X^* | H^2 (\mathbf{P}^1_{\bar k}, \underline{\mathbf{Z}/\ell\mathbf{Z}})) = q
$$ 
and if $\pi$ is the arithmetic frobenius then we get 
$$
\text{Tr}(\text{frob}_k^* | H^2 (\mathbf{P}^1_{\bar k}, \underline{\mathbf{Z}/\ell\mathbf{Z}})) = q^{-1}.
$$
The latter option is clearly wrong. 

\begin{remark}
The computation of the degrees can be done by lifting (in some obvious sense) to characteristic 0 and considering the situation with complex coefficients. This method almost never works, since lifting is in general impossible for schemes which are not projective space.
\end{remark}

The question remains as to why we have to consider compactly supported cohomology. In fact, in view of Poincar\'e duality, it is not strictly necessary. However, let us consider the case where $X = \mathbf{A}^1_k$ and $\mathcal{F} = \underline{\mathbf{Z}/\ell\mathbf{Z}}$. The action on stalks is again trivial, so we only need look at the action on cohomology. But then $\pi_X^*$ acts as the identity on $H^0(\mathbf{A}^1_{\bar k}, \underline{\mathbf{Z}/\ell\mathbf{Z}})$ and as multiplication by $q$ on $H^2_c(\mathbf{A}^1_{\bar k}, \underline{\mathbf{Z}/\ell\mathbf{Z}})$. 

\subsection{Traces}

We now explain how to take the trace of an endomorphism of a module over a noncommutative ring. Fix a finite ring $\Lambda$ with cardinality prime to $p$. Typically, $\Lambda$ is the group ring $(\mathbf{Z}/\ell^n\mathbf{Z})[G]$ for some finite group $G$. By convention, all the $\Lambda$-modules considered will be left $\Lambda$-modules.

Notation:
We set $\Lambda^\natural$ to be the quotient of $\Lambda$ by its additive subgroup generated by the commutators ({\it i.e.} the elements of the form $ab-ba$, $a, b \in \Lambda$). Note that $\Lambda^\natural$ is not a ring. 


For instance, the module $(\mathbf{Z}/\ell^n\mathbf{Z})[G]^\natural$ is the dual of the class functions, so 
$$
(\mathbf{Z}/\ell^n\mathbf{Z})[G]^\natural = \bigoplus_{\text{conjugacy } \atop \text{classes of $G$}} \mathbf{Z}/\ell^n\mathbf{Z}.
$$ 
For a free $\Lambda$-module, we have $\text{End}_\Lambda(\Lambda^{\oplus m}) = \text{Mat}_n(\Lambda)$. Note that since the modules are left modules, representation of endomorphism by matrices is a right action: if $a \in \text{End}(\Lambda^{\oplus m})$ has matrix $A$ and $v \in \Lambda$, then $a(v) = v A$.

\begin{definition}
The \emph{trace} of the endomorphism $a$ is the sum of the diagonal entries of a matrix representing it. This defines an additive map $\text{Tr} : \text{End}_\Lambda(\Lambda^{\oplus m}) \to \Lambda^\natural$.
\end{definition}

\begin{exercise} 
Given maps $\Lambda^{\oplus n} \xrightarrow{a} \Lambda^{\oplus n} \xrightarrow{b} \Lambda^{\oplus m}$ show that $\text{Tr}(ab) = \text{Tr}(ba)$. 
\end{exercise}

We extend the definition of the trace to a finite projective $\Lambda$-module $P$ and an endomorphism $\varphi$ of $P$ as follows. Write $P$ as the summand of a free $\Lambda$-module, {\it i.e.} consider maps $P \xrightarrow{a} \Lambda^{\oplus n} \xrightarrow{b} P$ with 
\begin{enumerate}
\item
$\Lambda^{\oplus n} = \Im a \oplus \ker b$; and
\item
$b\circ a = \text{id}_P$.
\end{enumerate}
Then we set $\text{Tr}(\varphi) = \text{Tr}(a\varphi b)$. It is easy to check that this is well-defined, using the previous exercise.

\subsubsection*{Derived categories}

With this definition of the trace, let us now discuss another issue with the formula as stated. Let $C$ be a smooth projective curve over $k$. Then there is a correspondence between finite locally constant sheaves $\mathcal{F}$ on $C_{et}$ which stalks are isomorphic to ${(\mathbf{Z}/\ell^n\mathbf{Z})}^{\oplus m}$ on the one hand, and continuous representations $\rho : \pi_1 (C,\bar c) \to \text{GL}_m(\mathbf{Z}/\ell^n\mathbf{Z}))$ (for some fixed choice of $\bar c$) on the other hand. We denote $\mathcal{F}_\rho$ the sheaf corresponding to $\rho$. Then $H^2 (C_{\bar k}, \mathcal{F}_\rho)$ is the group of coinvariants for the action of $\rho(\pi_1 (C,\bar c))$ on ${(\mathbf{Z}/\ell^n\mathbf{Z})}^{\oplus m}$, and there is a short exact sequence
$$
0 \longrightarrow \pi_1 (C_{\bar k},\bar c)  \longrightarrow \pi_1 (C,\bar c)  \longrightarrow G_k  \longrightarrow 0.
$$
For instance, let $\mathbf{Z} = \mathbf{Z} \sigma$ act on $\mathbf{Z}/\ell^2\mathbf{Z}$ via $\sigma(x) = (1+\ell) x$. The coinvariants are $(\mathbf{Z}/\ell^2\mathbf{Z})_{\sigma} = \mathbf{Z}/\ell\mathbf{Z}$, which is not a flat $\mathbf{Z}/\ell\mathbf{Z}$-module. Hence we cannot take the trace of some action on $H^2(C_{\bar k}, \mathcal{F}_\rho)$, at least not in the sense of the previous paragraph. 

In fact, our goal is to consider a trace formula for $\ell$-adic coefficients. But $\mathbf{Q}_\ell = \mathbf{Z}_\ell[1/\ell]$ and $\mathbf{Z}_\ell = \text{lim} \mathbf{Z}/\ell^n\mathbf{Z}$, and even for a flat $\mathbf{Z}/\ell^n\mathbf{Z}$ sheaf, the individual cohomology groups may not be flat, so we cannot compute traces. One possible remedy is consider the total derived complex $R\Gamma(C_{\bar k}, \mathcal{F}_\rho)$ in the derived category $D(\mathbf{Z}/\ell^n\mathbf{Z})$ and show that it is a perfect object, which means that it is quasi-isomorphic to a finite complex of finite free module. For such complexes, we can define the trace, but this will require an account of derived categories.

\subsection{Derived Categories}

To set up notation, let $\mathcal{A}$ be an abelian category. Let $\text{Comp}(\mathcal{A})$ be the abelian category of complexes in $\mathcal{A}$.  Let $K(\mathcal{A})$ be the category of complexes up to homotopy, with objects equal to complexes in $\mathcal{A}$ and objects equal to homotopy classes of morphisms of complexes. This is not an abelian category. Loosely speaking, $D(A)$ is defined to be the category obtained by inverting all quasi-isomorphisms in $\text{Comp}(\mathcal{A})$ or, equivalently, in $K(\mathcal{A})$.  Moreover, we can define $\text{Comp}^+(\mathcal{A}), K^+(\mathcal{A}), D^+(\mathcal{A})$ analogously using only bounded below complexes.  Similarly, we can define $\text{Comp}^-(\mathcal{A}), K^-(\mathcal{A}), D^-(\mathcal{A})$ using bounded above complexes, and we can define $\text{Comp}^b(\mathcal{A}), K^b(\mathcal{A}), D^b(\mathcal{A})$ using bounded complexes.

\begin{remark} $ $
\begin{itemize}
\item
There are some set-theoretical problems when $\mathcal{A}$ is somewhat arbitrary, which we will happily disregard.
\item
The categories $K(A)$ and $D(A)$ may be endowed with the structure of triangulated category, but we will not need these structures in the following discussion. 
\item
The categories $\text{Comp}(\mathcal{A})$ and $K(\mathcal{A})$ can also be defined when $\mathcal{A}$ is an additive category.
\end{itemize}
\end{remark}

The homology functor $H^i: \text{Comp}(\mathcal{A}) \to \mathcal{A}$ taking a complex $K^\bullet \mapsto H^i(K^\bullet)$ extends to functors $H^i: K(\mathcal{A}) \to \mathcal{A}$ and $H^i: D(\mathcal{A}) \to \mathcal{A}$.

\begin{lemma}
An object $E$ of $D(\mathcal{A})$ is contained in $D^+(\mathcal{A})$ if and only if $H^i(E) =0 $ for all $i \ll 0$.  Similar statements hold for $D^-$ and $D^+$.
\end{lemma}

The proof uses truncation functors.

\begin{lemma}
\begin{enumerate}
\item 
Let $I^\bullet$ be a complex in $\mathcal{A}$ with $I^n$ injective for all $n \in \mathbf{Z}$. Then 
$$
\text{Hom}_{D(\mathcal{A})}(K^\bullet, I^\bullet) 
= 
\text{Hom}_{K(\mathcal{A})}(K^\bullet, I^\bullet).
$$
\item 
Let $P^\bullet \in \text{Comp}^-(\mathcal{A})$ with $P^n$ is projective for all $n \in \mathbf{Z}$. Then 
$$
\text{Hom}_{D(\mathcal{A})}(P^\bullet, K^\bullet) 
= 
\text{Hom}_{K(\mathcal{A})}(P^\bullet, K^\bullet).
$$
\item 
If $\mathcal{A}$ has enough injectives and $\mathcal{I} \subseteq \mathcal{A}$ is the additive subcategory of injectives, then
$
D^+(\mathcal{A})\cong K^+(\mathcal{I})
$
(as triangulated categories).
\item 
If $\mathcal{A}$ has enough projectives and $\mathcal{P} \subseteq \mathcal{A}$ is the additive subcategory of projectives, then 
$
D^-(\mathcal{A}) \cong K^-(\mathcal{P}).
$
\end{enumerate}
\end{lemma}

\begin{proof}
  Omitted.
\end{proof}

\begin{definition}
Let $F: \mathcal{A} \to \mathcal{B}$ be a left exact functor and assume that $\mathcal{A}$ has enough injectives. We define the \emph{total right derived functor of $F$} as  the functor $RF: D^+(\mathcal{A}) \to D^+(\mathcal{B})$ fitting into the diagram
$$
\xymatrix{ 
D^+(\mathcal{A}) \ar[r]^{RF} & D^+(\mathcal{B}) \\
K^+(\mathcal I) \ar[u] \ar[r]^F & K^+(\mathcal{B}). \ar[u]
}
$$
This is possible since the left vertical arrow is invertible by the previous lemma.  Similarly, let $G: \mathcal{A} \to \mathcal{B}$ be a right exact functor and assume that $\mathcal{A}$ has enough projectives. We define the \emph{total right derived functor of $G$} as  the functor $LG: D^-(\mathcal{A}) \to D^-(\mathcal{B})$ fitting into the diagram 
$$
\xymatrix{ 
D^-(\mathcal{A}) \ar[r]^{LG} & D^-(\mathcal{B}) \\
K^-(\mathcal{P}) \ar[u] \ar[r]^G & K^-(\mathcal{B}). \ar[u]
}
$$
This is possible since the left vertical arrow is invertible by the previous lemma.
\end{definition}

\begin{remark} 
In these cases, it is true that $R^iF(K^\bullet) = H^i(RF(K^\bullet))$, where the left hand side is defined to be $i$th homology of the complex $F(K^\bullet)$.  
\end{remark}

\subsubsection{Filtered Derived Category}

\begin{definition}
Let $\mathcal{A}$ be an abelian category. Let $\text{Fil}(\mathcal{A})$ be the category of filtered objects $(A,F)$ of $\mathcal{A}$, where $F$ is a filtration of the form
$$
A \supseteq \cdots \supseteq F^n A \supseteq F^{n+1}A \supseteq \cdots \supseteq 0.
$$
This is an additive category. We denote $\text{Fil}^f(\mathcal{A})$ the full subcategory of $\text{Fil}(\mathcal{A})$ whose objects $(A,F)$ have finite filtration.  This is also an additive category. An object $I \in \text{Fil}^f(\mathcal{A})$ is called \emph{filtered injective} (respectively \emph{projective}) provided that $\text{gr}^p(I) = \text{gr}_F^p(I) = F^pI/F^{p+1}I$ is injective (resp. projective) in $\mathcal{A}$ for all $p$. The categories $\text{Comp}(\text{Fil}^f(\mathcal{A})) \supseteq \text{Comp}^+(\text{Fil}^f(\mathcal{A}))$ and $K(\text{Fil}^f(\mathcal{A})) \supseteq K^+(\text{Fil}^f(\mathcal A))$ are defined as before.
\\
A morphism $\alpha : K^\bullet \to L^\bullet$ of complexes in $\text{Comp}(\text{Fil}^f(\mathcal{A}))$ is called a \emph{filtered quasi-isomorphism} provided that 
$$
\mathrm{gr}^p(\alpha): \mathrm{gr}^p(K^\bullet) \to \mathrm{gr}^p(L^\bullet)
$$ 
is a quasi-isomorphism for all $p \in \mathbf{Z}$. Finally, we define $D F(\mathcal{A})$ (resp. $D F^+(\mathcal{A})$) by inverting the filtered quasi-isomorphisms in $K(\text{Fil}^f(\mathcal{A}))$ (resp. $K^+(\text{Fil}^f(\mathcal{A}))$). 
\end{definition}

\begin{lemma}
If $\mathcal{A}$ has enough injectives, then $D F^+(\mathcal{A}) \cong K^+(\mathcal{I})$, where $\mathcal{I}$ is the full additive subcategory of $\text{Fil}^f(\mathcal{A})$ consisting of filtered injective objects. Similarly, if $\mathcal{A}$ has enough projectives, then $D F^-(\mathcal{A}) \cong K^+(\mathcal{P})$, where $\mathcal P$ is the full additive subcategory of $\text{Fil}^f(\mathcal{A})$ consisting of filtered projective objects.
\end{lemma}

We omit the proof.

\subsubsection{Filtered Derived Functors}

\begin{definition}
Let $T: \mathcal{A} \to \mathcal{B}$ be a left exact functor and assume that $\mathcal{A}$ has enough injectives.  Define $RT: D F^+(\mathcal{A}) \to D F^+(\mathcal{B})$ to fit in the diagram
$$
\xymatrix{ 
D F^+(\mathcal{A}) \ar^{RT}[r] & D F^+(\mathcal{B}) \\
K^+(\mathcal{I}) \ar[u] \ar[r]^{T \quad} & K^+(\text{Fil}^f(\mathcal{B})). \ar[u]}
$$
This is well-defined by the previous lemma. Let $G: \mathcal{A} \to \mathcal{B}$ be a right exact functor  and assume that $\mathcal{A}$ has enough projectives.  Define $LG: D F^+(\mathcal{A}) \to D F^+(\mathcal{B})$ to fit in the diagram
$$
\xymatrix{ 
D F^-(\mathcal{A}) \ar^{LG}[r] & D F^-(\mathcal{B}) \\
K^-(\mathcal{P}) \ar[u] \ar[r]^{G \quad} & K^-(\text{Fil}^f(\mathcal{B})). \ar[u]}
$$
Again, this is well-defined by the previous lemma.
\end{definition}

\begin{proposition}
In the situation above, we have
$$
\mathrm{gr}^p \circ RT = RT \circ \mathrm{gr}^p
$$
where the $RT$ on the left is the filtered derived functor while the one on the right is the total derived functor. That is, there is a commuting diagram
$$
\xymatrix{
D F^+(\mathcal{A}) \ar[r]^{RT} \ar[d]_{\text{gr}^p} & D F^+(\mathcal{B}) \ar[d]^{\text{gr}^p}\\
D^+(\mathcal{A}) \ar[r]^{RT} & D^+(\mathcal{B}).}
$$
\end{proposition}

Given $K^\bullet \in D F^+(\mathcal{B})$, we get a spectral sequence 
$$
E_1^{p,q} = H^{p+q}(\text{gr}^p K^\bullet) \Rightarrow H^{p+q}(\text{forget filt}(K^\bullet)).
$$

\subsubsection{Applications}

Let $\mathcal{A}$ be an abelian category with enough injectives, and $0 \to L \to M \to N \to 0$ a short exact sequence in $\mathcal{A}$. Consider $\widetilde M \in \text{Fil}^f(\mathcal{A})$ to be $M$ along with the filtration defined by 
$$
F^1M = L, \ F^nM = M
\text{ for $n \leq 0$, and $F^nM = 0$ for $n \geq 2$.} 
$$
By definition, we have
$$
\text{forget filt}(\widetilde M) = M, \quad
\text{gr}^0(\widetilde M) = N, \quad
\text{gr}^1(\widetilde M) = L 
$$
and $\text{gr}^n(\widetilde M) = 0$ for all other $n \neq 0,1$. Let $T: \mathcal{A} \to \mathcal{B}$ be a left exact functor. Assume that $\mathcal{A}$ has enough injectives.  Then $RT(\widetilde M) \in D F^+(\mathcal{B})$ is a filtered complex with 
$$
\text{gr}^p(RT(\widetilde M)) \stackrel {\mathrm{qis}}{=} \left \{
\begin{matrix}
0 & \text{if} & p \neq 0,1, \\
RT(N) & \text{if} & p = 0, \\
RT(L) & \text{if} & p = 1.
\end{matrix}
\right . 
$$
and $\text{forget filt}(RT(\widetilde M))\stackrel{\text{qis}}{ = } RT(M)$. The spectral sequence applied to $RT(\widetilde M)$ gives
$$
E_1^{p,q} = R^{p+q}T(\mathrm{gr}^p(\widetilde M)) \Rightarrow R^{p+q}T(\text{forget filt}(\widetilde M)).
$$
Unwinding the spectral sequence gives us the long exact sequence
$$
\xymatrix{
  0 \ar[r] & T(L) \ar[r] & T(M) \ar[r] & T(N) \ar@(rd,ul)[rdllllr] \\
& R^1T(L) \ar[r] & R^1T(M) \ar[r] & \cdots
}
$$
This will be used as follows. Let $X/k$ be a scheme of finite type.  Let $\mathcal{F}$ be a flat constructible $\mathbf{Z}/\ell^n \mathbf{Z}$-module.  Then we want to show that the trace 
$$
\text{Tr}( \pi_X^\ast | R\Gamma_c(X_{\bar k}, \mathcal{F})) \in \mathbf{Z}/\ell^n \mathbf{Z}
$$
is additive on short exact sequences. To see this, it will not be enough to work with $R\Gamma_c(X_{\bar k}, -) \in D^+(\mathbf{Z}/\ell^n \mathbf{Z})$, but we will have to use the filtered derived category.

%10.29.09
\subsection{Perfectness}

Let $\Lambda$ be a (possibly noncommutative) ring, $\text{Mod}_{\Lambda}$ the category of left $\Lambda$-modules, $K(\Lambda) = K(\text{Mod}_\Lambda)$ its homotopy category, and $D(\Lambda)= D(\text{Mod}_\Lambda)$ the derived category. 

\begin{definition}
We denote by $K_{perf}(\Lambda)$ the category whose objects are bounded complexes of finite projective $\Lambda$-modules, and whose morphisms are morphisms of complexes up to homotopy. The functor $K_{perf}(\Lambda)\to D(\Lambda)$ is fully faithful, and we denote $D_{perf}(\Lambda)$ its essential image. An object of $D(\Lambda)$ is called \emph{perfect} if it is in $D_{perf}(\Lambda)$.
\end{definition}

\begin{proposition} \label{prop:TraceIsWellDefined}
Let $K\in D_{perf}(\Lambda)$ and $f\in \text{End}_{D(\Lambda)}(K)$. Then the trace $\text{Tr}(f)\in \Lambda^\natural$ is well defined. 
\end{proposition}

\begin{proof}
Let $P^\bullet$ be a bounded complex of finite projective $\Lambda$-modules and $\alpha: P^\bullet \cong K$ be an isomorphism in $D(\Lambda)$. Then $\alpha^{-1}\circ f\circ \alpha$ is the class of some morphism of complexes $f^\bullet: P^\bullet \to P^\bullet$ by ??? Set $$
\text{Tr}(f) = \sum_i (-1)^i \text{Tr}(f^i: P^i \to P^i) \in \Lambda^\natural.
$$
Given $P^\bullet$ and $\alpha$, this is independent of the choice of $f^\bullet$: any other choice is of the form $\tilde{f}^\bullet  = f^\bullet + dh +hd$ for some $h^i: P^i \to P^{i-1}(i\in \mathbf{Z})$. But
\begin{eqnarray*}
\text{Tr}(dh) & = & \sum_i (-1)^i \text{Tr}(P^i\xrightarrow{dh} P^i) \\
& = & \sum_i (-1)^i \text{Tr}(P^{i-1}\xrightarrow{hd} P^{i-1}) \\
& = & -\sum_i (-1)^{i-1}\text{Tr}(P^{i-1}\xrightarrow{hd} P^{i-1}) \\
& = &- \text{Tr}(hd)
\end{eqnarray*}
and so $\sum_i (-1)^i \text{Tr} ((dh+hd)|_{P^i})=0$. 
Furthermore, this is independent of the choice of $(P^\bullet , \alpha)$: suppose $(Q^\bullet, \beta)$ is another choice. Then by ???, the compositions  
$$ 
Q^\bullet \xrightarrow{\beta} K \xrightarrow{\alpha^{-1}} P^\bullet
\quad\text{and}\quad
P^\bullet \xrightarrow{\alpha} K \xrightarrow{\beta^{-1}} Q^\bullet
$$
are representable by morphisms of complexes $\gamma_1^\bullet$ and  $\gamma_2^\bullet$ respectively, such that $\gamma_1^\bullet \circ \gamma_2^\bullet$ is homotopic to the identity. Thus, the morphism of complexes $\gamma_2^\bullet\circ f^\bullet\circ \gamma_1^\bullet: Q^\bullet\to Q^\bullet$ represents the morphism $\beta^{-1}\circ f\circ\beta$ in $D(\Lambda)$. Now 
\begin{eqnarray*}
\text{Tr}(\gamma_2^\bullet\circ f^\bullet\circ\gamma_1^\bullet|_{Q^\bullet}) & = & \text{Tr}(\gamma_1^\bullet \circ\gamma_2^\bullet \circ f^\bullet|_{P^\bullet})\\ 
& = & \text{Tr}(f^\bullet|_{P^\bullet})
\end{eqnarray*}
by the fact that  $\gamma_1^\bullet \circ \gamma_2^\bullet$ is homotopic to the identity and the independence from $(P^\bullet, \alpha)$ already proved.
\end{proof}

\subsubsection*{Filtrations} 

We now present a filtered version of the category of perfect complexes. An object $(M,F)$ of $\text{Fil}^f(\text{Mod}_\Lambda)$ is called \emph{filtered finite projective} if for all $p$, $\text{gr}^p_F (M)$ is finite and projective. We then consider the homotopy category $K\mathrm{F}_{\text{perf}}(\Lambda)$ of bounded complexes of filtered finite projective objects of $\text{Fil}^f(\text{Mod}_\Lambda)$. We have a diagram of categories
$$
\begin{array}{ccccc}
K\mathrm{F}(\Lambda) & \supseteq & K\mathrm{F}_{\text{perf}}(\Lambda)\\
\downarrow & & \downarrow\\
D\mathrm{F}(\Lambda)  & \supseteq & D\mathrm{F}_{\text{perf}}(\Lambda)
\end{array}
$$
where the vertical functor on the right is fully faithful and the category $D\mathrm{F}_{\text{perf}}(\Lambda)$ is its essential image, as before.

\begin{lemma}[Additivity] 
Let $K\in D\mathrm{F}_{\text{perf}}(\Lambda)$ and $f\in \text{End}_{D\mathrm{F}}(K)$. Then 
$$
\text{Tr}(f|_K) = \sum_{p\in \mathbf{Z}} \text{Tr}(f|_{\text{gr}^p K}).
$$
\end{lemma}

\begin{proof} 
By proposition \ref{prop:TraceIsWellDefined}, we may assume we have a bounded complex $P^\bullet$ of filtered finite projectives of $\text{Fil}^f(\text{Mod}_\Lambda)$ and a map $f^\bullet: P^\bullet\to P^\bullet$ in $\text{Comp}(\text{Fil}^f(\text{Mod}_\Lambda))$. So the lemma follows from the following result, which proof is left to the reader.

\begin{lemma}
Let $P \in \text{Fil}^f(\text{Mod}_\Lambda)$ be filtered finite projective, and $f: P \to P$ an endomorphism in $\text{Fil}^f(\text{Mod}_\Lambda)$. Then
$$
\text{Tr}(f|_P) = \sum_p \text{Tr}(f|_{\text{gr}^p(P)}).
$$
\end{lemma}
\end{proof}

\paragraph{Characterizing Perfect Objects} 

\begin{definition}
An object $K\in D^-(\Lambda)$ is said to have \emph{finite $\text{Tor}$-dimension} if there exists $r\in \mathbf{Z}$ such that for any right $\Lambda$-module $N$, $H^i(N\otimes_{\Lambda}^\mathbf{L} K) = 0$ for all $i \leq r$ (in other words, $\text{Tor}^i_\Lambda (N, K) = 0$). Recall that $N\otimes^\mathbf{L}_{\Lambda}K$ is the total left derived functor of the functor $\text{Mod}_{\Lambda} \to \text{Ab}$, $M \mapsto N\otimes_{\Lambda} M$. It is thus a complex of abelian groups. 
\end{definition}

\begin{lemma} \label{lem:CharacterizingPerfectObjects}
Let $\Lambda$ be a left noetherian ring and $K\in D^-(\Lambda)$. Then $K$ is perfect if and only if the two following conditions hold:
\begin{enumerate}
\item
$K$ has finite $\text{Tor}$-dimension ; and
\item
for all $i \in \mathbf{Z}$, $H^i(K)$ is a finite $\Lambda$-module.
\end{enumerate} 
\end{lemma}

The reader is strongly urged to try and prove this. The proof relies on the fact that a finite module on a finitely left-presented ring is flat if and only if it is projective. 

\begin{remark}
A common variant of this lemma is to consider instead a noetherian scheme $X$ and the category $D_{perf}(\mathcal{O}_X)$ of complexes which are locally quasi-isomorphic to a finite complex of finite locally free $\mathcal{O}_X$-modules.
\end{remark}

\paragraph{The Lefschetz Trace Formula}

\begin{definition}
Let $\Lambda$ be a finite ring, $X$ a noetherian scheme,  $K(X, \Lambda)$ the homotopy category of sheaves of $\Lambda$-modules on $X_{et}$, and $D(X, \Lambda)$ the corresponding derived category. We denote by $D^b$ (respectively $D^+$, $D^-$) the full subcategory of bounded (resp. above, below) complexes in $D(X, \Lambda)$.

We consider the full subcategory $D_{ctf}^b (X, \Lambda)$ of $D^-(X, \Lambda)$ consisting of objects which are quasi-isomorphic to a bounded complex of constructible flat $\Lambda$-modules. Its objects are abusively called \emph{perfect complexes}.
\end{definition} 

\begin{remark}
In fact, a perfect complex is projective in each degree, because of the noetherian assumption on $X$. 
\end{remark}

\begin{remark}
This construction differs from the common variant mentioned above. It can happen that a complex of $\mathcal{O}_X$-modules is locally quasi-isomorphic to a finite complex of finite locally free $\mathcal{O}_X$-modules, without being globally quasi-isomorphic to a bounded complex of locally free $\mathcal{O}_X$-modules. This does not happen in the \'etale site for constructible sheaves.
\end{remark}

In this framework, lemma \ref{lem:CharacterizingPerfectObjects} reads as follows.

\begin{lemma} 
Let $K\in D^-(X, \Lambda)$. Then $K\in D_{ctf}^b(X, \Lambda)$ if and only if 
\begin{enumerate}
\item
$K$ has finite $\text{Tor}$-dimension ; and
\item
for all $i \in \mathbf{Z}$, $\underline{H}^i(K)$ is constructible.
\end{enumerate} 
\end{lemma}

The first condition can be checked on stalks (provided that the bounds are uniform).

\begin{remark} 
This lemma is used to prove that if $f: X\to Y$ is a separated, finite type morphism of schemes and $Y$ is noetherian, then $Rf_!$ induces a functor $D_{ctf}^b(X, \Lambda) \to D_{ctf}^b (Y, \Lambda)$. We only need this fact in the case where $Y$ is the spectrum of a field and $X$ is a curve. 
\end{remark}

\begin{proposition}
Let $X$ be a projective curve over a field $k$, $\Lambda$ a finite ring and $K\in D_{ctf}^b (X, \Lambda)$. Then $R\Gamma(X_{\bar k}, K)\in D_{perf}(\Lambda)$.
\end{proposition}
	
\begin{proof}[Sketch of proof.] $ $
\begin{enumerate}
\item
{\it The cohomology of $R\Gamma(X_{\bar k}, K)$ is bounded.}
\\
Consider the spectral sequence
$$
H^i(X_{\bar k}, \underline H^j(K)) \Rightarrow H^{i+j} (R\Gamma(X_{\bar k}, K)).
$$
Since $K$ is bounded and $\Lambda$ is finite, the sheaves $\underline H^j(K)$ are torsion. Moreover, $X_{\bar k}$ has finite cohomological dimension, so the left-hand side is nonzero for finitely many $i$ and $j$ only. Therefore, so is the right-hand side.
\item
{\it The cohomology groups $H^{i+j} (R\Gamma(X_{\bar k}, K))$ are finite.}
\\
Since the sheaves $\underline H^j(K)$ are constructible, the groups $H^i(X_{\bar k}, \underline H^j(K))$ are finite,\footnote{In section \ref{subsection:HigherVanishingForTorsionSheaves} where we proved vanishing of cohomology, we should have proved -- using the exact same arguments -- that \'etale cohomology with values in a torsion sheaf is finite. Maybe that section should be updated. It's flabbergasting that we even forgot to mention it.} so it follows by the spectral sequence again. 
\item
{\it $R\Gamma(X_{\bar k}, K)$ has finite $\text{Tor}$-dimension.}
\\
Let $N$ be a right $\Lambda$-module (in fact, since $\Lambda$ is finite, it suffices to assume that $N$ is finite). By the projection formula (change of module), 
$$
N\otimes^\mathbf{L}_\Lambda R \Gamma(X_{\bar k}, K) = R\Gamma(X_{\bar k}, N\otimes^\mathbf{L}_\Lambda K).
$$
Therefore,
$$
H^i (N\otimes^\mathbf{L}_\Lambda R\Gamma(X_{\bar k}, K)) = H^i(R\Gamma(X_{\bar k}, N \otimes_{\Lambda}^\mathbf{L} K)).
$$
Now consider the spectral sequence
$$
H^i (X_{\bar k}, \underline H^j (N\otimes_{\Lambda}^\mathbf{L} K))
\Rightarrow
H^{i+j}(R\Gamma(X_{\bar k}, N \otimes_{\Lambda}^\mathbf{L} K)).
$$
Since $K$ has finite $\text{Tor}$-dimension, $\underline H^j (N\otimes_{\Lambda}^\mathbf{L} K)$ vanishes universally for $j$ small enough, and the left-hand side vanishes whenever $i < 0$. Therefore $R\Gamma(X_{\bar k}, K)$ has finite $\text{Tor}$-dimension, as claimed. So it is a perfect complex by lemma \ref{lem:CharacterizingPerfectObjects}.
\end{enumerate}
\end{proof}

\subsection{Lefschetz Numbers}

The fact that the total cohomology of a constructible complex with finite coefficients is a perfect complex is the key technical reason why cohomology behaves well, and allows us to define rigorously the traces occurring in the trace formula.

\begin{definition}
Let $\Lambda$ be a finite ring, $X$ a projective curve over a finite field $k$ and $K\in D_{ctf}^b(X, \Lambda)$ (for instance $K=\underline\Lambda$). There is a canonical map $c_K: \pi_X^{-1}K \to K$, and its base change $c_K|_{X_{\bar k}}$ induces an action denoted $\pi_X^*$ on the perfect complex $R\Gamma(X_{\bar k}, K|_{X_{\bar k}})$. The \emph{global Lefschetz number} of $K$ is the trace $
\text{Tr}(\pi_X^*\left|_{R\Gamma(X_{\bar k}, K)}\right.)$ of that action. It is an element of $\Lambda^\natural$.
\\
Since $K\in D_{ctf}^b (X, \Lambda)$, for any geometric point $\bar x$ of $X$, the complex $K_{\bar x}$ is a perfect complex (in $D_{perf}(\Lambda)$). As we have seen in section \ref{subsection:Frobenii}, the Frobenius $\pi_X$ acts on $K_{\bar x}$. The \emph{local Lefschetz number} of $K$ is the sum  
$$
\sum_{x\in X(k)} \text{Tr}(\pi_X \left|_{K_{\overline x}}\right.)
$$
which is again an element of $\Lambda^\natural$.
\end{definition}

At last, we can formulate precisely the trace formula.

\begin{theorem}[Lefschetz Trace Formula] 
Let $X$ be a projective curve over a finite field $k$, $\Lambda$ a finite ring and $K \in D_{ctf}^b(X,\Lambda)$. Then the global and local Lefschetz numbers of $K$ are equal, {\it i.e.} 
$$
\text{Tr}(\pi^*_X\left|_{R\Gamma(X_{\bar k}, \Lambda)}\right.)
=
\sum_{x\in X(k)} \text{Tr}(\pi_X\left|_{K_{\bar x}}\right.)
$$
in $\Lambda^\natural$. 
\end{theorem}

%11.5.09
We will use, rather than prove, the trace formula. Nevertheless, let us discuss some ideas of proof. Since we only stated the formula for curves, it is a consequence of the following result.

\begin{theorem}[Weil] \label{thm:WeilTraceFormula}
Let $C$ be a nonsingular projective curve over an algebraically closed field $k$, and $\varphi : C \to C$ a $k$-endomorphism of $C$ distinct from the identity. Let $V(\varphi) = \Delta_C \cdot \Gamma_\varphi$, where $\Delta_C$ is the diagonal, $\Gamma_\varphi$ is the graph of $\varphi$, and the intersection number is taken on $C\times C$. Let $J = \underline{\text{Pic}}^0_{C/k}$ be the jacobian of $C$ and denote $\varphi^* : J \to J$ the action induced by $\varphi$ by taking pullbacks. Then
$$
V(\varphi) = 1 - \text{Tr}_J(\varphi^*) + \deg \varphi.
$$ 
\end{theorem}

The number $V(\varphi)$ is the number of fixed points of $\varphi$, it is equal to 
$$
V(\varphi) = \sum_{c \in |C| : \varphi(c) = c} m_\text{Fix$(\varphi)$} (c)
$$
where $m_\text{Fix$(\varphi)$} (c)$ is the multiplicity of $c$ as a fixed point of $\varphi$, namely the order or vanishing of the image of a local uniformizer under $\varphi - \text{id}_C$. Proofs of this theorem can be found in \cite{Lang, Weil}.

\begin{example}
Let $C = E$ be an elliptic curve and $\varphi = [n]$ be multiplication by $n$. Then $\varphi^* = \varphi^t$ is multiplication by $n$ on the jacobian, so it has trace $2n$ and degree $n^2$. On the other hand, the fixed points of $\varphi$ are the points $p \in E$ such that $n p = p$, which is the $(n-1)$-torsion, which has cardinality $(n-1)^2$. So the theorem reads
$$
(n-1)^2 = 1 - 2n + n^2.
$$
\end{example}

\paragraph{Jacobians}
We now discuss without proofs the correspondence between a curve and its jacobian which is used in Weil's proof. Let $C$ be a nonsingular projective curve over an algebraically closed field $k$ and choose a base point $c_0 \in C(k)$. Denote by $A^1(C \times C)$ (or $\text{Pic} (C\times C)$, or $\text{CaCl}(C\times C)$) the abelian group of codimension 1 divisors of $C\times C$. Then
$$
A^1(C\times C) = \text{pr}_1^* (A^1(C)) \oplus  \text{pr}_2^* (A^1(C)) \oplus R
$$  
where $R = \{ Z \in A^1(C\times C) \ | \ Z|_{C \times \{c_0\}} \sim_\text{rat} 0 \text{ and }  Z|_{\{c_0\} \times C} \sim_\text{rat} 0 \}$. In other words, $R$ is the subgroup of line bundles which pull back to the trivial one under either projection. Then there is a canonical isomorphism of abelian groups $R \cong \text{End}(J)$ which maps a divisor $Z$ in $R$ to the endomorphism
$$
\begin{array}{rcl}
J & \to & J \\
\left[ \mathcal{O}_C(D) \right] & \mapsto & (\text{pr}_1 |_Z)_* (\text{pr}_2 |_Z)^* (D).
\end{array}
$$
The aforementioned correspondence is the following. We denote by $\sigma $ the automorphism of $C \times C$ that switches the factors.
$$
\begin{array}{|c|c|}
\hline 
& \\
\text{End}(J)  & R 
\\ & \\
\hline
& \\
\text{composition of $\alpha, \beta$} 
& 
{\text{pr}_{13}}_* ({\text{pr}_{12}}^*(\alpha) \circ {\text{pr}_{23}}^*(\beta)) 
\\ & \\
\text{id}_J 
& 
\Delta_C - \{c_0\} \times C - C \times \{c_0\}
\\ & \\
\varphi^* 
& 
\Gamma_\varphi - C \times \{\varphi(c_0)\} - \sum_{\varphi(c) = c_0} \{c\} \times C 
\\ & \\
{\begin{array}{c}
\text{the trace form} \\
\alpha, \beta \mapsto \text{Tr}(\alpha \beta)
\end{array}}
& 
\alpha, \beta \mapsto - \int_{C\times C} \alpha . \sigma^*\beta
\\ & \\
{\begin{array}{c}
\text{the Rosati involution} \\
\alpha \mapsto \alpha^\dagger
\end{array}}
&  
\alpha \mapsto \sigma^*\alpha
\\ & \\
{\begin{array}{c}
\text{positivity of Rosati} \\
\text{Tr}(\alpha\alpha^\dagger) > 0
\end{array}}
&  
{\begin{array}{c}
\text{Hodge index theorem on $C\times C$} \\
- \int_{C\times C} \alpha \sigma^*\alpha > 0.
\end{array}}
\\ & \\
\hline
\end{array}
$$
In fact, in light of the Kunneth formula, the subgroup $R$ corresponds to the $1,1$ hodge classes in $H^1(C)\otimes H^1(C)$.

\paragraph{Weil's proof} Using this correspondence, we can prove the trace formula. We have
\begin{eqnarray*}
V(\varphi) & = & \int_{C\times C} \Gamma_\varphi.\Delta \\
& = & \int_{C\times C} \Gamma_\varphi. \left(\Delta_C - \{c_0\} \times C - C \times \{c_0\}\right) + \int_{C\times C} \Gamma_\varphi. \left(\{c_0\} \times C + C \times \{c_0\}\right).
\end{eqnarray*}
Now, on the one hand
$$
\int_{C\times C} \Gamma_\varphi. \left(\{c_0\} \times C + C \times \{c_0\}\right) 
= 
1 + \deg \varphi
$$
and on the other hand, since $R$ is the the orthogonal of the ample divisor $\{c_0\} \times C + C \times \{c_0\}$,
\begin{eqnarray*}
&&
\int_{C\times C} \Gamma_\varphi. \left(\Delta_C - \{c_0\} \times C - C \times \{c_0\}\right) \\
& = &
\int_{C\times C} \left(\Gamma_\varphi - C \times \{\varphi(c_0)\} - \sum_{\varphi(c) = c_0} \{c\} \times C \right). \left(\Delta_C - \{c_0\} \times C - C \times \{c_0\}\right) \\
& = & - \text{Tr}_J (\varphi^* \circ \text{id}_J).
\end{eqnarray*}
Recapitulating, we have
$$
V(\varphi) = 1  - \text{Tr}_J (\varphi^*) + \deg \varphi
$$
which is the trace formula.

\begin{lemma}
Consider the situation of theorem \ref{thm:WeilTraceFormula} and let $\ell$ be a prime number invertible in $k$. Then
$$
\sum_{i=0}^2 (-1)^i \text{Tr}(\varphi^* |_{H^i (C, \underline{\mathbf{Z}/\ell^n \mathbf{Z}})}) = V(\varphi) \mod \ell^n.
$$
\end{lemma}

\begin{proof}[Sketch of proof]
Observe first that the assumption makes sense because $H^i(C, \underline{\mathbf{Z}/\ell^n \mathbf{Z}})$ is a free $\mathbf{Z}/\ell^n \mathbf{Z}$-module for all $i$. The trace of $\varphi^*$ on the 0th degree cohomology is 1. The choice of a primitive $\ell^n$th root of unity in $k$ gives an isomorphism
$$
H^i(C,\underline{\mathbf{Z}/\ell^n \mathbf{Z}}) \cong H^i(C,\mu_{\ell^n})
$$
compatibly with the action of the geometric Frobenius. On the other hand, $H^1(C,\mu_{\ell^n}) = J[\ell^n]$. Therefore,
\begin{eqnarray*}
\text{Tr}(\varphi^* |_{H^1 (C, \underline{\mathbf{Z}/\ell^n \mathbf{Z}})})) & = & \text{Tr}_J (\varphi^*) \mod \ell^n \\
& = & \text{Tr}_{\mathbf{Z}/\ell^n \mathbf{Z}} (\varphi^* : J[\ell^n] \to J[\ell^n]).
\end{eqnarray*}
Moreover, $H^2(C,\mu_{\ell^n}) = \text{Pic}(C)/\ell^n\text{Pic}(C) \cong \mathbf{Z}/\ell^n \mathbf{Z}$ where $\varphi^*$ is multiplication by $\deg \varphi$. Hence
$$
\text{Tr} (\varphi^*|_{H^2 (C, \underline{\mathbf{Z}/\ell^n \mathbf{Z}})}) = \deg \varphi.
$$
Thus we have 
$$
\sum_{i=0}^2 (-1)^i \text{Tr}(\varphi^* |_{H^i (C, \underline{\mathbf{Z}/\ell^n \mathbf{Z}})}) = 1 - \text{Tr}_J(\varphi^*) + \deg \varphi \mod \ell^n
$$
and the corollary follows from theorem \ref{thm:WeilTraceFormula}.
\end{proof}

An alternative way to prove this corollary is to show that 
$$
X \mapsto H^* (X, \mathbf{Q}_\ell) = \mathbf{Q}_\ell \otimes \text{lim}_n H^*(X, \mathbf{Z}/\ell^n\mathbf{Z})
$$ 
defines a Weil cohomology theory on smooth projective varieties over $k$. Then the trace formula 
$$
V(\varphi) = \sum_{i=0}^2 (-1)^i \text{Tr}(\varphi^* |_{H^i (C,\mathbf{Q}_\ell)})
$$
is a formal consequence of the axioms (it's an exercise in linear algebra, the proof is the same as in the topological case).

%11.10.09

\subsection{Preliminaries and Sorites}

Notation:
We fix the notation for this section. We denote by $A$ a commutative ring, $\Lambda$ a (possibly noncommutative) ring with a ring map $A\to \Lambda$ which image lies in the center of $\Lambda$. We let $G$ be a finite group, $\Gamma$ a \emph{monoid extension of $G$ by $\mathbf{N}$}, meaning that there is an exact sequence
$$
1\to G\to \tilde\Gamma\to \mathbf{Z}\to 1
$$
and $\Gamma$ consists of those elements of $\tilde\Gamma$ which image is nonnegative. Finally, we let $P$ be an $A[\Gamma]$-module which is finite and projective as an $A[G]$-module, and $M$ a $\Lambda[\Gamma]$-module which is finite and projective as a $\Lambda$-module.


Our goal is to compute the trace of $1 \in \mathbf{N}$ acting over $\Lambda$ on the coinvariants of $G$ on $P\otimes_A M$, that is, the number
$$
\text{Tr}_{\Lambda}\left(1; \left(P\otimes_A M\right)_G\right) \in \Lambda^\natural.
$$
The element $1\in \mathbf{N}$ will correspond to the Frobenius.

\begin{lemma} 
Let $e\in G$ denote the neutral element. The map
$$\begin{array}{rcl}
\Lambda[G] & \longrightarrow & \Lambda^{\natural}\\
\sum \lambda_g\cdot g &\longmapsto& \lambda_e
\end{array}$$
factors through $\Lambda[G]^\natural$. We denote $\varepsilon: \Lambda[G]^\natural\to \Lambda^\natural$ the induced map.
\end{lemma}

\begin{proof}
We have to show the map annihilates commutators. One has
$$
\left(\sum\lambda_g g\right)\left(\sum\mu_g g\right)-\left(\sum \mu_g g\right)\left(\sum\lambda_g g\right) 
= \sum_g\left(\sum_{g_1g_2=g} \lambda_{g_1}\mu_{g_2}-\mu_{g_1}\lambda_{g_2}\right)g
$$
The coefficient of $e$ is
$$\sum_g\left(\lambda_g\mu_{g^{-1}}-\mu_g\lambda_{g^{-1}}\right) = \sum_g\left(\lambda_g\mu_{g^{-1}}-\mu_{g^{-1}}\lambda_g\right)$$
which is a sum of commutators, hence it it zero in $\Lambda^\natural$.
\end{proof}

\begin{definition} 
Let $f: P\to P$ be an endomorphism of a finite projective $\Lambda[G]$-module $P$. We define
$$
\text{Tr}_{\Lambda}^G(f; P) := \varepsilon\left(\text{Tr}_{\Lambda[G]}(f; P)\right).
$$
\end{definition}

\begin{lemma} \label{lem:|ambdaTrace} 
Let $f: P\to P$ be an endomorphism of the finite projective $\Lambda[G]$-module $P$. Then
$$
\text{Tr}_{\Lambda}(f; P) = \# G \cdot \text{Tr}_\Lambda^G(f; P).
$$
\end{lemma}

\begin{proof}
By additivity, reduce to the case $P=\Lambda[G]$. In that case, $f$ is given by right multiplication by some element $\sum\lambda_g\cdot g$ of $\Lambda[G]$. In the basis $(g)_{g \in G}$, the matrix of $f$ has coefficient $\lambda_{g_2^{-1}g_1}$ in the $(g_1, g_2)$ position. In particular, all diagonal coefficients are $\lambda_e$, and there are $\#G$ such coefficients.
\end{proof}

\begin{lemma} 
The map $A\to \Lambda$ defines an $A$-module structure on $\Lambda^\natural$. 
\end{lemma}

This is clear.

\begin{lemma} \label{lem:DiagonalActionProjectiveModule}
Let $P$ be a finite projective $A[G]$-module and $M$ a $\Lambda[G]$-module, finite projective as a $\Lambda$-module. Then $P\otimes_A M$ is a finite projective $\Lambda[G]$-module, for the structure induced by the diagonal action of $G$.
\end{lemma}

Note that $P\otimes_A M$ is naturally a $\Lambda$-module since $M$ is. Explictly, together with the diagonal action this reads
$$
\left(\sum\lambda_g g\right)\left(p\otimes m\right) = \sum g p\otimes \lambda_g g m.
$$

\begin{proof}
For any $\Lambda[G]$-module $N$ one has 
$$
\text{Hom}_{\Lambda[G]}\left(P\otimes_A M, N\right)= \text{Hom}_{A[G]}\left(P, \text{Hom}_{\Lambda}(M, N)\right)
$$
where the $G$-action on $\text{Hom}_{\Lambda}(M, N)$ is given by $(g\cdot \varphi)(m) = g \varphi (g^{-1} m) $. Now it suffices to observe that the right-hand side is a composition of exact functors, because of the projectivity of $P$ and $M$.
\end{proof}

\begin{lemma} \label{lem:MultiplicativityOfTrace}
With assumptions as in lemma \ref{lem:DiagonalActionProjectiveModule}, let $u\in \text{End}_{A[G]}(P)$ and $v\in \text{End}_{\Lambda[G]}(M)$. Then
$$
\text{Tr}_\Lambda^G \left(u\otimes v; P\otimes_A M\right) = \text{Tr}_A^G(u; P)\cdot \text{Tr}_\Lambda(v;M).
$$
\end{lemma}

\begin{proof}[Sketch of proof] 
Reduce to the case $P=A[G]$. In that case, $u$ is right multiplication by some element $a = \sum a_gg$ of $A[G]$, which we write $u = R_a$. There is an isomorphism of $\Lambda[G]$-modules
$$
\begin{array}{rrcl}
\varphi: &A[G]\otimes_A M &\cong &\left(A[G]\otimes_A M\right)'\\
&g\otimes m &\longmapsto & g\otimes g^{-1}m\end{array}
$$
where $\left(A[G]\otimes_A M\right)'$ has the module structure given by the left $G$-action, together with the $\Lambda$-linearity on $M$. This transport of structure changes $u \otimes v$ into $\sum_ga_gR_g\otimes g^{-1}v$. In other words,
$$
\varphi \circ (u \otimes v) \circ \varphi^{-1}
=
\sum_ga_gR_g\otimes g^{-1}v.
$$
Working out explicitly both sides of the equation, we have to show
$$
\text{Tr}_\Lambda^G\left(\sum_g a_gR_g\otimes g^{-1}v\right) = a_e\cdot \text{Tr}_\Lambda(v; M).
$$
This is done by showing that
$$\text{Tr}_\Lambda^G\left(a_gR_g\otimes g^{-1}v\right) = \left\{\begin{array}{lc} 0 & \text{ if } g\neq e\\
a_e\text{Tr}_\Lambda\left(v; M\right) & \text{ if }g=e\end{array}\right.$$
by reducing to $M=\Lambda$. 
\end{proof}

Notation: 
Consider the monoid extension $1 \to G\to \Gamma\to \mathbf{N} \to 1$ and let $\gamma\in \Gamma$. Then we write $Z_\gamma = \{g\in G\ | \ g\gamma=\gamma g\}$.


\begin{lemma} \label{lem:GammaZGammaTrace}
Let $P$ be a $\Lambda[\Gamma]$-module, finite and projective as a $\Lambda[G]$-module, and $\gamma \in \Gamma$. Then
$$
\text{Tr}_{\Lambda}(\gamma; P) = \# Z_\gamma \cdot \text{Tr}_\Lambda^{Z_\gamma}\left(\gamma; P\right).
$$	
\end{lemma}

\begin{proof} 
This follows readily from lemma \ref{lem:|ambdaTrace}. 
\end{proof}

\begin{lemma} \label{lem:AWeakTraceFormula}
Let $P$ be an $A[\Gamma]$-module, finite projective as $A[G]$-module. Let $M$ be a $\Lambda[\Gamma]$-module, finite projective as a $\Lambda$-module. Then 
$$
\text{Tr}_{\Lambda}^{Z_\gamma}(\gamma; P\otimes_A M) = \text{Tr}_A^{Z_\gamma} (\gamma;P)\cdot \text{Tr}_\Lambda(\gamma; M).
$$
\end{lemma}

\begin{proof}
This follows directly from lemma \ref{lem:MultiplicativityOfTrace}.
\end{proof}

\begin{lemma} \label{lem:TraceOf1}
Let $P$ be a $\Lambda[\Gamma]$-module, finite projective as $\Lambda[G]$-module. Then the coinvariants $P_G = \Lambda\otimes_{\Lambda[G]} P$
form a finite projective $\Lambda$-module, endowed with an action of $\Gamma/G = \mathbf{N}$. Moreover, we have
$$
\text{Tr}_\Lambda(1; P_G) = {\sum_{\gamma\mapsto 1}}' \text{Tr}_\Lambda^{Z_\gamma}(\gamma;P)
$$
where $\sum_{\gamma\mapsto 1}'$ means taking the sum over the $G$-conjugacy classes in $\Gamma$. 
\end{lemma}

\begin{proof}[Sketch of proof] 
We first prove this after multiplying by $\#G$. 
$$
\# G\cdot \text{Tr}_\Lambda(1; P_G)  
=  \text{Tr}_\Lambda\bigg(\sum_{\gamma\mapsto 1} \gamma; P_G\bigg) 
=  \text{Tr}_\Lambda\bigg(\sum_{\gamma\mapsto 1} \gamma; P\bigg)
$$
where the second equality follows by considering the commutative triangle 
$$
\xymatrix{
{P^G \ } \ar^a@{^(->}[r] & P \ar^b@{->>}[r] & P_G \ar^c@/^1pc/[ll]
}
$$
where $a$ is the canonical inclusion, $b$ the canonical surjection and $c = \sum_{\gamma \mapsto 1} \gamma$. Then we have
$$
\bigg(\sum_{\gamma \mapsto 1} \gamma\bigg) \bigg|_P = a \circ c \circ b 
\qquad \text{and} \qquad
\bigg(\sum_{\gamma \mapsto 1} \gamma\bigg) \bigg|_{P_G} = b \circ a \circ c 
$$
hence they have the same trace. We then have
$$
\# G\cdot \text{Tr}_\Lambda(1; P_G)
 =
{\sum_{\gamma\mapsto 1}}' \frac{\# G}{\# Z_\gamma}\text{Tr}_\Lambda(\gamma;P)
 =  \# G{\sum_{\gamma\mapsto 1}}' \text{Tr}_\Lambda^{Z_\gamma}(\gamma; P).
$$
To finish the proof, reduce to case $\Lambda$ torsion-free by some universality argument. See \cite{SGA4.5} for details.
\end{proof}

Let us try to illustrate the content of formula \ref{lem:AWeakTraceFormula}. Suppose that $\Lambda$, viewed as a trivial $\Gamma$-module, admits a finite resolution 
$
0\to P_r\to \ldots \to P_1 \to P_0\to \Lambda\to 0
$
by some $\Lambda[\Gamma]$-modules $P_i$ which are finite and projective as $\Lambda[G]$-modules. In that case
$$ 
H_*\left(\left(P_\bullet\right)_G\right) = \text{Tor}_*^{\Lambda[G]}\left(\Lambda, \Lambda\right) = H_*(G, \Lambda) ; 
$$
and
$$
\text{Tr}_\Lambda^{Z_\gamma}\left(\gamma; P_\bullet\right) =\frac{1}{\# Z_\gamma}\text{Tr}_\Lambda(\gamma; P_\bullet)=\frac{1}{\# Z_\gamma}\text{Tr}(\gamma;\Lambda) = \frac{1}{\# Z_\gamma}.
$$
Therefore, lemma \ref{lem:AWeakTraceFormula} says
$$
\text{Tr}_\Lambda (1 ; P_G) 
= \text{Tr}\left(1\left|_{H_*(G, \Lambda)}\right.\right) 
= {\sum_{\gamma\mapsto1}}'\frac{1}{\# Z_\gamma}.
$$
This can be interpreted as a point count on the stack $BG$. If $\Lambda= \mathbf{F}_\ell$ with $\ell$ prime to $\#G$, then $H_*(G, \Lambda)$ is $\mathbf{F}_\ell$ in degree 0 (and 0 in other degrees) and the formula reads 
$$
1 = \sum_{\sigma\text{-conjugacy}\atop \text{classes }\left<\gamma\right>}\frac{1}{\# Z_\gamma}\mod{\ell}.
$$

%11.12.09

\subsection{Proof of the Trace Formula}

\begin{theorem} \label{thm:SomeTraceFormulaInDim1}
Let $k$ be a finite field and $X$ a finite type, separated scheme of dimension at most 1 over $k$. Let $\Lambda$ be a finite ring whose cardinality is prime to that of $k$, and $K\in D_{ctf}^b(X, \Lambda)$. Then
\begin{equation} \label{eq:SomeTraceFormulaInDim1}
\text{Tr}\left(\pi_X^*\big|_{R\Gamma_c(X_{\bar k}, K)}\right) = \sum_{x\in X(k)} \text{Tr}\left(\pi_x\big|_{K_{\bar x}}\right)
\end{equation}
in $\Lambda^{\natural}$. 
\end{theorem}

\begin{remark} $ $
\begin{itemize}
\item 
This formula holds in any dimension. By a d\'evissage lemma (which uses proper base change etc.) it reduces to the current statement -- in that generality.
\item 
The complex $R\Gamma_c(X_{\bar k}, K)$ is defined by choosing an open immersion $j: X \hookrightarrow \bar X$ with $\bar X$ projective over $k$ of dimension at most 1 and setting
$$
R\Gamma_c(X_{\bar k}, K) := R\Gamma(\bar X_{\bar k}, j_!K).
$$
That this is independent of the choice made follows from (the missing section).
\end{itemize}
\end{remark}

Notation:
For short, we write $T'(X, K) = \sum_{x\in X(k)} \text{Tr}\big(\pi_x\big|_{K_{\bar x}}\big)$ for the right-hand side of (\ref{eq:SomeTraceFormulaInDim1}) and $T''(X, K) =\text{Tr}\big(\pi_x^*\big|_{R\Gamma_c(X_{\bar k}, K)}\big)$ for the left-hand side.


\begin{proof}[Proof of theorem \ref{thm:SomeTraceFormulaInDim1}] 
$ $
\begin{enumerate}
\item
{\it Let $j: \mathcal{U}\hookrightarrow X$ be an open immersion with complement $Y = X - \mathcal{U}$ and $i: Y \hookrightarrow X$. Then
\begin{enumerate}
\item $T''(X, K) = T''(\mathcal{U}, j^{-1} K)+ T''(Y, i^{-1}K)$ ; and
\item $T'(X, K) = T'(\mathcal{U}, j^{-1} K)+ T'(Y, i^{-1}K)$.
\end{enumerate}}
This is clear for $T'$. For {\it (a)}, use the exact sequence 
$$
0\to j_!j^{-1} K \to K \to i_* i^{-1} K \to 0
$$ 
to get a filtration on $K$. This gives rise to an object $\widetilde K \in D\mathrm{F}(X, \Lambda)$ whose graded pieces are $j_!j^{-1}K$ and $i_*i^{-1}K$, both of which lie in $D_{ctf}^b(X, \Lambda)$. Then, by filtered derived abstract nonsense (INSERT REFERENCE), $R\Gamma_c(X_{\bar k}, K)\in D\mathrm{F}_{perf}(\Lambda)$, and it comes equipped with $\pi_x^*$ in $D\mathrm{F}_{perf}(\Lambda)$. By the discussion of traces on filtered complexes (INSERT REFERENCE) we get
\begin{eqnarray*}
\text{Tr}\left(\pi_X^*\big|_{R\Gamma_c(X_{\bar k}, K)}\right) 
& = & \text{Tr}\left(\pi_X^*\big|_{R\Gamma_c(X_{\bar k}, j_!j^{-1}K)}\right)+ \text{Tr}\left(\pi_X^*\big|_{R\Gamma_c(X_{\bar k}, i_*i^{-1}K)}\right)
\\
& = & T''(U, i^{-1}K) + T''(Y, i^{-1}K).
\end{eqnarray*}
\item 
{\it The theorem holds if $\dim X\leq 0$. }
\\
Indeed, in that case
$$
R\Gamma_c(X_{\bar k}, K)  = R\Gamma(X_{\bar k}, K) = \Gamma(X_{\bar k}, K) = \bigoplus_{\bar x\in X_{\bar k}} K_{\bar x} \leftarrow \pi_X*.
$$
Since the fixed points of $\pi_X: X_{\bar k}\to X_{\bar k}$ are exactly the points $\bar x\in X_{\bar k}$ which lie over a $k$-rational point $x\in X(k)$ we get
$$
\text{Tr}\big(\pi_X^*|_{R\Gamma_c(X_{\bar k})}\big) = \sum_{x\in X(k)}\text{Tr}(\pi_{\bar x}|_{K_{\bar x}}).
$$
\item 
{\it It suffices to prove the equality $T'(\mathcal{U}, \mathcal{F}) = T''(\mathcal{U}, \mathcal{F})$ in the case where 
\begin{itemize}
\item $\mathcal{U}$ is a smooth irreducible affine curve over $k$ ;
\item $\mathcal{U}(k) = \emptyset$ ;
\item $K=\mathcal{F}$ is a finite locally constant sheaf of $\Lambda$-modules on $\mathcal{U}$ whose stalk(s) are finite projective $\Lambda$-modules ; and
\item $\Lambda$ is killed by a power of a prime $\ell$ and $\ell \in k^*$. 
\end{itemize}
}
Indeed, because of Step 2, we can throw out any finite set of points. But we have only finitely many rational points, so we may assume there are none\footnote{At this point, there should be an evil laugh in the background.}. We may assume that $\mathcal{U}$ is smooth irreducible and affine by passing to irreducible components and throwing away the bad points if necessary. The assumptions of $\mathcal{F}$ come from unwinding the definition of $D_{ctf}^b(X, \Lambda)$ and those on $\Lambda$ from considering its primary decomposition.
\end{enumerate}

For the remainder of the proof, we consider the situation
$$
\xymatrix{
{\mathcal{V} \ } \ar_f[d] \ar@{^(->}[r] & Y \ar^{\bar f}[d] \\	
{\mathcal{U} \ } \ar@{^(->}[r] & X
}
$$
where $\mathcal{U}$ is as above, $f$ is a finite \'etale Galois covering, $\mathcal{V}$ is connected and the horizontal arrows are projective completions. Denoting $G=\text{Aut}(\mathcal{V}|\mathcal{U})$, we also assume (as we may) that $f^{-1}\mathcal{F} =\underline M$ is constant, where the module $M = \Gamma(\mathcal{V}, f^{-1}\mathcal{F})$ is a $\Lambda[G]$-module which is finite and projective over $\Lambda$. This corresponds to the trivial monoid extension 
$$
1\to G\to \Gamma=G\times \mathbf{N}\to \mathbf{N}\to 1.
$$
In that context, using the reductions above, we need to show that $T''(\mathcal{U},\mathcal{F})=0$. We now present a series of lemmata in order to complete the proof.
\begin{enumerate}
\item
{\it There is a natural action of $G$ on $f_*f^{-1}\mathcal{F}$ and the trace map $f_*f^{-1}\mathcal{F}\to \mathcal{F}$ defines an isomorphism
$$
(f_*f^{-1}\mathcal{F})\otimes_{\Lambda[G]} \Lambda=(f_*f^{-1}\mathcal{F})_G \cong \mathcal{F}.
$$
}
\\
To prove this, simply unwind everything at a geometric point.
\item
{\it
Let $A=\mathbf{Z}/\ell^n \mathbf{Z}$ with $n\gg 0$. Then $f_*f^{-1}\mathcal{F} \cong (f_*\underline A)\otimes_{\underline A} \underline M$ with diagonal $G$-action.
}
\item
{\it
There is a canonical isomorphism $(f_* \underline A\otimes_{\underline A} \underline M)\otimes_{\Lambda[G]} \cong\mathcal{F}$.
}\\
In fact, this is a derived tensor product, because of the projectivity assumption on $\mathcal{F}$.
\item
{\it
There is a canonical isomorphism $R\Gamma_c(\mathcal{U}_{\bar k}, \mathcal{F}) = (R\Gamma_c(\mathcal{U}_{\bar k}, f_*A)\otimes_A^\mathbf{L} M)\otimes_{\Lambda[G]}^\mathbf{L} \Lambda$, compatible with the action of $\pi^*_\mathcal{U}$..
}
\\
This comes from the universal coefficient theorem, {\it i.e.} the fact that $R\Gamma_c$ commutes with $\otimes^\mathbf{L}$, and the flatness of $\mathcal{F}$ as a $\Lambda$-module. 
\end{enumerate}

We have
\begin{eqnarray*}
\text{Tr}(\pi_\mathcal{U}^*\big|_{R\Gamma_c(\mathcal{U}_{\bar k}, \mathcal{F})}) & = &  {\sum_{g\in G}}'\text{Tr}_{\Lambda}^{Z_g}\left((g,\pi_\mathcal{U}^*)\big|_{R\Gamma_c(\mathcal{U}_{\bar k}, f_*A)\otimes_A^\mathbf{L} M}\right)\\
& = &  {\sum_{g\in G}}' \text{Tr}_A^{Z_g}((g, \pi_\mathcal{U}^*)\big|_{R\Gamma_c(\mathcal{U}_{\bar k}, f_*A)})\cdot \text{Tr}_\Lambda(g|_M)
\end{eqnarray*}
where $\Gamma$ acts on $R\Gamma_c(\mathcal{U}_{\bar k}, \mathcal{F})$ by $G$ and $(e, 1)$ acts via $\pi_\mathcal{U}^*$. So the monoidal extension is given by $\Gamma = G \times \mathbf{N} \to \mathbf{N}$, $\gamma \mapsto 1$. The first equality follows from lemma \ref{lem:TraceOf1} and the second from lemma \ref{lem:AWeakTraceFormula}.

\begin{enumerate}
\setcounter{enumi}{4}
\item 
{\it  
It suffices to show that  $\text{Tr}_A^{Z_g}\left((g, \pi_\mathcal{U}^*)\big|_{R\Gamma_c(\mathcal{U}_{\bar k}, f_*A)}\right) \in A$ maps to zero in $\Lambda$. 
}\\
Recall that 
\begin{eqnarray*}
\# Z_g \cdot \text{Tr}_A^{Z_g}\left((g, \pi_\mathcal{U}^*)\big|_{R\Gamma_c(\mathcal{U}_{\bar k}, f_*A)}\right) 
& = & \text{Tr}_A\left((g, \pi_\mathcal{U}^*)\big|_{R\Gamma_c(\mathcal{U}_{\bar k}, f_*A)}\right)\\
& = & \text{Tr}_A\left((g^{-1}\pi_\mathcal{V})^*\big|_{R\Gamma_c(\mathcal{V}_{\bar k}, A)}\right).
\end{eqnarray*}
The first equality is lemma \ref{lem:GammaZGammaTrace}, the second is the Leray spectral sequence, using the finiteness of $f$ and the fact that we are only taking traces over $A$. Now since $A=\mathbf{Z}/\ell^n\mathbf{Z}$ with $n\gg 0$ and $\# Z_g=\ell^a$ for some (fixed) $a$, it suffices to show the following result.
\item
{\it $\text{Tr}_A\left((g^{-1}\pi_\mathcal{V})^*\big|_{R\Gamma_c(\mathcal{V}, A)}\right) = 0$ in $A$.} \\
By additivity again, we have
\begin{eqnarray*}
&  \text{Tr}_A\left((g^{-1}\pi_\mathcal{V})^*\big|_{R\Gamma_c(\mathcal{V}_{\bar k}, A)}\right)+\text{Tr}_A\left((g^{-1}\pi_\mathcal{V})^*\big|_{R\Gamma_c(Y-\mathcal{V})_{\bar k}, A)}\right) 
\\
& = \ \text{Tr}_A\left((g^{-1}\pi_Y)^*\big|_{R\Gamma(Y_{\bar k}, A)}\right)
\end{eqnarray*}
The latter trace is the number of fixed points of $g^{-1}\pi_Y$ on $Y$, by Weil's trace formula \ref{thm:SomeTraceFormulaInDim1}. Moreover, by the 0-dimensional case already proven in step 2, 
$$
\text{Tr}_A\left((g^{-1}\pi_\mathcal{V})^*\big|_{R\Gamma_c(Y-\mathcal{V})_{\bar k}, A)}\right)
$$ 
is the number of fixed points of $g^{-1}\pi_Y$ on $(Y-\mathcal{V})_{\bar k}$. Therefore, 
$$
\text{Tr}_A\left((g^{-1}\pi_\mathcal{V})^*\big|_{R\Gamma_c(\mathcal{V}_{\bar k}, A)}\right)
$$ 
is the number of fixed points of $g^{-1}\pi_Y$ on $\mathcal{V}_{\bar k}$. But there are no such points: if $\bar y\in Y_{\bar k}$ is fixed under $g^{-1}\pi_Y$, then $\bar f(\bar y) \in X_{\bar k}$ is fixed under $\pi_X$. But $\mathcal{U}$ has no $k$-rational point, so we must have $\bar f(\bar y)\in (X-\mathcal{U})_{\bar k}$ and so $\bar y\notin \mathcal{V}_{\bar k}$, a contradiction. 
\end{enumerate}
This finishes the proof.
\end{proof}

\begin{remark}
Even though all we did are reductions and mostly algebra, the trace formula \ref{thm:SomeTraceFormulaInDim1} is much stronger than Weil's geometric trace formula (theorem \ref{thm:WeilTraceFormula}) because it applies to coefficient systems (sheaves), not merely constant coefficients. 
\end{remark}
%11.17.09

\section{Applications}

%$L$-functions

\subsection{$\ell$-adic sheaves}

\begin{definition}
Let $X$ be a noetherian scheme. A \emph{$\mathbf{Z}_\ell$-sheaf} on $X$ is an inverse system $\left\{\mathcal{F}_n\right\}_{n\geq 1}$ where
\begin{itemize}
\item 
$\mathcal{F}_n$ is a constructible $\mathbf{Z}/\ell^n\mathbf{Z}$-module on $X_{et}$, and
\item 
the transition maps $\mathcal{F}_{n+1}\to \mathcal{F}_n$ induce isomorphisms $\mathcal{F}_{n+1}\otimes_{\mathbf{Z}/\ell^{n+1}\mathbf{Z}} \mathbf{Z}/\ell^n\mathbf{Z} \cong \mathcal{F}_n$.
\end{itemize}
We say that $\mathcal{F}$ is \emph{lisse} if each $\mathcal{F}_n$ is locally constant. A \emph{morphism} of such is merely a morphism of inverse systems.
\end{definition}

\begin{lemma} \label{lem:EventuallyCstInverseSystems}
Let $\{\mathcal{G}_n\}_{n\geq 1}$ be an inverse system of constructible $\mathbf{Z}/\ell^n\mathbf{Z}$-modules. Suppose that for all $k\geq 1$, the maps
$$
\mathcal{G}_{n+1}/\ell^k \mathcal{G}_{n+1}\to \mathcal{G}_n /\ell^k \mathcal{G}_n
$$
are isomorphisms for all $n\gg 0$ (where the bound possibly depends on $k$). In other words, assume that the system $\{\mathcal{G}/\ell^k\mathcal{G}_n\}_{n\geq 1}$ is eventually constant, and call $\mathcal{F}_k$ the corresponding sheaf. Then the system $\left\{\mathcal{F}_k\right\}_{k\geq 1}$ forms a $\mathbf{Z}_\ell$-sheaf on $X$.
\end{lemma}

The proof is obvious.

\begin{lemma} 
The category of $\mathbf{Z}_\ell$-sheaves on $X$ is abelian.
\end{lemma}

\begin{proof} 
Let  $\Phi=\left\{\varphi_n\right\}_{n\geq 1}: \left\{\mathcal{F}_n\right\}\to \left\{\mathcal{G}_n\right\}$ be a morphism of $\mathbf{Z}_\ell$-sheaves. Set
$$
\text{Coker} \Phi := \left\{ \text{Coker}\left(\mathcal{F}_n\xrightarrow{\varphi_n} \mathcal{G}_n\right)\right\}_{n\geq 1}
$$
and $\text{Ker}\Phi$ is the result of lemma \ref{lem:EventuallyCstInverseSystems} applied to the inverse system
$$
\left\{\bigcap_{m\geq n} \Im \left( \text{Ker} \varphi_m \to \text{Ker} \varphi_n \right)\right\}_{n\geq 1}.
$$
That this defines an abelian category is left to the reader.
\end{proof}

\begin{example}
Let $X=\text{Spec}(\mathbf{C})$ and $\Phi : \mathbf{Z}_\ell\to \mathbf{Z}_\ell$ be multiplication by $\ell$. More precisely,
$$
\Phi = \left\{ \mathbf{Z}/\ell^n\mathbf{Z} \xrightarrow{\ell} \mathbf{Z}/\ell^n\mathbf{Z}\right\}_{n \geq 1}.
$$ 
To compute the kernel, we consider the inverse system 
$$
\cdots\to \mathbf{Z}/\ell\mathbf{Z}\xrightarrow{0} \mathbf{Z}/\ell\mathbf{Z}\xrightarrow{0}\mathbf{Z}/\ell\mathbf{Z}.
$$
Since the images are always zero, $\text{Ker} \Phi$ is zero as a system. 
\end{example}

\begin{remark} 
If $\mathcal{F} = \left\{\mathcal{F}_n\right\}_{n\geq 1}$ is a $\mathbf{Z}_\ell$-sheaf on $X$ and $\bar x$ is a geometric point then $M_n=\left\{\mathcal{F}_{n, \bar x}\right\}$ is an inverse system of finite $\mathbf{Z}/\ell^n\mathbf{Z}$-modules such that $M_{n+1}\to M_n$ is surjective and $M_n=M_{n+1}/\ell^n M_{n+1}$. It follows that
$$
M= \text{lim}_n M_n = \text{lim} \mathcal{F}_{n, \bar x}
$$
is a finite $\mathbf{Z}_\ell$-module. Indeed, $M/\ell M= M_1$ is finite over $\mathbf{F}_\ell$, so by Nakayama $M$ is finite over $\mathbf{Z}_\ell$. Therefore, $M\cong \mathbf{Z}_\ell^{\oplus r} \oplus \oplus_{i=1}^t \mathbf{Z}_\ell/\ell^{e_i}\mathbf{Z}_\ell$ for some $r, t\geq 0$, $e_i\geq 1$. The module $M = \mathcal{F}_{\bar x}$ is called the \emph{stalk} of $\mathcal{F}$ at $\bar x$.
\end{remark}

\begin{definition} 
A $\mathbf{Z}_\ell$-sheaf $\mathcal{F}$ is \emph{torsion} if $\ell^n: \mathcal{F} \to \mathcal{F}$ is the zero map for some $n$. The abelian category of $\mathbf{Q}_\ell$-sheaves on $X$ is the quotient of the abelian category of $\mathbf{Z}_\ell$-sheaves by the Serre subcategory of torsion sheaves.  In other words, its objects are $\mathbf{Z}_\ell$-sheaves on $X$, and if $\mathcal{F}, \mathcal{G}$ are two such, then
$$
\text{Hom}_{\mathbf{Q}_\ell} \left(\mathcal{F}, \mathcal{G} \right) = \text{Hom}_{\mathbf{Z}_\ell} \left(\mathcal{F}, \mathcal{G}\right) \otimes_{\mathbf{Z}_\ell} \mathbf{Q}_\ell.
$$
We denote by $\mathcal{F} \mapsto \mathcal{F} \otimes \mathbf{Q}_\ell$ the quotient functor (right adjoint to the inclusion). If $\mathcal{F} = \mathcal{F}' \otimes \mathbf{Q}_\ell$ where $\mathcal{F}'$ is a $\mathbf{Z}_\ell$-sheaf and $\bar x$ is a geometric point, then the \emph{stalk} of $\mathcal{F}$ at $\bar x$ is $\mathcal{F}_{\bar x} = \mathcal{F}'_{\bar x} \otimes \mathbf{Q}_\ell$.
\end{definition}

\begin{remark}
Since a $\mathbf{Z}_\ell$-sheaf is only defined on a noetherian scheme, it is torsion if and only if its stalks are torsion. 
\end{remark}

\begin{definition} 
If $X$ is a separated scheme of finite type over an algebraically closed field $k$ and $\mathcal{F} = \left\{\mathcal{F}_n\right\}_{n\geq 1}$ is a $\mathbf{Z}_\ell$-sheaf on $X$, then we define
$$
H^i(X, \mathcal{F}) := \text{lim}_n H^i(X, \mathcal{F}_n)
\qquad\text{ and }\qquad
H_c^i(X, \mathcal{F}) := \text{lim}_n H_c^i(X, \mathcal{F}_n).
$$
If $\mathcal{F} = \mathcal{F}'\otimes \mathbf{Q}_\ell$ for a $\mathbf{Z}_\ell$-sheaf  $\mathcal{F}'$ then we set
$$
H_c^i(X ,\mathcal{F}) := H_c^i(X, \mathcal{F}')\otimes_{\mathbf{Z}_\ell}\mathbf{Q}_\ell.
$$
\end{definition}

\subsection{$L$-functions} 

\begin{definition}
Let $X$ be a scheme of finite type over a finite field $k$. Let $\Lambda$ be a finite ring of order prime to the characteristic of $k$ and $\mathcal{F}$ a constructible flat $\Lambda$-module on $X_{et}$. Then we set
$$
L(X, \mathcal{F}) := \text{pr}od_{x\in |X|}\det\left(1-\pi_x^*\ T^{\deg x}\Big|_{\mathcal{F}_{\bar x}}\right)^{-1}\in \Lambda [[ T ]]
$$
where $|X|$ is the set of closed points of $X$, $\deg x = [\kappa(x): k]$ and $\bar x$ is a geometric point lying over $x$. This definition clearly generalizes to the case where $\mathcal{F} =K \in D_{ctf}^b(X, \Lambda)$. 

\begin{remark}
Intuitively, $T$ should be thought of as $T = t^f$ where $p^f = \# k$. The definitions are then independent of the size of the ground field.
\end{remark}

Now assume that $\mathcal{F}$ is a $\mathbf{Q}_\ell$-sheaf on $X$. We define
$$
L(X, \mathcal{F}) := \text{pr}od_{x\in |X|}\det\left(1-\pi_x^*\ T^{\deg x}\Big|_{\mathcal{F}_{\bar x}}\right)^{-1}\in \mathbf{Q}_\ell [[ T ]].
$$
Note that this product converges since there are finitely many points of a given degree.
\end{definition}

\subsubsection*{Cohomological Interpretation} 

This is how Grothendieck interpreted the $L$-function.

\begin{theorem}[Finite Coefficients] \label{thmA} 
Let $X$ be a scheme of finite type over a finite field $k$. Let $\Lambda$ be a finite ring of order prime to the characteristic of $k$ and $\mathcal{F}$ a constructible flat $\Lambda$-module on $X_{et}$. Then
$$
L(X, \mathcal{F}) = \det\left(1-\pi_X^*\ T\Big|_{R\Gamma_c(X_{\bar k}, \mathcal{F})}\right)^{-1}\in \Lambda[[ T]].
$$
\end{theorem}

Thus far, we don't even know whether each cohomology group $H^i_c(X_{\bar k}, \mathcal{F})$ is free.

\begin{theorem}[$\mathbf{Q}_\ell$-sheaves] \label{thmB} 
Let $X$ be a scheme of finite type over a finite field $k$, and $\mathcal{F}$ a $\mathbf{Q}_\ell$-sheaf on $X$. Then
$$
L(X, \mathcal{F}) = \text{pr}od_i \det\left(1-\pi_X^*\ T\Big|_{H_c^i\left(X_{\bar k} , \mathcal{F}\right)}\right)^{(-1)^{i+1}}
\in \mathbf{Q}_\ell[[ T]]. 
$$
\end{theorem}

\begin{remark}
Since we have only developed some theory of traces and not of determinants, theorem \ref{thmA} is harder to prove than theorem \ref{thmB}. We will only prove the latter, for the former see \cite{SGA4.5}. Observe also that there is no version of this theorem more general for $\mathbf{Z}_\ell$ coefficients since there is no $\ell$-torsion.
\end{remark}

We reduce the proof of theorem \ref{thmB} to a trace formula. Since $\mathbf{Q}_\ell$ has characteristic 0, it suffices to prove the equality after taking logarithmic derivatives. More precisely, we apply $T\frac{d}{dT} \log $ to both sides. We have on the one hand
\begin{eqnarray*}
T\frac{d}{dT}\log L(X, \mathcal{F}) & = & 
T\frac{d}{dT} \log \text{pr}od_{x\in |X|} \det\left(1-\pi_x^*\ T^{\deg x}\Big|_{\mathcal{F}_{\bar x}}\right)^{-1}\\
& = & \sum_{x\in |X|} T\frac{d}{dT} \log \left( \det\left(1-\pi_x^*\ T^{\deg x}\Big|_{\mathcal{F}_{\bar x}}\right)^{-1}\right) \\
&= & \sum_{x \in |X|} \deg x\sum_{n \geq 1} \text{Tr}\left({\left(\pi_x^n\right)}^*\big|_{\mathcal{F}_{\bar x}}\right) T^{n\deg x}
\end{eqnarray*}
where the last equality results from the formula
$$
T\frac{d}{dT}\log\left(\det\left(1-fT|_M\right)^{-1}\right) = \sum_{n\geq 1} \text{Tr}(f^n|_M)T^n
$$
which holds for any commutative ring $\Lambda$ and any endomorphism $f$ of a finite projective $\Lambda$-module $M$. On the other hand, we have
\begin{align*}
&T\frac{d}{dT} \log\left(\text{pr}od_i \det\left(1-\pi_X^*\ T\Big|_{H_c^i\left(X_{\bar k} , \mathcal{F}\right)}\right)^{(-1)^{i+1}}\right) \\
&\qquad\qquad\qquad \qquad = \qquad
\sum_i (-1)^i \sum_{n\geq 1} \text{Tr}\left({\left(\pi_X^n\right)}^*\big|_{H_c^i(X_{\bar k}, \mathcal{F})}\right) T^n
\end{align*}
by the same formula again. Now, comparing powers of $T$ and using the Mobius inversion formula, we see that theorem \ref{thmB} is a consequence of the following equality
$$
\sum_{d | n} d \sum_{x\in |X| \atop \deg x = d} \text{Tr} \left( {\left( \pi_X^{n/d} \right)}^* \Big|_{\mathcal{F}_{\bar x}} \right)
=
\sum_i (-1)^i \text{Tr}\left(\left(\pi^n_X\right)^*\big|_{H^i_c(X_{\bar k}, \mathcal{F})} \right).
$$
Writing $k_n$ for the degree $n$ extension of $k$, $X_n = X \times_{\text{Spec} k} \text{Spec} k_n$ and $_n \mathcal{F} = \mathcal{F}|_{X_n}$, this boils down to
$$
\sum_{x \in X_n(k_n)} \text{Tr}\left( \pi_X^*\big|_{_n\mathcal{F}_{\bar x}} \right)
=
\sum_i (-1)^i \text{Tr}\left(\left(\pi^n_X\right)^*\big|_{H^i_c({(X_n)}_{\bar k}, _n\mathcal{F})} \right)
$$
which is a consequence of the following result.

%11.19.09

\begin{theorem} \label{thmC}
Let $X$ be a separated scheme of finite type over a finite field $k$ and $\mathcal{F}$ be a $\mathbf{Q}_\ell$-sheaf on $X$. Then $\dim_{\mathbf{Q}_\ell}H_c^i(X_{\bar k}, \mathcal{F})$ is finite for all $i$, and is nonzero for $0\leq i \leq 2 \dim X$ only. Furthermore, we have
$$
\sum_{x\in X(k)} \text{Tr}\left(\pi_x\left|_{\mathcal{F}_{\bar x}}\right.\right) = \sum_i (-1)^i\text{Tr}\left(\pi_X^*\big|_{H_c^i(X_{\bar k}, \mathcal{F})}\right).
$$
\end{theorem}

\begin{theorem} \label{thmD} 
Let $X/k$ be as above, let $\Lambda$ be a finite ring with $\#\Lambda \in k^*$ and $K\in D_{ctf}^b(X, \Lambda)$. Then $R\Gamma_c(X_{\bar k}, K)\in D_{perf}(\Lambda)$ and
$$
\sum_{x\in X(k)}\text{Tr}\left(\pi_x\left|_{K_{\bar x}}\right.\right) = \text{Tr}\left(\pi_X^*\left|_{R\Gamma_c(X_{\bar k}, K )}\right.\right).
$$
\end{theorem}

Note that we have already proved this (REFERENCE) when $\dim X \leq 1$. The general case follows easily from that case together with the proper base change theorem. We now explain how to deduce theorem \ref{thmC} from theorem \ref{thmD}. We first use some \'etale cohomology arguments to reduce the proof to an algebraic statement which we subsequently prove.

\paragraph{\it Proof of theorem \ref{thmC}.}
Let $\mathcal{F}$ be as in theorem \ref{thmC}. We can write $\mathcal{F}$ as $\mathcal{F}'\otimes \mathbf{Q}_\ell$ where $\mathcal{F}' = \left\{\mathcal{F}'_n\right\}$ is a $\mathbf{Z}_\ell$-sheaf without torsion, {\it i.e.} $\ell : \mathcal{F}'\to \mathcal{F}'$ has trivial kernel in the category of $\mathbf{Z}_\ell$-sheaves. Then each $\mathcal{F}_n'$ is a flat constructible $\mathbf{Z}/\ell^n\mathbf{Z}$-module on $X_{et}$, so $\mathcal{F}'_n \in D_{ctf}^b(X, \mathbf{Z}/\ell^n\mathbf{Z})$ and $\mathcal{F}_{n+1}'\otimes^{\mathbf{L}}_{\mathbf{Z}/\ell^{n+1}\mathbf{Z}}\mathbf{Z}/\ell^n\mathbf{Z}=\mathcal{F}_n'$. Note that the last equality holds also for standard (non-derived) tensor product, since $\mathcal{F}'_n$ is flat (it is the same equality). Therefore,
\begin{enumerate}
\item
the complex $K_n= R\Gamma_c\left(X_{\bar k}, \mathcal{F}_n'\right)$ is perfect, and it is endowed with an endomorphism $\pi_n: K_n\to K_n$ in $D(\mathbf{Z}/\ell^n\mathbf{Z})$ ; 
\item 
there are identifications 
$$
K_{n+1}\otimes^{\mathbf{L}}_{\mathbf{Z}/\ell^{n+1}\mathbf{Z}}\mathbf{Z}/\ell^n\mathbf{Z} = K_n
$$ 
in $D_{perf}(\mathbf{Z}/\ell^n\mathbf{Z})$, compatible with the endomorphisms  $\pi_{n+1}$ and $\pi_n$ (see \cite[Rapport 4.12]{SGA4.5}) ;
\item 
the equality $\text{Tr}\left(\pi_X^*\left|_{K_n}\right.\right) = \sum_{x\in X(k)}\text{Tr}\left(\pi_x\left|_{(\mathcal{F}'_n)_{\bar x}}\right.\right)$ holds ; and
\item 
for each $x\in X(k)$, the elements $\text{Tr}\left(\pi_x\big|_{\mathcal{F}'_{n, \bar x}}\right)\in \mathbf{Z}/\ell^n\mathbf{Z}$ form an element of $\mathbf{Z}_\ell$ which is equal to $\text{Tr}\left(\pi_x\left|_{\mathcal{F}_{\bar x}}\right.\right)\in \mathbf{Q}_\ell$.
\end{enumerate}
It thus suffices to prove the following algebra lemma.

\begin{lemma} 
Suppose we have $K_n\in D_{perf}(\mathbf{Z}/\ell^n\mathbf{Z})$, $\pi_n: K_n\to K_n$ and isomorphisms $\varphi_n: K_{n+1}\otimes^{\mathbf{L}}_{\mathbf{Z}/\ell^{n+1}\mathbf{Z}}\mathbf{Z}/\ell^n\mathbf{Z}\cong K_n$ compatible with $\pi_{n+1}$ and $\pi_n$. Then
\begin{enumerate}
\item 
the elements $t_n=\text{Tr}(\pi_n\left|_{K_n}\right.)\in \mathbf{Z}/\ell^n\mathbf{Z}$ form an element $t_\infty=\{t_n\}$ of $\mathbf{Z}_\ell$ ; 
\item 
the $\mathbf{Z}_\ell$-module $H_\infty^i=\text{lim}_n H^i(k_n)$ is  finite and is nonzero for finitely many $i$ only ; and 
\item 
the operators $H^i(\pi_n): H^i(K_n)\to H^i(K_n)$ are compatible and define $\pi_\infty^i: H_\infty^i\to H_\infty^i$ satisfying
$$
\sum(-1)^i \text{Tr}\left(\pi_\infty^i\big|_{H_\infty^i\otimes_{\mathbf{Z}_\ell}\mathbf{Q}_\ell}\right) = t_\infty.
$$
\end{enumerate}
\end{lemma}

\begin{proof}
Since $\mathbf{Z}/\ell^n\mathbf{Z}$ is a local ring and $K_n$ is perfect, each $K_n$ can be represented by a finite complex $K_n^\bullet$ of finite free $\mathbf{Z}/\ell^n \mathbf{Z}$-modules such that the map $K_n^p \to K_n^{p+1}$ has image contained in $\ell K_n^{p+1}$. It is a fact that such a complex is unique up to isomorphism. Moreover $\pi_n$ can be represented by a morphism of complexes $\pi_n^\bullet: K_n^\bullet\to K_n^\bullet$ (which is unique up to homotopy). By the same token the isomorphism $\varphi_n:K_{n+1}\otimes_{\mathbf{Z}/\ell^{n+1}\mathbf{Z}}^{\mathbf{L}} \mathbf{Z}/\ell^n\mathbf{Z}\to K_n$ is represented by a map of complexes
$$
\varphi_n^\bullet: K_{n+1}^\bullet\otimes_{\mathbf{Z}/\ell^{n+1}\mathbf{Z}}\mathbf{Z}/\ell^n\mathbf{Z}\to K_n^\bullet.
$$
In fact, $\varphi_n^\bullet$ is an isomorphism of complexes, thus we see that 
\begin{itemize}
\item
there exist $a, b\in \mathbf{Z}$ independent of $n$ such that $K_n^i = 0$ for all $i\notin[a, b]$ ; and 
\item 
the rank of $K_n^i$ is independent of $n$.
\end{itemize}	
Therefore, the module $K_\infty^i = \text{lim}_n \{K_n^i, \varphi_n^i\}$ is a finite free $\mathbf{Z}_\ell$-module and $K_\infty^\bullet$ is a finite complex of finite free $\mathbf{Z}_\ell$-modules. By induction on the number of nonzero terms, one can prove that $H^i\left(K_\infty^\bullet\right) = \text{lim}_n H^i\left(K_n^\bullet\right)$ (this is not true for unbounded complexes). We conclude that $H_\infty^i = H^i\left(K_\infty^\bullet\right)$ is a finite $\mathbf{Z}_\ell$-module. This proves {\it ii}. To prove the remainder of the lemma, we need to overcome the possible noncommutativity of the diagrams
$$
\xymatrix{
{K_{n+1}^\bullet} \ar_{\pi_{n+1}^\bullet}[d] \ar^{\varphi_n^\bullet}[r] & {K_n^\bullet} \ar^{\pi_n^\bullet}[d] \\
{K_{n+1}^\bullet} \ar_{\varphi_n^\bullet}[r] & {K_n^\bullet.}
}
$$
However, this diagram does commute in the derived category, hence it commutes up to homotopy. We inductively replace $\pi_n^\bullet$ for $n\geq 2$ by homotopic maps of complexes making these diagrams commute. Namely, if $h^i: K_{n+1}^i \to K_n^{i-1}$ is a homotopy, {\it i.e.}
$$
\pi_n^\bullet\circ\varphi_n^\bullet-\varphi_n^\bullet\circ\pi_{n+1}^\bullet = dh+hd,
$$
then we choose $\tilde h^i: K_{n+1}^i\to K_{n+1}^{i-1}$ lifting $h^i$. This is possible because $K_{n+1}^i$ free and $K_{n+1}^{i-1}\to K_n^{i-1}$ is surjective. Then replace $\pi_n^\bullet$ by $\tilde\pi_n^\bullet$ defined by 
$$
\tilde\pi_{n+1}^\bullet = \pi_{n+1}^\bullet +  d\tilde h+\tilde hd.
$$
With this choice of $\{\pi_n^\bullet\}$, the above diagrams commute, and the maps fit together to define an endomorphism $\pi_\infty^\bullet = \text{lim}_n\pi_n^\bullet$ of $K_\infty^\bullet$. Then part {\it i} is clear: the elements $t_n = \sum(-1)^i \text{Tr}\left(\pi_n^i\left|_{K_n^i}\right.\right)$ fit into an element $t_\infty$ of $\mathbf{Z}_\ell$. Moreover
\begin{eqnarray*}
t_\infty & = & \sum(-1)^i \text{Tr}_{\mathbf{Z}_\ell}\left(\pi_\infty^i\big|_{K_\infty^i}\right) \\
& = & \sum(-1)^i \text{Tr}_{\mathbf{Q}_\ell}\left(\pi_\infty^i\big|_{K_\infty^i\otimes_{\mathbf{Z}_\ell}\mathbf{Q}_\ell}\right)\\
& = &\sum(-1)^i\text{Tr}\left(\pi_\infty\big|_{H^i(K_\infty^\bullet\otimes\mathbf{Q}_\ell)}\right)
\end{eqnarray*}
where the last equality follows from the fact that $\mathbf{Q}_\ell$ is a field, so the complex $K_\infty^\bullet\otimes\mathbf{Q}_\ell$ is quasi-isomorphic to its cohomology $H^i(K_\infty^\bullet\otimes\mathbf{Q}_\ell)$. The latter is also equal to $H^i(K_\infty^\bullet)\otimes_{\mathbf{Z}}\mathbf{Q}_\ell = H_\infty^i\otimes \mathbf{Q}_\ell$, which finishes the proof of the lemma, and also that of theorem \ref{thmC}.
\end{proof}

%\paragraph{15 minutes..} What did we skip the proof of
%\begin{itemize}
%\item discussion of $\otimes^{\mathbb{L}}$
%\item $R\Gamma_c$ commutes with $\otimes^{\mathbb L}$ 
%\item proper base change theorem
%\item Inadequate discussion of $R\Gamma_c$
%\item Given $X\to^f S$ finite type, separated $S$ quasi-projective, discussion of $Rf_!$ on etale sheaves. 
%\end{itemize}

%11.24.09

\subsection{Examples of $L$-functions} 

We use theorem \ref{thmB} for curves to give examples of $L$-functions

\subsubsection*{Constant sheaves} 

Let $k$ be a finite field, $X$ a smooth, geometrically irreducible curve over $k$ and $\mathcal{F} = \underline{\mathbf{Q}_\ell}$ the constant sheaf. If $\bar x$ is a geometric point of $X$, the Galois module $\mathcal{F}_{\bar x} = \mathbf{Q}_\ell$ is trivial, so  
$$
\det\left(1-\pi_x^*\ T^{\deg x}\Big|_{\mathcal{F}_{\bar x}}\right)^{-1} = \frac{1}{1-T^{\deg x}}.
$$
Applying theorem \ref{thmB}, we get
\begin{eqnarray*}
L(X, \mathcal{F}) & = & \text{pr}od_{i=0}^2 \det\left(1-\pi_X^*\ T\big|_{H_c^i(X_{\bar k}, \mathbf{Q}_\ell)}\right)^{(-1)^{i+1}} \\
& = & 
\frac{\det\left(1-\pi_X^*\ T\big|_{H_c^1(X_{\bar k}, \mathbf{Q}_\ell)}\right)}{\det\left(1-\pi_X^*\ T\big|_{H_c^0(X_{\bar k}, \mathbf{Q}_\ell)}\right)\cdot\det\left(1-\pi_X^*\ T\big|_{H_c^2(X_{\bar k}, \mathbf{Q}_\ell)}\right)}.
\end{eqnarray*}
To compute the latter, we distinguish two cases.
\begin{description} 
\item[\it Projective case.]
Assume that $X$ is projective, so $H_c^i(X_{\bar k}, \mathbf{Q}_\ell) = H^i(X_{\bar k}, \mathbf{Q}_\ell)$, and we have
$$ 
H^i(X_{\bar k}, \mathbf{Q}_\ell) =
\left\{\begin{array}{ll}
\mathbf{Q}_\ell & \text{if $i = 0$, and $\pi_X^*$ acts as 1 ;} \\
\mathbf{Q}_\ell^{2g} & \text{if $i = 1$ ;} \\
\mathbf{Q}_\ell & \text{if $i = 2$, and $\pi_X^*$ acts as multiplication by $q=\deg \pi_X$.} 
\end{array}\right.
$$
We do not know much about the action of $\pi_X^*$ on the degree 1 cohomology. Let us call $\alpha_1, \ldots, \alpha_{2g}$ its eigenvalues in $\bar{\mathbf{Q}}_\ell$. Putting everything together, theorem \ref{thmB} yields the equality
$$
\text{pr}od_{x\in |X|} \frac{1}{1-T^{\deg x}} = \frac{\det\left(1- \pi_X^*\ T\big|_{H^1(X_{\bar k}, \mathbf{Q}_\ell)}\right)}{(1-T)(1-qT)}
$$
from which we deduce the following result.

\begin{lemma}
Let $X$ be a smooth, projective, geometrically irreducible curve over a finite field $k$. Then
\begin{enumerate}
\item the $L$-function $L(X, \mathbf{Q}_\ell)$ is a rational funtion ;
\item the eigenvalues $\alpha_1, \ldots, \alpha_{2g}$ of $\pi_X^*$ on $H^1(X_{\bar k}, \mathbf{Q}_\ell)$ are algebraic integers independent of $\ell$ ;
\item the number of rational points of $X$ on $k_n$, where $[k_n: k] = n$, is  
$$
\# X(k_n) = 1-\sum_{i=1}^{2g}\alpha_i^n +q^n ; 
$$ 
\item 
for each $i$, $|\alpha_i| < q$. 
\end{enumerate}
\end{lemma}	

Part {\it iii} is theorem \ref{thmC} applied to $\mathcal{F} = \underline{\mathbf{Q}_\ell}$ on $X\otimes k_n$. For part {\it iv}, use the following result.
\begin{exercise}
Let $\alpha_1, \dots, \alpha_n \in \mathbf{C}$. Then for any conic sector containing the positive real axis of the form $C_\varepsilon = \{ z \in \mathbf{C} \ | \ |\arg z| < \varepsilon \}$ with $\varepsilon >0$, there exists an integer $k \geq 1$ such that $\alpha_1^k, \dots, \alpha_n^k \in C_\varepsilon$.
\end{exercise}

Then prove that $|\alpha_i| \leq q$ for all $i$. Then, use elementary considerations on complex numbers to prove (as in the proof of the prime number theorem) that $|\alpha_i| < q$. In fact, the Riemann hypothesis says that for all $|\alpha_i| = \sqrt q$ for all $i$. We will come back to this later.

\item[\it Affine case.] 
Assume now that $X$ is affine, say $X= \bar X-\left\{x_1, \ldots, x_n\right\}$ where $j: X \hookrightarrow \bar X$ is a projective nonsingular completion. Then $H_c^0(X_{\bar k}, \mathbf{Q}_\ell) = 0$ and $H_c^2(X_{\bar k}, \mathbf{Q}_\ell) = H^2(\bar X_{\bar k}, \mathbf{Q}_\ell)$ so theorem \ref{thmB} reads
$$
L(X, \mathbf{Q}_\ell)  =  \text{pr}od_{x\in |X|}\frac{1}{1-T^{\deg x}} = \frac{\det\left(1-\pi_X^*\ T\big|_{H_c^1(X_{\bar k}, \mathbf{Q}_\ell)}\right)}{1-qT}.
$$
On the other hand, the previous case gives
\begin{eqnarray*}
L(X, \mathbf{Q}_\ell) & = & L(\bar X, \mathbf{Q}_\ell)\text{pr}od_{i=1}^n\left(1-T^{\deg x_i}\right) \\
& = & \frac{\text{pr}od_{i=1}^n(1-T^{\deg x_i})\text{pr}od_{j=1}^{2g}(1-\alpha_jT)}{(1-T)(1-qT)}.
\end{eqnarray*}
Therefore, we see that $\dim H_c^1(X_{\bar k}, \mathbf{Q}_\ell) = 2g+\sum_{i=1}^n \deg(x_i)-1$, and the eigenvalues $\alpha_1, \ldots, \alpha_{2g}$ of $\pi_{\bar X}^*$ acting on the degree 1 cohomology are roots of unity. More precisely, each $x_i$ gives a complete set of $\deg(x_i)$th roots of unity, and one occurrence of 1 is omitted.  To see this directly using coherent sheaves, consider the short exact sequence on $\bar X$
$$
0\to j_!\mathbf{Q}_\ell\to \mathbf{Q}_\ell\to\bigoplus_{i=1}^n \mathbf{Q}_{\ell, x_i}\to 0.
$$
The long exact cohomology sequence reads
$$
0\to \mathbf{Q}_\ell \to \bigoplus_{i=1}^n \mathbf{Q}_\ell^{\oplus \deg x_i} \to H_c^1(X_{\bar k}, \mathbf{Q}_\ell) \to H_c^1(\bar X_{\bar k}, \mathbf{Q}_\ell)\to 0
$$
where the action of Frobenius on $\bigoplus_{i=1}^n \mathbf{Q}_\ell^{\oplus \deg x_i}$ is by cyclic permutation of each term; and $H_c^2(X_{\bar k}, \mathbf{Q}_\ell) = H_c^2(\bar X_{\bar k}, \mathbf{Q}_\ell)$.
\end{description}

\subsubsection*{The Legendre family}

Let $k$ be a finite field of odd characteristic, $X=\text{Spec} k\left[\lambda, \frac{1}{\lambda(\lambda-1)}\right]$, and consider the family of elliptic curves  $f: E\to X$ on $\mathbb P^2_X$ whose affine equation is $y^2 = x(x-1)(x-\lambda)$. We set $\mathcal{F} = Rf_*^1\mathbf{Q}_\ell = \left\{R^1f_*\mathbf{Z}/\ell^n\mathbf{Z}\right\}_{n\geq 1}\otimes \mathbf{Q}_\ell$. In this situation, the following is true
\begin{itemize}	
\item for each $n \geq 1$, the sheaf $R^1f_*(\mathbf{Z}/\ell^n\mathbf{Z})$ is finite locally constant -- in fact, it is free of rank 2 over $\mathbf{Z}/\ell^n\mathbf{Z}$ ;
\item the system $\{R^1f_*\mathbf{Z}/\ell^n\mathbf{Z}\}_{n\geq 1}$ is a lisse $\ell$-adic sheaf ; and
\item for all $x\in |X|$, $\det\left(1-\pi_x\ T^{\deg x}\big|_{\mathcal{F}_{\bar x}}\right) = (1-\alpha_x T^{\deg x})(1-\beta_x T^{\deg x })$ where $\alpha_x, \beta_x$ are the eigenvalues of the geometric frobenius of $E_x$ acting on $H^1(E_{\bar x}, \mathbf{Q}_\ell)$. 
\end{itemize}
Note that $E_x$ is only defined over $\kappa(x)$ and not over $k$. The proof of these facts uses the proper base change theorem and the local acyclicity of smooth morphisms. For details, see \cite{SGA4.5}. It follows that
$$
L(E/X) := L(X, \mathcal{F}) = \text{pr}od_{x\in |X|} \frac{1}{(1-\alpha_xT^{\deg x})(1-\beta_xT^{\deg x })} .
$$
Applying theorem \ref{thmB} we get
$$
L(E/X) = \text{pr}od_{i=0}^2\det\left(1-\pi_X^*\ T\left|_{H_c^i(X_{\bar k}, \mathcal{F})}\right.\right)^{(-1)^{i+1}},
$$
and we see in particular that this is a rational function. Furthermore, it is relatively easy to show that $H_c^0(X_{\bar k}, \mathcal{F}) = H_c^2(X_{\bar k}, \mathcal{F}) = 0$, so we merely have
$$
L(E/X) = \det\left(1-\pi_X^*T\big|_{H_c^1(X, \mathcal{F})}\right).
$$
To compute this determinant explicitly, consider the Leray spectral sequence for the proper morphism $f:E \to X$ over $\mathbf{Q}_\ell$, namely
$$
H_c^i(X_{\bar k}, R^jf_*\mathbf{Q}_\ell) \Rightarrow H_c^{i+j}(E_{\bar k},\mathbf{Q}_\ell)
$$
which degenerates. We have $f_*\mathbf{Q}_\ell=\mathbf{Q}_\ell$ and $R^1f_*\mathbf{Q}_\ell = \mathcal{F}$. The sheaf $R^2f_*\mathbf{Q}_\ell = \mathbf{Q}_\ell(-1)$ is the \emph{Tate twist} of $\mathbf{Q}_\ell$, {\it i.e.} it is the sheaf $\mathbf{Q}_\ell$ where the Galois action is given by multiplication by $\#\kappa(x)$ on the stalk at $\bar x$.  It follows that,  for all $n\geq 1$,
\begin{eqnarray*}
\# E(k_n) & = & \sum(-1)^i\text{Tr}\left({\pi_E^n}^*\big|_{H_c^i(E_{\bar k}, \mathbf{Q}_\ell)}\right)\\
& = & \sum_{i, j}(-1)^{i+j}\text{Tr}\left({\pi^n_X}^*\big|_{H_c^i(X_{\bar k}, R^jf_*\mathbf{Q}_\ell)}\right)\\
& = & (q^n-2) + \text{Tr}\left( {\pi_X^n}^*\big|_{H_c^1(X_{\bar k}, \mathcal{F})}\right)+ q^n (q^n-2)\\
& = & q^{2n} - q^n-2 +\text{Tr}\left( {\pi_X^n}^*\big|_{H_c^1(X_{\bar k}, \mathcal{F})}\right)
\end{eqnarray*}
where the first equality follows from theorem \ref{thmC}, the second one from the Leray spectral sequence and the third one by writing down the higher direct images of $\mathbf{Q}_\ell$ under $f$. Alternatively, we could write 
$$
\#E(k_n) = \sum_{x \in X(k_n)} \#E_x(k_n)
$$
and use the trace formula for each curve. We can also find the number of $k_n$-rational points simply by counting. The zero section contributes $q^n -2$ points (we omit the points where $\lambda = 0, 1$) hence
$$
\# E(k_n) =  q^n-2 + \#\left\{y^2 = x(x-1)(x-\lambda), \lambda\neq 0, 1\right\}.
$$
Now we have
$$
\begin{array}{l}
 \#\left\{y^2 = x(x-1)(x-\lambda), \; \lambda\neq 0, 1\right\}\\
 \\
 \quad =  \#\left\{y^2 = x(x-1)(x-\lambda)\text{ in }\mathbb A^3\right\} -\#\left\{y^2 = x^2(x-1)\right\}-\#\left\{y^2 = x(x-1)^2\right\}\\
 \\
 \quad =  \#\left\{\lambda=\frac{-y^2}{x(x-1)}+x, \; x\neq 0, 1\right\} + \#\left\{y^2 = x(x-1)(x-\lambda), x=0, 1\right\}-2(q^n-\varepsilon_n) \\
 \\
 \quad =  q^n(q^n-2)+2q^n-2(q^n-\varepsilon_n)\\
 \\
 \quad =  q^{2n}-2q^n+2\varepsilon_n
\end{array}
$$
where $\varepsilon_n = 1$ if $-1$ is a square in $k_n$, 0 otherwise\footnote{I don't understand this: 
 $$
\#\left\{y^2 = x^2(x-1)\right\} = \sum_{x \in k_n} 1+\left(\frac{x^2(x-1)}{k_n}\right) = q^n+\sum_{x \in k_n} \left(\frac{x-1}{k_n}\right) = q^n
 $$
and similarly for $\#\left\{y^2 = x(x-1)^2\right\}$, so $\varepsilon_n = 0$?}, {\it i.e.} 
$$
\varepsilon_n = \frac{1}{2}\left(1+\left(\frac{-1}{k_n}\right)\right) = \frac{1}{2}\left(1+(-1)^{\frac{q^n-1}{2}}\right).
$$
Thus $ \# E(k_n) =  q^{2n}-q^n-2+ 2\varepsilon_n$. Comparing with the previous formula, we find
$$
\text{Tr}\left({\pi_X^n}^*\big|_{H_c^1(X_{\bar k}, \mathcal{F})}\right) = 2 \varepsilon_n =  1+(-1)^{\frac{q^n-1}{2}},
$$
which implies, by elementary algebra of complex numbers, that if $-1$ is a square in $k_n^*$, then $\dim H_c^1(X_{\bar k}, \mathcal{F}) = 2$ and the eigenvalues are $1$ and $1$. Therefore, in that case we have
 $$L(E/X) = (1-T)^2.$$

\subsubsection*{Exponential sums}

A standard problem in number theory is to evaluate sums of the form
$$
S_{a,b}(p) = \sum_{x\in \mathbb F_p-\left\{0, 1\right\}} e^{\frac{2\pi ix^a(x-1)^b}{p}}.
$$
In our context, this can be interpreted as a cohomological sum as follows. Consider the base scheme $S = \text{Spec} \mathbf{F}_p\left[x, \frac{1}{x(x-1)}\right]$ and the affine curve $f: X \to \mathbb P^1-\{0, 1, \infty\}$ over $S$ given by the equation $y^{p-1} = x^a(x-1)^b$. This is a finite \'etale Galois cover with group $\mathbf{F}_p^*$ and there is a splitting
$$
f_*(\bar{\mathbf{Q}}_\ell^*) =
\bigoplus_{\chi : \mathbb F_p^*\to \bar{\mathbf{Q}}_\ell^*} \mathcal{F}_\chi
$$
where $\chi$ varies over the characters of $\mathbf{F}_p^*$ and $\mathcal{F}_\chi$ is a rank 1 lisse $\mathbf{Q}_\ell$-sheaf on which $\mathbf{F}_p^*$ acts via $\chi$ on stalks. We get a corresponding decomposition
$$
H_c^1(X_{\bar k}, \mathbf{Q}_\ell) = \bigoplus_\chi H^1(\mathbb P_{\bar k}^1-\{0, 1, \infty\}, \mathcal{F}_\chi)
$$
and the cohomological interpretation of the exponential sum is given by the trace formula applied to $\mathcal{F}_\chi$ over $\mathbb P^1 - \{0, 1, \infty\}$ for some suitable $\chi$. It reads
$$
S_{a,b}(p) = -\text{Tr}\left(\pi_X^*\big|_{H^1(\mathbb P_{\bar k}^1-\{0, 1, \infty\}, \mathcal{F}_\chi)}\right).
$$
The general yoga of Weil suggests that there should be some cancellation in the sum. Applying (roughly) the Riemann-Hurwitz formula, we see that
$$
2g_X-2 \approx -2 (p-1) + 3(p-2) \approx p
$$
so $g_X\approx p/2$, which also suggests that the $\chi$-pieces are small.



%12.01.09

\section*{Trace formula in terms of fundamental groups}
\subsection{Fundamental groups} $X$ connected scheme $\overline x\to X$ geometric point consider the functor 
$$
\begin{array}{ccccccc}
F_{\overline x}: &
\text{ finite etale }\atop \text{ schemes over } X &
\to & \text{ finite sets} \\
&
Y/X &
\to &
F_{\overline x}(Y) =
\left\{\text{ geom points }\overline y\atop \text{ of } Y
\text{ lying over }\overline x\right\} \\
&
&
&
= Y_{\overline x} = \left\{ * \right\}
\end{array}
$$
where 
$$
* =
\xymatrix{
 &
Y \ar[d] &
\\
\overline x \ar[ur] &
{\overline y} \ar[dr] &
\\
 &
 &
 X
}
$$
Set 
$$
\pi_1(X, \overline x)
=
Aut(F_{\overline x})
=
\text{ set of automorphisms of the functor }F_{\overline x}
$$
	Note that for every finite etale $Y\to^\pi X$ there is an action
		$$\pi_1(X, \overline x) \times F_{\overline x}(Y) \to F_{\overline x}(Y)$$
		
\begin{definition}A subgroup of the form $\text{Stab}(\overline y\in F_{\overline x}(Y))\subset \pi_1(X, \overline x)$ is called open.
\end{definition}

\begin{theorem}[Grothendieck, SGA1] $X$ connected
	\begin{enumerate}
	\item there is a topology on $\pi_1(X, \overline x)$ such that the open subgroups form a fundamental system of open nbhds of $e\in \pi_1(X, \overline x)$.
	\item $\pi_1(X, \overline x)$ is a profinite group.
	\item the functor 
		$$\begin{array}{cccccc}
		\text{ schemes finite }\atop \text{ etale over }X & \to & \text{ finite discrete continuous }\atop \pi_1(X, \overline x)\text{-sets}\\
		Y /  X& \mapsto & F_{\overline x}(Y) \text{ with its natural action}
		\end{array}$$
		is an equivalence of categories. 
	\end{enumerate}
\end{theorem}

\begin{proposition} Let $X$ be an integral normal Netherian scheme. Let $\overline y\to X$ be an algebraic geometric point lying over the generic point $\eta\in X$. Then
	$$\pi_x(X, \overline \eta) = Gal(M/\kappa(\eta))$$
	($\kappa(\eta)$, function field of $X$) where
	$$\kappa(\overline \eta)\supset M\supset \kappa(\eta) = k(X)$$
	is the max sub-extension such that for every finite sub extension $M\supset L\supset \kappa(\eta)$ the normalization of $X$ in $L$ is finite etale over $X$. 
\end{proposition}

%%%%%%%%%%%%%%
\paragraph{Change of base point} For any $\overline x_1, \; \overline x_2$  geom. points of $X$ there exists an isom. of fibre functions
	$$\mathcal{F}_{\overline x_1} \cong \mathcal{F}_{\overline x_2}$$
	(This is a path from $\overline x_1$ to $\overline x_2$.) Conjugation by this path gives isom
		$$\pi_1(X, \overline x_1) \cong \pi_1(X, \overline x_2)$$
		well defined up to inner actions. 

\paragraph{Functoriality} For any morphism $X_1\to X_2$ of connected schemes any $\overline x\in X_1$ there is a canonical map
	$$\pi_1(X_1, \overline x) \to \pi_1(X_2, \overline x)$$
(Why? because the fibre functor ...)

\paragraph{Base field} Let $X$ be a variety over a field $k$. Then we get 	
	$$\pi_1(X, \overline x) \to \pi_1(Spec(k), \overline x) =^{\text{prop}} Gal(k^{\text{sep}}/k)$$
	This map is surjective iff $X$ is geom. connected over $k$.
	%The kernel of this map is $\pi_1(X_{\overline{k}}, \overline x)$.
	So in the geometrically connected case we get s.e.s. of profinite groups 
	$$1 \to \pi_1(X_{\overline k}, \overline x)\to \pi_1(X, \overline x)\to Gal(k^{\text{sep}}/k)\to 1$$
	($\pi_1(X_{\overline k}, \overline x)$: geometric fundamental group of $X$, $\pi_1(X, \overline x)$: arithmetic fundamental group of $X$)
	
\paragraph{Comparison} If $X$ is a variety over $\mathbf{C}$ then 
$$\pi_1(X, \overline x)=\text{ profinite completion of }\pi_1(X(\mathbf{C})(\text{ usual topology}), x)$$
(have $x\in X(\mathbf{C})$)

\paragraph{Frobenii} $X$ variety over $k$, $\sharp k<\infty$. For any $x\in X$ closed point, let 
	$$F_x\in \pi_1(x, \overline x)=\text{Gal}(\kappa(x)^{\text{sep}}/\kappa(x))$$ be the geometric frobenius. Let $\overline\eta$ be an alg. geom. gen. pt. Then
		$$\pi_1(X, \overline\eta) \leftarrow^{\cong}\pi_1(X, \overline x) {\text{\small fundtoriality}\atop\leftarrow}\pi_1(x, \overline x)$$
		
\noindent
Easy fact: 
$$
\begin{array}{cccccccc}
\pi_1(X, \overline \eta) & \to^{\deg} \pi_1(\text{Spec}(k), \overline \eta) * & = Gal(k^{sep}/k) \\
& & || \\
& & \widehat{\mathbf{Z}}\cdot F_{\text{Spec}(k)} \\
F_x & \mapsto & \deg(x)\cdot F_{\text{Spec}(k)}
\end{array}
$$
Recall: $\deg(x) = [\kappa(x):k]$

\paragraph{Fundamental groups and lisse sheaves}
	Let $X$ be a connected scheme, $\overline x$ geom. pt. There are equivalences of categories 
		$$\begin{array}{ccccccc}
		\text{($\Lambda$ finite ring)}& \text{fin. loc. const. sheaves of }
		\atop \Lambda\text{-modules of }X_{et} & \leftrightarrow & \text{ finite(discrete) }\Lambda\text{-modules}\atop\text{ with continuous }\pi_1(X, \overline x)\text{-action}\\
		\\
		(l\text{ a prime}) & \text{ lisse }l\text{-adic}\atop\text{ sheaves} & \leftrightarrow & \text{ finitely generated }\mathbf{Z}_l\text{-modules } M \text{ with continuous }\atop \pi_1(X, \overline x)\text{ action where we use $l$-adic topology on } M
		\end{array}$$
In particular lisse $\mathbf{Q}_l$-sheaves correspond to continuous homomorphisms 
	$$\pi_1(X, \overline x) \to GL_r(\mathbf{Q}_l)$$
	($r\geq 0$)
	
Notation: A module with action $(M, \rho)$ corresponds to the sheaf $\mathcal{F}_\rho$. 

\paragraph{Trace formulas} $X$ variety over $k$, $\sharp k<\infty$. 
\begin{enumerate}
	\item $\Lambda$ finite ring $(\sharp \Lambda, \sharp k)=1$
		$$\rho: \pi_1(X, \overline x)\to GL_r(\Lambda)$$
		continuous. For every $n\geq 1$ we have
		$$\sum_{d|n}d\left(\sum_{x\in |X|, \atop\deg(x)=d}\text{Tr}(\rho(F_x^{n/d}))\right) = \text{Tr}\left((\pi_x^n)^*\left|_{R\Gamma_c(X_{\overline k}, \mathcal{F}_\rho)}\right.\right)$$
		
	\item $l\neq char(k)$ prime, $\rho: \pi_1(X, \overline x)\to GL_r(\mathbf{Q}_l)$. For any $n\geq 1$
		$$\sum_{d|n}d\left(\sum_{x\in |X|\atop \deg(x)=d}\text{Tr}\left(\rho(F_x^{n/d})\right)\right)=\sum_{i=0}^{2\dim X}(-1)^i \text{Tr}\left(\pi_X^*\left|_{H_c^i(X_{\overline k}, \mathcal{F}_\rho)}\right.\right)$$
\end{enumerate}

\paragraph{Weil conjecture}  (Deligne-Weil I, 1974) $X$ smooth proj. over $k$, $\sharp k=q$, then the eigenvalues of $\pi_X^*$ on $H^i(X_{\overline k}, \mathbf{Q}_l)$ are algebraic integers $\alpha$ with $|\alpha|=q^{1/2}$. 

\paragraph{Deligne's conjectures}(almost completely proved by Lafforgue+$\cdots$) $X$ normal variety over $k$ finite
	$$\rho: \pi_1(X, \overline x)\to GL_r(\mathbf{Q}_l) $$
	continuous.\\
Assume: $\rho$ irreducible $\det(\rho)$ of finite order. Then 
	\begin{itemize}
	\item there exists a number field $E$ such that for all $x\in |X|$(closed points) the char. poly of $\rho(F_x)$ has coefficients in $E$. 
	\item for any $x\in |X|$ the eigenvalues $\alpha_{x, i}$, $i=1, \ldots, r$ of $\rho(F_x)$ have complex absolute value $1$. 
	(these are algebraic numbers not necessary integers)
	\item for every finite place $\lambda$( not dividing $p$), of $E$ (maybe after enlarging $E$ a bit) there exists 
	$$\rho\lambda: \pi_1(X, \overline x) \to GL_r(E_\lambda)$$
	compatible with $\rho$. (some char. polys of  $F_x$'s)
	\end{itemize}

\begin{theorem}[Deligne Weil II] (Not the original formulation.) For a sheaf $\mathcal{F}_\rho$ with $\rho$ satisfying the conclusions of the conjecture above then the eigenvalues of $\pi_X^*$ on $H_c^i(X_{\overline k}, \mathcal{F}_{\rho})$ are algebraic numbers $\alpha$ with absolute values
	$$|\alpha|=q^{w/2}, \text{ for }w\in \mathbf{Z}, \; w\leq i$$
	Moreover, if $X$ smooth and proj. then $w=i$.
\end{theorem}


%12.03.09

\section*{Cohomology of curves, revisited} 
Let $k$ be a field, $X$ be geometric connected, smooth curve over $k$, 
	$$1\to \pi_1(X_{\overline k}, \overline \eta)\to \pi_1(X, \overline\eta)\to \text{Gal}(k^{^{sep}}/k)\to 1$$
If $\Lambda$ is a finite ring ($\sharp\Lambda\in k^*$), $M$ a finite $\Lambda$-module
	$$\rho:\pi_1(X, \overline\eta)\to \text{Aut}_{\Lambda}(M)$$ continuous, 	then $\mathcal{F}_\rho$ denotes the associated sheaf on $X_{et}$.\\
	
\noindent\underline{Claim: } There is a canonical isomorphism
	$$H_c^2(X_{\overline k}, \mathcal{F}_\rho)=(M)_{\pi_1(X_{\overline k}, \overline\eta)}(-1)$$
	($\uparrow$ co-invariants) as $\text{Gal}(k^{^{sep}}/k)$-modules. Here $(-1)$ indicates the Tate twist i.e., $\sigma\in \text{Gal}(k^{^{sep}}/k)$ acts via
	$$\chi_{cycl}(\sigma)^{-1}.\sigma\text{ on RHS}$$
	where 
	$$\chi_{cycl}: \text{Gal}(k^{^{sep}}/k)\to \text{pr}od_{l\neq char(k)}\mathbf{Z}_l^*$$
	cyclotomic character. 

\paragraph{Reformulation} (Deligne, Weil II, page 338) For any finite locally constant sheaf $\mathcal{F}$ on $X$ there is a maximal quotient $\mathcal{F}\to \mathcal{F}''$ with $\mathcal{F}''/X_{\overline k}$ a constant sheaf, hence 
	$$\mathcal{F}'' = (X\to \text{Spec}(k))^{-1}F''$$
	where $F''$ is a sheaf $\text{Spec}(k)$, i.e., a $\text{Gal}(k^{^{sep}}/k)$-module. Then
	$$H_c^2(X_{\overline k}, \mathcal{F})\to H_c^2(X_{\overline k}, \mathcal{F}'')\to F''(-1)$$
	is an isomorphism. 
\begin{proof}(Claim) Let $Y\to^{\varphi}X$ be the finite etale Galois covering corresponding to $Ker(\rho)\subset \pi_1(X, \overline\eta)$. So 
	$$Aut(Y/X)=Ind(\rho)$$
	is Galois group. Then $\varphi^*\mathcal{F}_\rho =\underline M_Y$ and
		$$\varphi_*\varphi^*\mathcal{F}_\rho\to \mathcal{F}_\rho$$
		which gives
		\begin{align*}
		&H_c^2(X_{\overline k}, \varphi_*\varphi^*\mathcal{F}_\rho) \to H_c^2(X_{\overline k}, \mathcal{F}_\rho)\\
		&=H_c^2(Y_{\overline k}, \varphi^*\mathcal{F}_\rho)\\
		&=H_c^2(Y_{\overline k}, \underline M) = \oplus_{\text{irred. comp. of }\atop Y_{\overline k}}M
		\end{align*}
		$$Im(\rho)\rightarrow H_c^2(Y_{\overline k}, \underline M) = \oplus_{\text{irred. comp. of }\atop Y_{\overline k}}M\to_{Im(\rho)\text{ equivalent}} H_c^2(X_{\overline k}, \mathcal{F}_{\rho})\leftarrow^{\text{trivial }Im(\rho)\atop \text{action}}$$
		irreducible curve $C/\overline k$, $H_c^2(C, \underline M)=M$. \\
Since
	$${\text{set of irreducible }\atop\text{components of }Y_k} = \frac{Im(\rho)}{Im(\rho|_{\pi_1(X_{\overline k}, \overline \eta)})}$$
	We conclude that $H_c^2(X_{\overline k}, \mathcal{F}_\rho)$ is a quotient of $M_{\pi_1(X_{\overline k}, \overline \eta)}$. On the other hand, there is a surjection
		$$\mathcal{F}_\rho\to \mathcal{F}'' = {\text{ sheaf on } X\text{ associated to }\atop (M)_{\pi_1(X_{\overline k}, \overline \eta)}\leftarrow\pi_1(X, \overline \eta)}$$
			$$H_c^2(X_{\overline k}, \mathcal{F}_\rho)\to M_{\pi_1(X_{\overline k}, \overline\eta)}$$
			The twist in Galois action comes from the fact that $H_c^2(X_{\overline k}, \mu_n)=^{\text{can}} \mathbf{Z}/n\mathbf{Z}$. 
\end{proof}
	$$H^0(X_{\overline k}, \mathcal{F}_\rho) = M^{\pi_1(X_{\overline k}, \overline\eta)}$$
	$$H_c^2(X_{\overline k}, \mathcal{F}_\rho) = M_{\pi_1(X_{\overline k}, \overline \eta)}(-1)$$
	
\paragraph{Profinite groups, cohomology + homology} Let $G$ be a profinite group \underline{Cohomology}(torsion). Consider the category of discrete mordules with continuous $G$-action. This category has .. injection and 
	$$H^i(G, M) = R^iH^0(G, M) = R^i(M\mapsto M^G)$$
	Also there is a derived version $RH^0(G, -)$. \\

\noindent\underline{Homology: }(profinite) Consider the category of compact abelian groups with continuous $G$-action. This category has enough projectives and $H_i(G, M) = L_iH_0(G, M)=L_i(M\mapsto M_G)$ and there is also a derived version. 
\\

\noindent
\underline{Trivial duality: } The functor $M\mapsto  M^V = Hom_{cont}(M, S^1)$	 exchanges the categories above and
	$$ H^i(G, M)^V = H_i(G, M^V)$$
	
\paragraph{Notes on Homology} \begin{enumerate}
	\item If we look at $\Lambda$-module then we can identify
		$$H_i(G, M)=Tor_i^{\Lambda[[G]]}(M, \Lambda)$$
		
	\item If $G\vartriangleleft \Gamma$, and $\Gamma$-profinite then
		\begin{itemize}
		\item $H^0(G, -)$: discrete $\Gamma$-module$\to$ discrete $\Gamma/G$-modules
		\item $H_0(G, -)$: compact $\Gamma$-modules $\to$ compact $\Gamma/G$-modules
		\end{itemize}
	\end{enumerate}
	
\begin{proposition} Let $X/k$ as before but $X_{\overline k}\neq \mathbb P^1_{\overline k}$ 
\begin{enumerate}
	\item The functor
		$$(M, \rho)\mapsto H_c^{2-i}(X_{\overline k}, \mathcal{F}_\rho)$$
		are the left derived functor of $(M, \rho)\mapsto H_c^2(X_{\overline k}, \mathcal{F}_\rho)$ so
			$$H_c^{2-i}(X_{\overline k},\mathcal{F}_\rho) = H_i(\pi_1(X_{\overline k}, \overline \eta), M)(-1)$$
			
	\item The functor
		$$(M, \rho)\mapsto H^i(X_{\overline k}, \mathcal{F}_\rho)$$
		are the right derived functor of 
		$$(M, \rho)\mapsto M^{\pi_1(X_{\overline k}, \overline \eta)}$$
		so 
		$$H^i(X_{\overline k}, \mathcal{F}_\rho) = H^i(\pi_1(X_{\overline k}, \overline \eta), M)$$
		Moreover, there are derived versions. 
\end{enumerate}
\end{proposition}
\begin{proof}(Idea) Show both sides are univ. $\delta$-functors.
\end{proof}

\begin{remark} Proposition $+$ Trivial duality then you get
	$$H^{2-i}_c(X_{\overline k}, \mathcal{F}_\rho)\times H^i(X_{\overline k}, \mathcal{F}_\rho^V(1))\to \mathbf{Q}/\mathbf{Z}$$
	a perfect pairing. If $X$ proj then this is a Poincare duality. 
\end{remark}

\paragraph{Abstract Trace formula} Suppose given an extension of profinite groups, 
$$
1 \to G \to \Gamma \to^{\deg} \widehat{\mathbf{Z}} \to 1
$$
We say $\Gamma$ has an abstract trace formula iff there exist
	\begin{enumerate}
	\item an integer $q\geq 1$ such that for all $l \not | q$ have $cd_l(G)<\infty$ and
	\item for every $d\geq 1$ a set $S_d$ and for each $x\in S_d$ a conjugacy class $F_x \in \Gamma$ with $\deg(F_x)=d$ such that for all finite rings $\Lambda$ with $q\in \Lambda^*$, for all finite proj. $\Lambda$-modules $M$ with continuous $\Gamma$-action, for all $n>0$ we have
		$$\sum_{d|n}d\left(\sum_{x\in S_d}\text{Tr}(F_x^{n/d}\left|_M\right.)\right) = q^n\text{Tr}(F^n\left|_{M\otimes_{\Lambda[[G]]}^{\mathbb L}\Lambda}\right.)$$
		in $\Lambda^\natural$.  ($M\otimes_{\Lambda[[G]]}^{\mathbb L}\Lambda"=" LH_0(G, M)$ derived homology)
	($F=1$ in $\Gamma/G=\widehat{\mathbf{Z}}$)
	\end{enumerate}	
	
\begin{remark}
	\begin{enumerate}
	\item If modeling projective curves then we can use cohomology and we don't need factor $q^n$. 
	\item The only examples I know are $\Gamma=\pi_1(X, \overline \eta)$ where $X$ is smooth, geometrically irreducible and $K(\pi, 1)$ over finite field. In this case $q=(\sharp k)^{\dim X}$. Modulo the proposition, we proved this for curves in this course.
	\item Given the integer $q$ then the sets $S_d$ are uniquely determined. (You can multiple $q$ by an integer $m$ and then replace $S_d$ by $m^d$ copies of $S_d$ without changing the formula.) 
	\end{enumerate}
\end{remark}
	
\begin{example}
	Fix a $q\geq 1$
		$$\begin{array}{cccccccccccccccc}
		1&\to &G=\widehat{\mathbf{Z}}^{(q)}&\to &\Gamma&\to &
		\widehat{\mathbf{Z}} &\to &1\\
		& &= \text{pr}od_{l\not \mid q}\mathbf{Z}_l & & F & \mapsto & 1
		\end{array}$$
	$FxF^{-1}=ux, \; u\in (\widehat{\mathbf{Z}}^{(q)})^*$ Just using the trivial modules $\mathbf{Z}/m\mathbf{Z}$ we see
		$$q^n-(qu)^n\equiv \sum_{d|n} d\sharp S_d$$
		in $\mathbf{Z}/m\mathbf{Z}$ for all $(m, q)=1$ (up to $u\to u^{-1}$) this implies $qu=a\in \mathbf{Z}$ and $|a|<q$. The special case $a=1$ does occur 
		$$\pi_1^t(\mathbb G_{m, \mathbb F_p}, \overline \eta)$$
		$$\sharp S_1 = q-1$$
		$$\sharp S_2 = \frac{(q^2-1)-(q-1)}{2}$$


		
\end{example}

%12.08.09

References:\\

\paragraph{Unramified cusp forms} $k$ finite field, char $p$, $X$ geom. irred, proj. smooth curve over $k$, $K=k(X)$ function field. Let $v$ be a place of $K$ $\leftrightarrow$ closed point $x\in X$, $K_v$, a completion of $K$ at $v$ $=$ f.f. $(\widehat{O_{X, x}})$, $O_v\subset K_v$ integers. 
	$$O=\text{pr}od_v O_v\subset \mathbb A=\text{pr}od_v'K_v$$
	$\Lambda=$ any ring with $p$ invertible in $\Lambda$. 

\begin{definition} An unramified cusp form(today!) on $GL_2(\mathbf{A})$ with values in $\Lambda$ is a function
	$$f: GL_2(\mathbf{A})\to \Lambda$$
	such that
	\begin{enumerate}
	\item $f(x\gamma) = f(x)$ for all $x\in GL_2(\mathbf{A})$ and all $\gamma\in GL_2(K)$
	\item $f(ux) = f(x)$ for all $x\in GL_2(\mathbf{A})$ and all $u\in GL_2(O)$
	\item for all $x\in GL_2(\mathbf{A})$, 
	$$\int_{\mathbf{A} \mod K}f\left(x\begin{matrix}1 & z\\ 0 & 1\end{matrix} \right)dz=0$$
	(see[DJ 4.1])
	\end{enumerate}
\end{definition}

\paragraph{Hecke Operator} For $v$ a place of $K$ and $f$ an unramified cusp form we set
	$$T_v(f)(x) = \int_{g\in M_v}f(g^{-1}x)dg, \; U_v(f)(x) = f\left(\begin{matrix} \pi_v^{-1} & 0 \\ 0 & \pi_v^{-1}\end{matrix} x\right)$$

\noindent
notation used: $\pi_v\in O_v$ uniformizer
	$$M_v = \left\{h\in Mat(2\times 2, \; O_v)\left| \det h = \pi_vO_v^*\right.\right\}$$
	$dg=$ Haar measure on $GL_2(K_v)$ with $\int_{GL_2(O_v)} dg = 1$, 
	Explicitly
	$$T_v(f)(x) = f\left(\begin{matrix} \pi_v^{-1}& 0 \\ 0 & 1\end{matrix} x\right) + \sum_{i=1}^{q_v} f\left(\begin{matrix} 1 & 0 \\ -\pi_v^{-1}\lambda_i & \pi_v^{-1}\end{matrix} x\right)$$
	with $\lambda_i\in O_v$ a set of representatives of $O_v/(\pi_v)=\kappa_v$, $q_v = \sharp\kappa_v$. 
	
\paragraph{Eigenform} An $f$ unr. cusp form such that some value of $f$ is a unit and $T_vf = t_vf$ and $U_vf = u_vf$ for some (uniquely determined) $t_v, \; u_v\in \Lambda$. 

\begin{theorem}[D'] Given an eigenform $f$ with values in $\overline{\mathbf{Q}}_l$ and $u_v\in \overline{\mathbf{Z}}_l^*$ then there exists
	$$\rho: \pi_1(X)\to GL_2(E)$$
	continuous, absolutely irreducible where
	$$\mathbf{Q}_l\subset^{\text{finite}} E\subset \overline{\mathbf{Q}}_l$$
	such that
	$$t_v = T_v\left(\rho(F_v)\right), \; u_v = q_v^{-1}\det\left(\rho(F_v)\right).$$
\end{theorem}

\begin{theorem}[D'] Suppose $\mathbf{Q}_l\subset E$ finite, and 
	$$\rho: \pi_1(X)\to GL_2(E)$$
	abs. irreducible, continuous. Then there exists an eigenform $f$ with values in $\overline{\mathbf{Q}}_l$ such that 
		$$t_v=Tr\left(\rho(F_v)\right), \; u_v=q_v^{-1}\det(\rho(F_v))$$
\end{theorem}

\begin{remark} We now have a complete theorems like this two above for $GL_n$ and allowing ramification!
\end{remark}

\paragraph{Central character} If $f$ is an eigenform then 
	$$\begin{array}{ccccccccc}
	\chi_f:& O^*\backslash \mathbf{A}^*/K^*&\to &\Lambda^*\\
	&(1, \ldots, \pi_v, 1, \ldots, 1)& \mapsto & u_v^{-1}\end{array}$$
	is called the \underline{central character}. If corresponds to the determinant of $\rho$ via normalizations as above. Set
	$$C(\Lambda)=\left\{{\text{unr. cusp forms } f \text{ with coefficients in }\Lambda}\atop {\text{ such that } U_v f = \varphi_v^{-1}f\forall v}\right\}$$
	
\begin{proposition}[dJ. 4.7] If $\Lambda$ is Noetherian then $C(\Lambda)$ is a finitely generated $\Lambda$-module. Moreover, if $\Lambda$ is a field with prime subfield $\mathbb F\subset \Lambda$ then
	$$C(\Lambda)=(C(\mathbb F))\otimes_{\mathbf{F}}\Lambda$$
	compatibly with $T_v$ acting. 
\end{proposition}

\begin{lemma}
\begin{enumerate}
	\item If $\Lambda$ is a field then the eigenvalues $t_v$ for $f\in C(\Lambda)$ are algebraic over $\mathbf{F}$. 
	\item Switching $l$
	$$\rho: \pi_1(X)\to SL_2(E_\lambda)$$
	abs. irred. cont., $\mathbf{Q}_l\subset E_l$ finite. \\
	Then(D) $f\in C(\overline{\mathbf{Q}}_l)$ eigenform ass. to $\rho$\\
	$\Rightarrow$(Prop) $f\in C(E)$ eigenform with $\overline{\mathbf{Q}}_l\supset E\supset \mathbf{Q}$ finite ass. to $\rho$\\
	$\Rightarrow$(complete) $f\in C(E_{\lambda'})$ eigenform with $E_{\lambda'}$ a finite extension of $\mathbf{Q}_{l'}$\\
	$\Rightarrow$(D') $\rho': \pi_1(X)\to SL_2(E_{\lambda'}')$ abs. irred. $E_{\lambda'}\subset E_{\lambda'}'$ finite. 
\end{enumerate}
\end{lemma}

\noindent
(Speculation) If for a (topological) ring $\Lambda$ we have 
	$$\left({\rho: \pi_1(X)\to SL_2(\Lambda)\atop \text{ abs irred}}\right)\leftrightarrow^{1-1} \text{ eigen forms in } C(\Lambda)$$
	then all eigenvalues of $\rho(F_v)$ algebraic (won't work in an easy way of $\Lambda$ finite ring.)
	
\paragraph{Conjecture(dJ)} For any continuous 
	$$\rho: \pi_1(X)\to GL_n(\mathbf{F}_l[[t]])$$
	we have $\sharp \rho(\pi_1(X_{\overline k}))<\infty$. 
	
	"For any lisse sheaf of $\overline{\mathbf{F}_l((t))}$-modules the geom monodromy is finite. "
	
\begin{theorem}[dJ] Conj. holds if $n\leq 2$. 
\end{theorem}

\begin{theorem}[G] Conjecture holds if $l>2n$ modulo some unproven things. 
\end{theorem}

\begin{theorem}[dJ, 3.5] Suppose
	$$\rho_0: \pi_1(X)\to GL_n(\mathbf{F}_l)$$
	is a continuous, $l\neq p$. Assume
	\begin{enumerate}
	\item Conj. holds for $X$
	\item $\rho_0\left|_{\pi_1(X_{\overline k})}\right.$ abs. irred.
	\item $l\not \mid n$
	\end{enumerate}
	Then the universal determination ring $R_{\text{univ}}$ of $\rho_0$ is finite flat over $\mathbf{Z}_l$. 
\end{theorem}

\noindent
\underline{Explanation: } There is a representation $\rho_{\text{univ}}: \pi_1(X)\to GL_n(R_{\text{univ}})$ (Univ. Defo ring) $R_{\text{univ}}$ loc. complete, residue field $\mathbf{F}_l$ and $(R_{\text{univ}}\to \mathbf{F}_l)\circ\rho_{\text{univ}}\cong\rho_0$. \\
And given any $R\to \mathbf{F}_l$, $R$ local complete and $\rho: \pi_1(X)\to GL_n(R)$ then there exists $\psi: R_{\text{univ}}\to R$ such that $\psi\circ\rho_{\text{univ}}\cong \rho$. 
$$
\text{Spec}(R_{\text{univ}})
\xrightarrow{\text{finite defo}}\\
\text{Spec}(\mathbf{Z}_l)
$$
In particular, such a $\rho_0$ (there is a picture...)
lifts to a $\rho: \pi_1(X)\to GL_n(\overline{\mathbf{Q}}_l)$\\

\noindent
\underline{Notes } \begin{itemize}
	\item The therorem on defo. is easy
	\item Any rewrite on conj. seems hard.
	\item I'd like have conjectures on $\pi_1(X)$!
\end{itemize}

%12.10.09

\paragraph{Counting points} Let $X$ be a smooth, geometrically irreducible, projective curve over $k$. $q=\sharp k$. Trace formula gives:\\
there exists algebraic integers $w_1, \ldots, w_{2g}$ such that 
	$$\sharp X(k_n) = q^n-\sum_{i=1}^{2g_X}w_i^n+1.$$
If $\sigma\in \text{Aut}(X)$ then for all $i$, there exists $j$ such that $\sigma(w_i)=w_j$. 

\paragraph{Riemann-Hypothesis} (Formulated by Emil Artin, in 1924 for hyper elliptic curves. Proved by Weil 1940) Two proofs
	\begin{itemize}
	\item on $X\times X$, using Hodge index theorem
	\item Jacobian of $X$
	\end{itemize}
S?. Bombieri: function field $k(X)$ and Frobenius operator (1969)
	$$f\in k(X), \; f^q-f...$$

\paragraph{Application I} Precise form of Chebotorav.\\
Let $Y\to^\varphi_G X$, Galois covering, finite etale, 
	$$G = \text{Aut}(\overline Y/X) \leftarrow \pi_1(X).$$
	$G=Gal(Y/X)$. 
	Assume $Y_{\overline k} = $ irreducible. \\
If $C\subset G$ is a conjugacy class then for all $n>0$, we have
	$$\left|\sharp\left\{x\in X(k_n)\left|F_x\in C\right.\right\}-\frac{\sharp C}{\sharp G}\cdot\sharp X(k_n)\right|\leq (\sharp C)(2g-2)\sqrt{q^n}$$
	(Warning: Please check ($\sharp C$) carefully before using.)\\
	
\begin{proof}[Sketch]
	$$\varphi_*(\overline Q_l) = \oplus_{\pi\in \widehat G} \mathcal{F}_{\pi}$$
	where $\widehat G = $ set of isom. classes of irred representations of $G$ over $\overline{\mathbf{Q}}_l$. For $\pi\in \widehat G$, 
		$$\chi_{\pi}: G\to\overline{\mathbf{Q}}_l$$
		character of $\pi$. 
	$$H^*(Y_{\overline k}, \overline{\mathbf{Q}}_l) = \oplus_{\pi\in \widehat G} H^*(Y_{\overline k}, \overline{\mathbf{Q}}_l)_\pi =_{(\varphi\text{ finite })} \oplus_{\pi\in \widehat G}H^*(X_{\overline k}, \mathcal{F}_\pi)$$
	If $\pi\neq 1$
	$$H^0(X_{\overline k},\mathcal{F}_\pi) = H^2(X_{\overline k}, \mathcal{F}_\pi)=0, \; \dim H^1(X_{\overline k}, \mathcal{F}_\pi) = (2g_X-2)d_\pi^2$$
	(can get this from trace formula for acting on ...)
	$$\left|\sum_{x\in X(k_n)}\chi_\pi(\mathcal{F}_x)\right|\leq_{\pi\neq 1} (2g_X-2)d_\pi^2\sqrt{q^n}$$
	
	Write $1_C = \sum_\pi a_\pi\chi_\pi, \; a_\pi=\left<1_C, \chi_\pi\right>$, $a_1 = \left<1_C, \chi_1\right> = \frac{\sharp C}{\sharp G}$
	$$\left<f, h\right> = \frac{1}{\sharp G}\sum_{g\in G}f(g)\overline{h(g)}$$
	$$\frac{\sharp C}{\sharp G} = ||1_C||^2 = \sum|a_\pi|^2$$

\noindent
Final step: 
	\begin{align*}
	\sharp\left\{x\in X(k_n)\left| F_x\in C\right.\right\}&=\sum_{x\in X(k_n)}1_C(x) = \sum_{x\in X)k_n}\sum_{\pi}a_\pi\chi_\pi(F_x)\\
	&=\underbrace{\frac{\sharp C}{\sharp G}\sharp X(k_n)}_{\text{term for }\pi=1}+\underbrace{\sum_{\pi\neq 1}a_\pi\sum_{x\in X(k_n)}\chi_\pi(F_x)}_{\text{ error term (to be bounded by $E$)}}
	\end{align*}
	\begin{align*}
	|E|&\leq \sum_{\pi\in \widehat G, \atop \pi\neq 1}|a_\pi|(2g-2)d_\pi^2\sqrt{q^n}\\
	&\leq \sum_{\pi\neq 1} \frac{\sharp C}{\sharp G} (2g_X-2)d_\pi^3\sqrt{q^n}
	\end{align*}
By Weil's conjecture, $\sharp X(k_n)\sim q^n$. 
\end{proof}

\paragraph{Application II}(Ihara, How many prime decompose completely...)
	$$1\to \pi_1(X_{\overline k})\to \pi_1(X)\to^{\deg}\widehat{\mathbf{Z}}\to 1$$
	\begin{proposition}
	There exists a finite set $x_1, \ldots, x_n$ of closed points of $X$ such that that set of \underline{all} frobenius elements corresponding to these points topologically generate $\pi_1(X)$. \\
	
\noindent
	(Variant statement) There exists $x_1, \ldots, x_n\in |X|$ such that there exists the smallest normal closed subgroup $\Gamma$ of $\pi_1(X)$ containing $1$ frobenius element for each $x_i$ is all of $\pi_1(X)$. i.e., $\Gamma=\pi_1(X)$.
	\end{proposition}
\begin{proof} Pick $N\gg 0$ and let 
	$$\left\{x_1, \ldots, x_n\right\}={\text{ set of all closed points of }\atop X\text{ of degree} \leq N\text{ over } k}$$
	Let $\Gamma\subset \pi_1(X)$ be as in variant statement for these points. Assume $\Gamma\neq \pi_1(X)$. We can pick $\Gamma\lhd \pi_1(X)$ with $U\neq \pi_1(X)$. By R.H. for $X$ this set I will have some $x_{i_1}$ of degree $N$, some $x_{i_2}$ of degree $N-1$. This shows $\Gamma\to^{\deg}\widehat{\mathbf{Z}}$ and $z_0$ and so also $U$. This exactly means if $Y\to X$ is the finite etale Galois covering as to $U$, then $Y_{\overline k}$ irreducible. 
		$$Y\to^G X, \; G = \pi_1(X)/U$$
		By construction all points of $X$ of degree $\leq N$, split completely in $Y$. So, in particular 
			$$\sharp Y(k_N)\geq (\sharp G)\sharp X(k_N)$$
			Use R.H. on both sides. So you get
		$$q^N+1+2g_Yq^{N/2}\geq \sharp G\sharp X(k_N)\geq \sharp G(q^N+1-2g_Xq^{N/2})$$
		Since $2g_Y-2 = (\sharp G)(2g_X-2)$, 
		$$q^N+1+(\sharp G)(2g_X-1)+1)q^{N/2}\geq \sharp G(q^N+1-2g_Xq^{N/2})$$
\end{proof}	

\paragraph{Weird Question} Set $W_X = \deg^{-1}(\mathbf{Z})\subset \pi_1(X)$. Is it true that for some finite set of closed points $x_1, \ldots, x_n$ of $X$ the set of all frobenii corresponding to these points algebraically generate $W_X$. \\
$\iff$ Baire cat. some question for all Frobenii. 

\paragraph{Drinfeld-Vledust: Number of points of an algebraic curve} Fix $q$, set
	$$A(q) = \text{lim}\sup_{X/k} \frac{\sharp X(k)}{g_X}$$
	($X$ as behave $k=\mathbf{F}_q$)
	$$g_x\to \infty$$
	\begin{itemize}
	\item RH $\Rightarrow\; A(q)\leq 2\sqrt q$
	\item Ihara $\Rightarrow\; A(q)\leq \sqrt{2q}$
	\item DV $A(q)\leq \sqrt q-1$ (actually this is sharp of $q$ is a square)
	\end{itemize}
	
\begin{proof} $X\to w_1, \ldots, w_{2g}$, $g=g_{X}$. Set $\alpha_i=\frac{w_i}{\sqrt q}$, $|\alpha_i|=1$ If $\alpha_i$ occurs then $\overline{\alpha}_i=\alpha_i^{-1}$ also occurs. Then
	$$N=\sharp X(k)\leq X(k_r)=q^r+1-(\sum\alpha_i)q^{r/2}$$
Rewrite:
	$$-\sum\alpha_i^r\geq Nq^{-r/2}-q^{r/2}-q^{-r/2}$$
	$$0\leq |\alpha_i^n +\alpha_i^{n-1} +\cdots +\alpha_i +1|^2 =(n+1)+\sum_{j=1}^M(n+1-j)(\alpha_i^j+\alpha_i^{-j})$$
	So
	\begin{align*}
	2g(n+1) &\geq -\sum_i \left(\sum_{j=1}^n (n+1-j)(\alpha_i^j +\alpha_i^{-j})\right)\\
	&=-\sum_{j=1}^n (n+1-j)\left(\sum_i\alpha_i^j +\sum_i\alpha_i^{-j}\right)
	\end{align*}
	\begin{align*}
	g(n+1)&\geq -\sum_{j=1}^n (n+1-j)(\sum_i\alpha_i^j)\\
	&\geq N\sum_{j=1}^n (n+1-j)q^{-j/2}-\sum_{j=1}^n (n+1-j)(q^{j/2}+q^{-j/2})\end{align*}
	This gives
	$$\frac{N}{g}\leq \left(\sum_{j=1}^n \frac{n+1-j}{n+1}q^{-j/2} \right)^{-1}\cdot\left(1+\frac{1}{g}\sum_{j=1}^n\frac{n+1-j}{n+1}(q^{j/2}+q^{-j/2})\right)$$
	Fix $n$ let $g\to \infty$
	$$A(q)\leq \left(\sum_{j=1}^n \frac{n+1-j}{n+1}q^{-j/2}\right)^{-1}$$
	So 
	$$A(q)\leq \text{lim}_{n\to\infty}(\cdots) = \left(\sum_{j=1}^\infty q^{-j/2}\right)^{-1}=\sqrt q-1$$
\end{proof}

\section{Other chapters}

\begin{multicols}{2}
\begin{enumerate}
\item \hyperref[introduction-section-phantom]{Introduction}
\item \hyperref[conventions-section-phantom]{Conventions}
\item \hyperref[sets-section-phantom]{Set Theory}
\item \hyperref[categories-section-phantom]{Categories}
\item \hyperref[topology-section-phantom]{Topology}
\item \hyperref[sheaves-section-phantom]{Sheaves on Spaces}
\item \hyperref[algebra-section-phantom]{Commutative Algebra}
\item \hyperref[sites-section-phantom]{Sites and Sheaves}
\item \hyperref[homology-section-phantom]{Homological Algebra}
\item \hyperref[derived-section-phantom]{Derived Categories}
\item \hyperref[more-algebra-section-phantom]{More Algebra}
\item \hyperref[simplicial-section-phantom]{Simplicial Methods}
\item \hyperref[modules-section-phantom]{Sheaves of Modules}
\item \hyperref[sites-modules-section-phantom]{Modules on Sites}
\item \hyperref[injectives-section-phantom]{Injectives}
\item \hyperref[cohomology-section-phantom]{Cohomology of Sheaves}
\item \hyperref[sites-cohomology-section-phantom]{Cohomology on Sites}
\item \hyperref[hypercovering-section-phantom]{Hypercoverings}
\item \hyperref[schemes-section-phantom]{Schemes}
\item \hyperref[constructions-section-phantom]{Constructions of Schemes}
\item \hyperref[properties-section-phantom]{Properties of Schemes}
\item \hyperref[morphisms-section-phantom]{Morphisms of Schemes}
\item \hyperref[coherent-section-phantom]{Coherent Cohomology}
\item \hyperref[divisors-section-phantom]{Divisors}
\item \hyperref[limits-section-phantom]{Limits of Schemes}
\item \hyperref[varieties-section-phantom]{Varieties}
\item \hyperref[chow-section-phantom]{Chow Homology}
\item \hyperref[topologies-section-phantom]{Topologies on Schemes}
\item \hyperref[descent-section-phantom]{Descent}
\item \hyperref[more-morphisms-section-phantom]{More on Morphisms}
\item \hyperref[flat-section-phantom]{More on Flatness}
\item \hyperref[groupoids-section-phantom]{Groupoid Schemes}
\item \hyperref[more-groupoids-section-phantom]{More on Groupoid Schemes}
\item \hyperref[etale-section-phantom]{\'Etale Morphisms of Schemes}
\item \hyperref[etale-cohomology-section-phantom]{\'Etale Cohomology}
\item \hyperref[spaces-section-phantom]{Algebraic Spaces}
\item \hyperref[spaces-properties-section-phantom]{Properties of Algebraic Spaces}
\item \hyperref[spaces-morphisms-section-phantom]{Morphisms of Algebraic Spaces}
\item \hyperref[spaces-topologies-section-phantom]{Topologies on Algebraic Spaces}
\item \hyperref[spaces-descent-section-phantom]{Descent and Algebraic Spaces}
\item \hyperref[spaces-more-morphisms-section-phantom]{More on Morphisms of Spaces}
\item \hyperref[quot-section-phantom]{Quot and Hilbert Spaces}
\item \hyperref[stacks-section-phantom]{Stacks}
\item \hyperref[spaces-groupoids-section-phantom]{Groupoids in Algebraic Spaces}
\item \hyperref[spaces-more-groupoids-section-phantom]{More on Groupoids in Spaces}
\item \hyperref[bootstrap-section-phantom]{Bootstrap}
\item \hyperref[examples-stacks-section-phantom]{Examples of Stacks}
\item \hyperref[groupoids-quotients-section-phantom]{Quotients of Groupoids}
\item \hyperref[algebraic-section-phantom]{Algebraic Stacks}
\item \hyperref[criteria-section-phantom]{Criteria for Representability}
\item \hyperref[stacks-properties-section-phantom]{Properties of Algebraic Stacks}
\item \hyperref[stacks-morphisms-section-phantom]{Morphisms of Algebraic Stacks}
\item \hyperref[examples-section-phantom]{Examples}
\item \hyperref[exercises-section-phantom]{Exercises}
\item \hyperref[guide-section-phantom]{Guide to Literature}
\item \hyperref[desirables-section-phantom]{Desirables}
\item \hyperref[coding-section-phantom]{Coding Style}
\item \hyperref[fdl-section-phantom]{GNU Free Documentation License}
\item \hyperref[index-section-phantom]{Auto Generated Index}
\end{enumerate}
\end{multicols}


\bibliography{my}
\bibliographystyle{amsalpha}

\end{document}
