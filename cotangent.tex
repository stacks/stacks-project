\IfFileExists{stacks-project.cls}{%
\documentclass{stacks-project}
}{%
\documentclass{amsart}
}

% The following AMS packages are automatically loaded with
% the amsart documentclass:
%\usepackage{amsmath}
%\usepackage{amssymb}
%\usepackage{amsthm}

% For dealing with references we use the comment environment
\usepackage{verbatim}
\newenvironment{reference}{\comment}{\endcomment}
%\newenvironment{reference}{}{}
\newenvironment{slogan}{\comment}{\endcomment}
\newenvironment{history}{\comment}{\endcomment}

% For commutative diagrams you can use
% \usepackage{amscd}
\usepackage[all]{xy}

% We use 2cell for 2-commutative diagrams.
\xyoption{2cell}
\UseAllTwocells

% To put source file link in headers.
% Change "template.tex" to "this_filename.tex"
% \usepackage{fancyhdr}
% \pagestyle{fancy}
% \lhead{}
% \chead{}
% \rhead{Source file: \url{template.tex}}
% \lfoot{}
% \cfoot{\thepage}
% \rfoot{}
% \renewcommand{\headrulewidth}{0pt}
% \renewcommand{\footrulewidth}{0pt}
% \renewcommand{\headheight}{12pt}

\usepackage{multicol}

% For cross-file-references
\usepackage{xr-hyper}

% Package for hypertext links:
\usepackage{hyperref}

% For any local file, say "hello.tex" you want to link to please
% use \externaldocument[hello-]{hello}
\externaldocument[introduction-]{introduction}
\externaldocument[conventions-]{conventions}
\externaldocument[sets-]{sets}
\externaldocument[categories-]{categories}
\externaldocument[topology-]{topology}
\externaldocument[sheaves-]{sheaves}
\externaldocument[sites-]{sites}
\externaldocument[stacks-]{stacks}
\externaldocument[fields-]{fields}
\externaldocument[algebra-]{algebra}
\externaldocument[brauer-]{brauer}
\externaldocument[homology-]{homology}
\externaldocument[derived-]{derived}
\externaldocument[simplicial-]{simplicial}
\externaldocument[more-algebra-]{more-algebra}
\externaldocument[smoothing-]{smoothing}
\externaldocument[modules-]{modules}
\externaldocument[sites-modules-]{sites-modules}
\externaldocument[injectives-]{injectives}
\externaldocument[cohomology-]{cohomology}
\externaldocument[sites-cohomology-]{sites-cohomology}
\externaldocument[dga-]{dga}
\externaldocument[dpa-]{dpa}
\externaldocument[hypercovering-]{hypercovering}
\externaldocument[schemes-]{schemes}
\externaldocument[constructions-]{constructions}
\externaldocument[properties-]{properties}
\externaldocument[morphisms-]{morphisms}
\externaldocument[coherent-]{coherent}
\externaldocument[divisors-]{divisors}
\externaldocument[limits-]{limits}
\externaldocument[varieties-]{varieties}
\externaldocument[topologies-]{topologies}
\externaldocument[descent-]{descent}
\externaldocument[perfect-]{perfect}
\externaldocument[more-morphisms-]{more-morphisms}
\externaldocument[flat-]{flat}
\externaldocument[groupoids-]{groupoids}
\externaldocument[more-groupoids-]{more-groupoids}
\externaldocument[etale-]{etale}
\externaldocument[chow-]{chow}
\externaldocument[intersection-]{intersection}
\externaldocument[pic-]{pic}
\externaldocument[adequate-]{adequate}
\externaldocument[dualizing-]{dualizing}
\externaldocument[duality-]{duality}
\externaldocument[discriminant-]{discriminant}
\externaldocument[local-cohomology-]{local-cohomology}
\externaldocument[curves-]{curves}
\externaldocument[resolve-]{resolve}
\externaldocument[models-]{models}
\externaldocument[pione-]{pione}
\externaldocument[etale-cohomology-]{etale-cohomology}
\externaldocument[proetale-]{proetale}
\externaldocument[crystalline-]{crystalline}
\externaldocument[spaces-]{spaces}
\externaldocument[spaces-properties-]{spaces-properties}
\externaldocument[spaces-morphisms-]{spaces-morphisms}
\externaldocument[decent-spaces-]{decent-spaces}
\externaldocument[spaces-cohomology-]{spaces-cohomology}
\externaldocument[spaces-limits-]{spaces-limits}
\externaldocument[spaces-divisors-]{spaces-divisors}
\externaldocument[spaces-over-fields-]{spaces-over-fields}
\externaldocument[spaces-topologies-]{spaces-topologies}
\externaldocument[spaces-descent-]{spaces-descent}
\externaldocument[spaces-perfect-]{spaces-perfect}
\externaldocument[spaces-more-morphisms-]{spaces-more-morphisms}
\externaldocument[spaces-flat-]{spaces-flat}
\externaldocument[spaces-groupoids-]{spaces-groupoids}
\externaldocument[spaces-more-groupoids-]{spaces-more-groupoids}
\externaldocument[bootstrap-]{bootstrap}
\externaldocument[spaces-pushouts-]{spaces-pushouts}
\externaldocument[groupoids-quotients-]{groupoids-quotients}
\externaldocument[spaces-more-cohomology-]{spaces-more-cohomology}
\externaldocument[spaces-simplicial-]{spaces-simplicial}
\externaldocument[formal-spaces-]{formal-spaces}
\externaldocument[restricted-]{restricted}
\externaldocument[spaces-resolve-]{spaces-resolve}
\externaldocument[formal-defos-]{formal-defos}
\externaldocument[defos-]{defos}
\externaldocument[cotangent-]{cotangent}
\externaldocument[examples-defos-]{examples-defos}
\externaldocument[algebraic-]{algebraic}
\externaldocument[examples-stacks-]{examples-stacks}
\externaldocument[stacks-sheaves-]{stacks-sheaves}
\externaldocument[criteria-]{criteria}
\externaldocument[artin-]{artin}
\externaldocument[quot-]{quot}
\externaldocument[stacks-properties-]{stacks-properties}
\externaldocument[stacks-morphisms-]{stacks-morphisms}
\externaldocument[stacks-limits-]{stacks-limits}
\externaldocument[stacks-cohomology-]{stacks-cohomology}
\externaldocument[stacks-perfect-]{stacks-perfect}
\externaldocument[stacks-introduction-]{stacks-introduction}
\externaldocument[stacks-more-morphisms-]{stacks-more-morphisms}
\externaldocument[stacks-geometry-]{stacks-geometry}
\externaldocument[moduli-]{moduli}
\externaldocument[moduli-curves-]{moduli-curves}
\externaldocument[examples-]{examples}
\externaldocument[exercises-]{exercises}
\externaldocument[guide-]{guide}
\externaldocument[desirables-]{desirables}
\externaldocument[coding-]{coding}
\externaldocument[obsolete-]{obsolete}
\externaldocument[fdl-]{fdl}
\externaldocument[index-]{index}

% Theorem environments.
%
\theoremstyle{plain}
\newtheorem{theorem}[subsection]{Theorem}
\newtheorem{proposition}[subsection]{Proposition}
\newtheorem{lemma}[subsection]{Lemma}

\theoremstyle{definition}
\newtheorem{definition}[subsection]{Definition}
\newtheorem{example}[subsection]{Example}
\newtheorem{exercise}[subsection]{Exercise}
\newtheorem{situation}[subsection]{Situation}

\theoremstyle{remark}
\newtheorem{remark}[subsection]{Remark}
\newtheorem{remarks}[subsection]{Remarks}

\numberwithin{equation}{subsection}

% Macros
%
\def\lim{\mathop{\rm lim}\nolimits}
\def\colim{\mathop{\rm colim}\nolimits}
\def\Spec{\mathop{\rm Spec}}
\def\Hom{\mathop{\rm Hom}\nolimits}
\def\Ext{\mathop{\rm Ext}\nolimits}
\def\SheafHom{\mathop{\mathcal{H}\!{\it om}}\nolimits}
\def\SheafExt{\mathop{\mathcal{E}\!{\it xt}}\nolimits}
\def\Sch{\textit{Sch}}
\def\Mor{\mathop{\rm Mor}\nolimits}
\def\Ob{\mathop{\rm Ob}\nolimits}
\def\Sh{\mathop{\textit{Sh}}\nolimits}
\def\NL{\mathop{N\!L}\nolimits}
\def\proetale{{pro\text{-}\acute{e}tale}}
\def\etale{{\acute{e}tale}}
\def\QCoh{\textit{QCoh}}
\def\Ker{\mathop{\rm Ker}}
\def\Im{\mathop{\rm Im}}
\def\Coker{\mathop{\rm Coker}}
\def\Coim{\mathop{\rm Coim}}

%
% Macros for moduli stacks/spaces
%
\def\QCohstack{\mathcal{QC}\!{\it oh}}
\def\Cohstack{\mathcal{C}\!{\it oh}}
\def\Spacesstack{\mathcal{S}\!{\it paces}}
\def\Quotfunctor{{\rm Quot}}
\def\Hilbfunctor{{\rm Hilb}}
\def\Curvesstack{\mathcal{C}\!{\it urves}}
\def\Polarizedstack{\mathcal{P}\!{\it olarized}}
\def\Complexesstack{\mathcal{C}\!{\it omplexes}}
% \Pic is the operator that assigns to X its picard group, usage \Pic(X)
% \Picardstack_{X/B} denotes the Picard stack of X over B
% \Picardfunctor_{X/B} denotes the Picard functor of X over B
\def\Pic{\mathop{\rm Pic}\nolimits}
\def\Picardstack{\mathcal{P}\!{\it ic}}
\def\Picardfunctor{{\rm Pic}}
\def\Deformationcategory{\mathcal{D}\!{\it ef}}


% OK, start here.
%
\begin{document}

\title{The Cotangent Complex}


\maketitle

\phantomsection
\label{section-phantom}

\tableofcontents

\section{Introduction}
\label{section-introduction}

\noindent
The goal of this chapter is to construct the cotangent complex of a
morphism of schemes and of a morphism of algebraic spaces.
Some references are the notes \cite{quillenhomology}, the paper
\cite{quillencohomology}, and the books
\cite{Andre} and \cite{cotangent}.





\section{Calculating derived lower shriek}
\label{section-calculate}

\noindent
In this section we work out some special cases of the situation
discussed in Cohomology on Sites, Section
\ref{sites-cohomology-section-derived-lower-shriek}.
We make sure that we have equality between lower shriek on modules
and sheaves of abelian groups. We encourage the reader to skip
this section on a first reading.

\begin{situation}
\label{situation-fibred-category}
Here $(\mathcal{D}, \mathcal{O}_\mathcal{D})$ be a ringed site
and $p : \mathcal{C} \to \mathcal{D}$ is a fibred category. We endow
$\mathcal{C}$ with the topology inherited from $\mathcal{D}$
(Stacks, Section \ref{stacks-section-topology}). We denote
$\pi : \Sh(\mathcal{C}) \to \Sh(\mathcal{D})$ the morphism of
topoi associated to $p$
(Stacks, Lemma \ref{stacks-lemma-topology-inherited-functorial}).
We set $\mathcal{O}_\mathcal{C} = \pi^{-1}\mathcal{O}_\mathcal{D}$
so that we obtain a morphism of ringed topoi
$$
\pi :
(\Sh(\mathcal{C}), \mathcal{O}_\mathcal{C})
\longrightarrow
(\Sh(\mathcal{D}), \mathcal{O}_\mathcal{D})
$$
\end{situation}

\begin{lemma}
\label{lemma-fibred-category-with-object}
Assumptions and notation as in Situation \ref{situation-fibred-category}.
For $U \in \Ob(\mathcal{C})$ consider the induced morphism
of topoi
$$
\pi_U : \Sh(\mathcal{C}/U) \longrightarrow \Sh(\mathcal{D}/p(U))
$$
Then there exists a morphism of topoi
$$
\sigma : \Sh(\mathcal{D}/p(U)) \to \Sh(\mathcal{C}/U)
$$
such that $\pi_U \circ \sigma = \text{id}$ and $\sigma^{-1} = \pi_{U, *}$.
\end{lemma}

\begin{proof}
Observe that $\pi_U$ is the restriction of $\pi$ to the localizations, see
Sites, Lemma \ref{sites-lemma-localize-cocontinuous}.
For an object $V \to p(U)$ of $\mathcal{D}/p(U)$ denote
$V \times_{p(U)} U \to U$ the strongly cartesian morphism of $\mathcal{C}$
over $\mathcal{D}$ which exists as $p$ is a fibred category.
The functor
$$
v : \mathcal{D}/p(U) \to \mathcal{C}/U,\quad
V/p(U) \mapsto V \times_{p(U)} U/U
$$
is continuous by the definition of the topology on $\mathcal{C}$.
Moreover, it is a right adjoint to $p$ by the definition of strongly
cartesian morphisms. Hence we are in the situation discussed in
Sites, Section \ref{sites-section-cocontinuous-adjoint}
and we see that the sheaf $\pi_{U, *}\mathcal{F}$
is equal to $V \mapsto \mathcal{F}(V \times_{p(U)} U)$
(see especially Sites, Lemma
\ref{sites-lemma-have-functor-other-way-morphism}).

\medskip\noindent
But here we have more. Namely, the functor $v$
is also cocontinuous (as all morphisms in coverings of $\mathcal{C}$ 
are strongly cartesian). Hence $v$ defines a morphism $\sigma$ as
indicated in the lemma. The equality $\sigma^{-1} = \pi_{U, *}$
is immediate from the definition. Since $\pi_U^{-1}\mathcal{G}$
is given by the rule $U'/U \mapsto \mathcal{G}(p(U')/p(U))$
it follows that $\sigma^{-1} \circ \pi_U^{-1} = \text{id}$
which proves the equality
$\pi_U \circ \sigma = \text{id}$.
\end{proof}

\begin{situation}
\label{situation-morphism-fibred-categories}
Let $(\mathcal{D}, \mathcal{O}_\mathcal{D})$ be a ringed site.
Let $u : \mathcal{C}' \to \mathcal{C}$ be a $1$-morphism of fibred
categories over $\mathcal{D}$
(Categories, Definition \ref{categories-definition-fibred-categories-over-C}).
Endow $\mathcal{C}$ and $\mathcal{C}'$ with their inherited topologies
(Stacks, Definition \ref{stacks-definition-topology-inherited})
and let
$\pi : \Sh(\mathcal{C}) \to \Sh(\mathcal{D})$,
$\pi' : \Sh(\mathcal{C}') \to \Sh(\mathcal{D})$, and
$g : \Sh(\mathcal{C}') \to \Sh(\mathcal{C})$
be the corresponding morphisms of topoi
(Stacks, Lemma \ref{stacks-lemma-topology-inherited-functorial}).
Set $\mathcal{O}_\mathcal{C} = \pi^{-1}\mathcal{O}_\mathcal{D}$
and $\mathcal{O}_{\mathcal{C}'} = (\pi')^{-1}\mathcal{O}_\mathcal{D}$.
Observe that $g^{-1}\mathcal{O}_\mathcal{C} = \mathcal{O}_{\mathcal{C}'}$
so that
$$
\xymatrix{
(\Sh(\mathcal{C}'), \mathcal{O}_{\mathcal{C}'}) \ar[rd]_{\pi'} \ar[rr]_g & &
(\Sh(\mathcal{C}), \mathcal{O}_\mathcal{C}) \ar[ld]^\pi \\
& (\Sh(\mathcal{D}), \mathcal{O}_\mathcal{D})
}
$$
is a commutative diagram of morphisms of ringed topoi.
\end{situation}

\begin{lemma}
\label{lemma-morphism-fibred-categories-with-object}
Assumptions and notation as in
Situation \ref{situation-morphism-fibred-categories}.
For $U' \in \Ob(\mathcal{C}')$ set $U = u(U')$ and $V = p'(U')$ and
consider the induced morphisms of ringed topoi
$$
\xymatrix{
(\Sh(\mathcal{C}'/U'), \mathcal{O}_{U'}) \ar[rd]_{\pi'_{U'}} \ar[rr]_{g'} & &
(\Sh(\mathcal{C}), \mathcal{O}_U) \ar[ld]^{\pi_U} \\
& (\Sh(\mathcal{D}/V), \mathcal{O}_V)
}
$$
Then there exists a morphism of topoi
$$
\sigma' : \Sh(\mathcal{D}/V) \to \Sh(\mathcal{C}'/U'),
$$
such that setting $\sigma = g' \circ \sigma'$ we have
$\pi'_{U'} \circ \sigma' = \text{id}$, $\pi_U \circ \sigma = \text{id}$,
$(\sigma')^{-1} = \pi'_{U', *}$, and $\sigma^{-1} = \pi_{U, *}$.
\end{lemma}

\begin{proof}
Let $v' : \mathcal{D}/V \to \mathcal{C}'/U'$ be the functor constructed
in the proof of Lemma \ref{lemma-fibred-category-with-object} starting
with $p' : \mathcal{C}' \to \mathcal{D}'$ and the object $U'$.
Since $u$ is a $1$-morphism of fibred categories over $\mathcal{D}$
it transforms strongly cartesian morphisms into strongly cartesian morphisms,
hence the functor $v = u \circ v'$ is the functor of
the proof of Lemma \ref{lemma-fibred-category-with-object}
relative to $p : \mathcal{C} \to \mathcal{D}$ and $U$. Thus our lemma
follows from that lemma.
\end{proof}

\begin{lemma}
\label{lemma-properties-lower-shriek-fibred-category}
Assumption and notation as in
Situation \ref{situation-morphism-fibred-categories}.
\begin{enumerate}
\item There are left adjoints
$g_! : \textit{Mod}(\mathcal{O}_{\mathcal{C}'}) \to
\textit{Mod}(\mathcal{O}_\mathcal{C})$ and
$g_!^{\textit{Ab}} : \textit{Ab}(\mathcal{C}') \to \textit{Ab}(\mathcal{C})$
to $g^* = g^{-1}$ on modules and on abelian sheaves.
\item The diagram
$$
\xymatrix{
\textit{Mod}(\mathcal{O}_{\mathcal{C}'}) \ar[d] \ar[r]_{g_!} &
\textit{Mod}(\mathcal{O}_\mathcal{C}) \ar[d] \\
\textit{Ab}(\mathcal{C}') \ar[r]^{g_!^{\textit{Ab}}} &
\textit{Ab}(\mathcal{C})
}
$$
commutes.
\item There are left adjoints
$Lg_! : D(\mathcal{O}_{\mathcal{C}'}) \to D(\mathcal{O}_\mathcal{C})$
and
$Lg_!^{\textit{Ab}} : D(\mathcal{C}') \to D(\mathcal{C})$
to $g^* = g^{-1}$ on derived categories of modules and abelian sheaves.
\item The diagram
$$
\xymatrix{
D(\mathcal{O}_{\mathcal{C}'}) \ar[d] \ar[r]_{Lg_!} &
D(\mathcal{O}_\mathcal{C}) \ar[d] \\
D(\mathcal{C}') \ar[r]^{Lg_!^{\textit{Ab}}} &
D(\mathcal{C})
}
$$
commutes.
\end{enumerate}
\end{lemma}

\begin{proof}
The functor $u$ is continuous and cocontinuous
Stacks, Lemma \ref{stacks-lemma-topology-inherited-functorial}.
Hence the existence of the functors $g_!$, $g_!^{\textit{Ab}}$,
$Lg_!$, and $Lg_!^{\textit{Ab}}$ can be found in
Modules on Sites, Sections
\ref{sites-modules-section-exactness-lower-shriek} and
\ref{sites-modules-section-lower-shriek-modules}
and
Cohomology on Sites, Section
\ref{sites-cohomology-section-derived-lower-shriek}.

\medskip\noindent
To prove (2) it suffices to show that the canonical map
$$
g_!^{\textit{Ab}}j_{U'!}\mathcal{O}_{U'} \to j_{u(U')!}\mathcal{O}_{u(U')}
$$
is an isomorphism for all objects $U'$ of $\mathcal{C}'$, see
Modules on Sites, Remark \ref{sites-modules-remark-when-shriek-equal}.
Similarly, to prove (4) it suffices to show that the canonical map
$$
Lg_!^{\textit{Ab}}j_{U'!}\mathcal{O}_{U'} \to j_{u(U')!}\mathcal{O}_{u(U')}
$$
is an isomorphism in $D(\mathcal{C})$ for all objects $U'$ of $\mathcal{C}'$.
This will then also imply the previous formula hence this is what we will show.

\medskip\noindent
We will use that for a localization morphism $j$ the
functors $j_!$ and $j_!^{\textit{Ab}}$ agree (see
Modules on Sites, Remark \ref{sites-modules-remark-localize-shriek-equal})
and that $j_!$ is exact
(Modules on Sites, Lemma \ref{sites-modules-lemma-extension-by-zero-exact}).
Let us adopt the notation of
Lemma \ref{lemma-morphism-fibred-categories-with-object}.
Since $Lg_!^{\textit{Ab}} \circ j_{U'!} = j_{U!} \circ L(g')^{\textit{Ab}}_!$
(by commutativity of Sites, Lemma \ref{sites-lemma-localize-cocontinuous}
and uniqueness of adjoint functors) it suffices to prove that
$L(g')^{\textit{Ab}}_!\mathcal{O}_{U'} = \mathcal{O}_U$. Using the
results of
Lemma \ref{lemma-morphism-fibred-categories-with-object}
we have for any object $E$ of $D(\mathcal{C}/u(U'))$ the following
sequence of equalities
\begin{align*}
\Hom_{D(\mathcal{C}/U)}(L(g')_!^{\textit{Ab}}\mathcal{O}_{U'}, E)
& =
\Hom_{D(\mathcal{C}'/U')}(\mathcal{O}_{U'}, (g')^{-1}E) \\
& =
\Hom_{D(\mathcal{C}'/U')}((\pi'_{U'})^{-1}\mathcal{O}_V, (g')^{-1}E) \\
& =
\Hom_{D(\mathcal{D}/V)}(\mathcal{O}_V, R\pi'_{U', *}(g')^{-1}E) \\
& =
\Hom_{D(\mathcal{D}/V)}(\mathcal{O}_V, (\sigma')^{-1}(g')^{-1}E) \\
& =
\Hom_{D(\mathcal{D}/V)}(\mathcal{O}_V, \sigma^{-1}E) \\
& =
\Hom_{D(\mathcal{D}/V)}(\mathcal{O}_V, \pi_{U, *}E) \\
& =
\Hom_{D(\mathcal{C}/U)}(\pi_U^{-1}\mathcal{O}_V, E) \\
& =
\Hom_{D(\mathcal{C}/U)}(\mathcal{O}_U, E)
\end{align*}
By Yoneda's lemma we conclude.
\end{proof}

\begin{remark}
\label{remark-morphism-fibred-categories}
Assumptions and notation as in
Situation \ref{situation-morphism-fibred-categories}.
Let $\mathcal{F}$ be an abelian sheaf on $\mathcal{C}$,
let $\mathcal{F}'$ be an abelian sheaf on $\mathcal{C}'$,
and let $t : \mathcal{F}' \to g^{-1}\mathcal{F}$ be a map.
Then we obtain a canonical map
$$
L\pi'_!\mathcal{F}' \longrightarrow L\pi_!\mathcal{F}
$$
by using the adjoint $g_!\mathcal{F}' \to \mathcal{F}$ of $t$,
the map $Lg_!\mathcal{F}' \to g_!\mathcal{F}'$, and the
equality $L\pi'_! = L\pi_! \circ Lg_!$.
\end{remark}

\begin{lemma}
\label{lemma-compute-pi-shriek}
Assumptions and notation as in
Situation \ref{situation-fibred-category}.
For $\mathcal{F}$ in $\textit{Ab}(\mathcal{C})$
the sheaf $\pi_!\mathcal{F}$ is the
sheaf associated the presheaf
$$
V \longmapsto \colim_{\mathcal{C}_V^{opp}} \mathcal{F}|_{\mathcal{C}_V}
$$
with restriction maps as indicated in the proof.
\end{lemma}

\begin{proof}
Denote $\mathcal{H}$ be the rule of the lemma.
For a morphism $h : V' \to V$ of $\mathcal{D}$ there is a
pullback functor $h^* : \mathcal{C}_V \to \mathcal{C}_{V'}$ of fibre
categories (Categories, Definition
\ref{categories-definition-pullback-functor-fibred-category}).
Moreover for $U \in \Ob(\mathcal{C}_V)$ there is a
strongly cartesian morphism $h^*U \to U$ covering $h$.
Restriction along these strongly cartesian morphisms defines a
transformation of functors
$$
\mathcal{F}|_{\mathcal{C}_V}
\longrightarrow
\mathcal{F}|_{\mathcal{C}_{V'}} \circ h^*.
$$
Hence a map $\mathcal{H}(V) \to \mathcal{H}(V')$ between colimits, see
Categories, Lemma \ref{categories-lemma-functorial-colimit}.

\medskip\noindent
To prove the lemma we show that
$$
\Mor_{\textit{PSh}(\mathcal{D})}(\mathcal{H}, \mathcal{G}) =
\Mor_{\textit{Sh}(\mathcal{C})}(\mathcal{F}, \pi^{-1}\mathcal{G})
$$
for every sheaf $\mathcal{G}$ on $\mathcal{C}$. An element of the
left hand side is a compatible system of maps
$\mathcal{F}(U) \to \mathcal{G}(p(U))$ for all $U$ in $\mathcal{C}$.
Since $\pi^{-1}\mathcal{G}(U) = \mathcal{G}(p(U))$ by our choice
of topology on $\mathcal{C}$ we see the same thing is true for the
right hand side and we win.
\end{proof}

\begin{example}[Category over point]
\label{example-category-to-point}
Let $\mathcal{C}$ be a category. Endow $\mathcal{C}$ with the chaotic
topology (Sites, Example \ref{sites-example-indiscrete}). Thus
presheaves and sheaves agree on $\mathcal{C}$.
The functor $p : \mathcal{C} \to *$ where $*$ is the category with a single
object and a single morphism is cocontinuous and continuous. Let
$\pi : \Sh(\mathcal{C}) \to \Sh(*)$ be the corresponding morphism
of topoi. Let $B$ be a ring. We endow $*$ with the sheaf of rings $B$
and $\mathcal{C}$ with $\mathcal{O}_\mathcal{C} = \pi^{-1}B$ which
we will often denote $\underline{B}$. In this way
$$
\pi : (\Sh(\mathcal{C}), \underline{B}) \to (*, B)
$$
is an example of Situation \ref{situation-fibred-category}
and
Lemma \ref{lemma-properties-lower-shriek-fibred-category}
applies to $\pi$ so we do not need to distinguish between $\pi_!$ on
modules or abelian sheaves. By
Lemma \ref{lemma-compute-pi-shriek} we see that
$\pi_!\mathcal{F} = \colim_{\mathcal{C}^{opp}} \mathcal{F}$.
We write
$L_n\pi_!\mathcal{F} = L_n\colim_{\mathcal{C}^{opp}} \mathcal{F}$
in this setting.
\end{example}

\begin{example}[Computing left derived colimits]
\label{example-left-derived-colimits}
In the situation of Example \ref{example-category-to-point} we can compute
the functors $L_n\colim_{\mathcal{C}^{opp}}$ as follows. Let
$\mathcal{F} \in \Ob(\textit{Ab}(\mathcal{C}))$.
Consider the complex
$$
K^\bullet(\mathcal{F}) :
\ \ldots \to
\bigoplus\nolimits_{U_2 \to U_1 \to U_0} \mathcal{F}(U_0)
\to
\bigoplus\nolimits_{U_1 \to U_0} \mathcal{F}(U_0)
\to
\bigoplus\nolimits_{U_0} \mathcal{F}(U_0)
$$
where the transition maps are given by
$$
(U_2 \to U_1 \to U_0, s)
\longmapsto
(U_1 \to U_0, s) - (U_2 \to U_0, s) + (U_2 \to U_1, s|_{U_1})
$$
and similarly in other degrees. By construction
$H^0(K^\bullet(\mathcal{F})) = \colim_{\mathcal{C}^{opp}} \mathcal{F}$, see
Categories, Lemma \ref{categories-lemma-colimits-coproducts-coequalizers}.
The construction of $K^\bullet(\mathcal{F})$ is functorial in $\mathcal{F}$
and transforms short exact sequences of $\textit{Ab}(\mathcal{C})$ into
short exact sequences of complexes. Thus the sequence of functors
$\mathcal{F} \mapsto H^n(K^\bullet(\mathcal{F}))$ forms a $\delta$-functor, see
Homology, Definition \ref{homology-definition-cohomological-delta-functor} and
Lemma \ref{homology-lemma-long-exact-sequence-cochain}.
For $\mathcal{F} = j_{U!}\mathbf{Z}_U$ the complex $K^\bullet(\mathcal{F})$
is the complex associated to the free $\mathbf{Z}$-module on the simplicial
set $X_\bullet$ with terms
$$
X_n = \coprod\nolimits_{U_n \to \ldots \to U_1 \to U_0}
\Mor_\mathcal{C}(U_0, U)
$$
This simplicial set is homotopy equivalent to the constant simplicial
set on a singleton $\{*\}$. Namely, the map $X_\bullet \to \{*\}$
is obvious, the map $\{*\} \to X_n$ is given
by mapping $*$ to $(U \to \ldots \to U, \text{id}_U)$, and the
maps
$$
h_{n, i} : X_n \longrightarrow X_n
$$
(Simplicial, Lemma \ref{simplicial-lemma-relations-homotopy})
defining the homotopy between the two maps $X_\bullet \to X_\bullet$
are given by the rule
$$
h_{n, i} :
(U_n \to \ldots \to U_0, f)
\longmapsto
(U_n \to \ldots \to U_i \to U \to \ldots \to U, \text{id})
$$
for $i > 0$ and $h_{n, 0} = \text{id}$. Verifications omitted.
This implies that $K^\bullet(j_{U!}\mathbf{Z}_U)$ has trivial
cohomology in negative degrees
(by the functoriality of
Simplicial, Remark \ref{simplicial-remark-homotopy-better}
and the result of
Simplicial, Lemma \ref{simplicial-lemma-homotopy-s-N}).
Thus $K^\bullet(\mathcal{F})$ computes the left derived functors
of $\colim_{\mathcal{C}^{opp}}$ for example by
(the duals of) Homology, Lemma \ref{homology-lemma-efface-implies-universal}
and
Derived Categories, Lemma \ref{derived-lemma-right-derived-delta-functor}.
\end{example}

\begin{example}
\label{example-morphism-categories}
Let $u : \mathcal{C}' \to \mathcal{C}$ be a functor.
Endow $\mathcal{C}'$ and $\mathcal{C}$ with the chaotic
topology as in Example \ref{example-category-to-point}.
The functors $u$, $\mathcal{C}' \to *$, and $\mathcal{C} \to *$
where $*$ is the category with a single object and a single morphism
are cocontinuous and continuous. Let
$g : \Sh(\mathcal{C}') \to \Sh(\mathcal{C})$,
$\pi' : \Sh(\mathcal{C}') \to \Sh(*)$, and
$\pi : \Sh(\mathcal{C}) \to \Sh(*)$,
be the corresponding morphisms of topoi.
Let $B$ be a ring. We endow $*$ with the sheaf of rings $B$ and
$\mathcal{C}'$, $\mathcal{C}$ with the constant sheaf $\underline{B}$.
In this way
$$
\xymatrix{
(\Sh(\mathcal{C}'), \underline{B}) \ar[rd]_{\pi'} \ar[rr]_g & &
(\Sh(\mathcal{C}), \underline{B}) \ar[ld]^\pi \\
& (\Sh(*), B)
}
$$
is an example of Situation \ref{situation-morphism-fibred-categories}.
Thus
Lemma \ref{lemma-properties-lower-shriek-fibred-category}
applies to $g$ so we do not need to distinguish between $g_!$ on
modules or abelian sheaves. In particular
Remark \ref{remark-morphism-fibred-categories}
produces canonical maps
$$
L_n\colim_{(\mathcal{C}')^{opp}} \mathcal{F}'
\longrightarrow
L_n\colim_{\mathcal{C}^{opp}} \mathcal{F}
$$
whenever we have $\mathcal{F}$ in $\textit{Ab}(\mathcal{C})$,
$\mathcal{F}'$ in $\textit{Ab}(\mathcal{C}')$,
and a map $t : \mathcal{F}' \to g^{-1}\mathcal{F}$. In terms of the
computation of $L_n\colim$ given in
Example \ref{example-left-derived-colimits}
we see that these maps come from a map of complexes
$$
K^\bullet(\mathcal{F}') \longrightarrow K^\bullet(\mathcal{F})
$$
given by the rule
$$
(U'_n \to \ldots \to U'_0, s') \longmapsto
(u(U'_n) \to \ldots \to u(U'_0), t(s'))
$$
with obvious notation.
\end{example}

\begin{lemma}
\label{lemma-compute-left-derived-pi-shriek-pre}
Assumptions and notation as in Situation \ref{situation-fibred-category}.
For $\mathcal{F}$ in $\textit{PAb}(\mathcal{C})$ and $n \geq 0$
consider the abelian sheaf $L_n(\mathcal{F})$ on $\mathcal{D}$
which is the sheaf associated to the presheaf
$$
V \longmapsto
L_n\colim_{\mathcal{C}_V^{opp}} \mathcal{F}|_{\mathcal{C}_V}
$$
with restriction maps as indicated in the proof. Then
$L_n(\mathcal{F}) = L_n(\mathcal{F}^\#)$.
\end{lemma}

\begin{proof}
For a morphism $h : V' \to V$ of $\mathcal{D}$ there is a
pullback functor $h^* : \mathcal{C}_V \to \mathcal{C}_{V'}$ of fibre
categories (Categories, Definition
\ref{categories-definition-pullback-functor-fibred-category}).
Moreover for $U \in \Ob(\mathcal{C}_V)$ there is a
strongly cartesian morphism $h^*U \to U$ covering $h$.
Restriction along these strongly cartesian morphisms defines a
transformation of functors
$$
\mathcal{F}|_{\mathcal{C}_V}
\longrightarrow
\mathcal{F}|_{\mathcal{C}_{V'}} \circ h^*.
$$
By Example \ref{example-morphism-categories}
we obtain the desired restriction map
$$
L_n\colim_{\mathcal{C}_V^{opp}} \mathcal{F}|_{\mathcal{C}_V}
\longrightarrow
L_n\colim_{\mathcal{C}_{V'}^{opp}} \mathcal{F}|_{\mathcal{C}_{V'}}
$$
Let us denote $L_{n, p}(\mathcal{F})$ this presheaf, so that
$L_n(\mathcal{F}) = L_{n, p}(\mathcal{F})^\#$.
The canonical map $\gamma : \mathcal{F} \to \mathcal{F}^+$
(Sites, Theorem \ref{sites-theorem-plus})
defines a canonical
map $L_{n, p}(\mathcal{F}) \to L_{n, p}(\mathcal{F}^+)$.
We have to prove this map becomes an isomorphism after sheafification.

\medskip\noindent
Let us use the computation of $L_n\colim$ given in
Example \ref{example-left-derived-colimits}. Denote
$K^\bullet(\mathcal{F}|_{\mathcal{C}_V})$ the complex associated to
the restriction of $\mathcal{F}$ to the fibre category $\mathcal{C}_V$.
By the remarks above we obtain a presheaf $K^\bullet(\mathcal{F})$
of complexes
$$
V \longmapsto K^\bullet(\mathcal{F}|_{\mathcal{C}_V})
$$
whose cohomology presheaves are the presheaves $L_{n, p}(\mathcal{F})$.
Thus it suffices to show that
$$
K^\bullet(\mathcal{F}) \longrightarrow K^\bullet(\mathcal{F}^+)
$$
becomes an isomorphism on sheafification.

\medskip\noindent
Injectivity. Let $V$ be an object of $\mathcal{D}$ and let
$\xi \in K^{-n}(\mathcal{F})(V)$ be an element which maps
to zero in $K^{-n}(\mathcal{F}^+)(V)$. We have to show there exists a
covering $\{V_j \to V\}$ such that $\xi|_{V_j}$ is zero in
$K^{-n}(\mathcal{F})(V_j)$. We write
$$
\xi = \sum (U_{i, n + 1} \to \ldots \to U_{i, 0}, \sigma_i)
$$
with $\sigma_i \in \mathcal{F}(U_{i, 0})$. We arrange it so that
each sequence of morphisms $U_n \to \ldots \to U_0$ of $\mathcal{C}_V$
occurs are most once. Since the sums in the definition
of the complex $K^\bullet$ are direct sums, the only way this can map
to zero in $K^\bullet(\mathcal{F}^+)(V)$ is if all $\sigma_i$ map
to zero in $\mathcal{F}^+(U_{i, 0})$. By construction of
$\mathcal{F}^+$ there exist coverings $\{U_{i, 0, j} \to U_{i, 0}\}$
such that $\sigma_i|_{U_{i, 0, j}}$ is zero. By our construction of
the topology on $\mathcal{C}$ we can write $U_{i, 0, j} \to U_{i, 0}$
as the pullback (Categories, Definition
\ref{categories-definition-pullback-functor-fibred-category})
of some morphisms $V_{i, j} \to V$ and moreover each
$\{V_{i, j} \to V\}$ is a covering. Choose a covering
$\{V_j \to V\}$ dominating each of the coverings $\{V_{i, j} \to V\}$.
Then it is clear that $\xi|_{V_j} = 0$.

\medskip\noindent
Surjectivity. Proof omitted. Hint: Argue as in the proof of
injectivity.
\end{proof}

\begin{lemma}
\label{lemma-compute-left-derived-pi-shriek}
Assumptions and notation as in Situation \ref{situation-fibred-category}.
For $\mathcal{F}$ in $\textit{Ab}(\mathcal{C})$ and $n \geq 0$
the sheaf $L_n\pi_!\mathcal{F}$ is equal to the sheaf
$L_n(\mathcal{F})$ constructed in
Lemma \ref{lemma-compute-left-derived-pi-shriek-pre}.
\end{lemma}

\begin{proof}
Consider the sequence of functors $\mathcal{F} \mapsto L_n(\mathcal{F})$
from $\textit{PAb}(\mathcal{C}) \to \textit{Ab}(\mathcal{C})$.
Since for each $V \in \Ob(\mathcal{D})$ the sequence of functors
$L_n\colim_{\mathcal{C}_V^{opp}}$ forms a $\delta$-functor
so do the functors $\mathcal{F} \mapsto L_n(\mathcal{F})$.
Our goal is to show these form a universal $\delta$-functor.
In order to do this we construct some abelian presheaves
on which these functors vanish.

\medskip\noindent
For $U' \in \Ob(\mathcal{C})$ consider the abelian presheaf
$\mathcal{F}_{U'} = j_{U'!}^{\textit{PAb}}\mathbf{Z}_{U'}$
(Modules on Sites, Remark \ref{sites-modules-remark-localize-presheaves}).
Recall that
$$
\mathcal{F}_{U'}(U) =
\bigoplus\nolimits_{\Mor_\mathcal{C}(U, U')} \mathbf{Z}
$$
If $U$ lies over $V = p(U)$ in $\mathcal{D})$ and $U'$ lies over $V' = p(U')$
then any morphism $a : U \to U'$ factors uniquely as $U \to h^*U' \to U'$
where $h = p(a) : V \to V'$ (see
Categories, Definition
\ref{categories-definition-pullback-functor-fibred-category}).
Hence we see that
$$
\mathcal{F}_{U'}|_{\mathcal{C}_V}
=
\bigoplus\nolimits_{h \in \Mor_\mathcal{D}(V, V')}
j_{h^*U'!}\mathbf{Z}_{h^*U'}
$$
where $j_{h^*U'} : \Sh(\mathcal{C}_V/h^*U') \to \Sh(\mathcal{C}_V)$
is the localization morphism. The sheaves $j_{h^*U'!}\mathbf{Z}_{h^*U'}$
have vanishing higher left derived colimits (see
Example \ref{example-left-derived-colimits}).
We conclude that $L_n(\mathcal{F}_{U'}) = 0$ for all $n > 0$ and all $U'$.
It follows that any abelian presheaf $\mathcal{F}$ is a quotient
of an abelian presheaf $\mathcal{G}$ with $L_n(\mathcal{G}) = 0$ for
all $n > 0$ (Modules on Sites, Lemma
\ref{sites-modules-lemma-module-quotient-flat}).
Since $L_n(\mathcal{F}) = L_n(\mathcal{F}^\#)$ we see
that the same thing is true for abelian sheaves. Thus
the sequence of functors $L_n(-)$ is a universal delta functor
on $\textit{Ab}(\mathcal{C})$
(Homology, Lemma \ref{homology-lemma-efface-implies-universal}).
Since we have agreement with
$H^{-n}(L\pi_!(-))$ for $n = 0$ by
Lemma \ref{lemma-compute-pi-shriek}
we conclude by uniqueness of universal $\delta$-functors
(Homology, Lemma \ref{homology-lemma-uniqueness-universal-delta-functor})
and
Derived Categories, Lemma \ref{derived-lemma-right-derived-delta-functor}.
\end{proof}

\begin{lemma}
\label{lemma-compute-left-derived-g-shriek}
Assumptions and notation as in
Situation \ref{situation-morphism-fibred-categories}.
For an abelian sheaf $\mathcal{F}'$ on $\mathcal{C}'$ the sheaf
$L_ng_!(\mathcal{F}')$ is the sheaf associated to the presheaf
$$
U
\longmapsto
L_n\colim_{\mathcal{I}_U^{opp}} \mathcal{F}'_U
$$
For notation and restriction maps see proof.
\end{lemma}

\begin{proof}
Say $p(U) = V$. The category $\mathcal{I}_U$ is the category of pairs
$(U', \varphi)$ where $\varphi : U \to u(U')$ is a morphism of $\mathcal{C}$
with $p(\varphi) = \text{id}_V$, i.e., $\varphi$ is a morphism of the
fibre category $\mathcal{C}_V$. Morphisms
$(U'_1, \varphi_1) \to (U'_2, \varphi_2)$ are given by morphisms
$a : U'_1 \to U'_2$ of the fibre category $\mathcal{C}'_V$ such that
$\varphi_2 = u(a) \circ \varphi_1$. The presheaf $\mathcal{F}'_U$ sends
$(U', \varphi)$ to $\mathcal{F}'(U')$.
We will construct the restriction mappings below.

\medskip\noindent
Choose a factorization
$$
\xymatrix{
\mathcal{C}' \ar@<1ex>[r]^{u'} &
\mathcal{C}'' \ar[r]^{u''} \ar@<1ex>[l]^w & \mathcal{C}
}
$$
of $u$ as in
Categories, Lemma \ref{categories-lemma-ameliorate-morphism-fibred-categories}.
Then $g_! = g''_! \circ g'_!$ and similarly for derived functors.
On the other hand, the functor $g'_!$ is exact, see
Modules on Sites, Lemma \ref{sites-modules-lemma-have-left-adjoint}.
Thus we get $Lg_!\mathcal{F}' = Lg''_!\mathcal{F}''$ where
$\mathcal{F}'' = g'_!\mathcal{F}'$. Note that
$\mathcal{F}'' = h^{-1}\mathcal{F}'$ where
$h : \Sh(\mathcal{C}'') \to \Sh(\mathcal{C}')$ is the morphism of topoi
associated to $w$, see
Sites, Lemma \ref{sites-lemma-have-left-adjoint}.
The functor $u''$ turns $\mathcal{C}''$ into a fibred category
over $\mathcal{C}$, hence
Lemma \ref{lemma-compute-left-derived-pi-shriek}
applies to the computation of $L_ng''_!$. The result follows as the
construction of $\mathcal{C}''$ in the proof of
Categories, Lemma \ref{categories-lemma-ameliorate-morphism-fibred-categories}
shows that the fibre category $\mathcal{C}''_U$ is equal to
$\mathcal{I}_U$. Moreover, $h^{-1}\mathcal{F}'|_{\mathcal{C}''_U}$
is given by the rule described above
(as $w$ is continuous and cocontinuous by
Stacks, Lemma \ref{stacks-lemma-topology-inherited-functorial}
so we may apply
Sites, Lemma \ref{sites-lemma-when-shriek}).
\end{proof}

\begin{lemma}
\label{lemma-initial-final}
Notation and assumptions as in Example \ref{example-category-to-point}.
If $\mathcal{C}$ has either an initial or a final object, then
$L\pi_! \circ \pi^{-1} = \text{id}$ on $D(\textit{Ab})$, resp.\ $D(B)$.
\end{lemma}

\begin{proof}
If $\mathcal{C}$ has an initial object, then $\pi_!$ is computed by
evaluating on this object and the statement is clear. If $\mathcal{C}$
has a final object, then $R\pi_*$ is computed by evaluating on this
object, hence $R\pi_* \circ \pi^{-1} \cong \text{id}$ on
$D(\textit{Ab})$, resp.\ $D(B)$. This implies that
$\pi^{-1} : D(\textit{Ab}) \to D(\mathcal{C})$,
resp.\ $\pi^{-1} : D(B) \to D(\underline{B})$ is fully faithful, see
Categories, Lemma \ref{categories-lemma-adjoint-fully-faithful}.
Then the same lemma implies that $L\pi_! \circ \pi^{-1} = \text{id}$
as desired.
\end{proof}

\begin{lemma}
\label{lemma-change-of-rings}
Notation and assumptions as in Example \ref{example-category-to-point}.
Let $B \to B'$ be a ring map. Consider the commutative diagram
of ringed topoi
$$
\xymatrix{
(\Sh(\mathcal{C}), \underline{B}) \ar[d]^\pi &
(\Sh(\mathcal{C}), \underline{B'}) \ar[d]^{\pi'} \ar[l]^h \\
(*, B) & (*, B') \ar[l]_f
}
$$
Then $L\pi_! \circ Lh^* = Lf^* \circ L\pi'_!$.
\end{lemma}

\begin{proof}
Both functors are right adjoint to the obvious functor
$D(B') \to D(\underline{B})$.
\end{proof}

\begin{lemma}
\label{lemma-compute-by-cosimplicial-resolution}
Notation and assumptions as in Example \ref{example-category-to-point}.
Let $U_\bullet$ be a cosimplicial object in $\mathcal{C}$ such that
for every $U \in \Ob(\mathcal{C})$ the simplicial set
$\Mor_\mathcal{C}(U_\bullet, U)$
is homotopy equivalent to the constant simplical set on a singleton. Then
$$
L\pi_!\mathcal{F} = \mathcal{F}(U_\bullet)
$$
in $D(\textit{Ab})$, resp.\ $D(B)$ functorially in $\mathcal{F}$ in
$\textit{Ab}(\mathcal{C})$, resp.\ $\textit{Mod}(\underline{B})$.
\end{lemma}

\begin{proof}
As $L\pi_!$ agrees for modules and abelian sheaves by
Lemma \ref{lemma-properties-lower-shriek-fibred-category}
it suffices to prove this when $\mathcal{F}$ is an abelian sheaf.
For $U \in \Ob(\mathcal{C})$ the abelian sheaf $j_{U!}\mathbf{Z}_U$
is a projective object of $\textit{Ab}(\mathcal{C})$ since
$\Hom(j_{U!}\mathbf{Z}_U, \mathcal{F}) = \mathcal{F}(U)$
and taking sections is an exact functor as the topology is chaotic.
Every abelian sheaf is a quotient of a direct sum of $j_{U!}\mathbf{Z}_U$
by Modules on Sites, Lemma \ref{sites-modules-lemma-module-quotient-flat}.
Thus we can compute $L\pi_!\mathcal{F}$ by choosing a resolution
$$
\ldots \to \mathcal{G}^{-1} \to \mathcal{G}^0 \to \mathcal{F} \to 0
$$
whose terms are direct sums of sheaves of the form above and taking
$L\pi_!\mathcal{F} = \pi_!\mathcal{G}^\bullet$. Consider the
double complex
$A^{\bullet, \bullet} = \mathcal{G}^\bullet(U_\bullet)$.
The map $\mathcal{G}^0 \to \mathcal{F}$ gives a map of complexes
$A^{0, \bullet} \to \mathcal{F}(U_\bullet)$.
Since $\pi_!$ is computed by taking the colimit over
$\mathcal{C}^{opp}$ (Lemma \ref{lemma-compute-pi-shriek})
we see that the two compositions
$\mathcal{G}^m(U_1) \to \mathcal{G}^m(U_0) \to \pi_!\mathcal{G}^m$
are equal. Thus we obtain a canonical map of complexes
$$
\text{Tot}(A^{\bullet, \bullet})
\longrightarrow
\pi_!\mathcal{G}^\bullet = L\pi_!\mathcal{F}
$$
To prove the lemma it suffices to show that the complexes
$$
\ldots \to \mathcal{G}^m(U_1) \to \mathcal{G}^m(U_0) \to
 \pi_!\mathcal{G}^m \to 0
$$
are exact, see Homology, Lemma
\ref{homology-lemma-double-complex-gives-resolution}.
Since the sheaves $\mathcal{G}^m$ are direct sums of the sheaves
$j_{U!}\mathbf{Z}_U$ we reduce to $\mathcal{G} = j_{U!}\mathbf{Z}_U$.
The complex $j_{U!}\mathbf{Z}_U(U_\bullet)$
is the complex of abelian groups associated to the free
$\mathbf{Z}$-module on the simplicial set
$\Mor_\mathcal{C}(U_\bullet, U)$ which we assumed to be homotopy
equivalent to a singleton. We conclude that
$$
j_{U!}\mathbf{Z}_U(U_\bullet) \to \mathbf{Z}
$$
is a homotopy equivalence of abelian groups hence a quasi-isomorphism
(Simplicial, Remark \ref{simplicial-remark-homotopy-better} and
Lemma \ref{simplicial-lemma-homotopy-s-N}). This finishes the proof
since $\pi_!j_{U!}\mathbf{Z}_U = \mathbf{Z}$
as was shown in the proof of
Lemma \ref{lemma-properties-lower-shriek-fibred-category}.
\end{proof}

\begin{lemma}
\label{lemma-get-it-now}
Notation and assumptions as in Example \ref{example-morphism-categories}.
If there exists a cosimplicial object $U'_\bullet$ of $\mathcal{C}'$
such that Lemma \ref{lemma-compute-by-cosimplicial-resolution}
applies to both $U'_\bullet$ in $\mathcal{C}'$
and $u(U'_\bullet)$ in $\mathcal{C}$, then we have
$L\pi'_! \circ g^{-1} = L\pi_!$ as functors
$D(\mathcal{C}) \to D(\textit{Ab})$,
resp.\ $D(\mathcal{C}, \underline{B}) \to D(B)$.
\end{lemma}

\begin{proof}
Follows immediately from
Lemma \ref{lemma-compute-by-cosimplicial-resolution}
and the fact that $g^{-1}$ is given by precomposing with $u$.
\end{proof}



\section{The cotangent complex of a ring map}
\label{section-cotangent-ring-map}

\noindent
Let $A$ be a ring. Let $\textit{Alg}_A$ be the category of $A$-algebras.
Consider the pair of adjoint functors $(F, i)$ where
$i : \textit{Alg}_A \to \textit{Sets}$ is the forgetful functor and
$F : \textit{Sets} \to \textit{Alg}_A$ assigns to a set $E$ the polynomial
algebra $A[E]$ on $E$ over $A$. Let $X_\bullet$ be the simplicial object of
$\text{Fun}(\textit{Alg}_A, \textit{Alg}_A)$ constructed in
Simplicial, Section \ref{simplicial-section-standard}.

\medskip\noindent
Consider an $A$-algebra $B$. Denote $P_\bullet = X_\bullet(B)$ the resulting
simplicial $A$-algebra. Recall that $P_0 = A[B]$, $P_1 = A[A[B]]$, and so on.
In particular each term $P_n$ is a polynomial $A$-algebra.
Recall also that there is an augmentation
$$
\epsilon : P_\bullet \longrightarrow B
$$
where we view $B$ as a constant simplicial $A$-algebra.

\begin{definition}
\label{definition-standard-resolution}
Let $A \to B$ be a ring map. The {\it standard resolution of $B$ over $A$}
is the augmentation $\epsilon : P_\bullet \to A$ with terms
$$
P_0 = A[B], P_1 = A[A[B]], \ldots
$$
and maps as constructed above.
\end{definition}

\noindent
It will turn out that we can use the standard resolution
to compute left derived functors in certain settings.

\begin{definition}
\label{definition-cotangent-complex-ring-map}
The {\it cotangent complex} $L_{B/A}$ of a ring map $A \to B$
is the complex of $B$-modules associated to the simplicial $B$-module
$$
\Omega_{P_\bullet/A} \otimes_{P_\bullet, \epsilon} B
$$
where $\epsilon : P_\bullet \to B$ is the standard resolution
of $B$ over $A$.
\end{definition}

\noindent
In Simplicial, Section \ref{simplicial-section-complexes} we associate a
chain complex to a simplicial module, but here we work with cochain complexes.
Thus the term $L_{B/A}^{-n}$ in degree $-n$ is the $B$-module
$\Omega_{P_n/A} \otimes_{P_n, \epsilon_n} B$ and $L_{B/A}^m = 0$
for $m > 0$.

\begin{remark}
\label{remark-variant-cotangent-complex}
Let $A \to B$ be a ring map. Let $\mathcal{A}$ be the category of
arrows $\psi : C \to B$ of $A$-algebras and let $\mathcal{S}$ be
the category of maps $E \to B$ where $E$ is a set. There are adjoint
functors $i : \mathcal{A} \to \mathcal{S}$ (the forgetful functor)
and $F : \mathcal{S} \to \mathcal{A}$ which sends $E \to B$ to
$A[E] \to B$. Let $X_\bullet$ be the simplicial object of
$\text{Fun}(\mathcal{A}, \mathcal{A})$ constructed in
Simplicial, Section \ref{simplicial-section-standard}.
The diagram
$$
\xymatrix{
\mathcal{A} \ar[d] \ar[r] & \mathcal{S} \ar@<1ex>[l] \ar[d] \\
\textit{Alg}_A \ar[r] & \textit{Sets} \ar@<1ex>[l]
}
$$
commutes. It follows that $X_\bullet(\text{id}_B : B \to B)$
is equal to the standard resolution of $B$ over $A$.
\end{remark}





\section{Simplicial resolutions and derived lower shriek}
\label{section-compute-L-pi-shriek}

\noindent
Let $A \to B$ be a ring map. Consider the category of $A$-algebra maps
$\alpha : P \to B$ where $P$ is a polynomial algebra over $A$
(in some set\footnote{It suffices to consider sets of cardinality
at most the cardinality of $B$.} of variables).
Let $\mathcal{C} = \mathcal{C}_{B/A}$ denote the {\bf opposite}
of this category. The reason for
taking the opposite is that we want to think of objects
$(P, \alpha)$ as corresponding to the diagram of affine schemes
$$
\xymatrix{
\Spec(B) \ar[d] \ar[r] & \Spec(P) \ar[ld] \\
\Spec(A)
}
$$
We endow $\mathcal{C}$ with the chaotic topology
(Sites, Example \ref{sites-example-indiscrete}), i.e., we endow
$\mathcal{C}$ with the structure of a site where coverings are given by
identities so that all presheaves are sheaves.
Moreover, we endow $\mathcal{C}$ with two sheaves of rings. The first
is the sheaf $\mathcal{O}$ which sends to object $(P, \alpha)$ to $P$.
Then second is the constant sheaf $B$, which we will denote
$\underline{B}$. We obtain the following diagram of morphisms of
ringed topoi
\begin{equation}
\label{equation-pi}
\vcenter{
\xymatrix{
(\Sh(\mathcal{C}), \underline{B}) \ar[r]_i \ar[d]_\pi &
(\Sh(\mathcal{C}), \mathcal{O}) \\
(\Sh(*), B)
}
}
\end{equation}
The morphism $i$ is the identity on underlying topoi and
$i^\sharp : \mathcal{O} \to \underline{B}$ is the obvious surjective map.
The map $\pi$ is as in Example \ref{example-category-to-point}.
An important role will be played in the following
by the derived functors
$$
Li^* : D(\mathcal{O}) \longrightarrow D(\underline{B})
$$
left adjoint to $Ri_* = i_* : D(\underline{B}) \to D(\mathcal{O})$ and
$$
L\pi_! : D(\underline{B}) \longrightarrow D(B)
$$
left adjoint to $\pi^* = \pi^{-1} : D(B) \to D(\underline{B})$.

\begin{lemma}
\label{lemma-identify-pi-shriek}
With notation as above let $P_\bullet$ be a simplicial $A$-algebra
endowed with an augmentation $\epsilon : P_\bullet \to B$.
Assume each $P_n$ is a polynomial algebra over $A$ and $\epsilon$
is a trivial Kan fibration on underlying simplicial sets. Then
$$
L\pi_!\mathcal{F} = \mathcal{F}(P_\bullet, \epsilon)
$$
in $D(\textit{Ab})$, resp.\ $D(B)$ functorially in $\mathcal{F}$ in
$\textit{Ab}(\mathcal{C})$, resp.\ $\textit{Mod}(\underline{B})$.
\end{lemma}

\begin{proof}
We will use the criterion of
Lemma \ref{lemma-compute-by-cosimplicial-resolution} to prove this.
Given an object $U = (Q, \beta)$ of $\mathcal{C}$ we have to show that
$$
S_\bullet = \Mor_\mathcal{C}((Q, \beta), (P_\bullet, \epsilon))
$$
is homotopy equivalent to a singleton.
Write $Q = A[E]$ for some set $E$ (this is possible by our choice of
the category $\mathcal{C}$). We see that
$$
S_\bullet = \Mor_{\textit{Sets}}((E, \beta|_E), (P_\bullet, \epsilon))
$$
Let $*$ be the constant simplicial set on a singleton. For $b \in B$
let $F_{b, \bullet}$ be the simplicial set defined by the cartesian
diagram
$$
\xymatrix{
F_{b, \bullet} \ar[r] \ar[d] & P_\bullet \ar[d]_\epsilon \\
{*} \ar[r]^b & B
}
$$
With this notation $S_\bullet = \prod_{e \in E} F_{\beta(e), \bullet}$.
Since we assumed $\epsilon$ is a trival Kan fibration we see that
$F_{b, \bullet} \to *$ is a trivial Kan fibration
(Simplicial, Lemma \ref{simplicial-lemma-trivial-kan-base-change}).
Thus $S_\bullet \to *$ is a trivial Kan fibration
(Simplicial, Lemma \ref{simplicial-lemma-product-trivial-kan}).
Therefore $S_\bullet$ is homotopy equivalent to $*$
(Simplicial, Lemma \ref{simplicial-lemma-trivial-kan-homotopy}).
\end{proof}

\noindent
In particular, we can use the standard resolution of $B$ over $A$
to compute derived lower shriek.

\begin{lemma}
\label{lemma-pi-shriek-standard}
Let $A \to B$ be a ring map. Let $\epsilon : P_\bullet \to B$
be the standard resolution of $B$ over $A$. Let $\pi$ be as in
(\ref{equation-pi}). Then
$$
L\pi_!\mathcal{F} = \mathcal{F}(P_\bullet, \epsilon)
$$
in $D(\textit{Ab})$, resp.\ $D(B)$ functorially in $\mathcal{F}$ in
$\textit{Ab}(\mathcal{C})$, resp.\ $\textit{Mod}(\underline{B})$.
\end{lemma}

\begin{proof}[First proof]
We will apply Lemma \ref{lemma-identify-pi-shriek}.
Since the terms $P_n$ are polynomial algebras we see the first
assumption of that lemma is satisfied. We give two proofs of the second
assumption. By
Simplicial, Lemma \ref{simplicial-lemma-standard-simplicial-homotopy}
the map $\epsilon$ is a homotopy equivalence of underlying
simplicial sets. By
Simplicial, Lemma \ref{simplicial-lemma-homotopy-equivalence}
this implies $\epsilon$ induces a quasi-isomorphism of associated
complexes of abelian groups. By
Simplicial, Lemma \ref{simplicial-lemma-qis-simplicial-abelian-groups}
this implies that $\epsilon$ is a trivial Kan fibration of underlying
simplicial sets.
\end{proof}

\begin{proof}[Second proof]
We will use the criterion of
Lemma \ref{lemma-compute-by-cosimplicial-resolution}.
Let $U = (Q, \beta)$ be an object of $\mathcal{C}$.
We have to show that
$$
S_\bullet = \Mor_\mathcal{C}((Q, \beta), (P_\bullet, \epsilon))
$$
is homotopy equivalent to a singleton. Write $Q = A[E]$ for some set $E$
(this is possible by our choice of the category $\mathcal{C}$). Using the
notation of Remark \ref{remark-variant-cotangent-complex} we see that
$$
S_\bullet = \Mor_\mathcal{S}((E \to B), i(P_\bullet \to B))
$$
By Simplicial, Lemma \ref{simplicial-lemma-standard-simplicial-homotopy}
the map $i(P_\bullet \to B) \to i(B \to B)$ is a homotopy equivalence
in $\mathcal{S}$. Hence $S_\bullet$ is homotopy equivalent to
$$
\Mor_\mathcal{S}((E \to B), (B \to B)) = \{*\}
$$
as desired.
\end{proof}

\begin{lemma}
\label{lemma-compute-cotangent-complex}
Let $A \to B$ be a ring map.  Let $\pi$ and $i$ be as in (\ref{equation-pi}).
There is a canonical isomorphism
$$
L_{B/A} = L\pi_!(Li^*\Omega_{\mathcal{O}/A}) =
L\pi_!(i^*\Omega_{\mathcal{O}/A}) =
L\pi_!(\Omega_{\mathcal{O}/A} \otimes_\mathcal{O} \underline{B})
$$
in $D(B)$.
\end{lemma}

\begin{proof}
For an object $\alpha : P \to B$ of the category $\mathcal{C}$
the module $\Omega_{P/A}$ is a free $P$-module. Thus
$\Omega_{\mathcal{O}/A}$ is a flat $\mathcal{O}$-module. Hence
$Li^*\Omega_{\mathcal{O}/A} = i^*\Omega_{\mathcal{O}/A}$ is the sheaf
of $\underline{B}$-modules which associates to $\alpha : P \to A$ the
$B$-module $\Omega_{P/A} \otimes_{P, \alpha} B$.
By Lemma \ref{lemma-pi-shriek-standard}
we see that the right hand side is computed by
the value of this sheaf on the standard resolution which is our
definition of the left hand side
(Definition \ref{definition-cotangent-complex-ring-map}).
\end{proof}

\begin{lemma}
\label{lemma-pi-lower-shriek-constant-sheaf}
If $A \to B$ is a ring map, then $L\pi_!(\pi^{-1}M) = M$
with $\pi$ as in (\ref{equation-pi}).
\end{lemma}

\begin{proof}
This follows from Lemma \ref{lemma-identify-pi-shriek} which tells us
$L\pi_!(\pi^{-1}M)$ is computed by $(\pi^{-1}M)(P_\bullet, \epsilon)$
which is the constant simplicial object on $M$.
\end{proof}

\begin{lemma}
\label{lemma-identify-H0}
If $A \to B$ is a ring map, then $H^0(L_{B/A}) = \Omega_{B/A}$.
\end{lemma}

\begin{proof}
We will prove this by a direct calculation.
We will use the identification of Lemma \ref{lemma-compute-cotangent-complex}.
There is clearly a map from $\Omega_{\mathcal{O}/A} \otimes \underline{B}$
to the constant sheaf with value $\Omega_{B/A}$. Thus this map induces
a map
$$
H^0(L_{B/A}) = H^0(L\pi_!(\Omega_{\mathcal{O}/A} \otimes \underline{B}))
= \pi_!(\Omega_{\mathcal{O}/A} \otimes \underline{B}) \to \Omega_{B/A}
$$
By choosing an object $P \to B$ of $\mathcal{C}_{B/A}$ with $P \to B$
surjective we see that this map is surjective (by
Algebra, Lemma \ref{algebra-lemma-differential-surjective}).
To show that it is injective, suppose that $P \to B$ is an object
of $\mathcal{C}_{B/A}$ and that $\xi \in \Omega_{P/A} \otimes_P B$
is an element which maps to zero in $\Omega_{B/A}$.
We first choose factorization $P \to P' \to B$ such that $P' \to B$
is surjective and $P'$ is a polynomial algebra over $A$.
We may replace $P$ by $P'$. If $B = P/I$, then the kernel
$\Omega_{P/A} \otimes_P B \to \Omega_{B/A}$ is the image of
$I/I^2$ (Algebra, Lemma \ref{algebra-lemma-differential-seq}).
Say $\xi$ is the image of $f \in I$.
Then we consider the two maps $a, b : P' = P[x] \to P$, the first of which
maps $x$ to $0$ and the second of which maps $x$ to $f$ (in both
cases $P[x] \to B$ maps $x$ to zero). We see that $\xi$ and $0$
are the image of $\text{d}x \otimes 1$ in $\Omega_{P'/A} \otimes_{P'} B$.
Thus $\xi$ and $0$ have the same image in the colimit (see
Example \ref{example-category-to-point})
$\pi_!(\Omega_{\mathcal{O}/A} \otimes \underline{B})$ as desired.
\end{proof}

\begin{lemma}
\label{lemma-pi-lower-shriek-polynomial-algebra}
If $B$ is a polynomial algebra over the ring $A$, then
with $\pi$ as in (\ref{equation-pi}) we have that
$\pi_!$ is exact and $\pi_!\mathcal{F} = \mathcal{F}(B \to B)$.
\end{lemma}

\begin{proof}
This follows from Lemma \ref{lemma-identify-pi-shriek} which tells us
the constant simplicial algebra on $B$ can be used to compute $L\pi_!$.
\end{proof}

\begin{lemma}
\label{lemma-cotangent-complex-polynomial-algebra}
If $B$ is a polynomial algebra over the ring $A$, then
$L_{B/A}$ is quasi-isomorphic to $\Omega_{B/A}[0]$.
\end{lemma}

\begin{proof}
Immediate from
Lemma \ref{lemma-compute-cotangent-complex} and
\ref{lemma-pi-lower-shriek-polynomial-algebra}.
\end{proof}





\section{Constructing a resolution}
\label{section-polynomial}

\noindent
In the Noetherian finite type case we can construct a ``small'' simplicial
resolution for finite type ring maps.

\begin{lemma}
\label{lemma-polynomial}
Let $A$ be a Noetherian ring. Let $A \to B$ be a finite type ring map.
Let $\mathcal{A}$ be the category of $A$-algebra maps $C \to B$. Let
$n \geq 0$ and let $P_\bullet$ be a simplical object of $\mathcal{A}$
such that
\begin{enumerate}
\item $P_\bullet \to B$ is a trivial Kan fibration of simplicial sets,
\item $P_k$ is finite type over $A$ for $k \leq n$,
\item $P_\bullet = \text{cosk}_n \text{sk}_n P_\bullet$ as simplicial
objects of $\mathcal{A}$.
\end{enumerate}
Then $P_{n + 1}$ is a finite type $A$-algebra.
\end{lemma}

\begin{proof}
Although the proof we give of this lemma is straightforward, it is a bit
messy. To clarify the idea we explain what happens for low $n$ before giving
the proof in general. For example, if $n = 0$, then (3) means that
$P_1 = P_0 \times_B P_0$. Since the ring map $P_0 \to B$ is surjective, this
is of finite type over $A$ by
More on Algebra, Lemma \ref{more-algebra-lemma-fibre-product-finite-type}.

\medskip\noindent
If $n = 1$, then (3) means that
$$
P_2 = \{(f_0, f_1, f_2) \in P_1^3 \mid
d_0f_0 = d_0f_1,\ d_1f_0 = d_0f_2,\ d_1f_1 = d_1f_2 \}
$$
where the equalities take place in $P_0$. Observe that the triple
$$
(d_0f_0, d_1f_0, d_1f_1) = (d_0f_1, d_0f_2, d_1f_2)
$$
is an element of the fibre product $P_0 \times_B P_0 \times_B P_0$ over $B$
because the maps $d_i : P_1 \to P_0$ are morphisms over $B$. Thus we get
a map
$$
\psi : P_2 \longrightarrow P_0 \times_B P_0 \times_B P_0
$$
The fibre of $\psi$ over an element
$(g_0, g_1, g_2) \in P_0 \times_B P_0 \times_B P_0$
is the set of triples $(f_0, f_1, f_2)$ of $1$-simplices
with $(d_0, d_1)(f_0) = (g_0, g_1)$, $(d_0, d_1)(f_1) = (g_0, g_2)$,
and $(d_0, d_1)(f_2) = (g_1, g_2)$. As $P_\bullet \to B$ is a trivial
Kan fibration the map $(d_0, d_1) : P_1 \to P_0 \times_B P_0$ is
surjective. Thus we see that $P_2$ fits into the cartesian diagram
$$
\xymatrix{
P_2 \ar[d] \ar[r] & P_1^3 \ar[d] \\
P_0 \times_B P_0 \times_B P_0 \ar[r] & (P_0 \times_B P_0)^3
}
$$
By More on Algebra, Lemma \ref{more-algebra-lemma-formal-consequence}
we conclude. The general case is similar, but requires a bit more notation.

\medskip\noindent
The case $n > 1$. By Simplicial, Lemma \ref{simplicial-lemma-cosk-above-object}
the condition $P_\bullet = \text{cosk}_n \text{sk}_n P_\bullet$
implies the same thing is true in the category of simplicial
$A$-algebras and hence in the category of sets (as the forgetful
functor from $A$-algebras to sets commutes with limits). Thus
$$
P_{n + 1} =
\Mor(\Delta[n + 1], P_\bullet) =
\Mor(\text{sk}_n \Delta[n + 1], \text{sk}_n P_\bullet)
$$
by Simplicial, Lemma \ref{simplicial-lemma-simplex-map} and
Equation (\ref{simplicial-equation-cosk}). We will prove by induction
on $1 \leq k < m \leq n + 1$ that the ring
$$
Q_{k, m} = \Mor(\text{sk}_k \Delta[m], \text{sk}_k P_\bullet)
$$
is of finite type over $A$. The case $k = 1$, $1 < m \leq n + 1$
is entirely similar to the discussion above in the case $n = 1$.
Namely, there is a cartesian diagram
$$
\xymatrix{
Q_{1, m} \ar[d] \ar[r] & P_1^N \ar[d] \\
P_0 \times_B \ldots \times_B P_0 \ar[r] & (P_0 \times_B P_0)^N
}
$$
where $N = {m + 1 \choose 2}$. We conclude as before.

\medskip\noindent
Let $1 \leq k_0 \leq n$ and assume $Q_{k, m}$ is of finite type
over $A$ for all $1 \leq k \leq k_0$ and $k < m \leq n + 1$.
For $k_0 + 1 < m \leq n + 1$ we claim there is a cartesian square
$$
\xymatrix{
Q_{k_0 + 1, m} \ar[d] \ar[r] & P_{k_0 + 1}^N \ar[d] \\
Q_{k_0, m} \ar[r] & Q_{k_0, k_0 + 1}^N
}
$$
where $N$ is the number of nondegerate $(k_0 + 1)$-simplices
of $\Delta[m]$. Namely, to see this is true, think of an element of
$Q_{k_0 + 1, m}$ as a function $f$ from the $(k_0 + 1)$-skeleton
of $\Delta[m]$ to $P_\bullet$. We can restrict $f$ to the $k_0$-skeleton
which gives the left vertical map of the diagram. We can also restrict
to each nondegenerate $(k_0 + 1)$-simplex which gives the top horizontal
arrow. Moreover, to give such an $f$ is the same thing as giving its
restriction to $k_0$-skeleton and to each nondegenerate
$(k_0 + 1)$-face, provided these agree on the overlap, and this
is exactly the content of the diagram. Moreover, the fact that
$P_\bullet \to B$ is a trivial Kan fibration implies that
the map
$$
P_{k_0} \to Q_{k_0, k_0 + 1} = \Mor(\partial \Delta[k_0 + 1], P_\bullet)
$$
is surjective as every map $\partial \Delta[k_0 + 1] \to B$ can be extended
to $\Delta[k_0 + 1] \to B$ for $k_0 \geq 1$ (small argument about constant
simplicial sets omitted). Since by induction hypothesis the rings
$Q_{k_0, m}$, $Q_{k_0, k_0 + 1}$ are finite type $A$-algebras, so is
$Q_{k_0 + 1, m}$ by
More on Algebra, Lemma \ref{more-algebra-lemma-formal-consequence}
once more.
\end{proof}

\begin{proposition}
\label{proposition-polynomial}
Let $A$ be a Noetherian ring. Let $A \to B$ be a finite type ring map.
There exists a simplicial $A$-algebra $P_\bullet$ with an augmentation
$\epsilon : P_\bullet \to B$ such that each $P_n$ is a polynomial algebra
of finite type over $A$ and such that $\epsilon$ is a trivial
Kan fibration of simplicial sets.
\end{proposition}

\begin{proof}
Let $\mathcal{A}$ be the category of $A$-algebra maps $C \to B$.
In this proof our simplicial objects and skelet and coskelet
functors will be taken in this category.

\medskip\noindent
Choose a polynomial algebra $P_0$ of finite type over $A$ and a surjection
$P_0 \to B$. As a first approximation we take
$P_\bullet = \text{cosk}_0(P_0)$. In other words, $P_\bullet$ is the simplicial
$A$-algebra with terms $P_n = P_0 \times_A \ldots \times_A P_0$.
(In the final paragraph of the proof this simplicial object will
be denoted $P^0_\bullet$.) By
Simplicial, Lemma \ref{simplicial-lemma-cosk-minus-one-equivalence}
the map $P_\bullet \to B$ is a trivial Kan fibration of simplicial sets.
Also, observe that $P_\bullet = \text{cosk}_0 \text{sk}_0 P_\bullet$.

\medskip\noindent
Suppose for some $n \geq 0$ we have constructed $P_\bullet$
(in the final paragraph of the proof this will be $P^n_\bullet$)
such that
\begin{enumerate}
\item[(a)] $P_\bullet \to B$ is a trivial Kan fibration of simplicial sets,
\item[(b)] $P_k$ is a finitely generated polynomial algebra for
$0 \leq k \leq n$, and
\item[(c)] $P_\bullet = \text{cosk}_n \text{sk}_n P_\bullet$
\end{enumerate}
By Lemma \ref{lemma-polynomial}
we can find a finitely generated polynomial algebra $Q$ over $A$
and a surjection $Q \to P_{n + 1}$. Since $P_n$ is a polynomial algebra
the $A$-algebra maps $s_i : P_n \to P_{n + 1}$ lift to maps
$s'_i : P_n \to Q$. Set $d'_j : Q \to P_n$ equal to the composition of
$Q \to P_{n + 1}$ and $d_j : P_{n + 1} \to P_n$.
We obtain a truncated simplicial object $P'_\bullet$ of $\mathcal{A}$
by setting $P'_k = P_k$ for $k \leq n$ and $P'_{n + 1} = Q$ and morphisms
$d'_i = d_i$ and $s'_i = s_i$ in degrees $k \leq n - 1$ and using the
morphisms $d'_j$ and $s'_i$ in degree $n$. Extend this to a full simplicial
object $P'_\bullet$ of $\mathcal{A}$ using $\text{cosk}_{n + 1}$. By
functoriality of the coskelet functors there is a morphism
$P'_\bullet \to P_\bullet$ of simplical objects extending the
given morphism of $(n + 1)$-truncated simplicial objects.
(This morphism will be denoted $P^{n + 1}_\bullet \to P^n_\bullet$
in the final paragraph of the proof.)

\medskip\noindent
Note that conditions (b) and (c) are satisfied for $P'_\bullet$ with $n$
replaced by $n + 1$. We claim the map $P'_\bullet \to P_\bullet$ satisfies
assumptions (1), (2), (3), and (4) of
Simplicial, Lemmas \ref{simplicial-lemma-section}
with $n + 1$ instead of $n$. Conditions (1) and (2) hold by construction.
By Simplicial, Lemma \ref{simplicial-lemma-cosk-above-object}
we see that we have
$P_\bullet = \text{cosk}_{n + 1}\text{sk}_{n + 1}P_\bullet$
and
$P'_\bullet = \text{cosk}_{n + 1}\text{sk}_{n + 1}P'_\bullet$
not only in $\mathcal{A}$ but also in the category of $A$-algebras,
whence in the category of sets (as the forgetful functor from $A$-algebras
to sets commutes with all limits). This proves (3) and (4). Thus the lemma
applies and $P'_\bullet \to P_\bullet$ is a trivial Kan fibration. By
Simplicial, Lemma \ref{simplicial-lemma-trivial-kan-composition}
we conclude that $P'_\bullet \to B$ is a trivial Kan fibration and (a)
holds as well.

\medskip\noindent
To finish the proof we take the inverse limit $P_\bullet = \lim P^n_\bullet$
of the sequence of simplicial algebras
$$
\ldots \to P^2_\bullet \to P^1_\bullet \to P^0_\bullet
$$
constructed above. The map $P_\bullet \to B$ is a trivial Kan fibration by
Simplicial, Lemma \ref{simplicial-lemma-limit-trivial-kan}.
However, the construction above stabilizes in each degree
to a fixed finitely generated polynomial algebra as desired.
\end{proof}

\begin{lemma}
\label{lemma-pi-shriek-finite}
Let $A$ be a Noetherian ring. Let $A \to B$ be a finite type ring map.
Let $\pi$, $\underline{B}$ be as in (\ref{equation-pi}).
If $\mathcal{F}$ is an $\underline{B}$-module such that
$\mathcal{F}(P, \alpha)$ is a finite $B$-module for all
$\alpha : P = A[x_1, \ldots, x_n] \to B$, then the cohomology modules
of $L\pi_!\mathcal{F}$ are finite $B$-modules.
\end{lemma}

\begin{proof}
By Lemma \ref{lemma-identify-pi-shriek} and
Proposition \ref{proposition-polynomial}
we can compute $L\pi_!\mathcal{F}$ by a complex
constructed out of the values of $\mathcal{F}$ on finite type
polynomial algebras.
\end{proof}

\begin{lemma}
\label{lemma-cotangent-finite}
Let $A$ be a Noetherian ring. Let $A \to B$ be a finite type ring map.
Then $H^n(L_{B/A})$ is a finite $B$-module for all $n \in \mathbf{Z}$.
\end{lemma}

\begin{proof}
Apply Lemmas \ref{lemma-compute-cotangent-complex} and
\ref{lemma-pi-shriek-finite}.
\end{proof}

\begin{remark}
\label{remark-resolution}
Let $A \to B$ be any ring map. The proof of
Proposition \ref{proposition-polynomial}
shows that there exists a simplicial $A$-algebra $P_\bullet$
with an augmentation $\epsilon : P_\bullet \to B$ such that each $P_n$
is a polynomial algebra (on some set) over $A$ such that $\epsilon$
is a trivial Kan fibration of simplicial sets.
Let us call such a pair $(P_\bullet, \epsilon)$ a
{\it resolution} of $B$ over $A$. Of course, we have seen in
the proof of Lemma \ref{lemma-pi-shriek-standard}
that the standard resolution of $B$ over $A$ is a resolution
(so that this terminology doesn't lead to a conflict).
But the point of this remark is that we can prove the existence
of resolutions without appealing to the computations in
Simpicial, Section \ref{simplicial-section-standard}.
\end{remark}

\begin{lemma}
\label{lemma-O-homology-B-homology}
Let $A \to B$ be a ring map. Let $\pi$, $\mathcal{O}$, $\underline{B}$
be as in (\ref{equation-pi}). For any $\mathcal{O}$-module $\mathcal{F}$
we have
$$
L\pi_!\mathcal{F} = L\pi_!(Li^*\mathcal{F}) =
L\pi_!(\mathcal{F} \otimes_\mathcal{O}^\mathbf{L} \underline{B})
$$
in $D(\textit{Ab})$.
\end{lemma}

\begin{proof}
We will prove $L\pi_!K = L\pi_!(Li^*K)$ for all objects $K$ of
$D(\mathcal{O})$. By our construction of $L\pi_!$
(see Lemma \ref{lemma-properties-lower-shriek-fibred-category})
it suffices to prove the lemma for $j_{U!}\mathcal{O}_U$ where
$U \in \Ob(\mathcal{C}_{B/A})$. Choose a resolution
$\epsilon : P_\bullet \to B$ of $B$ over $A$ (see
Remark \ref{remark-resolution}). Say $U = (Q, \beta)$. Let
$$
S_\bullet = \Mor((Q, \beta), (P_\bullet, \epsilon))
$$
which is a simplicial set homotopy equivalent to the constant simplicial
set on a singleton (we saw this in the proof of
Lemma \ref{lemma-identify-pi-shriek}).
Observe that the complex associated to the simplicial abelian group
$$
X_\bullet : n \longmapsto \bigoplus\nolimits_{s \in S_n} P_n
$$
computes $L\pi_!(j_{U!}\mathcal{O}_U)$. Since $j_{U!}\mathcal{O}_U$ is
a flat $\mathcal{O}$-module we have
$j_{U!}\mathcal{O}_U \otimes^\mathbf{L}_\mathcal{O} \underline{B} =
j_{U!}\underline{B}_U$ and $L\pi_!$ of this is computes by the complex
associated to the simplicial abelian group
$$
Y_\bullet : n \longmapsto \bigoplus\nolimits_{s \in S_n} B
$$
As the rule which to a simplicial set $T_\bullet$ associated the simplicial
abelian group with terms $\bigoplus_{t \in T_n} P_n$ is a functor, we see
that $X_\bullet \to P_\bullet$ is a homotopy equivalence of simplicial
abelian groups. Similarly, the rule which to a simplicial set
$T_\bullet$ associates the simplicial abelian group with terms
$\bigoplus_{t \in T_n} B$ is a functor. Hence $Y_\bullet \to B$
is a homotopy equivalence of simplicial abelian groups.
Since $P_\bullet \to B$ induces a quasi-isomorphism on associated
complexes of abelian groups (for example by
Simplical, Lemmas \ref{simplicial-lemma-trivial-kan-homotopy} and
\ref{simplicial-lemma-homotopy-equivalence})
we conclude that $X_\bullet$ and $Y_\bullet$ are quasi-isomorphic as desired.
\end{proof}

\begin{lemma}
\label{lemma-apply-O-B-comparison}
Let $A \to B$ be a ring map. Let $\pi$, $\mathcal{O}$, $\underline{B}$
be as in (\ref{equation-pi}). We have
$$
L\pi_!\mathcal{O} = L\pi_!(\underline{B}) = B
\quad\text{and}\quad
L_{B/A} = L\pi_!(\Omega_{\mathcal{O}/A} \otimes_\mathcal{O} \underline{B}) =
L\pi_!(\Omega_{\mathcal{O}/A})
$$
in $D(\textit{Ab})$.
\end{lemma}

\begin{proof}
This is just an application of Lemma \ref{lemma-O-homology-B-homology}
(and the first equality of the sequence on the right is
Lemma \ref{lemma-compute-cotangent-complex}).
\end{proof}








\section{Functoriality}
\label{section-functoriality}

\noindent
In this section we consider a commutative square
\begin{equation}
\label{equation-commutative-square}
\vcenter{
\xymatrix{
B \ar[r] & B' \\
A \ar[u] \ar[r] & A' \ar[u]
}
}
\end{equation}
of ring maps. We claim there is a canonical $B$-linear map of complexes
$$
L_{B/A} \longrightarrow L_{B'/A'}
$$
associated to this diagram. Namely, if $P_\bullet \to B$ is the
standard resolution of $B$ over $A$ and $P'_\bullet \to B'$ is the
standard resoluton of $B'$ over $A'$, then there is a canonical map
$P_\bullet \to P'_\bullet$
of simplicial $A$-algebras compatible with the augmentations
$P_\bullet \to B$ and $P'_\bullet \to B'$. This can be seen in terms
of the construction of standard resolutions in
Simplicial, Section \ref{simplicial-section-standard}
but in the special case at hand it probably suffices to say simply
that the maps
$$
P_0 = A[B] \longrightarrow A'[B'] = P'_0,\quad
P_1 = A[A[B]] \longrightarrow A'[A'[B']] = P'_1,
$$
and so on are given by the given maps $A \to A'$ and $B \to B'$.
The desired map $L_{B/A} \to L_{B'/A'}$ then comes from the associated
maps $\Omega_{P_n/A} \to \Omega_{P'_n/A'}$.

\medskip\noindent
Another description of the functoriality map can be given as follows.
Let $\mathcal{C} = \mathcal{C}_{B/A}$ and $\mathcal{C}' = \mathcal{C}_{B'/A}'$
be the categories considered in Section \ref{section-compute-L-pi-shriek}.
There is a functor
$$
u : \mathcal{C} \longrightarrow \mathcal{C}',\quad
(P, \alpha) \longmapsto (P \otimes_A A', c \circ (\alpha \otimes 1))
$$
where $c : B \otimes_A A' \to B'$ is the obvious map. As discussed in
Example \ref{example-morphism-categories} we obtain a morphism of topoi
$g : \Sh(\mathcal{C}) \to \Sh(\mathcal{C}')$ and a commutative
diagram of maps of ringed topoi
\begin{equation}
\label{equation-double-square}
\vcenter{
\xymatrix{
(\Sh(\mathcal{C}), \underline{B}) \ar[d]_\pi &
(\Sh(\mathcal{C}'), \underline{B'}) \ar[d]_\pi \ar[l]^h \ar[r]_g &
(\Sh(\mathcal{C}), \underline{B'}) \ar[d]_{\pi'} \\
(\Sh(*), B) &
(\Sh(*), B') \ar[l]_f \ar[r] &
(\Sh(*), B')
}
}
\end{equation}
Here $h$ is the identity on underlying topoi and given by the ring map
$B \to B'$ on sheaves of rings. 
By Remark \ref{remark-morphism-fibred-categories}
given $\mathcal{F}$ on $\mathcal{C}$ and $\mathcal{F}'$ on $\mathcal{C}'$
and a transformation $t : \mathcal{F} \to g^{-1}\mathcal{F}'$
we obtain a canonical map $L\pi_!\mathcal{F} \to L\pi'_!\mathcal{F}'$.
If we apply this to the sheaves
$$
\mathcal{F} : (P, \alpha) \mapsto \Omega_{P/A} \otimes_P B,\quad
\mathcal{F}' : (P', \alpha') \mapsto \Omega_{P'/A'} \otimes_{P'} B',
$$
and the transformation $t$ given by the canonical maps
$$
\Omega_{P/A} \otimes_P B \longrightarrow
\Omega_{P \otimes_A A'/A'} \otimes_{P \otimes_A A'} B'
$$
to get a canonical map
$$
L\pi_!(\Omega_{\mathcal{O}/A} \otimes_\mathcal{O} \underline{B})
\longrightarrow
L\pi'_!(\Omega_{\mathcal{O}'/A'} \otimes_{\mathcal{O}'} \underline{B'})
$$
By Lemma \ref{lemma-compute-cotangent-complex} this gives
$L_{B/A} \to L_{B'/A'}$. We omit the verification that this map
agrees with the map defined above in terms of simplicial
resolutions.

\begin{lemma}
\label{lemma-flat-base-change}
Assume (\ref{equation-commutative-square}) induces a quasi-isomorphism
$B \otimes_A^\mathbf{L} A' = B'$. Then, with notation as in
(\ref{equation-double-square}) and
$\mathcal{F}' \in \textit{Ab}(\mathcal{C}')$,
we have $L\pi_!(g^{-1}\mathcal{F}') = L\pi'_!\mathcal{F}'$.
\end{lemma}

\begin{proof}
We will apply Lemma \ref{lemma-get-it-now}. Let $P_\bullet \to B$ be a
resolution, see Remark \ref{remark-resolution}. If we can show that
$u(P_\bullet) = P_\bullet \otimes_A A' \to B'$ is a quasi-isomorphism, then
we will be done by
Simplicial, Lemma \ref{simplicial-lemma-qis-simplicial-abelian-groups}
and Lemma \ref{lemma-identify-pi-shriek}.

\medskip\noindent
The complex of $A$-modules $s(P_\bullet)$ associated to $P_\bullet$
(viewed as a simplicial $A$-module) is a free $A$-module resolution of $B$.
Namely, $P_n$ is a free $A$-module and $s(P_\bullet) \to B$ is a
quasi-isomorphism (for example by
Simplical, Lemmas \ref{simplicial-lemma-trivial-kan-homotopy} and
\ref{simplicial-lemma-homotopy-equivalence}).
Thus $B \otimes_A^\mathbf{L} A'$ is computed by
$s(P_\bullet) \otimes_A A' = s(P_\bullet \otimes_A A')$.
Therefore the assumption of the lemma signifies that
$\epsilon' : P_\bullet \otimes_A A' \to B'$ is a quasi-isomorphism.
\end{proof}

\begin{lemma}
\label{lemma-flat-base-change-cotangent-complex}
If (\ref{equation-commutative-square}) induces a quasi-isomorphism
$B \otimes_A^\mathbf{L} A' = B'$, then the functoriality map
induces an isomorphism
$$
L_{B/A} \otimes_B^\mathbf{L} B' \longrightarrow L_{B'/A'}
$$
\end{lemma}

\begin{proof}
We will use the notation introduced in Equation (\ref{equation-double-square}).
We have
$$
L_{B/A} \otimes_B^\mathbf{L} B' =
L\pi_!(\Omega_{\mathcal{O}/A} \otimes_\mathcal{O} \underline{B})
\otimes_B^\mathbf{L} B' =
L\pi_! Lh^*(\Omega_{\mathcal{O}/A} \otimes_\mathcal{O} \underline{B})
$$
the first equality by Lemma \ref{lemma-compute-cotangent-complex}
and the second by Lemma \ref{lemma-change-of-rings}.
Since $\Omega_{\mathcal{O}/A}$ is a flat $\mathcal{O}$-module,
we see that $\Omega_{\mathcal{O}/A} \otimes_\mathcal{O} \underline{B}$
is a flat $\underline{B}$-module. Thus
$Lh^*(\Omega_{\mathcal{O}/A} \otimes_\mathcal{O} \underline{B}) =
\Omega_{\mathcal{O}/A} \otimes_\mathcal{O} \underline{B'}$
which is equal to
$g^{-1}(\Omega_{\mathcal{O}'/A'} \otimes_{\mathcal{O}'} \underline{B'})$
by inspection.
we conclude by Lemma \ref{lemma-flat-base-change}
and the fact that $L_{B'/A'}$ is computed by
$L\pi'_!(\Omega_{\mathcal{O}'/A'} \otimes_{\mathcal{O}'} \underline{B'})$.
\end{proof}




\section{The fundamental triangle}
\label{section-triangle}

\noindent
In this section we consider a sequence of ring maps $A \to B \to C$.
It is our goal to show that this triangle gives rise to a distinguished
triangle
\begin{equation}
\label{equation-triangle}
L_{B/A} \otimes_B^\mathbf{L} C \to L_{C/A} \to L_{C/B} \to
L_{B/A} \otimes_B^\mathbf{L} C[1]
\end{equation}
in $D(C)$. To prove this we consider the category $\mathcal{C}_{C/B/A}$
wich is the {\bf opposite} of the category whose objects are
$(P \to B, Q \to C)$ where
\begin{enumerate}
\item $P$ is a polynomial algebra over $A$,
\item $P \to B$ is an $A$-algebra homomorphism,
\item $Q$ is a polynomial algebra over $P$, and
\item $Q \to C$ is a $P$-algebra-homomorphism.
\end{enumerate}
We take the opposite as we want to think of $(P \to B, Q \to C)$
as corresponding to the commutative diagram
$$
\xymatrix{
\Spec(C) \ar[d] \ar[r] & \Spec(Q) \ar[d] \\
\Spec(B) \ar[d] \ar[r] & \Spec(P) \ar[dl] \\
\Spec(A)
}
$$
Let $\mathcal{C}_{B/A}$, $\mathcal{C}_{C/A}$, $\mathcal{C}_{C/B}$
be the categories considered in Section \ref{section-compute-L-pi-shriek}.
There are functors
$$
\begin{matrix}
u_1 : \mathcal{C}_{C/B/A} \to \mathcal{C}_{B/A}, &
(P \to B, Q \to C) \mapsto (P \to B) \\
u_2 : \mathcal{C}_{C/B/A} \to \mathcal{C}_{C/A}, &
(P \to B, Q \to C) \mapsto (Q \to C) \\
u_3 : \mathcal{C}_{C/B/A} \to \mathcal{C}_{C/B}, &
(P \to B, Q \to C) \mapsto (Q \otimes_P B \to C)
\end{matrix}
$$
These functors induce corresponding morphisms of topoi $g_i$.
Let us denote $\mathcal{O}_i = g_i^{-1}\mathcal{O}$ so that we
get morphisms of ringed topoi
\begin{equation}
\label{equation-three-maps}
\begin{matrix}
g_1 : (\Sh(\mathcal{C}_{C/B/A}), \mathcal{O}_1)
\longrightarrow (\Sh(\mathcal{C}_{B/A}), \mathcal{O}) \\
g_2 : (\Sh(\mathcal{C}_{C/B/A}), \mathcal{O}_2)
\longrightarrow (\Sh(\mathcal{C}_{C/A}), \mathcal{O}) \\
g_3 : (\Sh(\mathcal{C}_{C/B/A}), \mathcal{O}_3)
\longrightarrow (\Sh(\mathcal{C}_{C/B}), \mathcal{O})
\end{matrix}
\end{equation}
Let us denote
$\pi : \Sh(\mathcal{C}_{C/B/A}) \to \Sh(*)$,
$\pi_1 : \Sh(\mathcal{C}_{B/A}) \to \Sh(*)$,
$\pi_2 : \Sh(\mathcal{C}_{C/A}) \to \Sh(*)$, and
$\pi_3 : \Sh(\mathcal{C}_{C/B}) \to \Sh(*)$,
so that $\pi = \pi_i \circ g_i$.
We will obtain our distinguished triangle from the identification
of the cotangent complex in Lemma \ref{lemma-compute-cotangent-complex}
and the following lemmas.

\begin{lemma}
\label{lemma-triangle-ses}
With notation as in (\ref{equation-three-maps}) set
$$
\begin{matrix}
\Omega_1 = \Omega_{\mathcal{O}/A} \otimes_\mathcal{O} \underline{B}
\text{ on }\mathcal{C}_{B/A} \\
\Omega_2 = \Omega_{\mathcal{O}/A} \otimes_\mathcal{O} \underline{C}
\text{ on }\mathcal{C}_{C/A} \\
\Omega_3 = \Omega_{\mathcal{O}/B} \otimes_\mathcal{O} \underline{C}
\text{ on }\mathcal{C}_{C/B}
\end{matrix}
$$
Then we have a canonical short exact sequence of sheaves
of $\underline{C}$-modules
$$
0 \to g_1^{-1}\Omega_1 \otimes_{\underline{B}} \underline{C} \to
g_2^{-1}\Omega_2 \to
g_3^{-1}\Omega_3 \to 0
$$
on $\mathcal{C}_{C/B/A}$.
\end{lemma}

\begin{proof}
Recall that $g_i^{-1}$ is gotten by simply precomposing with $u_i$.
Given an object $U = (P \to B, Q \to C)$ we have a split
short exact sequence
$$
0 \to \Omega_{P/A} \otimes Q \to \Omega_{Q/A} \to \Omega_{Q/P} \to 0
$$
for example by Algebra, Lemma \ref{algebra-lemma-ses-formally-smooth}.
Tensoring with $C$ over $Q$ we obtain a short exact sequence
$$
0 \to \Omega_{P/A} \otimes C \to \Omega_{Q/A} \otimes C \to
\Omega_{Q/P} \otimes C \to 0
$$
We have $\Omega_{P/A} \otimes C = \Omega_{P/A} \otimes B \otimes C$
whence this is the value of
$g_1^{-1}\Omega_1 \otimes_{\underline{B}} \underline{C}$
on $U$. The module $\Omega_{Q/A} \otimes C$ is the value of
$g_2^{-1}\Omega_2$ on $U$.
We have $\Omega_{Q/P} \otimes C = \Omega_{Q \otimes_P B/B} \otimes C$
by Algebra, Lemma \ref{algebra-lemma-differentials-base-change}
hence this is the value of
$g_3^{-1}\Omega_3$ on $U$. Thus the short exact sequence of the
lemma comes from assigning to $U$ the last displayed short exact
sequence.
\end{proof}

\begin{lemma}
\label{lemma-polynomial-on-top}
With notation as in (\ref{equation-three-maps})
suppose that $C$ is a polynomial algebra over $B$. Then
$L\pi_!(g_3^{-1}\mathcal{F}) = L\pi_{3, !}\mathcal{F} = \pi_{3, !}\mathcal{F}$
for any abelian sheaf $\mathcal{F}$ on $\mathcal{C}_{C/B}$
\end{lemma}

\begin{proof}
Write $C = B[E]$ for some set $E$. Choose a resolution
$P_\bullet \to B$ of $B$ over $A$. For every $n$ consider
the object $U_n = (P_n \to B, P_n[E] \to C)$ of $\mathcal{C}_{C/B/A}$.
Then $U_\bullet$ is a cosimplicial object of $\mathcal{C}_{C/B/A}$.
Note that $u_3(U_\bullet)$ is the constant cosimplicial
object of $\mathcal{C}_{C/B}$ with value $(C \to C)$.
We will prove that the object $U_\bullet$ of $\mathcal{C}_{C/B/A}$
satisfies the hypotheses of
Lemma \ref{lemma-compute-by-cosimplicial-resolution}.
This implies the lemma as it shows that $L\pi_!(g_3^{-1}\mathcal{F})$
is computed by the constant simplicial abelian group
$\mathcal{F}(C \to C)$ which is the value of
$L\pi_{3, !}\mathcal{F} = \pi_{3, !}\mathcal{F}$ by
Lemma \ref{lemma-pi-lower-shriek-polynomial-algebra}.

\medskip\noindent
Let $U = (\beta : P \to B, \gamma : Q \to C)$ be an object of
$\mathcal{C}_{C/B/A}$. We may write $P = A[S]$ and $Q = A[S \amalg T]$
by the definition of our category $\mathcal{C}_{C/B/A}$. We have to show that
$$
\Mor_{\mathcal{C}_{C/B/A}}(U_\bullet, U)
$$
is homotopy equivalent to a singleton simplicial set. Observe that this
simplicial set is the product
$$
\prod\nolimits_{s \in S} F_s \times \prod\nolimits_{t \in T} F'_t
$$
where $F_s$ is the corresponding simplicial set for
$U_s = (A[\{s\}] \to B, A[\{s\}] \to C)$
and $F'_t$ is the corresponding simplicial set for
$U_t = (A \to B, A[\{t\}] \to C)$. (Namely, the object $U$
is the product $\prod U_s \times \prod U_t$ in $\mathcal{C}_{C/B/A}$.)
The case of $F_s$ follows as $P_\bullet \to B$ is a trivial Kan
fibration (as a resolution) and $F_s$ is the fibre of this map over
$\beta(s)$. (Use Simplicial, Lemmas
\ref{simplicial-lemma-trivial-kan-base-change} and
\ref{simplicial-lemma-trivial-kan-homotopy}).
The case of $F'_t$ is more interesting. Here we are saying that
the fibre of
$$
P_\bullet[E] \longrightarrow C = B[E]
$$
over $\gamma(t) \in C$ is homotopy equivalent to a point. In fact we
will show this map is a trivial Kan fibration. Namely,
$P_\bullet \to B$ is a trivial can fibration. For any ring $R$
we have
$$
R[E] =
\colim_{\Sigma \subset \text{Map}(E, \mathbf{Z}_{\geq 0})\text{ finite}}
\prod\nolimits_{I \in \Sigma} R
$$
(filtered colimit). Thus the displayed map of simplicial sets is a
filtered colimit of trivial Kan fibrations, whence a trivial Kan fibration
by Simplicial, Lemma \ref{simplicial-lemma-filtered-colimit-trivial-kan}.
\end{proof}

\begin{lemma}
\label{lemma-triangle-compute-lower-shriek}
With notation as in (\ref{equation-three-maps}) we have
$Lg_{i, !} \circ g_i^{-1} = \text{id}$ for $i = 1, 2, 3$
and hence also $L\pi_! \circ g_i^{-1} = L\pi_{i, !}$ for
$i = 1, 2, 3$.
\end{lemma}

\begin{proof}
Proof for $i = 1$. We claim the functor $\mathcal{C}_{C/B/A}$
is a fibred category over $\mathcal{C}_{B/A}$
Namely, suppose given $(P \to B, Q \to C)$
and a morphism $(P' \to B) \to (P \to B)$ of $\mathcal{C}_{B/A}$.
Recall that this means we have an $A$-algebra homomorphism
$P \to P'$ compatible with maps to $B$. Then we set $Q' = Q \otimes_P P'$
with induced map to $C$ and the morphism
$$
(P' \to B, Q' \to C) \longrightarrow (P \to B, Q \to C)
$$
in $\mathcal{C}_{C/B/A}$ (note reversal arrows again) is strongly cartesian
in $\mathcal{C}_{C/B/A}$ over $\mathcal{C}_{B/A}$. Moreover, observe
that the fibre category of $u_1$ over $P \to B$ is the category
$\mathcal{C}_{C/P}$. Let $\mathcal{F}$ be an abelian sheaf on
$\mathcal{C}_{B/A}$. Since we have a fibred category we may apply
Lemma \ref{lemma-compute-left-derived-pi-shriek}.
Thus $L_ng_{1, !}g_1^{-1}\mathcal{F}$ is the (pre)sheaf
which assigns to $U \in \Ob(\mathcal{C}_{B/A})$ the $L_n\colim$ of
$g_1^{-1}\mathcal{F}$ restricted to the fibre category over $U$.
Since these restrictions are constant the desired result follows from
Lemma \ref{lemma-pi-lower-shriek-constant-sheaf}
via our identifications of fibre categories above.

\medskip\noindent
The case $i = 2$.
We claim $\mathcal{C}_{C/B/A}$ is a fibred category over $\mathcal{C}_{C/A}$
is a fibred category. Namely, suppose given $(P \to B, Q \to C)$
and a morphism $(Q' \to C) \to (Q \to C)$ of $\mathcal{C}_{C/A}$.
Recall that this means we have a $B$-algebra homomorphism
$Q \to Q'$ compatible with maps to $C$. Then
$$
(P \to B, Q' \to C) \longrightarrow (P \to B, Q \to C)
$$
is strongly cartesian in $\mathcal{C}_{C/B/A}$ over $\mathcal{C}_{C/A}$.
Note that the fibre category of $u_2$ over $Q \to C$ has an final
(beware reversal arrows) object, namely, $(A \to B, Q \to C)$. Let
$\mathcal{F}$ be an abelian sheaf on $\mathcal{C}_{C/A}$.
Since we have a fibred category we may apply
Lemma \ref{lemma-compute-left-derived-pi-shriek}.
Thus $L_ng_{2, !}g_2^{-1}\mathcal{F}$ is the (pre)sheaf
which assigns to $U \in \Ob(\mathcal{C}_{C/A})$ the $L_n\colim$ of
$g_1^{-1}\mathcal{F}$ restricted to the fibre category over $U$.
Since these restrictions are constant the desired result follows from
Lemma \ref{lemma-initial-final}
because the fibre categories all have final objects.

\medskip\noindent
The case $i = 3$. In this case we will apply
Lemma \ref{lemma-compute-left-derived-g-shriek}
to $u = u_3 : \mathcal{C}_{C/B/A} \to \mathcal{C}_{C/B}$
and $\mathcal{F}' = g_3^{-1}\mathcal{F}$ for some abelian sheaf
$\mathcal{F}$ on $\mathcal{C}_{C/B}$.
Suppose $U = (\overline{Q} \to C)$ is an object of $\mathcal{C}_{C/B}$.
Then $\mathcal{I}_U = \mathcal{C}_{\overline{Q}/B/A}$ (again beware
of reversal of arrows). The sheaf $\mathcal{F}'_U$ is given by the
rule $(P \to B, Q \to \overline{Q}) \mapsto \mathcal{F}(Q \otimes_P B \to C)$.
In other words, this sheaf is the pullback of a sheaf
on $\mathcal{C}_{\overline{Q}/C}$ via the morphism
$\Sh(\mathcal{C}_{\overline{Q}/B/A}) \to \Sh(\mathcal{C}_{\overline{Q}/B})$.
Thus Lemma \ref{lemma-polynomial-on-top} shows that
$L_n\colim_{\mathcal{I}_U^{opp}} \mathcal{F}'_U = 0$ for $n > 0$
and equal to $\mathcal{F}(\overline{Q} \to C)$ for $n = 0$.
The aforementioned Lemma \ref{lemma-compute-left-derived-g-shriek}
implies that $Lg_{3, !}(g_3^{-1}\mathcal{F}) = \mathcal{F}$ and
the proof is done.
\end{proof}

\begin{proposition}
\label{proposition-triangle}
Let $A \to B \to C$ be ring maps. There is a canonical distinguished
triangle
$$
L_{B/A} \otimes_B^\mathbf{L} C \to L_{C/A} \to L_{C/B} \to
L_{B/A} \otimes_B^\mathbf{L} C[1]
$$
in $D(C)$.
\end{proposition}

\begin{proof}
Consider the short exact sequence of sheaves of
Lemma \ref{lemma-triangle-ses}
and apply the derived functor $L\pi_!$ to obtain a distinguished
triangle
$$
L\pi_!(g_1^{-1}\Omega_1 \otimes_{\underline{B}} \underline{C}) \to
L\pi_!(g_2^{-1}\Omega_2) \to
L\pi_!(g_3^{-1}\Omega_3) \to
L\pi_!(g_1^{-1}\Omega_1 \otimes_{\underline{B}} \underline{C})[1]
$$
in $D(C)$. Using Lemmas \ref{lemma-triangle-compute-lower-shriek} and
\ref{lemma-compute-cotangent-complex}
we see that the second and third terms agree with $L_{C/A}$ and $L_{C/B}$
and the first one equals
$$
L\pi_{1, !}(\Omega_1 \otimes_{\underline{B}} \underline{C}) =
L\pi_{1, !}(\Omega_1) \otimes_B^\mathbf{L} C =
L_{B/A} \otimes_B^\mathbf{L} C
$$
The first equality by Lemma \ref{lemma-change-of-rings}
(and flatness of $\Omega_1$ as a sheaf of modules over $\underline{B}$)
and the second by Lemma \ref{lemma-compute-cotangent-complex}.
\end{proof}





\section{Localization and \'etale ring maps}
\label{section-localization}

\noindent
In this section we study what happens if we localize our rings.
Let $A \to A' \to B$ be ring maps such that $B = B \otimes_A^\mathbf{L} A'$.
This happens for example if $A' = S^{-1}A$ is the localization of $A$
at a multiplicative subset $S \subset A$. In this
case for an abelian sheaf $\mathcal{F}$ on $\mathcal{C}_{B/A'}$
we have (loosely formulated)
$$
L\colim_{\mathcal{C}_{B/A}^{opp}} (g^{-1}\mathcal{F}) =
L\colim_{\mathcal{C}_{B/A'}^{opp}} \mathcal{F}
$$
in $D(\textit{Ab})$, see Lemma \ref{lemma-flat-base-change} for a precise
statement.

\begin{lemma}
\label{lemma-localize-at-bottom}
Let $A \to A' \to B$ be ring maps such that $B = B \otimes_A^\mathbf{L} A'$.
Then $L_{B/A} = L_{B/A'}$ in $D(B)$.
\end{lemma}

\begin{proof}
According to the discussion above (i.e., using
Lemma \ref{lemma-flat-base-change})
and Lemma \ref{lemma-compute-cotangent-complex}
we have to show that the sheaf given
by the rule $(P \to B) \mapsto \Omega_{P/A} \otimes_P B$ on $\mathcal{C}_{B/A}$
is the pullback of the sheaf given by the rule
$(P \to B) \mapsto \Omega_{P/A'} \otimes_P B$.
The pullback functor $g^{-1}$ is given by precomposing with the
functor $u : \mathcal{C}_{B/A} \to \mathcal{C}_{B/A'}$,
$(P \to B) \mapsto (P \otimes_A A' \to B)$.
Thus we have to show that
$$
\Omega_{P/A} \otimes_P B =
\Omega_{P \otimes_A A'/A'} \otimes_{(P \otimes_A A')} B
$$
By Algebra, Lemma \ref{algebra-lemma-differentials-base-change}
the right hand side is equal to
$$
(\Omega_{P/A} \otimes_A A') \otimes_{(P \otimes_A A')} B
$$
Since $P$ is a polynomial algebra over $A$ the module
$\Omega_{P/A}$ is free and the equality is obvious.
\end{proof}

\begin{lemma}
\label{lemma-derived-diagonal}
Let $A \to B$ be a ring map such that $B = B \otimes_A^\mathbf{L} B$.
Then $L_{B/A} = 0$ in $D(B)$.
\end{lemma}

\begin{proof}
This is true because $L_{B/A} = L_{B/B} = 0$ by
Lemmas \ref{lemma-localize-at-bottom} and
\ref{lemma-cotangent-complex-polynomial-algebra}.
\end{proof}

\begin{lemma}
\label{lemma-bootstrap}
Let $A \to B$ be a ring map such that $\text{Tor}^A_i(B, B) = 0$ for $i > 0$
and such that $L_{B/B \otimes_A B} = 0$.
Then $L_{B/A} = 0$ in $D(B)$.
\end{lemma}

\begin{proof}
By Lemma \ref{lemma-flat-base-change-cotangent-complex} we see that
$L_{B/A} \otimes_B^\mathbf{L} (B \otimes_A B) = L_{B \otimes_A B/B}$.
Now we use the distinguished triangle (\ref{equation-triangle})
$$
L_{B \otimes_A B/B} \otimes^\mathbf{L}_{(B \otimes_A B)} B \to
L_{B/B} \to L_{B/B \otimes_A B} \to
L_{B \otimes_A B/B} \otimes^\mathbf{L}_{(B \otimes_A B)} B[1]
$$
associated to the ring maps $B \to B \otimes_A B \to B$ and the vanishing of
$L_{B/B}$ (Lemma \ref{lemma-cotangent-complex-polynomial-algebra}) and
$L_{B/B \otimes_A B}$ (assumed) to see that
$$
0 =
L_{B \otimes_A B/B} \otimes^\mathbf{L}_{(B \otimes_A B)} B =
L_{B/A} \otimes_B^\mathbf{L} (B \otimes_A B)
\otimes^\mathbf{L}_{(B \otimes_A B)} B = L_{B/A}
$$
as desired.
\end{proof}

\begin{lemma}
\label{lemma-when-zero}
The cotangent complex $L_{B/A}$ is zero in each of the following cases:
\begin{enumerate}
\item $A \to B$ and $B \otimes_A B \to B$ are flat,
\item $A \to B$ is a flat epimorphism of rings,
\item $B = S^{-1}A$ for some multiplicative subset $S \subset A$,
\item $A \to B$ is unramified and flat,
\item $A \to B$ is \'etale, and
\item add more here.
\end{enumerate}
\end{lemma}

\begin{proof}
In case (1) we may apply
Lemma \ref{lemma-derived-diagonal}
to the surjective flat ring map $B \otimes_A B \to B$
to conclude that $L_{B \otimes_A B/B} = 0$ and then we use
Lemma \ref{lemma-bootstrap}
to conclude. The other cases are each special cases of (1).
\end{proof}

\noindent
The following lemma shows that $L_{S^{-1}B/A} = S^{-1}L_{B/A}$
in $D(S^{-1}B)$.

\begin{lemma}
\label{lemma-localize-on-top}
Let $A \to B \to C$ be ring maps such that $L_{C/B} = 0$.
Then $L_{C/A} = L_{B/A} \otimes_B^\mathbf{L} C$.
\end{lemma}

\begin{proof}
This is a trivial consequence of 
the distinguished triangle (\ref{equation-triangle}).
\end{proof}





\section{Smooth ring maps}
\label{section-smooth}

\noindent
Let $C \to B$ be a surjection of rings with kernel $I$. Let us call such
a ring map ``weakly quasi-regular'' if $I/I^2$ is a flat $B$-module and
$\text{Tor}_*^C(B, B)$ is the exterior algebra on $I/I^2$.
The generalization to ``smooth ring maps'' of what is done in
Lemma \ref{lemma-when-zero} for ``\'etale ring maps'' is to look
at flat ring maps $A \to B$ such that the multiplication map
$B \otimes_A B \to B$ is weakly quasi-regular. For the moment we just stick to
smooth ring maps.

\begin{lemma}
\label{lemma-when-projective}
If $A \to B$ is a smooth ring map, then $L_{B/A} = \Omega_{B/A}[0]$.
\end{lemma}

\begin{proof}
We have the agreement in cohomological degree $0$ by
Lemma \ref{lemma-identify-H0}.
Thus it suffices to prove the other cohomology groups
are zero. It suffices to prove this locally on $\Spec(B)$ as
$L_{B_g/A} = (L_{B/A})_g$ for $g \in B$ by Lemma \ref{lemma-localize-on-top}.
Thus we may assume that $A \to B$ is standard smooth
(Algebra, Lemma \ref{algebra-lemma-smooth-syntomic}), i.e.,
that we can factor $A \to B$ as
$A \to A[x_1, \ldots, x_n] \to B$ with $A[x_1, \ldots, x_n] \to B$
\'etale. In this case Lemmas \ref{lemma-when-zero} and
Lemma \ref{lemma-localize-on-top} show that
$L_{B/A} = L_{A[x_1, \ldots, x_n]/A} \otimes B$
whence the conclusion by
Lemma \ref{lemma-cotangent-complex-polynomial-algebra}.
\end{proof}



\section{Other chapters}

\begin{multicols}{2}
\begin{enumerate}
\item \hyperref[introduction-section-phantom]{Introduction}
\item \hyperref[conventions-section-phantom]{Conventions}
\item \hyperref[sets-section-phantom]{Set Theory}
\item \hyperref[categories-section-phantom]{Categories}
\item \hyperref[topology-section-phantom]{Topology}
\item \hyperref[sheaves-section-phantom]{Sheaves on Spaces}
\item \hyperref[algebra-section-phantom]{Commutative Algebra}
\item \hyperref[sites-section-phantom]{Sites and Sheaves}
\item \hyperref[homology-section-phantom]{Homological Algebra}
\item \hyperref[derived-section-phantom]{Derived Categories}
\item \hyperref[more-algebra-section-phantom]{More Algebra}
\item \hyperref[simplicial-section-phantom]{Simplicial Methods}
\item \hyperref[modules-section-phantom]{Sheaves of Modules}
\item \hyperref[sites-modules-section-phantom]{Modules on Sites}
\item \hyperref[injectives-section-phantom]{Injectives}
\item \hyperref[cohomology-section-phantom]{Cohomology of Sheaves}
\item \hyperref[sites-cohomology-section-phantom]{Cohomology on Sites}
\item \hyperref[hypercovering-section-phantom]{Hypercoverings}
\item \hyperref[schemes-section-phantom]{Schemes}
\item \hyperref[constructions-section-phantom]{Constructions of Schemes}
\item \hyperref[properties-section-phantom]{Properties of Schemes}
\item \hyperref[morphisms-section-phantom]{Morphisms of Schemes}
\item \hyperref[coherent-section-phantom]{Coherent Cohomology}
\item \hyperref[divisors-section-phantom]{Divisors}
\item \hyperref[limits-section-phantom]{Limits of Schemes}
\item \hyperref[varieties-section-phantom]{Varieties}
\item \hyperref[chow-section-phantom]{Chow Homology}
\item \hyperref[topologies-section-phantom]{Topologies on Schemes}
\item \hyperref[descent-section-phantom]{Descent}
\item \hyperref[more-morphisms-section-phantom]{More on Morphisms}
\item \hyperref[flat-section-phantom]{More on Flatness}
\item \hyperref[groupoids-section-phantom]{Groupoid Schemes}
\item \hyperref[more-groupoids-section-phantom]{More on Groupoid Schemes}
\item \hyperref[etale-section-phantom]{\'Etale Morphisms of Schemes}
\item \hyperref[etale-cohomology-section-phantom]{\'Etale Cohomology}
\item \hyperref[spaces-section-phantom]{Algebraic Spaces}
\item \hyperref[spaces-properties-section-phantom]{Properties of Algebraic Spaces}
\item \hyperref[spaces-morphisms-section-phantom]{Morphisms of Algebraic Spaces}
\item \hyperref[spaces-topologies-section-phantom]{Topologies on Algebraic Spaces}
\item \hyperref[spaces-descent-section-phantom]{Descent and Algebraic Spaces}
\item \hyperref[spaces-more-morphisms-section-phantom]{More on Morphisms of Spaces}
\item \hyperref[quot-section-phantom]{Quot and Hilbert Spaces}
\item \hyperref[stacks-section-phantom]{Stacks}
\item \hyperref[spaces-groupoids-section-phantom]{Groupoids in Algebraic Spaces}
\item \hyperref[spaces-more-groupoids-section-phantom]{More on Groupoids in Spaces}
\item \hyperref[bootstrap-section-phantom]{Bootstrap}
\item \hyperref[examples-stacks-section-phantom]{Examples of Stacks}
\item \hyperref[groupoids-quotients-section-phantom]{Quotients of Groupoids}
\item \hyperref[algebraic-section-phantom]{Algebraic Stacks}
\item \hyperref[criteria-section-phantom]{Criteria for Representability}
\item \hyperref[stacks-properties-section-phantom]{Properties of Algebraic Stacks}
\item \hyperref[stacks-morphisms-section-phantom]{Morphisms of Algebraic Stacks}
\item \hyperref[examples-section-phantom]{Examples}
\item \hyperref[exercises-section-phantom]{Exercises}
\item \hyperref[guide-section-phantom]{Guide to Literature}
\item \hyperref[desirables-section-phantom]{Desirables}
\item \hyperref[coding-section-phantom]{Coding Style}
\item \hyperref[fdl-section-phantom]{GNU Free Documentation License}
\item \hyperref[index-section-phantom]{Auto Generated Index}
\end{enumerate}
\end{multicols}


\bibliography{my}
\bibliographystyle{amsalpha}

\end{document}
