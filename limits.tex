\IfFileExists{stacks-project.cls}{%
\documentclass{stacks-project}
}{%
\documentclass{amsart}
}

% The following AMS packages are automatically loaded with
% the amsart documentclass:
%\usepackage{amsmath}
%\usepackage{amssymb}
%\usepackage{amsthm}

% For dealing with references we use the comment environment
\usepackage{verbatim}
\newenvironment{reference}{\comment}{\endcomment}
%\newenvironment{reference}{}{}
\newenvironment{slogan}{\comment}{\endcomment}
\newenvironment{history}{\comment}{\endcomment}

% For commutative diagrams you can use
% \usepackage{amscd}
\usepackage[all]{xy}

% We use 2cell for 2-commutative diagrams.
\xyoption{2cell}
\UseAllTwocells

% To put source file link in headers.
% Change "template.tex" to "this_filename.tex"
% \usepackage{fancyhdr}
% \pagestyle{fancy}
% \lhead{}
% \chead{}
% \rhead{Source file: \url{template.tex}}
% \lfoot{}
% \cfoot{\thepage}
% \rfoot{}
% \renewcommand{\headrulewidth}{0pt}
% \renewcommand{\footrulewidth}{0pt}
% \renewcommand{\headheight}{12pt}

\usepackage{multicol}

% For cross-file-references
\usepackage{xr-hyper}

% Package for hypertext links:
\usepackage{hyperref}

% For any local file, say "hello.tex" you want to link to please
% use \externaldocument[hello-]{hello}
\externaldocument[introduction-]{introduction}
\externaldocument[conventions-]{conventions}
\externaldocument[sets-]{sets}
\externaldocument[categories-]{categories}
\externaldocument[topology-]{topology}
\externaldocument[sheaves-]{sheaves}
\externaldocument[sites-]{sites}
\externaldocument[stacks-]{stacks}
\externaldocument[fields-]{fields}
\externaldocument[algebra-]{algebra}
\externaldocument[brauer-]{brauer}
\externaldocument[homology-]{homology}
\externaldocument[derived-]{derived}
\externaldocument[simplicial-]{simplicial}
\externaldocument[more-algebra-]{more-algebra}
\externaldocument[smoothing-]{smoothing}
\externaldocument[modules-]{modules}
\externaldocument[sites-modules-]{sites-modules}
\externaldocument[injectives-]{injectives}
\externaldocument[cohomology-]{cohomology}
\externaldocument[sites-cohomology-]{sites-cohomology}
\externaldocument[dga-]{dga}
\externaldocument[dpa-]{dpa}
\externaldocument[hypercovering-]{hypercovering}
\externaldocument[schemes-]{schemes}
\externaldocument[constructions-]{constructions}
\externaldocument[properties-]{properties}
\externaldocument[morphisms-]{morphisms}
\externaldocument[coherent-]{coherent}
\externaldocument[divisors-]{divisors}
\externaldocument[limits-]{limits}
\externaldocument[varieties-]{varieties}
\externaldocument[topologies-]{topologies}
\externaldocument[descent-]{descent}
\externaldocument[perfect-]{perfect}
\externaldocument[more-morphisms-]{more-morphisms}
\externaldocument[flat-]{flat}
\externaldocument[groupoids-]{groupoids}
\externaldocument[more-groupoids-]{more-groupoids}
\externaldocument[etale-]{etale}
\externaldocument[chow-]{chow}
\externaldocument[intersection-]{intersection}
\externaldocument[pic-]{pic}
\externaldocument[adequate-]{adequate}
\externaldocument[dualizing-]{dualizing}
\externaldocument[duality-]{duality}
\externaldocument[discriminant-]{discriminant}
\externaldocument[local-cohomology-]{local-cohomology}
\externaldocument[curves-]{curves}
\externaldocument[resolve-]{resolve}
\externaldocument[models-]{models}
\externaldocument[pione-]{pione}
\externaldocument[etale-cohomology-]{etale-cohomology}
\externaldocument[proetale-]{proetale}
\externaldocument[crystalline-]{crystalline}
\externaldocument[spaces-]{spaces}
\externaldocument[spaces-properties-]{spaces-properties}
\externaldocument[spaces-morphisms-]{spaces-morphisms}
\externaldocument[decent-spaces-]{decent-spaces}
\externaldocument[spaces-cohomology-]{spaces-cohomology}
\externaldocument[spaces-limits-]{spaces-limits}
\externaldocument[spaces-divisors-]{spaces-divisors}
\externaldocument[spaces-over-fields-]{spaces-over-fields}
\externaldocument[spaces-topologies-]{spaces-topologies}
\externaldocument[spaces-descent-]{spaces-descent}
\externaldocument[spaces-perfect-]{spaces-perfect}
\externaldocument[spaces-more-morphisms-]{spaces-more-morphisms}
\externaldocument[spaces-flat-]{spaces-flat}
\externaldocument[spaces-groupoids-]{spaces-groupoids}
\externaldocument[spaces-more-groupoids-]{spaces-more-groupoids}
\externaldocument[bootstrap-]{bootstrap}
\externaldocument[spaces-pushouts-]{spaces-pushouts}
\externaldocument[groupoids-quotients-]{groupoids-quotients}
\externaldocument[spaces-more-cohomology-]{spaces-more-cohomology}
\externaldocument[spaces-simplicial-]{spaces-simplicial}
\externaldocument[spaces-duality-]{spaces-duality}
\externaldocument[formal-spaces-]{formal-spaces}
\externaldocument[restricted-]{restricted}
\externaldocument[spaces-resolve-]{spaces-resolve}
\externaldocument[formal-defos-]{formal-defos}
\externaldocument[defos-]{defos}
\externaldocument[cotangent-]{cotangent}
\externaldocument[examples-defos-]{examples-defos}
\externaldocument[algebraic-]{algebraic}
\externaldocument[examples-stacks-]{examples-stacks}
\externaldocument[stacks-sheaves-]{stacks-sheaves}
\externaldocument[criteria-]{criteria}
\externaldocument[artin-]{artin}
\externaldocument[quot-]{quot}
\externaldocument[stacks-properties-]{stacks-properties}
\externaldocument[stacks-morphisms-]{stacks-morphisms}
\externaldocument[stacks-limits-]{stacks-limits}
\externaldocument[stacks-cohomology-]{stacks-cohomology}
\externaldocument[stacks-perfect-]{stacks-perfect}
\externaldocument[stacks-introduction-]{stacks-introduction}
\externaldocument[stacks-more-morphisms-]{stacks-more-morphisms}
\externaldocument[stacks-geometry-]{stacks-geometry}
\externaldocument[moduli-]{moduli}
\externaldocument[moduli-curves-]{moduli-curves}
\externaldocument[examples-]{examples}
\externaldocument[exercises-]{exercises}
\externaldocument[guide-]{guide}
\externaldocument[desirables-]{desirables}
\externaldocument[coding-]{coding}
\externaldocument[obsolete-]{obsolete}
\externaldocument[fdl-]{fdl}
\externaldocument[index-]{index}

% Theorem environments.
%
\theoremstyle{plain}
\newtheorem{theorem}[subsection]{Theorem}
\newtheorem{proposition}[subsection]{Proposition}
\newtheorem{lemma}[subsection]{Lemma}

\theoremstyle{definition}
\newtheorem{definition}[subsection]{Definition}
\newtheorem{example}[subsection]{Example}
\newtheorem{exercise}[subsection]{Exercise}
\newtheorem{situation}[subsection]{Situation}

\theoremstyle{remark}
\newtheorem{remark}[subsection]{Remark}
\newtheorem{remarks}[subsection]{Remarks}

\numberwithin{equation}{subsection}

% Macros
%
\def\lim{\mathop{\mathrm{lim}}\nolimits}
\def\colim{\mathop{\mathrm{colim}}\nolimits}
\def\Spec{\mathop{\mathrm{Spec}}}
\def\Hom{\mathop{\mathrm{Hom}}\nolimits}
\def\Ext{\mathop{\mathrm{Ext}}\nolimits}
\def\SheafHom{\mathop{\mathcal{H}\!\mathit{om}}\nolimits}
\def\SheafExt{\mathop{\mathcal{E}\!\mathit{xt}}\nolimits}
\def\Sch{\mathit{Sch}}
\def\Mor{\operatorname{Mor}\nolimits}
\def\Ob{\mathop{\mathrm{Ob}}\nolimits}
\def\Sh{\mathop{\mathit{Sh}}\nolimits}
\def\NL{\mathop{N\!L}\nolimits}
\def\proetale{{pro\text{-}\acute{e}tale}}
\def\etale{{\acute{e}tale}}
\def\QCoh{\mathit{QCoh}}
\def\Ker{\mathop{\mathrm{Ker}}}
\def\Im{\mathop{\mathrm{Im}}}
\def\Coker{\mathop{\mathrm{Coker}}}
\def\Coim{\mathop{\mathrm{Coim}}}

%
% Macros for moduli stacks/spaces
%
\def\QCohstack{\mathcal{QC}\!\mathit{oh}}
\def\Cohstack{\mathcal{C}\!\mathit{oh}}
\def\Spacesstack{\mathcal{S}\!\mathit{paces}}
\def\Quotfunctor{\mathrm{Quot}}
\def\Hilbfunctor{\mathrm{Hilb}}
\def\Curvesstack{\mathcal{C}\!\mathit{urves}}
\def\Polarizedstack{\mathcal{P}\!\mathit{olarized}}
\def\Complexesstack{\mathcal{C}\!\mathit{omplexes}}
% \Pic is the operator that assigns to X its picard group, usage \Pic(X)
% \Picardstack_{X/B} denotes the Picard stack of X over B
% \Picardfunctor_{X/B} denotes the Picard functor of X over B
\def\Pic{\mathop{\mathrm{Pic}}\nolimits}
\def\Picardstack{\mathcal{P}\!\mathit{ic}}
\def\Picardfunctor{\mathrm{Pic}}
\def\Deformationcategory{\mathcal{D}\!\mathit{ef}}


% OK, start here.
%
\begin{document}

\title{Limits of Schemes}


\maketitle

\phantomsection
\label{section-phantom}

\tableofcontents

\section{Introduction}
\label{section-introduction}

\noindent
In this chapter we put material related to limits of schemes. We mostly
study limits of inverse systems over directed sets
(Categories, Definition \ref{categories-definition-directed-set})
with affine transition maps. We discuss absolute Noetherian
approximation. We characterize schemes locally of finite presentation
over a base as those whose associated functor of points is limit
preserving. As an application of absolute Noetherian approximation
we prove that the image of an affine under an integral morphism is affine.
Moreover, we prove some very general variants of Chow's lemma.
A basic reference is \cite{EGA}.




\section{Directed limits of schemes with affine transition maps}
\label{section-limits}

\noindent
In this section we construct the limit.

\begin{lemma}
\label{lemma-directed-inverse-system-affine-schemes-has-limit}
Let $I$ be a directed set. Let $(S_i, f_{ii'})$ be an inverse system of
schemes over $I$.  If all the schemes $S_i$
are affine, then the limit $S = \lim_i S_i$ exists
in the category of schemes.
In fact $S$ is affine and $S = \Spec(\colim_i R_i)$
with $R_i = \Gamma(S_i, \mathcal{O})$.
\end{lemma}

\begin{proof}
Just define $S = \Spec(\colim_i R_i)$.
It follows from Schemes, Lemma \ref{schemes-lemma-morphism-into-affine}
that $S$ is the limit even in the category of locally ringed spaces.
\end{proof}

\begin{lemma}
\label{lemma-directed-inverse-system-has-limit}
Let $I$ be a directed set. Let $(S_i, f_{ii'})$ be an
inverse system of schemes over $I$. If all the morphisms
$f_{ii'} : S_i \to S_{i'}$ are affine, then the limit $S = \lim_i S_i$ exists
in the category of schemes. Moreover,
\begin{enumerate}
\item each of the morphisms $f_i : S \to S_i$ is affine,
\item for an element $0 \in I$ and any open subscheme $U_0 \subset S_0$
we have
$$
f_0^{-1}(U_0) = \lim_{i \geq 0} f_{i0}^{-1}(U_0)
$$
in the category of schemes.
\end{enumerate}
\end{lemma}

\begin{proof}
Choose an element $0 \in I$. Note that $I$ is nonempty as the limit is
directed. For every $i \geq 0$ consider the quasi-coherent sheaf of
$\mathcal{O}_{S_0}$-algebras $\mathcal{A}_i = f_{i0, *}\mathcal{O}_{S_i}$.
Recall that $S_i = \underline{\Spec}_{S_0}(\mathcal{A}_i)$,
see Morphisms, Lemma \ref{morphisms-lemma-characterize-affine}.
Set $\mathcal{A} = \colim_{i \geq 0} \mathcal{A}_i$.
This is a quasi-coherent sheaf of $\mathcal{O}_{S_0}$-algebras,
see Schemes, Section \ref{schemes-section-quasi-coherent}.
Set $S = \underline{\Spec}_{S_0}(\mathcal{A})$.
By Morphisms, Lemma \ref{morphisms-lemma-affine-equivalence-algebras}
we get for $i \geq 0$ morphisms $f_i : S \to S_i$ compatible with
the transition morphisms. Note that the morphisms $f_i$ are
affine by Morphisms, Lemma \ref{morphisms-lemma-affine-permanence} for example.
By Lemma \ref{lemma-directed-inverse-system-affine-schemes-has-limit} above
we see that for any affine open $U_0 \subset S_0$ the
inverse image $U = f_0^{-1}(U_0) \subset S$ is the limit of the
system of opens $U_i = f_{i0}^{-1}(U_0)$, $i \geq 0$ in the
category of schemes.

\medskip\noindent
Let $T$ be a scheme. Let $g_i : T \to S_i$ be a compatible system
of morphisms. To show that $S = \lim_i S_i$ we have
to prove there is a unique morphism $g : T \to S$ with
$g_i = f_i \circ g$ for all $i \in I$.
For every $t \in T$ there exists an affine open
$U_0 \subset S_0$ containing $g_0(t)$. Let $V \subset g_0^{-1}(U_0)$
be an affine open neighbourhood containing $t$.
By the remarks above we obtain a unique morphism
$g_V : V \to U = f_0^{-1}(U_0)$ such that $f_i \circ g_V = g_i|_{U_i}$
for all $i$. The open sets $V \subset T$ so constructed form
a basis for the topology of $T$. The morphisms $g_V$ glue to a morphism
$g : T \to S$ because of the uniqueness property. This gives the
desired morphism $g : T \to S$.

\medskip\noindent
The final statement is clear from the construction of the limit above.
\end{proof}

\begin{lemma}
\label{lemma-scheme-over-limit}
Let $I$ be a directed set.
Let $(S_i, f_{ii'})$ be an inverse system of schemes over $I$.
Assume all the morphisms $f_{ii'} : S_i \to S_{i'}$ are affine,
Let $S = \lim_i S_i$. Let $0 \in I$.
Suppose that $T$ is a scheme over $S_0$.
Then
$$
T \times_{S_0} S = \lim_{i \geq 0} T \times_{S_0} S_i
$$
\end{lemma}

\begin{proof}
The right hand side is a scheme by
Lemma \ref{lemma-directed-inverse-system-has-limit}.
The equality is formal, see
Categories, Lemma \ref{categories-lemma-colimits-commute}.
\end{proof}



\section{Infinite products}
\label{section-inifinite-products}

\noindent
Infinite products of schemes usually do not exist. For example in
Examples, Section \ref{examples-section-not-algebraic}
it is shown that an infinite product of
copies of $\mathbf{P}^1$ is not even an algebraic space.

\medskip\noindent
On the other hand, infinite products of affine schemes do exist
and are affine. Using Schemes, Lemma \ref{schemes-lemma-morphism-into-affine}
this corresponds to the fact that in the category of rings
we have infinite coproducts: if $I$ is a set and $R_i$ is a ring
for each $i$, then we can consider the ring
$$
R = \otimes R_i =
\colim_{\{i_1, \ldots, i_n\} \subset I}
R_{i_1} \otimes_\mathbf{Z} \ldots \otimes_\mathbf{Z} R_{i_n}
$$
Given another ring $A$ a map $R \to A$ is the same thing as a
collection of ring maps $R_i \to A$ for all $i \in I$ as
follows from the corresponding property of finite tensor products.

\begin{lemma}
\label{lemma-infinite-product}
Let $S$ be a scheme. Let $I$ be a set and for each $i \in I$
let $f_i : T_i \to S$ be an affine morphism. Then the
product $T = \prod T_i$ exists in the category of schemes
over $S$. In fact, we have
$$
T = \lim_{\{i_1, \ldots, i_n\} \subset I}
T_{i_1} \times_S \ldots \times_S T_{i_n}
$$
and the projection morphisms $T \to T_{i_1} \times_S \ldots \times_S T_{i_n}$
are affine.
\end{lemma}

\begin{proof}
Omitted. Hint: Argue as in the discussion preceding the lemma
and use Lemma \ref{lemma-directed-inverse-system-has-limit}
for existence of the limit.
\end{proof}

\begin{lemma}
\label{lemma-infinite-product-surjective}
Let $S$ be a scheme. Let $I$ be a set and for each $i \in I$
let $f_i : T_i \to S$ be a surjective affine morphism. Then the
product $T = \prod T_i$ in the category of schemes over $S$
(Lemma \ref{lemma-infinite-product})
maps surjectively to $S$.
\end{lemma}

\begin{proof}
Let $s \in S$. Choose $t_i \in T_i$ mapping to $s$.
Choose a huge field extension $K/\kappa(s)$ such that
$\kappa(s_i)$ embeds into $K$ for each $i$. Then we get
morphisms $\Spec(K) \to T_i$ with image $s_i$ agreeing
as morphisms to $S$. Whence a morphism $\Spec(K) \to T$
which proves there is a point of $T$ mapping to $s$.
\end{proof}

\begin{lemma}
\label{lemma-infinite-product-integral}
Let $S$ be a scheme. Let $I$ be a set and for each $i \in I$
let $f_i : T_i \to S$ be an integral morphism. Then the
product $T = \prod T_i$ in the category of schemes over $S$
(Lemma \ref{lemma-infinite-product})
is integral over $S$.
\end{lemma}

\begin{proof}
Omitted. Hint: On affine pieces this reduces to the following
algebra fact: if $A \to B_i$ is integral for all $i$, then
$A \to \otimes_A B_i$ is integral.
\end{proof}





\section{Descending properties}
\label{section-descent}

\noindent
First some basic lemmas describing the topology of a limit.

\begin{lemma}
\label{lemma-inverse-limit-sets}
Let $S = \lim S_i$ be the limit of a directed inverse system
of schemes with affine transition morphisms
(Lemma \ref{lemma-directed-inverse-system-has-limit}). Then
$S_{set} = \lim_i S_{i, set}$ where $S_{set}$
indicates the underlying set of the scheme $S$.
\end{lemma}

\begin{proof}
Pick $i \in I$. Take $U_i \subset S_i$ an affine open.
Denote $U_{i'} = f_{i'i}^{-1}(U_i)$ and $U = f_i^{-1}(U_i)$.
Here $f_{i'i} : S_{i'} \to S_i$ is the transtion morphism
and $f_i : S \to S_i$ is the projection.
By Lemma \ref{lemma-directed-inverse-system-has-limit}
we have $U = \lim_{i' \geq i} U_i$.
Suppose we can show that $U_{set} = \lim_{i' \geq i} U_{i', set}$. Then
the lemma follows by a simple argument using an affine covering of $S_i$.
Hence we may assume all $S_i$ and $S$ affine. This reduces us to the
algebra question considered in the next paragraph.

\medskip\noindent
Suppose given a system of rings $(A_i, \varphi_{ii'})$
over $I$. Set $A = \colim_i A_i$ with canonical maps $\varphi_i : A_i \to A$.
Then
$$
\Spec(A) = \lim_i \Spec(A_i)
$$
Namely, suppose that we are given primes $\mathfrak p_i \subset A_i$
such that $\mathfrak p_i = \varphi_{ii'}^{-1}(\mathfrak p_{i'})$
for all $i' \geq i$. Then we simply set
$$
\mathfrak p =
\{x \in A
\mid
\exists i, x_i \in \mathfrak p_i \text{ with }\varphi_i(x_i) = x\}
$$
It is clear that this is an ideal and has the property that
$\varphi_i^{-1}(\mathfrak p) = \mathfrak p_i$. Then it follows
easily that it is a prime ideal as well.
\end{proof}

\begin{lemma}
\label{lemma-inverse-limit-top}
\begin{reference}
\cite[IV, Proposition 8.2.9]{EGA}
\end{reference}
Let $S = \lim S_i$ be the limit of a directed inverse system
of schemes with affine transition morphisms
(Lemma \ref{lemma-directed-inverse-system-has-limit}). Then
$S_{top} = \lim_i S_{i, top}$ where $S_{top}$
indicates the underlying topological space of the scheme $S$.
\end{lemma}

\begin{proof}
We will use the criterion of
Topology, Lemma \ref{topology-lemma-characterize-limit}.
We have seen that $S_{set} = \lim_i S_{i, set}$ in
Lemma \ref{lemma-inverse-limit-sets}.
The maps $f_i : S \to S_i$ are morphisms of schemes
hence continuous. Thus $f_i^{-1}(U_i)$ is open for each
open $U_i \subset S_i$. Finally, 
let $s \in S$ and let $s \in V \subset S$ be an open
neighbourhood. Choose $0 \in I$ and choose an
affine open neighbourhood $U_0 \subset S_0$ of the image of $s$.
Then $f_0^{-1}(U_0) = \lim_{i \geq 0} f_{i0}^{-1}(U_0)$, see
Lemma \ref{lemma-directed-inverse-system-has-limit}.
Then $f_0^{-1}(U_0)$ and $f_{i0}^{-1}(U_0)$ are affine and
$$
\mathcal{O}_S(f_0^{-1}(U_0)) =
\colim_{i \geq 0} \mathcal{O}_{S_i}(f_{i0}^{-1}(U_0))
$$
either by the proof of
Lemma \ref{lemma-directed-inverse-system-has-limit}
or by
Lemma \ref{lemma-directed-inverse-system-affine-schemes-has-limit}.
Choose $a \in \mathcal{O}_S(f_0^{-1}(U_0))$ such that
$s \in D(a) \subset V$. This is possible because the principal
opens form a basis for the topology on the affine scheme $f_0^{-1}(U_0)$.
Then we can pick an $i \geq 0$ and
$a_i \in \mathcal{O}_{S_i}(f_{i0}^{-1}(U_0))$ mapping to $a$.
It follows that $D(a_i) \subset f_{i0}^{-1}(U_0) \subset S_i$
is an open subset whose inverse image in $S$ is $D(a)$.
This finishes the proof.
\end{proof}

\begin{lemma}
\label{lemma-limit-nonempty}
Let $S = \lim S_i$ be the limit of a directed inverse system
of schemes with affine transition morphisms
(Lemma \ref{lemma-directed-inverse-system-has-limit}).
If all the schemes $S_i$ are nonempty and quasi-compact,
then the limit $S = \lim_i S_i$ is nonempty.
\end{lemma}

\begin{proof}
Choose $0 \in I$. Note that $I$ is nonempty as the limit is directed.
Choose an affine open covering $S_0 = \bigcup_{j = 1, \ldots, m} U_j$.
Since $I$ is directed there exists a $j \in \{1, \ldots, m\}$
such that $f_{i0}^{-1}(U_j) \not = \emptyset$ for all
$i \geq 0$. Hence $\lim_{i \geq 0} f_{i0}^{-1}(U_j)$ is not
empty since a directed colimit of nonzero rings is nonzero
(because $1 \not = 0$). As $\lim_{i \geq 0} f_{i0}^{-1}(U_j)$
is an open subscheme of the limit we win.
\end{proof}

\begin{lemma}
\label{lemma-inverse-limit-irreducibles}
Let $S = \lim S_i$ be the limit of a directed inverse system
of schemes with affine transition morphisms
(Lemma \ref{lemma-directed-inverse-system-has-limit}).
Let $s \in S$ with images $s_i \in S_i$.
Then $\overline{\{s\}} = \lim_i \overline{\{s_i\}}$ as sets
and as schemes if endowed with the reduced induced scheme structure.
\end{lemma}

\begin{proof}
Choose $0 \in I$ and an affine open covering
$U_0 = \bigcup_{j \in J} U_{0, j}$.
For $i \geq 0$ let $U_{i, j} = f_{i, 0}^{-1}(U_{0, j})$
and set $U_j = f_0^{-1}(U_{0, j})$.
Here $f_{i'i} : S_{i'} \to S_i$ is the transtion morphism
and $f_i : S \to S_i$ is the projection.
For $j \in J$ the following are equivalent:
(a) $x \in U_j$, (b) $x_0 \in U_{0, j}$,
(c) $x_i \in U_{i, j}$ for all $i \geq 0$.
Let $J' \subset J$ be the set of indices for which (a), (b), (c) are true.
Then $\overline{\{s\}} = \bigcup_{j \in J'} (\overline{\{s\}} \cap U_j)$
and simiarly for $\overline{\{s_i\}}$ for $i \geq 0$.
Note that $\overline{\{s\}} \cap U_j$ is the closure of the
set $\{s\}$ in the topological space $U_j$. Similarly for
$\overline{\{s_i\}} \cap U_{i, j}$ for $i \geq 0$.
Hence it suffices to prove the lemma in the case
$S$ and $S_i$ affine for all $i$. This reduces us to the
algebra question considered in the next paragraph.

\medskip\noindent
Suppose given a system of rings $(A_i, \varphi_{ii'})$
over $I$. Set $A = \colim_i A_i$ with canonical maps
$\varphi_i : A_i \to A$. Let $\mathfrak p \subset A$ be a
prime and set $\mathfrak p_i = \varphi_i^{-1}(\mathfrak p)$.
Then
$$
V(\mathfrak p) = \lim_i V(\mathfrak p_i)
$$
This follows from Lemma \ref{lemma-inverse-limit-sets}
because $A/\mathfrak p = \colim A_i/\mathfrak p_i$.
This equality of rings also shows the final statement
about reduced induced scheme structures holds true.
\end{proof}

\noindent
In the rest of this section we work in the following situation.

\begin{situation}
\label{situation-descent}
Let $S = \lim_{i \in I} S_i$ be the limit of a directed system of schemes
with affine transition morphisms $f_{i'i} : S_{i'} \to S_i$
(Lemma \ref{lemma-directed-inverse-system-has-limit}).
We assume that $S_i$ is quasi-compact and quasi-separated for all $i \in I$.
We denote $f_i : S \to S_i$ the projection.
We also choose an element $0 \in I$.
\end{situation}

\noindent
The type of result we are looking for is the following:
If we have an object over $S$, then for some $i$ there is a
similar object over $S_i$.

\begin{lemma}
\label{lemma-topology-limit}
In Situation \ref{situation-descent}.
\begin{enumerate}
\item We have $S_{set} = \lim_i S_{i, set}$ where $S_{set}$
indicates the underlying set of the scheme $S$.
\item We have $S_{top} = \lim_i S_{i, top}$ where $S_{top}$
indicates the underlying topological space of the scheme $S$.
\item If $s, s' \in S$ and $s'$ is not a specialization of $s$
then for some $i \in I$ the image $s'_i \in S_i$ of $s'$ is not
a specialization of the image $s_i \in S_i$ of $s$.
\item Add more easy facts on topology of $S$ here.
(Requirement: whatever is added should be easy in the affine case.)
\end{enumerate}
\end{lemma}

\begin{proof}
Part (1) is a special case of Lemma \ref{lemma-inverse-limit-sets}.

\medskip\noindent
Part (2) is a special case of Lemma \ref{lemma-inverse-limit-top}.

\medskip\noindent
Part (3) is a special case of Lemma \ref{lemma-inverse-limit-irreducibles}.
\end{proof}

\begin{lemma}
\label{lemma-descend-section}
In Situation \ref{situation-descent}.
Suppose that $\mathcal{F}_0$ is a quasi-coherent sheaf on $S_0$.
Set $\mathcal{F}_i = f_{i0}^*\mathcal{F}_0$ for $i \geq 0$ and set
$\mathcal{F} = f_0^*\mathcal{F}_0$.
Then
$$
\Gamma(S, \mathcal{F}) = \colim_{i \geq 0} \Gamma(S_i, \mathcal{F}_i)
$$
\end{lemma}

\begin{proof}
Write $\mathcal{A}_j = f_{i0, *} \mathcal{O}_{S_i}$.
This is a quasi-coherent sheaf of $\mathcal{O}_{S_0}$-algebras
(see Morphisms, Lemma \ref{morphisms-lemma-affine-equivalence-algebras})
and $S_i$ is the relative spectrum of $\mathcal{A}_i$ over $S_0$.
In the proof of Lemma \ref{lemma-directed-inverse-system-has-limit}
we constructed $S$ as the relative spectrum of
$\mathcal{A} = \colim_{i \geq 0} \mathcal{A}_i$
over $S_0$. Set
$$
\mathcal{M}_i = \mathcal{F}_0 \otimes_{\mathcal{O}_{S_0}} \mathcal{A}_i
$$
and
$$
\mathcal{M} = \mathcal{F}_0 \otimes_{\mathcal{O}_{S_0}} \mathcal{A}.
$$
Then we have $f_{i0, *} \mathcal{F}_i = \mathcal{M}_i$
and $f_{0, *}\mathcal{F} = \mathcal{M}$. Since $\mathcal{A}$
is the colimit of the sheaves $\mathcal{A}_i$ and since tensor
product commutes with directed colimits, we conclude that
$\mathcal{M} = \colim_{i \geq 0} \mathcal{M}_i$.
Since $S_0$ is quasi-compact and quasi-separated we see that
\begin{eqnarray*}
\Gamma(S, \mathcal{F})
& = &
\Gamma(S_0, \mathcal{M}) \\
& = &
\Gamma(S_0, \colim_{i \geq 0} \mathcal{M}_i) \\
& = &
\colim_{i \geq 0} \Gamma(S_0, \mathcal{M}_i) \\
& = &
\colim_{i \geq 0} \Gamma(S_i, \mathcal{F}_i)
\end{eqnarray*}
see Sheaves, Lemma \ref{sheaves-lemma-directed-colimits-sections} and
Topology, Lemma \ref{topology-lemma-topology-quasi-separated-scheme}
for the middle equality.
\end{proof}

\begin{lemma}
\label{lemma-limit-closed-nonempty}
In Situation \ref{situation-descent}.
Suppose for each $i$ we are given a nonempty closed subset
$Z_i \subset S_i$ with $f_{ii'}(Z_i) \subset Z_{i'}$.
Then there exists a point $s \in S$ with $f_i(s) \in Z_i$ for
all $i$.
\end{lemma}

\begin{proof}
Let $Z_i \subset S_i$ also denote the reduced closed subscheme
associated to $Z_i$, see Schemes,
Definition \ref{schemes-definition-reduced-induced-scheme}.
A closed immersion is affine, and a composition of affine
morphisms is affine (see
Morphisms, Lemmas \ref{morphisms-lemma-closed-immersion-affine}
and \ref{morphisms-lemma-composition-affine}), and hence $Z_i \to S_{i'}$ is
affine when $i \geq i'$. We conclude that the morphism
$f_{ii'} : Z_i \to Z_{i'}$ is affine by
Morphisms, Lemma \ref{morphisms-lemma-affine-permanence}.
Each of the schemes $Z_i$ is quasi-compact as a closed
subscheme of a quasi-compact scheme. Hence we may apply
Lemma \ref{lemma-limit-nonempty} to see that
$Z = \lim_i Z_i$ is nonempty. Since there is a
canonical morphism $Z \to S$ we win.
\end{proof}

\begin{lemma}
\label{lemma-limit-fibre-product-empty}
In Situation \ref{situation-descent}.
Suppose we are given an $i$ and a morphism $T \to S_i$ such that
\begin{enumerate}
\item $T \times_{S_i} S = \emptyset$, and
\item $T$ is quasi-compact.
\end{enumerate}
Then $T \times_{S_i} S_{i'} = \emptyset$ for all sufficiently large $i'$.
\end{lemma}

\begin{proof}
By Lemma \ref{lemma-scheme-over-limit}
we see that $T \times_{S_i} S = \lim_{i' \geq i} T \times_{S_i} S_{i'}$.
Hence the result follows from
Lemma \ref{lemma-limit-nonempty}.
\end{proof}

\begin{lemma}
\label{lemma-limit-contained-in-constructible}
In Situation \ref{situation-descent}.
Suppose we are given an $i$ and a locally constructible subset
$E \subset S_i$ such that $f_i(S) \subset E$.
Then $f_{ii'}(S_{i'}) \subset E$ for all sufficiently large $i'$.
\end{lemma}

\begin{proof}
Writing $S_i$ as a finite union of open affine subschemes reduces
the question to the case that $S_i$ is affine and $E$ is constructible, see
Lemma \ref{lemma-directed-inverse-system-has-limit}
and
Properties, Lemma \ref{properties-lemma-locally-constructible}.
In this case the complement $S_i \setminus E$ is constructible too.
Hence there exists an affine scheme $T$ and a morphism $T \to S_i$
whose image is $S_i \setminus E$, see
Algebra, Lemma \ref{algebra-lemma-constructible-is-image}.
By
Lemma \ref{lemma-limit-fibre-product-empty}
we see that $T \times_{S_i} S_{i'}$ is empty for all sufficiently large
$i'$, and hence $f_{ii'}(S_{i'}) \subset E$ for all sufficiently large $i'$.
\end{proof}

\begin{lemma}
\label{lemma-descend-opens}
In Situation \ref{situation-descent} we have the following:
\begin{enumerate}
\item Given any quasi-compact open $V \subset S = \lim_i S_i$
there exists an $i \in I$ and a quasi-compact open $V_i \subset S_i$
such that $f_i^{-1}(V_i) = V$.
\item Given $V_i \subset S_i$ and $V_{i'} \subset S_{i'}$
quasi-compact opens such that $f_i^{-1}(V_i) = f_{i'}^{-1}(V_{i'})$
there exists an index $i'' \geq i, i'$ such that
$f_{i''i}^{-1}(V_i) = f_{i''i'}^{-1}(V_{i'})$.
\item If $V_{1, i}, \ldots, V_{n, i} \subset S_i$ are quasi-compact
opens and $S = f_i^{-1}(V_{1, i}) \cup \ldots \cup f_i^{-1}(V_{n, i})$
then $S_{i'} = f_{i'i}^{-1}(V_{1, i}) \cup \ldots \cup f_{i'i}^{-1}(V_{n, i})$
for some $i' \geq i$.
\end{enumerate}
\end{lemma}

\begin{proof}
Choose $i_0 \in I$. Note that $I$ is nonempty as the limit is directed.
For convenience we write $S_0 = S_{i_0}$ and $i_0 = 0$.
Choose an affine open covering $S_0 = U_{1, 0} \cup \ldots \cup U_{m, 0}$.
Denote $U_{j, i} \subset S_i$ the inverse image of $U_{j, 0}$
under the transition morphism for $i \geq 0$.
Denote $U_j$ the inverse image of $U_{j, 0}$ in $S$.
Note that $U_j = \lim_i U_{j, i}$ is a limit of affine
schemes.

\medskip\noindent
We first prove the uniqueness statement: Let
$V_i \subset S_i$ and $V_{i'} \subset S_{i'}$
quasi-compact opens such that $f_i^{-1}(V_i) = f_{i'}^{-1}(V_{i'})$.
It suffices to show that $f_{i''i}^{-1}(V_i \cap U_{j, i''})$ and
$f_{i''i'}^{-1}(V_{i'} \cap U_{j, i''})$ become equal
for $i''$ large enough. Hence we reduce to the case
of a limit of affine schemes. In this case write
$S = \Spec(R)$ and $S_i = \Spec(R_i)$ for all $i \in I$.
We may write $V_i = S_i \setminus V(h_1, \ldots, h_m)$
and $V_{i'} = S_{i'} \setminus V(g_1, \ldots, g_n)$.
The assumption means that the ideals
$\sum g_jR$ and $\sum h_jR$ have the same radical
in $R$. This means that $g_j^N = \sum a_{jj'}h_{j'}$ and
$h_j^N = \sum b_{jj'} g_{j'}$ for some $N \gg 0$ and $a_{jj'}$
and $b_{jj'}$ in $R$.
Since $R = \colim_i R_i$ we can chose an index
$i'' \geq i$ such that the equations
$g_j^N = \sum a_{jj'}h_{j'}$ and
$h_j^N = \sum b_{jj'} g_{j'}$ hold in $R_{i''}$ for some
$a_{jj'}$ and $b_{jj'}$ in $R_{i''}$. This implies that
the ideals $\sum g_jR_{i''}$ and $\sum h_jR_{i''}$ have the same radical
in $R_{i''}$ as desired.

\medskip\noindent
We prove existence: If $S_0$ is affine, then $S_i = \Spec(R_i)$ for all
$i \geq 0$ and $S = \Spec(R)$ with $R = \colim R_i$. Then
$V = S \setminus V(g_1, \ldots, g_n)$ for some $g_1, \ldots, g_n \in R$.
Choose any $i$ large enough so that each of the $g_j$ comes from an
element $g_{j, i} \in R_i$ and take
$V_i = S_i \setminus V(g_{1, i}, \ldots, g_{n, i})$.
If $S_0$ is general, then the opens $V \cap U_j$
are quasi-compact because $S$ is quasi-separated. Hence by the
affine case we see that for each $j = 1, \ldots, m$
there exists an $i_j \in I$ and a quasi-compact open
$V_{i_j} \subset U_{j, i_j}$ whose inverse image in $U_j$
is $V \cap U_j$. Set $i = \max(i_1, \ldots, i_m)$
and let $V_i = \bigcup f_{ii_j}^{-1}(V_{i_j})$.

\medskip\noindent
The statement on coverings follows from the uniqueness statement
for the opens $V_{1, i} \cup \ldots \cup V_{n, i}$ and $S_i$ of $S_i$.
\end{proof}

\begin{lemma}
\label{lemma-limit-quasi-affine}
In Situation \ref{situation-descent} if $S$ is quasi-affine, then
for some $i_0 \in I$ the schemes $S_i$ for $i \geq i_0$ are quasi-affine.
\end{lemma}

\begin{proof}
Choose $i_0 \in I$. Note that $I$ is nonempty as the limit is directed.
For convenience we write $S_0 = S_{i_0}$ and $i_0 = 0$.
Let $s \in S$. We may choose an affine open
$U_0 \subset S_0$ containing $f_0(s)$. Since $S$ is quasi-affine
we may choose an element $a \in \Gamma(S, \mathcal{O}_S)$ such
that $s \in D(a) \subset f_0^{-1}(U_0)$, and such that
$D(a)$ is affine. By Lemma \ref{lemma-descend-section}
there exists an $i \geq 0$ such that $a$
comes from an element $a_i \in \Gamma(S_i, \mathcal{O}_{S_i})$.
For any index $j \geq i$ we denote $a_j$
the image of $a_i$ in the global sections of the
structure sheaf of $S_j$.
Consider the opens $D(a_j) \subset S_j$
and $U_j = f_{j0}^{-1}(U_0)$. Note that
$U_j$ is affine and $D(a_j)$ is a quasi-compact open of $S_j$,
see Properties, Lemma \ref{properties-lemma-affine-cap-s-open}
for example. Hence we may apply Lemma \ref{lemma-descend-opens} to the opens
$U_j$ and $U_j \cup D(a_j)$ to conclude that
$D(a_j) \subset U_j$ for some  $j \geq i$.
For such an index $j$ we see that $D(a_j) \subset S_j$ is an affine open
(because $D(a_j)$ is a standard affine open of the affine open $U_j$)
containing the image $f_j(s)$.

\medskip\noindent
We conclude that for every $s \in S$ there exist
an index $i \in I$, and a global section
$a \in \Gamma(S_i, \mathcal{O}_{S_i})$
such that $D(a) \subset S_i$ is an affine open
containing $f_i(s)$. Because $S$ is quasi-compact we
may choose a single index $i \in I$ and global sections
$a_1, \ldots, a_m \in \Gamma(S_i, \mathcal{O}_{S_i})$
such that each $D(a_j) \subset S_i$ is affine open
and such that $f_i : S \to S_i$ has image contained
in the union $W_i = \bigcup_{j = 1, \ldots, m} D(a_j)$.
For $i' \geq i$ set $W_{i'} = f_{i'i}^{-1}(W_i)$.
Since $f_i^{-1}(W_i)$ is all of $S$ we see
(by Lemma \ref{lemma-descend-opens} again)
that for a suitable $i' \geq i$ we
have $S_{i'} = W_{i'}$. Thus we may replace $i$ by
$i'$ and assume that $S_i = \bigcup_{j = 1, \ldots, m} D(a_j)$.
This implies that $\mathcal{O}_{S_i}$ is an ample invertible
sheaf on $S_i$ (see Properties, Definition \ref{properties-definition-ample})
and hence that $S_i$ is quasi-affine, see
Properties, Lemma \ref{properties-lemma-quasi-affine-O-ample}.
Hence we win.
\end{proof}

\begin{lemma}
\label{lemma-limit-affine}
In Situation \ref{situation-descent} if $S$ is affine,
then for some $i_0 \in I$ the schemes $S_i$ for $i \geq i_0$
are affine.
\end{lemma}

\begin{proof}
By Lemma \ref{lemma-limit-quasi-affine} we may assume that $S_0$ is
quasi-affine for some $0 \in I$. Set $R_0 = \Gamma(S_0, \mathcal{O}_{S_0})$.
Then $S_0$ is a quasi-compact open of $T_0 = \Spec(R_0)$. Denote
$j_0 : S_0 \to T_0$ the corresponding quasi-compact open immersion.
For $i \geq 0$ set $\mathcal{A}_i = f_{0i, *}\mathcal{O}_{S_i}$.
Since $f_{0i}$ is affine we see that
$S_i = \underline{\Spec}_{S_0}(\mathcal{A}_i)$.
Set $T_i = \underline{\Spec}_{T_0}(j_{0, *}\mathcal{A}_i)$.
Then $T_i \to T_0$ is affine, hence $T_i$ is affine. Thus
$T_i$ is the spectrum of
$$
R_i = \Gamma(T_0, j_{0, *}\mathcal{A}_i) = \Gamma(S_0, \mathcal{A}_i) =
\Gamma(S_i, \mathcal{O}_{S_i}).
$$
Write $S = \Spec(R)$. We have $R = \colim_i R_i$
by Lemma \ref{lemma-descend-section}.
Hence also $S = \lim_i T_i$. As formation of the relative spectrum commutes
with base change, the inverse image
of the open $S_0 \subset T_0$ in $T_i$ is $S_i$.
Let $Z_0 = T_0 \setminus S_0$ and let $Z_i \subset T_i$
be the inverse image of $Z_0$. As $S_i = T_i \setminus Z_i$, it suffices
to show that $Z_i$ is empty for some $i$. Assume $Z_i$ is nonempty for all
$i$ to get a contradiction. By Lemma \ref{lemma-limit-closed-nonempty}
there exists a point $s$ of $S = \lim T_i$ which maps to a point of $Z_i$
for every $i$. But $S = \lim_i S_i$, and hence we arrive at a contradiction
by Lemma \ref{lemma-topology-limit}.
\end{proof}

\begin{lemma}
\label{lemma-limit-separated}
In Situation \ref{situation-descent} if $S$ is separated,
then for some $i_0 \in I$ the schemes $S_i$ for $i \geq i_0$
are separated.
\end{lemma}

\begin{proof}
Choose a finite affine open covering
$S_0 = U_{0, 1} \cup \ldots \cup U_{0, m}$.
Set $U_{i, j} \subset S_i$ and $U_j \subset S$
equal to the inverse image of $U_{0, j}$.
Note that $U_{i, j}$ and $U_j$ are affine. As $S$ is separated
the intersections $U_{j_1} \cap U_{j_2}$ are affine. Since
$U_{j_1} \cap U_{j_2} = \lim_{i \geq 0} U_{i, j_1} \cap U_{i, j_2}$
we see that $U_{i, j_1} \cap U_{i, j_2}$ is affine for large $i$
by Lemma \ref{lemma-limit-affine}. To show that $S_i$ is separated
for large $i$ it now suffices to show that
$$
\mathcal{O}_{S_i}(V_{i, j_1})
\otimes_{\mathcal{O}_S(S)}
\mathcal{O}_{S_i}(V_{i, j_2})
\longrightarrow
\mathcal{O}_{S_i}(V_{i, j_1} \cap V_{i, j_2})
$$
is surjective for large $i$
(Schemes, Lemma \ref{schemes-lemma-characterize-separated}).

\medskip\noindent
To get rid of the annoying indices, assume we have affine opens
$U, V \subset S_0$ such that $U \cap V$ is affine too.
Let $U_i, V_i \subset S_i$, resp.\ $U, V \subset S$ be the inverse images.
We have to show that
$\mathcal{O}(U_i) \otimes \mathcal{O}(V_i) \to
\mathcal{O}(U_i \cap V_i)$
is surjective for $i$ large enough and we know that
$\mathcal{O}(U_) \otimes \mathcal{O}(V) \to \mathcal{O}(U \cap V)$
is surjective. Note that
$\mathcal{O}(U_0) \otimes \mathcal{O}(V_0) \to
\mathcal{O}(U_0 \cap V_0)$
is of finite type, as the diagonal morphism $S_i \to S_i \times S_i$
is an immersion (Schemes, Lemma \ref{schemes-lemma-diagonal-immersion})
hence locally of finite type
(Morphisms, Lemmas \ref{morphisms-lemma-locally-finite-type-characterize} and
\ref{morphisms-lemma-immersion-locally-finite-type}).
Thus we can choose elements
$f_{0, 1}, \ldots, f_{0, n} \in \mathcal{O}(U_0 \cap V_0)$
which generate $\mathcal{O}(U_0 \cap V_0)$ over
$\mathcal{O}(U_0) \otimes \mathcal{O}(V_0)$.
Observe that for $i \geq 0$ the diagram of schemes
$$
\xymatrix{
U_i \cap V_i \ar[r] \ar[d] & U_i \ar[d] \\
U_0 \cap V_0 \ar[r] & U_0
}
$$
is cartesian. Thus we see that the images
$f_{i, 1}, \ldots, f_{i, n} \in \mathcal{O}(U_i \cap V_i)$
generate $\mathcal{O}(U_i \cap V_i)$ over
$\mathcal{O}(U_i) \otimes \mathcal{O}(V_0)$
and a fortiori over
$\mathcal{O}(U_i) \otimes \mathcal{O}(V_i)$.
By assumption the images $f_1, \ldots, f_n \in \mathcal{O}(U \otimes V)$
are in the image of the map
$\mathcal{O}(U) \otimes \mathcal{O}(V) \to \mathcal{O}(U \cap V)$.
Since
$\mathcal{O}(U) \otimes \mathcal{O}(V) =
\colim \mathcal{O}(U_i) \otimes \mathcal{O}(V_i)$
we see that they are in the image of the map at some finite level
and the lemma is proved.
\end{proof}

\begin{lemma}
\label{lemma-limit-ample}
In Situation \ref{situation-descent} let $\mathcal{L}_0$ be an invertible
sheaf of modules on $S_0$. If the pullback $\mathcal{L}$ to $S$ is ample,
then for some $i \in I$ the pullback $\mathcal{L}_i$ to $S_i$ is ample.
\end{lemma}

\begin{proof}
The assumption means there are finitely many sections
$s_1, \ldots, s_m \in \Gamma(S, \mathcal{L})$ such that
$S_{s_j}$ is affine and such that $S = \bigcup S_{s_j}$, see
Properties, Definition \ref{properties-definition-ample}.
By Lemma \ref{lemma-descend-section} we can find an $i \in I$
and sections $s_{i, j} \in \Gamma(S_i, \mathcal{L}_i)$ mapping to $s_j$.
By Lemma \ref{lemma-limit-affine} we may, after increasing $i$, assume
that $(S_i)_{s_{i, j}}$ is affine for $j = 1, \ldots, m$.
By Lemma \ref{lemma-descend-opens} we may, after increasing $i$ a
last time, assume that $S_i = \bigcup (S_i)_{s_{i, j}}$.
Then $\mathcal{L}_i$ is ample by definition.
\end{proof}

\begin{lemma}
\label{lemma-finite-type-eventually-closed}
Let $S$ be a scheme. Let $X = \lim X_i$ be a directed limit of
schemes over $S$ with affine transition morphisms. Let $Y \to X$
be a morphism of schemes over $S$.
\begin{enumerate}
\item If $Y \to X$ is a closed immersion, $X_i$ quasi-compact, and
$Y$ locally of finite type over $S$, then $Y \to X_i$ is a closed
immersion for $i$ large enough.
\item If $Y \to X$ is an immersion, $X_i$ quasi-separated, $Y \to S$ locally
of finite type, and $Y$ quasi-compact, then $Y \to X_i$ is an
immersion for $i$ large enough.
\end{enumerate}
\end{lemma}

\begin{proof}
Proof of (1). Choose $0 \in I$ and a finite affine open covering
$X_0 = U_{0, 1} \cup \ldots \cup U_{0, m}$ with the property that
$U_{0, j}$ maps into an affine open $W_j \subset S$.
Let $V_j \subset Y$, resp.\ $U_{i, j} \subset X_i$, $i \geq 0$,
resp. $U_j \subset X$ be the inverse image of $U_{0, j}$. It suffices
to prove that $V_j \to U_{i, j}$ is a closed immersion for $i$
sufficiently large and we know that $V_j \to U_j$ is a closed immersion.
Thus we reduce to the following algebra fact: If $A = \colim A_i$ is a
directed colimit of $R$-algebras, $A \to B$ is a surjection of $R$-algebras,
and $B$ is a finitely generated $R$-algebra, then
$A_i \to B$ is surjective for $i$ sufficiently large.

\medskip\noindent
Proof of (2). Choose $0 \in I$. Choose a quasi-compact open
$X'_0 \subset X_0$ such that $Y \to X_0$ factors through $X'_0$.
After replacing $X_i$ by the inverse image of $X'_0$ for $i \geq 0$
we may assume all $X_i'$ are quasi-compact and quasi-separated.
Let $U \subset X$ be a quasi-compact open such that $Y \to X$ factors
through a closed immersion $Y \to U$ ($U$ exists as $Y$ is quasi-compact). By
Lemma \ref{lemma-descend-opens}
we may assume that $U = \lim U_i$ with $U_i \subset X_i$ quasi-compact
open. By part (1) we see that $Y \to U_i$ is a closed immersion for some
$i$. Thus (2) holds.
\end{proof}

\begin{lemma}
\label{lemma-eventually-separated}
Let $S$ be a scheme. Let $X = \lim X_i$ be a directed
limit of schemes over $S$ with affine transition morphisms.
Assume
\begin{enumerate}
\item $S$ quasi-separated,
\item $X_i$ quasi-compact and quasi-separated,
\item $X \to S$ separated.
\end{enumerate}
Then $X_i \to S$ is separated for all $i$ large enough.
\end{lemma}

\begin{proof}
Let $0 \in I$. Note that $I$ is nonempty as the limit is directed.
As $X_0$ is quasi-compact we can find finitely many
affine opens $U_1, \ldots, U_n \subset S$ such that
$X_0 \to S$ maps into $U_1 \cup \ldots \cup U_n$.
Denote $h_i : X_i \to S$ the structure morphism.
It suffices to check that for some $i \geq 0$ the morphisms
$h_i^{-1}(U_j) \to U_j$ are separated for $j = 1, \ldots,  n$.
Since $S$ is quasi-separated the morphisms $U_j \to S$ are quasi-compact.
Hence $h_i^{-1}(U_j)$ is quasi-compact and quasi-separated.
In this way we reduce to the case $S$ affine. In this case we
have to show that $X_i$ is separated and we know that $X$ is separated.
Thus the lemma follows from Lemma \ref{lemma-limit-separated}.
\end{proof}

\begin{lemma}
\label{lemma-eventually-affine}
Let $S$ be a scheme. Let $X = \lim X_i$ be a directed limit of schemes
over $S$ with affine transition morphisms. Assume
\begin{enumerate}
\item $S$ quasi-compact and quasi-separated,
\item $X_i$ quasi-compact and quasi-separated,
\item $X \to S$ affine.
\end{enumerate}
Then $X_i \to S$ is affine for $i$ large enough.
\end{lemma}

\begin{proof}
Choose a finite affine open covering $S = \bigcup_{j = 1, \ldots, n} V_j$.
Denote $f : X \to S$ and $f_i : X_i \to S$ the structure morphisms.
For each $j$ the scheme $f^{-1}(V_j) = \lim_i f_i^{-1}(V_j)$
is affine (as a finite morphism is affine by definition). Hence by
Lemma \ref{lemma-limit-affine} there exists an $i \in I$ such that
each $f_i^{-1}(V_j)$ is affine. In other words, $f_i : X_i \to S$ is
affine for $i$ large enough, see
Morphisms, Lemma \ref{morphisms-lemma-characterize-affine}.
\end{proof}

\begin{lemma}
\label{lemma-eventually-finite}
Let $S$ be a scheme. Let $X = \lim X_i$ be a directed limit of schemes
over $S$ with affine transition morphisms. Assume
\begin{enumerate}
\item $S$ quasi-compact and quasi-separated,
\item $X_i$ quasi-compact and quasi-separated,
\item the transition morphisms $X_{i'} \to X_i$ are finite,
\item $X_i \to S$ locally of finite type
\item $X \to S$ integral.
\end{enumerate}
Then $X_i \to S$ is finite for $i$ large enough.
\end{lemma}

\begin{proof}
By Lemma \ref{lemma-eventually-affine}
we may assume $X_i \to S$ is affine for all $i$.
Choose a finite affine open covering $S = \bigcup_{j = 1, \ldots, n} V_j$.
Denote $f : X \to S$ and $f_i : X_i \to S$ the structure morphisms.
It suffices to show that there exists an $i$ such that
$f_i^{-1}(V_j)$ is finite over $V_j$ for $j = 1, \ldots, m$
(Morphisms, Lemma \ref{morphisms-lemma-finite-local}).
Namely, for $i' \geq i$ the composition $X_{i'} \to X_i \to S$
will be finite as a composition of finite morphisms
(Morphisms, Lemma \ref{morphisms-lemma-composition-finite}).
This reduces us to the affine case: Let $R$ be a ring and
$A = \colim A_i$ with $R \to A$ integral and $A_i \to A_{i'}$
finite for all $i \leq i'$. Moreover $R \to A_i$ is of finite type
for all $i$. Goal: Show that $A_i$ is finite over $R$ for some $i$.
To prove this choose an $i \in I$ and pick generators
$x_1, \ldots, x_m \in A_i$ of $A_i$ as an $R$-algebra.
Since $A$ is integral over $R$ we can find monic polynomials
$P_j \in R[T]$ such that $P_j(x_j) = 0$ in $A$.
Thus there exists an $i' \geq i$ such that $P_j(x_j) = 0$ in $A_{i'}$
for $j = 1, \ldots, m$. Then the image $A'_i$ of $A_i$ in $A_{i'}$
is finite over $R$ by
Algebra, Lemma \ref{algebra-lemma-characterize-finite-in-terms-of-integral}.
Since $A'_i \subset A_{i'}$ is finite too we conclude
that $A_{i'}$ is finite over $R$ by
Algebra, Lemma \ref{algebra-lemma-finite-transitive}.
\end{proof}

\begin{lemma}
\label{lemma-eventually-closed-immersion}
Let $S$ be a scheme. Let $X = \lim X_i$ be a directed limit of schemes
over $S$ with affine transition morphisms. Assume
\begin{enumerate}
\item $S$ quasi-compact and quasi-separated,
\item $X_i$ quasi-compact and quasi-separated,
\item the transition morphisms $X_{i'} \to X_i$ are closed immersions,
\item $X_i \to S$ locally of finite type
\item $X \to S$ a closed immersion.
\end{enumerate}
Then $X_i \to S$ is a closed immersion for $i$ large enough.
\end{lemma}

\begin{proof}
By Lemma \ref{lemma-eventually-affine}
we may assume $X_i \to S$ is affine for all $i$.
Choose a finite affine open covering $S = \bigcup_{j = 1, \ldots, n} V_j$.
Denote $f : X \to S$ and $f_i : X_i \to S$ the structure morphisms.
It suffices to show that there exists an $i$ such that
$f_i^{-1}(V_j)$ is a closed subscheme of $V_j$ for $j = 1, \ldots, m$
(Morphisms, Lemma \ref{morphisms-lemma-closed-immersion}).
This reduces us to the affine case: Let $R$ be a ring and
$A = \colim A_i$ with $R \to A$ surjective and $A_i \to A_{i'}$
surjective for all $i \leq i'$. Moreover $R \to A_i$ is of finite type
for all $i$. Goal: Show that $R \to A_i$ is surjective for some $i$.
To prove this choose an $i \in I$ and pick generators
$x_1, \ldots, x_m \in A_i$ of $A_i$ as an $R$-algebra.
Since $R \to A$ is surjective we can find
$r_j \in R$ such that $r_j$ maps to $x_j$ in $A$.
Thus there exists an $i' \geq i$ such that $r_j$ maps to the image
of $x_j$ in $A_{i'}$ for $j = 1, \ldots, m$. Since $A_i \to A_{i'}$
is surjective this implies that $R \to A_{i'}$ is surjective.
\end{proof}










\section{Absolute Noetherian Approximation}
\label{section-approximation}

\noindent
A nice reference for this section is Appendix C of the article
by Thomason and Trobaugh \cite{TT}.
See Categories, Section \ref{categories-section-posets-limits}
for our conventions regarding directed systems.
We will use the existence result and properties of the limit
from Section \ref{section-limits} without further mention.

\begin{lemma}
\label{lemma-quasi-affine-finite-type-over-Z}
Let $W$ be a quasi-affine scheme of finite type over
$\mathbf{Z}$. Suppose $W \to \Spec(R)$ is an
open immersion into an affine scheme. There exists a
finite type $\mathbf{Z}$-algebra $A \subset R$
which induces an open immersion $W \to \Spec(A)$.
Moreover, $R$ is the directed colimit of such subalgebras.
\end{lemma}

\begin{proof}
Choose an affine open covering $W = \bigcup_{i = 1, \ldots, n} W_i$
such that each $W_i$ is a standard affine open in $\Spec(R)$.
In other words, if we write $W_i = \Spec(R_i)$
then $R_i = R_{f_i}$ for some $f_i \in R$.
Choose finitely many $x_{ij} \in R_i$ which generate
$R_i$ over $\mathbf{Z}$.
Pick an $N \gg 0$ such that each $f_i^Nx_{ij}$ comes from an
element of $R$, say $y_{ij} \in R$.
Set $A$ equal to the $\mathbf{Z}$-algebra generated by
the $f_i$ and the $y_{ij}$ and (optionally) finitely many
additional elements of $R$. Then $A$ works. Details omitted.
\end{proof}

\begin{lemma}
\label{lemma-diagram}
Suppose given a cartesian diagram of rings
$$
\xymatrix{
B \ar[r]_s & R \\
B'\ar[u] \ar[r] & R' \ar[u]_t
}
$$
Let $W' \subset \Spec(R')$ be an open of
the form $W' = D(f_1) \cup \ldots \cup D(f_n)$
such that $t(f_i) = s(g_i)$ for some $g_i \in B$
and $B_{g_i} \cong R_{s(g_i)}$. Then $B' \to R'$
induces an open immersion of $W'$ into $\Spec(B')$.
\end{lemma}

\begin{proof}
Set $h_i = (g_i, f_i) \in B'$. More on Algebra,
Lemma \ref{more-algebra-lemma-diagram-localize} shows that
$(B')_{h_i} \cong (R')_{f_i}$ as desired.
\end{proof}

\noindent
The following lemma is a precise statement of Noetherian
approximation.

\begin{lemma}
\label{lemma-approximate}
Let $S$ be a quasi-compact and quasi-separated scheme. Let $V \subset S$
be a quasi-compact open. Let $I$ be a directed set
and let $(V_i, f_{ii'})$ be an inverse system of schemes over $I$
with affine transition maps, with each $V_i$ of finite type over $\mathbf{Z}$,
and with $V = \lim V_i$. Then there exist
\begin{enumerate}
\item a directed set $J$,
\item an inverse system of schemes $(S_j, g_{jj'})$ over $J$,
\item an order preserving map $\alpha : J \to I$,
\item open subschemes $V'_j \subset S_j$, and
\item isomorphisms $V'_j \to V_{\alpha(j)}$
\end{enumerate}
such that
\begin{enumerate}
\item the transition morphisms $g_{jj'} : S_j \to S_{j'}$ are affine,
\item each $S_j$ is of finite type over $\mathbf{Z}$,
\item $g_{jj'}^{-1}(V_{j'}) = V_j$,
\item $S = \lim S_j$ and $V = \lim V_j$, and
\item the diagrams
$$
\vcenter{
\xymatrix{
V \ar[d] \ar[rd] \\
V'_j \ar[r] & V_{\alpha(j)}
}
}
\quad\text{and}\quad
\vcenter{
\xymatrix{
V_j \ar[r] \ar[d] & V_{\alpha(j)} \ar[d] \\
V_{j'} \ar[r] & V_{\alpha(j')}
}
}
$$
are commutative.
\end{enumerate}
\end{lemma}

\begin{proof}
Set $Z = S \setminus V$. Choose affine opens $U_1, \ldots, U_m \subset S$
such that $Z \subset \bigcup_{l = 1, \ldots, m} U_l$. Consider the opens
$$
V \subset V \cup U_1 \subset V \cup U_1 \cup U_2 \subset
\ldots \subset V \cup \bigcup\nolimits_{l = 1, \ldots, m} U_l = S
$$
If we can prove the lemma successively for each of the cases
$$
V \cup U_1 \cup \ldots \cup U_l
\subset
V \cup U_1 \cup \ldots \cup U_{l + 1}
$$
then the lemma will follow for $V \subset S$. In each case we are adding
one affine open. Thus we may assume
\begin{enumerate}
\item $S = U \cup V$,
\item $U$ affine open in $S$,
\item $V$ quasi-compact open in $S$, and
\item $V = \lim_i V_i$ with $(V_i, f_{ii'})$
an inverse system over a directed set $I$, each $f_{ii'}$
affine and each $V_i$ of finite type over $\mathbf{Z}$.
\end{enumerate}
Set $W = U \cap V$. As $S$ is quasi-separated, this is a quasi-compact open
of $V$. By Lemma \ref{lemma-descend-opens}
(and after shrinking $I$) we may assume that there exist
opens $W_i \subset V_i$ such that $f_{ij}^{-1}(W_j) = W_i$
and such that $f_i^{-1}(W_i) = W$. Since $W$ is a quasi-compact open
of $U$ it is quasi-affine. Hence we may assume (after shrinking $I$ again)
that $W_i$ is quasi-affine for all $i$, see
Lemma \ref{lemma-limit-quasi-affine}.

\medskip\noindent
Write $U = \Spec(B)$. Set $R = \Gamma(W, \mathcal{O}_W)$,
and $R_i = \Gamma(W_i, \mathcal{O}_{W_i})$.
By Lemma \ref{lemma-descend-section} we have $R = \colim_i R_i$.
Now we have the maps of rings
$$
\xymatrix{
B \ar[r]_s & R \\
& R_i \ar[u]_{t_i}
}
$$
We set $B_i = \{(b, r) \in B \times R_i \mid s(b) = t_i(t)\}$ so that we
have a cartesian diagram
$$
\xymatrix{
B \ar[r]_s & R \\
B_i \ar[u] \ar[r] & R_i \ar[u]_{t_i}
}
$$
for each $i$. The transition maps $R_i \to R_{i'}$ induce maps
$B_i \to B_{i'}$. It is clear that $B = \colim_i B_i$.
In the next paragraph we show that for all sufficiently large $i$
the composition $W_i \to \Spec(R_i) \to \Spec(B_i)$ is an open immersion.

\medskip\noindent
As $W$ is a quasi-compact open of $U = \Spec(B)$
we can find a finitely many elements $g_l \in B$, $l = 1, \ldots, m$
such that $D(g_l) \subset W$ and such that
$W = \bigcup_{l = 1, \ldots, m} D(g_l)$.
Note that this implies $D(g_l) = W_{s(g_l)}$ as open subsets of $U$,
where $W_{s(g_l)}$ denotes the largest open subset of $W$ on which
$s(g_l)$ is invertible. Hence
$$
B_{g_l} =
\Gamma(D(g_l), \mathcal{O}_U) =
\Gamma(W_{s(g_l)}, \mathcal{O}_W) = R_{s(g_l)},
$$
where the last equality is
Properties, Lemma \ref{properties-lemma-invert-f-sections}.
Since $W_{s(g_l)}$ is affine this also
implies that $D(s(g_l)) = W_{s(g_l)}$ as open subsets of $\Spec(R)$.
Since $R = \colim_i R_i$ we can (after shrinking $I$)
assume there exist $g_{l, i} \in R_i$ for all $i \in I$ such that
$s(g_l) = t_i(g_{l, i})$. Of course we choose the $g_{l, i}$
such that $g_{l, i}$ maps to $g_{l, i'}$ under the transition maps
$R_i \to R_{i'}$. Then, by Lemma \ref{lemma-descend-opens} we can
(after shrinking $I$ again)
assume the corresponding opens $D(g_{l, i}) \subset \Spec(R_i)$
are contained in $W_i$ for $l = 1, \ldots, m$ and cover $W_i$.
We conclude that the morphism $W_i \to \Spec(R_i) \to \Spec(B_i)$
is an open immersion, see Lemma \ref{lemma-diagram}

\medskip\noindent
By Lemma \ref{lemma-quasi-affine-finite-type-over-Z}
we can write $B_i$ as a directed colimit of subalgebras
$A_{i, p} \subset B_i$, $p \in P_i$ each
of finite type over $\mathbf{Z}$ and such that $W_i$ is
identified with an open subscheme of $\Spec(A_{i, p})$.
Let $S_{i, p}$ be the scheme obtained by glueing
$V_i$ and $\Spec(A_{i, p})$ along the open $W_i$, see
Schemes, Section \ref{schemes-section-glueing-schemes}.
Here is the resulting commutative diagram of schemes:
$$
\xymatrix{
& & V \ar[lld] \ar[d] & W \ar[l] \ar[lld] \ar[d] \\
V_i \ar[d] & W_i \ar[l] \ar[d] & S \ar[lld] & U \ar[lld] \ar[l] \\
S_{i, p} & \Spec(A_{i, p}) \ar[l]
}
$$
The morphism $S \to S_{i, p}$ arises because the upper right
square is a pushout in the category of schemes.
Note that $S_{i, p}$ is of finite type over $\mathbf{Z}$ since
it has a finite affine open covering whose members are
spectra of finite type $\mathbf{Z}$-algebras.
We define a preorder on $J = \coprod_{i \in I} P_i$
by the rule $(i', p') \geq (i, p)$ if and only if
$i' \geq i$ and the map $B_i \to B_{i'}$ maps $A_{i, p}$ into
$A_{i', p'}$. This is exactly the condition needed to
define a morphism $S_{i', p'} \to S_{i, p}$: namely make a commutative
diagram as above using the transition morphisms $V_{i'} \to V_i$
and $W_{i'} \to W_i$ and
the morphism $\Spec(A_{i', p'}) \to \Spec(A_{i, p})$ induced
by the ring map $A_{i, p} \to A_{i', p'}$. The relevant commutativities
have been built into the constructions.
We claim that $S$ is the directed limit of the schemes $S_{i, p}$.
Since by construction the schemes $V_i$ have limit $V$ this boils
down to the fact that $B$ is the limit of the rings $A_{i, p}$
which is true by construction. The map $\alpha : J \to I$ is given
by the rule $j = (i, p) \mapsto i$. The open subscheme $V'_j$ is
just the image of $V_i \to S_{i, p}$ above. The commutativity of
the diagrams in (5) is clear from the construction.
This finishes the proof of the lemma.
\end{proof}

\begin{proposition}
\label{proposition-approximate}
Let $S$ be a quasi-compact and quasi-separated scheme.
There exist a directed set $I$
and an inverse system of schemes $(S_i, f_{ii'})$ over $I$
such that
\begin{enumerate}
\item the transition morphisms $f_{ii'}$ are affine
\item each $S_i$ is of finite type over $\mathbf{Z}$, and
\item $S = \lim_i S_i$.
\end{enumerate}
\end{proposition}

\begin{proof}
This is a special case of Lemma \ref{lemma-approximate}
with $V = \emptyset$.
\end{proof}






\section{Limits and morphisms of finite presentation}
\label{section-finite-presentation}

\noindent
The following is a generalization of
Algebra, Lemma \ref{algebra-lemma-characterize-finite-presentation}.

\begin{proposition}
\label{proposition-characterize-locally-finite-presentation}
\begin{reference}
\cite[IV, Proposition 8.14.2]{EGA}
\end{reference}
Let $f : X \to S$ be a morphism of schemes.
The following are equivalent:
\begin{enumerate}
\item The morphism $f$ is locally of finite presentation.
\item For any directed set $I$, and any
inverse system $(T_i, f_{ii'})$ of $S$-schemes over $I$
with each $T_i$ affine, we have
$$
\Mor_S(\lim_i T_i, X) =
\colim_i \Mor_S(T_i, X)
$$
\item For any directed set $I$, and any
inverse system $(T_i, f_{ii'})$ of $S$-schemes over $I$
with each $f_{ii'}$ affine and every $T_i$ quasi-compact and
quasi-separated as a scheme, we have
$$
\Mor_S(\lim_i T_i, X) =
\colim_i \Mor_S(T_i, X)
$$
\end{enumerate}
\end{proposition}

\begin{proof}
It is clear that (3) implies (2).

\medskip\noindent
Let us prove that (2) implies (1). Assume (2).
Choose any affine opens $U \subset X$ and $V \subset S$ such that
$f(U) \subset V$. We have to show that
$\mathcal{O}_S(V) \to \mathcal{O}_X(U)$ is of finite presentation.
Let $(A_i, \varphi_{ii'})$ be a directed system of
$\mathcal{O}_S(V)$-algebras. Set $A = \colim_i A_i$.
According to
Algebra, Lemma \ref{algebra-lemma-characterize-finite-presentation}
we have to show that
$$
\Hom_{\mathcal{O}_S(V)}(\mathcal{O}_X(U), A) =
\colim_i \Hom_{\mathcal{O}_S(V)}(\mathcal{O}_X(U), A_i)
$$
Consider the schemes $T_i = \Spec(A_i)$. They
form an inverse system of $V$-schemes over $I$
with transition morphisms $f_{ii'} : T_i \to T_{i'}$
induced by the $\mathcal{O}_S(V)$-algebra maps $\varphi_{i'i}$.
Set $T := \Spec(A) = \lim_i T_i$.
The formula above becomes in terms of morphism sets of schemes
$$
\Mor_V(\lim_i T_i, U) =
\colim_i \Mor_V(T_i, U).
$$
We first observe that
$\Mor_V(T_i, U) = \Mor_S(T_i, U)$
and
$\Mor_V(T, U) = \Mor_S(T, U)$.
Hence we have to show that
$$
\Mor_S(\lim_i T_i, U) =
\colim_i \Mor_S(T_i, U)
$$
and we are given that
$$
\Mor_S(\lim_i T_i, X) =
\colim_i \Mor_S(T_i, X).
$$
Hence it suffices to prove that given a morphism $g_i : T_i \to X$ over $S$
such that the composition $T \to T_i \to X$ ends up in $U$ there exists some
$i' \geq i$ such that the composition $g_{i'} : T_{i'} \to T_i \to X$ ends up
in $U$. Denote $Z_{i'} = g_{i'}^{-1}(X \setminus U)$.
Assume each $Z_{i'}$ is nonempty
to get a contradiction. By Lemma \ref{lemma-limit-closed-nonempty}
there exists a point $t$ of $T$ which is mapped into $Z_{i'}$ for all
$i' \geq i$. Such a point is not mapped into $U$. A contradiction.

\medskip\noindent
Finally, let us prove that (1) implies (3). Assume (1). Let an inverse directed
system $(T_i, f_{ii'})$ of $S$-schemes be given. Assume the morphisms $f_{ii'}$
are affine and each $T_i$ is quasi-compact and quasi-separated as a scheme. Let
$T = \lim_i T_i$. Denote $f_i : T \to T_i$ the projection morphisms.
We have to show:
\begin{enumerate}
\item[(a)] Given morphisms $g_i, g'_i : T_i \to X$ over $S$ such that
$g_i \circ f_i = g'_i \circ f_i$, then there exists an $i' \geq i$
such that $g_i \circ f_{i'i} = g'_i \circ f_{i'i}$.
\item[(b)] Given any morphism
$g : T \to X$ over $S$ there exists an $i \in I$ and a morphism
$g_i : T_i \to X$ such that $g = f_i \circ g_i$.
\end{enumerate}

\noindent
First let us prove the uniqueness part (a). Let $g_i, g'_i : T_i \to X$ be
morphisms such that $g_i \circ f_i = g'_i \circ f_i$. For any $i' \geq i$
we set $g_{i'} = g_i \circ f_{i'i}$ and $g'_{i'} = g'_i \circ f_{i'i}$.
We also set $g = g_i \circ f_i = g'_i \circ f_i$.
Consider the morphism
$(g_i, g'_i) : T_i \to X \times_S X$. Set
$$
W =
\bigcup\nolimits_{U \subset X\text{ affine open},
V \subset S\text{ affine open}, f(U) \subset V}
U \times_V U.
$$
This is an open in $X \times_S X$, with the property that the morphism
$\Delta_{X/S}$ factors through a closed immersion into $W$, see the proof
of Schemes, Lemma \ref{schemes-lemma-diagonal-immersion}.
Note that the composition
$(g_i, g'_i) \circ f_i : T \to X \times_S X$ is a morphism into $W$
because it factors through the diagonal by assumption.
Set $Z_{i'} = (g_{i'}, g'_{i'})^{-1}(X \times_S X \setminus W)$.
If each $Z_{i'}$ is nonempty, then by Lemma \ref{lemma-limit-closed-nonempty}
there exists a point $t \in T$ which maps to $Z_{i'}$ for all
$i' \geq i$. This is a contradiction with the fact that $T$ maps into $W$.
Hence we may increase $i$ and assume that $(g_i, g'_i) : T_i \to X \times_S X$
is a morphism into $W$. By construction of $W$, and since $T_i$ is
quasi-compact we can find a finite affine open covering
$T_i = T_{1, i} \cup \ldots \cup T_{n, i}$ such that
$(g_i, g'_i)|_{T_{j, i}}$ is a morphism into $U \times_V U$ for
some pair $(U, V)$ as in the definition of $W$ above.
Since it suffices to prove that $g_{i'}$ and $g'_{i'}$ agree
on each of the $f_{i'i}^{-1}(T_{j, i})$ this reduces us to the affine case.
The affine case follows from
Algebra, Lemma \ref{algebra-lemma-characterize-finite-presentation}
and the fact that the ring map
$\mathcal{O}_S(V) \to \mathcal{O}_X(U)$ is of finite presentation
(see Morphisms,
Lemma \ref{morphisms-lemma-locally-finite-presentation-characterize}).

\medskip\noindent
Finally, we prove the existence part (b).
Let $g : T \to X$ be a morphism of schemes over $S$.
We can find a finite affine open covering
$T = W_1 \cup \ldots \cup W_n$ such that for
each $j \in \{1, \ldots, n\}$ there exist affine opens
$U_j \subset X$ and $V_j \subset S$ with $f(U_j) \subset V_j$
and $g(W_j) \subset U_j$. By Lemmas \ref{lemma-descend-opens}
and \ref{lemma-limit-affine}
(after possibly shrinking $I$) we may assume that
there exist affine open coverings $T_i = W_{1, i} \cup \ldots \cup W_{n, i}$
compatible with transition maps such that $W_j = \lim_i W_{j, i}$.
We apply Algebra, Lemma \ref{algebra-lemma-characterize-finite-presentation}
to the rings corresponding to the affine schemes $U_j$, $V_j$, $W_{j, i}$ and
$W_j$ using that $\mathcal{O}_S(V_j) \to \mathcal{O}_X(U_j)$ is of finite
presentation (see Morphisms,
Lemma \ref{morphisms-lemma-locally-finite-presentation-characterize}).
Thus we can find for each $j$ an index $i_j \in I$ and a morphism
$g_{j, i_j} : W_{j, i_j} \to X$ such that
$g_{j, i_j} \circ f_i|_{W_j} : W_j \to W_{j, i} \to X$
equals $g|_{W_j}$. By part (a) proved above, using the quasi-compactness of
$W_{j_1, i} \cap W_{j_2, i}$ which follows as $T_i$ is quasi-separated,
we can find an index $i' \in I$ larger than all $i_j$ such that
$$
g_{j_1, i_{j_1}} \circ f_{i'i_{j_1}}|_{W_{j_1, i'} \cap W_{j_2, i'}} =
g_{j_2, i_{j_2}} \circ f_{i'i_{j_2}}|_{W_{j_1, i'} \cap W_{j_2, i'}}
$$
for all $j_1, j_2 \in \{1, \ldots, n\}$. Hence the morphisms
$g_{j, i_j} \circ f_{i'i_j}|_{W_{j, i'}}$ glue to given the
desired morphism $T_{i'} \to X$.
\end{proof}

\begin{remark}
\label{remark-limit-preserving}
Let $S$ be a scheme. Let us say that a functor
$F : (\Sch/S)^{opp} \to \textit{Sets}$ is
{\it limit preserving} if for every directed inverse system
$\{T_i\}_{i \in I}$ of affine schemes with limit $T$ we have
$F(T) = \colim_i F(T_i)$. Let $X$ be a scheme over $S$, and
let $h_X : (\Sch/S)^{opp} \to \textit{Sets}$ be its
functor of points, see
Schemes, Section \ref{schemes-section-representable}.
In this terminology
Proposition \ref{proposition-characterize-locally-finite-presentation}
says that a scheme $X$ is locally of finite presentation over
$S$ if and only if $h_X$ is limit preserving.
\end{remark}

\begin{lemma}
\label{lemma-surjection-is-enough}
Let $f : X \to S$ be a morphism of schemes. If for every directed limit
$T = \lim_{i \in I} T_i$ of affine schemes over $S$ the map
$$
\colim \Mor_S(T_i, X) \longrightarrow \Mor_S(T, X)
$$
is surjective, then $f$ is locally of finite presentation.
In other words, in
Proposition \ref{proposition-characterize-locally-finite-presentation}
parts (2) and (3) it suffices to check surjectivity of the map.
\end{lemma}

\begin{proof}
The proof is exactly the same as the proof of the implication
``(2) implies (1)'' in
Proposition \ref{proposition-characterize-locally-finite-presentation}.
Choose any affine opens $U \subset X$ and $V \subset S$ such that
$f(U) \subset V$. We have to show that
$\mathcal{O}_S(V) \to \mathcal{O}_X(U)$ is of finite presentation.
Let $(A_i, \varphi_{ii'})$ be a directed system of
$\mathcal{O}_S(V)$-algebras. Set $A = \colim_i A_i$.
According to
Algebra, Lemma \ref{algebra-lemma-characterize-finite-presentation}
it suffices to show that
$$
\colim_i \Hom_{\mathcal{O}_S(V)}(\mathcal{O}_X(U), A_i) \to
\Hom_{\mathcal{O}_S(V)}(\mathcal{O}_X(U), A)
$$
is surjective. Consider the schemes $T_i = \Spec(A_i)$. They
form an inverse system of $V$-schemes over $I$
with transition morphisms $f_{ii'} : T_i \to T_{i'}$
induced by the $\mathcal{O}_S(V)$-algebra maps $\varphi_{i'i}$.
Set $T := \Spec(A) = \lim_i T_i$.
The formula above becomes in terms of morphism sets of schemes
$$
\colim_i \Mor_V(T_i, U) \to \Mor_V(\lim_i T_i, U)
$$
We first observe that
$\Mor_V(T_i, U) = \Mor_S(T_i, U)$
and
$\Mor_V(T, U) = \Mor_S(T, U)$.
Hence we have to show that
$$
\colim_i \Mor_S(T_i, U) \to
\Mor_S(\lim_i T_i, U)
$$
is surjective and we are given that
$$
\colim_i \Mor_S(T_i, X) \to
\Mor_S(\lim_i T_i, X)
$$
is surjective.
Hence it suffices to prove that given a morphism $g_i : T_i \to X$ over $S$
such that the composition $T \to T_i \to X$ ends up in $U$ there exists some
$i' \geq i$ such that the composition $g_{i'} : T_{i'} \to T_i \to X$ ends up
in $U$. Denote $Z_{i'} = g_{i'}^{-1}(X \setminus U)$.
Assume each $Z_{i'}$ is nonempty
to get a contradiction. By Lemma \ref{lemma-limit-closed-nonempty}
there exists a point $t$ of $T$ which is mapped into $Z_{i'}$ for all
$i' \geq i$. Such a point is not mapped into $U$. A contradiction.
\end{proof}








\section{Relative approximation}
\label{section-relative-approximation}

\noindent
The title of this section refers to results of the following type.

\begin{lemma}
\label{lemma-relative-approximation}
Let $f : X \to S$ be a morphism of schemes. Assume that
\begin{enumerate}
\item $X$ is quasi-compact and quasi-separated, and
\item $S$ is quasi-separated.
\end{enumerate}
Then $X = \lim X_i$ is a limit of a directed system of schemes
$X_i$ of finite presentation over $S$ with affine transition morphisms
over $S$.
\end{lemma}

\begin{proof}
Since $f(X)$ is quasi-compact we may replace $S$ by a quasi-compact
open containing $f(X)$. Hence we may assume $S$ is quasi-compact as well.
Write $X = \lim X_a$ and $S = \lim S_b$ as in
Proposition \ref{proposition-approximate}, i.e., with $X_a$ and $S_b$
of finite type over $\mathbf{Z}$ and with affine transition morphisms.
By Proposition \ref{proposition-characterize-locally-finite-presentation}
we find that for each $b$ there exists an $a$ and a morphism
$f_{a, b} : X_a \to S_b$ making the diagram
$$
\xymatrix{
X \ar[d] \ar[r] & S \ar[d] \\
X_a \ar[r] & S_b
}
$$
commute. Moreover the same proposition implies that, given a second
triple $(a', b', f_{a', b'})$, there exists an $a'' \geq a'$ such that
the compositions $X_{a''} \to X_a \to X_b$ and
$X_{a''} \to X_{a'} \to X_{b'} \to X_b$ are equal.
Consider the set of triples $(a, b, f_{a, b})$ endowed with the preorder
$$
(a, b, f_{a, b}) \geq (a', b', f_{a', b'})
\Leftrightarrow
a \geq a',\ b' \geq b,\text{ and }
f_{a', b'} \circ h_{a, a'} = g_{b', b} \circ f_{a, b}
$$
where $h_{a, a'} : X_a \to X_{a'}$ and $g_{b', b} : S_{b'} \to S_b$
are the transition morphisms. The remarks above show that this system
is directed. It follows formally from the equalities
$X = \lim X_a$ and $S = \lim S_b$ that
$$
X = \lim_{(a, b, f_{a, b})} X_a \times_{f_{a, b}, S_b} S.
$$
where the limit is over our directed system above. The transition morphisms
$X_a \times_{S_b} S \to X_{a'} \times_{S_{b'}} S$ are affine as
the composition
$$
X_a \times_{S_b} S \to X_a \times_{S_{b'}} S \to X_{a'} \times_{S_{b'}} S
$$
where the first morphism is a closed immersion (by
Schemes, Lemma \ref{schemes-lemma-fibre-product-after-map})
and the second is a base change of an affine morphism
(Morphisms, Lemma \ref{morphisms-lemma-base-change-affine})
and the composition of affine morphisms is affine
(Morphisms, Lemma \ref{morphisms-lemma-composition-affine}).
The morphisms $f_{a, b}$ are of finite presentation
(Morphisms, Lemmas
\ref{morphisms-lemma-noetherian-finite-type-finite-presentation} and
\ref{morphisms-lemma-finite-presentation-permanence})
and hence the base changes $X_a \times_{f_{a, b}, S_b} S \to S$
are of finite presentation
(Morphisms, Lemma \ref{morphisms-lemma-base-change-finite-presentation}).
\end{proof}

\begin{lemma}
\label{lemma-integral-limit-finite-and-finite-presentation}
Let $X \to S$ be an integral morphism with $S$ quasi-compact and
quasi-separated. Then $X = \lim X_i$ with $X_i \to S$ finite and
of finite presentation.
\end{lemma}

\begin{proof}
Consider the sheaf $\mathcal{A} = f_*\mathcal{O}_X$.
This is a quasi-coherent sheaf of $\mathcal{O}_S$-algebras, see
Schemes, Lemma \ref{schemes-lemma-push-forward-quasi-coherent}.
Combining
Properties, Lemma
\ref{properties-lemma-integral-algebra-directed-colimit-finite}
we can write $\mathcal{A} = \colim_i \mathcal{A}_i$ as a filtered
colimit of finite and finitely presented $\mathcal{O}_S$-algebras.
Then
$$
X_i = \underline{\Spec}_S(\mathcal{A}_i)
\longrightarrow
S
$$
is a finite and finitely presented morphism of schemes. By construction
$X = \lim_i X_i$ which proves the lemma.
\end{proof}








\section{Descending properties of morphisms}
\label{section-descent-of-properties}

\noindent
This section is the analogue of
Section \ref{section-descent}
for properties of morphisms over $S$.
We will work in the following situation.

\begin{situation}
\label{situation-descent-property}
Let $S = \lim S_i$ be a limit of a directed system of schemes
with affine transition morphisms
(Lemma \ref{lemma-directed-inverse-system-has-limit}).
Let $0 \in I$ and let $f_0 : X_0 \to Y_0$ be a morphism of schemes over $S_0$.
Assume $S_0$, $X_0$, $Y_0$ are quasi-compact and quasi-separated.
Let $f_i : X_i \to Y_i$ be the base change of $f_0$ to $S_i$ and
let $f : X \to Y$ be the base change of $f_0$ to $S$.
\end{situation}

\begin{lemma}
\label{lemma-descend-affine-finite-presentation}
Notation and assumptions as in Situation \ref{situation-descent-property}.
If $f$ is affine, then there exists an index $i \geq 0$
such that $f_i$ is affine.
\end{lemma}

\begin{proof}
Let $Y_0 = \bigcup_{j = 1, \ldots, m} V_{j, 0}$ be a finite affine
open covering. Set $U_{j, 0} = f_0^{-1}(V_{j, 0})$. For $i \geq 0$
we denote $V_{j, i}$ the inverse image of $V_{j, 0}$ in $Y_i$ and
$U_{j, i} = f_i^{-1}(V_{j, i})$. Similarly we have
$U_j = f^{-1}(V_j)$. Then $U_j = \lim_{i \geq 0} U_{j, i}$
(see Lemma \ref{lemma-directed-inverse-system-has-limit}).
Since $U_j$ is affine by assumption we see that
each $U_{j, i}$ is affine for $i$ large enough, see
Lemma \ref{lemma-limit-affine}. As there are finitely many $j$ we
can pick an $i$ which works for all $j$. Thus $f_i$ is
affine for $i$ large enough, see
Morphisms, Lemma \ref{morphisms-lemma-characterize-affine}.
\end{proof}

\begin{lemma}
\label{lemma-descend-finite-finite-presentation}
Notation and assumptions as in Situation \ref{situation-descent-property}.
If
\begin{enumerate}
\item $f$ is a finite morphism, and
\item $f_0$ is locally of finite type,
\end{enumerate}
then there exists an $i \geq 0$ such that $f_i$ is finite.
\end{lemma}

\begin{proof}
A finite morphism is affine, see
Morphisms, Definition \ref{morphisms-definition-integral}.
Hence by Lemma \ref{lemma-descend-affine-finite-presentation} above
after increasing $0$ we may assume that $f_0$ is affine.
By writing $Y_0$ as a finite union of affines we reduce to proving
the result when $X_0$ and $Y_0$ are affine and map
into a common affine $W \subset S_0$. The corresponding algebra
statement follows from Algebra, Lemma \ref{algebra-lemma-colimit-finite}.
\end{proof}

\begin{lemma}
\label{lemma-descend-unramified}
Notation and assumptions as in Situation \ref{situation-descent-property}.
If
\begin{enumerate}
\item $f$ is unramified, and
\item $f_0$ is locally of finite type,
\end{enumerate}
then there exists an $i \geq 0$ such that $f_i$ is unramified.
\end{lemma}

\begin{proof}
Choose a finite affine open covering
$Y_0 = \bigcup_{j = 1, \ldots, m} Y_{j, 0}$
such that each $Y_{j, 0}$ maps into an affine open
$S_{j, 0} \subset S_0$. For each $j$ let
$f_0^{-1}Y_{j, 0} = \bigcup_{k = 1, \ldots, n_j} X_{k, 0}$ be a
finite affine open covering. Since the property of being unramified is
local we see that it suffices to prove the lemma for the morphisms
of affines $X_{k, i} \to Y_{j, i} \to S_{j, i}$ which are the base
changes of $X_{k, 0} \to Y_{j, 0} \to S_{j, 0}$ to $S_i$.
Thus we reduce to the case that $X_0, Y_0, S_0$ are affine

\medskip\noindent
In the affine case we reduce to the following algebra result.
Suppose that $R = \colim_{i \in I} R_i$. For some $0 \in I$
suppose given an $R_0$-algebra map $A_i \to B_i$ of finite type.
If $R \otimes_{R_0} A_0 \to R \otimes_{R_0} B_0$ is unramified, then
for some $i \geq 0$ the map
$R_i \otimes_{R_0} A_0 \to R_i \otimes_{R_0} B_0$ is unramified.
This follows from Algebra,
Lemma \ref{algebra-lemma-colimit-unramified}.
\end{proof}

\begin{lemma}
\label{lemma-descend-closed-immersion-finite-presentation}
Notation and assumptions as in Situation \ref{situation-descent-property}.
If
\begin{enumerate}
\item $f$ is a closed immersion, and
\item $f_0$ is locally of finite type,
\end{enumerate}
then there exists an $i \geq 0$ such that $f_i$ is a closed immersion.
\end{lemma}

\begin{proof}
A closed immersion is affine, see
Morphisms, Lemma \ref{morphisms-lemma-closed-immersion-affine}.
Hence by Lemma \ref{lemma-descend-affine-finite-presentation} above
after increasing $0$ we may assume that $f_0$ is affine.
By writing $Y_0$ as a finite union of affines we reduce to proving
the result when $X_0$ and $Y_0$ are affine and map
into a common affine $W \subset S_0$. The corresponding algebra
statement is a consequence of
Algebra, Lemma \ref{algebra-lemma-colimit-surjective}.
\end{proof}

\begin{lemma}
\label{lemma-descend-separated-finite-presentation}
Notation and assumptions as in Situation \ref{situation-descent-property}.
If $f$ is separated, then $f_i$ is separated for some $i \geq 0$.
\end{lemma}

\begin{proof}
Apply Lemma \ref{lemma-descend-closed-immersion-finite-presentation}
to the diagonal morphism $\Delta_{X_0/S_0} : X_0 \to X_0 \times_{S_0} X_0$.
(This is permissible as diagonal morphisms are locally of finite type
and the fibre product $X_0 \times_{S_0} X_0$ is quasi-compact and
quasi-separated, see
Schemes, Lemma \ref{schemes-lemma-diagonal-immersion},
Morphisms, Lemma \ref{morphisms-lemma-immersion-locally-finite-type}, and
Schemes, Remark \ref{schemes-remark-quasi-compact-and-quasi-separated}.
\end{proof}

\begin{lemma}
\label{lemma-descend-flat-finite-presentation}
Notation and assumptions as in Situation \ref{situation-descent-property}.
If
\begin{enumerate}
\item $f$ is flat,
\item $f_0$ is locally of finite presentation,
\end{enumerate}
then $f_i$ is flat for some $i \geq 0$.
\end{lemma}

\begin{proof}
Choose a finite affine open covering
$Y_0 = \bigcup_{j = 1, \ldots, m} Y_{j, 0}$
such that each $Y_{j, 0}$ maps into an affine open
$S_{j, 0} \subset S_0$. For each $j$ let
$f_0^{-1}Y_{j, 0} = \bigcup_{k = 1, \ldots, n_j} X_{k, 0}$ be a
finite affine open covering. Since the property of being flat is
local we see that it suffices to prove the lemma for the morphisms
of affines $X_{k, i} \to Y_{j, i} \to S_{j, i}$ which are the base
changes of $X_{k, 0} \to Y_{j, 0} \to S_{j, 0}$ to $S_i$.
Thus we reduce to the case that $X_0, Y_0, S_0$ are affine

\medskip\noindent
In the affine case we reduce to the following algebra result.
Suppose that $R = \colim_{i \in I} R_i$. For some $0 \in I$
suppose given an $R_0$-algebra map $A_i \to B_i$ of finite presentation.
If $R \otimes_{R_0} A_0 \to R \otimes_{R_0} B_0$ is flat, then
for some $i \geq 0$ the map
$R_i \otimes_{R_0} A_0 \to R_i \otimes_{R_0} B_0$ is flat.
This follows from Algebra,
Lemma \ref{algebra-lemma-flat-finite-presentation-limit-flat} part (3).
\end{proof}

\begin{lemma}
\label{lemma-descend-finite-locally-free}
Notation and assumptions as in Situation \ref{situation-descent-property}.
If
\begin{enumerate}
\item $f$ is finite locally free (of degree $d$),
\item $f_0$ is locally of finite presentation,
\end{enumerate}
then $f_i$ is finite locally free (of degree $d$) for some $i \geq 0$.
\end{lemma}

\begin{proof}
By
Lemmas \ref{lemma-descend-flat-finite-presentation} and
\ref{lemma-descend-finite-finite-presentation}
we find an $i$ such that $f_i$ is flat and finite.
On the other hand, $f_i$ is locally of finite presentation.
Hence $f_i$ is finite locally free by
Morphisms, Lemma \ref{morphisms-lemma-finite-flat}.
If moreover $f$ is finite locally free of degree $d$,
then the image of $Y \to Y_i$ is contained in the
open and closed locus $W_d \subset Y_i$ over which $f_i$ has degree
$d$. By
Lemma \ref{lemma-limit-contained-in-constructible}
we see that for some $i' \geq i$ the image of $Y_{i'} \to Y_i$
is contained in $W_d$.
Then $f_{i'}$ will be finite locally free of degree $d$.
\end{proof}

\begin{lemma}
\label{lemma-descend-smooth}
Notation and assumptions as in Situation \ref{situation-descent-property}.
If
\begin{enumerate}
\item $f$ is smooth,
\item $f_0$ is locally of finite presentation,
\end{enumerate}
then $f_i$ is smooth for some $i \geq 0$.
\end{lemma}

\begin{proof}
Being smooth is local on the source and the target (Morphisms,
Lemma \ref{morphisms-lemma-smooth-characterize}) hence we may assume
$S_0, X_0, Y_0$ affine (details omitted). The corresponding algebra fact is
Algebra, Lemma \ref{algebra-lemma-colimit-smooth}.
\end{proof}

\begin{lemma}
\label{lemma-descend-etale}
Notation and assumptions as in Situation \ref{situation-descent-property}.
If
\begin{enumerate}
\item $f$ is \'etale,
\item $f_0$ is locally of finite presentation,
\end{enumerate}
then $f_i$ is \'etale for some $i \geq 0$.
\end{lemma}

\begin{proof}
Being \'etale is local on the source and the target (Morphisms,
Lemma \ref{morphisms-lemma-etale-characterize}) hence we may assume
$S_0, X_0, Y_0$ affine (details omitted). The corresponding algebra fact is
Algebra, Lemma \ref{algebra-lemma-colimit-etale}.
\end{proof}

\begin{lemma}
\label{lemma-descend-isomorphism}
Notation and assumptions as in Situation \ref{situation-descent-property}.
If
\begin{enumerate}
\item $f$ is an isomorphism, and
\item $f_0$ is locally of finite presentation,
\end{enumerate}
then $f_i$ is an isomorphism for some $i \geq 0$.
\end{lemma}

\begin{proof}
By Lemmas \ref{lemma-descend-etale} and
\ref{lemma-descend-closed-immersion-finite-presentation}
we can find an $i$ such that $f_i$ is flat and a closed immersion.
Then $f_i$ identifies $X_i$ with an open and closed subscheme of
$Y_i$, see Morphisms, Lemma
\ref{morphisms-lemma-flat-closed-immersions-finite-presentation}.
By assumption the image of $Y \to Y_i$ maps into $f_i(X_i)$.
Thus by Lemma \ref{lemma-limit-contained-in-constructible}
we find that $Y_{i'}$ maps into $f_i(X_i)$ for some $i' \geq i$.
It follows that $X_{i'} \to Y_{i'}$ is surjective and we win.
\end{proof}

\begin{lemma}
\label{lemma-descend-monomorphism}
Notation and assumptions as in Situation \ref{situation-descent-property}.
If
\begin{enumerate}
\item $f$ is a monomorphism, and
\item $f_0$ is locally of finite type,
\end{enumerate}
then $f_i$ is a monomorphism for some $i \geq 0$.
\end{lemma}

\begin{proof}
Recall that a morphism of schemes $V \to W$ is a monomorphism if and
only if the diagonal $V \to V \times_W V$ is an isomorphism
(Schemes, Lemma \ref{schemes-lemma-monomorphism}).
The morphism $X_0 \to X_0 \times_{Y_0} X_0$ is locally of finite
presentation by
Morphisms, Lemma \ref{morphisms-lemma-diagonal-morphism-finite-type}.
Since $X_0 \times_{Y_0} X_0$ is quasi-compact and quasi-separated
(Schemes, Remark \ref{schemes-remark-quasi-compact-and-quasi-separated})
we conclude from
Lemma \ref{lemma-descend-isomorphism}
that $\Delta_i : X_i \to X_i \times_{Y_i} X_i$ is an isomorphism for
some $i \geq 0$. For this $i$ the morphism $f_i$ is a monomorphism.
\end{proof}

\begin{lemma}
\label{lemma-descend-surjective}
Notation and assumptions as in Situation \ref{situation-descent-property}.
If
\begin{enumerate}
\item $f$ is surjective, and
\item $f_0$ is locally of finite presentation,
\end{enumerate}
then there exists an $i \geq 0$ such that $f_i$ is surjective.
\end{lemma}

\begin{proof}
The morphism $f_0$ is of finite presentation.
Hence $E = f_0(X_0)$ is a constructible subset of $Y_0$, see
Morphisms, Lemma \ref{morphisms-lemma-chevalley}.
Since $f_i$ is the base change of $f_0$ by
$Y_i \to Y_0$ we see that the image of $f_i$ is the
inverse image of $E$ in $Y_i$. Moreover, we know that
$Y \to Y_0$ maps into $E$. Hence we win by
Lemma \ref{lemma-limit-contained-in-constructible}.
\end{proof}

\begin{lemma}
\label{lemma-descend-syntomic}
Notation and assumptions as in Situation \ref{situation-descent-property}.
If
\begin{enumerate}
\item $f$ is syntomic, and
\item $f_0$ is locally of finite presentation,
\end{enumerate}
then there exists an $i \geq 0$ such that $f_i$ is syntomic.
\end{lemma}

\begin{proof}
Choose a finite affine open covering
$Y_0 = \bigcup_{j = 1, \ldots, m} Y_{j, 0}$
such that each $Y_{j, 0}$ maps into an affine open
$S_{j, 0} \subset S_0$. For each $j$ let
$f_0^{-1}Y_{j, 0} = \bigcup_{k = 1, \ldots, n_j} X_{k, 0}$ be a
finite affine open covering. Since the property of being syntomic is
local we see that it suffices to prove the lemma for the morphisms
of affines $X_{k, i} \to Y_{j, i} \to S_{j, i}$ which are the base
changes of $X_{k, 0} \to Y_{j, 0} \to S_{j, 0}$ to $S_i$.
Thus we reduce to the case that $X_0, Y_0, S_0$ are affine

\medskip\noindent
In the affine case we reduce to the following algebra result.
Suppose that $R = \colim_{i \in I} R_i$. For some $0 \in I$
suppose given an $R_0$-algebra map $A_i \to B_i$ of finite presentation.
If $R \otimes_{R_0} A_0 \to R \otimes_{R_0} B_0$ is syntomic, then
for some $i \geq 0$ the map
$R_i \otimes_{R_0} A_0 \to R_i \otimes_{R_0} B_0$ is syntomic.
This follows from Algebra,
Lemma \ref{algebra-lemma-colimit-syntomic}.
\end{proof}





\section{Finite type closed in finite presentation}
\label{section-finite-type-closed-in-finite-presentation}

\noindent
A result of this type is \cite[Satz 2.10]{Kiehl}. Another
reference is \cite{Conrad-Nagata}.

\begin{lemma}
\label{lemma-locally-finite-type-in-finite-presentation}
Let $f : X \to S$ be a morphism of schemes.
Assume:
\begin{enumerate}
\item The morphism $f$ is locally of finite type.
\item The scheme $X$ is quasi-compact and quasi-separated.
\end{enumerate}
Then there exists a morphism of finite presentation
$f' : X' \to S$ and an immersion $X \to X'$ of schemes over $S$.
\end{lemma}

\begin{proof}
By
Proposition \ref{proposition-approximate}
we can write
$X = \lim_i X_i$ with each $X_i$ of finite type over $\mathbf{Z}$ and
with transition morphisms $f_{ii'} : X_i \to X_{i'}$ affine.
Consider the commutative diagram
$$
\xymatrix{
X \ar[r] \ar[rd] & X_{i, S} \ar[r] \ar[d] & X_i \ar[d] \\
& S \ar[r] & \Spec(\mathbf{Z})
}
$$
Note that $X_i$ is of finite presentation over $\Spec(\mathbf{Z})$, see
Morphisms,
Lemma \ref{morphisms-lemma-noetherian-finite-type-finite-presentation}.
Hence the base change $X_{i, S} \to S$ is of finite presentation by
Morphisms, Lemma \ref{morphisms-lemma-base-change-finite-presentation}.
Thus it suffices to show that the arrow $X \to X_{i, S}$ is an
immersion for $i$ sufficiently large.

\medskip\noindent
To do this we choose a finite affine open covering
$X = V_1 \cup \ldots \cup V_n$ such that
$f$ maps each $V_j$ into an affine open $U_j \subset S$.
Let $h_{j, a} \in \mathcal{O}_X(V_j)$ be a finite
set of elements which generate $\mathcal{O}_X(V_j)$ as
an $\mathcal{O}_S(U_j)$-algebra, see
Morphisms, Lemma \ref{morphisms-lemma-locally-finite-type-characterize}.
By Lemmas \ref{lemma-descend-opens} and \ref{lemma-limit-affine}
(after possibly shrinking $I$) we may assume that
there exist affine open coverings
$X_i = V_{1, i} \cup \ldots \cup V_{n, i}$
compatible with transition maps such that $V_j = \lim_i V_{j, i}$.
By Lemma \ref{lemma-descend-section} we can choose $i$ so large that each
$h_{j, a}$ comes from an element
$h_{j, a, i} \in \mathcal{O}_{X_i}(V_{j, i})$.
Thus the arrow in
$$
V_j \longrightarrow U_j \times_{\Spec(\mathbf{Z})} V_{j, i} =
(V_{j, i})_{U_j} \subset (V_{j, i})_S \subset X_{i, S}
$$
is a closed immersion. Since $\bigcup (V_{j, i})_{U_j}$ forms an open of
$X_{i, S}$ and since the inverse image of $(V_{j, i})_{U_j}$ in $X$
is $V_j$ it follows that $X \to X_{i, S}$ is an immersion.
\end{proof}

\begin{remark}
\label{remark-cannot-do-better}
We cannot do better than this if we do not assume
more on $S$ and the morphism $f : X \to S$.
For example, in general it will not be possible to
find a {\it closed} immersion $X \to X'$ as in the lemma.
The reason is that this would imply that $f$ is quasi-compact which
may not be the case. An example is to take $S$ to be infinite
dimensional affine space with $0$ doubled and $X$ to be one of
the two infinite dimensional affine spaces.
\end{remark}

\begin{lemma}
\label{lemma-finite-type-closed-in-finite-presentation}
Let $f : X \to S$ be a morphism of schemes.
Assume:
\begin{enumerate}
\item The morphism $f$ is of locally of finite type.
\item The scheme $X$ is quasi-compact and quasi-separated, and
\item The scheme $S$ is quasi-separated.
\end{enumerate}
Then there exists a morphism of finite presentation
$f' : X' \to S$ and a closed immersion $X \to X'$ of schemes over $S$.
\end{lemma}

\begin{proof}
By Lemma \ref{lemma-locally-finite-type-in-finite-presentation} above
there exists a morphism $Y \to S$ of finite presentation and an
immersion $i : X \to Y$ of schemes over $S$.
For every point $x \in X$, there exists an affine open
$V_x \subset Y$ such that $i^{-1}(V_x) \to V_x$ is a
closed immersion. Since $X$ is quasi-compact we can find
finitely may affine opens $V_1, \ldots, V_n \subset Y$
such that $i(X) \subset V_1 \cup \ldots \cup V_n$ and
$i^{-1}(V_j) \to V_j$ is a closed immersion. In other words
such that $i : X \to X' = V_1 \cup \ldots \cup V_n$ is a
closed immersion of schemes over $S$.
Since $S$ is quasi-separated and $Y$ is quasi-separated over $S$
we deduce that $Y$ is quasi-separated, see
Schemes, Lemma \ref{schemes-lemma-separated-permanence}.
Hence the open immersion $X' = V_1 \cup \ldots \cup V_n \to Y$
is quasi-compact. This implies that
$X' \to Y$ is of finite presentation, see
Morphisms,
Lemma \ref{morphisms-lemma-quasi-compact-open-immersion-finite-presentation}.
We conclude since then $X' \to Y \to S$ is a composition of morphisms
of finite presentation, and hence of finite presentation (see
Morphisms, Lemma \ref{morphisms-lemma-composition-finite-presentation}).
\end{proof}

\begin{lemma}
\label{lemma-closed-is-limit-closed-and-finite-presentation}
Let $X \to Y$ be a closed immersion of schemes. Assume $Y$ quasi-compact and
quasi-separated. Then $X$ can be written as a directed limit $X = \lim X_i$
of schemes over $Y$ where $X_i \to Y$ is a closed immersion
of finite presentation.
\end{lemma}

\begin{proof}
Let $\mathcal{I} \subset \mathcal{O}_Y$ be the quasi-coherent sheaf of
ideals defining $X$ as a closed subscheme of $Y$. By
Properties, Lemma \ref{properties-lemma-quasi-coherent-colimit-finite-type}
we can write $\mathcal{I}$ as a directed colimit
$\mathcal{I} = \colim_{i \in I} \mathcal{I}_i$ of its
quasi-coherent sheaves of ideals of finite type.
Let $X_i \subset Y$ be the closed subscheme defined by $\mathcal{I}_i$.
These form an inverse system of schemes indexed by $I$.
The transition morphisms $X_i \to X_{i'}$ are affine because
they are closed immersions. Each $X_i$ is quasi-compact and quasi-separated
since it is a closed subscheme of $Y$ and $Y$ is quasi-compact and
quasi-separated by our assumptions.
We have $X = \lim_i X_i$ as follows directly from the
fact that $\mathcal{I} = \colim_{i \in I} \mathcal{I}_a$.
Each of the morphisms $X_i \to Y$ is of finite presentation, see
Morphisms, Lemma \ref{morphisms-lemma-closed-immersion-finite-presentation}.
\end{proof}

\begin{lemma}
\label{lemma-finite-type-is-limit-finite-presentation}
Let $f : X \to S$ be a morphism of schemes. Assume
\begin{enumerate}
\item The morphism $f$ is of locally of finite type.
\item The scheme $X$ is quasi-compact and quasi-separated, and
\item The scheme $S$ is quasi-separated.
\end{enumerate}
Then $X = \lim X_i$ where the $X_i \to S$ are of
finite presentation, the $X_i$ are quasi-compact and quasi-separated,
and the transition morphisms $X_{i'} \to X_i$ are closed immersions
(which implies that $X \to X_i$ are closed immersions for all $i$).
\end{lemma}

\begin{proof}
By Lemma \ref{lemma-finite-type-closed-in-finite-presentation}
there is a closed immersion $X \to Y$ with $Y \to S$ of
finite presentation. Then $Y$ is quasi-separated by
Schemes, Lemma \ref{schemes-lemma-separated-permanence}.
Since $X$ is quasi-compact, we may assume
$Y$ is quasi-compact by replacing $Y$ with a quasi-compact open
containing $X$. We see that $X = \lim X_i$ with $X_i \to Y$ a closed
immersion of finite presentation by
Lemma \ref{lemma-closed-is-limit-closed-and-finite-presentation}.
The morphisms $X_i \to S$ are of finite presentation by
Morphisms, Lemma \ref{morphisms-lemma-composition-finite-presentation}.
\end{proof}

\begin{proposition}
\label{proposition-separated-closed-in-finite-presentation}
Let $f : X \to S$ be a morphism of schemes. Assume
\begin{enumerate}
\item $f$ is of finite type and separated, and
\item $S$ is quasi-compact and quasi-separated.
\end{enumerate}
Then there exists a separated morphism of finite presentation
$f' : X' \to S$ and a closed immersion $X \to X'$ of schemes over $S$.
\end{proposition}

\begin{proof}
Apply Lemma \ref{lemma-finite-type-is-limit-finite-presentation}
and note that $X_i \to S$ is separated for
large $i$ by Lemma \ref{lemma-eventually-separated} as we have
assumed that $X \to S$ is separated.
\end{proof}

\begin{lemma}
\label{lemma-finite-closed-in-finite-finite-presentation}
Let $f : X \to S$ be a morphism of schemes. Assume
\begin{enumerate}
\item $f$ is finite, and
\item $S$ is quasi-compact and quasi-separated.
\end{enumerate}
Then there exists a morphism which is finite and of finite presentation
$f' : X' \to S$ and a closed immersion $X \to X'$ of schemes over $S$.
\end{lemma}

\begin{proof}
We may write $X = \lim X_i$ as in
Lemma \ref{lemma-finite-type-is-limit-finite-presentation}.
Applying Lemma \ref{lemma-eventually-finite} we see that $X_i \to S$
is finite for large enough $i$.
\end{proof}

\begin{lemma}
\label{lemma-finite-in-finite-and-finite-presentation}
Let $f : X \to S$ be a morphism of schemes. Assume
\begin{enumerate}
\item $f$ is finite, and
\item $S$ quasi-compact and quasi-separated.
\end{enumerate}
Then $X$ is a directed limit $X = \lim X_i$
where the transition maps are closed immersions and the objects
$X_i$ are finite and of finite presentation over $S$.
\end{lemma}

\begin{proof}
We may write $X = \lim X_i$ as in
Lemma \ref{lemma-finite-type-is-limit-finite-presentation}.
Applying Lemma \ref{lemma-eventually-finite} we see that $X_i \to S$
is finite for large enough $i$.
\end{proof}








\section{Descending relative objects}
\label{section-descending-relative}

\noindent
The following lemma is typical of the type of results in this section.
We write out the ``standard'' proof completely. It may be faster to
convince yourself that the result is true than to read this proof.

\begin{lemma}
\label{lemma-descend-finite-presentation}
Let $I$ be a directed set.
Let $(S_i, f_{ii'})$ be an inverse system of schemes over $I$.
Assume
\begin{enumerate}
\item the morphisms $f_{ii'} : S_i \to S_{i'}$ are affine,
\item the schemes $S_i$ are quasi-compact and quasi-separated.
\end{enumerate}
Let $S = \lim_i S_i$. Then we have the following:
\begin{enumerate}
\item For any morphism of finite presentation $X \to S$
there exists an index $i \in I$ and a morphism of finite
presentation $X_i \to S_i$ such that $X \cong X_{i, S}$ as
schemes over $S$.
\item Given an index $i \in I$, schemes
$X_i$, $Y_i$ of finite presentation over $S_i$, and a morphism
$\varphi : X_{i, S} \to Y_{i, S}$ over $S$, there exists an index
$i' \geq i$ and a morphism
$\varphi_{i'} : X_{i, S_{i'}} \to Y_{i, S_{i'}}$
whose base change to $S$ is $\varphi$.
\item Given an index $i \in I$, schemes $X_i$, $Y_i$ of finite presentation
over $S_i$ and a pair of morphisms $\varphi_i, \psi_i : X_i \to Y_i$
whose base changes $\varphi_{i, S} = \psi_{i, S}$ are equal,
there exists an index $i' \geq i$ such that
$\varphi_{i, S_{i'}} = \psi_{i, S_{i'}}$.
\end{enumerate}
In other words, the category of schemes of finite presentation over
$S$ is the colimit over $I$ of the categories of schemes of finite
presentation over $S_i$.
\end{lemma}

\begin{proof}
In case each of the schemes $S_i$ is affine, and we consider
only affine schemes of finite presentation over $S_i$, resp.\ $S$
this lemma is equivalent to
Algebra, Lemma \ref{algebra-lemma-colimit-category-fp-algebras}.
We claim that the affine case implies the lemma in general.

\medskip\noindent
Let us prove (3). Suppose given an index $i \in I$, schemes
$X_i$, $Y_i$ of finite presentation over $S_i$ and a pair of morphisms
$\varphi_i, \psi_i : X_i \to Y_i$. Assume that the base changes are
equal: $\varphi_{i, S} = \psi_{i, S}$. We will use the notation
$X_{i'} = X_{i, S_{i'}}$ and $Y_{i'} = Y_{i, S_{i'}}$ for
$i' \geq i$. We also set $X = X_{i, S}$ and $Y = Y_{i, S}$.
Note that according to Lemma \ref{lemma-scheme-over-limit} we have
$X = \lim_{i' \geq i} X_{i'}$ and similarly for $Y$.
Additionally we denote $\varphi_{i'}$ and $\psi_{i'}$
(resp.\ $\varphi$ and $\psi$)
the base change of $\varphi_i$ and $\psi_i$ to $S_{i'}$
(resp.\ $S$). So our assumption means that $\varphi = \psi$.
Since $Y_i$ and $X_i$ are of finite presentation
over $S_i$, and since $S_i$ is quasi-compact and quasi-separated, also
$X_i$ and $Y_i$ are quasi-compact and quasi-separated
(see Morphisms,
Lemma \ref{morphisms-lemma-finite-presentation-quasi-compact-quasi-separated}).
Hence we may choose a finite affine open covering
$Y_i = \bigcup V_{j, i}$ such that each $V_{j, i}$ maps into
an affine open of $S$. As above, denote $V_{j, i'}$ the inverse
image of $V_{j, i}$ in $Y_{i'}$ and $V_j$ the inverse image in $Y$.
The immersions $V_{j, i'} \to Y_{i'}$ are quasi-compact, and the inverse images
$U_{j, i'} = \varphi_i^{-1}(V_{j, i'})$ and
$U_{j, i'}' = \psi_i^{-1}(V_{j, i'})$
are quasi-compact opens of $X_{i'}$. By assumption the inverse images of
$V_j$ under $\varphi$ and $\psi$ in $X$ are equal.
Hence by Lemma \ref{lemma-descend-opens}
there exists an index $i' \geq i$ such that
of $U_{j, i'} = U_{j, i'}'$ in $X_{i'}$.
Choose an finite affine open covering
$U_{j, i'} = U_{j, i'}' = \bigcup W_{j, k, i'}$
which induce coverings $U_{j, i''} = U_{j, i''}' = \bigcup W_{j, k, i''}$
for all $i'' \geq i'$.
By the affine case there exists
an index $i''$ such that
$\varphi_{i''}|_{W_{j, k, i''}} = \psi_{i''}|_{W_{j, k, i''}}$
for all $j, k$. Then $i''$ is an index such that
$\varphi_{i''} = \psi_{i''}$ and (3) is proved.

\medskip\noindent
Let us prove (2). Suppose given an index $i \in I$, schemes
$X_i$, $Y_i$ of finite presentation over $S_i$ and a morphism
$\varphi : X_{i, S} \to Y_{i, S}$. We will use the notation
$X_{i'} = X_{i, S_{i'}}$ and $Y_{i'} = Y_{i, S_{i'}}$ for
$i' \geq i$. We also set $X = X_{i, S}$ and $Y = Y_{i, S}$.
Note that according to Lemma \ref{lemma-scheme-over-limit} we have
$X = \lim_{i' \geq i} X_{i'}$ and similarly for $Y$.
Since $Y_i$ and $X_i$ are of finite presentation
over $S_i$, and since $S_i$ is quasi-compact and quasi-separated, also
$X_i$ and $Y_i$ are quasi-compact and quasi-separated
(see Morphisms,
Lemma \ref{morphisms-lemma-finite-presentation-quasi-compact-quasi-separated}).
Hence we may choose a finite affine open covering
$Y_i = \bigcup V_{j, i}$ such that each $V_{j, i}$ maps into
an affine open of $S$. As above, denote $V_{j, i'}$ the inverse
image of $V_{j, i}$ in $Y_{i'}$ and $V_j$ the inverse image in $Y$.
The immersions $V_j \to Y$ are quasi-compact, and the inverse images
$U_j = \varphi^{-1}(V_j)$ are quasi-compact opens of $X$.
Hence by Lemma \ref{lemma-descend-opens} there exists an index
$i' \geq i$ and quasi-compact opens $U_{j, i'}$ of $X_{i'}$
whose inverse image in $X$ is $U_j$. Choose an finite affine open covering
$U_{j, i'} = \bigcup W_{j, k, i'}$ which induce affine open
coverings $U_{j, i''} = \bigcup W_{j, k, i''}$
for all $i'' \geq i'$ and an affine open covering
$U_j = \bigcup W_{j, k}$. By the affine case there exists
an index $i''$ and morphisms
$\varphi_{j, k, i''} : W_{j, k, i''} \to V_{j, i''}$
such that
$\varphi|_{W_{j, k}} = \varphi_{j, k, i'', S}$ for all $j, k$.
By part (3) proved above, there is a further index $i''' \geq i''$
such that
$$
\varphi_{j_1, k_1, i'', S_{i'''}}|_{W_{j_1, k_1, i'''} \cap W_{j_2, k_2, i'''}}
=
\varphi_{j_2, k_2, i'', S_{i'''}}|_{W_{j_1, k_1, i'''} \cap W_{j_2, k_2, i'''}}
$$
for all $j_1, j_2, k_1, k_2$. Then $i'''$ is an index such that
there exists a morphism $\varphi_{i'''} : X_{i'''} \to Y_{i'''}$
whose base change to $S$ gives $\varphi$. Hence (2) holds.

\medskip\noindent
Let us prove (1). Suppose given a scheme $X$ of finite presentation
over $S$. Since $X$ is of finite presentation
over $S$, and since $S$ is quasi-compact and quasi-separated, also
$X$ is quasi-compact and quasi-separated
(see Morphisms,
Lemma \ref{morphisms-lemma-finite-presentation-quasi-compact-quasi-separated}).
Choose a finite affine open covering $X = \bigcup U_j$
such that each $U_j$ maps into an affine open $V_j \subset S$.
Denote $U_{j_1j_2} = U_{j_1} \cap U_{j_2}$ and
$U_{j_1j_2j_3} = U_{j_1} \cap U_{j_2} \cap U_{j_3}$.
By Lemmas \ref{lemma-descend-opens} and \ref{lemma-limit-affine}
we can find an index $i_1$ and affine opens $V_{j, i_1} \subset S_{i_1}$
such that each $V_j$ is the inverse of this in $S$.
Let $V_{j, i}$ be the inverse image of $V_{j, i_1}$ in $S_i$ for
$i \geq i_1$. By the affine case we may find an index $i_2 \geq i_1$ and
affine schemes $U_{j, i_2} \to V_{j, i_2}$ such
that $U_j = S \times_{S_{i_2}} U_{j, i_2}$ is the base change.
Denote $U_{j, i} = S_i \times_{S_{i_2}} U_{j, i_2}$ for $i \geq i_2$.
By Lemma \ref{lemma-descend-opens} there exists an index
$i_3 \geq i_2$ and open subschemes
$W_{j_1, j_2, i_3} \subset U_{j_1, i_3}$
whose base change to $S$ is equal to $U_{j_1j_2}$.
Denote $W_{j_1, j_2, i} = S_i \times_{S_{i_3}} W_{j_1, j_2, i_3}$
for $i \geq i_3$. By part (2) shown above there exists an index
$i_4 \geq i_3$ and morphisms
$\varphi_{j_1, j_2, i_4} : W_{j_1, j_2, i_4} \to W_{j_2, j_1, i_4}$
whose base change to $S$ gives the identity morphism
$U_{j_1j_2} = U_{j_2j_1}$ for all $j_1, j_2$.
For all $i \geq i_4$ denote
$\varphi_{j_1, j_2, i} = \text{id}_S \times \varphi_{j_1, j_2, i_4}$
the base change. We claim that for some $i_5 \geq i_4$ the system
$((U_{j, i_5})_j, (W_{j_1, j_2, i_5})_{j_1, j_2},
(\varphi_{j_1, j_2, i_5})_{j_1, j_2})$ forms a glueing datum
as in Schemes, Section \ref{schemes-section-glueing-schemes}.
In order to see this we have to verify that for $i$ large enough
we have
$$
\varphi_{j_1, j_2, i}^{-1}(W_{j_1, j_2, i} \cap W_{j_1, j_3, i})
=
W_{j_1, j_2, i} \cap W_{j_1, j_3, i}
$$
and that for large enough $i$ the cocycle condition holds.
The first condition follows from Lemma \ref{lemma-descend-opens}
and the fact that $U_{j_2j_1j_3} = U_{j_1j_2j_3}$.
The second from part (1) of the lemma proved above and the fact
that the cocycle condition holds for the maps
$\text{id} : U_{j_1j_2} \to U_{j_2j_1}$.
Ok, so now we can use Schemes, Lemma \ref{schemes-lemma-glue-schemes}
to glue the system
$((U_{j, i_5})_j, (W_{j_1, j_2, i_5})_{j_1, j_2},
(\varphi_{j_1, j_2, i_5})_{j_1, j_2})$ to get a scheme
$X_{i_5} \to S_{i_5}$. By construction the base change of
$X_{i_5}$ to $S$ is formed by glueing the open affines
$U_j$ along the opens $U_{j_1} \leftarrow U_{j_1j_2} \rightarrow U_{j_2}$.
Hence $S \times_{S_{i_5}} X_{i_5} \cong X$ as desired.
\end{proof}

\begin{lemma}
\label{lemma-descend-modules-finite-presentation}
Let $I$ be a directed set.
Let $(S_i, f_{ii'})$ be an inverse system of schemes over $I$.
Assume
\begin{enumerate}
\item all the morphisms $f_{ii'} : S_i \to S_{i'}$ are affine,
\item all the schemes $S_i$ are quasi-compact and quasi-separated.
\end{enumerate}
Let $S = \lim_i S_i$. Then we have the following:
\begin{enumerate}
\item For any sheaf of $\mathcal{O}_S$-modules
$\mathcal{F}$ of finite presentation there exists an index
$i \in I$ and a sheaf of $\mathcal{O}_{S_i}$-modules of finite
presentation $\mathcal{F}_i$ such that
$\mathcal{F} \cong f_i^*\mathcal{F}_i$.
\item Suppose given an index $i \in I$, sheaves
of $\mathcal{O}_{S_i}$-modules $\mathcal{F}_i$, $\mathcal{G}_i$
of finite presentation and a morphism
$\varphi : f_i^*\mathcal{F}_i \to f_i^*\mathcal{G}_i$ over $S$.
Then there exists an index $i' \geq i$ and a morphism
$\varphi_{i'} : f_{i'i}^*\mathcal{F}_i \to f_{i'i}^*\mathcal{G}_i$
whose base change to $S$ is $\varphi$.
\item Suppose given an index $i \in I$, sheaves of $\mathcal{O}_{S_i}$-modules
$\mathcal{F}_i$, $\mathcal{G}_i$ of finite presentation
and a pair of morphisms $\varphi_i, \psi_i : \mathcal{F}_i \to \mathcal{G}_i$.
Assume that the base changes are equal: $f_i^*\varphi_i = f_i^*\psi_i$.
Then there exists an index $i' \geq i$ such that
$f_{i'i}^*\varphi_i = f_{i'i}^*\psi_i$.
\end{enumerate}
In other words, the category of modules
of finite presentation over $S$ is the colimit over $I$
of the categories modules of finite presentation over $S_i$.
\end{lemma}

\begin{proof}
Omitted. Since we have written out completely the proof of
Lemma \ref{lemma-descend-finite-presentation} above
it seems wise to use this here and not completely write this
proof out also. For example we can use:
\begin{enumerate}
\item there is an equivalence of categories
between quasi-coherent $\mathcal{O}_S$-modules and
vector bundles over $S$, see
Constructions, Section \ref{constructions-section-vector-bundle}.
\item a vector bundle $\mathbf{V}(\mathcal{F}) \to S$ is
of finite presentation over $S$ if and only if $\mathcal{F}$
is an $\mathcal{O}_S$-module of finite presentation.
\end{enumerate}
Then you can descend morphisms in terms of morphisms of the
associated vectorbundles. Similarly for objects.
\end{proof}

\begin{lemma}
\label{lemma-descend-invertible-modules}
Let $S = \lim S_i$ be the limit of a directed system of quasi-compact and
quasi-separated schemes $S_i$ with affine transition morphisms. Then any
invertible $\mathcal{O}_S$-module is the pullback of an invertible
$\mathcal{O}_{S_i}$-module for some $i$.
\end{lemma}

\begin{proof}
Let $\mathcal{L}$ be an invertible $\mathcal{O}_S$-module. Since
invertible modules are of finite presentation we can find an $i$
and modules $\mathcal{L}_i$ and $\mathcal{N}_i$ of finite presentation
over $S_i$ such that $f_i^*\mathcal{L}_i \cong \mathcal{L}$ and
$f_i^*\mathcal{N}_i \cong \mathcal{L}^{\otimes -1}$, see
Lemma \ref{lemma-descend-modules-finite-presentation}.
Since pullback commutes with tensor product we see that
$f_i^*(\mathcal{L}_i \otimes_{\mathcal{O}_{S_i}} \mathcal{N}_i)$
is isomorphic to $\mathcal{O}_S$. Since the tensor product of
finitely presented modules is finitely presented, the same
lemma implies that
$f_{i'i}^*\mathcal{L}_i
\otimes_{\mathcal{O}_{S_{i'}}} f_{i'i}^*\mathcal{N}_i$
is isomorphic to $\mathcal{O}_{S_{i'}}$ for some $i' \geq i$.
It follows that $f_{i'i}^*\mathcal{L}_i$ is invertible
(Modules, Lemma \ref{modules-lemma-invertible}) and the proof is complete.
\end{proof}

\begin{lemma}
\label{lemma-descend-module-flat-finite-presentation}
With notation and assumptions as in
Lemma \ref{lemma-descend-finite-presentation}.
Let $i \in I$.
Suppose that $\varphi_i : X_i \to Y_i$ is a morphism of schemes
of finite presentation over $S_i$ and that $\mathcal{F}_i$ is a
quasi-coherent $\mathcal{O}_{X_i}$-module of finite presentation.
If the pullback of $\mathcal{F}_i$ to $X_i \times_{S_i} S$ is flat
over $Y_i \times_{S_i} S$, then there exists an index $i' \geq i$
such that the pullback of $\mathcal{F}_i$ to $X_i \times_{S_i} S_{i'}$
is flat over $Y_i \times_{S_i} S_{i'}$.
\end{lemma}

\begin{proof}
(This lemma is the analogue of
Lemma \ref{lemma-descend-flat-finite-presentation}
for modules.)
For $i' \geq i$ denote $X_{i'} = S_{i'} \times_{S_i} X_i$,
$\mathcal{F}_{i'} = (X_{i'} \to X_i)^*\mathcal{F}_i$ and similarly
for $Y_{i'}$. Denote $\varphi_{i'}$ the base change
of $\varphi_i$ to $S_{i'}$. Also set $X = S \times_{S_i} X_i$,
$Y =S \times_{S_i} X_i$, $\mathcal{F} = (X \to X_i)^*\mathcal{F}_i$
and $\varphi$ the base change of $\varphi_i$ to $S$.
Let $Y_i = \bigcup_{j = 1, \ldots, m} V_{j, i}$ be a finite affine open
covering such that each $V_{j, i}$ maps into some affine open of $S_i$.
For each $j = 1, \ldots m$ let
$\varphi_i^{-1}(V_{j, i}) = \bigcup_{k = 1, \ldots, m(j)} U_{k, j, i}$
be a finite affine open covering. For $i' \geq i$ we denote
$V_{j, i'}$ the inverse image of $V_{j, i}$ in $Y_{i'}$ and
$U_{k, j, i'}$ the inverse image of $U_{k, j, i}$ in $X_{i'}$.
Similarly we have $U_{k, j} \subset X$ and $V_j \subset Y$.
Then $U_{k, j} = \lim_{i' \geq i} U_{k, j, i'}$
and $V_j = \lim_{i' \geq i} V_j$
(see Lemma \ref{lemma-directed-inverse-system-has-limit}).
Since $X_{i'} = \bigcup_{k, j} U_{k, j, i'}$ is a finite open covering
it suffices to prove the lemma for each of the morphisms
$U_{k, j, i} \to V_{j, i}$ and the sheaf $\mathcal{F}_i|_{U_{k, j, i}}$.
Hence we see that the lemma reduces to the case that $X_i$ and
$Y_i$ are affine and map into an affine open of $S_i$, i.e., we
may also assume that $S$ is affine.

\medskip\noindent
In the affine case we reduce to the following algebra result.
Suppose that $R = \colim_{i \in I} R_i$. For some $i \in I$
suppose given a map $A_i \to B_i$ of finitely presented $R_i$-algebras.
Let $N_i$ be a finitely presented $B_i$-module.
Then, if $R \otimes_{R_i} N_i$ is flat over $R \otimes_{R_i} A_i$,
then for some $i' \geq i$ the module
$R_{i'} \otimes_{R_i} N_i$ is flat over $R_{i'} \otimes_{R_i} A$.
This is exactly the result proved in
Algebra,
Lemma \ref{algebra-lemma-flat-finite-presentation-limit-flat} part (3).
\end{proof}












\section{Characterizing affine schemes}
\label{section-affine}

\noindent
If $f : X \to S$ is a surjective integral morphism of schemes
such that $X$ is an affine scheme then $S$ is affine too.
See \cite[A.2]{Conrad-Nagata}. Our proof relies
on the Noetherian case which we stated and proved in Cohomology of Schemes,
Lemma \ref{coherent-lemma-image-affine-finite-morphism-affine-Noetherian}.
See also \cite[II 6.7.1]{EGA}.

\begin{lemma}
\label{lemma-affine}
Let $f : X \to S$ be a morphism of schemes.
Assume that $f$ is surjective and finite, and assume that $X$ is affine.
Then $S$ is affine.
\end{lemma}

\begin{proof}
Since $f$ is surjective and $X$ is quasi-compact we see that $S$ is
quasi-compact. Since $X$ is separated and $f$ is surjective and
universally closed (Morphisms, Lemma
\ref{morphisms-lemma-integral-universally-closed}), we see that $S$
is separated (Morphisms, Lemma
\ref{morphisms-lemma-image-universally-closed-separated}).

\medskip\noindent
By Lemma \ref{lemma-finite-in-finite-and-finite-presentation}
we can write $X = \lim_a X_a$ with $X_a \to S$ finite and of finite
presentation. By Lemma \ref{lemma-limit-affine} we see that $X_a$
is affine for some $a \in A$. Replacing $X$ by $X_a$ we may assume
that $X \to S$ is surjective, finite, of finite presentation and
that $X$ is affine.

\medskip\noindent
By Proposition \ref{proposition-approximate} we may write
$S = \lim_{i \in I} S_i$ as
a directed limits as schemes of finite type over $\mathbf{Z}$.
By Lemma \ref{lemma-descend-finite-presentation} we can
after shrinking $I$ assume there exist schemes $X_i \to S_i$
of finite presentation such that $X_{i'} = X_i \times_S S_{i'}$
for $i' \geq i$ and such that $X = \lim_i X_i$. By
Lemma \ref{lemma-descend-finite-finite-presentation} we may
assume that $X_i \to S_i$ is finite for all $i \in I$ as well.
By Lemma \ref{lemma-limit-affine} once again we may assume that $X_i$ is
affine for all $i \in I$. Hence the result follows from the
Noetherian case, see Cohomology of Schemes,
Lemma \ref{coherent-lemma-image-affine-finite-morphism-affine-Noetherian}.
\end{proof}

\begin{proposition}
\label{proposition-affine}
Let $f : X \to S$ be a morphism of schemes.
Assume that $f$ is surjective and integral, and assume that $X$ is affine.
Then $S$ is affine.
\end{proposition}

\begin{proof}
Since $f$ is surjective and $X$ is quasi-compact we see that $S$ is
quasi-compact. Since $X$ is separated and $f$ is surjective and
universally closed (Morphisms, Lemma
\ref{morphisms-lemma-integral-universally-closed}), we see that $S$
is separated (Morphisms, Lemma
\ref{morphisms-lemma-image-universally-closed-separated}).

\medskip\noindent
By Lemma \ref{lemma-integral-limit-finite-and-finite-presentation}
we can write $X = \lim_i X_i$ with $X_i \to S$ finite. By
Lemma \ref{lemma-limit-affine}
we see that for $i$ sufficiently large the scheme $X_i$ is affine.
Moreover, since $X \to S$ factors through each $X_i$ we see that
$X_i \to S$ is surjective. Hence we conclude that $S$ is affine by
Lemma \ref{lemma-affine}.
\end{proof}

\begin{lemma}
\label{lemma-affines-glued-in-closed-affine}
Let $X$ be a scheme which is set theoretically the union of
finitely many affine closed subschemes. Then $X$ is affine.
\end{lemma}

\begin{proof}
Let $Z_i \subset X$, $i = 1, \ldots, n$ be affine closed subschemes such that
$X = \bigcup Z_i$ set theoretically. Then $\coprod Z_i \to X$ is surjective
and integral with affine source. Hence $X$ is affine by
Proposition \ref{proposition-affine}.
\end{proof}

\begin{lemma}
\label{lemma-ample-on-reduction}
Let $i : Z \to X$ be a closed immersion of schemes
inducing a homeomorphism of underlying topological spaces.
Let $\mathcal{L}$ be an invertible sheaf on $X$.
Then $i^*\mathcal{L}$ is ample on $Z$, if and only if $\mathcal{L}$
is ample on $X$.
\end{lemma}

\begin{proof}
If $\mathcal{L}$ is ample, then $i^*\mathcal{L}$ is ample for
example by Morphisms, Lemma
\ref{morphisms-lemma-pullback-ample-tensor-relatively-ample}.
Assume $i^*\mathcal{L}$ is ample. Then $Z$ is quasi-compact
(Properties, Definition \ref{properties-definition-ample})
and separated
(Properties, Lemma \ref{properties-lemma-ample-separated}).
Since $i$ is surjective, we see that $X$ is quasi-compact.
Since $i$ is universally closed and surjective, we see that
$X$ is separated (Morphisms, Lemma
\ref{morphisms-lemma-image-universally-closed-separated}).

\medskip\noindent
By Proposition \ref{proposition-approximate} we can write
$X = \lim X_i$ as a directed limit of finite type schemes over $\mathbf{Z}$
with affine transition morphisms. We can find an $i$ and an invertible
sheaf $\mathcal{L}_i$ on $X_i$ whose pullback to $X$ is isomorphic to
$\mathcal{L}$, see Lemma \ref{lemma-descend-modules-finite-presentation}.

\medskip\noindent
For each $i$ let $Z_i \subset X_i$ be the scheme theoretic image
of the morphism $Z \to X$. If $\Spec(A_i) \subset X_i$ is an affine
open subscheme with inverse image of $\Spec(A)$ in $X$ and if
$Z \cap \Spec(A)$ is defined by the ideal $I \subset A$, then
$Z_i \cap \Spec(A_i)$ is defined by the ideal $I_i \subset A_i$
which is the inverse image of $I$ in $A_i$ under the ring map
$A_i \to A$, see
Morphisms, Example \ref{morphisms-example-scheme-theoretic-image}.
Since $\colim A_i/I_i = A/I$ it follows that $\lim Z_i = Z$.
By Lemma \ref{lemma-limit-ample} we see that $\mathcal{L}_i|_{Z_i}$
is ample for some $i$. Since $Z$ and hence $X$ maps into $Z_i$
set theoretically, we see that $X_{i'} \to X_i$ maps into $Z_i$
set theoretically for some $i' \geq i$, see
Lemma \ref{lemma-limit-contained-in-constructible}.
(Observe that since $X_i$ is Noetherian, every closed subset
of $X_i$ is constructible.) Let $T \subset X_{i'}$
be the scheme theoretic inverse image of $Z_i$ in $X_{i'}$.
Observe that $\mathcal{L}_{i'}|_T$ is the pullback
of $\mathcal{L}_i|_{Z_i}$ and hence ample by
Morphisms, Lemma \ref{morphisms-lemma-pullback-ample-tensor-relatively-ample}
and the fact that $T \to Z_i$ is an affine morphism.
Thus we see that $\mathcal{L}_{i'}$ is ample on $X_{i'}$
by Cohomology of Schemes, Lemma \ref{coherent-lemma-ample-on-reduction}.
Pulling back to $X$ (using the same lemma as above)
we find that $\mathcal{L}$ is ample.
\end{proof}

\begin{lemma}
\label{lemma-thickening-quasi-affine}
Let $i : Z \to X$ be a closed immersion of schemes
inducing a homeomorphism of underlying topological spaces.
Then $X$ is quasi-affine if and only if $Z$ is quasi-affine.
\end{lemma}

\begin{proof}
Recall that a scheme is quasi-affine
if and only if the structure sheaf is ample, see
Properties, Lemma \ref{properties-lemma-quasi-affine-O-ample}.
Hence if $Z$ is quasi-affine, then $\mathcal{O}_Z$ is ample,
hence $\mathcal{O}_X$ is ample by
Lemma \ref{lemma-ample-on-reduction}, hence
$X$ is quasi-affine. A proof of the converse, which
can also be seen in an elementary way, is gotten by
reading the argument just given backwards.
\end{proof}

\noindent
The following lemma does not really belong in this section.

\begin{lemma}
\label{lemma-ample-profinite-set-in-principal-affine}
Let $X$ be a scheme. Let $\mathcal{L}$ be an ample invertible sheaf on $X$.
Assume we have morphisms of schemes
$$
\Spec(k) \leftarrow \Spec(A) \to W \subset X
$$
where $k$ is a field, $A$ is an integral $k$-algebra, $W$ is open in $X$.
Then there exists an $n > 0$ and a section
$s \in \Gamma(X, \mathcal{L}^{\otimes n})$ such that
$X_s$ is affine, $X_s \subset W$, and $\Spec(A) \to W$ factors through $X_s$
\end{lemma}

\begin{proof}
Since $\Spec(A)$ is quasi-compact, we may replace $W$ by a quasi-compact
open still containing the image of $\Spec(A) \to X$.
Recall that $X$ is quasi-separated and quasi-compact by dint
of having an ample invertible sheaf, see
Properties, Definition \ref{properties-definition-ample} and
Lemma \ref{properties-lemma-affine-s-opens-cover-quasi-separated}.
By Proposition \ref{proposition-approximate} we can
write $X = \lim X_i$ as a limit of a directed system
of schemes of finite type over $\mathbf{Z}$
with affine transition morphisms.
For some $i$ the ample invertible sheaf $\mathcal{L}$ on $X$
descends to an ample invertible sheaf $\mathcal{L}_i$ on $X_i$
and the open $W$ is the inverse image of a quasi-compact
open $W_i \subset X_i$, see
Lemmas \ref{lemma-limit-ample}, \ref{lemma-descend-invertible-modules}, and
\ref{lemma-descend-opens}.
We may replace $X, W, \mathcal{L}$ by $X_i, W_i, \mathcal{L}_i$
and assume $X$ is of finite presentation over $\mathbf{Z}$.
Write $A = \colim A_j$ as the colimit of its finite $k$-subalgebras.
Then for some $j$ the morphism $\Spec(A) \to X$ factors through
a morphism $\Spec(A_j) \to X$, see
Proposition \ref{proposition-characterize-locally-finite-presentation}.
Since $\Spec(A_j)$ is finite this reduces the lemma to
Properties, Lemma \ref{properties-lemma-ample-finite-set-in-principal-affine}.
\end{proof}



















\section{Variants of Chow's Lemma}
\label{section-chows-lemma}

\noindent
In this section we prove a number of variants of Chow's lemma.
The most interesting version is probably just the Noetherian
case, which we stated and proved in
Cohomology of Schemes, Section \ref{coherent-section-chows-lemma}.

\begin{lemma}
\label{lemma-chow-finite-type}
Let $S$ be a quasi-compact and quasi-separated scheme.
Let $f : X \to S$ be a separated morphism of finite type.
Then there exists an $n \geq 0$ and a diagram
$$
\xymatrix{
X \ar[rd] & X' \ar[d] \ar[l]^\pi \ar[r] & \mathbf{P}^n_S \ar[dl] \\
& S &
}
$$
where $X' \to \mathbf{P}^n_S$ is an immersion, and
$\pi : X' \to X$ is proper and surjective.
\end{lemma}

\begin{proof}
By Proposition \ref{proposition-separated-closed-in-finite-presentation}
we can find a closed immersion $X \to Y$ where $Y$ is separated
and of finite presentation over $S$. Clearly, if we prove the assertion
for $Y$, then the result follows for $X$. Hence we may assume that
$X$ is of finite presentation over $S$.

\medskip\noindent
Write $S = \lim_i S_i$ as a directed limit of Noetherian schemes, see
Proposition \ref{proposition-approximate}. By
Lemma \ref{lemma-descend-finite-presentation} we can
find an index $i \in I$ and a scheme $X_i \to S_i$ of finite presentation
so that $X = S \times_{S_i} X_i$.
By Lemma \ref{lemma-descend-separated-finite-presentation}
we may assume that $X_i \to S_i$ is separated.
Clearly, if we prove the assertion for
$X_i$ over $S_i$, then the assertion holds for $X$. The case
$X_i \to S_i$ is treated by
Cohomology of Schemes, Lemma \ref{coherent-lemma-chow-Noetherian}.
\end{proof}

\noindent
Here is a variant of Chow's lemma where we assume the scheme
on top has finitely many irreducible components.

\begin{lemma}
\label{lemma-chow-EGA}
Let $S$ be a quasi-compact and quasi-separated scheme.
Let $f : X \to S$ be a separated morphism of finite type.
Assume that $X$ has finitely many irreducible components.
Then there exists an $n \geq 0$ and a diagram
$$
\xymatrix{
X \ar[rd] & X' \ar[d] \ar[l]^\pi \ar[r] & \mathbf{P}^n_S \ar[dl] \\
& S &
}
$$
where $X' \to \mathbf{P}^n_S$ is an immersion, and
$\pi : X' \to X$ is proper and surjective. Moreover, there exists
an open dense subscheme $U \subset X$ such that $\pi^{-1}(U) \to U$
is an isomorphism of schemes.
\end{lemma}

\begin{proof}
Let $X = Z_1 \cup \ldots \cup Z_n$ be the decomposition of $X$
into irreducible components. Let $\eta_j \in Z_j$ be the generic point.

\medskip\noindent
There are (at least) two ways to proceed with the proof.
The first is to redo the proof of
Cohomology of Schemes, Lemma \ref{coherent-lemma-chow-Noetherian}
using the general
Properties, Lemma \ref{properties-lemma-point-and-maximal-points-affine}
to find suitable affine opens in $X$. (This is the ``standard'' proof.)
The second is to use absolute Noetherian approximation as in
the proof of Lemma \ref{lemma-chow-finite-type} above.
This is what we will do here.

\medskip\noindent
By Proposition \ref{proposition-separated-closed-in-finite-presentation}
we can find a closed immersion $X \to Y$ where $Y$ is separated
and of finite presentation over $S$.
Write $S = \lim_i S_i$ as a directed limit of Noetherian schemes, see
Proposition \ref{proposition-approximate}. By
Lemma \ref{lemma-descend-finite-presentation} we can
find an index $i \in I$ and a scheme $Y_i \to S_i$ of finite presentation
so that $Y = S \times_{S_i} Y_i$.
By Lemma \ref{lemma-descend-separated-finite-presentation}
we may assume that $Y_i \to S_i$ is separated.
We have the following diagram
$$
\xymatrix{
\eta_j \in Z_j \ar[r] & X \ar[r] \ar[rd] & Y \ar[r] \ar[d] & Y_i \ar[d] \\
& & S \ar[r] & S_i
}
$$
Denote $h : X \to Y_i$ the composition.

\medskip\noindent
For $i' \geq i$ write $Y_{i'} = S_{i'} \times_{S_i} Y_i$.
Then $Y = \lim_{i' \geq i} Y_{i'}$, see
Lemma \ref{lemma-scheme-over-limit}.
Choose $j, j' \in \{1, \ldots, n\}$, $j \not = j'$.
Note that $\eta_j$ is not a specialization of $\eta_{j'}$.
By Lemma \ref{lemma-topology-limit}
we can replace $i$ by a bigger index and assume
that $h(\eta_j)$ is not a specialization of $h(\eta_{j'})$
for all pairs $(j, j')$ as above.
For such an index, let
$Y' \subset Y_i$ be the scheme theoretic image of
$h : X \to Y_i$, see
Morphisms, Definition \ref{morphisms-definition-scheme-theoretic-image}.
The morphism $h$ is quasi-compact as the composition of the quasi-compact
morphisms $X \to Y$ and $Y \to Y_i$ (which is affine).
Hence by
Morphisms, Lemma \ref{morphisms-lemma-quasi-compact-scheme-theoretic-image}
the morphism $X \to Y'$ is dominant. Thus the generic points
of $Y'$ are all contained in the set
$\{h(\eta_1), \ldots, h(\eta_n)\}$, see
Morphisms, Lemma \ref{morphisms-lemma-quasi-compact-dominant}.
Since none of the $h(\eta_j)$ is the specialization of another
we see that the points $h(\eta_1), \ldots, h(\eta_n)$ are pairwise
distinct and are each a generic point of $Y'$.

\medskip\noindent
We apply Cohomology of Schemes, Lemma
\ref{coherent-lemma-chow-Noetherian} above to the morphism
$Y' \to S_i$. This gives a diagram
$$
\xymatrix{
Y' \ar[rd] & Y^* \ar[d] \ar[l]^\pi \ar[r] & \mathbf{P}^n_{S_i} \ar[dl] \\
& S_i &
}
$$
such that $\pi$ is proper and surjective and an isomorphism over
a dense open subscheme $V \subset Y'$. By our choice of $i$ above
we know that $h(\eta_1), \ldots, h(\eta_n) \in V$. Consider
the commutative diagram
$$
\xymatrix{
X' \ar@{=}[r] &
X \times_{Y'} Y^* \ar[r] \ar[d] &
Y^* \ar[r] \ar[d] &
\mathbf{P}^n_{S_i} \ar[ddl] \\
& X \ar[r] \ar[d] & Y' \ar[d] & \\
& S \ar[r] & S_i &
}
$$
Note that $X' \to X$ is an isomorphism over the open subscheme
$U = h^{-1}(V)$ which contains each of the $\eta_j$ and hence is
dense in $X$. We conclude $X \leftarrow X' \rightarrow \mathbf{P}^n_S$
is a solution to the problem posed in the lemma.
\end{proof}













\section{Applications of Chow's lemma}
\label{section-apply-chow}

\noindent
Here is a first application of Chow's lemma.

\begin{lemma}
\label{lemma-eventually-proper}
\begin{slogan}
If the base change of a scheme to a limit is proper, then
already the base change is proper at a finite level.
\end{slogan}
Assumptions and notation as in Situation \ref{situation-descent-property}.
If
\begin{enumerate}
\item $f$ is proper, and
\item $f_0$ is locally of finite type,
\end{enumerate}
then there exists an $i$ such that $f_i$ is proper.
\end{lemma}

\begin{proof}
By Lemma \ref{lemma-descend-separated-finite-presentation} we see that
$f_i$ is separated for some $i \geq 0$. Replacing
$0$ by $i$ we may assume that $f_0$ is separated.
Observe that $f_0$ is quasi-compact, see
Schemes, Lemma \ref{schemes-lemma-quasi-compact-permanence}.
By Lemma \ref{lemma-chow-finite-type} we can choose a diagram
$$
\xymatrix{
X_0 \ar[rd] & X_0' \ar[d] \ar[l]^\pi \ar[r] & \mathbf{P}^n_{Y_0} \ar[dl] \\
& Y_0 &
}
$$
where $X_0' \to \mathbf{P}^n_{Y_0}$ is an immersion, and
$\pi : X_0' \to X_0$ is proper and surjective. Introduce
$X' = X_0' \times_{Y_0} Y$ and $X_i' = X_0' \times_{Y_0} Y_i$.
By Morphisms, Lemmas \ref{morphisms-lemma-composition-proper} and
\ref{morphisms-lemma-base-change-proper}
we see that $X' \to Y$ is proper. Hence $X' \to \mathbf{P}^n_Y$ is
a closed immersion (Morphisms, Lemma
\ref{morphisms-lemma-image-proper-scheme-closed}). By
Morphisms, Lemma \ref{morphisms-lemma-image-proper-is-proper}
it suffices to prove that $X'_i \to Y_i$ is proper for some $i$.
By Lemma \ref{lemma-descend-closed-immersion-finite-presentation}
we find that $X'_i \to \mathbf{P}^n_{Y_i}$ is
a closed immersion for $i$ large enough. Then $X'_i \to Y_i$
is proper and we win.
\end{proof}

\begin{lemma}
\label{lemma-proper-limit-of-proper-finite-presentation}
Let $f : X \to S$ be a proper morphism with $S$ quasi-compact and
quasi-separated. Then $X = \lim X_i$ is a directed limit of schemes
$X_i$ proper and of finite presentation over $S$ such that
all transition morphisms and the morphisms $X \to X_i$ are closed
immersions.
\end{lemma}

\begin{proof}
By Proposition \ref{proposition-separated-closed-in-finite-presentation}
we can find a closed immersion $X \to Y$ with $Y$ separated and of
finite presentation over $S$. By Lemma \ref{lemma-chow-finite-type}
we can find a diagram
$$
\xymatrix{
Y \ar[rd] & Y' \ar[d] \ar[l]^\pi \ar[r] & \mathbf{P}^n_S \ar[dl] \\
& S &
}
$$
where $Y' \to \mathbf{P}^n_S$ is an immersion, and
$\pi : Y' \to Y$ is proper and surjective. By
Lemma \ref{lemma-closed-is-limit-closed-and-finite-presentation}
we can write $X = \lim X_i$ with $X_i \to Y$ a closed immersion of
finite presentation. Denote $X'_i \subset Y'$, resp.\ $X' \subset Y'$
the scheme theoretic inverse image of $X_i \subset Y$, resp.\ $X \subset Y$.
Then $\lim X'_i = X'$. Since $X' \to S$ is proper
(Morphisms, Lemmas \ref{morphisms-lemma-composition-proper}), we see that
$X' \to \mathbf{P}^n_S$ is a closed immersion (Morphisms, Lemma
\ref{morphisms-lemma-image-proper-scheme-closed}). Hence for $i$ large enough
we find that $X'_i \to \mathbf{P}^n_S$ is a closed immersion by
Lemma \ref{lemma-eventually-closed-immersion}.
Thus $X'_i$ is proper over $S$.
For such $i$ the morphism $X_i \to S$ is proper by
Morphisms, Lemma \ref{morphisms-lemma-image-proper-is-proper}.
\end{proof}

\begin{lemma}
\label{lemma-proper-limit-of-proper-finite-presentation-noetherian}
Let $f : X \to S$ be a proper morphism with $S$ quasi-compact and
quasi-separated. Then there exists a directed set $I$, an
inverse system $(f_i : X_i \to S_i)$ of morphisms of schemes over $I$,
such that the transition morphisms $X_i \to X_{i'}$ and $S_i \to S_{i'}$
are affine, such that $f_i$ is proper, such that $S_i$ is of finite
type over $\mathbf{Z}$, and such that
$(X \to S) = \lim (X_i \to S_i)$.
\end{lemma}

\begin{proof}
By Lemma \ref{lemma-proper-limit-of-proper-finite-presentation}
we can write $X = \lim_{k \in K} X_k$ with $X_k \to S$ proper and
of finite presentation. Next, by absolute Noetherian approximation
(Proposition \ref{proposition-approximate}) we can
write $S = \lim_{j \in J} S_j$ with $S_j$ of finite type over $\mathbf{Z}$.
For each $k$ there exists a $j$ and a morphism $X_{k, j} \to S_j$
of finite presentation with $X_k \cong S \times_{S_j} X_{k, j}$
as schemes over $S$, see
Lemma \ref{lemma-descend-finite-presentation}.
After increasing $j$ we may assume $X_{k, j} \to S_j$ is proper, see
Lemma \ref{lemma-eventually-proper}. The set $I$ will be consist
of these pairs $(k, j)$ and the corresponding morphism is $X_{k, j} \to S_j$.
For every $k' \geq k$ we can find a $j' \geq j$ and a morphism
$X_{j', k'} \to X_{j, k}$ over $S_{j'} \to S_j$ whose base change to $S$
gives the morphism $X_{k'} \to X_k$ (follows again from
Lemma \ref{lemma-descend-finite-presentation}).
These morphisms form the transition morphisms of the system. Some details
omitted.
\end{proof}

\noindent
Recall the scheme theoretic support of
a finite type quasi-coherent module, see
Morphisms, Definition \ref{morphisms-definition-scheme-theoretic-support}.

\begin{lemma}
\label{lemma-eventually-proper-support}
Assumptions and notation as in Situation \ref{situation-descent-property}.
Let $\mathcal{F}_0$ be a quasi-coherent $\mathcal{O}_{X_0}$-module.
Denote $\mathcal{F}$ and $\mathcal{F}_i$ the pullbacks of
$\mathcal{F}_0$ to $X$ and $X_i$. Assume
\begin{enumerate}
\item $f_0$ is locally of finite type,
\item $\mathcal{F}_0$ is of finite type,
\item the scheme theoretic support of $\mathcal{F}$ is proper over $Y$.
\end{enumerate}
Then the scheme theoretic support of $\mathcal{F}_i$ is proper over $Y_i$
for some $i$.
\end{lemma}

\begin{proof}
We may replace $X_0$ by the scheme theoretic support of $\mathcal{F}_0$.
By Morphisms, Lemma \ref{morphisms-lemma-support-finite-type} this
guarantees that $X_i$ is the support of $\mathcal{F}_i$ and $X$ is the
support of $\mathcal{F}$. Then, if $Z \subset X$ denotes the scheme
theoretic support of $\mathcal{F}$, we see that $Z \to X$ is a universal
homeomorphism. We conclude that $X \to Y$ is proper as this is true for
$Z \to Y$ by assumption, see
Morphisms, Lemma \ref{morphisms-lemma-image-proper-is-proper}.
By Lemma \ref{lemma-eventually-proper} we see that $X_i \to Y$ is proper
for some $i$. Then it follows that the scheme theoretic support $Z_i$ of
$\mathcal{F}_i$ is proper over $Y$ by
Morphisms, Lemmas \ref{morphisms-lemma-closed-immersion-proper} and
\ref{morphisms-lemma-composition-proper}.
\end{proof}











\section{Universally closed morphisms}
\label{section-universally-closed}

\noindent
In this section we discuss when a quasi-compact (but not necessarily
separated) morphism is universally closed. We first prove a lemma which
will allow us to check universal closedness after a base change
which is locally of finite presentation.

\begin{lemma}
\label{lemma-separate}
Let $f : X \to S$ be a quasi-compact morphism of schemes.
Let $g : T \to S$ be a morphism of schemes.
Let $t \in T$ be a point and $Z \subset X_T$ be a closed
subscheme such that $Z \cap X_t = \emptyset$.
Then there exists an open neighbourhood
$V \subset T$ of $t$, a commutative diagram
$$
\xymatrix{
V \ar[d] \ar[r]_a & T' \ar[d]^b \\
T \ar[r]^g & S,
}
$$
and a closed subscheme $Z' \subset X_{T'}$ such that
\begin{enumerate}
\item the morphism $b : T' \to S$ is locally of finite presentation,
\item with $t' = a(t)$ we have $Z' \cap X_{t'} = \emptyset$, and
\item $Z \cap X_V$ maps into $Z'$ via the morphism $X_V \to X_{T'}$.
\end{enumerate}
Moreover, we may assume $V$ and $T'$ are affine.
\end{lemma}

\begin{proof}
Let $s = g(t)$. During the proof we may always replace $T$ by an
open neighbourhood of $t$. Hence we may also replace $S$ by an open
neighbourhood of $s$. Thus we may and do assume that $T$ and $S$ are affine.
Say $S = \Spec(A)$, $T = \Spec(B)$, $g$ is given by the
ring map $A \to B$, and $t$ correspond to the prime ideal
$\mathfrak q \subset B$.

\medskip\noindent
As $X \to S$ is quasi-compact and $S$ is affine we may write
$X = \bigcup_{i = 1, \ldots, n} U_i$ as a finite union of affine opens.
Write $U_i = \Spec(C_i)$. In particular we have
$X_T = \bigcup_{i = 1, \ldots, n} U_{i, T} =
\bigcup_{i = 1, \ldots n} \Spec(C_i \otimes_A B)$.
Let $I_i \subset C_i \otimes_A B$ be the ideal corresponding to the
closed subscheme $Z \cap U_{i, T}$. The condition that
$Z \cap X_t = \emptyset$ signifies that $I_i$ generates the
unit ideal in the ring
$$
C_i \otimes_A \kappa(\mathfrak q) =
(B \setminus \mathfrak q)^{-1}\left(
C_i \otimes_A B/\mathfrak q C_i \otimes_A B \right)
$$
Since $I_i (B \setminus \mathfrak q)^{-1}(C_i \otimes_A B) =
(B \setminus \mathfrak q)^{-1} I_i$ this means that $1 = x_i/g_i$
for some $x_i \in I_i$ and $g_i \in B$, $g_i \not \in \mathfrak q$.
Thus, clearing denominators we can find a relation of the form
$$
x_i + \sum\nolimits_j f_{i, j}c_{i, j} = g_i
$$
with $x_i \in I_i$, $f_{i, j} \in \mathfrak q$, $c_{i, j} \in C_i \otimes_A B$,
and $g_i \in B$, $g_i \not \in \mathfrak q$. After replacing $B$ by
$B_{g_1 \ldots g_n}$, i.e., after replacing $T$ by a smaller affine
neighbourhood of $t$, we may assume the equations read
$$
x_i + \sum\nolimits_j f_{i, j}c_{i, j} = 1
$$
with $x_i \in I_i$, $f_{i, j} \in \mathfrak q$, $c_{i, j} \in C_i \otimes_A B$.

\medskip\noindent
To finish the argument write $B$ as a colimit of finitely presented
$A$-algebras $B_\lambda$ over a directed set $\Lambda$.
For each $\lambda$ set
$\mathfrak q_\lambda = (B_\lambda \to B)^{-1}(\mathfrak q)$.
For sufficiently large $\lambda \in \Lambda$ we can find
\begin{enumerate}
\item an element
$x_{i, \lambda} \in C_i \otimes_A B_\lambda$ which maps to $x_i$,
\item elements $f_{i, j, \lambda} \in \mathfrak q_{i, \lambda}$
mapping to $f_{i, j}$, and
\item elements $c_{i, j, \lambda} \in C_i \otimes_A B_\lambda$
mapping to $c_{i, j}$.
\end{enumerate}
After increasing $\lambda$ a bit more the equation
$$
x_{i, \lambda} + \sum\nolimits_j f_{i, j, \lambda}c_{i, j, \lambda} = 1
$$
will hold. Fix such a $\lambda$ and set $T' = \Spec(B_\lambda)$.
Then $t' \in T'$ is the point corresponding to the prime $\mathfrak q_\lambda$.
Finally, let $Z' \subset X_{T'}$ be the scheme theoretic image of
$Z \to X_T \to X_{T'}$. As $X_T \to X_{T'}$ is affine, we can compute $Z'$
on the affine open pieces $U_{i, T'}$ as the closed subscheme associated
to $\Ker(C_i \otimes_A B_\lambda \to C_i \otimes_A B/I_i)$, see
Morphisms, Example \ref{morphisms-example-scheme-theoretic-image}.
Hence $x_{i, \lambda}$ is in the ideal defining $Z'$. Thus the last
displayed equation shows that $Z' \cap X_{t'}$ is empty.
\end{proof}

\begin{lemma}
\label{lemma-test-universally-closed}
Let $f : X \to S$ be a quasi-compact morphism of schemes.
The following are equivalent
\begin{enumerate}
\item $f$ is universally closed,
\item for every morphism $S' \to S$ which is locally of finite presentation
the base change $X_{S'} \to S'$ is closed, and
\item for every $n$ the morphism
$\mathbf{A}^n \times X \to \mathbf{A}^n \times S$
is closed.
\end{enumerate}
\end{lemma}

\begin{proof}
It is clear that (1) implies (2). Let us prove that (2) implies (1).
Suppose that the base change $X_T \to T$ is not closed for some
scheme $T$ over $S$. By
Schemes, Lemma \ref{schemes-lemma-quasi-compact-closed}
this means that there exists some specialization $t_1 \leadsto t$ in
$T$ and a point $\xi \in X_T$ mapping to $t_1$ such that $\xi$ does not
specialize to a point in the fibre over $t$. Set
$Z = \overline{\{\xi\}} \subset X_T$. Then $Z \cap X_t = \emptyset$. Apply
Lemma \ref{lemma-separate}.
We find an open neighbourhood $V \subset T$ of $t$, a commutative diagram
$$
\xymatrix{
V \ar[d] \ar[r]_a & T' \ar[d]^b \\
T \ar[r]^g & S,
}
$$
and a closed subscheme $Z' \subset X_{T'}$ such that
\begin{enumerate}
\item the morphism $b : T' \to S$ is locally of finite presentation,
\item with $t' = a(t)$ we have $Z' \cap X_{t'} = \emptyset$, and
\item $Z \cap X_V$ maps into $Z'$ via the morphism $X_V \to X_{T'}$.
\end{enumerate}
Clearly this means that $X_{T'} \to T'$ maps the closed subset $Z'$
to a subset of $T'$ which contains $a(t_1)$ but not $t' = a(t)$.
Since $a(t_1) \leadsto a(t) = t'$ we conclude that $X_{T'} \to T'$
is not closed. Hence we have shown that $X \to S$ not universally closed
implies that $X_{T'} \to T'$ is not closed for some $T' \to S$
which is locally of finite presentation. In order words (2)
implies (1).

\medskip\noindent
Assume that $\mathbf{A}^n \times X \to \mathbf{A}^n \times S$ is
closed for every integer $n$. We want to prove that $X_T \to T$ is
closed for every scheme $T$ which is locally of finite presentation
over $S$. We may of course assume that $T$ is affine and maps into
an affine open $V$ of $S$ (since $X_T \to T$ being a closed is local on $T$).
In this case there exists a closed immersion $T \to \mathbf{A}^n \times V$
because $\mathcal{O}_T(T)$ is a finitely presented
$\mathcal{O}_S(V)$-algebra, see
Morphisms,
Lemma \ref{morphisms-lemma-locally-finite-presentation-characterize}.
Then $T \to \mathbf{A}^n \times S$ is a locally closed immersion.
Hence we get a cartesian diagram
$$
\xymatrix{
X_T \ar[d]_{f_T} \ar[r] & \mathbf{A}^n \times X \ar[d]^{f_n} \\
T \ar[r] & \mathbf{A}^n \times S
}
$$
of schemes where the horizontal arrows are locally closed immersions.
Hence any closed subset $Z \subset X_T$ can be written as
$X_T \cap Z'$ for some closed subset $Z' \subset \mathbf{A}^n \times X$.
Then $f_T(Z) = T \cap f_n(Z')$ and we see that if $f_n$ is closed, then
also $f_T$ is closed.
\end{proof}

\begin{lemma}
\label{lemma-limited-base-change}
Let $S$ be a scheme.
Let $f : X \to S$ be a separated morphism of finite type.
The following are equivalent:
\begin{enumerate}
\item The morphism $f$ is proper.
\item For any morphism $S' \to S$ which is locally of finite type
the base change $X_{S'} \to S'$ is closed.
\item For every $n \geq 0$ the morphism
$\mathbf{A}^n \times X \to \mathbf{A}^n \times S$ is closed.
\end{enumerate}
\end{lemma}

\begin{proof}[First proof]
In view of the fact that a proper morphism is the same thing as
a separated, finite type, and universally closed morphism, this
lemma is a special case of Lemma \ref{lemma-test-universally-closed}.
\end{proof}

\begin{proof}[Second proof]
Clearly (1) implies (2), and (2) implies (3), so we just need to show (3)
implies (1).
First we reduce to the case when $S$ is affine.  Assume that (3) implies (1)
when the base is affine.  Now let $f: X \to S$ be a separated morphism of
finite type.  Being proper is local on the base
(see Morphisms, Lemma \ref{morphisms-lemma-proper-local-on-the-base}), so if
$S = \bigcup_\alpha S_\alpha$ is an open affine cover, and if
we denote $X_\alpha := f^{-1}(S_\alpha)$, then it is
enough to show that $f|_{X_\alpha}: X_\alpha \to S_\alpha$ is proper
for all $\alpha$.  Since $S_\alpha$ is affine, if
the map $f|_{X_\alpha}$ satisfies (3), then it will satisfy (1)
by assumption, and will be proper.  To finish the reduction to the
case $S$ is affine, we must show that if $f: X \to S$ is separated of
finite type satisfying (3), then $f|_{X_\alpha} : X_\alpha \to S_\alpha$
is separated of finite type satisfying (3).  Separatedness and finite
type are clear.  To see (3), notice that
$\mathbf{A}^n \times X_\alpha$ is the open preimage of
$\mathbf{A}^n \times S_\alpha$ under the map $1 \times f$.  Fix a closed
set $Z \subset \mathbf A^n \times X_\alpha$.  Let $\bar Z$ denote the
closure of $Z$ in $\mathbf{A}^n \times X$.  Then for topological
reasons,
$$
1 \times f(\bar Z) \cap \mathbf{A}^n \times S_\alpha  = 1 \times f(Z).
$$
Hence $1 \times f(Z)$ is closed, and we have reduced the proof of
(3) $\Rightarrow$ (1) to the affine case.

\medskip\noindent
Assume $S$ affine, and $f : X \to S$ separated of finite type.
We can apply Chow's Lemma \ref{lemma-chow-finite-type}
to get $\pi : X' \to X$ proper surjective and $X' \to \mathbf{P}^n_S$
an immersion. If $X$ is proper over $S$, then $X' \to S$ is proper
(Morphisms, Lemma \ref{morphisms-lemma-composition-proper}). Since
$\mathbf{P}^n_S \to S$ is separated, we conclude that $X' \to
\mathbf{P}^n_S$ is proper
(Morphisms, Lemma \ref{morphisms-lemma-image-proper-scheme-closed})
and hence a closed immersion
(Schemes, Lemma \ref{schemes-lemma-immersion-when-closed}).
Conversely, assume $X' \to \mathbf{P}^n_S$ is a closed immersion.
Consider the diagram:
\begin{equation}
\label{equation-check-proper}
\xymatrix{
X' \ar[r] \ar@{->>}[d]_{\pi} &
\mathbf{P}^n_S \ar[d] \\
X \ar[r]^f & S
}
\end{equation}
All maps are a priori proper except for $X \to S$.
Hence we conclude that $X \to S$ is proper by
Morphisms, Lemma \ref{morphisms-lemma-image-proper-is-proper}.
Therefore, we have shown that $X \to S$ is proper if and only if
$X' \to \mathbf{P}^n_S$ is a closed immersion.

\medskip\noindent
Assume $S$ is affine and (3) holds, and let $n, X', \pi$ be as above.
Since being a closed morphism is local on the base, the map
$X \times \mathbf{P}^n \to S \times \mathbf{P}^n$ is closed since by (3)
$X \times \mathbf{A}^n \to S \times \mathbf{A}^n$ is closed and since
projective space is covered by copies of affine $n$-space, see
Constructions,
Lemma \ref{constructions-lemma-standard-covering-projective-space}.
By Morphisms, Lemma \ref{morphisms-lemma-base-change-proper}
the morphism
$$
X' \times_S \mathbf{P}^n_S
\to
X \times_S \mathbf{P}^n_S =
X \times \mathbf{P}^n
$$
is proper. Since $\mathbf{P}^n$ is separated, the projection
$$
X' \times_S \mathbf{P}^n_S = \mathbf{P}^n_{X'} \to X'
$$
will be separated as it is just a base change of a separated
morphism. Therefore, the map $X' \to X' \times_S \mathbf{P}^n_S$ is proper,
since it is a section to a separated map (see
Schemes, Lemma \ref{schemes-lemma-section-immersion}).
Composing these morphisms
$$
X' \to X' \times_S \mathbf{P}^n_S \to X \times_S \mathbf{P}^n_S
= X \times \mathbf{P}^n \to S \times \mathbf{P}^n = \mathbf{P}^n_S
$$
we find that the immersion $X' \to \mathbf{P}^n_S$ is closed, and
hence a closed immersion.
\end{proof}





\section{Noetherian valuative criterion}
\label{section-Noetherian-valuative-criterion}

\noindent
If the base is Noetherian we can show that the valuative criterion holds
using only discrete valuation rings.

\medskip\noindent
Many of the results in this section can (and perhaps should)
be proved by appealing to the following lemma, although we have not
always done so.

\begin{lemma}
\label{lemma-reach-point-closure-Noetherian}
Let $f : X \to Y$ be a morphism of schemes.
Assume $f$ finite type and $Y$ locally Noetherian.
Let $y \in Y$ be a point in the closure of the image of $f$.
Then there exists a commutative diagram
$$
\xymatrix{
\Spec(K) \ar[r] \ar[d] & X \ar[d]^f \\
\Spec(A) \ar[r] & Y
}
$$
where $A$ is a discrete valuation ring and $K$ is its field of fractions
mapping the closed point of $\Spec(A)$ to $y$. Moreover, we can assume
that the image point of $\Spec(K) \to X$ is a generic point $\eta$
of an irreducible component of $X$ and that $K = \kappa(\eta)$.
\end{lemma}

\begin{proof}
By the non-Noetherian version of this lemma
(Morphisms, Lemma \ref{morphisms-lemma-reach-points-scheme-theoretic-image})
there exists a point $x \in X$ such that $f(x)$ specializes to $y$.
We may replace $x$ by any point specializing to $x$, hence we may
assume that $x$ is a generic point of an irreducible component of $X$.
This produces a ring map $\mathcal{O}_{Y, y} \to \kappa(x)$
(see Schemes, Section \ref{schemes-section-points}).
Let $R \subset \kappa(x)$ be the image. Then $R$ is Noetherian as a quotient
of the Noetherian local ring $\mathcal{O}_{Y, y}$.
On the other hand, the extension $\kappa(x)$ is
a finitely generated extension of the fraction field of $R$
as $f$ is of finite type.
Thus there exists a discrete valuation ring $A \subset \kappa(x)$
with fraction field $\kappa(x)$ dominating $R$ by
Algebra, Lemma \ref{algebra-lemma-exists-dvr}. Then
$$
\xymatrix{
\Spec(\kappa(x)) \ar[d] \ar[rrr] & & & X \ar[d] \\
\Spec(A) \ar[r] & \Spec(R) \ar[r] & \Spec(\mathcal{O}_{Y, y}) \ar[r] & Y
}
$$
gives the desired diagram.
\end{proof}

\noindent
First we state the result concerning
separation. We will often use solid commutative diagrams of morphisms of
schemes having the following shape
\begin{equation}
\label{equation-valuative}
\vcenter{
\xymatrix{
\Spec(K) \ar[r] \ar[d] & X \ar[d] \\
\Spec(A) \ar[r] \ar@{-->}[ru] & S
}
}
\end{equation}
with $A$ a valuation ring and $K$ its field of fractions.

\begin{lemma}
\label{lemma-Noetherian-dvr-valuative-separation}
Let $S$ be a locally Noetherian scheme.
Let $f : X \to S$ be a morphism of schemes.
Assume $f$ is locally of finite type.
The following are equivalent:
\begin{enumerate}
\item The morphism $f$ is separated.
\item For any diagram (\ref{equation-valuative}) there is at most
one dotted arrow.
\item For all diagrams (\ref{equation-valuative}) with $A$ a discrete
valuation ring there is at most one dotted arrow.
\item For any irreducible component $X_0$ of $X$ with
generic point $\eta \in X_0$, for any discrete valuation ring
$A \subset K = \kappa(\eta)$ with fraction field $K$ and any
diagram (\ref{equation-valuative}) such that
the morphism $\Spec(K) \to X$ is the canonical one
(see Schemes, Section \ref{schemes-section-points})
there is at most one dotted arrow.
\end{enumerate}
\end{lemma}

\begin{proof}
Clearly (1) implies (2), (2) implies (3), and (3) implies (4).  It
remains to show (4) implies (1). Assume (4).
We begin by reducing to $S$ affine.  Being separated is a local
on the base (see
Schemes, Lemma \ref{schemes-lemma-characterize-separated}).
Hence, if we can show that whenever
$X \to S$ has (4) that the restriction $X_\alpha \to S_\alpha$ has (4)
where $S_\alpha \subset S$ is an (affine) open subset and $X_\alpha :=
f^{-1}(S_\alpha)$, then we will be done.  The
generic points of the irreducible components of $X_\alpha$ will be the
generic points of irreducible components of $X$, since $X_\alpha$ is
open in $X$.  Therefore, any two distinct dotted arrows in the diagram
\begin{equation}
\label{equation-valuative-alpha}
\xymatrix{
\Spec(K) \ar[r] \ar[d] & X_\alpha \ar[d] \\
\Spec(A) \ar[r] \ar@{-->}[ru] & S_\alpha
}
\end{equation}
would then give two distinct arrows in diagram
(\ref{equation-valuative}) via the maps $X_\alpha \to X$ and
$S_\alpha \to S$, which is a contradiction.  Thus we have reduced
to the case $S$ is affine.  We remark that in the course of this
reduction, we prove that if $X \to S$ has (4) then the restriction $U
\to V$ has (4) for opens $U \subset X$ and $V \subset S$ with
$f(U) \subset V$.

\medskip\noindent
We next wish to reduce to the case $X \to S$ is finite type.  Assume
that we know (4) implies (1) when $X$ is finite type. Since
$S$ is Noetherian and $X$ is locally of finite type over $S$
we see $X$ is locally Noetherian as well (see Morphisms,
Lemma \ref{morphisms-lemma-finite-type-noetherian}).
Thus, $X \to S$ is quasi-separated (see
Properties, Lemma \ref{properties-lemma-locally-Noetherian-quasi-separated}),
and therefore we may apply the valuative criterion to check whether $X$
is separated (see
Schemes, Lemma \ref{schemes-lemma-valuative-criterion-separatedness}).
Let $X = \bigcup_\alpha X_\alpha$ be an affine open
cover of $X$. Given any two dotted arrows, in a diagram
(\ref{equation-valuative}), the image of the closed points of
$\Spec A$ will
fall in two sets $X_\alpha$ and $X_\beta$.  Since $X_\alpha \cup
X_\beta$ is open, for topological reasons it must contain the image of
$\Spec(A)$ under both maps. Therefore, the two dotted arrows factor
through $X_\alpha \cup X_\beta \to X$, which is a scheme of finite type over
$S$. Since $X_\alpha \cup X_\beta$ is an open subset of $X$, by our
previous remark, $X_\alpha \cup X_\beta$ satisfies (4), so by
assumption, is separated.  This implies the two given dotted
arrows are the same. Therefore, we have reduced to $X \to S$ is finite type.

\medskip\noindent
Assume $X \to S$ of finite type and assume (4).
Since $X \to S$ is finite type, and $S$ is an affine Noetherian
scheme, $X$ is also Noetherian (see
Morphisms, Lemma \ref{morphisms-lemma-finite-type-noetherian}).
Therefore, $X \to X \times_S X$ will
be a quasi-compact immersion of Noetherian schemes.  We proceed by
contradiction.  Assume that $X \to X \times_S X$ is not closed.  Then,
there is some $y \in X \times_S X$ in the closure of the image that is
not in the image. As $X$ is Noetherian it has finitely many irreducible
components. Therefore, $y$ is in the closure of the image of one of
the irreducible components $X_0 \subset X$.  Give $X_0$ the reduced
induced structure.  The composition $X_0 \to X \to X \times_S X$
factors through the closed subscheme $X_0 \times_S X_0 \subset X \times_S X$.
Denote the closure of $\Delta(X_0)$ in $X_0 \times_S X_0$
by $\bar X_0$ (again as a reduced closed subscheme). Thus $y \in \bar X_0$.
Since $X_0 \to X_0 \times_S X_0$ is an immersion, the image of $X_0$
will be open in $\bar X_0$. Hence $X_0$ and $\bar X_0$ are
birational. Since $\bar{X}_0$ is a closed subscheme of a
Noetherian scheme, it is Noetherian. Thus, the local ring
$\mathcal O_{{\bar X_0, y}}$ is a local Noetherian domain with fraction
field $K$ equal to the function field of $X_0$.  By the Krull-Akizuki
theorem (see Algebra, Lemma \ref{algebra-lemma-exists-dvr}), there exists a
discrete valuation ring $A$ dominating $\mathcal O_{{\bar X_0, y}}$
with fraction field $K$.  This allows to construct a diagram:
\begin{equation}
\label{equation-valuative-generic}
\xymatrix{
\Spec(K) \ar[r] \ar[d] & X_0 \ar[d]^{\Delta} \\
\Spec(A) \ar[r] \ar@{-->}[ur]& X_0 \times_S X_0 \\
}
\end{equation}
which sends $\Spec K$ to the generic point of $\Delta(X_0)$ and
the closed point of $A$ to $y \in X_0 \times_S X_0$ (use the material in
Schemes, Section \ref{schemes-section-points} to construct the arrows).
There cannot even exist
a set theoretic dotted arrow, since $y$ is not in the image of
$\Delta$ by our choice of $y$.  By categorical means, the existence of
the dotted arrow in the above diagram is equivalent to the uniqueness
of the dotted arrow in the following diagram:
\begin{equation}
\label{equation-valuative-nonexistent}
\xymatrix{
\Spec(K) \ar[r] \ar[d] & X_0 \ar[d]\\
\Spec(A) \ar[r] \ar@{-->}[ur] & S \\
}
\end{equation}
Therefore, we have non-uniqueness in this latter diagram by the
nonexistence in the first.  Therefore, $X_0$ does not satisfy
uniqueness for discrete valuation rings, and since $X_0$ is an
irreducible component of $X$, we have that $X \to S$ does not satisfy
(4).  Therefore, we have shown (4) implies (1).
\end{proof}

\begin{lemma}
\label{lemma-Noetherian-dvr-valuative-proper}
Let $S$ be a locally Noetherian scheme.
Let $f : X \to S$ be a morphism of finite type.
The following are equivalent:
\begin{enumerate}
\item The morphism $f$ is proper.
\item For any diagram (\ref{equation-valuative}) there exists exactly
one dotted arrow.
\item For all diagrams (\ref{equation-valuative}) with $A$ a discrete
valuation ring there exists exactly one dotted arrow.
\item For any irreducible component $X_0$ of $X$ with
generic point $\eta \in X_0$, for any discrete valuation ring
$A \subset K = \kappa(\eta)$ with fraction field $K$ and any
diagram (\ref{equation-valuative}) such that
the morphism $\Spec(K) \to X$ is the canonical one
(see
Schemes, Section \ref{schemes-section-points})
there exists exactly one dotted arrow.
\end{enumerate}
\end{lemma}

\begin{proof}
(1) implies (2) implies (3) implies (4).  We will now show (4) implies
(1).  As in the proof of Lemma \ref{lemma-Noetherian-dvr-valuative-separation},
we can reduce to the
case $S$ is affine, since properness is local on the base, and if $X
\to S$ satisfies (4), then $X_\alpha \to S_\alpha$ does as well for
open $S_\alpha \subset S$ and $X_\alpha = f^{-1}(S_\alpha)$.

\medskip\noindent
Now $S$ is a Noetherian scheme, and so $X$ is as well, since $X \to
S$ is of finite type.  Now we may use Chow's lemma
(Cohomology of Schemes, Lemma \ref{coherent-lemma-chow-Noetherian})
to get a surjective, proper, birational
$X' \to X$ and an immersion $X' \to \mathbf{P}^n_S$.  We wish to
show $X \to S$ is universally closed.  As in the proof of Lemma
\ref{lemma-limited-base-change}, it is enough to check that
$X' \to \mathbf{P}^n_S$ is a closed immersion.
For the sake of contradiction, assume that $X' \to
\mathbf{P}^n_S$ is not a closed immersion.  Then there is some $y
\in \mathbf{P}^n_S$ that is in the closure of the image of $X'$, but
is not in the image.  So $y$ is in the closure of the image of an
irreducible component $X_0'$ of $X'$, but not in the image.
Let $\bar X_0' \subset \mathbf{P}^n_S$ be the closure of
the image of $X_0'$. As $X' \to \mathbf{P}^n_S$ is an immersion
of Noetherian schemes, the morphism $X'_0 \to \bar X_0'$ is
open and dense. By
Algebra, Lemma \ref{algebra-lemma-exists-dvr}
or
Properties, Lemma \ref{properties-lemma-locally-Noetherian-specialization-dvr}
we can find a discrete valuation ring $A$ dominating
$\mathcal{O}_{\bar X_0', y}$ and with identical field
of fractions $K$. It is clear that
$K$ is the residue field at the generic point of $X_0'$.
Thus the solid commutative diagram
\begin{equation}
\label{equation-solid}
\xymatrix{
\Spec K \ar[r] \ar[d] & X' \ar [r] \ar[d] &
\mathbf{P}^n_S \ar[d] \\
\Spec A \ar@{-->}[r] \ar@{-->}[ru] \ar[urr] & X \ar[r] & S\\
}
\end{equation}
Note that the closed point of $A$ maps to $y \in \mathbf{P}^n_S$.  By
construction, there does not exist a set theoretic lift to $X'$.
As $X' \to X$ is birational, the image of $X'_0$ in $X$ is an
irreducible component $X_0$ of $X$ and $K$ is also identified with
the function field of $X_0$. Hence, as $X \to S$ is assumed to satisfy (4),
the dotted arrow $\Spec(A) \to X$ exists.
Since $X' \to X$ is proper, the dotted
arrow lifts to the dotted arrow $\Spec(A) \to X'$ (use Schemes,
Proposition \ref{schemes-proposition-characterize-universally-closed}).
We can compose this with the immersion $X' \to \mathbf{P}^n_S$ to obtain
another morphism (not depicted in the diagram) from
$\Spec(A) \to \mathbf{P}^n_S$.  Since $\mathbf{P}^n_S$
is proper over $S$, it satisfies (2), and so these two morphisms
agree.  This is a contradiction, for we have constructed the
forbidden lift of our original map $\Spec(A) \to \mathbf{P}^n_S$
to $X'$.
\end{proof}

\begin{lemma}
\label{lemma-check-universally-closed-Noetherian}
Let $f : X \to S$ be a finite type morphism of schemes.
Assume $S$ is locally Noetherian. Then the following are equivalent
\begin{enumerate}
\item $f$ is universally closed,
\item for every $n$ the morphism
$\mathbf{A}^n \times X \to \mathbf{A}^n \times S$ is closed,
\item for any diagram (\ref{equation-valuative}) there exists some
dotted arrow,
\item for all diagrams (\ref{equation-valuative}) with $A$ a discrete
valuation ring there exists some dotted arrow.
\end{enumerate}
\end{lemma}

\begin{proof}
The equivalence of (1) and (2) is a special case of
Lemma \ref{lemma-test-universally-closed}.
The equivalence of (1) and (3) is a special case of
Schemes, Proposition \ref{schemes-proposition-characterize-universally-closed}.
Trivially (3) implies (4).
Thus all we have to do is prove that (4) implies (2).
We will prove that $\mathbf{A}^n \times X \to \mathbf{A}^n \times S$
is closed by the criterion of
Schemes, Lemma \ref{schemes-lemma-quasi-compact-closed}.
Pick $n$ and a specialization $z \leadsto z'$ of points
in $\mathbf{A}^n \times S$ and a point $y \in \mathbf{A}^n \times X$
lying over $z$. Note that $\kappa(y)$ is a finitely generated field
extension of $\kappa(z)$ as $\mathbf{A}^n \times X \to \mathbf{A}^n \times S$
is of finite type. Hence by
Properties, Lemma \ref{properties-lemma-locally-Noetherian-specialization-dvr}
or
Algebra, Lemma \ref{algebra-lemma-exists-dvr}
implies that there exists a discrete valuation ring $A \subset \kappa(y)$
with fraction field $\kappa(z)$ dominating the image of
$\mathcal{O}_{\mathbf{A}^n \times S, z'}$ in $\kappa(z)$.
This gives a commutative diagram
$$
\xymatrix{
\Spec(\kappa(y)) \ar[r] \ar[d] &
\mathbf{A}^n \times X \ar[d] \ar[r] & X \ar[d] \\
\Spec(A) \ar[r] & \mathbf{A}^n \times S \ar[r] & S
}
$$
Now property (4) implies that there exists a morphism
$\Spec(A) \to X$ which fits into this diagram.
Since we already have the morphism $\Spec(A) \to \mathbf{A}^n$
from the left lower horizontal arrow we also get a morphism
$\Spec(A) \to \mathbf{A}^n \times X$ fitting into the
left square. Thus the image $y' \in \mathbf{A}^n \times X$
of the closed point is a specialization of $y$ lying over $z'$.
This proves that specializations lift along
$\mathbf{A}^n \times X \to \mathbf{A}^n \times S$
and we win.
\end{proof}

\noindent
In the Noetherian case one usually does not have to consider all
possible diagrams with discrete valuation rings when testing the
valuative criteria. We have already seen this
in Lemmas \ref{lemma-Noetherian-dvr-valuative-separation} and
\ref{lemma-Noetherian-dvr-valuative-proper}. Here is another variant.

\begin{lemma}
\label{lemma-refined-valuative-criterion-proper}
Let $f : X \to S$ and $h : U \to X$ be morphisms of schemes.
Assume that $S$ is locally Noetherian, that $f$ and $h$ are of finite type,
that $f$ is separated, and that $h(U)$ is dense in $X$.
If given any commutative solid diagram
$$
\xymatrix{
\Spec(K) \ar[r] \ar[d] & U \ar[r]^h & X \ar[d]^f \\
\Spec(A) \ar[rr] \ar@{-->}[rru] & & S
}
$$
where $A$ is a discrete valuation ring with field of fractions $K$, there
exists a dotted arrow making the diagram commute, then $f$ is proper.
\end{lemma}

\begin{proof}
There is an immediate reduction to the case where $S$ is affine.
Then $U$ is quasi-compact.
Let $U = U_1 \cup \ldots \cup U_n$ be an affine open covering.
We may replace $U$ by $U_1 \amalg \ldots \amalg U_n$ without
changing the assumptions, hence we may assume $U$ is affine.
Thus we can find an open immersion $U \to Y$ over $X$
with $Y$ proper over $X$. (First put $U$ inside $\mathbf{A}^n_X$
using Morphisms, Lemma \ref{morphisms-lemma-quasi-affine-finite-type-over-S}
and then take the closure inside $\mathbf{P}^n_X$, or you can directly
use Morphisms, Lemma \ref{morphisms-lemma-quasi-projective-open-projective}.)
We can assume $U$ is dense in $Y$ (replace $Y$ by the scheme theoretic
closure of $U$ if necessary, see Morphisms, Section
\ref{morphisms-section-scheme-theoretic-closure}).
Note that $g : Y \to X$ is surjective as the image is closed
and contains the dense subset $h(U)$.
We will show that $Y \to S$ is proper. This will imply that
$X \to S$ is proper by
Morphisms, Lemma \ref{morphisms-lemma-image-proper-is-proper}
thereby finishing the proof.
To show that $Y \to S$ is proper we will use
part (4) of Lemma \ref{lemma-Noetherian-dvr-valuative-proper}.
To do this consider a diagram
$$
\xymatrix{
\Spec(K) \ar[r]_y \ar[d] & Y \ar[d]^{f \circ g} \\
\Spec(A) \ar[r] \ar@{..>}[ru] & S
}
$$
where $A$ is a discrete valuation ring with fraction field $K$
and where $y : \Spec(K) \to Y$ is the inclusion of a generic point.
We have to show there exists a unique dotted arrow.
Uniqueness holds by the converse to the valuative criterion
for separatedness
(Schemes, Lemma \ref{schemes-lemma-separated-implies-valuative})
since $Y \to S$ is separated as the
composition of the separated morphisms $Y \to X$ and $X \to S$
(Schemes, Lemma \ref{schemes-lemma-separated-permanence}).
Existence can be seen as follows.
As $y$ is a generic point of $Y$, it is contained in $U$.
By assumption of the lemma
there exists a morphism $a : \Spec(A) \to X$ such that
$$
\xymatrix{
\Spec(K) \ar[r]_y \ar[d] & U \ar[r] & X \ar[d]^f \\
\Spec(A) \ar[rr] \ar[rru]^a & & S
}
$$
is commutative. Then since $Y \to X$ is proper, we can
apply the valuative criterion for properness
(Morphisms, Lemma \ref{morphisms-lemma-characterize-proper})
to find a morphism $b : \Spec(A) \to Y$ such that
$$
\xymatrix{
\Spec(K) \ar[r]_y \ar[d] & Y \ar[d]^g \\
\Spec(A) \ar[r]^a \ar[ru]^b & X
}
$$
is commutative. This finishes the proof since
$b$ can serve as the dotted arrow above.
\end{proof}

\begin{lemma}
\label{lemma-refined-valuative-criterion-separated}
Let $f : X \to S$ and $h : U \to X$ be morphisms of schemes.
Assume that $S$ is locally Noetherian, that $f$ is locally of finite type,
that $h$ is of finite type, and that $h(U)$ is dense in $X$.
If given any commutative solid diagram
$$
\xymatrix{
\Spec(K) \ar[r] \ar[d] & U \ar[r]^h & X \ar[d]^f \\
\Spec(A) \ar[rr] \ar@{-->}[rru] & & S
}
$$
where $A$ is a discrete valuation ring with field of fractions $K$, there
exists at most one dotted arrow making the diagram commute, then $f$ is
separated.
\end{lemma}

\begin{proof}
We will apply Lemma \ref{lemma-refined-valuative-criterion-proper}
to the morphisms $U \to X$ and $\Delta : X \to X \times_S X$.
We check the conditions. Observe that $\Delta$ is quasi-compact by
Properties, Lemma \ref{properties-lemma-locally-Noetherian-quasi-separated}
(and Schemes, Lemma \ref{schemes-lemma-compose-after-separated}).
Of course $\Delta$ is locally of finite type and separated (true
for any diagonal morphism).
Finally, suppose given a commutative solid diagram
$$
\xymatrix{
\Spec(K) \ar[r] \ar[d] & U \ar[r]^h & X \ar[d]^\Delta \\
\Spec(A) \ar[rr]^{(a, b)} \ar@{-->}[rru] & & X \times_S X
}
$$
where $A$ is a discrete valuation ring with field of fractions $K$.
Then $a$ and $b$ give two dotted arrows in the diagram of the lemma
and have to be equal. Hence as dotted arrow we can use $a = b$
which gives existence. This finishes the proof.
\end{proof}

\begin{lemma}
\label{lemma-refined-valuative-criterion-universally-closed}
Let $f : X \to S$ and $h : U \to X$ be morphisms of schemes.
Assume that $S$ is locally Noetherian, that $f$ and $h$ are of finite type, and
that $h(U)$ is dense in $X$. If given any commutative solid diagram
$$
\xymatrix{
\Spec(K) \ar[r] \ar[d] & U \ar[r]^h & X \ar[d]^f \\
\Spec(A) \ar[rr] \ar@{-->}[rru] & & S
}
$$
where $A$ is a discrete valuation ring with field of fractions $K$, there
exists a unique dotted arrow making the diagram commute, then $f$ is proper.
\end{lemma}

\begin{proof}
Combine Lemmas \ref{lemma-refined-valuative-criterion-separated} and
\ref{lemma-refined-valuative-criterion-proper}.
\end{proof}









\section{Limits and dimensions of fibres}
\label{section-limits-dimension}

\noindent
The following lemma is most often used in the situation of
Lemma \ref{lemma-descend-finite-presentation}
to assure that if the fibres of the limit have dimension $\leq d$,
then the fibres at some finite stage have dimension $\leq d$.

\begin{lemma}
\label{lemma-limit-dimension}
Let $I$ be a directed set.
Let $(f_i : X_i \to S_i)$ be an inverse system of morphisms of schemes
over $I$. Assume
\begin{enumerate}
\item all the morphisms $S_{i'} \to S_i$ are affine,
\item all the schemes $S_i$ are quasi-compact and quasi-separated,
\item the morphisms $f_i$ are of finite type, and
\item the morphisms $X_{i'} \to X_i \times_{S_i} S_{i'}$ are closed
immersions.
\end{enumerate}
Let $f : X = \lim_i X_i \to S = \lim_i S_i$ be the limit.
Let $d \geq 0$.
If every fibre of $f$ has dimension $\leq d$, then for some $i$
every fibre of $f_i$ has dimension $\leq d$.
\end{lemma}

\begin{proof}
For each $i$ let $U_i = \{x \in X_i \mid \dim_x((X_i)_{f_i(x)}) \leq d\}$.
This is an open subset of $X_i$, see
Morphisms, Lemma \ref{morphisms-lemma-openness-bounded-dimension-fibres}.
Set $Z_i = X_i \setminus U_i$ (with reduced induced scheme structure).
We have to show that $Z_i = \emptyset$ for some $i$.
If not, then $Z = \lim Z_i \not = \emptyset$, see
Lemma \ref{lemma-limit-nonempty}.
Say $z \in Z$ is a point. Note that $Z \subset X$ is a closed subscheme.
Set $s = f(z)$. For each $i$ let $s_i \in S_i$ be the image
of $s$. We remark that $Z_s$ is the limit of the schemes $(Z_i)_{s_i}$
and $Z_s$ is also the limit of the schemes $(Z_i)_{s_i}$ base
changed to $\kappa(s)$. Moreover, all the morphisms
$$
Z_s
\longrightarrow
(Z_{i'})_{s_{i'}} \times_{\Spec(\kappa(s_{i'}))} \Spec(\kappa(s))
\longrightarrow
(Z_i)_{s_i} \times_{\Spec(\kappa(s_i))} \Spec(\kappa(s))
\longrightarrow
X_s
$$
are closed immersions by assumption (4). Hence $Z_s$ is the scheme
theoretic intersection of the closed subschemes
$(Z_i)_{s_i} \times_{\Spec(\kappa(s_i))} \Spec(\kappa(s))$
in $X_s$. Since all the irreducible components of the schemes
$(Z_i)_{s_i} \times_{\Spec(\kappa(s_i))} \Spec(\kappa(s))$
have dimension $> d$ and contain $z$ we conclude that
$Z_s$ contains an irreducible component of dimension $> d$ passing
through $z$ which contradicts the fact that $Z_s \subset X_s$ and
$\dim(X_s) \leq d$.
\end{proof}

\begin{lemma}
\label{lemma-descend-quasi-finite}
Notation and assumptions as in Situation \ref{situation-descent-property}.
If
\begin{enumerate}
\item $f$ is a quasi-finite morphism, and
\item $f_0$ is locally of finite type,
\end{enumerate}
then there exists an $i \geq 0$ such that $f_i$ is quasi-finite.
\end{lemma}

\begin{proof}
Follows immediately from Lemma \ref{lemma-limit-dimension}.
\end{proof}

\begin{lemma}
\label{lemma-approximate-given-relative-dimension}
Let $S$ be a quasi-compact and quasi-separated scheme.
Let $f : X \to S$ be a morphism of finite presentation.
Let $d \geq 0$ be an integer.
If $Z \subset X$ be a closed subscheme such that
$\dim(Z_s) \leq d$ for all $s \in S$, then there exists a
closed subscheme $Z' \subset X$ such that
\begin{enumerate}
\item $Z \subset Z'$,
\item $Z' \to X$ is of finite presentation, and
\item $\dim(Z'_s) \leq d$ for all $s \in S$.
\end{enumerate}
\end{lemma}

\begin{proof}
By
Proposition \ref{proposition-approximate}
we can write $S = \lim S_i$ as the limit of a directed inverse
system of Noetherian schemes with affine transition maps. By
Lemma \ref{lemma-descend-finite-presentation}
we may assume that there exist a system of morphisms
$f_i : X_i \to S_i$ of finite presentation such that
$X_{i'} = X_i \times_{S_i} S_{i'}$
for all $i' \geq i$ and such that $X = X_i \times_{S_i} S$.
Let $Z_i \subset X_i$ be the scheme theoretic image of
$Z \to X \to X_i$. Then for $i' \geq i$ the morphism $X_{i'} \to X_i$
maps $Z_{i'}$ into $Z_i$ and the induced morphism
$Z_{i'} \to Z_i \times_{S_i} S_{i'}$ is a closed immersion. By
Lemma \ref{lemma-limit-dimension}
we see that the dimension of the fibres of $Z_i \to S_i$
all have dimension $\leq d$ for a suitable $i \in I$.
Fix such an $i$ and set $Z' = Z_i \times_{S_i} S \subset X$.
Since $S_i$ is Noetherian, we see that $X_i$ is Noetherian, and hence
the morphism $Z_i \to X_i$ is of finite presentation.
Therefore also the base change $Z' \to X$ is of finite presentation.
Moreover, the fibres of $Z' \to S$ are base changes of the fibres
of $Z_i \to S_i$ and hence have dimension $\leq d$.
\end{proof}

\begin{lemma}
\label{lemma-higher-direct-images-zero-above-dimension-fibre}
Let $f : X \to Y$ be a morphism of schemes. Let $y \in Y$.
Assume $f$ is proper and $\dim(X_y) = d$.
Then for any quasi-coherent $\mathcal{O}_X$-module $\mathcal{F}$
we have $(R^pf_*\mathcal{F})_y = 0$ for all $p > d$.
\end{lemma}

\begin{proof}
By Morphisms, Lemma
\ref{morphisms-lemma-openness-bounded-dimension-fibres}
and the fact that $f$ is closed, we can find an open neighbourhood
$V$ of $y$ such that the fibres over points of $V$ all have dimension $\leq d$.
Since the question is local on $Y$ we may assume $Y$ is affine
and all fibres of $f$ have dimension $\leq d$.
By Lemma \ref{lemma-proper-limit-of-proper-finite-presentation}
we can write $X = \lim X_i$ as a cofiltered limit with $X_i \to Y$ proper and
of finite presentation and such that both $X \to X_i$
and transition morphisms are closed immersions.
For some $i$ we have that $X_i \to Y$ has fibres of dimension $\leq d$,
see Lemma \ref{lemma-limit-dimension}.
For a quasi-coherent $\mathcal{O}_X$-module $\mathcal{F}$ we have
$R^pf_*\mathcal{F} = R^pf_{i, *}(X \to X_i)_*\mathcal{F}$ by
Cohomology of Schemes, Lemma \ref{coherent-lemma-relative-affine-vanishing}
and Leray (Cohomology, Lemma \ref{cohomology-lemma-relative-Leray}).
Thus we may replace $X$ by $X_i$ and
reduce to the case discussed in the next paragraph.

\medskip\noindent
Assume $Y$ is affine and $f : X \to Y$ is proper and of finite presentation
and all fibres have dimension $\leq d$. It suffices to show that
$H^p(X, \mathcal{F}) = 0$ for $p > d$. Namely, by
Cohomology of Schemes, Lemma
\ref{coherent-lemma-quasi-coherence-higher-direct-images-application}
we have $H^p(X, \mathcal{F}) = H^0(Y, R^pf_*\mathcal{F})$.
On the other hand, $R^pf_*\mathcal{F}$ is quasi-coherent on $Y$
by Cohomology of Schemes, Lemma
\ref{coherent-lemma-quasi-coherence-higher-direct-images},
hence vanishing of global sections implies vanishing.
Write $Y = \lim_{i \in I} Y_i$ as a cofiltered limit of affine schemes
with $Y_i$ the spectrum of a Noetherian ring
(for example a finite type $\mathbf{Z}$-algebra).
We can choose an element $0 \in I$ and a finite type morphism
$X_0 \to Y_0$ such that $X \cong Y \times_{Y_0} X_0$, see
Lemma \ref{lemma-descend-finite-presentation}.
After increasing $0$ we may assume $X_0 \to Y_0$ is proper
(Lemma \ref{lemma-eventually-proper})
and that the fibres of $X_0 \to Y_0$ have dimension $\leq d$
(Lemma \ref{lemma-limit-dimension}).
Since $X \to X_0$ is affine, we find that
$H^p(X, \mathcal{F}) = H^p(X_0, (X \to X_0)_*\mathcal{F})$ by
Cohomology of Schemes, Lemma \ref{coherent-lemma-relative-affine-cohomology}.
This reduces us to the case discussed in the next paragraph.

\medskip\noindent
Assume $Y$ is affine Noetherian and $f : X \to Y$ is proper
and all fibres have dimension $\leq d$.
In this case we can write $\mathcal{F} = \colim \mathcal{F}_i$
as a filtered colimit of coherent $\mathcal{O}_X$-modules, see
Properties, Lemma
\ref{properties-lemma-directed-colimit-finite-presentation}.
Then $H^p(X, \mathcal{F}) = \colim H^p(X, \mathcal{F}_i)$ by
Cohomology, Lemma \ref{cohomology-lemma-quasi-separated-cohomology-colimit}.
Thus we may assume $\mathcal{F}$ is coherent.
In this case we see that $(R^pf_*\mathcal{F})_y = 0$ for
all $y \in Y$ by Cohomology of Schemes, Lemma
\ref{coherent-lemma-higher-direct-images-zero-above-dimension-fibre}.
Thus $R^pf_*\mathcal{F} = 0$ and therefore
$H^p(X, \mathcal{F}) = 0$ (see above) and we win.
\end{proof}





\section{Glueing in closed fibres}
\label{section-change-over-closed-points}

\noindent
Applying our theory above to the spectrum of a local ring we obtain
the following pleasing glueing result for relative schemes.

\begin{lemma}
\label{lemma-glueing-near-closed-point}
Let $S$ be a scheme. Let $s \in S$ be a closed point such that
$U = S \setminus \{s\} \to S$ is quasi-compact. With
$V = \Spec(\mathcal{O}_{S, s}) \setminus \{s\}$ there is
an equivalence of categories
$$
\left\{
\begin{matrix}
X \to S\text{ of finite presentation}
\end{matrix}
\right\}
\longrightarrow
\left\{
\vcenter{
\xymatrix{
X' \ar[d] & Y' \ar[d] \ar[l] \ar[r] & Y \ar[d] \\
U & V \ar[l] \ar[r] & \Spec(\mathcal{O}_{S, s})
}
}
\right\}
$$
where on the right hand side we consider commutative diagrams
whose squares are cartesian and whose vertical arrows are
of finite presentation.
\end{lemma}

\begin{proof}
Let $W \subset S$ be an open neighbourhood of $s$. By
glueing of relative schemes, see
Constructions, Section \ref{constructions-section-relative-glueing},
the functor
$$
\left\{
\begin{matrix}
X \to S\text{ of finite presentation}
\end{matrix}
\right\}
\longrightarrow
\left\{
\vcenter{
\xymatrix{
X' \ar[d] & Y' \ar[d] \ar[l] \ar[r] & Y \ar[d] \\
U & W \setminus \{s\} \ar[l] \ar[r] & W
}
}
\right\}
$$
is an equivalence of categories. We have
$\mathcal{O}_{S, s} = \colim \mathcal{O}_W(W)$ where
$W$ runs over the affine open neighbourhoods of $s$.
Hence $\Spec(\mathcal{O}_{S, s}) = \lim W$ where $W$
runs over the affine open neighbourhoods of $s$.
Thus the category of schemes of finite presentation
over $\Spec(\mathcal{O}_{S, s})$ is the limit of the
category of schemes of finite presentation over
$W$ where $W$ runs over the affine open neighbourhoods
of $s$, see
Lemma \ref{lemma-descend-finite-presentation}.
For every affine open $s \in W$ we see that $U \cap W$
is quasi-compact as $U \to S$ is quasi-compact.
Hence $V = \lim W \cap U = \lim W \setminus \{s\}$ is a limit of
quasi-compact and quasi-separated schemes (see
Lemma \ref{lemma-directed-inverse-system-has-limit}).
Thus also the category of schemes of finite presentation
over $V$ is the limit of the
categories of schemes of finite presentation over
$W \cap U$ where $W$ runs over the affine open neighbourhoods
of $s$. The lemma follows formally from a combination
of these results.
\end{proof}

\begin{lemma}
\label{lemma-glueing-near-point}
Let $S$ be a scheme. Let $U \subset S$ be a retrocompact open.
Let $s \in S$ be a point in the complement of $U$. With
$V = \Spec(\mathcal{O}_{S, s}) \cap U$ there is
an equivalence of categories
$$
\colim_{s \in U' \supset U\text{ open}}
\left\{
\vcenter{
\xymatrix{
X \ar[d] \\
U'
}
}
\right\}
\longrightarrow
\left\{
\vcenter{
\xymatrix{
X' \ar[d] & Y' \ar[d] \ar[l] \ar[r] & Y \ar[d] \\
U & V \ar[l] \ar[r] & \Spec(\mathcal{O}_{S, s})
}
}
\right\}
$$
where on the left hand side the vertical arrow is of finite
presentation and on the right hand side we consider commutative diagrams
whose squares are cartesian and whose vertical arrows are
of finite presentation.
\end{lemma}

\begin{proof}
Let $W \subset S$ be an open neighbourhood of $s$. By
glueing of relative schemes, see
Constructions, Section \ref{constructions-section-relative-glueing},
the functor
$$
\left\{
\begin{matrix}
X \to U' = U \cup W \text{ of finite presentation}
\end{matrix}
\right\}
\longrightarrow
\left\{
\vcenter{
\xymatrix{
X' \ar[d] & Y' \ar[d] \ar[l] \ar[r] & Y \ar[d] \\
U & W \cap U \ar[l] \ar[r] & W
}
}
\right\}
$$
is an equivalence of categories. We have
$\mathcal{O}_{S, s} = \colim \mathcal{O}_W(W)$ where
$W$ runs over the affine open neighbourhoods of $s$.
Hence $\Spec(\mathcal{O}_{S, s}) = \lim W$ where $W$
runs over the affine open neighbourhoods of $s$.
Thus the category of schemes of finite presentation
over $\Spec(\mathcal{O}_{S, s})$ is the limit of the
category of schemes of finite presentation over
$W$ where $W$ runs over the affine open neighbourhoods
of $s$, see
Lemma \ref{lemma-descend-finite-presentation}.
For every affine open $s \in W$ we see that $U \cap W$
is quasi-compact as $U \to S$ is quasi-compact.
Hence $V = \lim W \cap U$ is a limit of
quasi-compact and quasi-separated schemes (see
Lemma \ref{lemma-directed-inverse-system-has-limit}).
Thus also the category of schemes of finite presentation
over $V$ is the limit of the
categories of schemes of finite presentation over
$W \cap U$ where $W$ runs over the affine open neighbourhoods
of $s$. The lemma follows formally from a combination
of these results.
\end{proof}

\begin{lemma}
\label{lemma-glueing-near-multiple-closed-points}
Let $S$ be a scheme. Let $s_1, \ldots, s_n \in S$ be pairwise distinct
closed points such that
$U = S \setminus \{s_1, \ldots, s_n\} \to S$ is quasi-compact. With
$S_i = \Spec(\mathcal{O}_{S, s_i})$ and $U_i = S_i \setminus \{s_i\}$
there is an equivalence of categories
$$
FP_S \longrightarrow
FP_U \times_{(FP_{U_1} \times \ldots \times FP_{U_n})}
(FP_{S_1} \times \ldots \times FP_{S_n})
$$
where $FP_T$ is the category of schemes of finite presentation over
the scheme $T$.
\end{lemma}

\begin{proof}
For $n = 1$ this is Lemma \ref{lemma-glueing-near-closed-point}.
For $n > 1$ the lemma can be proved in exactly the same way or it
can be deduced from it. For example, suppose that $f_i : X_i \to S_i$
are objects of $FP_{S_i}$ and $f : X \to U$ is an object
of $FP_U$ and we're given isomorphisms $X_i \times_{S_i} U_i = X \times_U U_i$.
By Lemma \ref{lemma-glueing-near-closed-point} we can find
a morphism $f' : X' \to U' = S \setminus \{s_1, \ldots, s_{n - 1}\}$
which is of finite presentation, which is isomorphic to
$X_i$ over $S_i$, which is isomorphic to $X$ over $U$, and
these isomorphisms are compatible with the given isomorphism
$X_i \times_{S_n} U_n = X \times_U U_n$.
Then we can apply induction to
$f_i : X_i \to S_i$, $i \leq n - 1$,
$f' : X' \to U'$, and the induced
isomorphisms $X_i \times_{S_i} U_i = X' \times_{U'} U_i$, $i \leq n - 1$.
This shows essential surjectivity. We omit the proof of
fully faithfulness.
\end{proof}






\section{Application to modifications}
\label{section-modifications-at-a-point}

\noindent
Using the results from Section \ref{section-change-over-closed-points}
we can describe the category of modifications of
a scheme over a closed point in terms of the local ring.

\begin{lemma}
\label{lemma-modifications}
Let $S$ be a scheme. Let $s \in S$ be a closed point such that
$U = S \setminus \{s\} \to S$ is quasi-compact. With
$V = \Spec(\mathcal{O}_{S, s}) \setminus \{s\}$ the base change functor
$$
\left\{
\begin{matrix}
f : X \to S\text{ of finite presentation} \\
f^{-1}(U) \to U\text{ is an isomorphism}
\end{matrix}
\right\}
\longrightarrow
\left\{
\begin{matrix}
g : Y \to \Spec(\mathcal{O}_{S, s})\text{ of finite presentation} \\
g^{-1}(V) \to V\text{ is an isomorphism}
\end{matrix}
\right\}
$$
is an equivalence of categories.
\end{lemma}

\begin{proof}
This is a special case of Lemma \ref{lemma-glueing-near-closed-point}.
\end{proof}

\begin{lemma}
\label{lemma-modifications-properties}
Notation and assumptions as in Lemma \ref{lemma-modifications}.
Let $f : X \to S$ correspond to $g : Y \to \Spec(\mathcal{O}_{S, s})$
via the equivalence. Then $f$ is separated, proper, finite,
and add more here if and only if $g$ is so.
\end{lemma}

\begin{proof}
The property of being separated, proper, integral, finite, etc
is stable under base change. See
Schemes, Lemma \ref{schemes-lemma-separated-permanence}
and
Morphisms, Lemmas \ref{morphisms-lemma-base-change-proper} and
\ref{morphisms-lemma-base-change-finite}.
Hence if $f$ has the property, then so does $g$.
Conversely, if $g$ does, then $f$ does in a neighbourhood of $s$ by
Lemmas \ref{lemma-descend-separated-finite-presentation},
\ref{lemma-eventually-proper}, and
\ref{lemma-descend-finite-finite-presentation}.
Since $f$ clearly has the given property over $S \setminus \{s\}$
we conclude as one can check the property locally on the base.
\end{proof}

\begin{remark}
\label{remark-more-general-modification}
The lemma above can be generalized as follows. Let $S$ be a scheme and
let $T \subset S$ be a closed subset. Assume there exists a cofinal
system of open neighbourhoods $T \subset W_i$ such that
(1) $W_i \setminus T$ is quasi-compact and
(2) $W_i \subset W_j$ is an affine morphism.
Then $W = \lim W_i$ is a scheme which contains $T$
as a closed subscheme. Set $U = X \setminus T$ and $V = W \setminus T$.
Then the base change functor
$$
\left\{
\begin{matrix}
f : X \to S\text{ of finite presentation} \\
f^{-1}(U) \to U\text{ is an isomorphism}
\end{matrix}
\right\}
\longrightarrow
\left\{
\begin{matrix}
g : Y \to W\text{ of finite presentation} \\
g^{-1}(V) \to V\text{ is an isomorphism}
\end{matrix}
\right\}
$$
is an equivalence of categories. If we ever need this we will
change this remark into a lemma and provide a detailed proof.
\end{remark}



\section{Descending finite type schemes}
\label{section-finite-type-quasi-separated}

\noindent
This section continues the theme of
Section \ref{section-finite-type-closed-in-finite-presentation}
in the spirit of the results discussed in
Section \ref{section-descending-relative}.

\begin{situation}
\label{situation-limit-noetherian}
Let $S = \lim_{i \in I} S_i$ be the limit of a directed system of
Noetherian schemes with affine transition morphisms $S_{i'} \to S_i$
for $i' \geq i$.
\end{situation}

\begin{lemma}
\label{lemma-good-diagram}
In Situation \ref{situation-limit-noetherian}.
Let $X \to S$ be quasi-separated and of finite type.
Then there exists an $i \in I$ and a diagram
\begin{equation}
\label{equation-good-diagram}
\vcenter{
\xymatrix{
X \ar[r] \ar[d] & W \ar[d] \\
S \ar[r] & S_i
}
}
\end{equation}
such that $W \to S_i$ is of finite type and such that
the induced morphism $X \to S \times_{S_i} W$ is a closed
immersion.
\end{lemma}

\begin{proof}
By Lemma \ref{lemma-finite-type-closed-in-finite-presentation}
we can find a closed immersion $X \to X'$
over $S$ where $X'$ is a scheme of finite presentation over $S$.
By Lemma \ref{lemma-descend-finite-presentation}
we can find an $i$ and a morphism of finite presentation
$X'_i \to S_i$ whose pull back is $X'$. Set $W = X'_i$.
\end{proof}

\begin{lemma}
\label{lemma-limit-from-good-diagram}
In Situation \ref{situation-limit-noetherian}.
Let $X \to S$ be quasi-separated and of finite type.
Given $i \in I$ and a diagram
$$
\vcenter{
\xymatrix{
X \ar[r] \ar[d] & W \ar[d] \\
S \ar[r] & S_i
}
}
$$
as in (\ref{equation-good-diagram}) for $i' \geq i$ let
$X_{i'}$ be the scheme theoretic image of $X \to S_{i'} \times_{S_i} W$.
Then $X = \lim_{i' \geq i} X_{i'}$.
\end{lemma}

\begin{proof}
Since $X$ is quasi-compact and quasi-separated formation of the
scheme theoretic image of $X \to S_{i'} \times_{S_i} W$
commutes with restriction to open subschemes
(Morphisms, Lemma \ref{morphisms-lemma-quasi-compact-scheme-theoretic-image}).
Hence we may and do assume $W$ is affine and maps into an affine open
$U_i$ of $S_i$. Let $U \subset S$, $U_{i'} \subset S_{i'}$
be the inverse image of $U_i$. Then $U$, $U_{i'}$, 
$S_{i'} \times_{S_i} W = U_{i'} \times_{U_i} W$, and
$S \times_{S_i} W = U \times_{U_i} W$ are all affine.
This implies $X$ is affine because $X \to S \times_{S_i} W$ is
a closed immersion. This also shows the ring map
$$
\mathcal{O}(U) \otimes_{\mathcal{O}(U_i)} \mathcal{O}(W) \to
\mathcal{O}(X)
$$
is surjective. Let $I$ be the kernel. Then we see that $X_{i'}$
is the spectrum of the ring
$$
\mathcal{O}(X_{i'}) =
\mathcal{O}(U_{i'}) \otimes_{\mathcal{O}(U_i)} \mathcal{O}(W)/I_{i'}
$$
where $I_{i'}$ is the inverse image of the ideal $I$ (see
Morphisms, Example \ref{morphisms-example-scheme-theoretic-image}).
Since $\mathcal{O}(U) = \colim \mathcal{O}(U_{i'})$
we see that $I = \colim I_{i'}$ and we conclude
that $\colim \mathcal{O}(X_{i'}) = \mathcal{O}(X)$.
\end{proof}

\begin{lemma}
\label{lemma-morphism-good-diagram}
In Situation \ref{situation-limit-noetherian}.
Let $f : X \to Y$ be a morphism of schemes quasi-separated
and of finite type over $S$. Let
$$
\vcenter{
\xymatrix{
X \ar[r] \ar[d] & W \ar[d] \\
S \ar[r] & S_{i_1}
}
}
\quad\text{and}\quad
\vcenter{
\xymatrix{
Y \ar[r] \ar[d] & V \ar[d] \\
S \ar[r] & S_{i_2}
}
}
$$
be diagrams as in (\ref{equation-good-diagram}). Let
$X = \lim_{i \geq i_1} X_i$ and
$Y = \lim_{i \geq i_2} Y_i$ be the corresponding
limit descriptions as in Lemma \ref{lemma-limit-from-good-diagram}.
Then there exists an $i_0 \geq \max(i_1, i_2)$ and a morphism
$$
(f_i)_{i \geq i_0} : (X_i)_{i \geq i_0} \to (Y_i)_{i \geq i_0}
$$
of inverse systems over $(S_i)_{i \geq i_0}$ such that
such that $f = \lim_{i \geq i_0} f_i$.
If $(g_i)_{i \geq i_0} : (X_i)_{i \geq i_0} \to (Y_i)_{i \geq i_0}$
is a second morphism of inverse systems over $(S_i)_{i \geq i_0}$ such that
such that $f = \lim_{i \geq i_0} g_i$
then $f_i = g_i$ for all $i \gg i_0$.
\end{lemma}

\begin{proof}
Since $V \to S_{i_2}$ is of finite presentation and
$X = \lim_{i \geq i_1} X_i$ we can appeal to Proposition
\ref{proposition-characterize-locally-finite-presentation}
to find an $i_0 \geq \max(i_1, i_2)$ and a morphism $h : X_{i_0} \to V$
over $S_{i_2}$ such that $X \to X_{i_0} \to V$ is equal to $X \to Y \to V$.
For $i \geq i_0$ we get a commutative solid diagram
$$
\xymatrix{
X \ar[d] \ar[r] &
X_i \ar[r] \ar@{..>}[d] \ar@/_2pc/[dd] |!{[d];[ld]}\hole &
X_{i_0} \ar[d]^h \\
Y \ar[r] \ar[d] & Y_i \ar[r] \ar[d] & V \ar[d] \\
S \ar[r] & S_i \ar[r] & S_{i_0}
}
$$
Since $X \to X_i$ has scheme theoretically dense image
and since $Y_i$ is the scheme theoretic image of
$Y \to S_i \times_{S_{i_2}} V$
we find that the morphism $X_i \to S_i \times_{S_{i_2}} V$
induced by the diagram
factors through $Y_i$ (Morphisms, Lemma \ref{morphisms-lemma-factor-factor}).
This proves existence.

\medskip\noindent
Uniqueness. Let $E_i \subset X_i$ be the equalizer of $f_i$ and $g_i$
for $i \geq i_0$. By
Schemes, Lemma \ref{schemes-lemma-where-are-they-equal}
$E_i$ is a locally closed subscheme of $X_i$.
Since $X_i$ is a closed subscheme of $S_i \times_{S_{i_0}} X_{i_0}$
and similarly for $Y_i$ we see that
$$
E_i = X_i \times_{(S_i \times_{S_{i_0}} X_{i_0})} (S_i \times_{S_{i_0}} E_{i_0})
$$
Thus to finish the proof it suffices to show that $X_i \to X_{i_0}$
factors through $E_{i_0}$ for some $i \geq i_0$.
To do this we will use that $X \to X_{i_0}$ factors through $E_{i_0}$
as both $f_{i_0}$ and $g_{i_0}$ are compatible with $f$.
Since $X_i$ is Noetherian, we see that the underlying
topological space $|E_{i_0}|$ is a constructible subset of $|X_{i_0}|$
(Topology, Lemma \ref{topology-lemma-constructible-Noetherian-space}).
Hence $X_i \to X_{i_0}$ factors through $E_{i_0}$ set theoretically
for large enough $i$ by Lemma \ref{lemma-limit-contained-in-constructible}.
For such an $i$ the scheme theoretic inverse image
$(X_i \to X_{i_0})^{-1}(E_{i_0})$ is a closed subscheme of $X_i$
through which $X$ factors and hence equal to $X_i$ since
$X \to X_i$ has scheme theoretically dense image by construction.
This concludes the proof.
\end{proof}

\begin{remark}
\label{remark-finite-type-gives-well-defined-system}
In Situation \ref{situation-limit-noetherian}
Lemmas \ref{lemma-good-diagram}, \ref{lemma-limit-from-good-diagram}, and
\ref{lemma-morphism-good-diagram}
tell us that the category of schemes quasi-separated and
of finite type over $S$ is equivalent to certain types of
inverse systems of schemes over $(S_i)_{i \in I}$, namely
the ones produced by applying Lemma \ref{lemma-limit-from-good-diagram}
to a diagram of the form (\ref{equation-good-diagram}).
For example, given $X \to S$ finite type and quasi-separated
if we choose two different diagrams $X \to V_1 \to S_{i_1}$
and $X \to V_2 \to S_{i_2}$ as in (\ref{equation-good-diagram}), then
applying Lemma \ref{lemma-morphism-good-diagram} to $\text{id}_X$
(in two directions)
we see that the corresponding limit descriptions of
$X$ are canonically isomorphic (up to shrinking the
directed set $I$). And so on and so forth.
\end{remark}

\begin{lemma}
\label{lemma-morphism-good-diagram-flat}
Notation and assumptions as in Lemma \ref{lemma-morphism-good-diagram}.
If $f$ is flat and of finite presentation, then
there exists an $i_3 \geq i_0$ such that for $i \geq i_3$ we have
$f_i$ is flat, $X_i = Y_i \times_{Y_{i_3}} X_{i_3}$, and
$X = Y \times_{Y_{i_3}} X_{i_3}$.
\end{lemma}

\begin{proof}
By Lemma \ref{lemma-descend-finite-presentation}
we can choose an $i \geq i_2$ and a morphism
$U \to Y_i$ of finite presentation such that $X = Y \times_{Y_i} U$
(this is where we use that $f$ is of finite presentation).
After increasing $i$ we may assume that $U \to Y_i$ is flat, see
Lemma \ref{lemma-descend-flat-finite-presentation}.
As discussed in Remark \ref{remark-finite-type-gives-well-defined-system}
we may and do replace the initial diagram used to define the system
$(X_i)_{i \geq i_1}$ by the system corresponding to
$X \to U \to S_i$. Thus $X_{i'}$ for $i' \geq i$ is defined as
the scheme theoretic image of $X \to S_{i'} \times_{S_i} U$.

\medskip\noindent
Because $U \to Y_i$ is flat (this is where we use that $f$ is flat),
because $X = Y \times_{Y_i} U$, and
because the scheme theoretic image of $Y \to Y_i$ is $Y_i$,
we see that the scheme theoretic image of $X \to U$ is $U$
(Morphisms, Lemma
\ref{morphisms-lemma-flat-base-change-scheme-theoretic-image}).
Observe that $Y_{i'} \to S_{i'} \times_{S_i} Y_i$ is a closed
immersion for $i' \geq i$ by construction of the system of $Y_j$.
Then the same argument as above shows that the scheme theoretic image
of $X \to S_{i'} \times_{S_i} U$
is equal to the closed subscheme $Y_{i'} \times_{Y_i} U$.
Thus we see that $X_{i'} = Y_{i'} \times_{Y_i} U$ for all $i' \geq i$
and hence the lemma holds with $i_3 = i$.
\end{proof}

\begin{lemma}
\label{lemma-morphism-good-diagram-smooth}
Notation and assumptions as in Lemma \ref{lemma-morphism-good-diagram}.
If $f$ is smooth, then there exists an $i_3 \geq i_0$ such that for
$i \geq i_3$ we have $f_i$ is smooth.
\end{lemma}

\begin{proof}
Combine Lemmas \ref{lemma-morphism-good-diagram-flat} and
\ref{lemma-descend-smooth}.
\end{proof}

\begin{lemma}
\label{lemma-morphism-good-diagram-proper}
Notation and assumptions as in Lemma \ref{lemma-morphism-good-diagram}.
If $f$ is proper, then there exists an $i_3 \geq i_0$ such that for
$i \geq i_3$ we have $f_i$ is proper.
\end{lemma}

\begin{proof}
By the discussion in 
Remark \ref{remark-finite-type-gives-well-defined-system}
the choice of $i_1$ and $W$ fitting into a diagram as in
(\ref{equation-good-diagram}) is immaterial for the truth of
the lemma. Thus we choose $W$ as follows.
First we choose a closed immersion $X \to X'$
with $X' \to S$ proper and of finite presentation, see
Lemma \ref{lemma-proper-limit-of-proper-finite-presentation}.
Then we choose an $i_3 \geq i_2$ and a proper morphism $W \to Y_{i_3}$
such that $X' = Y \times_{Y_{i_3}} W$. This is possible because
$Y = \lim_{i \geq i_2} Y_i$ and
Lemmas \ref{lemma-descend-finite-presentation} and
\ref{lemma-eventually-proper}.
With this choice of $W$ it is immediate from the construction that
for $i \geq i_3$ the scheme $X_i$ is a closed subscheme of
$Y_i \times_{Y_{i_3}} W \subset S_i \times_{S_{i_3}} W$
and hence proper over $Y_i$.
\end{proof}

\begin{lemma}
\label{lemma-good-diagram-fibre-product}
In Situation \ref{situation-limit-noetherian} suppose that we have a
cartesian diagram
$$
\xymatrix{
X^1 \ar[r]_p \ar[d]_q & X^3 \ar[d]^a \\
X^2 \ar[r]^b & X^4
}
$$
of schemes quasi-separated and of finite type over $S$.
For each $j = 1, 2, 3, 4$ choose $i_j \in I$ and a diagram
$$
\xymatrix{
X^j \ar[r] \ar[d] & W^j \ar[d] \\
S \ar[r] & S_{i_j}
}
$$
as in (\ref{equation-good-diagram}). Let
$X^j = \lim_{i \geq i_j} X^j_i$ be the corresponding limit descriptions
as in Lemma \ref{lemma-morphism-good-diagram}.
Let $(a_i)_{i \geq i_5}$, $(b_i)_{i \geq i_6}$, $(p_i)_{i \geq i_7}$, and
$(q_i)_{i \geq i_8}$ be the corresponding morphisms of systems contructed
in Lemma \ref{lemma-morphism-good-diagram}. Then there exists an
$i_9 \geq \max(i_5, i_6, i_7, i_8)$ such that for $i \geq i_9$ we have
$a_i \circ p_i = b_i \circ q_i$ and such that
$$
(q_i, p_i) : X^1_i \longrightarrow X^2_i \times_{b_i, X^4_i, a_i} X^3_i
$$
is a closed immersion.
If $a$ and $b$ are flat and of finite presentation, then there exists an
$i_{10} \geq \max(i_5, i_6, i_7, i_8, i_9)$ such that for $i \geq i_{10}$
the last displayed morphism is an isomorphism.
\end{lemma}

\begin{proof}
According to the discussion in
Remark \ref{remark-finite-type-gives-well-defined-system}
the choice of $W^1$ fitting into a diagram as in
(\ref{equation-good-diagram}) is immaterial for the truth of
the lemma. Thus we may choose $W^1 = W^2 \times_{W^4} W^3$.
Then it is immediate from the construction of $X^1_i$ that 
$a_i \circ p_i = b_i \circ q_i$ and that
$$
(q_i, p_i) : X^1_i \longrightarrow X^2_i \times_{b_i, X^4_i, a_i} X^3_i
$$
is a closed immersion.

\medskip\noindent
If $a$ and $b$ are flat and of finite presentation, then so are
$p$ and $q$ as base changes of $a$ and $b$. Thus we can apply
Lemma \ref{lemma-morphism-good-diagram-flat}
to each of $a$, $b$, $p$, $q$, and $a \circ p = b \circ q$.
It follows that there exists an $i_9 \in I$ such that
$$
(q_i, p_i) : X^1_i \to X^2_i \times_{X^4_i} X^3_i
$$
is the base change of $(q_{i_9}, p_{i_9})$ by the morphism
by the morphism $X^4_i \to X^4_{i_9}$ for all $i \geq i_9$.
We conclude that $(q_i, p_i)$ is an isomorphism for all sufficiently
large $i$ by Lemma \ref{lemma-descend-isomorphism}.
\end{proof}






\begin{multicols}{2}[\section{Other chapters}]
\noindent
Preliminaries
\begin{enumerate}
\item \hyperref[introduction-section-phantom]{Introduction}
\item \hyperref[conventions-section-phantom]{Conventions}
\item \hyperref[sets-section-phantom]{Set Theory}
\item \hyperref[categories-section-phantom]{Categories}
\item \hyperref[topology-section-phantom]{Topology}
\item \hyperref[sheaves-section-phantom]{Sheaves on Spaces}
\item \hyperref[sites-section-phantom]{Sites and Sheaves}
\item \hyperref[stacks-section-phantom]{Stacks}
\item \hyperref[fields-section-phantom]{Fields}
\item \hyperref[algebra-section-phantom]{Commutative Algebra}
\item \hyperref[brauer-section-phantom]{Brauer Groups}
\item \hyperref[homology-section-phantom]{Homological Algebra}
\item \hyperref[derived-section-phantom]{Derived Categories}
\item \hyperref[simplicial-section-phantom]{Simplicial Methods}
\item \hyperref[more-algebra-section-phantom]{More on Algebra}
\item \hyperref[smoothing-section-phantom]{Smoothing Ring Maps}
\item \hyperref[modules-section-phantom]{Sheaves of Modules}
\item \hyperref[sites-modules-section-phantom]{Modules on Sites}
\item \hyperref[injectives-section-phantom]{Injectives}
\item \hyperref[cohomology-section-phantom]{Cohomology of Sheaves}
\item \hyperref[sites-cohomology-section-phantom]{Cohomology on Sites}
\item \hyperref[dga-section-phantom]{Differential Graded Algebra}
\item \hyperref[dpa-section-phantom]{Divided Power Algebra}
\item \hyperref[hypercovering-section-phantom]{Hypercoverings}
\end{enumerate}
Schemes
\begin{enumerate}
\setcounter{enumi}{24}
\item \hyperref[schemes-section-phantom]{Schemes}
\item \hyperref[constructions-section-phantom]{Constructions of Schemes}
\item \hyperref[properties-section-phantom]{Properties of Schemes}
\item \hyperref[morphisms-section-phantom]{Morphisms of Schemes}
\item \hyperref[coherent-section-phantom]{Cohomology of Schemes}
\item \hyperref[divisors-section-phantom]{Divisors}
\item \hyperref[limits-section-phantom]{Limits of Schemes}
\item \hyperref[varieties-section-phantom]{Varieties}
\item \hyperref[topologies-section-phantom]{Topologies on Schemes}
\item \hyperref[descent-section-phantom]{Descent}
\item \hyperref[perfect-section-phantom]{Derived Categories of Schemes}
\item \hyperref[more-morphisms-section-phantom]{More on Morphisms}
\item \hyperref[flat-section-phantom]{More on Flatness}
\item \hyperref[groupoids-section-phantom]{Groupoid Schemes}
\item \hyperref[more-groupoids-section-phantom]{More on Groupoid Schemes}
\item \hyperref[etale-section-phantom]{\'Etale Morphisms of Schemes}
\end{enumerate}
Topics in Scheme Theory
\begin{enumerate}
\setcounter{enumi}{40}
\item \hyperref[chow-section-phantom]{Chow Homology}
\item \hyperref[intersection-section-phantom]{Intersection Theory}
\item \hyperref[weil-section-phantom]{Weil Cohomology Theories}
\item \hyperref[pic-section-phantom]{Picard Schemes of Curves}
\item \hyperref[adequate-section-phantom]{Adequate Modules}
\item \hyperref[dualizing-section-phantom]{Dualizing Complexes}
\item \hyperref[duality-section-phantom]{Duality for Schemes}
\item \hyperref[discriminant-section-phantom]{Discriminants and Differents}
\item \hyperref[local-cohomology-section-phantom]{Local Cohomology}
\item \hyperref[algebraization-section-phantom]{Algebraic and Formal Geometry}
\item \hyperref[curves-section-phantom]{Algebraic Curves}
\item \hyperref[resolve-section-phantom]{Resolution of Surfaces}
\item \hyperref[models-section-phantom]{Semistable Reduction}
\item \hyperref[pione-section-phantom]{Fundamental Groups of Schemes}
\item \hyperref[etale-cohomology-section-phantom]{\'Etale Cohomology}
\item \hyperref[crystalline-section-phantom]{Crystalline Cohomology}
\item \hyperref[proetale-section-phantom]{Pro-\'etale Cohomology}
\item \hyperref[more-etale-section-phantom]{More \'Etale Cohomology}
\item \hyperref[trace-section-phantom]{The Trace Formula}
\end{enumerate}
Algebraic Spaces
\begin{enumerate}
\setcounter{enumi}{59}
\item \hyperref[spaces-section-phantom]{Algebraic Spaces}
\item \hyperref[spaces-properties-section-phantom]{Properties of Algebraic Spaces}
\item \hyperref[spaces-morphisms-section-phantom]{Morphisms of Algebraic Spaces}
\item \hyperref[decent-spaces-section-phantom]{Decent Algebraic Spaces}
\item \hyperref[spaces-cohomology-section-phantom]{Cohomology of Algebraic Spaces}
\item \hyperref[spaces-limits-section-phantom]{Limits of Algebraic Spaces}
\item \hyperref[spaces-divisors-section-phantom]{Divisors on Algebraic Spaces}
\item \hyperref[spaces-over-fields-section-phantom]{Algebraic Spaces over Fields}
\item \hyperref[spaces-topologies-section-phantom]{Topologies on Algebraic Spaces}
\item \hyperref[spaces-descent-section-phantom]{Descent and Algebraic Spaces}
\item \hyperref[spaces-perfect-section-phantom]{Derived Categories of Spaces}
\item \hyperref[spaces-more-morphisms-section-phantom]{More on Morphisms of Spaces}
\item \hyperref[spaces-flat-section-phantom]{Flatness on Algebraic Spaces}
\item \hyperref[spaces-groupoids-section-phantom]{Groupoids in Algebraic Spaces}
\item \hyperref[spaces-more-groupoids-section-phantom]{More on Groupoids in Spaces}
\item \hyperref[bootstrap-section-phantom]{Bootstrap}
\item \hyperref[spaces-pushouts-section-phantom]{Pushouts of Algebraic Spaces}
\end{enumerate}
Topics in Geometry
\begin{enumerate}
\setcounter{enumi}{76}
\item \hyperref[spaces-chow-section-phantom]{Chow Groups of Spaces}
\item \hyperref[groupoids-quotients-section-phantom]{Quotients of Groupoids}
\item \hyperref[spaces-more-cohomology-section-phantom]{More on Cohomology of Spaces}
\item \hyperref[spaces-simplicial-section-phantom]{Simplicial Spaces}
\item \hyperref[spaces-duality-section-phantom]{Duality for Spaces}
\item \hyperref[formal-spaces-section-phantom]{Formal Algebraic Spaces}
\item \hyperref[restricted-section-phantom]{Restricted Power Series}
\item \hyperref[spaces-resolve-section-phantom]{Resolution of Surfaces Revisited}
\end{enumerate}
Deformation Theory
\begin{enumerate}
\setcounter{enumi}{84}
\item \hyperref[formal-defos-section-phantom]{Formal Deformation Theory}
\item \hyperref[defos-section-phantom]{Deformation Theory}
\item \hyperref[cotangent-section-phantom]{The Cotangent Complex}
\item \hyperref[examples-defos-section-phantom]{Deformation Problems}
\end{enumerate}
Algebraic Stacks
\begin{enumerate}
\setcounter{enumi}{88}
\item \hyperref[algebraic-section-phantom]{Algebraic Stacks}
\item \hyperref[examples-stacks-section-phantom]{Examples of Stacks}
\item \hyperref[stacks-sheaves-section-phantom]{Sheaves on Algebraic Stacks}
\item \hyperref[criteria-section-phantom]{Criteria for Representability}
\item \hyperref[artin-section-phantom]{Artin's Axioms}
\item \hyperref[quot-section-phantom]{Quot and Hilbert Spaces}
\item \hyperref[stacks-properties-section-phantom]{Properties of Algebraic Stacks}
\item \hyperref[stacks-morphisms-section-phantom]{Morphisms of Algebraic Stacks}
\item \hyperref[stacks-limits-section-phantom]{Limits of Algebraic Stacks}
\item \hyperref[stacks-cohomology-section-phantom]{Cohomology of Algebraic Stacks}
\item \hyperref[stacks-perfect-section-phantom]{Derived Categories of Stacks}
\item \hyperref[stacks-introduction-section-phantom]{Introducing Algebraic Stacks}
\item \hyperref[stacks-more-morphisms-section-phantom]{More on Morphisms of Stacks}
\item \hyperref[stacks-geometry-section-phantom]{The Geometry of Stacks}
\end{enumerate}
Topics in Moduli Theory
\begin{enumerate}
\setcounter{enumi}{102}
\item \hyperref[moduli-section-phantom]{Moduli Stacks}
\item \hyperref[moduli-curves-section-phantom]{Moduli of Curves}
\end{enumerate}
Miscellany
\begin{enumerate}
\setcounter{enumi}{104}
\item \hyperref[examples-section-phantom]{Examples}
\item \hyperref[exercises-section-phantom]{Exercises}
\item \hyperref[guide-section-phantom]{Guide to Literature}
\item \hyperref[desirables-section-phantom]{Desirables}
\item \hyperref[coding-section-phantom]{Coding Style}
\item \hyperref[obsolete-section-phantom]{Obsolete}
\item \hyperref[fdl-section-phantom]{GNU Free Documentation License}
\item \hyperref[index-section-phantom]{Auto Generated Index}
\end{enumerate}
\end{multicols}


\bibliography{my}
\bibliographystyle{amsalpha}

\end{document}

