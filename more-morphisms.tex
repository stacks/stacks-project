\IfFileExists{stacks-project.cls}{%
\documentclass{stacks-project}
}{%
\documentclass{amsart}
}

% The following AMS packages are automatically loaded with
% the amsart documentclass:
%\usepackage{amsmath}
%\usepackage{amssymb}
%\usepackage{amsthm}

% For dealing with references we use the comment environment
\usepackage{verbatim}
\newenvironment{reference}{\comment}{\endcomment}
%\newenvironment{reference}{}{}
\newenvironment{slogan}{\comment}{\endcomment}
\newenvironment{history}{\comment}{\endcomment}

% For commutative diagrams you can use
% \usepackage{amscd}
\usepackage[all]{xy}

% We use 2cell for 2-commutative diagrams.
\xyoption{2cell}
\UseAllTwocells

% To put source file link in headers.
% Change "template.tex" to "this_filename.tex"
% \usepackage{fancyhdr}
% \pagestyle{fancy}
% \lhead{}
% \chead{}
% \rhead{Source file: \url{template.tex}}
% \lfoot{}
% \cfoot{\thepage}
% \rfoot{}
% \renewcommand{\headrulewidth}{0pt}
% \renewcommand{\footrulewidth}{0pt}
% \renewcommand{\headheight}{12pt}

\usepackage{multicol}

% For cross-file-references
\usepackage{xr-hyper}

% Package for hypertext links:
\usepackage{hyperref}

% For any local file, say "hello.tex" you want to link to please
% use \externaldocument[hello-]{hello}
\externaldocument[introduction-]{introduction}
\externaldocument[conventions-]{conventions}
\externaldocument[sets-]{sets}
\externaldocument[categories-]{categories}
\externaldocument[topology-]{topology}
\externaldocument[sheaves-]{sheaves}
\externaldocument[sites-]{sites}
\externaldocument[stacks-]{stacks}
\externaldocument[fields-]{fields}
\externaldocument[algebra-]{algebra}
\externaldocument[brauer-]{brauer}
\externaldocument[homology-]{homology}
\externaldocument[derived-]{derived}
\externaldocument[simplicial-]{simplicial}
\externaldocument[more-algebra-]{more-algebra}
\externaldocument[smoothing-]{smoothing}
\externaldocument[modules-]{modules}
\externaldocument[sites-modules-]{sites-modules}
\externaldocument[injectives-]{injectives}
\externaldocument[cohomology-]{cohomology}
\externaldocument[sites-cohomology-]{sites-cohomology}
\externaldocument[dga-]{dga}
\externaldocument[dpa-]{dpa}
\externaldocument[hypercovering-]{hypercovering}
\externaldocument[schemes-]{schemes}
\externaldocument[constructions-]{constructions}
\externaldocument[properties-]{properties}
\externaldocument[morphisms-]{morphisms}
\externaldocument[coherent-]{coherent}
\externaldocument[divisors-]{divisors}
\externaldocument[limits-]{limits}
\externaldocument[varieties-]{varieties}
\externaldocument[topologies-]{topologies}
\externaldocument[descent-]{descent}
\externaldocument[perfect-]{perfect}
\externaldocument[more-morphisms-]{more-morphisms}
\externaldocument[flat-]{flat}
\externaldocument[groupoids-]{groupoids}
\externaldocument[more-groupoids-]{more-groupoids}
\externaldocument[etale-]{etale}
\externaldocument[chow-]{chow}
\externaldocument[intersection-]{intersection}
\externaldocument[pic-]{pic}
\externaldocument[adequate-]{adequate}
\externaldocument[dualizing-]{dualizing}
\externaldocument[duality-]{duality}
\externaldocument[discriminant-]{discriminant}
\externaldocument[local-cohomology-]{local-cohomology}
\externaldocument[curves-]{curves}
\externaldocument[resolve-]{resolve}
\externaldocument[models-]{models}
\externaldocument[pione-]{pione}
\externaldocument[etale-cohomology-]{etale-cohomology}
\externaldocument[proetale-]{proetale}
\externaldocument[crystalline-]{crystalline}
\externaldocument[spaces-]{spaces}
\externaldocument[spaces-properties-]{spaces-properties}
\externaldocument[spaces-morphisms-]{spaces-morphisms}
\externaldocument[decent-spaces-]{decent-spaces}
\externaldocument[spaces-cohomology-]{spaces-cohomology}
\externaldocument[spaces-limits-]{spaces-limits}
\externaldocument[spaces-divisors-]{spaces-divisors}
\externaldocument[spaces-over-fields-]{spaces-over-fields}
\externaldocument[spaces-topologies-]{spaces-topologies}
\externaldocument[spaces-descent-]{spaces-descent}
\externaldocument[spaces-perfect-]{spaces-perfect}
\externaldocument[spaces-more-morphisms-]{spaces-more-morphisms}
\externaldocument[spaces-flat-]{spaces-flat}
\externaldocument[spaces-groupoids-]{spaces-groupoids}
\externaldocument[spaces-more-groupoids-]{spaces-more-groupoids}
\externaldocument[bootstrap-]{bootstrap}
\externaldocument[spaces-pushouts-]{spaces-pushouts}
\externaldocument[groupoids-quotients-]{groupoids-quotients}
\externaldocument[spaces-more-cohomology-]{spaces-more-cohomology}
\externaldocument[spaces-simplicial-]{spaces-simplicial}
\externaldocument[formal-spaces-]{formal-spaces}
\externaldocument[restricted-]{restricted}
\externaldocument[spaces-resolve-]{spaces-resolve}
\externaldocument[formal-defos-]{formal-defos}
\externaldocument[defos-]{defos}
\externaldocument[cotangent-]{cotangent}
\externaldocument[examples-defos-]{examples-defos}
\externaldocument[algebraic-]{algebraic}
\externaldocument[examples-stacks-]{examples-stacks}
\externaldocument[stacks-sheaves-]{stacks-sheaves}
\externaldocument[criteria-]{criteria}
\externaldocument[artin-]{artin}
\externaldocument[quot-]{quot}
\externaldocument[stacks-properties-]{stacks-properties}
\externaldocument[stacks-morphisms-]{stacks-morphisms}
\externaldocument[stacks-limits-]{stacks-limits}
\externaldocument[stacks-cohomology-]{stacks-cohomology}
\externaldocument[stacks-perfect-]{stacks-perfect}
\externaldocument[stacks-introduction-]{stacks-introduction}
\externaldocument[stacks-more-morphisms-]{stacks-more-morphisms}
\externaldocument[stacks-geometry-]{stacks-geometry}
\externaldocument[moduli-]{moduli}
\externaldocument[moduli-curves-]{moduli-curves}
\externaldocument[examples-]{examples}
\externaldocument[exercises-]{exercises}
\externaldocument[guide-]{guide}
\externaldocument[desirables-]{desirables}
\externaldocument[coding-]{coding}
\externaldocument[obsolete-]{obsolete}
\externaldocument[fdl-]{fdl}
\externaldocument[index-]{index}

% Theorem environments.
%
\theoremstyle{plain}
\newtheorem{theorem}[subsection]{Theorem}
\newtheorem{proposition}[subsection]{Proposition}
\newtheorem{lemma}[subsection]{Lemma}

\theoremstyle{definition}
\newtheorem{definition}[subsection]{Definition}
\newtheorem{example}[subsection]{Example}
\newtheorem{exercise}[subsection]{Exercise}
\newtheorem{situation}[subsection]{Situation}

\theoremstyle{remark}
\newtheorem{remark}[subsection]{Remark}
\newtheorem{remarks}[subsection]{Remarks}

\numberwithin{equation}{subsection}

% Macros
%
\def\lim{\mathop{\rm lim}\nolimits}
\def\colim{\mathop{\rm colim}\nolimits}
\def\Spec{\mathop{\rm Spec}}
\def\Hom{\mathop{\rm Hom}\nolimits}
\def\Ext{\mathop{\rm Ext}\nolimits}
\def\SheafHom{\mathop{\mathcal{H}\!{\it om}}\nolimits}
\def\SheafExt{\mathop{\mathcal{E}\!{\it xt}}\nolimits}
\def\Sch{\textit{Sch}}
\def\Mor{\mathop{\rm Mor}\nolimits}
\def\Ob{\mathop{\rm Ob}\nolimits}
\def\Sh{\mathop{\textit{Sh}}\nolimits}
\def\NL{\mathop{N\!L}\nolimits}
\def\proetale{{pro\text{-}\acute{e}tale}}
\def\etale{{\acute{e}tale}}
\def\QCoh{\textit{QCoh}}
\def\Ker{\mathop{\rm Ker}}
\def\Im{\mathop{\rm Im}}
\def\Coker{\mathop{\rm Coker}}
\def\Coim{\mathop{\rm Coim}}

%
% Macros for moduli stacks/spaces
%
\def\QCohstack{\mathcal{QC}\!{\it oh}}
\def\Cohstack{\mathcal{C}\!{\it oh}}
\def\Spacesstack{\mathcal{S}\!{\it paces}}
\def\Quotfunctor{{\rm Quot}}
\def\Hilbfunctor{{\rm Hilb}}
\def\Curvesstack{\mathcal{C}\!{\it urves}}
\def\Polarizedstack{\mathcal{P}\!{\it olarized}}
\def\Complexesstack{\mathcal{C}\!{\it omplexes}}
% \Pic is the operator that assigns to X its picard group, usage \Pic(X)
% \Picardstack_{X/B} denotes the Picard stack of X over B
% \Picardfunctor_{X/B} denotes the Picard functor of X over B
\def\Pic{\mathop{\rm Pic}\nolimits}
\def\Picardstack{\mathcal{P}\!{\it ic}}
\def\Picardfunctor{{\rm Pic}}
\def\Deformationcategory{\mathcal{D}\!{\it ef}}


% OK, start here.
%
\begin{document}

\title{More on Morphisms}


\maketitle

\phantomsection
\label{section-phantom}

\tableofcontents

\section{Introduction}
\label{section-introduction}

\noindent
In this chapter we continue our study of properties of morphisms of schemes.
A fundamental reference is \cite{EGA}.









\section{Formally smooth morphisms}
\label{section-formally-smooth}

\noindent
Mike Artin's position on differential criteria of smoothness (e.g.,
Morphisms, Lemma \ref{morphisms-lemma-smooth-at-point}) is that they are
basically useless (in practice). It turns out that it is
often possible to prove that a morphism $X \to S$ is smooth by considering
infinitesimal deformations of $T$-valued points of $X$ over $S$. In this
section we introduce to relevant terminology and we show that this leads
to the same notion as before in the case of morphisms which are locally
of finite presentation, see Lemma \ref{lemma-smooth-formally-smooth}.

\medskip\noindent
Recall that a ring map $R \to A$ is called {\it formally smooth}
(see Algebra, Definition \ref{algebra-definition-formally-smooth})
if for every commutative solid diagram
$$
\xymatrix{
A \ar[r] \ar@{-->}[rd] & B/I \\
R \ar[r] \ar[u] & B \ar[u]
}
$$
where $I \subset B$ is an ideal of square zero, a dotted
arrow exists which makes the diagram commute. This motivates
the following analogue for morphisms of schemes.

\begin{definition}
\label{definition-formally-smooth}
Let $f : X \to S$ be a morphism of schemes.
We say $f$ is {\it formally smooth} if given any solid commutative diagram
$$
\xymatrix{
X \ar[d]_f & T \ar[d]^i \ar[l] \\
S & T' \ar[l] \ar@{-->}[lu]
}
$$
where $T$ and $T'$ are affine and $i$ is a closed immersion defined
by an ideal of square zero there exists a dotted arrow making the diagram
commute.
\end{definition}

\begin{lemma}
\label{lemma-composition-formally-smooth}
A composition of formally smooth morphisms is formally smooth.
\end{lemma}

\begin{proof}
Omitted.
\end{proof}

\begin{lemma}
\label{lemma-base-change-formally-smooth}
A base change of a formally smooth morphism is formally smooth.
\end{lemma}

\begin{proof}
Omitted, but see Algebra, Lemma \ref{algebra-lemma-base-change-fs}
for the algebraic version.
\end{proof}

\begin{lemma}
\label{lemma-formally-smooth-on-opens}
Let $f : X \to S$ be a morphism of schemes.
Let $U \subset X$ and $V \subset S$ be open subschemes such that
$f(U) \subset V$. If $f$ is formally smooth, so is $f|_U : U \to V$.
\end{lemma}

\begin{proof}
Consider a solid diagram
$$
\xymatrix{
U \ar[d]_{f|_U} & T \ar[d]^i \ar[l]^a \\
V & T' \ar[l] \ar@{-->}[lu]
}
$$
as in Definition \ref{definition-formally-smooth}. If $f$ is formally
smooth, then there exists an $S$-morphism $a' : T' \to X$ such that
$a'|_T = a$. Since the underlying sets of $T$ and $T'$ are the same
we see that $a'$ is a morphism into $U$ (see Schemes, Section
\ref{schemes-section-open-immersion}). And it clearly is a $V$-morphism
as well. Hence the dotted arrow above as desired.
\end{proof}

\begin{lemma}
\label{lemma-affine-formally-smooth}
Let $f : X \to S$ be a morphism of schemes.
Assume $X$ and $S$ are affine.
Then $f$ is formally smooth if and only if
$\mathcal{O}_S(S) \to \mathcal{O}_X(X)$ is a formally smooth
ring map.
\end{lemma}

\begin{proof}
This is immediate from the definitions
(Definition \ref{definition-formally-smooth} and
Algebra, Definition \ref{algebra-definition-formally-smooth})
by the equivalence of categories of rings and affine schemes,
see
Schemes, Lemma \ref{schemes-lemma-category-affine-schemes}.
\end{proof}

\noindent
To prove more we need a preliminary lemma.

\medskip\noindent
Let $i : T \to T'$ be a closed immersion defined by an ideal of
square zero. Note that the underlying topological spaces defined
by $T$ and $T'$ are identified via $i$. We will use the following
abuse of language in this section. Given an open $U \subset T$ we
let $U' \subset T'$ denote the corresponding open of $T'$, and
conversely of course. Note that we often think of $U$ and $U'$ as
schemes (by the abuse of language introduced in Schemes,
Section \ref{schemes-section-open-immersion}). And then we think of $U$ as
a closed subscheme of $U'$ (also defined by an ideal of square zero).
Using this identification we have a short exact sequence
$$
0 \to \mathcal{I} \to \mathcal{O}_{T'} \to \mathcal{O}_T \to 0
$$
of quasi-coherent $\mathcal{O}_{T'}$-modules
where $\mathcal{I}^2 = 0$. This in particular gives $\mathcal{I}$
the structure of a $\mathcal{O}_T$-module, see
Morphisms, Section \ref{morphisms-section-closed-immersions-quasi-coherent}.

\begin{lemma}
\label{lemma-action-by-derivations}
Let $S$ be a scheme. Let $i : T \to T'$ be a closed immersion of
schemes over $S$ defined by an ideal $\mathcal{I}$ of square zero
(notation as above). Let $X$ be a scheme over $S$ and let
$a : T \to X$ be a morphism of schemes over $S$.
\begin{enumerate}
\item The rule
$$
U \mapsto \{a' \in \text{Mor}_S(U', X) \text{ such that } a'|_U = a|_U\}
$$
defines a sheaf of sets $\mathcal{F}$ on $T$.
\item There is an action of the sheaf
$$
\mathcal{H} = \textit{Hom}_{\mathcal{O}_T}(a^*\Omega^1_{X/S}, \mathcal{I})
$$
on the sheaf $\mathcal{F}$.
\item Moreover, the action
$\mathcal{H}(U) \times \mathcal{F}(U) \to \mathcal{F}(U)$
is simply transitive for any open $U \subset T$ such that
$\mathcal{F}(U) \not = \emptyset$.
\end{enumerate}
\end{lemma}

\begin{proof}
The restriction mapping $\mathcal{F}(U) \to \mathcal{F}(V)$ for
$V \subset U \subset T$
of $\mathcal{F}$ is really the restriction map $a' \mapsto a'|_{V'}$.
With this definition in place it is clear that $\mathcal{F}$ is a
sheaf since morphisms are defined locally. This proves the first assertion.

\medskip\noindent
Let us define the action. Let $U \subset T$ be open. Let
$(a', (a')^\sharp) : U' \to X$ be an element of $\mathcal{F}(U)$.
Since $a' = a$ on underlying topological spaces we may as well
write this element as $(a|_U, (a')^\sharp)$. We think of $(a')^\sharp$
as a map
$$
(a')^\sharp : (a|_U)^{-1}\mathcal{O}_X \longrightarrow \mathcal{O}_{U'}.
$$
Let $\theta : a^*\Omega_{X/S}|_U \to \mathcal{I}|_U$ be an element of
$\mathcal{H}(U)$. We can think of $\theta$ also as a
$a^{-1}\mathcal{O}_X$-linear map
$\theta : a^{-1}\Omega_{X/S}|_U \to \mathcal{I}|_U$.
We define
$\theta \cdot (a, (a')^\sharp) = (a, (a'')^\sharp)$ where
$$
(a'')^\sharp = (a')^\sharp + \theta \circ a^{-1} \text{d}_{X/S} :
(a|_U)^{-1}\mathcal{O}_X \longrightarrow \mathcal{O}_{U'}.
$$
This makes sense since $\text{d}_{X/S} : \mathcal{O}_X \to \Omega_{X/S}$.
We have to check the following
\begin{enumerate}
\item $(a'')^\sharp$ is a homomorphism of sheaves of rings,
\item it is a homomorphism of sheaves of $p^{-1}\mathcal{O}_S$-algebras
where $p : T \to S$ is the structure morphism, and
\item $(a'')^\sharp \bmod \mathcal{I}$ is equal to $(a|_U)^\sharp$.
\end{enumerate}
One way to check these assertions is to argue as in the proof of
Morphisms, Lemma \ref{morphisms-lemma-double-structure-gives-derivation}
(and we encourage the reader to do so).

\medskip\noindent
Another possibility is to translate the above into algebra by choosing
an affine open covering $U = \bigcup U_i$ such that $a(U_i)$ is
contained in an affine open $W_i \subset X$ which in turn maps into an
affine open $V_i \subset S$. It is enough to prove the three assertions
on sections of the corresponding sheaves over each affine, and hence it
is enough to prove the three assertions
when $X = \text{Spec}(A)$, $U' = T' = \text{Spec}(B)$,
$U = T = \text{Spec}(B/I)$ and $S = \text{Spec}(R)$. In this case
we have a commutative diagram
$$
\xymatrix{
A \ar[r]_a \ar[rd]_{a'} & B/I \\
R \ar[r] \ar[u] & B \ar[u]
}
$$
and $\theta : \Omega_{A/R} \to I$ is $A$-linear. We have to check that
$$
a'' = a' + \theta \circ \text{d}_{A/R}
$$
is an $R$-algebra map from $A \to B$ which agrees with $a$ as a map into
$B/I$. This is proved using the Leibniz rule for the derivation
$D = \theta \circ \text{d}_{A/R}$ exactly as in the proof of the aforementioned
Morphisms, Lemma \ref{morphisms-lemma-double-structure-gives-derivation}.
This proves (2).

\medskip\noindent
Proof of (3). Consider a pair $a', a'' \in \mathcal{F}(U)$.
We have to show there exists a unique $\theta$ such that
$\theta \cdot a' = a''$. Let us first prove this in the algebraic
situation described above. Suppose given therefore two
commutative diagrams
$$
\xymatrix{
A \ar[r]_a \ar[rd]_{a'} & B/I \\
R \ar[r] \ar[u] & B \ar[u]
}
\quad \text{and} \quad
\xymatrix{
A \ar[r]_a \ar[rd]_{a''} & B/I \\
R \ar[r] \ar[u] & B \ar[u]
}
$$
The difference $D = a'' - a'$ is a map from $A$ to $I$ (obviously).
All we have to show that it is an $R$-derivation, since then it will
be of the form $D = \theta \circ \text{d}_{A/r}$ for some unique
$\theta$. It is clear that $D(r) = 0$ for $r \in R$ since both $a'$
and $a''$ are $R$-algebra homomorphsms. Finally
$$
D(xy) =
a''(xy) - a'(xy) = a''(x) a''(y) - a'(x) a'(y) =
a''(x)D(y) + D(x) a'(y).
$$
Since $a'$ and $a''$ agree with $a$ as maps into $B/I$ we see that
this is equal to $a(x) D(y) + D(x) a(y)$ as desired.

\medskip\noindent
One can use this algebraic result and a suitable choice of affine
open coverings to prove (3). A more global argument would be the following.
Suppose that $(a|_U, (a')^\sharp)$ and $(a|_U, (a'')^\sharp)$ are
two elements of $\mathcal{F}(U)$. Consider the difference
$$
D = (a'')^\sharp - (a')^\sharp : 
(a|_U)^{-1}\mathcal{O}_X \longrightarrow \mathcal{O}_{U'}.
$$
Since $(a')^\sharp$ and $(a'')^\sharp$ are equal to $(a|_U)^\sharp$
after composing with the map $\mathcal{O}_{U'} \to \mathcal{O}_{U}$
we see that $D : (a|_U)^{-1}\mathcal{O}_X \to \mathcal{I}|_U$.
Arguing in exactly the same way as before we see that this is
a $p^{-1}\mathcal{O}_S$-derivation.
Thinking of $D$ as a map $\mathcal{O}_X \to (a|_U)_*(\mathcal{I}|_U)$
by adjunction it is still a $q^{-1}\mathcal{O}_S$-derivation
where $q : X \to S$ is the structural morphism.
Hence it is of the form
$\theta \circ \text{d}_{X/S}$ for some unique
$\theta : \Omega_{X/S} \to (a|_U)_*(\mathcal{I}|_U)$.
Using adjunction again we obtain a $\theta$ as desired.
\end{proof}

\noindent
The following lemma is the main result of this section. It is a victory of the
functorial point of view in that it implies (combined with 
Limits,
Proposition \ref{limits-proposition-characterize-locally-finite-presentation})
that we can recognize whether a morphism $f : X \to S$ is smooth in terms of
``simple'' properties of the functor $h_X : \textit{Sch}/S \to \textit{Sets}$.

\begin{lemma}
\label{lemma-smooth-formally-smooth}
Let $f : X \to S$ be a morphism of schemes.
The following are equivalent:
\begin{enumerate}
\item The morphism $f$ is smooth, and
\item the morphism $f$ is locally of finite presentation and
formally smooth.
\end{enumerate}
\end{lemma}

\begin{proof}
Assume $f : X \to S$ is locally of finite presentation and formally smooth.
Consider a pair of affine opens $\text{Spec}(A) = U \subset X$ and
$\text{Spec}(R) = V \subset S$
such that $f(U) \subset V$. By Lemma \ref{lemma-formally-smooth-on-opens}
we see that $U \to V$ is formally smooth. By Lemma
\ref{lemma-affine-formally-smooth} we see that $R \to A$ is formally
smooth. By
Morphisms, Lemma \ref{morphisms-lemma-locally-finite-presentation-characterize}
we see that $R \to A$ is of finite presentation.
By Algebra, Lemma \ref{algebra-lemma-formally-smooth-smooth}
we see that $R \to A$ is smooth.
Hence by the definition of a smooth morphism we see that $X \to S$ is smooth.

\medskip\noindent
Conversely, assume that $f : X \to S$ is smooth. Consider a solid commutative
diagram
$$
\xymatrix{
X \ar[d]_f & T \ar[d]^i \ar[l]^a \\
S & T' \ar[l] \ar@{-->}[lu]
}
$$
where $T$ and $T'$ are affine and $i$ is a closed immersion defined
by an ideal of square zero. We will show the dotted arrow exists thereby
proving that $f$ is formally smooth.

\medskip\noindent
Let $\mathcal{F}$ be the sheaf of sets, and $\mathcal{H}$ be the sheaf
of $\mathcal{O}_T$-modules on $T$ introduced
in Lemma \ref{lemma-action-by-derivations}. Our goal is simply
to show that $\mathcal{F}(T) \not = \emptyset$. In other words we
are trying to show that $\mathcal{F}$ is a trivial $\mathcal{H}$-torsor
on $T$ (see Cohomology, Section \ref{cohomology-section-h1-torsors}).
There are two steps: (I) To show that $\mathcal{F}$ is a torsor
we have to show that $\mathcal{F}$ is a torsor we have to show
that $\mathcal{F}_t \not = \emptyset$ for all $t \in T$ (see
Cohomology, Definition \ref{cohomology-definition-torsor}).
(II) To show that $\mathcal{F}$ is the trivial torsor it suffices
to show that $H^1(T, \mathcal{H}) = 0$ (see
Cohomology, Lemma \ref{cohomology-lemma-torsors-h1} --
we may use either cohomology
of $\mathcal{H}$ as an abelian sheaf or as an $\mathcal{O}_T$-module,
see Cohomology, Lemma \ref{cohomology-lemma-modules-abelian}).

\medskip\noindent
First we prove (I). To see this, for every $t \in T$ we can
choose an affine open $U \subset T$ neighbourhood of $t$
such that $a(U)$ is contained
in an affine open $\text{Spec}(A) = W \subset X$
which maps to an affine open $\text{Spec}(R) = V \subset S$.
By Morphisms, Lemma \ref{morphisms-lemma-smooth-characterize}
the ring map $R \to A$ is smooth.
Hence by Algebra, Lemma \ref{algebra-lemma-formally-smooth-smooth}
the ring map $R \to A$ is formally smooth.
Lemma \ref{lemma-affine-formally-smooth}
in turn implies that $W \to V$ is formally smooth.
Hence we can lift $a|_U : U \to W$ to a $V$-morphism
$a' : U' \to W \subset X$ showing that $\mathcal{F}(U) \not = \emptyset$.

\medskip\noindent
Finally we prove (II).
By Morphisms, Lemma \ref{morphisms-lemma-finite-presentation-differentials}
we see that $\Omega_{X/S}$ is of finite presentation
(it is even finite locally free by
Morphisms, Lemma \ref{morphisms-lemma-smooth-omega-finite-locally-free}).
Hence $a^*\Omega_{X/S}$ is of finite presentation (see
Modules, Lemma \ref{modules-lemma-pullback-finite-presentation}).
Hence the sheaf
$\mathcal{H} = \textit{Hom}_{\mathcal{O}_T}(a^*\Omega^1_{X/S}, \mathcal{I})$
is quasi-coherent by the discussion in
Schemes, Section \ref{schemes-section-quasi-coherent}.
Thus by
Coherent, Lemma \ref{coherent-lemma-quasi-coherent-affine-cohomology-zero}
we have $H^1(X, \mathcal{H}) = 0$ as desired.
\end{proof}













\section{Formally unramified morphisms}
\label{section-formally-unramified}

\noindent
Recall that a ring map $R \to A$ is called {\it formally unramified}
(see Algebra, Definition \ref{algebra-definition-formally-unramified})
if for every commutative solid diagram
$$
\xymatrix{
A \ar[r] \ar@{-->}[rd] & B/I \\
R \ar[r] \ar[u] & B \ar[u]
}
$$
where $I \subset B$ is an ideal of square zero, at most one dotted
arrow exists which makes the diagram commute. This motivates
the following analogue for morphisms of schemes.

\begin{definition}
\label{definition-formally-unramified}
Let $f : X \to S$ be a morphism of schemes.
We say $f$ is {\it formally unramified} if given any solid commutative diagram
$$
\xymatrix{
X \ar[d]_f & T \ar[d]^i \ar[l] \\
S & T' \ar[l] \ar@{-->}[lu]
}
$$
where $T$ and $T'$ are affine and $i$ is a closed immersion defined
by an ideal of square zero there exists at most one
dotted arrow making the diagram commute.
\end{definition}

\noindent
Here is a characterization in terms of the sheaf of differentials.

\begin{lemma}
\label{lemma-formally-unramified-differentials}
Let $f : X \to S$ be a morphism of schemes.
Then $f$ is formally unramified if and only if $\Omega_{X/S} = 0$.
\end{lemma}

\begin{proof}
We recall some of the arguments of
Morphisms, Section \ref{morphisms-section-sheaf-differentials}.
Let $U \subset X \times_S X$ be an open such that
$\Delta : X \to X \times_S X$ induces a closed immersion into $U$.
Let $\mathcal{J} \subset \mathcal{O}_U$ be the ideal sheaf of this
closed immersion. Let $X' \subset U$ be the closed subscheme
defined by the quasi-coherent sheaf of ideals $\mathcal{J}^2$.
Consider the two morphisms $p_1, p_2 : X' \to X$ induced by
the two projections $X \times_S X \to X$.
Note that $p_1$ and $p_2$ agree when composed with $\Delta : X \to X'$
and that $X \to X'$ is a closed immersion defined by a an ideal
whose square is zero. Moreover there is a short exact sequence
$$
0 \to \mathcal{J}/\mathcal{J}^2 \to \mathcal{O}_{X'} \to \mathcal{O}_X \to 0
$$
and $\Omega_{X/S} = \mathcal{J}/\mathcal{J}^2$ is generated by the local
sections $p_1^\sharp(f) - p_2^\sharp(f)$ for $f$ a local section of
$\mathcal{O}_X$.

\medskip\noindent
Suppose that $f : X \to S$ is formally unramified.
By assumption this means that $p_1 = p_2$ when restricted to any
affine open $T' \subset X'$. Hence $p_1 = p_2$. By what was said above
we conclude that $\Omega_{X/S} = 0$.

\medskip\noindent
Conversely, suppose that $\Omega_{X/S} = 0$.
Then it is clear from Lemma \ref{lemma-action-by-derivations}
that there is always at most one morphism fitting into the diagram of
Definition \ref{definition-formally-unramified}.
\end{proof}

\begin{lemma}
\label{lemma-composition-formally-unramified}
A composition of formally unramified morphisms is formally unramified.
\end{lemma}

\begin{proof}
Omitted.
\end{proof}

\begin{lemma}
\label{lemma-base-change-formally-unramified}
A base change of a formally unramified morphism is formally unramified.
\end{lemma}

\begin{proof}
Omitted.
\end{proof}

\begin{lemma}
\label{lemma-formally-unramified-on-opens}
Let $f : X \to S$ be a morphism of schemes.
Let $U \subset X$ and $V \subset S$ be open subschemes such that
$f(U) \subset V$. If $f$ is formally unramified, so is $f|_U : U \to V$.
\end{lemma}

\begin{proof}
Consider a solid diagram
$$
\xymatrix{
U \ar[d]_{f|_U} & T \ar[d]^i \ar[l]^a \\
V & T' \ar[l] \ar@{-->}[lu]
}
$$
as in Definition \ref{definition-formally-unramified}. If $f$ is formally
ramified, then there exists at most one
$S$-morphism $a' : T' \to X$ such that $a'|_T = a$.
Hence clearly there exists at most one such morphism into $U$.
\end{proof}

\begin{lemma}
\label{lemma-affine-formally-unramified}
Let $f : X \to S$ be a morphism of schemes.
Assume $X$ and $S$ are affine.
Then $f$ is formally unramified if and only if
$\mathcal{O}_S(S) \to \mathcal{O}_X(X)$ is a formally unramified
ring map.
\end{lemma}

\begin{proof}
This is immediate from the definitions
(Definition \ref{definition-formally-unramified} and
Algebra, Definition \ref{algebra-definition-formally-unramified})
by the equivalence of categories of rings and affine schemes,
see
Schemes, Lemma \ref{schemes-lemma-category-affine-schemes}.
\end{proof}

\begin{lemma}
\label{lemma-unramified-formally-unramified}
Let $f : X \to S$ be a morphism of schemes.
The following are equivalent:
\begin{enumerate}
\item The morphism $f$ is unramified, and
\item the morphism $f$ is locally of finite presentation and
formally unramified.
\end{enumerate}
\end{lemma}

\begin{proof}
Use Lemma \ref{lemma-formally-unramified-differentials} and
Morphisms, Lemma \ref{morphisms-lemma-unramified-omega-zero}.
\end{proof}













\section{Formally etale morphisms}
\label{section-formally-etale}

\noindent
Recall that a ring map $R \to A$ is called {\it formally etale}
(see Algebra, Definition \ref{algebra-definition-formally-etale})
if for every commutative solid diagram
$$
\xymatrix{
A \ar[r] \ar@{-->}[rd] & B/I \\
R \ar[r] \ar[u] & B \ar[u]
}
$$
where $I \subset B$ is an ideal of square zero, there exists
exactly one dotted arrow which makes the diagram commute. This motivates
the following analogue for morphisms of schemes.

\begin{definition}
\label{definition-formally-etale}
Let $f : X \to S$ be a morphism of schemes.
We say $f$ is {\it formally etale} if given any solid commutative diagram
$$
\xymatrix{
X \ar[d]_f & T \ar[d]^i \ar[l] \\
S & T' \ar[l] \ar@{-->}[lu]
}
$$
where $T$ and $T'$ are affine and $i$ is a closed immersion defined
by an ideal of square zero there exists exactly one
dotted arrow making the diagram commute.
\end{definition}

\noindent
This in particular garantees that $\Omega_{X/S}$ is zero since
this is the case for formally unramified morphisms.

\begin{lemma}
\label{lemma-formally-etale-unramfied-smooth}
Let $f : X \to S$ be a morphism of schemes.
Then $f$ is formally etale if and only if
$f$ is formally smooth and formally unramified.
\end{lemma}

\begin{proof}
Omitted.
\end{proof}

\begin{lemma}
\label{lemma-composition-formally-etale}
A composition of formally etale morphisms is formally etale.
\end{lemma}

\begin{proof}
Omitted.
\end{proof}

\begin{lemma}
\label{lemma-base-change-formally-etale}
A base change of a formally etale morphism is formally etale.
\end{lemma}

\begin{proof}
Omitted.
\end{proof}

\begin{lemma}
\label{lemma-formally-etale-on-opens}
Let $f : X \to S$ be a morphism of schemes.
Let $U \subset X$ and $V \subset S$ be open subschemes such that
$f(U) \subset V$. If $f$ is formally etale, so is $f|_U : U \to V$.
\end{lemma}

\begin{proof}
Combine Lemmas \ref{lemma-formally-etale-unramfied-smooth},
\ref{lemma-formally-smooth-on-opens}, and
\ref{lemma-formally-unramified-on-opens}.
\end{proof}

\begin{lemma}
\label{lemma-affine-formally-etale}
Let $f : X \to S$ be a morphism of schemes.
Assume $X$ and $S$ are affine.
Then $f$ is formally etale if and only if
$\mathcal{O}_S(S) \to \mathcal{O}_X(X)$ is a formally etale
ring map.
\end{lemma}

\begin{proof}
This is immediate from the definitions
(Definition \ref{definition-formally-etale} and
Algebra, Definition \ref{algebra-definition-formally-etale})
by the equivalence of categories of rings and affine schemes,
see
Schemes, Lemma \ref{schemes-lemma-category-affine-schemes}.
\end{proof}

\begin{lemma}
\label{lemma-etale-formally-etale}
Let $f : X \to S$ be a morphism of schemes.
The following are equivalent:
\begin{enumerate}
\item The morphism $f$ is etale, and
\item the morphism $f$ is locally of finite presentation and
formally etale.
\end{enumerate}
\end{lemma}

\begin{proof}
Formally etale is the same as formally smooth and formally unramified
(see Lemma \ref{lemma-formally-etale-unramfied-smooth}).
Being etale is the same as being smooth and unramified
(see Morphisms, Lemma \ref{morphisms-lemma-etale-smooth-unramified}).
Hence this follows from Lemmas \ref{lemma-smooth-formally-smooth}
and \ref{lemma-unramified-formally-unramified}.
\end{proof}












\section{Smoothness over a Noetherian base}
\label{section-smooth-Noetherian}

\noindent
It turns out that if the base is Noetherian then we can get away with
less in the formulation of formal smoothness.

\begin{definition}
\end{definition}























\section{Other chapters}

\begin{multicols}{2}
\begin{enumerate}
\item \hyperref[introduction-section-phantom]{Introduction}
\item \hyperref[conventions-section-phantom]{Conventions}
\item \hyperref[sets-section-phantom]{Set Theory}
\item \hyperref[categories-section-phantom]{Categories}
\item \hyperref[topology-section-phantom]{Topology}
\item \hyperref[sheaves-section-phantom]{Sheaves on Spaces}
\item \hyperref[algebra-section-phantom]{Commutative Algebra}
\item \hyperref[sites-section-phantom]{Sites and Sheaves}
\item \hyperref[homology-section-phantom]{Homological Algebra}
\item \hyperref[derived-section-phantom]{Derived Categories}
\item \hyperref[more-algebra-section-phantom]{More Algebra}
\item \hyperref[simplicial-section-phantom]{Simplicial Methods}
\item \hyperref[modules-section-phantom]{Sheaves of Modules}
\item \hyperref[sites-modules-section-phantom]{Modules on Sites}
\item \hyperref[injectives-section-phantom]{Injectives}
\item \hyperref[cohomology-section-phantom]{Cohomology of Sheaves}
\item \hyperref[sites-cohomology-section-phantom]{Cohomology on Sites}
\item \hyperref[hypercovering-section-phantom]{Hypercoverings}
\item \hyperref[schemes-section-phantom]{Schemes}
\item \hyperref[constructions-section-phantom]{Constructions of Schemes}
\item \hyperref[properties-section-phantom]{Properties of Schemes}
\item \hyperref[morphisms-section-phantom]{Morphisms of Schemes}
\item \hyperref[coherent-section-phantom]{Coherent Cohomology}
\item \hyperref[divisors-section-phantom]{Divisors}
\item \hyperref[limits-section-phantom]{Limits of Schemes}
\item \hyperref[varieties-section-phantom]{Varieties}
\item \hyperref[chow-section-phantom]{Chow Homology}
\item \hyperref[topologies-section-phantom]{Topologies on Schemes}
\item \hyperref[descent-section-phantom]{Descent}
\item \hyperref[more-morphisms-section-phantom]{More on Morphisms}
\item \hyperref[flat-section-phantom]{More on Flatness}
\item \hyperref[groupoids-section-phantom]{Groupoid Schemes}
\item \hyperref[more-groupoids-section-phantom]{More on Groupoid Schemes}
\item \hyperref[etale-section-phantom]{\'Etale Morphisms of Schemes}
\item \hyperref[etale-cohomology-section-phantom]{\'Etale Cohomology}
\item \hyperref[spaces-section-phantom]{Algebraic Spaces}
\item \hyperref[spaces-properties-section-phantom]{Properties of Algebraic Spaces}
\item \hyperref[spaces-morphisms-section-phantom]{Morphisms of Algebraic Spaces}
\item \hyperref[spaces-topologies-section-phantom]{Topologies on Algebraic Spaces}
\item \hyperref[spaces-descent-section-phantom]{Descent and Algebraic Spaces}
\item \hyperref[spaces-more-morphisms-section-phantom]{More on Morphisms of Spaces}
\item \hyperref[quot-section-phantom]{Quot and Hilbert Spaces}
\item \hyperref[stacks-section-phantom]{Stacks}
\item \hyperref[spaces-groupoids-section-phantom]{Groupoids in Algebraic Spaces}
\item \hyperref[spaces-more-groupoids-section-phantom]{More on Groupoids in Spaces}
\item \hyperref[bootstrap-section-phantom]{Bootstrap}
\item \hyperref[examples-stacks-section-phantom]{Examples of Stacks}
\item \hyperref[groupoids-quotients-section-phantom]{Quotients of Groupoids}
\item \hyperref[algebraic-section-phantom]{Algebraic Stacks}
\item \hyperref[criteria-section-phantom]{Criteria for Representability}
\item \hyperref[stacks-properties-section-phantom]{Properties of Algebraic Stacks}
\item \hyperref[stacks-morphisms-section-phantom]{Morphisms of Algebraic Stacks}
\item \hyperref[examples-section-phantom]{Examples}
\item \hyperref[exercises-section-phantom]{Exercises}
\item \hyperref[guide-section-phantom]{Guide to Literature}
\item \hyperref[desirables-section-phantom]{Desirables}
\item \hyperref[coding-section-phantom]{Coding Style}
\item \hyperref[fdl-section-phantom]{GNU Free Documentation License}
\item \hyperref[index-section-phantom]{Auto Generated Index}
\end{enumerate}
\end{multicols}


\bibliography{my}
\bibliographystyle{alpha}

\end{document}
