\IfFileExists{stacks-project.cls}{%
\documentclass{stacks-project}
}{%
\documentclass{amsart}
}

% The following AMS packages are automatically loaded with
% the amsart documentclass:
%\usepackage{amsmath}
%\usepackage{amssymb}
%\usepackage{amsthm}

% For dealing with references we use the comment environment
\usepackage{verbatim}
\newenvironment{reference}{\comment}{\endcomment}
%\newenvironment{reference}{}{}
\newenvironment{slogan}{\comment}{\endcomment}
\newenvironment{history}{\comment}{\endcomment}

% For commutative diagrams you can use
% \usepackage{amscd}
\usepackage[all]{xy}

% We use 2cell for 2-commutative diagrams.
\xyoption{2cell}
\UseAllTwocells

% To put source file link in headers.
% Change "template.tex" to "this_filename.tex"
% \usepackage{fancyhdr}
% \pagestyle{fancy}
% \lhead{}
% \chead{}
% \rhead{Source file: \url{template.tex}}
% \lfoot{}
% \cfoot{\thepage}
% \rfoot{}
% \renewcommand{\headrulewidth}{0pt}
% \renewcommand{\footrulewidth}{0pt}
% \renewcommand{\headheight}{12pt}

\usepackage{multicol}

% For cross-file-references
\usepackage{xr-hyper}

% Package for hypertext links:
\usepackage{hyperref}

% For any local file, say "hello.tex" you want to link to please
% use \externaldocument[hello-]{hello}
\externaldocument[introduction-]{introduction}
\externaldocument[conventions-]{conventions}
\externaldocument[sets-]{sets}
\externaldocument[categories-]{categories}
\externaldocument[topology-]{topology}
\externaldocument[sheaves-]{sheaves}
\externaldocument[sites-]{sites}
\externaldocument[stacks-]{stacks}
\externaldocument[fields-]{fields}
\externaldocument[algebra-]{algebra}
\externaldocument[brauer-]{brauer}
\externaldocument[homology-]{homology}
\externaldocument[derived-]{derived}
\externaldocument[simplicial-]{simplicial}
\externaldocument[more-algebra-]{more-algebra}
\externaldocument[smoothing-]{smoothing}
\externaldocument[modules-]{modules}
\externaldocument[sites-modules-]{sites-modules}
\externaldocument[injectives-]{injectives}
\externaldocument[cohomology-]{cohomology}
\externaldocument[sites-cohomology-]{sites-cohomology}
\externaldocument[dga-]{dga}
\externaldocument[dpa-]{dpa}
\externaldocument[hypercovering-]{hypercovering}
\externaldocument[schemes-]{schemes}
\externaldocument[constructions-]{constructions}
\externaldocument[properties-]{properties}
\externaldocument[morphisms-]{morphisms}
\externaldocument[coherent-]{coherent}
\externaldocument[divisors-]{divisors}
\externaldocument[limits-]{limits}
\externaldocument[varieties-]{varieties}
\externaldocument[topologies-]{topologies}
\externaldocument[descent-]{descent}
\externaldocument[perfect-]{perfect}
\externaldocument[more-morphisms-]{more-morphisms}
\externaldocument[flat-]{flat}
\externaldocument[groupoids-]{groupoids}
\externaldocument[more-groupoids-]{more-groupoids}
\externaldocument[etale-]{etale}
\externaldocument[chow-]{chow}
\externaldocument[intersection-]{intersection}
\externaldocument[pic-]{pic}
\externaldocument[adequate-]{adequate}
\externaldocument[dualizing-]{dualizing}
\externaldocument[duality-]{duality}
\externaldocument[discriminant-]{discriminant}
\externaldocument[local-cohomology-]{local-cohomology}
\externaldocument[curves-]{curves}
\externaldocument[resolve-]{resolve}
\externaldocument[models-]{models}
\externaldocument[pione-]{pione}
\externaldocument[etale-cohomology-]{etale-cohomology}
\externaldocument[proetale-]{proetale}
\externaldocument[crystalline-]{crystalline}
\externaldocument[spaces-]{spaces}
\externaldocument[spaces-properties-]{spaces-properties}
\externaldocument[spaces-morphisms-]{spaces-morphisms}
\externaldocument[decent-spaces-]{decent-spaces}
\externaldocument[spaces-cohomology-]{spaces-cohomology}
\externaldocument[spaces-limits-]{spaces-limits}
\externaldocument[spaces-divisors-]{spaces-divisors}
\externaldocument[spaces-over-fields-]{spaces-over-fields}
\externaldocument[spaces-topologies-]{spaces-topologies}
\externaldocument[spaces-descent-]{spaces-descent}
\externaldocument[spaces-perfect-]{spaces-perfect}
\externaldocument[spaces-more-morphisms-]{spaces-more-morphisms}
\externaldocument[spaces-flat-]{spaces-flat}
\externaldocument[spaces-groupoids-]{spaces-groupoids}
\externaldocument[spaces-more-groupoids-]{spaces-more-groupoids}
\externaldocument[bootstrap-]{bootstrap}
\externaldocument[spaces-pushouts-]{spaces-pushouts}
\externaldocument[groupoids-quotients-]{groupoids-quotients}
\externaldocument[spaces-more-cohomology-]{spaces-more-cohomology}
\externaldocument[spaces-simplicial-]{spaces-simplicial}
\externaldocument[formal-spaces-]{formal-spaces}
\externaldocument[restricted-]{restricted}
\externaldocument[spaces-resolve-]{spaces-resolve}
\externaldocument[formal-defos-]{formal-defos}
\externaldocument[defos-]{defos}
\externaldocument[cotangent-]{cotangent}
\externaldocument[examples-defos-]{examples-defos}
\externaldocument[algebraic-]{algebraic}
\externaldocument[examples-stacks-]{examples-stacks}
\externaldocument[stacks-sheaves-]{stacks-sheaves}
\externaldocument[criteria-]{criteria}
\externaldocument[artin-]{artin}
\externaldocument[quot-]{quot}
\externaldocument[stacks-properties-]{stacks-properties}
\externaldocument[stacks-morphisms-]{stacks-morphisms}
\externaldocument[stacks-limits-]{stacks-limits}
\externaldocument[stacks-cohomology-]{stacks-cohomology}
\externaldocument[stacks-perfect-]{stacks-perfect}
\externaldocument[stacks-introduction-]{stacks-introduction}
\externaldocument[stacks-more-morphisms-]{stacks-more-morphisms}
\externaldocument[stacks-geometry-]{stacks-geometry}
\externaldocument[moduli-]{moduli}
\externaldocument[moduli-curves-]{moduli-curves}
\externaldocument[examples-]{examples}
\externaldocument[exercises-]{exercises}
\externaldocument[guide-]{guide}
\externaldocument[desirables-]{desirables}
\externaldocument[coding-]{coding}
\externaldocument[obsolete-]{obsolete}
\externaldocument[fdl-]{fdl}
\externaldocument[index-]{index}

% Theorem environments.
%
\theoremstyle{plain}
\newtheorem{theorem}[subsection]{Theorem}
\newtheorem{proposition}[subsection]{Proposition}
\newtheorem{lemma}[subsection]{Lemma}

\theoremstyle{definition}
\newtheorem{definition}[subsection]{Definition}
\newtheorem{example}[subsection]{Example}
\newtheorem{exercise}[subsection]{Exercise}
\newtheorem{situation}[subsection]{Situation}

\theoremstyle{remark}
\newtheorem{remark}[subsection]{Remark}
\newtheorem{remarks}[subsection]{Remarks}

\numberwithin{equation}{subsection}

% Macros
%
\def\lim{\mathop{\rm lim}\nolimits}
\def\colim{\mathop{\rm colim}\nolimits}
\def\Spec{\mathop{\rm Spec}}
\def\Hom{\mathop{\rm Hom}\nolimits}
\def\Ext{\mathop{\rm Ext}\nolimits}
\def\SheafHom{\mathop{\mathcal{H}\!{\it om}}\nolimits}
\def\SheafExt{\mathop{\mathcal{E}\!{\it xt}}\nolimits}
\def\Sch{\textit{Sch}}
\def\Mor{\mathop{\rm Mor}\nolimits}
\def\Ob{\mathop{\rm Ob}\nolimits}
\def\Sh{\mathop{\textit{Sh}}\nolimits}
\def\NL{\mathop{N\!L}\nolimits}
\def\proetale{{pro\text{-}\acute{e}tale}}
\def\etale{{\acute{e}tale}}
\def\QCoh{\textit{QCoh}}
\def\Ker{\mathop{\rm Ker}}
\def\Im{\mathop{\rm Im}}
\def\Coker{\mathop{\rm Coker}}
\def\Coim{\mathop{\rm Coim}}

%
% Macros for moduli stacks/spaces
%
\def\QCohstack{\mathcal{QC}\!{\it oh}}
\def\Cohstack{\mathcal{C}\!{\it oh}}
\def\Spacesstack{\mathcal{S}\!{\it paces}}
\def\Quotfunctor{{\rm Quot}}
\def\Hilbfunctor{{\rm Hilb}}
\def\Curvesstack{\mathcal{C}\!{\it urves}}
\def\Polarizedstack{\mathcal{P}\!{\it olarized}}
\def\Complexesstack{\mathcal{C}\!{\it omplexes}}
% \Pic is the operator that assigns to X its picard group, usage \Pic(X)
% \Picardstack_{X/B} denotes the Picard stack of X over B
% \Picardfunctor_{X/B} denotes the Picard functor of X over B
\def\Pic{\mathop{\rm Pic}\nolimits}
\def\Picardstack{\mathcal{P}\!{\it ic}}
\def\Picardfunctor{{\rm Pic}}
\def\Deformationcategory{\mathcal{D}\!{\it ef}}


% OK, start here.
%
\begin{document}

\title{Coherent Cohomology}


\maketitle

\phantomsection
\label{section-phantom}

\tableofcontents

\section{Introduction}
\label{section-introduction}

\noindent
The title of this chapter is a bit of a lie because we
will first prove a number of results on the cohomology of quasi-coherent
sheaves. A fundamental reference is \cite{EGA}.
Having done this we wil elaborate on cohomology of
coherent sheaves in the Noetherian setting. See also \cite{FAC}.







\section{Cech cohomology of quasi-coherent sheaves}
\label{section-cech-quasi-coherent}

\noindent
Let $X$ be a scheme.
Let $U \subset X$ be an affine open.
Recall that a {\it standard open covering} of $U$ is a covering
of the form $\mathcal{U} : U = \bigcup_{i = 1}^n D(f_i)$
where $f_1, \ldots, f_n \in \Gamma(U, \mathcal{O}_X)$ generate
the unit ideal, see
Schemes, Definition \ref{schemes-definition-standard-covering}.

\begin{lemma}
\label{lemma-cech-cohomology-quasi-coherent-trivial}
Let $X$ be a scheme.
Let $\mathcal{F}$ be a quasi-coherent $\mathcal{O}_X$-module.
Let $\mathcal{U} : U = \bigcup_{i = 1}^n D(f_i)$ be a standard
open covering of an affine open of $X$.
Then $\check{H}^p(\mathcal{U}, \mathcal{F}) = 0$ for
all $p > 0$.
\end{lemma}

\begin{proof}
Write $U = \text{Spec}(A)$ for some ring $A$.
In other words, $f_1, \ldots, f_n$ are elements of $A$
which generate the unit ideal of $A$.
Write $\mathcal{F}|_U = \widetilde{M}$ for some $A$-module $M$.
Clearly the Cech complex
$\check{\mathcal{C}}^\bullet(\mathcal{U}, \mathcal{F})$
is identified with the complex
$$
\prod_{i_0} M_{f_{i_0}} \to
\prod_{i_0i_1} M_{f_{i_0}f_{i_1}} \to
\prod_{i_0i_1i_2} M_{f_{i_0}f_{i_1}f_{i_2}} \to
\ldots
$$
We are asked to show that the extended complex
\begin{equation}
\label{equation-extended}
0 \to
M \to
\prod_{i_0} M_{f_{i_0}} \to
\prod_{i_0i_1} M_{f_{i_0}f_{i_1}} \to
\prod_{i_0i_1i_2} M_{f_{i_0}f_{i_1}f_{i_2}} \to
\ldots
\end{equation}
(whose truncation we have studied in
Algebra, Lemma \ref{algebra-lemma-cover-module}) is exact.
It suffices to show that (\ref{equation-extended})
is exact after localizing at a prime $\mathfrak p$, see
Algebra, Lemma \ref{algebra-lemma-characterize-zero-local}.
In fact we will show that the extended complex localized
at $\mathfrak p$ is homotopic to zero.

\medskip\noindent
There exists an index $i$ such that $f_i \not \in \mathfrak p$.
Choose and fix such an element $i_{\text{fix}}$. Note that
$M_{f_{i_{\text{fix}}}, \mathfrak p} = M_{\mathfrak p}$. Similarly
for a localization at a product $f_{i_0} \ldots f_{i_p}$ and $\mathfrak p$
we can drop any $f_{i_j}$ for which $i_j = i_{\text{fix}}$.
Let us define a homotopy
$$
h :
\prod\nolimits_{i_0 \ldots i_{p + 1}}
M_{f_{i_0} \ldots f_{i_{p + 1}}, \mathfrak p}
\longrightarrow
\prod\nolimits_{i_0 \ldots i_p}
M_{f_{i_0} \ldots f_{i_p}, \mathfrak p}
$$
by the rule
$$
h(s)_{i_0 \ldots i_p} = s_{i_{\text{fix}} i_0 \ldots i_p}
$$
(This is ``dual'' to the homotopy in the proof of
Cohomology, Lemma \ref{cohomology-lemma-homology-complex}.)
In other words, $h : \prod_{i_0} M_{f_{i_0}, \mathfrak p} \to M$
is projection onto the factor
$M_{f_{i_{\text{fix}}}, \mathfrak p} = M_{\mathfrak p}$ and in general
the map $h$ equal projection onto the factors
$M_{f_{i_{\text{fix}}} f_{i_1} \ldots f_{i_{p + 1}}, \mathfrak p}
= M_{f_{i_1} \ldots f_{i_{p + 1}}, \mathfrak p}$. We compute
\begin{align*}
(dh + hd)(s)_{i_0 \ldots i_p}
= &
\sum_{j = 0}^p
(-1)^j
h(s)_{i_0 \ldots \hat i_j \ldots i_p}
+
d(s)_{i_{\text{fix}} i_0 \ldots i_p}\\
= &
\sum_{j = 0}^p
(-1)^j
s_{i_{\text{fix}} i_0 \ldots \hat i_j \ldots i_p}
+
s_{i_0 \ldots i_p}
+
\sum_{j = 0}^p
(-1)^{j + 1}
s_{i_{\text{fix}} i_0 \ldots \hat i_j \ldots i_p} \\
= &
s_{i_0 \ldots i_p}
\end{align*}
This proves the identity map is homotopic to zero as desired.
\end{proof}

\noindent
The following lemma says in particular that for any affine scheme
$X$ and any quasi-coherent sheaf $\mathcal{F}$ on $X$ we have
$$
H^p(X, \mathcal{F}) = 0
$$
for all $p > 0$.

\begin{lemma}
\label{lemma-quasi-coherent-affine-cohomology-zero}
Let $X$ be a scheme.
Let $\mathcal{F}$ be a quasi-coherent $\mathcal{O}_X$-module.
For any affine open $U \subset X$ we have
$H^p(U, \mathcal{F}) = 0$ for all $p > 0$.
\end{lemma}

\begin{proof}
We are going to apply
Cohomology, Lemma \ref{cohomology-lemma-cech-vanish-basis}.
As our basis $\mathcal{B}$ for the topology of $X$ we are going to use
the affine opens of $X$.
As our set $\text{Cov}$ of open coverings we are going to use the standard
open coverings of affine opens of $X$.
Next we check that conditions (1), (2) and (3) of
Cohomology, Lemma \ref{cohomology-lemma-cech-vanish-basis}
hold. Note that the intersection of standard opens in an affine is
another standard open. Hence property (1) holds. 
The coverings form a cofinal system of open coverings of any element
of $\mathcal{B}$, see
Schemes, Lemma \ref{schemes-lemma-standard-open}.
Hence (2) holds.
Finally, condition (3) of the lemma follows from
Lemma \ref{lemma-cech-cohomology-quasi-coherent-trivial}.
\end{proof}

\noindent
Here is a relative version of the vanishing of cohomology of quasi-coherent
sheaves on affines.

\begin{lemma}
\label{lemma-relative-affine-vanishing}
Let $f : X \to S$ be a morphism of schemes.
Let $\mathcal{F}$ be a quasi-coherent $\mathcal{O}_X$-module.
If $f$ is affine then $R^if_*\mathcal{F} = 0$ for all $i > 0$.
\end{lemma}

\begin{proof}
According to
Cohomology, Lemma \ref{cohomology-lemma-describe-higher-direct-images}
the sheaf
$R^if_*\mathcal{F}$ is the sheaf associated to the presheaf
$V \mapsto H^i(f^{-1}(V), \mathcal{F}|_{f^{-1}(V)})$.
By assumption, whenever $V$ is affine we have that $f^{-1}(V)$ is
affine, see Morphisms, Definition \ref{morphisms-definition-affine}.
By Lemma \ref{lemma-quasi-coherent-affine-cohomology-zero} we conclude that
$H^i(f^{-1}(V), \mathcal{F}|_{f^{-1}(V)}) = 0$
whenever $V$ is affine. Since $S$ has a basis consisting of affine
opens we win.
\end{proof}

\begin{lemma}
\label{lemma-cech-cohomology-quasi-coherent}
Let $X$ be a scheme.
Let $\mathcal{U} : X = \bigcup_{i \in I} U_i$ be an open covering such that
$U_{i_0 \ldots i_p}$ is affine open for all $p \ge 0$ and all
$i_0, \ldots, i_p \in I$
In this case for any quasi-coherent sheaf $\mathcal{F}$ we have
$$
\check{H}^p(\mathcal{U}, \mathcal{F}) = H^p(X, \mathcal{F})
$$
as $\Gamma(X, \mathcal{O}_X)$-modules for all $p$.
\end{lemma}

\begin{proof}
In view of Lemma \ref{lemma-quasi-coherent-affine-cohomology-zero}
this is a special case of
Comology, Lemma \ref{cohomology-lemma-cech-spectral-sequence-application}.
\end{proof}







\section{Vanishing of cohomology}
\label{section-vanishing}

\noindent
We have seen that on an affine scheme the higher cohomology groups
of any quasi-coherent sheaf vanish
(Lemma \ref{lemma-quasi-coherent-affine-cohomology-zero}).
It turns out that this also
characterizes affine schemes. We give two versions allthough the
first covers all conceivable cases.

\begin{lemma}
\label{lemma-quasi-compact-h1-zero-covering}
Let $X$ be a scheme.
Assume that
\begin{enumerate}
\item $X$ is quasi-compact,
\item for every quasi-coherent sheaf of ideals
$\mathcal{I} \subset \mathcal{O}_X$ we have $H^1(X, \mathcal{I}) = 0$.
\end{enumerate}
Then $X$ is affine.
\end{lemma}

\begin{proof}
Let $x \in X$ be a closed point. Let $U \subset X$ be an affine open
neighbourhood of $x$. Write $U = \text{Spec}(A)$ and let
$\mathfrak m \subset A$ be the maximal ideal corresponding to $x$.
Set $Z = X \setminus U$ and $Z' = Z \cup \{x\}$.
By Schemes, Lemma \ref{schemes-lemma-reduced-closed-subscheme} there
are quasi-coherent sheaves of ideals
$\mathcal{I}$, resp.\ $\mathcal{I}'$ cutting out
the reduced closed subschemes $Z$, resp.\ $Z'$.
Consider the short exact sequence
$$
0 \to \mathcal{I}' \to \mathcal{I} \to \mathcal{I}/\mathcal{I}' \to 0.
$$
Since $x$ is a closed point of $X$ and $x \not \in Z$ we see that
$\mathcal{I}/\mathcal{I}'$ is supported at $x$. In fact, the restriction
of $\mathcal{I}/\mathcal{I'}$ to $U$ corresponds to the $A$-module
$A/\mathfrak m$. Hence we see that $\Gamma(X, \mathcal{I}/\mathcal{I'})
= A/\mathfrak m$. Since by assumption $H^1(X, \mathcal{I}') = 0$
we see there exists a global section $f \in \Gamma(X, \mathcal{I})$
which maps to the element $1 \in A/\mathfrak m$ as a section of
$\mathcal{I}/\mathcal{I'}$. Clearly we have
$x \in X_f \subset U$. This implies that $X_f = D(f_A)$ where
$f_A$ is the image of $f$ in $A = \Gamma(U, \mathcal{O}_X)$.
In particular $X_f$ is affine.

\medskip\noindent
Consider the union $W = \bigcup X_f$ over all $f \in \Gamma(X, \mathcal{O}_X)$
such that $X_f$ is affine. Obviously $W$ is open in $X$.
By the arguments above every closed point of
$X$ is contained in $W$. The closed subset $X \setminus W$ of $X$
is also quasi-compact
(see Topology, Lemma \ref{topology-lemma-closed-in-quasi-compact}).
Hence it has a closed point if it is nonempty (see
Topology, Lemma \ref{topology-lemma-quasi-compact-closed-point}).
This is a would contradict the fact that all closed points are in
$W$. Hence we conclude $X = W$.

\medskip\noindent
Choose finitely many $f_1, \ldots, f_n \in \Gamma(X, \mathcal{O}_X)$
such that $X = X_{f_1} \cup \ldots \cup X_{f_n}$ and such that each
$X_{f_i}$ is affine. This is possible as we've seen above.
By Properties, Lemma \ref{properties-lemma-characterize-affine}
to finish the proof it suffices
to show that $f_1, \ldots, f_n$ generate the unit ideal in
$\Gamma(X, \mathcal{O}_X)$. Conisder the short exact sequence
$$
\xymatrix{
0 \ar[r] &
\mathcal{F} \ar[r] &
\mathcal{O}_X^{\oplus n} \ar[rr]^{f_1, \ldots, f_n} & &
\mathcal{O}_X \ar[r] &
0
}
$$
The arrow defined by $f_1, \ldots, f_n$ is surjective since the
opens $X_{f_i}$ cover $X$. We let $\mathcal{F}$ be the kernel
of this surjective map.
Observe that $\mathcal{F}$ has a filtration
$$
0 = \mathcal{F}_0 \subset \mathcal{F}_1 \subset
\ldots \subset \mathcal{F}_n = \mathcal{F}
$$
so that each subquotient $\mathcal{F}_i/\mathcal{F}_{i - 1}$ is
isomorphic to a quasi-coherent sheaf of ideals.
Namely we can take $\mathcal{F}_i$ to be the intersection
of the first $i$ direct summands of $\mathcal{O}_X^{\oplus n}$.
The assumption
of the lemma implies that $H^1(X, \mathcal{F}_i/\mathcal{F}_{i - 1}) = 0$
for all $i$. This implies that
$H^1(X, \mathcal{F}_2) = 0$ because it is sandwiched between
$H^1(X, \mathcal{F}_1)$ and $H^1(X, \mathcal{F}_2/\mathcal{F}_1)$.
Continuing like this we deduce that $H^1(X, \mathcal{F}) = 0$.
Therefore we conclude that the map
$$
\xymatrix{
\bigoplus\nolimits_{i = 1, \ldots, n} \Gamma(X, \mathcal{O}_X)
\ar[rr]^{f_1, \ldots, f_n} & &
\Gamma(X, \mathcal{O}_X)
}
$$
is surjective as desired.
\end{proof}

\noindent
Note that if $X$ is a Noetherian scheme then every quasi-coherent
sheaf of ideals is automatically a coherent sheaf of ideals and a
finite type quasi-coherent sheaf of ideals. Hence
the preceding lemma and the next lemma both apply in this case.

\begin{lemma}
\label{lemma-quasi-separated-h1-zero-covering}
Let $X$ be a scheme. Assume that
\begin{enumerate}
\item $X$ is quasi-compact,
\item $X$ is quasi-separated, and
\item $H^1(X, \mathcal{I}) = 0$ for every quasi-coherent sheaf
of ideals $\mathcal{I}$ of finite type.
\end{enumerate}
Then $X$ is affine.
\end{lemma}

\begin{proof}
By
Properties, Lemma \ref{properties-lemma-quasi-coherent-colimit-finite-type}
every quasi-coherent sheaf of ideals is a directed colimit of
quasi-coherent sheaves of ideals of finite type.
By Cohomology, Lemma \ref{cohomology-lemma-quasi-separated-cohomology-colimit}
taking cohomology on $X$ commutes with directed colimits.
Hence we see that $H^1(X, \mathcal{I}) = 0$
for every quasi-coherent sheaf of ideals on $X$. In other words
we see that Lemma \ref{lemma-quasi-compact-h1-zero-covering} applies.
\end{proof}








\section{Quasi-coherence of higher direct images}
\label{section-quasi-coherence}

\begin{lemma}
\label{lemma-vanishing-nr-affines}
Let $X$ be a quasi-compact separated scheme.
Let $t \geq 1$ be the minimal number of affine opens needed to
cover $X$.
Then $H^n(X, \mathcal{F}) = 0$ for all $n \geq t$ and all
quasi-coherent sheaves $\mathcal{F}$.
\end{lemma}

\begin{proof}
First proof.
By induction on $t$.
If $t = 1$ the result follows from
Lemma \ref{lemma-quasi-coherent-affine-cohomology-zero}.
If $t > 1$ write $X = U \cup V$ with $V$ affine open and
$U = U_1 \cup \ldots \cup U_{t - 1}$ a union of $t - 1$ open affines.
Note that in this case
$U \cap V =  (U_1 \cap V) \cup \ldots (U_{t - 1} \cap V)$
is also a union of $t - 1$ affine open subschemes, see
Schemes, Lemma \ref{schemes-lemma-characterize-separated}.
We apply the Mayer-Vietoris long exact sequence
$$
0 \to
H^0(X, \mathcal{F}) \to
H^0(U, \mathcal{F}) \oplus H^0(V, \mathcal{F}) \to
H^0(U \cap V, \mathcal{F}) \to
H^1(X, \mathcal{F}) \to \ldots
$$
see Cohomology, Lemma \ref{cohomology-lemma-mayer-vietoris}.
By induction we see that $H^i(U, \mathcal{F})$,
$H^i(U, \mathcal{F})$, $H^i(U, \mathcal{F})$ are zero for
$i \geq t - 1$. It follows immediately that $H^i(X, \mathcal{F})$
is zero for $i \geq t$.

\medskip\noindent
Second proof. 
Let $\mathcal{U} : X = \bigcup_{i = 1}^t U_i$ be a finite affine open
covering. Since $X$ is separated the multiple intersections
$U_{i_0 \ldots i_p}$ are all affine, see 
Schemes, Lemma \ref{schemes-lemma-characterize-separated}.
By Lemma \ref{lemma-cech-cohomology-quasi-coherent} the Cech
cohomology groups $\check{H}^p(\mathcal{U}, \mathcal{F})$
agree with the cohomology groups. By
Cohomology, Lemma \ref{cohomology-lemma-alternating-usual}
the Cech cohomology groups may be computed using the alternating
Cech complex $\check{\mathcal{C}}_{alt}^\bullet(\mathcal{U}, \mathcal{F})$.
As the covering consists of $t$ elements we see immediately
that $\check{\mathcal{C}}_{alt}^p(\mathcal{U}, \mathcal{F}) = 0$
for all $p \geq t$. Hence the result follows.
\end{proof}

\begin{lemma}
\label{lemma-quasi-coherence-higher-direct-images}
Let $f : X \to S$ be a morphism of schemes.
Assume that $f$ is quasi-separated and quasi-compact.
\begin{enumerate}
\item For any quasi-coherent $\mathcal{O}_X$-module $\mathcal{F}$ the
higher direct images $R^pf_*\mathcal{F}$ are quasi-coherent on $S$.
\item If $S$ is quasi-compact, there exists an integer $n = n(X, S, f)$
such that $R^pf_*\mathcal{F} = 0$ for all $p \geq n$ and any
quasi-coherent sheaf $\mathcal{F}$ on $X$.
\end{enumerate}
\end{lemma}

\begin{proof}
Note that under the hypotheses of the lemma the sheaf
$R^0f_*\mathcal{F} = f_*\mathcal{F}$ is quasi-coherent by
Schemes, Lemma \ref{schemes-lemma-push-forward-quasi-coherent}.

\medskip\noindent
We will repeatedly use
Cohomology, Lemma \ref{cohomology-lemma-localize-higher-direct-images}
to see that forming higher direct images commutes with restriction.
Being quasi-coherent is local on $S$ and in order to prove (1) we
may assume $S$ is affine.

\medskip\noindent
Suppose $S = \bigcup_{i = 1, \ldots m} S_i$ is a finite affine open covering.
Assume $n_i$ works in (2) for the morphism $f^{-1}(S_i) \to S_i$.
Then we see that $n = \max\{n_i\}$ works for $X \to S$ in (2).
Hence we may assume $S$ is affine in order to prove (2) as well.

\medskip\noindent
Assume $S$ is affine and $f$ quasi-compact and separated.
Let $t \geq 1$ be the minimal number of affine opens needed to cover $X$.
We will prove this case of the lemma by induction on $t$ and we will in
addition show that setting $n = t$ in (2) works.
If $t = 1$ then the morphism $f$ is affine by
Morphisms, Lemma \ref{morphisms-lemma-morphism-affines-affine}
and (1) and (2) (with $n = 1$)
both follow from Lemma \ref{lemma-relative-affine-vanishing}.
If $t > 1$ write $X = U \cup V$ with $V$ affine open and
$U = U_1 \cup \ldots \cup U_{t - 1}$ a union of $t - 1$ open affines.
Note that in this case
$U \cap V =  (U_1 \cap V) \cup \ldots (U_{t - 1} \cap V)$
is also a union of $t - 1$ affine open subschemes, see
Schemes, Lemma \ref{schemes-lemma-characterize-separated}.
We will apply the relative Mayer-Vietoris sequence
$$
0 \to
f_*\mathcal{F} \to
a_*(\mathcal{F}|_U) \oplus b_*(\mathcal{F}|_V) \to
c_*(\mathcal{F}|_{U \cap V}) \to
R^1f_*\mathcal{F} \to \ldots
$$
see Cohomology, Lemma \ref{cohomology-lemma-relative-mayer-vietoris}.
By induction we see that
$R^pa_*\mathcal{F}$, $R^pb_*\mathcal{F}$ and $R^pc_*\mathcal{F}$
are all quasi-coherent and zero when $p \geq t - 1$. This immediately
implies the desired vanishing. It also implies that each of sheaves
$R^pf_*\mathcal{F}$ is quasi-coherent since it sits in the middle of a short
exact sequence with a cokernel of a map between quasi-coherent sheaves
on the left and a kernel of a map between quasi-coherent sheaves on the right.
Use the results on quasi-coherent sheaves in
Schemes, Section \ref{schemes-section-quasi-coherent} to see
this is sufficient to conclude.

\medskip\noindent
Assume $S$ is affine and $f$ quasi-compact and quasi-separated.
Let $t \geq 1$ be the minimal number of affine opens needed to cover $X$.
We will prove this case of lemma by induction on $t$. We have seen above
that this will imply the lemma in general.
If $t = 1$ then the morphism $f$ is affine by
Morphisms, Lemma \ref{morphisms-lemma-morphism-affines-affine}
and (1) and (2) (with $n = 1$)
both follow from Lemma \ref{lemma-relative-affine-vanishing}.
If $t > 1$ write $X = U \cup V$ with $V$ affine open and
$U$ a union of $t - 1$ open affines.
Note that in this case $U \cap V$ is an open subscheme of an affine
scheme and hence separated (see
Schemes, Lemma \ref{schemes-lemma-affine-separated}).
We will apply the relative Mayer-Vietoris sequence
$$
0 \to
f_*\mathcal{F} \to
a_*(\mathcal{F}|_U) \oplus b_*(\mathcal{F}|_V) \to
c_*(\mathcal{F}|_{U \cap V}) \to
R^1f_*\mathcal{F} \to \ldots
$$
see Cohomology, Lemma \ref{cohomology-lemma-relative-mayer-vietoris}.
By induction there exist integers $n_a, n_b, n_c$ such that
$R^pa_*\mathcal{F}$, $R^pb_*\mathcal{F}$ and $R^pc_*\mathcal{F}$
are all quasi-coherent and zero when $p \geq \max\{n_a, n_b, n_c\}$.
This immediately implies the vanishing in (2) with
$n = \max\{n_a, n_b, n_c\} + 1$. It also implies that each of sheaves
$R^pf_*\mathcal{F}$ is quasi-coherent since it sits in the middle of a short
exact sequence with a cokernel of a map between quasi-coherent sheaves
on the left and a kernel of a map between quasi-coherent sheaves on the right.
Use the results on quasi-coherent sheaves in
Schemes, Section \ref{schemes-section-quasi-coherent} to see
this is sufficient to conclude.
\end{proof}

\begin{lemma}
\label{lemma-quasi-coherence-higher-direct-images-application}
Let $f : X \to S$ be a morphism of schemes.
Assume that $f$ is quasi-separated and quasi-compact.
Assume $S$ is affine.
For any quasi-coherent $\mathcal{O}_X$-module $\mathcal{F}$
we have
$$
H^q(X, \mathcal{F}) = H^0(S, R^qf_*\mathcal{F})
$$
for all $q \in \mathbf{Z}$.
\end{lemma}

\begin{proof}
Consider the Leray spectral sequence $E_2^{p, q} = H^p(S, R^qf_*\mathcal{F})$
converging to $H^{p + q}(X, \mathcal{F})$, see
Cohomology, Lemma \ref{cohomology-lemma-Leray}.
By Lemma \ref{lemma-quasi-coherence-higher-direct-images}
we see that the sheaves $R^qf_*\mathcal{F}$ are quasi-coherent.
By Lemma \ref{lemma-quasi-coherent-affine-cohomology-zero}
we see that $E_2^{p, q} = 0$ when $p > 0$.
Hence the spectral sequence degenerates at $E_2$ and we win.
See also
Cohomology, Lemma \ref{cohomology-lemma-apply-Leray} (2)
for the general principle.
\end{proof}









\section{Ample invertible sheaves and cohomology}
\label{section-ample-cohomology}

\noindent
Given a ringed space $X$, an invertible $\mathcal{O}_X$-module $\mathcal{L}$,
a section $s \in \Gamma(X, \mathcal{L})$ and an $\mathcal{O}_X$-module
$\mathcal{F}$ we get a map
$\mathcal{F} \to \mathcal{F} \otimes_{\mathcal{O}_X} \mathcal{L}$,
$t \mapsto t \otimes s$ which we call multiplication by $s$.
We usually denote it $t \mapsto st$.

\begin{lemma}
\label{lemma-section-maps-back-into}
Let $X$ be a scheme.
Let $\mathcal{L}$ be an invertible $\mathcal{O}_X$-module.
Let $s \in \Gamma(X, \mathcal{L})$ be a section.
Let $\mathcal{F}' \subset \mathcal{F}$ be quasi-coherent
$\mathcal{O}_X$-modules. Assume that
\begin{enumerate}
\item $X$ is quasi-compact,
\item $\mathcal{F}$ is of finite type, and
\item $\mathcal{F}'|_{X_s} = \mathcal{F}|_{X_s}$.
\end{enumerate}
Then there exists an $n \geq 0$ such that
multiplication by $s^n$ on $\mathcal{F}$ factors
through $\mathcal{F}'$.
\end{lemma}

\begin{proof}
In other words we claim that
$s^n\mathcal{F} \subset
\mathcal{F}' \otimes_{\mathcal{O}_X} \mathcal{L}^{\otimes n}$
for some $n \geq 0$.
If this is true for $n_0$ then it is true for all $n \geq n_0$.
Hence it suffices to show there is a finite open covering such that
the result holds for each of the members of this open covering.
Since $X$ is quasi-compact we may therefore assume that $X$ is
affine and that $\mathcal{L} \cong \mathcal{O}_X$. Thus the lemma
translates into the following algebra problem (use Properties,
Lemma \ref{properties-lemma-finite-type-module}):
Let $A$ be a ring. Let $f \in A$. Let $M' \subset M$ be $A$-modules.
Assume $M$ is a finite $A$-module, and assume that
$(M')_f = M_f$. Then there exists an $n \geq 0$ such that
$f^n M \subset M'$. The proof of this is omitted.
\end{proof}

\medskip\noindent
Let $X$ be a scheme.
Let $\mathcal{L}$ be an invertible $\mathcal{O}_X$-module.
Let $s \in \Gamma(X, \mathcal{L})$ be a section.
Assume $X$ quasi-compact and quasi-separated.
The following lemma says roughly that the category of finitely
presented $\mathcal{O}_{X_s}$-modules is the category of 
finitely presented $\mathcal{O}_X$-modules where the map
multiplication by $s$ has been inverted.

\begin{lemma}
\label{lemma-section-maps-backwards}
Let $X$ be a scheme.
Let $\mathcal{L}$ be an invertible $\mathcal{O}_X$-module.
Let $s \in \Gamma(X, \mathcal{L})$ be a section.
Let $\mathcal{F}$, $\mathcal{F}'$ be quasi-coherent
$\mathcal{O}_X$-modules.
Let $\psi : \mathcal{F}|_{X_s} \to \mathcal{F}'|_{X_s}$ be a map
of $\mathcal{O}_{X_s}$-modules.
Assume that
\begin{enumerate}
\item $X$ is quasi-compact and quasi-separated, and
\item $\mathcal{F}$ is of finitely presented.
\end{enumerate}
Then there exists an $n \geq 0$ and a morphism
$\alpha : \mathcal{F} \to
\mathcal{F}' \otimes_{\mathcal{O}_X} \mathcal{L}^{\otimes n}$
whose restriction to $X_s$ equals $\psi$ via the identification
$\mathcal{L}^{\otimes n}|_{X_s} = \mathcal{O}_{X_s}$ coming from $s$.
Moreover, given a pair of solutions $(n, \alpha)$ and 
$(n', \alpha')$ there exists an $m \geq \max(n, n')$
such that $s^{m - n}\alpha = s^{m - n'}\alpha'$.
\end{lemma}

\begin{proof}
If the lemma holds for $n_0$ with map $\alpha_0$
then it holds for all $n \geq n_0$
simply by taking $\alpha = s^{n - n_0} \alpha_0$.
Choose a finite affine open covering $X = \bigcup U_i$ such
that $\mathcal{L}|_{U_i}$ is trivial. Choose finite affine
open coverings $U_i \cap U_{i'} = \bigcup U_{ii'j}$.
Suppose we can prove the lemma when $X$ is affine and
$\mathcal{L}$ is trivial. Then we can find $n_i \geq 0$
$\alpha_i : \mathcal{F}|_{U_i} \to
\mathcal{F}'|_{U_i} \otimes_{\mathcal{O}_{U_i}}
\mathcal{L}^{\otimes n_i}|_{U_i}$ satisfying the relation over $U_i$.
By the uniqueness assertion of the lemma, and the finiteness of
the number of affines $U_{ii'j}$ we can find a single large\
integer $m$ such that the maps $s^{m - n_i}\alpha_i$
and $s^{m - n_{i'}}\alpha_{i'}$
agree over $U_{ii'j}$ and hence over $U_i \cap U_{i'}$.
Thus the morphisms $s^{m - n_i}\alpha_i$ glue to give our global
map $\alpha$. Proof of the uniqueness statement is omitted.

\medskip\noindent
Assume $X$ affine and that $\mathcal{L} \cong \mathcal{O}_X$. Then the lemma
translates into the following algebra problem (use Properties,
Lemma \ref{properties-lemma-finite-presentation-module}):
Let $A$ be a ring. Let $f \in A$. Let $\psi : M_f \to (M')_f$ be
a map of $A_f$-modules. Assume $M$ is a finitely presented $A$-module.
Then there exists an $n \geq 0$ and an $A$-module map
$\alpha : M \to M'$ such that $\alpha \otimes 1_{A_f} = f^n\psi$.
Moreover, given any second solution $(n', \alpha')$
there exists an $m \geq \max(n, n')$
such that $f^{m - n}\alpha = f^{m - n'}\alpha'$.
The proof of this algebraic fact is omitted.
\end{proof}

\noindent
Cohomology is functorial. In particular, given a ringed space $X$,
an invertible $\mathcal{O}_X$-module $\mathcal{L}$, a section
$s \in \Gamma(X, \mathcal{L})$ we get maps
$$
H^p(X, \mathcal{F})
\longrightarrow
H^p(X, \mathcal{F} \otimes_{\mathcal{O}_X} \mathcal{L}), \quad
\xi \longmapsto s\xi
$$
induced by the map
$\mathcal{F} \to \mathcal{F} \otimes_{\mathcal{O}_X} \mathcal{L}$
which is multiplication by $s$.

\begin{lemma}
\label{lemma-section-affine-open-kills-classes}
Let $X$ be a scheme.
Let $\mathcal{L}$ be an invertible $\mathcal{O}_X$-module.
Let $s \in \Gamma(X, \mathcal{L})$ be a section.
Assume that
\begin{enumerate}
\item $X$ is quasi-compact and quasi-separated, and
\item $X_s$ is affine.
\end{enumerate}
Then for every quasi-coherent $\mathcal{O}_X$-module $\mathcal{F}$ and
every $p > 0$ and all $\xi \in H^p(X, \mathcal{F})$ there exists
an $n \geq 0$ such that $s^n\xi = 0$ in
$H^p(X, \mathcal{F} \otimes_{\mathcal{O}_X} \mathcal{L}^{\otimes n})$.
\end{lemma}

\begin{proof}
You can prove this lemma using a Mayer-Vietoris type argument and induction
on the number of affines needed to cover $X$
similar to the proof of Lemma \ref{lemma-quasi-coherence-higher-direct-images}.
This may be preferable to the proof that follows.

\medskip\noindent
Let $\mathcal{F}$ be a quasi-coherent $\mathcal{O}_X$-module.
Cohomology on $X$ commutes with directed colimits of sheaves of
$\mathcal{O}_X$-modules, see 
Cohomology, Lemma \ref{cohomology-lemma-quasi-separated-cohomology-colimit}.
By
Properties, Lemma \ref{properties-lemma-directed-colimit-finite-presentation}
we can write $\mathcal{F}$ as a directed colimit of
$\mathcal{O}_X$-submodules of finite presentation.
Hence every $\xi \in H^p(X, \mathcal{F})$ is the image of
$\xi' \in H^p(X, \mathcal{F}')$ for some
$\mathcal{O}_X$-submodule of finite presentation.
Thus we may replace $\mathcal{F}$
by $\mathcal{F}'$ and assume $\mathcal{F}$ is of finite presentation.

\medskip\noindent
Let $j : X_s \to X$ be the inclusion morphism.
Morphisms, Lemma \ref{morphisms-lemma-affine-s-open}
says that $j$ is an affine morphism.
Hence $R^qj_*(j^*\mathcal{F}) = 0$ for all $q > 0$,
see Lemma \ref{lemma-relative-affine-vanishing}.
Since also $H^p(X_s, j^*\mathcal{F}) = 0$ by
Lemma \ref{lemma-quasi-coherent-affine-cohomology-zero},
we conclude that $H^p(X, j_*j^*\mathcal{F}) = 0$ for all $p > 0$
for example by the Leray spectral sequence (
Cohomology, Lemma \ref{cohomology-lemma-Leray}).
Write
$$
j_*j^*\mathcal{F} = \text{colim}_{\lambda \in \Lambda}\ \mathcal{F}_\lambda
$$
as a directed colimit of $\mathcal{O}_X$-modules
$\mathcal{F}_\lambda$ of finite presentation
(Properties, Lemma \ref{properties-lemma-directed-colimit-finite-presentation}
again). By
Modules, Lemma \ref{modules-lemma-finite-presentation-quasi-compact-colimit}
there exists a $\lambda \in \Lambda$ such that
$\mathcal{F} \to j_*j^*\mathcal{F}$ factors through $\mathcal{F}_\lambda$.
After shrinking $\Lambda$ we may assume that we have a compatible
collection of morphisms
$\chi_\lambda : \mathcal{F} \to \mathcal{F}_\lambda$
for all $\lambda \in \Lambda$
which when taking the colimit gives the canonical map
$\mathcal{F} \to j_*j^*\mathcal{F}$.

\medskip\noindent
With these preparations the proof goes as follows.
Take $\xi \in H^p(X, \mathcal{F})$ for some $p > 0$.
It maps to zero in $H^p(X, j_*j^*\mathcal{F})$ because we saw above this
group is zero. By
Cohomology, Lemma \ref{cohomology-lemma-quasi-separated-cohomology-colimit}
again it follows that $\xi$ maps to zero in $H^p(X, \mathcal{F}_\lambda)$
via the map $\chi_\lambda$ for
some $\lambda$. Note that since $\mathcal{F} \to j_*j^*\mathcal{F}$
is an isomorphism over $X_s$ we see that there is an $\mathcal{O}_{X_s}$-module
map $\psi : \mathcal{F}_\lambda|_{X_s} \to \mathcal{F}|_{X_s}$ which
is a left inverse to $\chi_\lambda : \mathcal{F} \to \mathcal{F}_\lambda$.
By Lemma \ref{lemma-section-maps-backwards}
there exists an $n$ and a map
$\alpha : \mathcal{F}_\lambda \to
\mathcal{F} \otimes_{\mathcal{O}_X} \mathcal{L}^{\otimes n}$
such that $\alpha$ restricts to $\psi$ on $X_s$ (via
$\mathcal{L}^{\otimes n}|_{X_s} \cong \mathcal{O}_{X_s}$).
By the uniqueness part of Lemma \ref{lemma-section-maps-backwards}
applied to $\alpha \circ \chi_\lambda$ which restricts to
multiplication by $s^n$ on $X_s$
we may assume (after increasing $n$) that the composition
$$
\xymatrix{
\mathcal{F}
\ar[r]_-{\chi_\lambda} &
\mathcal{F}_\lambda
\ar[r]_-{\alpha} &
\mathcal{F} \otimes_{\mathcal{O}_X} \mathcal{L}^{\otimes n}
}
$$
is equal to multiplication by $s^n$ on $\mathcal{F}$.
Hence we see that $s^n\xi = 0$.
\end{proof}











\section{Cohomology of projective space}
\label{section-cohomology-projective-space}

\noindent
In this section we compute the cohomology of the twists of the
structure sheaf on $\mathbf{P}^n_S$ over a scheme $S$.
Recall that $\mathbf{P}^n_S$ was defined as the fibre product
$
\mathbf{P}^n_S = S \times_{\text{Spec}(\mathbf{Z})} \mathbf{P}^n_{\mathbf{Z}}
$
in Constructions, Definition \ref{constructions-definition-projective-space}.
It was shown to be equal to
$$
\mathbf{P}^n_S = \underline{\text{Proj}}_S(\mathcal{O}_S[T_0, \ldots, T_n])
$$
in Constructions, Lemma \ref{constructions-lemma-projective-space-bundle}.
In particular, projective space is a particular case of a projective bundle.
If $S = \text{Spec}(R)$ is affine then we have
$$
\mathbf{P}^n_S = \mathbf{P}^n_R = \text{Proj}(R[T_0, \ldots, T_n]).
$$
All these identifications are compatible and compatible with the constructions
of the twisted structure sheaves $\mathcal{O}_{\mathbf{P}^n_S}(d)$.

\medskip\noindent
Before we state the result we need some notation.
Let $R$ be a ring.
Recall that $R[T_0, \ldots, T_n]$ is a graded
$R$-algebra where each $T_i$ is homogenous of degree $1$.
Denote $(R[T_0, \ldots, T_n])_d$ the degree $d$ summand.
It is a finite free $R$-module of rank $\binom{n + d}{d}$
when $d \geq 0$ and zero else.
It has a basis consisting of monomials $T_0^{e_0} \ldots T_n^{e_n}$
with $\sum e_i = d$. We will also use the following notation:
$R[\frac{1}{T_0}, \ldots, \frac{1}{T_n}]$ denotes the $\mathbf{Z}$-graded
ring with $\frac{1}{T_i}$ in degree $-1$. In particular the
$\mathbf{Z}$-graded $R[\frac{1}{T_0}, \ldots, \frac{1}{T_n}]$ module
$$
\frac{1}{T_0 \ldots T_n} R[\frac{1}{T_0}, \ldots, \frac{1}{T_n}]
$$
which shows up in the statement below is zero in degrees
$\geq -n$, is free on the generator $\frac{1}{T_0 \ldots T_n}$
in degree $-n - 1$ and is free of rank $(-1)^n\binom{n + d}{d}$ for
$d \leq -n - 1$.

\begin{lemma}
\label{lemma-cohomology-projective-space-over-ring}
Let $R$ be a ring.
Let $n \geq 0$ be an integer.
We have
$$
H^q(\mathbf{P}^n, \mathcal{O}_{\mathbf{P}^n_R}(d)) =
\left\{
\begin{matrix}
(R[T_0, \ldots, T_n])_d & \text{if} & q = 0 \\
0 & \text{if} & q \not = 0, n \\
\left(\frac{1}{T_0 \ldots T_n} R[\frac{1}{T_0}, \ldots, \frac{1}{T_n}]\right)_d
& \text{if} & q = n
\end{matrix}
\right.
$$
as $R$-modules.
\end{lemma}

\begin{proof}
We will use the standard affine open convering
$$
\mathcal{U} : \mathbf{P}^n_R = \bigcup\nolimits_{i = 0}^n D_{+}(T_i)
$$
to compute the cohomology using the Cech complex.
This is permissible by Lemma \ref{lemma-cech-cohomology-quasi-coherent}
since any intersection of finitely many affine $D_{+}(T_i)$ is also a
standard affine open (see
Constructions, Section \ref{constructions-section-proj}).
In fact, we can use the alternating or ordered Cech complex according to
Cohomology, Lemmas \ref{cohomology-lemma-ordered-alternating} and
\ref{cohomology-lemma-alternating-usual}.

\medskip\noindent
The ordering we will use on $\{0, \ldots, n\}$ is the usual one.
Hence the complex we are looking at has terms
$$
\check{\mathcal{C}}_{ord}^p(\mathcal{U}, \mathcal{O}_{\mathbf{P}_R}(d))
=
\bigoplus\nolimits_{i_0 < \ldots < i_p}
(R[T_0, \ldots, T_n, \frac{1}{T_{i_0} \ldots T_{i_p}}])_d
$$
Moreover, the maps are given by the usual formula
$$
d(s)_{i_0 \ldots i_{p + 1}} =
\sum\nolimits_{j = 0}^{p + 1} (-1)^j s_{i_0 \ldots \hat i_j \ldots i_{p + 1}}
$$
see Cohomology, Section \ref{cohomology-section-alternating-cech}.
Note that each term of this complex has a natural
$\mathbf{Z}^{n + 1}$-grading. Namely, we get this by declaring a monomial
$T_0^{e_0} \ldots T_n^{e_n}$ to be homogeneous with weight
$(e_0, \ldots, e_n) \in \mathbf{Z}^{n + 1}$. It is clear that the differential
given above respects the grading. In a formula we have
$$
\check{\mathcal{C}}_{ord}^\bullet(\mathcal{U}, \mathcal{O}_{\mathbf{P}_R}(d))
=
\bigoplus\nolimits_{\vec{e} \in \mathbf{Z}^{n + 1}}
\check{\mathcal{C}}^\bullet(\vec{e})
$$
where not all summand on the right hand side occur (see below).
Hence in order to compute the cohomology
modules of the complex it suffices to compute the cohomology of the graded
pieces and take the direct sum at the end.

\medskip\noindent
Fix $\vec{e} = (e_0, \ldots, e_n) \in \mathbf{Z}^{n + 1}$. In order for this
weight to occur in the complex above we need to assume
$e_0 + \ldots + e_n = d$ (if not then it occurs for a different twist of
the structure sheaf of course). Assuming this set
$$
NEG(\vec{e}) = \{i \in \{0, \ldots, n\} \mid e_i < 0\}.
$$
With this notation the weight $\vec{e}$ summand
$\check{\mathcal{C}}^\bullet(\vec{e})$ of the Cech complex above has
the following terms
$$
\check{\mathcal{C}}^p(\vec{e})
=
\bigoplus\nolimits_{i_0 < \ldots < i_p,
\ NEG(\vec{e}) \subset \{i_0, \ldots, i_p\}}
R \cdot T_0^{e_0} \ldots T_n^{e_n}
$$
In other words, the terms corresponding to $i_0 < \ldots < i_p$ such
that $NEG(\vec{e})$ is not contained in $\{i_0 \ldots i_p\}$ are zero.
The differential of the complex $\check{\mathcal{C}}^\bullet(\vec{e})$
is still given by the exact same formula as above.

\medskip\noindent
Suppose that $NEG(\vec{e}) = \{0, \ldots, n\}$, i.e., that all
exponents $e_i$ are negative.
In this case the complex $\check{\mathcal{C}}^\bullet(\vec{e})$ has
only one term, namely $\check{\mathcal{C}}^n(\vec{e}) = 
R \cdot \frac{1}{T^{-e_0} \ldots T^{-e_n}}$. Hence in this
case
$$
H^q(\check{\mathcal{C}}^\bullet(\vec{e})) =
\left\{
\begin{matrix}
R \cdot \frac{1}{T^{-e_0} \ldots T^{-e_n}} & \text{if} & q = n \\
0 & \text{if} & \text{else}
\end{matrix}
\right.
$$
The direct sum of all of these terms clearly gives the value
$$
\left(\frac{1}{T_0 \ldots T_n} R[\frac{1}{T_0}, \ldots, \frac{1}{T_n}]\right)_d
$$
in degree $n$ as stated in the lemma. Moreover these terms do not contribute
to cohomology in other degrees (also in accordance with the statement of the
lemma).

\medskip\noindent
Assume $NEG(\vec{e}) = \emptyset$. In this case the complex
$\check{\mathcal{C}}^\bullet(\vec{e})$ has a summand $R$ corresponding
to all $i_0 < \ldots < i_p$.
Let us compare the complex $\check{\mathcal{C}}^\bullet(\vec{e})$
to another complex. Namely, consider the affine open open covering
$$
\mathcal{V} : \text{Spec}(R) = \bigcup\nolimits_{i \in \{0, \ldots, n\}} V_i
$$
where $V_i = \text{Spec}(R)$ for all $i$. Consider the alternating
Cech complex
$$
\check{\mathcal{C}}_{ord}^\bullet(\mathcal{V}, \mathcal{O}_{\text{Spec}(R)})
$$
By the same reasoning as above this computes the cohomology of the
structure sheaf on $\text{Spec}(R)$. Hence we see that
$H^p(
\check{\mathcal{C}}_{ord}^\bullet(\mathcal{V}, \mathcal{O}_{\text{Spec}(R)})
) = R$ if $p = 0$ and is $0$ whenever $p > 0$.
For these facts, see
Lemma \ref{lemma-cech-cohomology-quasi-coherent-trivial} and its proof.
Note that also
$\check{\mathcal{C}}_{ord}^\bullet(\mathcal{V}, \mathcal{O}_{\text{Spec}(R)})$
has a summand $R$ for every $i_0 < \ldots < i_p$ and has exactly the same
differential as $\check{\mathcal{C}}^\bullet(\vec{e})$. In other words
these complexes are isomorphic complexes and hence have the same cohomology.
We conclude that
$$
H^q(\check{\mathcal{C}}^\bullet(\vec{e})) =
\left\{
\begin{matrix}
R \cdot T^{e_0} \ldots T^{e_n} & \text{if} & q = 0 \\
0 & \text{if} & \text{else}
\end{matrix}
\right.
$$
in the case that $NEG(\vec{e}) = \emptyset$.
The direct sum of all of these terms clearly gives the value
$$
(R[T_0, \ldots, T_n])_d
$$
in degree $0$ as stated in the lemma. Moreover these terms do not contribute
to cohomology in other degrees (also in accordance with the statement of the
lemma).

\medskip\noindent
To finish the proof of the lemma we have to show that the complexes
$\check{\mathcal{C}}^\bullet(\vec{e})$ are acyclic when
$NEG(\vec{e})$ is neither empty nor equal to $\{0, \ldots, n\}$.
Pick an index $i_{\text{fix}} \not \in NEG(\vec{e})$ (such an index exists).
Consider the map
$$
h :
\check{\mathcal{C}}^{p + 1}(\vec{e})
\to
\check{\mathcal{C}}^p(\vec{e})
$$
given by the rule
$$
h(s)_{i_0 \ldots i_p} = s_{i_{\text{fix}} i_0 \ldots i_p}
$$
(compare with the proof of
Lemma \ref{lemma-cech-cohomology-quasi-coherent-trivial}).
It is clear that this is well defined since
$$
NEG(\vec{e}) \subset \{i_0, \ldots, i_p\}
\Leftrightarrow
NEG(\vec{e}) \subset \{i_{\text{fix}}, i_0, \ldots, i_p\}
$$
Also $\check{\mathcal{C}}^0(\vec{e}) = 0$ so that this
formula does work for all $p$ including $p = - 1$.
The exact same (combinatorial) computation as in the
proof of Lemma \ref{lemma-cech-cohomology-quasi-coherent-trivial}
shows that
$$
(hd + dh)(s)_{i_0 \ldots i_p}
=
s_{i_0 \ldots i_p}
$$
Hence we see that the identity map of the complex
$\check{\mathcal{C}}^\bullet(\vec{e})$ is homotopic to zero
which implies that it is acyclic.
\end{proof}

\noindent
In the following lemma we are going to use the pairing of free
$R$-modules
$$
R[T_0, \ldots, T_n]
\times
\frac{1}{T_0 \ldots T_n} R[\frac{1}{T_0}, \ldots, \frac{1}{T_n}]
\longrightarrow
R
$$
which is defined by the rule
$$
(f, g)
\longmapsto
\text{coefficient of }
\frac{1}{T_0 \ldots T_n}
\text{ in }fg.
$$
In other words, the basis element $T_0^{e_0} \ldots T_n^{e_n}$ paired
with the basis element $T_0^{d_0} \ldots T_n^{d_n}$ gives $1$ if and only
if $e_i + d_i = -1$ for all $i$, and pairs to zero in all other cases.
Using this pairing we get an identification
$$
\left(\frac{1}{T_0 \ldots T_n} R[\frac{1}{T_0}, \ldots, \frac{1}{T_n}]\right)_d
=
\text{Hom}_R((R[T_0, \ldots, T_n])_{-n - 1 - d}, R)
$$
Thus we can reformulate the result of
Lemma \ref{lemma-cohomology-projective-space-over-ring} as saying that
\begin{equation}
\label{equation-identify}
H^q(\mathbf{P}^n, \mathcal{O}_{\mathbf{P}^n_R}(d)) =
\left\{
\begin{matrix}
(R[T_0, \ldots, T_n])_d & \text{if} & q = 0 \\
0 & \text{if} & q \not = 0, n \\
\text{Hom}_R((R[T_0, \ldots, T_n])_{-n - 1 - d}, R)
& \text{if} & q = n
\end{matrix}
\right.
\end{equation}

\begin{lemma}
\label{lemma-identify-functorially}
The identifications of Equation (\ref{equation-identify}) are
compatible with base change w.r.t.\ ring maps $R \to R'$.
Moreover, for any $f \in R[T_0, \ldots, T_n]$ homogeneous
of degree $m$ the map multiplication by $f$
$$
\mathcal{O}_{\mathbf{P}^n_R}(d)
\longrightarrow
\mathcal{O}_{\mathbf{P}^n_R}(d + m)
$$
induces the map on the cohomology group via the identifications
of Equation (\ref{equation-identify}) which is multiplication by
$f$ for $H^0$ and the contragredient of multiplication by $f$
$$
(R[T_0, \ldots, T_n])_{-n - 1 - (d + m)}
\longrightarrow
(R[T_0, \ldots, T_n])_{-n - 1 - d}
$$
on $H^n$.
\end{lemma}

\begin{proof}
Suppose that $R \to R'$ is a ring map.
Let $\mathcal{U}$ be the standard affine open convering of $\mathbf{P}^n_R$,
and let $\mathcal{U}'$ be the standard affine open convering of
$\mathbf{P}^n_{R'}$. Note that $\mathcal{U}'$ is the pullback of the covering
$\mathcal{U}$ under the canonical morphism
$\mathbf{P}^n_{R'} \to \mathbf{P}^n_R$. Hence there
is a map of Cech complexes
$$
\gamma :
\check{\mathcal{C}}_{ord}^\bullet(\mathcal{U},
\mathcal{O}_{\mathbf{P}_R}(d))
\longrightarrow
\check{\mathcal{C}}_{ord}^\bullet(\mathcal{U}',
\mathcal{O}_{\mathbf{P}_{R'}}(d))
$$
which is compatible with the map on cohomology by
Cohomology, Lemma \ref{cohomology-lemma-functoriality-cech}.
It is clear from the computations in the proof of
Lemma \ref{lemma-cohomology-projective-space-over-ring}
that this map of Cech complexes is compatible with the identifications
of the cohomology groups in question. (Namely the basis elements for
the Cech complex over $R$ simply map to the corresponding basis elements
for the Cech complex over $R'$.) Whence the first statement of the lemma.

\medskip\noindent
Now fix the ring $R$ and consider two homogeneous polynomials
$f, g \in R[T_0, \ldots, T_n]$ both of the same degree $m$.
Since cohomology is an additive functor, it is clear that the
map induced by multiplication by $f + g$ is the same as the sum
of the maps induced by multiplication by $f$ and the map induced
by multiplication by $g$. Moreover, since cohomology is a functor
a similar result holds for multiplication by a product $fg$ where
$f, g$ are both homogeneous (but not necessarily of the same degree).
Hence to verify the second statement of the lemma it suffices to
prove this when $f = x \in R$ or when $f = T_i$.
In the case of multiplication by an element $x \in R$ the result
follows since every cohomology groups or complex in sight has the
structuce or an $R$-module or complex of $R$-modules.
Finally, we consider the case of multiplication by $T_i$
as a $\mathcal{O}_{\mathbf{P}^n_R}$-linear map 
$$
\mathcal{O}_{\mathbf{P}^n_R}(d)
\longrightarrow
\mathcal{O}_{\mathbf{P}^n_R}(d + 1)
$$
The statement on $H^0$ is clear. For the statement on $H^n$ 
consider multiplication by $T_i$ as a map on Cech complexes
$$
\check{\mathcal{C}}_{ord}^\bullet(\mathcal{U},
\mathcal{O}_{\mathbf{P}_R}(d))
\longrightarrow
\check{\mathcal{C}}_{ord}^\bullet(\mathcal{U},
\mathcal{O}_{\mathbf{P}_{R}}(d + 1))
$$
We are going to use the notation introduced in the proof of
Lemma \ref{lemma-cohomology-projective-space-over-ring}.
We consider the effect of multiplication by $T_i$
in terms of the decompositions
$$
\check{\mathcal{C}}_{ord}^\bullet(\mathcal{U}, \mathcal{O}_{\mathbf{P}_R}(d))
=
\bigoplus\nolimits_{\vec{e} \in \mathbf{Z}^{n + 1},\ \sum e_i = d}
\check{\mathcal{C}}^\bullet(\vec{e})
$$
and
$$
\check{\mathcal{C}}_{ord}^\bullet(\mathcal{U},
\mathcal{O}_{\mathbf{P}_R}(d + 1))
=
\bigoplus\nolimits_{\vec{e} \in \mathbf{Z}^{n + 1},\ \sum e_i = d + 1}
\check{\mathcal{C}}^\bullet(\vec{e})
$$
It is clear that it maps the subcomplex
$\check{\mathcal{C}}^\bullet(\vec{e})$ to the subcomplex
$\check{\mathcal{C}}^\bullet(\vec{e} + \vec{b}_i)$ where
$\vec{b}_i = (0, \ldots, 0, 1, 0, \ldots, 0))$ the $i$th basis vector.
In other words, it maps the summand of $H^n$ corresponding to
$\vec{e}$ with $e_i < 0$ and $\sum e_i = d$
to the summand of $H^n$ corresponding to
$\vec{e} + \vec{b}_i$ (which is zero if $e_i + b_i \geq 0$).
It is easy to see that this corresponds exactly to the action
of the contragredient of multiplication by $T_i$ as a map
$$
(R[T_0, \ldots, T_n])_{-n - 1 - (d + 1)}
\longrightarrow
(R[T_0, \ldots, T_n])_{-n - 1 - d}
$$
This proves the lemma.
\end{proof}

\noindent
Before we state the relative version we need some notation.
Namely, recall that $\mathcal{O}_S[T_0, \ldots, T_n]$ is a graded
$\mathcal{O}_S$-module where each $T_i$ is homogenous of degree $1$.
Denote $(\mathcal{O}_S[T_0, \ldots, T_n])_d$ the degree $d$ summand.
It is a finite locally free sheaf of rank $\binom{n + d}{d}$ on $S$.

\begin{lemma}
\label{lemma-cohomology-projective-space-over-base}
Let $S$ be a scheme.
Let $n \geq 0$ be an integer.
Consider the structure morphism
$$
f : \mathbf{P}^n_S \longrightarrow S.
$$
We have
$$
R^qf_*(\mathcal{O}_{\mathbf{P}^n_S}(d)) =
\left\{
\begin{matrix}
(\mathcal{O}_S[T_0, \ldots, T_n])_d & \text{if} & q = 0 \\
0 & \text{if} & q \not = 0, n \\
\textit{Hom}_{\mathcal{O}_S}(
(\mathcal{O}_S[T_0, \ldots, T_n])_{- n - 1 - d}, \mathcal{O}_S)
& \text{if} & q = n
\end{matrix}
\right.
$$
\end{lemma}

\begin{proof}
Omitted. Hint: This follows since the identifications in
(\ref{equation-identify}) are compatible with affine base change
by Lemma \ref{lemma-identify-functorially}.
\end{proof}

\noindent
Next we state the version for projective bundles associated to finite locally
free sheaves. Let $S$ be a scheme. Let $\mathcal{E}$ be a finite locally
free $\mathcal{O}_S$-module of constant rank $n + 1$, see
Modules, Section \ref{modules-section-locally-free}.
In this case we think of $\text{Sym}(\mathcal{E})$ as a graded
$\mathcal{O}_S$-module where $\mathcal{E}$ is the graded part of degree $1$.
And $\text{Sym}^d(\mathcal{E})$ is the degree $d$ summand.
It is a finite locally free sheaf of rank $\binom{n + d}{d}$ on $S$.
Recall that our normalization is that
$$
\pi :
\mathbf{P}(\mathcal{E})
=
\underline{\text{Proj}}_S(\text{Sym}(\mathcal{E}))
\longrightarrow
S
$$
and that there are natural maps
$\text{Sym}^d(\mathcal{E}) \to \pi_*\mathcal{O}_{\mathbf{P}(\mathcal{E})}(d)$.

\begin{lemma}
\label{lemma-cohomology-projective-bundle}
Let $S$ be a scheme. Let $n \geq 1$.
Let $\mathcal{E}$ be a finite locally
free $\mathcal{O}_S$-module of constant rank $n + 1$.
Consider the structure morphism
$$
\pi : \mathbf{P}(\mathcal{E}) \longrightarrow S.
$$
We have
$$
R^q\pi_*(\mathcal{O}_{\mathbf{P}(\mathcal{E})}(d)) =
\left\{
\begin{matrix}
\text{Sym}^d(\mathcal{E}) & \text{if} & q = 0 \\
0 & \text{if} & q \not = 0, n \\
\textit{Hom}_{\mathcal{O}_S}(
\text{Sym}^{- n - 1 - d}(\mathcal{E})
\otimes_{\mathcal{O}_S}
\wedge^{n + 1}\mathcal{E},
\mathcal{O}_S)
& \text{if} & q = n
\end{matrix}
\right.
$$
These identifications are compatible with base change and
isomorphism between locally free sheaves.
\end{lemma}

\begin{proof}
Consider the canonical map
$$
\pi^*\mathcal{E} \longrightarrow \mathcal{O}_{\mathbf{P}(\mathcal{E})}(1)
$$
and twist down by $1$ to get
$$
\pi^*(\mathcal{E})(-1) \longrightarrow \mathcal{O}_{\mathbf{P}(\mathcal{E})}
$$
This is a surjective map from a locally free rank $n + 1$ sheaf onto
the structure sheaf. Hence the corresponding Koszul complex is
exact (insert future reference here). In other words there is an
exact complex
$$
0 \to
\pi^*(\wedge^{n + 1}\mathcal{E})(-n - 1) \to
\ldots \to
\pi^*(\wedge^i\mathcal{E})(-i) \to
\ldots \to
\mathcal{E}(-1) \to
\mathcal{O}_{\mathbf{P}(\mathcal{E})} \to 0
$$
We will think of the term $\pi^*(\wedge^i\mathcal{E})(-i)$ as being
in degree $-i$.
We are going to compute the higher direct images
of this acyclic complex using the first spectral sequence of
Homology, Lemma \ref{homology-lemma-two-ss-complex-functor}.
Namely, we see that there is a spectral sequence with terms
$$
E_2^{p, q} = H^p(L^{\bullet, q})
\quad
\text{with}
\quad
L^{-i, q} = R^q\pi_*\left(\pi^*(\wedge^i\mathcal{E})(-i)\right)
$$
converging to zero!
By the projection formula (
Cohomology, Lemma \ref{cohomology-lemma-projection-formula})
we have
$$
L^{-i, q} = \wedge^i\mathcal{E} \otimes_{\mathcal{O}_S}
R^q\pi_*\left(\mathcal{O}_{\mathbf{P}(\mathcal{E})}(-i)\right).
$$
Note that locally on $S$ the sheaf $\mathcal{E}$ is trivial,
i.e., isomorphic to $\mathcal{O}_S^{\oplus n + 1}$, hence locally on
$S$ the morphism $\mathbf{P}(\mathcal{E}) \to S$ can be identified
with $\mathbf{P}^n_S \to S$. Hence
locally on $S$ we can use the result of Lemmas
\ref{lemma-cohomology-projective-space-over-ring},
\ref{lemma-identify-functorially}, or
\ref{lemma-cohomology-projective-space-over-base}.
It follows that $L^{-i, q} = 0$ unless $i = q = 0$
or $i = n + 1$ and $q = n$. This in turn implies that
$E_2^{p, q} = 0$ unless $(p, q) = (0, 0)$ or
$(p, q) = (-n - 1, n)$, and
\begin{align*}
E_2^{0, 0} & = \pi_*\mathcal{O}_{\mathbf{P}(\mathcal{E})} = \mathcal{O}_S
\\
E_2^{-n - 1, n} & = \wedge^{n + 1}\mathcal{E} \otimes_{\mathcal{O}_S}
R^n\pi_*\left(\mathcal{O}_{\mathbf{P}(\mathcal{E})}(-n - 1)\right).
\end{align*}
Hence there can only be one nonzero
differential in the spectral sequence namely the map
$d_{n + 1}$ inducing a map
$$
d_{n + 1}^{0, 0} :
\mathcal{O}_S
\longrightarrow
\wedge^{n + 1}\mathcal{E} \otimes_{\mathcal{O}_S}
R^n\pi_*\left(\mathcal{O}_{\mathbf{P}(\mathcal{E})}(-n - 1)\right)
$$
which has to be an isomorphism (because the spectral sequence converges
to the $0$ sheaf). Since $\wedge^{n + 1}\mathcal{E}$ is an invertible
sheaf, this implies that
$R^n\pi_*\mathcal{O}_{\mathbf{P}(\mathcal{E})}(-n - 1)$ is invertible
as well and canonically isomorphic to the inverse of
$\wedge^{n + 1}\mathcal{E}$. In other words we have proved the case
$d = - n - 1$ of the lemma.

\medskip\noindent
Working locally on $S$ we see immediately from the computation of
cohomology in Lemmas \ref{lemma-cohomology-projective-space-over-ring},
\ref{lemma-identify-functorially}, or
\ref{lemma-cohomology-projective-space-over-base} the statements on
vanishing of the lemma. Moreover the result on $R^0\pi_*$ is clear
as well, since there are canonical maps
$\text{Sym}^d(\mathcal{E}) \to \pi_* \mathcal{O}_{\mathbf{P}(\mathcal{E})}(d)$
for all $d$. It remains to show that the description of
$R^n\pi_*\mathcal{O}_{\mathbf{P}(\mathcal{E})}(d)$ is correct
for $d < -n - 1$. In other to do this we consider the map
$$
\pi^*(\text{Sym}^{-d + n + 1}(\mathcal{E}))
\otimes_{\mathcal{O}_{\mathbf{P}(\mathcal{E})}}
\mathcal{O}_{\mathbf{P}(\mathcal{E})}(d)
\longrightarrow
\mathcal{O}_{\mathbf{P}(\mathcal{E})}(-n - 1)
$$
Applying $R^n\pi_*$ and the projection formula (see above) we get a map
$$
\text{Sym}^{-d + n + 1}(\mathcal{E})
\otimes_{\mathcal{O}_S}
R^n\pi_*(\mathcal{O}_{\mathbf{P}(\mathcal{E})}(d))
\longrightarrow
R^n\pi_*\mathcal{O}_{\mathbf{P}(\mathcal{E})}(-n - 1) =
(\wedge^{n + 1}\mathcal{E})^{\otimes -1}
$$
(the last equality we have shown above).
Again by the local calculations of Lemmas
\ref{lemma-cohomology-projective-space-over-ring},
\ref{lemma-identify-functorially}, or
\ref{lemma-cohomology-projective-space-over-base}
it follows that this map induces a perfect pairing between
$R^n\pi_*(\mathcal{O}_{\mathbf{P}(\mathcal{E})}(d))$ and
$\text{Sym}^{-d + n + 1}(\mathcal{E}) \otimes \wedge^{n + 1}(\mathcal{E})$
as desired.
\end{proof}















\section{Coherent sheaves on locally Noetherian schemes}
\label{section-coherent-sheaves}

\noindent
Allthough it is possible to consider coherent sheaves on non-Noetherian
schemes we will always assume the base scheme is locally Noetherian when
we consider coherent sheaves. Here is a characterization of coherent
sheaves on locally Noetherian schemes.

\begin{lemma}
\label{lemma-coherent-Noetherian}
Let $X$ be a locally Noetherian scheme.
Let $\mathcal{F}$ be an $\mathcal{O}_X$-module.
The following are equivalent
\begin{enumerate}
\item $\mathcal{F}$ is coherent,
\item $\mathcal{F}$ is a quasi-coherent, finite type $\mathcal{O}_X$-module,
\item $\mathcal{F}$ is a finitely presented $\mathcal{O}_X$-module,
\item for any affine open $\text{Spec}(A) = U \subset X$ we have
$\mathcal{F}|_U = \widetilde M$ with $M$ a finite $A$-module, and
\item there exists an affine open covering $X = \bigcup U_i$,
$U_i = \text{Spec}(A_i)$ such that each
$\mathcal{F}|_{U_i} = \widetilde M_i$ with $M_i$ a finite $A_i$-module.
\end{enumerate}
In particular $\mathcal{O}_X$ is coherent, any invertible
$\mathcal{O}_X$-module is coherent, and more generally any
finite locally free $\mathcal{O}_X$-module is invertible.
\end{lemma}

\begin{proof}
The implications (1) $\Rightarrow$ (2) and (1) $\Rightarrow$ (3) hold
in general, see
Modules, Lemma \ref{modules-lemma-coherent-finite-presentation}.
If $\mathcal{F}$ is finitely presented then $\mathcal{F}$ is
quasi-coherent, see
Modules, Lemma \ref{modules-lemma-finite-presentation-quasi-coherent}.
Hence also (3) $\Rightarrow$ (2).

\medskip\noindent
Assume $\mathcal{F}$ is a quasi-coherent, finite type $\mathcal{O}_X$-module.
By
Properties, Lemma \ref{properties-lemma-finite-type-module}
we see that on any affine open
$\text{Spec}(A) = U \subset X$ we have $\mathcal{F}|_U = \widetilde M$
with $M$ a finite $A$-module. Since $A$ is Noetherian we see that
$M$ has a finite resolution
$$
A^{\oplus m} \to A^{\oplus n} \to M \to 0.
$$
Hence $\mathcal{F}$ is of finite presentation by
Properties, Lemma \ref{properties-lemma-finite-presentation-module}.
In other words (2) $\Rightarrow$ (3).

\medskip\noindent
By Modules, Lemma \ref{modules-lemma-coherent-structure-sheaf} it suffices
to show that $\mathcal{O}_X$ is coherent in order to show that (3)
implies (1). Thus we have to show: given any open $U \subset X$ and
any finite collection of sections $f_i \in \mathcal{O}_X(U)$,
$i = 1, \ldots, n$ the kernel of the map
$\bigoplus_{i = 1, \ldots, n} \mathcal{O}_U \to \mathcal{O}_U$
is of finite type. Since being of finite type is a local property
it suffices to check this in a neighbourhood of any $x \in U$.
Thus we may assume $U = \text{Spec}(A)$ is affine. In this case
$f_1, \ldots, f_n \in A$ are elements of $A$. Since $A$ is
Noetherian, see
Properties, Lemma \ref{properties-lemma-locally-Noetherian}
the kernel $K$ of the map $\bigoplus_{i = 1, \ldots, n} A \to A$
is a finite $A$-module. See for example
Algebra, Lemma \ref{algebra-lemma-Noetherian-basic}.
As the functor\ $\widetilde{ }$\ is exact, see
Schemes, Lemma \ref{schemes-lemma-spec-sheaves}
we get an exact sequence
$$
\widetilde K \to
\bigoplus\nolimits_{i = 1, \ldots, n} \mathcal{O}_U \to
\mathcal{O}_U
$$
and by
Properties, Lemma \ref{properties-lemma-finite-type-module}
again we see that $\widetilde K$ is of finite type. We conclude
that (1), (2) and (3) are all equivalent.

\medskip\noindent
It follows from
Properties, Lemma \ref{properties-lemma-finite-type-module}
that (2) implies (4). It is trivial that (4) implies (5).
The discussion in
Schemes, Section \ref{schemes-section-quasi-coherent}
show that (5) implies
that $\mathcal{F}$ is quasi-coherent and it is clear that (5)
implies that $\mathcal{F}$ is of finite type. Hence (5) implies
(2) and we win.
\end{proof}

\begin{lemma}
\label{lemma-coherent-abelian-Noetherian}
Let $X$ be a locally Noetherian scheme.
The category of coherent $\mathcal{O}_X$-modules is abelian.
More precisely, the kernel and cokernel of a map of coherent
$\mathcal{O}_X$-modules are coherent. Any extension
of coherent sheaves is coherent.
\end{lemma}

\begin{proof}
This is a restatement of
Modules, Lemma \ref{modules-lemma-coherent-abelian}
in a particular case.
\end{proof}

\noindent
The following lemma does not always hold for the category of coherent
$\mathcal{O}_X$-modules on a general ringed space $X$.

\begin{lemma}
\label{lemma-coherent-Noetherian-quasi-coherent-sub-quotient}
Let $X$ be a locally Noetherian scheme.
Let $\mathcal{F}$ be a coherent $\mathcal{O}_X$-module.
Any quasi-coherent submodule of $\mathcal{F}$ is coherent.
Any quasi-coherent quotient module of $\mathcal{F}$ is coherent.
\end{lemma}

\begin{proof}
We may assume that $X$ is affine, say $X = \text{Spec}(A)$.
Properties, Lemma \ref{properties-lemma-locally-Noetherian}
implies that $A$ is Noetherian. Lemma \ref{lemma-coherent-Noetherian}
turns this into algebra. The algebraic counter part of
the lemma is that a quotient, or a submodule of a finite $A$-module
is a finite $A$-module, see for example
Algebra, Lemma \ref{algebra-lemma-Noetherian-basic}.
\end{proof}

\begin{lemma}
\label{lemma-tensor-hom-coherent}
Let $X$ be a locally Noetherian scheme.
Let $\mathcal{F}$, $\mathcal{G}$ be coherent $\mathcal{O}_X$-modules.
The $\mathcal{O}_X$-modules $\mathcal{F} \otimes_{\mathcal{O}_X} \mathcal{G}$
and $\textit{Hom}_{\mathcal{O}_X}(\mathcal{F}, \mathcal{G})$ are
coherent.
\end{lemma}

\begin{proof}
It is shown in
Modules, Lemma \ref{modules-lemma-internal-hom-locally-kernel-direct-sum} that
$\textit{Hom}_{\mathcal{O}_X}(\mathcal{F}, \mathcal{G})$ is coherent.
The result for tensor products is
Modules, Lemma \ref{modules-lemma-tensor-product-permanence}
\end{proof}

\begin{lemma}
\label{lemma-local-isomorphism}
Let $X$ be a locally Noetherian scheme.
Let $\mathcal{F}$, $\mathcal{G}$ be coherent $\mathcal{O}_X$-modules.
Let $\varphi : \mathcal{G} \to \mathcal{F}$ be a homomorphism
of $\mathcal{O}_X$-modules. Let $x \in X$.
\begin{enumerate}
\item If $\mathcal{F}_x = 0$ then there exists an open neighbourhood
$U \subset X$ of $x$ such that $\mathcal{F}|_U = 0$.
\item If $\varphi_x : \mathcal{G}_x \to \mathcal{F}_x$ is injective,
then there exists an open neighbourhood $U \subset X$ of $x$ such that
$\varphi|_U$ is injective.
\item If $\varphi_x : \mathcal{G}_x \to \mathcal{F}_x$ is surjective,
then there exists an open neighbourhood $U \subset X$ of $x$ such that
$\varphi|_U$ is surjective.
\item If $\varphi_x : \mathcal{G}_x \to \mathcal{F}_x$ is bijective,
then there exists an open neighbourhood $U \subset X$ of $x$ such that
$\varphi|_U$ is an isomorphism.
\end{enumerate}
\end{lemma}

\begin{proof}
See Modules, Lemmas
\ref{modules-lemma-finite-type-surjective-on-stalk},
\ref{modules-lemma-finite-type-stalk-zero}, and
\ref{modules-lemma-finite-type-to-coherent-injective-on-stalk}.
\end{proof}

\begin{lemma}
\label{lemma-map-stalks-local-map}
Let $X$ be a locally Noetherian scheme.
Let $\mathcal{F}$, $\mathcal{G}$ be coherent $\mathcal{O}_X$-modules.
Let $x \in X$.
Suppose $\psi : \mathcal{G}_x \to \mathcal{F}_x$ is a map of
$\mathcal{O}_{X, x}$-modules.
Then there exists an open neighbourhood $U \subset X$ of $x$ and a map
$\varphi : \mathcal{G}|_U \to \mathcal{F}|_U$ such that
$\varphi_x = \psi$.
\end{lemma}

\begin{proof}
In view of Lemma \ref{lemma-coherent-Noetherian}
this is a reformulation of
Modules, Lemma \ref{modules-lemma-stalk-internal-hom}.
\end{proof}

\begin{lemma}
\label{lemma-coherent-support-closed}
Let $X$ be a locally Noetherian scheme.
Let $\mathcal{F}$ be a coherent $\mathcal{O}_X$-module.
Then $\text{Supp}(\mathcal{F})$ (see
Modules, Definition \ref{modules-definition-support}) is closed.
\end{lemma}

\begin{proof}
This holds more generally (Lemma \ref{lemma-coherent-Noetherian})
for any $\mathcal{O}_X$-module of finite type on any
ringed space, see
Modules, Lemma \ref{modules-lemma-support-finite-type-closed}.
\end{proof}

\begin{lemma}
\label{lemma-finite-pushforward-coherent}
Let $f : X \to Y$ be a morphism of schemes.
Assume $f$ is finite and $Y$ locally Noetherian.
Then $f_*\mathcal{F}$ is coherent if $\mathcal{F}$ is
coherent.
\end{lemma}

\begin{proof}
Note that the assumptions imply that also $X$ is locally Noetherian
(see Morphisms, Lemma \ref{morphisms-lemma-finite-type-noetherian})
and hence the statement makes sense.
Let $\text{Spec}(A) = V \subset Y$ be an affine open subset.
By Morphisms, Definition \ref{morphisms-definition-integral}
we see that $f^{-1}(V) = \text{Spec}(B)$ with $A \to B$ finite.
Lemma \ref{lemma-coherent-Noetherian}
turns the statement of the lemma into the following algebra
fact: If $M$ is a finite $B$-module, then $M$ is also finite
viewed as a $A$-module, see
Algebra, Lemma \ref{algebra-lemma-finite-module-over-finite-extension}.
\end{proof}

\noindent
In the situation of the lemma also the higher direct images are
coherent since they vanish, see Lemma \ref{lemma-relative-affine-vanishing}.
We will show that this is always the case for a proper morphism
between locally Noetherian schemes (insert future reference here).









\section{Coherent sheaves on Noetherian schemes}
\label{section-coherent-quasi-compact}

\noindent
In this section we mention some properties of coherent sheaves on
Noetherian schemes.

\begin{lemma}
\label{lemma-acc-coherent}
Let $X$ be a Noetherian scheme.
Let $\mathcal{F}$ be a coherent $\mathcal{O}_X$-module.
The ascending chain condition holds for quasi-coherent submodules
of $\mathcal{F}$. In other words, give any sequence
$$
\mathcal{F}_1 \subset \mathcal{F}_2 \subset \ldots \subset \mathcal{F}
$$
of quasi-coherent submodules, then
$\mathcal{F}_n = \mathcal{F}_{n + 1} = \ldots $ for some $n \geq 0$.
\end{lemma}

\begin{proof}
Choose a finite affine open covering.
On each member of the covering we get stabilization by
Algebra, Lemma \ref{algebra-lemma-Noetherian-basic}.
Hence the lemma follows.
\end{proof}

\begin{lemma}
\label{lemma-power-ideal-kills-sheaf}
Let $X$ be a Noetherian scheme.
Let $\mathcal{F}$ be a coherent sheaf on $X$.
Let $\mathcal{I} \subset \mathcal{O}_X$ be a quasi-coherent
sheaf of ideals corresponding to a closed subscheme $Z \subset X$.
Then there is some $n \geq 0$ such that $\mathcal{I}^n\mathcal{F} = 0$
if and only if $\text{Supp}(\mathcal{F}) \subset Z$ (set theoretically).
\end{lemma}

\begin{proof}
This follows immediately from 
Algebra, Lemma \ref{algebra-lemma-Noetherian-power-ideal-kills-module}
because $X$ has a finite covering by spectra of Noetherian rings.
\end{proof}

\begin{lemma}
\label{lemma-Artin-Rees}
(Artin-Rees.)
Let $X$ be a Noetherian scheme.
Let $\mathcal{F}$ be a coherent sheaf on $X$.
Let $\mathcal{G} \subset \mathcal{F}$ be a quasi-coherent subsheaf.
Let $\mathcal{I} \subset \mathcal{O}_X$ be a quasi-coherent sheaf of
ideals.
Then there exists a $c \geq 0$ such that for all $n \geq c$ we
have
$$
\mathcal{I}^{n - c}(\mathcal{I}^c\mathcal{F} \cap \mathcal{G})
=
\mathcal{I}^n\mathcal{F}
$$
\end{lemma}

\begin{proof}
This follows immediately from
Algebra, Lemma \ref{algebra-lemma-Artin-Rees}
because $X$ has a finite covering by spectra of Noetherian rings.
\end{proof}

\begin{lemma}
\label{lemma-homs-over-open}
Let $X$ be a Noetherian scheme.
Let $\mathcal{F}$, $\mathcal{G}$ be coherent $\mathcal{O}_X$-modules.
Let $\mathcal{I} \subset \mathcal{O}_X$ be a quasi-coherent sheaf of
ideals. Denote $Z \subset X$ the corresponding closed subscheme and
set $U = X \setminus Z$.
There is a canonical isomorphism
$$
\text{colim}_n\ \text{Hom}_{\mathcal{O}_X}(
\mathcal{I}^n\mathcal{G}, \mathcal{F})
\longrightarrow
\textit{Hom}_{\mathcal{O}_U}(\mathcal{G}|_U, \mathcal{F}|_U).
$$
In particular we have an isomorphism
$$
\text{colim}_n\ \text{Hom}_{\mathcal{O}_X}(
\mathcal{I}^n, \mathcal{F})
\longrightarrow
\Gamma(U, \mathcal{F}).
$$
\end{lemma}

\begin{proof}
We first prove the second equality.
Let $\mathcal{F}_n$ denote the quasi-coherent subsheaf of
$\mathcal{F}$ consisting of sections annihilated by $\mathcal{I}^n$,
see Properties, Lemma \ref{properties-lemma-sections-over-quasi-compact-open}.
Since $\mathcal{F}_1 \subset \mathcal{F}_2 \subset \ldots$ we see that
$\mathcal{F}_n = \mathcal{F}_{n + 1} = \ldots $ for some $n \geq 0$
by Lemma \ref{lemma-acc-coherent}. Set $\mathcal{H} = \mathcal{F}_n$
for this $n$. By Artin-Rees (Lemma \ref{lemma-Artin-Rees})
there exists an $c \geq 0$ such that
$\mathcal{I}^m\mathcal{F} \cap \mathcal{G}
\subset \mathcal{I}^{m - c}\mathcal{G}$. Picking $m = n + c$ we get
$\mathcal{I}^m\mathcal{F} \cap \mathcal{G} \subset \mathcal{I}^n\mathcal{G}
= 0$. Thus if we set $\mathcal{F}' = \mathcal{I}^m\mathcal{F}$ then we
see that $\mathcal{F}' \cap \mathcal{F}_n = 0$ and
$\mathcal{F}'|_U = \mathcal{F}|_U$. Note in particular that the subsheaf
$(\mathcal{F}')_N$ of sections annihilated by $\mathcal{I}^N$ is zero
for all $N \geq 0$. Hence by
Properties, Lemma \ref{properties-lemma-sections-over-quasi-compact-open}
we deduce that
the top horizontal arrow in the following commutative
diagram is a bijection:
$$
\xymatrix{
\text{colim}_n\ \text{Hom}_{\mathcal{O}_X}(
\mathcal{I}^n, \mathcal{F}')
\ar[r] \ar[d] &
\Gamma(U, \mathcal{F}') \ar[d] \\
\text{colim}_n\ \text{Hom}_{\mathcal{O}_X}(
\mathcal{I}^n, \mathcal{F})
\ar[r] &
\Gamma(U, \mathcal{F})
}
$$
Since also the right vertical arrow is a bijection we conclude that
the bottom horizontal arrow is surjective. The bottom horizontal
arrow is injective by
Properties, Lemma \ref{properties-lemma-sections-over-quasi-compact-open}.
This proves the bottom arrow is a bijection as desired.

\medskip\noindent
Next, we come to the general case.
By Lemma \ref{lemma-tensor-hom-coherent} the sheaf
$\mathcal{H} = \textit{Hom}_{\mathcal{O}_X}(\mathcal{G}, \mathcal{F})$
is coherent. By definition we have
$$
\mathcal{H}(U)
=
\textit{Hom}_{\mathcal{O}_U}(\mathcal{G}|_U, \mathcal{F}|_U)
$$
Pick a $\psi$ in the right hand side of the first arrow of the
lemma, i.e.,  $\psi \in \mathcal{H}(U)$. The result just proved applies
to $\mathcal{H}$ and hence there exists an $n \geq 0$ and an
$\varphi : \mathcal{I}^n \to \mathcal{H}$ which recovers
$\psi$ on restricition to $U$. By
Modules, Lemma \ref{modules-lemma-internal-hom}
$\varphi$ corresponds to a map
$$
\varphi :
\mathcal{I}^{\otimes n} \otimes_{\mathcal{O}_X} \mathcal{G}
\longrightarrow
\mathcal{F}.
$$
This is almost what we want except that the source of the arrow
is the tensor product of $\mathcal{I}^n$ and $\mathcal{G}$
and not the product. We will show that, at the cost of increasing $n$,
the difference is irrelevant. Consider the short exact sequence
$$
0 \to \mathcal{K} \to
\mathcal{I}^n \otimes_{\mathcal{O}_X} \mathcal{G} \to
\mathcal{I}^n\mathcal{G} \to 0
$$
where $\mathcal{K}$ is defined as the kernel. Note that
$\mathcal{I}^n\mathcal{K} = 0$ (proof omitted). By Artin-Rees
again we see that
$$
\mathcal{K}
\cap
\mathcal{I}^m(\mathcal{I}^n \otimes_{\mathcal{O}_X} \mathcal{G})
=
0
$$
for some $m$ large enough. In other words we see that
$$
\mathcal{I}^m(\mathcal{I}^n \otimes_{\mathcal{O}_X} \mathcal{G})
\longrightarrow
\mathcal{I}^{n + m}\mathcal{G}
$$
is an isomorphism. Let $\varphi'$ be the restriction of
$\varphi$ to this submodule thought of as a map
$\mathcal{I}^{m + n}\mathcal{G} \to \mathcal{F}$.
Then $\varphi'$ gives an element
of the left hand side of the first arrow of the lemma which
maps to $\psi$ via the arrow. In other words we have prove surjectivity
of the arrow. We omit the proof of injectivity.
\end{proof}


















\section{Devissage of coherent sheaves}
\label{section-devissage}

\noindent
Let $X$ be a Noetherian scheme. Consider an integral closed subscheme
$i : Z \to X$. It is often convenient to consider coherent sheaves of
the form $i_*\mathcal{G}$ where $\mathcal{G}$ is a coherent sheaf on
$Z$. In particular we are interested in these sheaves when $\mathcal{G}$
is a torsion free rank $1$ sheaf. For example $\mathcal{G}$ could be
a nonzero sheaf of ideals on $Z$, or even more specifically
$\mathcal{G} = \mathcal{O}_Z$.

\medskip\noindent
Throughout this section we will use that a coherent sheaf is the
same thing as a finite type quasi-coherent sheaf and that a
quasi-coherent subquotient of a coherent sheaf is coherent, see
Section \ref{section-coherent-sheaves}.
The support of a coherent sheaf is closed, see
Modules, Lemma \ref{modules-lemma-support-finite-type-closed}.

\begin{lemma}
\label{lemma-prepare-filter-support}
Let $X$ be a Noetherian scheme.
Let $\mathcal{F}$ be a coherent sheaf on $X$.
Suppose that $\text{Supp}(\mathcal{F}) = Z \cup Z'$ with $Z$, $Z'$ closed.
Then there exists a short exact sequence of coherent sheaves
$$
0 \to \mathcal{G}' \to \mathcal{F} \to \mathcal{G} \to 0
$$
with $\text{Supp}(\mathcal{G}') \subset Z'$ and
$\text{Supp}(\mathcal{G}) \subset Z$.
\end{lemma}

\begin{proof}
Throughout the proof we will use that a coherent sheaf is the
same thing as a finite type quasi-coherent sheaf and that a
quasi-coherent subquotient of a coherent sheaf is coherent, see
Section \ref{section-coherent-sheaves}.
The support of a coherent sheaf is closed, see
Modules, Lemma \ref{modules-lemma-support-finite-type-closed}.
Let $\mathcal{I} \subset \mathcal{O}_X$ be the sheaf of ideals
defining the reduced induced closed subscheme structure on $Z$, see
Schemes, Lemma \ref{schemes-lemma-reduced-closed-subscheme}.
Consider the subsheaves
$\mathcal{G}'_n = \mathcal{I}^n\mathcal{F}$ and the
quotients $\mathcal{G}_n = \mathcal{F}/\mathcal{I}^n\mathcal{F}$.
For each $n$ we have a short exact sequence
$$
0 \to \mathcal{G}'_n \to \mathcal{F} \to \mathcal{G}_n \to 0
$$
For every point $x$ of $Z' \setminus Z$ we have
$\mathcal{I}_x = \mathcal{O}_{X, x}$
and hence $\mathcal{G}_{n, x} = 0$. Thus we see that
$\text{Supp}(\mathcal{G}_n) \subset Z$. Note that $X \setminus Z'$
is a Noetherian scheme. Hence by Lemma \ref{lemma-power-ideal-kills-sheaf}
there exists an $n$ such that
$\mathcal{G}'_n|_{X \setminus Z'} =
\mathcal{I}^n\mathcal{F}|_{X \setminus Z'} = 0$.
For such an $n$ we see that $\text{Supp}(\mathcal{G}'_n) \subset Z'$.
Thus setting
$\mathcal{G}' = \mathcal{G}'_n$ and $\mathcal{G} = \mathcal{G}_n$
works.
\end{proof}

\begin{lemma}
\label{lemma-prepare-filter-irreducible}
Let $X$ be a Noetherian scheme.
Let $i : Z \to X$ be an integral closed subscheme.
Let $\xi \in Z$ be the generic point.
Let $\mathcal{F}$ be a coherent sheaf on $X$.
Assume that $\mathcal{F}_\xi$ is annihilated by
$\mathfrak m_\xi$. Then there exists an integer
$r \geq 0$ and a sheaf of ideals $\mathcal{I} \subset \mathcal{O}_Z$
and an injective map of coherent sheaves
$$
i_*\left(\mathcal{I}^{\oplus r}\right) \to \mathcal{F}
$$
which is an isomorphism in a neighbourhood of $\xi$.
\end{lemma}

\begin{proof}
Let $\mathcal{J} \subset \mathcal{O}_X$ be the ideal sheaf of $Z$.
Let $\mathcal{F}' \subset \mathcal{F}$ be the subsheaf of
local sections of $\mathcal{F}$ which are annihilated by
$\mathcal{J}$. It is a quasi-coherent sheaf by
Properties, Lemma \ref{properties-lemma-sections-annihilated-by-ideal}.
Moreover, $\mathcal{F}'_\xi = \mathcal{F}_\xi$ because
$\mathcal{J}_\xi = \mathfrak m_\xi$ and part (3) of
Properties, Lemma \ref{properties-lemma-sections-annihilated-by-ideal}.
By Lemma \ref{lemma-local-isomorphism} we see that
$\mathcal{F}' \to \mathcal{F}$
induces an isomorphism in a neighbourhood of $\xi$.
Hence we may replace $\mathcal{F}$ by $\mathcal{F}'$ and assume
that $\mathcal{F}$ is annihilated by $\mathcal{J}$.

\medskip\noindent
Assume $\mathcal{J}\mathcal{F} = 0$. By
Morphisms, Lemma \ref{morphisms-lemma-i-star-equivalence} we can write
$\mathcal{F} = i_*\mathcal{G}$ for some quasi-coherent
sheaf $\mathcal{G}$ on $Z$. By
Modules, Lemma \ref{modules-lemma-i-star-reflects-finite-type}
combined with the results of Section \ref{section-coherent-sheaves}
we also see $\mathcal{G}$ is coherent on $Z$.
Suppose we can find a morphism
$\mathcal{I}^{\oplus r} \to \mathcal{G}$ which is an isomorphism
in a neighbourhood of the generic point $\xi$ of $Z$.
Then applying $i_*$ (which is left exact) we get the result of the lemma.
Hence we have reduced to the case $X = Z$.

\medskip\noindent
Suppose $Z = X$ is an integral Noetherian scheme with generic point $\xi$.
Note that $\mathcal{O}_{X, \xi} = \kappa(\xi)$ is the function field of $X$
in this case.
Since $\mathcal{F}_\xi$ is a finite $\mathcal{O}_\xi$-module we see
that $r = \dim_{\kappa(\xi)} \mathcal{F}_\xi$ is finite.
Hence the sheaves $\mathcal{O}_X^{\oplus r}$ and $\mathcal{F}$
have isomorphic stalks at $\xi$.
By Lemma \ref{lemma-map-stalks-local-map} there exists a nonempty
open $U \subset X$ and a morphism
$\psi : \mathcal{O}_X^{\oplus r}|_U \to \mathcal{F}|_U$
which is an isomorphism
at $\xi$, and hence an isomorphism in a neighbourhood of $\xi$ by
Lemma \ref{lemma-local-isomorphism}.
By Schemes, Lemma \ref{schemes-lemma-reduced-closed-subscheme}
there exists a quasi-coherent sheaf of ideals
$\mathcal{I} \subset \mathcal{O}_X$
whose associated closed subscheme $Z \subset X$ is the complement
of $U$.
By Lemma \ref{lemma-homs-over-open} there exists an $n \geq 0$ and a morphism
$\mathcal{I}^n(\mathcal{O}_X)^{\oplus r}) \to \mathcal{F}$
which recovers our $\psi$ over $U$. Since
$\mathcal{I}^n(\mathcal{O}_X)^{\oplus r}) = (\mathcal{I}^n)^{\oplus r}$
we get a map as in the lemma. It is injective because $X$ is
integral and it is injective at the generic point of $X$
(easy proof omitted).
\end{proof}

\begin{lemma}
\label{lemma-coherent-filter}
Let $X$ be a Noetherian scheme.
Let $\mathcal{F}$ be a coherent sheaf on $X$.
There exists a filtration
$$
0 = \mathcal{F}_0 \subset \mathcal{F}_1 \subset
\ldots \subset \mathcal{F}_m = \mathcal{F}
$$
by coherent subsheaves such that for each $j = 1, \ldots, m$
there exists an integral closed subscheme $Z_j \subset X$
and a sheaf of ideals $\mathcal{I}_j \subset \mathcal{O}_{Z_j}$
such that
$$
\mathcal{F}_j/\mathcal{F}_{j - 1}
\cong (Z_j \to X)_* \mathcal{I}_j
$$
\end{lemma}

\begin{proof}
Consider the collection
$$
\mathcal{T} =
\left\{
\begin{matrix}
Z \subset X
\text{ closed such that there exists a coherent sheaf }
\mathcal{F} \\
\text{ with }
\text{Supp}(\mathcal{F}) = Z
\text{ for which the lemma is wrong}
\end{matrix}
\right\}
$$
We are trying to show that $\mathcal{T}$ is empty. If not, then
because $X$ is Noetherian we can choose a minimal element
$Z \in \mathcal{T}$. This means that there exists a coherent
sheaf $\mathcal{F}$ on $X$ whose support is $Z$ and for which the
lemma does not hold. Clearly $Z \not = \emptyset$ since the only
sheaf whose support is empty is the zero sheaf for which the
lemma does hold (with $m = 0$).

\medskip\noindent
If $Z$ is not irreducible, then we can write $Z = Z_1 \cup Z_2$
with $Z_1, Z_2$ closed and strictly smaller than $Z$.
Then we can apply Lemma \ref{lemma-prepare-filter-support}
to get a short exact sequence of coherent sheaves
$$
0 \to
\mathcal{G}_1 \to
\mathcal{F} \to
\mathcal{G}_2 \to 0
$$
with $\text{Supp}(\mathcal{G}_i) \subset Z_i$. By minimality of
$Z$ each of $\mathcal{G}_i$ has a filtration as in the statement
of the lemma. By considering the induced filtration on $\mathcal{F}$
we arrive at a contradiction. Hence we conclude
that $Z$ is irreducible.

\medskip\noindent
Suppose $Z$ is irreducible. Let $\mathcal{J}$ be the sheaf of ideals
cutting out the reduced induced closed subscheme structure of $Z$,
see Schemes, Lemma \ref{schemes-lemma-reduced-closed-subscheme}.
By Lemma \ref{lemma-power-ideal-kills-sheaf} we see there exists
an $n \geq 0$ such that $\mathcal{J}^n\mathcal{F} = 0$. Hence we obtain
a filtration
$$
0 = \mathcal{I}^n\mathcal{F} \subset \mathcal{I}^{n - 1}\mathcal{F}
\subset \ldots \subset \mathcal{I}\mathcal{F} \subset \mathcal{F}
$$
each of whose succesive subquotients is annihilated by $\mathcal{J}$.
Hence if each of these subquotients has a filtration as in the statement
of the lemma then also $\mathcal{F}$ does. In other words we may
assume that $\mathcal{J}$ does annihilate $\mathcal{F}$.

\medskip\noindent
In the case where $Z$ is irreducible and $\mathcal{J}\mathcal{F} = 0$
we can apply Lemma \ref{lemma-prepare-filter-irreducible}.
This gives a short exact sequence
$$
0 \to
i_*(\mathcal{I}^{\oplus r}) \to
\mathcal{F} \to
\mathcal{Q} \to 0
$$
where $\mathcal{Q}$ is defined as the quotient.
Since $\mathcal{Q}$ is zero in a neighbourhood of $\xi$ by
the lemma just cited we see that the support of $\mathcal{Q}$
is strictly smaller than $Z$. Hence we see that $\mathcal{Q}$
has a filtration of the desired type by minimality of $Z$.
But then clearly $\mathcal{F}$ does too, which is our final contradiction.
\end{proof}

\begin{lemma}
\label{lemma-property-initial}
Let $X$ be a Noetherian scheme.
Let $\mathcal{P}$ be a property of coherent sheaves on $X$ such that
\begin{enumerate}
\item For any short exact sequence of coherent sheaves
$$
0 \to \mathcal{F}_1 \to \mathcal{F} \to \mathcal{F}_2 \to 0
$$
if $\mathcal{F}_i$, $i = 1, 2$ have property $\mathcal{P}$
then so does $\mathcal{F}$.
\item For every integral closed subscheme $Z \subset X$
and every quasi-coherent sheaf of ideals
$\mathcal{I} \subset \mathcal{O}_Z$ we have
$\mathcal{P}$ for $i_*\mathcal{I}$.
\end{enumerate}
Then property $\mathcal{P}$ holds for every coherent sheaf
on $X$.
\end{lemma}

\begin{proof}
First note that if $\mathcal{F}$ is a coherent sheaf with a filtration
$$
0 = \mathcal{F}_0 \subset \mathcal{F}_1 \subset
\ldots \subset \mathcal{F}_m = \mathcal{F}
$$
by coherent subsheaves such that each of $\mathcal{F}_i/\mathcal{F}_{i - 1}$
has property $\mathcal{P}$, then so does $\mathcal{F}$.
This follows from the property (1) for $\mathcal{P}$.
On the other hand, by Lemma \ref{lemma-coherent-filter}
we can filter any $\mathcal{F}$
with succesive subquotients as in (2).
Hence the lemma follows.
\end{proof}

\begin{lemma}
\label{lemma-property-irreducible}
Let $X$ be a Noetherian scheme.
Let $Z_0 \subset X$ be an irreducible closed subset with generic point $\xi$.
Let $\mathcal{P}$ be a property of coherent sheaves on $X$ such that
\begin{enumerate}
\item For any short exact sequence of coherent sheaves if two
out of three of them have property $\mathcal{P}$ then so does the
third.
\item For every integral closed subscheme $Z \subset Z_0 \subset X$,
$Z \not = Z_0$ and every quasi-coherent sheaf of ideals
$\mathcal{I} \subset \mathcal{O}_Z$ we have
$\mathcal{P}$ for $(Z \to X)_*\mathcal{I}$.
\item There exists some coherent sheaf $\mathcal{G}$ on $X$ such that
\begin{enumerate}
\item $\text{Supp}(\mathcal{G}) = Z_0$,
\item $\mathcal{G}_\xi$ is annihilated by $\mathfrak m_\xi$,
\item $\dim_{\kappa(\xi)} \mathcal{G}_\xi = 1$, and
\item property $\mathcal{P}$ holds for $\mathcal{G}$.
\end{enumerate}
\end{enumerate}
Then property $\mathcal{P}$ holds for every coherent sheaf
$\mathcal{F}$ on $X$ whose support is contained in $Z_0$.
\end{lemma}

\begin{proof}
First note that if $\mathcal{F}$ is a coherent sheaf with a filtration
$$
0 = \mathcal{F}_0 \subset \mathcal{F}_1 \subset
\ldots \subset \mathcal{F}_m = \mathcal{F}
$$
by coherent subsheaves such that each of $\mathcal{F}_i/\mathcal{F}_{i - 1}$
has property $\mathcal{P}$, then so does $\mathcal{F}$. Or, if $\mathcal{F}$
has property $\mathcal{P}$ and all but one of the 
$\mathcal{F}_i/\mathcal{F}_{i - 1}$ has property $\mathcal{P}$ then
so does the last one.
This follows from the two-out-of-three property (1) for $\mathcal{P}$.

\medskip\noindent
As a first application of these remarks we conclude that
any coherent sheaf whose support is strictly contained in $Z_0$
has property $\mathcal{P}$. Namely, such a sheaf
has a filtration (see Lemma \ref{lemma-coherent-filter})
whose subquotients have property $\mathcal{P}$ according to (2).

\medskip\noindent
As a second application
we apply this remark to the sheaf $\mathcal{G}$ from assumption
(3) and a filtration
$$
0 = \mathcal{G}_0 \subset \mathcal{G}_1 \subset
\ldots \subset \mathcal{G}_m = \mathcal{G}
$$
by coherent subsheaves as in Lemma \ref{lemma-coherent-filter}.
Let $Z_j \to X$ be the integral closed subschemes, and
$\mathcal{I}_i \subset \mathcal{O}_{Z_i}$ the quasi-coherent sheaves
of ideals such that
$\mathcal{G}_i/\mathcal{G}_{i - 1} \cong (Z_i \to X)_*\mathcal{I}_i$.
We may obviously assume all the $\mathcal{I}_i$ are nonzero.
Since $\dim_{\kappa(\xi)} \mathcal{G}_\xi = 1$ we see that there
is exactly one $i = i_0$ such that $Z_{i_0} = Z_0$ and all other
$Z_i \subset Z_0$ are proper irreducible closed subsets of $Z_0$.
By the remark above and (2) we see that $(Z_{i_0} \to X)_*\mathcal{I}_{i_0}$
has property $\mathcal{P}$.
We conclude that there exists at least one quasi-coherent
sheaf of ideals $\mathcal{I}$ (namely $\mathcal{I}_{i_0}$) on $Z_0$
such that $(Z_0 \to X)_*\mathcal{I}$ has property $\mathcal{P}$.

\medskip\noindent
Next, suppose that $\mathcal{I}'$ is another quasi-coherent
sheaf of ideals $Z_0$. Then we can consider the intersection
$\mathcal{I}'' = \mathcal{I}' \cap \mathcal{I}$ and we get
two short exact sequences
$$
0 \to
(Z_0 \to X)_*\mathcal{I}'' \to
(Z_0 \to X)_*\mathcal{I} \to
\mathcal{Q} \to 0
$$
and
$$
0 \to
(Z_0 \to X)_*\mathcal{I}' \to
(Z_0 \to X)_*\mathcal{I} \to
\mathcal{Q}' \to 0.
$$
Note that the support of the coherent sheaves $\mathcal{Q}$ and
$\mathcal{Q}'$ are strictly contained in $Z_0$.
Hence $\mathcal{Q}$ and $\mathcal{Q}$ have property $\mathcal{P}$
(see above). Hence we conclude using (1)
that $(Z_0 \to X)_*\mathcal{I}''$ and $(Z_0 \to X)_*\mathcal{I}'$
both have $\mathcal{P}$ as well.

\medskip\noindent
The final step of the proof is to note that any coherent sheaf
$\mathcal{F}$ on $X$ whose support is contained in $Z_0$ has a filtration
(see Lemma \ref{lemma-coherent-filter} again) whose subquotients
all have property $\mathcal{P}$ by what we just said.
\end{proof}

\begin{lemma}
\label{lemma-property}
Let $X$ be a Noetherian scheme.
Let $\mathcal{P}$ be a property of coherent sheaves on $X$ such that
\begin{enumerate}
\item For any short exact sequence of coherent sheaves if two
out of three of them have property $\mathcal{P}$ then so does the
third.
\item For every integral closed subscheme $Z \subset X$
with generic point $\xi$ there exists
some coherent sheaf $\mathcal{G}$ such that
\begin{enumerate}
\item $\text{Supp}(\mathcal{G}) = Z$,
\item $\mathcal{G}_\xi$ is annihilated by $\mathfrak m_\xi$,
\item $\dim_{\kappa(\xi)} \mathcal{G}_\xi = 1$, and
\item property $\mathcal{P}$ holds for $\mathcal{G}$.
\end{enumerate}
\end{enumerate}
Then property $\mathcal{P}$ holds for every coherent sheaf
on $X$.
\end{lemma}

\begin{proof}
According to Lemma \ref{lemma-property-initial} it suffices to show
that given any integral closed subscheme $Z \subset X$
and every quasi-coherent sheaf of ideals
$\mathcal{I} \subset \mathcal{O}_Z$ we have
$\mathcal{P}$ for $(Z \to X)_*\mathcal{I}$.
If this fails, then since $X$ is Noetherian there
is a minimal integral closed subscheme $Z_0 \subset X$
such that $\mathcal{P}$ fails for $(Z_0 \to X)_*\mathcal{I}$ for
some quasi-coherent sheaf of ideals $\mathcal{I} \subset \mathcal{O}_{Z_0}$.
In other words, the result does hold for any integral closed
subscheme of $Z$. According to Lemma \ref{lemma-property-irreducible}
this cannot happen.
\end{proof}

\begin{lemma}
\label{lemma-property-irreducible-higher-rank}
(Variant of Lemma \ref{lemma-property-irreducible}.)
Let $X$ be a Noetherian scheme.
Let $Z_0 \subset X$ be an irreducible closed subset with generic point $\xi$.
Let $\mathcal{P}$ be a property of coherent sheaves on $X$ such that
\begin{enumerate}
\item For any short exact sequence of coherent sheaves if two
out of three of them have property $\mathcal{P}$ then so does the
third.
\item If $\mathcal{P}$ holds for a direct sum of coherent sheaves
then it holds for both.
\item For every integral closed subscheme $Z \subset Z_0 \subset X$,
$Z \not = Z_0$ and every quasi-coherent sheaf of ideals
$\mathcal{I} \subset \mathcal{O}_Z$ we have
$\mathcal{P}$ for $(Z \to X)_*\mathcal{I}$.
\item There exists some coherent sheaf $\mathcal{G}$ on $X$ such that
\begin{enumerate}
\item $\text{Supp}(\mathcal{G}) = Z_0$,
\item $\mathcal{G}_\xi$ is annihilated by $\mathfrak m_\xi$, and
\item property $\mathcal{P}$ holds for $\mathcal{G}$.
\end{enumerate}
\end{enumerate}
Then property $\mathcal{P}$ holds for every coherent sheaf
$\mathcal{F}$ on $X$ whose support is contained in $Z_0$.
\end{lemma}

\begin{proof}
The proof is a variant on the proof of Lemma \ref{lemma-property-irreducible}.
In exactly the same manner as in that proof we see that
any coherent sheaf whose support is strictly contained in $Z_0$
has property $\mathcal{P}$.

\medskip\noindent
Consider a coherent sheaf $\mathcal{G}$ as in (3).
By Lemma \ref{lemma-prepare-filter-irreducible}
there exists a sheaf of ideals $\mathcal{I}$ on $Z_0$ and
a short exact sequence
$$
0 \to
\left((Z_0 \to X)_*\mathcal{I}\right)^{\oplus r} \to
\mathcal{G} \to
\mathcal{Q} \to 0
$$
where the support of $\mathcal{Q}$ is stricly contained in $Z_0$.
In particular $r > 0$ and $\mathcal{I}$ is nonzero
because the support of $\mathcal{G}$ is equal to $Z$.
Since $\mathcal{Q}$ has property $\mathcal{P}$ we conclude that
also $\left((Z_0 \to X)_*\mathcal{I}\right)^{\oplus r}$
has property $\mathcal{P}$. 
By (2) we deduce property $\mathcal{P}$ for
$(Z_0 \to X)_*\mathcal{I}$. Slotting this into the proof of
Lemma \ref{lemma-property-irreducible}
at the appropriate point gives the lemma.
Some details omitted.
\end{proof}

\begin{lemma}
\label{lemma-property-higher-rank}
(Variant of Lemma \ref{lemma-property}.)
Let $X$ be a Noetherian scheme.
Let $\mathcal{P}$ be a property of coherent sheaves on $X$ such that
\begin{enumerate}
\item For any short exact sequence of coherent sheaves if two
out of three of them have property $\mathcal{P}$ then so does the
third.
\item If $\mathcal{P}$ holds for a direct sum of coherent sheaves
then it holds for both.
\item For every integral closed subscheme $Z \subset X$
with generic point $\xi$ there exists
some coherent sheaf $\mathcal{G}$ such that
\begin{enumerate}
\item $\text{Supp}(\mathcal{G}) = Z$,
\item $\mathcal{G}_\xi$ is annihilated by $\mathfrak m_\xi$, and
\item property $\mathcal{P}$ holds for $\mathcal{G}$.
\end{enumerate}
\end{enumerate}
Then property $\mathcal{P}$ holds for every coherent sheaf
on $X$.
\end{lemma}

\begin{proof}
This follows from Lemma \ref{lemma-property-irreducible-higher-rank}
by exactly the same argument as used in
the proof of Lemma \ref{lemma-property}.
\end{proof}

\begin{lemma}
\label{lemma-property-irreducible-higher-rank-cohomological}
(Cohomological variant of Lemma \ref{lemma-property-irreducible-higher-rank}.)
Let $X$ be a Noetherian scheme.
Let $Z_0 \subset X$ be an irreducible closed subset with generic point $\xi$.
Let $\mathcal{P}$ be a property of coherent sheaves on $X$ such that
\begin{enumerate}
\item For any short exact sequence of coherent sheaves
$$
0 \to \mathcal{F}_1 \to \mathcal{F} \to \mathcal{F}_2 \to 0
$$
if $\mathcal{F}_i$, $i = 1, 2$ have property $\mathcal{P}$
then so does $\mathcal{F}$.
\item If $\mathcal{P}$ holds for a direct sum of coherent sheaves
then it holds for both.
\item For every integral closed subscheme $Z \subset Z_0 \subset X$,
$Z \not = Z_0$ and every quasi-coherent sheaf of ideals
$\mathcal{I} \subset \mathcal{O}_Z$ we have
$\mathcal{P}$ for $(Z \to X)_*\mathcal{I}$.
\item There exists some coherent sheaf $\mathcal{G}$ such that
\begin{enumerate}
\item $\text{Supp}(\mathcal{G}) = Z_0$,
\item $\mathcal{G}_\xi$ is annihilated by $\mathfrak m_\xi$, and
\item for every quasi-coherent sheaf of ideals
$\mathcal{J} \subset \mathcal{O}_X$ such that
$\mathcal{J}_\xi = \mathcal{O}_{X, \xi}$ there exists a quasi-coherent
subsheaf $\mathcal{G}' \subset \mathcal{J}\mathcal{G}$ with
$\mathcal{G}'_\xi = \mathcal{G}_\xi$ and such that
$\mathcal{P}$ holds for $\mathcal{G}'$.
\end{enumerate}
\end{enumerate}
Then property $\mathcal{P}$ holds for every coherent sheaf
$\mathcal{F}$ on $X$ whose support is contained in $Z_0$.
\end{lemma}

\begin{proof}
The proof is a variant on the proof of Lemma \ref{lemma-property-irreducible}.
In exactly the same manner as in that proof we see that
any coherent sheaf whose support is strictly contained in $Z_0$
has property $\mathcal{P}$. Note that this does not use the full
strength of the two-out-of-three property of that lemma, only
the weaker variant (1) above which is in force in the current situation.

\medskip\noindent
Let us denote $i : Z_0 \to X$ the closed immersion.
Consider a coherent sheaf $\mathcal{G}$ as in (4).
By Lemma \ref{lemma-prepare-filter-irreducible}
there exists a sheaf of ideals $\mathcal{I}$ on $Z_0$ and
a short exact sequence
$$
0 \to
i_*\mathcal{I}^{\oplus r} \to
\mathcal{G} \to
\mathcal{Q} \to 0
$$
where the support of $\mathcal{Q}$ is stricly contained in $Z_0$.
In particular $r > 0$ and $\mathcal{I}$ is nonzero
because the support of $\mathcal{G}$ is equal to $Z_0$.
Let $\mathcal{I}' \subset \mathcal{I}$ be any nonzero quasi-coherent
sheaf of ideals on $Z$ contained in $\mathcal{I}$.
Then we also get a short exact sequence
$$
0 \to
i_*(\mathcal{I}')^{\oplus r} \to
\mathcal{G} \to
\mathcal{Q}' \to 0
$$
where $\mathcal{Q}'$ has support properly contained in $Z_0$.
Let $\mathcal{J} \subset \mathcal{O}_X$ be a quasi-coherent sheaf
of ideals cuttting out the support of $\mathcal{Q}'$ (for example
the ideal corresponding to the reduced induced closed subscheme
structure on the the support of $\mathcal{Q}'$). Then
$\mathcal{J}_\xi = \mathcal{O}_{X, \xi}$. By
Lemma \ref{lemma-power-ideal-kills-sheaf}
we see that $\mathcal{J}^n\mathcal{Q}' = 0$ for some $n$.
Hence $\mathcal{J}^n\mathcal{G} \subset i_*(\mathcal{I}')^{\oplus r}$.
By assumption (4)(c) of the lemma we see there exists
a quasi-coherent subsheaf $\mathcal{G}' \subset \mathcal{J}^n\mathcal{G}$
with $\mathcal{G}'_\xi = \mathcal{G}_\xi$
for which property $\mathcal{P}$ holds.
Hence we get a short exact sequence
$$
0 \to \mathcal{G}' \to
i_*(\mathcal{I}')^{\oplus r} \to
\mathcal{Q}'' \to 0
$$
where $\mathcal{Q}''$ has support properly contained in $Z_0$.
Thus by our initial remarks and property (1) of the lemma
we conclude that $i_*(\mathcal{I}')^{\oplus r}$ satisfies
$\mathcal{P}$. Hence we see that $i_*\mathcal{I}'$ satisfies
$\mathcal{P}$ by (2). Finally, for an arbitrary quasi-coherent
sheaf of ideals $\mathcal{I}'' \subset \mathcal{O}_{Z_0}$ we can set
$\mathcal{I} = \mathcal{I}'' \cap \mathcal{I}$ and we get
a short exact sequence
$$
0 \to
i_*(\mathcal{I}') \to
i_*(\mathcal{I}'') \to
\mathcal{Q}''' \to 0
$$
where $\mathcal{Q}'''$ has support properly contained in $Z_0$.
Hence we conclude that property $\mathcal{P}$ holds for
$i_*\mathcal{I}''$ as well. Slotting this into the proof of
Lemma \ref{lemma-property-irreducible}
at the appropriate point gives the lemma.
Some details omitted.
\end{proof}

\begin{lemma}
\label{lemma-property-higher-rank-cohomological}
(Cohomological variant of Lemma \ref{lemma-property-higher-rank}.)
Let $X$ be a Noetherian scheme.
Let $\mathcal{P}$ be a property of coherent sheaves on $X$ such that
\begin{enumerate}
\item For any short exact sequence of coherent sheaves
$$
0 \to \mathcal{F}_1 \to \mathcal{F} \to \mathcal{F}_2 \to 0
$$
if $\mathcal{F}_i$, $i = 1, 2$ have property $\mathcal{P}$
then so does $\mathcal{F}$.
\item If $\mathcal{P}$ holds for a direct sum of coherent sheaves
then it holds for both.
\item For every integral closed subscheme $Z \subset X$
with generic point $\xi$ there exists
some coherent sheaf $\mathcal{G}$ such that
\begin{enumerate}
\item $\text{Supp}(\mathcal{G}) = Z$,
\item $\mathcal{G}_\xi$ is annihilated by $\mathfrak m_\xi$, and
\item for every quasi-coherent sheaf of ideals
$\mathcal{J} \subset \mathcal{O}_X$ such that
$\mathcal{J}_\xi = \mathcal{O}_{X, \xi}$ there exists a quasi-coherent
subsheaf $\mathcal{G}' \subset \mathcal{J}\mathcal{G}$ with
$\mathcal{G}'_\xi = \mathcal{G}_\xi$ and such that
$\mathcal{P}$ holds for $\mathcal{G}'$.
\end{enumerate}
\end{enumerate}
Then property $\mathcal{P}$ holds for every coherent sheaf
on $X$.
\end{lemma}

\begin{proof}
Identical to the proofs of Lemmas \ref{lemma-property-higher-rank}
and \ref{lemma-property}.
\end{proof}








\section{Finite morphisms and affines}
\label{section-finite-affine}

\noindent
In this section we use the results of the preceding sections
to show that the image of a Noetherian affine scheme under a finite
morphism is affine. We will see later that this result holds more
generally (see Limits, Lemma \ref{limits-lemma-affine}).

\begin{lemma}
\label{lemma-finite-morphism-Noetherian}
Let $f : Y \to X$ be a morphism of schemes.
Assume $f$ is finite, surjective and $X$ locally Noetherian.
Let $Z \subset X$ be an integral closed subscheme with
generic point $\xi$. Then
there exists a coherent sheaf $\mathcal{F}$ on $Y$
such that the support of $f_*\mathcal{F}$ is equal to $Z$
and $(f_*\mathcal{F})_\xi$ is annihilated by $\mathfrak m_\xi$.
\end{lemma}

\begin{proof}
Note that $Y$ is locally Noetherian by
Morphisms, Lemma \ref{morphisms-lemma-finite-type-noetherian}.
Because $f$ is surjective the fibre $Y_\xi$ is not empty.
Pick $\xi' \in Y$ mapping to $\xi$. Let $Z' = \overline{\{\xi'\}}$.
We may think of $Z' \subset Y$ as a reduced closed subscheme,
see Schemes, Lemma \ref{schemes-lemma-reduced-closed-subscheme}.
Hence the sheaf $\mathcal{F} = (Z' \to Y)_*\mathcal{O}_{Z'}$
is a coherent sheaf on $Y$ (see
Lemma \ref{lemma-finite-pushforward-coherent}).
Look at the commutative diagram
$$
\xymatrix{
Z' \ar[r]_{i'} \ar[d]_{f'} &
Y \ar[d]^f \\
Z \ar[r]^i &
X
}
$$
We see that $f_*\mathcal{F} = i_*f'_*\mathcal{O}_{Z'}$.
Hence the stalk of $f_*\mathcal{F}$ at $\xi$ is the stalk
of $f'_*\mathcal{O}_{Z'}$ at $\xi$. Note that since $Z'$ is
integral with generic point $\xi'$ we have that
$\xi'$ is the only point of $Z'$ lying over $\xi$, see
Algebra, Lemmas \ref{algebra-lemma-finite-is-integral} and
\ref{algebra-lemma-integral-no-inclusion}.
Hence the stalk of $f'_*\mathcal{O}_{Z'}$ at $\xi$
equal $\mathcal{O}_{Z', \xi'} = \kappa(\xi')$. In particular
the stalk of $f_*\mathcal{F}$ at $\xi$ is not zero.
This combined with the fact that $f_*\mathcal{F}$ is
of the form $i_*f'_*(\text{something})$ implies the lemma.
\end{proof}

\begin{lemma}
\label{lemma-affine-morphism-projection-ideal}
Let $f : Y \to X$ be a morphism of schemes.
Let $\mathcal{F}$ be a quasi-coherent sheaf on $Y$.
Let $\mathcal{I}$ be a quasi-coherent sheaf of ideals on $X$.
If the morphism $f$ is affine then
$\mathcal{I}f_*\mathcal{F} = f_*(f^{-1}\mathcal{I}\mathcal{F})$.
\end{lemma}

\begin{proof}
The notation means the following. Since $f^{-1}$ is an exact functor
we see that $f^{-1}\mathcal{I}$ is a sheaf
of ideals of $f^{-1}\mathcal{O}_X$. Via the map
$f^\sharp : f^{-1}\mathcal{O}_X \to \mathcal{O}_Y$ this acts on
$\mathcal{F}$. Then $f^{-1}\mathcal{I}\mathcal{F}$ is the subsheaf
generated by sums of local sections of the form $as$ where $a$
is a local section of $f^{-1}\mathcal{I}$ and $s$ is a local section
of $\mathcal{F}$. It is a quasi-coherent $\mathcal{O}_Y$-submodule
of $\mathcal{F}$ because it is also the image of a natural map
$f^*\mathcal{I} \otimes_{\mathcal{O}_Y} \mathcal{F} \to \mathcal{F}$.

\medskip\noindent
Having said this the proof is straightforward. Namely, the question is local
and hence we may assume $X$ is affine. Since $f$ is affine we see that
$Y$ is affine too. Thus we may write
$Y = \text{Spec}(B)$, $X = \text{Spec}(A)$, $\mathcal{F} = \widetilde{M}$,
and $\mathcal{I} = \widetilde{I}$. The assertion of the lemma in this
case boils down to the statement that
$$
I(M_A) = ((IB)M)_A
$$
where $M_A$ indicates the $A$-module associated to the $B$-module $M$.
\end{proof}

\begin{lemma}
\label{lemma-image-affine-finite-morphism-affine-Noetherian}
Let $f : Y \to X$ be a morphism of schemes.
Assume
\begin{enumerate}
\item $f$ finite,
\item $f$ surjective,
\item $Y$ affine, and
\item $X$ Noetherian.
\end{enumerate}
Then $X$ is affine.
\end{lemma}

\begin{proof}
We will prove that under the assumptions of the lemma for any coherent
$\mathcal{O}_X$-module $\mathcal{F}$ we have $H^1(X, \mathcal{F}) = 0$.
Since this will in particular imply that $H^1(X, \mathcal{I}) = 0$
for every quasi-coherent sheaf of ideals of $\mathcal{O}_X$. Then it will
follows that $X$ is affine from either
Lemma \ref{lemma-quasi-compact-h1-zero-covering} or
Lemma \ref{lemma-quasi-separated-h1-zero-covering}.

\medskip\noindent
Let $\mathcal{P}$ be the property of coherent sheaves
$\mathcal{F}$ on $X$ defined by the rule
$$
\mathcal{P}(\mathcal{F}) \Leftrightarrow H^1(X, \mathcal{F}) = 0.
$$
We are going to apply Lemma \ref{lemma-property-higher-rank-cohomological}.
Thus we have to verify (1), (2) and (3) of that lemma for $\mathcal{P}$.
Property (1) follows from the long exact cohomology sequence associated
to a short exact sequence of sheaves. Property (2) follows since
$H^1(X, -)$ is an additive functor. To see (3) let $Z \subset X$ be
an integral closed subscheme with generic point $\xi$.
Let $\mathcal{F}$ be a coherent sheaf on $Y$ such that
the support of $f_*\mathcal{F}$ is equal to $Z$
and $(f_*\mathcal{F})_\xi$ is annihilated by $\mathfrak m_\xi$,
see Lemma \ref{lemma-finite-morphism-Noetherian}. We claim that
taking $\mathcal{G} = f_*\mathcal{F}$ works. We only have to verify
part (3)(c) of Lemma \ref{lemma-property-higher-rank-cohomological}.
Hence assume that $\mathcal{J} \subset \mathcal{O}_X$ is a 
quasi-coherent sheaf of ideals such that
$\mathcal{J}_\xi = \mathcal{O}_{X, \xi}$.
A finite morphism is affine hence by
Lemma \ref{lemma-affine-morphism-projection-ideal} we see that
$\mathcal{J}\mathcal{G} = f_*(f^{-1}\mathcal{J}\mathcal{F})$.
Also, as pointed out in the proof of
Lemma \ref{lemma-affine-morphism-projection-ideal} the sheaf
$f^{-1}\mathcal{J}\mathcal{F}$ is a quasi-coherent $\mathcal{O}_Y$-module.
Since $Y$ is affine we see that $H^1(Y, f^{-1}\mathcal{J}\mathcal{F}) = 0$,
see Lemma \ref{lemma-quasi-coherent-affine-cohomology-zero}.
Since $f$ is finite, hence affine, we see that
$R^qf_*(f^{-1}\mathcal{J}\mathcal{F}) = 0$ for all $q \geq 1$
by Lemma \ref{lemma-relative-affine-vanishing}.
By Cohomology, Lemma \ref{cohomology-lemma-apply-Leray} we see that
$$
H^1(X, \mathcal{J}\mathcal{G}) =
H^1(X, f_*(f^{-1}\mathcal{J}\mathcal{F})) =
H^1(Y, f^{-1}\mathcal{J}\mathcal{F}) = 0.
$$
Hence the quasi-coherent subsheaf $\mathcal{G}' = \mathcal{J}\mathcal{G}$
satisfies $\mathcal{P}$. This verifies property (3)(c) of
Lemma \ref{lemma-property-higher-rank-cohomological} as desired.
\end{proof}









\section{Coherent sheaves and projective morphisms}
\label{section-coherent-projective}

\noindent
It seems illuminating to formulate an all-in-one result for
projective space over a Noetherian ring.

\begin{lemma}
\label{lemma-coherent-projective}
Let $R$ be a Noetherian ring.
Let $n \geq 0$ be an integer.
For every coherent sheaf $\mathcal{F}$ on $\mathbf{P}^n_R$
we have the following:
\begin{enumerate}
\item There exists an $r \geq 0$ and
$d_1, \ldots, d_r \in \mathbf{Z}$ and a surjection
$$
\bigoplus\nolimits_{j = 1, \ldots, r}
\mathcal{O}_{\mathbf{P}^n_R}(d_j)
\longrightarrow
\mathcal{F}.
$$
\item We have $H^i(\mathbf{P}^n_R, \mathcal{F}) = 0$ unless
$0 \leq i \leq n$.
\item For any $i$ the cohomology group $H^i(\mathbf{P}^n_R, \mathcal{F})$
is a finite $R$-module.
\item If $i > 0$, then
$H^i(\mathbf{P}^n_R, \mathcal{F}(d)) = 0$ for all $d$ large enough.
\item For any $k \in \mathbf{Z}$ the graded $R[T_0, \ldots, T_n]$-module
$$
\bigoplus\nolimits_{d \geq k} H^0(\mathbf{P}^n_R, \mathcal{F}(d))
$$
is a finite $R[T_0, \ldots, T_n]$-module.
\end{enumerate}
\end{lemma}

\begin{proof}
We will use that $\mathcal{O}_{\mathbf{P}^n_R}(1)$ is an ample invertible
sheaf on
the scheme $\mathbf{P}^n_R$. This follows directly from the definition
since $\mathbf{P}^n_R$ covered by the standard affine opens $D_{+}(T_i)$.
Hence by
Properties, Proposition \ref{properties-proposition-characterize-ample}
every finite type quasi-coherent $\mathcal{O}_{\mathbf{P}^n_R}$-module
is a quotient of a finite direct sum of tensor powers of
$\mathcal{O}_{\mathbf{P}^n_R}(1)$. On the other hand a coherent sheaves
and finite type quasi-coherent sheaves are the same thing on projective
space over $R$ by Lemma \ref{lemma-coherent-Noetherian}. Thus we see (1).

\medskip\noindent
Projective $n$-space $\mathbf{P}^n_R$ is covered by $n + 1$ affines,
namely the standard opens $D_{+}(T_i)$, $i = 0, \ldots, n$, see Constructions,
Lemma \ref{constructions-lemma-standard-covering-projective-space}.
Hence we see that for any quasi-coherent
sheaf $\mathcal{F}$ on $\mathbf{P}^n_R$
we have $H^i(\mathbf{P}^n_R, \mathcal{F}) = 0$ for $i \geq n + 1$,
see Lemma \ref{lemma-vanishing-nr-affines}. Hence (2) holds.

\medskip\noindent
Let us prove (3) and (4) simultaneously for all coherent sheaves
on $\mathbf{P}^n_R$ by descending induction on $i$. Clearly the result
holds for $i \geq n + 1$ by (2). Suppose we know the result for
$i + 1$ and we want to show the result for $i$. (If $i = 0$, then
part (4) is vacuous.) Let $\mathcal{F}$ be a coherent sheaf on
$\mathbf{P}^n_R$. Choose a surjection as in (1) and denote
$\mathcal{G}$ the kernel so that we have a short exact sequence
$$
0 \to \mathcal{G} \to
\bigoplus\nolimits_{j = 1, \ldots, r}
\mathcal{O}_{\mathbf{P}^n_R}(d_j)
\to
\mathcal{F} \to 0
$$
By Lemma \ref{lemma-coherent-abelian-Noetherian}
we see that $\mathcal{G}$ is coherent. The long exact
cohomology sequence gives an exact sequence
$$
H^i(\mathbf{P}^n_R, \bigoplus\nolimits_{j = 1, \ldots, r}
\mathcal{O}_{\mathbf{P}^n_R}(d_j))
\to
H^i(\mathbf{P}^n_R, \mathcal{F})
\to
H^{i + 1}(\mathbf{P}^n_R, \mathcal{G}).
$$
By induction assumption the right $R$-module is finite and by
Lemma \ref{lemma-cohomology-projective-space-over-ring} the left
$R$-module is finite. Since $R$ is Noetherian it follows immediately
that $H^i(\mathbf{P}^n_R, \mathcal{F})$ is a finite $R$-module.
This proves the induction step for assertion (3).
Since $\mathcal{O}_{\mathbf{P}^n_R}(d)$ is invertible
we see that twisting on $\mathbf{P}^n_R$ is an exact functor (since
you get it by tensoring with an invertible sheaf, see
Constructions, Definition \ref{constructions-definition-twist}).
This means that for all $d \in \mathbf{Z}$ the sequence
$$
0 \to \mathcal{G}(d) \to
\bigoplus\nolimits_{j = 1, \ldots, r}
\mathcal{O}_{\mathbf{P}^n_R}(d_j + d)
\to
\mathcal{F}(d) \to 0
$$
is short exact. The resulting cohomology sequence is
$$
H^i(\mathbf{P}^n_R, \bigoplus\nolimits_{j = 1, \ldots, r}
\mathcal{O}_{\mathbf{P}^n_R}(d_j + d))
\to
H^i(\mathbf{P}^n_R, \mathcal{F}(d))
\to
H^{i + 1}(\mathbf{P}^n_R, \mathcal{G}(d)).
$$
By induction assumption we see the module on the right is zero
for $d \gg 0$ and by the computation in
Lemma \ref{lemma-cohomology-projective-space-over-ring}
the module on the left is zero as soon as $d \geq -\min\{d_j\}$
and $i \geq 1$. Hence the induction step for assertion (4).
This concludes the proof of (3) and (4).

\medskip\noindent
In order to prove (5) note that for all sufficiently large $d$
the map
$$
H^0(\mathbf{P}^n_R, \bigoplus\nolimits_{j = 1, \ldots, r}
\mathcal{O}_{\mathbf{P}^n_R}(d_j + d))
\to
H^0(\mathbf{P}^n_R, \mathcal{F}(d))
$$
is surjective by the vanishing of $H^1(\mathbf{P}^n_R, \mathcal{G}(d))$
we just proved. In other words, the module
$$
M_k
=
\bigoplus\nolimits_{d \geq k} H^0(\mathbf{P}^n_R, \mathcal{F}(d))
$$
is for $k$ large enough a quotient of the corresponding module
$$
N_k
=
\bigoplus\nolimits_{d \geq k} H^0(\mathbf{P}^n_R,
\bigoplus\nolimits_{j = 1, \ldots, r}
\mathcal{O}_{\mathbf{P}^n_R}(d_j + d)
)
$$
When $k$ is sufficiently small (e.g.\ $k < -d_j$ for all $j$) then
$$
N_k = \bigoplus\nolimits_{j = 1, \ldots, r}
R[T_0, \ldots, T_n](d_j)
$$
by our computations in Section \ref{section-cohomology-projective-space}.
In particular it is finitely generated.
Suppose $k \in \mathbf{Z}$ is arbitrary.
Choose $k_{-} \ll k \ll k_{+}$.
Consider the diagram
$$
\xymatrix{
N_{k_{-}} & N_{k_{+}} \ar[d] \ar[l] \\
M_k & M_{k_{+}} \ar[l]
}
$$
where the vertical arrow is the surjective map above and
the horizontal arrows are the obvious inclusion maps.
By what was said above we see that $N_{k_{-}}$ is a finitely
generated $R[T_0, \ldots, T_n]$-module. Hence $N_{k_{+}}$ is
a finitely generated $R[T_0, \ldots, T_n]$-module because it
is a submodule of a finitely generated module and the ring
$R[T_0, \ldots, T_n]$ is Noetherian. Since the vertical arrow
is surjective we conclude that $M_{k_{+}}$ is a finitely
generated $R[T_0, \ldots, T_n]$-module. The quotient
$M_k/M_{k_{+}}$ is finite as an $R$-module since it is a
finite direct sum of the finite $R$-modules
$H^0(\mathbf{P}^n_R, \mathcal{F}(d))$ for $k \leq d < k_{+}$.
Note that we use part (3) for $i = 0$ here. Hence
$M_k/M_{k_{+}}$ is a fortiori a finite $R[T_0, \ldots, T_n]$-module.
In other words, we have sandwiched $M_k$ between two finite
$R[T_0, \ldots, T_n]$-module and we win.
\end{proof}



































\section{Other chapters}

\begin{multicols}{2}
\begin{enumerate}
\item \hyperref[introduction-section-phantom]{Introduction}
\item \hyperref[conventions-section-phantom]{Conventions}
\item \hyperref[sets-section-phantom]{Set Theory}
\item \hyperref[categories-section-phantom]{Categories}
\item \hyperref[topology-section-phantom]{Topology}
\item \hyperref[sheaves-section-phantom]{Sheaves on Spaces}
\item \hyperref[algebra-section-phantom]{Commutative Algebra}
\item \hyperref[sites-section-phantom]{Sites and Sheaves}
\item \hyperref[homology-section-phantom]{Homological Algebra}
\item \hyperref[derived-section-phantom]{Derived Categories}
\item \hyperref[more-algebra-section-phantom]{More Algebra}
\item \hyperref[simplicial-section-phantom]{Simplicial Methods}
\item \hyperref[modules-section-phantom]{Sheaves of Modules}
\item \hyperref[sites-modules-section-phantom]{Modules on Sites}
\item \hyperref[injectives-section-phantom]{Injectives}
\item \hyperref[cohomology-section-phantom]{Cohomology of Sheaves}
\item \hyperref[sites-cohomology-section-phantom]{Cohomology on Sites}
\item \hyperref[hypercovering-section-phantom]{Hypercoverings}
\item \hyperref[schemes-section-phantom]{Schemes}
\item \hyperref[constructions-section-phantom]{Constructions of Schemes}
\item \hyperref[properties-section-phantom]{Properties of Schemes}
\item \hyperref[morphisms-section-phantom]{Morphisms of Schemes}
\item \hyperref[coherent-section-phantom]{Coherent Cohomology}
\item \hyperref[divisors-section-phantom]{Divisors}
\item \hyperref[limits-section-phantom]{Limits of Schemes}
\item \hyperref[varieties-section-phantom]{Varieties}
\item \hyperref[chow-section-phantom]{Chow Homology}
\item \hyperref[topologies-section-phantom]{Topologies on Schemes}
\item \hyperref[descent-section-phantom]{Descent}
\item \hyperref[more-morphisms-section-phantom]{More on Morphisms}
\item \hyperref[flat-section-phantom]{More on Flatness}
\item \hyperref[groupoids-section-phantom]{Groupoid Schemes}
\item \hyperref[more-groupoids-section-phantom]{More on Groupoid Schemes}
\item \hyperref[etale-section-phantom]{\'Etale Morphisms of Schemes}
\item \hyperref[etale-cohomology-section-phantom]{\'Etale Cohomology}
\item \hyperref[spaces-section-phantom]{Algebraic Spaces}
\item \hyperref[spaces-properties-section-phantom]{Properties of Algebraic Spaces}
\item \hyperref[spaces-morphisms-section-phantom]{Morphisms of Algebraic Spaces}
\item \hyperref[spaces-topologies-section-phantom]{Topologies on Algebraic Spaces}
\item \hyperref[spaces-descent-section-phantom]{Descent and Algebraic Spaces}
\item \hyperref[spaces-more-morphisms-section-phantom]{More on Morphisms of Spaces}
\item \hyperref[quot-section-phantom]{Quot and Hilbert Spaces}
\item \hyperref[stacks-section-phantom]{Stacks}
\item \hyperref[spaces-groupoids-section-phantom]{Groupoids in Algebraic Spaces}
\item \hyperref[spaces-more-groupoids-section-phantom]{More on Groupoids in Spaces}
\item \hyperref[bootstrap-section-phantom]{Bootstrap}
\item \hyperref[examples-stacks-section-phantom]{Examples of Stacks}
\item \hyperref[groupoids-quotients-section-phantom]{Quotients of Groupoids}
\item \hyperref[algebraic-section-phantom]{Algebraic Stacks}
\item \hyperref[criteria-section-phantom]{Criteria for Representability}
\item \hyperref[stacks-properties-section-phantom]{Properties of Algebraic Stacks}
\item \hyperref[stacks-morphisms-section-phantom]{Morphisms of Algebraic Stacks}
\item \hyperref[examples-section-phantom]{Examples}
\item \hyperref[exercises-section-phantom]{Exercises}
\item \hyperref[guide-section-phantom]{Guide to Literature}
\item \hyperref[desirables-section-phantom]{Desirables}
\item \hyperref[coding-section-phantom]{Coding Style}
\item \hyperref[fdl-section-phantom]{GNU Free Documentation License}
\item \hyperref[index-section-phantom]{Auto Generated Index}
\end{enumerate}
\end{multicols}


\bibliography{my}
\bibliographystyle{alpha}

\end{document}
