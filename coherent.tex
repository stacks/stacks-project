\IfFileExists{stacks-project.cls}{%
\documentclass{stacks-project}
}{%
\documentclass{amsart}
}

% The following AMS packages are automatically loaded with
% the amsart documentclass:
%\usepackage{amsmath}
%\usepackage{amssymb}
%\usepackage{amsthm}

% For dealing with references we use the comment environment
\usepackage{verbatim}
\newenvironment{reference}{\comment}{\endcomment}
%\newenvironment{reference}{}{}
\newenvironment{slogan}{\comment}{\endcomment}
\newenvironment{history}{\comment}{\endcomment}

% For commutative diagrams you can use
% \usepackage{amscd}
\usepackage[all]{xy}

% We use 2cell for 2-commutative diagrams.
\xyoption{2cell}
\UseAllTwocells

% To put source file link in headers.
% Change "template.tex" to "this_filename.tex"
% \usepackage{fancyhdr}
% \pagestyle{fancy}
% \lhead{}
% \chead{}
% \rhead{Source file: \url{template.tex}}
% \lfoot{}
% \cfoot{\thepage}
% \rfoot{}
% \renewcommand{\headrulewidth}{0pt}
% \renewcommand{\footrulewidth}{0pt}
% \renewcommand{\headheight}{12pt}

\usepackage{multicol}

% For cross-file-references
\usepackage{xr-hyper}

% Package for hypertext links:
\usepackage{hyperref}

% For any local file, say "hello.tex" you want to link to please
% use \externaldocument[hello-]{hello}
\externaldocument[introduction-]{introduction}
\externaldocument[conventions-]{conventions}
\externaldocument[sets-]{sets}
\externaldocument[categories-]{categories}
\externaldocument[topology-]{topology}
\externaldocument[sheaves-]{sheaves}
\externaldocument[sites-]{sites}
\externaldocument[stacks-]{stacks}
\externaldocument[fields-]{fields}
\externaldocument[algebra-]{algebra}
\externaldocument[brauer-]{brauer}
\externaldocument[homology-]{homology}
\externaldocument[derived-]{derived}
\externaldocument[simplicial-]{simplicial}
\externaldocument[more-algebra-]{more-algebra}
\externaldocument[smoothing-]{smoothing}
\externaldocument[modules-]{modules}
\externaldocument[sites-modules-]{sites-modules}
\externaldocument[injectives-]{injectives}
\externaldocument[cohomology-]{cohomology}
\externaldocument[sites-cohomology-]{sites-cohomology}
\externaldocument[dga-]{dga}
\externaldocument[dpa-]{dpa}
\externaldocument[hypercovering-]{hypercovering}
\externaldocument[schemes-]{schemes}
\externaldocument[constructions-]{constructions}
\externaldocument[properties-]{properties}
\externaldocument[morphisms-]{morphisms}
\externaldocument[coherent-]{coherent}
\externaldocument[divisors-]{divisors}
\externaldocument[limits-]{limits}
\externaldocument[varieties-]{varieties}
\externaldocument[topologies-]{topologies}
\externaldocument[descent-]{descent}
\externaldocument[perfect-]{perfect}
\externaldocument[more-morphisms-]{more-morphisms}
\externaldocument[flat-]{flat}
\externaldocument[groupoids-]{groupoids}
\externaldocument[more-groupoids-]{more-groupoids}
\externaldocument[etale-]{etale}
\externaldocument[chow-]{chow}
\externaldocument[intersection-]{intersection}
\externaldocument[pic-]{pic}
\externaldocument[adequate-]{adequate}
\externaldocument[dualizing-]{dualizing}
\externaldocument[duality-]{duality}
\externaldocument[discriminant-]{discriminant}
\externaldocument[local-cohomology-]{local-cohomology}
\externaldocument[curves-]{curves}
\externaldocument[resolve-]{resolve}
\externaldocument[models-]{models}
\externaldocument[pione-]{pione}
\externaldocument[etale-cohomology-]{etale-cohomology}
\externaldocument[proetale-]{proetale}
\externaldocument[crystalline-]{crystalline}
\externaldocument[spaces-]{spaces}
\externaldocument[spaces-properties-]{spaces-properties}
\externaldocument[spaces-morphisms-]{spaces-morphisms}
\externaldocument[decent-spaces-]{decent-spaces}
\externaldocument[spaces-cohomology-]{spaces-cohomology}
\externaldocument[spaces-limits-]{spaces-limits}
\externaldocument[spaces-divisors-]{spaces-divisors}
\externaldocument[spaces-over-fields-]{spaces-over-fields}
\externaldocument[spaces-topologies-]{spaces-topologies}
\externaldocument[spaces-descent-]{spaces-descent}
\externaldocument[spaces-perfect-]{spaces-perfect}
\externaldocument[spaces-more-morphisms-]{spaces-more-morphisms}
\externaldocument[spaces-flat-]{spaces-flat}
\externaldocument[spaces-groupoids-]{spaces-groupoids}
\externaldocument[spaces-more-groupoids-]{spaces-more-groupoids}
\externaldocument[bootstrap-]{bootstrap}
\externaldocument[spaces-pushouts-]{spaces-pushouts}
\externaldocument[groupoids-quotients-]{groupoids-quotients}
\externaldocument[spaces-more-cohomology-]{spaces-more-cohomology}
\externaldocument[spaces-simplicial-]{spaces-simplicial}
\externaldocument[formal-spaces-]{formal-spaces}
\externaldocument[restricted-]{restricted}
\externaldocument[spaces-resolve-]{spaces-resolve}
\externaldocument[formal-defos-]{formal-defos}
\externaldocument[defos-]{defos}
\externaldocument[cotangent-]{cotangent}
\externaldocument[examples-defos-]{examples-defos}
\externaldocument[algebraic-]{algebraic}
\externaldocument[examples-stacks-]{examples-stacks}
\externaldocument[stacks-sheaves-]{stacks-sheaves}
\externaldocument[criteria-]{criteria}
\externaldocument[artin-]{artin}
\externaldocument[quot-]{quot}
\externaldocument[stacks-properties-]{stacks-properties}
\externaldocument[stacks-morphisms-]{stacks-morphisms}
\externaldocument[stacks-limits-]{stacks-limits}
\externaldocument[stacks-cohomology-]{stacks-cohomology}
\externaldocument[stacks-perfect-]{stacks-perfect}
\externaldocument[stacks-introduction-]{stacks-introduction}
\externaldocument[stacks-more-morphisms-]{stacks-more-morphisms}
\externaldocument[stacks-geometry-]{stacks-geometry}
\externaldocument[moduli-]{moduli}
\externaldocument[moduli-curves-]{moduli-curves}
\externaldocument[examples-]{examples}
\externaldocument[exercises-]{exercises}
\externaldocument[guide-]{guide}
\externaldocument[desirables-]{desirables}
\externaldocument[coding-]{coding}
\externaldocument[obsolete-]{obsolete}
\externaldocument[fdl-]{fdl}
\externaldocument[index-]{index}

% Theorem environments.
%
\theoremstyle{plain}
\newtheorem{theorem}[subsection]{Theorem}
\newtheorem{proposition}[subsection]{Proposition}
\newtheorem{lemma}[subsection]{Lemma}

\theoremstyle{definition}
\newtheorem{definition}[subsection]{Definition}
\newtheorem{example}[subsection]{Example}
\newtheorem{exercise}[subsection]{Exercise}
\newtheorem{situation}[subsection]{Situation}

\theoremstyle{remark}
\newtheorem{remark}[subsection]{Remark}
\newtheorem{remarks}[subsection]{Remarks}

\numberwithin{equation}{subsection}

% Macros
%
\def\lim{\mathop{\rm lim}\nolimits}
\def\colim{\mathop{\rm colim}\nolimits}
\def\Spec{\mathop{\rm Spec}}
\def\Hom{\mathop{\rm Hom}\nolimits}
\def\Ext{\mathop{\rm Ext}\nolimits}
\def\SheafHom{\mathop{\mathcal{H}\!{\it om}}\nolimits}
\def\SheafExt{\mathop{\mathcal{E}\!{\it xt}}\nolimits}
\def\Sch{\textit{Sch}}
\def\Mor{\mathop{\rm Mor}\nolimits}
\def\Ob{\mathop{\rm Ob}\nolimits}
\def\Sh{\mathop{\textit{Sh}}\nolimits}
\def\NL{\mathop{N\!L}\nolimits}
\def\proetale{{pro\text{-}\acute{e}tale}}
\def\etale{{\acute{e}tale}}
\def\QCoh{\textit{QCoh}}
\def\Ker{\mathop{\rm Ker}}
\def\Im{\mathop{\rm Im}}
\def\Coker{\mathop{\rm Coker}}
\def\Coim{\mathop{\rm Coim}}

%
% Macros for moduli stacks/spaces
%
\def\QCohstack{\mathcal{QC}\!{\it oh}}
\def\Cohstack{\mathcal{C}\!{\it oh}}
\def\Spacesstack{\mathcal{S}\!{\it paces}}
\def\Quotfunctor{{\rm Quot}}
\def\Hilbfunctor{{\rm Hilb}}
\def\Curvesstack{\mathcal{C}\!{\it urves}}
\def\Polarizedstack{\mathcal{P}\!{\it olarized}}
\def\Complexesstack{\mathcal{C}\!{\it omplexes}}
% \Pic is the operator that assigns to X its picard group, usage \Pic(X)
% \Picardstack_{X/B} denotes the Picard stack of X over B
% \Picardfunctor_{X/B} denotes the Picard functor of X over B
\def\Pic{\mathop{\rm Pic}\nolimits}
\def\Picardstack{\mathcal{P}\!{\it ic}}
\def\Picardfunctor{{\rm Pic}}
\def\Deformationcategory{\mathcal{D}\!{\it ef}}


% OK, start here.
%
\begin{document}

\title{Coherent Cohomology}


\maketitle

\phantomsection
\label{section-phantom}

\tableofcontents

\section{Introduction}
\label{section-introduction}

\noindent
The title of this chapter is a bit of a lie because we
will first prove a number of results on the cohomology of quasi-coherent
sheaves. Having done this we wil elaborate on cohomology of
coherent sheaves in the Noetherian setting.
A fundamental reference is \cite{EGA}.



\section{Cech cohomology of quasi-coherent sheaves}
\label{section-cech-quasi-coherent}

\noindent
Let $X$ be a scheme.
Let $U \subset X$ be an affine open.
Recall that a {\it standard open covering} of $U$ is a covering
of the form $\mathcal{U} : U = \bigcup_{i = 1}^n D(f_i)$
where $f_1, \ldots, f_n \in \Gamma(U, \mathcal{O}_X)$ generate
the unit ideal, see
Schemes, Definition \ref{schemes-definition-standard-covering}.

\begin{lemma}
\label{lemma-cech-cohomology-quasi-coherent-trivial}
Let $X$ be a scheme.
Let $\mathcal{F}$ be a quasi-coherent $\mathcal{O}_X$-module.
Let $\mathcal{U} : U = \bigcup_{i = 1}^n D(f_i)$ be a standard
open covering of an affine open of $X$.
Then $\check{H}^p(\mathcal{U}, \mathcal{F}) = 0$ for
all $p > 0$.
\end{lemma}

\begin{proof}
Write $U = \text{Spec}(A)$ for some ring $A$.
In other words, $f_1, \ldots, f_n$ are elements of $A$
which generate the unit ideal of $A$.
Write $\mathcal{F}|_U = \widetilde{M}$ for some $A$-module $M$.
Clearly the Cech complex
$\check{\mathcal{C}}^\bullet(\mathcal{U}, \mathcal{F})$
is identified with the complex
$$
\prod_{i_0} M_{f_{i_0}} \to
\prod_{i_0i_1} M_{f_{i_0}f_{i_1}} \to
\prod_{i_0i_1i_2} M_{f_{i_0}f_{i_1}f_{i_2}} \to
\ldots
$$
We are asked to show that the extended complex
\begin{equation}
\label{equation-extended}
0 \to
M \to
\prod_{i_0} M_{f_{i_0}} \to
\prod_{i_0i_1} M_{f_{i_0}f_{i_1}} \to
\prod_{i_0i_1i_2} M_{f_{i_0}f_{i_1}f_{i_2}} \to
\ldots
\end{equation}
(whose truncation we have studied in
Algebra, Lemma \ref{algebra-lemma-cover-module}) is exact.
It suffices to show that (\ref{equation-extended})
is exact after localizing at a prime $\mathfrak p$, see
Algebra, Lemma \ref{algebra-lemma-characterize-zero-local}.
In fact we will show that the extended complex localized
at $\mathfrak p$ is homotopic to zero.

\medskip\noindent
There exists an index $i$ such that $f_i \not \in \mathfrak p$.
Choose and fix such an element $i_{\text{fix}}$. Note that
$M_{f_{i_{\text{fix}}}, \mathfrak p} = M_{\mathfrak p}$. Similarly
for a localization at a product $f_{i_0} \ldots f_{i_p}$ and $\mathfrak p$
we can drop any $f_{i_j}$ for which $i_j = i_{\text{fix}}$.
Let us define a homotopy
$$
h :
\prod\nolimits_{i_0 \ldots i_{p + 1}}
M_{f_{i_0} \ldots f_{i_{p + 1}}, \mathfrak p}
\longrightarrow
\prod\nolimits_{i_0 \ldots i_p}
M_{f_{i_0} \ldots f_{i_p}, \mathfrak p}
$$
by the rule
$$
h(s)_{i_0 \ldots i_p} = s_{i_{\text{fix}} i_0 \ldots i_p}
$$
(This is ``dual'' to the homotopy in the proof of
Cohomology, Lemma \ref{cohomology-lemma-homology-complex}.)
In other words, $h : \prod_{i_0} M_{f_{i_0}, \mathfrak p} \to M$
is projection onto the factor
$M_{f_{i_{\text{fix}}}, \mathfrak p} = M_{\mathfrak p}$ and in general
the map $h$ equal projection onto the factors
$M_{f_{i_{\text{fix}}} f_{i_1} \ldots f_{i_{p + 1}}, \mathfrak p}
= M_{f_{i_1} \ldots f_{i_{p + 1}}, \mathfrak p}$. We compute
\begin{align*}
(dh + hd)(s)_{i_0 \ldots i_p}
= &
\sum_{j = 0}^p
(-1)^j
h(s)_{i_0 \ldots \hat i_j \ldots i_p}
+
d(s)_{i_{\text{fix}} i_0 \ldots i_p}\\
= &
\sum_{j = 0}^p
(-1)^j
s_{i_{\text{fix}} i_0 \ldots \hat i_j \ldots i_p}
+
s_{i_0 \ldots i_p}
+
\sum_{j = 0}^p
(-1)^{j + 1}
s_{i_{\text{fix}} i_0 \ldots \hat i_j \ldots i_p} \\
= &
s_{i_0 \ldots i_p}
\end{align*}
This proves the identity map is homotopic to zero as desired.
\end{proof}

\begin{lemma}
\label{lemma-quasi-coherent-affine-cohomology-zero}
Let $X$ be a scheme.
Let $\mathcal{F}$ be a quasi-coherent $\mathcal{O}_X$-module.
For any affine open $U \subset X$ we have
$H^p(U, \mathcal{F}) = 0$ for all $p > 0$.
\end{lemma}

\begin{proof}
We are going to apply
Cohomology, Lemma \ref{cohomology-lemma-cech-vanish-basis}.
As our basis $\mathcal{B}$ for the topology of $X$ we are going to use
the affine opens of $X$.
As our set $\text{Cov}$ of open coverings we are going to use the standard
open coverings of affine opens of $X$.
Next we check that conditions (1), (2) and (3) of
Cohomology, Lemma \ref{cohomology-lemma-cech-vanish-basis}
hold. Note that the intersection of standard opens in an affine is
another standard open. Hence property (1) holds. 
The coverings form a cofinal system of open coverings of any element
of $\mathcal{B}$, see
Schemes, Lemma \ref{schemes-lemma-standard-open}.
Hence (2) holds.
Finally, condition (3) of the lemma follows from
Lemma \ref{lemma-cech-cohomology-quasi-coherent-trivial}.
\end{proof}

\begin{lemma}
\label{lemma-relative-affine-vanishing}
Let $f : X \to S$ be a morphism of schemes.
Let $\mathcal{F}$ be a quasi-coherent $\mathcal{O}_X$-module.
If $f$ is affine then $R^if_*\mathcal{F} = 0$ for all $i > 0$.
\end{lemma}

\begin{proof}
According to
Cohomology, Lemma \ref{cohomology-lemma-describe-higher-direct-images}
the sheaf
$R^if_*\mathcal{F}$ is the sheaf associated to the presheaf
$V \mapsto H^i(f^{-1}(V), \mathcal{F}|_{f^{-1}(V)})$.
By assumption, whenever $V$ is affine we have that $f^{-1}(V)$ is
affine, see Morphisms, Definition \ref{morphisms-definition-affine}.
By Lemma \ref{lemma-quasi-coherent-affine-cohomology-zero} we conclude that
$H^i(f^{-1}(V), \mathcal{F}|_{f^{-1}(V)}) = 0$
whenever $V$ is affine. Since $S$ has a basis consisting of affine
opens we win.
\end{proof}

\begin{lemma}
\label{lemma-cech-cohomology-quasi-coherent}
Let $X$ be a scheme.
Let $\mathcal{U} : X = \bigcup_{i \in I} U_i$ be an open covering such that
$U_{i_0 \ldots i_p}$ is affine open for all $p \ge 0$ and all
$i_0, \ldots, i_p \in I$
In this case for any quasi-coherent sheaf $\mathcal{F}$ we have
$$
\check{H}^p(\mathcal{U}, \mathcal{F}) = H^p(X, \mathcal{F})
$$
for all $p$.
\end{lemma}

\begin{proof}
In view of Lemma \ref{lemma-quasi-coherent-affine-cohomology-zero}
this is a special case of
Comology, Lemma \ref{cohomology-lemma-cech-spectral-sequence-application}.
\end{proof}















\section{Quasi-coherence of higher direct images}
\label{section-quasi-coherence}

\begin{lemma}
\label{lemma-vanishing-nr-affines}
Let $X$ be a quasi-compact separated scheme.
Let $t \geq 1$ be the minimal number of affine opens needed to
cover $X$.
Then $H^n(X, \mathcal{F}) = 0$ for all $n \geq t$ and all
quasi-coherent sheaves $\mathcal{F}$.
\end{lemma}

\begin{proof}
First proof.
By induction on $t$.
If $t = 1$ the result follows from
Lemma \ref{lemma-quasi-coherent-affine-cohomology-zero}.
If $t > 1$ write $X = U \cup V$ with $V$ affine open and
$U = U_1 \cup \ldots \cup U_{t - 1}$ a union of $t - 1$ open affines.
Note that in this case
$U \cap V =  (U_1 \cap V) \cup \ldots (U_{t - 1} \cap V)$
is also a union of $t - 1$ affine open subschemes, see
Schemes, Lemma \ref{schemes-lemma-characterize-separated}.
We apply the Mayer-Vietoris long exact sequence
$$
0 \to
H^0(X, \mathcal{F}) \to
H^0(U, \mathcal{F}) \oplus H^0(V, \mathcal{F}) \to
H^0(U \cap V, \mathcal{F}) \to
H^1(X, \mathcal{F}) \to \ldots
$$
see Cohomology, Lemma \ref{cohomology-lemma-mayer-vietoris}.
By induction we see that $H^i(U, \mathcal{F})$,
$H^i(U, \mathcal{F})$, $H^i(U, \mathcal{F})$ are zero for
$i \geq t - 1$. It follows immediately that $H^i(X, \mathcal{F})$
is zero for $i \geq t$.

\medskip\noindent
Second proof. 
Let $\mathcal{U} : X = \bigcup_{i = 1}^t U_i$ be a finite affine open
covering. Since $X$ is separated the multiple intersections
$U_{i_0 \ldots i_p}$ are all affine, see 
Schemes, Lemma \ref{schemes-lemma-characterize-separated}.
By Lemma \ref{lemma-cech-cohomology-quasi-coherent} the Cech
cohomology groups $\check{H}^p(\mathcal{U}, \mathcal{F})$
agree with the cohomology groups. By
Cohomology, Lemma \ref{cohomology-lemma-alternating-usual}
the Cech cohomology groups may be computed using the alternating
Cech complex $\check{\mathcal{C}}_{alt}^\bullet(\mathcal{U}, \mathcal{F})$.
As the covering consists of $t$ elements we see immediately
that $\check{\mathcal{C}}_{alt}^p(\mathcal{U}, \mathcal{F}) = 0$
for all $p \geq t$. Hence the result follows.
\end{proof}

\begin{lemma}
\label{lemma-quasi-coherence-higher-direct-images}
Let $f : X \to S$ be a morphism of schemes.
Assume that $f$ is quasi-separated and quasi-compact.
\begin{enumerate}
\item For any quasi-coherent $\mathcal{O}_X$-module $\mathcal{F}$ the
higher direct images $R^pf_*\mathcal{F}$ are quasi-coherent on $S$.
\item If $S$ is quasi-compact, there exists an integer $n = n(X, S, f)$
such that $R^pf_*\mathcal{F} = 0$ for all $p \geq n$ and any
quasi-coherent sheaf $\mathcal{F}$ on $X$.
\end{enumerate}
\end{lemma}

\begin{proof}
Note that under the hypotheses of the lemma the sheaf
$R^0f_*\mathcal{F} = f_*\mathcal{F}$ is quasi-coherent by
Schemes, Lemma \ref{schemes-lemma-push-forward-quasi-coherent}.

\medskip\noindent
We will repeatedly use
Cohomology, Lemma \ref{cohomology-lemma-localize-higher-direct-images}
to see that forming higher direct images commutes with restriction.
Being quasi-coherent is local on $S$ and in order to prove (1) we
may assume $S$ is affine.

\medskip\noindent
Suppose $S = \bigcup_{i = 1, \ldots m} S_i$ is a finite affine open covering.
Assume $n_i$ works in (2) for the morphism $f^{-1}(S_i) \to S_i$.
Then we see that $n = \max\{n_i\}$ works for $X \to S$ in (2).
Hence we may assume $S$ is affine in order to prove (2) as well.

\medskip\noindent
Assume $S$ is affine and $f$ quasi-compact and separated.
Let $t \geq 1$ be the minimal number of affine opens needed to cover $X$.
We will prove this case of the lemma by induction on $t$ and we will in
addition show that setting $n = t$ in (2) works.
If $t = 1$ then the morphism $f$ is affine by
Morphisms, Lemma \ref{morphisms-lemma-morphism-affines-affine}
and (1) and (2) (with $n = 1$)
both follow from Lemma \ref{lemma-relative-affine-vanishing}.
If $t > 1$ write $X = U \cup V$ with $V$ affine open and
$U = U_1 \cup \ldots \cup U_{t - 1}$ a union of $t - 1$ open affines.
Note that in this case
$U \cap V =  (U_1 \cap V) \cup \ldots (U_{t - 1} \cap V)$
is also a union of $t - 1$ affine open subschemes, see
Schemes, Lemma \ref{schemes-lemma-characterize-separated}.
We will apply the relative Mayer-Vietoris sequence
$$
0 \to
f_*\mathcal{F} \to
a_*(\mathcal{F}|_U) \oplus b_*(\mathcal{F}|_V) \to
c_*(\mathcal{F}|_{U \cap V}) \to
R^1f_*\mathcal{F} \to \ldots
$$
see Cohomology, Lemma \ref{cohomology-lemma-relative-mayer-vietoris}.
By induction we see that
$R^pa_*\mathcal{F}$, $R^pb_*\mathcal{F}$ and $R^pc_*\mathcal{F}$
are all quasi-coherent and zero when $p \geq t - 1$. This immediately
implies the desired vanishing. It also implies that each of sheaves
$R^pf_*\mathcal{F}$ is quasi-coherent since it sits in the middle of a short
exact sequence with a cokernel of a map between quasi-coherent sheaves
on the left and a kernel of a map between quasi-coherent sheaves on the right.
Use the results on quasi-coherent sheaves in
Schemes, Section \ref{schemes-section-quasi-coherent} to see
this is sufficient to conclude.

\medskip\noindent
Assume $S$ is affine and $f$ quasi-compact and quasi-separated.
Let $t \geq 1$ be the minimal number of affine opens needed to cover $X$.
We will prove this case of lemma by induction on $t$. We have seen above
that this will imply the lemma in general.
If $t = 1$ then the morphism $f$ is affine by
Morphisms, Lemma \ref{morphisms-lemma-morphism-affines-affine}
and (1) and (2) (with $n = 1$)
both follow from Lemma \ref{lemma-relative-affine-vanishing}.
If $t > 1$ write $X = U \cup V$ with $V$ affine open and
$U$ a union of $t - 1$ open affines.
Note that in this case $U \cap V$ is an open subscheme of an affine
scheme and hence separated (see
Schemes, Lemma \ref{schemes-lemma-affine-separated}).
We will apply the relative Mayer-Vietoris sequence
$$
0 \to
f_*\mathcal{F} \to
a_*(\mathcal{F}|_U) \oplus b_*(\mathcal{F}|_V) \to
c_*(\mathcal{F}|_{U \cap V}) \to
R^1f_*\mathcal{F} \to \ldots
$$
see Cohomology, Lemma \ref{cohomology-lemma-relative-mayer-vietoris}.
By induction there exist integers $n_a, n_b, n_c$ such that
$R^pa_*\mathcal{F}$, $R^pb_*\mathcal{F}$ and $R^pc_*\mathcal{F}$
are all quasi-coherent and zero when $p \geq \max\{n_a, n_b, n_c\}$.
This immediately implies the vanishing in (2) with
$n = \max\{n_a, n_b, n_c\} + 1$. It also implies that each of sheaves
$R^pf_*\mathcal{F}$ is quasi-coherent since it sits in the middle of a short
exact sequence with a cokernel of a map between quasi-coherent sheaves
on the left and a kernel of a map between quasi-coherent sheaves on the right.
Use the results on quasi-coherent sheaves in
Schemes, Section \ref{schemes-section-quasi-coherent} to see
this is sufficient to conclude.
\end{proof}







\section{Coherent sheaves on Noetherian schemes}
\label{section-coherent-sheaves}

\noindent
Allthough it is possible to consider coherent sheaves on non-Noetherian
schemes we will almost always assume the base scheme is Noetherian when
we consider coherent sheaves. Here is a characterization of coherent
sheaves on Noetherian schemes.

\begin{lemma}
\label{lemma-coherent-Noetherian}
Let $X$ be a locally Noetherian scheme.
Let $\mathcal{F}$ be an $\mathcal{O}_X$-module.
The following are equivalent
\begin{enumerate}
\item $\mathcal{F}$ is coherent,
\item $\mathcal{F}$ is a quasi-coherent, finite type $\mathcal{O}_X$-module,
\item $\mathcal{F}$ is a finitely presented $\mathcal{O}_X$-module,
\item for any affine open $\text{Spec}(A) = U \subset X$ we have
$\mathcal{F}|_U = \widetilde M$ with $M$ a finite $A$-module, and
\item there exists an affine open covering $X = \bigcup U_i$,
$U_i = \text{Spec}(A_i)$ such that each
$\mathcal{F}|_{U_i} = \widetilde M_i$ with $M_i$ a finite $A_i$-module.
\end{enumerate}
In particular $\mathcal{O}_X$ is coherent, any invertible
$\mathcal{O}_X$-module is coherent, and more generally any
finite locally free $\mathcal{O}_X$-module is invertible.
\end{lemma}

\begin{proof}
The implications (1) $\Rightarrow$ (2) and (1) $\Rightarrow$ (3) hold
in general, see
Modules, Lemma \ref{modules-lemma-coherent-finite-presentation}.
If $\mathcal{F}$ is finitely presented then $\mathcal{F}$ is
quasi-coherent, see
Modules, Lemma \ref{modules-lemma-finite-presentation-quasi-coherent}.
Hence also (3) $\Rightarrow$ (2).

\medskip\noindent
Assume $\mathcal{F}$ is a quasi-coherent, finite type $\mathcal{O}_X$-module.
By
Properties, Lemma \ref{properties-lemma-finite-type-module}
we see that on any affine open
$\text{Spec}(A) = U \subset X$ we have $\mathcal{F}|_U = \widetilde M$
with $M$ a finite $A$-module. Since $A$ is Noetherian we see that
$M$ has a finite resolution
$$
A^{\oplus m} \to A^{\oplus n} \to M \to 0.
$$
Hence $\mathcal{F}$ is of finite presentation by
Properties, Lemma \ref{properties-lemma-finite-presentation-module}.
In other words (2) $\Rightarrow$ (3).

\medskip\noindent
By Modules, Lemma \ref{modules-lemma-coherent-structure-sheaf} it suffices
to show that $\mathcal{O}_X$ is coherent in order to show that (3)
implies (1). Thus we have to show: given any open $U \subset X$ and
any finite collection of sections $f_i \in \mathcal{O}_X(U)$,
$i = 1, \ldots, n$ the kernel of the map
$\bigoplus_{i = 1, \ldots, n} \mathcal{O}_U \to \mathcal{O}_U$
is of finite type. Since being of finite type is a local property
it suffices to check this in a neighbourhood of any $x \in U$.
Thus we may assume $U = \text{Spec}(A)$ is affine. In this case
$f_1, \ldots, f_n \in A$ are elements of $A$. Since $A$ is
Noetherian, see
Properties, Lemma \ref{properties-lemma-locally-Noetherian}
the kernel $K$ of the map $\bigoplus_{i = 1, \ldots, n} A \to A$
is a finite $A$-module. See for example
Algebra, Lemma \ref{algebra-lemma-Noetherian-basic}.
As the functor\ $\widetilde{ }$\ is exact, see
Schemes, Lemma \ref{schemes-lemma-spec-sheaves}
we get an exact sequence
$$
\widetilde K \to
\bigoplus\nolimits_{i = 1, \ldots, n} \mathcal{O}_U \to
\mathcal{O}_U
$$
and by
Properties, Lemma \ref{properties-lemma-finite-type-module}
again we see that $\widetilde K$ is of finite type. We conclude
that (1), (2) and (3) are all equivalent.

\medskip\noindent
It follows from
Properties, Lemma \ref{properties-lemma-finite-type-module}
that (2) implies (4). It is trivial that (4) implies (5).
The discussion in
Schemes, Section \ref{schemes-section-quasi-coherent}
show that (5) implies
that $\mathcal{F}$ is quasi-coherent and it is clear that (5)
implies that $\mathcal{F}$ is of finite type. Hence (5) implies
(2) and we win.
\end{proof}

\begin{lemma}
\label{lemma-coherent-abelian-Noetherian}
Let $X$ be a locally Noetherian scheme.
The category of coherent $\mathcal{O}_X$-modules is abelian.
More precisely, the kernel and cokernel of a map of coherent
$\mathcal{O}_X$-modules are coherent. Any extension
of coherent sheaves is coherent.
\end{lemma}

\begin{proof}
This is simply a restatement of
Modules, Lemma \ref{modules-lemma-coherent-abelian}
in a particular case.
\end{proof}

\noindent
The following lemma does not always hold for the category of coherent
$\mathcal{O}_X$-modules on a general ringed space $X$.

\begin{lemma}
\label{lemma-coherent-Noetherian-quasi-coherent-sub-quotient}
Let $X$ be a locally Noetherian scheme.
Let $\mathcal{F}$ be a coherent $\mathcal{O}_X$-module.
Any quasi-coherent submodule of $\mathcal{F}$ is coherent.
Any quasi-coherent quotient module of $\mathcal{F}$ is coherent.
\end{lemma}

\begin{proof}
We may assume that $X$ is affine, say $X = \text{Spec}(A)$.
Properties, Lemma \ref{properties-lemma-locally-Noetherian}
implies that $A$ is Noetherian. The algebraic counter part of
the lemma is that a quotient, or a submodule of a finite $A$-module
is a finite $A$-module, see for example
Algebra, Section \ref{algebra-lemma-Noetherian-basic}.
\end{proof}










\section{Other chapters}

\begin{multicols}{2}
\begin{enumerate}
\item \hyperref[introduction-section-phantom]{Introduction}
\item \hyperref[conventions-section-phantom]{Conventions}
\item \hyperref[sets-section-phantom]{Set Theory}
\item \hyperref[categories-section-phantom]{Categories}
\item \hyperref[topology-section-phantom]{Topology}
\item \hyperref[sheaves-section-phantom]{Sheaves on Spaces}
\item \hyperref[algebra-section-phantom]{Commutative Algebra}
\item \hyperref[sites-section-phantom]{Sites and Sheaves}
\item \hyperref[homology-section-phantom]{Homological Algebra}
\item \hyperref[derived-section-phantom]{Derived Categories}
\item \hyperref[more-algebra-section-phantom]{More Algebra}
\item \hyperref[simplicial-section-phantom]{Simplicial Methods}
\item \hyperref[modules-section-phantom]{Sheaves of Modules}
\item \hyperref[sites-modules-section-phantom]{Modules on Sites}
\item \hyperref[injectives-section-phantom]{Injectives}
\item \hyperref[cohomology-section-phantom]{Cohomology of Sheaves}
\item \hyperref[sites-cohomology-section-phantom]{Cohomology on Sites}
\item \hyperref[hypercovering-section-phantom]{Hypercoverings}
\item \hyperref[schemes-section-phantom]{Schemes}
\item \hyperref[constructions-section-phantom]{Constructions of Schemes}
\item \hyperref[properties-section-phantom]{Properties of Schemes}
\item \hyperref[morphisms-section-phantom]{Morphisms of Schemes}
\item \hyperref[coherent-section-phantom]{Coherent Cohomology}
\item \hyperref[divisors-section-phantom]{Divisors}
\item \hyperref[limits-section-phantom]{Limits of Schemes}
\item \hyperref[varieties-section-phantom]{Varieties}
\item \hyperref[chow-section-phantom]{Chow Homology}
\item \hyperref[topologies-section-phantom]{Topologies on Schemes}
\item \hyperref[descent-section-phantom]{Descent}
\item \hyperref[more-morphisms-section-phantom]{More on Morphisms}
\item \hyperref[flat-section-phantom]{More on Flatness}
\item \hyperref[groupoids-section-phantom]{Groupoid Schemes}
\item \hyperref[more-groupoids-section-phantom]{More on Groupoid Schemes}
\item \hyperref[etale-section-phantom]{\'Etale Morphisms of Schemes}
\item \hyperref[etale-cohomology-section-phantom]{\'Etale Cohomology}
\item \hyperref[spaces-section-phantom]{Algebraic Spaces}
\item \hyperref[spaces-properties-section-phantom]{Properties of Algebraic Spaces}
\item \hyperref[spaces-morphisms-section-phantom]{Morphisms of Algebraic Spaces}
\item \hyperref[spaces-topologies-section-phantom]{Topologies on Algebraic Spaces}
\item \hyperref[spaces-descent-section-phantom]{Descent and Algebraic Spaces}
\item \hyperref[spaces-more-morphisms-section-phantom]{More on Morphisms of Spaces}
\item \hyperref[quot-section-phantom]{Quot and Hilbert Spaces}
\item \hyperref[stacks-section-phantom]{Stacks}
\item \hyperref[spaces-groupoids-section-phantom]{Groupoids in Algebraic Spaces}
\item \hyperref[spaces-more-groupoids-section-phantom]{More on Groupoids in Spaces}
\item \hyperref[bootstrap-section-phantom]{Bootstrap}
\item \hyperref[examples-stacks-section-phantom]{Examples of Stacks}
\item \hyperref[groupoids-quotients-section-phantom]{Quotients of Groupoids}
\item \hyperref[algebraic-section-phantom]{Algebraic Stacks}
\item \hyperref[criteria-section-phantom]{Criteria for Representability}
\item \hyperref[stacks-properties-section-phantom]{Properties of Algebraic Stacks}
\item \hyperref[stacks-morphisms-section-phantom]{Morphisms of Algebraic Stacks}
\item \hyperref[examples-section-phantom]{Examples}
\item \hyperref[exercises-section-phantom]{Exercises}
\item \hyperref[guide-section-phantom]{Guide to Literature}
\item \hyperref[desirables-section-phantom]{Desirables}
\item \hyperref[coding-section-phantom]{Coding Style}
\item \hyperref[fdl-section-phantom]{GNU Free Documentation License}
\item \hyperref[index-section-phantom]{Auto Generated Index}
\end{enumerate}
\end{multicols}


\bibliography{my}
\bibliographystyle{alpha}

\end{document}
