\IfFileExists{stacks-project.cls}{%
\documentclass{stacks-project}
}{%
\documentclass{amsart}
}

% The following AMS packages are automatically loaded with
% the amsart documentclass:
%\usepackage{amsmath}
%\usepackage{amssymb}
%\usepackage{amsthm}

% For dealing with references we use the comment environment
\usepackage{verbatim}
\newenvironment{reference}{\comment}{\endcomment}
%\newenvironment{reference}{}{}
\newenvironment{slogan}{\comment}{\endcomment}
\newenvironment{history}{\comment}{\endcomment}

% For commutative diagrams you can use
% \usepackage{amscd}
\usepackage[all]{xy}

% We use 2cell for 2-commutative diagrams.
\xyoption{2cell}
\UseAllTwocells

% To put source file link in headers.
% Change "template.tex" to "this_filename.tex"
% \usepackage{fancyhdr}
% \pagestyle{fancy}
% \lhead{}
% \chead{}
% \rhead{Source file: \url{template.tex}}
% \lfoot{}
% \cfoot{\thepage}
% \rfoot{}
% \renewcommand{\headrulewidth}{0pt}
% \renewcommand{\footrulewidth}{0pt}
% \renewcommand{\headheight}{12pt}

\usepackage{multicol}

% For cross-file-references
\usepackage{xr-hyper}

% Package for hypertext links:
\usepackage{hyperref}

% For any local file, say "hello.tex" you want to link to please
% use \externaldocument[hello-]{hello}
\externaldocument[introduction-]{introduction}
\externaldocument[conventions-]{conventions}
\externaldocument[sets-]{sets}
\externaldocument[categories-]{categories}
\externaldocument[topology-]{topology}
\externaldocument[sheaves-]{sheaves}
\externaldocument[sites-]{sites}
\externaldocument[stacks-]{stacks}
\externaldocument[fields-]{fields}
\externaldocument[algebra-]{algebra}
\externaldocument[brauer-]{brauer}
\externaldocument[homology-]{homology}
\externaldocument[derived-]{derived}
\externaldocument[simplicial-]{simplicial}
\externaldocument[more-algebra-]{more-algebra}
\externaldocument[smoothing-]{smoothing}
\externaldocument[modules-]{modules}
\externaldocument[sites-modules-]{sites-modules}
\externaldocument[injectives-]{injectives}
\externaldocument[cohomology-]{cohomology}
\externaldocument[sites-cohomology-]{sites-cohomology}
\externaldocument[dga-]{dga}
\externaldocument[dpa-]{dpa}
\externaldocument[hypercovering-]{hypercovering}
\externaldocument[schemes-]{schemes}
\externaldocument[constructions-]{constructions}
\externaldocument[properties-]{properties}
\externaldocument[morphisms-]{morphisms}
\externaldocument[coherent-]{coherent}
\externaldocument[divisors-]{divisors}
\externaldocument[limits-]{limits}
\externaldocument[varieties-]{varieties}
\externaldocument[topologies-]{topologies}
\externaldocument[descent-]{descent}
\externaldocument[perfect-]{perfect}
\externaldocument[more-morphisms-]{more-morphisms}
\externaldocument[flat-]{flat}
\externaldocument[groupoids-]{groupoids}
\externaldocument[more-groupoids-]{more-groupoids}
\externaldocument[etale-]{etale}
\externaldocument[chow-]{chow}
\externaldocument[intersection-]{intersection}
\externaldocument[pic-]{pic}
\externaldocument[adequate-]{adequate}
\externaldocument[dualizing-]{dualizing}
\externaldocument[duality-]{duality}
\externaldocument[discriminant-]{discriminant}
\externaldocument[local-cohomology-]{local-cohomology}
\externaldocument[curves-]{curves}
\externaldocument[resolve-]{resolve}
\externaldocument[models-]{models}
\externaldocument[pione-]{pione}
\externaldocument[etale-cohomology-]{etale-cohomology}
\externaldocument[proetale-]{proetale}
\externaldocument[crystalline-]{crystalline}
\externaldocument[spaces-]{spaces}
\externaldocument[spaces-properties-]{spaces-properties}
\externaldocument[spaces-morphisms-]{spaces-morphisms}
\externaldocument[decent-spaces-]{decent-spaces}
\externaldocument[spaces-cohomology-]{spaces-cohomology}
\externaldocument[spaces-limits-]{spaces-limits}
\externaldocument[spaces-divisors-]{spaces-divisors}
\externaldocument[spaces-over-fields-]{spaces-over-fields}
\externaldocument[spaces-topologies-]{spaces-topologies}
\externaldocument[spaces-descent-]{spaces-descent}
\externaldocument[spaces-perfect-]{spaces-perfect}
\externaldocument[spaces-more-morphisms-]{spaces-more-morphisms}
\externaldocument[spaces-flat-]{spaces-flat}
\externaldocument[spaces-groupoids-]{spaces-groupoids}
\externaldocument[spaces-more-groupoids-]{spaces-more-groupoids}
\externaldocument[bootstrap-]{bootstrap}
\externaldocument[spaces-pushouts-]{spaces-pushouts}
\externaldocument[groupoids-quotients-]{groupoids-quotients}
\externaldocument[spaces-more-cohomology-]{spaces-more-cohomology}
\externaldocument[spaces-simplicial-]{spaces-simplicial}
\externaldocument[formal-spaces-]{formal-spaces}
\externaldocument[restricted-]{restricted}
\externaldocument[spaces-resolve-]{spaces-resolve}
\externaldocument[formal-defos-]{formal-defos}
\externaldocument[defos-]{defos}
\externaldocument[cotangent-]{cotangent}
\externaldocument[examples-defos-]{examples-defos}
\externaldocument[algebraic-]{algebraic}
\externaldocument[examples-stacks-]{examples-stacks}
\externaldocument[stacks-sheaves-]{stacks-sheaves}
\externaldocument[criteria-]{criteria}
\externaldocument[artin-]{artin}
\externaldocument[quot-]{quot}
\externaldocument[stacks-properties-]{stacks-properties}
\externaldocument[stacks-morphisms-]{stacks-morphisms}
\externaldocument[stacks-limits-]{stacks-limits}
\externaldocument[stacks-cohomology-]{stacks-cohomology}
\externaldocument[stacks-perfect-]{stacks-perfect}
\externaldocument[stacks-introduction-]{stacks-introduction}
\externaldocument[stacks-more-morphisms-]{stacks-more-morphisms}
\externaldocument[stacks-geometry-]{stacks-geometry}
\externaldocument[moduli-]{moduli}
\externaldocument[moduli-curves-]{moduli-curves}
\externaldocument[examples-]{examples}
\externaldocument[exercises-]{exercises}
\externaldocument[guide-]{guide}
\externaldocument[desirables-]{desirables}
\externaldocument[coding-]{coding}
\externaldocument[obsolete-]{obsolete}
\externaldocument[fdl-]{fdl}
\externaldocument[index-]{index}

% Theorem environments.
%
\theoremstyle{plain}
\newtheorem{theorem}[subsection]{Theorem}
\newtheorem{proposition}[subsection]{Proposition}
\newtheorem{lemma}[subsection]{Lemma}

\theoremstyle{definition}
\newtheorem{definition}[subsection]{Definition}
\newtheorem{example}[subsection]{Example}
\newtheorem{exercise}[subsection]{Exercise}
\newtheorem{situation}[subsection]{Situation}

\theoremstyle{remark}
\newtheorem{remark}[subsection]{Remark}
\newtheorem{remarks}[subsection]{Remarks}

\numberwithin{equation}{subsection}

% Macros
%
\def\lim{\mathop{\rm lim}\nolimits}
\def\colim{\mathop{\rm colim}\nolimits}
\def\Spec{\mathop{\rm Spec}}
\def\Hom{\mathop{\rm Hom}\nolimits}
\def\Ext{\mathop{\rm Ext}\nolimits}
\def\SheafHom{\mathop{\mathcal{H}\!{\it om}}\nolimits}
\def\SheafExt{\mathop{\mathcal{E}\!{\it xt}}\nolimits}
\def\Sch{\textit{Sch}}
\def\Mor{\mathop{\rm Mor}\nolimits}
\def\Ob{\mathop{\rm Ob}\nolimits}
\def\Sh{\mathop{\textit{Sh}}\nolimits}
\def\NL{\mathop{N\!L}\nolimits}
\def\proetale{{pro\text{-}\acute{e}tale}}
\def\etale{{\acute{e}tale}}
\def\QCoh{\textit{QCoh}}
\def\Ker{\mathop{\rm Ker}}
\def\Im{\mathop{\rm Im}}
\def\Coker{\mathop{\rm Coker}}
\def\Coim{\mathop{\rm Coim}}

%
% Macros for moduli stacks/spaces
%
\def\QCohstack{\mathcal{QC}\!{\it oh}}
\def\Cohstack{\mathcal{C}\!{\it oh}}
\def\Spacesstack{\mathcal{S}\!{\it paces}}
\def\Quotfunctor{{\rm Quot}}
\def\Hilbfunctor{{\rm Hilb}}
\def\Curvesstack{\mathcal{C}\!{\it urves}}
\def\Polarizedstack{\mathcal{P}\!{\it olarized}}
\def\Complexesstack{\mathcal{C}\!{\it omplexes}}
% \Pic is the operator that assigns to X its picard group, usage \Pic(X)
% \Picardstack_{X/B} denotes the Picard stack of X over B
% \Picardfunctor_{X/B} denotes the Picard functor of X over B
\def\Pic{\mathop{\rm Pic}\nolimits}
\def\Picardstack{\mathcal{P}\!{\it ic}}
\def\Picardfunctor{{\rm Pic}}
\def\Deformationcategory{\mathcal{D}\!{\it ef}}


% OK, start here.
%
\begin{document}

\title{Moduli of Curves}

\maketitle

\phantomsection
\label{section-phantom}

\tableofcontents




\section{Introduction}
\label{section-introduction}

\noindent
In this chapter we discuss some of the familiar moduli stacks of curves.
A reference is the celebrated article of Deligne and Mumford, see \cite{DM}.




\section{Conventions and abuse of language}
\label{section-conventions}

\noindent
We continue to use the conventions and the abuse of language
introduced in
Properties of Stacks, Section \ref{stacks-properties-section-conventions}.
Unless otherwise mentioned our base scheme will be $\Spec(\mathbf{Z})$.







\section{The stack of curves}
\label{section-stack-curves}

\noindent
This section is the continuation of Quot, Section \ref{quot-section-curves}.
Let $\Curvesstack$ be the stack whose category of sections over a
scheme $S$ is the category of families of curves over $S$.
We already know that $\Curvesstack$ is an
algebraic stack over $\mathbf{Z}$, see Quot, Theorem
\ref{quot-theorem-curves-algebraic}.

\medskip\noindent
Often base change is denoted by a subscript, but we cannot use
this notation for $\Curvesstack$ because $\Curvesstack_S$
is our notation for the fibre category over $S$.
This is why in Quot, Remark \ref{quot-remark-curves-base-change}
we used $B\text{-}\Curvesstack$ for the base change
$$
B\text{-}\Curvesstack = \Curvesstack \times B
$$
to the algebraic space $B$. The product on the right is over the
final object, i.e., over $\Spec(\mathbf{Z})$. The object on the left
is the stack classifying families of curves on the category of schemes
over $B$. In particular, if $k$ is a field, then
$$
k\text{-}\Curvesstack = \Curvesstack \times \Spec(k)
$$
is the moduli stack classifying families of curves on the category
of schemes over $k$.
Before we continue, here is a sanity check.

\begin{lemma}
\label{lemma-extend-curves-to-spaces}
Let $T \to B$ be a morphism of algebraic spaces. The category
$$
\Mor_B(T, B\text{-}\Curvesstack) = \Mor(T, \Curvesstack)
$$
is the category of families of curves over $T$.
\end{lemma}

\begin{proof}
A family of curves over $T$ is a morphism $f : X \to T$ of algebraic
spaces, which is flat, proper, of finite presentation, and has
relative dimension $\leq 1$ (Morphisms of Spaces, Definition
\ref{spaces-morphisms-definition-relative-dimension}).
This is exactly the same as the definition in
Quot, Situation \ref{quot-situation-curves}
except that $T$ the base is allowed to be an algebraic space.
Our default base category for algebraic stacks/spaces
is the category of schemes, hence the lemma does not follow
immediately from the definitions. Having said this, we encourage
the reader to skip the proof.

\medskip\noindent
By the product description of $B\text{-}\Curvesstack$ given above,
it suffices to prove the lemma in the absolute case. Choose a scheme
$U$ and a surjective \'etale morphism $p : U \to T$.
Let $R = U \times_T U$ with projections $s, t : R \to U$.

\medskip\noindent
Let $v : T \to \Curvesstack$ be a morphism. Then $v \circ p$
corresponds to a family of curves $X_U \to U$. The canonical
$2$-morphism $v \circ p \circ t \to v \circ p \circ s$
is an isomorphism $\varphi : X_U \times_{U, s} R \to X_U \times_{U, t} R$.
This isomorphism satisfies the cocycle condition on
$R \times_{s, t} R$.
By Bootstrap, Lemma \ref{bootstrap-lemma-descend-algebraic-space}
we obtain a morphism of algebraic spaces $X \to T$
whose pullback to $U$ is equal to $X_U$ compatible with $\varphi$.
Since $\{U \to T\}$ is an \'etale covering, we see that
$X \to T$ is flat, proper, of finite presentation by
Descent on Spaces, Lemmas
\ref{spaces-descent-lemma-descending-property-flat},
\ref{spaces-descent-lemma-descending-property-proper}, and
\ref{spaces-descent-lemma-descending-property-finite-presentation}.
Also $X \to T$ has relative dimension $\leq 1$ because this is
an \'etale local property. Hence $X \to T$ is a family of curves over $T$.

\medskip\noindent
Conversely, let $X \to T$ be a family of curves. Then the
base change $X_U$ determines a morphism $w : U \to \Curvesstack$
and the canonical isomorphism $X_U \times_{U, s} R \to X_U \times_{U, t} R$
determines a $2$-arrow $w \circ s \to w \circ t$ satisfying the
cocycle condition. Thus a morphism $v : T = [U/R] \to \Curvesstack$
by the universal property of the quotient $[U/R]$, see
Groupoids in Spaces, Lemma
\ref{spaces-groupoids-lemma-quotient-stack-2-coequalizer}.
(Actually, it is much easier in this case to go back to before
we introduced our abuse of language and direct construct
the functor $\Sch/T \to \Curvesstack$ which ``is'' the
morphsim $T \to \Curvesstack$.)

\medskip\noindent
We omit the verification that the constructions given above
extend to morphisms between objects and are mutually quasi-inverse.
\end{proof}






\section{The stack of polarized curves}
\label{section-polarized-curves}

\noindent
In this section we work out some of the material
discussed in Quot, Remark \ref{quot-remark-alternative-approach-curves}.
Consider the $2$-fibre product
$$
\xymatrix{
\Curvesstack \times_{\Spacesstack'_{fp, flat, proper}}
\Polarizedstack \ar[r] \ar[d] &
\Polarizedstack \ar[d] \\
\Curvesstack \ar[r] &
\Spacesstack'_{fp, flat, proper}
}
$$
We denote this $2$-fibre product by
$$
\textit{PolarizedCurves} =
\Curvesstack
\times_{\Spacesstack'_{fp, flat, proper}}
\Polarizedstack
$$
This fibre product parametrizes polarized curves, i.e., families
of curves endowed with a relatively ample invertible sheaf.
More precisely, an object of
$\textit{PolarizedCurves}$
is a pair $(X \to S, \mathcal{L})$ where
\begin{enumerate}
\item $X \to S$ is a morphism of schemes which is proper, flat,
of finite presentation, and has relative dimension $\leq 1$, and
\item $\mathcal{L}$ is an invertible $\mathcal{O}_X$-module
which is relatively ample on $X/S$.
\end{enumerate}
A morphism $(X' \to S', \mathcal{L}') \to (X \to S, \mathcal{L})$
between objects of
$\textit{PolarizedCurves}$
is given by a triple $(f, g, \varphi)$
where $f : X' \to X$ and $g : S' \to S$
are morphisms of schemes which fit into a commutative diagram
$$
\xymatrix{
X' \ar[d] \ar[r]_f & X \ar[d] \\
S' \ar[r]^g & S
}
$$
inducing an isomorphism $X' \to S' \times_S X$, in other words, the
diagram is cartesian, and $\varphi : f^*\mathcal{L} \to \mathcal{L}'$
is an isomorphism. Composition is defined in the obvious manner.

\begin{lemma}
\label{lemma-polarized-curves-in-polarized}
The morphism
$\textit{PolarizedCurves} \to
\Polarizedstack$ is an open and closed immersion.
\end{lemma}

\begin{proof}
This is true because the $1$-morphism
$\Curvesstack \to \Spacesstack'_{fp, flat, proper}$
is representable by open and closed immersions, see
Quot, Lemma \ref{quot-lemma-curves-open-and-closed-in-spaces}.
\end{proof}

\begin{lemma}
\label{lemma-polarized-curves-over-curves}
The morphism
$\textit{PolarizedCurves} \to \Curvesstack$
is smooth and surjective.
\end{lemma}

\begin{proof}
Surjective. Given a field $k$ and a proper algebraic space
$X$ over $k$ of dimension $\leq 1$, i.e., an object of $\Curvesstack$ over $k$.
By Spaces over Fields, Lemma
\ref{spaces-over-fields-lemma-codim-1-point-in-schematic-locus}
the algebraic space $X$ is a scheme. Hence $X$
is a proper scheme of dimension $\leq 1$ over $k$.
By Varieties, Lemma \ref{varieties-lemma-dim-1-proper-projective}
we see that $X$ is H-projective over $\kappa$.
In particular, there exists an ample invertible $\mathcal{O}_X$-module
$\mathcal{L}$ on $X$. Then $(X, \mathcal{L})$ is an object
of $\textit{PolarizedCurves}$ over
$k$ which maps to $X$.

\medskip\noindent
Smooth. Let $X \to S$ be an object of $\Curvesstack$, i.e., a
morphism $S \to \Curvesstack$. It is clear that
$$
\textit{PolarizedCurves}
\times_{\Curvesstack} S
\subset \Picardstack_{X/S}
$$
is the substack of objects $(T/S, \mathcal{L}/X_T)$ such that
$\mathcal{L}$ is ample on $X_T/T$. This is an open substack by
Descent on Spaces, Lemma \ref{spaces-descent-lemma-ample-in-neighbourhood}.
Since $\Picardstack_{X/S} \to S$ is smooth by
Moduli Stacks, Lemma \ref{moduli-lemma-pic-curves-smooth}
we win.
\end{proof}






\section{Properties of the stack of curves}
\label{section-properties}

\noindent
The following lemma isn't true for moduli of surfaces, see
Remark \ref{remark-boundedness-aut-does-not-work-surfaces}.

\begin{lemma}
\label{lemma-curves-diagonal-separated-fp}
The diagonal of $\Curvesstack$ is separated
and of finite presentation.
\end{lemma}

\begin{proof}
Recall that $\Curvesstack$ is a limit preserving algebraic stack, see
Quot, Lemma \ref{quot-lemma-curves-limits}.
By Limits of Stacks, Lemma \ref{stacks-limits-lemma-limit-preserving-diagonal}
this implies that
$\Delta : \Polarizedstack \to \Polarizedstack \times \Polarizedstack$
is limit preserving. Hence $\Delta$ is locally of finite presentation
by Limits of Stacks, Proposition
\ref{stacks-limits-proposition-characterize-locally-finite-presentation}.

\medskip\noindent
Let us prove that $\Delta$ is separated. To see this, it suffices to show
that given a scheme $U$ and two objects $Y \to U$ and $X \to U$ of
$\Curvesstack$ over $U$, the algebraic space
$$
\mathit{Isom}_U(Y, X)
$$
is separated. This we have seen in
Moduli Stacks, Lemmas \ref{moduli-lemma-Mor-s-lfp} and
\ref{moduli-lemma-Isom-in-Mor} that the target is
a separated algebraic space.

\medskip\noindent
To finish the proof we show that $\Delta$ is quasi-compact. Since
$\Delta$ is representable by algebraic spaces, it suffice to check
the base change of $\Delta$ by a surjective smooth morphism
$U \to \Curvesstack \times \Curvesstack$ is quasi-compact
(see for example Properties of Stacks, Lemma
\ref{stacks-properties-lemma-check-property-covering}).
We choose $U = \coprod U_i$ to be a disjoint union of affine opens
with a surjective smooth morphism
$$
U \longrightarrow
\textit{PolarizedCurves} \times \textit{PolarizedCurves}
$$
Then $U \to \Curvesstack \times \Curvesstack$ will be surjective
and smooth since $\textit{PolarizedCurves} \to \Curvesstack$
is surjective and smooth by Lemma \ref{lemma-polarized-curves-over-curves}.
Since $\textit{PolarizedCurves}$ is limit preserving
(by Artin's Axioms, Lemma \ref{artin-lemma-fibre-product-limit-preserving}
and Quot, Lemmas \ref{quot-lemma-curves-limits},
\ref{quot-lemma-polarized-limits}, and
\ref{quot-lemma-spaces-limits}), we
see that $\textit{PolarizedCurves} \to \Spec(\mathbf{Z})$ is locally of
finite presentation, hence $U_i \to \Spec(\mathbf{Z})$ is
locally of finite presentation
(Limits of Stacks, Proposition
\ref{stacks-limits-proposition-characterize-locally-finite-presentation}
and Morphisms of Stacks, Lemmas
\ref{stacks-morphisms-lemma-composition-finite-presentation} and
\ref{stacks-morphisms-lemma-smooth-locally-finite-presentation}).
In particular, $U_i$ is Noetherian affine. This reduces us to the
case discussed in the next paragraph.

\medskip\noindent
In this paragraph, given a Noetherian affine scheme $U$ and two objects
$(Y, \mathcal{N})$ and $(X, \mathcal{L})$
of $\textit{PolarizedCurves}$ over $U$, we show the algebraic space
$$
\mathit{Isom}_U(Y, X)
$$
is quasi-compact. Since the connected components of $U$ are open and closed
we may replace $U$ by these. Thus we may and do assume $U$ is connected.
Let $u \in U$ be a point. Let $Q$, $P$ be the Hilbert polynomials
of these families, i.e.,
$$
Q(n) = \chi(Y_u, \mathcal{N}_u^{\otimes n})
\quad\text{and}\quad
P(n) = \chi(X_u, \mathcal{L}_u^{\otimes n})
$$
see Varieties, Lemma \ref{varieties-lemma-numerical-polynomial-from-euler}.
Since $U$ is connected and since
the functions
$u \mapsto \chi(Y_u, \mathcal{N}_u^{\otimes n})$ and
$u \mapsto \chi(X_u, \mathcal{L}_u^{\otimes n})$
are locally constant (see 
Derived Categories of Schemes, Lemma
\ref{perfect-lemma-chi-locally-constant-geometric})
we see that we get the same Hilbert polynomial in every point of $U$.
Set
$$
\mathcal{M} = \text{pr}_1^*\mathcal{N}
\otimes_{\mathcal{O}_{Y \times_U X}} \text{pr}_2^*\mathcal{L}
$$
on $Y \times_U X$. Given $(f, \varphi) \in \mathit{Isom}_U(Y, X)(T)$
for some scheme $T$ over $U$ then for every $t \in T$ we have
\begin{align*}
\chi(Y_t, (\text{id} \times f)^*\mathcal{M}^{\otimes n})
& =
\chi(Y_t,
\mathcal{N}_t^{\otimes n} \otimes_{\mathcal{O}_{Y_t}}
f_t^*\mathcal{L}_t^{\otimes n}) \\
& =
n\deg(\mathcal{N}_t) + n\deg(f_t^*\mathcal{L}_t) +
\chi(Y_t, \mathcal{O}_{Y_t}) \\
& =
Q(n) + n\deg(\mathcal{L}_t) \\
& =
Q(n) + P(n) - P(0)
\end{align*}
by Riemann-Roch for proper curves, more precisely by
Varieties, Definition \ref{varieties-definition-degree-invertible-sheaf} and
Lemma \ref{varieties-lemma-degree-tensor-product}
and the fact that $f_t$ is an isomorphism.
Setting $P'(t) = Q(t) + P(t) - P(0)$ we find
$$
\mathit{Isom}_U(Y, X) =
\mathit{Isom}_U(Y, X) \cap \mathit{Mor}^{P', \mathcal{M}}_U(Y, X)
$$
The intersection is an intersection of open subspaces of
$\mathit{Mor}_U(Y, X)$, see
Moduli Stacks, Lemma \ref{moduli-lemma-Isom-in-Mor} and
Remark \ref{moduli-remark-Mor-numerical}.
Now $\mathit{Mor}^{P', \mathcal{M}}_U(Y, X)$
is a Noetherian algebraic space as it is of finite
presentation over $U$ by
Moduli Stacks, Lemma \ref{moduli-lemma-Mor-qc-over-base}.
Thus the intersection is a Noetherian algebraic space too
and the proof is finished.
\end{proof}

\begin{remark}
\label{remark-boundedness-aut-does-not-work-surfaces}
The boundedness argument in the proof of
Lemma \ref{lemma-curves-diagonal-separated-fp}
does not work for moduli of surfaces and in fact,
the result is wrong, for example because K3 surfaces
over fields can have infinite discrete automorphism groups.
The ``reason'' the argument does not work is that on a
projective surface $S$ over a field,
given ample invertible sheaves $\mathcal{N}$
and $\mathcal{L}$ with Hilbert polynomials $Q$ and $P$,
there is no a priori bound on the Hilbert polynomial
of $\mathcal{N} \otimes_{\mathcal{O}_S} \mathcal{L}$.
In terms of intersection theory, if $H_1$, $H_2$ are ample effective
Cartier divisors on $S$,
then there is no (upper) bound on the intersection number $H_1 \cdot H_2$
in terms of $H_1 \cdot H_1$ and $H_2 \cdot H_2$.
\end{remark}

\begin{lemma}
\label{lemma-curves-qs-lfp}
The morphism $\Curvesstack \to \Spec(\mathbf{Z})$ is quasi-separated and
locally of finite presentation.
\end{lemma}

\begin{proof}
To check $\Curvesstack \to \Spec(\mathbf{Z})$ is quasi-separated we have to
show that its diagonal is quasi-compact and quasi-separated.
This is immediate from Lemma \ref{lemma-curves-diagonal-separated-fp}.
To prove that $\Curvesstack \to \Spec(\mathbf{Z})$ is locally of finite
presentation, it suffices to show that $\Curvesstack$
is limit preserving, see Limits of Stacks, Proposition
\ref{stacks-limits-proposition-characterize-locally-finite-presentation}.
This is Quot, Lemma \ref{quot-lemma-curves-limits}.
\end{proof}








\section{Finite, reduced automorphism groups}
\label{section-finite-aut}

\noindent
Let $X$ be a proper scheme over a field $k$ of dimension $\leq 1$, i.e.,
an object of $\Curvesstack$ over $k$.
By Lemma \ref{lemma-curves-diagonal-separated-fp}
the automorphism group algebraic space $\mathit{Aut}(X)$
is finite type and separated over $k$.
In particular, $\mathit{Aut}(X)$ is a group scheme, see
More on Groupoids of Spaces, Lemma
\ref{spaces-more-groupoids-lemma-group-space-scheme-locally-finite-type-over-k}.
If the characteristic of $k$ is zero, then $\mathit{Aut}(X)$
is reduced. However, in general $\mathit{Aut}(X)$ is not reduced, even
if $X$ is geometrically reduced.

\begin{example}[Non-reduced automorphism group]
\label{example-non-reduced}
Let $k$ be an algebraically closed field of characteristic $2$.
Set $Y = Z = \mathbf{P}^1_k$. Choose three pairwise distinct $k$-valued points
$a, b, c$ in $\mathbf{A}^1_k$. Thinking of
$\mathbf{A}^1_k \subset \mathbf{P}^1_k = Y = Z$ as an open subschemes,
we get a closed immersion
$$
T =  \Spec(k[t]/(t - a)^2) \amalg \Spec(k[t]/(t - b)^2)
\amalg \Spec(k[t]/(t - c)^2)
\longrightarrow
\mathbf{P}^1_k
$$
Let $X$ be the pushout in the diagram
$$
\xymatrix{
T \ar[r] \ar[d] & Y \ar[d] \\
Z \ar[r] & X
}
$$
Let $U \subset X$ be the affine open part which is the image of
$\mathbf{A}^1_k \amalg \mathbf{A}^1_k$. Then we have an equalizer
diagram
$$
\xymatrix{
\mathcal{O}_X(U) \ar[r] &
k[t] \times k[t] \ar@<1ex>[r] \ar@<-1ex>[r] &
k[t]/(t - a)^2 \times k[t]/(t - b)^2 \times k[t]/(t - c)^2
}
$$
Over the dual numbers $A = k[\epsilon]$ we have a nontrivial automorphism
of this equalizer diagram sending $t$ to $t + \epsilon$. We leave it to the
reader to see that this automorphism extends to an automorphism of $X$
over $A$. On the other hand, the reader easily shows that the
automorphism group of $X$ over $k$ is finite.
Thus $\mathit{Aut}(X)$ must be non-reduced.
\end{example}

\noindent
Let $X$ be a proper scheme over a field $k$ of dimension $\leq 1$, i.e.,
an object of $\Curvesstack$ over $k$. If $\mathit{Aut}(X)$
is geometrically reduced, then it need not be the case that
it has dimension $0$, even if $X$ is smooth and geometrically connected.

\begin{example}[Smooth positive dimensional automorphism group]
\label{example-pos-dim}
Let $k$ be an algebraically closed field. If $X$ is a smooth
genus $0$, resp.\ $1$ curve, then the automorphism group has
dimension $3$, resp.\ $1$. Namely, in the genus $0$ case we have
$X \cong \mathbf{P}^1_k$ by Curves, Proposition
\ref{curves-proposition-projective-line}. Since
$$
\mathit{Aut}(\mathbf{P}^1_k) = \text{PGL}_{2, k}
$$
as functors we see that the dimension is $3$. On the other hand,
if the genus of $X$ is $1$, then we see that the map
$X = \underline{\Hilbfunctor}^1_{X/k} \to
\underline{\Picardfunctor}^1_{X/k}$ is an isomorphism, see
Picard Schemes of Curves, Lemma \ref{pic-lemma-picard-pieces}
and
Curves, Theorem \ref{curves-theorem-curves-rational-maps}.
Thus $X$ has the structure of an abelian variety
(since $\underline{\Picardfunctor}^1_{X/k} \cong
\underline{\Picardfunctor}^0_{X/k}$).
In particular the (co)tangent bundle of $X$ are trivial
(Groupoids, Lemma \ref{groupoids-lemma-group-scheme-module-differentials}).
We conclude that $\dim_k H^0(X, T_X) = 1$ hence
$\dim \mathit{Aut}(X) \leq 1$. On the other hand, the translations
(viewing $X$ as a group scheme) provide a $1$-dimensional
piece of $\text{Aut}(X)$ and we conlude its dimension is indeed $1$.
\end{example}

\noindent
It turns out that there is an open substack of
$\Curvesstack$ parametrizing curves whose automorphism
group is geometrically reduced and finite.
Here is a precise statement.

\begin{lemma}
\label{lemma-DM-part-curves}
There exist an open substack $\Curvesstack^{DM} \subset \Curvesstack$
with the following properties
\begin{enumerate}
\item $\Curvesstack^{DM} \subset \Curvesstack$ is the maximal
open substack which is DM,
\item given $X$ a proper scheme over a field $k$ of dimension $\leq 1$
the following are equivalent
\begin{enumerate}
\item the classifying morphism $\Spec(k) \to \Curvesstack$ factors
through $\Curvesstack^{DM}$,
\item $\mathit{Aut}(X)$ is geometrically reduced over $k$ and
has dimension $0$,
\item $\mathit{Aut}(X) \to \Spec(k)$ is unramified.
\end{enumerate}
\end{enumerate}
\end{lemma}

\begin{proof}
The existence of an open substack with property (1) is
Morphisms of Stacks, Lemma \ref{stacks-morphisms-lemma-open-DM-locus}.
The points of this open substack are characterized by (2)(c) by
Morphisms of Stacks, Lemma \ref{stacks-morphisms-lemma-points-DM-locus}.
The equivalence of (2)(b) and (2)(c) is the statement that an
algebraic space $G$ which is locally of finite type, geometrically reduced,
and of dimension $0$ over a field $k$, is unramified over $k$.
First, $G$ is a scheme by Spaces over Fields, Lemma
\ref{spaces-over-fields-lemma-locally-finite-type-dim-zero}.
Then we can take an affine open in $G$ and observe
that it will be proper over $k$ and apply
Varieties, Lemma
\ref{varieties-lemma-proper-geometrically-reduced-global-sections}.
Some details omitted.
\end{proof}







\section{Nodal curves}
\label{section-nodal-curves}

\noindent
In algebraic geometry a special role is played by nodal curves.
We suggest the reader take a brief look at some of the discussion
in Curves, Sections \ref{curves-section-nodal} and
\ref{curves-section-families-nodal}
and More on Morphisms of Spaces, Section
\ref{spaces-more-morphisms-section-families-nodal}.

\begin{lemma}
\label{lemma-nodal-part-curves}
There exist an open substack $\Curvesstack^{nodal} \subset \Curvesstack$
such that for a family of curves $f : X \to S$ the following are equivalent
\begin{enumerate}
\item $f$ is at-worst-nodal of relative dimension $1$, and
\item the classifying morphism $S \to \Curvesstack$ factors
through $\Curvesstack^{nodal}$.
\end{enumerate}
\end{lemma}

\begin{proof}
In fact, it suffices to show that given a family of curves
$f : X \to S$, there is an open subscheme $S' \subset S$
such that $S' \times_S X \to S'$ is at-worst-nodal of relative dimension $1$
and such that formation of $S'$ commutes with arbitrary base change.
By More on Morphisms of Spaces, Lemma
\ref{spaces-more-morphisms-lemma-locus-where-nodal}
there is a maximal open subspace $X' \subset X$ such
that $f|_{X'} : X' \to S$ is at-worst-nodal of relative dimension $1$.
Moreover, formation of $X'$ commutes with base change.
Hence we can take
$$
S' = S \setminus |f|(|X| \setminus |X'|)
$$
This is open because a proper morphism is universally closed by
definition.
\end{proof}







\section{Other chapters}

\begin{multicols}{2}
\begin{enumerate}
\item \hyperref[introduction-section-phantom]{Introduction}
\item \hyperref[conventions-section-phantom]{Conventions}
\item \hyperref[sets-section-phantom]{Set Theory}
\item \hyperref[categories-section-phantom]{Categories}
\item \hyperref[topology-section-phantom]{Topology}
\item \hyperref[sheaves-section-phantom]{Sheaves on Spaces}
\item \hyperref[algebra-section-phantom]{Commutative Algebra}
\item \hyperref[sites-section-phantom]{Sites and Sheaves}
\item \hyperref[homology-section-phantom]{Homological Algebra}
\item \hyperref[derived-section-phantom]{Derived Categories}
\item \hyperref[more-algebra-section-phantom]{More Algebra}
\item \hyperref[simplicial-section-phantom]{Simplicial Methods}
\item \hyperref[modules-section-phantom]{Sheaves of Modules}
\item \hyperref[sites-modules-section-phantom]{Modules on Sites}
\item \hyperref[injectives-section-phantom]{Injectives}
\item \hyperref[cohomology-section-phantom]{Cohomology of Sheaves}
\item \hyperref[sites-cohomology-section-phantom]{Cohomology on Sites}
\item \hyperref[hypercovering-section-phantom]{Hypercoverings}
\item \hyperref[schemes-section-phantom]{Schemes}
\item \hyperref[constructions-section-phantom]{Constructions of Schemes}
\item \hyperref[properties-section-phantom]{Properties of Schemes}
\item \hyperref[morphisms-section-phantom]{Morphisms of Schemes}
\item \hyperref[coherent-section-phantom]{Coherent Cohomology}
\item \hyperref[divisors-section-phantom]{Divisors}
\item \hyperref[limits-section-phantom]{Limits of Schemes}
\item \hyperref[varieties-section-phantom]{Varieties}
\item \hyperref[chow-section-phantom]{Chow Homology}
\item \hyperref[topologies-section-phantom]{Topologies on Schemes}
\item \hyperref[descent-section-phantom]{Descent}
\item \hyperref[more-morphisms-section-phantom]{More on Morphisms}
\item \hyperref[flat-section-phantom]{More on Flatness}
\item \hyperref[groupoids-section-phantom]{Groupoid Schemes}
\item \hyperref[more-groupoids-section-phantom]{More on Groupoid Schemes}
\item \hyperref[etale-section-phantom]{\'Etale Morphisms of Schemes}
\item \hyperref[etale-cohomology-section-phantom]{\'Etale Cohomology}
\item \hyperref[spaces-section-phantom]{Algebraic Spaces}
\item \hyperref[spaces-properties-section-phantom]{Properties of Algebraic Spaces}
\item \hyperref[spaces-morphisms-section-phantom]{Morphisms of Algebraic Spaces}
\item \hyperref[spaces-topologies-section-phantom]{Topologies on Algebraic Spaces}
\item \hyperref[spaces-descent-section-phantom]{Descent and Algebraic Spaces}
\item \hyperref[spaces-more-morphisms-section-phantom]{More on Morphisms of Spaces}
\item \hyperref[quot-section-phantom]{Quot and Hilbert Spaces}
\item \hyperref[stacks-section-phantom]{Stacks}
\item \hyperref[spaces-groupoids-section-phantom]{Groupoids in Algebraic Spaces}
\item \hyperref[spaces-more-groupoids-section-phantom]{More on Groupoids in Spaces}
\item \hyperref[bootstrap-section-phantom]{Bootstrap}
\item \hyperref[examples-stacks-section-phantom]{Examples of Stacks}
\item \hyperref[groupoids-quotients-section-phantom]{Quotients of Groupoids}
\item \hyperref[algebraic-section-phantom]{Algebraic Stacks}
\item \hyperref[criteria-section-phantom]{Criteria for Representability}
\item \hyperref[stacks-properties-section-phantom]{Properties of Algebraic Stacks}
\item \hyperref[stacks-morphisms-section-phantom]{Morphisms of Algebraic Stacks}
\item \hyperref[examples-section-phantom]{Examples}
\item \hyperref[exercises-section-phantom]{Exercises}
\item \hyperref[guide-section-phantom]{Guide to Literature}
\item \hyperref[desirables-section-phantom]{Desirables}
\item \hyperref[coding-section-phantom]{Coding Style}
\item \hyperref[fdl-section-phantom]{GNU Free Documentation License}
\item \hyperref[index-section-phantom]{Auto Generated Index}
\end{enumerate}
\end{multicols}


\bibliography{my}
\bibliographystyle{amsalpha}

\end{document}
