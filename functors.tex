\IfFileExists{stacks-project.cls}{%
\documentclass{stacks-project}
}{%
\documentclass{amsart}
}

% The following AMS packages are automatically loaded with
% the amsart documentclass:
%\usepackage{amsmath}
%\usepackage{amssymb}
%\usepackage{amsthm}

% For dealing with references we use the comment environment
\usepackage{verbatim}
\newenvironment{reference}{\comment}{\endcomment}
%\newenvironment{reference}{}{}
\newenvironment{slogan}{\comment}{\endcomment}
\newenvironment{history}{\comment}{\endcomment}

% For commutative diagrams you can use
% \usepackage{amscd}
\usepackage[all]{xy}

% We use 2cell for 2-commutative diagrams.
\xyoption{2cell}
\UseAllTwocells

% To put source file link in headers.
% Change "template.tex" to "this_filename.tex"
% \usepackage{fancyhdr}
% \pagestyle{fancy}
% \lhead{}
% \chead{}
% \rhead{Source file: \url{template.tex}}
% \lfoot{}
% \cfoot{\thepage}
% \rfoot{}
% \renewcommand{\headrulewidth}{0pt}
% \renewcommand{\footrulewidth}{0pt}
% \renewcommand{\headheight}{12pt}

\usepackage{multicol}

% For cross-file-references
\usepackage{xr-hyper}

% Package for hypertext links:
\usepackage{hyperref}

% For any local file, say "hello.tex" you want to link to please
% use \externaldocument[hello-]{hello}
\externaldocument[introduction-]{introduction}
\externaldocument[conventions-]{conventions}
\externaldocument[sets-]{sets}
\externaldocument[categories-]{categories}
\externaldocument[topology-]{topology}
\externaldocument[sheaves-]{sheaves}
\externaldocument[sites-]{sites}
\externaldocument[stacks-]{stacks}
\externaldocument[fields-]{fields}
\externaldocument[algebra-]{algebra}
\externaldocument[brauer-]{brauer}
\externaldocument[homology-]{homology}
\externaldocument[derived-]{derived}
\externaldocument[simplicial-]{simplicial}
\externaldocument[more-algebra-]{more-algebra}
\externaldocument[smoothing-]{smoothing}
\externaldocument[modules-]{modules}
\externaldocument[sites-modules-]{sites-modules}
\externaldocument[injectives-]{injectives}
\externaldocument[cohomology-]{cohomology}
\externaldocument[sites-cohomology-]{sites-cohomology}
\externaldocument[dga-]{dga}
\externaldocument[dpa-]{dpa}
\externaldocument[hypercovering-]{hypercovering}
\externaldocument[schemes-]{schemes}
\externaldocument[constructions-]{constructions}
\externaldocument[properties-]{properties}
\externaldocument[morphisms-]{morphisms}
\externaldocument[coherent-]{coherent}
\externaldocument[divisors-]{divisors}
\externaldocument[limits-]{limits}
\externaldocument[varieties-]{varieties}
\externaldocument[topologies-]{topologies}
\externaldocument[descent-]{descent}
\externaldocument[perfect-]{perfect}
\externaldocument[more-morphisms-]{more-morphisms}
\externaldocument[flat-]{flat}
\externaldocument[groupoids-]{groupoids}
\externaldocument[more-groupoids-]{more-groupoids}
\externaldocument[etale-]{etale}
\externaldocument[chow-]{chow}
\externaldocument[intersection-]{intersection}
\externaldocument[pic-]{pic}
\externaldocument[adequate-]{adequate}
\externaldocument[dualizing-]{dualizing}
\externaldocument[duality-]{duality}
\externaldocument[discriminant-]{discriminant}
\externaldocument[local-cohomology-]{local-cohomology}
\externaldocument[curves-]{curves}
\externaldocument[resolve-]{resolve}
\externaldocument[models-]{models}
\externaldocument[pione-]{pione}
\externaldocument[etale-cohomology-]{etale-cohomology}
\externaldocument[proetale-]{proetale}
\externaldocument[crystalline-]{crystalline}
\externaldocument[spaces-]{spaces}
\externaldocument[spaces-properties-]{spaces-properties}
\externaldocument[spaces-morphisms-]{spaces-morphisms}
\externaldocument[decent-spaces-]{decent-spaces}
\externaldocument[spaces-cohomology-]{spaces-cohomology}
\externaldocument[spaces-limits-]{spaces-limits}
\externaldocument[spaces-divisors-]{spaces-divisors}
\externaldocument[spaces-over-fields-]{spaces-over-fields}
\externaldocument[spaces-topologies-]{spaces-topologies}
\externaldocument[spaces-descent-]{spaces-descent}
\externaldocument[spaces-perfect-]{spaces-perfect}
\externaldocument[spaces-more-morphisms-]{spaces-more-morphisms}
\externaldocument[spaces-flat-]{spaces-flat}
\externaldocument[spaces-groupoids-]{spaces-groupoids}
\externaldocument[spaces-more-groupoids-]{spaces-more-groupoids}
\externaldocument[bootstrap-]{bootstrap}
\externaldocument[spaces-pushouts-]{spaces-pushouts}
\externaldocument[groupoids-quotients-]{groupoids-quotients}
\externaldocument[spaces-more-cohomology-]{spaces-more-cohomology}
\externaldocument[spaces-simplicial-]{spaces-simplicial}
\externaldocument[formal-spaces-]{formal-spaces}
\externaldocument[restricted-]{restricted}
\externaldocument[spaces-resolve-]{spaces-resolve}
\externaldocument[formal-defos-]{formal-defos}
\externaldocument[defos-]{defos}
\externaldocument[cotangent-]{cotangent}
\externaldocument[examples-defos-]{examples-defos}
\externaldocument[algebraic-]{algebraic}
\externaldocument[examples-stacks-]{examples-stacks}
\externaldocument[stacks-sheaves-]{stacks-sheaves}
\externaldocument[criteria-]{criteria}
\externaldocument[artin-]{artin}
\externaldocument[quot-]{quot}
\externaldocument[stacks-properties-]{stacks-properties}
\externaldocument[stacks-morphisms-]{stacks-morphisms}
\externaldocument[stacks-limits-]{stacks-limits}
\externaldocument[stacks-cohomology-]{stacks-cohomology}
\externaldocument[stacks-perfect-]{stacks-perfect}
\externaldocument[stacks-introduction-]{stacks-introduction}
\externaldocument[stacks-more-morphisms-]{stacks-more-morphisms}
\externaldocument[stacks-geometry-]{stacks-geometry}
\externaldocument[moduli-]{moduli}
\externaldocument[moduli-curves-]{moduli-curves}
\externaldocument[examples-]{examples}
\externaldocument[exercises-]{exercises}
\externaldocument[guide-]{guide}
\externaldocument[desirables-]{desirables}
\externaldocument[coding-]{coding}
\externaldocument[obsolete-]{obsolete}
\externaldocument[fdl-]{fdl}
\externaldocument[index-]{index}

% Theorem environments.
%
\theoremstyle{plain}
\newtheorem{theorem}[subsection]{Theorem}
\newtheorem{proposition}[subsection]{Proposition}
\newtheorem{lemma}[subsection]{Lemma}

\theoremstyle{definition}
\newtheorem{definition}[subsection]{Definition}
\newtheorem{example}[subsection]{Example}
\newtheorem{exercise}[subsection]{Exercise}
\newtheorem{situation}[subsection]{Situation}

\theoremstyle{remark}
\newtheorem{remark}[subsection]{Remark}
\newtheorem{remarks}[subsection]{Remarks}

\numberwithin{equation}{subsection}

% Macros
%
\def\lim{\mathop{\rm lim}\nolimits}
\def\colim{\mathop{\rm colim}\nolimits}
\def\Spec{\mathop{\rm Spec}}
\def\Hom{\mathop{\rm Hom}\nolimits}
\def\Ext{\mathop{\rm Ext}\nolimits}
\def\SheafHom{\mathop{\mathcal{H}\!{\it om}}\nolimits}
\def\SheafExt{\mathop{\mathcal{E}\!{\it xt}}\nolimits}
\def\Sch{\textit{Sch}}
\def\Mor{\mathop{\rm Mor}\nolimits}
\def\Ob{\mathop{\rm Ob}\nolimits}
\def\Sh{\mathop{\textit{Sh}}\nolimits}
\def\NL{\mathop{N\!L}\nolimits}
\def\proetale{{pro\text{-}\acute{e}tale}}
\def\etale{{\acute{e}tale}}
\def\QCoh{\textit{QCoh}}
\def\Ker{\mathop{\rm Ker}}
\def\Im{\mathop{\rm Im}}
\def\Coker{\mathop{\rm Coker}}
\def\Coim{\mathop{\rm Coim}}

%
% Macros for moduli stacks/spaces
%
\def\QCohstack{\mathcal{QC}\!{\it oh}}
\def\Cohstack{\mathcal{C}\!{\it oh}}
\def\Spacesstack{\mathcal{S}\!{\it paces}}
\def\Quotfunctor{{\rm Quot}}
\def\Hilbfunctor{{\rm Hilb}}
\def\Curvesstack{\mathcal{C}\!{\it urves}}
\def\Polarizedstack{\mathcal{P}\!{\it olarized}}
\def\Complexesstack{\mathcal{C}\!{\it omplexes}}
% \Pic is the operator that assigns to X its picard group, usage \Pic(X)
% \Picardstack_{X/B} denotes the Picard stack of X over B
% \Picardfunctor_{X/B} denotes the Picard functor of X over B
\def\Pic{\mathop{\rm Pic}\nolimits}
\def\Picardstack{\mathcal{P}\!{\it ic}}
\def\Picardfunctor{{\rm Pic}}
\def\Deformationcategory{\mathcal{D}\!{\it ef}}


% OK, start here.
%
\begin{document}

\title{Functors versus morphisms}


\maketitle

\phantomsection
\label{section-phantom}

\tableofcontents

\section{Introduction}
\label{section-introduction}

\noindent
Let $X$ and $Y$ be schemes. In this chapter circles around the relationship
between functors $\QCoh(\mathcal{O}_Y) \to \QCoh(\mathcal{O}_X)$ and
morphisms of schemes $X \to Y$. More broadly speaking we study the
relationship between $\QCoh(\mathcal{O}_X)$ and $X$ or, if $X$ is Noetherian,
the relationship between $\textit{Coh}(\mathcal{O}_X)$ and $X$.
This relationship was studied in \cite{Gabriel}.







\section{Functors between categories of modules}
\label{section-functors}

\noindent
The following lemma is archetypical of the results in this chapter.

\begin{lemma}
\label{lemma-functor}
Let $A$ and $B$ be rings. Let $F : \text{Mod}_A \to \text{Mod}_B$
be a functor. The following are equivalent
\begin{enumerate}
\item $F$ is isomorphic to the functor $M \mapsto M \otimes_A K$
for some $A \otimes_\mathbf{Z} B$-module $K$,
\item $F$ is right exact and commutes with all direct sums,
\item $F$ commutes with all colimits,
\item $F$ has a right adjoint $G$.
\end{enumerate}
\end{lemma}

\begin{proof}
If (1), then (4) as a right adjoint for $M \mapsto M \otimes_A K$
is $N \mapsto \Hom_B(K, N)$, see
Differential Graded Algebra, Lemma \ref{dga-lemma-tensor-hom-adjunction}.
If (4), then (3) by Categories, Lemma \ref{categories-lemma-adjoint-exact}.
The implication (3) $\Rightarrow$ (2) is immediate from the definitions.

\medskip\noindent
Assume (2). We will prove (1). By the discussion in
Homology, Section \ref{homology-section-functors}
the functor $F$ is additive. Hence $F$ induces
a ring map $A \to \text{End}_B(F(M))$, $a \mapsto F(a \cdot \text{id}_M)$
for every $A$-module $M$. We conclude that $F(M)$ is an
$A \otimes_\mathbf{Z} B$-module functorially in $M$.
Set $K = F(A)$. Define
$$
M \otimes_A K = M \otimes_A F(A) \longrightarrow F(M),
\quad m \otimes k \longmapsto F(\varphi_m)(k)
$$
Here $\varphi_m : A \to M$ sends $a \to am$. The rule
$(m, k) \mapsto F(\varphi_m)(k)$ is $A$-bilinear (and $B$-linear
on the right) as required to obtain the displayed
$A \otimes_\mathbf{Z} B$-linear map.
This construction is functorial in $M$, hence defines a transformation
of functors $- \otimes_A K \to F(-)$ which is an isomorphism when
evaluated on $A$. For every $A$-module $M$ we can choose an exact sequence
$$
\bigoplus\nolimits_{j \in J} A \to
\bigoplus\nolimits_{i \in I} A \to
M \to 0
$$
Using the maps constructed above we find a commutative diagram
$$
\xymatrix{
(\bigoplus\nolimits_{j \in J} A) \otimes_A K \ar[r] \ar[d] &
(\bigoplus\nolimits_{i \in I} A) \otimes_A K \ar[r] \ar[d] &
M \otimes_A K \ar[r] \ar[d] &
0 \\
F(\bigoplus\nolimits_{j \in J} A) \ar[r] &
F(\bigoplus\nolimits_{i \in I} A) \ar[r] &
F(M) \ar[r] &  0
}
$$
The lower row is exact as $F$ is right exact.
The upper row is exact as tensor product with $K$ is right exact.
Since $F$ commutes with direct sums the left two vertical arrows
are bijections. Hence we conclude.
\end{proof}

\begin{example}
\label{example-functor-modules}
Let $R$ be a ring. Let $A$ and $B$ be $R$-algebras. Let $K$ be a
$A \otimes_R B$-module. Then we can consider the functor
\begin{equation}
\label{equation-FM-modules}
F : \text{Mod}_A \longrightarrow \text{Mod}_B,\quad
M \longmapsto M \otimes_A K
\end{equation}
This functor is $R$-linear, right exact,
commutes with arbitrary direct sums, commutes
with all colimits, has a right adjoint (Lemma \ref{lemma-functor}).
\end{example}

\begin{lemma}
\label{lemma-functor-modules}
Let $R$ be a ring. Let $A$ and $B$ be $R$-algebras. There is an
equivalence of categories between
\begin{enumerate}
\item the category of $R$-linear functors
$F : \text{Mod}_A \to \text{Mod}_B$ which
are right exact and commute with arbitrary direct sums, and
\item the category $\text{Mod}_{A \otimes_R B}$.
\end{enumerate}
given by sending $K$ to the functor $F$ in (\ref{equation-FM-modules}).
\end{lemma}

\begin{proof}
Let $F$ be an object of the first category. By
Lemma \ref{lemma-functor} we may assume $F(M) = M \otimes_A K$
functorially in $M$ for some $A \otimes_\mathbf{Z} B$-module $K$.
The $R$-linearity of $F$ immediately implies that the
$A \otimes_\mathbf{Z} B$-module structure on $K$ comes
from a (unique) $A \otimes_R B$-module structure on $K$.
Thus we see that sending $K$ to $F$ as in (\ref{equation-FM-modules})
is essentially surjective.

\medskip\noindent
To prove that our functor is fully faithful, we have to show that
given $A \otimes_R B$-modules $K$ and $K'$ any transformation
$t : F \to F'$ between the corresponding functors, comes from
a unique $\varphi : K \to K'$. Since $K = F(A)$ and $K' = F'(A)$
we can take $\varphi$ to be the value $t_A : F(A) \to F'(A)$
of $t$ at $A$. This maps is $A \otimes_R B$-linear by the
definition of the $A \otimes B$-module structure on $F(A)$
and $F'(A)$ given in the proof of Lemma \ref{lemma-functor}.
\end{proof}

\begin{remark}
\label{remark-composition}
Let $R$ be a ring. Let $A$, $B$, $C$ be $R$-algebras.
Let $F : \text{Mod}_A \to \text{Mod}_B$ and
$F' : \text{Mod}_B \to \text{Mod}_C$ be
$R$-linear, right exact functors which commute with arbitrary direct sums.
If by the equivalence of Lemma \ref{lemma-functor-modules} the object
$K$ in $\text{Mod}_{A \otimes_R B}$ corresponds to $F$ and the object
$K'$ in $\text{Mod}_{B \otimes_R C}$ corresponds to $F'$, then
$K \otimes_B K'$ viewed as an object of
$\text{Mod}_{A \otimes_R C}$ corresponds to $F' \circ F$.
\end{remark}

\begin{remark}
\label{remark-exact-flat}
In the situation of Lemma \ref{lemma-functor-modules}
suppose that $F$ corresponds to $K$. Then
$F$ is exact $\Leftrightarrow$ $K$ is flat over $A$.
\end{remark}

\begin{remark}
\label{remark-finite}
In the situation of Lemma \ref{lemma-functor-modules}
suppose that $F$ corresponds to $K$. Then
$F$ sends finite $A$-modules to finite $B$-modules
$\Leftrightarrow$ $K$ is finite as a $B$-module.
\end{remark}

\begin{remark}
\label{remark-finite-presentation}
In the situation of Lemma \ref{lemma-functor-modules}
suppose that $F$ corresponds to $K$. Then
$F$ sends finitely presented $A$-modules to finitely presented $B$-modules
$\Leftrightarrow$ $K$ is finitely presented as a $B$-module.
\end{remark}

\begin{lemma}
\label{lemma-functor-equivalence}
Let $A$ and $B$ be rings. If
$$
F : \text{Mod}_A \longrightarrow \text{Mod}_B
$$
is an equivalence of categories, then there exists an isomorphism
$A \to B$ of rings and an invertible $B$-module $L$ such that
$F$ is isomorphic to the functor $M \mapsto (M \otimes_A B) \otimes_B L$.
\end{lemma}

\begin{proof}
Since an equivalence commutes with all colimits, we see that
Lemmas \ref{lemma-functor} applies. Let $K$ be the
$A \otimes_\mathbf{Z} B$-module such that $F$ is
isomorphic to the functor $M \mapsto M \otimes_A K$.
Let $K'$ be the $B \otimes_\mathbf{Z} A$-module such that
a quasi-inverse of $F$ is
isomorphic to the functor $N \mapsto N \otimes_B K'$.
By Remark \ref{remark-composition} and
Lemma \ref{lemma-functor-modules} we have an isomorphism
$$
\psi : K \otimes_B K' \longrightarrow A
$$
of $A \otimes_\mathbf{Z} A$-modules.
Similarly, we have an isomorphism
$$
\psi' : K' \otimes_A K \longrightarrow B
$$
of $B \otimes_\mathbf{Z} B$-modules. Choose an element
$\xi = \sum_{i = 1, \ldots, n} x_i \otimes y_i \in K \otimes_B K'$
such that $\psi(\xi) = 1$. Consider the isomorphisms
$$
K \xrightarrow{\psi^{-1} \otimes \text{id}_K}
K \otimes_B K' \otimes_A K \xrightarrow{\text{id}_K \otimes \psi'} K
$$
The composition is an isomorphism and given by
$$
k \longmapsto \sum x_i \psi'(y_i \otimes k)
$$
We conclude this automorphism factors as
$$
K \to B^{\oplus n} \to K
$$
as a map of $B$-modules. It follows that $K$ is finite
projective as a $B$-module.

\medskip\noindent
We claim that $K$ is invertible as a $B$-module. This is equivalent
to asking the rank of $K$ as a $B$-module to have the constant value $1$,
see More on Algebra, Lemma \ref{more-algebra-lemma-invertible} and
Algebra, Lemma \ref{algebra-lemma-finite-projective}.
If not, then there exists a maximal ideal $\mathfrak m \subset B$
such that either (a) $K \otimes_B B/\mathfrak m = 0$ or
(b) there is a surjection $K \to (B/\mathfrak m)^{\oplus 2}$ of
$B$-modules. Case (a) is absurd as $K' \otimes_A K \otimes_B N = N$
for all $B$-modules $N$. Case (b) would imply we get a surjection
$$
A = K \otimes_B K' \longrightarrow (B/\mathfrak m \otimes_B K')^{\oplus 2}
$$
of (right) $A$-modules. This is impossible as the target is an $A$-module
which needs at least two generators: $B/\mathfrak m \otimes_B K'$
is nonzero as the image of the nonzero module $B/\mathfrak m$ under
the quasi-inverse of $F$.

\medskip\noindent
Since $K$ is invertible as a $B$-module we see that $\Hom_B(K, K) = B$.
Thus the $A$-module structure on $K$ defines a ring map
$A \to B$. The lemma follows.
\end{proof}

\begin{lemma}
\label{lemma-functor-equivalence-linear}
Let $R$ be a ring. Let $A$ and $B$ be $R$-algebras. If
$$
F : \text{Mod}_A \longrightarrow \text{Mod}_B
$$
is an $R$-linear equivalence of categories, then there exists an isomorphism
$A \to B$ of $R$-algebras and an invertible $B$-module $L$ such that
$F$ is isomorphic to the functor $M \mapsto (M \otimes_A B) \otimes_B L$.
\end{lemma}

\begin{proof}
We get $A \to B$ and $L$ from Lemma \ref{lemma-functor-equivalence}.
To finish the proof, we need to show that the $R$-linearity
of $F$ forces $A \to B$ to be an $R$-algebra map. We omit the details.
\end{proof}

\begin{remark}
\label{remark-monoidal}
Let $A$ and $B$ be rings. Let us endow $\text{Mod}_A$ and $\text{Mod}_B$
with the usual monoidal structure given by tensor products of modules.
Let $F : \text{Mod}_A \to \text{Mod}_B$ be a functor of
monoidal categories, see
Categories, Definition \ref{categories-definition-functor-monoidal-categories}.
Here are some comments:
\begin{enumerate}
\item Since $F(A)$ is a unit (by our definitions) we have $F(A) = B$.
\item We obtain a multiplicative map $\varphi : A \to B$
by sending $a \in A$ to its action on $F(A) = B$.
\item If $F$ is additive, then $\varphi$ is a ring map.
\item Take $A = B$ and $F(M) = M \otimes_A M$. In this case $\varphi(a) = a^2$.
\item If $F$ is right exact and commutes with direct sums,
then $F(M) = M \otimes_{A, \varphi} B$ by Lemma \ref{lemma-functor}
and the discussion so far.
\end{enumerate}
In other words, ring maps $A \to B$ are in bijection with isomorphism classes
of functors of monoidal categories $\text{Mod}_A \to \text{Mod}_B$
which commute with all colimits.
\end{remark}










\section{Other chapters}

\begin{multicols}{2}
\begin{enumerate}
\item \hyperref[introduction-section-phantom]{Introduction}
\item \hyperref[conventions-section-phantom]{Conventions}
\item \hyperref[sets-section-phantom]{Set Theory}
\item \hyperref[categories-section-phantom]{Categories}
\item \hyperref[topology-section-phantom]{Topology}
\item \hyperref[sheaves-section-phantom]{Sheaves on Spaces}
\item \hyperref[algebra-section-phantom]{Commutative Algebra}
\item \hyperref[sites-section-phantom]{Sites and Sheaves}
\item \hyperref[homology-section-phantom]{Homological Algebra}
\item \hyperref[derived-section-phantom]{Derived Categories}
\item \hyperref[more-algebra-section-phantom]{More Algebra}
\item \hyperref[simplicial-section-phantom]{Simplicial Methods}
\item \hyperref[modules-section-phantom]{Sheaves of Modules}
\item \hyperref[sites-modules-section-phantom]{Modules on Sites}
\item \hyperref[injectives-section-phantom]{Injectives}
\item \hyperref[cohomology-section-phantom]{Cohomology of Sheaves}
\item \hyperref[sites-cohomology-section-phantom]{Cohomology on Sites}
\item \hyperref[hypercovering-section-phantom]{Hypercoverings}
\item \hyperref[schemes-section-phantom]{Schemes}
\item \hyperref[constructions-section-phantom]{Constructions of Schemes}
\item \hyperref[properties-section-phantom]{Properties of Schemes}
\item \hyperref[morphisms-section-phantom]{Morphisms of Schemes}
\item \hyperref[coherent-section-phantom]{Coherent Cohomology}
\item \hyperref[divisors-section-phantom]{Divisors}
\item \hyperref[limits-section-phantom]{Limits of Schemes}
\item \hyperref[varieties-section-phantom]{Varieties}
\item \hyperref[chow-section-phantom]{Chow Homology}
\item \hyperref[topologies-section-phantom]{Topologies on Schemes}
\item \hyperref[descent-section-phantom]{Descent}
\item \hyperref[more-morphisms-section-phantom]{More on Morphisms}
\item \hyperref[flat-section-phantom]{More on Flatness}
\item \hyperref[groupoids-section-phantom]{Groupoid Schemes}
\item \hyperref[more-groupoids-section-phantom]{More on Groupoid Schemes}
\item \hyperref[etale-section-phantom]{\'Etale Morphisms of Schemes}
\item \hyperref[etale-cohomology-section-phantom]{\'Etale Cohomology}
\item \hyperref[spaces-section-phantom]{Algebraic Spaces}
\item \hyperref[spaces-properties-section-phantom]{Properties of Algebraic Spaces}
\item \hyperref[spaces-morphisms-section-phantom]{Morphisms of Algebraic Spaces}
\item \hyperref[spaces-topologies-section-phantom]{Topologies on Algebraic Spaces}
\item \hyperref[spaces-descent-section-phantom]{Descent and Algebraic Spaces}
\item \hyperref[spaces-more-morphisms-section-phantom]{More on Morphisms of Spaces}
\item \hyperref[quot-section-phantom]{Quot and Hilbert Spaces}
\item \hyperref[stacks-section-phantom]{Stacks}
\item \hyperref[spaces-groupoids-section-phantom]{Groupoids in Algebraic Spaces}
\item \hyperref[spaces-more-groupoids-section-phantom]{More on Groupoids in Spaces}
\item \hyperref[bootstrap-section-phantom]{Bootstrap}
\item \hyperref[examples-stacks-section-phantom]{Examples of Stacks}
\item \hyperref[groupoids-quotients-section-phantom]{Quotients of Groupoids}
\item \hyperref[algebraic-section-phantom]{Algebraic Stacks}
\item \hyperref[criteria-section-phantom]{Criteria for Representability}
\item \hyperref[stacks-properties-section-phantom]{Properties of Algebraic Stacks}
\item \hyperref[stacks-morphisms-section-phantom]{Morphisms of Algebraic Stacks}
\item \hyperref[examples-section-phantom]{Examples}
\item \hyperref[exercises-section-phantom]{Exercises}
\item \hyperref[guide-section-phantom]{Guide to Literature}
\item \hyperref[desirables-section-phantom]{Desirables}
\item \hyperref[coding-section-phantom]{Coding Style}
\item \hyperref[fdl-section-phantom]{GNU Free Documentation License}
\item \hyperref[index-section-phantom]{Auto Generated Index}
\end{enumerate}
\end{multicols}


\bibliography{my}
\bibliographystyle{amsalpha}

\end{document}

