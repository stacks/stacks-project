\IfFileExists{stacks-project.cls}{%
\documentclass{stacks-project}
}{%
\documentclass{amsart}
}

% The following AMS packages are automatically loaded with
% the amsart documentclass:
%\usepackage{amsmath}
%\usepackage{amssymb}
%\usepackage{amsthm}

% For dealing with references we use the comment environment
\usepackage{verbatim}
\newenvironment{reference}{\comment}{\endcomment}
%\newenvironment{reference}{}{}
\newenvironment{slogan}{\comment}{\endcomment}
\newenvironment{history}{\comment}{\endcomment}

% For commutative diagrams you can use
% \usepackage{amscd}
\usepackage[all]{xy}

% We use 2cell for 2-commutative diagrams.
\xyoption{2cell}
\UseAllTwocells

% To put source file link in headers.
% Change "template.tex" to "this_filename.tex"
% \usepackage{fancyhdr}
% \pagestyle{fancy}
% \lhead{}
% \chead{}
% \rhead{Source file: \url{template.tex}}
% \lfoot{}
% \cfoot{\thepage}
% \rfoot{}
% \renewcommand{\headrulewidth}{0pt}
% \renewcommand{\footrulewidth}{0pt}
% \renewcommand{\headheight}{12pt}

\usepackage{multicol}

% For cross-file-references
\usepackage{xr-hyper}

% Package for hypertext links:
\usepackage{hyperref}

% For any local file, say "hello.tex" you want to link to please
% use \externaldocument[hello-]{hello}
\externaldocument[introduction-]{introduction}
\externaldocument[conventions-]{conventions}
\externaldocument[sets-]{sets}
\externaldocument[categories-]{categories}
\externaldocument[topology-]{topology}
\externaldocument[sheaves-]{sheaves}
\externaldocument[sites-]{sites}
\externaldocument[stacks-]{stacks}
\externaldocument[fields-]{fields}
\externaldocument[algebra-]{algebra}
\externaldocument[brauer-]{brauer}
\externaldocument[homology-]{homology}
\externaldocument[derived-]{derived}
\externaldocument[simplicial-]{simplicial}
\externaldocument[more-algebra-]{more-algebra}
\externaldocument[smoothing-]{smoothing}
\externaldocument[modules-]{modules}
\externaldocument[sites-modules-]{sites-modules}
\externaldocument[injectives-]{injectives}
\externaldocument[cohomology-]{cohomology}
\externaldocument[sites-cohomology-]{sites-cohomology}
\externaldocument[dga-]{dga}
\externaldocument[dpa-]{dpa}
\externaldocument[hypercovering-]{hypercovering}
\externaldocument[schemes-]{schemes}
\externaldocument[constructions-]{constructions}
\externaldocument[properties-]{properties}
\externaldocument[morphisms-]{morphisms}
\externaldocument[coherent-]{coherent}
\externaldocument[divisors-]{divisors}
\externaldocument[limits-]{limits}
\externaldocument[varieties-]{varieties}
\externaldocument[topologies-]{topologies}
\externaldocument[descent-]{descent}
\externaldocument[perfect-]{perfect}
\externaldocument[more-morphisms-]{more-morphisms}
\externaldocument[flat-]{flat}
\externaldocument[groupoids-]{groupoids}
\externaldocument[more-groupoids-]{more-groupoids}
\externaldocument[etale-]{etale}
\externaldocument[chow-]{chow}
\externaldocument[intersection-]{intersection}
\externaldocument[pic-]{pic}
\externaldocument[adequate-]{adequate}
\externaldocument[dualizing-]{dualizing}
\externaldocument[duality-]{duality}
\externaldocument[discriminant-]{discriminant}
\externaldocument[local-cohomology-]{local-cohomology}
\externaldocument[curves-]{curves}
\externaldocument[resolve-]{resolve}
\externaldocument[models-]{models}
\externaldocument[pione-]{pione}
\externaldocument[etale-cohomology-]{etale-cohomology}
\externaldocument[proetale-]{proetale}
\externaldocument[crystalline-]{crystalline}
\externaldocument[spaces-]{spaces}
\externaldocument[spaces-properties-]{spaces-properties}
\externaldocument[spaces-morphisms-]{spaces-morphisms}
\externaldocument[decent-spaces-]{decent-spaces}
\externaldocument[spaces-cohomology-]{spaces-cohomology}
\externaldocument[spaces-limits-]{spaces-limits}
\externaldocument[spaces-divisors-]{spaces-divisors}
\externaldocument[spaces-over-fields-]{spaces-over-fields}
\externaldocument[spaces-topologies-]{spaces-topologies}
\externaldocument[spaces-descent-]{spaces-descent}
\externaldocument[spaces-perfect-]{spaces-perfect}
\externaldocument[spaces-more-morphisms-]{spaces-more-morphisms}
\externaldocument[spaces-flat-]{spaces-flat}
\externaldocument[spaces-groupoids-]{spaces-groupoids}
\externaldocument[spaces-more-groupoids-]{spaces-more-groupoids}
\externaldocument[bootstrap-]{bootstrap}
\externaldocument[spaces-pushouts-]{spaces-pushouts}
\externaldocument[groupoids-quotients-]{groupoids-quotients}
\externaldocument[spaces-more-cohomology-]{spaces-more-cohomology}
\externaldocument[spaces-simplicial-]{spaces-simplicial}
\externaldocument[formal-spaces-]{formal-spaces}
\externaldocument[restricted-]{restricted}
\externaldocument[spaces-resolve-]{spaces-resolve}
\externaldocument[formal-defos-]{formal-defos}
\externaldocument[defos-]{defos}
\externaldocument[cotangent-]{cotangent}
\externaldocument[examples-defos-]{examples-defos}
\externaldocument[algebraic-]{algebraic}
\externaldocument[examples-stacks-]{examples-stacks}
\externaldocument[stacks-sheaves-]{stacks-sheaves}
\externaldocument[criteria-]{criteria}
\externaldocument[artin-]{artin}
\externaldocument[quot-]{quot}
\externaldocument[stacks-properties-]{stacks-properties}
\externaldocument[stacks-morphisms-]{stacks-morphisms}
\externaldocument[stacks-limits-]{stacks-limits}
\externaldocument[stacks-cohomology-]{stacks-cohomology}
\externaldocument[stacks-perfect-]{stacks-perfect}
\externaldocument[stacks-introduction-]{stacks-introduction}
\externaldocument[stacks-more-morphisms-]{stacks-more-morphisms}
\externaldocument[stacks-geometry-]{stacks-geometry}
\externaldocument[moduli-]{moduli}
\externaldocument[moduli-curves-]{moduli-curves}
\externaldocument[examples-]{examples}
\externaldocument[exercises-]{exercises}
\externaldocument[guide-]{guide}
\externaldocument[desirables-]{desirables}
\externaldocument[coding-]{coding}
\externaldocument[obsolete-]{obsolete}
\externaldocument[fdl-]{fdl}
\externaldocument[index-]{index}

% Theorem environments.
%
\theoremstyle{plain}
\newtheorem{theorem}[subsection]{Theorem}
\newtheorem{proposition}[subsection]{Proposition}
\newtheorem{lemma}[subsection]{Lemma}

\theoremstyle{definition}
\newtheorem{definition}[subsection]{Definition}
\newtheorem{example}[subsection]{Example}
\newtheorem{exercise}[subsection]{Exercise}
\newtheorem{situation}[subsection]{Situation}

\theoremstyle{remark}
\newtheorem{remark}[subsection]{Remark}
\newtheorem{remarks}[subsection]{Remarks}

\numberwithin{equation}{subsection}

% Macros
%
\def\lim{\mathop{\rm lim}\nolimits}
\def\colim{\mathop{\rm colim}\nolimits}
\def\Spec{\mathop{\rm Spec}}
\def\Hom{\mathop{\rm Hom}\nolimits}
\def\Ext{\mathop{\rm Ext}\nolimits}
\def\SheafHom{\mathop{\mathcal{H}\!{\it om}}\nolimits}
\def\SheafExt{\mathop{\mathcal{E}\!{\it xt}}\nolimits}
\def\Sch{\textit{Sch}}
\def\Mor{\mathop{\rm Mor}\nolimits}
\def\Ob{\mathop{\rm Ob}\nolimits}
\def\Sh{\mathop{\textit{Sh}}\nolimits}
\def\NL{\mathop{N\!L}\nolimits}
\def\proetale{{pro\text{-}\acute{e}tale}}
\def\etale{{\acute{e}tale}}
\def\QCoh{\textit{QCoh}}
\def\Ker{\mathop{\rm Ker}}
\def\Im{\mathop{\rm Im}}
\def\Coker{\mathop{\rm Coker}}
\def\Coim{\mathop{\rm Coim}}

%
% Macros for moduli stacks/spaces
%
\def\QCohstack{\mathcal{QC}\!{\it oh}}
\def\Cohstack{\mathcal{C}\!{\it oh}}
\def\Spacesstack{\mathcal{S}\!{\it paces}}
\def\Quotfunctor{{\rm Quot}}
\def\Hilbfunctor{{\rm Hilb}}
\def\Curvesstack{\mathcal{C}\!{\it urves}}
\def\Polarizedstack{\mathcal{P}\!{\it olarized}}
\def\Complexesstack{\mathcal{C}\!{\it omplexes}}
% \Pic is the operator that assigns to X its picard group, usage \Pic(X)
% \Picardstack_{X/B} denotes the Picard stack of X over B
% \Picardfunctor_{X/B} denotes the Picard functor of X over B
\def\Pic{\mathop{\rm Pic}\nolimits}
\def\Picardstack{\mathcal{P}\!{\it ic}}
\def\Picardfunctor{{\rm Pic}}
\def\Deformationcategory{\mathcal{D}\!{\it ef}}


% OK, start here.
%
\begin{document}

\title{Etale Cohomology: Cohomology}


\maketitle

\phantomsection
\label{section-phantom}

\tableofcontents

%9.29.09
\section{\'Etale cohomology}
\label{section-etale-cohomology}


\section{Colimits}
\label{section-colimit}

\noindent
Let us start by recalling that if $\left\{\mathcal{F}_i\right\}_{i\in I}$ is a
system of sheaves on $X$, its colimit (in the category of sheaves) is the
sheafification of the presheaf $\mathcal{U} \mapsto \text{colim}_{i\in I}
\mathcal{F}_i(\mathcal{U})$. In the case where $X$ is noetherian, the
sheafification is superfluous. See \cite{H}.

\begin{theorem}
\label{theorem-colimit}
Let $X$ be a quasi-compact and quasi-separated scheme. Let
$\left(\mathcal{F}_i, \varphi_{ij}\right)$ be a system of abelian sheaves on
$X_{et}$ over the partially ordered set $I$. If $I$ is directed then
$$
\text{colim}_{i\in I} H_{et}^p(X, \mathcal{F}_i) = H_{et}^p(X,
\text{colim}_{i\in I} \mathcal{F}_i).
$$
\end{theorem}

\begin{proof}[Sketch of proof.]
This is proven for all $X$ at the same time, by induction on $p$.
\begin{enumerate}
\item
For any quasi-compact and quasi-separated scheme $X$ and any \'etale covering
$\mathcal{U}$ of $X$, show that there exists a refinement $\mathcal{V}
=\left\{\mathcal{V}_j \to X\right\}_{j\in J}$ with $J$ finite and each $V_j$
quasi-compact and quasi-separated such that all the $\mathcal{V}_{j_0} \times_X
\cdots \times_X \mathcal{V}_{j_p}$ are also quasi-compact and quasi-separated.
\item
Using the previous step and the definition of colimits in the category of
sheaves, show that the theorem holds for $p=0$, all $X$. (Exercise.)
\item
Using the locality of cohomology
(Lemma \ref{lemma-locality-cohomology}),
the \u Cech-to-cohomology spectral sequence
(Theorem \ref{theorem-cech-ss}) and the fact that the induction
hypothesis applies to all $\mathcal{V}_{j_0}\times_X \cdots \times_X
\mathcal{V}_{j_p}$ in the above situation, prove the induction step $p\to p+1$.
\end{enumerate}
\end{proof}

\begin{theorem}
\label{theorem-directed-colimit-cohomology}
Let $A$ be a ring, $(I, \leq)$ a directed poset and $(B_i, \varphi_{ij})$ a
system of $A$-algebras. Set $B=\text{colim}_{i\in I} B_i$. Let $X \to
\text{Spec} A$ be a quasi-compact and quasi-separated morphism of schemes and
$\mathcal{F}$ an abelian sheaf on $X_{et}$. Denote $X_i = X\times_{\text{Spec}
A} \text{Spec} B_i$, $Y= X \times_{\text{Spec} A}\text{Spec} B$,
$\mathcal{F}_i = (X_i\to X)^{-1}\mathcal{F}$ and $\mathcal{G} = (Y\to
X)^{-1}\mathcal{F}$. Then
$$
H_{et}^p(Y, \mathcal{G}) = \text{colim}_{i\in I} H_{et}^p ((X_i),
\mathcal{F}_i).
$$
\end{theorem}

\begin{proof}[Sketch of proof.]
The proof proceeds along the following steps.
\begin{enumerate}
\item Given $\mathcal{V}\to Y$ \'etale with $\mathcal{V}$ quasi-compact and
quasi-separated, there exist $i\in I$ and $\mathcal{U}_i \to X_i$ such that
$\mathcal{V} = \mathcal{U}_i \times_{X_i} Y$.
\end{enumerate}
If all the schemes considered were affine, this would correspond to the
following algebra statement: if $B=\text{colim} B_i$ and $B\to C$ is \'etale,
then there exist $i\in I$ and $B_i\to C_i$ \'etale such that $C \cong B
\otimes_{B_i} C_i$.

\medskip\noindent
This is proven as follows: write $C \cong B\left[x_1,\ldots, x_n\right]/(f_1,
\ldots, f_n)$ with $\det (f_j(x_k)) \in C^*$ and pick $i\in I$ large enough so
that all the coefficients of the $f_j$s lie in $B_i$, and let $C_i =
B_i\left[x_1, \ldots, x_n\right]/(f_1, \dots, f_n)$. This makes sense by the
assumption. After further increasing $i$, $\det (f_j(x_k))$ will be invertible
in $C_i$, and $C_i$ will be \'etale over $B_i$.
\begin{enumerate}
\item[(2)]
By (1), we see that for every \'etale covering $\mathcal{V} =
\left\{\mathcal{V}_j\to Y\right\}_{j\in J}$ with $J$ finite and the
$\mathcal{V}_j$s quasi-compact and quasi-separated, there exists $i\in I$ and
an \'etale covering $\mathcal{V}_i = \left\{\mathcal{V}_{ij} \to X_i
\right\}_{j\in J}$ such that $\mathcal{V} \cong \mathcal{V}_i\times_{X_i} Y$.
\item[(3)]
Show that (2) implies
$$
\check H^*(\mathcal{V}, \mathcal{G})=\text{colim}_{i\in I}\check
H^*(\mathcal{V}_i, \mathcal{F}_i).
$$
This is not clear, as we have not explained how to deal with $\mathcal{F}_i$
and $\mathcal{G}$, in particular with the dual.
\item[(4)] Use the \u Cech-to-cohomology spectral sequence
(Theorem \ref{theorem-cech-ss}).
\end{enumerate}
\end{proof}





\section{Stalks of higher direct images}
\label{section-stalks-direct-image}

\begin{lemma}
\label{lemma-higher-direct-images}
Let $f: X\to Y$ be a morphism of schemes and $\mathcal{F}\in
\textit{Ab}(X_{et})$. Then $R^pf_*\mathcal{F}$ is the sheaf associated to the
presheaf
$$
(V\to Y)\longmapsto H_{et}^0 \left(X\times_Y V,
\mathcal{F}|_{X\times_YV}\right).
$$
\end{lemma}

\noindent
This lemma is valid for topological spaces, and the proof in this case is the
same.

\begin{theorem}
\label{theorem-higher-direct-images}
Let $f: X\to S$ be a quasi-compact and quasi-separated morphism of schemes,
$\mathcal{F}$ an abelian sheaf on $X_{et}$, and $\bar s$ a geometric point of
$S$. Then
$$
\left(R^pf_* \mathcal{F}\right)_{\bar s} = H_{et}^p\left( X\times_S
\text{Spec}(\mathcal{O}_{S, \bar s}^\mathrm{sh}),
\text{pr}^{-1}\mathcal{F}\right)
$$
where $\text{pr}$ is the projection $X\times_S \text{Spec}(\mathcal{O}_{S,
\bar{s}}^\mathrm{sh}) \to X$.
\end{theorem}

\begin{proof}
Let $\mathcal{I}$ be the category opposite to the category of \'etale
neighborhoods of $\bar s$ on $S$. By lemma \ref{lemma-higher-direct-images}
we have
$$
\left(R^pf_*\mathcal{F}\right)_{\bar{s}} = \text{colim}_{(\mathcal{V},
\bar{v})\in \mathcal{I}} H^p(X\times_S\mathcal{V},
\mathcal{F}|_{X\times_S\mathcal{V}}).
$$
On the other hand,
$$
\mathcal{O}_{S, \bar{s}}^\mathrm{sh} = \text{colim}_{(\mathcal{V}, \bar v)\in
\mathcal{I}} \Gamma(\mathcal{V}, \mathcal{O}_\mathcal{V}).
$$
Replacing $\mathcal{I}$ with its cofinal subset $\mathcal{I}^\mathrm{aff}$
consisting of affine \'etale neighborhoods $\mathcal{V}_i= \text{Spec} B_i$ of
$\bar s$ mapping into some fixed affine open $\text{Spec} A \subset S$, we get
$$
\mathcal{O}_{S, \bar{s}}^\mathrm{sh} = \text{colim}_{i\in
\mathcal{I}^\mathrm{aff}} B_i,
$$
and the result follows from theorem \ref{theorem-directed-colimit-cohomology}.
\end{proof}





\section{The Leray spectral sequence}
\label{section-leray}

\begin{lemma}
\label{lemma-prepare-leray}
Let $f: X\to Y$ be a morphism and $\mathcal{I}$ an injective sheaf in
$\textit{Ab}(X_{et})$. Then
\begin{enumerate}
\item
for any $\mathcal{V}\in\text{Ob}(Y_{et})$ and any \'etale covering
$\mathcal{V}=\left\{\mathcal{V}_j\to \mathcal{V}\right\}_{j\in J}$ we have
$\check H^p(\mathcal{V}, f_*\mathcal{I}) = 0$ for all $p>0$ ;
\item
$f_*\mathcal{I}$ is acyclic for the functors $\Gamma(Y, -)$ and
$\Gamma(\mathcal{V},-)$ ; and
\item
if $g: Y\to Z$, then $f_*\mathcal{I}$ is acyclic for $g_*$.
\end{enumerate}
\end{lemma}

\begin{proof}
Observe that $\check{\mathcal{C}}^\bullet(\mathcal{V}, f_*\mathcal{I}) =
\check{\mathcal{C}}^\bullet(\mathcal{V}\times_Y X, \mathcal{I})$ which has no
cohomology by lemma \ref{lemma-hom-injective}, which proves {\it i}. The
second statement is a great exercise in using the \u Cech-to-cohomology
spectral sequence. See (insert future reference) for more details. Part {\it
iii} is a consequence of {\it ii} and the description of $R^pg_*$ from theorem
\ref{theorem-higher-direct-images}.
\end{proof}

\noindent
Using the formalism of Grothendieck spectral sequences, this gives the
following.

\begin{proposition}
\label{proposition-leray}
(Leray spectral sequence)
Let $f: X \to Y$ be a morphism of schemes and $\mathcal{F}$ an \'etale sheaf on
$X$. Then there is a spectral sequence
$$
E_2^{p,q} = H_{et}^p(Y, R^qf_*\mathcal{F}) \Rightarrow H_{et}^{p+q}(X,
\mathcal{F}).
$$
\end{proposition}





\section{Henselian rings}
\label{section-heselian-ring}

\begin{theorem}
\label{theorem-quasi-finite-etale-locally}
Let $A\to B$ be finite type ring map and $\mathfrak p \subset A$ a prime
ideal. Then there exist an \'etale ring map $A \to A'$ and a prime
$\mathfrak p' \subset A'$ lying over $\mathfrak p$ such that
\begin{enumerate}
\item
$\kappa(\mathfrak p) = \kappa(\mathfrak p')$ ;
\item
$ B \otimes_A A' = B_1\times \cdots \times B_r \times C$ ;
\item
$ A'\to B_i$ is finite and there exists a unique prime $q_i\subset B_i$ lying
over $\mathfrak p'$ ; and
\item
all irreducible components of the fibre
$\text{Spec}(C\otimes_{A'} \kappa(\mathfrak p'))$ of $C$ over $\mathfrak p'$
have dimension at least 1.
\end{enumerate}
\end{theorem}

\begin{proof}
See Algebra, Lemma \ref{algebra-lemma-etale-makes-quasi-finite-finite}, or
see \cite[Th\'eor\`eme 18.12.1]{EGA4}.
\end{proof}

\noindent
Recall Hensel's lemma: if $f\in \mathbf{Z}_p[T]$ monic, $\bar{f}\pmod{p}$
factors as $\bar g\bar h$ with $gcd(\bar{g}, \bar{h})=1$ then $f$ factors
as $f = gh$ with $\bar{g}=\bar{g}, \; \bar{h}=\bar{h}$.
In particular, if $f \in \mathbf{Z}_p[T]$, monic
$\alpha\in \mathbf{F}_p$, $\bar f(\alpha) =0$ but $\bar f'(\alpha)\neq 0$
then $\exists $ root of $f$ in $\mathbf{Z}_p$ which residue to $\alpha$.

%10.01.09

\begin{definition}
\label{definition-henselian}
A local ring $(R, \mathfrak m, \kappa)$ is called {\it henselian} if for all
$f\in R[T]$ monic, for all $\alpha\in \kappa$ such that $\bar f(\alpha)=0$ and
$\bar f'(\alpha)\neq 0$, there exists $\tilde\alpha\in R$ such that
$f(\tilde\alpha) = 0$ and $\tilde\alpha\mod\mathfrak m = \alpha$.
\end{definition}

\noindent
Recall that a complete local ring is a local ring $(R, \mathfrak m)$ such that
$R\cong \text{lim}_n R/\mathfrak m^n$, i.e., it is complete and separated
for the $\mathfrak m$-adic topology.

\begin{theorem}
\label{theorem-hensel}
Complete local rings are henselian.
\end{theorem}

\begin{proof}
Newton's method.
\end{proof}

\begin{theorem}
\label{theorem-henselian}
Let $(R, \mathfrak m, \kappa)$ be a local ring. The following are equivalent:
\begin{enumerate}
\item $R$ is henselian ;
\item for any $f\in R[T]$ and any factorization $\bar f = g_0 h_0$ in
$\kappa[T]$ with $\gcd(g_0,h_0)=1$, there exists a factorization $f=gh$ in
$R[T]$ with $\bar g = g_0$ and $\bar h=h_0$ ;
\item any finite $R$-module is isomorphic to a product of (finite) local rings ;
\item any finite type $R$-algebra $A$ is isomorphic to a product $A \cong A'
\times C$ where $A' \cong A_1 \times \cdots \times A_r$ is a product of finite
local rings and all the irreducible components of $C\otimes_R\kappa$ have
dimension at least 1 ;
\item if $A$ is an \'etale $R$-algebra and $\mathfrak n$ is a maximal ideal of
$A$ lying over $\mathfrak m$ such that $\kappa \cong A/\mathfrak n$, then there
exists an isomorphism $\varphi: A \cong R\times A'$ such that
$\varphi(\mathfrak n) = \mathfrak m \times A'\subset R\times A'$.
\end{enumerate}
\end{theorem}

\begin{example}
\label{example-powerseries}
In the case $R = \mathbf{C}[[t]]$, the finite henselian $R$-algebras are the
trivial one $R \to R$ and the extensions
$\mathbf{C}[[t]] \to \mathbf{C}[[t]][X, X^{-1}]/(X^n-t)$.
The latter ones always miss the origin, so any
\'etale covering contains the identity and thus has the trivial covering as
refinement. We will see below that this is in fact a somewhat general fact and
this will give us the vanishing of higher direct images for a finite morphism.
\end{example}

\begin{lemma}
\label{lemma-finite-over-henselian}
If $R$ is henselian and $A$ is a finite $R$-module, then $A$ is a finite
product of henselian local rings.
\end{lemma}

\begin{definition}
\label{definition-strictly-henselian}
A local ring $R$ is called {\it strictly henselian} if it is henselian and its
residue field is separably closed.
\end{definition}

\begin{theorem}
\label{theorem-henselization}
Let $(R, \mathfrak m, \kappa)$ be a local ring and
$\kappa\subset\kappa^{sep}$ a separable closure. There exist canonical local
ring maps $ R\to R^\text{h} \to R^\text{sh}$ where
\begin{enumerate}
\item $R^\text{h}$, $R^\text{sh}$ are colimits of \'etale $R$-algebras ;
\item $R^\text{h}$, $R^\text{sh}$ are henselian ;
\item $\mathfrak m R^\text{h}$ (respectively $\mathfrak m R^\text{sh}$) is the
maximal ideal of $R^\text{h}$ (resp. $R^\text{sh}$) ; and
\item the first residue field extension is trivial $\kappa=R^\text{h}/\mathfrak
m R^\text{h}$, and the second is the separable closure $\kappa^{sep} =
R^\text{sh}/\mathfrak m R^\text{sh}$.
\end{enumerate}
Moreover,
$R^\text{sh} \cong \mathcal{O}^\text{sh}_{\text{Spec}(R), \;
\text{Spec}(\kappa^\text{sh})}$
as defined in \ref{definition-etale-local-rings}.
\end{theorem}

\begin{remark}
\label{remark-henselization-Noetherian}
If $R$ is noetherian then $R^\text{h}$ is also noetherian and they have the
same completion: $\hat R\cong \widehat{R^\text{h}}$. In particular, $R\subset
R^\text{h} \subset \hat R$. The henselization of $R$ is in general much
smaller than its completion and inherits many of its properties ({\it e.g} if
$R$ is reduced, then so is $R^\text{h}$, but not $\hat R$ in general).
\end{remark}




\section{Vanishing of finite higher direct images}
\label{section-vanishing-finite-morphism}

\noindent
The next goal is to prove that the higher direct images of a finite morphism of
schemes vanish.

\begin{lemma}
\label{lemma-vanishing-etale-cohomology-strictly-henselian}
Let $R$ be a strictly henselian ring and $S=\text{Spec}(R)$. Then the global
sections functor $\Gamma(S, -): \textit{Ab}(S_{et})\to \textit{Ab}$ is exact. In
particular
$$
\forall p\geq 1, \quad H_{et}^p(S, \mathcal{F})=0
$$
for all $\mathcal{F}\in \textit{Ab}(S_{et})$.
\end{lemma}

\begin{proof}
Let $\mathcal{U} = \left\{f_i : \mathcal{U}_i \to S \right\}_{i\in I}$ be an
\'etale covering, and denote $s$ the closed point of $S$. Then $s = f_i (u_i)$
for some $i\in I$ and some $u_i \in U_i$ by lemma
\ref{lemma-geometric-lift-to-cover}. Pick an affine open neighborhood
$\text{Spec} A$ of $u_i$ in $\mathcal{U}_i$. Then there is a commutative diagram
$$
\xymatrix{
R \ar[r] \ar[d] & A \ar[d] \\
{\kappa(s)} \ar[r] & {\kappa(u_i)}
}
$$
where $\kappa(s)$ is separably closed, and the residue extension is finite
separable. Therefore, $\kappa(s) \cong \kappa(u_i)$, and using part {\it v} of
theorem \ref{theorem-henselian}, we see that $A \cong R\times A'$ and we get a
section
$$
\xymatrix{
{\text{Spec} A\ }\ar[dr] \ar@{^{(}->}[r] & {\mathcal{U}_i} \ar[d]\\
& {S.} \ar@/^1pc/[ul]
}
$$
In particular, the covering $\left\{\text{id} : S\to S\right\}$ refines
$\mathcal{U}$. This implies that if
$$
0 \to \mathcal{F}_1\to \mathcal{F}_2 \xrightarrow{\alpha} \mathcal{F}_3\to 0
$$
is a short exact sequence in $\textit{Ab}(S_{et})$, then the sequence
$$
0 \to \Gamma(S_{et}, \mathcal{F}_1) \to \Gamma(S_{et}, \mathcal{F}_2) \to
\Gamma(S_{et}, \mathcal{F}_3)\to 0
$$
is also exact. Indeed, exactness is clear except possibly at the last step. But
given a section $s \in \Gamma(S_{et}, \mathcal{F}_3)$, we know that there exist
a covering $\mathcal{U}$ and local sections $s_i$ such that $\alpha (s_i) =
s|_{\mathcal{U}_i}$. But since this covering can be refined by the identity,
the $s_i$ must agree locally with $s$, hence they glue to a global section of
$\mathcal{F}_2$.
\end{proof}

\begin{proposition}
\label{proposition-finite-higher-direct-image-zero}
Let $f: X\to Y$ be a finite morphism of schemes. Then for all $q\geq 1$ and all
$\mathcal{F}\in \textit{Ab}(X_{et})$, $R^q f_*\mathcal{F}=0$.
\end{proposition}

\begin{proof}
Let $X_{\bar y}^\text{sh}$ denote the fiber product $X\times_Y
\text{Spec}(\mathcal{O}_{Y, \bar y}^\text{sh})$. It suffices to show that for
all $q\geq 1$, $H_{et}^q(X_{\bar y}^\text{sh}, \mathcal{G})=0$. Since $f$ is
finite, $X_{\bar y}^\text{sh}$ is finite over $\text{Spec}(\mathcal{O}_{Y, \bar
y}^\text{sh})$, thus $X_{\bar y}^\text{sh} = \text{Spec} A$ for some ring $A$
finite over $\mathcal{O}_{Y, \bar y}^\text{sh}$. Since the latter is strictly
henselian, corollary \ref{lemma-finite-over-henselian} implies that $A$
is henselian and therefore splits as a product of henselian local rings $A_1
\times \cdots \times A_r$. Furthermore, $\kappa(\mathcal{O}_{Y, \bar
y}^\text{sh})$ is separably closed and for each $i$, the residue field
extension $\kappa(\mathcal{O}_{Y, \bar y}^\text{sh}) \subset \kappa(A_i)$ is
finite, hence $\kappa(A_i)$ is separably closed and $A_i$ is strictly
henselian. This implies that $\text{Spec} A = \coprod_{i=1}^r \text{Spec} A_i$,
and we can apply
Lemma \ref{lemma-vanishing-etale-cohomology-strictly-henselian} to get
the result.
\end{proof}





\section{Cohomology of a point}
\label{section-cohomology-point}

\begin{lemma}
\label{lemma-sheaves-point}
Let $K$ be a field and $K^{sep}$ a separable closure of $K$. Consider
$$
\mathcal{G} := \text{Aut}_{\text{Spec}(K)}(\text{Spec}(K^{sep}))^{opp} =
\text{Gal}(K^{sep} | K)
$$
as a topological group, and denote $\mathcal{G}\textit{-Sets}$
(resp.\ $\mathcal{C}^0(\mathcal{G}\text{-Sets})$)
the category of (resp.\ continuous) left $\mathcal{G}$-sets.
Then the functor
$$
\begin{matrix}
\text{Spec} K _{et} & \longrightarrow & \mathcal{G}\textit{-Sets} \\
(X\to\text{Spec} K) & \longmapsto & \text{Hom}_{\text{Spec}
K}\left(\text{Spec}(K^{sep}), X\right)
\end{matrix}
$$
is an equivalence of categories, with essential image
$\mathcal{C}^0(\mathcal{G}\text{-Sets})$.
\end{lemma}

\noindent
Recall that a $\mathcal{G}$-set is {\it continuous} if each of its elements
has an open stabilizer. This means that the action is continuous when
$\mathcal{G}$ is endowed with its profinite topology and the $\mathcal{G}$-sets
have the discrete topology.

\begin{proof}
Recall that $X$ is \'etale over $K$ if and only if $X=\coprod_{i\in I}
\text{Spec} K_i$ with $K_i | K$ finite and separable. Then use standard Galois
theory.
\end{proof}

\begin{remark}
\label{remark-covering-surjective}
Under the correspondence of the lemma, the coverings in
$\text{Spec}(K)_{et}$ correspond to surjective families of maps in
$\mathcal{C}^0(\mathcal{G}\text{-Sets})$.
\end{remark}

\begin{lemma}
\label{lemma-sheaves-equivalence-point}
The stalk functor
$$
\begin{matrix}
\textit{Sh}(\text{Spec} K_{et}) & \longrightarrow &
\mathcal{C}^0(\mathcal{G}\text{-Sets}) \\
\mathcal{F} & \longmapsto & \mathcal{F}_{\text{Spec} K^{sep}}
\end{matrix}
$$
is an equivalence of categories. In other words, every sheaf on
$\text{Spec}(K)_{et}$ is representable.
\end{lemma}

\begin{proof}
The category $\text{Spec} K_{et}$ has some extra structure (maybe pushouts or
something -- figure it out) which makes it automatic that all sheaves are
representable. In the language of Section \ref{section-G-sets}, we have
identifications $\textit{Sh}(\mathcal{T}_G) =
\textit{Sh}(\mathcal{G}\textit{-Sets}) = \mathcal{G}\textit{-Sets}$.
\end{proof}

\begin{lemma}
\label{lemma-compare-cohomology-point}
Let $\mathcal{F}$ be an abelian sheaf on $\text{Spec}(K)_{et}$. Then
\begin{enumerate}
\item the $\mathcal{G}$-module $M = \mathcal{F}_{\text{Spec}(K^{sep})}$ is
continuous ;
\item $H_{et}^0(\text{Spec}(K), \mathcal{F})=M^{\mathcal{G}}$ ; and
\item $H_{et}^q(\text{Spec}(K), \mathcal{F}) = H_{\mathcal{C}^0}^q(\mathcal{G},
M)$.
\end{enumerate}
\end{lemma}

\begin{proof}
Part {\it i} is clear (use that the stalk functor is exact). For {\it ii}, we
have
\begin{eqnarray*}
\Gamma(\text{Spec}(K)_{et}, \mathcal{F}) & = &
\text{Hom}_{\textit{Sh}(\text{Spec}(K)_{et})}(h_{\text{Spec}(K)}, \mathcal{F})
\\
& = & \text{Hom}_{\mathcal{C}^0(\mathcal{G}\text{-Sets})}(\{*\}, M) \\
& = & M^{\mathcal{G}}
\end{eqnarray*}
where the first identification is of Yoneda type, the second results from
Lemma \ref{lemma-sheaves-equivalence-point}
and the third is clear. Part {\it iii} is also a
straightforward consequence of lemma \ref{lemma-sheaves-equivalence-point}.
\end{proof}

\begin{example}
\label{example-sheaves-point}
Sheaves on $\text{Spec}(K)_{et}$.
\begin{enumerate}
\item The constant sheaf $\underline{\mathbf{Z}/n\mathbf{Z}}$ corresponds to
the module $\mathbf{Z}/n\mathbf{Z}$ with trivial action ;
\item the sheaf $\mathbf{G}_m|_{\text{Spec}(K)_{et}}$ corresponds to
$(K^{sep})^*$ with the canonical (left) $\mathcal{G}$-action ;
\item the sheaf $\mathbf{G}_a|_{\text{Spec}(K^{sep})}$ corresponds to
$(K^{sep}, +)$ with the canonical (left) $\mathcal{G}$-action ;
\item the sheaf $\mu_n|_{\text{Spec}(K^{sep})}$ corresponds to $\mu_n(K^{sep})$
with the canonical $\mathcal{G}$-action.
\end{enumerate}

\noindent
The same arguments as in the fpqc case (see
Section \ref{section-cech-cohomology})
give the following identifications for cohomology groups:
$$
\begin{matrix}
H_{et}^0(S_{et}, \mathbf{G}_m) & = & \Gamma(S, \mathcal{O}_S^*) \; ; \\
H_{et}^1(S_{et}, \mathbf{G}_m) & = & H^1(S, \mathcal{O}_S^*) \, = \, H^1(S,
\mathcal{O}_S^*) \, = \, \text{Pic}(S) \; ;\\
H_{et}^i(S_{et}, \mathbf{G}_a) & = & H_{Zar}^i(S, \mathcal{O}_S).
\end{matrix}
$$
Also, for any quasi-coherent sheaf $\mathcal{F}$ on $S_{et}$, $H^i(S_{et},
\mathcal{F}) = H_{Zar}^i(S, \mathcal{F})$.
In particular, this gives the following sequence of equalities
\begin{eqnarray*}
0 & = & \text{Pic}(\text{Spec}(K)) \\
& = & H_{et}^1(\text{Spec}(K)_{et}, \mathbf{G}_m) \\
& = & H^1_{\mathcal{C}^0}(\mathcal{G}, (K^{sep})^*)
\end{eqnarray*}
which is none other than Hilbert's 90 theorem. Similarly, for $i \geq 1$,
\begin{eqnarray*}
0 & = & H^i(\text{Spec}(K), \mathcal{O}) \\
& = & H_{et}^i(\text{Spec}(K)_{et}, \mathbf{G}_a) \\
& = & H^i(\mathcal{G}, (K^{sep}, +)).
\end{eqnarray*}
\end{example}





%10.06.09
\section{Galois action on stalks}
\label{section-galois-action-stalks}

\begin{definition}
\label{definition-algebraic-geometric-point}
Let $S$ be a scheme. A geometric point $\bar s$ of $S$ is called
{\it algebraic} if $\kappa(s) \subset \kappa(\bar s)$ is
algebraic, i.e., if $\kappa(\bar s)$ is a separable algebraic
closure of $\kappa(s)$.
\end{definition}

\begin{example}
\label{example-stupid}
The geometric point $\text{Spec} \mathbf{C} \to \text{Spec} \mathbf{Q}$ is not
algebraic.
\end{example}

\noindent
Stalks as sets. Let $S$ be a scheme, $\mathcal{F}$ an
\'etale sheaf on $X$, and $\bar s$ a geometric point of $S$. Then
$$
\mathcal{F}_{\bar s} = \left\{
(\mathcal{U},\bar u, t) \ \big| \ \mathcal{U} \to S \text{ is \'etale, } \bar u
: \bar s \to \mathcal{U} \text{ is an $S$-morphism and } t \in
\mathcal{F}(\mathcal{U})
\right\}
\big/\sim
$$
where $(\mathcal{U},\bar u, t) \sim (\mathcal{U}',\bar u', t')$ if there exist
an \'etale neighborhood $(\mathcal{U}'',\bar u'')$ of $\bar s$ and a
commutative diagram
$$
\xymatrix{
\bar s \ar^{\bar u''}[rd] \ar^{\bar u'}@/^1pc/[rrd] \ar^{\bar u}@/_1pc/[rdd] \\
& \mathcal{U}'' \ar^{\beta}[r] \ar^{\alpha}[d] & \mathcal{U}' \ar[d] \\
& \mathcal{U} \ar[r] & S
}
$$
such that $\alpha^*(t) = \beta^*(t')$ in $\mathcal{F}(\mathcal{U}'')$.

\medskip\noindent
Galois action on stalks.
Given an algebraic geometric point $\bar s$ of $S$, set $\mathcal{G} =
\text{Gal}(\kappa(\bar s) | \kappa(s))$ and define an action of $\mathcal{G}$
on $\mathcal{F}_{\bar s}$ as follows
$$
\begin{matrix}
\mathcal{G} & \times & \mathcal{F}_{\bar s} & \longrightarrow &
\mathcal{F}_{\bar s} \\
(\sigma & , & (\mathcal{U},\bar u, t)) & \longmapsto & (\mathcal{U},\bar u
\circ \text{Spec} \sigma, t).
\end{matrix}
$$
It is easy to check that this is a well-defined left action on the stalk
$\mathcal{F}_{\bar s}$. We can thus restate the theorem of last time as follows.

\begin{theorem}
\label{theorem-equivalence-sheaves-point}
The action of $\mathcal{G}$ on the stalks $\mathcal{F}_{\bar s}$ is continuous
(for the discrete topology on $\mathcal{F}_{\bar s}$. Moreover, if $\bar s=
\text{Spec} K$ then it induces an equivalence of categories
$$
\begin{matrix}
\textit{Sh}(S_{et}) & \longrightarrow & \{ \text{discrete $\mathcal{G}$-sets
with continuous action} \} \\
\mathcal{F} & \longmapsto & \mathcal{F}_{\bar s}.
\end{matrix}
$$
\end{theorem}

\noindent
In particular, the category $\textit{Ab}(S_{et})$ corresponds to the full
subcategory of discrete $\mathcal{G}$-modules, and we have the identification
$H_{et}^0(S,\mathcal{F}) = (\mathcal{F}_{\bar s})^\mathcal{G}$, and more
generally $H_{et}^q(S,\mathcal{F}) = H_{cont}^q (\mathcal{G}, \mathcal{F}_{\bar
s})$ for $q \geq 1$.

\section{Other chapters}

\begin{multicols}{2}
\begin{enumerate}
\item \hyperref[introduction-section-phantom]{Introduction}
\item \hyperref[conventions-section-phantom]{Conventions}
\item \hyperref[sets-section-phantom]{Set Theory}
\item \hyperref[categories-section-phantom]{Categories}
\item \hyperref[topology-section-phantom]{Topology}
\item \hyperref[sheaves-section-phantom]{Sheaves on Spaces}
\item \hyperref[algebra-section-phantom]{Commutative Algebra}
\item \hyperref[sites-section-phantom]{Sites and Sheaves}
\item \hyperref[homology-section-phantom]{Homological Algebra}
\item \hyperref[derived-section-phantom]{Derived Categories}
\item \hyperref[more-algebra-section-phantom]{More Algebra}
\item \hyperref[simplicial-section-phantom]{Simplicial Methods}
\item \hyperref[modules-section-phantom]{Sheaves of Modules}
\item \hyperref[sites-modules-section-phantom]{Modules on Sites}
\item \hyperref[injectives-section-phantom]{Injectives}
\item \hyperref[cohomology-section-phantom]{Cohomology of Sheaves}
\item \hyperref[sites-cohomology-section-phantom]{Cohomology on Sites}
\item \hyperref[hypercovering-section-phantom]{Hypercoverings}
\item \hyperref[schemes-section-phantom]{Schemes}
\item \hyperref[constructions-section-phantom]{Constructions of Schemes}
\item \hyperref[properties-section-phantom]{Properties of Schemes}
\item \hyperref[morphisms-section-phantom]{Morphisms of Schemes}
\item \hyperref[coherent-section-phantom]{Coherent Cohomology}
\item \hyperref[divisors-section-phantom]{Divisors}
\item \hyperref[limits-section-phantom]{Limits of Schemes}
\item \hyperref[varieties-section-phantom]{Varieties}
\item \hyperref[chow-section-phantom]{Chow Homology}
\item \hyperref[topologies-section-phantom]{Topologies on Schemes}
\item \hyperref[descent-section-phantom]{Descent}
\item \hyperref[more-morphisms-section-phantom]{More on Morphisms}
\item \hyperref[flat-section-phantom]{More on Flatness}
\item \hyperref[groupoids-section-phantom]{Groupoid Schemes}
\item \hyperref[more-groupoids-section-phantom]{More on Groupoid Schemes}
\item \hyperref[etale-section-phantom]{\'Etale Morphisms of Schemes}
\item \hyperref[etale-cohomology-section-phantom]{\'Etale Cohomology}
\item \hyperref[spaces-section-phantom]{Algebraic Spaces}
\item \hyperref[spaces-properties-section-phantom]{Properties of Algebraic Spaces}
\item \hyperref[spaces-morphisms-section-phantom]{Morphisms of Algebraic Spaces}
\item \hyperref[spaces-topologies-section-phantom]{Topologies on Algebraic Spaces}
\item \hyperref[spaces-descent-section-phantom]{Descent and Algebraic Spaces}
\item \hyperref[spaces-more-morphisms-section-phantom]{More on Morphisms of Spaces}
\item \hyperref[quot-section-phantom]{Quot and Hilbert Spaces}
\item \hyperref[stacks-section-phantom]{Stacks}
\item \hyperref[spaces-groupoids-section-phantom]{Groupoids in Algebraic Spaces}
\item \hyperref[spaces-more-groupoids-section-phantom]{More on Groupoids in Spaces}
\item \hyperref[bootstrap-section-phantom]{Bootstrap}
\item \hyperref[examples-stacks-section-phantom]{Examples of Stacks}
\item \hyperref[groupoids-quotients-section-phantom]{Quotients of Groupoids}
\item \hyperref[algebraic-section-phantom]{Algebraic Stacks}
\item \hyperref[criteria-section-phantom]{Criteria for Representability}
\item \hyperref[stacks-properties-section-phantom]{Properties of Algebraic Stacks}
\item \hyperref[stacks-morphisms-section-phantom]{Morphisms of Algebraic Stacks}
\item \hyperref[examples-section-phantom]{Examples}
\item \hyperref[exercises-section-phantom]{Exercises}
\item \hyperref[guide-section-phantom]{Guide to Literature}
\item \hyperref[desirables-section-phantom]{Desirables}
\item \hyperref[coding-section-phantom]{Coding Style}
\item \hyperref[fdl-section-phantom]{GNU Free Documentation License}
\item \hyperref[index-section-phantom]{Auto Generated Index}
\end{enumerate}
\end{multicols}


\bibliography{my}
\bibliographystyle{amsalpha}

\end{document}
