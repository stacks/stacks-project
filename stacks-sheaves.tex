\IfFileExists{stacks-project.cls}{%
\documentclass{stacks-project}
}{%
\documentclass{amsart}
}

% The following AMS packages are automatically loaded with
% the amsart documentclass:
%\usepackage{amsmath}
%\usepackage{amssymb}
%\usepackage{amsthm}

% For dealing with references we use the comment environment
\usepackage{verbatim}
\newenvironment{reference}{\comment}{\endcomment}
%\newenvironment{reference}{}{}
\newenvironment{slogan}{\comment}{\endcomment}
\newenvironment{history}{\comment}{\endcomment}

% For commutative diagrams you can use
% \usepackage{amscd}
\usepackage[all]{xy}

% We use 2cell for 2-commutative diagrams.
\xyoption{2cell}
\UseAllTwocells

% To put source file link in headers.
% Change "template.tex" to "this_filename.tex"
% \usepackage{fancyhdr}
% \pagestyle{fancy}
% \lhead{}
% \chead{}
% \rhead{Source file: \url{template.tex}}
% \lfoot{}
% \cfoot{\thepage}
% \rfoot{}
% \renewcommand{\headrulewidth}{0pt}
% \renewcommand{\footrulewidth}{0pt}
% \renewcommand{\headheight}{12pt}

\usepackage{multicol}

% For cross-file-references
\usepackage{xr-hyper}

% Package for hypertext links:
\usepackage{hyperref}

% For any local file, say "hello.tex" you want to link to please
% use \externaldocument[hello-]{hello}
\externaldocument[introduction-]{introduction}
\externaldocument[conventions-]{conventions}
\externaldocument[sets-]{sets}
\externaldocument[categories-]{categories}
\externaldocument[topology-]{topology}
\externaldocument[sheaves-]{sheaves}
\externaldocument[sites-]{sites}
\externaldocument[stacks-]{stacks}
\externaldocument[fields-]{fields}
\externaldocument[algebra-]{algebra}
\externaldocument[brauer-]{brauer}
\externaldocument[homology-]{homology}
\externaldocument[derived-]{derived}
\externaldocument[simplicial-]{simplicial}
\externaldocument[more-algebra-]{more-algebra}
\externaldocument[smoothing-]{smoothing}
\externaldocument[modules-]{modules}
\externaldocument[sites-modules-]{sites-modules}
\externaldocument[injectives-]{injectives}
\externaldocument[cohomology-]{cohomology}
\externaldocument[sites-cohomology-]{sites-cohomology}
\externaldocument[dga-]{dga}
\externaldocument[dpa-]{dpa}
\externaldocument[hypercovering-]{hypercovering}
\externaldocument[schemes-]{schemes}
\externaldocument[constructions-]{constructions}
\externaldocument[properties-]{properties}
\externaldocument[morphisms-]{morphisms}
\externaldocument[coherent-]{coherent}
\externaldocument[divisors-]{divisors}
\externaldocument[limits-]{limits}
\externaldocument[varieties-]{varieties}
\externaldocument[topologies-]{topologies}
\externaldocument[descent-]{descent}
\externaldocument[perfect-]{perfect}
\externaldocument[more-morphisms-]{more-morphisms}
\externaldocument[flat-]{flat}
\externaldocument[groupoids-]{groupoids}
\externaldocument[more-groupoids-]{more-groupoids}
\externaldocument[etale-]{etale}
\externaldocument[chow-]{chow}
\externaldocument[intersection-]{intersection}
\externaldocument[pic-]{pic}
\externaldocument[adequate-]{adequate}
\externaldocument[dualizing-]{dualizing}
\externaldocument[duality-]{duality}
\externaldocument[discriminant-]{discriminant}
\externaldocument[local-cohomology-]{local-cohomology}
\externaldocument[curves-]{curves}
\externaldocument[resolve-]{resolve}
\externaldocument[models-]{models}
\externaldocument[pione-]{pione}
\externaldocument[etale-cohomology-]{etale-cohomology}
\externaldocument[proetale-]{proetale}
\externaldocument[crystalline-]{crystalline}
\externaldocument[spaces-]{spaces}
\externaldocument[spaces-properties-]{spaces-properties}
\externaldocument[spaces-morphisms-]{spaces-morphisms}
\externaldocument[decent-spaces-]{decent-spaces}
\externaldocument[spaces-cohomology-]{spaces-cohomology}
\externaldocument[spaces-limits-]{spaces-limits}
\externaldocument[spaces-divisors-]{spaces-divisors}
\externaldocument[spaces-over-fields-]{spaces-over-fields}
\externaldocument[spaces-topologies-]{spaces-topologies}
\externaldocument[spaces-descent-]{spaces-descent}
\externaldocument[spaces-perfect-]{spaces-perfect}
\externaldocument[spaces-more-morphisms-]{spaces-more-morphisms}
\externaldocument[spaces-flat-]{spaces-flat}
\externaldocument[spaces-groupoids-]{spaces-groupoids}
\externaldocument[spaces-more-groupoids-]{spaces-more-groupoids}
\externaldocument[bootstrap-]{bootstrap}
\externaldocument[spaces-pushouts-]{spaces-pushouts}
\externaldocument[groupoids-quotients-]{groupoids-quotients}
\externaldocument[spaces-more-cohomology-]{spaces-more-cohomology}
\externaldocument[spaces-simplicial-]{spaces-simplicial}
\externaldocument[formal-spaces-]{formal-spaces}
\externaldocument[restricted-]{restricted}
\externaldocument[spaces-resolve-]{spaces-resolve}
\externaldocument[formal-defos-]{formal-defos}
\externaldocument[defos-]{defos}
\externaldocument[cotangent-]{cotangent}
\externaldocument[examples-defos-]{examples-defos}
\externaldocument[algebraic-]{algebraic}
\externaldocument[examples-stacks-]{examples-stacks}
\externaldocument[stacks-sheaves-]{stacks-sheaves}
\externaldocument[criteria-]{criteria}
\externaldocument[artin-]{artin}
\externaldocument[quot-]{quot}
\externaldocument[stacks-properties-]{stacks-properties}
\externaldocument[stacks-morphisms-]{stacks-morphisms}
\externaldocument[stacks-limits-]{stacks-limits}
\externaldocument[stacks-cohomology-]{stacks-cohomology}
\externaldocument[stacks-perfect-]{stacks-perfect}
\externaldocument[stacks-introduction-]{stacks-introduction}
\externaldocument[stacks-more-morphisms-]{stacks-more-morphisms}
\externaldocument[stacks-geometry-]{stacks-geometry}
\externaldocument[moduli-]{moduli}
\externaldocument[moduli-curves-]{moduli-curves}
\externaldocument[examples-]{examples}
\externaldocument[exercises-]{exercises}
\externaldocument[guide-]{guide}
\externaldocument[desirables-]{desirables}
\externaldocument[coding-]{coding}
\externaldocument[obsolete-]{obsolete}
\externaldocument[fdl-]{fdl}
\externaldocument[index-]{index}

% Theorem environments.
%
\theoremstyle{plain}
\newtheorem{theorem}[subsection]{Theorem}
\newtheorem{proposition}[subsection]{Proposition}
\newtheorem{lemma}[subsection]{Lemma}

\theoremstyle{definition}
\newtheorem{definition}[subsection]{Definition}
\newtheorem{example}[subsection]{Example}
\newtheorem{exercise}[subsection]{Exercise}
\newtheorem{situation}[subsection]{Situation}

\theoremstyle{remark}
\newtheorem{remark}[subsection]{Remark}
\newtheorem{remarks}[subsection]{Remarks}

\numberwithin{equation}{subsection}

% Macros
%
\def\lim{\mathop{\rm lim}\nolimits}
\def\colim{\mathop{\rm colim}\nolimits}
\def\Spec{\mathop{\rm Spec}}
\def\Hom{\mathop{\rm Hom}\nolimits}
\def\Ext{\mathop{\rm Ext}\nolimits}
\def\SheafHom{\mathop{\mathcal{H}\!{\it om}}\nolimits}
\def\SheafExt{\mathop{\mathcal{E}\!{\it xt}}\nolimits}
\def\Sch{\textit{Sch}}
\def\Mor{\mathop{\rm Mor}\nolimits}
\def\Ob{\mathop{\rm Ob}\nolimits}
\def\Sh{\mathop{\textit{Sh}}\nolimits}
\def\NL{\mathop{N\!L}\nolimits}
\def\proetale{{pro\text{-}\acute{e}tale}}
\def\etale{{\acute{e}tale}}
\def\QCoh{\textit{QCoh}}
\def\Ker{\mathop{\rm Ker}}
\def\Im{\mathop{\rm Im}}
\def\Coker{\mathop{\rm Coker}}
\def\Coim{\mathop{\rm Coim}}

%
% Macros for moduli stacks/spaces
%
\def\QCohstack{\mathcal{QC}\!{\it oh}}
\def\Cohstack{\mathcal{C}\!{\it oh}}
\def\Spacesstack{\mathcal{S}\!{\it paces}}
\def\Quotfunctor{{\rm Quot}}
\def\Hilbfunctor{{\rm Hilb}}
\def\Curvesstack{\mathcal{C}\!{\it urves}}
\def\Polarizedstack{\mathcal{P}\!{\it olarized}}
\def\Complexesstack{\mathcal{C}\!{\it omplexes}}
% \Pic is the operator that assigns to X its picard group, usage \Pic(X)
% \Picardstack_{X/B} denotes the Picard stack of X over B
% \Picardfunctor_{X/B} denotes the Picard functor of X over B
\def\Pic{\mathop{\rm Pic}\nolimits}
\def\Picardstack{\mathcal{P}\!{\it ic}}
\def\Picardfunctor{{\rm Pic}}
\def\Deformationcategory{\mathcal{D}\!{\it ef}}


% OK, start here.
%
\begin{document}

\title{Sheaves on Algebraic Stacks}


\maketitle

\phantomsection
\label{section-phantom}

\tableofcontents

\section{Introduction}
\label{section-introduction}

\noindent
There is a myriad of ways to think about sheaves on algebraic stacks.
In this chapter we discuss one approach, which is particularly well
adapted to our foundations for algebraic stacks. Whenever we introduce
a type of sheaves we will indicate the precise relationship with
similar notions in the literature.
The goal of this chapter is to state those results
that are either obviously true or straightforward to prove
and leave more intricate constructions till later.

\medskip\noindent
In fact, it turns out that to develop a fully fledged theory of
constructible \'etale sheaves and/or an adequate discussion of
derived categories of complexes $\mathcal{O}$-modules whose
cohomology sheaves are quasi-coherent takes a significant amount of work, see
\cite{olsson_sheaves}. We will return to this in
Cohomology of Stacks, Section \ref{stacks-cohomology-section-introduction}.

\medskip\noindent
In the literature and in research papers on sheaves on algebraic stacks
the lisse-\'etale site of an algebraic stack often plays a prominent role.
However, it is a problematic beast, because it turns out that a morphism of
algebraic stacks does not induce a morphism of lisse-\'etale topoi. We have
therefore made the design decision to avoid any mention of the lisse-\'etale
site as long as possible. Arguments that traditionally use the lisse-\'etale
site will be replaced by an argument using a {\v C}ech covering
in the site $\mathcal{X}_{smooth}$ defined below.

\medskip\noindent
Some of the notation, conventions and terminology in this chapter is awkward
and may seem backwards to the more experienced reader. This is intentional.
Please see Quot, Section \ref{quot-section-conventions} for an
explanation.




\section{Conventions}
\label{section-conventions}

\noindent
The conventions we use in this chapter are the same as those in the
chapter on algebraic stacks, see
Algebraic Stacks, Section \ref{algebraic-section-conventions}.
For convenience we repeat them here.

\medskip\noindent
We work in a suitable big fppf site $\Sch_{fppf}$ as in
Topologies, Definition \ref{topologies-definition-big-fppf-site}.
So, if not explicitly stated otherwise all schemes will be objects
of $\Sch_{fppf}$. We record what changes if you change the big
fppf site elsewhere (insert future reference here).

\medskip\noindent
We will always work relative to a base $S$ contained in $\Sch_{fppf}$.
And we will then work with the big fppf site $(\Sch/S)_{fppf}$, see
Topologies, Definition \ref{topologies-definition-big-small-fppf}.
The absolute case can be recovered by taking
$S = \Spec(\mathbf{Z})$.





\section{Presheaves}
\label{section-presheaves}

\noindent
In this section we define presheaves on categories fibred in groupoids
over $(\Sch/S)_{fppf}$, but most of the discussion works
for categories over any base category. This section also serves to
introduce the notation we will use later on.

\begin{definition}
\label{definition-presheaves}
Let $p : \mathcal{X} \to (\Sch/S)_{fppf}$ be a category fibred in
groupoids.
\begin{enumerate}
\item A {\it presheaf on $\mathcal{X}$} is a presheaf on the
underlying category of $\mathcal{X}$.
\item A {\it morphism of presheaves on $\mathcal{X}$} is a morphism of
presheaves on the underlying category of $\mathcal{X}$.
\end{enumerate}
We denote $\textit{PSh}(\mathcal{X})$ the category of presheaves on
$\mathcal{X}$.
\end{definition}

\noindent
This defines presheaves of sets. Of course we can also talk about
presheaves of pointed sets, abelian groups, groups, monoids, rings,
modules over a fixed ring, and lie algebras over a fixed field, etc.
The category of {\it abelian presheaves}, i.e., presheaves of abelian
groups, is denoted $\textit{PAb}(\mathcal{X})$.

\medskip\noindent
Let $f : \mathcal{X} \to \mathcal{Y}$ be a $1$-morphism of categories
fibred in groupoids over $(\Sch/S)_{fppf}$. Recall that this
means just that $f$ is a functor over $(\Sch/S)_{fppf}$.
The material in
Sites, Section \ref{sites-section-more-functoriality-PSh}
provides us with a pair of adjoint functors\footnote{These functors
will be denoted $f^{-1}$ and $f_*$ after
Lemma \ref{lemma-functoriality-sheaves}
has been proved.}
\begin{equation}
\label{equation-pushforward-pullback}
f^p : \textit{PSh}(\mathcal{Y}) \longrightarrow \textit{PSh}(\mathcal{X})
\quad\text{and}\quad
{}_pf : \textit{PSh}(\mathcal{X}) \longrightarrow \textit{PSh}(\mathcal{Y}).
\end{equation}
The adjointness is
$$
\Mor_{\textit{PSh}(\mathcal{X})}(f^p\mathcal{G}, \mathcal{F})
=
\Mor_{\textit{PSh}(\mathcal{Y})}(\mathcal{G}, {}_pf\mathcal{F})
$$
where $\mathcal{F} \in \Ob(\textit{PSh}(\mathcal{X}))$ and
$\mathcal{G} \in \Ob(\textit{PSh}(\mathcal{Y}))$. We call
$f^p\mathcal{G}$ the {\it pullback} of $\mathcal{G}$. It follows
from the definitions that
$$
f^p\mathcal{G}(x) = \mathcal{G}(f(x))
$$
for any $x \in \Ob(\mathcal{X})$. The presheaf ${}_pf\mathcal{F}$
is called the {\it pushforward} of $\mathcal{F}$. It is described
by the formula
$$
({}_pf\mathcal{F})(y) = \lim_{f(x) \to y} \mathcal{F}(x).
$$
The rest of this section should probably be moved to the chapter
on sites and in any case should be skipped on a first reading.

\begin{lemma}
\label{lemma-1-morphisms-presheaves}
Let $f : \mathcal{X} \to \mathcal{Y}$ and $g : \mathcal{Y} \to \mathcal{Z}$
be $1$-morphisms of categories fibred in groupoids over
$(\Sch/S)_{fppf}$. Then $(g \circ f)^p = f^p \circ g^p$ and
there is a canonical isomorphism
${}_p(g \circ f) \to {}_pg \circ {}_pf$
compatible with adjointness of $(f^p, {}_pf)$, $(g^p, {}_pg)$, and
$((g \circ f)^p, {}_p(g \circ f))$.
\end{lemma}

\begin{proof}
Let $\mathcal{H}$ be a presheaf on $\mathcal{Z}$. Then
$(g \circ f)^p\mathcal{H} = f^p (g^p\mathcal{H})$ is given
by the equalities
$$
(g \circ f)^p\mathcal{H}(x) = \mathcal{H}((g \circ f)(x))
= \mathcal{H}(g(f(x))) = f^p (g^p\mathcal{H})(x).
$$
We omit the verification that this is compatible with restriction maps.

\medskip\noindent
Next, we define the transformation ${}_p(g \circ f) \to {}_pg \circ {}_pf$.
Let $\mathcal{F}$ be a presheaf on $\mathcal{X}$.
If $z$ is an object of $\mathcal{Z}$ then we get a
category $\mathcal{J}$ of quadruples
$(x, f(x) \to y, y, g(y) \to z)$ and a category $\mathcal{I}$
of pairs $(x, g(f(x)) \to z)$. There is a canonical functor
$\mathcal{J} \to \mathcal{I}$ sending the object
$(x, \alpha : f(x) \to y, y, \beta : g(y) \to z)$ to
$(x, \beta \circ f(\alpha) : g(f(x)) \to z)$. This gives the arrow in
\begin{align*}
({}_p(g \circ f)\mathcal{F})(z) & =
\lim_{g(f(x)) \to z} \mathcal{F}(x) \\
& = \lim_\mathcal{I} \mathcal{F} \\
& \to \lim_\mathcal{J} \mathcal{F} \\
& = \lim_{g(y) \to z}
\Big(\lim_{f(x) \to y} \mathcal{F}(x)\Big) \\
& =
({}_pg \circ {}_pf\mathcal{F})(x)
\end{align*}
by
Categories, Lemma \ref{categories-lemma-functorial-limit}.
We omit the verification that this is compatible with restriction maps.
An alternative to this direct construction is to define
${}_p(g \circ f) \cong {}_pg \circ {}_pf$
as the unique map compatible with the adjointness properties. This also
has the advantage that one does not need to prove the compatibility.

\medskip\noindent
Compatibility with adjointness of $(f^p, {}_pf)$, $(g^p, {}_pg)$, and
$((g \circ f)^p, {}_p(g \circ f))$ means that given presheaves
$\mathcal{H}$ and $\mathcal{F}$ as above we have a commutative diagram
$$
\xymatrix{
\Mor_{\textit{PSh}(\mathcal{X})}(f^pg^p\mathcal{H}, \mathcal{F})
\ar@{=}[r] \ar@{=}[d] &
\Mor_{\textit{PSh}(\mathcal{Y})}(g^p\mathcal{H}, {}_pf\mathcal{F})
\ar@{=}[r] &
\Mor_{\textit{PSh}(\mathcal{Y})}(\mathcal{H}, {}_pg{}_pf\mathcal{F})
\\
\Mor_{\textit{PSh}(\mathcal{X})}((g \circ f)^p\mathcal{G}, \mathcal{F})
\ar@{=}[rr] & &
\Mor_{\textit{PSh}(\mathcal{Y})}(\mathcal{G}, {}_p(g \circ f)\mathcal{F})
\ar[u]
}
$$
Proof omitted.
\end{proof}

\begin{lemma}
\label{lemma-2-morphisms-presheaves}
Let $f, g : \mathcal{X} \to \mathcal{Y}$ be $1$-morphisms of categories
fibred in groupoids over $(\Sch/S)_{fppf}$. Let $t : f \to g$
be a $2$-morphism of categories fibred in groupoids over
$(\Sch/S)_{fppf}$. Assigned to $t$ there are canonical
isomorphisms of functors
$$
t^p : g^p \longrightarrow f^p
\quad\text{and}\quad
{}_pt : {}_pf \longrightarrow {}_pg
$$
which compatible with adjointness of $(f^p, {}_pf)$ and
$(g^p, {}_pg)$ and with
vertical and horizontal composition of $2$-morphisms.
\end{lemma}

\begin{proof}
Let $\mathcal{G}$ be a presheaf on $\mathcal{Y}$. Then
$t^p : g^p\mathcal{G} \to f^p\mathcal{G}$ is given by the family
of maps
$$
g^p\mathcal{G}(x) = \mathcal{G}(g(x))
\xrightarrow{\mathcal{G}(t_x)}
\mathcal{G}(f(x)) = f^p\mathcal{G}(x)
$$
parametrized by $x \in \Ob(\mathcal{X})$. This makes sense as
$t_x : f(x) \to g(x)$ and $\mathcal{G}$ is a contravariant functor.
We omit the verification that this is compatible with restriction
mappings.

\medskip\noindent
To define the transformation ${}_pt$ for $y \in \Ob(\mathcal{Y})$
define ${}_y^f\mathcal{I}$, resp.\ ${}_y^g\mathcal{I}$ to be the category
of pairs $(x, \psi : f(x) \to y)$, resp.\ $(x, \psi : g(x) \to y)$, see
Sites, Section \ref{sites-section-more-functoriality-PSh}.
Note that $t$ defines a functor
${}_yt : {}_y^g\mathcal{I} \to {}_y^f\mathcal{I}$
given by the rule
$$
(x, g(x) \to y) \longmapsto (x, f(x) \xrightarrow{t_x} g(x) \to y).
$$
Note that for $\mathcal{F}$ a presheaf on $\mathcal{X}$ the composition
of ${}_yt$ with $\mathcal{F} : {}_y^f\mathcal{I}^{opp} \to \textit{Sets}$,
$(x, f(x) \to y) \mapsto \mathcal{F}(x)$ is equal to
$\mathcal{F} : {}_y^g\mathcal{I}^{opp} \to \textit{Sets}$. Hence by
Categories, Lemma \ref{categories-lemma-functorial-limit}
we get for every $y \in \Ob(\mathcal{Y})$ a canonical map
$$
({}_pf\mathcal{F})(y) = \lim_{{}_y^f\mathcal{I}} \mathcal{F}
\longrightarrow
\lim_{{}_y^g\mathcal{I}} \mathcal{F} = ({}_pg\mathcal{F})(y)
$$
We omit the verification that this is compatible with restriction
mappings. An alternative to this direct construction is to define
${}_pt$ as the unique map compatible with the adjointness properties
of the pairs $(f^p, {}_pf)$ and $(g^p, {}_pg)$ (see below). This also
has the advantage that one does not need to prove the compatibility.

\medskip\noindent
Compatibility with adjointness of $(f^p, {}_pf)$ and $(g^p, {}_pg)$ means
that given presheaves $\mathcal{G}$ and $\mathcal{F}$ as above we have
a commutative diagram
$$
\xymatrix{
\Mor_{\textit{PSh}(\mathcal{X})}(f^p\mathcal{G}, \mathcal{F})
\ar@{=}[r] \ar[d]_{- \circ t^p} &
\Mor_{\textit{PSh}(\mathcal{Y})}(\mathcal{G}, {}_pf\mathcal{F})
\ar[d]^{{}_pt \circ -} \\
\Mor_{\textit{PSh}(\mathcal{X})}(g^p\mathcal{G}, \mathcal{F})
\ar@{=}[r] &
\Mor_{\textit{PSh}(\mathcal{Y})}(\mathcal{G}, {}_pg\mathcal{F})
}
$$
Proof omitted. Hint: Work through the proof of
Sites, Lemma \ref{sites-lemma-adjoints-pu}
and observe the compatibility from the explicit description of the
horizontal and vertical maps in the diagram.

\medskip\noindent
We omit the verification that this is compatible with vertical and horizontal
compositions. Hint: The proof of this for $t^p$ is straightforward and
one can conclude that this holds for the ${}_pt$ maps using compatibility
with adjointness.
\end{proof}







\section{Sheaves}
\label{section-sheaves}

\noindent
We first make an observation that is important and trivial
(especially for those readers who do not worry about set theoretical
issues).

\medskip\noindent
Consider a big fppf site $\Sch_{fppf}$ as in
Topologies, Definition \ref{topologies-definition-big-fppf-site}
and denote its underlying category $\Sch_\alpha$.
Besides being the underlying category of a fppf site,
the category $\Sch_\alpha$ can also can serve as the underlying
category for a big Zariski site, a big \'etale site, a big smooth site,
and a big syntomic site, see
Topologies, Remark \ref{topologies-remark-choice-sites}.
We denote these sites $\Sch_{Zar}$, $\Sch_\etale$,
$\Sch_{smooth}$, and $\Sch_{syntomic}$.
In this situation, since we have defined
the big Zariski site $(\Sch/S)_{Zar}$ of $S$,
the big \'etale site $(\Sch/S)_\etale$ of $S$,
the big smooth site $(\Sch/S)_{smooth}$ of $S$,
the big syntomic site $(\Sch/S)_{syntomic}$ of $S$, and
the big fppf site $(\Sch/S)_{fppf}$ of $S$
as the localizations (see
Sites, Section \ref{sites-section-localize})
$\Sch_{Zar}/S$, $\Sch_\etale/S$,
$\Sch_{smooth}/S$, $\Sch_{syntomic}/S$, and
$\Sch_{fppf}/S$
of these (absolute) big sites we see that all of these have the
same underlying category, namely $\Sch_\alpha/S$.

\medskip\noindent
It follows that if we have a category
$p : \mathcal{X} \to (\Sch/S)_{fppf}$ fibred in groupoids, then
$\mathcal{X}$ inherits a Zariski, \'etale, smooth, syntomic, and
fppf topology, see
Stacks, Definition \ref{stacks-definition-topology-inherited}.

\begin{definition}
\label{definition-inherited-topologies}
Let $\mathcal{X}$ be a category fibred in groupoids over
$(\Sch/S)_{fppf}$.
\begin{enumerate}
\item The {\it associated Zariski site}, denoted $\mathcal{X}_{Zar}$,
is the structure of site on $\mathcal{X}$ inherited from
$(\Sch/S)_{Zar}$.
\item The {\it associated \'etale site}, denoted $\mathcal{X}_\etale$,
is the structure of site on $\mathcal{X}$ inherited from
$(\Sch/S)_\etale$.
\item The {\it associated smooth site}, denoted $\mathcal{X}_{smooth}$,
is the structure of site on $\mathcal{X}$ inherited from
$(\Sch/S)_{smooth}$.
\item The {\it associated syntomic site}, denoted $\mathcal{X}_{syntomic}$,
is the structure of site on $\mathcal{X}$ inherited from
$(\Sch/S)_{syntomic}$.
\item The {\it associated fppf site}, denoted $\mathcal{X}_{fppf}$,
is the structure of site on $\mathcal{X}$ inherited from
$(\Sch/S)_{fppf}$.
\end{enumerate}
\end{definition}

\noindent
This definition makes sense by the discussion above. If $\mathcal{X}$
is an algebraic stack, the literature calls $\mathcal{X}_{fppf}$ (or a
site equivalent to it) the {\it big fppf site} of $\mathcal{X}$ and similarly
for the other ones. We may occasionally use this terminology to
distinguish this construction from others.

\begin{remark}
\label{remark-ambiguity}
We only use this notation when the symbol $\mathcal{X}$ refers to a
category fibred in groupoids, and not a scheme, an algebraic space, etc.
In this way we will avoid confusion with the small \'etale site of a
scheme, or algebraic space which is denoted $X_\etale$ (in which
case we use a roman capital instead of a calligraphic one).
\end{remark}

\noindent
Now that we have these topologies defined we can say what it means
to have a sheaf on $\mathcal{X}$, i.e., define the corresponding topoi.

\begin{definition}
\label{definition-sheaves}
Let $\mathcal{X}$ be a category fibred in groupoids over
$(\Sch/S)_{fppf}$. Let $\mathcal{F}$ be a presheaf on $\mathcal{X}$.
\begin{enumerate}
\item We say $\mathcal{F}$ is a {\it Zariski sheaf}, or a
{\it sheaf for the Zariski topology} if $\mathcal{F}$
is a sheaf on the associated Zariski site $\mathcal{X}_{Zar}$.
\item We say $\mathcal{F}$ is an {\it \'etale sheaf}, or a
{\it sheaf for the \'etale topology} if $\mathcal{F}$
is a sheaf on the associated \'etale site $\mathcal{X}_\etale$.
\item We say $\mathcal{F}$ is a {\it smooth sheaf}, or a
{\it sheaf for the smooth topology} if $\mathcal{F}$
is a sheaf on the associated smooth site $\mathcal{X}_{smooth}$.
\item We say $\mathcal{F}$ is a {\it syntomic sheaf}, or a
{\it sheaf for the syntomic topology} if $\mathcal{F}$
is a sheaf on the associated syntomic site $\mathcal{X}_{syntomic}$.
\item We say $\mathcal{F}$ is an {\it fppf sheaf}, or a {\it sheaf},
or a {\it sheaf for the fppf topology} if $\mathcal{F}$
is a sheaf on the associated fppf site $\mathcal{X}_{fppf}$.
\end{enumerate}
A morphism of sheaves is just a morphism of presheaves. We denote
these categories of sheaves
$\Sh(\mathcal{X}_{Zar})$,
$\Sh(\mathcal{X}_\etale)$,
$\Sh(\mathcal{X}_{smooth})$,
$\Sh(\mathcal{X}_{syntomic})$, and
$\Sh(\mathcal{X}_{fppf})$.
\end{definition}

\noindent
Of course we can also talk about sheaves of pointed sets, abelian groups,
groups, monoids, rings, modules over a fixed ring, and lie algebras over
a fixed field, etc. The category of {\it abelian sheaves}, i.e., sheaves
of abelian groups, is denoted $\textit{Ab}(\mathcal{X}_{fppf})$
and similarly for the other topologies. If $\mathcal{X}$ is an algebraic
stack, then $\Sh(\mathcal{X}_{fppf})$ is equivalent (modulo
set theoretical problems) to what in the literature would be termed
the {\it category of sheaves on the big fppf site of $\mathcal{X}$}. Similar
for other topologies. We may occasionally use this terminology to
distinguish this construction from others.

\medskip\noindent
Since the topologies are listed in increasing order of strength we have
the following strictly full inclusions
$$
\Sh(\mathcal{X}_{fppf}) \subset
\Sh(\mathcal{X}_{syntomic}) \subset
\Sh(\mathcal{X}_{smooth}) \subset
\Sh(\mathcal{X}_\etale) \subset
\Sh(\mathcal{X}_{Zar}) \subset \textit{PSh}(\mathcal{X})
$$
We sometimes write
$\Sh(\mathcal{X}_{fppf}) = \Sh(\mathcal{X})$
and
$\textit{Ab}(\mathcal{X}_{fppf}) = \textit{Ab}(\mathcal{X})$
in accordance with our terminology that a sheaf on $\mathcal{X}$
is an fppf sheaf on $\mathcal{X}$.

\medskip\noindent
With this setup functoriality of these topoi is straightforward, and
moreover, is compatible with the inclusion functors above.

\begin{lemma}
\label{lemma-functoriality-sheaves}
Let $f : \mathcal{X} \to \mathcal{Y}$ be a $1$-morphism of categories
fibred in groupoids over $(\Sch/S)_{fppf}$. Let
$\tau \in \{Zar, \etale, smooth, syntomic, fppf\}$.
The functors ${}_pf$ and $f^p$ of (\ref{equation-pushforward-pullback})
transform $\tau$ sheaves into $\tau$ sheaves and define a morphism
of topoi
$f : \Sh(\mathcal{X}_\tau) \to \Sh(\mathcal{Y}_\tau)$.
\end{lemma}

\begin{proof}
This follows immediately from
Stacks, Lemma \ref{stacks-lemma-topology-inherited-functorial}.
\end{proof}

\noindent
In other words, pushforward and pullback of presheaves as defined in
Section \ref{section-presheaves}
also produces {\it pushforward} and {\it pullback} of $\tau$-sheaves.
Having said all of the above we see that we can write $f^p = f^{-1}$
and ${}_pf = f_*$ without any possibility of confusion.

\begin{definition}
\label{definition-morphism}
Let $f : \mathcal{X} \to \mathcal{Y}$ be a morphism of categories
fibred in groupoids over $(\Sch/S)_{fppf}$. We denote
$$
f = (f^{-1}, f_*) :
\Sh(\mathcal{X}_{fppf})
\longrightarrow
\Sh(\mathcal{Y}_{fppf})
$$
the {\it associated morphism of fppf topoi} constructed above.
Similarly for the associated Zariski, \'etale, smooth, and syntomic topoi.
\end{definition}

\noindent
As discussed in
Sites, Section \ref{sites-section-sheaves-algebraic-structures}
the same formula (on the underlying sheaf of sets) defines
pushforward and pullback for sheaves (for one of our topologies)
of pointed sets, abelian groups, groups, monoids, rings, modules
over a fixed ring, and lie algebras over a fixed field, etc.








\section{Computing pushforward}
\label{section-pushforward}

\noindent
Let $f : \mathcal{X} \to \mathcal{Y}$ be a $1$-morphism of categories
fibred in groupoids over $(\Sch/S)_{fppf}$. Let $\mathcal{F}$
be a presheaf on $\mathcal{X}$. Let $y \in \Ob(\mathcal{Y})$.
We can compute $f_*\mathcal{F}(y)$ in the following way. Suppose that
$y$ lies over the scheme $V$ and using the $2$-Yoneda lemma think
of $y$ as a $1$-morphism. Consider the projection
$$
\text{pr} :
(\Sch/V)_{fppf} \times_{y, \mathcal{Y}} \mathcal{X}
\longrightarrow
\mathcal{X}
$$
Then we have a canonical identification
\begin{equation}
\label{equation-pushforward}
f_*\mathcal{F}(y) = \Gamma\Big(
(\Sch/V)_{fppf} \times_{y, \mathcal{Y}} \mathcal{X},
\ \text{pr}^{-1}\mathcal{F}\Big)
\end{equation}
Namely, objects of the $2$-fibre product are triples
$(h : U \to V, x, f(x) \to h^*y)$. Dropping the $h$ from the
notation we see that this is equivalent to the data of an object
$x$ of $\mathcal{X}$ and a morphism $\alpha : f(x) \to y$ of $\mathcal{Y}$.
Since $f_*\mathcal{F}(y) = \lim_{f(x) \to y} \mathcal{F}(x)$ by definition
the equality follows.

\medskip\noindent
As a consequence we have the following ``base change'' result for
pushforwards. This result is trivial and hinges on the fact that
we are using ``big'' sites.

\begin{lemma}
\label{lemma-base-change}
Let $S$ be a scheme. Let
$$
\xymatrix{
\mathcal{Y}' \times_\mathcal{Y} \mathcal{X} \ar[r]_{g'} \ar[d]_{f'} &
\mathcal{X} \ar[d]^f \\
\mathcal{Y}' \ar[r]^g & \mathcal{Y}
}
$$
be a $2$-cartesian diagram of categories fibred in groupoids over $S$.
Then we have a canonical isomorphism
$$
g^{-1}f_*\mathcal{F} \longrightarrow f'_*(g')^{-1}\mathcal{F}
$$
functorial in the presheaf $\mathcal{F}$ on $\mathcal{X}$.
\end{lemma}

\begin{proof}
Given an object $y'$ of $\mathcal{Y}'$ over $V$
there is an equivalence
$$
(\Sch/V)_{fppf} \times_{g(y'), \mathcal{Y}} \mathcal{X}
=
(\Sch/V)_{fppf} \times_{y', \mathcal{Y}'}
(\mathcal{Y}' \times_\mathcal{Y} \mathcal{X})
$$
Hence by (\ref{equation-pushforward}) a bijection
$g^{-1}f_*\mathcal{F}(y') \to f'_*(g')^{-1}\mathcal{F}(y')$.
We omit the verification that this is compatible with restriction
mappings.
\end{proof}

\noindent
In the case of a representable morphism of categories fibred in groupoids
this formula (\ref{equation-pushforward}) simplifies. We suggest the
reader skip the rest of this section.

\begin{lemma}
\label{lemma-representable}
Let $f : \mathcal{X} \to \mathcal{Y}$ be a $1$-morphism of categories
fibred in groupoids over $(\Sch/S)_{fppf}$. The following are
equivalent
\begin{enumerate}
\item $f$ is representable, and
\item for every $y \in \Ob(\mathcal{Y})$ the functor
$\mathcal{X}^{opp} \to \textit{Sets}$,
$x \mapsto \Mor_\mathcal{Y}(f(x), y)$
is representable.
\end{enumerate}
\end{lemma}

\begin{proof}
According to the discussion in
Algebraic Stacks, Section \ref{algebraic-section-representable-morphism}
we see that $f$ is representable if and only if
for every $y \in \Ob(\mathcal{Y})$
lying over $U$ the $2$-fibre product
$(\Sch/U)_{fppf} \times_{y, \mathcal{Y}} \mathcal{X}$
is representable, i.e., of the form $(\Sch/V_y)_{fppf}$ for some
scheme $V_y$ over $U$. Objects in this $2$-fibre products are triples
$(h : V \to U, x, \alpha : f(x) \to h^*y)$ where $\alpha$ lies
over $\text{id}_V$. Dropping the $h$ from the notation we see that this
is equivalent to the data of an object $x$ of $\mathcal{X}$ and a
morphism $f(x) \to y$. Hence the $2$-fibre product is
representable by $V_y$ and $f(x_y) \to y$ where $x_y$ is an object
of $\mathcal{X}$ over $V_y$ if and only if the functor in (2) is representable
by $x_y$ with universal object a map $f(x_y) \to y$.
\end{proof}

\noindent
Let
$$
\xymatrix{
\mathcal{X} \ar[rr]_f \ar[rd]_p & &  \mathcal{Y} \ar[ld]^q \\
& (\Sch/S)_{fppf}
}
$$
be a $1$-morphism of categories fibred in groupoids. Assume $f$ is
representable. For every $y \in \Ob(\mathcal{Y})$ we choose
an object $u(y) \in \Ob(\mathcal{X})$ representing the functor
$x \mapsto \Mor_\mathcal{Y}(f(x), y)$ of
Lemma \ref{lemma-representable}
(this is possible by the axiom of choice).
The objects come with canonical morphisms $f(u(y)) \to y$ by
construction.
For every morphism $\beta : y' \to y$ in $\mathcal{Y}$ we obtain a unique
morphism $u(\beta) : u(y') \to u(y)$ in $\mathcal{X}$ such that the diagram
$$
\xymatrix{
f(u(y')) \ar[d] \ar[rr]_{f(u(\beta))} & & f(u(y)) \ar[d] \\
y' \ar[rr] & & y
}
$$
commutes. In other words, $u : \mathcal{Y} \to \mathcal{X}$ is a functor.
In fact, we can say a little bit more. Namely, suppose that
$V' = q(y')$, $V = q(y)$, $U' = p(u(y'))$ and $U = p(u(y))$. Then
$$
\xymatrix{
U' \ar[rr]_{p(u(\beta))} \ar[d] & & U \ar[d] \\
V' \ar[rr]^{q(\beta)} & & V
}
$$
is a fibre product square. This is true because $U' \to U$ represents
the base change
$(\Sch/V')_{fppf} \times_{y', \mathcal{Y}} \mathcal{X} \to
(\Sch/V)_{fppf} \times_{y, \mathcal{Y}} \mathcal{X}$
of $V' \to V$.

\begin{lemma}
\label{lemma-representable-pushforward}
Let $f : \mathcal{X} \to \mathcal{Y}$ be a representable $1$-morphism of
categories fibred in groupoids over $(\Sch/S)_{fppf}$. Let
$\tau \in \{Zar, \etale, smooth, syntomic, fppf\}$.
Then the functor $u : \mathcal{Y}_\tau \to \mathcal{X}_\tau$ is continuous
and defines a morphism of sites $\mathcal{X}_\tau \to \mathcal{Y}_\tau$
which induces the same morphism of topoi
$\Sh(\mathcal{X}_\tau) \to \Sh(\mathcal{Y}_\tau)$
as the morphism $f$ constructed in
Lemma \ref{lemma-functoriality-sheaves}.
Moreover, $f_*\mathcal{F}(y) = \mathcal{F}(u(y))$ for any presheaf
$\mathcal{F}$ on $\mathcal{X}$.
\end{lemma}

\begin{proof}
Let $\{y_i \to y\}$ be a $\tau$-covering in $\mathcal{Y}$. By definition
this simply means that $\{q(y_i) \to q(y)\}$ is a $\tau$-covering of
schemes. By the final remark above the lemma we see that
$\{p(u(y_i)) \to p(u(y))\}$ is the base change of the $\tau$-covering
$\{q(y_i) \to q(y)\}$ by $p(u(y)) \to q(y)$, hence is itself a
$\tau$-covering by the axioms of a site. Hence $\{u(y_i) \to u(y)\}$
is a $\tau$-covering of $\mathcal{X}$. This proves that $u$ is
continuous.

\medskip\noindent
Let's use the notation $u_p, u_s, u^p, u^s$ of
Sites, Sections \ref{sites-section-functoriality-PSh} and
\ref{sites-section-continuous-functors}.
If we can show the final assertion of the lemma, then we see that
$f_* = u^p = u^s$ (by continuity of $u$ seen above) and hence by adjointness
$f^{-1} = u_s$ which will prove $u_s$ is exact, hence that $u$ determines
a morphism of sites, and the equality will be clear as well.
To see that $f_*\mathcal{F}(y) = \mathcal{F}(u(y))$ note that by
definition
$$
f_*\mathcal{F}(y) = ({}_pf\mathcal{F})(y) =
\lim_{f(x) \to y} \mathcal{F}(x).
$$
Since $u(y)$ is a final object in the category the limit is taken
over we conclude.
\end{proof}





\section{The structure sheaf}
\label{section-structure-sheaf}

\noindent
Let $\tau \in \{Zar, \etale, smooth, syntomic, fppf\}$.
Let $p : \mathcal{X} \to (\Sch/S)_{fppf}$ be a category
fibred in groupoids. The 2-category of categories fibred in groupoids over
$(\Sch/S)_{fppf}$ has a final object, namely,
$\text{id} : (\Sch/S)_{fppf} \to (\Sch/S)_{fppf}$
and $p$ is a $1$-morphism from $\mathcal{X}$ to this final object.
Hence any presheaf $\mathcal{G}$ on $(\Sch/S)_{fppf}$ gives a
presheaf $p^{-1}\mathcal{G}$ on $\mathcal{X}$ defined by the rule
$p^{-1}\mathcal{G}(x) = \mathcal{G}(p(x))$. Moreover, the discussion in
Section \ref{section-sheaves}
shows that $p^{-1}\mathcal{G}$ is a $\tau$ sheaf whenever
$\mathcal{G}$ is a $\tau$-sheaf.

\medskip\noindent
Recall that the site $(\Sch/S)_{fppf}$ is a ringed site
with structure sheaf $\mathcal{O}$ defined by the rule
$$
(\Sch/S)^{opp} \longrightarrow \textit{Rings},
\quad
U/S \longmapsto \Gamma(U, \mathcal{O}_U)
$$
see
Descent, Definition \ref{descent-definition-structure-sheaf}.

\begin{definition}
\label{definition-structure-sheaf}
Let $p : \mathcal{X} \to (\Sch/S)_{fppf}$ be a category
fibred in groupoids. The
{\it structure sheaf of $\mathcal{X}$} is the sheaf of rings
$\mathcal{O}_\mathcal{X} = p^{-1}\mathcal{O}$.
\end{definition}

\noindent
For an object $x$ of $\mathcal{X}$ lying over $U$ we have
$\mathcal{O}_\mathcal{X}(x) = \mathcal{O}(U) = \Gamma(U, \mathcal{O}_U)$.
Needless to say $\mathcal{O}_\mathcal{X}$ is also a Zariski, \'etale,
smooth, and syntomic sheaf, and hence each of the sites
$\mathcal{X}_{Zar}$, $\mathcal{X}_\etale$, $\mathcal{X}_{smooth}$,
$\mathcal{X}_{syntomic}$, and $\mathcal{X}_{fppf}$ is a ringed site.
This construction is functorial as well.

\begin{lemma}
\label{lemma-functoriality-structure-sheaf}
Let $f : \mathcal{X} \to \mathcal{Y}$ be a $1$-morphism of categories
fibred in groupoids over $(\Sch/S)_{fppf}$. Let
$\tau \in \{Zar, \etale, smooth, syntomic, fppf\}$.
There is a canonical identification
$f^{-1}\mathcal{O}_\mathcal{X} = \mathcal{O}_\mathcal{Y}$
which turns
$f : \Sh(\mathcal{X}_\tau) \to \Sh(\mathcal{Y}_\tau)$
into a morphism of ringed topoi.
\end{lemma}

\begin{proof}
Denote $p : \mathcal{X} \to (\Sch/S)_{fppf}$ and
$q : \mathcal{Y} \to (\Sch/S)_{fppf}$ the structural functors.
Then $q = p \circ f$, hence $q^{-1} = f^{-1} \circ p^{-1}$ by
Lemma \ref{lemma-1-morphisms-presheaves}.
The result follows.
\end{proof}

\begin{remark}
\label{remark-flat}
In the situation of
Lemma \ref{lemma-functoriality-structure-sheaf}
the morphism of ringed topoi
$f : \Sh(\mathcal{X}_\tau) \to \Sh(\mathcal{Y}_\tau)$
is flat as is clear from the equality
$f^{-1}\mathcal{O}_\mathcal{X} = \mathcal{O}_\mathcal{Y}$.
This is a bit counter intuitive, for example because a closed
immersion of algebraic stacks is typically not flat (as a morphism of
algebraic stacks).
However, exactly the same thing happens when taking a closed
immersion $i : X \to Y$ of schemes: in this case the associated
morphism of big $\tau$-sites
$i : (\Sch/X)_\tau \to (\Sch/Y)_\tau$
also is flat.
\end{remark}




\section{Sheaves of modules}
\label{section-modules}

\noindent
Since we have a structure sheaf we have modules.

\begin{definition}
\label{definition-modules}
Let $\mathcal{X}$ be a category fibred in groupoids over
$(\Sch/S)_{fppf}$.
\begin{enumerate}
\item A {\it presheaf of modules on $\mathcal{X}$} is a
presheaf of $\mathcal{O}_\mathcal{X}$-modules. The category of
presheaves of modules is denoted $\textit{PMod}(\mathcal{O}_\mathcal{X})$.
\item We say a presheaf of modules $\mathcal{F}$ is an
{\it $\mathcal{O}_\mathcal{X}$-module}, or more precisely a
{\it sheaf of $\mathcal{O}_\mathcal{X}$-modules} if $\mathcal{F}$
is an fppf sheaf. The category of $\mathcal{O}_\mathcal{X}$-modules
is denoted $\textit{Mod}(\mathcal{O}_\mathcal{X})$.
\end{enumerate}
\end{definition}

\noindent
These (pre)sheaves of modules occur in the literature as {\it (pre)sheaves
of $\mathcal{O}_\mathcal{X}$-modules on the big fppf site of $\mathcal{X}$}.
We will occasionally use this terminology if we want to distinguish these
categories from others. We will also encounter presheaves of modules which
are sheaves in the Zariski, \'etale, smooth, or syntomic topologies
(without necessarily being sheaves). If need be these will be denoted
$\textit{Mod}(\mathcal{X}_\etale, \mathcal{O}_\mathcal{X})$
and similarly for the other topologies.

\medskip\noindent
Next, we address functoriality -- first for presheaves of modules. Let
$$
\xymatrix{
\mathcal{X} \ar[rr]_f \ar[rd]_p & &  \mathcal{Y} \ar[ld]^q \\
& (\Sch/S)_{fppf}
}
$$
be a $1$-morphism of categories fibred in groupoids.
The functors $f^{-1}$, $f_*$ on abelian presheaves extend to functors
\begin{equation}
\label{equation-functoriality-presheaves-modules}
f^{-1} :
\textit{PMod}(\mathcal{O}_\mathcal{Y})
\longrightarrow
\textit{PMod}(\mathcal{O}_\mathcal{X})
\quad\text{and}\quad
f_* :
\textit{PMod}(\mathcal{O}_\mathcal{X})
\longrightarrow
\textit{PMod}(\mathcal{O}_\mathcal{Y})
\end{equation}
This is immediate for $f^{-1}$ because
$f^{-1}\mathcal{G}(x) = \mathcal{G}(f(x))$ which is a module over
$\mathcal{O}_\mathcal{Y}(f(x)) = \mathcal{O}(q(f(x))) = \mathcal{O}(p(x)) =
\mathcal{O}_\mathcal{X}(x)$. Alternatively it follows because
$f^{-1}\mathcal{O}_\mathcal{Y} = \mathcal{O}_\mathcal{X}$
and because $f^{-1}$ commutes with limits (on presheaves).
Since $f_*$ is a right adjoint it commutes with all limits
(on presheaves) in particular products. Hence we can extend
$f_*$ to a functor on presheaves of modules as in the proof of
Modules on Sites, Lemma \ref{sites-modules-lemma-pushforward-module}.
We claim that the functors (\ref{equation-functoriality-presheaves-modules})
form an adjoint pair of functors:
$$
\Mor_{\textit{PMod}(\mathcal{O}_\mathcal{X})}(
f^{-1}\mathcal{G}, \mathcal{F})
=
\Mor_{\textit{PMod}(\mathcal{O}_\mathcal{Y})}(
\mathcal{G}, f_*\mathcal{F}).
$$
As $f^{-1}\mathcal{O}_\mathcal{Y} = \mathcal{O}_\mathcal{X}$
this follows from
Modules on Sites, Lemma \ref{sites-modules-lemma-adjoint-push-pull-modules}
by endowing $\mathcal{X}$ and $\mathcal{Y}$ with the chaotic
topology.

\medskip\noindent
Next, we discuss functoriality for modules, i.e., for sheaves of modules
in the fppf topology. Denote by $f$ also the induced morphism of ringed
topoi, see
Lemma \ref{lemma-functoriality-structure-sheaf}
(for the fppf topologies right now). Note that the functors
$f^{-1}$ and $f_*$ of (\ref{equation-functoriality-presheaves-modules})
preserve the subcategories of sheaves of modules, see
Lemma \ref{lemma-functoriality-sheaves}.
Hence it follows immediately that
\begin{equation}
\label{equation-functoriality-sheaves-modules}
f^{-1} :
\textit{Mod}(\mathcal{O}_\mathcal{Y})
\longrightarrow
\textit{Mod}(\mathcal{O}_\mathcal{X})
\quad\text{and}\quad
f_* :
\textit{Mod}(\mathcal{O}_\mathcal{X})
\longrightarrow
\textit{Mod}(\mathcal{O}_\mathcal{Y})
\end{equation}
form an adjoint pair of functors:
$$
\Mor_{\textit{Mod}(\mathcal{O}_\mathcal{X})}(
f^{-1}\mathcal{G}, \mathcal{F})
=
\Mor_{\textit{Mod}(\mathcal{O}_\mathcal{Y})}(
\mathcal{G}, f_*\mathcal{F}).
$$
By uniqueness of adjoints we conclude that
$f^* = f^{-1}$ where $f^*$ is as defined in
Modules on Sites, Section \ref{sites-modules-section-functoriality-modules}
for the morphism of ringed topoi $f$ above. Of course we could have
seen this directly because
$f^*(-) = f^{-1}(-) \otimes_{f^{-1}\mathcal{O}_\mathcal{Y}}
\mathcal{O}_\mathcal{X}$ and because
$f^{-1}\mathcal{O}_\mathcal{Y} = \mathcal{O}_\mathcal{X}$.

\medskip\noindent
Similarly for sheaves of modules in the Zariski, \'etale, smooth, syntomic
topology.



\section{Representable categories}
\label{section-representable}

\noindent
In this short section we compare our definitions with what happens
in case the algebraic stacks in question are representable.

\begin{lemma}
\label{lemma-compare-with-scheme}
Let $S$ be a scheme. Let $\mathcal{X}$ be a category fibred
in groupoids over $(\Sch/S)$. Assume $\mathcal{X}$ is representable
by a scheme $X$. For $\tau \in \{Zar,\linebreak[0] \etale,\linebreak[0]
smooth,\linebreak[0] syntomic,\linebreak[0] fppf\}$
there is a canonical equivalence
$$
(\mathcal{X}_\tau, \mathcal{O}_\mathcal{X}) =
((\Sch/X)_\tau, \mathcal{O}_X)
$$
of ringed sites.
\end{lemma}

\begin{proof}
This follows by choosing an equivalence
$(\Sch/X)_\tau \to \mathcal{X}$ of categories fibred in groupoids
over $(\Sch/S)_{fppf}$ and using the functoriality of
the construction $\mathcal{X} \leadsto \mathcal{X}_\tau$.
\end{proof}

\begin{lemma}
\label{lemma-compare-with-morphism-of-schemes}
Let $S$ be a scheme. Let $f : \mathcal{X} \to \mathcal{Y}$ be a morphism
of categories fibred in groupoids over $S$.
Assume $\mathcal{X}$, $\mathcal{Y}$ are representable by schemes
$X$, $Y$. Let $f : X \to Y$ be the morphism of schemes corresponding
to $f$. For $\tau \in \{Zar,\linebreak[0] \etale,\linebreak[0]
smooth,\linebreak[0] syntomic,\linebreak[0] fppf\}$
the morphism of ringed topoi
$f : (\Sh(\mathcal{X}_\tau), \mathcal{O}_\mathcal{X}) \to
(\Sh(\mathcal{Y}_\tau), \mathcal{O}_\mathcal{Y})$
agrees with the morphism of ringed topoi
$f : (\Sh((\Sch/X)_\tau), \mathcal{O}_X) \to 
(\Sh((\Sch/Y)_\tau), \mathcal{O}_Y)$ via the identifications of
Lemma \ref{lemma-compare-with-scheme}.
\end{lemma}

\begin{proof}
Follows by unwinding the definitions.
\end{proof}




\section{Restriction}
\label{section-restriction}


\noindent
A trivial but useful observation is that the localization
of a category fibred in groupoids at an object
is equivalent to the big site of the scheme it lies over.

\begin{lemma}
\label{lemma-localizing}
Let $p : \mathcal{X} \to (\Sch/S)_{fppf}$ be a category fibred
in groupoids. Let $\tau \in \{Zar, \etale, smooth, syntomic, fppf\}$.
Let $x \in \Ob(\mathcal{X})$ lying over $U = p(x)$.
The functor $p$ induces an equivalence of sites
$\mathcal{X}_\tau/x \to (\Sch/U)_\tau$.
\end{lemma}

\begin{proof}
Special case of Stacks, Lemma \ref{stacks-lemma-localizing}.
\end{proof}

\noindent
We use the lemma above to talk about the pullback and the restriction
of a (pre)sheaf to a scheme.

\begin{definition}
\label{definition-pullback}
Let $p : \mathcal{X} \to (\Sch/S)_{fppf}$ be a category fibred
in groupoids. Let $x \in \Ob(\mathcal{X})$ lying over $U = p(x)$.
Let $\mathcal{F}$ be a presheaf on $\mathcal{X}$.
\begin{enumerate}
\item The {\it pullback $x^{-1}\mathcal{F}$ of $\mathcal{F}$} is the
restriction $\mathcal{F}|_{(\mathcal{X}/x)}$ viewed as a presheaf on
$(\Sch/U)_{fppf}$ via the equivalence
$\mathcal{X}/x \to (\Sch/U)_{fppf}$ of
Lemma \ref{lemma-localizing}.
\item The {\it restriction of $\mathcal{F}$ to $U_\etale$}
is $x^{-1}\mathcal{F}|_{U_\etale}$, abusively written
$\mathcal{F}|_{U_\etale}$.
\end{enumerate}
\end{definition}

\noindent
This notation makes sense because to the object $x$ the $2$-Yoneda lemma, see
Algebraic Stacks, Section \ref{algebraic-section-2-yoneda}
associates a $1$-morphism $x : (\Sch/U)_{fppf} \to \mathcal{X}/x$
which is quasi-inverse to $p : \mathcal{X}/x \to (\Sch/U)_{fppf}$.
Hence $x^{-1}\mathcal{F}$ truly is the pullback of $\mathcal{F}$ via this
$1$-morphism. In particular, by the material above, if $\mathcal{F}$
is a sheaf (or a Zariski, \'etale, smooth, syntomic sheaf), then
$x^{-1}\mathcal{F}$ is a sheaf on $(\Sch/U)_{fppf}$ (or on
$(\Sch/U)_{Zar}$, $(\Sch/U)_\etale$,
$(\Sch/U)_{smooth}$, $(\Sch/U)_{syntomic}$).

\medskip\noindent
Let $p : \mathcal{X} \to (\Sch/S)_{fppf}$ be a category fibred
in groupoids. Let $\varphi : x \to y$ be a morphism of $\mathcal{X}$
lying over the morphism of schemes $a : U \to V$.
Recall that $a$ induces a morphism of small \'etale
sites $a_{small} : U_\etale \to V_\etale$, see
\'Etale Cohomology, Section \ref{etale-cohomology-section-functoriality}.
Let $\mathcal{F}$ be a presheaf on $\mathcal{X}$.
Let $\mathcal{F}|_{U_\etale}$ and
$\mathcal{F}|_{V_\etale}$ be the restrictions of $\mathcal{F}$
via $x$ and $y$. There is a natural {\it comparison} map
\begin{equation}
\label{equation-comparison-push}
c_\varphi :
\mathcal{F}|_{V_\etale}
\longrightarrow
a_{small, *}(\mathcal{F}|_{U_\etale})
\end{equation}
of presheaves on $U_\etale$. Namely, if $V' \to V$ is \'etale,
set $U' = V' \times_V U$ and define $c_\varphi$ on sections over $V'$
via
$$
\xymatrix{
a_{small, *}(\mathcal{F}|_{U_\etale})(V') &
\mathcal{F}|_{U_\etale}(U') \ar@{=}[l] &
\mathcal{F}(x') \ar@{=}[l] \\
\mathcal{F}|_{V_\etale}(V') \ar@{=}[rr] \ar[u]^{c_\varphi} &
&
\mathcal{F}(y') \ar[u]_{\mathcal{F}(\varphi')}
}
$$
Here $\varphi' : x' \to y'$ is a morphism of $\mathcal{X}$
fitting into a commutative diagram
$$
\vcenter{
\xymatrix{
x' \ar[r] \ar[d]_{\varphi'} & x \ar[d]^\varphi \\
y' \ar[r] & y
}
}
\quad\text{lying over}\quad
\vcenter{
\xymatrix{
U' \ar[r] \ar[d] & U \ar[d]^a \\
V' \ar[r] & V
}
}
$$
The existence and uniqueness of $\varphi'$ follow from the axioms
of a category fibred in groupoids.
We omit the verification that $c_\varphi$ so defined is indeed a map
of presheaves (i.e., compatible with restriction mappings) and that
it is functorial in $\mathcal{F}$. In case $\mathcal{F}$ is a sheaf for the
\'etale topology we obtain a {\it comparison} map
\begin{equation}
\label{equation-comparison}
c_\varphi : a_{small}^{-1}(\mathcal{F}|_{V_\etale})
\longrightarrow
\mathcal{F}|_{U_\etale}
\end{equation}
which is also denoted $c_\varphi$ as indicated (this is the customary abuse
of notation in not distinguishing between adjoint maps).

\begin{lemma}
\label{lemma-comparison}
Let $\mathcal{F}$ be an \'etale sheaf on $\mathcal{X} \to (\Sch/S)_{fppf}$.
\begin{enumerate}
\item If $\varphi : x \to y$ and $\psi : y \to z$
are morphisms of $\mathcal{X}$ lying over $a : U \to V$ and
$b : V \to W$, then the composition
$$
a_{small}^{-1}(b_{small}^{-1} (\mathcal{F}|_{W_\etale}))
\xrightarrow{a_{small}^{-1}c_\psi}
a_{small}^{-1}(\mathcal{F}|_{V_\etale})
\xrightarrow{c_\varphi}
\mathcal{F}|_{U_\etale}
$$
is equal to $c_{\psi \circ \varphi}$ via the identification
$$
(b \circ a)_{small}^{-1}(\mathcal{F}|_{W_\etale}) =
a_{small}^{-1}(b_{small}^{-1} (\mathcal{F}|_{W_\etale})).
$$
\item If $\varphi : x \to y$ lies over an \'etale morphism of schemes
$a : U \to V$, then (\ref{equation-comparison}) is an isomorphism.
\item Suppose $f : \mathcal{Y} \to \mathcal{X}$ is a $1$-morphism of
categories fibred in groupoids over $(\Sch/S)_{fppf}$ and $y$ is
an object of $\mathcal{Y}$ lying over the scheme $U$ with image
$x = f(y)$. Then there is a canonical identification
$f^{-1}\mathcal{F}|_{U_\etale} = \mathcal{F}|_{U_\etale}$.
\item Moreover, given $\psi : y' \to y$ in $\mathcal{Y}$ lying over
$a : U' \to U$ the comparison map
$c_\psi : a_{small}^{-1}(F^{-1}\mathcal{F}|_{U_\etale}) \to
F^{-1}\mathcal{F}|_{U'_\etale}$ is equal to the
comparison map $c_{f(\psi)} : a_{small}^{-1}\mathcal{F}|_{U_\etale}
\to \mathcal{F}|_{U'_\etale}$ via the identifications in (3).
\end{enumerate}
\end{lemma}

\begin{proof}
The verification of these properties is omitted.
\end{proof}

\noindent
Next, we turn to the restriction of (pre)sheaves of modules.

\begin{lemma}
\label{lemma-localizing-structure-sheaf}
Let $p : \mathcal{X} \to (\Sch/S)_{fppf}$ be a category fibred
in groupoids. Let $\tau \in \{Zar, \etale, smooth, syntomic, fppf\}$.
Let $x \in \Ob(\mathcal{X})$ lying over $U = p(x)$.
The equivalence of
Lemma \ref{lemma-localizing}
extends to an equivalence of ringed sites
$(\mathcal{X}_\tau/x, \mathcal{O}_\mathcal{X}|_x) \to
((\Sch/U)_\tau, \mathcal{O})$.
\end{lemma}

\begin{proof}
This is immediate from the construction of the structure sheaves.
\end{proof}

\noindent
Let $\mathcal{X}$ be a category fibred in groupoids over $(\Sch/S)_{fppf}$.
Let $\mathcal{F}$ be a (pre)sheaf of modules on $\mathcal{X}$ as in
Definition \ref{definition-modules}.
Let $x$ be an object of $\mathcal{X}$ lying over $U$. Then
Lemma \ref{lemma-localizing-structure-sheaf}
guarantees that the restriction
$x^{-1}\mathcal{F}$ is a (pre)sheaf of modules on $(\Sch/U)_{fppf}$.
We will sometimes write $x^*\mathcal{F} = x^{-1}\mathcal{F}$ in this case.
Similarly, if $\mathcal{F}$ is a sheaf for the Zariski, \'etale, smooth,
or syntomic topology, then $x^{-1}\mathcal{F}$ is as well. Moreover, the
restriction
$\mathcal{F}|_{U_\etale} = x^{-1}\mathcal{F}|_{U_\etale}$
to $U$ is a presheaf of $\mathcal{O}_{U_\etale}$-modules.
If $\mathcal{F}$ is a sheaf for the \'etale topology, then
$\mathcal{F}|_{U_\etale}$ is a sheaf of modules. Moreover,
if $\varphi : x \to y$ is a morphism of $\mathcal{X}$ lying over
$a : U \to V$ then the comparison map (\ref{equation-comparison})
is compatible with $a_{small}^\sharp$ (see
Descent, Remark \ref{descent-remark-change-topologies-ringed})
and induces a {\it comparison} map
\begin{equation}
\label{equation-comparison-modules}
c_\varphi : a_{small}^*(\mathcal{F}|_{V_\etale})
\longrightarrow
\mathcal{F}|_{U_\etale}
\end{equation}
of $\mathcal{O}_{U_\etale}$-modules.
Note that the properties (1), (2), (3), and (4) of
Lemma \ref{lemma-comparison}
hold in the setting of \'etale sheaves of modules as well.
We will use this in the following without further mention.

\begin{lemma}
\label{lemma-enough-points}
Let $p : \mathcal{X} \to (\Sch/S)_{fppf}$ be a category fibred
in groupoids. Let $\tau \in \{Zar, \etale, smooth, syntomic, fppf\}$.
The site $\mathcal{X}_\tau$ has enough points.
\end{lemma}

\begin{proof}
By
Sites, Lemma \ref{sites-lemma-enough-points-local}
we have to show that there exists a family of objects $x$ of $\mathcal{X}$
such that $\mathcal{X}_\tau/x$ has enough points and such that the sheaves
$h_x^\#$ cover the final object of the category of sheaves.
By
Lemma \ref{lemma-localizing}
and
\'Etale Cohomology, Lemma \ref{etale-cohomology-lemma-points-fppf}
we see that $\mathcal{X}_\tau/x$ has enough points for every object
$x$ and we win.
\end{proof}







\section{Restriction to algebraic spaces}
\label{section-restriction-algebraic-spaces}

\noindent
In this section we consider sheaves on categories representable by
algebraic spaces. The following lemma is the analogue of
Topologies, Lemma \ref{topologies-lemma-at-the-bottom-etale}
for algebraic spaces.

\begin{lemma}
\label{lemma-compare}
Let $S$ be a scheme. Let $\mathcal{X} \to (\Sch/S)_{fppf}$ be a category
fibred in groupoids. Assume $\mathcal{X}$ is representable by an algebraic
space $F$. Then there exists a continuous and cocontinuous functor
$
F_\etale \to \mathcal{X}_\etale
$
which induces a morphism of ringed sites
$$
\pi_F :
(\mathcal{X}_\etale, \mathcal{O}_\mathcal{X})
\longrightarrow
(F_\etale, \mathcal{O}_F)
$$
and a morphism of ringed topoi
$$
i_F :
(\Sh(F_\etale), \mathcal{O}_F)
\longrightarrow
(\Sh(\mathcal{X}_\etale), \mathcal{O}_\mathcal{X})
$$
such that $\pi_F \circ i_F = \text{id}$. Moreover $\pi_{F, *} = i_F^{-1}$.
\end{lemma}

\begin{proof}
Choose an equivalence $j : \mathcal{S}_F \to \mathcal{X}$, see
Algebraic Stacks, Sections \ref{algebraic-section-split} and
\ref{algebraic-section-representable-by-algebraic-spaces}.
An object of $F_\etale$ is a scheme $U$ together with an
\'etale morphism $\varphi : U \to F$. Then $\varphi$ is an object
of $\mathcal{S}_F$ over $U$. Hence $j(\varphi)$ is an object of
$\mathcal{X}$ over $U$. In this way $j$ induces a functor
$u : F_\etale \to \mathcal{X}$. It is clear that
$u$ is continuous and cocontinuous for the \'etale topology on
$\mathcal{X}$. Since $j$ is an equivalence, the functor $u$ is fully
faithful. Also, fibre products and equalizers exist in $F_\etale$
and $u$ commutes with them because these are computed on the level
of underlying schemes in $F_\etale$. Thus
Sites, Lemmas \ref{sites-lemma-when-shriek},
\ref{sites-lemma-preserve-equalizers}, and
\ref{sites-lemma-back-and-forth}
apply. In particular $u$ defines a morphism of topoi
$i_F : \Sh(F_\etale) \to \Sh(\mathcal{X}_\etale)$
and there exists a left adjoint $i_{F, !}$ of $i_F^{-1}$ which commutes
with fibre products and equalizers.

\medskip\noindent
We claim that $i_{F, !}$ is exact. If this is true, then we can define
$\pi_F$ by the rules $\pi_F^{-1} = i_{F, !}$ and $\pi_{F, *} = i_F^{-1}$
and everything is clear. To prove the claim, note that we already know
that $i_{F, !}$
is right exact and preserves fibre products. Hence it suffices to show
that $i_{F, !}* = *$ where $*$ indicates the final object in the category
of sheaves of sets. Let $U$ be a scheme and let
$\varphi : U \to F$ be surjective and \'etale. Set $R = U \times_F U$.
Then
$$
\xymatrix{
h_R \ar@<1ex>[r] \ar@<-1ex>[r] & h_U \ar[r] & {*}
}
$$
is a coequalizer diagram in $\Sh(F_\etale)$. Using the
right exactness of $i_{F, !}$, using $i_{F, !} = (u_p\ )^\#$, and using
Sites, Lemma \ref{sites-lemma-pullback-representable-presheaf}
we see that
$$
\xymatrix{
h_{u(R)} \ar@<1ex>[r] \ar@<-1ex>[r] & h_{u(U)} \ar[r] & i_{F, !}{*}
}
$$
is a coequalizer diagram in $\Sh(F_\etale)$. Using that
$j$ is an equivalence and that $F = U/R$ it follows that
the coequalizer in $\Sh(\mathcal{X}_\etale)$ of the
two maps $h_{u(R)} \to h_{u(U)}$ is $*$. We omit the proof that
these morphisms are compatible with structure sheaves.
\end{proof}

\noindent
Assume $\mathcal{X}$ is an algebraic stack represented by the
algebraic space $F$.
Let $j : \mathcal{S}_F \to \mathcal{X}$ be an equivalence and denote
$u : F_\etale \to \mathcal{X}_\etale$ the
functor of the proof of Lemma \ref{lemma-compare} above.
Given a sheaf $\mathcal{F}$ on $\mathcal{X}_\etale$ we have
$$
\pi_{F, *}\mathcal{F}(U) = i_F^{-1}\mathcal{F}(U) = \mathcal{F}(u(U)).
$$
This is why we often think of $i_F^{-1}$ as a {\it restriction functor}
similarly to
Definition \ref{definition-pullback}
and to the restriction of a sheaf on the big \'etale site of
a scheme to the small \'etale site of a scheme. We often use the notation
\begin{equation}
\label{equation-restrict}
\mathcal{F}|_{F_\etale} = i_F^{-1}\mathcal{F} = \pi_{F, *}\mathcal{F}
\end{equation}
in this situation.

\begin{lemma}
\label{lemma-compare-morphism}
Let $S$ be a scheme. Let $f : \mathcal{X} \to \mathcal{Y}$ be a morphism
of categories fibred in groupoids over $(\Sch/S)_{fppf}$. Assume
$\mathcal{X}$, $\mathcal{Y}$ are representable by algebraic spaces $F$, $G$.
Denote $f : F \to G$ the induced morphism of algebraic spaces, and
$f_{small} : F_\etale \to G_\etale$
the corresponding morphism of ringed topoi. Then
$$
\xymatrix{
(\Sh(\mathcal{X}_\etale), \mathcal{O}_\mathcal{X})
\ar[d]_{\pi_F} \ar[rr]_f & &
(\Sh(\mathcal{Y}_\etale), \mathcal{O}_\mathcal{Y}) \ar[d]^{\pi_G} \\
(\Sh(F_\etale), \mathcal{O}_F) \ar[rr]^{f_{small}} & &
(\Sh(G_\etale), \mathcal{O}_G)
}
$$
is a commutative diagram of ringed topoi.
\end{lemma}

\begin{proof}
This is similar to
Topologies, Lemma \ref{topologies-lemma-morphism-big-small-etale} (3)
but there is a small snag due to the fact that $F \to G$ may not be
representable by schemes. In particular we don't get a commutative diagram
of ringed sites, but only a commutative diagram of ringed topoi.

\medskip\noindent
Before we start the proof proper, we choose equivalences
$j : \mathcal{S}_F \to \mathcal{X}$ and
$j' : \mathcal{S}_G \to \mathcal{Y}$ which induce functors
$u : F_\etale \to \mathcal{X}$ and
$u' : G_\etale \to \mathcal{Y}$ as in the proof of
Lemma \ref{lemma-compare}. Because of the 2-functoriality of
sheaves on categories fibred in groupoids over $\Sch_{fppf}$
(see discussion in Section \ref{section-presheaves})
we may assume that $\mathcal{X} = \mathcal{S}_F$ and
$\mathcal{Y} = \mathcal{S}_G$ and that $f : \mathcal{S}_F \to \mathcal{S}_G$
is the functor associated to the morphism $f : F \to G$. Correspondingly
we will omit $u$ and $u'$ from the notation, i.e., given an object
$U \to F$ of $F_\etale$ we denote $U/F$
the corresponding object of $\mathcal{X}$. Similarly for $G$.

\medskip\noindent
Let $\mathcal{G}$ be a sheaf on $\mathcal{X}_\etale$.
To prove (2) we compute $\pi_{G, *}f_*\mathcal{G}$ and
$f_{small, *}\pi_{F, *}\mathcal{G}$. To do this let $V \to G$ be an object
of $G_\etale$. Then
$$
\pi_{G, *}f_*\mathcal{G}(V) = f_*\mathcal{G}(V/G) =
\Gamma\Big(
(\Sch/V)_{fppf} \times_{\mathcal{Y}} \mathcal{X},
\ \text{pr}^{-1}\mathcal{G}\Big)
$$
see (\ref{equation-pushforward}). The fibre product in the formula is
$$
(\Sch/V)_{fppf} \times_{\mathcal{Y}} \mathcal{X} =
(\Sch/V)_{fppf} \times_{\mathcal{S}_G} \mathcal{S}_F =
\mathcal{S}_{V \times_G F}
$$
i.e., it is the split category fibred in groupoids associated to the
algebraic space $V \times_G F$. And $\text{pr}^{-1}\mathcal{G}$ is a
sheaf on $\mathcal{S}_{V \times_G F}$ for the \'etale topology.

\medskip\noindent
In particular, if $V \times_G F$ is representable, i.e., if it is a scheme,
then $\pi_{G, *}f_*\mathcal{G}(V) = \mathcal{G}(V \times_G F/F)$ and
also
$$
f_{small, *}\pi_{F, *}\mathcal{G}(V) =
\pi_{F, *}\mathcal{G}(V \times_G F) =
\mathcal{G}(V \times_G F/F)
$$
which proves the desired equality in this special case.

\medskip\noindent
In general, choose a scheme $U$ and a surjective \'etale morphism
$U \to V \times_G F$. Set $R = U \times_{V \times_G F} U$. Then
$U/V \times_G F$ and $R/V \times_G F$ are objects of the fibre
product category above. Since $\text{pr}^{-1}\mathcal{G}$ is a
sheaf for the \'etale topology on $\mathcal{S}_{V \times_G F}$
the diagram
$$
\xymatrix{
\Gamma\Big(
(\Sch/V)_{fppf} \times_{\mathcal{Y}} \mathcal{X},
\ \text{pr}^{-1}\mathcal{G}\Big)
\ar[r] &
\text{pr}^{-1}\mathcal{G}(U/V \times_G F) \ar@<1ex>[r] \ar@<-1ex>[r] &
\text{pr}^{-1}\mathcal{G}(R/V \times_G F)
}
$$
is an equalizer diagram. Note that
$\text{pr}^{-1}\mathcal{G}(U/V \times_G F) = \mathcal{G}(U/F)$
and $\text{pr}^{-1}\mathcal{G}(R/V \times_G F) = \mathcal{G}(R/F)$
by the definition of pullbacks. Moreover, by the material in
Properties of Spaces, Section \ref{spaces-properties-section-etale-site}
(especially,
Properties of Spaces,
Remark \ref{spaces-properties-remark-explain-equivalence} and
Lemma \ref{spaces-properties-lemma-functoriality-etale-site})
we see that there is an equalizer diagram
$$
\xymatrix{
f_{small, *}\pi_{F, *}\mathcal{G}(V)
\ar[r] &
\pi_{F, *}\mathcal{G}(U/F) \ar@<1ex>[r] \ar@<-1ex>[r] &
\pi_{F, *}\mathcal{G}(R/F)
}
$$
Since we also have $\pi_{F, *}\mathcal{G}(U/F) = \mathcal{G}(U/F)$
and $\pi_{F, *}\mathcal{G}(U/F) = \mathcal{G}(U/F)$
we obtain a canonical identification
$f_{small, *}\pi_{F, *}\mathcal{G}(V) = \pi_{G, *}f_*\mathcal{G}(V)$.
We omit the proof that this is compatible with restriction mappings
and that it is functorial in $\mathcal{G}$.
\end{proof}

\noindent
Let $f : \mathcal{X} \to \mathcal{Y}$ and $f : F \to G$ be as in the
second part of the lemma above. A consequence of the lemma, using
(\ref{equation-restrict}), is that
\begin{equation}
\label{equation-compare-big-small}
(f_*\mathcal{F})|_{G_\etale} =
f_{small, *}(\mathcal{F}|_{F_\etale})
\end{equation}
for any sheaf $\mathcal{F}$ on $\mathcal{X}_\etale$.
Moreover, if $\mathcal{F}$ is a sheaf of $\mathcal{O}$-modules, then
(\ref{equation-compare-big-small}) is an isomorphism of
$\mathcal{O}_G$-modules on $G_\etale$.

\medskip\noindent
Finally, suppose that we have a $2$-commutative diagram
$$
\xymatrix{
\mathcal{U} \ar[r]^a \ar[dr]_f \drtwocell<\omit>{<-2>\varphi} &
\mathcal{V} \ar[d]^g \\
& \mathcal{X}
}
$$
of $1$-morphisms of categories fibred in groupoids over $(\Sch/S)_{fppf}$,
that $\mathcal{F}$ is a sheaf on $\mathcal{X}_\etale$,
and that $\mathcal{U}, \mathcal{V}$ are representable by algebraic
spaces $U, V$. Then we obtain a comparison map
\begin{equation}
\label{equation-comparison-algebraic-spaces}
c_\varphi : a_{small}^{-1}(g^{-1}\mathcal{F}|_{V_\etale})
\longrightarrow
f^{-1}\mathcal{F}|_{U_\etale}
\end{equation}
where $a : U \to V$ denotes the morphism of algebraic spaces corresponding
to $a$. This is the analogue of (\ref{equation-comparison}). We define
$c_\varphi$ as the adjoint to the map
$$
g^{-1}\mathcal{F}|_{V_\etale}
\longrightarrow
a_{small, *}(f^{-1}\mathcal{F}|_{U_\etale}) =
(a_*f^{-1}\mathcal{F})|_{V_\etale}
$$
(equality by (\ref{equation-compare-big-small}))
which is the restriction to $V$ (\ref{equation-restrict}) of the map
$$
g^{-1}\mathcal{F} \to a_*a^{-1}g^{-1}\mathcal{F} = a_*f^{-1}\mathcal{F}
$$
where the last equality uses the $2$-commutativity of the diagram above.
In case $\mathcal{F}$ is a sheaf of $\mathcal{O}_\mathcal{X}$-modules
$c_\varphi$ induces a {\it comparison} map
\begin{equation}
\label{equation-comparison-algebraic-spaces-modules}
c_\varphi : a_{small}^*(g^*\mathcal{F}|_{V_\etale})
\longrightarrow
f^*\mathcal{F}|_{U_\etale}
\end{equation}
of $\mathcal{O}_{U_\etale}$-modules.
Note that the properties (1), (2), (3), and (4) of
Lemma \ref{lemma-comparison}
hold in this setting as well.












\section{Quasi-coherent modules}
\label{section-quasi-coherent}

\noindent
At this point we can apply the general definition of a quasi-coherent
module to the situation discussed in this chapter.

\begin{definition}
\label{definition-quasi-coherent}
Let $p : \mathcal{X} \to (\Sch/S)_{fppf}$ be a category fibred
in groupoids. A {\it quasi-coherent module on $\mathcal{X}$}, or a
{\it quasi-coherent $\mathcal{O}_\mathcal{X}$-module} is a
quasi-coherent module on the ringed site
$(\mathcal{X}_{fppf}, \mathcal{O}_\mathcal{X})$ as in
Modules on Sites, Definition \ref{sites-modules-definition-site-local}.
The category of quasi-coherent sheaves on $\mathcal{X}$
is denoted $\QCoh(\mathcal{O}_\mathcal{X})$.
\end{definition}

\noindent
If $\mathcal{X}$ is an algebraic stack, then this definition agrees with all
definitions in the literature in the sense that $\QCoh(\mathcal{O}_\mathcal{X})$
is equivalent (modulo set theoretic issues) to any variant of this category
defined in the literature. For example, we will match our definition with
the definition in \cite[Definition 6.1]{olsson_sheaves} in
Cohomology on Stacks, Lemma \ref{lemma-quasi-coherent}.
We will also see alternative constructions of this category later on.

\medskip\noindent
In general (as is the case for morphisms of schemes) the pushforward
of quasi-coherent sheaf along a $1$-morphism is not quasi-coherent.
Pullback does preserve quasi-coherence.

\begin{lemma}
\label{lemma-pullback-quasi-coherent}
Let $f : \mathcal{X} \to \mathcal{Y}$ be a $1$-morphism of categories
fibred in groupoids over $(\Sch/S)_{fppf}$.
The pullback functor
$f^* = f^{-1} : \textit{Mod}(\mathcal{O}_\mathcal{Y}) \to
\textit{Mod}(\mathcal{O}_\mathcal{X})$
preserves quasi-coherent sheaves.
\end{lemma}

\begin{proof}
This is a general fact, see
Modules on Sites, Lemma \ref{sites-modules-lemma-local-pullback}.
\end{proof}

\noindent
It turns out that quasi-coherent sheaves have a very simple
characterization in terms of their pullbacks. See also
Lemma \ref{lemma-quasi-coherent}
for a characterization in terms of restrictions.

\begin{lemma}
\label{lemma-characterize-quasi-coherent}
Let $p : \mathcal{X} \to (\Sch/S)_{fppf}$ be a category
fibred in groupoids. Let $\mathcal{F}$
be a sheaf of $\mathcal{O}_\mathcal{X}$-modules. Then $\mathcal{F}$
is quasi-coherent if and only if $x^*\mathcal{F}$ is a quasi-coherent
sheaf on $(\Sch/U)_{fppf}$ for every object $x$ of
$\mathcal{X}$ with $U = p(x)$.
\end{lemma}

\begin{proof}
By
Lemma \ref{lemma-pullback-quasi-coherent}
the condition is necessary. Conversely, since $x^*\mathcal{F}$
is just the restriction to $\mathcal{X}_{fppf}/x$ we see that it
is sufficient directly from the definition of a quasi-coherent sheaf
(and the fact that the notion of being quasi-coherent is an intrinsic
property of sheaves of modules, see
Modules on Sites, Section \ref{sites-modules-section-intrinsic}).
\end{proof}

\begin{lemma}
\label{lemma-characterize-quasi-coherent-bis}
Let $p : \mathcal{X} \to (\Sch/S)_{fppf}$ be a category
fibred in groupoids. Let $\mathcal{F}$ be a presheaf of
modules on $\mathcal{X}$. The following are equivalent
\begin{enumerate}
\item $\mathcal{F}$ is an object of
$\textit{Mod}(\mathcal{X}_{Zar}, \mathcal{O}_\mathcal{X})$
and $\mathcal{F}$ is a quasi-coherent module on
$(\mathcal{X}_{Zar}, \mathcal{O}_\mathcal{X})$ in the sense of
Modules on Sites, Definition \ref{sites-modules-definition-site-local},
\item $\mathcal{F}$ is an object of
$\textit{Mod}(\mathcal{X}_\etale, \mathcal{O}_\mathcal{X})$
and $\mathcal{F}$ is a quasi-coherent module on
$(\mathcal{X}_\etale, \mathcal{O}_\mathcal{X})$ in the sense of
Modules on Sites, Definition \ref{sites-modules-definition-site-local}, and
\item $\mathcal{F}$ is a quasi-coherent module on $\mathcal{X}$
in the sense of Definition \ref{definition-quasi-coherent}.
\end{enumerate}
\end{lemma}

\begin{proof}
Assume either (1), (2), or (3) holds.
Let $x$ be an object of $\mathcal{X}$ lying over the scheme $U$.
Recall that $x^*\mathcal{F} = x^{-1}\mathcal{F}$ is just the
restriction to $\mathcal{X}/x = (\Sch/U)_\tau$ where
$\tau = fppf$, $\tau = \etale$, or $\tau = Zar$, see
Section \ref{section-restriction}.
By the definition of quasi-coherent modules on a ringed site
this restriction is quasi-coherent provided $\mathcal{F}$ is.
By Descent, Proposition \ref{descent-proposition-equivalence-quasi-coherent}
we see that $x^*\mathcal{F}$ is the sheaf associated to
a quasi-coherent $\mathcal{O}_U$-module and is therefore
a quasi-coherent module in the fppf, \'etale, and Zariski
topology; here we also use
Descent, Lemma \ref{descent-lemma-sheaf-condition-holds} and
Definition \ref{descent-definition-structure-sheaf}.
Since this holds for every object $x$ of $\mathcal{X}$,
we see that $\mathcal{F}$ is a sheaf in any of the three topologies.
Moreover, we find that $\mathcal{F}$ is quasi-coherent in any
of the three topologies directly from the definition of being
quasi-coherent and the fact that $x$ is an arbitrary object of $\mathcal{X}$.
\end{proof}

\noindent
Although there is a variant for the Zariski topology, it seems
that the \'etale topology is the natural topology to use in the
following definition.

\begin{definition}
\label{definition-locally-quasi-coherent}
Let $p : \mathcal{X} \to (\Sch/S)_{fppf}$ be a category
fibred in groupoids. Let $\mathcal{F}$
be a presheaf of $\mathcal{O}_\mathcal{X}$-modules.
We say $\mathcal{F}$ is {\it locally quasi-coherent}\footnote{This is
nonstandard notation.} if
$\mathcal{F}$ is a sheaf for the \'etale topology and
for every object $x$ of $\mathcal{X}$ the restriction
$x^*\mathcal{F}|_{U_\etale}$ is a quasi-coherent
sheaf. Here $U = p(x)$.
\end{definition}

\noindent
We use $\textit{LQCoh}(\mathcal{O}_\mathcal{X})$ to indicate the category of
locally quasi-coherent modules. We now have the following diagram
of categories of modules
$$
\xymatrix{
\QCoh(\mathcal{O}_\mathcal{X}) \ar[r] \ar[d] &
\textit{Mod}(\mathcal{O}_\mathcal{X}) \ar[d] \\
\textit{LQCoh}(\mathcal{O}_\mathcal{X}) \ar[r] &
\textit{Mod}(\mathcal{X}_\etale, \mathcal{O}_\mathcal{X})
}
$$
where the arrows are strictly full embeddings.
It turns out that many results for quasi-coherent sheaves have a
counter part for locally quasi-coherent modules. Moreover, from many
points of view (as we shall see later) this is a natural category to consider.
For example the quasi-coherent sheaves are exactly those
locally quasi-coherent modules that are ``cartesian'', i.e., satisfy
the second condition of the lemma below.

\begin{lemma}
\label{lemma-quasi-coherent}
Let $p : \mathcal{X} \to (\Sch/S)_{fppf}$ be a category
fibred in groupoids. Let $\mathcal{F}$
be a presheaf of $\mathcal{O}_\mathcal{X}$-modules. Then $\mathcal{F}$
is quasi-coherent if and only if the following two conditions hold
\begin{enumerate}
\item $\mathcal{F}$ is locally quasi-coherent, and
\item for any morphism $\varphi : x \to y$ of $\mathcal{X}$ lying over
$f : U \to V$ the comparison map
$c_\varphi : f_{small}^*\mathcal{F}|_{V_\etale} \to
\mathcal{F}|_{U_\etale}$ of
(\ref{equation-comparison-modules}) is an isomorphism.
\end{enumerate}
\end{lemma}

\begin{proof}
Assume $\mathcal{F}$ is quasi-coherent. Then $\mathcal{F}$ is a sheaf
for the fppf topology, hence a sheaf for the \'etale topology. Moreover,
any pullback of $\mathcal{F}$ to a ringed topos is quasi-coherent, hence
the restrictions $x^*\mathcal{F}|_{U_\etale}$ are quasi-coherent.
This proves $\mathcal{F}$ is locally quasi-coherent.
Let $y$ be an object of $\mathcal{X}$ with $V = p(y)$.
We have seen that $\mathcal{X}/y = (\Sch/V)_{fppf}$. By
Descent, Proposition \ref{descent-proposition-equivalence-quasi-coherent}
it follows that $y^*\mathcal{F}$ is the quasi-coherent module
associated to a (usual) quasi-coherent module $\mathcal{F}_V$ on
the scheme $V$. Hence certainly the comparison maps
(\ref{equation-comparison-modules}) are isomorphisms.

\medskip\noindent
Conversely, suppose that $\mathcal{F}$ satisfies (1) and (2).
Let $y$ be an object of $\mathcal{X}$ with $V = p(y)$. Denote
$\mathcal{F}_V$ the quasi-coherent module on
the scheme $V$ corresponding to the restriction
$y^*\mathcal{F}|_{V_\etale}$ which is quasi-coherent by
assumption (1), see
Descent, Proposition \ref{descent-proposition-equivalence-quasi-coherent}.
Condition (2) now signifies that the restrictions
$x^*\mathcal{F}|_{U_\etale}$ for $x$ over $y$ are each
isomorphic to the (\'etale sheaf associated to the) pullback of $\mathcal{F}_V$
via the corresponding morphism of schemes $U \to V$.
Hence $y^*\mathcal{F}$ is the sheaf on $(\Sch/V)_{fppf}$
associated to $\mathcal{F}_V$. Hence it is quasi-coherent (by
Descent, Proposition \ref{descent-proposition-equivalence-quasi-coherent}
again) and we see that $\mathcal{F}$ is quasi-coherent on $\mathcal{X}$ by
Lemma \ref{lemma-characterize-quasi-coherent}.
\end{proof}

\begin{lemma}
\label{lemma-pullback-lqc}
Let $f : \mathcal{X} \to \mathcal{Y}$ be a $1$-morphism of categories
fibred in groupoids over $(\Sch/S)_{fppf}$. The pullback functor
$f^* = f^{-1} :
\textit{Mod}(\mathcal{Y}_\etale, \mathcal{O}_\mathcal{Y})
\to
\textit{Mod}(\mathcal{X}_\etale, \mathcal{O}_\mathcal{X})$
preserves locally quasi-coherent sheaves.
\end{lemma}

\begin{proof}
Let $\mathcal{G}$ be locally quasi-coherent on $\mathcal{Y}$.
Choose an object $x$ of $\mathcal{X}$ lying over the scheme $U$.
The restriction $x^*f^*\mathcal{G}|_{U_\etale}$ equals
$(f \circ x)^*\mathcal{G}|_{U_\etale}$
hence is a quasi-coherent sheaf by assumption on $\mathcal{G}$.
\end{proof}

\begin{lemma}
\label{lemma-lqc-colimits}
Let $p : \mathcal{X} \to (\Sch/S)_{fppf}$ be a category fibred in
groupoids.
\begin{enumerate}
\item The category $\textit{LQCoh}(\mathcal{O}_\mathcal{X})$
has colimits and they agree with colimits in the category
$\textit{Mod}(\mathcal{X}_\etale, \mathcal{O}_\mathcal{X})$.
\item The category $\textit{LQCoh}(\mathcal{O}_\mathcal{X})$
is abelian with kernels and cokernels computed in
$\textit{Mod}(\mathcal{X}_\etale, \mathcal{O}_\mathcal{X})$,
in other words the inclusion functor is exact.
\item Given a short exact sequence
$0 \to \mathcal{F}_1 \to \mathcal{F}_2 \to \mathcal{F}_3 \to 0$ of
$\textit{Mod}(\mathcal{X}_\etale, \mathcal{O}_\mathcal{X})$
if two out of three are locally quasi-coherent so is the third.
\item Given $\mathcal{F}, \mathcal{G}$ in
$\textit{LQCoh}(\mathcal{O}_\mathcal{X})$
the tensor product $\mathcal{F} \otimes_{\mathcal{O}_\mathcal{X}} \mathcal{G}$
in $\textit{Mod}(\mathcal{X}_\etale, \mathcal{O}_\mathcal{X})$
is an object of $\textit{LQCoh}(\mathcal{O}_\mathcal{X})$.
\item Given $\mathcal{F}, \mathcal{G}$ in
$\textit{LQCoh}(\mathcal{O}_\mathcal{X})$
with $\mathcal{F}$ locally of finite presentation on
$\mathcal{X}_\etale$ the sheaf
$\SheafHom_{\mathcal{O}_\mathcal{X}}(\mathcal{F}, \mathcal{G})$
in $\textit{Mod}(\mathcal{X}_\etale, \mathcal{O}_\mathcal{X})$
is an object of $\textit{LQCoh}(\mathcal{O}_\mathcal{X})$.
\end{enumerate}
\end{lemma}

\begin{proof}
Each of these statements follows from the corresponding statement of
Descent, Lemma \ref{descent-lemma-equivalence-quasi-coherent-limits}.
For example, suppose that
$\mathcal{I} \to \textit{LQCoh}(\mathcal{O}_\mathcal{X})$,
$i \mapsto \mathcal{F}_i$ is a diagram.
Consider the object $\mathcal{F} = \colim_i \mathcal{F}_i$ of
$\textit{Mod}(\mathcal{X}_\etale, \mathcal{O}_\mathcal{X})$.
For any object $x$ of $\mathcal{X}$ with $U = p(x)$ the pullback functor
$x^*$ commutes with all colimits as it is a left adjoint. Hence
$x^*\mathcal{F} = \colim_i x^*\mathcal{F}_i$. Similarly we have
$x^*\mathcal{F}|_{U_\etale} =
\colim_i x^*\mathcal{F}_i|_{U_\etale}$.
Now by assumption each $x^*\mathcal{F}_i|_{U_\etale}$
is quasi-coherent, hence the colimit is quasi-coherent by the
aforementioned
Descent, Lemma \ref{descent-lemma-equivalence-quasi-coherent-limits}.
This proves (1).

\medskip\noindent
It follows from (1) that cokernels exist in
$\textit{LQCoh}(\mathcal{O}_\mathcal{X})$ and agree with the cokernels computed
in $\textit{Mod}(\mathcal{X}_\etale, \mathcal{O}_\mathcal{X})$.
Let $\varphi : \mathcal{F} \to \mathcal{G}$ be a morphism of
$\textit{LQCoh}(\mathcal{O}_\mathcal{X})$ and let
$\mathcal{K} = \Ker(\varphi)$ computed in
$\textit{Mod}(\mathcal{X}_\etale, \mathcal{O}_\mathcal{X})$.
If we can show that $\mathcal{K}$ is a locally quasi-coherent module,
then the proof of (2) is complete. To see this, note that kernels
are computed in the category of presheaves (no sheafification necessary).
Hence $\mathcal{K}|_{U_\etale}$ is the kernel of the map
$\mathcal{F}|_{U_\etale} \to \mathcal{G}|_{U_\etale}$,
i.e., is the kernel of a map of quasi-coherent sheaves on $U_\etale$
whence quasi-coherent by
Descent, Lemma \ref{descent-lemma-equivalence-quasi-coherent-limits}.
This proves (2).

\medskip\noindent
Parts (3), (4), and (5) follow in exactly the same way. Details omitted.
\end{proof}

\noindent
In the generality discussed here the category of quasi-coherent sheaves
is not abelian. See Examples, Section \ref{examples-section-nonabelian-QCoh}.
Here is what we can prove without any further work.

\begin{lemma}
\label{lemma-qc-colimits}
Let $p : \mathcal{X} \to (\Sch/S)_{fppf}$ be a category
fibred in groupoids.
\begin{enumerate}
\item The category $\QCoh(\mathcal{O}_\mathcal{X})$
has colimits and they agree with colimits in the categories
$\textit{Mod}(\mathcal{X}_{Zar}, \mathcal{O}_\mathcal{X})$,
$\textit{Mod}(\mathcal{X}_\etale, \mathcal{O}_\mathcal{X})$,
$\textit{Mod}(\mathcal{O}_\mathcal{X})$, and
$\textit{LQCoh}(\mathcal{O}_\mathcal{X})$.
\item Given $\mathcal{F}, \mathcal{G}$ in
$\QCoh(\mathcal{O}_\mathcal{X})$
the tensor product $\mathcal{F} \otimes_{\mathcal{O}_\mathcal{X}} \mathcal{G}$
in $\textit{Mod}(\mathcal{O}_\mathcal{X})$
is an object of $\QCoh(\mathcal{O}_\mathcal{X})$.
\item Given $\mathcal{F}, \mathcal{G}$ in
$\QCoh(\mathcal{O}_\mathcal{X})$
with $\mathcal{F}$ locally of finite presentation on
$\mathcal{X}_{fppf}$ the sheaf
$\SheafHom_{\mathcal{O}_\mathcal{X}}(\mathcal{F}, \mathcal{G})$
in $\textit{Mod}(\mathcal{O}_\mathcal{X})$
is an object of $\QCoh(\mathcal{O}_\mathcal{X})$.
\end{enumerate}
\end{lemma}

\begin{proof}
Let $\mathcal{I} \to \QCoh(\mathcal{O}_\mathcal{X})$,
$i \mapsto \mathcal{F}_i$ be a diagram.
Viewing $\mathcal{F}_i$ as quasi-coherent modules
in the Zariski topology (Lemma \ref{lemma-characterize-quasi-coherent-bis}), we
may consider the object $\mathcal{F} = \colim_i \mathcal{F}_i$ of
$\textit{Mod}(\mathcal{X}_{Zar}, \mathcal{O}_\mathcal{X})$.
For any object $x$ of $\mathcal{X}$ with $U = p(x)$ the restriction functor
$x^*$ (Section \ref{section-restriction})
commutes with all colimits as it is a left adjoint. Hence
$x^*\mathcal{F} = \colim_i x^*\mathcal{F}_i$ in
$\textit{Mod}((\Sch/U)_{Zar}, \mathcal{O})$.
Observe that $x_i^*\mathcal{F}_i$ is a quasi-coherent object
(because restrictions of quasi-coherent modules are quasi-coherent).
Thus by the equivalence in
Descent, Proposition \ref{descent-proposition-equivalence-quasi-coherent}
and by the compatibility with colimits in
Descent, Lemma \ref{descent-lemma-equivalence-quasi-coherent-limits}
we conclude that $x^*\mathcal{F}$ is quasi-coherent.
Thus $\mathcal{F}$ is quasi-coherent, see
Lemma \ref{lemma-characterize-quasi-coherent-bis}.
Thus we see that $\QCoh(\mathcal{O}_\mathcal{X})$
has colimits and they agree with colimits in the category
$\textit{Mod}(\mathcal{X}_{Zar}, \mathcal{O}_\mathcal{X})$.
Since the other categories listed are full subcategories
of $\textit{Mod}(\mathcal{X}_{Zar}, \mathcal{O}_\mathcal{X})$
we conclude part (1) holds.

\medskip\noindent
Parts (2) and (3) are proved in the same way.
Details omitted.
\end{proof}






\section{Stackification and sheaves}
\label{section-stackification}

\noindent
It turns out that the category of sheaves on a category fibred in
groupoids only ``knows about'' the stackification.

\begin{lemma}
\label{lemma-stackification}
Let $f : \mathcal{X} \to \mathcal{Y}$ be a $1$-morphism of categories
fibred in groupoids over $(\Sch/S)_{fppf}$. If
$f$ induces an equivalence of stackifications, then the morphism
of topoi
$f : \Sh(\mathcal{X}_{fppf}) \to \Sh(\mathcal{Y}_{fppf})$
is an equivalence.
\end{lemma}

\begin{proof}
We may assume $\mathcal{Y}$ is the stackification of $\mathcal{X}$.
We claim that $f : \mathcal{X} \to \mathcal{Y}$ is a special cocontinuous
functor, see
Sites, Definition \ref{sites-definition-special-cocontinuous-functor}
which will prove the lemma. By
Stacks, Lemma \ref{stacks-lemma-topology-inherited-functorial}
the functor $f$ is continuous and cocontinuous. By
Stacks, Lemma \ref{stacks-lemma-stackify}
we see that conditions (3), (4), and (5) of
Sites, Lemma \ref{sites-lemma-equivalence}
hold.
\end{proof}

\begin{lemma}
\label{lemma-stackification-quasi-coherent}
Let $f : \mathcal{X} \to \mathcal{Y}$ be a $1$-morphism of categories
fibred in groupoids over $(\Sch/S)_{fppf}$. If
$f$ induces an equivalence of stackifications, then $f^*$
induces equivalences
$\textit{Mod}(\mathcal{O}_\mathcal{X}) \to
\textit{Mod}(\mathcal{O}_\mathcal{Y})$
and
$\QCoh(\mathcal{O}_\mathcal{X}) \to
\QCoh(\mathcal{O}_\mathcal{Y})$.
\end{lemma}

\begin{proof}
We may assume $\mathcal{Y}$ is the stackification of $\mathcal{X}$.
The first assertion is clear from
Lemma \ref{lemma-stackification}
and
$\mathcal{O}_\mathcal{X} = f^{-1}\mathcal{O}_\mathcal{Y}$.
Pullback of quasi-coherent sheaves are quasi-coherent, see
Lemma \ref{lemma-pullback-quasi-coherent}.
Hence it suffices to show that if $f^*\mathcal{G}$ is
quasi-coherent, then $\mathcal{G}$ is.
To see this, let $y$ be an object of $\mathcal{Y}$.
Translating the condition that $\mathcal{Y}$ is the stackification
of $\mathcal{X}$ we see there exists an fppf covering $\{y_i \to y\}$
in $\mathcal{Y}$ such that $y_i \cong f(x_i)$ for some
$x_i$ object of $\mathcal{X}$. Say $x_i$ and $y_i$ lie over the scheme $U_i$.
Then $f^*\mathcal{G}$ being quasi-coherent, means that $x_i^*f^*\mathcal{G}$
is quasi-coherent. Since $x_i^*f^*\mathcal{G}$ is isomorphic to
$y_i^*\mathcal{G}$ (as sheaves on $(\Sch/U_i)_{fppf}$ we
see that $y_i^*\mathcal{G}$ is quasi-coherent.
It follows from
Modules on Sites, Lemma \ref{sites-modules-lemma-local-final-object}
that the restriction of $\mathcal{G}$ to $\mathcal{Y}/y$ is
quasi-coherent. Hence $\mathcal{G}$ is quasi-coherent by
Lemma \ref{lemma-characterize-quasi-coherent}.
\end{proof}





\section{Quasi-coherent sheaves and presentations}
\label{section-quasi-coherent-presentation}

\noindent
In
Groupoids in Spaces, Definition
\ref{spaces-groupoids-definition-groupoid-module}
we have the defined the notion of a quasi-coherent module
on an arbitrary groupoid. The following (formal) proposition tells us
that we can study quasi-coherent sheaves on quotient stacks in
terms of quasi-coherent modules on presentations.

\begin{proposition}
\label{proposition-quasi-coherent}
Let $(U, R, s, t, c)$ be a groupoid in algebraic spaces over $S$.
Let $\mathcal{X} = [U/R]$ be the quotient stack.
The category of quasi-coherent modules on $\mathcal{X}$
is equivalent to the category of quasi-coherent modules
on $(U, R, s, t, c)$.
\end{proposition}

\begin{proof}
Denote $\QCoh(U, R, s, t, c)$ the category of quasi-coherent modules
on the groupoid $(U, R, s, t, c)$. We will construct quasi-inverse functors
$$
\QCoh(\mathcal{O}_\mathcal{X})
\longleftrightarrow
\QCoh(U, R, s, t, c).
$$
According to
Lemma \ref{lemma-stackification-quasi-coherent}
the stackification map $[U/_{\!p}R] \to [U/R]$ (see
Groupoids in Spaces, Definition
\ref{spaces-groupoids-definition-quotient-stack})
induces an equivalence of categories of quasi-coherent sheaves.
Thus it suffices to prove the lemma with $\mathcal{X} = [U/_{\!p}R]$.

\medskip\noindent
Recall that an object $x = (T, u)$ of $\mathcal{X} = [U/_{\!p}R]$
is given by a scheme $T$ and a morphism $u : T \to U$. A morphism
$(T, u) \to (T', u')$ is given by a pair $(f, r)$ where $f : T \to T'$
and $r : T \to R$ with $s \circ r = u$ and $t \circ r = u' \circ f$.
Let us call a {\it special morphism} any morphism of the form
$(f, e \circ u' \circ f) : (T, u' \circ f) \to (T', u')$.
The category of $(T, u)$ with special morphisms is just the
category of schemes over $U$.

\medskip\noindent
Let $\mathcal{F}$ be a quasi-coherent sheaf on $\mathcal{X}$.
Then we obtain for every $x = (T, u)$ a quasi-coherent sheaf
$\mathcal{F}_{(T, u)} = x^*\mathcal{F}|_{T_\etale}$ on $T$.
Moreover, for any morphism $(f, r) : x = (T, u) \to (T', u') = x'$
we obtain a comparison isomorphism
$$
c_{(f, r)} :
f_{small}^*\mathcal{F}_{(T', u')}
\longrightarrow
\mathcal{F}_{(T, u)}
$$
see Lemma \ref{lemma-quasi-coherent}. Moreover, these isomorphisms are
compatible with compositions, see
Lemma \ref{lemma-comparison}.
If $U$, $R$ are schemes, then we can
construct the quasi-coherent sheaf on the groupoid as follows: First
the object $(U, \text{id})$ corresponds to a quasi-coherent sheaf
$\mathcal{F}_{(U, \text{id})}$ on $U$. Next, the isomorphism
$\alpha : t_{small}^*\mathcal{F}_{(U, \text{id})} \to
s_{small}^*\mathcal{F}_{(U, \text{id})}$
comes from
\begin{enumerate}
\item the morphism $(R, \text{id}_R) : (R, s) \to (R, t)$
in the category $[U/_{\!p}R]$ which produces an isomorphism
$\mathcal{F}_{(R, t)} \to \mathcal{F}_{(R, s)}$,
\item the special morphism $(R, s) \to (U, \text{id})$ which produces an
isomorphism
$s_{small}^*\mathcal{F}_{(U, \text{id})} \to \mathcal{F}_{(R, s)}$, and
\item the special morphism $(R, t) \to (U, \text{id})$ which produces an
isomorphism $t_{small}^*\mathcal{F}_{(U, \text{id})} \to \mathcal{F}_{(R, t)}$.
\end{enumerate}
The cocycle condition for $\alpha$ follows from the condition
that $(U, R, s, t, c)$ is groupoid, i.e., that composition is
associative (details omitted).

\medskip\noindent
To do this in general, i.e., when $U$ and $R$ are algebraic spaces,
it suffices to explain how to associate to an algebraic space $(W, u)$ over
$U$ a quasi-coherent sheaf $\mathcal{F}_{(W, u)}$ and to construct the
comparison maps for morphisms between these. We set
$\mathcal{F}_{(W, u)} = x^*\mathcal{F}|_{W_\etale}$
where $x$ is the $1$-morphism
$\mathcal{S}_W \to \mathcal{S}_U \to [U/_{\!p}R]$ and the comparison
maps are explained in (\ref{equation-comparison-algebraic-spaces-modules}).

\medskip\noindent
Conversely, suppose that $(\mathcal{G}, \alpha)$ is a quasi-coherent
module on $(U, R, s, t, c)$. We are going to define a presheaf of modules
$\mathcal{F}$ on $\mathcal{X}$ as follows. Given an object
$(T, u)$ of $[U/_{\!p}R]$ we set
$$
\mathcal{F}(T, u) : = \Gamma(T, u_{small}^*\mathcal{G}).
$$
Given a morphism $(f, r) : (T, u) \to (T', u')$ we get a map
\begin{align*}
\mathcal{F}(T', u') & = \Gamma(T', (u')_{small}^*\mathcal{G}) \\
& \to \Gamma(T, f_{small}^*(u')_{small}^*\mathcal{G}) =
\Gamma(T, (u' \circ f)_{small}^*\mathcal{G}) \\
& = \Gamma(T, (t \circ r)_{small}^*\mathcal{G}) =
\Gamma(T, r_{small}^*t_{small}^*\mathcal{G}) \\
& \to \Gamma(T, r_{small}^*s_{small}^*\mathcal{G}) =
\Gamma(T, (s \circ r)_{small}^*\mathcal{G}) \\
& = \Gamma(T, u_{small}^*\mathcal{G}) \\
& = \mathcal{F}(T, u)
\end{align*}
where the first arrow is pullback along $f$ and the second arrow is
$\alpha$. Note that if $(T, r)$ is a special morphism, then this
map is just pullback along $f$ as $e_{small}^*\alpha = \text{id}$ by
the axioms of a sheaf of quasi-coherent modules on a groupoid.
The cocycle condition implies that $\mathcal{F}$ is a presheaf
of modules (details omitted). It is immediate from the definition that
$\mathcal{F}$ is quasi-coherent when pulled back to
$(\Sch/T)_{fppf}$ (by the simple description of the
restriction maps of $\mathcal{F}$ in case of a special morphism).

\medskip\noindent
We omit the verification that the functors constructed above are
quasi-inverse to each other.
\end{proof}

\noindent
We finish this section with a technical lemma on maps out of quasi-coherent
sheaves. It is an analogue of
Schemes, Lemma \ref{schemes-lemma-compare-constructions}.
We will see later (Criteria for Representability, Theorem
\ref{criteria-theorem-flat-groupoid-gives-algebraic-stack})
that the assumptions on the groupoid imply that $\mathcal{X}$ is
an algebraic stack.

\begin{lemma}
\label{lemma-map-from-quasi-coherent}
Let $(U, R, s, t, c)$ be a groupoid in algebraic spaces over $S$.
Assume $s, t$ are flat and locally of finite presentation.
Let $\mathcal{X} = [U/R]$ be the quotient stack. Denote
$\pi : \mathcal{S}_U \to \mathcal{X}$ the quotient map.
Let $\mathcal{F}$ be a quasi-coherent
$\mathcal{O}_\mathcal{X}$-module, and let $\mathcal{H}$ be any object
of $\textit{Mod}(\mathcal{O}_\mathcal{X})$.
The map
$$
\Hom_{\mathcal{O}_\mathcal{X}}(\mathcal{F}, \mathcal{H})
\longrightarrow
\Hom_{\mathcal{O}_U}(x^*\mathcal{F}|_{U_\etale},
x^*\mathcal{H}|_{U_\etale}),
\quad
\phi \longmapsto x^*\phi|_{U_\etale}
$$
is injective and its image consists of exactly those
$\varphi : x^*\mathcal{F}|_{U_\etale} \to
x^*\mathcal{H}|_{U_\etale}$ which give rise to a commutative
diagram
$$
\xymatrix{
s_{small}^*(x^*\mathcal{F}|_{U_\etale})
\ar[r] \ar[d]^{s_{small}^*\varphi} &
(x \circ s)^*\mathcal{F}|_{R_\etale} =
(x \circ t)^*\mathcal{F}|_{R_\etale} &
t_{small}^*(x^*\mathcal{F}|_{U_\etale})
\ar[l] \ar[d]_{t_{small}^*\varphi} \\
s_{small}^*(x^*\mathcal{H}|_{U_\etale})
\ar[r] &
(x \circ s)^*\mathcal{H}|_{R_\etale} =
(x \circ t)^*\mathcal{H}|_{R_\etale} &
t_{small}^*(x^*\mathcal{H}|_{U_\etale})
\ar[l]
}
$$
of modules on $R_\etale$
where the horizontal arrows are the comparison maps
(\ref{equation-comparison-algebraic-spaces-modules}).
\end{lemma}

\begin{proof}
According to
Lemma \ref{lemma-stackification-quasi-coherent}
the stackification map $[U/_{\!p}R] \to [U/R]$ (see
Groupoids in Spaces, Definition
\ref{spaces-groupoids-definition-quotient-stack})
induces an equivalence of categories of quasi-coherent sheaves
and of fppf $\mathcal{O}$-modules.
Thus it suffices to prove the lemma with $\mathcal{X} = [U/_{\!p}R]$.
By Proposition \ref{proposition-quasi-coherent}
and its proof there exists a quasi-coherent module
$(\mathcal{G}, \alpha)$ on $(U, R, s, t, c)$ such that
$\mathcal{F}$ is given by the rule
$\mathcal{F}(T, u) = \Gamma(T, u^*\mathcal{G})$.
In particular $x^*\mathcal{F}|_{U_\etale} = \mathcal{G}$
and it is clear that the map of the statement of the
lemma is injective. Moreover, given a map
$\varphi : \mathcal{G} \to x^*\mathcal{H}|_{U_\etale}$
and given any object
$y = (T, u)$ of $[U/_{\!p}R]$ we can consider the map
$$
\mathcal{F}(y) = \Gamma(T, u^*\mathcal{G})
\xrightarrow{u_{small}^*\varphi}
\Gamma(T, u_{small}^*x^*\mathcal{H}|_{U_\etale})
\rightarrow
\Gamma(T, y^*\mathcal{H}|_{T_\etale}) = \mathcal{H}(y)
$$
where the second arrow is the comparison map
(\ref{equation-comparison-modules}) for the sheaf $\mathcal{H}$.
This assignment is compatible with the restriction mappings of the
sheaves $\mathcal{F}$ and $\mathcal{G}$ for morphisms of
$[U/_{\!p}R]$ if the cocycle condition of
the lemma is satisfied. Proof omitted. Hint: the restriction maps
of $\mathcal{F}$ are made explicit in terms of $(\mathcal{G}, \alpha)$
in the proof of
Proposition \ref{proposition-quasi-coherent}.
\end{proof}







\section{Quasi-coherent sheaves on algebraic stacks}
\label{section-quasi-coherent-algebraic-stacks}

\noindent
Let $\mathcal{X}$ be an algebraic stack over $S$. By
Algebraic Stacks, Lemma \ref{algebraic-lemma-stack-presentation}
we can find an equivalence $[U/R] \to \mathcal{X}$
where $(U, R, s, t, c)$ is a smooth groupoid in algebraic spaces.
Then
$$
\QCoh(\mathcal{O}_\mathcal{X})
\cong
\QCoh(\mathcal{O}_{[U/R]})
\cong
\QCoh(U, R, s, t, c)
$$
where the second equivalence is
Proposition \ref{proposition-quasi-coherent}.
Hence the category of quasi-coherent sheaves on an algebraic stack
is equivalent to the category of quasi-coherent modules on a smooth
groupoid in algebraic spaces. In particular, by
Groupoids in Spaces, Lemma \ref{spaces-groupoids-lemma-abelian}
we see that $\QCoh(\mathcal{O}_\mathcal{X})$ is abelian!

\medskip\noindent
There is something slightly disconcerting about our current setup.
It is that the fully faithful embedding
$$
\QCoh(\mathcal{O}_\mathcal{X})
\longrightarrow
\textit{Mod}(\mathcal{O}_\mathcal{X})
$$
is in general {\bf not} exact. However, exactly the same thing happens
for schemes: for most schemes $X$ the embedding
$$
\QCoh(\mathcal{O}_X) \cong
\QCoh((\Sch/X)_{fppf}, \mathcal{O}_X) \longrightarrow
\textit{Mod}((\Sch/X)_{fppf}, \mathcal{O}_X)
$$
isn't exact, see
Descent, Lemma \ref{descent-lemma-equivalence-quasi-coherent-limits}.
Parenthetically, the example in the proof of
Descent, Lemma \ref{descent-lemma-equivalence-quasi-coherent-limits}
shows that in general the strictly full embedding
$\QCoh(\mathcal{O}_\mathcal{X}) \to
\textit{LQCoh}(\mathcal{O}_\mathcal{X})$ isn't exact either.

\medskip\noindent
We collect all the positive results obtained so far in a single statement.

\begin{lemma}
\label{lemma-quasi-coherent-algebraic-stack}
Let $\mathcal{X}$ be an algebraic stack over $S$.
\begin{enumerate}
\item If $[U/R] \to \mathcal{X}$ is a presentation of $\mathcal{X}$
then there is a canonical equivalence
$\QCoh(\mathcal{O}_\mathcal{X}) \cong
\QCoh(U, R, s, t, c)$.
\item The category $\QCoh(\mathcal{O}_\mathcal{X})$ is abelian.
\item The category $\QCoh(\mathcal{O}_\mathcal{X})$
has colimits and they agree with colimits in the category
$\textit{Mod}(\mathcal{O}_\mathcal{X})$.
\item Given $\mathcal{F}, \mathcal{G}$ in
$\QCoh(\mathcal{O}_\mathcal{X})$
the tensor product $\mathcal{F} \otimes_{\mathcal{O}_\mathcal{X}} \mathcal{G}$
in $\textit{Mod}(\mathcal{O}_\mathcal{X})$
is an object of $\QCoh(\mathcal{O}_\mathcal{X})$.
\item Given $\mathcal{F}, \mathcal{G}$ in
$\QCoh(\mathcal{O}_\mathcal{X})$
with $\mathcal{F}$ locally of finite presentation on
$\mathcal{X}_{fppf}$ the sheaf
$\SheafHom_{\mathcal{O}_\mathcal{X}}(\mathcal{F}, \mathcal{G})$
in $\textit{Mod}(\mathcal{O}_\mathcal{X})$
is an object of $\QCoh(\mathcal{O}_\mathcal{X})$.
\end{enumerate}
\end{lemma}

\begin{proof}
Properties (3), (4), and (5) were proven in
Lemma \ref{lemma-qc-colimits}.
Part (1) is
Proposition \ref{proposition-quasi-coherent}.
Part (2) follows from
Groupoids in Spaces, Lemma \ref{spaces-groupoids-lemma-abelian}
as discussed above.
\end{proof}

\begin{proposition}
\label{proposition-coherator}
Let $\mathcal{X}$ be an algebraic stack over $S$.
\begin{enumerate}
\item The category $\QCoh(\mathcal{O}_\mathcal{X})$ is a Grothendieck
abelian category. Consequently, $\QCoh(\mathcal{O}_\mathcal{X})$
has enough injectives and all limits.
\item The inclusion functor
$\QCoh(\mathcal{O}_\mathcal{X}) \to
\textit{Mod}(\mathcal{O}_\mathcal{X})$ has a right adjoint\footnote{This
functor is sometimes called the {\it coherator}.}
$$
Q :
\textit{Mod}(\mathcal{O}_\mathcal{X})
\longrightarrow
\QCoh(\mathcal{O}_\mathcal{X})
$$
such that for every quasi-coherent sheaf $\mathcal{F}$ the adjunction mapping
$Q(\mathcal{F}) \to \mathcal{F}$ is an isomorphism.
\end{enumerate}
\end{proposition}

\begin{proof}
This proof is a repeat of the proof in the case of schemes, see
Properties, Proposition \ref{properties-proposition-coherator}
and the case of algebraic spaces, see
Properties of Spaces, Proposition
\ref{spaces-properties-proposition-coherator}.
We advise the reader to read either of those proofs first.

\medskip\noindent
Part (1) means $\QCoh(\mathcal{O}_\mathcal{X})$ (a) has all colimits,
(b) filtered colimits are exact, and (c) has a generator, see
Injectives, Section \ref{injectives-section-grothendieck-conditions}.
By Lemma \ref{lemma-quasi-coherent-algebraic-stack}
colimits in $\QCoh(\mathcal{O}_X)$ exist and agree
with colimits in $\textit{Mod}(\mathcal{O}_X)$. By
Modules on Sites, Lemma \ref{sites-modules-lemma-limits-colimits}
filtered colimits are exact. Hence (a) and (b) hold.

\medskip\noindent
Choose a presentation $\mathcal{X} = [U/R]$ so that $(U, R, s, t, c)$
is a smooth groupoid in algebraic spaces and in particular $s$ and $t$
are flat morphisms of algebraic spaces. By
Lemma \ref{lemma-quasi-coherent-algebraic-stack}
above we have
$\QCoh(\mathcal{O}_\mathcal{X}) = \QCoh(U, R, s, t, c)$.
By Groupoids in Spaces, Lemma \ref{spaces-groupoids-lemma-set-generators}
there exists a set $T$ and a family $(\mathcal{F}_t)_{t \in T}$ of
quasi-coherent sheaves on $\mathcal{X}$ such that every quasi-coherent
sheaf on $\mathcal{X}$ is the directed colimit of its subsheaves
which are isomorphic to one of the $\mathcal{F}_t$.
Thus $\bigoplus_t \mathcal{F}_t$ is
a generator of $\QCoh(\mathcal{O}_X)$ and we conclude that (c) holds.
The assertions on limits and injectives hold in any
Grothendieck abelian category, see
Injectives, Theorem
\ref{injectives-theorem-injective-embedding-grothendieck} and
Lemma \ref{injectives-lemma-grothendieck-products}.

\medskip\noindent
Proof of (2). To construct $Q$ we use the following general procedure.
Given an object $\mathcal{F}$ of $\textit{Mod}(\mathcal{O}_\mathcal{X})$
we consider the functor
$$
\QCoh(\mathcal{O}_\mathcal{X})^{opp}
\longrightarrow
\textit{Sets},
\quad
\mathcal{G}
\longmapsto
\Hom_\mathcal{X}(\mathcal{G}, \mathcal{F})
$$
This functor transforms colimits into limits,
hence is representable, see
Injectives, Lemma \ref{injectives-lemma-grothendieck-brown}.
Thus there exists a quasi-coherent sheaf $Q(\mathcal{F})$
and a functorial isomorphism
$\Hom_\mathcal{X}(\mathcal{G}, \mathcal{F}) =
\Hom_\mathcal{X}(\mathcal{G}, Q(\mathcal{F}))$
for $\mathcal{G}$ in $\QCoh(\mathcal{O}_\mathcal{X})$.
By the Yoneda lemma
(Categories, Lemma \ref{categories-lemma-yoneda})
the construction $\mathcal{F} \leadsto Q(\mathcal{F})$ is
functorial in $\mathcal{F}$. By construction $Q$ is a right
adjoint to the inclusion functor.
The fact that $Q(\mathcal{F}) \to \mathcal{F}$ is an isomorphism
when $\mathcal{F}$ is quasi-coherent is a formal consequence of the fact
that the inclusion functor
$\QCoh(\mathcal{O}_\mathcal{X}) \to
\textit{Mod}(\mathcal{O}_\mathcal{X})$
is fully faithful.
\end{proof}








\section{Cohomology}
\label{section-cohomology-general}

\noindent
Let $S$ be a scheme and let $\mathcal{X}$ be a category fibred in groupoids
over $(\Sch/S)_{fppf}$. For any $\tau \in \{Zariski, \etale, smooth,
syntomic, fppf\}$ the categories $\textit{Ab}(\mathcal{X}_\tau)$ and
$\textit{Mod}(\mathcal{X}_\tau, \mathcal{O}_\mathcal{X})$ have
enough injectives, see
Injectives, Theorems \ref{injectives-theorem-sheaves-injectives} and
\ref{injectives-theorem-sheaves-modules-injectives}.
Thus we can use the machinery of
Cohomology on Sites, Section \ref{sites-cohomology-section-cohomology-sheaves}
to define the cohomology groups
$$
H^p(\mathcal{X}_\tau, \mathcal{F}) = H^p_\tau(\mathcal{X}, \mathcal{F})
\quad\text{and}\quad
H^p(x, \mathcal{F}) = H^p_\tau(x, \mathcal{F})
$$
for any $x \in \Ob(\mathcal{X})$ and any object $\mathcal{F}$ of
$\textit{Ab}(\mathcal{X}_\tau)$ or
$\textit{Mod}(\mathcal{X}_\tau, \mathcal{O}_\mathcal{X})$. Moreover, if
$f : \mathcal{X} \to \mathcal{Y}$ is a $1$-morphism of categories fibred
in groupoids over $(\Sch/S)_{fppf}$, then we obtain the higher direct
images $R^if_*\mathcal{F}$ in $\textit{Ab}(\mathcal{Y}_\tau)$ or
$\textit{Mod}(\mathcal{Y}_\tau, \mathcal{O}_\mathcal{Y})$.
Of course, as explained in
Cohomology on Sites, Section \ref{sites-cohomology-section-derived-functors}
there are also derived versions of $H^p(-)$ and $R^if_*$.

\begin{lemma}
\label{lemma-cohomology-restriction}
Let $S$ be a scheme. Let $\mathcal{X}$ be a category fibred in groupoids
over $(\Sch/S)_{fppf}$. Let $\tau \in \{Zariski, \etale, smooth,
syntomic, fppf\}$. Let $x \in \Ob(\mathcal{X})$ be an object lying
over the scheme $U$. Let $\mathcal{F}$ be
an object of $\textit{Ab}(\mathcal{X}_\tau)$ or
$\textit{Mod}(\mathcal{X}_\tau, \mathcal{O}_\mathcal{X})$. Then
$$
H^p_\tau(x, \mathcal{F}) = H^p((\Sch/U)_\tau, x^{-1}\mathcal{F})
$$
and if $\tau = \etale$, then we also have
$$
H^p_\etale(x, \mathcal{F}) =
H^p(U_\etale, \mathcal{F}|_{U_\etale}).
$$
\end{lemma}

\begin{proof}
The first statement follows from
Cohomology on Sites, Lemma \ref{sites-cohomology-lemma-cohomology-of-open}
and the equivalence of 
Lemma \ref{lemma-localizing-structure-sheaf}.
The second statement follows from the first combined with
\'Etale Cohomology, Lemma
\ref{etale-cohomology-lemma-compare-cohomology-big-small}.
\end{proof}














\section{Injective sheaves}
\label{section-lower-shriek}

\noindent
The pushforward of an injective abelian sheaf or module is injective.

\begin{lemma}
\label{lemma-pushforward-injective}
Let $f : \mathcal{X} \to \mathcal{Y}$ be a $1$-morphism of categories
fibred in groupoids over $(\Sch/S)_{fppf}$. Let
$\tau \in \{Zar, \etale, smooth, syntomic, fppf\}$.
\begin{enumerate}
\item $f_*\mathcal{I}$ is injective in $\textit{Ab}(\mathcal{Y}_\tau)$
for $\mathcal{I}$ injective in $\textit{Ab}(\mathcal{X}_\tau)$, and
\item $f_*\mathcal{I}$ is injective in
$\textit{Mod}(\mathcal{Y}_\tau, \mathcal{O}_\mathcal{Y})$
for $\mathcal{I}$ injective in
$\textit{Mod}(\mathcal{X}_\tau, \mathcal{O}_\mathcal{X})$.
\end{enumerate}
\end{lemma}

\begin{proof}
This follows formally from the fact that $f^{-1}$ is an exact
left adjoint of $f_*$, see
Homology, Lemma \ref{homology-lemma-adjoint-preserve-injectives}.
\end{proof}

\noindent
In the rest of this section we prove that pullback $f^{-1}$ has a left
adjoint $f_!$ on abelian sheaves and modules. If $f$ is representable (by
schemes or by algebraic spaces), then it will turn out that $f_!$ is exact
and $f^{-1}$ will preserve injectives. We first prove a few
preliminary lemmas about fibre products and equalizers in categories
fibred in groupoids and their behaviour with respect to morphisms.

\begin{lemma}
\label{lemma-fibre-products}
Let $p : \mathcal{X} \to (\Sch/S)_{fppf}$
be a category fibred in groupoids.
\begin{enumerate}
\item The category $\mathcal{X}$ has fibre products.
\item If the $\mathit{Isom}$-presheaves of $\mathcal{X}$
are representable by algebraic spaces, then $\mathcal{X}$ has equalizers.
\item If $\mathcal{X}$ is an algebraic stack (or more generally
a quotient stack), then $\mathcal{X}$ has equalizers.
\end{enumerate}
\end{lemma}

\begin{proof}
Part (1) follows
Categories, Lemma \ref{categories-lemma-fibred-groupoids-fibre-product-goes-up}
as $(\Sch/S)_{fppf}$ has fibre products.

\medskip\noindent
Let $a, b : x \to y$ be morphisms of $\mathcal{X}$.
Set $U = p(x)$ and $V = p(y)$. The category of schemes has equalizers
hence we can let $W \to U$ be the equalizer of $p(a)$ and $p(b)$.
Denote $c : z \to x$ a morphism of $\mathcal{X}$ lying over $W \to U$.
The equalizer of $a$ and $b$, if it exists, is the equalizer of $a \circ c$
and $b \circ c$. Thus we may assume that $p(a) = p(b) = f : U \to V$.
As $\mathcal{X}$ is fibred in groupoids, there exists a unique automorphism
$i : x \to x$ in the fibre category of $\mathcal{X}$ over $U$ such that
$a \circ i = b$. Again the equalizer of $a$ and $b$ is the equalizer
of $\text{id}_x$ and $i$. Recall that the $\mathit{Isom}_\mathcal{X}(x)$
is the presheaf on $(\Sch/U)_{fppf}$ which to
$T/U$ associates the set of automorphisms of $x|_T$ in the fibre category
of $\mathcal{X}$ over $T$, see
Stacks, Definition \ref{stacks-definition-mor-presheaf}.
If $\mathit{Isom}_\mathcal{X}(x)$ is representable by an algebraic space
$G \to U$, then we see that $\text{id}_x$ and $i$ define morphisms
$e, i : U \to G$ over $U$. Set $M = U \times_{e, G, i} U$, which by
Morphisms of Spaces, Lemma \ref{spaces-morphisms-lemma-section-immersion}
is a scheme. Then it is clear that $x|_M \to x$ is the equalizer of
the maps $\text{id}_x$ and $i$ in $\mathcal{X}$.
This proves (2).

\medskip\noindent
If $\mathcal{X} = [U/R]$ for some groupoid in algebraic spaces
$(U, R, s, t, c)$ over $S$, then the hypothesis of (2) holds by
Bootstrap, Lemma \ref{bootstrap-lemma-quotient-stack-isom}.
If $\mathcal{X}$ is an algebraic stack, then we can choose a
presentation $[U/R] \cong \mathcal{X}$ by
Algebraic Stacks, Lemma \ref{algebraic-lemma-stack-presentation}.
\end{proof}

\begin{lemma}
\label{lemma-fibre-products-morphism}
Let $f : \mathcal{X} \to \mathcal{Y}$ be a $1$-morphism of categories
fibred in groupoids over $(\Sch/S)_{fppf}$.
\begin{enumerate}
\item The functor $f$ transforms fibre products into fibre products.
\item If $f$ is faithful, then $f$ transforms equalizers into equalizers.
\end{enumerate}
\end{lemma}

\begin{proof}
By
Categories, Lemma \ref{categories-lemma-fibred-groupoids-fibre-product-goes-up}
we see that a fibre product in $\mathcal{X}$ is any commutative square lying
over a fibre product diagram in $(\Sch/S)_{fppf}$. Similarly for
$\mathcal{Y}$. Hence (1) is clear.

\medskip\noindent
Let $x \to x'$ be the equalizer of two morphisms $a, b : x' \to x''$
in $\mathcal{X}$. We will show that $f(x) \to f(x')$ is the equalizer
of $f(a)$ and $f(b)$. Let $y \to f(x)$ be a morphism of $\mathcal{Y}$
equalizing $f(a)$ and $f(b)$. Say $x, x', x''$ lie over the schemes
$U, U', U''$ and $y$ lies over $V$. Denote $h : V \to U'$ the image
of $y \to f(x)$ in the category of schemes. The morphism
$y \to f(x)$ is isomorphic to $f(h^*x') \to f(x')$ by the axioms of
fibred categories. Hence, as $f$ is faithful, we see that
$h^*x' \to x'$ equalizes $a$ and $b$. Thus we obtain a unique morphism
$h^*x' \to x$ whose image $y = f(h^*x') \to f(x)$ is the desired morphism
in $\mathcal{Y}$.
\end{proof}

\begin{lemma}
\label{lemma-fibre-products-preserve-properties}
Let $f : \mathcal{X} \to \mathcal{Y}$, $g : \mathcal{Z} \to \mathcal{Y}$
be faithful $1$-morphisms of categories
fibred in groupoids over $(\Sch/S)_{fppf}$.
\begin{enumerate}
\item the functor $\mathcal{X} \times_\mathcal{Y} \mathcal{Z} \to \mathcal{Y}$
is faithful, and
\item if $\mathcal{X}, \mathcal{Z}$ have equalizers, so does
$\mathcal{X} \times_\mathcal{Y} \mathcal{Z}$.
\end{enumerate}
\end{lemma}

\begin{proof}
We think of objects in $\mathcal{X} \times_\mathcal{Y} \mathcal{Z}$ as
quadruples $(U, x, z, \alpha)$ where $\alpha : f(x) \to g(z)$ is an
isomorphism over $U$, see
Categories, Lemma \ref{categories-lemma-2-product-categories-over-C}.
A morphism $(U, x, z, \alpha) \to (U', x', z', \alpha')$ is a
pair of morphisms $a : x \to x'$ and $b : z \to z'$ compatible
with $\alpha$ and $\alpha'$. Thus it is clear that if $f$ and
$g$ are faithful, so is the functor
$\mathcal{X} \times_\mathcal{Y} \mathcal{Z} \to \mathcal{Y}$.
Now, suppose that
$(a, b), (a', b') : (U, x, z, \alpha) \to (U', x', z', \alpha')$
are two morphisms of the $2$-fibre product. Then consider the equalizer
$x'' \to x$ of $a$ and $a'$ and the equalizer $z'' \to z$ of $b$ and $b'$.
Since $f$ commutes with equalizers (by
Lemma \ref{lemma-fibre-products-morphism})
we see that $f(x'') \to f(x)$ is the equalizer of $f(a)$ and $f(a')$.
Similarly, $g(z'') \to g(z)$ is the equalizer of $g(b)$ and $g(b')$.
Picture
$$
\xymatrix{
f(x'') \ar[r] \ar@{..>}[d]_{\alpha''}&
f(x) \ar[d]_\alpha
\ar@<0.5ex>[r]^{f(a)}
\ar@<-0.5ex>[r]_{f(a')}
 &
f(x') \ar[d]^{\alpha'} \\
g(z'') \ar[r] &
g(z)
\ar@<0.5ex>[r]^{g(b)}
\ar@<-0.5ex>[r]_{g(b')}
 &
g(z')
}
$$
It is clear that the dotted arrow exists and is an isomorphism.
However, it is not a priori the case that the image of $\alpha''$
in the category of schemes is the identity of its source. On the other
hand, the existence of $\alpha''$ means that we can assume that $x''$
and $z''$ are defined over the same scheme and that the morphisms
$x'' \to x$ and $z'' \to z$ have the same image in the category of schemes.
Redoing the diagram above we see that the dotted arrow now does
project to an identity morphism and we win. Some details omitted.
\end{proof}

\noindent
As we are working with big sites we have the following somewhat
counter intuitive result (which also holds for morphisms of big sites
of schemes). Warning: This result isn't true if we drop the hypothesis
that $f$ is faithful.

\begin{lemma}
\label{lemma-pullback-injective}
Let $f : \mathcal{X} \to \mathcal{Y}$ be a $1$-morphism of categories
fibred in groupoids over $(\Sch/S)_{fppf}$. Let
$\tau \in \{Zar, \etale, smooth, syntomic, fppf\}$.
The functor
$f^{-1} : \textit{Ab}(\mathcal{Y}_\tau) \to \textit{Ab}(\mathcal{X}_\tau)$
has a left adjoint
$f_! : \textit{Ab}(\mathcal{X}_\tau) \to \textit{Ab}(\mathcal{Y}_\tau)$.
If $f$ is faithful and $\mathcal{X}$ has equalizers, then
\begin{enumerate}
\item $f_!$ is exact, and
\item $f^{-1}\mathcal{I}$ is injective in $\textit{Ab}(\mathcal{X}_\tau)$
for $\mathcal{I}$ injective in $\textit{Ab}(\mathcal{Y}_\tau)$.
\end{enumerate}
\end{lemma}

\begin{proof}
By
Stacks, Lemma \ref{stacks-lemma-topology-inherited-functorial}
the functor $f$ is continuous and cocontinuous. Hence by
Modules on Sites, Lemma \ref{sites-modules-lemma-g-shriek-adjoint}
the functor
$f^{-1} : \textit{Ab}(\mathcal{Y}_\tau) \to \textit{Ab}(\mathcal{X}_\tau)$
has a left adjoint
$f_! : \textit{Ab}(\mathcal{X}_\tau) \to \textit{Ab}(\mathcal{Y}_\tau)$.
To see (1) we apply
Modules on Sites, Lemma \ref{sites-modules-lemma-exactness-lower-shriek}
and to see that the hypotheses of that lemma are satisfied use
Lemmas \ref{lemma-fibre-products} and
\ref{lemma-fibre-products-morphism}
above. Part (2) follows from this formally, see
Homology, Lemma \ref{homology-lemma-adjoint-preserve-injectives}.
\end{proof}

\begin{lemma}
\label{lemma-pullback-injective-modules}
Let $f : \mathcal{X} \to \mathcal{Y}$ be a $1$-morphism of categories
fibred in groupoids over $(\Sch/S)_{fppf}$. Let
$\tau \in \{Zar, \etale, smooth, syntomic, fppf\}$.
The functor
$f^* : \textit{Mod}(\mathcal{Y}_\tau, \mathcal{O}_\mathcal{Y}) \to
\textit{Mod}(\mathcal{X}_\tau, \mathcal{O}_\mathcal{X})$
has a left adjoint
$f_! : \textit{Mod}(\mathcal{X}_\tau, \mathcal{O}_\mathcal{X}) \to
\textit{Mod}(\mathcal{Y}_\tau, \mathcal{O}_\mathcal{Y})$ which
agrees with the functor $f_!$ of Lemma \ref{lemma-pullback-injective}
on underlying abelian sheaves.
If $f$ is faithful and $\mathcal{X}$ has equalizers, then
\begin{enumerate}
\item $f_!$ is exact, and
\item $f^{-1}\mathcal{I}$ is injective in
$\textit{Mod}(\mathcal{X}_\tau, \mathcal{O}_\mathcal{X})$
for $\mathcal{I}$ injective in
$\textit{Mod}(\mathcal{Y}_\tau, \mathcal{O}_\mathcal{X})$.
\end{enumerate}
\end{lemma}

\begin{proof}
Recall that $f$ is a continuous and cocontinuous functor of sites
and that $f^{-1}\mathcal{O}_\mathcal{Y} = \mathcal{O}_\mathcal{X}$. Hence
Modules on Sites, Lemma \ref{sites-modules-lemma-lower-shriek-modules}
implies $f^*$ has a left adjoint $f_!^{Mod}$.
Let $x$ be an object of $\mathcal{X}$ lying over the scheme $U$.
Then $f$ induces an equivalence of ringed sites
$$
\mathcal{X}/x \longrightarrow \mathcal{Y}/f(x)
$$
as both sides are equivalent to $(\Sch/U)_\tau$, see
Lemma \ref{lemma-localizing-structure-sheaf}.
Modules on Sites, Remark \ref{sites-modules-remark-when-shriek-equal}
shows that $f_!$ agrees with the functor on abelian sheaves.

\medskip\noindent
Assume now that $\mathcal{X}$ has equalizers and that $f$ is faithful.
Lemma \ref{lemma-pullback-injective}
tells us that $f_!$ is exact. Finally,
Homology, Lemma \ref{homology-lemma-adjoint-preserve-injectives}
implies the statement on pullbacks of injective modules.
\end{proof}




\section{The {\v C}ech complex}
\label{section-cech}

\noindent
To compute the cohomology of a sheaf on an algebraic stack we compare
it to the cohomology of the sheaf restricted to coverings of the
given algebraic stack.

\medskip\noindent
Throughout this section the situation will be as follows. We are given
a $1$-morphism of categories fibred in groupoids
\begin{equation}
\label{equation-covering}
\vcenter{
\xymatrix{
\mathcal{U} \ar[rr]_f \ar[rd]_q & &  \mathcal{X} \ar[ld]^p \\
& (\Sch/S)_{fppf}
}
}
\end{equation}
We are going to think about $\mathcal{U}$ as a ``covering'' of $\mathcal{X}$.
Hence we want to consider the simplicial object
$$
\xymatrix{
\mathcal{U} \times_\mathcal{X} \mathcal{U} \times_\mathcal{X} \mathcal{U}
\ar@<1ex>[r]
\ar@<0ex>[r]
\ar@<-1ex>[r]
&
\mathcal{U} \times_\mathcal{X} \mathcal{U}
\ar@<0.5ex>[r]
\ar@<-0.5ex>[r]
&
\mathcal{U}
}
$$
in the category of categories fibred in groupoids over
$(\Sch/S)_{fppf}$. However, since this is a $(2, 1)$-category and
not a category, we should say explicitly what we mean. Namely, we let
$\mathcal{U}_n$ be the category with objects
$(u_0, \ldots, u_n, x, \alpha_0, \ldots, \alpha_n)$
where $\alpha_i : f(u_i) \to x$ is an isomorphism in $\mathcal{X}$.
We denote $f_n : \mathcal{U}_n \to \mathcal{X}$ the $1$-morphism
which assigns to $(u_0, \ldots, u_n, x, \alpha_0, \ldots, \alpha_n)$
the object $x$. Note that $\mathcal{U}_0 = \mathcal{U}$ and $f_0 = f$.
Given a map $\varphi : [m] \to [n]$ we consider the $1$-morphism
$\mathcal{U}_\varphi : \mathcal{U}_n \longrightarrow \mathcal{U}_n$
given by
$$
(u_0, \ldots, u_n, x, \alpha_0, \ldots, \alpha_n)
\longmapsto
(u_{\varphi(0)}, \ldots, u_{\varphi(m)}, x,
\alpha_{\varphi(0)}, \ldots, \alpha_{\varphi(m)})
$$
on objects. All of these $1$-morphisms compose correctly on the nose
(no $2$-morphisms required) and all of these $1$-morphisms are $1$-morphisms
over $\mathcal{X}$. We denote $\mathcal{U}_\bullet$ this simplicial object.
If $\mathcal{F}$ is a presheaf of sets on $\mathcal{X}$, then we obtain a
cosimplicial set
$$
\xymatrix{
\Gamma(\mathcal{U}_0, f_0^{-1}\mathcal{F})
\ar@<0.5ex>[r]
\ar@<-0.5ex>[r]
&
\Gamma(\mathcal{U}_1, f_1^{-1}\mathcal{F})
\ar@<1ex>[r]
\ar@<0ex>[r]
\ar@<-1ex>[r]
&
\Gamma(\mathcal{U}_2, f_2^{-1}\mathcal{F})
}
$$
Here the arrows are the pullback maps along the given morphisms of
the simplicial object.
If $\mathcal{F}$ is a presheaf of abelian groups, this is a cosimplicial
abelian group.

\medskip\noindent
Let $\mathcal{U} \to \mathcal{X}$ be as above and let $\mathcal{F}$
be an abelian presheaf on $\mathcal{X}$.
The {\it {\v C}ech complex} associated to the situation is denoted
$\check{\mathcal{C}}^\bullet(\mathcal{U} \to \mathcal{X}, \mathcal{F})$.
It is the cochain complex associated to the cosimplicial abelian group
above, see
Simplicial, Section \ref{simplicial-section-dold-kan-cosimplicial}.
It has terms
$$
\check{\mathcal{C}}^n(\mathcal{U} \to \mathcal{X}, \mathcal{F})
= \Gamma(\mathcal{U}_n, f_n^{-1}\mathcal{F}).
$$
The boundary maps are the maps
$$
d^n = \sum\nolimits_{i = 0}^{n + 1} (-1)^i \delta^{n + 1}_i :
\Gamma(\mathcal{U}_n, f_n^{-1}\mathcal{F})
\longrightarrow
\Gamma(\mathcal{U}_{n + 1}, f_{n + 1}^{-1}\mathcal{F})
$$
where $\delta^{n + 1}_i$ corresponds to the map
$[n] \to [n + 1]$ omitting the index $i$. Note that the map
$\Gamma(\mathcal{X}, \mathcal{F}) \to
\Gamma(\mathcal{U}_0, f_0^{-1}\mathcal{F}_0)$
is in the kernel of the differential $d^0$. Hence we define
the {\it extended {\v C}ech complex} to be the complex
$$
\ldots \to 0 \to
\Gamma(\mathcal{X}, \mathcal{F}) \to
\Gamma(\mathcal{U}_0, f_0^{-1}\mathcal{F}_0) \to
\Gamma(\mathcal{U}_1, f_1^{-1}\mathcal{F}_1) \to \ldots
$$
with $\Gamma(\mathcal{X}, \mathcal{F})$ placed in degree $-1$.
The extended {\v C}ech complex is acyclic if and only if the canonical map
$$
\Gamma(\mathcal{X}, \mathcal{F})[0]
\longrightarrow
\check{\mathcal{C}}^\bullet(\mathcal{U} \to \mathcal{X}, \mathcal{F})
$$
is a quasi-isomorphism of complexes.

\begin{lemma}
\label{lemma-generalities}
Generalities on {\v C}ech complexes.
\begin{enumerate}
\item If
$$
\xymatrix{
\mathcal{V} \ar[d]_g \ar[r]_h & \mathcal{U} \ar[d]^f \\
\mathcal{Y} \ar[r]^e & \mathcal{X}
}
$$
is $2$-commutative diagram of categories fibred in groupoids over
$(\Sch/S)_{fppf}$, then there is a morphism of {\v C}ech complexes
$$
\check{\mathcal{C}}^\bullet(\mathcal{U} \to \mathcal{X}, \mathcal{F})
\longrightarrow
\check{\mathcal{C}}^\bullet(\mathcal{V} \to \mathcal{Y}, e^{-1}\mathcal{F})
$$
\item if $h$ and $e$ are equivalences, then the map of (1) is an isomorphism,
\item if $f, f' : \mathcal{U} \to \mathcal{X}$ are $2$-isomorphic, then
the associated {\v C}ech complexes are isomorphic,
\end{enumerate}
\end{lemma}

\begin{proof}
In the situation of (1) let $t : f \circ h \to e \circ g$ be a $2$-morphism.
The map on complexes is given in degree $n$ by
pullback along the $1$-morphisms
$\mathcal{V}_n \to \mathcal{U}_n$ given by the rule
$$
(v_0, \ldots, v_n, y, \beta_0, \ldots, \beta_n)
\longmapsto
(h(v_0), \ldots, h(v_n), e(y),
e(\beta_0) \circ t_{v_0}, \ldots, e(\beta_n) \circ t_{v_n}).
$$
For (2), note that pullback on global sections is an isomorphism
for any presheaf of sets when the pullback is along an equivalence
of categories. Part (3) follows on combining (1) and (2).
\end{proof}

\begin{lemma}
\label{lemma-homotopy}
If there exists a $1$-morphism $s : \mathcal{X} \to \mathcal{U}$
such that $f \circ s$ is $2$-isomorphic to $\text{id}_\mathcal{X}$
then the extended {\v C}ech complex is homotopic to zero.
\end{lemma}

\begin{proof}
Set $\mathcal{U}' = \mathcal{U} \times_\mathcal{X} \mathcal{X}$
equal to the fibre product as described in
Categories, Lemma \ref{categories-lemma-2-product-categories-over-C}.
Set $f' : \mathcal{U}' \to \mathcal{X}$ equal to the second projection.
Then $\mathcal{U} \to \mathcal{U}'$, $u \mapsto (u, f(x), 1)$
is an equivalence over $\mathcal{X}$, hence we may replace
$(\mathcal{U}, f)$ by $(\mathcal{U}', f')$ by
Lemma \ref{lemma-generalities}.
The advantage of this is that now $f'$ has a section $s'$ such
that $f' \circ s' = \text{id}_\mathcal{X}$ on the nose. Namely, if
$t : s \circ f \to \text{id}_\mathcal{X}$ is a $2$-isomorphism
then we can set $s'(x) = (s(x), x, t_x)$. Thus we may assume that
$f \circ s = \text{id}_\mathcal{X}$.

\medskip\noindent
In the case that $f \circ s = \text{id}_\mathcal{X}$ the result follows
from general principles. We give the homotopy explicitly. Namely,
for $n \geq 0$ define $s_n : \mathcal{U}_n \to \mathcal{U}_{n + 1}$
to be the $1$-morphism defined by the rule on objects
$$
(u_0, \ldots, u_n, x, \alpha_0, \ldots, \alpha_n)
\longmapsto
(u_0, \ldots, u_n, s(x), x,
\alpha_0, \ldots, \alpha_n, \text{id}_x).
$$
Define
$$
h^{n + 1} :
\Gamma(\mathcal{U}_{n + 1}, f_{n + 1}^{-1}\mathcal{F})
\longrightarrow
\Gamma(\mathcal{U}_n, f_n^{-1}\mathcal{F})
$$
as pullback along $s_n$. We also set $s_{-1} = s$ and
$h^0 : \Gamma(\mathcal{U}_0, f_0^{-1}\mathcal{F}) \to
\Gamma(\mathcal{X}, \mathcal{F})$ equal to pullback along $s_{-1}$.
Then the family of maps $\{h^n\}_{n \geq 0}$ is a homotopy between
$1$ and $0$ on the extended {\v C}ech complex.
\end{proof}









\section{The relative {\v C}ech complex}
\label{section-sheaf-cech-complex}

\noindent
Let $f : \mathcal{U} \to \mathcal{X}$ be a $1$-morphism of categories
fibred in groupoids over $(\Sch/S)_{fppf}$ as in
(\ref{equation-covering}). Consider the associated simplicial
object $\mathcal{U}_\bullet$ and the maps
$f_n : \mathcal{U}_n \to \mathcal{X}$. Let
$\tau \in \{Zar, \etale, smooth, syntomic, fppf\}$.
Finally, suppose that $\mathcal{F}$ is a sheaf (of sets)
on $\mathcal{X}_\tau$. Then
$$
\xymatrix{
f_{0, *}f_0^{-1}\mathcal{F}
\ar@<0.5ex>[r]
\ar@<-0.5ex>[r]
&
f_{1, *}f_1^{-1}\mathcal{F}
\ar@<1ex>[r]
\ar@<0ex>[r]
\ar@<-1ex>[r]
&
f_{2, *}f_2^{-1}\mathcal{F}
}
$$
is a cosimplicial sheaf on $\mathcal{X}_\tau$ where we use the pullback maps
introduced in
Sites, Section \ref{sites-section-pullback}.
If $\mathcal{F}$ is an abelian sheaf, then $f_{n, *}f_n^{-1}\mathcal{F}$
form a cosimplicial abelian sheaf on $\mathcal{X}_\tau$.
The associated complex (see
Simplicial, Section \ref{simplicial-section-dold-kan-cosimplicial})
$$
\ldots \to 0 \to
f_{0, *}f_0^{-1}\mathcal{F} \to
f_{1, *}f_1^{-1}\mathcal{F} \to
f_{2, *}f_2^{-1}\mathcal{F} \to \ldots
$$
is called the {\it relative {\v C}ech complex} associated to the situation.
We will denote this complex $\mathcal{K}^\bullet(f, \mathcal{F})$.
The {\it extended relative {\v C}ech complex} is the complex
$$
\ldots \to 0 \to
\mathcal{F} \to
f_{0, *}f_0^{-1}\mathcal{F} \to
f_{1, *}f_1^{-1}\mathcal{F} \to
f_{2, *}f_2^{-1}\mathcal{F} \to \ldots
$$
with $\mathcal{F}$ in degree $-1$. The extended relative {\v C}ech complex
is acyclic if and only if the map
$\mathcal{F}[0] \to \mathcal{K}^\bullet(f, \mathcal{F})$
is a quasi-isomorphism of complexes of sheaves.

\begin{remark}
\label{remark-cech-complex-presheaves}
We can define the complex $\mathcal{K}^\bullet(f, \mathcal{F})$
also if $\mathcal{F}$ is a presheaf, only we cannot use the reference to
Sites, Section \ref{sites-section-pullback}
to define the pullback maps. To explain the pullback maps, suppose
given a commutative diagram
$$
\xymatrix{
\mathcal{V} \ar[rd]_g \ar[rr]_h & &  \mathcal{U} \ar[ld]^f \\
& \mathcal{X}
}
$$
of categories fibred in groupoids over $(\Sch/S)_{fppf}$
and a presheaf $\mathcal{G}$ on $\mathcal{U}$
we can define the pullback map $f_*\mathcal{G} \to g_*h^{-1}\mathcal{G}$
as the composition
$$
f_*\mathcal{G} \longrightarrow
f_*h_*h^{-1}\mathcal{G} = g_*h^{-1}\mathcal{G}
$$
where the map comes from the adjunction map
$\mathcal{G} \to h_*h^{-1}\mathcal{G}$. This works because in our situation
the functors $h_*$ and $h^{-1}$ are adjoint in presheaves (and agree with
their counter parts on sheaves). See
Sections \ref{section-presheaves} and \ref{section-sheaves}.
\end{remark}

\begin{lemma}
\label{lemma-generalities-sheafified}
Generalities on relative {\v C}ech complexes.
\begin{enumerate}
\item If
$$
\xymatrix{
\mathcal{V} \ar[d]_g \ar[r]_h & \mathcal{U} \ar[d]^f \\
\mathcal{Y} \ar[r]^e & \mathcal{X}
}
$$
is $2$-commutative diagram of categories fibred in groupoids over
$(\Sch/S)_{fppf}$, then there is a morphism
$e^{-1}\mathcal{K}^\bullet(f, \mathcal{F}) \to
\mathcal{K}^\bullet(g, e^{-1}\mathcal{F})$.
\item if $h$ and $e$ are equivalences, then the map of (1) is an isomorphism,
\item if $f, f' : \mathcal{U} \to \mathcal{X}$ are $2$-isomorphic, then
the associated relative {\v C}ech complexes are isomorphic,
\end{enumerate}
\end{lemma}

\begin{proof}
Literally the same as the proof of
Lemma \ref{lemma-generalities}
using the pullback maps of
Remark \ref{remark-cech-complex-presheaves}.
\end{proof}

\begin{lemma}
\label{lemma-homotopy-sheafified}
If there exists a $1$-morphism $s : \mathcal{X} \to \mathcal{U}$
such that $f \circ s$ is $2$-isomorphic to $\text{id}_\mathcal{X}$
then the extended relative {\v C}ech complex is homotopic to zero.
\end{lemma}

\begin{proof}
Literally the same as the proof of
Lemma \ref{lemma-homotopy}.
\end{proof}

\begin{remark}
\label{remark-cech-complex-sections}
Let us ``compute'' the value of the relative {\v C}ech complex on an
object $x$ of $\mathcal{X}$. Say $p(x) = U$.
Consider the $2$-fibre product diagram (which serves to introduce
the notation $g : \mathcal{V} \to \mathcal{Y}$)
$$
\xymatrix{
\mathcal{V} \ar@{=}[r] \ar[d]_g &
(\Sch/U)_{fppf} \times_{x, \mathcal{X}} \mathcal{U} \ar[r] \ar[d] &
\mathcal{U} \ar[d]^f \\
\mathcal{Y} \ar@{=}[r] &
(\Sch/U)_{fppf} \ar[r]^-x & \mathcal{X}
}
$$
Note that the morphism $\mathcal{V}_n \to \mathcal{U}_n$ of the proof of
Lemma \ref{lemma-generalities}
induces an equivalence
$\mathcal{V}_n =
(\Sch/U)_{fppf} \times_{x, \mathcal{X}} \mathcal{U}_n$.
Hence we see from
(\ref{equation-pushforward})
that
$$
\Gamma(x, \mathcal{K}^\bullet(f, \mathcal{F})) =
\check{\mathcal{C}}^\bullet(\mathcal{V} \to \mathcal{Y}, x^{-1}\mathcal{F})
$$
In words: The value of the relative {\v C}ech complex on an object $x$ of
$\mathcal{X}$ is the {\v C}ech complex of the base change of $f$ to
$\mathcal{X}/x \cong (\Sch/U)_{fppf}$. This implies for example that
Lemma \ref{lemma-homotopy}
implies
Lemma \ref{lemma-homotopy-sheafified}
and more generally that results on the (usual) {\v C}ech complex imply
results for the relative {\v C}ech complex.
\end{remark}

\begin{lemma}
\label{lemma-base-change-cech-complex}
Let
$$
\xymatrix{
\mathcal{V} \ar[d]_g \ar[r]_h & \mathcal{U} \ar[d]^f \\
\mathcal{Y} \ar[r]^e & \mathcal{X}
}
$$
be a $2$-fibre product of categories fibred in groupoids over
$(\Sch/S)_{fppf}$ and let $\mathcal{F}$ be an abelian presheaf
on $\mathcal{X}$. Then the map
$e^{-1}\mathcal{K}^\bullet(f, \mathcal{F}) \to
\mathcal{K}^\bullet(g, e^{-1}\mathcal{F})$
of
Lemma \ref{lemma-generalities-sheafified}
is an isomorphism of complexes of abelian presheaves.
\end{lemma}

\begin{proof}
Let $y$ be an object of $\mathcal{Y}$ lying over the scheme $T$.
Set $x = e(y)$. We are going to show that the map induces an isomorphism
on sections over $y$. Note that
$$
\Gamma(y, e^{-1}\mathcal{K}^\bullet(f, \mathcal{F})) =
\Gamma(x, \mathcal{K}^\bullet(f, \mathcal{F})) =
\check{\mathcal{C}}^\bullet(
(\Sch/T)_{fppf} \times_{x, \mathcal{X}} \mathcal{U} \to
(\Sch/T)_{fppf}, x^{-1}\mathcal{F})
$$
by
Remark \ref{remark-cech-complex-sections}. On the other hand,
$$
\Gamma(y, \mathcal{K}^\bullet(g, e^{-1}\mathcal{F})) =
\check{\mathcal{C}}^\bullet(
(\Sch/T)_{fppf} \times_{y, \mathcal{Y}} \mathcal{V} \to
(\Sch/T)_{fppf}, y^{-1}e^{-1}\mathcal{F})
$$
also by
Remark \ref{remark-cech-complex-sections}.
Note that $y^{-1}e^{-1}\mathcal{F} = x^{-1}\mathcal{F}$
and since the diagram is $2$-cartesian the $1$-morphism
$$
(\Sch/T)_{fppf} \times_{y, \mathcal{Y}} \mathcal{V} \to
(\Sch/T)_{fppf} \times_{x, \mathcal{X}} \mathcal{U}
$$
is an equivalence. Hence the map on sections over $y$ is an
isomorphism by
Lemma \ref{lemma-generalities}.
\end{proof}

\noindent
Exactness can be checked on a ``covering''.

\begin{lemma}
\label{lemma-check-exactness-covering}
Let $f : \mathcal{U} \to \mathcal{X}$ be a $1$-morphism of categories fibred
in groupoids over $(\Sch/S)_{fppf}$. Let
$\tau \in \{Zar, \etale, smooth, syntomic, fppf\}$.
Let
$$
\mathcal{F} \to \mathcal{G} \to \mathcal{H}
$$
be a complex in $\textit{Ab}(\mathcal{X}_\tau)$. Assume that
\begin{enumerate}
\item for every object $x$ of $\mathcal{X}$ there exists a covering
$\{x_i \to x\}$ in $\mathcal{X}_\tau$ such that each $x_i$ is isomorphic
to $f(u_i)$ for some object $u_i$ of $\mathcal{U}$, and
\item $f^{-1}\mathcal{F} \to f^{-1}\mathcal{G} \to f^{-1}\mathcal{H}$ is exact.
\end{enumerate}
Then the sequence $\mathcal{F} \to \mathcal{G} \to \mathcal{H}$
is exact.
\end{lemma}

\begin{proof}
Let $x$ be an object of $\mathcal{X}$ lying over the scheme $T$.
Consider the sequence
$x^{-1}\mathcal{F} \to x^{-1}\mathcal{G} \to x^{-1}\mathcal{H}$
of abelian sheaves on $(\Sch/T)_\tau$. It suffices to show
this sequence is exact. By assumption there exists a $\tau$-covering
$\{T_i \to T\}$ such that $x|_{T_i}$ is isomorphic to $f(u_i)$ for
some object $u_i$ of $\mathcal{U}$ over $T_i$ and moreover the sequence
$u_i^{-1}f^{-1}\mathcal{F} \to u_i^{-1}f^{-1}\mathcal{G} \to
u_i^{-1}f^{-1}\mathcal{H}$ of abelian sheaves on $(\Sch/T_i)_\tau$
is exact. Since
$u_i^{-1}f^{-1}\mathcal{F} = x^{-1}\mathcal{F}|_{(\Sch/T_i)_\tau}$
we conclude that the sequence
$x^{-1}\mathcal{F} \to x^{-1}\mathcal{G} \to x^{-1}\mathcal{H}$
become exact after localizing at each of the members of a covering,
hence the sequence is exact.
\end{proof}

\begin{proposition}
\label{proposition-exactness-cech-complex}
Let $f : \mathcal{U} \to \mathcal{X}$ be a $1$-morphism of categories fibred
in groupoids over $(\Sch/S)_{fppf}$. Let
$\tau \in \{Zar, \etale, smooth, syntomic, fppf\}$.
If
\begin{enumerate}
\item $\mathcal{F}$ is an abelian sheaf on $\mathcal{X}_\tau$, and
\item for every object $x$ of $\mathcal{X}$ there exists a covering
$\{x_i \to x\}$ in $\mathcal{X}_\tau$ such that each $x_i$ is isomorphic
to $f(u_i)$ for some object $u_i$ of $\mathcal{U}$,
\end{enumerate}
then the extended relative {\v C}ech complex
$$
\ldots \to 0 \to
\mathcal{F} \to
f_{0, *}f_0^{-1}\mathcal{F} \to
f_{1, *}f_1^{-1}\mathcal{F} \to
f_{2, *}f_2^{-1}\mathcal{F} \to \ldots
$$
is exact in $\textit{Ab}(\mathcal{X}_\tau)$.
\end{proposition}

\begin{proof}
By
Lemma \ref{lemma-check-exactness-covering}
it suffices to check exactness after pulling back to $\mathcal{U}$.
By
Lemma \ref{lemma-base-change-cech-complex}
the pullback of the extended relative {\v C}ech complex is isomorphic
to the extend relative {\v C}ech complex for the morphism
$\mathcal{U} \times_\mathcal{X} \mathcal{U} \to \mathcal{U}$
and an abelian sheaf on $\mathcal{U}_\tau$. Since there is a section
$\Delta_{\mathcal{U}/\mathcal{X}} : \mathcal{U} \to
\mathcal{U} \times_\mathcal{X} \mathcal{U}$ exactness follows from
Lemma \ref{lemma-homotopy-sheafified}.
\end{proof}

\noindent
Using this we can construct the {\v C}ech-to-cohomology spectral sequence
as follows. We first give a technical, precise version. In the next section
we give a version that applies only to algebraic stacks.

\begin{lemma}
\label{lemma-cech-to-cohomology}
Let $f : \mathcal{U} \to \mathcal{X}$ be a $1$-morphism of categories fibred
in groupoids over $(\Sch/S)_{fppf}$. Let
$\tau \in \{Zar, \etale, smooth, syntomic, fppf\}$.
Assume
\begin{enumerate}
\item $\mathcal{F}$ is an abelian sheaf on $\mathcal{X}_\tau$,
\item for every object $x$ of $\mathcal{X}$ there exists a covering
$\{x_i \to x\}$ in $\mathcal{X}_\tau$ such that each $x_i$ is isomorphic
to $f(u_i)$ for some object $u_i$ of $\mathcal{U}$,
\item the category $\mathcal{U}$ has equalizers, and
\item the functor $f$ is faithful.
\end{enumerate}
Then there is a first quadrant spectral sequence of abelian groups
$$
E_1^{p, q} = H^q((\mathcal{U}_p)_\tau, f_p^{-1}\mathcal{F})
\Rightarrow
H^{p + q}(\mathcal{X}_\tau, \mathcal{F})
$$
converging to the cohomology of $\mathcal{F}$ in the $\tau$-topology.
\end{lemma}

\begin{proof}
Before we start the proof we make some remarks. By
Lemma \ref{lemma-fibre-products-preserve-properties}
(and induction) all of the categories fibred in groupoids $\mathcal{U}_p$
have equalizers and all of the morphisms $f_p : \mathcal{U}_p \to \mathcal{X}$
are faithful. Let $\mathcal{I}$ be an injective object
of $\textit{Ab}(\mathcal{X}_\tau)$. By
Lemma \ref{lemma-pullback-injective}
we see $f_p^{-1}\mathcal{I}$ is an injective object of
$\textit{Ab}((\mathcal{U}_p)_\tau)$.
Hence $f_{p, *}f_p^{-1}\mathcal{I}$ is an injective object of
$\textit{Ab}(\mathcal{X}_\tau)$ by
Lemma \ref{lemma-pushforward-injective}.
Hence
Proposition \ref{proposition-exactness-cech-complex}
shows that the extended relative {\v C}ech complex
$$
\ldots \to 0 \to
\mathcal{I} \to
f_{0, *}f_0^{-1}\mathcal{I} \to
f_{1, *}f_1^{-1}\mathcal{I} \to
f_{2, *}f_2^{-1}\mathcal{I} \to \ldots
$$
is an exact complex in $\textit{Ab}(\mathcal{X}_\tau)$ all of whose
terms are injective. Taking global sections of this complex is exact
and we see that the {\v C}ech complex
$\check{\mathcal{C}}^\bullet(\mathcal{U} \to \mathcal{X}, \mathcal{I})$
is quasi-isomorphic to $\Gamma(\mathcal{X}_\tau, \mathcal{I})[0]$.

\medskip\noindent
With these preliminaries out of the way consider the two spectral sequences
associated to the double complex (see
Homology, Section \ref{homology-section-double-complex})
$$
\check{\mathcal{C}}^\bullet(\mathcal{U} \to \mathcal{X}, \mathcal{I}^\bullet)
$$
where $\mathcal{F} \to \mathcal{I}^\bullet$ is an injective resolution
in $\textit{Ab}(\mathcal{X}_\tau)$.
The discussion above shows that
Homology, Lemma \ref{homology-lemma-double-complex-gives-resolution}
applies which shows that
$\Gamma(\mathcal{X}_\tau, \mathcal{I}^\bullet)$
is quasi-isomorphic to the total complex associated to the double complex.
By our remarks above the complex $f_p^{-1}\mathcal{I}^\bullet$ is an
injective resolution of $f_p^{-1}\mathcal{F}$. Hence the other spectral
sequence is as indicated in the lemma.
\end{proof}

\noindent
To be sure there is a version for modules as well.

\begin{lemma}
\label{lemma-cech-to-cohomology-modules}
Let $f : \mathcal{U} \to \mathcal{X}$ be a $1$-morphism of categories fibred
in groupoids over $(\Sch/S)_{fppf}$. Let
$\tau \in \{Zar, \etale, smooth, syntomic, fppf\}$.
Assume
\begin{enumerate}
\item $\mathcal{F}$ is an object of
$\textit{Mod}(\mathcal{X}_\tau, \mathcal{O}_\mathcal{X})$,
\item for every object $x$ of $\mathcal{X}$ there exists a covering
$\{x_i \to x\}$ in $\mathcal{X}_\tau$ such that each $x_i$ is isomorphic
to $f(u_i)$ for some object $u_i$ of $\mathcal{U}$,
\item the category $\mathcal{U}$ has equalizers, and
\item the functor $f$ is faithful.
\end{enumerate}
Then there is a first quadrant spectral sequence of
$\Gamma(\mathcal{O}_\mathcal{X})$-modules
$$
E_1^{p, q} = H^q((\mathcal{U}_p)_\tau, f_p^*\mathcal{F})
\Rightarrow
H^{p + q}(\mathcal{X}_\tau, \mathcal{F})
$$
converging to the cohomology of $\mathcal{F}$ in the $\tau$-topology.
\end{lemma}

\begin{proof}
The proof of this lemma is identical to the proof of
Lemma \ref{lemma-cech-to-cohomology}
except that it uses an injective resolution in
$\textit{Mod}(\mathcal{X}_\tau, \mathcal{O}_\mathcal{X})$
and it uses
Lemma \ref{lemma-pullback-injective-modules}
instead of
Lemma \ref{lemma-pullback-injective}.
\end{proof}

\noindent
Here is a lemma that translates a more usual kind of covering in the
kinds of coverings we have encountered above.

\begin{lemma}
\label{lemma-surjective-flat-locally-finite-presentation}
Let $f : \mathcal{X} \to \mathcal{Y}$ be a $1$-morphism of
categories fibred in groupoids over $(\Sch/S)_{fppf}$.
\begin{enumerate}
\item Assume that $f$ is representable by algebraic spaces, surjective,
flat, and locally of finite presentation. Then for any object $y$ of
$\mathcal{Y}$ there exists an fppf covering $\{y_i \to y\}$ and objects
$x_i$ of $\mathcal{X}$ such that $f(x_i) \cong y_i$ in $\mathcal{Y}$.
\item Assume that $f$ is representable by algebraic spaces, surjective,
and smooth. Then for any object $y$ of
$\mathcal{Y}$ there exists an \'etale covering $\{y_i \to y\}$ and objects
$x_i$ of $\mathcal{X}$ such that $f(x_i) \cong y_i$ in $\mathcal{Y}$.
\end{enumerate}
\end{lemma}

\begin{proof}
Proof of (1). Suppose that $y$ lies over the scheme $V$.
We may think of $y$ as a morphism $(\Sch/V)_{fppf} \to \mathcal{Y}$.
By definition the $2$-fibre product
$\mathcal{X} \times_\mathcal{Y} (\Sch/V)_{fppf}$
is representable by an algebraic space $W$ and the morphism
$W \to V$ is surjective, flat, and locally of finite presentation.
Choose a scheme $U$ and a surjective \'etale morphism $U \to W$.
Then $U \to V$ is also surjective, flat, and locally of finite presentation
(see Morphisms of Spaces, Lemmas
\ref{spaces-morphisms-lemma-etale-flat},
\ref{spaces-morphisms-lemma-etale-locally-finite-presentation},
\ref{spaces-morphisms-lemma-composition-surjective},
\ref{spaces-morphisms-lemma-composition-finite-presentation}, and
\ref{spaces-morphisms-lemma-composition-flat}).
Hence $\{U \to V\}$ is an fppf covering. Denote $x$ the object of
$\mathcal{X}$ over $U$ corresponding to the $1$-morphism
$(\Sch/U)_{fppf} \to \mathcal{X}$. Then $\{f(x) \to y\}$ is
the desired fppf covering of $\mathcal{Y}$.

\medskip\noindent
Proof of (1). Suppose that $y$ lies over the scheme $V$.
We may think of $y$ as a morphism $(\Sch/V)_{fppf} \to \mathcal{Y}$.
By definition the $2$-fibre product
$\mathcal{X} \times_\mathcal{Y} (\Sch/V)_{fppf}$
is representable by an algebraic space $W$ and the morphism
$W \to V$ is surjective and smooth.
Choose a scheme $U$ and a surjective \'etale morphism $U \to W$.
Then $U \to V$ is also surjective and smooth
(see Morphisms of Spaces, Lemmas
\ref{spaces-morphisms-lemma-etale-smooth},
\ref{spaces-morphisms-lemma-composition-surjective}, and
\ref{spaces-morphisms-lemma-composition-smooth}).
Hence $\{U \to V\}$ is a smooth covering. By
More on Morphisms, Lemma \ref{more-morphisms-lemma-etale-dominates-smooth}
there exists an \'etale covering $\{V_i \to V\}$ such that
each $V_i \to V$ factors through $U$. Denote $x_i$ the object of
$\mathcal{X}$ over $V_i$ corresponding to the $1$-morphism
$$
(\Sch/V_i)_{fppf} \to (\Sch/U)_{fppf} \to \mathcal{X}.
$$
Then $\{f(x_i) \to y\}$ is
the desired \'etale covering of $\mathcal{Y}$.
\end{proof}

\begin{lemma}
\label{lemma-cech-to-cohomology-relative}
Let $f : \mathcal{U} \to \mathcal{X}$ and
$g : \mathcal{X} \to \mathcal{Y}$
be composable $1$-morphisms of categories fibred
in groupoids over $(\Sch/S)_{fppf}$. Let
$\tau \in \{Zar, \etale, smooth, syntomic, \linebreak[0] fppf\}$.
Assume
\begin{enumerate}
\item $\mathcal{F}$ is an abelian sheaf on $\mathcal{X}_\tau$,
\item for every object $x$ of $\mathcal{X}$ there exists a covering
$\{x_i \to x\}$ in $\mathcal{X}_\tau$ such that each $x_i$ is isomorphic
to $f(u_i)$ for some object $u_i$ of $\mathcal{U}$,
\item the category $\mathcal{U}$ has equalizers, and
\item the functor $f$ is faithful.
\end{enumerate}
Then there is a first quadrant spectral sequence of abelian sheaves
on $\mathcal{Y}_\tau$
$$
E_1^{p, q} = R^q(g \circ f_p)_*f_p^{-1}\mathcal{F}
\Rightarrow
R^{p + q}g_*\mathcal{F}
$$
where all higher direct images are computed in the $\tau$-topology.
\end{lemma}

\begin{proof}
Note that the assumptions on $f : \mathcal{U} \to \mathcal{X}$
and $\mathcal{F}$ are identical to those in
Lemma \ref{lemma-cech-to-cohomology}.
Hence the preliminary remarks made in the proof of that lemma
hold here also. These remarks imply in particular that
$$
0 \to g_*\mathcal{I} \to
(g \circ f_0)_*f_0^{-1}\mathcal{I} \to
(g \circ f_1)_*f_1^{-1}\mathcal{I} \to \ldots
$$
is exact if $\mathcal{I}$ is an injective object of
$\textit{Ab}(\mathcal{X}_\tau)$.
Having said this, consider the two spectral sequences of
Homology, Section \ref{homology-section-double-complex}
associated to the double complex $\mathcal{C}^{\bullet, \bullet}$ with terms
$$
\mathcal{C}^{p, q} = (g \circ f_p)_*\mathcal{I}^q
$$
where $\mathcal{F} \to \mathcal{I}^\bullet$ is an injective resolution
in $\textit{Ab}(\mathcal{X}_\tau)$. The first spectral sequence implies, via
Homology, Lemma \ref{homology-lemma-double-complex-gives-resolution},
that $g_*\mathcal{I}^\bullet$ is quasi-isomorphic to the total complex
associated to $\mathcal{C}^{\bullet, \bullet}$.
Since $f_p^{-1}\mathcal{I}^\bullet$ is an injective resolution of
$f_p^{-1}\mathcal{F}$ (see
Lemma \ref{lemma-pullback-injective})
the second spectral sequence has terms
$E_1^{p, q} = R^q(g \circ f_p)_*f_p^{-1}\mathcal{F}$ as in the statement
of the lemma.
\end{proof}

\begin{lemma}
\label{lemma-cech-to-cohomology-relative-modules}
Let $f : \mathcal{U} \to \mathcal{X}$ and
$g : \mathcal{X} \to \mathcal{Y}$
be composable $1$-morphisms of categories fibred
in groupoids over $(\Sch/S)_{fppf}$. Let
$\tau \in \{Zar, \etale, smooth, syntomic, \linebreak[0] fppf\}$.
Assume
\begin{enumerate}
\item $\mathcal{F}$ is an object of
$\textit{Mod}(\mathcal{X}_\tau, \mathcal{O}_\mathcal{X})$,
\item for every object $x$ of $\mathcal{X}$ there exists a covering
$\{x_i \to x\}$ in $\mathcal{X}_\tau$ such that each $x_i$ is isomorphic
to $f(u_i)$ for some object $u_i$ of $\mathcal{U}$,
\item the category $\mathcal{U}$ has equalizers, and
\item the functor $f$ is faithful.
\end{enumerate}
Then there is a first quadrant spectral sequence in
$\textit{Mod}(\mathcal{Y}_\tau, \mathcal{O}_\mathcal{Y})$
$$
E_1^{p, q} = R^q(g \circ f_p)_*f_p^{-1}\mathcal{F}
\Rightarrow
R^{p + q}g_*\mathcal{F}
$$
where all higher direct images are computed in the $\tau$-topology.
\end{lemma}

\begin{proof}
The proof is identical to the proof of
Lemma \ref{lemma-cech-to-cohomology-relative}
except that it uses an injective resolution in
$\textit{Mod}(\mathcal{X}_\tau, \mathcal{O}_\mathcal{X})$
and it uses
Lemma \ref{lemma-pullback-injective-modules}
instead of
Lemma \ref{lemma-pullback-injective}.
\end{proof}






\section{Cohomology on algebraic stacks}
\label{section-cohomology}

\noindent
Let $\mathcal{X}$ be an algebraic stack over $S$. In the sections above
we have seen how to define sheaves for the \'etale, ..., fppf
topologies on $\mathcal{X}$. In fact, we have constructed a site
$\mathcal{X}_\tau$ for each $\tau \in \{Zar, \etale, smooth, syntomic,
fppf\}$. There is a notion of an abelian sheaf $\mathcal{F}$ on these sites.
In the chapter on cohomology of sites we have explained how to define
cohomology. Putting all of this together, let's define the
{\it derived global sections}
$$
R\Gamma_{Zar}(\mathcal{X}, \mathcal{F}),
R\Gamma_\etale(\mathcal{X}, \mathcal{F}), \ldots,
R\Gamma_{fppf}(\mathcal{X}, \mathcal{F})
$$
as $\Gamma(\mathcal{X}_\tau, \mathcal{I}^\bullet)$ where
$\mathcal{F} \to \mathcal{I}^\bullet$ is an injective resolution
in $\textit{Ab}(\mathcal{X}_\tau)$. The $i$th cohomology group is the
$i$th cohomology of the total derived cohomology. We will denote
this
$$
H^i_{Zar}(\mathcal{X}, \mathcal{F}),
H^i_\etale(\mathcal{X}, \mathcal{F}), \ldots,
H^i_{fppf}(\mathcal{X}, \mathcal{F}).
$$
It will turn out that $H^i_\etale = H^i_{smooth}$
because of
More on Morphisms, Lemma \ref{more-morphisms-lemma-etale-dominates-smooth}.
If $\mathcal{F}$ is a presheaf of $\mathcal{O}_\mathcal{X}$-modules
which is a sheaf in the $\tau$-topology, then we use injective
resolutions in $\textit{Mod}(\mathcal{X}_\tau, \mathcal{O}_\mathcal{X})$
to compute total derived global sections and cohomology groups; of course
the end result is quasi-isomorphic resp.\ isomorphic by the general fact
Cohomology on Sites, Lemma
\ref{sites-cohomology-lemma-cohomology-modules-abelian-agree}.

\medskip\noindent
So far our only tool to compute cohomology groups is the result on
{\v C}ech complexes proved above. We rephrase it here in the language
of algebraic stacks for the \'etale and the fppf topology. Let
$f : \mathcal{U} \to \mathcal{X}$ be a $1$-morphism of algebraic stacks.
Recall that
$$
f_p : \mathcal{U}_p =
\mathcal{U} \times_\mathcal{X} \ldots \times_\mathcal{X} \mathcal{U}
\longrightarrow
\mathcal{X}
$$
is the structure morphism where there are $(p + 1)$-factors. Also, recall
that a sheaf on $\mathcal{X}$ is a sheaf for the fppf topology. Note
that if $\mathcal{U}$ is an algebraic space, then
$f : \mathcal{U} \to \mathcal{X}$ is representable by algebraic spaces,
see
Algebraic Stacks, Lemma \ref{algebraic-lemma-representable-diagonal}.
Thus the proposition applies in particular to a smooth cover of the
algebraic stack $\mathcal{X}$ by a scheme.

\begin{proposition}
\label{proposition-smooth-covering-compute-cohomology}
Let $f : \mathcal{U} \to \mathcal{X}$ be a $1$-morphism of algebraic stacks.
\begin{enumerate}
\item Let $\mathcal{F}$ be an abelian \'etale sheaf on $\mathcal{X}$.
Assume that $f$ is representable by algebraic spaces, surjective, and smooth.
Then there is a spectral sequence
$$
E_1^{p, q} = H^q_\etale(\mathcal{U}_p, f_p^{-1}\mathcal{F})
\Rightarrow
H^{p + q}_\etale(\mathcal{X}, \mathcal{F})
$$
\item Let $\mathcal{F}$ be an abelian sheaf on $\mathcal{X}$.
Assume that $f$ is representable by algebraic spaces, surjective, flat,
and locally of finite presentation. Then there is
a spectral sequence
$$
E_1^{p, q} = H^q_{fppf}(\mathcal{U}_p, f_p^{-1}\mathcal{F})
\Rightarrow
H^{p + q}_{fppf}(\mathcal{X}, \mathcal{F})
$$
\end{enumerate}
\end{proposition}

\begin{proof}
To see this we will check the hypotheses (1) -- (4) of
Lemma \ref{lemma-cech-to-cohomology}.
The $1$-morphism $f$ is faithful by
Algebraic Stacks, Lemma
\ref{algebraic-lemma-characterize-representable-by-algebraic-spaces}.
This proves (4).
Hypothesis (3) follows from the fact that $\mathcal{U}$ is an algebraic
stack, see
Lemma \ref{lemma-fibre-products}.
To see (2) apply
Lemma \ref{lemma-surjective-flat-locally-finite-presentation}.
Condition (1) is satisfied by fiat.
\end{proof}










\section{Higher direct images and algebraic stacks}
\label{section-higher-direct-images}

\noindent
Let $g : \mathcal{X} \to \mathcal{Y}$ be a $1$-morphism of algebraic stacks
over $S$. In the sections above we have constructed a morphism of ringed
topoi $g : \Sh(\mathcal{X}_\tau) \to \Sh(\mathcal{Y}_\tau)$
for each $\tau \in \{Zar, \etale, smooth, syntomic, fppf\}$.
In the chapter on cohomology of sites we have explained how to
define higher direct images. Hence the {\it derived direct image}
$Rg_*\mathcal{F}$ is defined as $g_*\mathcal{I}^\bullet$ where
$\mathcal{F} \to \mathcal{I}^\bullet$ is an injective resolution in
$\textit{Ab}(\mathcal{X}_\tau)$. The $i$th higher direct image
$R^ig_*\mathcal{F}$ is the $i$th cohomology of the derived direct image.
Important: it matters which topology $\tau$ is used here!

\medskip\noindent
If $\mathcal{F}$ is a presheaf of $\mathcal{O}_\mathcal{X}$-modules
which is a sheaf in the $\tau$-topology, then we use injective
resolutions in $\textit{Mod}(\mathcal{X}_\tau, \mathcal{O}_\mathcal{X})$
to compute derived direct image and higher direct images.

\medskip\noindent
So far our only tool to compute the higher direct images of $g_*$
is the result on {\v C}ech complexes proved above. This requires the choice
of a ``covering'' $f : \mathcal{U} \to \mathcal{X}$. If $\mathcal{U}$
is an algebraic space, then $f : \mathcal{U} \to \mathcal{X}$
is representable by algebraic spaces, see
Algebraic Stacks, Lemma \ref{algebraic-lemma-representable-diagonal}.
Thus the proposition applies in particular to a smooth cover of the
algebraic stack $\mathcal{X}$ by a scheme.

\begin{proposition}
\label{proposition-smooth-covering-compute-direct-image}
Let $f : \mathcal{U} \to \mathcal{X}$ and $g : \mathcal{X} \to \mathcal{Y}$
be composable $1$-morphisms of algebraic stacks.
\begin{enumerate}
\item Assume that $f$ is representable by algebraic spaces, surjective and
smooth.
\begin{enumerate}
\item If $\mathcal{F}$ is in $\textit{Ab}(\mathcal{X}_\etale)$
then there is a spectral sequence
$$
E_1^{p, q} = R^q(g \circ f_p)_*f_p^{-1}\mathcal{F}
\Rightarrow
R^{p + q}g_*\mathcal{F}
$$
in $\textit{Ab}(\mathcal{Y}_\etale)$ with higher direct images
computed in the \'etale topology.
\item If $\mathcal{F}$ is in
$\textit{Mod}(\mathcal{X}_\etale, \mathcal{O}_\mathcal{X})$ then
there is a spectral sequence
$$
E_1^{p, q} = R^q(g \circ f_p)_*f_p^{-1}\mathcal{F}
\Rightarrow
R^{p + q}g_*\mathcal{F}
$$
in $\textit{Mod}(\mathcal{Y}_\etale, \mathcal{O}_\mathcal{Y})$.
\end{enumerate}
\item Assume that $f$ is representable by algebraic spaces, surjective,
flat, and locally of finite presentation.
\begin{enumerate}
\item If $\mathcal{F}$ is in $\textit{Ab}(\mathcal{X})$ then there is
a spectral sequence
$$
E_1^{p, q} = R^q(g \circ f_p)_*f_p^{-1}\mathcal{F}
\Rightarrow
R^{p + q}g_*\mathcal{F}
$$
in $\textit{Ab}(\mathcal{Y})$ with higher direct images
computed in the fppf topology.
\item If $\mathcal{F}$ is in $\textit{Mod}(\mathcal{O}_\mathcal{X})$ then
there is a spectral sequence
$$
E_1^{p, q} = R^q(g \circ f_p)_*f_p^{-1}\mathcal{F}
\Rightarrow
R^{p + q}g_*\mathcal{F}
$$
in $\textit{Mod}(\mathcal{O}_\mathcal{Y})$.
\end{enumerate}
\end{enumerate}
\end{proposition}

\begin{proof}
To see this we will check the hypotheses (1) -- (4) of
Lemma \ref{lemma-cech-to-cohomology-relative} and
Lemma \ref{lemma-cech-to-cohomology-relative-modules}.
The $1$-morphism $f$ is faithful by
Algebraic Stacks, Lemma
\ref{algebraic-lemma-characterize-representable-by-algebraic-spaces}.
This proves (4).
Hypothesis (3) follows from the fact that $\mathcal{U}$ is an algebraic
stack, see
Lemma \ref{lemma-fibre-products}.
To see (2) apply
Lemma \ref{lemma-surjective-flat-locally-finite-presentation}.
Condition (1) is satisfied by fiat in all four cases.
\end{proof}

\noindent
Here is a description of higher direct images for a
morphism of algebraic stacks.

\begin{lemma}
\label{lemma-pushforward-restriction}
Let $S$ be a scheme. Let $f : \mathcal{X} \to \mathcal{Y}$ be a
$1$-morphism of algebraic stacks\footnote{This result should hold
for any $1$-morphism of categories fibred in groupoids over
$(\Sch/S)_{fppf}$.} over $S$.
Let $\tau \in \{Zariski,\linebreak[0] \etale,\linebreak[0]
smooth,\linebreak[0] syntomic,\linebreak[0] fppf\}$.
Let $\mathcal{F}$ be
an object of $\textit{Ab}(\mathcal{X}_\tau)$ or
$\textit{Mod}(\mathcal{X}_\tau, \mathcal{O}_\mathcal{X})$.
Then the sheaf $R^if_*\mathcal{F}$ is the sheaf associated to the
presheaf
$$
y \longmapsto
H^i_\tau\Big((\Sch/V)_{fppf} \times_{y, \mathcal{Y}} \mathcal{X},
\ \text{pr}^{-1}\mathcal{F}\Big)
$$
Here $y$ is a typical object of $\mathcal{Y}$ lying over the scheme $V$.
\end{lemma}

\begin{proof}
Choose an injective resolution $\mathcal{F}[0] \to \mathcal{I}^\bullet$.
By the formula for pushforward (\ref{equation-pushforward}) we see that
$R^if_*\mathcal{F}$ is the sheaf associated to the presheaf which associates
to $y$ the cohomology of the complex
$$
\begin{matrix}
\Gamma\Big((\Sch/V)_{fppf} \times_{y, \mathcal{Y}} \mathcal{X},
\ \text{pr}^{-1}\mathcal{I}^{i - 1}\Big) \\
\downarrow \\
\Gamma\Big((\Sch/V)_{fppf} \times_{y, \mathcal{Y}} \mathcal{X},
\ \text{pr}^{-1}\mathcal{I}^i\Big) \\
\downarrow \\
\Gamma\Big((\Sch/V)_{fppf} \times_{y, \mathcal{Y}} \mathcal{X},
\ \text{pr}^{-1}\mathcal{I}^{i + 1}\Big)
\end{matrix}
$$
Since $\text{pr}^{-1}$ is exact, it suffices to show that
$\text{pr}^{-1}$ preserves injectives. This follows from
Lemmas \ref{lemma-pullback-injective} and
\ref{lemma-pullback-injective-modules}
as well as the fact that $\text{pr}$ is a representable morphism of
algebraic stacks (so that $\text{pr}$ is faithful by
Algebraic Stacks, Lemma
\ref{algebraic-lemma-characterize-representable-by-algebraic-spaces}
and that
$(\Sch/V)_{fppf} \times_{y, \mathcal{Y}} \mathcal{X}$
has equalizers by 
Lemma \ref{lemma-fibre-products}).
\end{proof}

\noindent
Here is a trivial base change result.

\begin{lemma}
\label{lemma-base-change-higher-direct-images}
Let $S$ be a scheme. Let
$\tau \in \{Zariski,\linebreak[0] \etale,\linebreak[0]
smooth,\linebreak[0] syntomic,\linebreak[0] fppf\}$. Let
$$
\xymatrix{
\mathcal{Y}' \times_\mathcal{Y} \mathcal{X} \ar[r]_{g'} \ar[d]_{f'} &
\mathcal{X} \ar[d]^f \\
\mathcal{Y}' \ar[r]^g & \mathcal{Y}
}
$$
be a $2$-cartesian diagram of algebraic stacks over $S$. Then the base change
map is an isomorphism
$$
g^{-1}Rf_*\mathcal{F} \longrightarrow Rf'_*(g')^{-1}\mathcal{F}
$$
functorial for $\mathcal{F}$ in $\textit{Ab}(\mathcal{X}_\tau)$
or $\mathcal{F}$ in $\textit{Mod}(\mathcal{X}_\tau, \mathcal{O}_\mathcal{X})$.
\end{lemma}

\begin{proof}
The isomorphism $g^{-1}f_*\mathcal{F} = f'_*(g')^{-1}\mathcal{F}$ is
Lemma \ref{lemma-base-change} (and it holds for arbitrary presheaves).
For the derived direct images, there is a base change map because the
morphisms $g$ and $g'$ are flat, see
Cohomology on Sites, Section \ref{sites-cohomology-section-base-change-map}.
To see that this map is a quasi-isomorphism we can use that for
an object $y'$ of $\mathcal{Y}'$ over a scheme $V$ there is an equivalence
$$
(\Sch/V)_{fppf} \times_{g(y'), \mathcal{Y}} \mathcal{X}
=
(\Sch/V)_{fppf} \times_{y', \mathcal{Y}'}
(\mathcal{Y}' \times_\mathcal{Y} \mathcal{X})
$$
We conclude that the induced map
$g^{-1}R^if_*\mathcal{F} \to R^if'_*(g')^{-1}\mathcal{F}$
is an isomorphism by
Lemma \ref{lemma-pushforward-restriction}.
\end{proof}












\section{Comparison}
\label{section-compare}

\noindent
In this section we collect some results on comparing cohomology defined
using stacks and using algebraic spaces.

\begin{lemma}
\label{lemma-compare-injectives}
Let $S$ be a scheme. Let $\mathcal{X}$ be an algebraic stack over $S$
representable by the algebraic space $F$.
\begin{enumerate}
\item If $\mathcal{I}$ injective in $\textit{Ab}(\mathcal{X}_\etale)$, then
$\mathcal{I}|_{F_\etale}$ is injective in $\textit{Ab}(F_\etale)$,
\item If $\mathcal{I}^\bullet$ is a K-injective complex in
$\textit{Ab}(\mathcal{X}_\etale)$, then $\mathcal{I}^\bullet|_{F_\etale}$
is a K-injective complex in $\textit{Ab}(F_\etale)$.
\end{enumerate}
The same does not hold for modules.
\end{lemma}

\begin{proof}
This follows formally from the fact that the restriction functor
$\pi_{F, *} = i_F^{-1}$ (see Lemma \ref{lemma-compare})
is right adjoint to the exact functor $\pi_F^{-1}$, see
Homology, Lemma \ref{homology-lemma-adjoint-preserve-injectives} and
Derived Categories, Lemma \ref{derived-lemma-adjoint-preserve-K-injectives}.
To see that the lemma does not hold for modules, we refer the
reader to \'Etale Cohomology, Lemma
\ref{etale-cohomology-lemma-compare-injectives}.
\end{proof}

\begin{lemma}
\label{lemma-compare-morphism-cohomology}
Let $S$ be a scheme. Let $f : \mathcal{X} \to \mathcal{Y}$ be a morphism
of algebraic stacks over $S$. Assume $\mathcal{X}$, $\mathcal{Y}$ are
representable by algebraic spaces $F$, $G$. Denote $f : F \to G$ the
induced morphism of algebraic spaces.
\begin{enumerate}
\item For any $\mathcal{F} \in \textit{Ab}(\mathcal{X}_\etale)$
we have
$$
(Rf_*\mathcal{F})|_{G_\etale} =
Rf_{small, *}(\mathcal{F}|_{F_\etale})
$$
in $D(G_\etale)$.
\item For any object $\mathcal{F}$ of
$\textit{Mod}(\mathcal{X}_\etale, \mathcal{O}_\mathcal{X})$
we have
$$
(Rf_*\mathcal{F})|_{G_\etale} =
Rf_{small, *}(\mathcal{F}|_{F_\etale})
$$
in $D(\mathcal{O}_G)$.
\end{enumerate}
\end{lemma}

\begin{proof}
Part (1) follows immediately from
Lemma \ref{lemma-compare-injectives}
and (\ref{equation-compare-big-small})
on choosing an injective resolution of $\mathcal{F}$.

\medskip\noindent
Part (2) can be proved as follows. In Lemma \ref{lemma-compare-morphism}
we have seen that $\pi_G \circ f = f_{small} \circ \pi_F$ as morphisms
of ringed sites. Hence we obtain
$R\pi_{G, *} \circ Rf_* = Rf_{small, *} \circ R\pi_{F, *}$
by Cohomology on Sites, Lemma
\ref{sites-cohomology-lemma-derived-pushforward-composition}.
Since the restriction functors $\pi_{F, *}$ and $\pi_{G, *}$
are exact, we conclude.
\end{proof}

\begin{lemma}
\label{lemma-compare-representable-morphism-cohomology}
Let $S$ be a scheme. Consider a $2$-fibre product square
$$
\xymatrix{
\mathcal{X}' \ar[r]_{g'} \ar[d]_{f'} & \mathcal{X} \ar[d]^f \\
\mathcal{Y}' \ar[r]^g & \mathcal{Y}
}
$$
of algebraic stacks over $S$. Assume that $f$ is representable by algebraic
spaces and that $\mathcal{Y}'$ is representable by an algebraic space $G'$.
Then $\mathcal{X}'$ is representable by an algebraic space $F'$ and
denoting $f' : F' \to G'$ the induced morphism of algebraic spaces
we have
$$
g^{-1}(Rf_*\mathcal{F})|_{G'_\etale} =
Rf'_{small, *}((g')^{-1}\mathcal{F}|_{F'_\etale})
$$
for any $\mathcal{F}$ in $\textit{Ab}(\mathcal{X}_\etale)$
or in
$\textit{Mod}(\mathcal{X}_\etale, \mathcal{O}_\mathcal{X})$
\end{lemma}

\begin{proof}
Follows formally on combining
Lemmas \ref{lemma-base-change-higher-direct-images} and
\ref{lemma-compare-morphism-cohomology}.
\end{proof}











\section{Change of topology}
\label{section-change-topology}

\noindent
Here is a technical lemma which tells us that the
fppf cohomology of a locally quasi-coherent sheaf is equal to its
\'etale cohomology provided the comparison maps are isomorphisms
for morphisms of $\mathcal{X}$ lying over flat morphisms.

\begin{lemma}
\label{lemma-lqc-flat-base-change-fppf-sheaf}
Let $S$ be a scheme. Let $\mathcal{X}$ be an algebraic stack over $S$.
Let $\mathcal{F}$ be a presheaf of $\mathcal{O}_\mathcal{X}$-modules.
Assume
\begin{enumerate}
\item[(a)] $\mathcal{F}$ is locally quasi-coherent, and
\item[(b)] for any morphism $\varphi : x \to y$ of $\mathcal{X}$ which lies
over a morphism of schemes $f : U \to V$ which is flat and
locally of finite presentation the comparison map
$c_\varphi : f_{small}^*\mathcal{F}|_{V_\etale} \to
\mathcal{F}|_{U_\etale}$ of
(\ref{equation-comparison-modules}) is an isomorphism.
\end{enumerate}
Then $\mathcal{F}$ is a sheaf for the fppf topology.
\end{lemma}

\begin{proof}
Let $\{x_i \to x\}$ be an fppf covering of $\mathcal{X}$ lying over the
fppf covering $\{f_i : U_i \to U\}$ of schemes over $S$.
By assumption the restriction $\mathcal{G} = \mathcal{F}|_{U_\etale}$
is quasi-coherent and the comparison maps
$f_{i, small}^*\mathcal{G} \to \mathcal{F}|_{U_{i, \etale}}$
are isomorphisms. Hence the sheaf condition for $\mathcal{F}$
and the covering $\{x_i \to x\}$ is equivalent to the sheaf condition
for $\mathcal{G}^a$ on $(\Sch/U)_{fppf}$ and the covering $\{U_i \to U\}$
which holds by
Descent, Lemma \ref{descent-lemma-sheaf-condition-holds}.
\end{proof}

\begin{lemma}
\label{lemma-compare-fppf-etale}
Let $S$ be a scheme. Let $\mathcal{X}$ be an algebraic stack over $S$.
Let $\mathcal{F}$ be a presheaf $\mathcal{O}_\mathcal{X}$-module such that
\begin{enumerate}
\item[(a)] $\mathcal{F}$ is locally quasi-coherent, and
\item[(b)] for any morphism $\varphi : x \to y$ of $\mathcal{X}$ which lies
over a morphism of schemes $f : U \to V$ which is flat and
locally of finite presentation, the comparison map
$c_\varphi : f_{small}^*\mathcal{F}|_{V_\etale} \to
\mathcal{F}|_{U_\etale}$ of
(\ref{equation-comparison-modules}) is an isomorphism.
\end{enumerate}
Then $\mathcal{F}$ is an $\mathcal{O}_\mathcal{X}$-module and
we have the following
\begin{enumerate}
\item If $\epsilon : \mathcal{X}_{fppf} \to \mathcal{X}_\etale$
is the comparison morphism, then
$R\epsilon_*\mathcal{F} = \epsilon_*\mathcal{F}$.
\item The cohomology groups $H^p_{fppf}(\mathcal{X}, \mathcal{F})$ are equal
to the cohomology groups computed in the \'etale topology on $\mathcal{X}$.
Similarly for the cohomology groups $H^p_{fppf}(x, \mathcal{F})$ and the
derived versions $R\Gamma(\mathcal{X}, \mathcal{F})$ and
$R\Gamma(x, \mathcal{F})$.
\item If $f : \mathcal{X} \to \mathcal{Y}$ is a $1$-morphism of
categories fibred in groupoids over $(\Sch/S)_{fppf}$ then
$R^if_*\mathcal{F}$ is equal to the fppf-sheafification of the
higher direct image computed in the \'etale cohomology.
Similarly for derived pullback.
\end{enumerate}
\end{lemma}

\begin{proof}
The assertion that $\mathcal{F}$ is an $\mathcal{O}_\mathcal{X}$-module
follows from
Lemma \ref{lemma-lqc-flat-base-change-fppf-sheaf}.
Note that $\epsilon$ is a morphism of sites given by the identity
functor on $\mathcal{X}$. The sheaf $R^p\epsilon_*\mathcal{F}$ is therefore
the sheaf associated to the presheaf
$x \mapsto H^p_{fppf}(x, \mathcal{F})$, see
Cohomology on Sites, Lemma \ref{sites-cohomology-lemma-higher-direct-images}.
To prove (1) it suffices to show that
$H^p_{fppf}(x, \mathcal{F}) = 0$ for $p > 0$
whenever $x$ lies over an affine scheme $U$. By
Lemma \ref{lemma-cohomology-restriction}
we have
$H^p_{fppf}(x, \mathcal{F}) = H^p((\Sch/U)_{fppf}, x^{-1}\mathcal{F})$.
Combining
Descent, Lemma \ref{descent-lemma-quasi-coherent-and-flat-base-change}
with Cohomology of Schemes, Lemma
\ref{coherent-lemma-quasi-coherent-affine-cohomology-zero}
we see that these cohomology groups are zero.

\medskip\noindent
We have seen above that $\epsilon_*\mathcal{F}$ and $\mathcal{F}$ are the
sheaves on $\mathcal{X}_\etale$ and $\mathcal{X}_{fppf}$
corresponding to the same presheaf on $\mathcal{X}$ (and this is true more
generally for any sheaf in the fppf topology on $\mathcal{X}$).
We often abusively identify $\mathcal{F}$ and $\epsilon_*\mathcal{F}$
and this is the sense in which parts (2) and (3) of the lemma should be
understood. Thus part (2) follows formally from (1) and the Leray spectral
sequence, see
Cohomology on Sites, Lemma \ref{sites-cohomology-lemma-apply-Leray}.

\medskip\noindent
Finally we prove (3). The sheaf $R^if_*\mathcal{F}$
(resp.\ $Rf_{\etale, *}\mathcal{F}$)
is the sheaf associated to the presheaf
$$
y \longmapsto
H^i_\tau\Big((\Sch/V)_{fppf} \times_{y, \mathcal{Y}} \mathcal{X},
\ \text{pr}^{-1}\mathcal{F}\Big)
$$
where $\tau$ is $fppf$ (resp.\ $\etale$), see
Lemma \ref{lemma-pushforward-restriction}.
Note that $\text{pr}^{-1}\mathcal{F}$ satisfies properties (a) and (b)
also (by Lemmas \ref{lemma-pullback-lqc} and \ref{lemma-comparison}),
hence these two presheaves are equal by (2).
This immediately implies (3).
\end{proof}

\noindent
We will use the following lemma to compare \'etale cohomology of sheaves
on algebraic stacks with cohomology on the lisse-\'etale topos.

\begin{lemma}
\label{lemma-cohomology-on-subcategory}
Let $S$ be a scheme. Let $\mathcal{X}$ be an algebraic stack over $S$.
Let $\tau = \etale$ (resp.\ $\tau = fppf$). Let
$\mathcal{X}' \subset \mathcal{X}$ be a full subcategory with the
following properties
\begin{enumerate}
\item if $x \to x'$ is a morphism of $\mathcal{X}$ which lies over a
smooth (resp.\ flat and locally finitely presented) morphism of
schemes and $x' \in \Ob(\mathcal{X}')$, then $x \in \Ob(\mathcal{X}')$, and
\item there exists an object $x \in \Ob(\mathcal{X}')$ lying over
a scheme $U$ such that the associated $1$-morphism
$x : (\Sch/U)_{fppf} \to \mathcal{X}$ is smooth and surjective.
\end{enumerate}
We get a site $\mathcal{X}'_\tau$ by declaring a covering of $\mathcal{X}'$
to be any family of morphisms $\{x_i \to x\}$ in $\mathcal{X}'$ which is a
covering in $\mathcal{X}_\tau$. Then the inclusion functor
$\mathcal{X}' \to \mathcal{X}_\tau$ is fully faithful, cocontinuous, and
continuous, whence defines a morphism of topoi
$$
g : \Sh(\mathcal{X}'_\tau) \longrightarrow \Sh(\mathcal{X}_\tau)
$$
and $H^p(\mathcal{X}'_\tau, g^{-1}\mathcal{F}) =
H^p(\mathcal{X}_\tau, \mathcal{F})$ for all $p \geq 0$ and all
$\mathcal{F} \in \textit{Ab}(\mathcal{X}_\tau)$.
\end{lemma}

\begin{proof}
Note that assumption (1) implies that if $\{x_i \to x\}$ is a covering
of $\mathcal{X}_\tau$ and $x \in \Ob(\mathcal{X}')$, then we have
$x_i \in \Ob(\mathcal{X}')$. Hence we see that $\mathcal{X}' \to \mathcal{X}$
is continuous and cocontinuous as the coverings of objects of
$\mathcal{X}'_\tau$ agree with their coverings seen as objects of
$\mathcal{X}_\tau$. We obtain the morphism $g$ and the functor
$g^{-1}$ is identified with the restriction functor, see
Sites, Lemma \ref{sites-lemma-when-shriek}.

\medskip\noindent
In particular, if $\{x_i \to x\}$ is a covering in $\mathcal{X}'_\tau$,
then for any abelian sheaf $\mathcal{F}$ on $\mathcal{X}$ then
$$
\check H^p(\{x_i \to x\}, g^{-1}\mathcal{F}) =
\check H^p(\{x_i \to x\}, \mathcal{F})
$$
Thus if $\mathcal{I}$ is an injective abelian sheaf on $\mathcal{X}_\tau$
then we see that the higher {\v C}ech cohomology groups are zero
(Cohomology on Sites,
Lemma \ref{sites-cohomology-lemma-injective-trivial-cech}).
Hence $H^p(x, g^{-1}\mathcal{I}) = 0$ for all objects $x$
of $\mathcal{X}'$
(Cohomology on Sites,
Lemma \ref{sites-cohomology-lemma-cech-vanish-collection}).
In other words injective abelian sheaves on $\mathcal{X}_\tau$
are right acyclic for the functor $H^0(x, g^{-1}-)$.
It follows that $H^p(x, g^{-1}\mathcal{F}) = H^p(x, \mathcal{F})$
for all $\mathcal{F} \in \textit{Ab}(\mathcal{X})$ and all
$x \in \Ob(\mathcal{X}')$.

\medskip\noindent
Choose an object $x \in \mathcal{X}'$ lying over a scheme $U$
as in assumption (2). In particular $\mathcal{X}/x \to \mathcal{X}$
is a morphism of algebraic stacks which representable by algebraic spaces,
surjective, and smooth. (Note that $\mathcal{X}/x$ is equivalent to
$(\Sch/U)_{fppf}$, see Lemma \ref{lemma-localizing}.)
The map of sheaves
$$
h_x \longrightarrow *
$$
in $\Sh(\mathcal{X}_\tau)$ is surjective. Namely, for any object $x'$
of $\mathcal{X}$ there exists a $\tau$-covering $\{x'_i \to x'\}$
such that there exist morphisms $x'_i \to x$, see
Lemma \ref{lemma-surjective-flat-locally-finite-presentation}.
Since $g$ is exact, the map of sheaves
$$
g^{-1}h_x \longrightarrow * = g^{-1}*
$$
in $\Sh(\mathcal{X}'_\tau)$ is surjective also. Let $h_{x, n}$ be
the $(n + 1)$-fold product $h_x \times \ldots \times h_x$.
Then we have spectral sequences
\begin{equation}
\label{equation-spectral-sequence-one}
E_1^{p, q} = H^q(h_{x, p}, \mathcal{F}) \Rightarrow
H^{p + q}(\mathcal{X}_\tau, \mathcal{F})
\end{equation}
and
\begin{equation}
\label{equation-spectral-sequence-two}
E_1^{p, q} = H^q(g^{-1}h_{x, p}, g^{-1}\mathcal{F}) \Rightarrow
H^{p + q}(\mathcal{X}'_\tau, g^{-1}\mathcal{F})
\end{equation}
see Cohomology on Sites,
Lemma \ref{sites-cohomology-lemma-cech-to-cohomology-sheaf-sets}.

\medskip\noindent
Case I: $\mathcal{X}$ has a final object $x$ which is also an object of
$\mathcal{X}'$. This case follows immediately from the discussion
in the second paragraph above.

\medskip\noindent
Case II: $\mathcal{X}$ is representable by an algebraic space $F$.
In this case the sheaves $h_{x, n}$ are representable by an
object $x_n$ in $\mathcal{X}$. (Namely, if $\mathcal{S}_F = \mathcal{X}$
and $x : U \to F$ is the given object, then $h_{x, n}$ is representable
by the object $U \times_F \ldots \times_F U \to F$ of $\mathcal{S}_F$.)
It follows that $H^q(h_{x, p}, \mathcal{F}) = H^q(x_p, \mathcal{F})$.
The morphisms $x_n \to x$ lie over smooth morphisms of schemes, hence
$x_n \in \mathcal{X}'$ for all $n$. Hence
$H^q(g^{-1}h_{x, p}, g^{-1}\mathcal{F}) = H^q(x_p, g^{-1}\mathcal{F})$.
Thus in the two spectral sequences
(\ref{equation-spectral-sequence-one}) and
(\ref{equation-spectral-sequence-two}) above the $E_1^{p, q}$ terms agree
by the discussion in the second paragraph. The lemma follows in Case II
as well.

\medskip\noindent
Case III: $\mathcal{X}$ is an algebraic stack. We claim that in this case
the cohomology groups $H^q(h_{x, p}, \mathcal{F})$ and
$H^q(g^{-1}h_{x, n}, g^{-1}\mathcal{F})$ agree by Case II above.
Once we have proved this the result will follow as before.

\medskip\noindent
Namely, consider the category $\mathcal{X}/h_{x, n}$, see
Sites, Lemma \ref{sites-lemma-localize-topos-site}.
Since $h_{x, n}$ is the $(n + 1)$-fold product of $h_x$ an
object of this category is an $(n + 2)$-tuple
$(y, s_0, \ldots, s_n)$ where $y$ is an object of $\mathcal{X}$ and each
$s_i : y \to x$ is a morphism of $\mathcal{X}$.
This is a category over $(\Sch/S)_{fppf}$. There is an equivalence
$$
\mathcal{X}/h_{x, n}
\longrightarrow
(\Sch/U)_{fppf} \times_\mathcal{X} \ldots \times_\mathcal{X} (\Sch/U)_{fppf}
=: \mathcal{U}_n
$$
over $(\Sch/S)_{fppf}$. Namely, if $x : (\Sch/U)_{fppf} \to \mathcal{X}$ also
denotes the $1$-morphism associated with $x$ and
$p : \mathcal{X} \to (\Sch/S)_{fppf}$ the structure functor,
then we can think of $(y, s_0, \ldots, s_n)$ as
$(y, f_0, \ldots, f_n, \alpha_0, \ldots, \alpha_n)$
where $y$ is an object of $\mathcal{X}$, $f_i : p(y) \to p(x)$ is a
morphism of schemes, and $\alpha_i : y \to x(f_i)$ an isomorphism.
The category of $2n+3$-tuples
$(y, f_0, \ldots, f_n, \alpha_0, \ldots, \alpha_n)$
is an incarnation of the $(n + 1)$-fold fibred product $\mathcal{U}_n$
of algebraic stacks displayed above, as we discussed in
Section \ref{section-cech}.
By Cohomology on Sites, Lemma
\ref{sites-cohomology-lemma-cohomology-on-sheaf-sets}
we have
$$
H^p(\mathcal{U}_n, \mathcal{F}|_{\mathcal{U}_n}) =
H^p(\mathcal{X}/h_{x, n}, \mathcal{F}|_{\mathcal{X}/h_{x, n}}) =
H^p(h_{x, n}, \mathcal{F}).
$$
Finally, we discuss the ``primed'' analogue of this. Namely,
$\mathcal{X}'/h_{x, n}$ corresponds, via the equivalence above
to the full subcategory $\mathcal{U}'_n \subset \mathcal{U}_n$
consisting of those tuples
$(y, f_0, \ldots, f_n, \alpha_0, \ldots, \alpha_n)$
with $y \in \mathcal{X}'$. Hence certainly property (1) of the
statement of the lemma holds
for the inclusion $\mathcal{U}'_n \subset \mathcal{U}_n$.
To see property (2) choose an object $\xi = (y, s_0, \ldots, s_n)$ which
lies over a scheme $W$ such that $(\Sch/W)_{fppf} \to \mathcal{U}_n$
is smooth and surjective (this is possible as $\mathcal{U}_n$ is
an algebraic stack). Then
$(\Sch/W)_{fppf} \to \mathcal{U}_n \to (\Sch/U)_{fppf}$
is smooth as a composition of base changes of the morphism
$x : (\Sch/U)_{fppf} \to \mathcal{X}$, see
Algebraic Stacks, Lemmas
\ref{algebraic-lemma-base-change-representable-transformations-property} and
\ref{algebraic-lemma-composition-representable-transformations-property}.
Thus axiom (1) for $\mathcal{X}$ implies that $y$ is an object of
$\mathcal{X}'$ whence $\xi$ is an object of $\mathcal{U}'_n$.
Using again
$$
H^p(\mathcal{U}'_n, \mathcal{F}|_{\mathcal{U}'_n}) =
H^p(\mathcal{X}'/h_{x, n}, \mathcal{F}|_{\mathcal{X}'/h_{x, n}}) =
H^p(g^{-1}h_{x, n}, g^{-1}\mathcal{F}).
$$
we now can use Case II for 
$\mathcal{U}'_n \subset \mathcal{U}_n$
to conclude.
\end{proof}









\section{Other chapters}

\begin{multicols}{2}
\begin{enumerate}
\item \hyperref[introduction-section-phantom]{Introduction}
\item \hyperref[conventions-section-phantom]{Conventions}
\item \hyperref[sets-section-phantom]{Set Theory}
\item \hyperref[categories-section-phantom]{Categories}
\item \hyperref[topology-section-phantom]{Topology}
\item \hyperref[sheaves-section-phantom]{Sheaves on Spaces}
\item \hyperref[algebra-section-phantom]{Commutative Algebra}
\item \hyperref[sites-section-phantom]{Sites and Sheaves}
\item \hyperref[homology-section-phantom]{Homological Algebra}
\item \hyperref[derived-section-phantom]{Derived Categories}
\item \hyperref[more-algebra-section-phantom]{More Algebra}
\item \hyperref[simplicial-section-phantom]{Simplicial Methods}
\item \hyperref[modules-section-phantom]{Sheaves of Modules}
\item \hyperref[sites-modules-section-phantom]{Modules on Sites}
\item \hyperref[injectives-section-phantom]{Injectives}
\item \hyperref[cohomology-section-phantom]{Cohomology of Sheaves}
\item \hyperref[sites-cohomology-section-phantom]{Cohomology on Sites}
\item \hyperref[hypercovering-section-phantom]{Hypercoverings}
\item \hyperref[schemes-section-phantom]{Schemes}
\item \hyperref[constructions-section-phantom]{Constructions of Schemes}
\item \hyperref[properties-section-phantom]{Properties of Schemes}
\item \hyperref[morphisms-section-phantom]{Morphisms of Schemes}
\item \hyperref[coherent-section-phantom]{Coherent Cohomology}
\item \hyperref[divisors-section-phantom]{Divisors}
\item \hyperref[limits-section-phantom]{Limits of Schemes}
\item \hyperref[varieties-section-phantom]{Varieties}
\item \hyperref[chow-section-phantom]{Chow Homology}
\item \hyperref[topologies-section-phantom]{Topologies on Schemes}
\item \hyperref[descent-section-phantom]{Descent}
\item \hyperref[more-morphisms-section-phantom]{More on Morphisms}
\item \hyperref[flat-section-phantom]{More on Flatness}
\item \hyperref[groupoids-section-phantom]{Groupoid Schemes}
\item \hyperref[more-groupoids-section-phantom]{More on Groupoid Schemes}
\item \hyperref[etale-section-phantom]{\'Etale Morphisms of Schemes}
\item \hyperref[etale-cohomology-section-phantom]{\'Etale Cohomology}
\item \hyperref[spaces-section-phantom]{Algebraic Spaces}
\item \hyperref[spaces-properties-section-phantom]{Properties of Algebraic Spaces}
\item \hyperref[spaces-morphisms-section-phantom]{Morphisms of Algebraic Spaces}
\item \hyperref[spaces-topologies-section-phantom]{Topologies on Algebraic Spaces}
\item \hyperref[spaces-descent-section-phantom]{Descent and Algebraic Spaces}
\item \hyperref[spaces-more-morphisms-section-phantom]{More on Morphisms of Spaces}
\item \hyperref[quot-section-phantom]{Quot and Hilbert Spaces}
\item \hyperref[stacks-section-phantom]{Stacks}
\item \hyperref[spaces-groupoids-section-phantom]{Groupoids in Algebraic Spaces}
\item \hyperref[spaces-more-groupoids-section-phantom]{More on Groupoids in Spaces}
\item \hyperref[bootstrap-section-phantom]{Bootstrap}
\item \hyperref[examples-stacks-section-phantom]{Examples of Stacks}
\item \hyperref[groupoids-quotients-section-phantom]{Quotients of Groupoids}
\item \hyperref[algebraic-section-phantom]{Algebraic Stacks}
\item \hyperref[criteria-section-phantom]{Criteria for Representability}
\item \hyperref[stacks-properties-section-phantom]{Properties of Algebraic Stacks}
\item \hyperref[stacks-morphisms-section-phantom]{Morphisms of Algebraic Stacks}
\item \hyperref[examples-section-phantom]{Examples}
\item \hyperref[exercises-section-phantom]{Exercises}
\item \hyperref[guide-section-phantom]{Guide to Literature}
\item \hyperref[desirables-section-phantom]{Desirables}
\item \hyperref[coding-section-phantom]{Coding Style}
\item \hyperref[fdl-section-phantom]{GNU Free Documentation License}
\item \hyperref[index-section-phantom]{Auto Generated Index}
\end{enumerate}
\end{multicols}


\bibliography{my}
\bibliographystyle{amsalpha}

\end{document}
