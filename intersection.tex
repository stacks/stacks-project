\IfFileExists{stacks-project.cls}{%
\documentclass{stacks-project}
}{%
\documentclass{amsart}
}

% The following AMS packages are automatically loaded with
% the amsart documentclass:
%\usepackage{amsmath}
%\usepackage{amssymb}
%\usepackage{amsthm}

% For dealing with references we use the comment environment
\usepackage{verbatim}
\newenvironment{reference}{\comment}{\endcomment}
%\newenvironment{reference}{}{}
\newenvironment{slogan}{\comment}{\endcomment}
\newenvironment{history}{\comment}{\endcomment}

% For commutative diagrams you can use
% \usepackage{amscd}
\usepackage[all]{xy}

% We use 2cell for 2-commutative diagrams.
\xyoption{2cell}
\UseAllTwocells

% To put source file link in headers.
% Change "template.tex" to "this_filename.tex"
% \usepackage{fancyhdr}
% \pagestyle{fancy}
% \lhead{}
% \chead{}
% \rhead{Source file: \url{template.tex}}
% \lfoot{}
% \cfoot{\thepage}
% \rfoot{}
% \renewcommand{\headrulewidth}{0pt}
% \renewcommand{\footrulewidth}{0pt}
% \renewcommand{\headheight}{12pt}

\usepackage{multicol}

% For cross-file-references
\usepackage{xr-hyper}

% Package for hypertext links:
\usepackage{hyperref}

% For any local file, say "hello.tex" you want to link to please
% use \externaldocument[hello-]{hello}
\externaldocument[introduction-]{introduction}
\externaldocument[conventions-]{conventions}
\externaldocument[sets-]{sets}
\externaldocument[categories-]{categories}
\externaldocument[topology-]{topology}
\externaldocument[sheaves-]{sheaves}
\externaldocument[sites-]{sites}
\externaldocument[stacks-]{stacks}
\externaldocument[fields-]{fields}
\externaldocument[algebra-]{algebra}
\externaldocument[brauer-]{brauer}
\externaldocument[homology-]{homology}
\externaldocument[derived-]{derived}
\externaldocument[simplicial-]{simplicial}
\externaldocument[more-algebra-]{more-algebra}
\externaldocument[smoothing-]{smoothing}
\externaldocument[modules-]{modules}
\externaldocument[sites-modules-]{sites-modules}
\externaldocument[injectives-]{injectives}
\externaldocument[cohomology-]{cohomology}
\externaldocument[sites-cohomology-]{sites-cohomology}
\externaldocument[dga-]{dga}
\externaldocument[dpa-]{dpa}
\externaldocument[hypercovering-]{hypercovering}
\externaldocument[schemes-]{schemes}
\externaldocument[constructions-]{constructions}
\externaldocument[properties-]{properties}
\externaldocument[morphisms-]{morphisms}
\externaldocument[coherent-]{coherent}
\externaldocument[divisors-]{divisors}
\externaldocument[limits-]{limits}
\externaldocument[varieties-]{varieties}
\externaldocument[topologies-]{topologies}
\externaldocument[descent-]{descent}
\externaldocument[perfect-]{perfect}
\externaldocument[more-morphisms-]{more-morphisms}
\externaldocument[flat-]{flat}
\externaldocument[groupoids-]{groupoids}
\externaldocument[more-groupoids-]{more-groupoids}
\externaldocument[etale-]{etale}
\externaldocument[chow-]{chow}
\externaldocument[intersection-]{intersection}
\externaldocument[pic-]{pic}
\externaldocument[adequate-]{adequate}
\externaldocument[dualizing-]{dualizing}
\externaldocument[duality-]{duality}
\externaldocument[discriminant-]{discriminant}
\externaldocument[local-cohomology-]{local-cohomology}
\externaldocument[curves-]{curves}
\externaldocument[resolve-]{resolve}
\externaldocument[models-]{models}
\externaldocument[pione-]{pione}
\externaldocument[etale-cohomology-]{etale-cohomology}
\externaldocument[proetale-]{proetale}
\externaldocument[crystalline-]{crystalline}
\externaldocument[spaces-]{spaces}
\externaldocument[spaces-properties-]{spaces-properties}
\externaldocument[spaces-morphisms-]{spaces-morphisms}
\externaldocument[decent-spaces-]{decent-spaces}
\externaldocument[spaces-cohomology-]{spaces-cohomology}
\externaldocument[spaces-limits-]{spaces-limits}
\externaldocument[spaces-divisors-]{spaces-divisors}
\externaldocument[spaces-over-fields-]{spaces-over-fields}
\externaldocument[spaces-topologies-]{spaces-topologies}
\externaldocument[spaces-descent-]{spaces-descent}
\externaldocument[spaces-perfect-]{spaces-perfect}
\externaldocument[spaces-more-morphisms-]{spaces-more-morphisms}
\externaldocument[spaces-flat-]{spaces-flat}
\externaldocument[spaces-groupoids-]{spaces-groupoids}
\externaldocument[spaces-more-groupoids-]{spaces-more-groupoids}
\externaldocument[bootstrap-]{bootstrap}
\externaldocument[spaces-pushouts-]{spaces-pushouts}
\externaldocument[groupoids-quotients-]{groupoids-quotients}
\externaldocument[spaces-more-cohomology-]{spaces-more-cohomology}
\externaldocument[spaces-simplicial-]{spaces-simplicial}
\externaldocument[spaces-duality-]{spaces-duality}
\externaldocument[formal-spaces-]{formal-spaces}
\externaldocument[restricted-]{restricted}
\externaldocument[spaces-resolve-]{spaces-resolve}
\externaldocument[formal-defos-]{formal-defos}
\externaldocument[defos-]{defos}
\externaldocument[cotangent-]{cotangent}
\externaldocument[examples-defos-]{examples-defos}
\externaldocument[algebraic-]{algebraic}
\externaldocument[examples-stacks-]{examples-stacks}
\externaldocument[stacks-sheaves-]{stacks-sheaves}
\externaldocument[criteria-]{criteria}
\externaldocument[artin-]{artin}
\externaldocument[quot-]{quot}
\externaldocument[stacks-properties-]{stacks-properties}
\externaldocument[stacks-morphisms-]{stacks-morphisms}
\externaldocument[stacks-limits-]{stacks-limits}
\externaldocument[stacks-cohomology-]{stacks-cohomology}
\externaldocument[stacks-perfect-]{stacks-perfect}
\externaldocument[stacks-introduction-]{stacks-introduction}
\externaldocument[stacks-more-morphisms-]{stacks-more-morphisms}
\externaldocument[stacks-geometry-]{stacks-geometry}
\externaldocument[moduli-]{moduli}
\externaldocument[moduli-curves-]{moduli-curves}
\externaldocument[examples-]{examples}
\externaldocument[exercises-]{exercises}
\externaldocument[guide-]{guide}
\externaldocument[desirables-]{desirables}
\externaldocument[coding-]{coding}
\externaldocument[obsolete-]{obsolete}
\externaldocument[fdl-]{fdl}
\externaldocument[index-]{index}

% Theorem environments.
%
\theoremstyle{plain}
\newtheorem{theorem}[subsection]{Theorem}
\newtheorem{proposition}[subsection]{Proposition}
\newtheorem{lemma}[subsection]{Lemma}

\theoremstyle{definition}
\newtheorem{definition}[subsection]{Definition}
\newtheorem{example}[subsection]{Example}
\newtheorem{exercise}[subsection]{Exercise}
\newtheorem{situation}[subsection]{Situation}

\theoremstyle{remark}
\newtheorem{remark}[subsection]{Remark}
\newtheorem{remarks}[subsection]{Remarks}

\numberwithin{equation}{subsection}

% Macros
%
\def\lim{\mathop{\mathrm{lim}}\nolimits}
\def\colim{\mathop{\mathrm{colim}}\nolimits}
\def\Spec{\mathop{\mathrm{Spec}}}
\def\Hom{\mathop{\mathrm{Hom}}\nolimits}
\def\Ext{\mathop{\mathrm{Ext}}\nolimits}
\def\SheafHom{\mathop{\mathcal{H}\!\mathit{om}}\nolimits}
\def\SheafExt{\mathop{\mathcal{E}\!\mathit{xt}}\nolimits}
\def\Sch{\mathit{Sch}}
\def\Mor{\operatorname{Mor}\nolimits}
\def\Ob{\mathop{\mathrm{Ob}}\nolimits}
\def\Sh{\mathop{\mathit{Sh}}\nolimits}
\def\NL{\mathop{N\!L}\nolimits}
\def\proetale{{pro\text{-}\acute{e}tale}}
\def\etale{{\acute{e}tale}}
\def\QCoh{\mathit{QCoh}}
\def\Ker{\mathop{\mathrm{Ker}}}
\def\Im{\mathop{\mathrm{Im}}}
\def\Coker{\mathop{\mathrm{Coker}}}
\def\Coim{\mathop{\mathrm{Coim}}}

%
% Macros for moduli stacks/spaces
%
\def\QCohstack{\mathcal{QC}\!\mathit{oh}}
\def\Cohstack{\mathcal{C}\!\mathit{oh}}
\def\Spacesstack{\mathcal{S}\!\mathit{paces}}
\def\Quotfunctor{\mathrm{Quot}}
\def\Hilbfunctor{\mathrm{Hilb}}
\def\Curvesstack{\mathcal{C}\!\mathit{urves}}
\def\Polarizedstack{\mathcal{P}\!\mathit{olarized}}
\def\Complexesstack{\mathcal{C}\!\mathit{omplexes}}
% \Pic is the operator that assigns to X its picard group, usage \Pic(X)
% \Picardstack_{X/B} denotes the Picard stack of X over B
% \Picardfunctor_{X/B} denotes the Picard functor of X over B
\def\Pic{\mathop{\mathrm{Pic}}\nolimits}
\def\Picardstack{\mathcal{P}\!\mathit{ic}}
\def\Picardfunctor{\mathrm{Pic}}
\def\Deformationcategory{\mathcal{D}\!\mathit{ef}}


% OK, start here.
%
\begin{document}

\title{Intersection Theory}


\maketitle

\phantomsection
\label{section-phantom}

\tableofcontents


\section{Introduction}
\label{section-introduction}

\noindent
In this chapter we construct the intersection product on the Chow groups
modulo rational equivalence on a nonsingular projective variety over an
algebraically closed field. Our tools are Serre's Tor formula
(see \cite[Chapter V]{Serre_algebre_locale}), reduction to the diagonal,
and the moving lemma.

\medskip\noindent
We first recall cycles and how to construct proper pushforward and
flat pullback of cycles. Next, we introduce rational equivalence of cycles
which gives us the Chow groups $A_*(X)$. Proper pushforward and flat pullback
factor through rational equivalence to give operations on Chow groups.
This takes up Sections 
\ref{section-cycles},
\ref{section-cycle-of-closed},
\ref{section-cycle-of-coherent-sheaf},
\ref{section-pushforward},
\ref{section-flat-pullback},
\ref{section-rational-equivalence},
\ref{section-alternative},
\ref{section-pushforward-and-rational-equivalence}, and
\ref{section-flat-pullback-and-rational-equivalence}.
For proofs we mostly refer to the chapter on Chow homology
where these results have been proven in the setting of
schemes locally of finite type over a universally catenary Noetherian base, see
Chow Homology, Section \ref{chow-section-setup} ff.

\medskip\noindent
Since we work on a nonsingular projective $X$ any irreducible component
of the intersection $V \cap W$ of two irreducible closed subvarieties
has dimension at least $\dim(V) + \dim(W) - \dim(X)$. We say $V$ and $W$
intersect properly if equality holds for every irreducible component $Z$.
In this case we define the intersection multiplicity
$e_Z = e(X, V \cdot W, Z)$ by the formula
$$
e_Z = \sum\nolimits_i
(-1)^i
\text{length}_{\mathcal{O}_{X, Z}}
\text{Tor}_i^{\mathcal{O}_{X, Z}}(\mathcal{O}_{W, Z}, \mathcal{O}_{V, Z})
$$
We need to do a little bit of commutative algebra to show that these
intersection multiplicities agree with intuition in simple cases,
namely, that sometimes
$$
e_Z = \text{length}_{\mathcal{O}_{X, Z}} \mathcal{O}_{V \cap W, Z},
$$
in other words, only $\text{Tor}_0$ contributes. This happens when
$V$ and $W$ are Cohen-Macaulay in the generic point of $Z$ or when
$W$ is cut out by a regular sequence in $\mathcal{O}_{X, Z}$ which
also defines a regular sequence on $\mathcal{O}_{V, Z}$. However,
Example \ref{example-naive-multiplicity-wrong} shows that higher
tors are necessary in general. Moreover, there is a relationship
with the Samuel multiplicity. These matters are discussed in
Sections
\ref{section-intersect-properly},
\ref{section-tor-formula},
\ref{section-multiplicities},
\ref{section-computing-intersection-multiplicities}, and
\ref{section-intersection-product}. 

\medskip\noindent
Reduction to the diagonal is the statement that we can intersect
$V$ and $W$ by intersecting $V \times W$ with the diagonal in $X \times X$.
This innocuous statement, which is clear on the level of scheme
theoretic intersections, reduces an intersection of a general pair
of closed subschemes, to the case where one of the two is locally cut
out by a regular sequence. We use this, following Serre, to obtain positivity
of intersection multiplicities. Moreover, reduction to the diagonal
leads to additivity of intersection multiplicities, associativity, and a
projection formula. This can be found in Sections
\ref{section-exterior-product},
\ref{section-reduction-diagonal},
\ref{section-associative},
\ref{section-flat-pullback-and-intersection-products}, and
\ref{section-projection-formula-flat}.

\medskip\noindent
Finally, we come to the moving lemmas and applications. There are two
parts to the moving lemma. The first is that given closed subvarieties
$$
Z \subset X \subset \mathbf{P}^N
$$
with $X$ nonsingular, we can find a subvariety $C \subset \mathbf{P}^N$
intersecting $X$ properly such that
$$
C \cdot X = [Z] + \sum m_j [Z_j]
$$
and such that the other components $Z_j$ are ``more general'' than $Z$.
The second part is that one can move $C \subset \mathbf{P}^N$ over
a rational curve to a subvariety in general position with respect to
any given list of subvarieties. Combined these results imply that it suffices
to define the intersection product of cycles on $X$ which intersect
properly which was done above. Of course this only leads to an intersection
product on $A_*(X)$ if one can show, as we do in the text, that these products
pass through rational equivalence. This and some applications are discussed 
in Sections
\ref{section-projection},
\ref{section-moving-lemma},
\ref{section-intersections-and-rational-equivalence},
\ref{section-chow-rings},
\ref{section-general-pullback}, and
\ref{section-pullback-cycles}.



\section{Conventions}
\label{section-conventions}

\noindent
We fix an algebraically closed ground field $\mathbf{C}$ of any
characteristic. All schemes and varieties are over $\mathbf{C}$ and all
morphisms are over $\mathbf{C}$. A variety $X$ is
{\it nonsingular} if $X$ is a regular scheme (see
Properties, Definition \ref{properties-definition-regular}).
In our case this means that the morphism $X \to \Spec(\mathbf{C})$
is smooth (see
Varieties, Lemma \ref{varieties-lemma-geometrically-regular-smooth}).


\section{Cycles}
\label{section-cycles}

\noindent
Let $X$ be a variety. A {\it closed subvariety} of $X$ is an integral
closed subscheme $Z \subset X$. A {\it $k$-cycle} on $X$ is a finite
formal sum $\sum n_i [Z_i]$ where each $Z_i$ is a closed subvariety
of dimension $k$. Whenever we use the notation $\alpha = \sum n_i[Z_i]$
for a $k$-cycle we always assume the subvarieties $Z_i$ are pairwise
distinct and $n_i \not = 0$ for all $i$. In this case the
{\it support} of $\alpha$ is the closed subset
$$
\text{Supp}(\alpha) = \bigcup Z_i \subset X
$$
of dimension $k$. The group of $k$-cycles is denoted $Z_k(X)$.
See Chow Homology, Section \ref{chow-section-cycles}.


\section{Cycle associated to closed subscheme}
\label{section-cycle-of-closed}

\noindent
Suppose that $X$ is a variety and that $Z \subset X$ be a closed subscheme
with $\dim(Z) \leq k$. Let $Z_i$ be the irreducible components of $Z$ of
dimension $k$ and let $n_i$ be the {\it multiplicity of $Z_i$ in $Z$}
defined as
$$
n_i = \text{length}_{\mathcal{O}_{X, Z_i}} \mathcal{O}_{Z, Z_i}
$$
where $\mathcal{O}_{X, Z_i}$, resp.\ $\mathcal{O}_{Z, Z_i}$ is the
local ring of $X$, resp.\ $Z$ at the generic point of $Z_i$.
We define the $k$-cycle associated to $Z$ to be the $k$-cycle
$$
[Z]_k = \sum n_i [Z_i].
$$
See Chow Homology, Section \ref{chow-section-cycle-of-closed-subscheme}.


\section{Cycle associated to a coherent sheaf}
\label{section-cycle-of-coherent-sheaf}

\noindent
Suppose that $X$ is a variety and that
$\mathcal{F}$ is a coherent $\mathcal{O}_X$-module with
$\dim(\text{Supp}(\mathcal{F})) \leq k$.
Let $Z_i$ be the irreducible components of $\text{Supp}(\mathcal{F})$
of dimension $k$ and let $n_i$ be the
{\it multiplicity of $Z_i$ in $\mathcal{F}$} defined as
$$
n_i = \text{length}_{\mathcal{O}_{X, Z_i}} \mathcal{F}_{\xi_i}
$$
where $\mathcal{O}_{X, Z_i}$ is the
local ring of $X$ at the generic point $\xi_i$ of $Z_i$
and $\mathcal{F}_{\xi_i}$ is the stalk of $\mathcal{F}$ at this point.
We define the $k$-cycle associated to $\mathcal{F}$ to be the $k$-cycle
$$
[\mathcal{F}]_k = \sum n_i [Z_i].
$$
See Chow Homology, Section \ref{chow-section-cycle-of-coherent-sheaf}.
Note that, if $Z \subset X$ is a closed subscheme with $\dim(Z) \leq k$, then
$[Z]_k = [\mathcal{O}_Z]_k$ by definition.


\section{Proper pushforward}
\label{section-pushforward}

\noindent
Suppose that $f : X \to Y$ is a proper morphism of varieties.
Let $Z \subset X$ be a $k$-dimensional
closed subvariety. We define $f_*[Z]$ to be $0$ if $\dim(f(Z))<k$
and $d \cdot [f(Z)]$ if $\dim(f(Z)) = k$ where
$$
d = [\mathbf{C}(Z) : \mathbf{C}(f(Z))] = \deg(Z/f(Z))
$$
is the degree of the dominant morphism $Z \to f(Z)$, see
Morphisms, Definition \ref{morphisms-definition-degree}.
Let $\alpha = \sum n_i [Z_i]$ be a $k$-cycle on $Y$. The
{\it pushforward} of $\alpha$ is the sum $f_* \alpha = \sum n_i f_*[Z_i]$
where each $f_*[Z_i]$ is defined as above. This defines a homomorphism
$$
f_* : Z_k(X) \longrightarrow Z_k(Y)
$$
See Chow Homology, Section \ref{chow-section-proper-pushforward}.

\begin{lemma}
\label{lemma-push-coherent}
\begin{reference}
See \cite[Chapter V]{Serre_algebre_locale}.
\end{reference}
Suppose that $f : X \to Y$ is a proper morphism of varieties.
Let $\mathcal{F}$ be a coherent sheaf with
$\dim(\text{Supp}(\mathcal{F})) \leq k$, then
$f_*[\mathcal{F}]_k = [f_*\mathcal{F}]_k$. In particular, if
$Z \subset X$ is a closed subscheme of dimension $\leq k$, then
$f_*[Z] = [f_*\mathcal{O}_Z]_k$.
\end{lemma}

\begin{proof}
See Chow Homology, Lemma \ref{chow-lemma-cycle-push-sheaf}.
\end{proof}

\begin{lemma}
\label{lemma-compose-pushforward}
Let $f : X \to Y$ and $g : Y \to Z$ be proper morphisms of
varieties. Then $g_* \circ f_* = (g \circ f)_*$ as maps $Z_k(X) \to Z_k(Z)$.
\end{lemma}

\begin{proof}
Special case of Chow Homology, Lemma \ref{chow-lemma-compose-pushforward}.
\end{proof}



\section{Flat pullback}
\label{section-flat-pullback}

\noindent
Suppose that $f : X \to Y$ is a flat morphism of varieties.
By Morphisms, Lemma
\ref{morphisms-lemma-dimension-fibre-at-a-point-additive} 
every fibre of $f$ has dimension $r = \dim(X) - \dim(Y)$\footnote{Conversely,
if $f : X \to Y$ is a dominant morphism of varieties,
$X$ is Cohen-Macaulay, $Y$ is nonsingular, and all fibres have
the same dimension $r$, then $f$ is flat. This follows from
Algebra, Lemma \ref{algebra-lemma-CM-over-regular-flat} and
Varieties, Lemma \ref{varieties-lemma-dimension-fibres-locally-algebraic}
showing $\dim(X) = \dim(Y) + r$.}.
Let $Z \subset X$ be a $k$-dimensional closed subvariety. We define
$f^*[Z]$ to be the $(k + r)$-cycle associated to the scheme theoretic
inverse image: $f^*[Z] = [f^{-1}(Z)]_{k + r}$. Let
$\alpha = \sum n_i [Z_i]$ be a $k$-cycle on $Y$. The {\it pullback} of
$\alpha$ is the sum $f_* \alpha = \sum n_i f^*[Z_i]$ where each $f^*[Z_i]$
is defined as above. This defines a homomorphism
$$
f^* : Z_k(Y) \longrightarrow Z_{k + r}(X)
$$
See Chow Homology, Section \ref{chow-section-flat-pullback}.

\begin{lemma}
\label{lemma-pullback}
Let $f : X \to Y$ be a flat morphism of varieties. Set $r = \dim(X) - \dim(Y)$.
Then $f^*[\mathcal{F}]_k = [f^*\mathcal{F}]_{k + r}$
if $\mathcal{F}$ is a coherent sheaf on $Y$ and the dimension of the
support of $\mathcal{F}$ is at most $k$.
\end{lemma}

\begin{proof}
See Chow Homology, Lemma \ref{chow-lemma-pullback-coherent}.
\end{proof}

\begin{lemma}
\label{lemma-compose-flat-pullback}
Let $f : X \to Y$ and $g : Y \to Z$ be flat morphisms of
varieties. Then $g \circ f$ is flat and $f^* \circ g^* = (g \circ f)^*$
as maps $Z_k(Z) \to Z_{k + \dim(X) - \dim(Z)}(X)$.
\end{lemma}

\begin{proof}
Special case of Chow Homology, Lemma \ref{chow-lemma-compose-flat-pullback}.
\end{proof}


\section{Rational Equivalence}
\label{section-rational-equivalence}

\noindent
We are going to define rational equivalence in a way which at first
glance may seem different from what you are used to, or from what
is in \cite[Chapter I]{F} or
Chow Homology, Section \ref{chow-section-rational-equivalence}.
However, in Section \ref{section-alternative} we will show that
the two notions agree.

\medskip\noindent
Let $X$ be a variety. Let $W \subset X \times \mathbf{P}^1$
be a closed subvariety of dimension $k + 1$. Let $a, b$ be distinct closed
points of $\mathbf{P}^1$. Assume that $X \times a$, $X \times b$ and $W$
intersect properly:
$$
\dim (W \cap X \times a) \leq k,\quad
\dim (W \cap X \times b) \leq k.
$$
This is true as soon as $W \to \mathbf{P}^1$ is dominant or if $W$ is
contained in a fibre of the projection over a closed point different from
$a$ or $b$ (this is an uninteresting case which we will discard). In this
situation the scheme theoretic fibre $W_a$ of the morphism
$W \to \mathbf{P}^1$ is equal to the scheme theoretic intersection
$W \cap X \times a$ in $X \times \mathbf{P}^1$. Identifying $X \times a$
and $X \times b$ with $X$ we may think of the fibres $W_a$ and $W_b$
as closed subschemes of $X$ of dimension $\leq k$\footnote{We will sometimes
think of $W_a$ as a closed subscheme of $X \times \mathbf{P}^1$ and sometimes
as a closed subscheme of $X$. It should always be clear from context which
point of view is taken.}. A basic example of a
rational equivalence is
$$
[W_a]_k \sim_{rat} [W_b]_k
$$
The cycles $[W_a]_k$ and $[W_b]_k$ are easy to compute in practice
(given $W$) because they are obtained by proper intersection with
a Cartier divisor (we will see this in
Section \ref{section-intersection-product}).
Since the automorphism group of $\mathbf{P}^1$ is $2$-transitive we may
move the pair of closed points $a, b$ to any pair we like. A traditional
choice is to choose $a = 0$ and $b = \infty$.

\medskip\noindent
More generally, let $\alpha = \sum n_i [W_i]$ be a $(k + 1)$-cycle on
$X \times \mathbf{P}^1$.  Let $a_i, b_i$ be pairs of distinct closed points of
$\mathbf{P}^1$. Assume that $X \times a_i$, $X \times b_i$ and $W_i$ intersect
properly, in other words, each $W_i, a_i, b_i$ satisfies the condition
discussed above. A {\it cycle rationally equivalent to zero} is any cycle
of the form
$$
\sum n_i([W_{i, a_i}]_k - [W_{i, b_i}]_k).
$$
This is indeed a $k$-cycle. The collection of $k$-cycles rationally
equivalent to zero is an additive subgroup of the group of $k$-cycles.
We say two $k$-cycles are {\it rationally equivalent}, notation
$\alpha \sim_{rat} \alpha'$, if $\alpha - \alpha'$ is a cycle rationally
equivalent to zero.

\medskip\noindent
We define
$$
A_k(X) = Z_k(X)/ \sim_{rat}
$$
to be the {\it Chow group of $k$-cycles on $X$}. We will see in
Lemma \ref{lemma-rational-equivalence}
that this agrees with the Chow group as defined in
Chow Homology, Definition \ref{chow-definition-rational-equivalence}.


\section{Rational equivalence and rational functions}
\label{section-alternative}

\noindent
Let $X$ be a variety. Let $W \subset X$ be a subvariety
of dimension $k + 1$. Let $f \in \mathbf{C}(W)^*$ be a nonzero rational
function on $W$. For every subvariety $Z \subset W$ of dimension $k$
one can define the order of vanishing $\text{ord}_{W, Z}(f)$ of $f$ at
$Z$. If $f$ is an element of the local ring $\mathcal{O}_{W, Z}$,
then one has
$$
\text{ord}_{W, Z}(f) =
\text{length}_{\mathcal{O}_{X, z}} \mathcal{O}_{W, Z}/f\mathcal{O}_{W, Z}
$$
where $\mathcal{O}_{X, Z}$, resp.\ $\mathcal{O}_{W, Z}$ is the
local ring of $X$, resp.\ $W$ at the generic point of $Z$. In general one
extends the definition by multiplicativity. The {\it principal divisor
associated to $f$} is
$$
\text{div}_W(f) = \sum \text{ord}_{W, Z}(f)[Z]
$$
in $Z_k(W)$. Since $W \subset X$ is a closed subvariety we may think
of $\text{div}_W(f)$ as a cycle on $X$.
See Chow Homology, Section \ref{chow-section-principal-divisors}.

\begin{lemma}
\label{lemma-rational-equivalence}
Let $X$ be a variety. Let $W \subset X$ be a subvariety
of dimension $k + 1$. Let $f \in \mathbf{C}(W)^*$ be a nonzero rational
function on $W$. Then $\text{div}_W(f)$ is rationally equivalent to zero on
$X$. Conversely, these principal divisors generate the abelian group of
cycles rationally equivalent to zero on $X$.
\end{lemma}

\begin{proof}
The first assertion follows from
Chow Homology, Lemma \ref{chow-lemma-rational-function}.
More precisely, let $W' \subset X \times \mathbf{P}^1$ be the closure
of the graph of $f$. Then $\text{div}_W(f) = [W'_0]_k - [W'_\infty]$
in $Z_k(W) \subset Z_k(X)$, see part (6) of
Chow Homology, Lemma \ref{chow-lemma-rational-function}.

\medskip\noindent
For the second, let $W' \subset X \times \mathbf{P}^1$ be a closed
subvariety of dimension $k + 1$ which dominates $\mathbf{P}^1$.
We will show that $[W'_0]_k - [W'_\infty]_k$ is a principal divisor
which will finish the proof. Let $W \subset X$ be the image of $W'$
under the projection to $X$. Then $W' \to W$ is proper and generically
finite\footnote{If $W' \to W$ is birational, then the result follows
from Chow Homology, Lemma \ref{chow-lemma-rational-function}.
Our task is to show that even if $W' \to W$
has degree $>1$ the basic rational equivalence
$[W'_0]_k \sim_{rat} [W'_\infty]_k$ comes from a principal divisor
on a subvariety of $X$.}. Let $f$ denote the projection $W' \to \mathbf{P}^1$
viewed as an element of $\mathbf{C}(W')^*$. Let
$g = \text{Nm}(f) \in \mathbf{C}(W)^*$ be the norm. By
Chow Homology, Lemma \ref{chow-lemma-proper-pushforward-alteration}
we have
$$
\text{div}_W(g) = \text{pr}_{X, *}\text{div}_{W'}(f)
$$
Since it is clear that 
$\text{div}_{W'}(f) = [W'_0]_k - [W'_\infty]_k$ the proof is complete.
\end{proof}


\section{Proper pushforward and rational equivalence}
\label{section-pushforward-and-rational-equivalence}

\noindent
Suppose that $f : X \to Y$ is a proper morphism of varieties.
Let $\alpha \sim_{rat} 0$ be a $k$-cycle on
$X$ rationally equivalent to $0$. Then the {pushforward}
of $\alpha$ is rationally equivalent to zero:
$f_* \alpha \sim_{rat} 0$. See Chapter I of \cite{F} or
Chow Homology, Lemma \ref{chow-lemma-proper-pushforward-rational-equivalence}.

\medskip\noindent
Therefore we obtain a commutative diagram
$$
\xymatrix{
Z_k(X) \ar[r] \ar[d]_{f_*} & A_k(X) \ar[d]^{f_*} \\
Z_k(Y) \ar[r] & A_k(Y)
}
$$
of groups of $k$-cycles.


\section{Flat pullback and rational equivalence}
\label{section-flat-pullback-and-rational-equivalence}

\noindent
Suppose that $f : X \to Y$ is a flat morphism of varieties.
Set $r = \dim(X) - \dim(Y)$.
Let $\alpha \sim_{rat} 0$ be a $k$-cycle on
$Y$ rationally equivalent to $0$. Then the pullback
of $\alpha$ is rationally equivalent to zero:
$f^* \alpha \sim_{rat} 0$. See Chapter I of \cite{F} or
Chow Homology, Lemma \ref{chow-lemma-flat-pullback-rational-equivalence}.

\medskip\noindent
Therefore we obtain a commutative diagram
$$
\xymatrix{
Z_{k + r}(X) \ar[r] & A_{k + r}(X) \\
Z_k(Y) \ar[r] \ar[u]^{f^*} & A_k(Y) \ar[u]_{f^*}
}
$$
of groups of $k$-cycles.


\section{The short exact sequence for an open}
\label{section-ses}

\noindent
Let $X$ be a variety and let $U \subset X$ be an open subvariety.
Let $X \setminus U = \bigcup Z_i$ be the decomposition into irreducible
components\footnote{Since in this chapter we only consider Chow groups
of varieties, we are prohibited from taking $Z_k(X \setminus U)$
and $A_k(X \setminus U)$, hence the approach using the varieties $Z_i$.}.
Then for each $k \geq 0$ there exists a commutative diagram
$$
\xymatrix{
\bigoplus Z_k(Z_i) \ar[r] \ar[d] &
Z_k(X) \ar[r] \ar[d] &
Z_k(U) \ar[d] \ar[r] &
0 \\
\bigoplus A_k(Z_i) \ar[r] &
A_k(X) \ar[r] &
A_k(U) \ar[r] &
0
}
$$
with exact rows. Here the vertical arrows are the canonical quotient maps.
The left horizontal arrows are given by proper pushforward along the closed
immersions $Z_i \to X$. The right horizontal arrows are given by flat
pullback along the open immersion $j : U \to X$. Since we have seen that
these maps factor through rational equivalence we obtain the commutativity
of the squares. The top row is exact simply because every subvariety
of $X$ is either contained in some $Z_i$ or has irreducible intersection
with $U$. The bottom row is exact because every principal divisor
$\text{div}_W(f)$ on $U$ is the restriction of a principal divisor on $X$.
More precisely, if $W \subset U$ is a $(k + 1)$-dimensional closed subvariety
and $f \in \mathbf{C}(W)^*$, then denote $\overline{W}$ the closure of $W$
in $X$. Then $W \subset \overline{W}$ is an open immersion, so
$\mathbf{C}(W) = \mathbf{C}(\overline{W})$ and we may think of $f$
as a nonconstant rational function on $\overline{W}$. Then clearly
$$
j^*\text{div}_{\overline{W}}(f) = \text{div}_W(f)
$$
in $Z_k(X)$. The exactness of the lower row follows easily from this.
For details see Chow Homology, Lemma \ref{chow-lemma-restrict-to-open}.


\section{Proper intersections}
\label{section-intersect-properly}

\noindent
First a few lemmas to get dimension estimates.

\begin{lemma}
\label{lemma-dimension-product-varieties}
Let $X$ and $Y$ be varieties. Then $X \times Y$ is a variety and
$\dim(X \times Y) = \dim(X) + \dim(Y)$.
\end{lemma}

\begin{proof}
The scheme $X \times Y = X \times_{\Spec(\mathbf{C})} Y$ is a variety by
Varieties, Lemma \ref{varieties-lemma-product-varieties}.
The statement on dimension is
Varieties, Lemma \ref{varieties-lemma-dimension-product-locally-algebraic}.
\end{proof}

\noindent
Recall that a regular immersion $i : X \to Y$ of schemes
is a closed immersion whose
corresponding sheaf of ideals is locally generated by a regular sequence, see
Divisors, Section \ref{divisors-section-regular-immersions}.
Moreover, the conormal sheaf $\mathcal{C}_{X/Y}$ is finite locally free of
rank equal to the length of the regular sequence. Let us say $i$ is a
{\it regular immersion of codimension $c$}
if $\mathcal{C}_{X/Y}$ is locally free of rank $c$.

\medskip\noindent
More generally, recall
(More on Morphisms, Section \ref{more-morphisms-section-lci})
that $f : X \to Y$ is a local complete intersection
morphism if we can cover $X$ by opens $U$ such that we can factor
$f|_U$ as
$$
\xymatrix{
U \ar[rr]_i \ar[rd] & & \mathbf{A}^n_Y \ar[ld] \\
& Y
}
$$
where $i$ is a Koszul regular immersion (if $Y$ is locally Noetherian
this is the same as asking $i$ to be a regular immersion, see
Divisors, Lemma \ref{divisors-lemma-regular-immersion-noetherian}).
Let us say that $f$ is a {\it local complete intersection morphism
of relative dimension $r$} if for any factorization as above, the
closed immersion $i$ has conormal sheaf of rank $n - r$ (in other
words if $i$ is a Koszul-regular immersion of codimension $n - r$
which in the Noetherian case just means it is regular immersion of
codimension $n - r$).

\begin{lemma}
\label{lemma-pullback-by-regular-immersion}
Let $f : X \to Y$ be a morphism of varieties.
\begin{enumerate}
\item If $Z \subset Y$ is a subvariety dimension $d$ and $f$ is a regular
immersion of codimension $c$, then every irreducible component
of $f^{-1}(Z)$ has dimension $\geq d - c$.
\item If $Z \subset Y$ is a subvariety of dimension $d$ and
$f$ is a local complete intersection morphism of relative dimension $r$,
then every irreducible component of $f^{-1}(Z)$ has dimension $\geq d + r$.
\end{enumerate}
\end{lemma}

\begin{proof}
Proof of (1). We may work locally, hence we may assume that
$Y = \Spec(A)$ and $X = V(f_1, \ldots, f_c)$ where $f_1, \ldots, f_c$
is a regular sequence in $A$. If $Z = \Spec(A/\mathfrak p)$, then
we see that $f^{-1}(Z) = \Spec(A/\mathfrak p + (f_1, \ldots, f_c))$.
If $V$ is an irreducible component of $f^{-1}(Z)$, then we can
choose a closed point $v \in V$ not contained in any other irreducible
component of $f^{-1}(Z)$. Then
$$
\dim(Z) = \dim \mathcal{O}_{Z, v}
\quad\text{and}\quad
\dim(V) = \dim \mathcal{O}_{V, v} = \dim \mathcal{O}_{Z, v}/(f_1, \ldots, f_c)
$$
The first equality for example by
Algebra, Lemma \ref{algebra-lemma-dimension-prime-polynomial-ring}
and the second equality by our choice of closed point.
The result now follows from the fact that dividing by one element
in the maximal ideal decreases the dimension by at most $1$, see
Algebra, Lemma \ref{algebra-lemma-one-equation}.

\medskip\noindent
Proof of (2). Choose a factorization as in the definition of a
local complete intersection and apply (1). Some details omitted.
\end{proof}

\begin{lemma}
\label{lemma-diagonal-regular-immersion}
Let $X$ be a nonsingular variety. Then the diagonal
$\Delta : X \to X \times X$ is a regular immersion of codimension $\dim(X)$.
\end{lemma}

\begin{proof}
In fact, any closed immersion between nonsingular projective
varieties is a regular immersion, see Divisors,
Lemma \ref{divisors-lemma-immersion-smooth-into-smooth-regular-immersion}.
\end{proof}

\noindent
The following lemma demonstrates how reduction to the diagonal works.

\begin{lemma}
\label{lemma-intersect-in-smooth}
Let $X$ be a nonsingular variety and let $W,V \subset X$
be closed subvarieties with $\dim(W) = s$ and $\dim(V) = r$. Then every
irreducible component $Z$ of $V \cap W$ has dimension $\geq r + s - \dim(X)$.
\end{lemma}

\begin{proof}
Since $V \cap W = \Delta^{-1}(V \times W)$ (scheme theoretically)
we conclude by Lemmas \ref{lemma-diagonal-regular-immersion} and
\ref{lemma-pullback-by-regular-immersion}.
\end{proof}

\noindent
This lemma suggests the following definition.

\begin{definition}
\label{definition-proper-intersection}
Let $X$ be a nonsingular variety.
\begin{enumerate}
\item Let $W,V \subset X$ be closed subvarieties with
$\dim(W) = s$ and $\dim(V) = r$. We say that $W$ and $V$
{\it intersect properly} if $\dim(V \cap W) \leq r + s - \dim(X)$.
\item Let $\alpha = \sum n_i [W_i]$ be an $s$-cycle,
and $\beta = \sum_j m_j [V_j]$ be an $r$-cycle on $X$. We say
that $\alpha$ and $\beta$ {\it intersect properly} if
$W_i$ and $V_j$ intersect properly for all $i$ and $j$.
\end{enumerate}
\end{definition}


\section{Intersection multiplicities using Tor formula}
\label{section-tor-formula}

\noindent
A basic fact we will use frequently is that given sheaves of
modules $\mathcal{F}$, $\mathcal{G}$ on a ringed space $(X, \mathcal{O}_X)$
and a point $x \in X$ we have
$$
\text{Tor}_p^{\mathcal{O}_X}(\mathcal{F}, \mathcal{G})_x =
\text{Tor}_p^{\mathcal{O}_{X, x}}(\mathcal{F}_x, \mathcal{G}_x)
$$
as $\mathcal{O}_{X, x}$-modules. This can be seen in several ways
from our construction of derived tensor products in
Cohomology, Section \ref{cohomology-section-flat}, for example it follows from
Cohomology, Lemma \ref{cohomology-lemma-check-K-flat-stalks}.
Moreover, if $X$ is a scheme and $\mathcal{F}$ and $\mathcal{G}$
are quasi-coherent, then the modules
$\text{Tor}_p^{\mathcal{O}_X}(\mathcal{F}, \mathcal{G})$ are
quasi-coherent too, see
Derived Categories of Schemes, Lemma
\ref{perfect-lemma-quasi-coherence-tensor-product}.
More important for our purposes is the following result.

\begin{lemma}
\label{lemma-tensor-coherent}
Let $X$ be a locally Noetherian scheme.
\begin{enumerate}
\item If $\mathcal{F}$ and $\mathcal{G}$ are coherent $\mathcal{O}_X$-modules,
then $\text{Tor}_p^{\mathcal{O}_X}(\mathcal{F}, \mathcal{G})$ is too.
\item If $L$ and $K$ are in $D^-_{\textit{Coh}}(\mathcal{O}_X)$, then
so is $L \otimes_{\mathcal{O}_X}^\mathbf{L} K$.
\end{enumerate}
\end{lemma}

\begin{proof}
Let us explain how to prove (1) in a more elementary way and part (2)
using previously developed general theory.

\medskip\noindent
Proof of (1). Since formation of $\text{Tor}$ commutes with localization
we may assume $X$ is affine. Hence $X = \Spec(A)$ for some Noetherian
ring $A$ and $\mathcal{F}$, $\mathcal{G}$ correspond to finite $A$-modules
$M$ and $N$ (Cohomology of Schemes, Lemma
\ref{coherent-lemma-coherent-Noetherian}).
By Derived Categories of Schemes, Lemma
\ref{perfect-lemma-quasi-coherence-tensor-product} we may
compute the $\text{Tor}$'s by first computing the $\text{Tor}$'s
of $M$ and $N$ over $A$, and then taking the associated $\mathcal{O}_X$-module.
Since the modules $\text{Tor}_p^A(M, N)$ are finite by
Algebra, Lemma \ref{algebra-lemma-tor-noetherian}
we conclude.

\medskip\noindent
By Derived Categories of Schemes, Lemma
\ref{perfect-lemma-identify-pseudo-coherent-noetherian}
the assumption is equivalent to asking $L$ and $K$ to be
(locally) pseudo-coherent. Then $L \otimes_{\mathcal{O}_X}^\mathbf{L} K$
is pseudo-coherent by
Cohomology, Lemma \ref{cohomology-lemma-tensor-pseudo-coherent}.
\end{proof}

\begin{lemma}
\label{lemma-compute-tor-nonsingular}
Let $X$ be a nonsingular variety.
Let $\mathcal{F}$, $\mathcal{G}$ be coherent $\mathcal{O}_X$-modules.
The $\mathcal{O}_X$-module
$\text{Tor}_p^{\mathcal{O}_X}(\mathcal{F}, \mathcal{G})$
is coherent, has stalk at $x$ equal to
$\text{Tor}_p^{\mathcal{O}_{X, x}}(\mathcal{F}_x, \mathcal{G}_x)$,
is supported on
$\text{Supp}(\mathcal{F}) \cap \text{Supp}(\mathcal{G})$, and
is nonzero only for $p \in \{0, \ldots, \dim(X)\}$.
\end{lemma}

\begin{proof}
The result on stalks was discussed above and it implies the support
condition. The $\text{Tor}$'s are coherent by
Lemma \ref{lemma-tensor-coherent}. The vanishing of negative
$\text{Tor}$'s is immediate from the construction. The
vanishing of $\text{Tor}_p$ for $p > \dim(X)$ can be seen as follows:
the local rings $\mathcal{O}_{X, x}$ are regular
(as $X$ is nonsingular) of dimension $\leq \dim(X)$
(Algebra, Lemma \ref{algebra-lemma-dimension-prime-polynomial-ring}),
hence $\mathcal{O}_{X, x}$ has finite global dimension $\leq \dim(X)$
(Algebra, Lemma \ref{algebra-lemma-finite-gl-dim-finite-dim-regular})
which implies that $\text{Tor}$-groups of modules vanish beyond the dimension
(More on Algebra, Lemma \ref{more-algebra-lemma-finite-gl-dim-tor-dimension}).
\end{proof}

\noindent
Let $X$ be a nonsingular variety and $W, V \subset X$
be closed subvarieties with $\dim(W) = s$ and $\dim(V) = r$.
Assume $V$ and $W$ intersect properly.
In this case Lemma \ref{lemma-intersect-in-smooth} tells us all irreducible
components of $V \cap W$ have dimension equal to $r + s - \dim(X)$.
The sheaves $\text{Tor}_j^{\mathcal{O}_X}(\mathcal{O}_W, \mathcal{O}_V)$ are
coherent, supported on $V \cap W$, and zero if $j < 0$ or $j > \dim(X)$
(Lemma \ref{lemma-compute-tor-nonsingular}).
We define the {\it intersection product} as
$$
W \cdot V = \sum\nolimits_i (-1)^i
[\text{Tor}_i^{\mathcal{O}_X}(\mathcal{O}_W, \mathcal{O}_V)]_{r + s - \dim(X)}.
$$
We stress that this makes sense only because of our assumption that
$V$ and $W$ intersect properly. This fact will necessitate a moving
lemma in order to define the intersection product in general.

\medskip\noindent
With this notation, the cycle $V \cdot W$ is a formal linear
combination $\sum e_Z Z$ of the irreducible components $Z$
of the intersection $V \cap W$. The integers $e_Z$ are called
the {\it intersection multiplicities}
$$
e_Z = e(X, V \cdot W, Z) =
\sum\nolimits_i
(-1)^i
\text{length}_{\mathcal{O}_{X, Z}}
\text{Tor}_i^{\mathcal{O}_{X, Z}}(\mathcal{O}_{W, Z}, \mathcal{O}_{V, Z})
$$
where $\mathcal{O}_{X, Z}$, resp.\ $\mathcal{O}_{W, Z}$,
resp.\ $\mathcal{O}_{V, Z}$ denotes the local ring of $X$, resp.\ $W$,
resp.\ $V$ at the generic point of $Z$.
These alternating sums of lengths of $\text{Tor}$'s satisfy many good
properties, as we will see later on.

\medskip\noindent
In the case of transversal intersections, the intersection number is $1$.

\begin{lemma}
\label{lemma-transversal}
Let $X$ be a nonsingular variety. Let $V, W \subset X$ be
closed subvarieties which intersect properly. Let $Z$ be an irreducible
component of $V \cap W$ and assume that the multiplicity
(in the sense of Section \ref{section-cycle-of-closed}) of $Z$
in the closed subscheme $V \cap W$ is $1$.
Then $e(X, V \cdot W, Z) = 1$ and $V$ and $W$ are smooth
in a general point of $Z$.
\end{lemma}

\begin{proof}
Let $(A, \mathfrak m, \kappa) =
(\mathcal{O}_{X, \xi}, \mathfrak m_\xi, \kappa(\xi))$ where $\xi \in Z$
is the generic point. Then $\dim(A) = \dim(X) - \dim(Z)$, see
Varieties, Lemma \ref{varieties-lemma-dimension-locally-algebraic}.
Let $I, J \subset A$ cut out the trace of $V$ and $W$
in $\Spec(A)$. Set $\overline{I} = I + \mathfrak m^2/\mathfrak m^2$.
Then $\dim_\kappa \overline{I} \leq \dim(X) - \dim(V)$ with equality
if and only if $A/I$ is regular (this follows from the lemma cited
above and the definition of regular rings, see
Algebra, Definition \ref{algebra-definition-regular-local}
and the discussion preceding it). Similarly for $\overline{J}$.
If the multiplicity is $1$, then
$\text{length}_A(A/I + J) = 1$, hence $I + J = \mathfrak m$, hence
$\overline{I} + \overline{J} = \mathfrak m/\mathfrak m^2$.
Then we get equality everywhere (because the intersection is
proper). Hence we find $f_1, \ldots, f_a \in I$ and $g_1, \ldots g_b \in J$
such that $\overline{f}_1, \ldots, \overline{g}_b$ is a basis
for $\mathfrak m/\mathfrak m^2$. Then $f_1, \ldots, g_b$ is a
regular system of parameters and a regular sequence
(Algebra, Lemma \ref{algebra-lemma-regular-ring-CM}).
The same lemma shows $A/(f_1, \ldots, f_a)$ is a regular local ring
of dimension $\dim(X) - \dim(V)$, hence $A/(f_1, \ldots, f_a) \to A/I$
is an isomorphism (if the kernel is nonzero, then the dimension
of $A/I$ is strictly less, see
Algebra, Lemmas \ref{algebra-lemma-regular-domain} and
\ref{algebra-lemma-one-equation}).
We conclude $I = (f_1, \ldots, f_a)$ and $J = (g_1, \ldots, g_b)$
by symmetry. Thus the Koszul complex $K_\bullet(A, f_1, \ldots, f_a)$
on $f_1, \ldots, f_a$ is a resolution of $A/I$, see
More on Algebra, Lemma \ref{more-algebra-lemma-regular-koszul-regular}.
Hence
\begin{align*}
\text{Tor}_p^A(A/I, A/J)
& =
H_p(K_\bullet(A, f_1, \ldots, f_a) \otimes_A A/J) \\
& =
H_p(K_\bullet(A/J, f_1 \bmod J, \ldots, f_a \bmod J))
\end{align*}
Since we've seen above that $f_1 \bmod J, \ldots, f_a \bmod J$ is
a regular system of parameters in the regular local ring $A/J$
we conclude that there is only one cohomology group, namely
$H_0 = A/(I + J) = \kappa$. This finishes the proof.
\end{proof}

\begin{example}
\label{example-naive-multiplicity-wrong}
In this example we show that it is necessary to use the higher
tors in the formula for the intersection multiplicities above.
Let $X$ be a nonsingular variety of dimension $4$.
Let $p \in X$ be a closed point. Let $V, W \subset X$
be closed subvarieties in $X$. Assume that there is an
isomorphism
$$
\mathcal{O}_{X, p}^\wedge \cong \mathbf{C}[[x, y, z, w]]
$$
such that the ideal of $V$ is $(xz, xw, yz, yw)$ and the ideal
of $W$ is $(x - z, y - w)$. Then a computation shows that
$$
\text{length}\ \mathbf{C}[[x, y, z, w]]/
(xz, xw, yz, yw, x - z, y - w) = 3
$$
On the other hand, the multiplicity $e(X, V \cdot W, p) = 2$
as can be seen from the fact that formal locally $V$ is the
union of two smooth planes $x = y = 0$ and $z = w = 0$ at $p$,
each of which has intersection multiplicity $1$ with the plane
$x - z = y - w = 0$ (Lemma \ref{lemma-transversal}). To make an
actual example, take
a general morphism $f : \mathbf{P}^2 \to \mathbf{P}^4$ given by
$5$ homogeneous polynomials of degree $> 1$. The image
$V \subset \mathbf{P}^4 = X$ will have singularities of the type
described above, because there will be $p_1, p_2 \in \mathbf{P}^2$
with $f(p_1) = f(p_2)$. To find $W$ take a general plane passing
through such a point.
\end{example}






\section{Algebraic multiplicities}
\label{section-multiplicities}

\noindent
Let $(A, \mathfrak m, \kappa)$ be a Noetherian local ring.
Let $M$ be a finite $A$-module and let $I \subset A$ be an ideal
of definition (Algebra, Definition \ref{algebra-definition-ideal-definition}).
Recall that the function
$$
\chi_{I, M}(n) = \text{length}_A(M/I^nM) =
\sum\nolimits_{p = 0, \ldots, n - 1} \text{length}_A(I^pM/I^{p + 1}M)
$$
is a numerical polynomial
(Algebra, Proposition \ref{algebra-proposition-hilbert-function-polynomial}).
The degree of this polynomial is equal to $\dim(\text{Supp}(M))$ by
Algebra, Lemma \ref{algebra-lemma-support-dimension-d}.

\begin{definition}
\label{definition-multiplicity}
In the situation above, if $d \geq \dim(\text{Supp}(M))$, then we set
$e_I(M, d)$ equal to $0$ if $d > \dim(\text{Supp}(M))$
and equal to $d!$ times the
leading coefficient of the numerical polynomial $\chi_{I, M}$ so that
$$
\chi_{I, M}(n) \sim e_I(M, d) \frac{n^d}{d!} + \text{lower order terms}
$$
The {\it multiplicity of $M$ for the ideal of definition $I$}
is $e_I(M) = e_I(M, \dim(\text{Supp}(M)))$.
\end{definition}

\noindent
We have the following properties of these multiplicities.

\begin{lemma}
\label{lemma-multiplicity-ses}
Let $A$ be a Noetherian local ring. Let $I \subset A$ be an ideal of
definition. Let $0 \to M' \to M \to M'' \to 0$ be a short exact sequence
of finite $A$-modules. Let $d \geq \dim(\text{Supp}(M))$. Then
$$
e_I(M, d) = e_I(M', d) + e_I(M'', d)
$$
\end{lemma}

\begin{proof}
Immediate from the definitions and
Algebra, Lemma \ref{algebra-lemma-hilbert-ses-chi}.
\end{proof}

\begin{lemma}
\label{lemma-multiplicity-as-a-sum}
Let $A$ be a Noetherian local ring. Let $I \subset A$ be an ideal of
definition. Let $M$ be a finite $A$-module. Let $d \geq \dim(\text{Supp}(M))$.
Then
$$
e_I(M, d) =
\sum \text{length}_{A_\mathfrak p}(M_\mathfrak p) e_I(A/\mathfrak p, d)
$$
where the sum is over primes $\mathfrak p \subset A$ with
$\dim(A/\mathfrak p) = d$.
\end{lemma}

\begin{proof}
Both the left and side and the right hand side are additive in short
exact sequences of modules of dimension $\leq d$, see
Lemma \ref{lemma-multiplicity-ses} and
Algebra, Lemma \ref{algebra-lemma-length-additive}.
Hence by Algebra, Lemma \ref{algebra-lemma-filter-Noetherian-module}
it suffices to prove this when $M = A/\mathfrak q$ for some
prime $\mathfrak q$ of $A$ with $\dim(A/\mathfrak q) \leq d$.
This case is obvious.
\end{proof}

\begin{lemma}
\label{lemma-leading-coefficient}
Let $P$ be a polynomial of degree $r$ with leading coefficient $a$.
Then
$$
r! a = \sum\nolimits_{i = 0, \ldots, r} (-1)^i{r \choose i} P(t - i)
$$
for any $t$.
\end{lemma}

\begin{proof}
Let us write $\Delta$ the operator which to a polynomial $P$ associates
the polynomial $\Delta(P) = P(t) - P(t - 1)$. We claim that
$$
\Delta^r(P) = \sum\nolimits_{i = 0, \ldots, r} (-1)^i {r \choose i} P(t - i)
$$
This is true for $r = 0, 1$ by inspection. Assume it is true for $r$.
Then we compute
\begin{align*}
\Delta^{r + 1}(P)
& =
\sum\nolimits_{i = 0, \ldots, r} (-1)^i {r \choose i} \Delta(P)(t - i) \\
& =
\sum\nolimits_{n = -r, \ldots, 0} (-1)^i {r \choose i}
(P(t - i) - P(t - i - 1))
\end{align*}
Thus the claim follows from the equality
$$
{r + 1 \choose i} = {r \choose i} + {r \choose i - 1}
$$
The lemma follows from the fact that $\Delta(P)$ is of degree $r - 1$
with leading coefficient $ra$ if the degree of $P$ is $r$.
\end{proof}

\noindent
An important fact is that one can compute the multiplicity in terms
of the Koszul complex. Recall that if $R$ is a ring and
$f_1, \ldots, f_r \in R$, then $K_\bullet(f_1, \ldots, f_r)$
denotes the Koszul complex, see
More on Algebra, Section \ref{more-algebra-section-koszul}.

\begin{theorem}
\label{theorem-multiplicity-with-koszul}
\begin{reference}
\cite[Theorem 1 in part B of Chapter IV]{Serre_algebre_locale}
\end{reference}
Let $A$ be a Noetherian local ring. Let $I = (f_1, \ldots, f_r) \subset A$
be an ideal of definition. Let $M$ be a finite $A$-module. Then
$$
e_I(M, r) = \sum
(-1)^i\text{length}_A H_i(K_\bullet(f_1, \ldots, f_r) \otimes_A M)
$$
\end{theorem}

\begin{proof}
Let us change the Koszul complex $K_\bullet(f_1, \ldots, f_r)$ into a cochain
complex $K^\bullet$ by setting $K^n = K_{-n}(f_1, \ldots, f_r)$.
Then $K^\bullet$ is sitting in degrees $-r, \ldots, 0$ and
$H^i(K^\bullet \otimes_A M) = H_{-i}(K_\bullet(f_1, \ldots, f_r) \otimes_A M)$.
The statement of the theorem makes sense as the modules
$H^i(K^\bullet \otimes M)$ are annihilated by $f_1, \ldots, f_r$
(More on Algebra, Lemma \ref{more-algebra-lemma-homotopy-koszul})
hence have finite length.
Define a filtration on the complex $K^\bullet$ by setting
$$
F^p(K^n \otimes_A M) =
I^{\max(0, p + n)}(K^n \otimes_A M),\quad p \in \mathbf{Z}
$$
Since $f_i I^p \subset I^{p + 1}$ this is a filtration by subcomplexes.
Thus we have a filtered complex and we obtain a spectral sequence, see
Homology, Section \ref{homology-section-filtered-complex}.
We have
$$
E_0 = \bigoplus\nolimits_{p, q} E_0^{p, q} =
\bigoplus\nolimits_{p, q} \text{gr}^p(K^{p + q} \otimes_A M) =
\text{Gr}_I(K^\bullet \otimes_A M)
$$
Since $K^n$ is finite free we have
$$
\text{Gr}_I(K^\bullet \otimes_A M) =
\text{Gr}_I(K^\bullet) \otimes_{\text{Gr}_I(A)} \text{Gr}_I(M)
$$
Note that $\text{Gr}_I(K^\bullet)$ is the Koszul
complex over $\text{Gr}_I(A)$ on the elements
$\overline{f}_1, \ldots, \overline{f}_r \in I/I^2$.
A simple calculation (omitted)
shows that the differential $d_0$ on $E_0$
agrees with the differential coming from the Koszul complex.
Since $\text{Gr}_I(M)$ is a finite $\text{Gr}_I(A)$-module
and since $\text{Gr}_I(A)$ is Noetherian (as a quotient
of $A/I[x_1, \ldots, x_r]$ with $x_i \mapsto \overline{f}_i$), the
cohomology module $E_1 = \bigoplus E_1^{p, q}$
is a finite $\text{Gr}_I(A)$-module. However, as above
$E_1$ is annihilated by $\overline{f}_1, \ldots, \overline{f}_r$.
We conclude $E_1$ has finite length.
In particular we find that $\text{Gr}^p_F(K^\bullet \otimes M)$ is
acyclic for $p \gg 0$.

\medskip\noindent
Next, we check that the spectral sequence above converges
using Homology, Lemma \ref{homology-lemma-filtered-complex-ss-converges}.
The required equalities follow easily from the Artin-Rees lemma
in the form stated in Algebra, Lemma \ref{algebra-lemma-map-AR}.
Thus we see that
\begin{align*}
\sum (-1)^i\text{length}_A(H^i(K^\bullet \otimes_A M))
& =
\sum (-1)^{p + q} \text{length}_A(E_\infty^{p, q}) \\
& =
\sum (-1)^{p + q} \text{length}_A(E_1^{p, q})
\end{align*}
because as we've seen above the length of $E_1$ is finite
(of course this uses additivity of lengths). Pick $t$ so
large that $\text{Gr}^p_F(K^\bullet \otimes M)$
is acyclic for $p \geq t$ (see above). Using
additivity again we see that
$$
\sum (-1)^{p + q} \text{length}_A(E_1^{p, q}) =
\sum\nolimits_n \sum\nolimits_{p \leq t}
(-1)^n \text{length}_A(\text{gr}^p(K^n \otimes_A M))
$$
This is equal to
$$
\sum\nolimits_{n = -r, \ldots, 0} (-1)^n{r \choose |n|} \chi_{I, M}(t + n)
$$
by our choice of filtration above and the definition of $\chi_{I, M}$ in
Algebra, Section \ref{algebra-section-Noetherian-local}.
The lemma follows from Lemma \ref{lemma-leading-coefficient}
and the definition of $e_I(M, r)$.
\end{proof}

\begin{remark}[Trivial generalization]
\label{remark-trivial-generalization}
Let $(A, \mathfrak m, \kappa)$ be a Noetherian local ring.
Let $M$ be a finite $A$-module. Let $I \subset A$ be an ideal.
The following are equivalent
\begin{enumerate}
\item $I' = I + \text{Ann}(M)$ is an ideal of definition
(Algebra, Definition \ref{algebra-definition-ideal-definition}),
\item the image $\overline{I}$ of $I$ in $\overline{A} = A/\text{Ann}(M)$
is an ideal of definition,
\item $\text{Supp}(M/IM) \subset \{\mathfrak m\}$,
\item $\dim(\text{Supp}(M/IM)) \leq 0$, and
\item $\text{length}_A(M/IM) < \infty$.
\end{enumerate}
This follows from Algebra, Lemma \ref{algebra-lemma-support-point}
(details omitted). If this is the case we have $M/I^nM = M/(I')^nM$
for all $n$ and $M/I^nM = M/\overline{I}^nM$ for all $n$
if $M$ is viewed as an $\overline{A}$-module.
Thus we can define
$$
\chi_{I, M}(n) = \text{length}_A(M/I^nM) =
\sum\nolimits_{p = 0, \ldots, n - 1} \text{length}_A(I^pM/I^{p + 1}M)
$$
and we get
$$
\chi_{I, M}(n) = \chi_{I', M}(n) = \chi_{\overline{I}, M}(n)
$$
for all $n$ by the equalities above.
All the results of Algebra, Section \ref{algebra-section-Noetherian-local}
and all the results in this section, have analogues in this setting.
In particular we can define multiplicities $e_I(M, d)$ for
$d \geq \dim(\text{Supp}(M))$ and we have
$$
\chi_{I, M}(n) \sim e_I(M, d) \frac{n^d}{d!} + \text{lower order terms}
$$
as in the case where $I$ is an ideal of definition.
\end{remark}



\section{Computing intersection multiplicities}
\label{section-computing-intersection-multiplicities}

\noindent
In this section we discuss some cases where the intersection multiplicities
can be computed by different means. Here is a first example.

\begin{lemma}
\label{lemma-intersection-multiplicity-CM}
Let $X$ be a nonsingular variety and $W, V \subset X$ closed
subvarieties which intersect properly. Let $Z$ be an irreducible component
of $V \cap W$ with generic point $\xi$. Assume that $\mathcal{O}_{W, \xi}$
and $\mathcal{O}_{V, \xi}$ are Cohen-Macaulay. Then
$$
e(X, V \cdot W, Z) =
\text{length}_{\mathcal{O}_{X, \xi}}(\mathcal{O}_{V \cap W, \xi})
$$
where $V \cap W$ is the scheme theoretic intersection.
In particular, if both $V$ and $W$ are Cohen-Macaulay, then
$V \cdot W = [V \cap W]_{\dim(V) + \dim(W) - \dim(X)}$.
\end{lemma}

\begin{proof}
Set $A = \mathcal{O}_{X, \xi}$, $B = \mathcal{O}_{V, \xi}$, and
$C = \mathcal{O}_{W, \xi}$. By Auslander-Buchsbaum
(Algebra, Proposition \ref{algebra-proposition-Auslander-Buchsbaum})
we can find a finite free resolution $F_\bullet \to B$ of
length
$$
\text{depth}(A) - \text{depth}(B) =
\dim(A) - \dim(B) = \dim(C)
$$
First equality as $A$ and $B$ are Cohen-Macaulay and the second
as $V$ and $W$ intersect properly. Then $F_\bullet \otimes_A C$ is a
complex of finite free modules representing $B \otimes_A^\mathbf{L} C$
hence has cohomology modules with support in $\{\mathfrak m_A\}$.
By the Acyclicity lemma (Algebra, Lemma \ref{algebra-lemma-acyclic})
which applies as $C$ is Cohen-Macaulay
we conclude that $F_\bullet \otimes_A C$ has nonzero
cohomology only in degree $0$. This finishes the proof.
\end{proof}

\begin{lemma}
\label{lemma-one-ideal-ci}
Let $A$ be a Noetherian local ring. Let $I = (f_1, \ldots, f_r)$ be an ideal
generated by a regular sequence. Let $M$ be a finite $A$-module. Assume that
$\dim(\text{Supp}(M/IM)) = 0$. Then
$$
e_I(M, r) = \sum (-1)^i\text{length}_A(\text{Tor}_i^A(A/I, M))
$$
Here $e_I(M, r)$ is as in Remark \ref{remark-trivial-generalization}.
\end{lemma}

\begin{proof}
Since $f_1, \ldots, f_r$ is a regular sequence the Koszul complex
$K_\bullet(f_1, \ldots, f_r)$ is a resolution of $A/I$ over $A$, see
More on Algebra, Lemma
\ref{more-algebra-lemma-noetherian-finite-all-equivalent}.
Thus the right hand side is equal to
$$
\sum (-1)^i\text{length}_A H_i(K_\bullet(f_1, \ldots, f_r) \otimes_A M)
$$
Now the result follows immediately from
Theorem \ref{theorem-multiplicity-with-koszul} if $I$ is an ideal
of definition. In general, we replace $A$ by $\overline{A} = A/\text{Ann}(M)$
and $f_1, \ldots, f_r$ by $\overline{f}_1, \ldots, \overline{f}_r$
which is allowed because
$$
K_\bullet(f_1, \ldots, f_r) \otimes_A M =
K_\bullet(\overline{f}_1, \ldots, \overline{f}_r) \otimes_{\overline{A}} M
$$
Since $e_I(M, r) = e_{\overline{I}}(M, r)$ where
$\overline{I} = (\overline{f}_1, \ldots, \overline{f}_r) \subset \overline{A}$
is an ideal of definition the result follows from
Theorem \ref{theorem-multiplicity-with-koszul} in this case as well.
\end{proof}

\begin{lemma}
\label{lemma-multiplicity-with-lci}
Let $X$ be a nonsingular variety. Let $W,V \subset X$ be
closed subvarieties which intersect properly. Let $Z$ be an irreducible
component of $V \cap W$ with generic point $\xi$.
Suppose the ideal of $V$ in $\mathcal{O}_{X, \xi}$ is cut out by
a regular sequence $f_1, \ldots, f_c \in \mathcal{O}_{X, \xi}$.
Then $e(X, V\cdot W, Z)$ is equal to $c!$ times the leading coefficient in
the Hilbert polynomial
$$
t \mapsto \text{length}_{\mathcal{O}_{X, \xi}}
\mathcal{O}_{W, \xi}/(f_1, \ldots, f_c)^t,\quad t \gg 0.
$$
In particular, this coefficient is $> 0$.
\end{lemma}

\begin{proof}
The equality
$$
e(X, V\cdot W, Z) = e_{(f_1, \ldots, f_c)}(\mathcal{O}_{W, \xi}, c)
$$
follows from the more general Lemma \ref{lemma-one-ideal-ci}.
To see that $e_{(f_1, \ldots, f_c)}(\mathcal{O}_{W, \xi}, c)$ is
$> 0$ or equivalently that $e_{(f_1, \ldots, f_c)}(\mathcal{O}_{W, \xi}, c)$
is the leading coefficient of the Hilbert polynomial
it suffices to show that the
dimension of $\mathcal{O}_{W, \xi}$ is $c$, because the degree of the
Hilbert polynomial is equal to the dimension by
Algebra, Proposition \ref{algebra-proposition-dimension}.
Say $\dim(V) = r$, $\dim(W) = s$, and $\dim(X) = n$. Then
$\dim(Z) = r + s - n$ as the intersection is proper. Thus
the transcendence degree of $\kappa(\xi)$ over $\mathbf{C}$ is
$r + s - n$, see Algebra, Lemma
\ref{algebra-lemma-dimension-prime-polynomial-ring}.
We have $r + c = n$ because $V$ is cut out by a regular sequence
in a neighbourhood of $\xi$, see
Divisors, Lemma \ref{divisors-lemma-Noetherian-scheme-regular-ideal}
and then Lemma \ref{lemma-pullback-by-regular-immersion}
applies (for example). Thus
$$
\dim(\mathcal{O}_{W, \xi}) = s - (r + s - n) = s - ((n - c) + s - n) = c
$$
the first equality by Algebra, Lemma
\ref{algebra-lemma-dimension-at-a-point-finite-type-field}.
\end{proof}

\begin{lemma}
\label{lemma-multiplicity-with-effective-Cartier-divisor}
In Lemma \ref{lemma-multiplicity-with-lci} assume that  $c = 1$, i.e., $V$
is an effective Cartier divisor. Then
$$
e(X, V \cdot W, Z) =
\text{length}_{\mathcal{O}_{X, \xi}}
(\mathcal{O}_{W, \xi}/f_1\mathcal{O}_{W, \xi}).
$$
\end{lemma}

\begin{proof}
In this case the image of $f_1$ in $\mathcal{O}_{W, \xi}$ is nonzero by
properness of intersection, hence a nonzerodivisor divisor. Moreover,
$\mathcal{O}_{W, \xi}$ is a Noetherian local domain of dimension $1$.
Thus
$$
\text{length}_{\mathcal{O}_{X, \xi}}
(\mathcal{O}_{W, \xi}/f_1^t\mathcal{O}_{W, \xi}) =
t \text{length}_{\mathcal{O}_{X, \xi}}
(\mathcal{O}_{W, \xi}/f_1\mathcal{O}_{W, \xi})
$$
for all $t \geq 1$, see Algebra, Lemma \ref{algebra-lemma-ord-additive}.
This proves the lemma.
\end{proof}

\begin{lemma}
\label{lemma-multiplicity-lci-CM}
In Lemma \ref{lemma-multiplicity-with-lci} assume that
the local ring $\mathcal{O}_{W, \xi}$ is Cohen-Macaulay. Then we
have
$$
e(X, V \cdot W, Z) =
\text{length}_{\mathcal{O}_{X, \xi}} (\mathcal{O}_{W, \xi}/
f_1\mathcal{O}_{W, \xi} + \ldots + f_c\mathcal{O}_{W, \xi}).
$$
\end{lemma}

\begin{proof}
This follows immediately from Lemma \ref{lemma-intersection-multiplicity-CM}.
Alternatively, we can deduce it from Lemma \ref{lemma-multiplicity-with-lci}.
Namely, by Algebra, Lemma \ref{algebra-lemma-reformulate-CM}
we see that $f_1, \ldots, f_c$ is a regular sequence in
$\mathcal{O}_{W, \xi}$. Then
Algebra, Lemma \ref{algebra-lemma-regular-quasi-regular} shows that
$f_1, \ldots, f_c$ is a quasi-regular sequence.
This easily implies the length of
$\mathcal{O}_{W, \xi}/(f_1, \ldots, f_c)^t$ is
$$
{c + t \choose c}
\text{length}_{\mathcal{O}_{X, \xi}} (\mathcal{O}_{W, \xi}/
f_1\mathcal{O}_{W, \xi} + \ldots + f_c\mathcal{O}_{W, \xi}).
$$
Looking at the leading coefficient we conclude.
\end{proof}


\section{Intersection product using Tor formula}
\label{section-intersection-product}

\noindent
Let $X$ be a nonsingular variety. Let
$\alpha = \sum n_i [W_i]$ be an $r$-cycle and
$\beta = \sum_j m_j [V_j]$ be an $s$-cycle on $X$.
Assume that $\alpha$ and $\beta$ intersect properly, see
Definition \ref{definition-proper-intersection}.
In this case we define
$$
\alpha \cdot \beta = \sum\nolimits_{i,j} n_i m_j W_i \cdot V_j.
$$
where $W_i \cdot V_j$ is as defined in Section \ref{section-tor-formula}.
If $\beta = [V]$ where $V$ is a closed subvariety of dimension $s$,
then we sometimes write $\alpha \cdot \beta = \alpha \cdot V$.

\begin{lemma}
\label{lemma-rational-equivalence-and-intersection}
Let $X$ be a nonsingular variety. Let $a, b \in \mathbf{P}^1$
be distinct closed points. Let $k \geq 0$.
\begin{enumerate}
\item If $W \subset X \times \mathbf{P}^1$ is a closed subvariety
of dimension $k + 1$ which intersects $X \times a$ properly, then
\begin{enumerate}
\item $[W_a]_k = W \cdot X \times a$ as cycles on $X \times \mathbf{P}^1$, and
\item $[W_a]_k = \text{pr}_{X, *}(W \cdot X \times a)$ as cycles on $X$.
\end{enumerate}
\item Let $\alpha$ be a $(k + 1)$-cycle on $X \times \mathbf{P}^1$
which intersects $X \times a$ and $X \times b$ properly. Then
$pr_{X,*}( \alpha \cdot X \times a - \alpha \cdot X \times b)$
is rationally equivalent to zero.
\item Conversely, any $k$-cycle which is
rationally equivalent to $0$ is of this form.
\end{enumerate}
\end{lemma}

\begin{proof}
First we observe that $X \times a$ is an effective Cartier divisor in
$X \times \mathbf{P}^1$ and that $W_a$ is the scheme theoretic intersection
of $W$ with $X \times a$. Hence the equality in (1)(a) is
immediate from the definitions and the calculation of intersection
multiplicity in case of a Cartier divisor given in
Lemma \ref{lemma-multiplicity-with-effective-Cartier-divisor}.
Part (1)(b) holds because $W_a \to X \times \mathbf{P}^1 \to X$ maps
isomorphically onto its image which is how we viewed $W_a$
as a closed subscheme of $X$ in Section \ref{section-rational-equivalence}.
Parts (2) and (3) are formal consequences of part (1) and the definitions.
\end{proof}

\noindent
For transversal intersections of closed subschemes the intersection
multiplicity is $1$.

\begin{lemma}
\label{lemma-transversal-subschemes}
Let $X$ be a nonsingular variety. Let $r, s \geq 0$ and let
$Y, Z \subset X$ be closed subschemes with $\dim(Y) \leq r$ and
$\dim(Z) \leq s$. Assume $[Y]_r = \sum n_i[Y_i]$ and
$[Z]_s = \sum m_j[Z_j]$ intersect properly.
Let $T$ be an irreducible component of $Y_{i_0} \cap Z_{j_0}$
for some $i_0$ and $j_0$ and assume that the multiplicity
(in the sense of Section \ref{section-cycle-of-closed}) of $T$
in the closed subscheme $Y \cap Z$ is $1$.
Then
\begin{enumerate}
\item the coefficient of $T$ in $[Y]_r \cdot [Z]_s$ is $1$,
\item $Y$ and $Z$ are nonsingular at the generic point of $Z$,
\item $n_{i_0} = 1$, $m_{j_0} = 1$, and
\item $T$ is not contained in $Y_i$ or $Z_j$ for $i \not = i_0$ and
$j \not = j_0$.
\end{enumerate}
\end{lemma}

\begin{proof}
Set $n = \dim(X)$, $a = n - r$, $b = n - s$. Observe that
$\dim(T) = r + s - n = n - a - b$ by the assumption that the
intersections are transversal. Let $(A, \mathfrak m, \kappa) =
(\mathcal{O}_{X, \xi}, \mathfrak m_\xi, \kappa(\xi))$ where $\xi \in T$
is the generic point. Then $\dim(A) = a + b$, see
Varieties, Lemma \ref{varieties-lemma-dimension-locally-algebraic}.
Let $I_0, I, J_0, J \subset A$ cut out the trace of
$Y_{i_0}$, $Y$, $Z_{j_0}$, $Z$ in $\Spec(A)$.
Then $\dim(A/I) = \dim(A/I_0) = b$ and $\dim(A/J) = \dim(A/J_0) = a$
by the same reference. Set $\overline{I} = I + \mathfrak m^2/\mathfrak m^2$.
Then $I \subset I_0 \subset \mathfrak m$ and
$J \subset J_0 \subset \mathfrak m$ and $I + J = \mathfrak m$.
By Lemma \ref{lemma-transversal} and its proof we see that
$I_0 = (f_1, \ldots, f_a)$ and $J_0 = (g_1, \ldots, g_b)$
where $f_1, \ldots, g_b$ is a regular system of parameters
for the regular local ring $A$. Since $I + J = \mathfrak m$, the map
$$
I \oplus J \to
\mathfrak m/\mathfrak m^2 =
\kappa f_1
\oplus \ldots \oplus
\kappa f_a
\oplus
\kappa g_1
\oplus \ldots \oplus
\kappa g_b
$$
is surjective. We conclude that we can find
$f_1', \ldots, f_a' \in I$ and $g'_1, \ldots, g_b' \in J$
whose residue classes in $\mathfrak m/\mathfrak m^2$ are equal to the
residue classes of $f_1, \ldots, f_a$ and $g_1, \ldots, g_b$.
Then $f'_1, \ldots, g'_b$ is a regular system of parameters of $A$.
By Algebra, Lemma \ref{algebra-lemma-regular-ring-CM} we find that
$A/(f'_1, \ldots, f'_a)$ is a regular local ring of dimension $b$.
Thus any nontrivial quotient of $A/(f'_1, \ldots, f'_a)$
has strictly smaller dimension
(Algebra, Lemmas \ref{algebra-lemma-regular-domain} and
\ref{algebra-lemma-one-equation}). Hence $I = (f'_1, \ldots, f'_a) = I_0$.
By symmetry $J = J_0$. This proves (2), (3), and (4).
Finally, the coefficient of $T$ in $[Y]_r \cdot [Z]_s$
is the coefficient of $T$ in $Y_{i_0} \cdot Z_{j_0}$ which is
$1$ by Lemma \ref{lemma-transversal}.
\end{proof}



\section{Exterior product}
\label{section-exterior-product}

\noindent
Let $X$ and $Y$ be varieties.
Let $V$, resp.\ $W$ be a closed subvariety of $X$, resp.\ $Y$.
The product $V\times W$ is a closed subvariety of $X\times Y$
(Lemma \ref{lemma-dimension-product-varieties}).
For a $k$-cycle $\alpha = \sum n_i [V_i]$ and a $l$-cycle
$\beta = \sum m_j [V_j]$ on $Y$ we define the
{\it exterior product} of $\alpha$ and $\beta$ to be the cycle
$\alpha \times \beta = \sum n_i m_j [V_i \times W_j]$.
Exterior product defines a $\mathbf{Z}$-linear map
$$
Z_r(X) \otimes_\mathbf{Z} Z_s(Y) \longrightarrow Z_{r + s}(X \times Y)
$$
Let us prove that exterior product factors through rational equivalence.

\begin{lemma}
\label{lemma-exterior-product-rational-equivalence}
Let $X$ and $Y$ be varieties.
Let $\alpha \in Z_r(X)$ and $\beta \in Z_s(Y)$.
If $\alpha \sim_{rat} 0$ or $\beta \sim_{rat} 0$, then
$\alpha \times \beta \sim_{rat} 0$.
\end{lemma}

\begin{proof}
By linearity and symmetry in $X$ and $Y$, it suffices to prove this when
$\alpha = [V]$ for some subvariety $V \subset X$ of dimension $s$ and
$\beta = [W_a]_s - [W_b]_s$ for some closed subvariety
$W \subset Y \times \mathbf{P}^1$ of dimension $s + 1$ which
intersects $Y \times a$ and $Y \times b$ properly. In this case
the lemma follows if we can prove
$$
[(V \times W)_a]_{r + s} = [V] \times [W_a]_s
$$
and similarly with $a$ replaced by $b$. Namely, then we see that
$\alpha \times \beta = [(V \times W)_a]_{r + s} - [(V \times W)_b]_{r + s}$
as desired. To see the displayed equality we note the equality
$$
V \times W_a = (V \times W)_a
$$
of schemes. The projection $V \times W_a \to W_a$ induces a bijection
of irreducible components (see for example
Varieties, Lemma \ref{varieties-lemma-bijection-irreducible-components}).
Let $W' \subset W_a$ be an irreducible component with generic point $\zeta$.
Then $V \times W'$ is the corresponding irreducible component of
$V \times W_a$ (see Lemma \ref{lemma-dimension-product-varieties}).
Let $\xi$ be the generic point of $V \times W'$. We have to show that
$$
\text{length}_{\mathcal{O}_{Y, \zeta}}(\mathcal{O}_{W_a, \zeta}) =
\text{length}_{\mathcal{O}_{X \times Y, \xi}}(
\mathcal{O}_{V \times W_a, \xi})
$$
In this formula we may replace
$\mathcal{O}_{Y, \zeta}$ by $\mathcal{O}_{W_a, \zeta}$ and
we may replace
$\mathcal{O}_{X \times Y, \zeta}$ by $\mathcal{O}_{V \times W_a, \zeta}$
(see Algebra, Lemma \ref{algebra-lemma-length-independent}).
As $\mathcal{O}_{W_a, \zeta} \to \mathcal{O}_{V \times W_a, \xi}$ is flat,
by Algebra, Lemma \ref{algebra-lemma-pullback-module} it suffices
to show that
$$
\text{length}_{\mathcal{O}_{V \times W_a, \xi}}(
\mathcal{O}_{V \times W_a, \xi}/
\mathfrak m_\zeta\mathcal{O}_{V \times W_a, \xi}) = 1
$$
This is true because the quotient on the right is the local ring
$\mathcal{O}_{V \times W', \xi}$ of a variety at a generic point
hence equal to $\kappa(\xi)$.
\end{proof}

\noindent
We conclude that exterior product defines a commutative diagram
$$
\xymatrix{
Z_r(X) \otimes_\mathbf{Z} Z_s(Y) \ar[r] \ar[d] &
Z_{r + s}(X \times Y) \ar[d] \\
A_r(X) \otimes_\mathbf{Z} A_s(Y) \ar[r] &
A_{r + s}(X \times Y)
}
$$
for any pair of varieties $X$ and $Y$. For nonsingular varieties
we can think of the exterior product as an intersection product
of pullbacks.

\begin{lemma}
\label{lemma-exterior-product}
Let $X$ and $Y$ be nonsingular varieties.
Let $\alpha \in Z_r(X)$ and $\beta \in Z_s(Y)$.
Then
\begin{enumerate}
\item $\text{pr}_Y^*(\beta) = [X] \times \beta$ and
$\text{pr}_X^*(\alpha) = \alpha \times [Y]$,
\item $\alpha \times [Y]$ and $[X]\times \beta$
intersect properly on $X\times Y$, and
\item we have
$\alpha \times \beta =
(\alpha \times [Y])\cdot ([X]\times\beta) =
pr_Y^*(\alpha) \cdot pr_X^*(\beta)$
in $Z_{r + s}(X \times Y)$.
\end{enumerate}
\end{lemma}

\begin{proof}
By linearity we may assume $\alpha = [V]$ and $\beta = [W]$.
Then (1) says that $\text{pr}_Y^{-1}(W) = X \times W$ and
$\text{pr}_X^{-1}(V) = V \times Y$. This is clear.
Part (2) holds because $X \times W \cap V \times Y = V \times W$ and
$\dim(V \times W) = r + s$ by Lemma \ref{lemma-dimension-product-varieties}.

\medskip\noindent
Proof of (3).
Let $\xi$ be the generic point of $V \times W$.
Since the projections $X \times W \to W$ is smooth as a base change of
$X \to \Spec(\mathbf{C})$, we see that $X \times W$ is nonsingular
at every point lying over the generic point of $W$, in particular at $\xi$.
Similarly for $V \times Y$. Hence $\mathcal{O}_{X \times W, \xi}$
and $\mathcal{O}_{V \times Y, \xi}$ are Cohen-Macaulay local rings
and Lemma \ref{lemma-intersection-multiplicity-CM} applies.
Since $V \times Y \cap X \times W = V \times W$ scheme theoretically
the proof is complete.
\end{proof}



\section{Reduction to the diagonal}
\label{section-reduction-diagonal}

\noindent
Let $X$ be a nonsingular variety. We will use $\Delta$
to denote either the diagonal morphism $\Delta : X \to X \times X$
or the image $\Delta \subset X \times X$.
Reduction to the diagonal is the statement that
intersection products on $X$ can be reduced to intersection products
of exterior products with the diagonal on $X \times X$.

\begin{lemma}
\label{lemma-tor-and-diagonal}
Let $X$ be a nonsingular variety.
\begin{enumerate}
\item If $\mathcal{F}$ and $\mathcal{G}$ are coherent $\mathcal{O}_X$-modules,
then there are canonical isomorphisms
$$
\text{Tor}_i^{\mathcal{O}_{X \times X}}(\mathcal{O}_\Delta,
\text{pr}_1^*\mathcal{F} \otimes_{\mathcal{O}_{X \times X}}
\text{pr}_2^*\mathcal{G})
=
\Delta_*\text{Tor}_i^{\mathcal{O}_X}(\mathcal{F}, \mathcal{G})
$$
\item If $K$ and $M$ are in $D_\QCoh(\mathcal{O}_X)$, then
there is a canonical isomorphism
$$
L\Delta^* \left(
L\text{pr}_1^*K \otimes_{\mathcal{O}_{X \times X}}^\mathbf{L} L\text{pr}_2^*M
\right)
= K \otimes_{\mathcal{O}_X}^\mathbf{L} M
$$
in $D_\QCoh(\mathcal{O}_X)$ and a canonical isomorphism
$$
\mathcal{O}_\Delta \otimes_{\mathcal{O}_{X \times X}}^\mathbf{L}
L\text{pr}_1^*K \otimes_{\mathcal{O}_{X \times X}}^\mathbf{L} L\text{pr}_2^*M
= \Delta_*(K \otimes_{\mathcal{O}_X}^\mathbf{L} M)
$$
in $D_\QCoh(X \times X)$.
\end{enumerate}
\end{lemma}

\begin{proof}
Let us explain how to prove (1) in a more elementary way and part (2) using
more general theory. As (2) implies (1) the reader can skip the proof
of (1).

\medskip\noindent
Proof of (1). Choose an affine open $\Spec(A) \subset X$.
Then $A$ is a Noetherian $\mathbf{C}$-algebra and
$\mathcal{F}$, $\mathcal{G}$ correspond to finite $A$-modules
$M$ and $N$ (Cohomology of Schemes, Lemma
\ref{coherent-lemma-coherent-Noetherian}).
By Derived Categories of Schemes, Lemma
\ref{perfect-lemma-quasi-coherence-tensor-product} we may
compute $\text{Tor}_i$ over $\mathcal{O}_X$
by first computing the $\text{Tor}$'s
of $M$ and $N$ over $A$, and then taking the associated $\mathcal{O}_X$-module.
For the $\text{Tor}_i$ over $\mathcal{O}_{X \times X}$ we compute
the tor of $A$ and $M \otimes_\mathbf{C} N$ over $A \otimes_\mathbf{C} A$
and then take the associated $\mathcal{O}_{X \times X}$-module.
Hence on this affine patch we have to prove that
$$
\text{Tor}_i^{A \otimes_\mathbf{C} A}(A, M \otimes_\mathbf{C} N) =
\text{Tor}_i^A(M, N)
$$
To see this choose resolutions $F_\bullet \to M$ and $G_\bullet \to M$
by finite free $A$-modules
(Algebra, Lemma \ref{algebra-lemma-resolution-by-finite-free}).
Note that $\text{Tot}(F_\bullet \otimes_\mathbf{C} G_\bullet)$
is a resolution of $M \otimes_\mathbf{C} N$ as it computes
Tor groups over $\mathbf{C}$! Of course the terms of
$F_\bullet \otimes_\mathbf{C} G_\bullet$ are finite free
$A \otimes_\mathbf{C} A$-modules. Hence the left hand side
of the displayed equation is the module
$$
H_i(A \otimes_{A \otimes_\mathbf{C} A}
\text{Tot}(F_\bullet \otimes_\mathbf{C} G_\bullet))
$$
and the right hand side is the module
$$
H_i(\text{Tot}(F_\bullet \otimes_A G_\bullet))
$$
Since $A \otimes_{A \otimes_\mathbf{C} A} (F_p \otimes_\mathbf{C} G_q)
= F_p \otimes_A G_q$ we see that these modules are equal.
This defines an isomorphism over the affine open $\Spec(A) \times \Spec(A)$
(which is good enough for the application to equality of intersection numbers).
We omit the proof that these isomorphisms glue.

\medskip\noindent
Proof of (2). The second statement follows from the first by the
projection formula as stated in
Derived Categories of Schemes, Lemma \ref{perfect-lemma-cohomology-base-change}.
To see the first, represent $K$ and $M$ by K-flat complexes
$\mathcal{K}^\bullet$ and $\mathcal{M}^\bullet$.
Since pullback and tensor product preserve K-flat complexes
(Cohomology, Lemmas \ref{cohomology-lemma-tensor-product-K-flat} and
\ref{cohomology-lemma-pullback-K-flat})
we see that it suffices to show
$$
\Delta^*\text{Tot}(
\text{pr}_1^*\mathcal{K}^\bullet
\otimes_{\mathcal{O}_{X \times X}} \text{pr}_2^*\mathcal{M}^\bullet)
=
\text{Tot}(
\mathcal{K}^\bullet \otimes_{\mathcal{O}_X} \mathcal{M}^\bullet)
$$
Thus it suffices to see that there are canonical isomorphisms
$$
\Delta^*(\text{pr}_1^*\mathcal{K}
\otimes_{\mathcal{O}_{X \times X}} \text{pr}_2^*\mathcal{M})
\longrightarrow
\mathcal{K} \otimes_{\mathcal{O}_X} \mathcal{M}
$$
whenever $\mathcal{K}$ and $\mathcal{M}$ are $\mathcal{O}_X$-modules
(not necessarily quasi-coherent or flat).
We omit the details.
\end{proof}

\begin{lemma}
\label{lemma-reduction-diagonal}
Let $X$ be a nonsingular variety. Let $\alpha$, resp.\ $\beta$
be an $r$-cycle, resp.\ $s$-cycle on $X$. Assume $\alpha$ and $\beta$
intersect properly. Then
\begin{enumerate}
\item $\alpha \times \beta$ and $[\Delta]$ intersect properly
\item we have $\Delta_*(\alpha \cdot \beta) = [\Delta] \cdot \alpha\times\beta$
as cycles on $X \times X$,
\item if $X$ is proper, then
$\text{pr}_{1, *}([\Delta] \cdot \alpha\times\beta) = \alpha\cdot\beta$,
where $pr_1 : X\times X \to X$ is the projection.
\end{enumerate}
\end{lemma}

\begin{proof}
By linearity it suffices to prove this when $\alpha = [V]$ and $\beta = [W]$
for some closed subvarieties $V \subset X$ and $W \subset Y$ which intersect
properly. Recall that $V \times W$ is a closed subvariety of dimension $r + s$.
Observe that scheme theoretically we have
$V \cap W = \Delta^{-1}(V \times W)$ as well as
$\Delta(V \cap W) = \Delta \cap V \times W$.
This proves (1).

\medskip\noindent
Proof of (2). Let $Z \subset V \cap W$ be an irreducible component
with generic point $\xi$. We have to show that the coefficient of
$Z$ in $\alpha \cdot \beta$ is the same as the coefficient of
$\Delta(Z)$ in $[\Delta] \cdot \alpha \times \beta$. The first is given
by the integer
$$
\sum (-1)^i
\text{length}_{\mathcal{O}_{X, \xi}}
\text{Tor}_i^{\mathcal{O}_X}(\mathcal{O}_V, \mathcal{O}_W)_\xi
$$
and the second by the integer
$$
\sum (-1)^i
\text{length}_{\mathcal{O}_{X \times Y, \Delta(\xi)}}
\text{Tor}_i^{\mathcal{O}_{X \times Y}}(
\mathcal{O}_\Delta, \mathcal{O}_{V \times W})_{\Delta(\xi)}
$$
However, by Lemma \ref{lemma-tor-and-diagonal} we have
$$
\text{Tor}_i^{\mathcal{O}_X}(\mathcal{O}_V, \mathcal{O}_W)_\xi \cong
\text{Tor}_i^{\mathcal{O}_{X \times Y}}(
\mathcal{O}_\Delta, \mathcal{O}_{V \times W})_{\Delta(\xi)}
$$
as $\mathcal{O}_{X \times X, \Delta(\xi)}$-modules. Thus equality
of lengths (by Algebra, Lemma \ref{algebra-lemma-length-independent}
to be precise).

\medskip\noindent
Part (2) implies (3) because
$\text{pr}_{1, *} \circ \Delta_* = \text{id}$ by
Lemma \ref{lemma-compose-pushforward}.
\end{proof}

\begin{proposition}
\label{proposition-positivity}
\begin{reference}
This is one of the main results of \cite{Serre_algebre_locale}.
\end{reference}
Let $X$ be a nonsingular variety. Let $V \subset X$ and
$W \subset Y$ be closed subvarieties which intersect properly.
Let $Z \subset V \cap W$ be an irreducible component.
Then $e(X, V \cdot W, Z) > 0$.
\end{proposition}

\begin{proof}
By Lemma \ref{lemma-reduction-diagonal} we have
$$
e(X, V \cdot W, Z) = e(X \times X, \Delta \cdot V \times W, \Delta(Z))
$$
Since $\Delta : X \to X \times X$ is a regular immersion
(see Lemma \ref{lemma-diagonal-regular-immersion}), we see that
$e(X \times X, \Delta \cdot V \times W, \Delta(Z))$ is a positive
integer by Lemma \ref{lemma-multiplicity-with-lci}.
\end{proof}

\noindent
The following is a key lemma in the development of the theory as
is done in this chapter. Essentially, this lemma tells us that
the intersection numbers have a suitable additivity property.

\begin{lemma}
\label{lemma-tor-sheaf}
\begin{reference}
\cite[Chapter V]{Serre_algebre_locale}
\end{reference}
Let $X$ be a nonsingular variety. Let $\mathcal{F}$ and
$\mathcal{G}$ be coherent sheaves on $X$ with
$\dim(\text{Supp}(\mathcal{F})) \leq r$,
$\dim(\text{Supp}(\mathcal{G})) \leq s$, and
$\dim(\text{Supp}(\mathcal{F}) \cap \text{Supp}(\mathcal{G}) )
\leq r + s - \dim X$. In this case $[\mathcal{F}]_r$ and $[\mathcal{G}]_s$
intersect properly and
$$
[\mathcal{F}]_r \cdot [\mathcal{G}]_s =
\sum (-1)^p
[\text{Tor}_p^{\mathcal{O}_X}(\mathcal{F}, \mathcal{G})]_{r + s - \dim(X)}.
$$
\end{lemma}

\begin{proof}
The statement that $[\mathcal{F}]_r$ and $[\mathcal{G}]_s$ intersect properly
is immediate. Since we are proving an equality of cycles we may work
locally on $X$. (Observe that the formation of the intersection
product of cycles, the formation of $\text{Tor}$-sheaves, and
forming the cycle associated to a coherent sheaf, each commute with
restriction to open subschemes.) Thus we may and do assume that $X$ is affine.

\medskip\noindent
Denote
$$
RHS(\mathcal{F}, \mathcal{G}) = [\mathcal{F}]_r \cdot [\mathcal{G}]_s
\quad\text{and}\quad
LHS(\mathcal{F}, \mathcal{G}) = \sum (-1)^p
[\text{Tor}_p^{\mathcal{O}_X}(\mathcal{F}, \mathcal{G})]_{r + s - \dim(X)}
$$
Consider a short exact sequence
$$
0 \to \mathcal{F}_1 \to \mathcal{F}_2 \to \mathcal{F}_3 \to 0
$$
of coherent sheaves on $X$ with
$\text{Supp}(\mathcal{F}_i) \subset \text{Supp}(\mathcal{F})$,
then both $LHS(\mathcal{F}_i, \mathcal{G})$ and
$RHS(\mathcal{F}_i, \mathcal{G})$ are defined for $i = 1, 2, 3$
and we have
$$
RHS(\mathcal{F}_2, \mathcal{G}) =
RHS(\mathcal{F}_1, \mathcal{G}) + RHS(\mathcal{F}_3, \mathcal{G})
$$
and similarly for LHS. Namely, the support condition guarantees that
everything is defined, the short exact sequence and additivity of lengths
gives
$$
[\mathcal{F}_2]_r = [\mathcal{F}_1]_r  + [\mathcal{F}_3]_r
$$
(Chow Homology, Lemma \ref{chow-lemma-additivity-sheaf-cycle})
which implies additivity for RHS. The long exact sequence of $\text{Tor}$s
$$
\ldots \to \text{Tor}_1(\mathcal{F}_3, \mathcal{G}) \to
\text{Tor}_0(\mathcal{F}_1, \mathcal{G}) \to
\text{Tor}_0(\mathcal{F}_2, \mathcal{G}) \to
\text{Tor}_0(\mathcal{F}_3, \mathcal{G}) \to 0
$$
and additivity of lengths as before implies additivity for LHS.

\medskip\noindent
By Algebra, Lemma \ref{algebra-lemma-filter-Noetherian-module}
and the fact that $X$ is affine, we can find a filtration of $\mathcal{F}$
whose graded pieces are structure sheaves of closed subvarieties of
$\text{Supp}(\mathcal{F})$. The additivity shown in the previous paragraph,
implies that it suffices to prove $LHS = RHS$ with
$\mathcal{F}$ replaced by $\mathcal{O}_V$ where
$V \subset \text{Supp}(\mathcal{F})$.
By symmetry we can do the same for $\mathcal{G}$.
This reduces us to proving that
$$
LHS(\mathcal{O}_V, \mathcal{O}_W) = RHS(\mathcal{O}_V, \mathcal{O}_W)
$$
where $W \subset \text{Supp}(\mathcal{G})$ is a closed subvariety.
If $\dim(V) = r$ and $\dim(W) = s$, then this equality is the
{\bf definition} of $V \cdot W$. On the other hand, if
$\dim(V) < r$ or $\dim(W) < s$, i.e., $[V]_r = 0$ or $[W]_s = 0$,
then we have to prove that $RHS(\mathcal{O}_V, \mathcal{O}_W) = 0$
\footnote{The reader can see that this is not a triviality by
taking $r = s = 1$ and $X$ a nonsingular surface and $V = W$
a closed point $x$ of $X$. In this case there are $3$ nonzero
$\text{Tor}$s of lengths $1, 2, 1$ at $x$.}.

\medskip\noindent
Let $Z \subset V \cap W$ be an irreducible component of dimension
$r + s - \dim(X)$. This is the maximal dimension of a component
and it suffices to show that the coefficient of $Z$ in $RHS$ is zero.
Let $\xi \in Z$ be the generic point. Write $A = \mathcal{O}_{X, \xi}$,
$B = \mathcal{O}_{X \times X, \Delta(\xi)}$, and
$C = \mathcal{O}_{V \times W, \Delta(\xi)}$.
By Lemma \ref{lemma-tor-and-diagonal} we have
$$
\text{coeff of }Z\text{ in }
RHS(\mathcal{O}_V, \mathcal{O}_W) = 
\sum (-1)^i
\text{length}_B \text{Tor}_i^B(A, C)
$$
Since $\dim(V) < r$ or $\dim(W) < s$ we have $\dim(V \times W) < r + s$
which implies $\dim(C) < \dim(X)$ (small detail omitted). Moreover, the
kernel $I$ of $B \to A$ is generated by a regular sequence of
length $\dim(X)$ (Lemma \ref{lemma-diagonal-regular-immersion}).
Hence vanishing by Lemma \ref{lemma-one-ideal-ci} because
the Hilbert function of $C$ with respect to $I$ has degree $\dim(C) < n$
by Algebra, Proposition \ref{algebra-proposition-dimension}.
\end{proof}

\begin{remark}
\label{remark-Serre-conjectures}
Let $(A, \mathfrak m, \kappa)$ be a regular local ring.
Let $M$ and $N$ be nonzero finite $A$-modules such that $M \otimes_A N$
is supported in $\{\mathfrak m\}$. Then
$$
\chi(M, N) = \sum (-1)^i \text{length}_A \text{Tor}_i^A(M, N)
$$
is finite. Let $r = \dim(\text{Supp}(M))$ and $s = \dim(\text{Supp}(N))$.
In \cite{Serre_algebre_locale} it is shown that $r + s \leq \dim(A)$
and the following conjectures are made:
\begin{enumerate}
\item if $r + s < \dim(A)$, then $\chi(M, N) = 0$, and
\item if $r + s = \dim(A)$, then $\chi(M, N) > 0$.
\end{enumerate}
The arguments that prove Lemma \ref{lemma-tor-sheaf} and
Proposition \ref{proposition-positivity} can be leveraged
(as is done in Serre's text) to show that (1) and (2) are
true if $A$ contains a field. Currently, conjecture (1) is known
in general and it is known that $\chi(M, N) \geq 0$ in general (Gabber).
Positivity is, as far as we know, still an open problem.
\end{remark}



\section{Associativity of intersections}
\label{section-associative}

\noindent
It is clear that proper intersections as defined above are commutative.
Using the key Lemma \ref{lemma-tor-sheaf} we can prove that (proper)
intersection products are associative.

\begin{lemma}
\label{lemma-associative}
Let $X$ be a nonsingular variety. Let $U, V, W$ be closed
subvarieties. Assume that $U, V, W$ intersect properly pairwise
and that $\dim(U \cap V \cap W) \leq \dim(U) + \dim(V) + \dim(W) - 2\dim(X)$.
Then
$$
U \cdot (V \cdot W) = (U \cdot V) \cdot W
$$
as cycles on $X$.
\end{lemma}

\begin{proof}
We are going to use Lemma \ref{lemma-tor-sheaf} without further mention.
This implies that
\begin{align*}
V \cdot W
& =
\sum (-1)^i [\text{Tor}_i(\mathcal{O}_V, \mathcal{O}_W)]_{b + c - n} \\
U \cdot (V \cdot W)
& =
\sum (-1)^{i + j}
[
\text{Tor}_j(\mathcal{O}_U, \text{Tor}_i(\mathcal{O}_V, \mathcal{O}_W))
]_{a + b + c - 2n} \\
U \cdot V
& =
\sum (-1)^i [\text{Tor}_i(\mathcal{O}_U, \mathcal{O}_V)]_{a + b - n} \\
(U \cdot V) \cdot W
& =
\sum (-1)^{i + j}
[
\text{Tor}_j(\text{Tor}_i(\mathcal{O}_U, \mathcal{O}_V), \mathcal{O}_W))
]_{a + b + c - 2n}
\end{align*}
where $\dim(U) = a$, $\dim(V) = b$, $\dim(W) = c$, $\dim(X) = n$.
The assumptions in the lemma guarantee that the coherent sheaves
in the formulae above satisfy the required bounds on dimensions
of supports in order to make sense of these. Now consider the object
$$
K =
\mathcal{O}_U \otimes^\mathbf{L}_{\mathcal{O}_X} \mathcal{O}_V
\otimes^\mathbf{L}_{\mathcal{O}_X} \mathcal{O}_W
$$
of the derived category $D_{\textit{Coh}}(\mathcal{O}_X)$.
We claim that the expressions obtained above for
$U \cdot (V \cdot W)$ and $(U \cdot V) \cdot W$
are equal to
$$
\sum (-1)^k [H^k(K)]_{a + b + c - 2n}
$$
This will prove the lemma. By symmetry it suffices to prove one
of these equalities. To do this we represent $\mathcal{O}_U$
and $\mathcal{O}_V \otimes_{\mathcal{O}_X}^\mathbf{L} \mathcal{O}_W$
by K-flat complexes $M^\bullet$ and $L^\bullet$ and use the
spectral sequence associated to the double complex
$K^\bullet \otimes_{\mathcal{O}_X} L^\bullet$ in
Homology, Section \ref{homology-section-double-complex}.
This is a spectral sequence with $E_2$ page
$$
E_2^{p, q} =
\text{Tor}_{-p}(\mathcal{O}_U, \text{Tor}_{-q}(\mathcal{O}_V, \mathcal{O}_W))
$$
converging to $H^{p + q}(K)$ (details omitted; compare with
More on Algebra, Example \ref{more-algebra-example-tor}).
Since lengths are additive in short
exact sequences we see that the result is true.
\end{proof}




\section{Flat pullback and intersection products}
\label{section-flat-pullback-and-intersection-products}

\noindent
Short discussion of the interaction between intersections and
flat pullback.

\begin{lemma}
\label{lemma-flat-pull-back-and-intersections-sheaves}
Let $f : X \to Y$ be a flat morphism of nonsingular varieties. Set
$e = \dim(X) - \dim(Y)$. Let $\mathcal{F}$ and $\mathcal{G}$ be coherent
sheaves on $Y$ with $\dim(\text{Supp}(\mathcal{F})) \leq r$,
$\dim(\text{Supp}(\mathcal{G})) \leq s$, and
$\dim(\text{Supp}(\mathcal{F}) \cap \text{Supp}(\mathcal{G}) )
\leq r + s - \dim(Y)$. In this case the cycles
$[f^*\mathcal{F}]_{r + e}$ and $[f^*\mathcal{G}]_{s + e}$
intersect properly and
$$
f^*([\mathcal{F}]_r \cdot [\mathcal{G}]_s) =
[f^*\mathcal{F}]_{r + e} \cdot [f^*\mathcal{G}]_{s + e}
$$
\end{lemma}

\begin{proof}
The statement that $[f^*\mathcal{F}]_{r + e}$ and $[f^*\mathcal{G}]_{s + e}$
intersect properly is immediate from the assumption that $f$ has
relative dimension $e$. By
Lemmas \ref{lemma-tor-sheaf} and \ref{lemma-pullback}
it suffices to show that
$$
f^*\text{Tor}_i^{\mathcal{O}_Y}(\mathcal{F}, \mathcal{G}) =
\text{Tor}_i^{\mathcal{O}_X}(f^*\mathcal{F}, f^*\mathcal{G})
$$
as $\mathcal{O}_X$-modules. This follows from
Cohomology, Lemma \ref{cohomology-lemma-pullback-tensor-product}
and the fact that $f^*$ is exact, so $Lf^*\mathcal{F} = f^*\mathcal{F}$
and similarly for $\mathcal{G}$.
\end{proof}

\begin{lemma}
\label{lemma-flat-pullback-and-intersections}
Let $f : X \to Y$ be a flat morphism of nonsingular varieties.
Let $\alpha$ be a $r$-cycle on $Y$ and $\beta$ an $s$-cycle on $Y$.
Assume that $\alpha$ and $\beta$ intersect properly. Then $f^*\alpha$
and $f^*\beta$ intersect properly and
$f^*( \alpha \cdot \beta ) = f^*\alpha \cdot f^*\beta$.
\end{lemma}

\begin{proof}
By linearity we may assume that $\alpha = [V]$ and $\beta = [W]$
for some closed subvarieties $V, W \subset Y$ of dimension $r, s$.
Say $f$ has relative dimension $e$. Then the lemma is a special case of
Lemma \ref{lemma-flat-pull-back-and-intersections-sheaves}
because $[V] = [\mathcal{O}_V]_r$, $[W] = [\mathcal{O}_W]_r$,
$f^*[V] = [f^{-1}(V)]_{r + e} = [f^*\mathcal{O}_V]_{r + e}$, and
$f^*[W] = [f^{-1}(W)]_{s + e} = [f^*\mathcal{O}_W]_{s + e}$.
\end{proof}


\section{Projection formula for flat proper morphisms}
\label{section-projection-formula-flat}

\noindent
Short discussion of the projection formula for flat proper morphisms.

\begin{lemma}
\label{lemma-projection-formula-flat}
\begin{reference}
See \cite[Chapter V, Section 7, formula (10)]{Serre_algebre_locale}
for a more general formula.
\end{reference}
Let $f : X \to Y$ be a flat proper morphism of nonsingular varieties.
Set $e = \dim(X) - \dim(Y)$. Let $\alpha$ be an $r$-cycle on $X$ and let
$\beta$ be a $s$-cycle on $Y$. Assume that $\alpha$ and $f^*(\beta)$ intersect
properly. Then $f_*(\alpha)$ and $\beta$ intersect properly and
$$
f_*(\alpha) \cdot \beta = f_*( \alpha \cdot f^*\beta)
$$
\end{lemma}

\begin{proof}
By linearity we reduce to the case where $\alpha = [V]$ and
$\beta = [W]$ for some closed subvariety $V \subset X$ and
$W \subset Y$ of dimension $r$ and $s$. Then $f^{-1}(W)$ has
pure dimension $s + e$. We assume the cycles
$[V]$ and $f^*[W]$ intersect properly. We will use without
further mention the fact that $V \cap f^{-1}(W) \to f(V) \cap W$
is surjective.

\medskip\noindent
Let $a$ be the dimension of the generic fibre of $V \to f(V)$.
If $a > 0$, then $f_*[V] = 0$. In particular $f_*\alpha$ and $\beta$
intersect properly. To finish this case we have to show that
$f_*([V] \cdot f^*[W]) = 0$. However, since every fibre of
$V \to f(V)$ has dimension $\geq a$ (see
Morphisms, Lemma \ref{morphisms-lemma-openness-bounded-dimension-fibres})
we conclude that every irreducible component $Z$ of $V \cap f^{-1}(W)$
has fibres of dimension $\geq a$ over $f(Z)$. This certainly
implies what we want.

\medskip\noindent
Assume that $V \to f(V)$ is generically finite. Let $Z \subset f(V) \cap W$
be an irreducible component. Let $Z_i \subset V \cap f^{-1}(W)$,
$i = 1, \ldots, t$ be the irreducible components of $V \cap f^{-1}(W)$
dominating $Z$. By assumption each $Z_i$ has dimension
$r + s + e - \dim(X) = r + s - \dim(Y)$. Hence
$\dim(Z) \leq r + s - \dim(Y)$. Thus we see that $f(V)$ and $W$
intersect properly, $\dim(Z) = r + s - \dim(Y)$, and each
$Z_i \to Z$ is generically finite. In particular, it follows that
$V \to f(V)$ has finite fibre over the generic point $\xi$ of $Z$.
Thus $V \to Y$ is finite in an open neighbourhood of $\xi$, see
Cohomology of Schemes, Lemma
\ref{coherent-lemma-proper-finite-fibre-finite-in-neighbourhood}.
Using a very general projection formula\footnote{This can be avoided
by working in an affine neighbourhood of $\xi$ as above, choosing
an affine open of $X$ containing the generic points of the $Z_i$, and
translating the question into algebra. Doing this will produce a
relatively elementary proof of (\ref{equation-stalks}).}
for derived tensor products, we get
$$
Rf_*(\mathcal{O}_V \otimes_{\mathcal{O}_X}^\mathbf{L} Lf^*\mathcal{O}_W) =
Rf_*\mathcal{O}_V \otimes_{\mathcal{O}_Y}^\mathbf{L} \mathcal{O}_W
$$
see Derived Categories of Schemes, Lemma
\ref{perfect-lemma-cohomology-base-change}.
Since $f$ is flat, we see that $Lf^*\mathcal{O}_W = f^*\mathcal{O}_W$.
Since $f|_V$ is finite in an open neighbourhood of $\xi$ we have
$$
(Rf_*\mathcal{F})_\xi = (f_*\mathcal{F})_\xi
$$
for any coherent sheaf on $X$ whose support is contained in $V$
(see Cohomology of Schemes, Lemma
\ref{coherent-lemma-higher-direct-images-zero-finite-fibre}). Thus
we conclude that
\begin{equation}
\label{equation-stalks}
\left(
f_*\text{Tor}_i^{\mathcal{O}_X}(\mathcal{O}_V, f^*\mathcal{O}_W)
\right)_\xi =
\left(\text{Tor}_i^{\mathcal{O}_Y}(f_*\mathcal{O}_V, \mathcal{O}_W)\right)_\xi
\end{equation}
for all $i$. Since $f^*[W] = [f^*\mathcal{O}_W]_{s + e}$ by
Lemma \ref{lemma-pullback} we have
$$
[V] \cdot f^*[W] =
\sum (-1)^i
[\text{Tor}_i^{\mathcal{O}_X}(\mathcal{O}_V,
f^*\mathcal{O}_W)]_{r + s - \dim(Y)}
$$
by Lemma \ref{lemma-tor-sheaf}. Applying
Lemma \ref{lemma-push-coherent}
we find
$$
f_*([V] \cdot f^*[W]) =
\sum (-1)^i
[f_*\text{Tor}_i^{\mathcal{O}_X}(\mathcal{O}_V,
f^*\mathcal{O}_W)]_{r + s - \dim(Y)}
$$
Since $f_*[V] = [f_*\mathcal{O}_V]_r$ by
Lemma \ref{lemma-push-coherent} we have
$$
[f_*V] \cdot [W] =
\sum (-1)^i
[\text{Tor}_i^{\mathcal{O}_X}(f_*\mathcal{O}_V,
\mathcal{O}_W)]_{r + s - \dim(Y)}
$$
again by Lemma \ref{lemma-tor-sheaf}.
Comparing the formula for $f_*([V] \cdot f^*[W])$ with
the formula for $f_*[V] \cdot [W]$ and looking at the
coefficient of $Z$ by taking lengths of stalks at $\xi$, we see that
(\ref{equation-stalks}) finishes the proof.
\end{proof}

\begin{lemma}
\label{lemma-transfer}
Let $X \to P$ be a closed immersion of nonsingular varieties.
Let $C' \subset P \times \mathbf{P}^1$ be a closed subvariety of dimension
$r + 1$. Assume
\begin{enumerate}
\item the fibre $C = C'_0$ has dimension $r$, i.e., $C' \to \mathbf{P}^1$
is dominant,
\item $C'$ intersects $X \times \mathbf{P}^1$ properly,
\item $[C]_r$ intersects $X$ properly.
\end{enumerate}
Then setting $\alpha = [C]_r \cdot X$ viewed as cycle on $X$ and
$\beta = C' \cdot X \times \mathbf{P}^1$ viewed as cycle on
$X \times \mathbf{P}^1$, we have
$$
\alpha = \text{pr}_{X, *}(\beta \cdot X \times 0)
$$
as cycles on $X$ where $\text{pr}_X : X \times \mathbf{P}^1 \to X$ is the
projection.
\end{lemma}

\begin{proof}
Let $\text{pr} : P \times \mathbf{P}^1 \to P$ be the projection.
Since we are proving an equality of cycles it suffices to think of
$\alpha$, resp.\ $\beta$ as a cycle on $P$, resp.\ $P \times \mathbf{P}^1$
and prove the result for pushing forward by $\text{pr}$.
Because $\text{pr}^*X = X \times \mathbf{P}^1$ and
$\text{pr}$ defines an isomorphism of $C'_0$ onto $C$
the projection formula (Lemma \ref{lemma-projection-formula-flat})
gives
$$
\text{pr}_*([C'_0]_r \cdot X \times \mathbf{P}^1) = [C]_r \cdot X = \alpha
$$
On the other hand, we have $[C'_0]_r = C' \cdot P \times 0$
as cycles on $P \times \mathbf{P}^1$
by Lemma \ref{lemma-rational-equivalence-and-intersection}.
Hence
$$
[C'_0]_r \cdot X \times \mathbf{P}^1 =
(C' \cdot P \times 0) \cdot X \times \mathbf{P}^1 =
(C' \cdot X \times \mathbf{P}^1) \cdot P \times 0
$$
by associativity (Lemma \ref{lemma-associative}) and commutativity of the
intersection product. It remains to show that the intersection product of
$C' \cdot X \times \mathbf{P}^1$ with $P \times 0$ on
$P \times \mathbf{P}^1$ is equal (as a cycle) to the intersection product of
$\beta$ with $X \times 0$ on $X \times \mathbf{P}^1$. Write
$C' \cdot X \times \mathbf{P}^1 = \sum n_k[E_k]$ and hence
$\beta = \sum n_k[E_k]$ for some subvarieties
$E_k \subset X \times \mathbf{P}^1 \subset P \times \mathbf{P}^1$.
Then both intersections are equal to $\sum m_k[E_{k, 0}]$ by
Lemma \ref{lemma-rational-equivalence-and-intersection} applied twice.
This finishes the proof.
\end{proof}




\section{Projections}
\label{section-projection}

\noindent
Recall that we are working over a fixed algebraically closed ground field
$\mathbf{C}$. If $V$ is a finite dimensional vector space over $\mathbf{C}$
then we set
$$
\mathbf{P}(V) = \text{Proj}(\text{Sym}(V))
$$
where $\text{Sym}(V)$ is the symmetric algebra on $V$ over $\mathbf{C}$.
The normalization is chosen such that
$V = \Gamma(\mathbf{P}(V), \mathcal{O}_{\mathbf{P}(V)}(1))$.
Of course we have $\mathbf{P}(V) \cong \mathbf{P}^n_{\mathbf{C}}$ if
$\dim(V) = n + 1$. We note that $\mathbf{P}(V)$ is a nonsingular projective
variety.

\medskip\noindent
Let $p \in \mathbf{P}(V)$ be a closed point. The point $p$ corresponds to a
surjection $V \to L_p$ of vector spaces where $\dim(L_p) = 1$, see
Constructions, Lemma \ref{constructions-lemma-apply}.
Let us denote $W_p = \Ker(V \to L_p)$.
{\it Projection from $p$} is the morphism
$$
r_p : \mathbf{P}(V) \setminus \{p\} \longrightarrow \mathbf{P}(W_p)
$$
of Constructions, Lemma \ref{constructions-lemma-morphism-proj}.
Here is a lemma to warm up.

\begin{lemma}
\label{lemma-projection-generically-finite}
Let $V$ be a vector space of dimension $n + 1$.
Let $X \subset \mathbf{P}(V)$ be a closed subscheme.
If $X \not = \mathbf{P}(V)$, then there is a nonempty Zariski open
$U \subset \mathbf{P}(V)$
such that for all closed points $p \in U$ the restriction
of the projection $r_p$ defines a finite morphism
$r_p|_X : X \to \mathbf{P}(W_p)$.
\end{lemma}

\begin{proof}
We claim the lemma holds with $U = \mathbf{P}(V) \setminus X$. For a closed
point $p$ of $U$ we indeed obtain a morphism $r_p|_X : X \to \mathbf{P}(W_p)$.
This morphism is proper because $X$ is a proper scheme
(Morphisms, Lemmas \ref{morphisms-lemma-locally-projective-proper} and
\ref{morphisms-lemma-image-proper-scheme-closed}). On the other hand, the
fibres of $r_p$ are affine lines as can be seen by a direct calculation.
Hence the fibres of $r_p|X$ are proper and affine, whence finite
(Morphisms, Lemma \ref{morphisms-lemma-finite-proper}).
Finally, a proper morphism with finite fibres is finite
(Cohomology of Schemes, Lemma \ref{coherent-lemma-characterize-finite}).
\end{proof}

\begin{lemma}
\label{lemma-projection-generically-immersion}
Let $V$ be a vector space of dimension $n + 1$.
Let $X \subset \mathbf{P}(V)$ be a closed subvariety.
Let $x \in X$ be a nonsingular point.
\begin{enumerate}
\item If $\dim(X) < n - 1$, then there is a nonempty Zariski open
$U \subset \mathbf{P}(V)$ such that for all closed points $p \in U$ the
morphism $r_p|_X : X \to r_p(X)$ is an
isomorphism over an open neighbourhood of $r_p(x)$.
\item If $\dim(X) = n - 1$, then there is a nonempty Zariski open
$U \subset \mathbf{P}(V)$ such that for all closed points $p \in U$ the
morphism $r_p|_X : X \to \mathbf{P}(W_p)$ is \'etale at $x$.
\end{enumerate}
\end{lemma}

\begin{proof}
Proof of (1). Note that if $x, y \in X$ have the same image under
$r_p$ then $p$ is on the line $\overline{xy}$.
Consider the finite type scheme
$$
T = \{(y, p) \mid
y \in X \setminus \{x\},\ p \in \mathbf{P}(V),\ p \in \overline{xy}\}
$$
and the morphisms $T \to X$ and $T \to \mathbf{P}(V)$ given by
$(y, p) \mapsto y$ and $(y, p) \mapsto p$.
Since each fibre of $T \to X$ is a line, we see that
the dimension of $T$ is $\dim(X) + 1 < \dim(\mathbf{P}(V))$.
Hence $T \to \mathbf{P}(V)$ is not surjective. On the other hand,
consider the finite type scheme
$$
T' = \{p \mid
p \in \mathbf{P}(V) \setminus \{x\},
\ \overline{xp}\text{ tangent to }X\text{ at }x\}
$$
Then the dimension of $T'$ is $\dim(X) < \dim(\mathbf{P}(V))$.
Thus the morphism $T' \to \mathbf{P}(V)$ is not surjective either.
Let $U \subset \mathbf{P}(V) \setminus X$ be nonempty open and disjoint
from these images; such a $U$ exists because the images of $T$ and $T'$
in $\mathbf{P}(V)$ are constructible by
Morphisms, Lemma \ref{morphisms-lemma-chevalley}.
Then for $p \in U$ closed the projection
$r_p|_X : X \to \mathbf{P}(W_p)$ is injective on the
tangent space at $x$ and $r_p^{-1}(\{r_p(x)\}) = \{x\}$.
This means that $r_p$ is unramified at $x$
(Varieties, Lemma \ref{varieties-lemma-injective-tangent-spaces-unramified}),
finite by Lemma \ref{lemma-projection-generically-finite},
and $r_p^{-1}(\{r_p(x)\}) = \{x\}$ thus \'Etale Morphisms, Lemma
\ref{etale-lemma-finite-unramified-one-point} applies and
there is an open neighbourhood $R$ of $r_p(x)$
in $\mathbf{P}(W_p)$ such that $(r_p|_X)^{-1}(R) \to R$ is a
closed immersion which proves (1).

\medskip\noindent
Proof of (2). In this case we still conclude that the morphism
$T' \to \mathbf{P}(V)$ is not surjective.
Arguing as above we conclude that for $U \subset \mathbf{P}(V)$
avoiding $X$ and the image of $T'$, the projection
$r_p|_X : X \to \mathbf{P}(W_p)$ is \'etale at $x$ and finite.
\end{proof}

\begin{lemma}
\label{lemma-projection-injective}
Let $V$ be a vector space of dimension $n + 1$.
Let $Y, Z \subset \mathbf{P}(V)$ be closed subvarieties.
There is a nonempty Zariski open $U \subset \mathbf{P}(V)$
such that for all closed points $p \in U$ we have
$$
Y \cap r_p^{-1}(r_p(Z)) = (Y \cap Z) \cup E
$$
with $E \subset Y$ closed and
$\dim(E) \leq \dim(Y) + \dim(Z) + 1 - n$.
\end{lemma}

\begin{proof}
Set $Y' = Y \setminus Y \cap Z$.
Let $y \in Y'$, $z \in Z$ be closed points with $r_p(y) = r_p(z)$.
Then $p$ is on the line $\overline{yz}$ passing through $y$ and $z$.
Consider the finite type scheme
$$
T = \{(y, z, p) \mid y \in Y', z \in Z, p \in \overline{yz}\}
$$
and the morphism $T \to \mathbf{P}(V)$ given by $(y, z, p) \mapsto p$.
Observe that $T$ is irreducible and that $\dim(T) = \dim(Y) + \dim(Z) + 1$.
Hence the general fibre of $T \to \mathbf{P}(V)$ has dimension at most
$\dim(Y) + \dim(Z) + 1 - n$, more precisely, there exists a nonempty
open $U \subset \mathbf{P}(V) \setminus (Y \cup Z)$ over
which the fibre has dimension at most $\dim(Y) + \dim(Z) + 1 - n$
(Varieties, Lemma \ref{varieties-lemma-dimension-fibres-locally-algebraic}).
Let $p \in U$ be a closed point and let $F \subset T$ be the fibre
of $T \to \mathbf{P}(V)$ over $p$. Then
$$
(Y \cap r_p^{-1}(r_p(Z))) \setminus (Y \cap Z)
$$
is the image of $F \to Y$, $(y, z, p) \mapsto y$. Again by
Varieties, Lemma \ref{varieties-lemma-dimension-fibres-locally-algebraic}
the closure of the image of $F \to Y$ has dimension at most
$\dim(Y) + \dim(Z) + 1 - n$.
\end{proof}

\begin{lemma}
\label{lemma-find-lines}
Let $V$ be a vector space. Let $B \subset \mathbf{P}(V)$
be a closed subvariety of codimension $\geq 2$.
Let $p \in \mathbf{P}(V)$ be a closed point, $p \not \in B$.
Then there exists a line $\ell \subset \mathbf{P}(V)$
with $\ell \cap B = \emptyset$. Moreover, these lines
sweep out an open subset of $\mathbf{P}(V)$.
\end{lemma}

\begin{proof}
Consider the image of $B$ under the projection
$r_p : \mathbf{P}(V) \to \mathbf{P}(W_p)$.
Since $\dim(W_p) = \dim(V) - 1$, we see that $r_p(B)$
has codimension $\geq 1$ in $\mathbf{P}(W_p)$.
For any $q \in \mathbf{P}(V)$ with $r_p(q) \not \in r_p(B)$
we see that the line $\ell = \overline{pq}$ connecting $p$ and $q$ works.
\end{proof}

\begin{lemma}
\label{lemma-doubly-transitive}
Let $V$ be a vector space. Let $G = \text{PGL}(V)$.
Then $G \times \mathbf{P}(V) \to \mathbf{P}(V)$ is
doubly transitive.
\end{lemma}

\begin{proof}
Omitted. Hint: This follows from the fact that $\text{GL}(V)$ acts doubly
transitive on pairs of linearly independent vectors.
\end{proof}

\begin{lemma}
\label{lemma-determinant}
Let $k$ be a field. Let $n \geq 1$ be an integer and let
$x_{ij}, 1 \leq i, j \leq n$ be variables. Then
$$
\det
\left(
\begin{matrix}
x_{11} & x_{12} & \ldots & x_{1n} \\
x_{21} & \ldots & \ldots & \ldots \\
\ldots & \ldots & \ldots & \ldots \\
x_{n1} & \ldots & \ldots & x_{nn}
\end{matrix}
\right)
$$
is an irreducible element of the polynomial ring $k[x_{ij}]$.
\end{lemma}

\begin{proof}
Let $V$ be an $n$ dimensional vector space. Translating into geometry
the lemma signifies that the variety $C$ of non-invertible linear maps
$V \to V$ is irreducible. Let $W$ be a vector space of dimension $n - 1$.
By elementary linear algebra, the morphism
$$
\Hom(W, V) \times \Hom(V, W) \longrightarrow \Hom(V, V),\quad
(\psi, \varphi) \longmapsto \psi \circ \varphi
$$
has image $C$. Since the source is irreducible, so is the image.
\end{proof}

\noindent
Let $V$ be a vector space of dimension $n + 1$. Set $E = \text{End}(V)$.
Let $E^\vee = \Hom(E, \mathbf{C})$ be the dual vector space.
Write $\mathbf{P} = \mathbf{P}(E^\vee)$.
There is a canonical linear map
$$
V \longrightarrow V \otimes_\mathbf{C} E^\vee = \Hom(E, V)
$$
sending $v \in V$ to the map $g \mapsto g(v)$ in $\Hom(E, V)$.
Recall that we have a canonical map
$E^\vee \to \Gamma(\mathbf{P}, \mathcal{O}_\mathbf{P}(1))$
which is an isomorphism. Hence we obtain a canonical map
$$
\psi : V \otimes \mathcal{O}_\mathbf{P} \to V \otimes \mathcal{O}_\mathbf{P}(1)
$$
of sheaves of modules on $\mathbf{P}$ which on global sections recovers
the given map. Recall that a projective bundle $\mathbf{P}(\mathcal{E})$
is defined as the relative Proj of the symmetric algebra on $\mathcal{E}$, see
Constructions, Definition \ref{constructions-definition-projective-bundle}.
We are going to study the rational
map between $\mathbf{P}(V \otimes \mathcal{O}_\mathbf{P}(1))$ and
$\mathbf{P}(V \otimes \mathcal{O}_\mathbf{P})$ associated to $\psi$. By
Constructions, Lemma \ref{constructions-lemma-relative-proj-base-change}
we have a canonical isomorphism
$$
\mathbf{P}(V \otimes \mathcal{O}_\mathbf{P}) = \mathbf{P} \times \mathbf{P}(V)
$$
By Constructions, Lemma \ref{constructions-lemma-twisting-and-proj}
we see that
$$
\mathbf{P}(V \otimes \mathcal{O}_\mathbf{P}(1)) =
\mathbf{P}(V \otimes \mathcal{O}_\mathbf{P}) = \mathbf{P} \times \mathbf{P}(V)
$$
Combining this with
Constructions, Lemma \ref{constructions-lemma-morphism-relative-proj}
we obtain
\begin{equation}
\label{equation-r-psi}
\mathbf{P} \times \mathbf{P}(V) \supset
U(\psi) \xrightarrow{r_\psi} \mathbf{P} \times \mathbf{P}(V)
\end{equation}
To understand this better we work out what happens on fibres over
$\mathbf{P}$. Let $g \in E$ be nonzero. This defines a nonzero map
$E^\vee \to \mathbf{C}$, hence a point $[g] \in \mathbf{P}$.
On the other hand, $g$ defines a $\mathbf{C}$-linear map $g : V \to V$.
Hence we obtain, by
Constructions, Lemma \ref{constructions-lemma-morphism-proj}
a map
$$
\mathbf{P}(V) \supset U(g) \xrightarrow{r_g} \mathbf{P}(V)
$$
What we will use below is that $U(g)$ is the fibre $U(\psi)_{[g]}$ and
that $r_g$ is the fibre of $r_\psi$ over the point $[g]$. Another observation
we will use is that the complement of $U(g)$ in $\mathbf{P}(V)$ is
the image of the closed immersion
$$
\mathbf{P}(\Coker(g)) \longrightarrow \mathbf{P}(V)
$$
and the image of $r_g$ is the image of the closed immersion
$$
\mathbf{P}(\Im(g)) \longrightarrow \mathbf{P}(V)
$$

\begin{lemma}
\label{lemma-make-family}
With notation as above. Let $X, Y$ be closed subvarieties of $\mathbf{P}(V)$
which intersect properly such that $X \not = \mathbf{P}(V)$. There exists
a line $\ell \subset \mathbf{P}$ such that
\begin{enumerate}
\item $[\text{id}_V] \in \ell$,
\item $X \subset U_g$ for all $[g] \in \ell$,
\item $g(X)$ intersects $Y$ properly for all $[g] \in \ell$.
\end{enumerate}
\end{lemma}

\begin{proof}
Let $B \subset \mathbf{P}$ be the set of ``bad'' points, i.e., those
points $[g]$ that violate either (2) or (3). Note that
$[\text{id}_V] \not \in B$ by assumption. Moreover, $B$ is closed.
Hence it suffices to prove that $\dim(B) \leq \dim(\mathbf{P}) - 2$
(Lemma \ref{lemma-find-lines}).

\medskip\noindent
First, consider the open $G = \text{PGL}(V) \subset \mathbf{P}$
consisting of points $[g]$ such that $g : V \to V$ is invertible.
Since $G$ acts doubly transitively on $\mathbf{P}(V)$
(Lemma \ref{lemma-doubly-transitive})
we see that
$$
T = \{(x, y, [g]) \mid x \in X, y \in Y, [g] \in G, r_g(x) = y\}
$$
is a locally trivial fibration over $X \times Y$ with fibre equal
to the stabilizer of a pair of points in $G$. Hence $T$ is a variety.
Observe that the fibre of $T \to G$ over $[g]$ is $r_g(X) \cap Y$.
The morphism $T \to G$ is surjective, because any translate of $X$
intersects $Y$
(Varieties, Lemma \ref{varieties-lemma-intersection-in-projective-space}).
Since the dimensions of fibres of a dominant morphism of varieties do
not jump in codimension $1$
(Varieties, Lemma \ref{varieties-lemma-dimension-fibres-locally-algebraic})
we conclude that $B \cap G$ has codimension $\geq 2$.

\medskip\noindent
Next we look at the complement $Z = \mathbf{P} \setminus G$.
This is an irreducible variety because the determinant is an
irreducible polynomial (Lemma \ref{lemma-determinant}).
Thus it suffices to prove that $B$ does not contain the
generic point of $Z$. For a general point $[g] \in Z$
the cokernel $V \to \Coker(g)$ has dimension $1$, hence
$U(g)$ is the complement of a point. Since $X \not = \mathbf{P}(V)$
we see that for a general $[g] \in Z$ we have $X \subset U(g)$.
Moreover, the morphism $r_g|_X : X \to r_g(X)$ is finite, hence
$\dim(r_g(X)) = \dim(X)$.
On the other hand, for such a $g$ the image of $r_g$ is the
closed subspace $H = \mathbf{P}(\Im(g)) \subset \mathbf{P}(V)$
which has codimension $1$.
For general point of $Z$ we see that $H \cap Y$ has dimension $1$
less than $Y$ (compare with
Varieties, Lemma \ref{varieties-lemma-exact-sequence-induction}).
Thus we see that we have to show that $r_g(X)$ and $H \cap Y$
intersect properly in $H$. For a fixed choice of $H$, we can
by postcomposing $g$ by an automorphism, move $r_g(X)$ by
an arbitrary automorphism of $H = \mathbf{P}(\Im(g))$.
Thus we can argue as above to conclude that the intersection
of $H \cap Y$ with $r_g(X)$ is proper for general $g$ with given
$H = \mathbf{P}(\Im(g))$. Some details omitted.
\end{proof}







\section{Moving Lemma}
\label{section-moving-lemma}

\noindent
The moving lemma states that given an $r$-cycle $\alpha$ and an $s$-cycle
$\beta$ there exists $\alpha'$, $\alpha' \sim_{rat} \alpha$ such that
$\alpha'$ and $\beta$ intersect properly (Lemma \ref{lemma-moving-move}).
See \cite{Samuel}, \cite{ChevalleyI}, \cite{ChevalleyII}.
The key to this is Lemma \ref{lemma-moving}; the reader may find
this lemma in the form stated in
\cite[Example 11.4.1]{F} and find a proof in \cite{Roberts}.

\begin{lemma}
\label{lemma-moving}
\begin{reference}
See \cite{Roberts}.
\end{reference}
Let $X \subset \mathbf{P}^N$ be a nonsingular closed subvariety.
Let $n = \dim(X)$ and $0 \leq d, d' < n$. Let $Z \subset X$ be a closed
subvariety of dimension $d$ and $T_i \subset X$, $i \in I$ be a finite
collection of closed subvarieties of dimension $d'$. Then there exists
a subvariety $C \subset \mathbf{P}^N$ such that $C$ intersects $X$
properly and such that
$$
C \cdot X = Z + \sum\nolimits_{j \in J} m_j Z_j
$$
where $Z_j \subset X$ are irreducible of dimension $d$, distinct from $Z$, and
$$
\dim(Z_j \cap T_i) \leq \dim(Z \cap T_i)
$$
with strict inequality if $Z$ does not intersect $T_i$ properly in $X$.
\end{lemma}

\begin{proof}
Write $\mathbf{P}^N = \mathbf{P}(V_N)$ so $\dim(V_N) = N + 1$ and set
$X_N = X$. We are going to choose a sequence of projections from points
\begin{align*}
& r_N : 
\mathbf{P}(V_N) \setminus \{p_N\} \to \mathbf{P}(V_{N - 1}), \\
& r_{N - 1} :
\mathbf{P}(V_{N - 1}) \setminus \{p_{N - 1}\} \to \mathbf{P}(V_{N - 2}), \\
& \ldots,\\
& r_{n + 1} :
\mathbf{P}(V_{n + 1}) \setminus \{p_{n + 1}\} \to \mathbf{P}(V_n)
\end{align*}
as in Section \ref{section-projection}. At each step we will choose
$p_N, p_{N - 1}, \ldots, p_{n + 1}$ in a suitable Zariski open set.
Pick a closed point $x \in Z \subset X$. For every $i$ pick
closed points $x_{it} \in T_i \cap Z$, at least one in each irreducible
component of $T_i \cap Z$. Taking the composition we obtain
a morphism
$$
\pi = (r_{n + 1} \circ \ldots \circ r_N)|_X :
X \longrightarrow \mathbf{P}(V_n)
$$
which has the following properties
\begin{enumerate}
\item $\pi$ is finite,
\item $\pi$ is \'etale at $x$ and all $x_{it}$,
\item $\pi|_Z : Z \to \pi(Z)$ is an isomorphism
over an open neighbourhood of $\pi(x_{it})$,
\item $T_i \cap \pi^{-1}(\pi(Z)) = (T_i \cap Z) \cup E_i$ with
$E_i \subset T_i$ closed and
$\dim(E_i) \leq d + d' + 1 - (n + 1) = d + d' - n$.
\end{enumerate}
It follows in a straightforward manner from
Lemmas \ref{lemma-projection-generically-finite},
\ref{lemma-projection-generically-immersion}, and
\ref{lemma-projection-injective} and induction that we can do this;
observe that the last projection is from $\mathbf{P}(V_{n + 1})$ and that
$\dim(V_{n + 1}) = n + 2$ which explains the inequality in (4).

\medskip\noindent
Let $C \subset \mathbf{P}(V_N)$ be the scheme theoretic closure of
$(r_{n + 1} \circ \ldots \circ r_N)^{-1}(\pi(Z))$. Because $\pi$
is \'etale at the point $x$ of $Z$, we see that the closed subscheme
$C \cap X$ contains $Z$ with multiplicity $1$ (local calculation omitted).
Hence by Lemma \ref{lemma-transversal-subschemes} we conclude that
$$
C \cdot X = [Z] + \sum m_j[Z_j]
$$
for some subvarieties $Z_j \subset X$ of dimension $d$. Note that
$$
C \cap X = \pi^{-1}(\pi(Z))
$$
set theoretically. Hence
$T_i \cap Z_j \subset T_i \cap \pi^{-1}(\pi(Z)) \subset T_i \cap Z \cup E_i$.
For any irreducible component of $T_i \cap Z$ contained in $E_i$ we
have the desired dimension bound. Finally, let $V$ be an irreducible
component of $T_i \cap Z_j$ which is contained in $T_i \cap Z$. To finish
the proof it suffices to show that $V$ does not contain any of the
points $x_{it}$, because then $\dim(V) < \dim(Z \cap T_i)$.
To show this it suffices to show that $x_{it} \not \in Z_j$
for all $i, t, j$.

\medskip\noindent
Set $Z' = \pi(Z)$ and $Z'' = \pi^{-1}(Z')$, scheme theoretically. By
condition (3) we can find an open $U \subset \mathbf{P}(V_n)$ containing
$\pi(x_{it})$ such that $\pi^{-1}(U) \cap Z \to U \cap Z'$ is an isomorphism.
In particular, $Z \to Z'$ is a local isomorphism at $x_{it}$.
On the other hand, $Z'' \to Z'$ is \'etale at $x_{it}$ by condition (2).
Hence the closed immersion $Z \to Z''$ is \'etale at $x_{it}$
(Morphisms, Lemma \ref{morphisms-lemma-etale-permanence}).
Thus $Z = Z''$ in a Zariski neighbourhood of $x_{it}$ which proves
the assertion.
\end{proof}

\noindent
The actual moving is done using the following lemma.

\begin{lemma}
\label{lemma-move}
Let $C \subset \mathbf{P}^N$ be a closed subvariety.
Let $X \subset \mathbf{P}^N$ be subvariety and let $T_i \subset X$
be a finite collection of closed subvarieties.
Assume that $C$ and $X$ intersect properly.
Then there exists a closed subvariety
$C' \subset \mathbf{P}^N \times \mathbf{P}^1$ such that
\begin{enumerate}
\item $C' \to \mathbf{P}^1$ is dominant,
\item $C'_0 = C$ scheme theoretically,
\item $C'$ and $X \times \mathbf{P}^1$ intersect properly,
\item $C'_\infty$ properly intersects each of the given $T_i$.
\end{enumerate}
\end{lemma}

\begin{proof}
Write $\mathbf{P}^N = \mathbf{P}(V)$ so $\dim(V) = N + 1$. Let
$E = \text{End}(V)$. Let $E^\vee = \Hom(E, \mathbf{C})$. Set
$\mathbf{P} = \mathbf{P}(E^\vee)$ as in Lemma \ref{lemma-make-family}.
Choose a general line $\ell \subset \mathbf{P}$ passing through $\text{id}_V$.
Set $C' \subset \ell \times \mathbf{P}(V)$ equal to the
closed subscheme having fibre $r_g(C)$ over $[g] \in \ell$.
More precisely, $C'$ is the image of
$$
\ell \times C \subset \mathbf{P} \times \mathbf{P}(V)
$$
under the morphism (\ref{equation-r-psi}). By Lemma \ref{lemma-make-family}
this makes sense, i.e., $\ell \times C \subset U(\psi)$. The morphism
$\ell \times C \to C'$ is finite and $C'_{[g]} = r_g(C)$ set theoretically
for all $[g] \in \ell$. Parts (1) and (2) are clear with
$0 = [\text{id}_V] \in \ell$. Part (3) follows from the fact
that $r_g(C)$ and $X$ intersect properly for all $[g] \in \ell$.
Part (4) follows from the fact that a general point $\infty = [g] \in \ell$
is a general point of $\mathbf{P}$ and for such as point
$r_g(C) \cap T$ is proper for any closed subvariety $T$ of $\mathbf{P}(V)$
(see proof of Lemma \ref{lemma-make-family}).
Some details omitted.
\end{proof}

\begin{lemma}
\label{lemma-moving-move}
Let $X$ be a nonsingular projective variety. Let $\alpha$ be an
$r$-cycle and $\beta$ be an $s$-cycle on $X$. Then there exists
an $r$-cycle $\alpha'$ such that $\alpha' \sim_{rat} \alpha$ and
such that $\alpha'$ and $\beta$ intersect properly.
\end{lemma}

\begin{proof}
Write $\beta = \sum n_i[T_i]$ for some subvarieties $T_i \subset X$
of dimension $s$. By linearity we may assume that $\alpha = [Z]$ for
some irreducible closed subvariety $Z \subset X$ of dimension $r$.
We will prove the lemma by induction on the maximum $e$ of the integers
$$
\dim(Z \cap T_i)
$$
The base case is $e = r + s - \dim(X)$. In this case $Z$ intersects
$\beta$ properly and the lemma is trivial.

\medskip\noindent
Induction step. Assume that $e > r + s - \dim(X)$. Choose an embedding
$X \subset \mathbf{P}^N$ and apply Lemma \ref{lemma-moving} to find a
closed subvariety $C \subset \mathbf{P}^N$ such that
$C \cdot X = [Z] + \sum m_j[Z_j]$ and such that the induction
hypothesis applies to each $Z_j$. Next, apply Lemma \ref{lemma-move}
to $C$, $X$, $T_i$ to find $C' \subset \mathbf{P}^N \times \mathbf{P}^1$.
Let $\gamma = C' \cdot X \times \mathbf{P}^1$ viewed as a cycle
on $X \times \mathbf{P}^1$. By Lemma \ref{lemma-transfer} we have
$$
[Z] + \sum m_j[Z_j] = \text{pr}_{X, *}(\gamma \cdot X \times 0)
$$
On the other hand the cycle
$\gamma_\infty = \text{pr}_{X, *}(\gamma \cdot X \times \infty)$
is supported on $C'_\infty \cap X$ hence intersects $\beta$ transversally.
Thus we see that $[Z] \sim_{rat} - \sum m_j[Z_j] + \gamma_\infty$
by Lemma \ref{lemma-rational-equivalence-and-intersection}. Since by
induction each $[Z_j]$ is rationally equivalent to a cycle which properly
intersects $\beta$ this finishes the proof.
\end{proof}



\section{Intersection products and rational equivalence}
\label{section-intersections-and-rational-equivalence}

\noindent
With definitions as above we show that the intersection product
is well defined modulo rational equivalence. We first deal with a
special case.

\begin{lemma}
\label{lemma-well-defined-special-case}
Let $X$ be a nonsingular variety. Let
$W \subset X \times \mathbf{P}^1$ be an $(s + 1)$-dimensional subvariety
dominating $\mathbf{P}^1$. Let $W_a$, resp.\ $W_b$ be the fibre of
$W \to \mathbf{P}^1$ over $a$, resp.\ $b$. Let $V$ be a $r$-dimensional
subvariety of $X$ such that $V$ intersects both $W_a$ and
$W_b$ properly. Then $[V] \cdot [W_a]_r \sim_{rat} [V] \cdot [W_b]_r$.
\end{lemma}

\begin{proof}
We have $[W_a]_r = \text{pr}_{X,*}(W \cdot X \times a)$ and similarly for
$[W_b]_r$, see Lemma \ref{lemma-rational-equivalence-and-intersection}.
Thus we reduce to showing
$$
V \cdot \text{pr}_{X,*}( W \cdot X \times a) \sim_{rat} V \cdot
\text{pr}_{X,*}( W \cdot X\times b).
$$
Applying the projection formula
Lemma \ref{lemma-projection-formula-flat} we get
$$
V \cdot \text{pr}_{X,*}( W \cdot X\times a) =
\text{pr}_{X,*}(V \times \mathbf{P}^1 \cdot (W \cdot X\times a))
$$
and similarly for $b$. Thus we reduce to showing
$$
\text{pr}_{X,*}(V \times \mathbf{P}^1 \cdot (W \cdot X\times a))
\sim_{rat}
\text{pr}_{X,*}(V \times \mathbf{P}^1 \cdot (W \cdot X\times b))
$$
If $V \times \mathbf{P}^1$ intersects $W$ properly, then
associativity for the intersection multiplicities
(Lemma \ref{lemma-associative})
gives $V \times \mathbf{P}^1 \cdot (W \cdot X\times a) =
(V \times \mathbf{P}^1 \cdot W) \cdot X \times a$
and similarly for $b$.  Thus we reduce to showing
$$
\text{pr}_{X,*}((V \times \mathbf{P}^1 \cdot W) \cdot X\times a)
\sim_{rat}
\text{pr}_{X,*}((V \times \mathbf{P}^1 \cdot W) \cdot X\times b)
$$
which is true by Lemma \ref{lemma-rational-equivalence-and-intersection}.

\medskip\noindent
The argument above does not quite work. The obstruction is that
we do not know that $V \times \mathbf{P}^1$ and $W$ intersect properly.
We only know that $V$ and $W_a$ and $V$ and $W_b$ intersect properly.
Let $Z_i$, $i \in I$ be the irreducible components of
$V \times \mathbf{P}^1 \cap W$. Then we know that
$\dim(Z_i) \geq r + 1 + s + 1 - n - 1 = r + s + 1 - n$ where $n = \dim(X)$, see
Lemma \ref{lemma-intersect-in-smooth}. Since we have assumed
that $V$ and $W_a$ intersect properly, we see that
$\dim(Z_{i, a}) = r + s - n$ or $Z_{i, a} = \emptyset$.
On the other hand, if $Z_{i, a} \not = \emptyset$, then
$\dim(Z_{i, a}) \geq \dim(Z_i) - 1 = r + s - n$.
It follows that $\dim(Z_i) = r + s + 1 - n$ if $Z_i$ meets $X \times a$
and in this case $Z_i \to \mathbf{P}^1$ is surjective.
Thus we may write $I = I' \amalg I''$ where $I'$ is the set of $i \in I$
such that $Z_i \to \mathbf{P}^1$ is surjective and $I''$ is the set of
$i \in I$ such that $Z_i$ lies over a closed point $t_i \in \mathbf{P}^1$
with $t_i \not = a$ and $t_i \not = b$. Consider the cycle
$$
\gamma = \sum\nolimits_{i \in I'} e_i [Z_i]
$$
where we take
$$
e_i = \sum\nolimits_p (-1)^p
\text{length}_{\mathcal{O}_{X \times \mathbf{P}^1, Z_i}}
\text{Tor}_p^{\mathcal{O}_{X \times \mathbf{P}^1, Z_i}}(
\mathcal{O}_{V \times \mathbf{P}^1, Z_i}, \mathcal{O}_{W, Z_i})
$$
We will show that $\gamma$ can be used as a replacement for
the intersection product of $V \times \mathbf{P}^1$ and $W$.

\medskip\noindent
We will show this using associativity of intersection products in exactly
the same way as above. Let $U = \mathbf{P}^1 \setminus \{t_i, i \in I''\}$.
Note that $X \times a$ and $X \times b$ are contained in $X \times U$.
The subvarieties
$$
V \times U,\quad W_U,\quad X \times a\quad\text{of}\quad X \times U
$$
intersect transversally pairwise by our choice of $U$ and moreover
$\dim(V \times U \cap W_U \cap X \times a) = \dim(V \cap W_a)$ has
the expected dimension. Thus we see that
$$
V \times U \cdot (W_U \cdot X \times a) =
(V \times U \cdot W_U) \cdot X \times a
$$
as cycles on $X \times U$ by Lemma \ref{lemma-associative}.
By construction $\gamma$ restricts to the cycle $V \times U \cdot W_U$
on $X \times U$. Trivially,
$V \times \mathbf{P}^1 \cdot (W \times X \times a)$ restricts
to $V \times U \cdot (W_U \cdot X \times a)$ on $X \times U$.
Hence
$$
V \times \mathbf{P}^1 \cdot (W \cdot X \times a) =
\gamma \cdot X \times a
$$
as cycles on $X \times \mathbf{P}^1$ (because both sides
are contained in $X \times U$ and are equal after restricting
to $X \times U$ by what was said before). Since we have the same for $b$
we conclude
\begin{align*}
V \cdot [W_a]
& =
\text{pr}_{X,*}(V \times \mathbf{P}^1 \cdot (W \cdot X\times a)) \\
& =
\text{pr}_{X, *}(\gamma \cdot X \times a) \\
& \sim_{rat} 
\text{pr}_{X, *}(\gamma \cdot X \times b) \\
& =
\text{pr}_{X,*}(V \times \mathbf{P}^1 \cdot (W \cdot X\times b)) \\
& =
V \cdot [W_b]
\end{align*}
The first and the last equality by the first paragraph of the proof,
the second and penultimate equalities were shown in this paragraph, and
the middle equivalence is
Lemma \ref{lemma-rational-equivalence-and-intersection}.
\end{proof}

\begin{theorem}
\label{theorem-well-defined}
Let $X$ be a nonsingular projective variety. Let $\alpha$, resp.\ $\beta$
be an $r$, resp.\ $s$ cycle on $X$. Assume that $\alpha$ and $\beta$
intersect properly so that $\alpha \cdot \beta$ is defined. Finally,
assume that $\alpha \sim_{rat} 0$. Then $\alpha \cdot \beta \sim_{rat} 0$.
\end{theorem}

\begin{proof}
Pick a closed immersion $X \subset \mathbf{P}^N$.
By linearity it suffices to prove the result when $\beta = [Z]$ for some
$s$-dimensional closed subvariety $Z \subset X$ which intersects $\alpha$
properly. The condition $\alpha \sim_{rat} 0$ means there
are finitely many $(r + 1)$-dimensional closed subvarieties
$W_i \subset X \times \mathbf{P}^1$ such that
$$
\alpha = \sum [W_{i, a_i}]_r - [W_{i, b_i}]_r
$$
for some pairs of points $a_i, b_i$ of $\mathbf{P}^1$.
Let $W_{i, a_i}^t$ and $W_{i, b_i}^t$ be the irreducible components
of $W_{i, a_i}$ and $W_{i, b_i}$.
We will use induction on the maximum $d$ of the integers
$$
\dim(Z \cap W_{i, a_i}^t),\quad \dim(Z \cap W_{i, b_i}^t)
$$
The main problem in the rest of the proof is that although we know that $Z$
intersects $\alpha$ properly, it may not be the case that $Z$ intersects the
``intermediate'' varieties $W_{i, a_i}^t$
and $W_{i, b_i}^t$ properly, i.e., it may happen that $d >  r + s - \dim(X)$.

\medskip\noindent
Base case: $d = r + s - \dim(X)$. In this case all the intersections of
$Z$ with the $W_{i, a_i}^t$ and $W_{i, b_i}^t$ are proper and the
desired result follows from Lemma \ref{lemma-well-defined-special-case},
because it applies to show that
$[Z] \cdot [W_{i, a_i}]_r \sim_{rat} [Z] \cdot [W_{i, b_i}]_r$ for
each $i$.

\medskip\noindent
Induction step: $d >  r + s - \dim(X)$. Apply
Lemma \ref{lemma-moving} to $Z \subset X$ and
the family of subvarieties $\{W_{i, a_i}^t, W_{i, b_i}^t\}$. Then we find a
closed subvariety $C \subset \mathbf{P}^N$ intersecting $X$
properly such that
$$
C \cdot X = [Z] + \sum m_j [Z_j]
$$
and such that
$$
\dim(Z_j \cap W_{i, a_i}^t) \leq \dim(Z \cap W_{i, a_i}^t),\quad
\dim(Z_j \cap W_{i, b_i}^t) \leq \dim(Z \cap W_{i, b_i}^t)
$$
with strict inequality if the right hand side is $> r + s - \dim(X)$.
This implies two things: (a) the induction hypothesis applies to
each $Z_j$, and (b) $C \cdot X$ and $\alpha$ intersect properly (because
$\alpha$ is a linear combination of those $[W_{i, a_i}^t]$ and
$[W_{i, a_i}^t]$ which intersect $Z$ properly).
Next, pick $C' \subset \mathbf{P}^N \times \mathbf{P}^1$
as in Lemma \ref{lemma-move} with respect to $C$, $X$, and
$W_{i, a_i}^t$, $W_{i, b_i}^t$.
Write $C' \cdot X \times \mathbf{P}^1 = \sum n_k [E_k]$ for
some subvarieties $E_k \subset X \times \mathbf{P}^1$ of
dimension $s + 1$. Note that $n_k > 0$ for all $k$ by
Proposition \ref{proposition-positivity}.
By Lemma \ref{lemma-transfer} we have
$$
[Z] + \sum m_j [Z_j] = \sum n_k[E_{k, 0}]_s
$$
Since $E_{k, 0} \subset C \cap X$ we see that $[E_{k, 0}]_s$ and $\alpha$
intersect properly. On the other hand, the cycle
$$
\gamma = \sum n_k[E_{k, \infty}]_s
$$
is supported on $C'_\infty \cap X$ and hence
properly intersects each $W_{i, a_i}^t$, $W_{i, b_i}^t$.
Thus by the base case and linearity, we see that
$$
\gamma \cdot \alpha \sim_{rat} 0
$$
As we have seen that $E_{k, 0}$ and $E_{k, \infty}$
intersect $\alpha$ properly Lemma \ref{lemma-well-defined-special-case}
applied to $E_k \subset X \times \mathbf{P}^1$ and $\alpha$ gives
$$
[E_{k, 0}] \cdot \alpha \sim_{rat} [E_{k, \infty}] \cdot \alpha
$$
Putting everything together we have
\begin{align*}
[Z] \cdot \alpha
& =
(\sum n_k[E_{k, 0}]_r - \sum m_j[Z_j]) \cdot \alpha \\
& \sim_{rat}
\sum n_k [E_{k, 0}] \cdot \alpha \quad (\text{by induction hypothesis})\\
& \sim_{rat}
\sum n_k [E_{k, \infty}] \cdot \alpha \quad (\text{by the lemma})\\
& =
\gamma \cdot \alpha \\
& \sim_{rat}
0 \quad (\text{by base case})
\end{align*}
This finishes the proof.
\end{proof}

\begin{remark}
\label{remark-quasi-projective}
Lemma \ref{lemma-moving-move} and
Theorem \ref{theorem-well-defined}
also hold for nonsingular quasi-projective varieties with
the same proof. The only change is that one needs to prove the following
version of the moving Lemma \ref{lemma-moving}: Let  $X \subset \mathbf{P}^N$
be a closed subvariety. Let $n = \dim(X)$ and $0 \leq d, d' < n$. Let
$X^{reg} \subset X$ be the open subset of nonsingular points. Let
$Z \subset X^{reg}$ be a closed subvariety of dimension $d$ and
$T_i \subset X^{reg}$, $i \in I$ be a finite collection of closed subvarieties
of dimension $d'$. Then there exists a subvariety $C \subset \mathbf{P}^N$
such that $C$ intersects $X$ properly and such that
$$
(C \cdot X)|_{X^{reg}} = Z + \sum\nolimits_{j \in J} m_j Z_j
$$
where $Z_j \subset X^{reg}$ are irreducible of dimension $d$, distinct
from $Z$, and
$$
\dim(Z_j \cap T_i) \leq \dim(Z \cap T_i)
$$
with strict inequality if $Z$ does not intersect $T_i$ properly in $X^{reg}$.
\end{remark}



\section{Chow rings}
\label{section-chow-rings}

\noindent
Let $X$ be a nonsingular projective variety. We define the intersection
product
$$
A_r(X) \times A_s(X) \longrightarrow A_{r + s - \dim(X)}(X),\quad
(\alpha, \beta) \longmapsto \alpha \cdot \beta
$$
as follows. Let $\alpha \in Z_r(X)$ and $\beta \in Z_s(X)$.
If $\alpha$ and $\beta$ intersect properly, we use the
definition given in Section \ref{section-intersection-product}.
If not, then we choose $\alpha \sim_{rat} \alpha'$ as in
Lemma \ref{lemma-moving-move} and we set
$$
\alpha \cdot \beta =
\text{class of }\alpha' \cdot \beta \in A_{r + s - \dim(X)}(X)
$$
This is well defined and passes through rational equivalence by
Theorem \ref{theorem-well-defined}. The intersection product
on $A_*(X)$ is commutative (this is clear), associative
(Lemma \ref{lemma-associative}) and has a unit $[X] \in A_{\dim(X)}(X)$.

\medskip\noindent
Often it is convenient to use $A^c(X) = A_{\dim X - c}(X)$ to denote the
group of codimension $c$ cycles modulo rational equivalence.
The intersection product defines a product
$$
A^k(X) \times A^l(X) \longrightarrow A^{k+l}(X)
$$
which is commutative, associative, and has a unit $1 = [X] \in A^0(X)$.


\section{Pullback for a general morphism}
\label{section-general-pullback}

\noindent
Let $f : X \to Y$ be a morphism of nonsingular projective varieties.
We define
$$
f^* : A_k(Y) \to A_{k+\dim X - \dim Y}(X)
$$
by the rule
$$
f^*(\alpha) = pr_{X, *}(\Gamma_f \cdot pr_Y^*(\alpha))
$$
where $\Gamma_f \subset X\times Y$ is the graph of $f$. Note that in this
generality, it is defined only on cycle classes and not on cycles. With the
notation $A^*$ introduced in Section \ref{section-chow-rings}
we may think of pullback as a map
$$
f^* : A^*(Y) \to A^*(X)
$$
in other words, it is a map of graded abelian groups.

\begin{lemma}
\label{lemma-pullback-and-intersection-product}
Let $f : X \to Y$ be a morphism of nonsingular projective varieties.
The pullback map on chow groups satisfies:
\begin{enumerate}
\item $f^* : A^*(Y) \to A^*(X)$ is a ring map,
\item $(g \circ f)^* = f^* \circ g^*$ for a composable pair $f, g$,
\item the projection formula holds: $f_*(\alpha) \cdot \beta =
f_*( \alpha \cdot f^*\beta)$, and
\item if $f$ is flat then it agrees with the previous definition.
\end{enumerate}
\end{lemma}

\begin{proof}
All of these follow readily from the results above.

\medskip\noindent
For (1) it suffices to show that
$\text{pr}_{X,*}( \Gamma_f \cdot \alpha \cdot \beta) =
\text{pr}_{X,*}(\Gamma_f \cdot \alpha) \cdot
\text{pr}_{X,*}(\Gamma_f \cdot \beta)$
for cycles $\alpha$, $\beta$ on $X \times Y$. If $\alpha$ is a cycle on
$X \times Y$ which intersects $\Gamma_f$ properly, then it is easy
to see that
$$
\Gamma_f \cdot \alpha =
\Gamma_f \cdot \text{pr}_X^*(\text{pr}_{X,*}(\Gamma_f \cdot \alpha))
$$
as cycles because $\Gamma_f$ is a graph. Thus we get the first
equality in
\begin{align*}
\text{pr}_{X,*}(\Gamma_f \cdot \alpha \cdot \beta)
& =
\text{pr}_{X,*}(
\Gamma_f \cdot
\text{pr}_X^*(\text{pr}_{X,*}(\Gamma_f \cdot \alpha)) \cdot \beta) \\
& =
\text{pr}_{X,*}(\text{pr}_X^*(\text{pr}_{X,*}(\Gamma_f \cdot \alpha))
\cdot (\Gamma_f \cdot \beta)) \\
& =
\text{pr}_{X,*}(\Gamma_f \cdot \alpha) \cdot
\text{pr}_{X,*}(\Gamma_f \cdot \beta)
\end{align*}
the last step by the projection formula in the flat case
(Lemma \ref{lemma-projection-formula-flat}).

\medskip\noindent
If $g : Y \to Z$ then property (2) follows formally from the observation that
$$
\Gamma =
\text{pr}_{X \times Y}^*\Gamma_f \cdot
\text{pr}_{Y \times Z}^*\Gamma_g
$$
in $Z_*(X \times Y \times Z)$ where $\Gamma = \{(x, f(x), g(f(x))\}$
and maps isomorphically to $\Gamma_{g \circ f}$ in $X \times Z$.
The equality follows from the scheme theoretic equality and
Lemma \ref{lemma-transversal}.

\medskip\noindent
For (3) we use the projection formula for flat maps twice
\begin{align*}
f_*(\alpha \cdot pr_{X, *}(\Gamma_f \cdot pr_Y^*(\beta)))
& =
f_*(pr_{X, *}(pr_X^*\alpha \cdot \Gamma_f \cdot pr_Y^*(\beta))) \\
& =
pr_{Y, *}(pr_X^*\alpha \cdot \Gamma_f \cdot pr_Y^*(\beta))) \\
& =
pt_{Y, *}(pr_X^*\alpha \cdot \Gamma_f) \cdot \beta \\
& =
f_*(\alpha) \cdot \beta
\end{align*}
where in the last equality we use the remark on graphs made above.
This proves (3).

\medskip\noindent
Property (4) rests on identifying the intersection product
$\Gamma_f \cdot pr_Y^*\alpha$ in the case $f$ is flat. Namely, in this
case if $V \subset Y$ is a closed subvariety, then every generic point
$\xi$ of the scheme $f^{-1}(V) \cong \Gamma_f \cap pr_Y^{-1}(V)$
lies over the generic point of $V$. Hence the local ring of
$pr_Y^{-1}(V) = X \times V$ at $\xi$ is Cohen-Macaulay. Since
$\Gamma_f \subset X \times Y$ is a regular immersion (as a morphism of
smooth projective varieties) we find that
$$
\Gamma_f \cdot pr_Y^*[V] = [\Gamma_f \cap pr_Y^{-1}(V)]_d
$$
with $d$ the dimension of $\Gamma_f \cap pr_Y^{-1}(V)$, see
Lemma \ref{lemma-multiplicity-lci-CM}. Since $\Gamma_f \cap pr_Y^{-1}(V)$
maps isomorphically to $f^{-1}(V)$ we conclude.
\end{proof}


\section{Pullback of cycles}
\label{section-pullback-cycles}

\noindent
Suppose that $X$ and $Y$ be nonsingular
projective varieties, and let $f : X \to Y$ be a morphism.
Suppose that $Z \subset Y$ is a closed subvariety. Let $f^{-1}(Z)$
be the scheme theoretic inverse image:
$$
\xymatrix{
f^{-1}(Z) \ar[r] \ar[d] & Z \ar[d] \\
X \ar[r] & Y
}
$$
is a fibre product diagram of schemes. In particular $f^{-1}(Z) \subset X$
is a closed subscheme of $X$. In this case we always have
$$
\dim f^{-1}(Z) \geq \dim Z + \dim X - \dim Y.
$$
If equality holds in the formula above, then
$f^*[Z] = [f^{-1}(Z)]_{\dim Z + \dim X - \dim Y}$
provided that the scheme $Z$ is Cohen-Macaulay at the images
of the generic points of $f^{-1}(Z)$. This follows by identifying
$f^{-1}(Z)$ with the scheme theoretic intersection of $\Gamma_f$
and $X \times Z$ and using Lemma \ref{lemma-multiplicity-lci-CM}.
Details are similar to the proof of part (4) of
Lemma \ref{lemma-pullback-and-intersection-product} above.





\begin{multicols}{2}[\section{Other chapters}]
\noindent
Preliminaries
\begin{enumerate}
\item \hyperref[introduction-section-phantom]{Introduction}
\item \hyperref[conventions-section-phantom]{Conventions}
\item \hyperref[sets-section-phantom]{Set Theory}
\item \hyperref[categories-section-phantom]{Categories}
\item \hyperref[topology-section-phantom]{Topology}
\item \hyperref[sheaves-section-phantom]{Sheaves on Spaces}
\item \hyperref[sites-section-phantom]{Sites and Sheaves}
\item \hyperref[stacks-section-phantom]{Stacks}
\item \hyperref[fields-section-phantom]{Fields}
\item \hyperref[algebra-section-phantom]{Commutative Algebra}
\item \hyperref[brauer-section-phantom]{Brauer Groups}
\item \hyperref[homology-section-phantom]{Homological Algebra}
\item \hyperref[derived-section-phantom]{Derived Categories}
\item \hyperref[simplicial-section-phantom]{Simplicial Methods}
\item \hyperref[more-algebra-section-phantom]{More on Algebra}
\item \hyperref[smoothing-section-phantom]{Smoothing Ring Maps}
\item \hyperref[modules-section-phantom]{Sheaves of Modules}
\item \hyperref[sites-modules-section-phantom]{Modules on Sites}
\item \hyperref[injectives-section-phantom]{Injectives}
\item \hyperref[cohomology-section-phantom]{Cohomology of Sheaves}
\item \hyperref[sites-cohomology-section-phantom]{Cohomology on Sites}
\item \hyperref[dga-section-phantom]{Differential Graded Algebra}
\item \hyperref[dpa-section-phantom]{Divided Power Algebra}
\item \hyperref[hypercovering-section-phantom]{Hypercoverings}
\end{enumerate}
Schemes
\begin{enumerate}
\setcounter{enumi}{24}
\item \hyperref[schemes-section-phantom]{Schemes}
\item \hyperref[constructions-section-phantom]{Constructions of Schemes}
\item \hyperref[properties-section-phantom]{Properties of Schemes}
\item \hyperref[morphisms-section-phantom]{Morphisms of Schemes}
\item \hyperref[coherent-section-phantom]{Cohomology of Schemes}
\item \hyperref[divisors-section-phantom]{Divisors}
\item \hyperref[limits-section-phantom]{Limits of Schemes}
\item \hyperref[varieties-section-phantom]{Varieties}
\item \hyperref[topologies-section-phantom]{Topologies on Schemes}
\item \hyperref[descent-section-phantom]{Descent}
\item \hyperref[perfect-section-phantom]{Derived Categories of Schemes}
\item \hyperref[more-morphisms-section-phantom]{More on Morphisms}
\item \hyperref[flat-section-phantom]{More on Flatness}
\item \hyperref[groupoids-section-phantom]{Groupoid Schemes}
\item \hyperref[more-groupoids-section-phantom]{More on Groupoid Schemes}
\item \hyperref[etale-section-phantom]{\'Etale Morphisms of Schemes}
\end{enumerate}
Topics in Scheme Theory
\begin{enumerate}
\setcounter{enumi}{40}
\item \hyperref[chow-section-phantom]{Chow Homology}
\item \hyperref[intersection-section-phantom]{Intersection Theory}
\item \hyperref[weil-section-phantom]{Weil Cohomology Theories}
\item \hyperref[pic-section-phantom]{Picard Schemes of Curves}
\item \hyperref[adequate-section-phantom]{Adequate Modules}
\item \hyperref[dualizing-section-phantom]{Dualizing Complexes}
\item \hyperref[duality-section-phantom]{Duality for Schemes}
\item \hyperref[discriminant-section-phantom]{Discriminants and Differents}
\item \hyperref[local-cohomology-section-phantom]{Local Cohomology}
\item \hyperref[algebraization-section-phantom]{Algebraic and Formal Geometry}
\item \hyperref[curves-section-phantom]{Algebraic Curves}
\item \hyperref[resolve-section-phantom]{Resolution of Surfaces}
\item \hyperref[models-section-phantom]{Semistable Reduction}
\item \hyperref[pione-section-phantom]{Fundamental Groups of Schemes}
\item \hyperref[etale-cohomology-section-phantom]{\'Etale Cohomology}
\item \hyperref[crystalline-section-phantom]{Crystalline Cohomology}
\item \hyperref[proetale-section-phantom]{Pro-\'etale Cohomology}
\item \hyperref[more-etale-section-phantom]{More \'Etale Cohomology}
\item \hyperref[trace-section-phantom]{The Trace Formula}
\end{enumerate}
Algebraic Spaces
\begin{enumerate}
\setcounter{enumi}{59}
\item \hyperref[spaces-section-phantom]{Algebraic Spaces}
\item \hyperref[spaces-properties-section-phantom]{Properties of Algebraic Spaces}
\item \hyperref[spaces-morphisms-section-phantom]{Morphisms of Algebraic Spaces}
\item \hyperref[decent-spaces-section-phantom]{Decent Algebraic Spaces}
\item \hyperref[spaces-cohomology-section-phantom]{Cohomology of Algebraic Spaces}
\item \hyperref[spaces-limits-section-phantom]{Limits of Algebraic Spaces}
\item \hyperref[spaces-divisors-section-phantom]{Divisors on Algebraic Spaces}
\item \hyperref[spaces-over-fields-section-phantom]{Algebraic Spaces over Fields}
\item \hyperref[spaces-topologies-section-phantom]{Topologies on Algebraic Spaces}
\item \hyperref[spaces-descent-section-phantom]{Descent and Algebraic Spaces}
\item \hyperref[spaces-perfect-section-phantom]{Derived Categories of Spaces}
\item \hyperref[spaces-more-morphisms-section-phantom]{More on Morphisms of Spaces}
\item \hyperref[spaces-flat-section-phantom]{Flatness on Algebraic Spaces}
\item \hyperref[spaces-groupoids-section-phantom]{Groupoids in Algebraic Spaces}
\item \hyperref[spaces-more-groupoids-section-phantom]{More on Groupoids in Spaces}
\item \hyperref[bootstrap-section-phantom]{Bootstrap}
\item \hyperref[spaces-pushouts-section-phantom]{Pushouts of Algebraic Spaces}
\end{enumerate}
Topics in Geometry
\begin{enumerate}
\setcounter{enumi}{76}
\item \hyperref[spaces-chow-section-phantom]{Chow Groups of Spaces}
\item \hyperref[groupoids-quotients-section-phantom]{Quotients of Groupoids}
\item \hyperref[spaces-more-cohomology-section-phantom]{More on Cohomology of Spaces}
\item \hyperref[spaces-simplicial-section-phantom]{Simplicial Spaces}
\item \hyperref[spaces-duality-section-phantom]{Duality for Spaces}
\item \hyperref[formal-spaces-section-phantom]{Formal Algebraic Spaces}
\item \hyperref[restricted-section-phantom]{Restricted Power Series}
\item \hyperref[spaces-resolve-section-phantom]{Resolution of Surfaces Revisited}
\end{enumerate}
Deformation Theory
\begin{enumerate}
\setcounter{enumi}{84}
\item \hyperref[formal-defos-section-phantom]{Formal Deformation Theory}
\item \hyperref[defos-section-phantom]{Deformation Theory}
\item \hyperref[cotangent-section-phantom]{The Cotangent Complex}
\item \hyperref[examples-defos-section-phantom]{Deformation Problems}
\end{enumerate}
Algebraic Stacks
\begin{enumerate}
\setcounter{enumi}{88}
\item \hyperref[algebraic-section-phantom]{Algebraic Stacks}
\item \hyperref[examples-stacks-section-phantom]{Examples of Stacks}
\item \hyperref[stacks-sheaves-section-phantom]{Sheaves on Algebraic Stacks}
\item \hyperref[criteria-section-phantom]{Criteria for Representability}
\item \hyperref[artin-section-phantom]{Artin's Axioms}
\item \hyperref[quot-section-phantom]{Quot and Hilbert Spaces}
\item \hyperref[stacks-properties-section-phantom]{Properties of Algebraic Stacks}
\item \hyperref[stacks-morphisms-section-phantom]{Morphisms of Algebraic Stacks}
\item \hyperref[stacks-limits-section-phantom]{Limits of Algebraic Stacks}
\item \hyperref[stacks-cohomology-section-phantom]{Cohomology of Algebraic Stacks}
\item \hyperref[stacks-perfect-section-phantom]{Derived Categories of Stacks}
\item \hyperref[stacks-introduction-section-phantom]{Introducing Algebraic Stacks}
\item \hyperref[stacks-more-morphisms-section-phantom]{More on Morphisms of Stacks}
\item \hyperref[stacks-geometry-section-phantom]{The Geometry of Stacks}
\end{enumerate}
Topics in Moduli Theory
\begin{enumerate}
\setcounter{enumi}{102}
\item \hyperref[moduli-section-phantom]{Moduli Stacks}
\item \hyperref[moduli-curves-section-phantom]{Moduli of Curves}
\end{enumerate}
Miscellany
\begin{enumerate}
\setcounter{enumi}{104}
\item \hyperref[examples-section-phantom]{Examples}
\item \hyperref[exercises-section-phantom]{Exercises}
\item \hyperref[guide-section-phantom]{Guide to Literature}
\item \hyperref[desirables-section-phantom]{Desirables}
\item \hyperref[coding-section-phantom]{Coding Style}
\item \hyperref[obsolete-section-phantom]{Obsolete}
\item \hyperref[fdl-section-phantom]{GNU Free Documentation License}
\item \hyperref[index-section-phantom]{Auto Generated Index}
\end{enumerate}
\end{multicols}


\bibliography{my}
\bibliographystyle{amsalpha}

\end{document}
