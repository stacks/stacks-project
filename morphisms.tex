\IfFileExists{stacks-project.cls}{%
\documentclass{stacks-project}
}{%
\documentclass{amsart}
}

% The following AMS packages are automatically loaded with
% the amsart documentclass:
%\usepackage{amsmath}
%\usepackage{amssymb}
%\usepackage{amsthm}

% For dealing with references we use the comment environment
\usepackage{verbatim}
\newenvironment{reference}{\comment}{\endcomment}
%\newenvironment{reference}{}{}
\newenvironment{slogan}{\comment}{\endcomment}
\newenvironment{history}{\comment}{\endcomment}

% For commutative diagrams you can use
% \usepackage{amscd}
\usepackage[all]{xy}

% We use 2cell for 2-commutative diagrams.
\xyoption{2cell}
\UseAllTwocells

% To put source file link in headers.
% Change "template.tex" to "this_filename.tex"
% \usepackage{fancyhdr}
% \pagestyle{fancy}
% \lhead{}
% \chead{}
% \rhead{Source file: \url{template.tex}}
% \lfoot{}
% \cfoot{\thepage}
% \rfoot{}
% \renewcommand{\headrulewidth}{0pt}
% \renewcommand{\footrulewidth}{0pt}
% \renewcommand{\headheight}{12pt}

\usepackage{multicol}

% For cross-file-references
\usepackage{xr-hyper}

% Package for hypertext links:
\usepackage{hyperref}

% For any local file, say "hello.tex" you want to link to please
% use \externaldocument[hello-]{hello}
\externaldocument[introduction-]{introduction}
\externaldocument[conventions-]{conventions}
\externaldocument[sets-]{sets}
\externaldocument[categories-]{categories}
\externaldocument[topology-]{topology}
\externaldocument[sheaves-]{sheaves}
\externaldocument[sites-]{sites}
\externaldocument[stacks-]{stacks}
\externaldocument[fields-]{fields}
\externaldocument[algebra-]{algebra}
\externaldocument[brauer-]{brauer}
\externaldocument[homology-]{homology}
\externaldocument[derived-]{derived}
\externaldocument[simplicial-]{simplicial}
\externaldocument[more-algebra-]{more-algebra}
\externaldocument[smoothing-]{smoothing}
\externaldocument[modules-]{modules}
\externaldocument[sites-modules-]{sites-modules}
\externaldocument[injectives-]{injectives}
\externaldocument[cohomology-]{cohomology}
\externaldocument[sites-cohomology-]{sites-cohomology}
\externaldocument[dga-]{dga}
\externaldocument[dpa-]{dpa}
\externaldocument[hypercovering-]{hypercovering}
\externaldocument[schemes-]{schemes}
\externaldocument[constructions-]{constructions}
\externaldocument[properties-]{properties}
\externaldocument[morphisms-]{morphisms}
\externaldocument[coherent-]{coherent}
\externaldocument[divisors-]{divisors}
\externaldocument[limits-]{limits}
\externaldocument[varieties-]{varieties}
\externaldocument[topologies-]{topologies}
\externaldocument[descent-]{descent}
\externaldocument[perfect-]{perfect}
\externaldocument[more-morphisms-]{more-morphisms}
\externaldocument[flat-]{flat}
\externaldocument[groupoids-]{groupoids}
\externaldocument[more-groupoids-]{more-groupoids}
\externaldocument[etale-]{etale}
\externaldocument[chow-]{chow}
\externaldocument[intersection-]{intersection}
\externaldocument[pic-]{pic}
\externaldocument[adequate-]{adequate}
\externaldocument[dualizing-]{dualizing}
\externaldocument[duality-]{duality}
\externaldocument[discriminant-]{discriminant}
\externaldocument[local-cohomology-]{local-cohomology}
\externaldocument[curves-]{curves}
\externaldocument[resolve-]{resolve}
\externaldocument[models-]{models}
\externaldocument[pione-]{pione}
\externaldocument[etale-cohomology-]{etale-cohomology}
\externaldocument[proetale-]{proetale}
\externaldocument[crystalline-]{crystalline}
\externaldocument[spaces-]{spaces}
\externaldocument[spaces-properties-]{spaces-properties}
\externaldocument[spaces-morphisms-]{spaces-morphisms}
\externaldocument[decent-spaces-]{decent-spaces}
\externaldocument[spaces-cohomology-]{spaces-cohomology}
\externaldocument[spaces-limits-]{spaces-limits}
\externaldocument[spaces-divisors-]{spaces-divisors}
\externaldocument[spaces-over-fields-]{spaces-over-fields}
\externaldocument[spaces-topologies-]{spaces-topologies}
\externaldocument[spaces-descent-]{spaces-descent}
\externaldocument[spaces-perfect-]{spaces-perfect}
\externaldocument[spaces-more-morphisms-]{spaces-more-morphisms}
\externaldocument[spaces-flat-]{spaces-flat}
\externaldocument[spaces-groupoids-]{spaces-groupoids}
\externaldocument[spaces-more-groupoids-]{spaces-more-groupoids}
\externaldocument[bootstrap-]{bootstrap}
\externaldocument[spaces-pushouts-]{spaces-pushouts}
\externaldocument[groupoids-quotients-]{groupoids-quotients}
\externaldocument[spaces-more-cohomology-]{spaces-more-cohomology}
\externaldocument[spaces-simplicial-]{spaces-simplicial}
\externaldocument[formal-spaces-]{formal-spaces}
\externaldocument[restricted-]{restricted}
\externaldocument[spaces-resolve-]{spaces-resolve}
\externaldocument[formal-defos-]{formal-defos}
\externaldocument[defos-]{defos}
\externaldocument[cotangent-]{cotangent}
\externaldocument[examples-defos-]{examples-defos}
\externaldocument[algebraic-]{algebraic}
\externaldocument[examples-stacks-]{examples-stacks}
\externaldocument[stacks-sheaves-]{stacks-sheaves}
\externaldocument[criteria-]{criteria}
\externaldocument[artin-]{artin}
\externaldocument[quot-]{quot}
\externaldocument[stacks-properties-]{stacks-properties}
\externaldocument[stacks-morphisms-]{stacks-morphisms}
\externaldocument[stacks-limits-]{stacks-limits}
\externaldocument[stacks-cohomology-]{stacks-cohomology}
\externaldocument[stacks-perfect-]{stacks-perfect}
\externaldocument[stacks-introduction-]{stacks-introduction}
\externaldocument[stacks-more-morphisms-]{stacks-more-morphisms}
\externaldocument[stacks-geometry-]{stacks-geometry}
\externaldocument[moduli-]{moduli}
\externaldocument[moduli-curves-]{moduli-curves}
\externaldocument[examples-]{examples}
\externaldocument[exercises-]{exercises}
\externaldocument[guide-]{guide}
\externaldocument[desirables-]{desirables}
\externaldocument[coding-]{coding}
\externaldocument[obsolete-]{obsolete}
\externaldocument[fdl-]{fdl}
\externaldocument[index-]{index}

% Theorem environments.
%
\theoremstyle{plain}
\newtheorem{theorem}[subsection]{Theorem}
\newtheorem{proposition}[subsection]{Proposition}
\newtheorem{lemma}[subsection]{Lemma}

\theoremstyle{definition}
\newtheorem{definition}[subsection]{Definition}
\newtheorem{example}[subsection]{Example}
\newtheorem{exercise}[subsection]{Exercise}
\newtheorem{situation}[subsection]{Situation}

\theoremstyle{remark}
\newtheorem{remark}[subsection]{Remark}
\newtheorem{remarks}[subsection]{Remarks}

\numberwithin{equation}{subsection}

% Macros
%
\def\lim{\mathop{\rm lim}\nolimits}
\def\colim{\mathop{\rm colim}\nolimits}
\def\Spec{\mathop{\rm Spec}}
\def\Hom{\mathop{\rm Hom}\nolimits}
\def\Ext{\mathop{\rm Ext}\nolimits}
\def\SheafHom{\mathop{\mathcal{H}\!{\it om}}\nolimits}
\def\SheafExt{\mathop{\mathcal{E}\!{\it xt}}\nolimits}
\def\Sch{\textit{Sch}}
\def\Mor{\mathop{\rm Mor}\nolimits}
\def\Ob{\mathop{\rm Ob}\nolimits}
\def\Sh{\mathop{\textit{Sh}}\nolimits}
\def\NL{\mathop{N\!L}\nolimits}
\def\proetale{{pro\text{-}\acute{e}tale}}
\def\etale{{\acute{e}tale}}
\def\QCoh{\textit{QCoh}}
\def\Ker{\mathop{\rm Ker}}
\def\Im{\mathop{\rm Im}}
\def\Coker{\mathop{\rm Coker}}
\def\Coim{\mathop{\rm Coim}}

%
% Macros for moduli stacks/spaces
%
\def\QCohstack{\mathcal{QC}\!{\it oh}}
\def\Cohstack{\mathcal{C}\!{\it oh}}
\def\Spacesstack{\mathcal{S}\!{\it paces}}
\def\Quotfunctor{{\rm Quot}}
\def\Hilbfunctor{{\rm Hilb}}
\def\Curvesstack{\mathcal{C}\!{\it urves}}
\def\Polarizedstack{\mathcal{P}\!{\it olarized}}
\def\Complexesstack{\mathcal{C}\!{\it omplexes}}
% \Pic is the operator that assigns to X its picard group, usage \Pic(X)
% \Picardstack_{X/B} denotes the Picard stack of X over B
% \Picardfunctor_{X/B} denotes the Picard functor of X over B
\def\Pic{\mathop{\rm Pic}\nolimits}
\def\Picardstack{\mathcal{P}\!{\it ic}}
\def\Picardfunctor{{\rm Pic}}
\def\Deformationcategory{\mathcal{D}\!{\it ef}}


% OK, start here.
%
\begin{document}

\title{Morphisms of schemes}

%\begin{abstract}
%\end{abstract}

\maketitle

\tableofcontents

\section{Introduction}
\label{section-introduction}

\noindent
In this chapter we introduce some types of morphisms of schemes.
A basic reference is \cite{EGA}.




\section{Radicial morphisms}
\label{section-radicial}

\begin{definition}
\label{definition-radicial}
Let $f : X \to S$ be a morphism.
We say $f$ is {\it radicial}
if for every field $K$ the induced map
$\text{Mor}(\text{Spec}(K), X) \to \text{Mor}(\text{Spec}(K), S)$
is injective.
\end{definition}

\begin{lemma}
\label{lemma-radicial-universally-injective}
Let $f : X \to S$ be a morphism of schemes.
The following are equivalent:
\begin{enumerate}
\item The morphism $f$ is radicial.
\item The morphism $f$ is universally injective, i.e.,
for every morphism $S' \to S$ the base change $f' : X_{S'} \to S$
is injective on underlying topological spaces.
\item For every $s \in S$ there is at most one $x \in X$
with $f(s) = x$ and the field extension $\kappa(s) \subset \kappa(x)$
is either an equality or purely inseparable.
\end{enumerate}
\end{lemma}

\begin{proof}
Let $K$ be a field, and let $s : \text{Spec}(K) \to S$ be a morphism.
Giving a morphism $x : \text{Spec}(K) \to X$ such that $f \circ x = s$
is the same as giving a section of the projection
$X_K = \text{Spec}(K) \times_S X \to \text{Spec}(K)$, which in turn
is the same as giving a point $x \in X_K$ whose residue field is $K$.
Hence we see that (2) implies (1).

\medskip\noindent
Conversely, suppose that (1) holds. Assume that $x, x' \in X_{S'}$
map to the same point $s' \in S'$. Choose a commutative diagram
$$
\xymatrix{
K & \kappa(x) \ar[l] \\
\kappa(x') \ar[u] & \kappa(s') \ar[l] \ar[u]
}
$$
of fields. By Schemes, Lemma \ref{schemes-lemma-characterize-points}
we get two morphisms $a, a' : \text{Spec}(K) \to X_{S'}$. One corresponding
to the point $x$ and the embedding $\kappa(x) \subset K$ and
the other corresponding to the  point $x$ and the embedding
$\kappa(x') \subset K$. Also we have $f' \circ a = f' \circ a'$.
Condition (1) now implies that the compositions of $a$ and $a'$ with
$X_{S'} \to X$ are equal. Since $X_{S'}$ is the fibre product
of $S'$ and $X$ over $S$ we see that $a = a'$. Hence $x = x'$ as
desired.

\medskip\noindent
If there are two points $x, x'$ mapping to the same point of $s$
then $f$ is not radicial by the equivalence of (1) and (2) above.
If for some $s = f(x)$, $x \in X$ the field extension
$\kappa(s) \subset \kappa(x)$ is not purely inseparable, then 
we may find a field extension $\kappa(s) \subset K$ such that
$\kappa(x)$ has two $\kappa(s)$-homomorphisms into $K$. By
Schemes, Lemma \ref{schemes-lemma-characterize-points} this
implies that the map
$\text{Mor}(\text{Spec}(K), X) \to \text{Mor}(\text{Spec}(K), S)$
is not injective, and hence $f$ is not radicial.
Thus we see that (1) implies (3).

\medskip\noindent
Finally, assume (3). By
Schemes, Lemma \ref{schemes-lemma-characterize-points} a morphism
$\text{Spec}(K) \to X$ is given by a pair $(x, \kappa(x) \to K)$.
Property (3) says exactly that associating to the pair
$(x, \kappa(x) \to K)$ the pair $(s, \kappa(s) \to \kappa(x) \to K)$
is injective.
\end{proof}



\section{Open morphisms}
\label{section-open}

\begin{definition}
\label{definition-open}
Let $f : X \to S$ be a morphism.
\begin{enumerate}
\item We say $f$ is {\it open} if the map on underlying
topological spaces is open.
\item We say $f$ is {\it universally open} if for any morphism of
schemes $S' \to S$ the base change $f' : X_{S'} \to S'$ is open.
\end{enumerate}
\end{definition}







\section{Affine morphisms}
\label{section-affine}

\begin{definition}
\label{definition-affine}
A morphism of schemes $f : X \to S$ is called {\it affine} if 
the inverse image of every affine open of $S$ is an affine
open of $X$.
\end{definition}







\section{Quasi-affine morphisms}
\label{section-quasi-affine}

\noindent
Recall that a scheme $X$ is called {\it quasi-affine} if it is quasi-compact
and isomorphic to an open subscheme of an affine scheme, see
Properties, Definition \ref{properties-definition-quasi-affine}.


\begin{definition}
\label{definition-quasi-affine}
A morphism of schemes $f : X \to S$ is called {\it quasi-affine} if
it is quasi-compact and the inverse image of every affine open
of $S$ is a quasi-affine scheme.
\end{definition}







\section{Finite type morphisms}
\label{section-finite-type}

\begin{definition}
\label{definition-finite-type}
Let $f : X \to S$ be a morphism of schemes.
\begin{enumerate}
\item We say that $f$ is {\it of finite type at $x \in X$} if 
there exists a affine open neighbourhood $\text{Spec}(A) = U \subset X$
of $x$ and and affine open $\text{Spec}(R) = V \subset S$
with $f(U) \subset V$ such that the induced ring map
$R \to A$ is of finite type.
\item We say that $f$ is {\it locally of finite type} if it is 
of finite type at every point of $X$.
\item We say that $f$ is {\it of finite type} if it is locally of
finite type and quasi-compact.
\end{enumerate}
\end{definition}





\section{Quasi-finite morphisms}
\label{section-quasi-finite}

\begin{definition}
\label{definition-quasi-finite}
Let $f : X \to S$ be a morphism of schemes.
\begin{enumerate}
\item We say that {\it $f$ is quasi-finite at a point $x \in X$} if there
exist an affine neighbourhood $\text{Spec}(A) = U \subset X$
and an affine open $\text{Spec}(R) = V \subset S$ such that
(a) $f(U) \subset V$, (b) $A$ is a finite type $R$-algebra,
and (c) $x$ is an isolated point of its fibre $X_{f(x)}$.
\item We say {\it $f$ is locally quasi-finite} if $f$ is
quasi-finite at every point $x$ of $X$.
\item We say that {\it $f$ is quasi-finite} if $f$ is of finite type
and every point $x$ is an isolated point of its fibre.
\end{enumerate}
\end{definition}





\section{Morphisms of finite presentation}
\label{section-finite-presentation}

\begin{definition}
\label{definition-finite-presentation}
Let $f : X \to S$ be a morphism of schemes.
\begin{enumerate}
\item We say that $f$ is {\it of finite presentation at $x \in X$} if 
there exists a affine open neighbourhood $\text{Spec}(A) = U \subset X$
of $x$ and and affine open $\text{Spec}(R) = V \subset S$
with $f(U) \subset V$ such that the induced ring map
$R \to A$ is of finite presentation.
\item We say that $f$ is {\it locally of finite presentation} if it is 
of finite presentation at every point of $X$.
\item We say that $f$ is {\it of finite presentation} if it is locally of
finite presentation, quasi-compact and quasi-separated.
\end{enumerate}
\end{definition}










\section{Relatively ample sheaves}
\label{section-relatively-ample}


\begin{definition}
\label{definition-relatively-ample}
Let $f : X \to S$ be a morphism of schemes.
Let $\mathcal{L}$ be an invertible $\mathcal{O}_X$-module.
We say {\it $\mathcal{L}$ is $f$-relatively ample}, or
{\it $\mathcal{L}$ is ample on $X/S$} if $f : X \to S$
is quasi-compact, and if for every affine open $V \subset S$
the restriction of $\mathcal{L}$ to the open subscheme
$f^{-1}(V)$ of $X$ is ample.
\end{definition}







\section{Very ample sheaves}
\label{section-very-ample}


\begin{definition}
\label{definition-very-ample}
Let $f : X \to S$ be a morphism of schemes.
Let $\mathcal{L}$ be an invertible $\mathcal{O}_X$-module.
We say {\it $\mathcal{L}$ is $f$-relatively very ample}, or
{\it $\mathcal{L}$ is very ample on $X/S$} if
there exist a quasi-coherent $\mathcal{O}_S$-module
$\mathcal{E}$ and an immersion $i : X \to \mathbf{P}(\mathcal{E})$
over $S$ such that
$\mathcal{L} \cong i^*\mathcal{O}_{\mathbf{P}(\mathcal{E})}(1)$.
\end{definition}








\section{Quasi-projective morphisms}
\label{section-quasi-projective}



\begin{definition}
\label{definition-quasi-projective}
Let $f : X \to S$ be a morphism of schemes.
We say $f$ is {\it quasi-projective} if $f$ is of finite type
and there exists an $f$-relatively ample invertible $\mathcal{O}_X$-module.
\end{definition}








\section{Proper morphisms}
\label{section-proper}


\begin{definition}
\label{definition-proper}
Let $f : X \to S$ be a morphism of schemes.
We say {\it $f$ is proper} if $f$ is separated, finite type, and
universally closed.
\end{definition}




\section{Projective morphisms}
\label{section-projective}


\begin{definition}
\label{definition-projective}
Let $f : X \to S$ be a morphism of schemes.
We say {\it $f$ is projective} if $X$ is isomorphic as
an $S$-scheme to a closed subscheme of a projective
bundle $\mathbf{P}(\mathcal{E})$ for some quasi-coherent, finite type
$\mathcal{O}_S$-module $\mathcal{E}$.
\end{definition}









\section{Integral and finite morphisms}
\label{section-integral}


\begin{definition}
\label{definition-integral}
Let $f : X \to S$ be a morphism of schemes.
\begin{enumerate}
\item We say that $f$ is {\it integral} if $f$ is affine
and if for every affine open $\text{Spec}(R) = V \subset S$
with inverse image $\text{Spec}(A) = f^{-1}(V) \subset X$
the associated ring map $R \to A$ is integral.
\item We say that $f$ is {\it finite} if $f$ is affine
and if for every affine open $\text{Spec}(R) = V \subset S$
with inverse image $\text{Spec}(A) = f^{-1}(V) \subset X$
the associated ring map $R \to A$ is finite.
\end{enumerate}
\end{definition}
















\section{Flat morphisms}
\label{section-flat}

\begin{definition}
\label{definition-flat}
Let $f : X \to S$ be a morphism of schemes.
\begin{enumerate}
\item We say $f$ is {\it flat at a point $x \in X$} if the
local ring $\mathcal{O}_{X, x}$ is flat over the local ring
$\mathcal{O}_{S, f(x)}$.
\item We say $f$ is {\it flat} if $f$ is flat at every point of $X$.
\end{enumerate}
\end{definition}


























\section{Other chapters}

\begin{multicols}{2}
\begin{enumerate}
\item \hyperref[introduction-section-phantom]{Introduction}
\item \hyperref[conventions-section-phantom]{Conventions}
\item \hyperref[sets-section-phantom]{Set Theory}
\item \hyperref[categories-section-phantom]{Categories}
\item \hyperref[topology-section-phantom]{Topology}
\item \hyperref[sheaves-section-phantom]{Sheaves on Spaces}
\item \hyperref[algebra-section-phantom]{Commutative Algebra}
\item \hyperref[sites-section-phantom]{Sites and Sheaves}
\item \hyperref[homology-section-phantom]{Homological Algebra}
\item \hyperref[derived-section-phantom]{Derived Categories}
\item \hyperref[more-algebra-section-phantom]{More Algebra}
\item \hyperref[simplicial-section-phantom]{Simplicial Methods}
\item \hyperref[modules-section-phantom]{Sheaves of Modules}
\item \hyperref[sites-modules-section-phantom]{Modules on Sites}
\item \hyperref[injectives-section-phantom]{Injectives}
\item \hyperref[cohomology-section-phantom]{Cohomology of Sheaves}
\item \hyperref[sites-cohomology-section-phantom]{Cohomology on Sites}
\item \hyperref[hypercovering-section-phantom]{Hypercoverings}
\item \hyperref[schemes-section-phantom]{Schemes}
\item \hyperref[constructions-section-phantom]{Constructions of Schemes}
\item \hyperref[properties-section-phantom]{Properties of Schemes}
\item \hyperref[morphisms-section-phantom]{Morphisms of Schemes}
\item \hyperref[coherent-section-phantom]{Coherent Cohomology}
\item \hyperref[divisors-section-phantom]{Divisors}
\item \hyperref[limits-section-phantom]{Limits of Schemes}
\item \hyperref[varieties-section-phantom]{Varieties}
\item \hyperref[chow-section-phantom]{Chow Homology}
\item \hyperref[topologies-section-phantom]{Topologies on Schemes}
\item \hyperref[descent-section-phantom]{Descent}
\item \hyperref[more-morphisms-section-phantom]{More on Morphisms}
\item \hyperref[flat-section-phantom]{More on Flatness}
\item \hyperref[groupoids-section-phantom]{Groupoid Schemes}
\item \hyperref[more-groupoids-section-phantom]{More on Groupoid Schemes}
\item \hyperref[etale-section-phantom]{\'Etale Morphisms of Schemes}
\item \hyperref[etale-cohomology-section-phantom]{\'Etale Cohomology}
\item \hyperref[spaces-section-phantom]{Algebraic Spaces}
\item \hyperref[spaces-properties-section-phantom]{Properties of Algebraic Spaces}
\item \hyperref[spaces-morphisms-section-phantom]{Morphisms of Algebraic Spaces}
\item \hyperref[spaces-topologies-section-phantom]{Topologies on Algebraic Spaces}
\item \hyperref[spaces-descent-section-phantom]{Descent and Algebraic Spaces}
\item \hyperref[spaces-more-morphisms-section-phantom]{More on Morphisms of Spaces}
\item \hyperref[quot-section-phantom]{Quot and Hilbert Spaces}
\item \hyperref[stacks-section-phantom]{Stacks}
\item \hyperref[spaces-groupoids-section-phantom]{Groupoids in Algebraic Spaces}
\item \hyperref[spaces-more-groupoids-section-phantom]{More on Groupoids in Spaces}
\item \hyperref[bootstrap-section-phantom]{Bootstrap}
\item \hyperref[examples-stacks-section-phantom]{Examples of Stacks}
\item \hyperref[groupoids-quotients-section-phantom]{Quotients of Groupoids}
\item \hyperref[algebraic-section-phantom]{Algebraic Stacks}
\item \hyperref[criteria-section-phantom]{Criteria for Representability}
\item \hyperref[stacks-properties-section-phantom]{Properties of Algebraic Stacks}
\item \hyperref[stacks-morphisms-section-phantom]{Morphisms of Algebraic Stacks}
\item \hyperref[examples-section-phantom]{Examples}
\item \hyperref[exercises-section-phantom]{Exercises}
\item \hyperref[guide-section-phantom]{Guide to Literature}
\item \hyperref[desirables-section-phantom]{Desirables}
\item \hyperref[coding-section-phantom]{Coding Style}
\item \hyperref[fdl-section-phantom]{GNU Free Documentation License}
\item \hyperref[index-section-phantom]{Auto Generated Index}
\end{enumerate}
\end{multicols}


\bibliography{my}
\bibliographystyle{alpha}

\end{document}
