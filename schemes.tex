\IfFileExists{stacks-project.cls}{%
\documentclass{stacks-project}
}{%
\documentclass{amsart}
}

% The following AMS packages are automatically loaded with
% the amsart documentclass:
%\usepackage{amsmath}
%\usepackage{amssymb}
%\usepackage{amsthm}

% For dealing with references we use the comment environment
\usepackage{verbatim}
\newenvironment{reference}{\comment}{\endcomment}
%\newenvironment{reference}{}{}
\newenvironment{slogan}{\comment}{\endcomment}
\newenvironment{history}{\comment}{\endcomment}

% For commutative diagrams you can use
% \usepackage{amscd}
\usepackage[all]{xy}

% We use 2cell for 2-commutative diagrams.
\xyoption{2cell}
\UseAllTwocells

% To put source file link in headers.
% Change "template.tex" to "this_filename.tex"
% \usepackage{fancyhdr}
% \pagestyle{fancy}
% \lhead{}
% \chead{}
% \rhead{Source file: \url{template.tex}}
% \lfoot{}
% \cfoot{\thepage}
% \rfoot{}
% \renewcommand{\headrulewidth}{0pt}
% \renewcommand{\footrulewidth}{0pt}
% \renewcommand{\headheight}{12pt}

\usepackage{multicol}

% For cross-file-references
\usepackage{xr-hyper}

% Package for hypertext links:
\usepackage{hyperref}

% For any local file, say "hello.tex" you want to link to please
% use \externaldocument[hello-]{hello}
\externaldocument[introduction-]{introduction}
\externaldocument[conventions-]{conventions}
\externaldocument[sets-]{sets}
\externaldocument[categories-]{categories}
\externaldocument[topology-]{topology}
\externaldocument[sheaves-]{sheaves}
\externaldocument[sites-]{sites}
\externaldocument[stacks-]{stacks}
\externaldocument[fields-]{fields}
\externaldocument[algebra-]{algebra}
\externaldocument[brauer-]{brauer}
\externaldocument[homology-]{homology}
\externaldocument[derived-]{derived}
\externaldocument[simplicial-]{simplicial}
\externaldocument[more-algebra-]{more-algebra}
\externaldocument[smoothing-]{smoothing}
\externaldocument[modules-]{modules}
\externaldocument[sites-modules-]{sites-modules}
\externaldocument[injectives-]{injectives}
\externaldocument[cohomology-]{cohomology}
\externaldocument[sites-cohomology-]{sites-cohomology}
\externaldocument[dga-]{dga}
\externaldocument[dpa-]{dpa}
\externaldocument[hypercovering-]{hypercovering}
\externaldocument[schemes-]{schemes}
\externaldocument[constructions-]{constructions}
\externaldocument[properties-]{properties}
\externaldocument[morphisms-]{morphisms}
\externaldocument[coherent-]{coherent}
\externaldocument[divisors-]{divisors}
\externaldocument[limits-]{limits}
\externaldocument[varieties-]{varieties}
\externaldocument[topologies-]{topologies}
\externaldocument[descent-]{descent}
\externaldocument[perfect-]{perfect}
\externaldocument[more-morphisms-]{more-morphisms}
\externaldocument[flat-]{flat}
\externaldocument[groupoids-]{groupoids}
\externaldocument[more-groupoids-]{more-groupoids}
\externaldocument[etale-]{etale}
\externaldocument[chow-]{chow}
\externaldocument[intersection-]{intersection}
\externaldocument[pic-]{pic}
\externaldocument[adequate-]{adequate}
\externaldocument[dualizing-]{dualizing}
\externaldocument[duality-]{duality}
\externaldocument[discriminant-]{discriminant}
\externaldocument[local-cohomology-]{local-cohomology}
\externaldocument[curves-]{curves}
\externaldocument[resolve-]{resolve}
\externaldocument[models-]{models}
\externaldocument[pione-]{pione}
\externaldocument[etale-cohomology-]{etale-cohomology}
\externaldocument[proetale-]{proetale}
\externaldocument[crystalline-]{crystalline}
\externaldocument[spaces-]{spaces}
\externaldocument[spaces-properties-]{spaces-properties}
\externaldocument[spaces-morphisms-]{spaces-morphisms}
\externaldocument[decent-spaces-]{decent-spaces}
\externaldocument[spaces-cohomology-]{spaces-cohomology}
\externaldocument[spaces-limits-]{spaces-limits}
\externaldocument[spaces-divisors-]{spaces-divisors}
\externaldocument[spaces-over-fields-]{spaces-over-fields}
\externaldocument[spaces-topologies-]{spaces-topologies}
\externaldocument[spaces-descent-]{spaces-descent}
\externaldocument[spaces-perfect-]{spaces-perfect}
\externaldocument[spaces-more-morphisms-]{spaces-more-morphisms}
\externaldocument[spaces-flat-]{spaces-flat}
\externaldocument[spaces-groupoids-]{spaces-groupoids}
\externaldocument[spaces-more-groupoids-]{spaces-more-groupoids}
\externaldocument[bootstrap-]{bootstrap}
\externaldocument[spaces-pushouts-]{spaces-pushouts}
\externaldocument[groupoids-quotients-]{groupoids-quotients}
\externaldocument[spaces-more-cohomology-]{spaces-more-cohomology}
\externaldocument[spaces-simplicial-]{spaces-simplicial}
\externaldocument[formal-spaces-]{formal-spaces}
\externaldocument[restricted-]{restricted}
\externaldocument[spaces-resolve-]{spaces-resolve}
\externaldocument[formal-defos-]{formal-defos}
\externaldocument[defos-]{defos}
\externaldocument[cotangent-]{cotangent}
\externaldocument[examples-defos-]{examples-defos}
\externaldocument[algebraic-]{algebraic}
\externaldocument[examples-stacks-]{examples-stacks}
\externaldocument[stacks-sheaves-]{stacks-sheaves}
\externaldocument[criteria-]{criteria}
\externaldocument[artin-]{artin}
\externaldocument[quot-]{quot}
\externaldocument[stacks-properties-]{stacks-properties}
\externaldocument[stacks-morphisms-]{stacks-morphisms}
\externaldocument[stacks-limits-]{stacks-limits}
\externaldocument[stacks-cohomology-]{stacks-cohomology}
\externaldocument[stacks-perfect-]{stacks-perfect}
\externaldocument[stacks-introduction-]{stacks-introduction}
\externaldocument[stacks-more-morphisms-]{stacks-more-morphisms}
\externaldocument[stacks-geometry-]{stacks-geometry}
\externaldocument[moduli-]{moduli}
\externaldocument[moduli-curves-]{moduli-curves}
\externaldocument[examples-]{examples}
\externaldocument[exercises-]{exercises}
\externaldocument[guide-]{guide}
\externaldocument[desirables-]{desirables}
\externaldocument[coding-]{coding}
\externaldocument[obsolete-]{obsolete}
\externaldocument[fdl-]{fdl}
\externaldocument[index-]{index}

% Theorem environments.
%
\theoremstyle{plain}
\newtheorem{theorem}[subsection]{Theorem}
\newtheorem{proposition}[subsection]{Proposition}
\newtheorem{lemma}[subsection]{Lemma}

\theoremstyle{definition}
\newtheorem{definition}[subsection]{Definition}
\newtheorem{example}[subsection]{Example}
\newtheorem{exercise}[subsection]{Exercise}
\newtheorem{situation}[subsection]{Situation}

\theoremstyle{remark}
\newtheorem{remark}[subsection]{Remark}
\newtheorem{remarks}[subsection]{Remarks}

\numberwithin{equation}{subsection}

% Macros
%
\def\lim{\mathop{\rm lim}\nolimits}
\def\colim{\mathop{\rm colim}\nolimits}
\def\Spec{\mathop{\rm Spec}}
\def\Hom{\mathop{\rm Hom}\nolimits}
\def\Ext{\mathop{\rm Ext}\nolimits}
\def\SheafHom{\mathop{\mathcal{H}\!{\it om}}\nolimits}
\def\SheafExt{\mathop{\mathcal{E}\!{\it xt}}\nolimits}
\def\Sch{\textit{Sch}}
\def\Mor{\mathop{\rm Mor}\nolimits}
\def\Ob{\mathop{\rm Ob}\nolimits}
\def\Sh{\mathop{\textit{Sh}}\nolimits}
\def\NL{\mathop{N\!L}\nolimits}
\def\proetale{{pro\text{-}\acute{e}tale}}
\def\etale{{\acute{e}tale}}
\def\QCoh{\textit{QCoh}}
\def\Ker{\mathop{\rm Ker}}
\def\Im{\mathop{\rm Im}}
\def\Coker{\mathop{\rm Coker}}
\def\Coim{\mathop{\rm Coim}}

%
% Macros for moduli stacks/spaces
%
\def\QCohstack{\mathcal{QC}\!{\it oh}}
\def\Cohstack{\mathcal{C}\!{\it oh}}
\def\Spacesstack{\mathcal{S}\!{\it paces}}
\def\Quotfunctor{{\rm Quot}}
\def\Hilbfunctor{{\rm Hilb}}
\def\Curvesstack{\mathcal{C}\!{\it urves}}
\def\Polarizedstack{\mathcal{P}\!{\it olarized}}
\def\Complexesstack{\mathcal{C}\!{\it omplexes}}
% \Pic is the operator that assigns to X its picard group, usage \Pic(X)
% \Picardstack_{X/B} denotes the Picard stack of X over B
% \Picardfunctor_{X/B} denotes the Picard functor of X over B
\def\Pic{\mathop{\rm Pic}\nolimits}
\def\Picardstack{\mathcal{P}\!{\it ic}}
\def\Picardfunctor{{\rm Pic}}
\def\Deformationcategory{\mathcal{D}\!{\it ef}}


% OK, start here.
%
\begin{document}

\title{Schemes}

%\begin{abstract}
%\end{abstract}

\maketitle

\tableofcontents

\section{Introduction}
\label{section-introduction}

\noindent
In this document we explain how we will think of schemes as stacks over the
category of affine schemes.

\section{Topological prelimiaries}
\label{section-preliminaries}

\noindent
Let $f : X \to Y$ be a continuous map of topological spaces.


\subsection{Ringed spaces}
\label{subsection-ringed-sapces}

\begin{definition}
\label{definition-ringed-space}
A {\it ringed space} is a pair $(X, \mathcal{O}_X)$
consisting of a topological space $X$ and a sheaf of rings
$\mathcal{O}_X$.
\end{definition}

\section{Affine schemes}
\label{section-schemes}

A locally ringed space $(X,\mathcal{O}_X)$ is a pair consisting of a
topological space $X$ and a sheaf of rings $\mathcal{O}_X$ all of whose stalks
are local rings. Morphisms in the category of locally ringed spaces are
maps of pairs $f : (X, \mathcal{O}_X) \to (Y,\mathcal{O}_Y)$ so that
all the induced ring maps $\mathcal{O}_{Y,f(x)} \to \mathcal{O}_{X,x}$ are
local ring maps.

\smallskip\noindent
A reference for this section is \cite{EGA}, I.

\subsection{Affine schemes}
\label{subsection-affine-schemes}

\noindent
An affine scheme is a locally ringed space isomorphic to a locally ringed
space of the form $\text{Spec}(A)$, for some commutative (unital) ring $A$.
(Note that $A$ can be the zero ring in which case $\text{Spec}(A)$ is
the empty space.) As a set $\text{Spec}(A)$ is the set of prime ideals of
$A$. The topology on $\text{Spec}(A)$ is the unique one that has a basis
of opens of the form $D(f) = \{ \wp \in\text{Spec}(A) \mid
f \not\in \wp \}$,
$f\in A$. The structure sheaf $\mathcal{O} = 
\mathcal{O}_{\text{Spec}(A)}$ is the unique
sheaf of rings such that (1) $\Gamma(D(f), \mathcal{O}) = A_f$ and
(2) the restriction map $\Gamma(D(f), \mathcal{O}) \to \Gamma(D(fg),
\mathcal{O})$ is the canonical map $A_f \to A_{fg}$.

\smallskip\noindent
A morphism of affine schemes is a morphism in the category of locally ringed 
spaces.

\subsection{The category of affine schemes}
\label{subsection-affine-schemes}

\noindent
It should be clear what the category of affine schemes is, except for a
little bit of set-theoretical discussion. We will use the notation
$\text{Aff}$ to denote this category. Our approach is to use only
categories which are sets. Thus we will choose a supply of affines and
work with this. For a precise mathematical discussion, see
Subsection \ref{subsection-sets-of-affines}.

\smallskip\noindent
The topology on $\text{Aff}$ will be the fppf topology. A covering is
given by a finite family of maps $\{U_i \to U\}$, where each $U_i \to U$
is a finitely presented flat morphism of affines, and $\coprod U_i \to U$
is surjective. 

\smallskip\noindent
Sometimes we consider $\text{Aff}$ with other topologies, such as the
etale, Zariski, or fpqc topologies. Notation $\text{Aff}_{etale}$, etc.
FIXME. Put in internal reference to topology discussion.

\subsubsection{Sets of affine schemes}
\label{subsection-sets-of-affines}

\noindent
Choose an ordinal $\alpha$ and denote $\text{Aff}_\alpha$ the
category of affine schemes which are elements of $V_\alpha$, see
Sets, Subsection \ref{sets-subsection-sets-hierarchy}. So there is a
theory of algebraic stacks for any $\alpha$. There are some minimal
conditions on $\alpha$ needed to imply that $\text{Aff}_\alpha$ is a site. 
These minimal required properties are expressed in the following lemma.

\begin{lemma}
\label{lemma-Aff-site}
For any set $S$ may choose an ordinal $\alpha$ with $S \in V_\alpha$ 
such that $\text{Aff}_\alpha$ has (finite) fibre products, and finite disjoint
unions. In addition we may assume that for any finitely presented morphism
of affines $X \to Y$, such that $Y \in \text{Aff}_\alpha$, there exists
an affine $X' \in \text{Ob}(\text{Aff}_\alpha)$ such that $X' \cong X$.
\end{lemma}

\begin{proof}
Consider the following statement: ``For any finite directed graph $\Gamma$,
for any assignment $v \mapsto F(v)$, $\forall v\in \text{Vertices}(\Gamma)$,
where $F(v)$ is an affine scheme, and any assignment
$\big(e : v_1 \to v_2\big) \mapsto \big(F(e) : F(v_1) \to F(v_2)\big)$,
$\forall e \in \text{Edges}(\Gamma)$ where $F(e)$ is a morphism of affine
schemes, there exists an affine scheme $X$ and morphisms $f(v) : X \to F(v)$,
$\forall v\in \text{Vertices}(\Gamma)$ such that $f(v_2) = F(e) \circ f(v_1)$,
$\forall \big(e : v_1 \to v_2\big) \in \text{Edges}(\Gamma)$, such that
$(X, \{f(v)\}_{v\in \text{Vertices}(\Gamma)})$ is universal among all such.''
This statement says that finite limits exist for affine schemes. It is
proved in a standard way (for example by turning it into ring theory).

\smallskip\noindent
On the other hand, upon formalizing the statement we obtain a provable
formula $\phi(\Gamma, F)$ of ZFC set theory. Hence, according to the reflection
principle, see Sets, Lemma \ref{sets-section-reflection-principle}
there exists an ordinal $\alpha$ such that the formula is true in
$V_\alpha$: If you take $\Gamma \in V_\alpha$ and the $F(v)$ to be in
$\text{Aff}_\alpha$, then you can find a solution
$(X, \{f(v)\}_{v\in \text{Vertices}(\Gamma)})$
with $X$ in $V_\alpha$. This takes care of the statement about fibre 
products. (Of course as soon as $\alpha$ is infinite then every
graph is isomorphic to a graph in $V_\alpha$; we can also simply apriori
require this for $V_\alpha$.).

\smallskip\noindent
We can similarly write out the condition of the existence of disjoint unions
as a set theory formula, and similarly the existence of the affine $X'$
given $X \to Y$. The reflection principle states we can have $S$ inside of
$V_\alpha$ as well.
\end{proof}

\noindent
Clearly, we may assume that $\text{Aff}_\alpha$ is closed under any reasonable
operation (see Sets, Section \ref{sets-section-reflection-principle}).
Of course, whenever we require such a condition we will need to write out
the proof that this is so.

\smallskip\noindent
So, in the following we will work with stacks (or categories) over
$\text{Aff}_\alpha$\footnote{As per our general philosophy, if we ever need
an actual 2-category of stacks, we also choose another cardinal $\gamma$ and
consider only those categories over $\text{Aff}_\alpha$ contained in
$V_\gamma$.}. If $\alpha < \beta$, then there is an inclusion
$\text{Aff}_\alpha \subset \text{Aff}_\beta$, and hence any category
over $\text{Aff}_\beta$ gives rise to a category over $\text{Aff}_\alpha$.
But this is not the correct thing to do when studying algebraic stacks.
Instead we want to show that algebraic stacks over $\text{Aff}_\alpha$
give rise to algebraic stacks over $\text{Aff}_\beta$. In other words we will
need a theorem saying that the 2-category of algebraic
stacks over $\text{Aff}_\alpha$ is equivalent to a full sub-2-category of
algebraic stacks over $\text{Aff}_\beta$. Here it is.

\smallskip\noindent
FIXME. Improve the theorem below and move it to a more appropriate spot.

\begin{theorem}
\label{theorem-change-alpha}
Suppose that $p : \mathcal{S} \to \text{Aff}_\alpha$ is an algebraic stack.
Let $\beta > \alpha$. Then there exists an algebraic stack
$p' : \mathcal{S}' \to \text{Aff}_\beta$ and an equivalence
$$
\xymatrix{
(p')^{-1}(\text{Aff}_\alpha) \ar[rd]_{p'} \ar[rr]^c && \mathcal{S}\ar[ld]^p\\
&\text{Aff}_\alpha.&}
$$
The pair $((\mathcal{S'},p'),c)$ is well determined up to a 1-isomorphism
(which is itself unique up to unique 2-isomorphism).
\end{theorem}

\begin{proof}
FIXME. Hint. Choose a representation (in $\text{Stacks}/\text{Aff}_\alpha$)
$\mathcal{S} = [ \mathcal{X}/\mathcal{R} ]$, with $\mathcal{X}$ representable
by a scheme $X$ and $\mathcal{R}$ representable by an algebraic space.
Choose a presentation $\mathcal{R} = [ \mathcal{U}/\mathcal{R}_\mathcal{U} ]$
where $\mathcal{U}$ and $\mathcal{R}_\mathcal{U}$ are representable
by schemes $U$ and $R_U$. Now define (in $\text{Stacks}/\text{Aff}_\beta$)
$\mathcal{U}'$ to be the stack associated to $U$, $\mathcal{R}'_\mathcal{U}$
to be the stack associated to $R_U$, $\mathcal{R}'$ the stack
$\mathcal{R}' = [ \mathcal{U}'/\mathcal{R}'_\mathcal{U} ]$, $\mathcal{X}'$
the stack associated to $X$, and finally
$\mathcal{S}' = [ \mathcal{X}'/\mathcal{R}' ]$.
\end{proof}

\noindent
From now on $\text{Aff}$ will denote a category of affines $\text{Aff}_\alpha$
such as in Lemma \ref{lemma-Aff-site}. By the theorem above we may increase
$\alpha$ whenever this is needed.

\begin{remark}
\label{remark-other-approach}
There is another approach. Allow yourself to enlarge $\alpha$ at any moment.
Think of every statement in the text as being preceded by ``There exist
arbitrarily large $\alpha$ such that''. 
\end{remark}

\subsection{Schemes}
\label{subsection-schemes}

\noindent
We recall the definition of a scheme.

\smallskip\noindent
A scheme $(X,\mathcal{O}_X)$ is a locally ringed space
with the property that every point has a neighbourhood which is an
affine scheme.

\smallskip\noindent
A scheme $X$ gives rise to a functor (or presheaf)
$$
\xymatrix{
\text{Aff}^{\text{opp}} \ar[r]^{h_X} & \text{Sets}, &
U \ar@{|->}[r] & \text{Mor}(U, X).}
$$
The usual Yoneda lemma tells us that we can recover the scheme from this
functor. 

\begin{lemma}
\label{lemma-yoneda-schemes}
Suppose that $X$, $Y$ are schemes with that have open coverings
by affines isomorphic to objects of $\text{Aff}$. Then $\text{Mor}(X,Y)
= \text{Mor}(h_X, h_Y)$.
\end{lemma}

\begin{proof}
FIXME.
\end{proof}

\subsection{Stacks representable by a scheme}
\label{subsection-stack-representable-by-scheme}

\noindent
In Categories, Definition
\ref{categories-definition-representable-fibred-category} we
defined the notion of a representable category fibred in groupoids. This,
applied to a stack (or a category) over $\text{Aff}$ will define the notion of
a stack representable by an affine scheme. 

\smallskip\noindent
Here is the formal definition of a category over $\text{Aff}$ representable by
a scheme. Please also see the informal discussion below.

\begin{definition}
\label{definition-representable-by-scheme}
A category fibred in groupoids $p : \mathcal{S} \to \text{Aff}$ is
called representable by a scheme, if the following conditions are satisfied:
\begin{enumerate}
\item all fibre categories $\mathcal{S}_U$ are setlike, and
\item the presheaf $U \mapsto \text{Ob}(\mathcal{S}_U)/\cong$ is 
is isomorphic to $h_S$ for a scheme $S$ as in
Lemma \ref{lemma-yoneda-schemes}.
\end{enumerate}
\end{definition}

\begin{lemma}
\label{lemma-representable-by-scheme-implies-stack}
If $\mathcal{S} \to \text{Aff}$ is representable by a scheme then $\mathcal{S}$
is a stack over $\text{Aff}$.
\end{lemma}

\begin{proof}
FIXME.
\end{proof}

\begin{example}
\label{example-standard-representable-scheme}
Let $X$ be a scheme that has a covering by open affines which are isomorphic
to objects of $\text{Aff}$. There is a standard stack over $\text{Aff}$
representable by $X$, namely the stack of affines over $X$. Compare
Categories, Example \ref {categories-example-comma-category}.
This stack will be denoted $\text{Aff}/X$, and it is described as follows.
\begin{enumerate}
\item An object of $\text{Aff}/X$ is a morphism of schemes
$U \to X$, with $U \in \text{Ob}(\text{Aff})$.
\item A morphism between $U\to X$ and $V \to X$ is a commutative diagram
$$
\xymatrix{
U \ar[rr] \ar[rd] && V \ar[ld] \\
&X.&}
$$
\item The functor $(\text{Aff}/X) \to \text{Aff}$ maps $U\to X$ to $U$.
\end{enumerate}
It is clear from the definition that $\text{Aff}/X$ is representable by
a scheme. 

\smallskip\noindent
The construction is clearly functorial in $X$, so that a morphism
of schemes $f : X \to Y$ induces a morphisms of stacks 
$\text{Aff}/X \to \text{Aff}/Y$. FIXME: more?
\end{example}

\begin{situation}
\label{situation-stack-represented-by-scheme}
The following situation will appear repeatedly in the text. Suppose that
$\mathcal{S} \to \text{Aff}$ is a stack representable by a scheme. If we
say the scheme $S$ represents $\mathcal{S}$, then we mean that besides 
being given the scheme $S$, we are given an equivalence $j : \mathcal{S}
\to \text{Aff}/S$ of stacks over $\text{Aff}$.
\end{situation}

\begin{lemma}
\label{lemma-morphism-stacks-representable-by-schemes}
Suppose that the stacks $\mathcal{X}$, $\mathcal{Y}$ are represented
by the schemes $X$ and $Y$. For any morphism of stacks $F : \mathcal{X}
\to \mathcal{Y}$ there is a unique morphism of schemes $f : X \to Y$
such that the diagram
$$
\xymatrix{
\mathcal{X} \ar[r]^F \ar[d]_j & \mathcal{Y} \ar[d]^j \\
\text{Aff}/X \ar[r]^f & \text{Aff}/Y}
$$
2-commutes and then the diagram actually commutes.
\end{lemma}

\begin{proof}
FIXME.
\end{proof}

\section{Morphisms representable by schemes}
\label{section-morphisms-representable-by-schemes}

\noindent
In this section we define the notion of moprhisms of stacks over $\text{Aff}$
representable by schemes.

\subsection{Definition}
\label{subsection-definition-representable-by-schemes}

\noindent
Here is the formal definition. Please also see the informal discussion below.

\begin{definition}
\label{definition-representable-by-schemes}
Let $f : \mathcal{X} \to \mathcal{Y}$ be a morphism of categories
fibred in groupoids over $\text{Aff}$. We say $f$ is representable by
schemes if for every stack $\mathcal{S}$ representable by a scheme
(see Definition \ref{definition-representable-by-scheme}), and every morphism
$\mathcal{U} \to \mathcal{Y}$, the 2-fibre product
$\mathcal{S}\times_\mathcal{Y}\mathcal{X}$ is representable by a scheme.
\end{definition}

\noindent
Informal discussion. In the situation of the definition we sometimes 
say that $\mathcal{X}$ is relatively representable over $\mathcal{Y}$.
Suppose that, with the notation of the definition, $S$ represents
$\mathcal{S}$ and $W$ represents $\mathcal{S}\times_\mathcal{Y}\mathcal{X}$.
According to Lemma \ref{lemma-morphism-stacks-representable-by-schemes}
we get a morphism of schemes $g : W \to S$ and a 2-commutative diagram
of stacks
$$
\xymatrix{
\text{Aff}/W \ar[d]^g &
\mathcal{S}\times_\mathcal{X}\mathcal{Y} \ar[d] \ar[l]^j \ar[r] &
\mathcal{Y} \ar[d] \\
\text{Aff}/S &
\mathcal{S} \ar[l]^j \ar[r] & \mathcal{X}
}
$$
FIXME: more.

\smallskip\noindent
FIXME. It seems to me that you can define the notion even if 
$\mathcal{X}$ and $\mathcal{Y}$ are just categories over $\text{Aff}$. Does
it make sense in this generality?

\begin{definition}
\label{definition-property-morphism-representable-by-schemes}
Let $P$ be a property of morphisms of schemes such that
if the morphism $f : X \to Y$ has property $P$, then so does
every base change of $f$. (FIXME: introduce base change.)
We say that a morphism of stacks $\mathcal{X}
\to \mathcal{Y}$ representable by schemes has property
$P$ if for every diagram as above the morphism of schemes
$g : W \to S$ has property $P$.
\end{definition}

\noindent
FIXME. Explain rationale behind this definition: what else could it be?

\section{Other chapters}

\begin{multicols}{2}
\begin{enumerate}
\item \hyperref[introduction-section-phantom]{Introduction}
\item \hyperref[conventions-section-phantom]{Conventions}
\item \hyperref[sets-section-phantom]{Set Theory}
\item \hyperref[categories-section-phantom]{Categories}
\item \hyperref[topology-section-phantom]{Topology}
\item \hyperref[sheaves-section-phantom]{Sheaves on Spaces}
\item \hyperref[algebra-section-phantom]{Commutative Algebra}
\item \hyperref[sites-section-phantom]{Sites and Sheaves}
\item \hyperref[homology-section-phantom]{Homological Algebra}
\item \hyperref[derived-section-phantom]{Derived Categories}
\item \hyperref[more-algebra-section-phantom]{More Algebra}
\item \hyperref[simplicial-section-phantom]{Simplicial Methods}
\item \hyperref[modules-section-phantom]{Sheaves of Modules}
\item \hyperref[sites-modules-section-phantom]{Modules on Sites}
\item \hyperref[injectives-section-phantom]{Injectives}
\item \hyperref[cohomology-section-phantom]{Cohomology of Sheaves}
\item \hyperref[sites-cohomology-section-phantom]{Cohomology on Sites}
\item \hyperref[hypercovering-section-phantom]{Hypercoverings}
\item \hyperref[schemes-section-phantom]{Schemes}
\item \hyperref[constructions-section-phantom]{Constructions of Schemes}
\item \hyperref[properties-section-phantom]{Properties of Schemes}
\item \hyperref[morphisms-section-phantom]{Morphisms of Schemes}
\item \hyperref[coherent-section-phantom]{Coherent Cohomology}
\item \hyperref[divisors-section-phantom]{Divisors}
\item \hyperref[limits-section-phantom]{Limits of Schemes}
\item \hyperref[varieties-section-phantom]{Varieties}
\item \hyperref[chow-section-phantom]{Chow Homology}
\item \hyperref[topologies-section-phantom]{Topologies on Schemes}
\item \hyperref[descent-section-phantom]{Descent}
\item \hyperref[more-morphisms-section-phantom]{More on Morphisms}
\item \hyperref[flat-section-phantom]{More on Flatness}
\item \hyperref[groupoids-section-phantom]{Groupoid Schemes}
\item \hyperref[more-groupoids-section-phantom]{More on Groupoid Schemes}
\item \hyperref[etale-section-phantom]{\'Etale Morphisms of Schemes}
\item \hyperref[etale-cohomology-section-phantom]{\'Etale Cohomology}
\item \hyperref[spaces-section-phantom]{Algebraic Spaces}
\item \hyperref[spaces-properties-section-phantom]{Properties of Algebraic Spaces}
\item \hyperref[spaces-morphisms-section-phantom]{Morphisms of Algebraic Spaces}
\item \hyperref[spaces-topologies-section-phantom]{Topologies on Algebraic Spaces}
\item \hyperref[spaces-descent-section-phantom]{Descent and Algebraic Spaces}
\item \hyperref[spaces-more-morphisms-section-phantom]{More on Morphisms of Spaces}
\item \hyperref[quot-section-phantom]{Quot and Hilbert Spaces}
\item \hyperref[stacks-section-phantom]{Stacks}
\item \hyperref[spaces-groupoids-section-phantom]{Groupoids in Algebraic Spaces}
\item \hyperref[spaces-more-groupoids-section-phantom]{More on Groupoids in Spaces}
\item \hyperref[bootstrap-section-phantom]{Bootstrap}
\item \hyperref[examples-stacks-section-phantom]{Examples of Stacks}
\item \hyperref[groupoids-quotients-section-phantom]{Quotients of Groupoids}
\item \hyperref[algebraic-section-phantom]{Algebraic Stacks}
\item \hyperref[criteria-section-phantom]{Criteria for Representability}
\item \hyperref[stacks-properties-section-phantom]{Properties of Algebraic Stacks}
\item \hyperref[stacks-morphisms-section-phantom]{Morphisms of Algebraic Stacks}
\item \hyperref[examples-section-phantom]{Examples}
\item \hyperref[exercises-section-phantom]{Exercises}
\item \hyperref[guide-section-phantom]{Guide to Literature}
\item \hyperref[desirables-section-phantom]{Desirables}
\item \hyperref[coding-section-phantom]{Coding Style}
\item \hyperref[fdl-section-phantom]{GNU Free Documentation License}
\item \hyperref[index-section-phantom]{Auto Generated Index}
\end{enumerate}
\end{multicols}


\bibliography{my}
\bibliographystyle{alpha}

\end{document}
