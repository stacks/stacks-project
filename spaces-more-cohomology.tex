\IfFileExists{stacks-project.cls}{%
\documentclass{stacks-project}
}{%
\documentclass{amsart}
}

% The following AMS packages are automatically loaded with
% the amsart documentclass:
%\usepackage{amsmath}
%\usepackage{amssymb}
%\usepackage{amsthm}

% For dealing with references we use the comment environment
\usepackage{verbatim}
\newenvironment{reference}{\comment}{\endcomment}
%\newenvironment{reference}{}{}
\newenvironment{slogan}{\comment}{\endcomment}
\newenvironment{history}{\comment}{\endcomment}

% For commutative diagrams you can use
% \usepackage{amscd}
\usepackage[all]{xy}

% We use 2cell for 2-commutative diagrams.
\xyoption{2cell}
\UseAllTwocells

% To put source file link in headers.
% Change "template.tex" to "this_filename.tex"
% \usepackage{fancyhdr}
% \pagestyle{fancy}
% \lhead{}
% \chead{}
% \rhead{Source file: \url{template.tex}}
% \lfoot{}
% \cfoot{\thepage}
% \rfoot{}
% \renewcommand{\headrulewidth}{0pt}
% \renewcommand{\footrulewidth}{0pt}
% \renewcommand{\headheight}{12pt}

\usepackage{multicol}

% For cross-file-references
\usepackage{xr-hyper}

% Package for hypertext links:
\usepackage{hyperref}

% For any local file, say "hello.tex" you want to link to please
% use \externaldocument[hello-]{hello}
\externaldocument[introduction-]{introduction}
\externaldocument[conventions-]{conventions}
\externaldocument[sets-]{sets}
\externaldocument[categories-]{categories}
\externaldocument[topology-]{topology}
\externaldocument[sheaves-]{sheaves}
\externaldocument[sites-]{sites}
\externaldocument[stacks-]{stacks}
\externaldocument[fields-]{fields}
\externaldocument[algebra-]{algebra}
\externaldocument[brauer-]{brauer}
\externaldocument[homology-]{homology}
\externaldocument[derived-]{derived}
\externaldocument[simplicial-]{simplicial}
\externaldocument[more-algebra-]{more-algebra}
\externaldocument[smoothing-]{smoothing}
\externaldocument[modules-]{modules}
\externaldocument[sites-modules-]{sites-modules}
\externaldocument[injectives-]{injectives}
\externaldocument[cohomology-]{cohomology}
\externaldocument[sites-cohomology-]{sites-cohomology}
\externaldocument[dga-]{dga}
\externaldocument[dpa-]{dpa}
\externaldocument[hypercovering-]{hypercovering}
\externaldocument[schemes-]{schemes}
\externaldocument[constructions-]{constructions}
\externaldocument[properties-]{properties}
\externaldocument[morphisms-]{morphisms}
\externaldocument[coherent-]{coherent}
\externaldocument[divisors-]{divisors}
\externaldocument[limits-]{limits}
\externaldocument[varieties-]{varieties}
\externaldocument[topologies-]{topologies}
\externaldocument[descent-]{descent}
\externaldocument[perfect-]{perfect}
\externaldocument[more-morphisms-]{more-morphisms}
\externaldocument[flat-]{flat}
\externaldocument[groupoids-]{groupoids}
\externaldocument[more-groupoids-]{more-groupoids}
\externaldocument[etale-]{etale}
\externaldocument[chow-]{chow}
\externaldocument[intersection-]{intersection}
\externaldocument[pic-]{pic}
\externaldocument[adequate-]{adequate}
\externaldocument[dualizing-]{dualizing}
\externaldocument[duality-]{duality}
\externaldocument[discriminant-]{discriminant}
\externaldocument[local-cohomology-]{local-cohomology}
\externaldocument[curves-]{curves}
\externaldocument[resolve-]{resolve}
\externaldocument[models-]{models}
\externaldocument[pione-]{pione}
\externaldocument[etale-cohomology-]{etale-cohomology}
\externaldocument[proetale-]{proetale}
\externaldocument[crystalline-]{crystalline}
\externaldocument[spaces-]{spaces}
\externaldocument[spaces-properties-]{spaces-properties}
\externaldocument[spaces-morphisms-]{spaces-morphisms}
\externaldocument[decent-spaces-]{decent-spaces}
\externaldocument[spaces-cohomology-]{spaces-cohomology}
\externaldocument[spaces-limits-]{spaces-limits}
\externaldocument[spaces-divisors-]{spaces-divisors}
\externaldocument[spaces-over-fields-]{spaces-over-fields}
\externaldocument[spaces-topologies-]{spaces-topologies}
\externaldocument[spaces-descent-]{spaces-descent}
\externaldocument[spaces-perfect-]{spaces-perfect}
\externaldocument[spaces-more-morphisms-]{spaces-more-morphisms}
\externaldocument[spaces-flat-]{spaces-flat}
\externaldocument[spaces-groupoids-]{spaces-groupoids}
\externaldocument[spaces-more-groupoids-]{spaces-more-groupoids}
\externaldocument[bootstrap-]{bootstrap}
\externaldocument[spaces-pushouts-]{spaces-pushouts}
\externaldocument[groupoids-quotients-]{groupoids-quotients}
\externaldocument[spaces-more-cohomology-]{spaces-more-cohomology}
\externaldocument[spaces-simplicial-]{spaces-simplicial}
\externaldocument[formal-spaces-]{formal-spaces}
\externaldocument[restricted-]{restricted}
\externaldocument[spaces-resolve-]{spaces-resolve}
\externaldocument[formal-defos-]{formal-defos}
\externaldocument[defos-]{defos}
\externaldocument[cotangent-]{cotangent}
\externaldocument[examples-defos-]{examples-defos}
\externaldocument[algebraic-]{algebraic}
\externaldocument[examples-stacks-]{examples-stacks}
\externaldocument[stacks-sheaves-]{stacks-sheaves}
\externaldocument[criteria-]{criteria}
\externaldocument[artin-]{artin}
\externaldocument[quot-]{quot}
\externaldocument[stacks-properties-]{stacks-properties}
\externaldocument[stacks-morphisms-]{stacks-morphisms}
\externaldocument[stacks-limits-]{stacks-limits}
\externaldocument[stacks-cohomology-]{stacks-cohomology}
\externaldocument[stacks-perfect-]{stacks-perfect}
\externaldocument[stacks-introduction-]{stacks-introduction}
\externaldocument[stacks-more-morphisms-]{stacks-more-morphisms}
\externaldocument[stacks-geometry-]{stacks-geometry}
\externaldocument[moduli-]{moduli}
\externaldocument[moduli-curves-]{moduli-curves}
\externaldocument[examples-]{examples}
\externaldocument[exercises-]{exercises}
\externaldocument[guide-]{guide}
\externaldocument[desirables-]{desirables}
\externaldocument[coding-]{coding}
\externaldocument[obsolete-]{obsolete}
\externaldocument[fdl-]{fdl}
\externaldocument[index-]{index}

% Theorem environments.
%
\theoremstyle{plain}
\newtheorem{theorem}[subsection]{Theorem}
\newtheorem{proposition}[subsection]{Proposition}
\newtheorem{lemma}[subsection]{Lemma}

\theoremstyle{definition}
\newtheorem{definition}[subsection]{Definition}
\newtheorem{example}[subsection]{Example}
\newtheorem{exercise}[subsection]{Exercise}
\newtheorem{situation}[subsection]{Situation}

\theoremstyle{remark}
\newtheorem{remark}[subsection]{Remark}
\newtheorem{remarks}[subsection]{Remarks}

\numberwithin{equation}{subsection}

% Macros
%
\def\lim{\mathop{\rm lim}\nolimits}
\def\colim{\mathop{\rm colim}\nolimits}
\def\Spec{\mathop{\rm Spec}}
\def\Hom{\mathop{\rm Hom}\nolimits}
\def\Ext{\mathop{\rm Ext}\nolimits}
\def\SheafHom{\mathop{\mathcal{H}\!{\it om}}\nolimits}
\def\SheafExt{\mathop{\mathcal{E}\!{\it xt}}\nolimits}
\def\Sch{\textit{Sch}}
\def\Mor{\mathop{\rm Mor}\nolimits}
\def\Ob{\mathop{\rm Ob}\nolimits}
\def\Sh{\mathop{\textit{Sh}}\nolimits}
\def\NL{\mathop{N\!L}\nolimits}
\def\proetale{{pro\text{-}\acute{e}tale}}
\def\etale{{\acute{e}tale}}
\def\QCoh{\textit{QCoh}}
\def\Ker{\mathop{\rm Ker}}
\def\Im{\mathop{\rm Im}}
\def\Coker{\mathop{\rm Coker}}
\def\Coim{\mathop{\rm Coim}}

%
% Macros for moduli stacks/spaces
%
\def\QCohstack{\mathcal{QC}\!{\it oh}}
\def\Cohstack{\mathcal{C}\!{\it oh}}
\def\Spacesstack{\mathcal{S}\!{\it paces}}
\def\Quotfunctor{{\rm Quot}}
\def\Hilbfunctor{{\rm Hilb}}
\def\Curvesstack{\mathcal{C}\!{\it urves}}
\def\Polarizedstack{\mathcal{P}\!{\it olarized}}
\def\Complexesstack{\mathcal{C}\!{\it omplexes}}
% \Pic is the operator that assigns to X its picard group, usage \Pic(X)
% \Picardstack_{X/B} denotes the Picard stack of X over B
% \Picardfunctor_{X/B} denotes the Picard functor of X over B
\def\Pic{\mathop{\rm Pic}\nolimits}
\def\Picardstack{\mathcal{P}\!{\it ic}}
\def\Picardfunctor{{\rm Pic}}
\def\Deformationcategory{\mathcal{D}\!{\it ef}}


% OK, start here.
%
\begin{document}

\title{More on Cohomology of Spaces}

\maketitle

\phantomsection
\label{section-phantom}

\tableofcontents




\section{Introduction}
\label{section-introduction}

\noindent
In this chapter continues the discussion started in
Cohomology of Spaces, Section \ref{spaces-cohomology-section-introduction}.
One can also view this chapter as the analogue for algebraic spaces
of the chapter on \'etale cohomology for schemes, see
\'Etale Cohomology, Section \ref{etale-cohomology-section-introduction}.

\medskip\noindent
In fact, we intend this chapter to be mainly a translation of the
results already proved for schemes into the language of algebraic
spaces. Some of our results can be found in \cite{Kn}.





\section{Conventions}
\label{section-conventions}

\noindent
The standing assumption is that all schemes are contained in
a big fppf site $\Sch_{fppf}$. And all rings $A$ considered
have the property that $\Spec(A)$ is (isomorphic) to an
object of this big site.

\medskip\noindent
Let $S$ be a scheme and let $X$ be an algebraic space over $S$.
In this chapter and the following we will write $X \times_S X$
for the product of $X$ with itself (in the category of algebraic
spaces over $S$), instead of $X \times X$.







\section{Transporting results from schemes}
\label{section-api}

\noindent
In this section we explain briefly how results for schemes
imply results for (representable) algebraic spaces and
(representable) morphisms of algebraic spaces.
For quasi-coherent modules more is true
(because \'etale cohomology of a quasi-coherent module
over a scheme agrees with Zariski cohomology) and this
has already been discussed in Cohomology of Spaces, Section
\ref{spaces-cohomology-section-higher-direct-image}.

\medskip\noindent
Let $S$ be a scheme. Let $X$ be an algebraic space over $S$.
Now suppose that $X$ is representable by the scheme $X_0$
(awkward but temporary notation; we usually just say ``$X$
is a scheme''). In this case $X$ and $X_0$ have the same small
\'etale sites:
$$
X_\etale = (X_0)_\etale
$$
This is pointed out in
Properties of Spaces, Section \ref{spaces-properties-section-etale-site}.
Moreover, if $f : X \to Y$ is a morphism of representable algebraic spaces
over $S$ and if $f_0 : X_0 \to Y_0$ is a morphism of
schemes representing $f$, then the induced morphisms of small
\'etale topoi agree:
$$
\xymatrix{
\Sh(X_\etale) \ar[rr]_{f_{small}} \ar@{=}[d] & &
\Sh(Y_\etale) \ar@{=}[d] \\
\Sh((X_0)_\etale) \ar[rr]^{(f_0)_{small}} & &
\Sh((Y_0)_\etale)
}
$$
See Properties of Spaces, Lemma
\ref{spaces-properties-lemma-functoriality-etale-site} and
Topologies, Lemma \ref{topologies-lemma-morphism-big-small-etale}.

\medskip\noindent
Thus there is absolutely no difference between \'etale cohomology
of a scheme and the \'etale cohomology of the corresponding algebraic space.
Similarly for higher direct images along morphisms of schemes.
In fact, if $f : X \to Y$ is a morphism of algebraic spaces over $S$
which is representable (by schemes), then the higher direct images
$R^if_*\mathcal{F}$ of a sheaf $\mathcal{F}$ on $X_\etale$
can be computed \'etale locally on $Y$ (Cohomology on Sites,
Lemma \ref{sites-cohomology-lemma-higher-direct-images})
hence this often reduces computations and proofs to the case
where $Y$ and $X$ are schemes.

\medskip\noindent
We will use the above without further mention in this chapter.
For other topologies the same thing is true; we state it
explicitly as a lemma for cohomology here.

\begin{lemma}
\label{lemma-compare-cohomology-other-topologies}
Let $S$ be a scheme. Let $\tau \in \{\etale, fppf, ph\}$ (add more here).
The inclusion functor
$$
(\Sch/S)_\tau \longrightarrow (\textit{Spaces}/S)_\tau
$$
is a special cocontinuous functor
(Sites, Definition \ref{sites-definition-special-cocontinuous-functor})
and hence identifies topoi.
\end{lemma}

\begin{proof}
The conditions of Sites, Lemma \ref{sites-lemma-equivalence}
are immediately verified as our functor is fully faithful
and as every algebraic space has an \'etale covering by schemes.
\end{proof}







\section{Proper base change}
\label{section-proper-base-change}

\noindent
The proper base change theorem for algebraic spaces follows from
the proper base change theorem for schemes and Chow's lemma
with a little bit of work.

\begin{lemma}
\label{lemma-surjective-proper}
Let $S$ be a scheme. Let $f : Y \to X$ be a surjective proper morphism
of algebraic spaces over $S$. Let $\mathcal{F}$ be a sheaf on $X_\etale$.
Then $\mathcal{F} \to f_*f^{-1}\mathcal{F}$ is injective with
image the equalizer of the two maps
$f_*f^{-1}\mathcal{F} \to g_*g^{-1}\mathcal{F}$ where
$g$ is the structure morphism $g : Y \times_X Y \to X$.
\end{lemma}

\begin{proof}
For any surjective morphism $f : Y \to X$ of algebraic spaces over $S$,
the map $\mathcal{F} \to f_*f^{-1}\mathcal{F}$ is injective.
Namely, if $\overline{x}$ is a geometric point of $X$, then we
choose a geometric point $\overline{y}$ of $Y$ lying over $\overline{x}$
and we consider
$$
\mathcal{F}_{\overline{x}} \to
(f_*f^{-1}\mathcal{F})_{\overline{x}} \to
(f^{-1}\mathcal{F})_{\overline{y}} = \mathcal{F}_{\overline{x}}
$$
See Properties of Spaces, Lemma \ref{spaces-properties-lemma-stalk-pullback}
for the last equality.

\medskip\noindent
The second statement is local on $X$ in the \'etale topology, hence we may
and do assume $Y$ is an affine scheme.

\medskip\noindent
Choose a surjective proper morphism $Z \to Y$ where $Z$ is a scheme, see
Cohomology of Spaces, Lemma \ref{spaces-cohomology-lemma-weak-chow}.
The result for $Z \to X$ implies the result for $Y \to X$.
Since $Z \to X$ is a surjective proper morphism of schemes
and hence a ph covering
(Topologies, Lemma \ref{topologies-lemma-surjective-proper-ph})
the result for $Z \to X$ follows from
\'Etale Cohomology, Lemma \ref{etale-cohomology-lemma-describe-pullback-pi-ph}
(in fact it is in some sense equivalent to this lemma).
\end{proof}

\begin{lemma}
\label{lemma-h0-proper-over-henselian-pair}
Let $(A, I)$ be a henselian pair. Let $X$ be an algebraic space over $A$
such that the structure morphism $f : X \to \Spec(A)$ is proper.
Let $i : X_0 \to X$ be the inclusion of $X \times_{\Spec(A)} \Spec(A/I)$.
For any sheaf $\mathcal{F}$ on $X_\etale$ we
have $\Gamma(X, \mathcal{F}) = \Gamma(Z, i^{-1}\mathcal{F})$.
\end{lemma}

\begin{proof}
Choose a surjective proper morphism $Y \to X$ where $Y$ is a scheme, see
Cohomology of Spaces, Lemma \ref{spaces-cohomology-lemma-weak-chow}.
Consider the diagram
$$
\xymatrix{
\Gamma(X_0, \mathcal{F}_0) \ar[r] \ar[d] &
\Gamma(Y_0, \mathcal{G}_0) \ar@<1ex>[r] \ar@<-1ex>[r] \ar[d] &
\Gamma((Y \times_X Y)_0, \mathcal{H}_0) \ar[d] \\
\Gamma(X, \mathcal{F}) \ar[r] &
\Gamma(Y, \mathcal{G}) \ar@<1ex>[r] \ar@<-1ex>[r] &
\Gamma(Y \times_X Y, \mathcal{H})
}
$$
Here $\mathcal{G}$, resp.\ $\mathcal{H}$ is the pullbackf or
$\mathcal{F}$ to $Y$, resp.\ $Y \times_X Y$ and the index $0$
indicates base change to $\Spec(A/I)$. By the case of schemes
(\'Etale Cohomology, Lemma
\ref{etale-cohomology-lemma-h0-proper-over-henselian-pair})
we see that the middle and right vertical arrows are bijective.
By Lemma \ref{lemma-surjective-proper} it follows that the left one is too.
\end{proof}

\begin{lemma}
\label{lemma-h0-proper-over-henselian-local}
Let $A$ be a henselian local ring. Let $X$ be an algebraic space
over $A$ such that $f : X \to \Spec(A)$
be a proper morphism. Let $X_0 \subset X$ be the fibre of
$f$ over the closed point. For any sheaf $\mathcal{F}$ on $X_\etale$ we
have $\Gamma(X, \mathcal{F}) = \Gamma(X_0, \mathcal{F}|_{X_0})$.
\end{lemma}

\begin{proof}
This is a special case of Lemma \ref{lemma-h0-proper-over-henselian-pair}.
\end{proof}

\begin{lemma}
\label{lemma-proper-base-change-f-star}
Let $S$ be a scheme. Let $f : X \to Y$ and $g : Y' \to Y$
be a morphisms of algebraic spaces over $S$. Assume $f$ is proper.
Set $X' = Y' \times_Y X$ with projections $f' : X' \to Y'$ and $g' : X' \to X$.
Let $\mathcal{F}$ be any sheaf on $X_\etale$. Then
$g^{-1}f_*\mathcal{F} = f'_*(g')^{-1}\mathcal{F}$.
\end{lemma}

\begin{proof}
The question is \'etale local on $Y'$. Choose a scheme $V$ and a surjective
\'etale morphism $V \to Y$. Choose a scheme $V'$ and a surjective \'etale
morphism $V' \to V \times_Y Y'$. Then we may replace $Y'$ by $V'$ and
$Y$ by $V$. Hence we may assume $Y$ and $Y'$ are schemes.
Then we may work Zariski locally on $Y$ and $Y'$ and hence we may
assume $Y$ and $Y'$ are affine schemes.

\medskip\noindent
Assume $Y$ and $Y'$ are affine schemes. Choose a surjective proper morphism
$h_1 : X_1 \to X$ where $X_1$ is a scheme, see
Cohomology of Spaces, Lemma \ref{spaces-cohomology-lemma-weak-chow}.
Set $X_2 = X_1 \times_X X_1$ and denote
$h_2 : X_2 \to X$ the structure morphism. Observe this is a scheme.
By the case of schemes
(\'Etale Cohomology, Lemma
\ref{etale-cohomology-lemma-proper-base-change-f-star})
we know the lemma is true for the cartesian diagrams
$$
\vcenter{
\xymatrix{
X'_1 \ar[r] \ar[d] & X_1 \ar[d] \\
Y' \ar[r] & Y
}
}
\quad\text{and}\quad
\vcenter{
\xymatrix{
X'_2 \ar[r] \ar[d] & X_2 \ar[d] \\
Y' \ar[r] & Y
}
}
$$
and the sheaves $\mathcal{F}_i = (X_i \to X)^{-1}\mathcal{F}$.
By Lemma \ref{lemma-surjective-proper} we have an exact sequence
$0 \to \mathcal{F} \to h_{1, *}\mathcal{F}_1 \to h_{2, *}\mathcal{F}_2$
and similarly for $(g')^{-1}\mathcal{F}$ because
$X'_2 = X'_1 \times_{X'} X'_1$. Hence we conlude that the
lemma is true (some details omitted).
\end{proof}

\noindent
Let $S$ be a scheme.
Let $f : Y \to X$ be a morphism of algebraic spaces over $S$. Let
$\overline{x} : \Spec(k) \to S$ be a geometric point. The fibre
of $f$ at $\overline{x}$ is the algebraic space
$Y_{\overline{x}} = \Spec(k) \times_{\overline{x}, X} Y$ over $\Spec(k)$.
If $\mathcal{F}$ is a sheaf on $Y_\etale$, then denote
$\mathcal{F}_{\overline{x}} = p^{-1}\mathcal{F}$
the pullback of $\mathcal{F}$ to $(Y_{\overline{x}})_\etale$.
Here $p : Y_{\overline{x}} \to Y$ is the projection.
In the following we will consider the set
$\Gamma(Y_{\overline{x}}, \mathcal{F}_{\overline{x}})$.

\begin{lemma}
\label{lemma-proper-pushforward-stalk}
Let $S$ be a scheme.
Let $f : Y \to X$ be a proper morphism of algebraic spaces over $S$. Let
$\overline{x} \to X$ be a geometric point.
For any sheaf $\mathcal{F}$ on $Y_\etale$
the canonical map
$$
(f_*\mathcal{F})_{\overline{x}} \longrightarrow
\Gamma(Y_{\overline{x}}, \mathcal{F}_{\overline{x}})
$$
is bijective.
\end{lemma}

\begin{proof}
This is a special case of Lemma \ref{lemma-proper-base-change-f-star}.
\end{proof}

\begin{theorem}
\label{theorem-proper-base-change}
Let $S$ be a scheme. Let
$$
\xymatrix{
X' \ar[r]_{g'} \ar[d]_{f'} & X \ar[d]^f \\
Y' \ar[r]^g & Y
}
$$
be a cartesian square of algebraic spaces over $S$.
Assume $f$ is proper.
Let $\mathcal{F}$ be an abelian torsion sheaf on $X_\etale$.
Then the base change map
$$
g^{-1}Rf_*\mathcal{F} \longrightarrow Rf'_*(g')^{-1}\mathcal{F}
$$
is an isomorphism.
\end{theorem}

\begin{proof}
This proof repeats a few of the arguments given in the proof of the
proper base change theorem for schemes. See
\'Etale Cohomology, Section \ref{etale-cohomology-section-proper-base-change}
for more details.

\medskip\noindent
The statement is \'etale local on $Y'$ and $Y$, hence we may assume
both $Y$ and $Y'$ are affine schemes. Observe that this in particular
proves the theorem in case $f$ is representable (we will use this
below).

\medskip\noindent
For every $n \geq 1$ let $\mathcal{F}[n]$ be the subsheaf of sections
of $\mathcal{F}$ annihilated by $n$. Then
$\mathcal{F} = \colim \mathcal{F}[n]$. By 
Cohomology of Spaces, Lemma \ref{spaces-cohomology-lemma-colimit-cohomology}
the functors $g^{-1}R^pf_*$ and $R^pf'_*(g')^{-1}$ commute
with filtered colimits. Hence it suffices to prove the theorem
if $\mathcal{F}$ is killed by $n$.

\medskip\noindent
Let $\mathcal{F} \to \mathcal{I}^\bullet$ be a resolution by
injective sheaves of $\mathbf{Z}/n\mathbf{Z}$-modules.
Observe that
$g^{-1}f_*\mathcal{I}^\bullet = f'_*(g')^{-1}\mathcal{I}^\bullet$
by Lemma \ref{lemma-proper-base-change-f-star}.
Applying Leray's acyclicity lemma
(Derived Categories, Lemma \ref{derived-lemma-leray-acyclicity})
we conclude it suffices to prove
$R^pf'_*(g')^{-1}\mathcal{I}^m = 0$ for $p > 0$ and $m \in \mathbf{Z}$.

\medskip\noindent
Choose a surjective proper morphism
$h : Z \to X$ where $Z$ is a scheme, see
Cohomology of Spaces, Lemma \ref{spaces-cohomology-lemma-weak-chow}.
Choose an injective map $h^{-1}\mathcal{I}^m \to \mathcal{J}$
where $\mathcal{J}$ is an injective sheaf of
$\mathbf{Z}/n\mathbf{Z}$-modules on $Z_\etale$.
Since $h$ is surjective the map $\mathcal{I}^m \to h_*\mathcal{J}$
is injective (see Lemma \ref{lemma-surjective-proper}).
Since $\mathcal{I}^m$ is injective we see that $\mathcal{I}^m$
is a direct summand of $h_*\mathcal{J}$. Thus it suffices
to prove the desired vanishing for $h_*\mathcal{J}$.

\medskip\noindent
Denote $h'$ the base change by $g$ and denote $g'' : Z' \to Z$
the projection. There is a spectral sequence
$$
E_2^{p, q} = R^pf'_* R^qh'_* (g'')^{-1}\mathcal{J}
$$
converging to $R^{p + q}(f' \circ h')_*(g'')^{-1}\mathcal{J}$.
Since $h$ and $f \circ h$ are representable (by schemes)
we know the result we want holds for them. Thus in the
spectral sequence we see that $E_2^{p, q} = 0$ for $q > 0$
and $R^{p + q}(f' \circ h')_*(g'')^{-1}\mathcal{J} = 0$
for $p + q > 0$. It follows that $E_2^{p, 0} = 0$ for $p > 0$.
Now
$$
E_2^{p, 0} = R^pf'_* h'_* (g'')^{-1}\mathcal{J} =
R^pf'_* (g')^{-1}h_*\mathcal{J}
$$
by Lemma \ref{lemma-proper-base-change-f-star}. This finishes the proof.
\end{proof}

\begin{lemma}
\label{lemma-proper-base-change}
Let $S$ be a scheme. Let
$$
\xymatrix{
X' \ar[r]_{g'} \ar[d]_{f'} & X \ar[d]^f \\
Y' \ar[r]^g & Y
}
$$
be a cartesian square of algebraic spaces over $S$. Assume $f$ is proper.
Let $E \in D^+(X_\etale)$ have torsion cohomology sheaves.
Then the base change map $g^{-1}Rf_*E \to Rf'_*(g')^{-1}E$
is an isomorphism.
\end{lemma}

\begin{proof}
This is a simple consequence of the proper base change theorem
(Theorem \ref{theorem-proper-base-change}) using the spectral
sequences
$$
E_2^{p, q} = R^pf_*H^q(E)
\quad\text{and}\quad
{E'}_2^{p, q} = R^pf'_*(g')^{-1}H^q(E)
$$
converging to $R^nf_*E$ and $R^nf'_*(g')^{-1}E$.
The spectral sequences are constructed in
Derived Categories, Lemma \ref{derived-lemma-two-ss-complex-functor}.
Some details omitted.
\end{proof}

\begin{lemma}
\label{lemma-proper-base-change-stalk}
Let $S$ be a scheme.
Let $f : X \to Y$ be a proper morphism of algebraic spaces.
Let $\overline{y} \to Y$ be a geometric point.
\begin{enumerate}
\item For a torsion abelian sheaf $\mathcal{F}$ on $X_\etale$ we have
$(R^nf_*\mathcal{F})_{\overline{y}} =
H^n_\etale(X_{\overline{y}}, \mathcal{F}_{\overline{y}})$.
\item For $E \in D^+(X_\etale)$ with torsion cohomology sheaves we have
$(R^nf_*E)_{\overline{y}} = H^n_\etale(X_{\overline{y}}, E_{\overline{y}})$.
\end{enumerate}
\end{lemma}

\begin{proof}
In the statement, $\mathcal{F}_{\overline{y}}$ denotes the pullback
of $\mathcal{F}$ to $X_{\overline{y}} = \overline{y} \times_Y X$.
Since pulling back by $\overline{y} \to Y$ produces the
stalk of $\mathcal{F}$, the first statement of the lemma
is a special case of Theorem \ref{theorem-proper-base-change}.
The second one is a special case of Lemma \ref{lemma-proper-base-change}.
\end{proof}

\begin{lemma}
\label{lemma-base-change-separably-closed}
Let $k \subset k'$ be an extension of separably closed fields.
Let $X$ be a proper algebraic space over $k$.
Let $\mathcal{F}$ be a torsion abelian sheaf on $X$.
Then the map $H^q_\etale(X, \mathcal{F}) \to
H^q_\etale(X_{k'}, \mathcal{F}|_{X_{k'}})$ is an isomorphism
for $q \geq 0$.
\end{lemma}

\begin{proof}
This is a special case of Theorem \ref{theorem-proper-base-change}.
\end{proof}










\section{Comparing big and small topoi}
\label{section-compare}

\noindent
Let $S$ be a scheme and let $X$ be an algebraic space over $S$.
In Topologies on Spaces, Lemma
\ref{spaces-topologies-lemma-at-the-bottom-etale}
we have introduced comparison morphisms
$\pi_X : (\textit{Spaces}/X)_\etale \to X_{spaces, \etale}$ and
$i_X : \Sh(X_\etale) \to \Sh((\textit{Spaces}/X)_\etale)$
with $\pi_X \circ i_X = \text{id}$ as morphisms of topoi and
$\pi_{X, *} = i_X^{-1}$.
More generally, if $f : Y \to X$ is an object of $(\textit{Spaces}/X)_\etale$,
then there is a morphism
$i_f : \Sh(Y_\etale) \to \Sh((\textit{Spaces}/X)_\etale)$
such that $f_{small} = \pi_X \circ i_f$, see
Topologies on Spaces, Lemmas \ref{spaces-topologies-lemma-put-in-T-etale} and
\ref{spaces-topologies-lemma-morphism-big-small-etale}. In
Topologies on Spaces, Remark
\ref{spaces-topologies-remark-change-topologies-ringed}
we have extended these to a morphism of ringed sites
$$
\pi_X :
((\textit{Spaces}/X)_\etale, \mathcal{O})
\to
(X_{spaces, \etale}, \mathcal{O}_X)
$$
and morphisms of ringed topoi
$$
i_X :
(\Sh(X_\etale), \mathcal{O}_X)
\to
(\Sh((\textit{Spaces}/X)_\etale), \mathcal{O})
$$
and
$$
i_f :
(\Sh(Y_\etale), \mathcal{O}_Y)
\to
(\Sh((\textit{Spaces}/X)_\etale, \mathcal{O}))
$$
Note that the restriction $i_X^{-1} = \pi_{X, *}$ (see
Topologies, Definition \ref{topologies-definition-restriction-small-etale})
transforms $\mathcal{O}$ into $\mathcal{O}_X$.
Similarly, $i_f^{-1}$ transforms $\mathcal{O}$ into $\mathcal{O}_Y$.
See Topologies on Spaces, Remark
\ref{spaces-topologies-remark-change-topologies-ringed}.
Hence $i_X^*\mathcal{F} = i_X^{-1}\mathcal{F}$ and
$i_f^*\mathcal{F} = i_f^{-1}\mathcal{F}$ for any $\mathcal{O}$-module
$\mathcal{F}$ on $(\textit{Spaces}/X)_\etale$. In particular $i_X^*$ and $i_f^*$
are exact functors. The functor $i_X^*$ is often denoted
$\mathcal{F} \mapsto \mathcal{F}|_{X_\etale}$ (and this does not
conflict with the notation in
Topologies on Spaces, Definition
\ref{spaces-topologies-definition-restriction-small-etale}).

\begin{lemma}
\label{lemma-describe-pullback}
Let $S$ be a scheme. Let $X$ be an algebraic space over $S$.
Let $\mathcal{F}$ be a sheaf on $X_\etale$. Then
$\pi_X^{-1}\mathcal{F}$ is given by the rule
$$
(\pi_X^{-1}\mathcal{F})(Y) = \Gamma(Y_\etale, f_{small}^{-1}\mathcal{F})
$$
for $f : Y \to X$ in $(\textit{Spaces}/X)_\etale$.
Moreover, $\pi_Y^{-1}\mathcal{F}$ satisfies the
sheaf condition with respect to smooth, syntomic, fppf, fpqc, and ph coverings.
\end{lemma}

\begin{proof}
Since pullback is transitive and $f_{small} = \pi_X \circ i_f$
(see above) we see that
$i_f^{-1} \pi_X^{-1}\mathcal{F} = f_{small}^{-1}\mathcal{F}$.
This shows that $\pi_X^{-1}$ has the description given in the lemma.

\medskip\noindent
To prove that $\pi_X^{-1}\mathcal{F}$ is a sheaf for the ph topology
it suffices by Topologies on Spaces, Lemma
\ref{spaces-topologies-lemma-characterize-sheaf}
to show that for a surjective proper morphism
$V \to U$ of algebraic spaces over $X$ we have
$(\pi_X^{-1}\mathcal{F})(U)$ is the equalizer of the two maps
$(\pi_X^{-1}\mathcal{F})(V) \to (\pi_X^{-1}\mathcal{F})(V \times_U V)$.
This we have seen in Lemma \ref{lemma-surjective-proper}.

\medskip\noindent
The case of smooth, syntomic, fppf coverings follows from the case
of ph coverings by Topologies on Spaces, Lemma
\ref{spaces-topologies-lemma-zariski-etale-smooth-syntomic-fppf-ph}.

\medskip\noindent
Let $\mathcal{U} = \{U_i \to U\}_{i \in I}$ be an fpqc covering of algebraic
spaces over $X$. Let $s_i \in (\pi_X^{-1}\mathcal{F})(U_i)$ be sections
which agree over $U_i \times_U U_j$. We have to prove there exists a unique
$s \in (\pi_X^{-1}\mathcal{F})(U)$ restricting to $s_i$ over $U_i$.
Case I: $U$ and $U_i$ are schemes. This case follows from
\'Etale Cohomology, Lemma \ref{etale-cohomology-lemma-describe-pullback}.
Case II: $U$ is a scheme. Here we choose surjective \'etale morphisms
$T_i \to U_i$ where $T_i$ is a scheme. Then $\mathcal{T} = \{T_i \to U\}$ is an
fpqc covering by schemes and by case I the result holds for $\mathcal{T}$.
We omit the verification that this implies the result for $\mathcal{U}$.
Case III: general case. Let $W \to U$ be a surjective \'etale
morphism, where $W$ is a scheme. Then $\mathcal{W} = \{U_i \times_U W \to W\}$
is an fpqc covering (by algebraic spaces) of the scheme $W$.
By case II the result hold for $\mathcal{W}$.
We omit the verification that this implies the result for $\mathcal{U}$.
\end{proof}

\begin{lemma}
\label{lemma-compare-injectives}
Let $S$ be a scheme.
Let $Y \to X$ be a morphism of $(\textit{Spaces}/S)_\etale$.
\begin{enumerate}
\item If $\mathcal{I}$ is injective in
$\textit{Ab}((\textit{Spaces}/X)_\etale)$, then
\begin{enumerate}
\item $i_f^{-1}\mathcal{I}$ is injective in $\textit{Ab}(Y_\etale)$,
\item $\mathcal{I}|_{X_\etale}$ is injective in $\textit{Ab}(X_\etale)$,
\end{enumerate}
\item If $\mathcal{I}^\bullet$ is a K-injective complex
in $\textit{Ab}((\textit{Spaces}/X)_\etale)$, then
\begin{enumerate}
\item $i_f^{-1}\mathcal{I}^\bullet$ is a K-injective complex in
$\textit{Ab}(Y_\etale)$,
\item $\mathcal{I}^\bullet|_{X_\etale}$ is a K-injective complex in
$\textit{Ab}(X_\etale)$,
\end{enumerate}
\end{enumerate}
The corresponding statements for modules do not hold.
\end{lemma}

\begin{proof}
Parts (1)(b) and (2)(b)
follow formally from the fact that the restriction functor
$\pi_{X, *} = i_X^{-1}$ is a right adjoint of the exact functor
$\pi_X^{-1}$, see
Homology, Lemma \ref{homology-lemma-adjoint-preserve-injectives} and
Derived Categories, Lemma \ref{derived-lemma-adjoint-preserve-K-injectives}.

\medskip\noindent
Parts (1)(a) and (2)(a) can be seen in two ways. First proof: We can use
that $i_f^{-1}$ is a right adjoint of the exact functor $i_{f, !}$.
This functor is constructed in
Topologies, Lemma \ref{topologies-lemma-put-in-T-etale}
for sheaves of sets and for abelian sheaves in
Modules on Sites, Lemma \ref{sites-modules-lemma-g-shriek-adjoint}.
It is shown in Modules on Sites, Lemma
\ref{sites-modules-lemma-exactness-lower-shriek} that it is exact.
Second proof. We can use that $i_f = i_Y \circ f_{big}$ as is shown
in Topologies, Lemma \ref{topologies-lemma-morphism-big-small-etale}.
Since $f_{big}$ is a localization, we see that pullback by it
preserves injectives and K-injectives, see
Cohomology on Sites, Lemmas \ref{sites-cohomology-lemma-cohomology-of-open} and
\ref{sites-cohomology-lemma-restrict-K-injective-to-open}.
Then we apply the already proved parts (1)(b) and (2)(b)
to the functor $i_Y^{-1}$ to conclude.

\medskip\noindent
To see a counter example for the case of modules we refer to
\'Etale Cohomology, Lemma \ref{etale-cohomology-lemma-compare-injectives}.
\end{proof}

\noindent
Let $S$ be a scheme.
Let $f : Y \to X$ be a morphism of algebraic spaces over $S$.
The commutative diagram of
Topologies on Spaces, Lemma
\ref{spaces-topologies-lemma-morphism-big-small-etale} (3)
leads to a commutative diagram of ringed sites
$$
\xymatrix{
(Y_{spaces, \etale}, \mathcal{O}_Y) \ar[d]_{f_{spaces, \etale}} &
((\textit{Spaces}/Y)_\etale, \mathcal{O}) \ar[d]^{f_{big}} \ar[l]^{\pi_Y} \\
(X_{spaces, \etale}, \mathcal{O}_X) &
((\textit{Spaces}/X)_\etale, \mathcal{O}) \ar[l]_{\pi_X}
}
$$
as one easily sees by writing out the definitions of
$f_{small}^\sharp$, $f_{big}^\sharp$, $\pi_X^\sharp$, and $\pi_Y^\sharp$.
In particular this means that
\begin{equation}
\label{equation-compare-big-small}
(f_{big, *}\mathcal{F})|_{X_\etale} =
f_{small, *}(\mathcal{F}|_{Y_\etale})
\end{equation}
for any sheaf $\mathcal{F}$ on $(\textit{Spaces}/Y)_\etale$ and if
$\mathcal{F}$ is a sheaf of $\mathcal{O}$-modules, then
(\ref{equation-compare-big-small})
is an isomorphism of $\mathcal{O}_X$-modules on $X_\etale$.

\begin{lemma}
\label{lemma-compare-higher-direct-image}
Let $S$ be a scheme.
Let $f : Y \to X$ be a morphism of algebraic spaces over $S$.
\begin{enumerate}
\item For $K$ in $D((\textit{Spaces}/Y)_\etale)$ we have
$
(Rf_{big, *}K)|_{X_\etale} = Rf_{small, *}(K|_{Y_\etale})
$
in $D(X_\etale)$.
\item For $K$ in $D((\textit{Spaces}/Y)_\etale, \mathcal{O})$ we have
$
(Rf_{big, *}K)|_{X_\etale} = Rf_{small, *}(K|_{Y_\etale})
$
in $D(\textit{Mod}(X_\etale, \mathcal{O}_X))$.
\end{enumerate}
More generally, let $g : X' \to X$ be an object of
$(\textit{Spaces}/X)_\etale$. Consider the fibre product
$$
\xymatrix{
Y' \ar[r]_{g'} \ar[d]_{f'} & Y \ar[d]^f \\
X' \ar[r]^g & X
}
$$
Then
\begin{enumerate}
\item[(3)] For $K$ in $D((\textit{Spaces}/Y)_\etale)$ we have
$i_g^{-1}(Rf_{big, *}K) = Rf'_{small, *}(i_{g'}^{-1}K)$
in $D(X'_\etale)$.
\item[(4)] For $K$ in $D((\textit{Spaces}/Y)_\etale, \mathcal{O})$ we have
$i_g^*(Rf_{big, *}K) = Rf'_{small, *}(i_{g'}^*K)$
in $D(\textit{Mod}(X'_\etale, \mathcal{O}_{X'}))$.
\item[(5)] For $K$ in $D((\textit{Spaces}/Y)_\etale)$ we have
$g_{big}^{-1}(Rf_{big, *}K) = Rf'_{big, *}((g'_{big})^{-1}K)$
in $D((\textit{Spaces}/X')_\etale)$.
\item[(6)] For $K$ in $D((\textit{Spaces}/Y)_\etale, \mathcal{O})$ we have
$g_{big}^*(Rf_{big, *}K) = Rf'_{big, *}((g'_{big})^*K)$
in $D(\textit{Mod}(X'_\etale, \mathcal{O}_{X'}))$.
\end{enumerate}
\end{lemma}

\begin{proof}
Part (1) follows from
Lemma \ref{lemma-compare-injectives}
and (\ref{equation-compare-big-small})
on choosing a K-injective complex of abelian sheaves representing $K$.

\medskip\noindent
Part (3) follows from Lemma \ref{lemma-compare-injectives}
and Topologies, Lemma
\ref{topologies-lemma-morphism-big-small-cartesian-diagram-etale}
on choosing a K-injective complex of abelian sheaves representing $K$.

\medskip\noindent
Part (5) is Cohomology on Sites, Lemma
\ref{sites-cohomology-lemma-localize-cartesian-square}.

\medskip\noindent
Part (6) is  Cohomology on Sites, Lemma
\ref{sites-cohomology-lemma-localize-cartesian-square-modules}.

\medskip\noindent
Part (2) can be proved as follows. Above we have seen
that $\pi_X \circ f_{big} = f_{small} \circ \pi_Y$ as morphisms
of ringed sites. Hence we obtain
$R\pi_{X, *} \circ Rf_{big, *} = Rf_{small, *} \circ R\pi_{Y, *}$
by Cohomology on Sites, Lemma
\ref{sites-cohomology-lemma-derived-pushforward-composition}.
Since the restriction functors $\pi_{X, *}$ and $\pi_{Y, *}$
are exact, we conclude.

\medskip\noindent
Part (4) follows from part (6) and part (2) applied to $f' : Y' \to X'$.
\end{proof}

\noindent
Let $S$ be a scheme. Let $X$ be an algebraic space over $S$.
Let $\mathcal{H}$ be an abelian sheaf on
$(\textit{Spaces}/X)_\etale$. Recall that $H^n_\etale(U, \mathcal{H})$
denotes the cohomology of $\mathcal{H}$ over an object
$U$ of $(\textit{Spaces}/X)_\etale$.

\begin{lemma}
\label{lemma-compare-cohomology}
Let $S$ be a scheme.
Let $f : Y \to X$ be a morphism of algebraic spaces over $S$. Then
\begin{enumerate}
\item For $K$ in $D(X_\etale)$ we have
$H^n_\etale(X, \pi_X^{-1}K) = H^n(X_\etale, K)$.
\item For $K$ in $D(X_\etale, \mathcal{O}_X)$ we have
$H^n_\etale(X, L\pi_X^*K) = H^n(X_\etale, K)$.
\item For $K$ in $D(X_\etale)$ we have
$H^n_\etale(Y, \pi_X^{-1}K) = H^n(Y_\etale, f_{small}^{-1}K)$.
\item For $K$ in $D(X_\etale, \mathcal{O}_X)$ we have
$H^n_\etale(Y, L\pi_X^*K) = H^n(Y_\etale, Lf_{small}^*K)$.
\item For $M$ in $D((\textit{Spaces}/X)_\etale)$ we have
$H^n_\etale(Y, M) = H^n(Y_\etale, i_f^{-1}M)$.
\item For $M$ in $D((\textit{Spaces}/X)_\etale, \mathcal{O})$ we have
$H^n_\etale(Y, M) = H^n(Y_\etale, i_f^*M)$.
\end{enumerate}
\end{lemma}

\begin{proof}
To prove (5) represent $M$ by a K-injective complex of abelian sheaves
and apply Lemma \ref{lemma-compare-injectives}
and work out the definitions. Part (3) follows from
this as $i_f^{-1}\pi_X^{-1} = f_{small}^{-1}$. Part (1) is a special
case of (3).

\medskip\noindent
Part (6) follows from the very general Cohomology on Sites, Lemma
\ref{sites-cohomology-lemma-pullback-same-cohomology}. Then part
(4) follows because $Lf_{small}^* = i_f^* \circ L\pi_X^*$.
Part (2) is a special case of (4).
\end{proof}

\begin{lemma}
\label{lemma-cohomological-descent-etale}
Let $S$ be a scheme. Let $X$ be an algebraic space over $S$.
For $K \in D(X_\etale)$ the map
$$
K \longrightarrow R\pi_{X, *}\pi_X^{-1}K
$$
is an isomorphism where
$\pi_X : \Sh((\textit{Spaces}/X)_\etale) \to \Sh(X_\etale)$ is as above.
\end{lemma}

\begin{proof}
This is true because both $\pi_X^{-1}$ and $\pi_{X, *} = i_X^{-1}$
are exact functors and the composition $\pi_{X, *} \circ \pi_X^{-1}$
is the identity functor.
\end{proof}

\begin{lemma}
\label{lemma-compare-higher-direct-image-proper}
Let $S$ be a scheme.
Let $f : Y \to X$ be a proper morphism of algebraic spaces over $S$.
Then we have
\begin{enumerate}
\item $\pi_X^{-1} \circ f_{small, *} = f_{big, *} \circ \pi_Y^{-1}$
as functors $\Sh(Y_\etale) \to \Sh((\textit{Spaces}/X)_\etale)$,
\item $\pi_X^{-1}Rf_{small, *}K = Rf_{big, *}\pi_Y^{-1}K$
for $K$ in $D^+(Y_\etale)$ whose cohomology sheaves are torsion, and
\item $\pi_X^{-1}Rf_{small, *}K = Rf_{big, *}\pi_Y^{-1}K$
for all $K$ in $D(Y_\etale)$ if $f$ is finite.
\end{enumerate}
\end{lemma}

\begin{proof}
Proof of (1). Let $\mathcal{F}$ be a sheaf on $Y_\etale$.
Let $g : X' \to X$ be an object of $(\textit{Spaces}/X)_\etale$.
Consider the fibre product
$$
\xymatrix{
Y' \ar[r]_{f'} \ar[d]_{g'} & X' \ar[d]^g \\
Y \ar[r]^f & X
}
$$
Then we have
$$
(f_{big, *}\pi_Y^{-1}\mathcal{F})(X') =
(\pi_Y^{-1}\mathcal{F})(Y') =
((g'_{small})^{-1}\mathcal{F})(Y')  =
(f'_{small, *}(g'_{small})^{-1}\mathcal{F})(X')
$$
the second equality by Lemma \ref{lemma-describe-pullback}.
On the other hand
$$
(\pi_X^{-1}f_{small, *}\mathcal{F})(X') =
(g_{small}^{-1}f_{small, *}\mathcal{F})(X')
$$
again by Lemma \ref{lemma-describe-pullback}.
Hence by proper base change for sheaves of sets
(Lemma \ref{lemma-proper-base-change-f-star})
we conclude the two sets are canonically isomorphic.
The isomorphism is compatible with restriction mappings
and defines an isomorphism
$\pi_X^{-1}f_{small, *}\mathcal{F} = f_{big, *}\pi_Y^{-1}\mathcal{F}$.
Thus an isomorphism of functors
$\pi_X^{-1} \circ f_{small, *} = f_{big, *} \circ \pi_Y^{-1}$.

\medskip\noindent
Proof of (2). There is a canonical base change map
$\pi_X^{-1}Rf_{small, *}K \to Rf_{big, *}\pi_Y^{-1}K$
for any $K$ in $D(Y_\etale)$, see
Cohomology on Sites, Remark \ref{sites-cohomology-remark-base-change}.
To prove it is an isomorphism, it suffices to prove the pull back of
the base change map by $i_g : \Sh(X'_\etale) \to \Sh((\Sch/X)_\etale)$
is an isomorphism for any object $g : X' \to X$ of $(\Sch/X)_\etale$.
Let $T', g', f'$ be as in the previous paragraph.
The pullback of the base change map is
\begin{align*}
g_{small}^{-1}Rf_{small, *}K
& =
i_g^{-1}\pi_X^{-1}Rf_{small, *}K \\
& \to
i_g^{-1}Rf_{big, *}\pi_Y^{-1}K \\
& =
Rf'_{small, *}(i_{g'}^{-1}\pi_Y^{-1}K) \\
& =
Rf'_{small, *}((g'_{small})^{-1}K)
\end{align*}
where we have used $\pi_X \circ i_g = g_{small}$,
$\pi_Y \circ i_{g'} = g'_{small}$, and
Lemma \ref{lemma-compare-higher-direct-image}.
This map is an isomorphism by the proper base change theorem
(Lemma \ref{lemma-proper-base-change}) provided $K$ is bounded
below and the cohomology sheaves of $K$ are torsion.

\medskip\noindent
Proof of (3). If $f$ is finite, then the functors
$f_{small, *}$ and $f_{big, *}$ are exact. This follows
from Cohomology of Spaces, Lemma
\ref{spaces-cohomology-lemma-finite-higher-direct-image-zero}
for $f_{small}$. Since any base change $f'$ of $f$ is finite too,
we conclude from Lemma \ref{lemma-compare-higher-direct-image} part (3)
that $f_{big, *}$ is exact too (as the higher derived functors are zero).
Thus this case follows from part (1).
\end{proof}







\section{Comparing fppf and \'etale topologies}
\label{section-fppf-etale}

\noindent
This section is the analogue of
\'Etale Cohomology, Section \ref{etale-cohomology-section-fppf-etale}.

\medskip\noindent
Let $S$ be a scheme. Let $X$ be an algebraic space over $S$.
On the category $\textit{Spaces}/X$ we consider the fppf
and \'etale topologies. The identity functor
$(\textit{Spaces}/X)_\etale \to (\textit{Spaces}/X)_{fppf}$
is continuous and defines a morphism of sites
$$
\epsilon_X :
(\textit{Spaces}/X)_{fppf} \longrightarrow (\textit{Spaces}/X)_\etale
$$
by an application of Sites, Proposition \ref{sites-proposition-get-morphism}.
Please note that $\epsilon_{X, *}$ is the identity functor on underlying
presheaves and that $\epsilon_X^{-1}$ associates to an \'etale sheaf the
fppf sheafification.
Consider the morphism of sites
$$
\pi_X : (\textit{Spaces}/X)_\etale \longrightarrow X_{spaces, \etale}
$$
comparing big and small \'etale sites, see Section \ref{section-compare}.
The composition determines a morphism of sites
$$
a_X = \pi_X \circ \epsilon_X :
(\textit{Spaces}/X)_{fppf}
\longrightarrow
X_{spaces, \etale}
$$
If $\mathcal{H}$ is an abelian sheaf on $(\textit{Spaces}/X)_{fppf}$,
then we will write $H^n_{fppf}(U, \mathcal{H})$ for the cohomology
of $\mathcal{H}$ over an object $U$ of $(\textit{Spaces}/X)_{fppf}$.

\begin{lemma}
\label{lemma-comparison-fppf-etale}
Let $S$ be a scheme. Let $X$ be an algebraic space over $S$.
\begin{enumerate}
\item For $\mathcal{F} \in \Sh(X_\etale)$ we have
$\epsilon_{X, *}a_X^{-1}\mathcal{F} = \pi_X^{-1}\mathcal{F}$
and $a_{X, *}a_X^{-1}\mathcal{F} = \mathcal{F}$.
\item For $\mathcal{F} \in \textit{Ab}(X_\etale)$ we have
$R^i\epsilon_{X, *}(a_X^{-1}\mathcal{F}) = 0$ for $i > 0$.
\end{enumerate}
\end{lemma}

\begin{proof}
We have $a_X^{-1}\mathcal{F} = \epsilon_X^{-1} \pi_X^{-1}\mathcal{F}$.
By Lemma \ref{lemma-describe-pullback} the \'etale sheaf
$\pi_X^{-1}\mathcal{F}$ is a sheaf for the fppf topology
and therefore is equal to $a_X^{-1}\mathcal{F}$ (as pulling
back by $\epsilon_X$ is given by fppf sheafification).
Recall moreover that $\epsilon_{X, *}$ is the identity
on underlying presheaves.
Now part (1) is immediate from the explicit description of $\pi_X^{-1}$
in Lemma \ref{lemma-describe-pullback}.

\medskip\noindent
We will prove part (2) by reducing it to the case of schemes --
see part (1) of
\'Etale Cohomology, Lemma \ref{etale-cohomology-lemma-V-C-all-n-etale-fppf}.
This will ``clearly work'' as every algebraic space is
\'etale locally a scheme. The details are given below but we urge
the reader to skip the proof.

\medskip\noindent
For an abelian sheaf $\mathcal{H}$ on $(\textit{Spaces}/X)_{fppf}$ the
higher direct image $R^p\epsilon_{X, *}\mathcal{H}$ is the sheaf
associated to the presheaf $U \mapsto H^p_{fppf}(U, \mathcal{H})$
on $(\textit{Spaces}/X)_\etale$. See
Cohomology on Sites, Lemma \ref{sites-cohomology-lemma-higher-direct-images}.
Since every object of $(\textit{Spaces}/X)_\etale$ has a covering
by schemes, it suffices to prove that given $U/X$ a scheme and
$\xi \in H^p_{fppf}(U, a_X^{-1}\mathcal{F})$ we can find
an \'etale covering $\{U_i \to U\}$ such that $\xi$
restricts to zero on $U_i$. We have
\begin{align*}
H^p_{fppf}(U, a_X^{-1}\mathcal{F})
& =
H^p((\textit{Spaces}/U)_{fppf}, (a_X^{-1}\mathcal{F})|_{\textit{Spaces}/U}) \\
& =
H^p((\Sch/U)_{fppf}, (a_X^{-1}\mathcal{F})|_{\Sch/U})
\end{align*}
where the second identification is
Lemma \ref{lemma-compare-cohomology-other-topologies}
and the first is a general fact about restriction
(Cohomology on Sites, Lemma \ref{sites-cohomology-lemma-cohomology-of-open}).
Looking at the first paragraph and the corresponding result in the
case of schemes (\'Etale Cohomology, Lemma
\ref{etale-cohomology-lemma-describe-pullback-pi-fppf})
we conclude that the sheaf $(a_X^{-1}\mathcal{F})|_{\Sch/U}$
matches the pullback by the ``schemes version of $a_U$''.
Therefore we can find an \'etale covering
$\{U_i \to U\}$ such that our class dies in
$H^p((\Sch/U_i)_{fppf}, (a_X^{-1}\mathcal{F})|_{\Sch/U_i})$
for each $i$, see
\'Etale Cohomology, Lemma \ref{etale-cohomology-lemma-V-C-all-n-etale-fppf}
(the precise statement one should use here is that $V_n$ holds for all $n$
which is the statement of part (2) for the case of schemes).
Transporting back (using the same formulas as above but now for
$U_i$) we conclude $\xi$ restricts to zero over $U_i$ as desired.
\end{proof}

\noindent
The hard work done in the case of schemes now tells us that
\'etale and fppf cohomology agree for sheaves coming from the
small \'etale site.

\begin{lemma}
\label{lemma-cohomological-descent-etale-fppf}
Let $S$ be a scheme. Let $X$ be an algebraic space over $S$.
For $K \in D^+(X_\etale)$ the maps
$$
\pi_X^{-1}K \longrightarrow R\epsilon_{X, *}a_X^{-1}K
\quad\text{and}\quad
K \longrightarrow Ra_{X, *}a_X^{-1}K
$$
are isomorphisms with
$a_X : \Sh((\textit{Spaces}/X)_{fppf}) \to \Sh(X_\etale)$ as above.
\end{lemma}

\begin{proof}
We only prove the second statement; the first is easier and proved in exactly
the same manner.
There is an immediate reduction to the case where
$K$ is given by a single abelian sheaf. Namely, represent $K$
by a bounded below complex $\mathcal{F}^\bullet$. By the case of a
sheaf we see that
$\mathcal{F}^n = a_{X, *} a_X^{-1} \mathcal{F}^n$
and that the sheaves $R^qa_{X, *}a_X^{-1}\mathcal{F}^n$
are zero for $q > 0$. By Leray's acyclicity lemma
(Derived Categories, Lemma \ref{derived-lemma-leray-acyclicity})
applied to $a_X^{-1}\mathcal{F}^\bullet$
and the functor $a_{X, *}$ we conclude. From now on assume $K = \mathcal{F}$.

\medskip\noindent
By Lemma \ref{lemma-comparison-fppf-etale} we have
$a_{X, *}a_X^{-1}\mathcal{F} = \mathcal{F}$. Thus it suffices to show that
$R^qa_{X, *}a_X^{-1}\mathcal{F} = 0$ for $q > 0$.
For this we can use $a_X = \epsilon_X \circ \pi_X$ and
the Leray spectral sequence
(Cohomology on Sites, Lemma \ref{sites-cohomology-lemma-relative-Leray}).
By Lemma \ref{lemma-comparison-fppf-etale}
we have $R^i\epsilon_{X, *}(a_X^{-1}\mathcal{F}) = 0$ for $i > 0$.
We have
$\epsilon_{X, *}a_X^{-1}\mathcal{F} = \pi_X^{-1}\mathcal{F}$
and by Lemma \ref{lemma-cohomological-descent-etale} we have
$R^j\pi_{X, *}(\pi_X^{-1}\mathcal{F}) = 0$ for $j > 0$.
This concludes the proof.
\end{proof}

\begin{lemma}
\label{lemma-compare-cohomology-etale-fppf}
Let $S$ be a scheme and let $X$ be an algebraic space over $S$.
With $a_X : \Sh((\textit{Spaces}/X)_{fppf}) \to \Sh(X_\etale)$
as above:
\begin{enumerate}
\item $H^q(X_\etale, \mathcal{F}) = H^q_{fppf}(X, a_X^{-1}\mathcal{F})$
for an abelian sheaf $\mathcal{F}$ on $X_\etale$,
\item $H^q(X_\etale, K) = H^q_{fppf}(X, a_X^{-1}K)$ for $K \in D^+(X_\etale)$.
\end{enumerate}
Example: if $A$ is an abelian group, then
$H^q_\etale(X, \underline{A}) = H^q_{fppf}(X, \underline{A})$.
\end{lemma}

\begin{proof}
This follows from Lemma \ref{lemma-cohomological-descent-etale-fppf}
by Cohomology on Sites, Remark \ref{sites-cohomology-remark-before-Leray}.
\end{proof}

\begin{lemma}
\label{lemma-push-pull-fppf-etale}
Let $S$ be a scheme.
Let $f : X \to Y$ be a morphism of algebraic spaces over $S$.
Then there are commutative diagrams of topoi
$$
\xymatrix{
\Sh((\textit{Spaces}/X)_{fppf}) \ar[rr]_{f_{big, fppf}} \ar[d]_{\epsilon_X} & &
\Sh((\textit{Spaces}/Y)_{fppf}) \ar[d]^{\epsilon_Y} \\
\Sh((\textit{Spaces}/X)_\etale) \ar[rr]^{f_{big, \etale}} & &
\Sh((\textit{Spaces}/Y)_\etale)
}
$$
and
$$
\xymatrix{
\Sh((\textit{Spaces}/X)_{fppf}) \ar[rr]_{f_{big, fppf}} \ar[d]_{a_X} & &
\Sh((\textit{Spaces}/Y)_{fppf}) \ar[d]^{a_Y} \\
\Sh(X_\etale) \ar[rr]^{f_{small}} & &
\Sh(Y_\etale)
}
$$
with $a_X = \pi_X \circ \epsilon_X$ and $a_Y = \pi_X \circ \epsilon_X$.
\end{lemma}

\begin{proof}
This follows immediately from working out the definitions of the
morphisms involved, see
Topologies on Spaces, Section \ref{spaces-topologies-section-fppf}
and Section \ref{section-compare}.
\end{proof}

\begin{lemma}
\label{lemma-proper-push-pull-fppf-etale}
In Lemma \ref{lemma-push-pull-fppf-etale} if $f$ is proper, then we have
\begin{enumerate}
\item $a_Y^{-1} \circ f_{small, *} = f_{big, fppf, *} \circ a_X^{-1}$, and
\item
$a_Y^{-1}(Rf_{small, *}K) = Rf_{big, fppf, *}(a_X^{-1}K)$
for $K$ in $D^+(X_\etale)$ with torsion cohomology sheaves.
\end{enumerate}
\end{lemma}

\begin{proof}
Proof of (1). You can prove this by repeating the proof of
Lemma \ref{lemma-compare-higher-direct-image-proper} part (1);
we will instead deduce the result from this.
As $\epsilon_{Y, *}$ is the identity functor on underlying presheaves,
it reflects isomorphisms. Lemma \ref{lemma-comparison-fppf-etale}
shows that $\epsilon_{Y, *} \circ a_Y^{-1} = \pi_Y^{-1}$
and similarly for $X$. To show that the canonical map
$a_Y^{-1}f_{small, *}\mathcal{F} \to f_{big, fppf, *}a_X^{-1}\mathcal{F}$
is an isomorphism, it suffices to show that
\begin{align*}
\pi_Y^{-1}f_{small, *}\mathcal{F}
& =
\epsilon_{Y, *}a_Y^{-1}f_{small, *}\mathcal{F} \\
& \to 
\epsilon_{Y, *}f_{big, fppf, *}a_X^{-1}\mathcal{F} \\
& =
f_{big, \etale, *} \epsilon_{X, *}a_X^{-1}\mathcal{F} \\
& =
f_{big, \etale, *}\pi_X^{-1}\mathcal{F}
\end{align*}
is an isomorphism. This is part
(1) of Lemma \ref{lemma-compare-higher-direct-image-proper}.

\medskip\noindent
To see (2) we use that
\begin{align*}
R\epsilon_{Y, *}Rf_{big, fppf, *}a_X^{-1}K
& =
Rf_{big, \etale, *}R\epsilon_{X, *}a_X^{-1}K \\
& =
Rf_{big, \etale, *}\pi_X^{-1}K \\
& =
\pi_Y^{-1}Rf_{small, *}K \\
& =
R\epsilon_{Y, *} a_Y^{-1}Rf_{small, *}K
\end{align*}
The first equality by the commutative diagram in
Lemma \ref{lemma-push-pull-fppf-etale}
and Cohomology on Sites, Lemma
\ref{sites-cohomology-lemma-derived-pushforward-composition}.
Then second equality is Lemma \ref{lemma-cohomological-descent-etale-fppf}.
The third is
Lemma \ref{lemma-compare-higher-direct-image-proper} part (2).
The fourth is Lemma \ref{lemma-cohomological-descent-etale-fppf} again.
Thus the base change map
$a_Y^{-1}(Rf_{small, *}K) \to Rf_{big, fppf, *}(a_X^{-1}K)$
induces an isomorphism
$$
R\epsilon_{Y, *}a_Y^{-1}Rf_{small, *}K \to
R\epsilon_{Y, *}Rf_{big, fppf, *}a_X^{-1}K
$$
The proof is finished by the following remark: a map
$\alpha : a_Y^{-1}L \to M$ with $L$ in $D^+(Y_\etale)$
and $M$ in $D^+((\textit{Spaces}/Y)_{fppf})$ such that
$R\epsilon_{Y, *}\alpha$ is an isomorphism, is an isomorphism. Namely, 
we show by induction on $i$ that $H^i(\alpha)$ is an isomorphism.
This is true for all sufficiently small $i$.
If it holds for $i \leq i_0$, then we see that
$R^j\epsilon_{Y, *}H^i(M) = 0$ for $j > 0$ and $i \leq i_0$
by Lemma \ref{lemma-comparison-fppf-etale}
because $H^i(M) = a_Y^{-1}H^i(L)$ in this range.
Hence $\epsilon_{Y, *}H^{i_0 + 1}(M) = H^{i_0 + 1}(R\epsilon_{Y, *}M)$
by a spectral sequence argument.
Thus $\epsilon_{Y, *}H^{i_0 + 1}(M) = \pi_Y^{-1}H^{i_0 + 1}(L) =
\epsilon_{Y, *}a_Y^{-1}H^{i_0 + 1}(L)$.
This implies $H^{i_0 + 1}(\alpha)$ is an isomorphism
(because $\epsilon_{Y, *}$ reflects isomorphisms as it is the
identity on underlying presheaves) as desired.
\end{proof}

\begin{lemma}
\label{lemma-finite-push-pull-fppf-etale}
In Lemma \ref{lemma-push-pull-fppf-etale} if $f$ is finite, then
$a_Y^{-1}(Rf_{small, *}K) = Rf_{big, fppf, *}(a_X^{-1}K)$
for $K$ in $D^+(X_\etale)$.
\end{lemma}

\begin{proof}
Let $V \to Y$ be a surjective \'etale morphism where $V$ is a scheme.
It suffices to prove the base change map is an isomorphism after
restricting to $V$. Hence we may assume that $Y$ is a scheme.
As the morphism is finite, hence representable, we conclude
that we may assume both $X$ and $Y$ are schemes. In this case
the result follows from the case of schemes
(\'Etale Cohomology, Lemma \ref{etale-cohomology-lemma-V-C-all-n-etale-fppf}
part (2)) using the comparison of topoi discussed in
Section \ref{section-api}
and in particular given in
Lemma \ref{lemma-compare-cohomology-other-topologies}.
Some details omitted.
\end{proof}

\begin{lemma}
\label{lemma-descent-sheaf-fppf-etale}
In Lemma \ref{lemma-push-pull-fppf-etale} assume
$f$ is flat, locally of finite presentation, and surjective.
Then the functor
$$
\Sh(Y_\etale) \longrightarrow
\left\{
(\mathcal{G}, \mathcal{H}, \alpha)
\middle|
\begin{matrix}
\mathcal{G} \in \Sh(X_\etale),\ \mathcal{H} \in \Sh((\Sch/Y)_{fppf}), \\
\alpha : a_X^{-1}\mathcal{G} \to f_{big, fppf}^{-1}\mathcal{H}
\text{ an isomorphism}
\end{matrix}
\right\}
$$
sending $\mathcal{F}$ to
$(f_{small}^{-1}\mathcal{F}, a_Y^{-1}\mathcal{F}, can)$ is an equivalence.
\end{lemma}

\begin{proof}
The functor $a_X^{-1}$ is fully faithful (as $a_{X, *}a_X^{-1} = \text{id}$ by
Lemma \ref{lemma-comparison-fppf-etale}). Hence the forgetful functor
$(\mathcal{G}, \mathcal{H}, \alpha) \mapsto \mathcal{H}$ identifies the
category of triples with a full subcategory of $\Sh((\Sch/Y)_{fppf})$.
Moreover, the functor $a_Y^{-1}$ is fully faithful, hence the functor
in the lemma is fully faithful as well.

\medskip\noindent
Suppose that we have an \'etale covering $\{Y_i \to Y\}$.
Let $f_i : X_i \to Y_i$ be the base change of $f$.
Denote $f_{ij} = f_i \times f_j : X_i \times_X X_j  \to Y_i \times_Y Y_j$.
Claim: if the lemma is true for $f_i$ and $f_{ij}$ for all $i, j$, then
the lemma is true for $f$. To see this, note that the given \'etale covering
determines an \'etale covering of the final object in each of
the four sites $Y_\etale, X_\etale, (\Sch/Y)_{fppf}, (\Sch/X)_{fppf}$.
Thus the category of sheaves is equivalent to the category of
glueing data for this covering
(Sites, Lemma \ref{sites-lemma-mapping-property-glue})
in each of the four cases. A huge commutative diagram of
categories then finishes the proof of the claim. We omit the details.
The claim shows that we may work \'etale locally on $Y$.
In particular, we may assume $Y$ is a scheme.

\medskip\noindent
Assume $Y$ is a scheme. Choose a scheme $X'$ and a surjective \'etale
morphism $s : X' \to X$. Set $f' = f \circ s : X' \to Y$ and observe
that $f'$ is surjective, locally of finite presentation, and flat.
Claim: if the lemma is true for $f'$, then it is true for $f$.
Namely, given a triple $(\mathcal{G}, \mathcal{H}, \alpha)$
for $f$, we can pullback by $s$ to get a triple
$(s_{small}^{-1}\mathcal{G}, \mathcal{H}, s_{big, fppf}^{-1}\alpha)$
for $f'$. A solution for this triple gives a sheaf
$\mathcal{F}$ on $Y_\etale$ with $a_Y^{-1}\mathcal{F} = \mathcal{H}$.
By the first paragraph of the proof this means the triple is
in the essential image. This reduces us to the case
where both $X$ and $Y$ are schemes. This case follows from
\'Etale Cohomology, Lemma
\ref{etale-cohomology-lemma-descent-sheaf-fppf-etale}
via the discussion in Section \ref{section-api}
and in particular Lemma \ref{lemma-compare-cohomology-other-topologies}.
\end{proof}



\section{Comparing fppf and \'etale topologies: modules}
\label{section-fppf-etale-modules}

\noindent
We continue the discussion in Section \ref{section-fppf-etale} but in this
section we briefly discuss what happens for sheaves of modules.

\medskip\noindent
Let $S$ be a scheme. Let $X$ be an algebraic space over $S$.
The morphisms of sites $\epsilon_X$, $\pi_X$, and
their composition $a_X$ introduced in Section \ref{section-fppf-etale}
have natural enhancements to morphisms of ringed sites. The first
is written as
$$
\epsilon_X :
((\textit{Spaces}/X)_{fppf}, \mathcal{O})
\longrightarrow
((\textit{Spaces}/X)_\etale, \mathcal{O})
$$
Note that we can use the same symbol for the structure sheaf as indeed
the sheaves have the same underlying presheaf. The second is
$$
\pi_X :
((\textit{Spaces}/X)_\etale, \mathcal{O})
\longrightarrow
(X_\etale, \mathcal{O}_X)
$$
The third is the morphism
$$
a_X :
((\textit{Spaces}/X)_{fppf}, \mathcal{O})
\longrightarrow
(X_\etale, \mathcal{O}_X)
$$
Let us review what we already know about quasi-coherent
modules on these sites.

\begin{lemma}
\label{lemma-review-quasi-coherent}
Let $S$ be a scheme. Let $X$ be an algebraic space over $S$.
Let $\mathcal{F}$ be a quasi-coherent $\mathcal{O}_X$-module.
\begin{enumerate}
\item The rule
$$
\mathcal{F}^a : (\textit{Spaces}/X)_\etale \longrightarrow \textit{Ab},\quad
(f : Y \to X) \longmapsto \Gamma(Y, f^*\mathcal{F})
$$
satisfies the sheaf condition for fpqc and a fortiori
fppf and \'etale coverings,
\item $\mathcal{F}^a = \pi_X^*\mathcal{F}$ on $(\textit{Spaces}/X)_\etale$,
\item $\mathcal{F}^a = a_X^*\mathcal{F}$ on $(\textit{Spaces}/X)_{fppf}$,
\item the rule $\mathcal{F} \mapsto \mathcal{F}^a$ defines
an equivalence between quasi-coherent $\mathcal{O}_X$-modules
and quasi-coherent modules on
$((\textit{Spaces}/X)_\etale, \mathcal{O})$,
\item the rule $\mathcal{F} \mapsto \mathcal{F}^a$ defines
an equivalence between quasi-coherent $\mathcal{O}_X$-modules
and quasi-coherent modules on
$((\textit{Spaces}/X)_{fppf}, \mathcal{O})$,
\item we have $\epsilon_{X, *}a_X^*\mathcal{F} = \pi_X^*\mathcal{F}$
and $a_{X, *}a_X^*\mathcal{F} = \mathcal{F}$,
\item we have $R^i\epsilon_{X, *}(a_X^*\mathcal{F}) = 0$
and $R^ia_{X, *}(a_X^*\mathcal{F}) = 0$ for $i > 0$.
\end{enumerate}
\end{lemma}

\begin{proof}
Part (1) is a consequence of fppf descent of quasi-coherent modules.
Namely, suppose that $\{f_i : U_i \to U\}$ is an fpqc covering
in $(\textit{Spaces}/X)_\etale$. Denote $g : U \to X$ the structure
morphism. Suppose that
we have a family of sections $s_i \in \Gamma(U_i , f_i^*g^*\mathcal{F})$
such that $s_i|_{U_i \times_U U_j} = s_j|_{U_i \times_U U_j}$.
We have to find the correspond section $s \in \Gamma(U, g^*\mathcal{F})$.
We can reinterpret the $s_i$ as a family of maps
$\varphi_i : f_i^*\mathcal{O}_U = \mathcal{O}_{U_i} \to f_i^*g^*\mathcal{F}$
compatible with the canonical descent data associated to the
quasi-coherent sheaves $\mathcal{O}_U$ and $g^*\mathcal{F}$ on $U$.
Hence by Descent on Spaces, Proposition
\ref{spaces-descent-proposition-fpqc-descent-quasi-coherent}
we see that we may (uniquely) descend
these to a map $\mathcal{O}_U \to g^*\mathcal{F}$ which gives
us our section $s$.

\medskip\noindent
We will deduce (2) -- (7) from the corresponding statement for schemes.
Choose an \'etale covering $\{X_i \to X\}_{i \in I}$
where each $X_i$ is a scheme. Observe that $X_i \times_X X_j$
is a scheme too. This covering induces a covering of
the final object in each of the three sites
$(\textit{Spaces}/X)_{fppf}$, $(\textit{Spaces}/X)_\etale$, and $X_\etale$.
Hence we see that the category of sheaves on these sites
are equivalent to descent data for these coverings, see
Sites, Lemma \ref{sites-lemma-mapping-property-glue}.
Parts (2), (3) are local (because we have the glueing statement).
Being quasi-coherent is a local property, hence parts
(4), (5) are local. Clearly (6) and (7) are local.
It follows that it suffices to prove parts (2) -- (7)
of the lemma when $X$ is a scheme.

\medskip\noindent
Assume $X$ is a scheme. The embeddings
$(\Sch/X)_\etale \subset (\textit{Spaces}/X)_\etale$ and
$(\Sch/X)_{fppf} \subset (\textit{Spaces}/X)_{fppf}$
determine equivalences of ringed topoi by
Lemma \ref{lemma-compare-cohomology-other-topologies}.
We conclude that (2) -- (7) follows from the case of schemes.
\'Etale Cohomology, Lemma
\ref{etale-cohomology-lemma-review-quasi-coherent}.
To transport the property of being quasi-coherent via
this equivalence use that being quasi-coherent is an
intrinsic property of modules as explained in
Modules on Sites, Section \ref{sites-modules-section-local}.
Some minor details omitted.
\end{proof}

\begin{lemma}
\label{lemma-cohomological-descent-etale-fppf-modules}
Let $S$ be a scheme. Let $X$ be an algebraic space over $S$.
For $\mathcal{F}$ a quasi-coherent $\mathcal{O}_X$-module the maps
$$
\pi_X^*\mathcal{F} \longrightarrow R\epsilon_{X, *}(a_X^*\mathcal{F})
\quad\text{and}\quad
\mathcal{F} \longrightarrow Ra_{X, *}(a_X^*\mathcal{F})
$$
are isomorphisms.
\end{lemma}

\begin{proof}
This is an immediate consequence of parts (6) and (7) of
Lemma \ref{lemma-review-quasi-coherent}.
\end{proof}

\begin{lemma}
\label{lemma-vanishing-adequate}
Let $S$ be a scheme. Let $X$ be an algebraic space over $S$.
Let $\mathcal{F}_1 \to \mathcal{F}_2 \to \mathcal{F}_3$
be a complex of quasi-coherent $\mathcal{O}_X$-modules.
Set
$$
\mathcal{H}_\etale =
\Ker(\pi_X^*\mathcal{F}_2 \to \pi_X^*\mathcal{F}_3)/
\Im(\pi_X^*\mathcal{F}_1 \to \pi_X^*\mathcal{F}_2)
$$
on $(\textit{Spaces}/X)_\etale$ and set
$$
\mathcal{H}_{fppf} =
\Ker(a_X^*\mathcal{F}_2 \to a_X^*\mathcal{F}_3)/
\Im(a_X^*\mathcal{F}_1 \to a_X^*\mathcal{F}_2)
$$
on $(\textit{Spaces}/X)_{fppf}$.
Then $\mathcal{H}_\etale = \epsilon_{X, *}\mathcal{H}_{fppf}$
and
$$
H^p_\etale(U, \mathcal{H}_\etale) = H^p_{fppf}(U, \mathcal{H}_{fppf}) = 0
$$
for $p > 0$ and any affine object $U$ of $(\textit{Spaces}/X)_\etale$.
\end{lemma}

\noindent
More is true, namely the collection of modules on
$(\textit{Spaces}/X)_{fppf}$ which fppf locally look like those
in the lemma are called adquate modules. They form a
weak Serre subcategory of the category of all $\mathcal{O}$-modules
and their cohomology is studied in
Adequate Modules, Section \ref{adequate-section-adequate}.

\begin{proof}
For any object $f : U \to X$ of $(\textit{Spaces}/X)_\etale$
consider the restriction
$\mathcal{H}_\etale|_{U_\etale}$ of $\mathcal{H}_\etale$ to $U_\etale$ via
the functor $i_f^* = i_f^{-1}$ discussed in Section \ref{section-compare}.
The sheaf $\mathcal{H}_\etale|_{U_\etale}$
is equal to the homology of complex $f^*\mathcal{F}_\bullet$ in degree $1$.
This is true because $i_f \circ \pi_X = f$ as morphisms of ringed sites
$U_\etale \to X_\etale$. In particular we see that
$\mathcal{H}_\etale|_{U_\etale}$ is a quasi-coherent $\mathcal{O}_U$-module.
Next, let $g : V \to U$ be a flat
morphism in $(\textit{Spaces}/X)_\etale$. Since
$$
i_{f \circ g}^* \circ \pi_X^* = (f \circ g)^* = g^* \circ f^*
$$
as morphisms of sites $V_\etale \to X_\etale$ and since $g$ is flat
hence $g^*$ is exact, we obtain
$$
\mathcal{H}_\etale|_{V_\etale} =
g^*\left(\mathcal{H}_\etale|_{U_\etale}\right)
$$
With these preparations we are ready to prove the lemma.

\medskip\noindent
Let $\mathcal{U} = \{g_i : U_i \to U\}_{i \in I}$ be an fppf covering
with $f : U \to X$ as above. The sheaf propery holds for
$\mathcal{H}_\etale$ and the covering $\mathcal{U}$
by (1) of Lemma \ref{lemma-review-quasi-coherent}
applied to $\mathcal{H}_\etale|_{U_\etale}$ and the above.
Therefore we see that $\mathcal{H}_\etale$ is already an fppf
sheaf and this means that $\mathcal{H}_{fppf}$ is
equal to $\mathcal{H}_\etale$
as a presheaf. In particular
$\mathcal{H}_\etale = \epsilon_{X, *}\mathcal{H}_{fppf}$.

\medskip\noindent
Finally, to prove the vanishing, we use
Cohomology on Sites, Lemma \ref{sites-cohomology-lemma-cech-vanish-collection}.
We let $\mathcal{B}$ be the affine objects of
$(\textit{Spaces}/X)_{fppf}$ and we let
$\text{Cov}$ be the set of finite fppf coverings
$\mathcal{U} = \{U_i \to U\}_{i = 1, \ldots, n}$ with $U$, $U_i$ affine.
We have
$$
{\check H}^p(\mathcal{U}, \mathcal{H}_\etale) =
{\check H}^p(\mathcal{U}, \left(\mathcal{H}_\etale|_{U_\etale}\right)^a)
$$
because the values of $\mathcal{H}_\etale$ on the
affine schemes $U_{i_0} \times_U \ldots \times_U U_{i_p}$
flat over $U$ agree with the values of the pullback
of the quasi-coherent module $\mathcal{H}_\etale|_{U_\etale}$ by
the first paragraph. Hence we obtain vanishing by
Descent, Lemma \ref{descent-lemma-standard-covering-Cech-quasi-coherent}.
This finishes the proof.
\end{proof}

\begin{lemma}
\label{lemma-cohomological-descent-etale-fppf-modules-unbounded}
Let $S$ be a scheme. Let $X$ be an algebraic space over $S$.
For $K \in D_\QCoh(\mathcal{O}_X)$ the maps
$$
L\pi_X^*K \longrightarrow R\epsilon_{X, *}(La_X^*\mathcal{F})
\quad\text{and}\quad
K \longrightarrow Ra_{X, *}(La_X^*K)
$$
are isomorphisms. Here
$a_X : \Sh((\textit{Spaces}/X)_{fppf}) \to \Sh(X_\etale)$ is as above.
\end{lemma}

\begin{proof}
The question is \'etale local on $X$ hence we may assume $X$ is affine.
Say $X = \Spec(A)$. Then we have $D_\QCoh(\mathcal{O}_X) = D(A)$ by
Derived Categories of Spaces, Lemma
\ref{spaces-perfect-lemma-derived-quasi-coherent-small-etale-site}
and
Derived Categories of Schemes, Lemma
\ref{perfect-lemma-affine-compare-bounded}.
Hence we can choose an K-flat complex of $A$-modules
$K^\bullet$ whose corresponding complex
$\mathcal{K}^\bullet$ of quasi-coherent $\mathcal{O}_X$-modules
represents $K$.
We claim that $\mathcal{K}^\bullet$ is a K-flat complex
of $\mathcal{O}_X$-modules.

\medskip\noindent
Proof of the claim. By
Derived Categories of Schemes, Lemma
\ref{perfect-lemma-affine-K-flat}
we see that $\widetilde{K}^\bullet$ is K-flat on the scheme
$(\Spec(A), \mathcal{O}_{\Spec(A)})$.
Next, note that $\mathcal{K}^\bullet = \epsilon^*\widetilde{K}^\bullet$
where $\epsilon$ is as in Derived Categories of Spaces, Lemma
\ref{spaces-perfect-lemma-derived-quasi-coherent-small-etale-site}
whence $\mathcal{K}^\bullet$ is K-flat by
Cohomology on Sites, Lemma \ref{sites-cohomology-lemma-pullback-K-flat-points}
and the fact that the \'etale site of a scheme has enough points
(\'Etale Cohomology, Remarks \ref{etale-cohomology-remarks-enough-points}).

\medskip\noindent
By the claim we see that
$La_X^*K = a_X^*\mathcal{K}^\bullet$ and
$L\pi_X^*K = \pi_X^*\mathcal{K}^\bullet$.
Since the first part of the proof shows that the pullback
$a_X^*\mathcal{K}^n$ of the quasi-coherent module
is acyclic for $\epsilon_{X, *}$, resp.\ $a_{X, *}$, surely the proof is done
by Leray's acyclicity lemma? Actually..., no because Leray's
acyclicity lemma only applies to bounded below complexes.
However, in the next paragraph we will show the result does follow
from the bounded below case because our complex is the derived limit
of bounded below complexes of quasi-coherent modules.

\medskip\noindent
The cohomology sheaves of
$\pi_X^*\mathcal{K}^\bullet$ and $a_X^*\mathcal{K}^\bullet$
have vanishing higher cohomology
groups over affine objects of $(\textit{Spaces}/X)_\etale$ by
Lemma \ref{lemma-vanishing-adequate}.
Therefore we have
$$
L\pi_X^*K = R\lim \tau_{\geq -n}(L\pi_X^*K)
\quad\text{and}\quad
La_X^*K = R\lim \tau_{\geq -n}(La_X^*K)
$$
by Cohomology on Sites, Lemma \ref{sites-cohomology-lemma-is-limit-dimension}.

\medskip\noindent
Proof of $L\pi_X^*K = R\epsilon_{X, *}(La_X^*\mathcal{F})$.
By the above we have
$$
R\epsilon_{X, *}La_X^*K =
R\lim R\epsilon_{X, *}(\tau_{\geq -n}(La_X^*K))
$$
by Cohomology on Sites, Lemma
\ref{sites-cohomology-lemma-Rf-commutes-with-Rlim}.
Note that $\tau_{\geq -n}(La_X^*K)$ is represented by
$\tau_{\geq -n}(a_X^*\mathcal{K}^\bullet)$ which may not be the
same as $a_X^*(\tau_{\geq -n}\mathcal{K}^\bullet)$.
But clearly the systems
$$
\{\tau_{\geq -n}(a_X^*\mathcal{K}^\bullet)\}_{n \geq 1}
\quad\text{and}\quad
\{a_X^*(\tau_{\geq -n}\mathcal{K}^\bullet)\}_{n \geq 1}
$$
are isomorphic as pro-systems.
By Leray's acyclicity lemma
(Derived Categories, Lemma \ref{derived-lemma-leray-acyclicity})
and the first part of the lemma we see that
$$
R\epsilon_{X, *}(a_X^*(\tau_{\geq -n}\mathcal{K}^\bullet)) =
\pi_X^*(\tau_{\geq -n}\mathcal{K}^\bullet)
$$
Then we can use that the systems
$$
\{\tau_{\geq -n}(\pi_X^*\mathcal{K}^\bullet)\}_{n \geq 1}
\quad\text{and}\quad
\{\pi_X^*(\tau_{\geq -n}\mathcal{K}^\bullet)\}_{n \geq 1}
$$
are isomorphic as pro-systems. Finally, we put everything together as follows
\begin{align*}
R\epsilon_{X, *}La_X^*K
& =
R\epsilon_{X, *} (R\lim \tau_{\geq -n}(La_X^*K)) \\
& =
R\lim R\epsilon_{X, *}(\tau_{\geq -n}(La_X^*K)) \\
& =
R\lim R\epsilon_{X, *}(\tau_{\geq -n}(a_X^*\mathcal{K}^\bullet)) \\
& =
R\lim R\epsilon_{X, *}(a_X^*(\tau_{\geq -n}\mathcal{K}^\bullet)) \\
& =
R\lim \pi_X^*(\tau_{\geq -n}\mathcal{K}^\bullet) \\
& =
R\lim \tau_{\geq -n}(\pi_X^*\mathcal{K}^\bullet) \\
& =
R\lim \tau_{\geq -n}(L\pi_X^*K) \\
& =
L\pi_X^*K
\end{align*}
Here in equalities four and six we have used that isomorphic
pro-systems have the same $R\lim$ (small detail omitted).
You can avoid this step by using more about cohomology of the terms
of the complex $\tau_{\geq -n}a_X^*\mathcal{K}^\bullet$ proved
in Lemma \ref{lemma-vanishing-adequate} as this will prove
directly that $R\epsilon_{X, *}(\tau_{\geq -n}(a_X^*\mathcal{K}^\bullet)) =
\tau_{\geq -n}(\pi_X^*\mathcal{K}^\bullet)$.

\medskip\noindent
The equality $K = Ra_{X, *}(La_X^*\mathcal{F})$ is
proved in exactly the same way using in the final step that
$K = R\lim \tau_{\geq -n}K$ by
Derived Categories of Spaces, Lemma \ref{spaces-perfect-lemma-nice-K-injective}.
\end{proof}








\section{Comparing ph and \'etale topologies}
\label{section-ph-etale}

\noindent
This section is the analogue of
\'Etale Cohomology, Section \ref{etale-cohomology-section-ph-etale}.

\medskip\noindent
Let $S$ be a scheme. Let $X$ be an algebraic space over $S$.
On the category $\textit{Spaces}/X$ we consider the ph
and \'etale topologies. The identity functor
$(\textit{Spaces}/X)_\etale \to (\textit{Spaces}/X)_{ph}$
is continuous as every \'etale covering is a ph covering by
Topologies on Spaces, Lemma
\ref{spaces-topologies-lemma-zariski-etale-smooth-syntomic-fppf-ph}.
Hence it defines a morphism of sites
$$
\epsilon_X :
(\textit{Spaces}/X)_{ph} \longrightarrow (\textit{Spaces}/X)_\etale
$$
by an application of Sites, Proposition \ref{sites-proposition-get-morphism}.
Please note that $\epsilon_{X, *}$ is the identity functor on underlying
presheaves and that $\epsilon_X^{-1}$ associates to an \'etale sheaf the
ph sheafification.
Consider the morphism of sites
$$
\pi_X : (\textit{Spaces}/X)_\etale \longrightarrow X_{spaces, \etale}
$$
comparing big and small \'etale sites, see Section \ref{section-compare}.
The composition determines a morphism of sites
$$
a_X = \pi_X \circ \epsilon_X :
(\textit{Spaces}/X)_{ph}
\longrightarrow
X_{spaces, \etale}
$$
If $\mathcal{H}$ is an abelian sheaf on $(\textit{Spaces}/X)_{ph}$,
then we will write $H^n_{ph}(U, \mathcal{H})$ for the cohomology
of $\mathcal{H}$ over an object $U$ of $(\textit{Spaces}/X)_{ph}$.

\begin{lemma}
\label{lemma-comparison-ph-etale}
Let $S$ be a scheme. Let $X$ be an algebraic space over $S$.
\begin{enumerate}
\item For $\mathcal{F} \in \Sh(X_\etale)$ we have
$\epsilon_{X, *}a_X^{-1}\mathcal{F} = \pi_X^{-1}\mathcal{F}$
and $a_{X, *}a_X^{-1}\mathcal{F} = \mathcal{F}$.
\item For $\mathcal{F} \in \textit{Ab}(X_\etale)$ torsion we have
$R^i\epsilon_{X, *}(a_X^{-1}\mathcal{F}) = 0$ for $i > 0$.
\end{enumerate}
\end{lemma}

\begin{proof}
We have $a_X^{-1}\mathcal{F} = \epsilon_X^{-1} \pi_X^{-1}\mathcal{F}$.
By Lemma \ref{lemma-describe-pullback} the \'etale sheaf
$\pi_X^{-1}\mathcal{F}$ is a sheaf for the ph topology
and therefore is equal to $a_X^{-1}\mathcal{F}$ (as pulling
back by $\epsilon_X$ is given by ph sheafification).
Recall moreover that $\epsilon_{X, *}$ is the identity
on underlying presheaves.
Now part (1) is immediate from the explicit description of $\pi_X^{-1}$
in Lemma \ref{lemma-describe-pullback}.

\medskip\noindent
We will prove part (2) by reducing it to the case of schemes --
see part (1) of
\'Etale Cohomology, Lemma \ref{etale-cohomology-lemma-V-C-all-n-etale-ph}.
This will ``clearly work'' as every algebraic space is
\'etale locally a scheme. The details are given below but we urge
the reader to skip the proof.

\medskip\noindent
For an abelian sheaf $\mathcal{H}$ on $(\textit{Spaces}/X)_{ph}$ the
higher direct image $R^p\epsilon_{X, *}\mathcal{H}$ is the sheaf
associated to the presheaf $U \mapsto H^p_{ph}(U, \mathcal{H})$
on $(\textit{Spaces}/X)_\etale$. See
Cohomology on Sites, Lemma \ref{sites-cohomology-lemma-higher-direct-images}.
Since every object of $(\textit{Spaces}/X)_\etale$ has a covering
by schemes, it suffices to prove that given $U/X$ a scheme and
$\xi \in H^p_{ph}(U, a_X^{-1}\mathcal{F})$ we can find
an \'etale covering $\{U_i \to U\}$ such that $\xi$
restricts to zero on $U_i$. We have
\begin{align*}
H^p_{ph}(U, a_X^{-1}\mathcal{F})
& =
H^p((\textit{Spaces}/U)_{ph}, (a_X^{-1}\mathcal{F})|_{\textit{Spaces}/U}) \\
& =
H^p((\Sch/U)_{ph}, (a_X^{-1}\mathcal{F})|_{\Sch/U})
\end{align*}
where the second identification is
Lemma \ref{lemma-compare-cohomology-other-topologies}
and the first is a general fact about restriction
(Cohomology on Sites, Lemma \ref{sites-cohomology-lemma-cohomology-of-open}).
Looking at the first paragraph and the corresponding result in the
case of schemes (\'Etale Cohomology, Lemma
\ref{etale-cohomology-lemma-describe-pullback-pi-ph})
we conclude that the sheaf $(a_X^{-1}\mathcal{F})|_{\Sch/U}$
matches the pullback by the ``schemes version of $a_U$''.
Therefore we can find an \'etale covering
$\{U_i \to U\}$ such that our class dies in
$H^p((\Sch/U_i)_{ph}, (a_X^{-1}\mathcal{F})|_{\Sch/U_i})$
for each $i$, see
\'Etale Cohomology, Lemma \ref{etale-cohomology-lemma-V-C-all-n-etale-ph}
(the precise statement one should use here is that $V_n$ holds for all $n$
which is the statement of part (2) for the case of schemes).
Transporting back (using the same formulas as above but now for
$U_i$) we conclude $\xi$ restricts to zero over $U_i$ as desired.
\end{proof}

\noindent
The hard work done in the case of schemes now tells us that
\'etale and ph cohomology agree for torsion abelian sheaves coming from the
small \'etale site.

\begin{lemma}
\label{lemma-cohomological-descent-etale-ph}
Let $S$ be a scheme. Let $X$ be an algebraic space over $S$.
For $K \in D^+(X_\etale)$ with torsion cohomology sheaves the maps
$$
\pi_X^{-1}K \longrightarrow R\epsilon_{X, *}a_X^{-1}K
\quad\text{and}\quad
K \longrightarrow Ra_{X, *}a_X^{-1}K
$$
are isomorphisms with
$a_X : \Sh((\textit{Spaces}/X)_{ph}) \to \Sh(X_\etale)$ as above.
\end{lemma}

\begin{proof}
We only prove the second statement; the first is easier and proved in exactly
the same manner. There is a reduction to the case where
$K$ is given by a single torsion abelian sheaf. Namely, represent $K$
by a bounded below complex $\mathcal{F}^\bullet$ of torsion
abelian sheaves. This is possible by Cohomology on Sites, Lemma
\ref{sites-cohomology-lemma-torsion}.
By the case of a sheaf we see that
$\mathcal{F}^n = a_{X, *} a_X^{-1} \mathcal{F}^n$
and that the sheaves $R^qa_{X, *}a_X^{-1}\mathcal{F}^n$
are zero for $q > 0$. By Leray's acyclicity lemma
(Derived Categories, Lemma \ref{derived-lemma-leray-acyclicity})
applied to $a_X^{-1}\mathcal{F}^\bullet$
and the functor $a_{X, *}$ we conclude. From now on assume
$K = \mathcal{F}$ where $\mathcal{F}$ is a torsion abelian sheaf.

\medskip\noindent
By Lemma \ref{lemma-comparison-ph-etale} we have
$a_{X, *}a_X^{-1}\mathcal{F} = \mathcal{F}$. Thus it suffices to show that
$R^qa_{X, *}a_X^{-1}\mathcal{F} = 0$ for $q > 0$.
For this we can use $a_X = \epsilon_X \circ \pi_X$ and
the Leray spectral sequence
(Cohomology on Sites, Lemma \ref{sites-cohomology-lemma-relative-Leray}).
By Lemma \ref{lemma-comparison-ph-etale}
we have $R^i\epsilon_{X, *}(a_X^{-1}\mathcal{F}) = 0$ for $i > 0$.
We have
$\epsilon_{X, *}a_X^{-1}\mathcal{F} = \pi_X^{-1}\mathcal{F}$
and by Lemma \ref{lemma-cohomological-descent-etale} we have
$R^j\pi_{X, *}(\pi_X^{-1}\mathcal{F}) = 0$ for $j > 0$.
This concludes the proof.
\end{proof}

\begin{lemma}
\label{lemma-compare-cohomology-etale-ph}
Let $S$ be a scheme and let $X$ be an algebraic space over $S$.
With $a_X : \Sh((\textit{Spaces}/X)_{ph}) \to \Sh(X_\etale)$
as above:
\begin{enumerate}
\item $H^q(X_\etale, \mathcal{F}) = H^q_{ph}(X, a_X^{-1}\mathcal{F})$
for a torsion abelian sheaf $\mathcal{F}$ on $X_\etale$,
\item $H^q(X_\etale, K) = H^q_{ph}(X, a_X^{-1}K)$ for $K \in D^+(X_\etale)$
with torsion cohomology sheaves
\end{enumerate}
Example: if $A$ is a torsion abelian group, then
$H^q_\etale(X, \underline{A}) = H^q_{ph}(X, \underline{A})$.
\end{lemma}

\begin{proof}
This follows from Lemma \ref{lemma-cohomological-descent-etale-ph}
by Cohomology on Sites, Remark \ref{sites-cohomology-remark-before-Leray}.
\end{proof}

\begin{lemma}
\label{lemma-push-pull-ph-etale}
Let $S$ be a scheme.
Let $f : X \to Y$ be a morphism of algebraic spaces over $S$.
Then there are commutative diagrams of topoi
$$
\xymatrix{
\Sh((\textit{Spaces}/X)_{ph}) \ar[rr]_{f_{big, ph}} \ar[d]_{\epsilon_X} & &
\Sh((\textit{Spaces}/Y)_{ph}) \ar[d]^{\epsilon_Y} \\
\Sh((\textit{Spaces}/X)_\etale) \ar[rr]^{f_{big, \etale}} & &
\Sh((\textit{Spaces}/Y)_\etale)
}
$$
and
$$
\xymatrix{
\Sh((\textit{Spaces}/X)_{ph}) \ar[rr]_{f_{big, ph}} \ar[d]_{a_X} & &
\Sh((\textit{Spaces}/Y)_{ph}) \ar[d]^{a_Y} \\
\Sh(X_\etale) \ar[rr]^{f_{small}} & &
\Sh(Y_\etale)
}
$$
with $a_X = \pi_X \circ \epsilon_X$ and $a_Y = \pi_X \circ \epsilon_X$.
\end{lemma}

\begin{proof}
This follows immediately from working out the definitions of the
morphisms involved, see
Topologies on Spaces, Section \ref{spaces-topologies-section-ph}
and Section \ref{section-compare}.
\end{proof}

\begin{lemma}
\label{lemma-proper-push-pull-ph-etale}
In Lemma \ref{lemma-push-pull-ph-etale} if $f$ is proper, then we have
\begin{enumerate}
\item $a_Y^{-1} \circ f_{small, *} = f_{big, ph, *} \circ a_X^{-1}$, and
\item
$a_Y^{-1}(Rf_{small, *}K) = Rf_{big, ph, *}(a_X^{-1}K)$
for $K$ in $D^+(X_\etale)$ with torsion cohomology sheaves.
\end{enumerate}
\end{lemma}

\begin{proof}
Proof of (1). You can prove this by repeating the proof of
Lemma \ref{lemma-compare-higher-direct-image-proper} part (1);
we will instead deduce the result from this.
As $\epsilon_{Y, *}$ is the identity functor on underlying presheaves,
it reflects isomorphisms. Lemma \ref{lemma-comparison-ph-etale}
shows that $\epsilon_{Y, *} \circ a_Y^{-1} = \pi_Y^{-1}$
and similarly for $X$. To show that the canonical map
$a_Y^{-1}f_{small, *}\mathcal{F} \to f_{big, ph, *}a_X^{-1}\mathcal{F}$
is an isomorphism, it suffices to show that
\begin{align*}
\pi_Y^{-1}f_{small, *}\mathcal{F}
& =
\epsilon_{Y, *}a_Y^{-1}f_{small, *}\mathcal{F} \\
& \to 
\epsilon_{Y, *}f_{big, ph, *}a_X^{-1}\mathcal{F} \\
& =
f_{big, \etale, *} \epsilon_{X, *}a_X^{-1}\mathcal{F} \\
& =
f_{big, \etale, *}\pi_X^{-1}\mathcal{F}
\end{align*}
is an isomorphism. This is part
(1) of Lemma \ref{lemma-compare-higher-direct-image-proper}.

\medskip\noindent
To see (2) we use that
\begin{align*}
R\epsilon_{Y, *}Rf_{big, ph, *}a_X^{-1}K
& =
Rf_{big, \etale, *}R\epsilon_{X, *}a_X^{-1}K \\
& =
Rf_{big, \etale, *}\pi_X^{-1}K \\
& =
\pi_Y^{-1}Rf_{small, *}K \\
& =
R\epsilon_{Y, *} a_Y^{-1}Rf_{small, *}K
\end{align*}
The first equality by the commutative diagram in
Lemma \ref{lemma-push-pull-ph-etale}
and Cohomology on Sites, Lemma
\ref{sites-cohomology-lemma-derived-pushforward-composition}.
Then second equality is Lemma \ref{lemma-cohomological-descent-etale-ph}.
The third is
Lemma \ref{lemma-compare-higher-direct-image-proper} part (2).
The fourth is Lemma \ref{lemma-cohomological-descent-etale-ph} again.
Thus the base change map
$a_Y^{-1}(Rf_{small, *}K) \to Rf_{big, ph, *}(a_X^{-1}K)$
induces an isomorphism
$$
R\epsilon_{Y, *}a_Y^{-1}Rf_{small, *}K \to
R\epsilon_{Y, *}Rf_{big, ph, *}a_X^{-1}K
$$
The proof is finished by the following remark: consider a map
$\alpha : a_Y^{-1}L \to M$ with $L$ in $D^+(Y_\etale)$
having torsion cohomology sheaves
and $M$ in $D^+((\textit{Spaces}/Y)_{ph})$. If
$R\epsilon_{Y, *}\alpha$ is an isomorphism, then $\alpha$ is an isomorphism.
Namely, we show by induction on $i$ that $H^i(\alpha)$ is an isomorphism.
This is true for all sufficiently small $i$.
If it holds for $i \leq i_0$, then we see that
$R^j\epsilon_{Y, *}H^i(M) = 0$ for $j > 0$ and $i \leq i_0$
by Lemma \ref{lemma-comparison-ph-etale}
because $H^i(M) = a_Y^{-1}H^i(L)$ in this range.
Hence $\epsilon_{Y, *}H^{i_0 + 1}(M) = H^{i_0 + 1}(R\epsilon_{Y, *}M)$
by a spectral sequence argument.
Thus $\epsilon_{Y, *}H^{i_0 + 1}(M) = \pi_Y^{-1}H^{i_0 + 1}(L) =
\epsilon_{Y, *}a_Y^{-1}H^{i_0 + 1}(L)$.
This implies $H^{i_0 + 1}(\alpha)$ is an isomorphism
(because $\epsilon_{Y, *}$ reflects isomorphisms as it is the
identity on underlying presheaves) as desired.
\end{proof}













\section{Other chapters}

\begin{multicols}{2}
\begin{enumerate}
\item \hyperref[introduction-section-phantom]{Introduction}
\item \hyperref[conventions-section-phantom]{Conventions}
\item \hyperref[sets-section-phantom]{Set Theory}
\item \hyperref[categories-section-phantom]{Categories}
\item \hyperref[topology-section-phantom]{Topology}
\item \hyperref[sheaves-section-phantom]{Sheaves on Spaces}
\item \hyperref[algebra-section-phantom]{Commutative Algebra}
\item \hyperref[sites-section-phantom]{Sites and Sheaves}
\item \hyperref[homology-section-phantom]{Homological Algebra}
\item \hyperref[derived-section-phantom]{Derived Categories}
\item \hyperref[more-algebra-section-phantom]{More Algebra}
\item \hyperref[simplicial-section-phantom]{Simplicial Methods}
\item \hyperref[modules-section-phantom]{Sheaves of Modules}
\item \hyperref[sites-modules-section-phantom]{Modules on Sites}
\item \hyperref[injectives-section-phantom]{Injectives}
\item \hyperref[cohomology-section-phantom]{Cohomology of Sheaves}
\item \hyperref[sites-cohomology-section-phantom]{Cohomology on Sites}
\item \hyperref[hypercovering-section-phantom]{Hypercoverings}
\item \hyperref[schemes-section-phantom]{Schemes}
\item \hyperref[constructions-section-phantom]{Constructions of Schemes}
\item \hyperref[properties-section-phantom]{Properties of Schemes}
\item \hyperref[morphisms-section-phantom]{Morphisms of Schemes}
\item \hyperref[coherent-section-phantom]{Coherent Cohomology}
\item \hyperref[divisors-section-phantom]{Divisors}
\item \hyperref[limits-section-phantom]{Limits of Schemes}
\item \hyperref[varieties-section-phantom]{Varieties}
\item \hyperref[chow-section-phantom]{Chow Homology}
\item \hyperref[topologies-section-phantom]{Topologies on Schemes}
\item \hyperref[descent-section-phantom]{Descent}
\item \hyperref[more-morphisms-section-phantom]{More on Morphisms}
\item \hyperref[flat-section-phantom]{More on Flatness}
\item \hyperref[groupoids-section-phantom]{Groupoid Schemes}
\item \hyperref[more-groupoids-section-phantom]{More on Groupoid Schemes}
\item \hyperref[etale-section-phantom]{\'Etale Morphisms of Schemes}
\item \hyperref[etale-cohomology-section-phantom]{\'Etale Cohomology}
\item \hyperref[spaces-section-phantom]{Algebraic Spaces}
\item \hyperref[spaces-properties-section-phantom]{Properties of Algebraic Spaces}
\item \hyperref[spaces-morphisms-section-phantom]{Morphisms of Algebraic Spaces}
\item \hyperref[spaces-topologies-section-phantom]{Topologies on Algebraic Spaces}
\item \hyperref[spaces-descent-section-phantom]{Descent and Algebraic Spaces}
\item \hyperref[spaces-more-morphisms-section-phantom]{More on Morphisms of Spaces}
\item \hyperref[quot-section-phantom]{Quot and Hilbert Spaces}
\item \hyperref[stacks-section-phantom]{Stacks}
\item \hyperref[spaces-groupoids-section-phantom]{Groupoids in Algebraic Spaces}
\item \hyperref[spaces-more-groupoids-section-phantom]{More on Groupoids in Spaces}
\item \hyperref[bootstrap-section-phantom]{Bootstrap}
\item \hyperref[examples-stacks-section-phantom]{Examples of Stacks}
\item \hyperref[groupoids-quotients-section-phantom]{Quotients of Groupoids}
\item \hyperref[algebraic-section-phantom]{Algebraic Stacks}
\item \hyperref[criteria-section-phantom]{Criteria for Representability}
\item \hyperref[stacks-properties-section-phantom]{Properties of Algebraic Stacks}
\item \hyperref[stacks-morphisms-section-phantom]{Morphisms of Algebraic Stacks}
\item \hyperref[examples-section-phantom]{Examples}
\item \hyperref[exercises-section-phantom]{Exercises}
\item \hyperref[guide-section-phantom]{Guide to Literature}
\item \hyperref[desirables-section-phantom]{Desirables}
\item \hyperref[coding-section-phantom]{Coding Style}
\item \hyperref[fdl-section-phantom]{GNU Free Documentation License}
\item \hyperref[index-section-phantom]{Auto Generated Index}
\end{enumerate}
\end{multicols}


\bibliography{my}
\bibliographystyle{amsalpha}

\end{document}
