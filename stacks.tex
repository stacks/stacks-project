\IfFileExists{stacks-project.cls}{%
\documentclass{stacks-project}
}{%
\documentclass{amsart}
}

% The following AMS packages are automatically loaded with
% the amsart documentclass:
%\usepackage{amsmath}
%\usepackage{amssymb}
%\usepackage{amsthm}

% For dealing with references we use the comment environment
\usepackage{verbatim}
\newenvironment{reference}{\comment}{\endcomment}
%\newenvironment{reference}{}{}
\newenvironment{slogan}{\comment}{\endcomment}
\newenvironment{history}{\comment}{\endcomment}

% For commutative diagrams you can use
% \usepackage{amscd}
\usepackage[all]{xy}

% We use 2cell for 2-commutative diagrams.
\xyoption{2cell}
\UseAllTwocells

% To put source file link in headers.
% Change "template.tex" to "this_filename.tex"
% \usepackage{fancyhdr}
% \pagestyle{fancy}
% \lhead{}
% \chead{}
% \rhead{Source file: \url{template.tex}}
% \lfoot{}
% \cfoot{\thepage}
% \rfoot{}
% \renewcommand{\headrulewidth}{0pt}
% \renewcommand{\footrulewidth}{0pt}
% \renewcommand{\headheight}{12pt}

\usepackage{multicol}

% For cross-file-references
\usepackage{xr-hyper}

% Package for hypertext links:
\usepackage{hyperref}

% For any local file, say "hello.tex" you want to link to please
% use \externaldocument[hello-]{hello}
\externaldocument[introduction-]{introduction}
\externaldocument[conventions-]{conventions}
\externaldocument[sets-]{sets}
\externaldocument[categories-]{categories}
\externaldocument[topology-]{topology}
\externaldocument[sheaves-]{sheaves}
\externaldocument[sites-]{sites}
\externaldocument[stacks-]{stacks}
\externaldocument[fields-]{fields}
\externaldocument[algebra-]{algebra}
\externaldocument[brauer-]{brauer}
\externaldocument[homology-]{homology}
\externaldocument[derived-]{derived}
\externaldocument[simplicial-]{simplicial}
\externaldocument[more-algebra-]{more-algebra}
\externaldocument[smoothing-]{smoothing}
\externaldocument[modules-]{modules}
\externaldocument[sites-modules-]{sites-modules}
\externaldocument[injectives-]{injectives}
\externaldocument[cohomology-]{cohomology}
\externaldocument[sites-cohomology-]{sites-cohomology}
\externaldocument[dga-]{dga}
\externaldocument[dpa-]{dpa}
\externaldocument[hypercovering-]{hypercovering}
\externaldocument[schemes-]{schemes}
\externaldocument[constructions-]{constructions}
\externaldocument[properties-]{properties}
\externaldocument[morphisms-]{morphisms}
\externaldocument[coherent-]{coherent}
\externaldocument[divisors-]{divisors}
\externaldocument[limits-]{limits}
\externaldocument[varieties-]{varieties}
\externaldocument[topologies-]{topologies}
\externaldocument[descent-]{descent}
\externaldocument[perfect-]{perfect}
\externaldocument[more-morphisms-]{more-morphisms}
\externaldocument[flat-]{flat}
\externaldocument[groupoids-]{groupoids}
\externaldocument[more-groupoids-]{more-groupoids}
\externaldocument[etale-]{etale}
\externaldocument[chow-]{chow}
\externaldocument[intersection-]{intersection}
\externaldocument[pic-]{pic}
\externaldocument[adequate-]{adequate}
\externaldocument[dualizing-]{dualizing}
\externaldocument[duality-]{duality}
\externaldocument[discriminant-]{discriminant}
\externaldocument[local-cohomology-]{local-cohomology}
\externaldocument[curves-]{curves}
\externaldocument[resolve-]{resolve}
\externaldocument[models-]{models}
\externaldocument[pione-]{pione}
\externaldocument[etale-cohomology-]{etale-cohomology}
\externaldocument[proetale-]{proetale}
\externaldocument[crystalline-]{crystalline}
\externaldocument[spaces-]{spaces}
\externaldocument[spaces-properties-]{spaces-properties}
\externaldocument[spaces-morphisms-]{spaces-morphisms}
\externaldocument[decent-spaces-]{decent-spaces}
\externaldocument[spaces-cohomology-]{spaces-cohomology}
\externaldocument[spaces-limits-]{spaces-limits}
\externaldocument[spaces-divisors-]{spaces-divisors}
\externaldocument[spaces-over-fields-]{spaces-over-fields}
\externaldocument[spaces-topologies-]{spaces-topologies}
\externaldocument[spaces-descent-]{spaces-descent}
\externaldocument[spaces-perfect-]{spaces-perfect}
\externaldocument[spaces-more-morphisms-]{spaces-more-morphisms}
\externaldocument[spaces-flat-]{spaces-flat}
\externaldocument[spaces-groupoids-]{spaces-groupoids}
\externaldocument[spaces-more-groupoids-]{spaces-more-groupoids}
\externaldocument[bootstrap-]{bootstrap}
\externaldocument[spaces-pushouts-]{spaces-pushouts}
\externaldocument[groupoids-quotients-]{groupoids-quotients}
\externaldocument[spaces-more-cohomology-]{spaces-more-cohomology}
\externaldocument[spaces-simplicial-]{spaces-simplicial}
\externaldocument[formal-spaces-]{formal-spaces}
\externaldocument[restricted-]{restricted}
\externaldocument[spaces-resolve-]{spaces-resolve}
\externaldocument[formal-defos-]{formal-defos}
\externaldocument[defos-]{defos}
\externaldocument[cotangent-]{cotangent}
\externaldocument[examples-defos-]{examples-defos}
\externaldocument[algebraic-]{algebraic}
\externaldocument[examples-stacks-]{examples-stacks}
\externaldocument[stacks-sheaves-]{stacks-sheaves}
\externaldocument[criteria-]{criteria}
\externaldocument[artin-]{artin}
\externaldocument[quot-]{quot}
\externaldocument[stacks-properties-]{stacks-properties}
\externaldocument[stacks-morphisms-]{stacks-morphisms}
\externaldocument[stacks-limits-]{stacks-limits}
\externaldocument[stacks-cohomology-]{stacks-cohomology}
\externaldocument[stacks-perfect-]{stacks-perfect}
\externaldocument[stacks-introduction-]{stacks-introduction}
\externaldocument[stacks-more-morphisms-]{stacks-more-morphisms}
\externaldocument[stacks-geometry-]{stacks-geometry}
\externaldocument[moduli-]{moduli}
\externaldocument[moduli-curves-]{moduli-curves}
\externaldocument[examples-]{examples}
\externaldocument[exercises-]{exercises}
\externaldocument[guide-]{guide}
\externaldocument[desirables-]{desirables}
\externaldocument[coding-]{coding}
\externaldocument[obsolete-]{obsolete}
\externaldocument[fdl-]{fdl}
\externaldocument[index-]{index}

% Theorem environments.
%
\theoremstyle{plain}
\newtheorem{theorem}[subsection]{Theorem}
\newtheorem{proposition}[subsection]{Proposition}
\newtheorem{lemma}[subsection]{Lemma}

\theoremstyle{definition}
\newtheorem{definition}[subsection]{Definition}
\newtheorem{example}[subsection]{Example}
\newtheorem{exercise}[subsection]{Exercise}
\newtheorem{situation}[subsection]{Situation}

\theoremstyle{remark}
\newtheorem{remark}[subsection]{Remark}
\newtheorem{remarks}[subsection]{Remarks}

\numberwithin{equation}{subsection}

% Macros
%
\def\lim{\mathop{\rm lim}\nolimits}
\def\colim{\mathop{\rm colim}\nolimits}
\def\Spec{\mathop{\rm Spec}}
\def\Hom{\mathop{\rm Hom}\nolimits}
\def\Ext{\mathop{\rm Ext}\nolimits}
\def\SheafHom{\mathop{\mathcal{H}\!{\it om}}\nolimits}
\def\SheafExt{\mathop{\mathcal{E}\!{\it xt}}\nolimits}
\def\Sch{\textit{Sch}}
\def\Mor{\mathop{\rm Mor}\nolimits}
\def\Ob{\mathop{\rm Ob}\nolimits}
\def\Sh{\mathop{\textit{Sh}}\nolimits}
\def\NL{\mathop{N\!L}\nolimits}
\def\proetale{{pro\text{-}\acute{e}tale}}
\def\etale{{\acute{e}tale}}
\def\QCoh{\textit{QCoh}}
\def\Ker{\mathop{\rm Ker}}
\def\Im{\mathop{\rm Im}}
\def\Coker{\mathop{\rm Coker}}
\def\Coim{\mathop{\rm Coim}}

%
% Macros for moduli stacks/spaces
%
\def\QCohstack{\mathcal{QC}\!{\it oh}}
\def\Cohstack{\mathcal{C}\!{\it oh}}
\def\Spacesstack{\mathcal{S}\!{\it paces}}
\def\Quotfunctor{{\rm Quot}}
\def\Hilbfunctor{{\rm Hilb}}
\def\Curvesstack{\mathcal{C}\!{\it urves}}
\def\Polarizedstack{\mathcal{P}\!{\it olarized}}
\def\Complexesstack{\mathcal{C}\!{\it omplexes}}
% \Pic is the operator that assigns to X its picard group, usage \Pic(X)
% \Picardstack_{X/B} denotes the Picard stack of X over B
% \Picardfunctor_{X/B} denotes the Picard functor of X over B
\def\Pic{\mathop{\rm Pic}\nolimits}
\def\Picardstack{\mathcal{P}\!{\it ic}}
\def\Picardfunctor{{\rm Pic}}
\def\Deformationcategory{\mathcal{D}\!{\it ef}}


% OK, start here.
%
\begin{document}

\title{Stacks}


\maketitle

\phantomsection
\label{section-phantom}

\tableofcontents

\section{Introduction}
\label{section-introduction}

\noindent
In this very short chapter we introduce stacks, and
stacks in groupoids. See \cite{DM}, and \cite{Vis2}.


\section{Presheaves of morphisms associated to fibred categories}
\label{section-morphisms}

\noindent
Let $\mathcal{C}$ be a category.
Let $p : \mathcal{S} \to \mathcal{C}$ be a fibred category,
see Categories, Section \ref{categories-section-fibred-categories}.
Suppose that $x, y\in \text{Ob}(\mathcal{S}_U)$ are
objects in the fibre category over $U$. We are going to define
a functor
$$
\mathit{Mor}(x, y) : (\mathcal{C}/U)^{opp} \longrightarrow \textit{Sets}.
$$
In other words this will be a presheaf on $\mathcal{C}/U$, see
Sites, Definition \ref{sites-definition-presheaf}.
Make a choice of pullbacks as in
Categories,
Definition \ref{categories-definition-pullback-functor-fibred-category}.
Then, for $f : V \to U$ we set
$$
\mathit{Mor}(x, y)(f : V \to U) =
\text{Mor}_{\mathcal{S}_V}(f^\ast x, f^\ast y).
$$
Let $f' : V' \to U$ be a second object of $\mathcal{C}/U$.
We also have to define the restriction map corresponding to a
morphism $g : V'/U  \to V/U$ in $\mathcal{C}/U$,
in other words $g : V' \to V$ and $f' = f \circ g$.
This will be a map
$$
\text{Mor}_{\mathcal{S}_V}(f^\ast x, f^\ast y)
\longrightarrow
\text{Mor}_{\mathcal{S}_{V'}}({f'}^\ast x, {f'}^\ast y),\quad
\phi \longmapsto \phi|_{V'}
$$
This map will basically be $g^\ast$, except that this transforms
an element $\phi$ of the left hand side into an element
$g^\ast \phi$
of $\text{Mor}_{\mathcal{S}_{V'}}(g^\ast f^\ast x, g^\ast f^\ast y)$.
At this point we use the transformation $\alpha_{g, f}$ of
Categories, Lemma \ref{categories-lemma-fibred}.
In a formula, the restriction map is described by
$$
\phi|_{V'} =
(\alpha_{g, f})_y^{-1} \circ
g^\ast \phi \circ
(\alpha_{g, f})_x.
$$
Of course, nobody thinks of this restriction map in this way.
We will only do this once in order to verify the following
lemma.

\begin{lemma}
\label{lemma-painfull}
This actually does give a presheaf.
\end{lemma}

\begin{proof}
Let $g : V'/U \to V/U$ be as above and similarly
$g' : V''/U \to V'/U$ be morphisms in $\mathcal{C}/U$.
So $f' = f \circ g$ and $f'' = f' \circ g' = f \circ g \circ g'$.
Let $\phi \in {Mor}_{\mathcal{S}_V}(f^\ast x, f^\ast y)$.
Then we have
\begin{eqnarray*}
& &
(\alpha_{g \circ g', f})_y^{-1} \circ
(g \circ g')^\ast \phi \circ
(\alpha_{g \circ g', f})_x
\\
& = &
(\alpha_{g \circ g', f})_y^{-1} \circ
(\alpha_{g', g})_{f^*y}^{-1} \circ
(g')^*g^\ast \phi \circ
(\alpha_{g', g})_{f^*x} \circ
(\alpha_{g \circ g', f})_x
\\
& = &
(\alpha_{g', f'})_y^{-1} \circ
(g')^*(\alpha_{g, f})_y^{-1} \circ
(g')^* g^\ast \phi \circ
(g')^*(\alpha_{g, f})_x
\circ
(\alpha_{g', f'})_x
\\
& = &
(\alpha_{g', f'})_y^{-1} \circ
(g')^*\Big(
(\alpha_{g, f})_y^{-1} \circ
g^\ast \phi \circ
(\alpha_{g, f})_x
\Big) \circ
(\alpha_{g', f'})_x
\end{eqnarray*}
which is what we want. The first equality holds because
$\alpha_{g', g}$ is a transformation of functors, and hence
$$
\xymatrix{
(g \circ g')^*f^*x
\ar[rr]_{(g \circ g')^\ast \phi}
\ar[d]_{(\alpha_{g', g})_{f^*x}} & &
(g \circ g')^*f^*y
\ar[d]^{(\alpha_{g', g})_{f^*y}} \\
(g')^*g^*f^*x
\ar[rr]^{(g')^*g^\ast \phi} & &
(g')^*g^*f^*x
}
$$
commutes. The second equality holds because of property (d) of
a pseudo functor since $f' = f \circ g$ (see
Categories, Definition \ref{categories-definition-functor-into-2-category}).
The last equality follows from the fact that $(g')^*$ is a functor.
\end{proof}

\noindent
From now on we often omit mentioning the transformations
$\alpha_{g, f}$ and we simply identify the functors
$g^* \circ f^*$ and $(f \circ g)^*$. In particular,
given $g : V'/U \to V/U$ the restriction
mappings for the presheaf $\mathit{Mor}(x, y)$
will sometimes be denoted $\phi \mapsto g^*\phi$.

\begin{remark}
\label{remark-alternative}
Suppose that $p : \mathcal{S} \to \mathcal{C}$ is fibred in groupoids.
In this case we can argue using
Categories, Lemma \ref{categories-lemma-fibred-strict}
which says that $\mathcal{S} \to \mathcal{C}$ is equivalent to the
category associated to a contravariant function
$F : \mathcal{C} \to \text{Groupoids}$.
In the case of the fibred category associated to $F$
we have $g^* \circ f^* = (f \circ g)^*$ on the nose
and there is no need to use the maps $\alpha_{g, f}$.
In this case the lemma is (even more) trivial. Of course then
one has to show that the $\mathit{Hom}(x, y)$ presheaf is
unchanged when passing to an equivalent fibred category (details
omitted).
\end{remark}

\noindent
We formalize the construction in a definition.

\begin{definition}
\label{definition-mor-presheaf}
Let $\mathcal{C}$ be a category.
Let $p : \mathcal{S} \to \mathcal{C}$ be a fibred category,
see Categories, Section \ref{categories-section-fibred-categories}.
Given an object $U$ of $\mathcal{C}$ and objects
$x$, $y$ of the fibre category, the {\it presheaf
of morphisms from $x$ to $y$} is the presheaf
$$
(f : V \to U) \longmapsto \text{Mor}_{\mathcal{S}_V}(f^*x, f^*y)
$$
described above. It is denoted $\mathit{Mor}(x, y)$.
The sub-presheaf $\mathit{Isom}(x, y)$ whose values
over $V$ is the set of isomorphisms
$f^*x \to f^*y$ in the fibre category $\mathcal{S}_V$
is called the {\it presheaf of isomorphisms from $x$ to $y$}.
\end{definition}

\noindent
If $\mathcal{S}$ is fibred in groupoids then of course
$\mathit{Isom}(x, y) = \mathit{Mor}(x, y)$, and it is
customary to use the $\mathit{Isom}$ notation.




\section{Descent data in fibred categories}
\label{section-descent-data}

\noindent
In this section we define the notion of a descent datum
in the abstract setting of a fibred category. Before we
do so we point out that this is completely analogous to
descent data for quasi-coherent sheaves
(Descent, Section \ref{descent-section-equivalence})
and descent data for schemes over schemes
(Descent, Section \ref{descent-section-descent-datum}).

\medskip\noindent
We will use the convention where the projection maps
$\text{pr}_i : X \times \ldots \times X \to X$
are labeled starting with $i = 0$. Hence we have
$\text{pr}_0, \text{pr}_1 : X \times X  \to X$,
$\text{pr}_0, \text{pr}_1, \text{pr}_2 : X \times X \times X  \to X$,
etc.

\begin{definition}
\label{definition-descent-data}
Let $\mathcal{C}$ be a category.
Let $p : \mathcal{S} \to \mathcal{C}$ be a fibred category.
Make a choice of pullbacks as in Categories,
Definition \ref{categories-definition-pullback-functor-fibred-category}.
Let $\mathcal{U} = \{f_i : U_i \to U\}_{i \in I}$
be a family of morphisms of $\mathcal{C}$. Assume all the fibre products
$U_i \times_U U_j$, and $U_i \times_U U_j \times_U U_k$ exist.
\begin{enumerate}
\item A {\it descent datum $(X_i, \varphi_{ij})$ in $\mathcal{S}$
relative to the family $\{f_i : U_i \to U\}$} is given by an object $X_i$
of $\mathcal{S}_{U_i}$ for each $i \in I$, an isomorphism
$\varphi_{ij} : \text{pr}_0^*X_i \to \text{pr}_1^*X_j$
in $\mathcal{S}_{U_i \times_U U_j}$ for each pair $(i, j) \in I^2$
such that for every triple of indices $(i, j, k) \in I^3$ the
diagram
$$
\xymatrix{
\text{pr}_0^*X_i \ar[rd]_{\text{pr}_{01}^*\varphi_{ij}}
\ar[rr]_{\text{pr}_{02}^*\varphi_{ik}} & &
\text{pr}_2^*X_k \\
& \text{pr}_1^*X_j \ar[ru]_{\text{pr}_{12}^*\varphi_{jk}} &
}
$$
in the category $\mathcal{S}_{U_i \times_U U_j \times_U U_k}$
commutes. This is called the {\it cocycle condition}.
\item A {\it morphism $\psi : (X_i, \varphi_{ij}) \to
(X'_i, \varphi'_{ij})$ of descent data} is given
by a family $\psi = (\psi_i)_{i\in I}$ of morphisms
$\psi_i : \mathcal{F}_i \to \mathcal{F}'_i$ in $\mathcal{S}_{U_i}$
such that all the diagrams
$$
\xymatrix{
\text{pr}_0^*X_i \ar[r]_{\varphi_{ij}} \ar[d]_{\text{pr}_0^*\psi_i}
& \text{pr}_1^*X_j \ar[d]^{\text{pr}_1^*\psi_j} \\
\text{pr}_0^*X'_i \ar[r]^{\varphi'_{ij}} &
\text{pr}_1^*X'_j \\
}
$$
in the categories $\mathcal{S}_{U_i \times_U U_j}$ commute.
\item The category of descent data relative to
$\mathcal{U}$ is denoted $DD(\mathcal{U})$.
\end{enumerate}
\end{definition}

\noindent
The fibre products $U_i \times_U U_j$ and $U_i \times_U U_j \times_U U_k$
will exist if each of the morphisms $f_i : U_i \to U$ is {\it representable},
see Categories, Definition \ref{categories-definition-representable-morphism}.
Recall that in a site one of the conditions for a covering $\{U_i \to U\}$ is
that each of the morphisms is representable, see
Sites, Definition \ref{sites-definition-site} part (3).
In fact the main interest in the definition above is where $\mathcal{C}$
is a site and $\{U_i \to U\}$ is a covering of $\mathcal{C}$. However,
a descent datum is just an abstract gadget that can be defined as above.
This is useful: for example, given a fibred category over $\mathcal{C}$
one can look at the collection of families with respect to which descent data
are effective, and try to use these as the family of coverings for a site.

\begin{remarks}
\label{remarks-definition-descent-datum}
Two remarks Definition \ref{definition-descent-data} are in order.
Let $p : \mathcal{S} \to \mathcal{C}$ be a fibred category.
Let $\{f_i : U_i \to U\}_{i \in I}$, and $(X_i, \varphi_{ij})$
be as in Definition \ref{definition-descent-data}.
\begin{enumerate}
\item There is a diagonal morphism $\Delta : U_i \to U_i \times_U U_i$.
We can pull back $\varphi_{ii}$ via this morphism to get an automorphism
$\Delta^\ast \varphi_{ii} \in \text{Aut}_{U_i}(x_i)$.
On pulling back the cocycle condition for the triple $(i, i, i)$
by $\Delta_{123} : U_i \to U_i\times_U U_i \times_U U_i$ we deduce that
$\Delta^\ast \varphi_{ii} \circ \Delta^\ast \varphi_{ii} =
\Delta^\ast \varphi_{ii}$; thus $\Delta^\ast \varphi_{ii} =
\text{id}_{x_i}$.
\item There is a morphism
$\Delta_{13}: U_i \times_U U_j \to U_i \times_U U_j \times_U U_i$
and we can pull back the
cocycle condition for the triple $(i, j, i)$ to get the
identity $(\sigma^\ast \varphi_{ji}) \circ \varphi_{ij} =
\text{id}_{\text{pr}_0^\ast x_i}$, where
$\sigma: U_i \times_U U_j \to U_j \times_U U_i$ is the switching morphism.
\end{enumerate}
\end{remarks}

\begin{lemma}
\label{lemma-pullback}
(Pullback of descent data.)
Let $\mathcal{C}$ be a category.
Let $p : \mathcal{S} \to \mathcal{C}$ be a fibred category.
Make a choice pullbacks as in Categories,
Definition \ref{categories-definition-pullback-functor-fibred-category}.
Let $\mathcal{U} = \{f_i : U_i \to U\}_{i \in I}$, and
$\mathcal{V} = \{V_j \to V\}_{j \in J}$ 
be a families of morphisms of $\mathcal{C}$ with fixed target.
Assume all the fibre products
$U_i \times_U U_{i'}$, $U_i \times_U U_{i'} \times_U U_{i''}$,
$V_j \times_V V_{j'}$, and $V_j \times_V V_{j'} \times_V V_{j''}$ exist.
Let $\alpha : I \to J$, $h : U \to V$ and
$g_i : U_i \to V_{\alpha(i)}$ be a morphism of families
of maps with fixed target, see
Sites, Definition \ref{sites-definition-morphism-coverings}.
\begin{enumerate}
\item Let $(Y_j, \varphi_{jj'})$ be a descent datum relative to the
family $\{V_j \to V\}$. The system
$$
\left(
g_i^*Y_{\alpha(i)},
(g_i \times g_{i'})^*\varphi_{\alpha(i)\alpha(i')}
\right)
$$
is a descent datum relative to $\mathcal{U}$.
\item This construction defines a functor between descent data relative
to $\mathcal{V}$ and descent data relative to $\mathcal{U}$.
\item Given a second $\alpha' : I \to J$, $h' : U \to V$ and
$g'_i : U_i \to V_{\alpha'(i)}$ morphism of families
of maps with fixed target, then if $h = h'$ the two resulting functors
between descent data are canonically isomorphic.
\end{enumerate}
\end{lemma}

\begin{proof}
Omitted.
\end{proof}

\begin{definition}
\label{definition-pullback-functor}
With $\mathcal{U} = \{U_i \to U\}_{i \in I}$,
$\mathcal{V} = \{V_j \to V\}_{j \in J}$,
$\alpha : I \to J$, $h : U \to V$,
and $g_i : U_i \to V_{\alpha(i)}$ as in Lemma \ref{lemma-pullback}
the functor
$$
(Y_j, \varphi_{jj'}) \longmapsto
(g_i^*Y_{\alpha(i)}, (g_i \times g_{i'})^*\varphi_{\alpha(i)\alpha(i')})
$$
constructed in that lemma
is called the {\it pullback functor} on descent data.
\end{definition}

\noindent
Given $h : U \to V$, if there exists a morphism
$\tilde h : \mathcal{U} \to \mathcal{V}$ covering $h$
then $\tilde h^*$ is independent of the choice of
$\tilde h$ as we saw in Lemma \ref{lemma-pullback}.
Hence we will sometimes simply write $h^*$ to indicate
the pullback functor.

\begin{definition}
\label{definition-effective-descent-datum}
Let $\mathcal{C}$ be a category.
Let $p : \mathcal{S} \to \mathcal{C}$ be a fibred category.
Make a choice of pullbacks as in Categories,
Definition \ref{categories-definition-pullback-functor-fibred-category}.
Let $\mathcal{U} = \{f_i : U_i \to U\}_{i \in I}$ be a family of morphisms
with target $U$. Assume all the fibre products
$U_i \times_U U_j$ and $U_i \times_U U_j \times_U U_k$ exist.
\begin{enumerate}
\item Given an object $X$ of $\mathcal{S}_U$ the {\it trivial descent datum}
is the descent datum $(X, \text{id}_X)$ with respect to the family
$\{\text{id}_U : U \to U\}$.
\item Given an object $X$ of $\mathcal{S}_U$
we have a {\it canonical descent datum} on the family of
objects $f_i^*X$ by pulling back the trivial
descent datum $(X, \text{id}_X)$ via the
obvious map $\{f_i : U_i \to U\} \to \{\text{id}_U : U \to U\}$.
We denote this descent datum $(f_i^*X, can)$.
\item A descent datum $(X_i, \varphi_{ij})$
relative to $\{f_i : U_i \to U\}$ is called {\it effective}
if there exists an object $X$ of $\mathcal{S}_U$ such that
$(X_i, \varphi_{ij})$ is isomorphic to $(f_i^*X, can)$.
\end{enumerate}
\end{definition}

\noindent
Note that the rule that associates to $X \in \mathcal{S}_U$ its
canonical descent datum relative to $\mathcal{U}$ defines a
functor
$$
\mathcal{S}_U \longrightarrow DD(\mathcal{U}).
$$
A descent datum is effective if and only if it is in the essential
image of this functor.
Let us make explicit the canonical descent datum as follows.

\begin{lemma}
\label{lemma-trivial-cocycle}
In the situation of
Definition \ref{definition-effective-descent-datum} part (2) the maps
$can_{ij} : \text{pr}_0^*f_i^*X \to \text{pr}_1^*f_j^*X$ are equal to
$(\alpha_{\text{pr}_1, f_j})_X \circ (\alpha_{\text{pr}_0, f_i})_X^{-1}$
where $\alpha_{\cdot, \cdot}$ is as in
Categories, Lemma \ref{categories-lemma-fibred}
and where we
use the equality $f_i \circ \text{pr}_0 = f_j \circ \text{pr}_1$
as maps $U_i \times_U U_j \to U$.
\end{lemma}

\begin{proof}
Omitted.
\end{proof}












\section{Definition of stacks}
\label{section-definition}

\noindent
Here is the definition of a stack. It mixes the notion of a fibred
category with the notion of descent.

\begin{definition}
\label{definition-stack}
Let $\mathcal{C}$ be a site. A {\it stack} over $\mathcal{C}$
is a category $p : \mathcal{S} \to \mathcal{C}$ over $\mathcal{C}$ which
satisfies the following conditions:
\begin{enumerate}
\item $p : \mathcal{S} \to \mathcal{C}$ is a fibred category
see Categories, Definition \ref{categories-definition-fibred-category},
\item for any $U \in \text{Ob}(\mathcal{C})$ and any $x, y \in \mathcal{S}_U$
the presheaf $\mathit{Mor}(x, y)$ (see
Definition \ref{definition-mor-presheaf}) is a sheaf on
the site $\mathcal{C}/U$, and
\item for any covering $\mathcal{U} = \{f_i : U_i \to U\}_{i \in I}$
of the site $\mathcal{C}$, any descent datum in $\mathcal{S}$
relative to $\mathcal{U}$ is effective.
\end{enumerate}
\end{definition}

\noindent
We find the formulation above the most convenient way to think about
a stack. Namely, given a category over $\mathcal{C}$ in order to verify
that it is a stack you proceed to check properties (1), (2) and
(3) in that order. Certainly properties (2) and (3) do not make sense
if the category isn't fibred, and property (3) scarcely makes any sense
if the morphisms presheaves aren't sheaves.

\medskip\noindent
The following lemma provides an alternative definition.

\begin{lemma}
\label{lemma-stack-equivalences}
Let $\mathcal{C}$ be a site.
Let $p : \mathcal{S} \to \mathcal{C}$ be a fibred category
over $\mathcal{C}$. The following are equivalent
\begin{enumerate}
\item $\mathcal{S}$ is a stack over $\mathcal{C}$, and
\item for any covering $\mathcal{U} = \{f_i : U_i \to U\}_{i \in I}$
of the site $\mathcal{C}$ the functor
$$
\mathcal{S}_U \longrightarrow DD(\mathcal{U})
$$
which associates to an
object its canonical descent datum is an equivalence.
\end{enumerate}
\end{lemma}

\begin{proof}
Omitted.
\end{proof}

\noindent
The $2$-category of stacks over $\mathcal{C}$
is defined as follows.

\begin{definition}
\label{definition-stacks-over-C}
Let $\mathcal{C}$ be a site.
The {\it $2$-category of stacks over $\mathcal{C}$}
is the sub $2$-category of the $2$-category of fibred categories
over $\mathcal{C}$ (see
Categories, Definition \ref{categories-definition-fibred-categories-over-C})
defined as follows:
\begin{enumerate}
\item Its objects will be stacks $p : \mathcal{S} \to \mathcal{C}$.
\item Its $1$-morphisms $(\mathcal{S}, p) \to (\mathcal{S}', p')$
will be functors $G : \mathcal{S} \to \mathcal{S}'$ such that
$p' \circ G = p$ and such that $G$ maps strongly cartesian
morphisms to strongly cartesian morphisms.
\item Its $2$-morphisms $t : G \to H$ for
$G, H : (\mathcal{S}, p) \to (\mathcal{S}', p')$
will be morphisms of functors
such that $p'(t_x) = \text{id}_{p(x)}$
for all $x \in \text{Ob}(\mathcal{S})$.
\end{enumerate}
\end{definition}

\begin{lemma}
\label{lemma-2-product-stacks}
Let $\mathcal{C}$ be a category.
The $(2, 1)$-category of stacks over $\mathcal{C}$
has 2-fibre products, and they are described as in
Categories, Lemma \ref{categories-lemma-2-product-categories-over-C}.
\end{lemma}

\begin{proof}
Let $f : \mathcal{X} \to \mathcal{S}$ and
$g : \mathcal{Y} \to \mathcal{S}$ be
$1$-morphisms of stacks over $\mathcal{C}$
as defined above. The category
$\mathcal{X} \times_{\mathcal{S}} \mathcal{Y}$
described in
Categories, Lemma \ref{categories-lemma-2-product-categories-over-C} is a
fibred category according to
Categories, Lemma \ref{categories-lemma-2-product-fibred-categories-over-C}.
(This is where we use that $f$ and $g$ preserve strongly cartesian
morphisms.) It remains to show that the morphism presheaves are sheaves
and that descent relative to coverings of $\mathcal{C}$ is effective.

\medskip\noindent
Recall that an object of $\mathcal{X} \times_{\mathcal{S}} \mathcal{Y}$
is given by a quadruple $(U, x, y, \phi)$.
It lies over the object
$U$ of $\mathcal{C}$. Next, let $(U, x', y', \phi')$ be second
object lying over $U$.
Recall that $\phi : f(x) \to g(y)$, and $\phi' : f(x') \to g(y')$
are isomorphisms in the category $\mathcal{S}_U$. Let us
use these isomorphisms to identify $z = f(x) = g(y)$ and
$z' = f(x') = g(y')$. With this identifications
it is clear that
$$
\mathit{Mor}((U, x, y, \phi), (U, x, y, \phi))
=
\mathit{Mor}(x, x')
\times_{\mathit{Mor}(z, z')}
\mathit{Mor}(y, y')
$$
as presheaves. However, as the fibred product in the category of
presheaves preserves sheaves (Sites, Lemma \ref{sites-lemma-limit-sheaf})
we see that this is a sheaf.

\medskip\noindent
Let $\mathcal{U} = \{f_i : U_i \to U\}_{i \in I}$ be a covering of the site
$\mathcal{C}$. Let $(X_i, \chi_{ij})$ be a descent datum
in $\mathcal{X} \times_{\mathcal{S}} \mathcal{Y}$ relative to $\mathcal{U}$.
Write $X_i = (U_i, x_i, y_i, \phi_i)$ as above. Write
$\chi_{ij} = (\varphi_{ij}, \psi_{ij})$ as in the definition of
the category $\mathcal{X} \times_{\mathcal{S}} \mathcal{Y}$ (see
Categories, Lemma \ref{categories-lemma-2-product-categories-over-C}).
It is clear that $(x_i, \varphi_{ij})$ is a descent datum in
$\mathcal{X}$ and that $(y_i, \psi_{ij})$ is a descent datum in 
$\mathcal{Y}$. Since $\mathcal{X}$ and $\mathcal{Y}$ are stacks these
descent data are effective. Thus we get
$x \in \text{Ob}(\mathcal{X}_U)$, and $y \in \text{Ob}(\mathcal{Y}_U)$
with $x_i = x|_{U_i}$, and $y_i = y|_{U_i}$ compatibly with descent data.
Set $z = f(x)$ and $z' = g(y)$ which are both objects of $\mathcal{S}_U$.
The morphisms $\phi_i$ are elements of
$\mathit{Isom}(z, z')(U_i)$ with the property that
$\phi_i|_{U_i \times_U U_j} = \phi_j|_{U_i \times_U U_j}$.
Hence by the sheaf property of $\mathit{Isom}(z, z')$
we obtain an isomorphism $\phi : z = f(x) \to z' = g(y)$.
We omit the verification that the canonical descent datum associated to
the object $(U, x, y, \phi)$ of
$(\mathcal{X} \times_{\mathcal{S}} \mathcal{Y})_U$ is isomorphic
to the descent datum we started with.
\end{proof}











\section{Stacks in groupoids}
\label{section-stacks-in-groupoids}

\noindent
Among stacks those which are fibred in groupoids are somewhat easier
to comprehend. We redefine them as follows.

\begin{definition}
\label{definition-stack-in-groupoids}
A {\it stack in groupoids} over a site $\mathcal{C}$ is a
category $p : \mathcal{S} \to \mathcal{C}$ over $\mathcal{C}$
such that
\begin{enumerate}
\item $p : \mathcal{S} \to \mathcal{C}$ is fibred
in groupoids over $\mathcal{C}$ (see
Categories, Definition \ref{categories-definition-fibred-groupoids}),
\item for all $U \in \text{Ob}(\mathcal{C})$,
for all $x, y\in \text{Ob}(\mathcal{S}_U)$ the presheaf
$\mathit{Isom}(x, y)$ is a sheaf on the site $\mathcal{C}/U$, and
\item for all coverings $\mathcal{U} = \{U_i \to U\}$ in $\mathcal{C}$,
all descent data $(x_i, \phi_{ij})$ for $\mathcal{U}$ are effective.
\end{enumerate}
\end{definition}

\noindent
Usually the hardest part to check is the third condition.
Here is the lemma comparing this with the notion of a stack.

\begin{lemma}
\label{lemma-stack-in-groupoids-stack}
Let $\mathcal{C}$ be a site.
Let $p : \mathcal{S} \to \mathcal{C}$ be a category over $\mathcal{C}$.
The following are equivalent
\begin{enumerate}
\item $\mathcal{S}$ is a stack in groupoids over $\mathcal{C}$,
\item $\mathcal{S}$ is a stack over $\mathcal{C}$ and all
fibre categories are groupoids, and
\item $\mathcal{S}$ is fibred in groupoids over $\mathcal{C}$
and is a stack over $\mathcal{C}$.
\end{enumerate}
\end{lemma}

\begin{proof}
Omitted, but see Categories, Lemma \ref{categories-lemma-fibred-groupoids}.
\end{proof}

\noindent
The $2$-category of stacks over $\mathcal{C}$
is defined as follows.

\begin{definition}
\label{definition-stacks-in-groupoids-over-C}
Let $\mathcal{C}$ be a site.
The {\it $2$-category of stacks in groupoids over $\mathcal{C}$}
is the sub $2$-category of the $2$-category of stacks
over $\mathcal{C}$ (see Definition \ref{definition-stacks-over-C})
defined as follows:
\begin{enumerate}
\item Its objects will be stacks in groupoids
$p : \mathcal{S} \to \mathcal{C}$.
\item Its $1$-morphisms $(\mathcal{S}, p) \to (\mathcal{S}', p')$
will be functors $G : \mathcal{S} \to \mathcal{S}'$ such that
$p' \circ G = p$. (Since every morphism is strongly cartesian
every functor preserves them.)
\item Its $2$-morphisms $t : G \to H$ for
$G, H : (\mathcal{S}, p) \to (\mathcal{S}', p')$
will be morphisms of functors
such that $p'(t_x) = \text{id}_{p(x)}$
for all $x \in \text{Ob}(\mathcal{S})$.
\end{enumerate}
\end{definition}

\noindent
Note that any $2$-morphism is automatically an isomorphism, so
that in fact the $2$-category of stacks in groupoids over $\mathcal{C}$
is a (strict) $(2,1)$-category.

\begin{lemma}
\label{lemma-2-product-stacks-in-groupoids}
Let $\mathcal{C}$ be a category.
The $2$-category of stacks in groupoids over $\mathcal{C}$
has 2-fibre products, and they are described as in
Categories, Lemma \ref{categories-lemma-2-product-categories-over-C}.
\end{lemma}

\begin{proof}
This is clear from
Categories, Lemma \ref{categories-lemma-2-product-fibred-categories}
and Lemmas \ref{lemma-stack-in-groupoids-stack}
and \ref{lemma-2-product-stacks}.
\end{proof}



\section{The inertia stack}
\label{section-the-inertia-stack}

\noindent
Let $p : \mathcal{S} \to \mathcal{C}$ be a stack over the site $\mathcal{C}$.
Recall that we have defined in
Categories, Definition \ref{categories-definition-inertia-fibred-category}
an {\it inertia fibred category} $I_\mathcal{S} \to \mathcal{C}$ as the
category whose objects are pairs $(x ,\alpha)$ where
$x \in \text{Ob}(\mathcal{S})$ and $\alpha : x \to x$ with
$p(\alpha) = \text{id}_{p(x)}$. This inertia category is actually
a stack over $\mathcal{C}$.

\begin{lemma}
\label{lemma-inertia}
Let $\mathcal{C}$ be a site.
If $\mathcal{S}$ is a stack over $S$,
then so is the intertia fibred category $I_\mathcal{S}$.
\end{lemma}

\begin{proof}
Follows from Lemma \ref{lemma-2-product-stacks-in-groupoids}
and the equivalence in
Categories, Lemma \ref{categories-lemma-intertia-fibred-category}
assertion (1).
\end{proof}













\section{Stackification of fibred categories}
\label{section-stackify}

\noindent
Here is the result.

\begin{lemma}
\label{lemma-stackify}
Let $\mathcal{C}$ be a site.
Let $p : \mathcal{S} \to \mathcal{C}$ be a fibred category over $\mathcal{C}$.
There exists a stack $p' : \mathcal{S}' \to \mathcal{C}$ and a
$1$-morphism $G : \mathcal{S} \to \mathcal{S}'$
of fibred categories over $\mathcal{C}$ (see
Categories, Definition \ref{categories-definition-fibred-categories-over-C})
such that
\begin{enumerate}
\item for every $U \in \text{Ob}(\mathcal{C})$, and any
$x, y \in \text{Ob}(\mathcal{S}_U)$ the map
$$
\mathit{Mor}(x, y) \longrightarrow \mathit{Mor}(G(x), G(y))
$$
induced by $G$ identifies the right hand side with the sheafification
of the left hand side, and
\item for every $U \in \text{Ob}(\mathcal{C})$, and any
$x' \in \text{Ob}(\mathcal{S}'_U)$ there exists a covering
$\{U_i \to U\}_{i \in I}$ such that for every $i \in I$ the
object $x'|_{U_i}$ is in the essential image of the
functor $G : \mathcal{S}_U \to \mathcal{S}'_U$.
\end{enumerate}
Moreover the stack $\mathcal{S}'$ is determined up to unique
$2$-isomorphism by these conditions.
\end{lemma}

\begin{proof}[Less naive proof]
Here is a less naive proof.
By Categories, Lemma \ref{categories-lemma-fibred-strict}
there exists an equivalence of
fibred categories $\mathcal{S} \to \mathcal{S}'$ where $\mathcal{S}'$
is a split fibred category, i.e., one in which the pullback
functors compose on the nose. Obviously the lemma for $\mathcal{S}'$
implies the lemma for $\mathcal{S}$. Hence we may think of $\mathcal{S}$
as a presheaf in categories.

\medskip\noindent
Consider the $2$-category $\textit{Cat}$ temporarily as a
category by forgetting about $2$-morphisms.
Let us think of a category as a quintuple
$(\text{Ob}, \text{Arrows}, s, t, \circ)$ as in
Categories, Section \ref{categories-section-definition-categories}.
Consider the forgetful functor
$$
forget : \textit{Cat} \to \textit{Sets}, \quad
(\text{Ob}, \text{Arrows}, s, t, \circ)
\mapsto
\text{Ob} \coprod \text{Arrows}.
$$
Then $forget$ is faithful, $\textit{Cat}$ has direct limits and
$forget$ commutes with them, $\textit{Cat}$ has directed colimits and
$forget$ commutes with them, and $forget$ reflects isomorphisms.
Hence, according to the first part of
Sites, Section \ref{sites-section-sheaves-algebraic-structures}
we can sheafify presheaves with values in $\textit{Cat}$, and
the result commutes with $forget$. Applying this to
$\mathcal{S}$ we obtain a sheafification $\mathcal{S}^\#$
which has a sheaf of objects and a sheaf of morphisms
both of which are the sheafifications of the corresponding
presheaves for $\mathcal{S}$. In this case it is quite
easy to see that the map $\mathcal{S} \to \mathcal{S}^\#$
has the properties (1) and (2) of the lemma.

\medskip\noindent
However, the category $\mathcal{S}^\#$ may not yet be a
stack since, allthough the presheaf of objects is a sheaf,
the descent condition may not yet be satisfied.
To remedy this we have to add more objects. But the argument
above does reduce us to the case where $\mathcal{S} = \mathcal{S}_F$
for some sheaf(!) $F : \mathcal{C}^{opp} \to \textit{Cat}$ of
categories. In this case consider the functor
$F' : \mathcal{C}^{opp} \to \textit{Cat}$ defined by
\begin{enumerate}
\item The set $\text{Ob}(F'(U))$ is the set of pairs
$(\mathcal{U}, \xi)$ where $\mathcal{U} = \{U_i \to U\}$
is a covering of $U$ and $\xi = (x_i, \varphi_{ii'})$ is
a descent datum relative to $\mathcal{U}$.
\item A morphism in $F'(U)$ from
$(\mathcal{U}, \xi)$ to $(\mathcal{V}, \eta)$
is an element of
$$
\text{colim}\ \text{Mor}_{DD(\mathcal{W})}(a^*\xi, b^*\eta)
$$
where the colimit is over all common refinements
$a : \mathcal{W} \to \mathcal{U}$, $b : \mathcal{W} \to \mathcal{V}$.
This colimit is directed (verification omitted).
Hence composition of morphisms in $F(U)$ is defined by
finding a common refinement and composing in $DD(\mathcal{W})$.
\item Given $h : V \to U$ and an object
$(\mathcal{U}, \xi)$ of $F'(U)$ we set $F'(h)(\mathcal{U}, \xi)$
equal to $(V \times_U \mathcal{U}, \text{pr}_1^*\xi)$.
More precisely, if $\mathcal{U} = \{U_i \to U\}$
and $\xi = (x_i, \varphi_{ii'})$, then
$V \times_U \mathcal{U} = \{V \times_U U_i \to V\}$
which comes with a canonical morphism
$\text{pr}_1 : V \times_U \mathcal{U} \to \mathcal{U}$ and
$\text{pr}_1^*\xi$ is the pullback of $\xi$ with respect to
this morphism (see Definition \ref{definition-pullback-functor}).
\item Given $h : V \to U$, objects $(\mathcal{U}, \xi)$ 
and $(\mathcal{V}, \eta)$ and a morphism between them, represented by
$a : \mathcal{W} \to \mathcal{U}$, $b : \mathcal{W} \to \mathcal{V}$,
and $\alpha : a^*\xi \to b^*\eta$, then $F'(h)(\alpha)$ is
represented by
$a' : V\times_U\mathcal{W} \to V\times_U\mathcal{U}$,
$b' : V\times_U\mathcal{W} \to V\times_U\mathcal{V}$,
and the pullback $\alpha'$ of the morphism $\alpha$ via
the map $V \times_U \mathcal{W} \to \mathcal{W}$. This works
since pullbacks in $\mathcal{S}_F$ commute on the nose.
\end{enumerate}
There is a map $F \to F'$ given by associating to
an object $x$ of $F(U)$ the object $(\{U \to U\}, (x, triv))$ of
$F'(U)$. At this point you have to check that the corresponding
functor $\mathcal{S}_F \to \mathcal{S}_{F'}$ has properties (1)
and (2) of the lemma, and finally that $\mathcal{S}_{F'}$ is
a stack. Details omitted.
\end{proof}

\begin{proof}[Proof by naive method]
In this proof method we proceed in stages:

\medskip\noindent
First, given $x$ lying over $U$ and any object $y$ of
$\mathcal{S}$, we say that two morphisms
$a, b : x \to y$ of $\mathcal{S}$
lying over the same arrow of $\mathcal{C}$
are {\it locally equal}
if there exists a covering $\{f_i : U_i \to U\}$ of $\mathcal{C}$
such that the compositions
$$
f_i^*x \to x \xrightarrow{a} y,
\quad
f_i^*x \to x \xrightarrow{b} y
$$
are equal. This gives an equivalence relation $\sim$
on arrows of $\mathcal{S}$. If $b \sim b'$ then
$a \circ b \circ c \sim a \circ b' \circ c$ (verification omitted).
Hence we can quotient out by this equivalence relation to
obtain a new category $\mathcal{S}^1$ over $\mathcal{C}$
to gether with a morphism $G^1 : \mathcal{S} \to \mathcal{S}^1$.

\medskip\noindent
One checks that $G^1$ preserves strongly cartesian morphisms
and that $\mathcal{S}^1$ is a fibred category over $\mathcal{C}$.
Checks omitted. Thus we reduce to the case where locally equal
morphisms are equal.

\medskip\noindent
Next, we add morphisms as follows. Given
$x$ lying over $U$ and any object $y$ of lying over $V$
a {\it locally defined morphism from $x$ to $y$} is given by
\begin{enumerate}
\item a morphism $f : U \to V$,
\item a covering $\{f_i : U_i \to U\}$ of $U$, and
\item morphisms $a_i : f_i^*x \to Y$ with $p(a_i) = h \circ f_i$
\end{enumerate}
with the property that the compositions
$$
(f_i \times f_j)^*x \to f_i^*x \xrightarrow{a_i} y,
\quad
(f_i \times f_j)^*x \to f_j^*x \xrightarrow{a_j} y
$$
are equal. Note that a usual morphism $a : x \to y$ gives a locally
defined morphism $(p(a) : U \to V, \{\text{id}_U\}, a)$.
We say two locally defined morphisms
$(f, \{f_i : U_i \to U\}, a_i)$ and $(g, \{g_j : U_i \to U\}, b_j)$
are {\it equal} if $f = g$ and the compositions
$$
(f_i \times g_j)^*x \to f_i^*x \xrightarrow{a_i} y,
\quad
(f_i \times g_j)^*x \to g_j^*x \xrightarrow{b_j} y
$$
are equal (this is the right condition since we are in the
situation where locally equal morphisms are equal).
To compose locally defined morphisms
$(f, \{f_i : U_i \to U\}, a_i)$ from $x$ to $y$ and
$(g, \{g_j : V_j \to V\}, b_j)$ from $y$ to $z$ lying over $W$,
just take $g \circ f : U \to W$, the covering
$\{U_i \times_V V_j \to U\}$, and as maps the compositions
$$
x|_{U_i \times_V V_j}
\xrightarrow{\text{pr}_0^*a_i}
y|_{V_j}
\xrightarrow{b_j}
z
$$
We omit the verification that this is a locally defined morphism.

\medskip\noindent
One checks that $\mathcal{S}^2$ with the same objects as
$\mathcal{S}$ and with locally defined morphisms as morphisms
is a category over $\mathcal{C}$, that there is a functor
$G^2 : \mathcal{S} \to \mathcal{S}^2$ over $\mathcal{C}$,
that this functor preserves strongly cartesian objects,
and that $\mathcal{S}^2$ is a fibred category over $\mathcal{C}$.
Checks omitted. This reduces one to the case where the
morphism presheaves of $\mathcal{S}$ are all sheaves, by
checking that the effect of using locally defined morphisms
is to take the sheafification of the (separated) morphisms
presheaves.

\medskip\noindent
Finally, in the case where the morphism presheaves are all sheaves
we have to add objects in order to make sure descent conditions are
effective in the end result. The simplest way to do this is to
consider the category $\mathcal{S}'$ whose objects are
pairs $(\mathcal{U}, \xi)$ where
$\mathcal{U} = \{U_i \to U\}$ is a covering of $\mathcal{C}$ and
$\xi = (X_i, \varphi_{ii'})$ is a descent datum relative $\mathcal{U}$.
Suppose given two such data
$(\mathcal{U}, \xi) = (\{f_i : U_i \to U\},  x_i, \varphi_{ii'})$ and
$(\mathcal{V}, \eta) = (\{g_j : V_j \to V\},  y_j, \psi_{jj'})$.
We define
$$
\text{Mor}_{\mathcal{S}'}((\mathcal{U}, \xi), (\mathcal{V}, \eta))
$$
as the set of $(f, a_{ij})$, where $f : U \to V$ and
$$
a_{ij} :
x_i|_{U_i \times_V V_j}
\longrightarrow 
y_j
$$
are morphisms of $\mathcal{S}$ lying over $U_i \times_V V_j \to V_j$.
These have to satisfy the following condition: for any
$i, i' \in I$ and $j, j' \in J$ set 
$W = (U_i \times_U U_{i'}) \times_V (V_j \times_V V_{j'})$. Then
$$
\xymatrix{
x_i|_W \ar[r]_{a_{ij}|_W} \ar[d]_{\varphi_{ii'}|_W} &
y_j|_W \ar[d]^{\psi_{jj'}|_W} \\
x_{i'}|_W \ar[r]^{a_{i'j'}|_W} &
y_{j'}|_W
}
$$
commutes. At this point you have to verify the following things:
\begin{enumerate}
\item there is a well defined composition on morphisms as above,
\item this turns $\mathcal{S}'$ into a category over $\mathcal{C}$,
\item there is a functor $G : \mathcal{S} \to \mathcal{S}'$ over $\mathcal{C}$,
\item for $x, y$ objects of $\mathcal{S}$ we have
$\text{Mor}_{\mathcal{S}}(x, y) = \text{Mor}_{\mathcal{S}'}(G(x), G(y))$,
\item any object of $\mathcal{S}'$ locally comes from an object of
$\mathcal{S}$, i.e., part (2) of the lemma holds,
\item $G$ preserves strongly cartesian morphisms,
\item $\mathcal{S}'$ is a fibred category over $\mathcal{C}$, and
\item $\mathcal{S}'$ is a stack over $\mathcal{C}$.
\end{enumerate}
This is all not hard but there is a lot of it. Details omitted.
\end{proof}



\section{Stackification of categories fibred in groupoids}
\label{section-stackify-groupoids}

\noindent
Here is the result.

\begin{lemma}
\label{lemma-stackify-groupoids}
Let $\mathcal{C}$ be a site.
Let $p : \mathcal{S} \to \mathcal{C}$ be a category
fibred in groupoids over $\mathcal{C}$.
There exists a stack in groupoids
$p' : \mathcal{S}' \to \mathcal{C}$ and a
$1$-morphism $G : \mathcal{S} \to \mathcal{S}'$
of categories fibred in groupoids over $\mathcal{C}$ (see
Categories, Definition
\ref{categories-definition-categories-fibred-in-groupoids-over-C})
such that
\begin{enumerate}
\item for every $U \in \text{Ob}(\mathcal{C})$, and any
$x, y \in \text{Ob}(\mathcal{S}_U)$ the map
$$
\mathit{Mor}(x, y) \longrightarrow \mathit{Mor}(G(x), G(y))
$$
induced by $G$ identifies the right hand side with the sheafification
of the left hand side, and
\item for every $U \in \text{Ob}(\mathcal{C})$, and any
$x' \in \text{Ob}(\mathcal{S}'_U)$ there exists a covering
$\{U_i \to U\}_{i \in I}$ such that for every $i \in I$ the
object $x'|_{U_i}$ is in the essential image of the
functor $G : \mathcal{S}_U \to \mathcal{S}'_U$.
\end{enumerate}
Moreover the stack in groupoids $\mathcal{S}'$ is determined up to unique
$2$-isomorphism by these conditions.
\end{lemma}

\begin{proof}
Apply Lemma \ref{lemma-stackify}. The result will be a
stack in groupoids by applying Lemma \ref{lemma-stack-in-groupoids-stack}.
\end{proof}







\section{Other chapters}

\begin{multicols}{2}
\begin{enumerate}
\item \hyperref[introduction-section-phantom]{Introduction}
\item \hyperref[conventions-section-phantom]{Conventions}
\item \hyperref[sets-section-phantom]{Set Theory}
\item \hyperref[categories-section-phantom]{Categories}
\item \hyperref[topology-section-phantom]{Topology}
\item \hyperref[sheaves-section-phantom]{Sheaves on Spaces}
\item \hyperref[algebra-section-phantom]{Commutative Algebra}
\item \hyperref[sites-section-phantom]{Sites and Sheaves}
\item \hyperref[homology-section-phantom]{Homological Algebra}
\item \hyperref[derived-section-phantom]{Derived Categories}
\item \hyperref[more-algebra-section-phantom]{More Algebra}
\item \hyperref[simplicial-section-phantom]{Simplicial Methods}
\item \hyperref[modules-section-phantom]{Sheaves of Modules}
\item \hyperref[sites-modules-section-phantom]{Modules on Sites}
\item \hyperref[injectives-section-phantom]{Injectives}
\item \hyperref[cohomology-section-phantom]{Cohomology of Sheaves}
\item \hyperref[sites-cohomology-section-phantom]{Cohomology on Sites}
\item \hyperref[hypercovering-section-phantom]{Hypercoverings}
\item \hyperref[schemes-section-phantom]{Schemes}
\item \hyperref[constructions-section-phantom]{Constructions of Schemes}
\item \hyperref[properties-section-phantom]{Properties of Schemes}
\item \hyperref[morphisms-section-phantom]{Morphisms of Schemes}
\item \hyperref[coherent-section-phantom]{Coherent Cohomology}
\item \hyperref[divisors-section-phantom]{Divisors}
\item \hyperref[limits-section-phantom]{Limits of Schemes}
\item \hyperref[varieties-section-phantom]{Varieties}
\item \hyperref[chow-section-phantom]{Chow Homology}
\item \hyperref[topologies-section-phantom]{Topologies on Schemes}
\item \hyperref[descent-section-phantom]{Descent}
\item \hyperref[more-morphisms-section-phantom]{More on Morphisms}
\item \hyperref[flat-section-phantom]{More on Flatness}
\item \hyperref[groupoids-section-phantom]{Groupoid Schemes}
\item \hyperref[more-groupoids-section-phantom]{More on Groupoid Schemes}
\item \hyperref[etale-section-phantom]{\'Etale Morphisms of Schemes}
\item \hyperref[etale-cohomology-section-phantom]{\'Etale Cohomology}
\item \hyperref[spaces-section-phantom]{Algebraic Spaces}
\item \hyperref[spaces-properties-section-phantom]{Properties of Algebraic Spaces}
\item \hyperref[spaces-morphisms-section-phantom]{Morphisms of Algebraic Spaces}
\item \hyperref[spaces-topologies-section-phantom]{Topologies on Algebraic Spaces}
\item \hyperref[spaces-descent-section-phantom]{Descent and Algebraic Spaces}
\item \hyperref[spaces-more-morphisms-section-phantom]{More on Morphisms of Spaces}
\item \hyperref[quot-section-phantom]{Quot and Hilbert Spaces}
\item \hyperref[stacks-section-phantom]{Stacks}
\item \hyperref[spaces-groupoids-section-phantom]{Groupoids in Algebraic Spaces}
\item \hyperref[spaces-more-groupoids-section-phantom]{More on Groupoids in Spaces}
\item \hyperref[bootstrap-section-phantom]{Bootstrap}
\item \hyperref[examples-stacks-section-phantom]{Examples of Stacks}
\item \hyperref[groupoids-quotients-section-phantom]{Quotients of Groupoids}
\item \hyperref[algebraic-section-phantom]{Algebraic Stacks}
\item \hyperref[criteria-section-phantom]{Criteria for Representability}
\item \hyperref[stacks-properties-section-phantom]{Properties of Algebraic Stacks}
\item \hyperref[stacks-morphisms-section-phantom]{Morphisms of Algebraic Stacks}
\item \hyperref[examples-section-phantom]{Examples}
\item \hyperref[exercises-section-phantom]{Exercises}
\item \hyperref[guide-section-phantom]{Guide to Literature}
\item \hyperref[desirables-section-phantom]{Desirables}
\item \hyperref[coding-section-phantom]{Coding Style}
\item \hyperref[fdl-section-phantom]{GNU Free Documentation License}
\item \hyperref[index-section-phantom]{Auto Generated Index}
\end{enumerate}
\end{multicols}



\bibliography{my}
\bibliographystyle{amsalpha}

\end{document}
