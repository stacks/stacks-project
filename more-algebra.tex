\IfFileExists{stacks-project.cls}{%
\documentclass{stacks-project}
}{%
\documentclass{amsart}
}

% The following AMS packages are automatically loaded with
% the amsart documentclass:
%\usepackage{amsmath}
%\usepackage{amssymb}
%\usepackage{amsthm}

% For dealing with references we use the comment environment
\usepackage{verbatim}
\newenvironment{reference}{\comment}{\endcomment}
%\newenvironment{reference}{}{}
\newenvironment{slogan}{\comment}{\endcomment}
\newenvironment{history}{\comment}{\endcomment}

% For commutative diagrams you can use
% \usepackage{amscd}
\usepackage[all]{xy}

% We use 2cell for 2-commutative diagrams.
\xyoption{2cell}
\UseAllTwocells

% To put source file link in headers.
% Change "template.tex" to "this_filename.tex"
% \usepackage{fancyhdr}
% \pagestyle{fancy}
% \lhead{}
% \chead{}
% \rhead{Source file: \url{template.tex}}
% \lfoot{}
% \cfoot{\thepage}
% \rfoot{}
% \renewcommand{\headrulewidth}{0pt}
% \renewcommand{\footrulewidth}{0pt}
% \renewcommand{\headheight}{12pt}

\usepackage{multicol}

% For cross-file-references
\usepackage{xr-hyper}

% Package for hypertext links:
\usepackage{hyperref}

% For any local file, say "hello.tex" you want to link to please
% use \externaldocument[hello-]{hello}
\externaldocument[introduction-]{introduction}
\externaldocument[conventions-]{conventions}
\externaldocument[sets-]{sets}
\externaldocument[categories-]{categories}
\externaldocument[topology-]{topology}
\externaldocument[sheaves-]{sheaves}
\externaldocument[sites-]{sites}
\externaldocument[stacks-]{stacks}
\externaldocument[fields-]{fields}
\externaldocument[algebra-]{algebra}
\externaldocument[brauer-]{brauer}
\externaldocument[homology-]{homology}
\externaldocument[derived-]{derived}
\externaldocument[simplicial-]{simplicial}
\externaldocument[more-algebra-]{more-algebra}
\externaldocument[smoothing-]{smoothing}
\externaldocument[modules-]{modules}
\externaldocument[sites-modules-]{sites-modules}
\externaldocument[injectives-]{injectives}
\externaldocument[cohomology-]{cohomology}
\externaldocument[sites-cohomology-]{sites-cohomology}
\externaldocument[dga-]{dga}
\externaldocument[dpa-]{dpa}
\externaldocument[hypercovering-]{hypercovering}
\externaldocument[schemes-]{schemes}
\externaldocument[constructions-]{constructions}
\externaldocument[properties-]{properties}
\externaldocument[morphisms-]{morphisms}
\externaldocument[coherent-]{coherent}
\externaldocument[divisors-]{divisors}
\externaldocument[limits-]{limits}
\externaldocument[varieties-]{varieties}
\externaldocument[topologies-]{topologies}
\externaldocument[descent-]{descent}
\externaldocument[perfect-]{perfect}
\externaldocument[more-morphisms-]{more-morphisms}
\externaldocument[flat-]{flat}
\externaldocument[groupoids-]{groupoids}
\externaldocument[more-groupoids-]{more-groupoids}
\externaldocument[etale-]{etale}
\externaldocument[chow-]{chow}
\externaldocument[intersection-]{intersection}
\externaldocument[pic-]{pic}
\externaldocument[adequate-]{adequate}
\externaldocument[dualizing-]{dualizing}
\externaldocument[duality-]{duality}
\externaldocument[discriminant-]{discriminant}
\externaldocument[local-cohomology-]{local-cohomology}
\externaldocument[curves-]{curves}
\externaldocument[resolve-]{resolve}
\externaldocument[models-]{models}
\externaldocument[pione-]{pione}
\externaldocument[etale-cohomology-]{etale-cohomology}
\externaldocument[proetale-]{proetale}
\externaldocument[crystalline-]{crystalline}
\externaldocument[spaces-]{spaces}
\externaldocument[spaces-properties-]{spaces-properties}
\externaldocument[spaces-morphisms-]{spaces-morphisms}
\externaldocument[decent-spaces-]{decent-spaces}
\externaldocument[spaces-cohomology-]{spaces-cohomology}
\externaldocument[spaces-limits-]{spaces-limits}
\externaldocument[spaces-divisors-]{spaces-divisors}
\externaldocument[spaces-over-fields-]{spaces-over-fields}
\externaldocument[spaces-topologies-]{spaces-topologies}
\externaldocument[spaces-descent-]{spaces-descent}
\externaldocument[spaces-perfect-]{spaces-perfect}
\externaldocument[spaces-more-morphisms-]{spaces-more-morphisms}
\externaldocument[spaces-flat-]{spaces-flat}
\externaldocument[spaces-groupoids-]{spaces-groupoids}
\externaldocument[spaces-more-groupoids-]{spaces-more-groupoids}
\externaldocument[bootstrap-]{bootstrap}
\externaldocument[spaces-pushouts-]{spaces-pushouts}
\externaldocument[groupoids-quotients-]{groupoids-quotients}
\externaldocument[spaces-more-cohomology-]{spaces-more-cohomology}
\externaldocument[spaces-simplicial-]{spaces-simplicial}
\externaldocument[formal-spaces-]{formal-spaces}
\externaldocument[restricted-]{restricted}
\externaldocument[spaces-resolve-]{spaces-resolve}
\externaldocument[formal-defos-]{formal-defos}
\externaldocument[defos-]{defos}
\externaldocument[cotangent-]{cotangent}
\externaldocument[examples-defos-]{examples-defos}
\externaldocument[algebraic-]{algebraic}
\externaldocument[examples-stacks-]{examples-stacks}
\externaldocument[stacks-sheaves-]{stacks-sheaves}
\externaldocument[criteria-]{criteria}
\externaldocument[artin-]{artin}
\externaldocument[quot-]{quot}
\externaldocument[stacks-properties-]{stacks-properties}
\externaldocument[stacks-morphisms-]{stacks-morphisms}
\externaldocument[stacks-limits-]{stacks-limits}
\externaldocument[stacks-cohomology-]{stacks-cohomology}
\externaldocument[stacks-perfect-]{stacks-perfect}
\externaldocument[stacks-introduction-]{stacks-introduction}
\externaldocument[stacks-more-morphisms-]{stacks-more-morphisms}
\externaldocument[stacks-geometry-]{stacks-geometry}
\externaldocument[moduli-]{moduli}
\externaldocument[moduli-curves-]{moduli-curves}
\externaldocument[examples-]{examples}
\externaldocument[exercises-]{exercises}
\externaldocument[guide-]{guide}
\externaldocument[desirables-]{desirables}
\externaldocument[coding-]{coding}
\externaldocument[obsolete-]{obsolete}
\externaldocument[fdl-]{fdl}
\externaldocument[index-]{index}

% Theorem environments.
%
\theoremstyle{plain}
\newtheorem{theorem}[subsection]{Theorem}
\newtheorem{proposition}[subsection]{Proposition}
\newtheorem{lemma}[subsection]{Lemma}

\theoremstyle{definition}
\newtheorem{definition}[subsection]{Definition}
\newtheorem{example}[subsection]{Example}
\newtheorem{exercise}[subsection]{Exercise}
\newtheorem{situation}[subsection]{Situation}

\theoremstyle{remark}
\newtheorem{remark}[subsection]{Remark}
\newtheorem{remarks}[subsection]{Remarks}

\numberwithin{equation}{subsection}

% Macros
%
\def\lim{\mathop{\rm lim}\nolimits}
\def\colim{\mathop{\rm colim}\nolimits}
\def\Spec{\mathop{\rm Spec}}
\def\Hom{\mathop{\rm Hom}\nolimits}
\def\Ext{\mathop{\rm Ext}\nolimits}
\def\SheafHom{\mathop{\mathcal{H}\!{\it om}}\nolimits}
\def\SheafExt{\mathop{\mathcal{E}\!{\it xt}}\nolimits}
\def\Sch{\textit{Sch}}
\def\Mor{\mathop{\rm Mor}\nolimits}
\def\Ob{\mathop{\rm Ob}\nolimits}
\def\Sh{\mathop{\textit{Sh}}\nolimits}
\def\NL{\mathop{N\!L}\nolimits}
\def\proetale{{pro\text{-}\acute{e}tale}}
\def\etale{{\acute{e}tale}}
\def\QCoh{\textit{QCoh}}
\def\Ker{\mathop{\rm Ker}}
\def\Im{\mathop{\rm Im}}
\def\Coker{\mathop{\rm Coker}}
\def\Coim{\mathop{\rm Coim}}

%
% Macros for moduli stacks/spaces
%
\def\QCohstack{\mathcal{QC}\!{\it oh}}
\def\Cohstack{\mathcal{C}\!{\it oh}}
\def\Spacesstack{\mathcal{S}\!{\it paces}}
\def\Quotfunctor{{\rm Quot}}
\def\Hilbfunctor{{\rm Hilb}}
\def\Curvesstack{\mathcal{C}\!{\it urves}}
\def\Polarizedstack{\mathcal{P}\!{\it olarized}}
\def\Complexesstack{\mathcal{C}\!{\it omplexes}}
% \Pic is the operator that assigns to X its picard group, usage \Pic(X)
% \Picardstack_{X/B} denotes the Picard stack of X over B
% \Picardfunctor_{X/B} denotes the Picard functor of X over B
\def\Pic{\mathop{\rm Pic}\nolimits}
\def\Picardstack{\mathcal{P}\!{\it ic}}
\def\Picardfunctor{{\rm Pic}}
\def\Deformationcategory{\mathcal{D}\!{\it ef}}


% OK, start here.
%
\begin{document}

\title{More Algebra}


\maketitle

\phantomsection
\label{section-phantom}

\tableofcontents



\section{Introduction}
\label{section-introduction}

\noindent
In this chapter we prove some results in commutative algebra which
are less elementary than those in the first chapter on commutative
algebra, see
Algebra, Section \ref{algebra-section-phantom}.
A reference is \cite{MatCA}.







\section{Formal glueing of module categories}
\label{section-formal-glueing}

\noindent
Fix a noetherian scheme $X$, and a closed subscheme $Z$ with complement $U$. 
Our goal is to explain a result of Artin that describes how coherent sheaves on 
$X$ can be constructed (uniquely) from coherent sheaves on the formal 
completion of $X$ along $Z$, and those on $U$ with a suitable compatibility on 
the overlap.

\begin{definition}
\label{definition-f-power-torsion}
Let $R$ be a ring. Let $M$ be an $R$-module.
\begin{enumerate}
\item Let $I \subset R$ be an ideal. We say $M$ is an
{\it $I$-power torsion module} if for every $m \in M$ there exists an $n > 0$
such that $I^n m = 0$.
\item Let $f \in R$. We say $M$ is 
{\it an $f$-power torsion module} if for each 
$m \in M$, there exists an $n > 0$ such that $f^n m = 0$.
\end{enumerate}
\end{definition}

\noindent
Thus an $f$-power torsion module is the same thing as a $I$-power torsion
module for $I = (f)$. We sometimes use the notation
$M[I^n] = \{m \in M \mid I^nm = 0\}$ and $M[I^\infty] = \bigcup M[I^n]$
for an $R$-module $M$. Thus $M$ is $I$-power torsion if and only if
$M = M[I^\infty]$ if and only if $M = \bigcup M[I^n]$.

\begin{lemma}
\label{lemma-characterize-flatness-on-torsion}
Let $\varphi : R \to S$ be a ring map. Let $I \subset R$ be an ideal.
The following are equivalent
\begin{enumerate}
\item $\varphi$ is flat and $R/I \to S/IS$ is faithfully flat,
\item $\varphi$ is flat, and the map
$\text{Spec}(S/IS) \to \text{Spec}(R/I)$ is surjective.
\item $\varphi$ is flat, and the base change functor
$M \mapsto M \otimes_R S$ is faithful on modules annihilated by $I$, and
\item $\varphi$ is flat, and the base change functor
$M \mapsto M \otimes_R S$ is faithful on $I$-power torsion modules.
\end{enumerate}
\end{lemma}

\begin{proof}
If $R \to S$ is flat, then $R/I^n \to S/I^nS$ is flat for every $n$, see
Algebra, Lemma \ref{algebra-lemma-flat-base-change}.
Hence (1) and (2) are equivalent by
Algebra, Lemma \ref{algebra-lemma-ff-rings}.
The equivalence of (1) with (3) follows by identifying $I$-torsion
$R$-modules with $R/I$-modules, using that
$$
M \otimes_R S = M \otimes_{R/I} S/IS
$$ 
for $R$-modules $M$ annihilated by $I$, and
Algebra, Lemma \ref{algebra-lemma-easy-ff}.
The implication (4) $\Rightarrow$ (3) is immediate. Assume (3). We have
seen above that $R/I^n \to S/I^nS$ is flat, and by assumption it induces
a surjection on spectra, as $\text{Spec}(R/I^n) = \text{Spec}(R/I)$ and
similarly for $S$. Hence the base change functor is faithful on modules
annihilated by $I^n$. Since any $I$-power torsion module $M$ is the union
$M = \bigcup M_n$ where $M_n$ is annihilated by $I^n$ we see that the base
change functor is faithful on the category of all $I$-power torsion modules
(as tensor product commutes with colimits).
\end{proof}

\begin{lemma}
\label{lemma-I-power-torsion-presentation}
Let $R$ be a ring.
Let $I$ be an ideal of $R$.
Let $M$ be an $I$-power torsion module.
Then $M$ admits a resolution
$$
\ldots \to K_2 \to K_1 \to K_0 \to M \to 0
$$
with each $K_i$ a direct sum of copies of $R/I^n$ for $n$ variable.
\end{lemma}

\begin{proof}
There is a canonical surjection
$$
\oplus_{m \in M} R/I^{n_m} \to M \to 0
$$
where $n_m$ is the smallest positive integer such that $I^{n_m} \cdot m = 0$.
The kernel of the preceding surjection is also an $I$-power torsion module.
Proceeding inductively, we construct the desired resolution of $M$.
\end{proof}

\begin{lemma}
\label{lemma-neighbourhood-isomorphism}
Assume $(\varphi : R \to S, I)$ satisfies the equivalent conditions of
Lemma \ref{lemma-characterize-flatness-on-torsion}.
The following are equivalent
\begin{enumerate}
\item for any $I$-power torsion module $M$, the natural map
$M \to M \otimes_R S$ is an isomorphism, and
\item $R/I \to S/IS$ is an isomorphism.
\end{enumerate}
\end{lemma}

\begin{proof}
The implication (1) $\Rightarrow$ (2) is immediate.
Assume (2). First assume that $M$ is annihilated by $I$.
In this case, $M$ is an $R/I$-module. Hence, we have an isomorphism  
$$
M \otimes_R S = M \otimes_{R/I} S/IS = M \otimes_{R/I} R/I = M
$$
proving the claim. Next we prove by induction that $M \to M \otimes_R S$
is an isomorphism for any module $M$ is annihilated by $I^n$. Assume
the induction hypothesis holds for $n$ and assume $M$ is annihilated by
$I^{n + 1}$. Then we have a short exact sequence
$$
0 \to I^nM \to M \to M/I^nM \to 0
$$
and as $R \to S$ is flat this gives rise to a short exact sequence
$$
0 \to I^nM \otimes_R S \to M \otimes_R S \to M/I^nM \otimes_R S \to 0
$$
Using that the canonical map is an isomorphism for $M' = I^nM$ and
$M'' = M/I^nM$ (by induction hypothesis) we conclude the same thing is
true for $M$. Finally, suppose that $M$ is a general $I$-power torsion
module. Then $M = \bigcup M_n$ where $M_n$ is annihilated by $I^n$
and we conclude using that tensor products commute with colimits.
\end{proof}

\begin{lemma}
\label{lemma-torsion-free}
Let $R$ be a ring. Let $I$ be an ideal of $R$.
For any $R$-module $M$ set $M[I^n] = \{m \in M \mid I^nm = 0\}$.
If $I$ is finitely generated then the following are equivalent
\begin{enumerate}
\item $M[I] = 0$,
\item $M[I^n] = 0$ for all $n \geq 1$, and
\item if $I = (f_1, \ldots, f_t)$, then the map
$M \to \bigoplus M_{f_i}$ is injective.
\end{enumerate}
\end{lemma}

\begin{proof}
This follows from
Algebra, Lemma \ref{algebra-lemma-when-injective-covering}.
\end{proof}

\begin{lemma}
\label{lemma-divide-by-torsion}
Let $R$ be a ring. Let $I$ be an ideal of $R$. For any $R$-module $M$
set $M[I^\infty] = \bigcup_{n \geq 1} M[I^n]$.
If $I$ is finitely generated, then $(M/M[I^\infty])[I] = 0$.
\end{lemma}

\begin{proof}
Let $m \in M$. If $m$ maps to an element of $(M/M[I^\infty])[I]$
then $Im \subset M[I^\infty]$.
Write $I = (f_1, \ldots, f_t)$. Then we see that
$f_i m \in M[I^\infty]$, i.e., $I^{n_i}f_i m = 0$ for some $n_i > 0$.
Thus we see that $I^Nm = 0$ with $N = \sum n_i + 2$.
Hence $m$ maps to zero in $(M/M[I^\infty])$ which proves the lemma.
\end{proof}

\begin{lemma}
\label{lemma-neighbourhood-equivalence}
Assume $\varphi : R \to S$ is a flat ring map and $I \subset R$ is a
finitely generated ideal such that $R/I \to S/IS$ is an isomorphism. Then
\begin{enumerate}
\item for any $R$-module $M$ the map $M \to M \otimes_R S$ induces
an isomorphism
$M[I^\infty] \to (M \otimes_R S)[(IS)^\infty]$ of $I$-power
torsion submodules,
\item the natural map
$$
\text{Hom}_R(M, N) \longrightarrow \text{Hom}_S(M \otimes_R S, N \otimes_R S)
$$
is an isomorphism if either $M$ or $N$ is $I$-power torsion, and
\item the base change functor $M \mapsto M \otimes_R S$ defines an
equivalence of categories between $I$-power torsion modules
and $IS$-power torsion modules.
\end{enumerate}
\end{lemma}

\begin{proof}
Note that the equivalent conditions of both
Lemma \ref{lemma-characterize-flatness-on-torsion} and
Lemma \ref{lemma-neighbourhood-isomorphism}
are satisfied. We will use these without further mention.
We first prove (1). Let $M$ be any $R$-module.
Set $M' = M/M[I^\infty]$ and consider the exact sequence
$$
0 \to M[I^\infty] \to M \to M' \to 0
$$
As $M[I^\infty] = M[I^\infty] \otimes_R S$ we see that it suffices to
show that $(M' \otimes_R S)[(IS)^\infty] = 0$.
Write $I = (f_1, \ldots, f_t)$. By
Lemma \ref{lemma-divide-by-torsion}
we see that $M'[I^\infty] = 0$. Hence for every $n > 0$ the map
$$
M' \longrightarrow \bigoplus\nolimits_{i = 1, \ldots t} M',
\quad
x \longmapsto (f_1^n x, \ldots, f_t^n x)
$$
is injective. As $S$ is flat over $R$ also the corresponding map
$M' \otimes_R S \to \bigoplus_{i = 1, \ldots t} M' \otimes_R S$
is injective. This means that $(M' \otimes_R S)[I^n] = 0$ as desired.

\medskip\noindent
Next we prove (2). If $N$ is $I$-power torsion, then
$N \otimes_R S = N$ and the displayed map of (2) is an isomorphism by
Algebra, Lemma \ref{algebra-lemma-adjoint-tensor-restrict}.
If $M$ is $I$-power torsion, then the image of any map
$M \to N$ factors through $M[I^\infty]$ and the image of any map
$M \otimes_R S \to N \otimes_R S$ factors through
$(N \otimes_R S)[(IS)^\infty]$. Hence in this case
part (1) guarantees that we may replace $N$ by $N[I^\infty]$
and the result follows from the case where $N$ is $I$-power torsion
we just discussed.

\medskip\noindent
Next we prove (3). The functor is fully faithful by (2).
For essential surjectivity, we simply note that for any $IS$-power torsion
$S$-module $N$, the natural map $N \otimes_R S \to N$ is an isomorphism.
\end{proof}

\begin{lemma}
\label{lemma-naive-Koszul-complex}
Let $R$ be a ring. Let $I = (f_1, \ldots, f_n)$ be a finitely generated ideal
of $R$. Let $M$ be the $R$-module generated by elements
$e_1, \ldots, e_n$ subject to the relations $f_i e_j - f_j e_i = 0$.
There exists a short exact sequence
$$
0 \to K \to M \to I \to 0
$$
such that $K$ is annihilated by $I$.
\end{lemma}

\begin{proof}
This is just a truncation of the Koszul complex, see (insert future
reference here).
The map $M \to I$ is is determined by the rule $e_i \mapsto f_i$. If
$m = \sum a_i e_i$ is in the kernel of $M \to I$, i.e., $\sum a_i f_i = 0$,
then $f_j m = \sum f_j a_i e_i = (\sum f_i a_i) e_j = 0$.
\end{proof}

\begin{lemma}
\label{lemma-explicit-ext}
Let $R$ be a ring. Let $I = (f_1, \ldots, f_n)$ be a finitely generated ideal
of $R$. For any $R$-module $N$ set
$$
H_1(N, f_\bullet) =
\frac{\{(x_1, \ldots, x_n) \in N^{\oplus n} \mid f_i x_j = f_j x_i \}}
{\{f_1x, \ldots, f_nx) \mid x \in N\}}
$$
For any $R$-module $N$ there exists a canonical short exact sequence
$$
0 \to \text{Ext}_R(R/I, N) \to H_1(N, f_\bullet) \to \text{Hom}_R(K, N)
$$
where $K$ is as in
Lemma \ref{lemma-naive-Koszul-complex}.
\end{lemma}

\begin{proof}
The notation above indicates the $\text{Ext}$-groups in $\text{Mod}_R$
as defined in
Homology, Section \ref{homology-section-extensions}.
These are denoted $\text{Ext}_R(M, N)$. Using the long exact sequence of
Homology, Lemma \ref{homology-lemma-six-term-sequence-ext}
associated to the short exact sequence $0 \to I \to R \to R/I \to 0$
and the fact that $\text{Ext}_R(R, N) = 0$ we see that
$$
\text{Ext}_R(R/I, N) =
\text{Coker}(N \longrightarrow \text{Hom}(I, N))
$$
Using the short exact sequence of
Lemma \ref{lemma-naive-Koszul-complex}
we see that we get a complex
$$
N \to \text{Hom}(M, N) \to \text{Hom}_R(K, N)
$$
whose homology in the middle is canonically isomorphic to
$\text{Ext}_R(R/I, N)$. The proof of the lemma is now complete
as the cokernel of the first map
is canonically isomorphic to $H_1(N, f_\bullet)$.
\end{proof}

\begin{lemma}
\label{lemma-koszul-homology-annihilated}
Let $R$ be a ring. Let $I = (f_1, \ldots, f_n)$ be a finitely generated ideal
of $R$. For any $R$-module $N$ the Koszul homology group
$H_1(N, f_\bullet)$ defined in
Lemma \ref{lemma-explicit-ext}
is annihilated by $I$.
\end{lemma}

\begin{proof}
Let $(x_1, \ldots, x_n) \in N^{\oplus n}$ with $f_i x_j = f_j x_i$.
Then we have $f_i(x_1, \ldots, x_n) = (f_i x_i, \ldots, f_i x_n)$.
In other words $f_i$ annihilates $H_1(N, f_\bullet)$.
\end{proof}

\noindent
We can improve on the full faithfulness of
Lemma \ref{lemma-neighbourhood-equivalence}
by showing that $\text{Ext}$-groups whose source is $I$-power torsion
are insensitive to passing to $S$ as well. See
Remark \ref{remark-neighbourhood-extensions}
below for a more highbrow version of the following lemma.

\begin{lemma}
\label{lemma-neighbourhood-extensions}
Assume $\varphi : R \to S$ is a flat ring map and $I \subset R$ is a
finitely generated ideal such that $R/I \to S/IS$ is an isomorphism.
Let $M$, $N$ be $R$-modules. Assume $M$ is $I$-power torsion.
Given an short exact sequence
$$
0 \to N \otimes_R S \to \tilde E \to M \otimes_R S \to 0
$$
there exists a commutative diagram
$$
\xymatrix{
0 \ar[r] &
N \ar[r] \ar[d] &
E \ar[r] \ar[d] &
M \ar[r] \ar[d] &
0 \\
0 \ar[r] &
N \otimes_R S \ar[r] &
\tilde E \ar[r] &
M \otimes_R S \ar[r] &
0
}
$$
with exact rows.
\end{lemma}

\begin{proof}
As $M$ is $I$-power torsion we see that $M \otimes_R S = M$, see
Lemma \ref{lemma-neighbourhood-isomorphism}.
We will use this identification without further mention.
As $R \to S$ is flat, the base change functor is exact and we
obtain a functorial map of $\text{Ext}$-groups
$$
\text{Ext}_R(M, N)
\longrightarrow
\text{Ext}_S(M \otimes_R S, N \otimes_R S),
$$
see
Homology, Lemma \ref{homology-lemma-exact-functor-ext}.
The claim of the lemma is that this map is surjective when
$M$ is $I$-power torsion. In fact we will show that it is an
isomorphism. By
Lemma \ref{lemma-I-power-torsion-presentation}
we can find a surjection $M' \to M$ with $M'$ a direct sum of
modules of the form $R/I^n$. Using the long exact sequence of
Homology, Lemma \ref{homology-lemma-six-term-sequence-ext}
we see that it suffices to prove the lemma for $M'$.
Using compatibility of $\text{Ext}$ with direct sums (details omitted)
we reduce to the case where $M = R/I^n$ for some $n$.

\medskip\noindent
Let $f_1, \ldots, f_t$ be generators for $I^n$. By
Lemma \ref{lemma-explicit-ext}
we have a commutative diagram
$$
\xymatrix{
0 \ar[r] &
\text{Ext}_R(R/I^n, N) \ar[r] \ar[d] &
H_1(N, f_\bullet) \ar[r] \ar[d] &
\text{Hom}_R(K, N) \ar[d] \\
0 \ar[r] &
\text{Ext}_S(S/I^nS, N \otimes S) \ar[r] &
H_1(N \otimes S, f_\bullet) \ar[r] &
\text{Hom}_S(K \otimes S, N \otimes S)
}
$$
with exact rows where $K$ is as in
Lemma \ref{lemma-naive-Koszul-complex}.
Hence it suffices to prove that the two right vertical arrows are
isomorphisms. Since $K$ is annihilated by $I^n$ we see that
$\text{Hom}_R(K, N) = \text{Hom}_S(K \otimes_R S, N \otimes_R S)$ by
Lemma \ref{lemma-neighbourhood-equivalence}.
As $R \to S$ is flat we have
$H_1(N, f_\bullet) \otimes_R S = H_1(N \otimes_R S, f_\bullet)$.
As $H_1(N, f_\bullet)$ is annihilated by $I^n$, see
Lemma \ref{lemma-koszul-homology-annihilated}
we have $H_1(N, f_\bullet) \otimes_R S = H_1(N, f_\bullet)$ by
Lemma \ref{lemma-neighbourhood-isomorphism}.
\end{proof}

\begin{remark}
\label{remark-neighbourhood-extensions}
Assume $\varphi : R \to S$ is a flat ring map and $I \subset R$ is a
finitely generated ideal such that $R/I \to S/IS$ is an isomorphism.
Let $M$, $N$ be $R$-modules and assume $M$ is $I$-power torsion.
Then the canonical map
$$
\text{Ext}^i_R(M, N)
\longrightarrow
\text{Ext}^i_S(M \otimes_R S, N \otimes_R S)
$$
is an isomorphism for all $i$. We sketch a proof of this strengthening of
Lemma \ref{lemma-neighbourhood-extensions}.
Consider the Koszul complex $K_\bullet = K_\bullet(R, f_\bullet)$ which is
the complex
$$
0 \to \wedge^n R^n \to \wedge^{n-1} R^n \to
\ldots \to \wedge^i R^n \to \ldots \to R^n \to R \to 0
$$
where the last term $R$ is placed in degree $0$ with maps given by
$$
e_{j_1} \wedge \ldots \wedge e_{j_i}
\longmapsto
\sum\nolimits_{a = 1}^i (-1)^{i + 1} f_{j_a} e_{j_1} \wedge \ldots
\wedge \hat e_{j_a} \wedge \ldots \wedge e_{j_i}
$$
Then $H_0(K_\bullet) = R/I$ and every homology module $H_i(K_\bullet)$
is annihilated by $I$. Having said this, we prove the statement
on $\text{Ext}$-groups by induction on $i$. The case $i = 0$ is
Lemma \ref{lemma-neighbourhood-equivalence}.
Assume that the result holds for all $i \leq i_0$ and all modules
$N$, $M$ with $M$ being $I$-power torsion. Pick a pair of modules
$N$ and $M$ with $M$ being $I$-power torsion and let's prove that
the map
$\text{Ext}^{i_0 + 1}_R(M, N) \to
\text{Ext}^{i_0 + 1}_S(M \otimes_R S, N \otimes_R S)$
is an isomorphism. By
Lemma \ref{lemma-I-power-torsion-presentation}
and the long exact sequence of $\text{Ext}$-groups and compatibility of
$\text{Ext}$ with direct sums we reduce to the case that $M = R/I^n$.
Since $I^n$ is finitely generated we can choose finitely many generators
$f_1, \ldots, f_t \in I^n$ and consider the Koszul complex
$K_\bullet = K_\bullet(R, f_\bullet)$. Note that
$K_\bullet \otimes_R S = K_\bullet(S, f_\bullet)$.
As $K_\bullet$ is a finite complex of finite free $R$-modules we
see that the map
\begin{equation}
\label{equation-comparison}
\text{Hom}_R(K_\bullet, N) \otimes_R S
\longrightarrow
\text{Hom}_S(K_\bullet \otimes_R S, N \otimes_R S)
\end{equation}
is an isomorphism of complexes. As $R \to S$ is flat and using
Lemmas \ref{lemma-neighbourhood-equivalence}
we see that
$$
H_b(K_\bullet) = H_b(K_\bullet) \otimes_R S = H_b(K_\bullet \otimes_R S).
$$
Below we will use the spectral sequences
\begin{align*}
E(R)_2^{a, b} = \text{Ext}^a_R(H_b(K_\bullet), N)
& \Rightarrow
H^{a + b}(\text{Hom}_R(K_\bullet, N)), \\
E(S)_2^{a, b} = \text{Ext}^a_R(H_b(K_\bullet \otimes_R S), N \otimes_R S)
& \Rightarrow
H^{a + b}(\text{Hom}_R(K_\bullet \otimes_R S, N \otimes_R S))
\end{align*}
see (insert future reference here).
The first one combined with the fact that each $H_b(K_\bullet)$
is annihilated by $I^n$ implies that $H^c(\text{Hom}_R(K_\bullet, N))$
is annihilated by $I^{n(t + 1)}$. Hence using
Lemma \ref{lemma-neighbourhood-equivalence}
once more we see that
$$
H^c(\text{Hom}_R(K_\bullet, N)) =
H^c(\text{Hom}_R(K_\bullet, N)) \otimes_R S =
H^c(\text{Hom}_S(K_\bullet \otimes_R S, N \otimes_R S))
$$
because (\ref{equation-comparison}) is an isomorphism and $R \to S$ is flat.
Combined we see that the map $E(R)_r^{a, b} \to E(S)_r^{a, b}$ of spectral
sequences is an isomorphism for $r = 2$ and $a \leq i_0$ (induction hypothesis)
and an isomorphism on abutments in all degrees.
Then a formal argument on spectral sequences (insert future
reference here) implies that
$E(R)_2^{i_0 + 1, 0} \to E(R)_2^{i_0 + 1, 0}$
is an isomorphism as well, which is the result we wanted to prove.
This ends the sketch of the proof of the result on $\text{Ext}$-groups;
if we ever need to use this result in the stacks project we will put in
a detailed proof.
\end{remark}

\noindent
Let $R \to S$ be a ring map.
Let $f_1, \ldots, f_t \in R$ and $I = (f_1, \ldots, f_t)$.
Then for any $R$-module $M$ we can define a complex
\begin{equation}
\label{equation-glueing-complex}
0 \to M \xrightarrow{\alpha}
M \otimes_R S \times \prod M_{f_i} \xrightarrow{\beta}
\prod (M \otimes_R S)_{f_i}
\times
\prod M_{f_if_j}
\end{equation}
where $\alpha(m) = (m \otimes 1, m/1, \ldots, m/1)$ and
$$
\beta(m', m_1, \ldots, m_t) =
((m'/1 - m_1 \otimes 1, \ldots, m'/1 - m_t \otimes 1),
(m_1 - m_2, \ldots, m_{t - 1} - m_t).
$$
We would like to know when this complex is exact.

\begin{lemma}
\label{lemma-recover-module-from-glueing-data}
Assume $\varphi : R \to S$ is a flat ring map and
$I = (f_1, \ldots, f_t) \subset R$ is an ideal such that
$R/I \to S/IS$ is an isomorphism.
Let $M$ be an $R$-module. Then the
complex (\ref{equation-glueing-complex})
is exact.
\end{lemma}

\begin{proof}
Let $m \in M$. If $\alpha(m) = 0$, then $m \in M[I^\infty]$, see
Lemma \ref{lemma-torsion-free}. Pick $n$ such that $I^n m = 0$
and consider the map $\varphi : R/I^n \to M$.
If $m \otimes 1 = 0$, then $\varphi \otimes 1_S = 0$, hence
$\varphi = 0$ (see
Lemma \ref{lemma-neighbourhood-equivalence})
hence $m = 0$. In this way we see that $\alpha$ is injective.

\medskip\noindent
Let $(m', m'_1, \ldots, m'_t) \in \text{Ker}(\beta)$.
Write $m'_i = m_i/f_i^n$ for some $n > 0$ and $m_i \in M$.
We may, after possibly enlarging $n$ assume that
$f_i^n m' = m_i \otimes 1$ in $M \otimes_R S$ and
$f_j^nm_i - f_i^nm_j = 0$ in $M$.
In particular we see that
$(m_1, \ldots, m_t)$ defines an element $\xi$ of
$H_1(M, (f_1^n, \ldots, f_t^n))$.
Since $H_1(M, (f_1^n, \ldots, f_t^n))$ is annihilated by $I^{tn + 1}$ (see
Lemma \ref{lemma-koszul-homology-annihilated})
and since $R \to S$ is flat we see that
$$
H_1(M, (f_1^n, \ldots, f_t^n)) =
H_1(M, (f_1^n, \ldots, f_t^n)) \otimes_R S =
H_1(M \otimes_R S, (f_1^n, \ldots, f_t^n))
$$
by
Lemma \ref{lemma-neighbourhood-isomorphism}
The existence of $m'$ implies that $\xi$ maps to zero in the last group, i.e.,
the element $\xi$ is zero. Thus there exists an $m \in M$ such that
$m_i = f_i^n m$. Then $(m', m'_1, \ldots, m'_t) - \alpha(m)
= (m'', 0, \ldots, 0)$ for some $m'' \in (M \otimes_R S)[(IS)^\infty]$.
By
Lemma \ref{lemma-neighbourhood-equivalence}
we conclude that $m'' \in M[I^\infty]$ and we win.
\end{proof}

\begin{remark}
\label{remark-glueing-data}
In this remark we define a category of glueing data.
Let $R \to S$ be a ring map.
Let $f_1, \ldots, f_t \in R$ and $I = (f_1, \ldots, f_t)$.
Consider the category $\text{Glue}(R \to S, f_1, \ldots, f_t)$
as the category whose
\begin{enumerate}
\item objects are systems $(M', M_i, \alpha_i, \alpha_{ij})$, where
$M'$ is an $S$-module, $M_i$ is an $R_{f_i}$-module,
$\alpha_i : (M')_{f_i} \to M_i \otimes_R S$ is an isomorphism, and
$\alpha_{ij} : (M_i)_{f_j} \to (M_j)_{f_i}$ are isomorphisms
such that
\begin{enumerate}
\item $\alpha_{ij} \circ \alpha_i = \alpha_j$ as maps
$(M')_{f_if_j} \to (M_j)_{f_i}$, and
\item $\alpha_{jk} \circ \alpha_{ij} = \alpha_{ik}$ as maps
$(M_i)_{f_jf_k} \to (M_k)_{f_if_j}$ (cocycle condition).
\end{enumerate}
\item morphisms
$(M', M_i, \alpha_i, \alpha_{ij}) \to (N', N_i, \beta_i, \beta_{ij})$
are given by maps $\varphi' : M' \to N'$ and $\varphi_i : M_i \to N_i$
compatible with the given maps $\alpha_i, \beta_i, \alpha_{ij}, \beta_{ij}$.
\end{enumerate}
There is a canonical functor
$$
\text{Can} : \text{Mod}_R
\longrightarrow
\text{Glue}(R \to S, f_1, \ldots, f_t),
\quad
M \longmapsto (M \otimes_R S, M_{f_i}, \text{can}_i, \text{can}_{ij})
$$
where $\text{can}_i : (M \otimes_R S)_{f_i} \to M_{f_i} \otimes_R S$
and $\text{can}_{ij} : (M_{f_i})_{f_j} \to (M_{f_j})_{f_i}$
are the canonical isomorphisms. For any object
$\mathbf{M} = (M', M_i, \alpha_i, \alpha_{ij})$ of the category
$\text{Glue}(R \to S, f_1, \ldots, f_t)$ we define
$$
H^0(\mathbf{M}) =
\{(m', m_i) \mid \alpha_i(m') = m_i \otimes 1, \alpha_{ij}(m_i) = m_j\}
$$
in other words defined by the exact sequence
$$
0 \to H^0(\mathbf{M}) \to
M' \times \prod M_i \to
\prod M'_{f_i}
\times
\prod (M_i)_{f_j}
$$
similar to (\ref{equation-glueing-complex}).
We think of $H^0(\mathbf{M})$ as an $R$-module. Thus we also get a functor
$$
H^0 : 
\text{Glue}(R \to S, f_1, \ldots, f_t)
\longrightarrow
\text{Mod}_R
$$
Our next goal is to show that the functors
$\text{Can}$ and $H^0$ are sometimes quasi-inverse to each other.
\end{remark}

\begin{lemma}
\label{lemma-H0-inverse}
Assume $\varphi : R \to S$ is a flat ring map and
$I = (f_1, \ldots, f_t) \subset R$ is an ideal such that
$R/I \to S/IS$ is an isomorphism. Then the functor $H^0$
is a left quasi-inverse to the functor $\text{Can}$ of
Remark \ref{remark-glueing-data}.
\end{lemma}

\begin{proof}
This is a reformulation of
Lemma \ref{lemma-recover-module-from-glueing-data}.
\end{proof}

\begin{lemma}
\label{lemma-exact}
Assume $\varphi : R \to S$ is a flat ring map and let
$I = (f_1, \ldots, f_t) \subset R$ be an ideal.
Then $\text{Glue}(R \to S, f_1, \ldots, f_t)$ is an abelian category, and
the functor $\text{Can}$ is exact and commutes with arbitrary colimits.
\end{lemma}

\begin{proof}
Given a morphism
$(\varphi', \varphi_i) :
(M', M_i, \alpha_i, \alpha_{ij})
\to
(N', N_i, \beta_i, \beta_{ij})$
of the category $\text{Glue}(R \to S, f_1, \ldots, f_t)$
we see that its kernel exists and is equal to the object
$(\text{Ker}(\varphi'), \text{Ker}(\varphi_i), \alpha_i, \alpha_{ij})$
and its cokernel exists and is equal to the object
$(\text{Coker}(\varphi'), \text{Coker}(\varphi_i), \beta_i, \beta_{ij})$.
This works because $R \to S$ is flat, hence taking kernels/cokernels
commutes with $- \otimes_R S$. Details omitted.
The exactness follows from the $R$-flatness of $R_{f_i}$ and $S$, while
commuting with colimits follows as tensor products commute with colimits.
\end{proof}

\begin{lemma}
\label{lemma-equivalence-I-unit}
Let $\varphi : R \to S$ be a flat ring map and $(f_1, \ldots, f_t) = R$.
Then $\text{Can}$ and $H^0$ are quasi-inverse equivalences of categories
$$
\text{Mod}_R = \text{Glue}(R \to S, f_1, \ldots, f_t)
$$
\end{lemma}

\begin{proof}
Consider an object $\mathbf{M} = (M', M_i, \alpha_i, \alpha_{ij})$
of $\text{Glue}(R \to S, f_1, \ldots, f_t)$. By
Algebra, Lemma \ref{algebra-lemma-glue-modules}
there exists a unique module $M$ and isomorphisms
$M_{f_i} \to M_i$ which recover the glueing data $\alpha_{ij}$.
Then both $M'$ and $M \otimes_R S$ are $S$-modules
which recover the modules $M_i \otimes_R S$ upon localizing at $f_i$.
Whence there is a canonical isomorphism $M \otimes_R S \to M'$.
This shows that $\mathbf{M}$ is in the essential image of $\text{Can}$.
Combined with
Lemma \ref{lemma-H0-inverse}
the lemma follows.
\end{proof}

\begin{lemma}
\label{lemma-base-change-glue}
Let $\varphi : R \to S$ be a flat ring map and $I = (f_1, \ldots, f_t)$
and ideal. Let $R \to R'$ be a flat ring map, and set $S' = S \otimes_R R'$.
Then we obtain a commutative diagram of categories and functors
$$
\xymatrix{
\text{Mod}_R \ar[r]_-{\text{Can}} \ar[d]_{-\otimes_R R'} &
\text{Glue}(R \to S, f_1, \ldots, f_t) \ar[r]_-{H^0} \ar[d]^{-\otimes_R R'} &
\text{Mod}_R \ar[d]^{-\otimes_R R'} \\
\text{Mod}_{R'} \ar[r]^-{\text{Can}} &
\text{Glue}(R' \to S', f_1, \ldots, f_t) \ar[r]^-{H^0} &
\text{Mod}_{R'}
}
$$
\end{lemma}

\begin{proof}
Omitted.
\end{proof}

\begin{proposition}
\label{proposition-equivalence}
Assume $\varphi : R \to S$ is a flat ring map and
$I = (f_1, \ldots, f_t) \subset R$ is an ideal such that
$R/I \to S/IS$ is an isomorphism. Then $\text{Can}$ and
$H^0$ are quasi-inverse equivalences of categories
$$
\text{Mod}_R = \text{Glue}(R \to S, f_1, \ldots, f_t)
$$
\end{proposition}

\begin{proof}
We have already seen that $H^0 \circ \text{Can}$ is isomorphic to the
identity functor, see
Lemma \ref{lemma-H0-inverse}.
Consider an object $\mathbf{M} = (M', M_i, \alpha_i, \alpha_{ij})$
of $\text{Glue}(R \to S, f_1, \ldots, f_t)$.
We get a natural morphism
$$
\Psi :
(H^0(\mathbf{M}) \otimes_R S, H^0(\mathbf{M})_{f_i},
\text{can}_i, \text{can}_{ij})
\longrightarrow
(M', M_i, \alpha_i, \alpha_{ij}).
$$
Namely, by definition $H^0(\mathbf{M})$ comes equipped with compatible
$R$-module maps $H^0(\mathbf{M}) \to M'$ and $H^0(\mathbf{M}) \to M_i$.
We have to show that this map is an isomorphism.

\medskip\noindent
Pick an index $i$ and set $R' = R_{f_i}$. Combining
Lemmas \ref{lemma-base-change-glue} and \ref{lemma-equivalence-I-unit}
we see that $\Psi \otimes_R R'$ is an isomorphism.
Hence the kernel, resp.\ cokernel of $\Psi$ is a system of the form
$(K, 0, 0, 0)$, resp.\ $(Q, 0, 0, 0)$. Note that
$H^0((K, 0, 0, 0)) = K$, that $H^0$ is left exact, and that by
construction $H^0(\Psi)$ is bijective. Hence we see $K = 0$, i.e.,
the kernel of $\Psi$ is zero.

\medskip\noindent
The conclusion of the above is that we obtain a short exact sequence
$$
0 \to H^0(\mathbf{M}) \otimes_R S \to M' \to Q \to 0
$$
and that $M_i = H^0(\mathbf{M})_{f_i}$.
Note that we may think of $Q$ as an $R$-module which is $I$-power
torsion so that $Q = Q \otimes_R S$. By
Lemma \ref{lemma-neighbourhood-extensions}
we see that there exists a commutative diagram
$$
\xymatrix{
0 \ar[r] &
H^0(\mathbf{M}) \ar[r] \ar[d] &
E \ar[r] \ar[d] &
Q \ar[r] \ar[d] &
0 \\
0 \ar[r] &
H^0(\mathbf{M}) \otimes_R S \ar[r] &
M' \ar[r] &
Q \ar[r] &
0
}
$$
with exact rows. This clearly determines an isomorphism
$\text{Can}(E) \to (M', M_i, \alpha_i, \alpha_{ij})$
in the category $\text{Glue}(R \to S, f_1, \ldots, f_t)$
and we win. (Of course, a posteriori we have $Q = 0$.)
\end{proof}

\noindent
Next, we specialize this very general proposition to get something
more useable. Namely, if $I = (f)$ is a principal ideal then the objects
of $\text{Glue}(R \to S, f)$ are simply triples $(M', M_1, \alpha_1)$
and there is {\it no} cocycle condition to check!

\begin{theorem}
\label{theorem-formal-glueing}
Let $R$ be a ring, and let $f \in R$.
Let $\varphi : R \to S$ be a flat ring map inducing an isomorphism
$R/fR \to S/fS$. Then the functor
$$
\text{Mod}_R
\longrightarrow
\text{Mod}_S \times_{\text{Mod}_{S_f}} \text{Mod}_{R_f},
\quad
M
\longmapsto
(M \otimes_R S, M_f, \text{can})
$$
is an equivalence.
\end{theorem}

\begin{proof}
The category appearing on the right side of the arrow
is the category of triples $(M', M_1, \alpha_1)$ where $M'$ is an
$S$-module, $M_1$ is a $R_f$-module, and
$\alpha_1 : M'_f \to M_1 \otimes_R S$ is a $S_f$-isomorphism, see
Categories, Example \ref{categories-example-2-fibre-product-categories}.
Hence this theorem is a special case of
Proposition \ref{proposition-equivalence}.
\end{proof}

\noindent
A useful special case of
Theorem \ref{theorem-formal-glueing}
is when $R$ is noetherian, and $S$ is a completion of $R$ at an
element $f$. The completion $R \to S$ is flat, and the functor
$M \mapsto M \otimes_R S$ can be identified with the $f$-adic
completion functor when $M$ is finitely generated. To state
this more precisely, let $\text{Mod}_{fg}(R)$ denote the category
of finitely generated $R$-modules.

\begin{proposition}
\label{proposition-formal-glueing}
Let $R$ be a noetherian ring.
Let $f \in R$ be an element.
Let $R^\wedge$ be the $f$-adic completion of $R$.
Then the functor $M \mapsto (M^\wedge, M_f, \text{can})$
defines an equivalence
$$
\text{Mod}_{fg}(R)
\longrightarrow
\text{Mod}_{fg}(R^\wedge)
\times_{\text{Mod}_{fg}(R^\wedge_f)} 
\text{Mod}_{fg}(R_f)
$$
\end{proposition}

\begin{proof}
The ring map $R \to R^\wedge$ is flat by
Algebra, Lemma \ref{algebra-lemma-completion-flat}.
It is clear that $R/fR = R^\wedge/fR^\wedge$.
By
Algebra, Lemma \ref{algebra-lemma-completion-tensor}
the completion of a finite $R$-module $M$ is equal to $M \otimes_R R^\wedge$.
Hence the displayed functor of the proposition is equal to the
functor occuring in
Theorem \ref{theorem-formal-glueing}.
In particular it is fully faithful. Let $(M_1, M_2, \psi)$ be an
object of the right hand side. By
Theorem \ref{theorem-formal-glueing}
there exists an $R$-module $M$ such that
$M_1 = M \otimes_R R^\wedge$ and $M_2 = M_f$. As $R \to R^\wedge \times R_f$
is faithfully flat we conclude from
Algebra, Lemma \ref{algebra-lemma-cover}
that $M$ is finitely generated, i.e., $M \in \text{Mod}_{fg}(R)$.
This proves the proposition.
\end{proof}

\begin{remark}
\label{remark-formal-glueing-algebras}
The equivalences of
Propostion \ref{proposition-equivalence},
Theorem \ref{theorem-formal-glueing}, and
Proposition \ref{proposition-formal-glueing}
preserve the $\otimes$-structures on either side.
Thus, it defines equivalences of various categories
built out of the pair $(\text{Mod}_R,\otimes)$, such as the category of 
$R$-algebras.
\end{remark}

\begin{remark}
\label{remark-topological-analogue}
Given a differential manifold $X$ with a compact closed submanifold $Z$
having complement $U$, specifying a sheaf on $X$ is the same as specifying 
a sheaf on $U$, a sheaf on an unspecified tubular neighbourhood $T$ of $Z$ in 
$X$, and an isomorphism between the two resulting sheaves along $T \cap U$.
Tubular neighbourhoods do not exist in algebraic geometry as such, but
results such as
Propostion \ref{proposition-equivalence},
Theorem \ref{theorem-formal-glueing}, and
Proposition \ref{proposition-formal-glueing}
allow us to work with formal neighbourhoods instead.
\end{remark}



\section{Other chapters}

\begin{multicols}{2}
\begin{enumerate}
\item \hyperref[introduction-section-phantom]{Introduction}
\item \hyperref[conventions-section-phantom]{Conventions}
\item \hyperref[sets-section-phantom]{Set Theory}
\item \hyperref[categories-section-phantom]{Categories}
\item \hyperref[topology-section-phantom]{Topology}
\item \hyperref[sheaves-section-phantom]{Sheaves on Spaces}
\item \hyperref[algebra-section-phantom]{Commutative Algebra}
\item \hyperref[sites-section-phantom]{Sites and Sheaves}
\item \hyperref[homology-section-phantom]{Homological Algebra}
\item \hyperref[derived-section-phantom]{Derived Categories}
\item \hyperref[more-algebra-section-phantom]{More Algebra}
\item \hyperref[simplicial-section-phantom]{Simplicial Methods}
\item \hyperref[modules-section-phantom]{Sheaves of Modules}
\item \hyperref[sites-modules-section-phantom]{Modules on Sites}
\item \hyperref[injectives-section-phantom]{Injectives}
\item \hyperref[cohomology-section-phantom]{Cohomology of Sheaves}
\item \hyperref[sites-cohomology-section-phantom]{Cohomology on Sites}
\item \hyperref[hypercovering-section-phantom]{Hypercoverings}
\item \hyperref[schemes-section-phantom]{Schemes}
\item \hyperref[constructions-section-phantom]{Constructions of Schemes}
\item \hyperref[properties-section-phantom]{Properties of Schemes}
\item \hyperref[morphisms-section-phantom]{Morphisms of Schemes}
\item \hyperref[coherent-section-phantom]{Coherent Cohomology}
\item \hyperref[divisors-section-phantom]{Divisors}
\item \hyperref[limits-section-phantom]{Limits of Schemes}
\item \hyperref[varieties-section-phantom]{Varieties}
\item \hyperref[chow-section-phantom]{Chow Homology}
\item \hyperref[topologies-section-phantom]{Topologies on Schemes}
\item \hyperref[descent-section-phantom]{Descent}
\item \hyperref[more-morphisms-section-phantom]{More on Morphisms}
\item \hyperref[flat-section-phantom]{More on Flatness}
\item \hyperref[groupoids-section-phantom]{Groupoid Schemes}
\item \hyperref[more-groupoids-section-phantom]{More on Groupoid Schemes}
\item \hyperref[etale-section-phantom]{\'Etale Morphisms of Schemes}
\item \hyperref[etale-cohomology-section-phantom]{\'Etale Cohomology}
\item \hyperref[spaces-section-phantom]{Algebraic Spaces}
\item \hyperref[spaces-properties-section-phantom]{Properties of Algebraic Spaces}
\item \hyperref[spaces-morphisms-section-phantom]{Morphisms of Algebraic Spaces}
\item \hyperref[spaces-topologies-section-phantom]{Topologies on Algebraic Spaces}
\item \hyperref[spaces-descent-section-phantom]{Descent and Algebraic Spaces}
\item \hyperref[spaces-more-morphisms-section-phantom]{More on Morphisms of Spaces}
\item \hyperref[quot-section-phantom]{Quot and Hilbert Spaces}
\item \hyperref[stacks-section-phantom]{Stacks}
\item \hyperref[spaces-groupoids-section-phantom]{Groupoids in Algebraic Spaces}
\item \hyperref[spaces-more-groupoids-section-phantom]{More on Groupoids in Spaces}
\item \hyperref[bootstrap-section-phantom]{Bootstrap}
\item \hyperref[examples-stacks-section-phantom]{Examples of Stacks}
\item \hyperref[groupoids-quotients-section-phantom]{Quotients of Groupoids}
\item \hyperref[algebraic-section-phantom]{Algebraic Stacks}
\item \hyperref[criteria-section-phantom]{Criteria for Representability}
\item \hyperref[stacks-properties-section-phantom]{Properties of Algebraic Stacks}
\item \hyperref[stacks-morphisms-section-phantom]{Morphisms of Algebraic Stacks}
\item \hyperref[examples-section-phantom]{Examples}
\item \hyperref[exercises-section-phantom]{Exercises}
\item \hyperref[guide-section-phantom]{Guide to Literature}
\item \hyperref[desirables-section-phantom]{Desirables}
\item \hyperref[coding-section-phantom]{Coding Style}
\item \hyperref[fdl-section-phantom]{GNU Free Documentation License}
\item \hyperref[index-section-phantom]{Auto Generated Index}
\end{enumerate}
\end{multicols}


\bibliography{my}
\bibliographystyle{amsalpha}

\end{document}
