\IfFileExists{stacks-project.cls}{%
\documentclass{stacks-project}
}{%
\documentclass{amsart}
}

% The following AMS packages are automatically loaded with
% the amsart documentclass:
%\usepackage{amsmath}
%\usepackage{amssymb}
%\usepackage{amsthm}

% For dealing with references we use the comment environment
\usepackage{verbatim}
\newenvironment{reference}{\comment}{\endcomment}
%\newenvironment{reference}{}{}
\newenvironment{slogan}{\comment}{\endcomment}
\newenvironment{history}{\comment}{\endcomment}

% For commutative diagrams you can use
% \usepackage{amscd}
\usepackage[all]{xy}

% We use 2cell for 2-commutative diagrams.
\xyoption{2cell}
\UseAllTwocells

% To put source file link in headers.
% Change "template.tex" to "this_filename.tex"
% \usepackage{fancyhdr}
% \pagestyle{fancy}
% \lhead{}
% \chead{}
% \rhead{Source file: \url{template.tex}}
% \lfoot{}
% \cfoot{\thepage}
% \rfoot{}
% \renewcommand{\headrulewidth}{0pt}
% \renewcommand{\footrulewidth}{0pt}
% \renewcommand{\headheight}{12pt}

\usepackage{multicol}

% For cross-file-references
\usepackage{xr-hyper}

% Package for hypertext links:
\usepackage{hyperref}

% For any local file, say "hello.tex" you want to link to please
% use \externaldocument[hello-]{hello}
\externaldocument[introduction-]{introduction}
\externaldocument[conventions-]{conventions}
\externaldocument[sets-]{sets}
\externaldocument[categories-]{categories}
\externaldocument[topology-]{topology}
\externaldocument[sheaves-]{sheaves}
\externaldocument[sites-]{sites}
\externaldocument[stacks-]{stacks}
\externaldocument[fields-]{fields}
\externaldocument[algebra-]{algebra}
\externaldocument[brauer-]{brauer}
\externaldocument[homology-]{homology}
\externaldocument[derived-]{derived}
\externaldocument[simplicial-]{simplicial}
\externaldocument[more-algebra-]{more-algebra}
\externaldocument[smoothing-]{smoothing}
\externaldocument[modules-]{modules}
\externaldocument[sites-modules-]{sites-modules}
\externaldocument[injectives-]{injectives}
\externaldocument[cohomology-]{cohomology}
\externaldocument[sites-cohomology-]{sites-cohomology}
\externaldocument[dga-]{dga}
\externaldocument[dpa-]{dpa}
\externaldocument[hypercovering-]{hypercovering}
\externaldocument[schemes-]{schemes}
\externaldocument[constructions-]{constructions}
\externaldocument[properties-]{properties}
\externaldocument[morphisms-]{morphisms}
\externaldocument[coherent-]{coherent}
\externaldocument[divisors-]{divisors}
\externaldocument[limits-]{limits}
\externaldocument[varieties-]{varieties}
\externaldocument[topologies-]{topologies}
\externaldocument[descent-]{descent}
\externaldocument[perfect-]{perfect}
\externaldocument[more-morphisms-]{more-morphisms}
\externaldocument[flat-]{flat}
\externaldocument[groupoids-]{groupoids}
\externaldocument[more-groupoids-]{more-groupoids}
\externaldocument[etale-]{etale}
\externaldocument[chow-]{chow}
\externaldocument[intersection-]{intersection}
\externaldocument[pic-]{pic}
\externaldocument[adequate-]{adequate}
\externaldocument[dualizing-]{dualizing}
\externaldocument[duality-]{duality}
\externaldocument[discriminant-]{discriminant}
\externaldocument[local-cohomology-]{local-cohomology}
\externaldocument[curves-]{curves}
\externaldocument[resolve-]{resolve}
\externaldocument[models-]{models}
\externaldocument[pione-]{pione}
\externaldocument[etale-cohomology-]{etale-cohomology}
\externaldocument[proetale-]{proetale}
\externaldocument[crystalline-]{crystalline}
\externaldocument[spaces-]{spaces}
\externaldocument[spaces-properties-]{spaces-properties}
\externaldocument[spaces-morphisms-]{spaces-morphisms}
\externaldocument[decent-spaces-]{decent-spaces}
\externaldocument[spaces-cohomology-]{spaces-cohomology}
\externaldocument[spaces-limits-]{spaces-limits}
\externaldocument[spaces-divisors-]{spaces-divisors}
\externaldocument[spaces-over-fields-]{spaces-over-fields}
\externaldocument[spaces-topologies-]{spaces-topologies}
\externaldocument[spaces-descent-]{spaces-descent}
\externaldocument[spaces-perfect-]{spaces-perfect}
\externaldocument[spaces-more-morphisms-]{spaces-more-morphisms}
\externaldocument[spaces-flat-]{spaces-flat}
\externaldocument[spaces-groupoids-]{spaces-groupoids}
\externaldocument[spaces-more-groupoids-]{spaces-more-groupoids}
\externaldocument[bootstrap-]{bootstrap}
\externaldocument[spaces-pushouts-]{spaces-pushouts}
\externaldocument[groupoids-quotients-]{groupoids-quotients}
\externaldocument[spaces-more-cohomology-]{spaces-more-cohomology}
\externaldocument[spaces-simplicial-]{spaces-simplicial}
\externaldocument[formal-spaces-]{formal-spaces}
\externaldocument[restricted-]{restricted}
\externaldocument[spaces-resolve-]{spaces-resolve}
\externaldocument[formal-defos-]{formal-defos}
\externaldocument[defos-]{defos}
\externaldocument[cotangent-]{cotangent}
\externaldocument[examples-defos-]{examples-defos}
\externaldocument[algebraic-]{algebraic}
\externaldocument[examples-stacks-]{examples-stacks}
\externaldocument[stacks-sheaves-]{stacks-sheaves}
\externaldocument[criteria-]{criteria}
\externaldocument[artin-]{artin}
\externaldocument[quot-]{quot}
\externaldocument[stacks-properties-]{stacks-properties}
\externaldocument[stacks-morphisms-]{stacks-morphisms}
\externaldocument[stacks-limits-]{stacks-limits}
\externaldocument[stacks-cohomology-]{stacks-cohomology}
\externaldocument[stacks-perfect-]{stacks-perfect}
\externaldocument[stacks-introduction-]{stacks-introduction}
\externaldocument[stacks-more-morphisms-]{stacks-more-morphisms}
\externaldocument[stacks-geometry-]{stacks-geometry}
\externaldocument[moduli-]{moduli}
\externaldocument[moduli-curves-]{moduli-curves}
\externaldocument[examples-]{examples}
\externaldocument[exercises-]{exercises}
\externaldocument[guide-]{guide}
\externaldocument[desirables-]{desirables}
\externaldocument[coding-]{coding}
\externaldocument[obsolete-]{obsolete}
\externaldocument[fdl-]{fdl}
\externaldocument[index-]{index}

% Theorem environments.
%
\theoremstyle{plain}
\newtheorem{theorem}[subsection]{Theorem}
\newtheorem{proposition}[subsection]{Proposition}
\newtheorem{lemma}[subsection]{Lemma}

\theoremstyle{definition}
\newtheorem{definition}[subsection]{Definition}
\newtheorem{example}[subsection]{Example}
\newtheorem{exercise}[subsection]{Exercise}
\newtheorem{situation}[subsection]{Situation}

\theoremstyle{remark}
\newtheorem{remark}[subsection]{Remark}
\newtheorem{remarks}[subsection]{Remarks}

\numberwithin{equation}{subsection}

% Macros
%
\def\lim{\mathop{\rm lim}\nolimits}
\def\colim{\mathop{\rm colim}\nolimits}
\def\Spec{\mathop{\rm Spec}}
\def\Hom{\mathop{\rm Hom}\nolimits}
\def\Ext{\mathop{\rm Ext}\nolimits}
\def\SheafHom{\mathop{\mathcal{H}\!{\it om}}\nolimits}
\def\SheafExt{\mathop{\mathcal{E}\!{\it xt}}\nolimits}
\def\Sch{\textit{Sch}}
\def\Mor{\mathop{\rm Mor}\nolimits}
\def\Ob{\mathop{\rm Ob}\nolimits}
\def\Sh{\mathop{\textit{Sh}}\nolimits}
\def\NL{\mathop{N\!L}\nolimits}
\def\proetale{{pro\text{-}\acute{e}tale}}
\def\etale{{\acute{e}tale}}
\def\QCoh{\textit{QCoh}}
\def\Ker{\mathop{\rm Ker}}
\def\Im{\mathop{\rm Im}}
\def\Coker{\mathop{\rm Coker}}
\def\Coim{\mathop{\rm Coim}}

%
% Macros for moduli stacks/spaces
%
\def\QCohstack{\mathcal{QC}\!{\it oh}}
\def\Cohstack{\mathcal{C}\!{\it oh}}
\def\Spacesstack{\mathcal{S}\!{\it paces}}
\def\Quotfunctor{{\rm Quot}}
\def\Hilbfunctor{{\rm Hilb}}
\def\Curvesstack{\mathcal{C}\!{\it urves}}
\def\Polarizedstack{\mathcal{P}\!{\it olarized}}
\def\Complexesstack{\mathcal{C}\!{\it omplexes}}
% \Pic is the operator that assigns to X its picard group, usage \Pic(X)
% \Picardstack_{X/B} denotes the Picard stack of X over B
% \Picardfunctor_{X/B} denotes the Picard functor of X over B
\def\Pic{\mathop{\rm Pic}\nolimits}
\def\Picardstack{\mathcal{P}\!{\it ic}}
\def\Picardfunctor{{\rm Pic}}
\def\Deformationcategory{\mathcal{D}\!{\it ef}}


% OK, start here.
%
\begin{document}

\title{More on flatness}

\maketitle

\phantomsection
\label{section-phantom}

\tableofcontents



\section{Introduction}
\label{section-introduction}

\noindent
In this chapter, we discuss some advanced results on flat modules and
flat morphisms of schemes. Most of these results can be
found in the paper \cite{GruRay} by Raynaud and Gruson.

\medskip\noindent
Before reading this chapter we advise the reader to take a look
at the following results (this list also serves as a pointer to
previous results):
\begin{enumerate}
\item General discussion on flat modules in
Algebra, Section \ref{algebra-section-flat}.
\item The relationship between $\text{Tor}$-groups and flatness, see
Algebra, Section \ref{algebra-section-tor}.
\item The sections on flatness criteria, namely,
Algebra, Section \ref{algebra-section-criteria-flatness}
(Noetherian case),
Algebra, Section \ref{algebra-section-flatness-artinian}
(Artinian case),
Algebra, Section \ref{algebra-section-more-flatness-criteria}
(non-Noetherian case), and finally
More on Morphisms, Section \ref{more-morphisms-section-criterion-flat-fibres}.
\item Generic flatness, see
Algebra, Section \ref{algebra-section-generic-flatness}
and
Morphisms, Section \ref{morphisms-section-generic-flatness}.
\item Openness of the flat locus, see
Algebra, Section \ref{algebra-section-open-flat}
and
More on Morphisms, Section \ref{more-morphisms-section-open-flat}.
\item Flattening stratification, see
Algebra, Section \ref{algebra-section-flattening}.
\item Additional algebraic in
Algebra, Sections \ref{algebra-section-descent-flatness-integral},
\ref{algebra-section-torsion-flat},
\ref{algebra-section-flat-finite}, and
\ref{algebra-section-blowup-flat}.
\end{enumerate}








\section{The local structure of a finite type module}
\label{section-local-structure-module}

\noindent
The key technical lemma that makes a lot of the arguments in this
chapter work is the following geometric lemma.

\begin{lemma}
\label{lemma-elementary-devissage}
Let $f : X \to S$ be a finite type morphism of schemes.
Let $\mathcal{F}$ be a finite type quasi-coherent $\mathcal{O}_X$-module.
Let $x \in X$ with image $s = f(x)$ in $S$.
Set $\mathcal{F}_s = \mathcal{F}|_{X_s}$ and
$n = \dim_x(\text{Supp}(\mathcal{F}_s))$.
Then we can construct
\begin{enumerate}
\item elementary etale neighbourhoods $g : (X', x') \to (X, x)$,
$e : (S', s') \to (S, s)$,
\item a commutative diagram
$$
\xymatrix{
X' \ar[d]_g & Z' \ar[l]^i \ar[r]_\pi & Y' \ar[r]_h & S' \ar[d]^e \\
X \ar[rrr]^f & & & S
}
$$
\item points $z' \in Z'$, $y' = \pi(z') \in Y'$ mapping to $x'$, $s'$, and
\item a finite type quasi-coherent $\mathcal{O}_{Z'}$-module $\mathcal{G}$
\end{enumerate}
such that the following properties hold
\begin{enumerate}
\item $X'$, $Z'$, $Y'$, $S'$ are affine schemes,
\item $i$ is a closed immersion of finite presentation,
\item $i_*(\mathcal{G}) \cong \mathcal{F}$,
\item $\pi$ is finite and $\pi^{-1}(\{y'\}) = \{x'\}$,
\item $h$ is smooth of relative dimension $n$
with geometrically integral fibres,
\item $\kappa(s') \subset \kappa(y')$ is
purely transcendental, and
\item an integer $r \geq 0$ and a map
$\alpha : \mathcal{O}_{Y'}^{\oplus r} \to \pi_*(\mathcal{G})$
such that $\alpha_{\xi} \otimes \kappa(\xi)$ is an isomorphism
where $\xi \in Y'_{s'}$ is the generic point.
\end{enumerate}
Moreover, if $f$ is locally of finite presentation and
$\mathcal{F}$ is an $\mathcal{O}_X$-module of finite presentation,
then $\mathcal{G}$ is an $\mathcal{O}_{Z'}$-module of finite presentation.
\end{lemma}

\begin{proof}
Let $V \subset S$ be an affine neighbourhood of $s$.
Let $U \subset f^{-1}(V)$ be an affine neighbourhood of $x$.
Then it suffices to prove the lemma for $f|_U : U \to V$ and
$\mathcal{F}|_U$. Hence we may assume that $X$ and $S$ are affine.

\medskip\noindent
Say the morphism $f : X \to S$ is given by the ring map
$A \to B$ and that $\mathcal{F}$ is the quasi-coherent sheaf
associated to the $B$-module $M$. By
Morphisms, Lemma \ref{morphisms-lemma-locally-finite-type-characterize}
we know that $A \to B$ is a finite type ring map, and by
Properties, Lemma \ref{properties-lemma-finite-type-module}
we know that $M$ is a finite $B$-module. In particular the
support of $\mathcal{F}$ is the closed subscheme of $\text{Spec}(B)$
cut out by $I = \{x \in B \mid xm = 0\ \forall m \in M\}$
the annihilator of $M$, see
Algebra, Lemma \ref{algebra-lemma-support-closed}.
Let $\mathfrak q \subset B$ be the prime ideal corresponding to $x$
and let $\mathfrak p \subset A$ be the prime ideal corresponding to $s$.
Note that $X_s = \text{Spec}(B \otimes_A \kappa(\mathfrak p))$ and
that $\mathcal{F}_s$ is the quasi-coherent sheaf associated to the
$B \otimes_A \kappa(\mathfrak p)$ module $M \otimes_A \kappa(\mathfrak p)$. By
Divisors, Lemma \ref{divisors-lemma-support-finite-type}
the support of $\mathcal{F}_s$ is equal to
$V(I(B \otimes_A \kappa(\mathfrak p)))$. Since
$B \otimes_A \kappa(\mathfrak p)$ is of finite type over $\kappa(\mathfrak p)$
there exist finitely many elements $f_1, \ldots, f_m \in I$
such that
$$
I(B \otimes_A \kappa(\mathfrak p)) =
(f_1, \ldots, f_n)(B \otimes_A \kappa(\mathfrak p)).
$$
Denote $i : Z \to X$ the closed subscheme cut out by
$(f_1, \ldots, f_m)$, in a formula $Z = \text{Spec}(B/(f_1, \ldots, f_m))$.
Since $M$ is annihilated by $I$ we can think of $M$ as an
$B/(f_1, \ldots, f_m)$-module. In other words, $\mathcal{F}$ is the
pushforward of a finite type module on $Z$.

\end{proof}





\section{Flattening stratification}
\label{section-flattening}

\noindent
If $f : X \to S$ is a proper, finitely presented morphism
of schemes then one can find a stratification of the base
over whose members the morphism $f$ is flat. It is not so hard
to find this stratification, but what is a bit tricky is to prove
that this stratification is characterized by a universal property.
In this section we discuss this and some of its variants.

\medskip\noindent
The first case is where the base is the spectrum of a complete
local Noetherian ring. In this case the closed stratum is defined
and satisfies the desired universal property even for a general
finite type morphism, provided we {\it only} look for flatness in
points lying over the closed point.

\begin{lemma}
\label{lemma-flattening-complete-noetherian}
Let $f : X \to S$ be a morphism of schemes.
Let $\mathcal{F}$ be a quasi-coherent sheaf on $X$.
Assume
\begin{enumerate}
\item $S = \text{Spec}(A)$ is the spectrum of a Noetherian complete
local ring $A$,
\item $f$ is of finite type, and
\item $\mathcal{F}$ is a finite type $\mathcal{O}_X$-module.
\end{enumerate}
Then there exists a closed subscheme $Z \subset S$ with the following
universal property: Given a morphism $S' \to S$ which corresponds to
a local homomorphism $A' \to A$ of local rings the following
are equivalent
\begin{enumerate}
\item[(a)] the pullback $\mathcal{F}'$ of $\mathcal{F}$ to
$X' = S' \times_S X$ is flat over $S'$ at all points $x' \in X'$
lying over the closed point of $S'$, and
\item[(b)] $S' \to S$ factors through $Z$.
\end{enumerate}
In particular, $Z$ is the largest closed subscheme of $S$ such that
$\mathcal{F}|_{X_Z}$ is flat over $Z$ at all points
of $X_Z$ lying over the closed point of $Z$.
\end{lemma}

\begin{proof}
As $f$ is of finite type it is quasi-compact. Hence $X$ is quasi-compact.
Choose a finite affine open covering $X = \bigcup X_i$.
Write $X_i = \text{Spec}(B_i)$ so that $A \to B_i$ is a finite type
ring map. Note that $\mathcal{F}|_{X_i}$ corresponds to a finite
$B_i$-module $M_i$. Set $Y = \coprod X_i = \text{Spec}(\prod B_i)$
and $\mathcal{G} = \widetilde{\prod M_i}$ on $Y$.
Now if $S' \to S$ is a morphism as in the lemma, then (a) holds
for $\mathcal{F}'$ relative to $X' \to S'$ if and only if (a) holds for
$\mathcal{G}'$ relative to $Y' \to S'$. Hence this
reduces us to the case where $X$ is affine.
In this case $Z$ exists by
Algebra,
Lemma \ref{algebra-lemma-flattening-complete-local-universal-property}.
(Hint: We strongly suggest the reader only read the construction of
$Z = \text{Spec}(A/I)$ in
Algebra, Lemma \ref{algebra-lemma-flattening-complete-local-noetherian}.)
\end{proof}





\section{Other chapters}

\begin{multicols}{2}
\begin{enumerate}
\item \hyperref[introduction-section-phantom]{Introduction}
\item \hyperref[conventions-section-phantom]{Conventions}
\item \hyperref[sets-section-phantom]{Set Theory}
\item \hyperref[categories-section-phantom]{Categories}
\item \hyperref[topology-section-phantom]{Topology}
\item \hyperref[sheaves-section-phantom]{Sheaves on Spaces}
\item \hyperref[algebra-section-phantom]{Commutative Algebra}
\item \hyperref[sites-section-phantom]{Sites and Sheaves}
\item \hyperref[homology-section-phantom]{Homological Algebra}
\item \hyperref[derived-section-phantom]{Derived Categories}
\item \hyperref[more-algebra-section-phantom]{More Algebra}
\item \hyperref[simplicial-section-phantom]{Simplicial Methods}
\item \hyperref[modules-section-phantom]{Sheaves of Modules}
\item \hyperref[sites-modules-section-phantom]{Modules on Sites}
\item \hyperref[injectives-section-phantom]{Injectives}
\item \hyperref[cohomology-section-phantom]{Cohomology of Sheaves}
\item \hyperref[sites-cohomology-section-phantom]{Cohomology on Sites}
\item \hyperref[hypercovering-section-phantom]{Hypercoverings}
\item \hyperref[schemes-section-phantom]{Schemes}
\item \hyperref[constructions-section-phantom]{Constructions of Schemes}
\item \hyperref[properties-section-phantom]{Properties of Schemes}
\item \hyperref[morphisms-section-phantom]{Morphisms of Schemes}
\item \hyperref[coherent-section-phantom]{Coherent Cohomology}
\item \hyperref[divisors-section-phantom]{Divisors}
\item \hyperref[limits-section-phantom]{Limits of Schemes}
\item \hyperref[varieties-section-phantom]{Varieties}
\item \hyperref[chow-section-phantom]{Chow Homology}
\item \hyperref[topologies-section-phantom]{Topologies on Schemes}
\item \hyperref[descent-section-phantom]{Descent}
\item \hyperref[more-morphisms-section-phantom]{More on Morphisms}
\item \hyperref[flat-section-phantom]{More on Flatness}
\item \hyperref[groupoids-section-phantom]{Groupoid Schemes}
\item \hyperref[more-groupoids-section-phantom]{More on Groupoid Schemes}
\item \hyperref[etale-section-phantom]{\'Etale Morphisms of Schemes}
\item \hyperref[etale-cohomology-section-phantom]{\'Etale Cohomology}
\item \hyperref[spaces-section-phantom]{Algebraic Spaces}
\item \hyperref[spaces-properties-section-phantom]{Properties of Algebraic Spaces}
\item \hyperref[spaces-morphisms-section-phantom]{Morphisms of Algebraic Spaces}
\item \hyperref[spaces-topologies-section-phantom]{Topologies on Algebraic Spaces}
\item \hyperref[spaces-descent-section-phantom]{Descent and Algebraic Spaces}
\item \hyperref[spaces-more-morphisms-section-phantom]{More on Morphisms of Spaces}
\item \hyperref[quot-section-phantom]{Quot and Hilbert Spaces}
\item \hyperref[stacks-section-phantom]{Stacks}
\item \hyperref[spaces-groupoids-section-phantom]{Groupoids in Algebraic Spaces}
\item \hyperref[spaces-more-groupoids-section-phantom]{More on Groupoids in Spaces}
\item \hyperref[bootstrap-section-phantom]{Bootstrap}
\item \hyperref[examples-stacks-section-phantom]{Examples of Stacks}
\item \hyperref[groupoids-quotients-section-phantom]{Quotients of Groupoids}
\item \hyperref[algebraic-section-phantom]{Algebraic Stacks}
\item \hyperref[criteria-section-phantom]{Criteria for Representability}
\item \hyperref[stacks-properties-section-phantom]{Properties of Algebraic Stacks}
\item \hyperref[stacks-morphisms-section-phantom]{Morphisms of Algebraic Stacks}
\item \hyperref[examples-section-phantom]{Examples}
\item \hyperref[exercises-section-phantom]{Exercises}
\item \hyperref[guide-section-phantom]{Guide to Literature}
\item \hyperref[desirables-section-phantom]{Desirables}
\item \hyperref[coding-section-phantom]{Coding Style}
\item \hyperref[fdl-section-phantom]{GNU Free Documentation License}
\item \hyperref[index-section-phantom]{Auto Generated Index}
\end{enumerate}
\end{multicols}



\bibliography{my}
\bibliographystyle{amsalpha}

\end{document}
