\IfFileExists{stacks-project.cls}{%
\documentclass{stacks-project}
}{%
\documentclass{amsart}
}

% The following AMS packages are automatically loaded with
% the amsart documentclass:
%\usepackage{amsmath}
%\usepackage{amssymb}
%\usepackage{amsthm}

% For dealing with references we use the comment environment
\usepackage{verbatim}
\newenvironment{reference}{\comment}{\endcomment}
%\newenvironment{reference}{}{}
\newenvironment{slogan}{\comment}{\endcomment}
\newenvironment{history}{\comment}{\endcomment}

% For commutative diagrams you can use
% \usepackage{amscd}
\usepackage[all]{xy}

% We use 2cell for 2-commutative diagrams.
\xyoption{2cell}
\UseAllTwocells

% To put source file link in headers.
% Change "template.tex" to "this_filename.tex"
% \usepackage{fancyhdr}
% \pagestyle{fancy}
% \lhead{}
% \chead{}
% \rhead{Source file: \url{template.tex}}
% \lfoot{}
% \cfoot{\thepage}
% \rfoot{}
% \renewcommand{\headrulewidth}{0pt}
% \renewcommand{\footrulewidth}{0pt}
% \renewcommand{\headheight}{12pt}

\usepackage{multicol}

% For cross-file-references
\usepackage{xr-hyper}

% Package for hypertext links:
\usepackage{hyperref}

% For any local file, say "hello.tex" you want to link to please
% use \externaldocument[hello-]{hello}
\externaldocument[introduction-]{introduction}
\externaldocument[conventions-]{conventions}
\externaldocument[sets-]{sets}
\externaldocument[categories-]{categories}
\externaldocument[topology-]{topology}
\externaldocument[sheaves-]{sheaves}
\externaldocument[sites-]{sites}
\externaldocument[stacks-]{stacks}
\externaldocument[fields-]{fields}
\externaldocument[algebra-]{algebra}
\externaldocument[brauer-]{brauer}
\externaldocument[homology-]{homology}
\externaldocument[derived-]{derived}
\externaldocument[simplicial-]{simplicial}
\externaldocument[more-algebra-]{more-algebra}
\externaldocument[smoothing-]{smoothing}
\externaldocument[modules-]{modules}
\externaldocument[sites-modules-]{sites-modules}
\externaldocument[injectives-]{injectives}
\externaldocument[cohomology-]{cohomology}
\externaldocument[sites-cohomology-]{sites-cohomology}
\externaldocument[dga-]{dga}
\externaldocument[dpa-]{dpa}
\externaldocument[hypercovering-]{hypercovering}
\externaldocument[schemes-]{schemes}
\externaldocument[constructions-]{constructions}
\externaldocument[properties-]{properties}
\externaldocument[morphisms-]{morphisms}
\externaldocument[coherent-]{coherent}
\externaldocument[divisors-]{divisors}
\externaldocument[limits-]{limits}
\externaldocument[varieties-]{varieties}
\externaldocument[topologies-]{topologies}
\externaldocument[descent-]{descent}
\externaldocument[perfect-]{perfect}
\externaldocument[more-morphisms-]{more-morphisms}
\externaldocument[flat-]{flat}
\externaldocument[groupoids-]{groupoids}
\externaldocument[more-groupoids-]{more-groupoids}
\externaldocument[etale-]{etale}
\externaldocument[chow-]{chow}
\externaldocument[intersection-]{intersection}
\externaldocument[pic-]{pic}
\externaldocument[adequate-]{adequate}
\externaldocument[dualizing-]{dualizing}
\externaldocument[duality-]{duality}
\externaldocument[discriminant-]{discriminant}
\externaldocument[local-cohomology-]{local-cohomology}
\externaldocument[curves-]{curves}
\externaldocument[resolve-]{resolve}
\externaldocument[models-]{models}
\externaldocument[pione-]{pione}
\externaldocument[etale-cohomology-]{etale-cohomology}
\externaldocument[proetale-]{proetale}
\externaldocument[crystalline-]{crystalline}
\externaldocument[spaces-]{spaces}
\externaldocument[spaces-properties-]{spaces-properties}
\externaldocument[spaces-morphisms-]{spaces-morphisms}
\externaldocument[decent-spaces-]{decent-spaces}
\externaldocument[spaces-cohomology-]{spaces-cohomology}
\externaldocument[spaces-limits-]{spaces-limits}
\externaldocument[spaces-divisors-]{spaces-divisors}
\externaldocument[spaces-over-fields-]{spaces-over-fields}
\externaldocument[spaces-topologies-]{spaces-topologies}
\externaldocument[spaces-descent-]{spaces-descent}
\externaldocument[spaces-perfect-]{spaces-perfect}
\externaldocument[spaces-more-morphisms-]{spaces-more-morphisms}
\externaldocument[spaces-flat-]{spaces-flat}
\externaldocument[spaces-groupoids-]{spaces-groupoids}
\externaldocument[spaces-more-groupoids-]{spaces-more-groupoids}
\externaldocument[bootstrap-]{bootstrap}
\externaldocument[spaces-pushouts-]{spaces-pushouts}
\externaldocument[groupoids-quotients-]{groupoids-quotients}
\externaldocument[spaces-more-cohomology-]{spaces-more-cohomology}
\externaldocument[spaces-simplicial-]{spaces-simplicial}
\externaldocument[formal-spaces-]{formal-spaces}
\externaldocument[restricted-]{restricted}
\externaldocument[spaces-resolve-]{spaces-resolve}
\externaldocument[formal-defos-]{formal-defos}
\externaldocument[defos-]{defos}
\externaldocument[cotangent-]{cotangent}
\externaldocument[examples-defos-]{examples-defos}
\externaldocument[algebraic-]{algebraic}
\externaldocument[examples-stacks-]{examples-stacks}
\externaldocument[stacks-sheaves-]{stacks-sheaves}
\externaldocument[criteria-]{criteria}
\externaldocument[artin-]{artin}
\externaldocument[quot-]{quot}
\externaldocument[stacks-properties-]{stacks-properties}
\externaldocument[stacks-morphisms-]{stacks-morphisms}
\externaldocument[stacks-limits-]{stacks-limits}
\externaldocument[stacks-cohomology-]{stacks-cohomology}
\externaldocument[stacks-perfect-]{stacks-perfect}
\externaldocument[stacks-introduction-]{stacks-introduction}
\externaldocument[stacks-more-morphisms-]{stacks-more-morphisms}
\externaldocument[stacks-geometry-]{stacks-geometry}
\externaldocument[moduli-]{moduli}
\externaldocument[moduli-curves-]{moduli-curves}
\externaldocument[examples-]{examples}
\externaldocument[exercises-]{exercises}
\externaldocument[guide-]{guide}
\externaldocument[desirables-]{desirables}
\externaldocument[coding-]{coding}
\externaldocument[obsolete-]{obsolete}
\externaldocument[fdl-]{fdl}
\externaldocument[index-]{index}

% Theorem environments.
%
\theoremstyle{plain}
\newtheorem{theorem}[subsection]{Theorem}
\newtheorem{proposition}[subsection]{Proposition}
\newtheorem{lemma}[subsection]{Lemma}

\theoremstyle{definition}
\newtheorem{definition}[subsection]{Definition}
\newtheorem{example}[subsection]{Example}
\newtheorem{exercise}[subsection]{Exercise}
\newtheorem{situation}[subsection]{Situation}

\theoremstyle{remark}
\newtheorem{remark}[subsection]{Remark}
\newtheorem{remarks}[subsection]{Remarks}

\numberwithin{equation}{subsection}

% Macros
%
\def\lim{\mathop{\rm lim}\nolimits}
\def\colim{\mathop{\rm colim}\nolimits}
\def\Spec{\mathop{\rm Spec}}
\def\Hom{\mathop{\rm Hom}\nolimits}
\def\Ext{\mathop{\rm Ext}\nolimits}
\def\SheafHom{\mathop{\mathcal{H}\!{\it om}}\nolimits}
\def\SheafExt{\mathop{\mathcal{E}\!{\it xt}}\nolimits}
\def\Sch{\textit{Sch}}
\def\Mor{\mathop{\rm Mor}\nolimits}
\def\Ob{\mathop{\rm Ob}\nolimits}
\def\Sh{\mathop{\textit{Sh}}\nolimits}
\def\NL{\mathop{N\!L}\nolimits}
\def\proetale{{pro\text{-}\acute{e}tale}}
\def\etale{{\acute{e}tale}}
\def\QCoh{\textit{QCoh}}
\def\Ker{\mathop{\rm Ker}}
\def\Im{\mathop{\rm Im}}
\def\Coker{\mathop{\rm Coker}}
\def\Coim{\mathop{\rm Coim}}

%
% Macros for moduli stacks/spaces
%
\def\QCohstack{\mathcal{QC}\!{\it oh}}
\def\Cohstack{\mathcal{C}\!{\it oh}}
\def\Spacesstack{\mathcal{S}\!{\it paces}}
\def\Quotfunctor{{\rm Quot}}
\def\Hilbfunctor{{\rm Hilb}}
\def\Curvesstack{\mathcal{C}\!{\it urves}}
\def\Polarizedstack{\mathcal{P}\!{\it olarized}}
\def\Complexesstack{\mathcal{C}\!{\it omplexes}}
% \Pic is the operator that assigns to X its picard group, usage \Pic(X)
% \Picardstack_{X/B} denotes the Picard stack of X over B
% \Picardfunctor_{X/B} denotes the Picard functor of X over B
\def\Pic{\mathop{\rm Pic}\nolimits}
\def\Picardstack{\mathcal{P}\!{\it ic}}
\def\Picardfunctor{{\rm Pic}}
\def\Deformationcategory{\mathcal{D}\!{\it ef}}


% OK, start here.
%
\begin{document}

\title{Intersection Theory}

\begin{abstract}
Incomplete version of document! Reader beware!
Goal: Get a reasonably complete setup of
cycles class group module rational equivalence
over a universally catenary base endowed with a dimension
function. Moreover, define capping with chern classes of
vector bundles. Finally define intersection products using
the Tor formula, and a moving lemma (if such exists).
No attempt is made to keep the exposition short.
\end{abstract}

\maketitle

\phantomsection
\label{section-phantom}


\tableofcontents



\section{Algebraic lemmas}
\label{section-algebraic-lemmas}

\noindent
In this section we collect algebraic lemmas used below in order to
slot them into the chapter on commutative algebra later on.

\begin{lemma}
\label{lemma-pushdown-module}
Let $A$ be a local Noetherian ring with maximal ideal $\mathfrak m$.
Let $B$ be a semi-local Noetherian ring with maximal ideals $\mathfrak m_i$,
$i = 1, \ldots, n$.
Suppose that $A \to B$ is a homomorphism such that each $\mathfrak m_i$
lies over $\mathfrak m$ and such that
$$
[\kappa(\mathfrak m_i) : \kappa(\mathfrak m)] < \infty.
$$
Let $M$ be a $B$-module of finite length.
Then
$$
\text{length}_A(M) = \sum\nolimits_{i = 1, \ldots, n}
[\kappa(\mathfrak m_i) : \kappa(\mathfrak m)]
\text{length}_{B_{\mathfrak m_i}}(M_{\mathfrak m_i}).
$$
\end{lemma}

\begin{proof}
Omitted.
\end{proof}

\begin{lemma}
\label{lemma-pullback-module}
Let $A \to B$ be a flat local homomorphism of local Noetherian rings.
Assume that $\sqrt{\mathfrak m_A B} = \mathfrak m_B$.
Then for any $A$-module $M$ of finite length we have
$$
\text{length}_A(M) \text{length}_B(B/\mathfrak m_AB)
=
\text{length}_B(M \otimes_A B)
$$
\end{lemma}

\begin{proof}
Omitted.
\end{proof}

\begin{lemma}
\label{lemma-pullback-transitive}
Let $A \to B \to C$ be flat local homomorphisms of local Noetherian rings.
Let $M$ be an $A$-module of finite length.
Then
$$
\text{length}_B(M \otimes_A B) \text{length}_C(C/\mathfrak m_B C)
=
\text{length}_C(M \otimes_A C)
$$
\end{lemma}

\begin{proof}
Omitted.
\end{proof}

\begin{lemma}
\label{lemma-flat-pullback-pushdown-module}
Let
$$
\xymatrix{
A' = B' \otimes_B A & A \ar[l] \\
B' \ar[u] & B \ar[l] \ar[u]
}
$$
be a diagram of Noetherian rings. Assume that $A$, $A'$, $B'$ are local,
the homomorphisms $B \to A$ and $B \to B'$ are local,
$\kappa(\mathfrak m_B) \to \kappa(\mathfrak m_A)$ is finite,
$B \to B'$ is flat and $\sqrt{\mathfrak m_A B'} = \mathfrak m_{B'}$.
Then
\begin{enumerate}
\item Any prime $\mathfrak m'$ of $A'$ lying over $\mathfrak m_A$
is maximal and $[\kappa(\mathfrak m') : \kappa(\mathfrak m_{B'})] < \infty$.
Moreover, there are only a finite number, say
$\mathfrak m'_1, \ldots, \mathfrak m'_n$, of these.
\item We have
\begin{align*}
[\kappa(\mathfrak m_A) : \kappa(\mathfrak m_B)]
&
\text{length}_{B'}(B'/\mathfrak m_BB')
= \\
& \sum\nolimits_{i = 1, \ldots, n}
[\kappa(\mathfrak m'_i) : \kappa(\mathfrak m_{B'})]
\text{length}_{A'_{\mathfrak m'_i}}((A'/\mathfrak m_AA')_{\mathfrak m'_i})
\end{align*}
\end{enumerate}
\end{lemma}

\begin{proof}
Omitted.
\end{proof}

\begin{lemma}
\label{lemma-length-additive}
Let $A$ be a semi-local Noetherian ring of dimension $1$.
If $a, b \in A$ are not zero divisors then
$$
\text{length}_A(A/(ab)) =
\text{length}_A(A/(a)) +
\text{length}_A(A/(b))
$$
and these lengths are finite.
\end{lemma}

\begin{proof}
Omitted.
\end{proof}

\begin{definition}
\label{definition-ord}
Suppose that $K$ is a field, and $A \subset K$ is a
local\footnote{We could also define this when $A$ is only
semi-local but this is probably never really what you want!}
Noetherian subring of dimension $1$ with fraction field $K$.
In this case we define the {\it order of vanishing along $A$}
$$
\text{ord}_A : K^* \longrightarrow \mathbf{Z}
$$
by the rule
$$
\text{ord}_A(x) = \text{length}_A(A/(x))
$$
if $x \in A$ and we set
$\text{ord}_A(x/y) = \text{ord}_A(x) - \text{ord}_A(y)$
for $x, y \in A$ both nonzero.
\end{definition}

\begin{lemma}
\label{lemma-finite-extension-dim-1}
Let $A \subset B$ be a finite extension of Noetherian domains.
Assume that $A$ is a local with maximal ideal $\mathfrak m$.
Let $K = f.f.(A)$ and $L = B \otimes_A K$ so that $L$ is
a finite field extension of $K$.
Let $y \in L^*$ and $x = \text{Nm}_{L/K}(y)$.
Let $\mathfrak m_i$, $i = 1, \ldots, n$ be the maximal ideals of $B$.
Then
$$
\text{ord}_A(x) =
\sum\nolimits_i
[\kappa(\mathfrak m_i) : \kappa(\mathfrak m)]
\text{ord}_{B_{\mathfrak m_i}}(y)
$$
where $\text{ord}$ is defined as in Definition \ref{definition-ord}.
\end{lemma}

\begin{proof}
Omitted.
\end{proof}









\section{Setup}
\label{section-setup}

\noindent
We will throughout work over a locally Noetherian universally
catenary base $S$ endowed with a dimension function $\delta$.
Allthough it is likely possible to generalize (parts of) the
discussion in the chapter, it seems that this is a good first
approximation. We usually do not assume our schemes are
separated or quasi-compact. Many interesting algebraic stacks
are non-separated and/or non-quasi-compact and this is a good
case study to see how to develop a reasonable theory for those as well.
In order to reference these hypotheses we give it a number.

\begin{situation}
\label{situation-setup}
Here $S$ is a locally Noetherian, and universally catenary scheme.
Moreover, we assume $S$ is endowed with a dimension function
$\delta : S \longrightarrow \mathbf{Z}$.
\end{situation}

\noindent
See Morphisms, Definition \ref{morphisms-definition-universally-catenary}
for the notion of a universally catenary scheme, and see
Topology, Definition \ref{topology-definition-dimension-function}
for the notion of a dimension function. Recall that any locally
Noetherian catenary scheme locally has a dimension function, see
Properties, Lemma \ref{properties-lemma-catenary-dimension-function}.
Moreover, there are lots of schemes which are universally catenary,
see Morphisms, Lemma \ref{morphisms-lemma-ubiquity-uc}.

\medskip\noindent
Let $(S, \delta)$ be as in Situation \ref{situation-setup}.
Any scheme $X$ locally of finite type over $S$ is locally Noetherian
and catenary. In fact, $X$ has a canonical dimension function
$$
\delta = \delta_{X/S} : X \longrightarrow \mathbf{Z}
$$
associated to $(f : X \to S, \delta)$ given by the rule
$\delta_{X/S}(x) = \delta(f(x)) + \text{trdeg}_{\kappa(f(x))}\kappa(x)$.
See Morphisms, Lemma \ref{morphisms-lemma-dimension-function-propagates}.
Moreover, if $h : X \to Y$ is a morphism of schemes locally of finite
type over $S$, and $x \in X$, $y = h(x)$,
then obviously
$\delta_{X/S}(x) = \delta_{Y/S}(y) + \text{trdeg}_{\kappa(y)}\kappa(x)$.
We will freely use this function and its properties in the following.

\medskip\noindent
Here are the basic examples of setups as above.
In fact, the main interest lies in the case where the base
is the spectrum of a field, or the case where the base
is the spectrum of a Dedekind ring (e.g.\ $\mathbf{Z}$,
or a discrete valuation ring).

\begin{example}
\label{example-field}
Here $S = \text{Spec}(k)$ and $k$ is a field.
We set $\delta(pt) = 0$ where $pt$ indicates the unique point of $S$.
The pair $(S, \delta)$ is an example of a situation as in
Situation \ref{situation-setup} by
Morphisms, Lemma \ref{morphisms-lemma-ubiquity-uc}.
\end{example}

\begin{example}
\label{example-domain-dimension-1}
Here $S = \text{Spec}(A)$, where $A$ is a Noetherian domain
of dimension $1$.
For example we could consider $A = \mathbf{Z}$.
We set $\delta(\mathfrak p) = 0$ if
$\mathfrak p$ is a maximal ideal and $\delta(\mathfrak p) = 1$
if $\mathfrak p = (0)$ corresponds to the generic point.
Situation \ref{situation-setup} by
Morphisms, Lemma \ref{morphisms-lemma-ubiquity-uc}.
\end{example}

\noindent
In good cases $\delta$ corresponds to the dimension function.

\begin{lemma}
\label{lemma-delta-is-dimension}
Let $(S, \delta)$ be as in Situation \ref{situation-setup}.
Assume in addition $S$ is a Jacobson scheme, and $\delta(s) = 0$ for every
closed point $s$ of $S$. Let $X$ be locally of finite type over $S$.
Let $Z \subset X$ be an irreducible closed subset and let
$\xi \in Z$ be its generic point.
Then $\delta_{X/S}(\xi) = \dim(Z)$.
\end{lemma}

\begin{proof}
Let $X \to S$, $\xi \in Z \subset X$ be as in the lemma.
Since $X$ is locally of finite type over $S$ we see that
$X$ is Jacobson, see
Morphisms, Lemma \ref{morphisms-lemma-Jacobson-universally-Jacobson}.
Hence closed points of $X$ are dense in every closed subset of $Z$
and map to closed points of $S$. Hence every maximal chain
of irreducible closed subsets of $Z$ ends with a closed point of $Z$.
It follows that $\dim(Z) = \text{sup}_z(\delta(\xi) - \delta(z))$
where $z \in Z$ runs over the closed points of $Z$, by the properties
of a dimension function. For each closed $z \in Z$ the field extension
$\kappa(z) \supset \kappa(f(z))$ is finite, see Morphisms,
Lemma \ref{morphisms-lemma-jacobson-finite-type-points}.
Hence $\delta_{X/S}(z) = \delta(f(z)) = 0$ for $z \in Z$ closed.
It follows that $\dim(Z) = \delta(x)$ as desired.
\end{proof}

\noindent
In the situation of the lemma above the
value of $\delta$ at the generic point of a closed irreducible subset
is the dimension of the irreducible closed subset.
However, in general we cannot expect the equality to hold.
For example if $S = \text{Spec}(\mathbf{C}[[t]])$ and
$X = \text{Spec}(\mathbf{C}((t)))$ then we would get
$\delta(x) = 1$ for the unique point of $X$, but $\dim(X) = 0$.
Still we want to think of $\delta_{X/S}$ as giving the
dimension of the irreducible closed subschemes. Thus we introduce
the following terminology.

\begin{definition}
\label{definition-delta-dimension}
Let $(S, \delta)$ as in Sitation \ref{situation-setup}.
For any scheme $X$ locally of finite type over $S$
and any irreducible closed subset $Z \subset X$ we define
$$
\dim_\delta(Z) = \delta(\xi)
$$
where $\xi \in Z$ is the generic point of $Z$.
We will call this the {\it $\delta$-dimension of $Z$}.
If $Z$ is a closed subscheme of $X$, then we define
$\dim_\delta(Z)$ as the supremum of the $\delta$-dimensions
of its irreducible components.
\end{definition}







\section{Cycles}
\label{section-cycles}

\noindent
Since we are not assuming our schemes are quasi-compact we have
to be a little careful when defining cycles. We have to allow
infinite sums because a rational function may have infinitely many
poles for example. In any case, if $X$ is quasi-compact then a
cycle is a finite sum as usual.

\begin{definition}
\label{definition-cycles}
Let $(S, \delta)$ be as in Sitation \ref{situation-setup}.
Let $X$ be locally of finite type over $S$.
Let $k \in \mathbf{Z}$.
\begin{enumerate}
\item A collection of closed subschemes $\{Z_i\}_{i \in I}$ of $X$
is said to be {\it locally finite} if for every quasi-compact open
$U \subset X$ the set
$$
\# \{i\in I \mid Z_i \cap U \not = \emptyset\}
$$
is finite.
\item A {\it cycle on $X$} is a formal sum
$$
\alpha = \sum n_Z [Z]
$$
where the sum is over integral closed subschemes $Z \subset X$,
each $n_Z \in \mathbf{Z}$, and the collection
$\{Z; n_Z \not = 0\}$ is locally finite.
\item A {\it $k$-cycle}, on $X$ is
a cycle
$$
\alpha = \sum n_Z [Z]
$$
where $n_Z \not = 0 \Rightarrow \dim_\delta(Z) = k$.
\item The abelian group of all $k$-cycles on $X$ is denoted $Z_k(X)$.
\end{enumerate}
\end{definition}

\noindent
In other words, a $k$-cycle on $X$
is a locally finite formal $\mathbf{Z}$-linear
combination of integral closed subschemes of $\delta$-dimension $k$.
Addition of $k$-cycles $\alpha = \sum n_Z[Z]$ and 
$\beta = \sum m_Z[Z]$ is given by
$$
\alpha + \beta = \sum (n_Z + m_Z)[Z],
$$
i.e., by adding the coefficients.




\section{Cycle associated to a closed subscheme}
\label{section-cycle-of-closed-subscheme}

\begin{lemma}
\label{lemma-multiplicity-finite}
Let $(S, \delta)$ be as in Sitation \ref{situation-setup}.
Let $X$ be locally of finite type over $S$.
Let $Z \subset X$ be a closed subscheme.
\begin{enumerate}
\item The collection of irreducible components of $Z$
is locally finite.
\item Let $Z' \subset Z$ be an irreducible component and
let $\xi \in Z'$ be its generic point.
Then
$$
\text{length}_{\mathcal{O}_{X, \xi}} \mathcal{O}_{Z, \xi} < \infty
$$
\item If $\dim_\delta(Z) \leq k$ and $\xi \in Z$ with
$\delta(\xi) = k$, then $\xi$ is a generic point of an
irreducible component of $Z$.
\end{enumerate}
\end{lemma}

\begin{proof}
Omitted.
\end{proof}

\begin{definition}
\label{definition-cycle-associated-to-closed-subscheme}
Let $(S, \delta)$ be as in Sitation \ref{situation-setup}.
Let $X$ be locally of finite type over $S$.
Let $Z \subset X$ be a closed subscheme.
\begin{enumerate}
\item For any irreducible component $Z' \subset Z$ with generic point $\xi$
the integer
$m_{Z', Z} = \text{length}_{\mathcal{O}_{X, \xi}} \mathcal{O}_{Z, \xi}$
(Lemma \ref{lemma-multiplicity-finite})
is called the {\it multiplicity of $Z'$ in $Z$}.
\item Assume $\dim_\delta(Z) \leq k$.
The {\it $k$-cycle associated to $Z$} is
$$
[Z]_k
=
\sum m_{Z', Z}[Z']
$$
where the sum is over the irreducible components of $Z$
of $\delta$-dimension $k$. (This is a $k$-cycle by
Lemma \ref{lemma-multiplicity-finite}.)
\end{enumerate}
\end{definition}

\noindent
It is important to note that we only define $[Z]_k$ if the $\delta$-dimension
of $Z$ does not exceed $k$. In other words, by convention, if we write
$[Z]_k$ then this implies that $\dim_\delta(Z) \leq k$.



\section{Cycle associated to a coherent sheaf}
\label{section-cycle-of-coherent-sheaf}



\begin{lemma}
\label{lemma-length-finite}
Let $(S, \delta)$ be as in Sitation \ref{situation-setup}.
Let $X$ be locally of finite type over $S$.
Let $\mathcal{F}$ be a coherent $\mathcal{O}_X$-module.
\begin{enumerate}
\item The collection of irreducible components of the support of
$\mathcal{F}$ is locally finite.
\item Let $Z' \subset \text{Supp}(\mathcal{F})$
be an irreducible component and
let $\xi \in Z'$ be its generic point.
Then
$$
\text{length}_{\mathcal{O}_{X, \xi}} \mathcal{F}_\xi < \infty
$$
\item If $\dim_\delta(\text{Supp}(\mathcal{F})) \leq k$
and $\xi \in Z$ with $\delta(\xi) = k$, then $\xi$ is a
generic point of an irreducible component of $\text{Supp}(\mathcal{F})$.
\end{enumerate}
\end{lemma}

\begin{proof}
Omitted.
\end{proof}

\begin{definition}
\label{definition-cycle-associated-to-coherent-sheaf}
Let $(S, \delta)$ be as in Sitation \ref{situation-setup}.
Let $X$ be locally of finite type over $S$.
Let $\mathcal{F}$ be a coherent $\mathcal{O}_X$-module.
\begin{enumerate}
\item For any irreducible component $Z' \subset \text{Supp}(\mathcal{F})$
with generic point $\xi$ the integer
$m_{Z', \mathcal{F}} = \text{length}_{\mathcal{O}_{X, \xi}} \mathcal{F}_\xi$
(Lemma \ref{lemma-length-finite})
is called the {\it multiplicity of $Z'$ in $\mathcal{F}$}.
\item Assume $\dim_\delta(\text{Supp}(\mathcal{F})) \leq k$.
The {\it $k$-cycle associated to $\mathcal{F}$} is
$$
[\mathcal{F}]_k
=
\sum m_{Z', \mathcal{F}}[Z']
$$
where the sum is over the irreducible components of
$\text{Supp}(\mathcal{F})$ of $\delta$-dimension $k$.
(This is a $k$-cycle by Lemma \ref{lemma-length-finite}.)
\end{enumerate}
\end{definition}

\noindent
It is important to note that we only define $[\mathcal{F}]_k$
if $\mathcal{F}$ is coherent and the $\delta$-dimension
of $\text{Supp}(\mathcal{F})$ does not exceed $k$. In other words,
by convention, if we write $[\mathcal{F}]_k$ then this implies that
$\mathcal{F}$ is coherent on $X$ and
$\dim_\delta(\text{Supp}(\mathcal{F})) \leq k$.

\begin{lemma}
\label{lemma-cycle-closed-coherent}
Let $(S, \delta)$ be as in Sitation \ref{situation-setup}.
Let $X$ be locally of finite type over $S$.
Let $Z \subset X$ be a closed subscheme.
If $\dim_\delta(Z) \leq k$, then $[Z]_k = [{\mathcal O}_Z]_k$.
\end{lemma}

\begin{proof}
Omitted.
\end{proof}





\section{Degrees of maps between integral schemes}
\label{section-degrees}

\noindent
Here is a lemma characterizing maps between integral schemes
which are locally of finite type and which are ``generically finite''.

\begin{lemma}
\label{lemma-finite-degree}
Let $X$, $Y$ be integral schemes.
Let $f : X \to Y$ be locally of finite type.
Assume $f$ is dominant.
The following are equivalent:
\begin{enumerate}
\item the extension $R(Y) \subset R(X)$ has
transcendence degree $0$,
\item the extension $R(Y) \subset R(X)$ is finite,
\item there exists nonempty affine opens $U \subset X$
and $V \subset Y$ such that $f(U) \subset V$
and $f|_U : U \to V$ is finite, and
\item the generic point of $X$ is the only point of $X$ mapping to
the generic point of $Y$.
\end{enumerate}
If $f$ is separated, or if $f$ is quasi-compact, then these are
also equivalent to
\begin{enumerate}
\item[(5)] there exists a nonempty affine open $V \subset Y$ such
that $f^{-1}(V) \to V$ is finite.
\end{enumerate}
\end{lemma}

\begin{proof}
Choose any affine opens $\text{Spec}(A) = U \subset X$
and $\text{Spec}(R) = V \subset Y$ such that $f(U) \subset V$.
Then $R$ and $A$ are domains by definition. The ring map
$R \to A$ is of finite type
(Morphisms, Lemma \ref{morphisms-lemma-locally-finite-type-characterize}).
Let $K = f.f.(R) = R(Y)$ and $L = f.f.(A) = R(X)$. Then $K \subset L$
is a finitely generated field extension. Hence we see that
(1) is equivalent to (2).

\medskip\noindent
Suppose (2) holds. Let $x_1, \ldots, x_n \in A$ be generators
of $A$ over $R$. By assumption there exist nonzero polynomials
$P_i(X) \in R[X]$ such that $P_i(x_i) = 0$. Let $f_i \in R$ be the
leading coefficient of $P_i$. Then we conclude that
$R_{f_1 \ldots f_n} \to A_{f_1 \ldots f_n}$ is finite, i.e., (3) holds.
Note that (3) implies (2). So now we see that (1), (2) and (3) are all
equivalent.

\medskip\noindent
Let $\eta$ be the generic point of $X$, and let $\eta' \in Y$ be the
generic point of $Y$. Assume (4). Then
$\dim_\eta(X_{\eta'}) = 0$ and we see that $R(X) = \kappa(\eta)$ has
transcendence degree $0$ over $R(Y) = \kappa(\eta')$ by
Morphisms, Lemma \ref{morphisms-lemma-dimension-fibre-at-a-point}.
In other words (1) holds. Assume the equivalent conditions (1), (2) and
(3). Suppose that $x \in X$ is a point mapping to $\eta'$.
As $x$ is a specialization of $\eta$,
this gives inclusions $R(Y) \subset \mathcal{O}_{X, x} \subset R(X)$,
which implies $\mathcal{O}_{X, x}$ is a field, see
Algebra, Lemma \ref{algebra-lemma-integral-over-field}.
Hence $x = \eta$. Thus we see that (1) -- (3)
are all equivalent.

\medskip\noindent
Assume $U, V$ as in (3) and assume $f$ is separated.
Then $U \to f^{-1}(V)$ is a morphism from a scheme proper over $V$
(Morphisms, Lemma \ref{morphisms-lemma-finite-proper})
into a scheme separated over $V$.
Hence $U \subset f^{-1}(V)$ is closed
(Morphisms, Lemma \ref{morphisms-lemma-image-proper-scheme-closed}).
Since $X$ is irreducible we conclude $U = f^{-1}(V)$. This
proves (5).

\medskip\noindent
Assume $f$ is quasi-compact. Let $U, V$ be as in (3).
Write $f^{-1}(V) = U_1 \cup \ldots \cup U_n$ as a finite union
of affines. Consider the proper closed set $Z_i = U_i \setminus U$.
Since $Z_i$ does not contain the generic point of $X$ we see that
$f : Z_i \to Y$ is not dominant (use property (4) and
Morphisms, Lemma \ref{morphisms-lemma-quasi-compact-dominant}).
Hence after shrinking $V$ we may assume that $U_i \subset U$
which implies that (5) holds.
\end{proof}

\begin{definition}
\label{definition-degree}
Let $X$ and $Y$ be integral schemes.
Let $f : X \to Y$ be locally of finite type and dominant.
Assume $[R(X) : R(Y)] < \infty$, or any other of the equivalent
conditions (1) -- (4) of Lemma \ref{lemma-finite-degree}.
Then the positive integer
$$
\text{deg}(X/Y) = [R(X) : R(Y)]
$$
is called the {\it degree of $X$ over $Y$}.
\end{definition}

\begin{lemma}
\label{lemma-degree-composition}
Let $X$, $Y$, $Z$ be integral schemes.
Let $f : X \to Y$ and $g : Y \to Z$ be dominant morphisms locally
of finite type. Assume that $[R(X) : R(Y)] < \infty$ and
$[R(Y) : R(Z)] < \infty$. Then
$$
\deg(X/Z) = \deg(X/Y) \deg(Y/Z).
$$
\end{lemma}

\begin{proof}
This comes from the multiplicativity of the degree in towers
of finite field extensions.
\end{proof}




\section{Preparation for proper pushforward}
\label{section-preparation-pushforward}

\begin{lemma}
\label{lemma-equal-dimension}
Let $(S, \delta)$ be as in Sitation \ref{situation-setup}.
Let $X$, $Y$ be locally of finite type over $S$.
Let $f : X \to Y$ be a morphism.
Assume $X$, $Y$ integral and $\dim_\delta(X) = \dim_\delta(Y)$.
Then either $f(X)$ is contained in a proper closed subscheme
of $Y$, or $f$ is dominant and the extension of function fields
$R(Y) \subset R(X)$ is finite.
\end{lemma}

\begin{proof}
The closure $\overline{f(X)} \subset Y$ is irreducible as $X$
is irreducible. If $\overline{f(X)} \not = Y$, then we are done.
If $\overline{f(X)} = Y$, then $f$ is dominant and by
Morphisms,
Lemma \ref{morphisms-lemma-dominant-finite-number-irreducible-components}
we see that the generic point $\eta_Y$ of $Y$ is in the image of $f$.
Of course this implies that $f(\eta_X) = \eta_Y$, where $\eta_X \in X$
is the generic point of $X$. Since $\delta(\eta_X) = \delta(\eta_Y)$
we see that $R(Y) = \kappa(\eta_Y) \subset \kappa(\eta_X) = R(X)$
is an extension of transcendence degree $0$.
Hence Lemma \ref{lemma-finite-degree} applies.
\end{proof}

\begin{lemma}
\label{lemma-quasi-compact-locally-finite}
Let $(S, \delta)$ be as in Sitation \ref{situation-setup}.
Let $X$, $Y$ be locally of finite type over $S$.
Let $f : X \to Y$ be a morphism.
Assume $f$ is quasi-compact, and $\{Z_i\}_{i \in I}$ is a locally
finite collection of closed subsets of $X$.
Then $\{\overline{f(Z_i)}\}_{i \in I}$ is a locally finite
collection of closed subsets of $Y$.
\end{lemma}

\begin{proof}
Omitted.
\end{proof}









\section{Proper pushforward}
\label{section-proper-pushforward}

\begin{definition}
\label{definition-proper-pushforward}
Let $(S, \delta)$ be as in Sitation \ref{situation-setup}.
Let $X$, $Y$ be locally of finite type over $S$.
Let $f : X \to Y$ be a morphism.
Assume $f$ is proper.
\begin{enumerate}
\item Let $Z \subset X$ be an integral closed subscheme
with $\dim_\delta(Z) = k$. We define
$$
f_*[Z] =
\left\{
\begin{matrix}
0 & \text{if} & \dim_\delta(f(Z))< k, \\
\deg(Z/f(Z)) [f(Z)] & \text{if} & \dim_\delta(f(Z)) = k.
\end{matrix}
\right.
$$
Here we think of $f(Z) \subset Y$ as an integral closed subscheme.
The degree of $Z$ over $f(Z)$ is finite if
$\dim_\delta(f(Z)) = \dim_\delta(Z)$
by Lemma \ref{lemma-equal-dimension}.
\item Let $\alpha = \sum n_Z [Z]$ be a $k$-cycle on $X$. The
{\it pushforward} of $\alpha$ as the sum
$$
f_* \alpha = \sum n_Z f_*[Z]
$$
where each $f_*[Z]$ is defined as above. The sum is locally finite
by Lemma \ref{lemma-quasi-compact-locally-finite} above.
\end{enumerate}
\end{definition}

\noindent
By definition the proper pushforward of cycles
$$
f_* : Z_k(X) \longrightarrow Z_k(Y)
$$
is a homomorphism of abelian groups. It turns $X \mapsto Z_k(X)$
into a contravariant functor on the category of schemes locally of
finite type over $S$ with morphisms equal to proper morphisms.

\begin{lemma}
\label{lemma-compose-pushforward}
Let $(S, \delta)$ be as in Sitation \ref{situation-setup}.
Let $X$, $Y$, and $Z$ be locally of finite type over $S$.
Let $f : X \to Y$ and $g : Y \to Z$ be proper morphisms.
Then $g_* \circ f_* = (g \circ f)_*$ as maps $Z_k(X) \to Z_k(Z)$.
\end{lemma}

\begin{proof}
Omitted, but see Lemma \ref{lemma-degree-composition}.
\end{proof}

\begin{lemma}
\label{lemma-cycle-push-sheaf}
Let $(S, \delta)$ be as in Sitation \ref{situation-setup}.
Let $X$, $Y$ be locally of finite type over $S$.
Let $f : X \to Y$ be a morphism.
Assume $f$ is proper.
\begin{enumerate}
\item Let $Z \subset X$ be a closed subscheme with $\dim_\delta(Z) \leq k$.
Then
$$
f_*[Z]_k = [f_*{\mathcal O}_Z]_k.
$$
\item Let $\mathcal{F}$ be a coherent sheaf on $X$ such that
$\dim_\delta(\text{Supp}(\mathcal{F})) \leq k$. Then
$$
f_*[\mathcal{F}]_k = [f_*{\mathcal F}]_k.
$$
\end{enumerate}
Note that the statement makes sense since $f_*\mathcal{F}$ and
$f_*\mathcal{O}_Z$ are coherent $\mathcal{O}_Y$-modules by
Coherent, Lemma \ref{coherent-lemma-proper-pushforward-coherent}.
\end{lemma}

\begin{proof}
Omitted. This eventually comes down to the algebra
Lemma \ref{lemma-pushdown-module}.
\end{proof}



















\section{Morphisms of given relative dimension}
\label{section-relative-dimension}

\noindent
In order to be able to speak comfortably about flat morphisms of a
given relative dimension we introduce the following notion.

\begin{definition}
\label{definition-relative-dimension-d}
Let $f : X \to S$ be a morphism of schemes.
Assume $f$ is locally of finite type.
We say $f$ is of {\it relative dimension $d$} if
all nonempty fibres $X_s$ are equidimensional of dimension $d$.
\end{definition}

\noindent
This is not a particularly well behaved notion, but it works well
in a number of situations.

\begin{lemma}
\label{lemma-base-change-relative-dimension-d}
Let $f : X \to S$ be a locally finite type morphism.
If $f$ has relative dimension $d$, then so does any base
change of $f$.
\end{lemma}

\begin{proof}
This is immediate from
Morphisms, Lemma \ref{morphisms-lemma-dimension-fibre-after-base-change}.
\end{proof}

\begin{lemma}
\label{lemma-composition-relative-dimension-d}
Let $f : X \to Y$, $g : Y \to Z$ be locally of finite type.
Assume
\begin{enumerate}
\item $f$ has relative dimension $d$,
\item $g$ has relative dimension $e$, and
\item $f$ is flat.
\end{enumerate}
Then $g \circ f$ has relative dimension $d + e$.
\end{lemma}

\begin{proof}
This is immediate from
Morphisms, Lemma \ref{morphisms-lemma-dimension-fibre-at-a-point-additive}.
\end{proof}

\noindent
In general it is not possible to decompose a morphism
into its pieces where the relative dimension is a given
one. However, it is possible if the morphism has Cohen-Macaulay
fibres and is flat (plus of finite presentation).

\begin{lemma}
\label{lemma-flat-finite-presentation-CM-fibres-relative-dimension}
Let $f : X \to S$ be a morphism of schemes.
Assume that
\begin{enumerate}
\item $f$ is flat,
\item $f$ is locally of finite presentation, and
\item for every $s \in S$ the fibre $X_s$ is Cohen-Macaulay
(see Properties, Definition \ref{properties-definition-Cohen-Macaulay})
\end{enumerate}
Then there exist open and closed subschemes $X_d \subset X$
such that $X = \coprod_{d \geq 0} X_d$ and $f|_{X_d} : X_d \to S$
has relative dimension $d$.
\end{lemma}

\begin{proof}
This is immediate from
Algebra,
Lemma
\ref{algebra-lemma-relative-dimension-CM}.
\end{proof}







\section{Preparation for flat pullback}
\label{section-preparation-flat-pullback}


\begin{lemma}
\label{lemma-inverse-image-locally-finite}
Let $(S, \delta)$ be as in Sitation \ref{situation-setup}.
Let $X$, $Y$ be locally of finite type over $S$.
Let $f : X \to Y$ be a morphism.
Assume $\{Z_i\}_{i \in I}$ is a locally
finite collection of closed subsets of $Y$.
Then $\{f^{-1}(Z_i)\}_{i \in I}$ is a locally finite
collection of closed subsets of $Y$.
\end{lemma}

\begin{proof}
Omitted.
\end{proof}

\begin{lemma}
\label{lemma-flat-inverse-image-dimension}
Let $(S, \delta)$ be as in Sitation \ref{situation-setup}.
Let $X$, $Y$ be locally of finite type over $S$.
Let $f : X \to Y$ be a morphism.
Assume $f$ is flat of relative dimension $r$, see
Definition \ref{definition-relative-dimension-d}.
For any closed subset $Z \subset Y$ we have
$$
\dim_\delta(f^{-1}(Z)) = \dim_\delta(Z) + r.
$$
If $Z$ is irreducible and $Z' \subset f^{-1}(Z)$ is an irreducible
component, then $Z'$ dominates $Z$ and
$\dim_\delta(Z') = \dim_\delta(Z) + r$.
\end{lemma}

\begin{proof}
It suffices to prove the final statement.
We may replace $Y$ by the integral closed subscheme $Z$ and
$X$ by the scheme theoretic inverse image $f^{-1}(Z) = Z \times_Y X$.
Hence we may assume $Z = Y$ is integral and $f$ is a flat morphism
of relative dimension $r$. Since $Y$ is locally Noetherian the
morphism $f$ which is locally of finite type,
is actually locally of finite presentation. Hence
Morphisms, Lemma \ref{morphisms-lemma-fppf-open}
applies and we see that $f$ is open.
Let $\xi \in X$ be a generic point of an irreducible component
of $X$. By the openness of $f$ we see that $f(\xi)$ is the
generic point $\eta$ of $Z = Y$. Note that $\dim_\xi(X_\eta) = r$
by assumption that $f$ has relative dimension $r$. On the other
hand, since $\xi$ is a generic point of $X$ we see that
$\mathcal{O}_{X, \xi} = \mathcal{O}_{X_\eta, \xi}$ has only one
prime ideal and hence has dimension $0$. Thus by
Morphisms, Lemma \ref{morphisms-lemma-dimension-fibre-at-a-point}
we conclude that the transcendence
degree of $\kappa(\xi)$ over $\kappa(\eta)$ is $r$.
In other words, $\delta(\xi) = \delta(\eta) + r$ as desired.
\end{proof}




\section{Flat pullback}
\label{section-flat-pullback}

\noindent
In the following we use $f^{-1}(Z)$ to denote the
{\it scheme theoretic inverse image} of a closed subscheme
$Z \subset Y$ for a morphism of schemes $f : X \to Y$.
We recall that the scheme theoretic inverse image is the fibre product
$$
\xymatrix{
f^{-1}(Z) \ar[r] \ar[d] & X \ar[d] \\
Z \ar[r] & Y
}
$$
and it is also the closed subscheme of $X$ cut out by the
quasi-coherent sheaf of ideals $f^{-1}(\mathcal{I})\mathcal{O}_X$, if
$\mathcal{I} \subset \mathcal{O}_Y$ is the quasi-coherent sheaf of ideals
corresponding to $Z$ in $Y$.
(This is discussed in
Schemes, Section \ref{schemes-section-closed-immersion} and
Lemma \ref{schemes-lemma-fibre-product-immersion}
and Defintion \ref{schemes-definition-inverse-image-closed-subscheme}.)

\begin{definition}
\label{definition-flat-pullback}
Let $(S, \delta)$ be as in Sitation \ref{situation-setup}.
Let $X$, $Y$ be locally of finite type over $S$.
Let $f : X \to Y$ be a morphism.
Assume $f$ is flat of relative dimension $r$.
\begin{enumerate}
\item Let $Z \subset Y$ be an integral closed subscheme of
$\delta$-dimension $k$. We define $f^*[Z]$ to be the
$(k+r)$-cycle on $X$ to the scheme theoretic inverse image
$$
f^*[Z] = [f^{-1}(Z)]_{k+r}.
$$
This makes sense since $\dim_\delta(f^{-1}(Z)) = k + r$
by Lemma \ref{lemma-flat-inverse-image-dimension}.
\item Let $\alpha = \sum n_i [Z_i]$ be
a $k$-cycle on $Y$. The {\it flat pullback of $\alpha$ by $f$}
is the sum
$$
f^* \alpha = \sum n_i f^*[Z_i]
$$
where each $f^*[Z_i]$ is defined as above.
\item We denote $f^* : Z_k(Y) \to Z_{k + r}(Y)$ the map of abelian
groups so obtained.
\end{enumerate}
\end{definition}

\noindent
An open immersion is flat. This is an important though trivial special
case of a flat morphism. If $U \subset X$ is open then sometimes the
pullback by $j : U \to X$ of a cycle is called the {\it restriction} of the
cycle to $U$. Note that in this case the maps
$$
j^* : Z_k(X) \longrightarrow Z_k(U)
$$
are all {\it surjective}. The reason is that given any integral closed
subscheme $Z' \subset U$, we can take the closure of $Z$ of $Z'$ in $X$
and think of it as a reduced closed subscheme of $X$ (see
Schemes, Lemma \ref{schemes-lemma-reduced-closed-subscheme}).
And clearly $Z \cap U = Z'$, ion other words
$j^*[Z] = [Z']$ whence the surjectivity. In fact a little bit more
is true.

\begin{lemma}
\label{lemma-exact-sequence-open}
Let $(S, \delta)$ be as in Sitation \ref{situation-setup}.
Let $X$ be locally of finite type over $S$.
Let $U \subset X$ be an open subscheme, and denote
$i : Y = X \setminus U \to X$ as a reduced closed subscheme of $X$.
For every $k \in \mathbf{Z}$ the sequence
$$
\xymatrix{
Z_k(Y) \ar[r]^{i_*} & Z_k(X) \ar[r]^{j_*} & Z_k(U) \ar[r] & 0
}
$$
is an exact complex of abelian groups.
\end{lemma}

\begin{proof}
By the description above the basis elements $[Z]$ of the free
abelian group $Z_k(X)$ map either to (distinct) basis elements
$[Z \cap U]$ or to zero if $Z \subset Y$. Hence the lemma is clear.
\end{proof}

\begin{lemma}
\label{lemma-compose-flat-pullback}
Let $(S, \delta)$ be as in Sitation \ref{situation-setup}.
Let $X, Y, Z$ be locally of finite type over $S$.
Let $f : X \to Y$ and $g : Y \to Z$ be flat morphisms of relative dimensions
$r$ and $s$. Then $f^* \circ g^* = (g \circ f)^*$ as maps
$Z_k(Z) \to Z_{k + r + s}(X)$.
\end{lemma}

\begin{proof}
Omitted, but see Lemma \ref{lemma-pullback-transitive}.
\end{proof}

\begin{lemma}
\label{lemma-pullback-coherent}
Let $(S, \delta)$ be as in Sitation \ref{situation-setup}.
Let $X, Y$ be locally of finite type over $S$.
Let $f : X \to Y$ be a flat morphism of relative dimension $r$.
Let $\mathcal{F}$ be a coherent sheaf on $Y$ with
$\dim_\delta(\text{Supp}(\mathcal{F})) \leq k$.
Then $f^*[{\mathcal F}]_k = [f^*{\mathcal F}]_{k+r}$.
\end{lemma}

\begin{proof}
Omitted, but see Lemma \ref{lemma-pullback-module}.
\end{proof}



\section{Push and pull}
\label{section-push-pull}

\noindent
In this section we verify that proper pushforward and flat pullback
are compatible when this makes sense.

\begin{lemma}
\label{lemma-flat-pullback-proper-pushforward}
Let $(S, \delta)$ be as in Sitation \ref{situation-setup}.
Let
$$
\xymatrix{
X' \ar[r]_{g'} \ar[d]_{f'} & X \ar[d]^f \\
Y' \ar[r]^g & Y
}
$$
be a fibre product diagram of schemes locally of finite type over $S$.
Assume $f : X \to Y$ proper and $g : Y' \to Y$ flat of relative dimension $r$.
Then also $f'$ is proper and $g'$ is flat of relative dimension $r$.
For any $k$-cycle $\alpha$ on $X$ we have
$$
g^*f_*\alpha = f'_*(g')^*\alpha
$$
in $Z_{k + r}(Y')$.
\end{lemma}

\begin{proof}
Omitted, but see Lemma \ref{lemma-flat-pullback-pushdown-module}.
\end{proof}









\section{Preparation for principal divisors}
\label{section-preparation-principal-divisors}

\noindent
Recall that if $Z$ is an irreducible closed subset of a scheme $X$,
then the codimension of $Z$ in $X$ is equal to the dimension
of the local ring $\mathcal{O}_{X, \xi}$, where $\xi \in Z$
is the generic point. See
Properties, Lemma \ref{properties-lemma-codimension-local-ring}.

\begin{definition}
\label{definition-order-vanishing}
Let $X$ be a locally Noetherian scheme. Assume $X$ is integral.
Let $f \in R(X)^*$. For every integral closed subscheme
$Z \subset X$ of codimension $1$ we define
the {\it order of vanishing of $f$ along $Z$} as the integer
$$
\text{ord}_Z(f) = \text{ord}_{\mathcal{O}_{X, \xi}}(f)
$$
where the right hand side is the notion of
Definition \ref{definition-ord} and $\xi$ is the generic point of $Z$.
\end{definition}

\begin{lemma}
\label{lemma-divisor-delta-dimension}
Let $(S, \delta)$ be as in Sitation \ref{situation-setup}.
Let $X$ be locally of finite type over $S$. Assume $X$ is
integral. If $Z \subset X$ is an integral closed subscheme
of codimension $1$, then $\dim_\delta(Z) = \dim_\delta(X) - 1$.
\end{lemma}

\begin{proof}
This is more or less the defining property of a dimension function.
\end{proof}

\begin{lemma}
\label{lemma-divisor-locally-finite}
Let $(S, \delta)$ be as in Sitation \ref{situation-setup}.
Let $X$ be locally of finite type over $S$. Assume $X$ is
integral. Let $f \in R(X)^*$. Then the set
$$
\{Z \subset X \mid Z \text{ is irreducible, closed of codimension }1
\text{ and }\text{ord}_Z(f) \not = 0\}
$$
is locally finite in $X$.
\end{lemma}

\begin{proof}
This is true simply because there exists a nonempty open subscheme
$U \subset X$ such that $f$ corresponds to a section of
$\Gamma(U, \mathcal{O}_X^*)$, and hence the codimension $1$
irreducibles which can occur in the set of the lemma are all
irreducible components of $X \setminus U$.
Hence Lemma \ref{lemma-multiplicity-finite} gives the desired result.
\end{proof}


\section{Principal divisors}
\label{section-principal-divisors}

\noindent
The following definition is the key to everything that follows.

\begin{definition}
\label{definition-principal-divisor}
Let $(S, \delta)$ be as in Sitation \ref{situation-setup}.
Let $X$ be locally of finite type over $S$. Assume $X$ is
integral with $\dim_\delta(X) = n$.
Let $f \in R(X)^*$. The {\it principal divisor
associated to $f$} is the $(n - 1)$-cycle
$$
\text{div}(f) = \text{div}_X(f) = \sum \text{ord}_Z(f) [Z]
$$
where the sum is over integral closed subscheme of
codimension $1$ and $\text{ord}_Z(f)$ is as in
Definition \ref{definition-order-vanishing}. This makes sense
by Lemmas \ref{lemma-divisor-delta-dimension} and
\ref{lemma-divisor-locally-finite}.
\end{definition}


\begin{lemma}
\label{lemma-div-additive}
Let $(S, \delta)$ be as in Sitation \ref{situation-setup}.
Let $X$ be locally of finite type over $S$. Assume $X$ is
integral with $\dim_\delta(X) = n$.
Let $f, g \in R(X)^*$.
Then
$$
\text{div}(fg) = \text{div}(f) + \text{div}(g)
$$
in $Z_{n - 1}(X)$.
\end{lemma}

\begin{proof}
This is clear from the additivity of the $\text{ord}$ functions.
\end{proof}

\noindent
An important role in the discussion of principal divisors
is played by the ``universal'' principal divisor $[0] - [\infty]$
on $\mathbf{P}^1_S$. To make this more precise, let us denote
$$
D_0, D_\infty \subset
\mathbf{P}^1_S = \underline{\text{Proj}}_S(\mathcal{O}_S[X_0, X_1])
$$
the closed subscheme cut out by the section $X_1$, resp.\ $X_0$
of $\mathcal{O}(1)$. This is an effective Cartier divisor, see
Divisors, Definition \ref{divisors-definition-effective-Cartier-divisor}
and Lemma \ref{divisors-lemma-characterize-OD}.
The following lemma say that loosely speaking\footnote{It does not make
sense since $S$ may be nonreduced or not irreducible.} we have
``$\text{div}(X_1/X_0) = [D_0] - [D_1]$'' and that this is the
universal principal divisor.

\begin{lemma}
\label{lemma-rational-function}
Let $(S, \delta)$ be as in Sitation \ref{situation-setup}.
Let $X$ be locally of finite type over $S$. Assume $X$ is
integral and $n = \dim_\delta(X)$. Let $f \in R(X)^*$.
Let $U \subset X$ be a nonempty open such that $f$
corresponds to a section $f \in \Gamma(U, \mathcal{O}_X^*)$.
Let $Y \subset X \times_S \mathbf{P}^1_S$ be the
closure of the graph of $f : U \to \mathbf{P}^1_S$.
\begin{enumerate}
\item The projection morphism $p : Y \to X$ is
proper,
\item $p|_{p^{-1}(U)} : p^{-1}(U) \to U$ is an isomorphism,
\item the pullbacks $q^{-1}D_0$ and $q^{-1}D_\infty$ are effective
Cartier divisors on $Y$, and
\item we have
$$
\text{div}(f) = p_*([q^{-1}D_0]_{n - 1} - [q^{-1}D_\infty]_{n - 1})
$$
\end{enumerate}
\end{lemma}

\begin{proof}
Omitted.
\end{proof}




\section{Two fun results on principal divisors}
\label{section-two-fun}

\noindent
The first lemma implies that the pushforward of a principal
divisor along a generically finite morphism is a principal divisor.

\begin{lemma}
\label{lemma-proper-pushforward-alteration}
Let $(S, \delta)$ be as in Sitation \ref{situation-setup}.
Let $X$, $Y$ be locally of finite type over $S$. Assume $X$, $Y$
are integral and $n = \dim_\delta(X) = \dim_\delta(Y)$.
Let $p : X \to Y$ be a dominant proper morphism.
Let $f \in R(X)^*$. Set
$$
g = \text{Nm}_{R(X)/R(Y)}(f).
$$
Then we have
$p_*\text{div}(f) = \text{div}(g)$.
\end{lemma}

\begin{proof}
Omitted, but see Lemma \ref{lemma-finite-extension-dim-1}.
\end{proof}

\noindent
The following lemma says that the degree of a principal divisor on
a proper curve is zero.

\begin{lemma}
\label{lemma-curve-principal-divisor}
Let $K$ be any field. Let $X$ be a $1$-dimensional integral scheme
endowed with a proper morphism $p : X \to \text{Spec}(K)$.
Let $f \in K(X)^*$ be an invertible rational function.
Then
$$
\sum\nolimits_{x \in X \text{ closed}}
[\kappa(x) : K] \text{ord}_{\mathcal{O}_{X, x}}(f)
=
0
$$
where $\text{ord}$ is as in Definition \ref{definition-ord}.
In other words, $p_*\text{div}(f) = 0$.
\end{lemma}

\begin{proof}
Omitted.
\end{proof}





\section{Rational equivalence}
\label{section-rational-equivalence}

\noindent
We define {\it rational equivalence} on $k$-cycles on $X$ as follows.
In this definition we allow locally finite sums of images of
principal divisors (under closed immersions). This leads to some
pretty strange phenomena, see Example \ref{example-weird}.
However, if we do not allow these then we do not know how to prove that
capping with chern classes of line bundles factors through rational
equivalence.

\begin{definition}
\label{definition-rational-equivalence}
Let $(S, \delta)$ be as in Sitation \ref{situation-setup}.
Let $X$ be a scheme locally of finite type over $S$.
Let $k \in \mathbf{Z}$.
\begin{enumerate}
\item Given any locally finite collection $\{W_j \subset X\}$
of integral closed subschemes with $\dim_\delta(W_j) = k + 1$,
and any $f_j \in R(W_j)^*$ we may consider
$$
\sum (i_j)_*\text{div}(f_j) \in Z_k(X)
$$
where $i_j : W_j \to X$ is the inclusion morphism.
This makes sense as the morphism
$\coprod i_j : \coprod W_j \to X$ is proper.
\item We say that $\alpha \in Z_k(X)$ is {\it rationally equivalent to zero}
if $\alpha$ is a cycle of the form displayed above.
\item We say $\alpha, \beta \in Z_k(X)$ are
{\it rationally equivalent} and we write $\alpha \sim_{rat} \beta$
if $\alpha - \beta$ is rationally equivalent to zero.
\item We define
$$
A_k(X) = Z_k(X) / \sim_{rat}
$$
to be the {\it Chow group of $k$-cycles on $X$}. This is sometimes called
the {\it Chow group of $k$-cycles module rational equivalence on $X$}.
\end{enumerate}
\end{definition}

\noindent
There are many other interesting (adequate) equivalence relations.
Rational equivalence is the coarsest one of them all.
A very simple but important lemma is the following.

\begin{lemma}
\label{lemma-restrict-to-open}
Let $(S, \delta)$ be as in Sitation \ref{situation-setup}.
Let $X$ be a scheme locally of finite type over $S$.
Let $U \subset X$ be an open subscheme, and denote
$i : Y = X \setminus U \to X$ as a reduced closed subscheme of $X$.
Let $k \in \mathbf{Z}$.
Suppose $\alpha, \beta \in Z_k(X)$.
If $\alpha|_U \sim_{rat} \beta|_U$ then there exist a cycle
$\gamma \in Z_k(Y)$ such that
$$
\alpha \sim_{rat} \beta + i_*\gamma.
$$
In other words, the sequence
$$
\xymatrix{
A_k(Y) \ar[r]^{i_*} & A_k(X) \ar[r]^{j_*} & A_k(U) \ar[r] & 0
}
$$
is an exact complex of abelian groups.
\end{lemma}

\begin{proof}
Let $\{W_j\}_{j \in J}$ be a locally finite collection of integral closed
subschemes of $\delta$-dimension $k + 1$, and let $f_j \in R(W_j)^*$
be elements such that $(\alpha - \beta)|_U = \sum (i_j)_*\text{div}(f_j)$
as in the definition. Set $W_j' \subset X$ equal
to the closure of $W_j$. Suppose that $V \subset X$ is a quasi-compact
open. Then also $V \cap U$ is quasi-compact open in $U$ as
$V$ is Noetherian. Hence the set
$\{j \in J \mid W_j \cap V \not = \emptyset\}
= \{j \in J \mid W'_j \cap V \not = \emptyset\}$
is finite since $\{W_j\}$ is locally finite. In other words we see that
$\{W'_j\}$ is also locally finite. Since $R(W_j) = R(W'_j)$ we see
that
$$
\alpha - \beta - \sum (i'_j)_*\text{div}(f_j)
$$
is a cycle supported on $Y$ and the lemma follows (see
Lemma \ref{lemma-exact-sequence-open}).
\end{proof}

\begin{example}
\label{example-weird}
Here is a ``strange'' example.
Suppose that $S$ is the spectrum of a field $k$
with $\delta$ as in Example \ref{example-field}.
Suppose that $X = C_1 \cup C_2 \cup \ldots$ is an infinite
union of curves $C_j \cong \mathbf{P}^1_k$ glued together
in the following way: The point $\infty \in C_i$ is glued
transversally to the point $0 \in C_{j + 1}$ for $j = 1, 2, 3, \ldots$.
Take the point $0 \in C_1$. This gives a zero cycle
$[0] \in Z_0(X)$. The ``strangeness'' in this situation is
that actually $[0] \sim_{rat} 0$! Namely we can choose
the rational function $f_j \in R(C_j)$ to be the function
which has a simple zero at $0$ and a simple pole at $\infty$
and no other zeros or poles. Then we see that the sum
$\sum (i_j)_*\text{div}(f_j)$ is exactly the $0$-cycle
$[0]$. In fact it turns out that $A_0(X) = 0$ in this example.
If you find this too bizarre, then you can just
make sure your spaces are always quasi-compact
(so $X$ does not even exist for you).
\end{example}









\section{Properties of rational equivalence}
\label{section-properties-rational-equivalence}

\begin{lemma}
\label{lemma-flat-pullback-rational-equivalence}
Let $(S, \delta)$ be as in Sitation \ref{situation-setup}.
Let $X$, $Y$ be schemes locally of finite type over $S$.
Let $f : X \to Y$ be a flat morphism of relative dimension $r$.
Let $\alpha \sim_{rat} \beta$ be rationally equivalent $k$-cycles on $Y$.
Then $f^*\alpha \sim_{rat} f^*\beta$ as $(k + r)$-cycles on $X$.
\end{lemma}

\begin{proof}
Omitted.
\end{proof}

\begin{lemma}
\label{lemma-proper-pushforward-rational-equivalence}
Let $(S, \delta)$ be as in Sitation \ref{situation-setup}.
Let $X$, $Y$ be schemes locally of finite type over $S$.
Let $p : X \to Y$ be a proper morphism.
Suppose $\alpha, \beta \in Z_k(X)$ are rationally equivalent.
Then $p_*\alpha$ is rationally equivalent to $p_*\beta$.
\end{lemma}

\begin{proof}
Let $i : W \to X$ be a $(k + 1)$-dimensional integral closed subscheme.
Let $f \in R(W)^*$. It suffices to show that
$p_*i_*\text{div}(f)$ is rationally equivalent to zero.
Let $W' = p(W)$. This is an integral closed subscheme of $Y$.
We get a commutative diagram of morphisms
$$
\xymatrix{
W \ar[r]_i \ar[d]_{p'} & X \ar[d]^p \\
W' \ar[r]^{i'} & Y
}
$$
Note that $p_*i_*\text{div}(f) = i'_*(p')_*\text{div}(f)$ by
Lemma \ref{lemma-compose-pushforward}. Hence
it suffices to show that $(p')_*\text{div}(f)$
is the divisor of a rational function on $W'$ or zero.
There are three cases to distinguish.

\medskip\noindent
The case $\dim_\delta(W') < k$. In this case automatically
$(p')_*\text{div}(f) = 0$ and there is nothing to prove.

\medskip\noindent
The case $\dim_\delta(W') = k$. Here you have to show that
$(p')_*\text{div}(f) = 0$. This eventually boils down to
the fact that the degree of a principal divisor on a proper
curve is zero. Details omitted, but see the lemma on curves below.

\medskip\noindent
The case $\dim_\delta(W') = k + 1$. In this case
Lemma \ref{lemma-proper-pushforward-alteration} applies.
\end{proof}















\section{Different characterizations of rational equivalence}
\label{section-different-rational-equivalence}

\noindent
Let $(S, \delta)$ be as in Sitation \ref{situation-setup}.
Let $X$ be a scheme locally of finite type over $S$.
Given any closed subscheme
$Z \subset X \times_S \mathbf{P}^1_S = X \times \mathbf{P}^1$
we let $Z_0$, resp.\ $Z_\infty$ be the scheme theoretic closed 
subscheme $Z_0 = \text{pr}_2^{-1}(D_0)$,
resp.\ $Z_\infty = \text{pr}_2^{-1}(D_\infty)$.
Here $D_0$, $D_\infty$ are as defined just above
Lemma \ref{lemma-rational-function}.

\begin{lemma}
\label{lemma-rational-equivalence-family}
Let $(S, \delta)$ be as in Sitation \ref{situation-setup}.
Let $X$ be a scheme locally of finite type over $S$.
Let $W \subset X \times_S \mathbf{P}^1_S$ be an integral
closed subscheme of $\delta$-dimension $k + 1$.
Assume $W \not = W_0$, and $W \not = W_\infty$. Then
\begin{enumerate}
\item $W_0$, $W_\infty$ are effective Cartier divisors of $W$,
\item $W_0$, $W_\infty$ can be viewed as closed subschemes
of $X$ and
$$
[W_0]_k \sim_{rat} [W_\infty]_k,
$$
\item for any $\alpha \in Z_k(X)$ with $\alpha \sim_{rat} 0$
there exists a locally finite family of
integral closed subschemes $W_i \subset X \times_S \mathbf{P}^1_S$
of $\delta$-dimension $k + 1$
as above such that $\alpha = \sum ([(W_i)_0]_k - [(W_i)_\infty]_k)$.
\end{enumerate}
\end{lemma}

\begin{proof}
Omitted.
\end{proof}

\begin{lemma}
\label{lemma-closed-subscheme-cross-p1}
Let $(S, \delta)$ be as in Sitation \ref{situation-setup}.
Let $X$ be a scheme locally of finite type over $S$.
Let $Z$ be a closed subscheme of $X \times \mathbf{P}^1$.
Assume $\dim_\delta(Z) \leq k + 1$.
Assume that any irreducible component $Z' \subset Z$
which is contained in $Z_0$ or $Z_\infty$ has
$\dim_\delta(Z') < k$. Then
$$
[Z_0]_k \sim_{rat} [Z_\infty]_k
$$
as $k$-cycles on $X$.
\end{lemma}

\begin{proof}
Omitted.
\end{proof}

\begin{lemma}
\label{lemma-coherent-sheaf-cross-p1}
Let $(S, \delta)$ be as in Sitation \ref{situation-setup}.
Let $X$ be a scheme locally of finite type over $S$.
Let $\mathcal{F}$ be a coherent sheaf on $X \times \mathbf{P}^1$.
Assume
$
\dim_\delta(\text{Supp}(\mathcal{F})) \leq k + 1.
$
Let $i_0, i_\infty : X \to X \times \mathbf{P}^1$ be the closed immersion
such that $i_t(x) = (x, t)$. Denote $\mathcal{F}_0 = i_0^*\mathcal{F}$ and
$\mathcal{F}_\infty = i_\infty^*\mathcal{F}$.
Assume that any irreducible component $W$ of $\text{Supp}(\mathcal{F})$
which is contained in $X \times \{0\}$ or $X \times \{\infty\}$ has
$\dim_\delta(W) < k$. Then
$$
[\mathcal{F}_0]_k \sim_{rat} [\mathcal{F}_\infty]_k
$$
as $k$-cycles on $X$.
\end{lemma}

\begin{proof}
Omitted.
\end{proof}

\begin{lemma}
\label{lemma-maps-between-coherent-sheaves}
Let $(S, \delta)$ be as in Sitation \ref{situation-setup}.
Let $X$ be a scheme locally of finite type over $S$.
Let $\mathcal{F}$, $\mathcal{G}$ be two coherent sheaves on $X$.
Assume that
$$
\dim_\delta(\text{Supp}(\mathcal{F})),
\dim_\delta(\text{Supp}(\mathcal{G}))
\leq
k + 1.
$$
Suppose that $\alpha, \beta : \mathcal{F} \to \mathcal{G}$
are two maps such that
$$
\text{Ker}(\alpha),
\text{Ker}(\beta),
\text{Coker}(\alpha),
\text{Coker}(\beta)
$$
all have support of $\delta$-dimension $\leq k$.
Then we have
$$
[\text{Coker}(\alpha)]_k - [\text{Ker}(\alpha)]_k
\sim_{rat}
[\text{Coker}(\beta)]_k - [\text{Ker}(\beta)]_k
$$
as $k$-cycles on $X$.
\end{lemma}

\begin{proof}
Omitted.
\end{proof}








\section{Preparation for the divisor associated to an invertible sheaf}
\label{section-preparation-divisor-sheaf}

\noindent
Let $X$ be a scheme.
Let $\mathcal{L}$ be an invertible $\mathcal{O}_X$-module.
Let $\xi \in X$ be a point.
If $s_\xi, s'_\xi \in \mathcal{L}_\xi$ generate $\mathcal{L}_\xi$
as $\mathcal{O}_{X, \xi}$-module, then there exists a unit
$u \in \mathcal{O}_{X, \xi}^*$ such that $s_\xi = u s'_\xi$.
Hence the following definition makes sense.

\begin{definition}
\label{definition-order-vanishing-meromorphic}
Let $X$ be a locally Noetherian scheme. Assume $X$ is integral.
Let $\mathcal{L}$ be an invertible $\mathcal{O}_X$-module.
Let $s \in \mathcal{K}_X(\mathcal{L})$ be a regular (i.e., nonzero)
meromorphic section of $\mathcal{L}$, see
Divisors, Definition \ref{divisors-definition-meromorphic-section}.
For every integral closed subscheme
$Z \subset X$ of codimension $1$ we define
the {\it order of vanishing of $s$ along $Z$} as the integer
$$
\text{ord}_{Z, \mathcal{L}}(s)
= \text{ord}_{\mathcal{O}_{X, \xi}}(s/s_\xi)
$$
where the right hand side is the notion of
Definition \ref{definition-ord}, $\xi \in Z$ is the generic point,
and $s_\xi \in \mathcal{L}_\xi$ is a generator.
\end{definition}

\begin{lemma}
\label{lemma-divisor-meromorphic-locally-finite}
Let $(S, \delta)$ be as in Sitation \ref{situation-setup}.
Let $X$ be locally of finite type over $S$. Assume $X$ is
integral. Let $\mathcal{L}$ be an invertible $\mathcal{O}_X$-module.
Let $s \in \mathcal{K}_X(\mathcal{L})$ be a regular (i.e., nonzero)
meromorphic section of $\mathcal{L}$. Then the set
$$
\{Z \subset X \mid Z \text{ is irreducible, closed of codimension }1
\text{ and }\text{ord}_{Z, \mathcal{L}}(s) \not = 0\}
$$
is locally finite in $X$.
\end{lemma}

\begin{proof}
This is true simply because there exists a nonempty open subscheme
$U \subset X$ such that $s$ corresponds to a section of
$\Gamma(U, \mathcal{L})$ which generates $\mathcal{L}$ over $U$.
Hence the codimension $1$
irreducibles which can occur in the set of the lemma are all
irreducible components of $X \setminus U$.
Hence Lemma \ref{lemma-multiplicity-finite} gives the desired result.
\end{proof}

\begin{lemma}
\label{lemma-divisor-meromorphic-well-defined}
Let $(S, \delta)$ be as in Sitation \ref{situation-setup}.
Let $X$ be locally of finite type over $S$. Assume $X$ is
integral and $n = \dim_\delta(X)$.
Let $\mathcal{L}$ be an invertible $\mathcal{O}_X$-module.
Let $s, s' \in \mathcal{K}_X(\mathcal{L})$ be nonzero
meromorphic sections of $\mathcal{L}$. Then $f = s/s'$
is an element of $R(X)$ and we have
$$
\sum \text{ord}_{Z, \mathcal{L}}(s)[Z]
=
\sum \text{ord}_{Z, \mathcal{L}}(s')[Z]
+
\text{div}(f)
$$
(where the sums are over integral closed subschemes $Z \subset X$
of $\delta$-dimension $n - 1$) as elements of $Z_{n - 1}(X)$.
\end{lemma}

\begin{proof}
This is clear from the definitions.
Note that Lemma \ref{lemma-divisor-meromorphic-locally-finite}
garantees that the sums are indeed
elements of $Z_{n - 1}(X)$.
\end{proof}









\section{The divisor associated to an invertible sheaf}
\label{section-divisor-invertible-sheaf}

\noindent
The material above allows us to define the divisor
associated to an invertible sheaf.

\begin{definition}
\label{definition-divisor-invertible-sheaf}
Let $(S, \delta)$ be as in Sitation \ref{situation-setup}.
Let $X$ be locally of finite type over $S$. Assume $X$ is
integral and $n = \dim_\delta(X)$.
Let $\mathcal{L}$ be an invertible $\mathcal{O}_X$-module.
\begin{enumerate}
\item For any nonzero meromorphic section $s$ of $\mathcal{L}$
we define the {\it Weil divisor associated to $s$} as
$$
\text{div}_{\mathcal{L}}(s) :=
\sum \text{ord}_{Z, \mathcal{L}}(s) [Z] \in Z_{n - 1}(X)
$$
where the sum is over integral closed subschemes $Z \subset X$
of $\delta$-dimension $n - 1$.
\item We define {\it Weil divisor associated to $\mathcal{L}$}
$$
c_1(\mathcal{L}) \cap [X] =
\text{class of }\text{div}_{\mathcal{L}}(s) \in A_{n - 1}(X)
$$
where $s$ is any nonzero meromorphic section of $\mathcal{L}$ over
$X$. This is well defined by
Lemma \ref{lemma-divisor-meromorphic-well-defined}.
\end{enumerate}
\end{definition}

\noindent
There are some cases where it is easy to compute the
Weil divisor associated to an invertible sheaf.

\begin{lemma}
\label{lemma-compute-c1}
Let $(S, \delta)$ be as in Sitation \ref{situation-setup}.
Let $X$ be locally of finite type over $S$. Assume $X$ is
integral and $n = \dim_\delta(X)$.
Let $\mathcal{L}$ be an invertible $\mathcal{O}_X$-module.
Let $s \in \Gamma(X, \mathcal{L})$ be a nonzero global section.
Then
$$
c_1(\mathcal{L}) \cap [X] = [Z(s)]_{n - 1}
$$
in $A_{n - 1}(X)$.
\end{lemma}

\begin{proof}
Omitted.
\end{proof}

\begin{lemma}
\label{lemma-c1-additive}
Let $(S, \delta)$ be as in Sitation \ref{situation-setup}.
Let $X$ be locally of finite type over $S$. Assume $X$ is
integral and $n = \dim_\delta(X)$.
Let $\mathcal{L}$, $\mathcal{N}$ be invertible $\mathcal{O}_X$-modules.
Then
$$
c_1(\mathcal{L}) \cap [X] + c_1(\mathcal{N}) \cap [X] =
c_1(\mathcal{L} \otimes_{\mathcal{O}_X} \mathcal{N}) \cap [X]
$$
in $A_{n - 1}(X)$.
\end{lemma}

\begin{proof}
Omitted.
\end{proof}

\noindent
The following lemma will be superseded by a more general result later.

\begin{lemma}
\label{lemma-flat-pullback-divisor-invertible-sheaf}
Let $(S, \delta)$ be as in Sitation \ref{situation-setup}.
Let $X$, $Y$ be locally of finite type over $S$. Assume $X$, $Y$
are integral and $n = \dim_\delta(X)$.
Let $\mathcal{L}$ be an invertible $\mathcal{O}_Y$-module.
Let $f : X \to Y$ be a flat morphism of relative dimension $r$.
Let $\mathcal{L}$ be an invertible sheaf on $Y$. Then
$$
f^*(c_1(\mathcal{L}) \cap [Y]) = c_1(f^*\mathcal{L}) \cap [X]
$$
in $A_{n - 1}(X)$.
\end{lemma}

\begin{proof}
Omitted.
\end{proof}



\section{Intersecting with Cartier divisors}
\label{section-intersecting-with-divisors}

\begin{definition}
\label{definition-cap-c1}
Let $(S, \delta)$ be as in Sitation \ref{situation-setup}.
Let $X$ be locally of finite type over $S$.
Let $\mathcal{L}$ be an invertible $\mathcal{O}_X$-module.
We define, for every integer $k$, an operation
$$
c_1(\mathcal{L}) \cap - :
Z_{k + 1}(X) \to A_k(X)
$$
called {\it intersection with the first chern class of $\mathcal{L}$}.
\begin{enumerate}
\item Given an integral closed subscheme $i : W \to X$ with
$\dim_\delta(W) = k + 1$ we define
$$
c_1(\mathcal{L}) \cap [W] = i_*(c_1({i^*\mathcal{L}}) \cap [W])
$$
where the right hand side is defined in
Definition \ref{definition-divisor-invertible-sheaf}.
\item For a general $(k + 1)$-cycle $\alpha = \sum n_i [W_i]$ we set
$$
c_1(\mathcal{L}) \cap \alpha = \sum n_i c_1(\mathcal{L}) \cap [W_i]
$$
\end{enumerate}
\end{definition}

\noindent
The main goal for the next few sections is to show that intersecting with
$c_1(\mathcal{L})$ factors through rational equivalence, and is commutative.
This is not a trviality.

\begin{lemma}
\label{lemma-c1-cap-additive}
Let $(S, \delta)$ be as in Sitation \ref{situation-setup}.
Let $X$ be locally of finite type over $S$.
Let $\mathcal{L}$, $\mathcal{N}$ be an invertible sheaves on $X$.
Then
$$
c_1(\mathcal{L}) \cap \alpha  + c_1(\mathcal{N}) \cap \alpha =
c_1(\mathcal{L} \otimes_{\mathcal{O}_X} \mathcal{N}) \cap \alpha
$$
in $A_k(X)$ for every $\alpha \in Z_{k - 1}(X)$. Moreover, 
$c_1(\mathcal{O}_X) \cap \alpha = 0$ for all $\alpha$.
\end{lemma}

\begin{proof}
Omitted.
\end{proof}

\begin{lemma}
\label{lemma-flat-pullback-cap-c1}
Let $(S, \delta)$ be as in Sitation \ref{situation-setup}.
Let $X$, $Y$ be locally of finite type over $S$.
Let $f : X \to Y$ be a flat morphism of relative dimension $r$.
Let $\mathcal{L}$ be an invertible sheaf on $Y$.
Let $\alpha$ be a $k$-cycle on $Y$.
Then
$$
f^*(c_1(\mathcal{L}) \cap \alpha) = c_1(f^*\mathcal{L}) \cap f^*\alpha
$$
in $A_{k + r}(X)$.
\end{lemma}

\begin{proof}
Omitted.
\end{proof}

\begin{lemma}
\label{lemma-pushforward-cap-c1}
Let $(S, \delta)$ be as in Sitation \ref{situation-setup}.
Let $X$, $Y$ be locally of finite type over $S$.
Let $p : X \to Y$ be a proper morphism.
Let $\alpha \in Z_{k + 1}(X)$.
Let $\mathcal{L}$ be an invertible sheaf on $Y$.
Then
$$
p_*(c_1(p^*\mathcal{L}) \cap \alpha) = c_1(\mathcal{L}) \cap p_*\alpha
$$
in $A_k(Y)$.
\end{lemma}

\begin{proof}
Omitted. Idea: Denote $Z' = p(Z)$ the image. Let $s'$ be a nonzero
meromorphic section of $\mathcal{L}|_{Z'}$. Denote $s$ the corresponding
meromorphic section of $p^*\mathcal{L}|_Z$. 

\medskip\noindent
If $\dim(Z') < k$ then $s'$ is being an invertible section  over a nonempty
open of $Z'$ easily implies that $p_*(c_1(p^*\mathcal{L}) \cap [Z]) = 0$
for dimension reasons. Hence this case follows.

\medskip\noindent
If $\dim(Z') = k$, then let $\xi \in Z'$ be any point of codimension
$1$ of $Z'$. Choose an element $s_\xi \in \mathcal{L}_\xi$ which
is a local generator of $\mathcal{L}$ in a neighbourhood $U$ of $\xi$.
Consider the rational function $f' = s/s_{\xi} \in \mathbf{C}(Z')$.
Denote $f \in \mathbf{C}(Z)$ the pullback of $f'$ to a rational function on
$Z$. The result we have to show is that
$$
[\mathbf{C}(Z) : \mathbf{C}(Z')] \text{div}(f')|_{U \cap Z'} =
p_*\text{div}(f)|_{U \cap Z'}
$$
This follows from Lemma \ref{lemma-proper-pushforward-alteration} since
$$
\text{Nm}_{\mathbf{C}(Z)/\mathbf{C}(Z')}(f)
=
(f')^{[\mathbf{C}(Z) : \mathbf{C}(Z')]}
$$
by our choice of $f$.
\end{proof}

\begin{lemma}
\label{lemma-geometric-cap}
Let $(S, \delta)$ be as in Sitation \ref{situation-setup}.
Let $X$ be locally of finite type over $S$.
Let $\mathcal{L}$ be an invertible $\mathcal{O}_X$-module.
Let $Z \subset X$ be a closed subscheme.
Assume $\dim_\delta(Z) \leq k + 1$.
Let $s \in \Gamma(Z, \mathcal{L}|_Z)$.
Denote $Z(s) \subset Z$ the closed subscheme where $s$ vanishes.
Assume
\begin{enumerate}
\item $\dim_\delta(Z(s)) \leq k$, and
\item for every generic point $\xi$ of an irreducible component of
$Z(s)$ of dimension $k$ the multiplication by $s$
induces an injection $\mathcal{O}_{Z, \xi} \to (\mathcal{L}|_Z)_\xi$.
\end{enumerate}
Then
$$
[Z(s)]_k = c_1(\mathcal{L}) \cap [Z]_{k + 1}
$$
in $A_k(X)$.
\end{lemma}

\begin{proof}
Omitted.
\end{proof}







\section{Blowing up lemmas}
\label{section-blowing-up-lemmas}


\begin{lemma}
\label{lemma-blowing-up-denominators}
Let $(S, \delta)$ be as in Sitation \ref{situation-setup}.
Let $X$ be locally of finite type over $S$.
Assume $X$ integral with $\dim_\delta(X) = n$.
Let $\mathcal{L}$ be an invertible $\mathcal{O}_X$-module.
Let $s$ be a nonzero meromorphic section of $\mathcal{L}$.
Let $U \subset X$ be the maximal open subscheme such that
$s$ corresponds to a section of $\mathcal{L}$ over $U$.
There exists a projective morphism
$$
\pi : X' \longrightarrow X
$$
such that
\begin{enumerate}
\item $X'$ is integral,
\item $\pi|_{\pi^{-1}(U)} : \pi^{-1}(U) \to U$ is an isomorphism,
\item there exist effective Cartier divisors $D, E \subset X'$
such that 
$$
\pi^*\mathcal{L} = \mathcal{O}_{X'}(D - E),
$$
\item the meromorphic section $s$ corresponds, via the isomorphism above,
to the meromorphic section $1_D \otimes (1_E)^{-1}$ (see Divisors,
Definition
\ref{divisors-definition-invertible-sheaf-effective-Cartier-divisor}),
\item we have
$$
\pi_*([D]_{n - 1} - [E]_{n - 1}) = \text{div}_{\mathcal{L}}(s)
$$
in $Z_{n - 1}(X)$.
\end{enumerate}
\end{lemma}

\begin{proof}
Omitted.
\end{proof}

\noindent
Let $X$ be a scheme. Let $\{D_i\}_{i \in I}$ be a locally finite collection
of effective Cartier divisors on $X$. Suppose given a function
$I \to \mathbf{Z}_{\geq 0}$, $i \mapsto n_i$. Then it makes sense to talk
about the effective Cartier divisor $\sum n_i D_i$. This is a mild
generalization of Divisors,
Definition \ref{divisors-definition-sum-effective-Cartier-divisors}.

\begin{lemma}
\label{lemma-blowing-up-intersections}
Let $(S, \delta)$ be as in Sitation \ref{situation-setup}.
Let $X$ be locally of finite type over $S$.
Assume $X$ integral and $\dim_\delta(X) = n$.
Let $\{D_i\}_{i \in I}$ be a locally finite collection of
effective Cartier divisors on $X$.
Assume that for all $\{i, j, k\} \subset I$, $\#\{i, j, k\} = 3$
we have $D_i \cap D_j \cap D_k = \emptyset$.
Then there exist $b : X' \to X$ and $\{D'_j\}_{j \in J}$ with
the following properties:
\begin{enumerate}
\item $b$ is proper morphism $b : X' \to X$,
\item $X'$ is integral,
\item we have $b|_{b^{-1}(U)} : b^{-1}(U) \to U$ an isomorphism, where
$U \subset X$ is the complement of the union of the pairwise
intersections of the $D_i$,
\item $\{D'_j\}_{j \in J}$ is a locally finite collection of effective
Cartier divisors on $X'$,
\item $\dim_\delta(D'_j \cap D'_{j'}) \leq n - 2$ if $j \not = j'$, and
\item $b^{-1}(D_i) = \sum n_{ij} D'_j$ for certain $n_{ij} \geq 0$.
\end{enumerate}
\end{lemma}

\begin{proof}
Omitted. See \cite[Section 2.4]{F}. We have adapted the formulation
so it also works in the non finite type setting.
Note that $b$ may be the composition of infinitely many blow
ups, but over any given quasi-compact open $W \subset X$ one needs
only finitely many blow-ups (this is the result of loc.\ cit.).
\end{proof}







\section{Intersecting with effective Cartier divisors}
\label{section-intersecting-effective-Cartier}

\begin{lemma}
\label{lemma-support-cap-effective-Cartier}
Let $(S, \delta)$ be as in Sitation \ref{situation-setup}.
Let $X$ be locally of finite type over $S$.
Let $D$ be an effective Cartier divisor on $X$.
Let $\alpha$ be a $k$-cycle on $X$.
There exists a representative $\beta \in Z_{k - 1}(X)$ of
$c_1(\mathcal{O}_X(D)) \cap \alpha$ which is
supported on $D \cap \text{Supp}(\alpha)$.
\end{lemma}

\begin{proof}
Omitted.
\end{proof}


\begin{lemma}
\label{lemma-improved-additivity}
Let $(S, \delta)$ be as in Sitation \ref{situation-setup}.
Let $X$ be locally of finite type over $S$.
Let $\{D_i\}_{i \in I}$ be a locally finite collection
of effective Cartier divisors on $X$. Let $n_i \geq 0$
and set $D = \sum n_i D_i$.
Let $\alpha \in Z_k(X)$. Choose representatives
$\alpha_i$ of $c_1(\mathcal{O}_X(D_i)) \cap \alpha$
supported on $\text{Supp}(\alpha) \cap D_i$, see
Lemma \ref{lemma-support-cap-effective-Cartier}.
Then $\sum n_i \alpha _i$ is a cycle and is equal
to $c_1(\mathcal{O}_X(D)) \cap \alpha$.
\end{lemma}

\begin{proof}
Omitted.
\end{proof}

\begin{lemma}
\label{lemma-commutativity-effective-Cartier-proper-intersection}
Let $(S, \delta)$ be as in Sitation \ref{situation-setup}.
Let $X$ be locally of finite type over $S$.
Assume $X$ integral and $\dim_\delta(X) = n$.
Let $D_1$, $D_2$ be effective Cartier divisors on $X$.
Assume $\dim_\delta(D_1 \cap D_2) = n - 2$. Then
$$
c_1(\mathcal{O}_X(D_1)) \cap [D_2]_{n - 1}
=
c_1(\mathcal{O}_X(D_2)) \cap [D_1]_{n - 1}
$$
in $A_{n - 2}(X)$.
\end{lemma}

\begin{proof}
Omitted. Use the canonical sections $1_{D_i}$ and
Lemma \ref{lemma-geometric-cap}.
\end{proof}

\begin{lemma}
\label{lemma-commutativity-effective-Cartier}
Let $(S, \delta)$ be as in Sitation \ref{situation-setup}.
Let $X$ be locally of finite type over $S$.
Assume $X$ integral and $\dim_\delta(X) = n$.
Let $D_1$, $D_2$ be effective Cartier divisors on $X$.
Then
$$
c_1(\mathcal{O}_X(D_1)) \cap [D_2]_{n - 1}
=
c_1(\mathcal{O}_X(D_2)) \cap [D_1]_{n - 1}
$$
in $A_{n - 2}(X)$.
\end{lemma}

\begin{proof}
Apply Lemma \ref{lemma-blowing-up-intersections} to the set $\{D_1, D_2\}$.
Thus we get a proper birational morphism $b : X' \to X$
and a locally finite collection of effective Cartier
divisors $\{D'_j\}_{j \in J}$ of $X'$ such that
$b^{-1}D_1 = \sum n_j D'_j$ and such that
$b^{-1}D_2 = \sum m_j D'_j$ for some $n_j, m_j \geq 0$.
Moreover $\dim_\delta(D'_j \cap D'_{j'}) = n - 2$ whenever
$j \not = j'$.
Whence $[D_1]_{n - 1} = b_* ( \sum n_j[D'_j]_{n - 1} )$
and $[D_2]_{n - 1} = b_* ( \sum m_j[D'_j]_{n - 1} )$.
By the projection formula of Lemma \ref{lemma-pushforward-cap-c1}
and the additivity of Lemmas \ref{lemma-c1-cap-additive}
and \ref{lemma-improved-additivity}
it is enough to show the equalities
$$
c_1(\mathcal{O}_{X'}(D'_j)) \cap [D_{j'}]_{n - 1}
=
c_1(\mathcal{O}_{X'}(D'_{j'})) \cap [D'_j]_{n - 1}
$$
in $A_{n - 2}(X')$. This is trivial if $j = j'$ and
if $j \not = j'$, then we see this from
Lemma \ref{lemma-commutativity-effective-Cartier-proper-intersection}.
\end{proof}








\section{Commutativity}
\label{section-commutativity}



\begin{lemma}
\label{lemma-commutativity-on-integral}
Let $(S, \delta)$ be as in Sitation \ref{situation-setup}.
Let $X$ be locally of finite type over $S$.
Assume $X$ integral and $\dim_\delta(X) = n$.
Let $\mathcal{L}$, $\mathcal{N}$ be invertible on $X$.
Choose a nonzero meromorphic section $s$ of $\mathcal{L}$
and a nonzero meromorphic section $t$ of $\mathcal{N}$.
Set $\alpha = \text{div}_{\mathcal{L}}(s)$ and
$\beta = \text{div}_{\mathcal{N}}(t)$.
Then
$$
c_1(\mathcal{N}) \cap \alpha
=
c_1(\mathcal{L}) \cap \beta
$$
in $A_{n - 2}(X)$.
\end{lemma}

\begin{proof}
By Lemma \ref{lemma-blowing-up-denominators} (applied twice)
there exists a proper morphism
$\pi : X' \to X$ and effective Cartier divisors
$D_1, E_1, D_2, E_2$ on $X'$ such that
$$
b^*\mathcal{L} = \mathcal{O}_{X'}(D_1 - E_1),
\quad
b^*\mathcal{L} = \mathcal{O}_{X'}(D_2 - E_2),
$$
and such that
$$
\alpha = \pi_*([D_1]_{n - 1} - [E_1]_{n - 1}),
\quad
\beta = \pi_*([D_2]_{n - 1} - [E_2]_{n - 1}).
$$
By the projection formula of Lemma \ref{lemma-pushforward-cap-c1}
and the additivity of Lemma \ref{lemma-c1-cap-additive}
it is enough to show the equality
$$
c_1(\mathcal{O}_{X'}(D_1)) \cap [D_2]_{n - 1}
=
c_1(\mathcal{O}_{X'}(D_2)) \cap [D_1]_{n - 1}
$$
and three other similar equalities involving $D_i$ and $E_j$.
Thus the result follows from Lemma \ref{lemma-commutativity-effective-Cartier}.
\end{proof}

\begin{lemma}
\label{lemma-factors}
The operation $\alpha \mapsto c_1(\mathcal{L}) \cap \alpha$
factors through rational equivalence to give an operation
$$
c_1(\mathcal{L}) \cap - :
A_{k + 1}(X) \to A_k(X)
$$
\end{lemma}

\begin{proof}
Omitted. Hint:
Apply Lemma \ref{lemma-commutativity-on-integral}
with one of the invertible sheaves equal to the
trivial one.
\end{proof}

\noindent
For any integer $s \geq 0$ we will denote
$$
c_1(\mathcal{L})^s \cap - : A_{k + s}(X) \to A_k(X)
$$
the $s$-fold iterate of the operation $c_1(\mathcal{L}) \cap - $.
This makes sense by the lemma above.

\begin{lemma}
\label{lemma-cap-commutative}
Let $X$ be a variety.
Let $\mathcal{L}$, $\mathcal{N}$ be invertible sheaves on $X$.
For any $\alpha \in A_{k + 2}(X)$ we have
$$
c_1(\mathcal{L}) \cap c_1(\mathcal{N}) \cap \alpha
=
c_1(\mathcal{N}) \cap c_1(\mathcal{L}) \cap \alpha
$$
as elements of $A_k(X)$.
\end{lemma}

\begin{proof}
Omitted. Hint: use Lemma \ref{lemma-commutativity-on-integral}.
\end{proof}









\section{Affine and vector bundles}
\label{section-affine-vector}

\begin{lemma}
\label{lemma-cap-P1}
Let $X = Y \times \mathbf{P}^1$ with projection morphisms
$p : X \to Y$ and $q : X \to \mathbf{P}^1$.
Let $\mathcal{L}$ be the pullback by $q$ of $\mathcal{O}(1)$ on $\mathbf{P}^1$.
The composition
$$
p_* \circ c_1(\mathcal{L}) \cap - \circ p^*
$$
is the identity morphism on $A_k(Y)$ for all $k$.
\end{lemma}

\begin{proof}
To see this let
$Z \subset X$ be a closed subvariety of $X$. Then it is clear that
$p^*[Z] = [Z \times \mathbf{P}^1]$. Let
$s \in \Gamma(\mathbf{P}^1, \mathcal{O}(1))$ be any nonzero global
section. There is a single rational point $t \in \mathbf{P}^1$
where $s$ vanishes exactly to order $1$. Hence the pullback of
$s$ to a section of $\mathcal{L}$ over $Z \times \mathbf{P}^1$ has
zero scheme exactly $Z \times \{t\}$. In other words, we see that
$$
c_1(\mathcal{L}) \cap p^*[Z] = [Z \times \{t\}]
$$
in $A_k(X)$. Since $p|_{Z \times \{t\}} : Z \times \{t\} \to Z$
is an isomorphism we see that also
$$
p_*(c_1(\mathcal{L}) \cap p^*[Z]) = [Z]
$$
in $A_k(Y)$ as desired.
\end{proof}

\begin{lemma}
\label{lemma-affine-over-variety}
The projection $p : Y \times \mathbf{A}^1 \to Y$ induces an isomorphism
$$
p^* : A_k(Y) \to A_{k + 1}(Y \times \mathbf{A}^1)
$$
More generally $A_k(Y) \cong A_{k + r}(Y \times \mathbf{A}^r)$
for all $r \geq 0$.
\end{lemma}

\begin{proof}
Hints: Argue directly this map is surjective.
Denote $\overline{p} : Y \times \mathbf{P}^1 \to Y$ the projection
morphism. Suppose that $p^*\alpha = 0$. Then $\overline{p}^*\alpha$
restricts to zero on $Y \times \mathbf{A}^1$, hence is rationally
equivalent to a cycle $\beta$ supported in $Y \times \{\infty\}$ by
Lemma \ref{lemma-restrict-to-open}.
Choose $s \in \Gamma(\mathbf{P}^1, \mathcal{O}(1))$
whose unique zero is $0 \in \mathbf{P}^1$. Since the zero set of $s$
misses $\beta$ we conclude that $c_1(\mathcal{O}(1)) \cap \beta = 0$.
Hence $p_*(c_1(\mathcal{O}(1)) \cap \beta) = 0$. This contradicts
Lemma \ref{lemma-cap-P1}.
\end{proof}


\begin{lemma}
\label{lemma-relative-effective-cartier}
Let $X$, $Y$ be varieties.
Let $p : X \to Y$ be a flat morphism of relative dimension $r$.
Let $i : D \to X$ be an effective Cartier divisor such that
$p|_D : D \to Y$ is flat of relative dimension $r - 1$. Assume that
$D$ is a variety. Let $\mathcal{L} = \mathcal{O}_X(D)$.
Then
$$
c_1(\mathcal{L}) \cap p^*\alpha = i_* ((p|_D)^*\alpha)
$$
in $A_{k + r - 1}(X)$ for every $\alpha \in A_k(Y)$.
\end{lemma}

\begin{proof}
Let $Z \subset Y$ be a closed subvariety of dimension $k$.
Then $p^{-1}(Z) \subset X$ is a closed subscheme of dimension
$k + r$. Also the scheme theoretic intersection $f^{-1}(Z) \cap D$ is
the zero scheme of the canonical section of $\mathcal{L} = \mathcal{O}_X(D)$
restricted to $f^{-1}(Z)$. Moreover, $f^{-1}(Z) \cap D \to Z$ is an
effective Cartier divisor by the algebra lemma below. Hence
Lemma \ref{lemma-geometric-cap} applies and we see that
$$
c_1(\mathcal{L}) \cap [f^{-1}(Z)]_{k + r} =
[f^{-1}(Z) \cap D]_{k + r - 1}.
$$
Clearly the cycle $[f^{-1}(Z) \cap D]_{k + r - 1}$ is the
pushforward of the cycle $[(p|_D)^{-1}(Z)]_{k + r - 1}$ on $D$.
The result follows from the definition of flat pullback.
\end{proof}

\begin{lemma}
\label{lemma-relative-effective-cartier-algebra}
Let $A \to B$ be a ring map. Let $f \in B$. Assume that
\begin{enumerate}
\item $A \to B$ is flat,
\item $f$ is a nonzero divisor, and
\item $A \to B/fB$ is flat.
\end{enumerate}
Then for every ideal $I \subset A$ the map
$f : B/IB \to B/IB$ is injective.
\end{lemma}

\begin{proof}
Note that $IB = I \otimes_A B$ and $I(B/fB) = I\otimes_A B/fB$
by the flatness of $B$ and $B/fB$ over $A$.
In particular $IB/fIB \cong I \otimes_A B/fB$ maps injectively
into $B/fB$. Hence the result follows from the snake lemma applied
to the diagram
$$
\xymatrix{
0 \ar[r] &
I \otimes_A B \ar[r] \ar[d]^f &
B \ar[r] \ar[d]^f &
B/IB \ar[r] \ar[d]^f &
0 \\
0 \ar[r] &
I \otimes_A B \ar[r] &
B \ar[r] &
B/IB \ar[r] &
0
}
$$
with exact rows.
\end{proof}




\section{Projective space bundle formula}
\label{section-projective-space-bundle-formula}

\noindent
Consider a variety $X$ and a finite locally free
${\mathcal O}_X$-module ${\mathcal E}$ of rank $r$.
Our convention is that
$$
\xymatrix{
{\bf P}({\mathcal E}) = Proj(Sym^*({\mathcal E}))
\ar[r]^-\pi
& X
}
$$
over $X$ with ${\mathcal O}_{{\bf P}({\mathcal E}}(1)$ normalized so that
$\pi_*({\mathcal O}_{{\bf P}({\mathcal E})}(1)) = {\mathcal E}$.
In particular there is a surjection
$\pi^*{\mathcal E} \to {\mathcal O}_{{\bf P}({\mathcal E})}(1)$.

\begin{lemma}
\label{lemma-cap-projective-bundle}
With notation as above the composition
$$
\pi_* \circ c_1(\mathcal{O}_{\mathbf{P}(\mathcal{E})}(1))^s \cap -
\circ \pi^*
$$
is $0$ if $s < r - 1$ and is the identity on $A_k(X)$ for all $k$
when $s = r - 1$.
\end{lemma}

\begin{proof}
Let $Z \subset X$ be a subvariety of dimension $k$.
Note that $\pi^*[Z] = [\mathbf{P}(\mathcal{E}|_Z)]$.
If $s < r - 1$, then by construction
$c_1(\mathcal{O}_{\mathbf{P}(\mathcal{E})}(1))^s \cap \pi^*[Z]$
is represented by a $k + r - 1 - s$ cycle supported on
$\mathbf{P}(\mathcal{E}|_Z)$. Hence the pushforward of this cycle
is zero for dimension reasons.
Let $\xi$ be the generic point of $Z$.
We can choose elements $e_1, \ldots, e_{r - 1} \in \mathcal{E}_\xi$
which form part of a basis of $\mathcal{E}_\xi$. These give rational
sections of $\mathcal{O}(1)$ over $\mathbf{P}(\mathcal{E}|_Z)$
whose common zero set is the closure of the image a rational section of
$\mathbf{P}(\mathcal{E}|_Z) \to Z$ union a closed subset whose
support maps to a proper closed subset of $Z$. This implies that
$$
\pi_* c_1(\mathcal{O}_{\mathbf{P}(\mathcal{E})}(1))^{r - 1} \cap \pi^*[Z]
=
[Z].
$$
Details omitted.
\end{proof}

\begin{lemma}
\label{lemma-chow-ring-projective-bundle}
With notation as above the map
\begin{align*}
\bigoplus\nolimits_{i=0}^{r-1} A_{k + i}(X)
& \longrightarrow
A_k({\bf P}({\mathcal E})), \\
(\alpha_0,\ldots,\alpha_{r-1}) &
\longmapsto
\pi^*\alpha_0 +
c_1(\mathcal{O}_{\mathbf{P}(\mathcal{E})}(1)) \cap \pi^*\alpha_1
+ \ldots + \\
&
\quad \quad \quad \quad
\quad \quad \quad \quad
\quad \quad \quad \quad
c_1(\mathcal{O}_{\mathbf{P}(\mathcal{E})}(1))^{r - 1} \cap \pi^*\alpha_{r-1}
\end{align*}
is an isomorphism where $r = rank({\mathcal E})$.
\end{lemma}

\begin{proof}
By the result of the previous lemma this map is injective.
We have to show it is surjective. We do this by induction
on $\dim(X)$. Let $\beta \in A_k(\mathbf{P}(\mathcal{E}))$.
If for some nonempty open $U \subset X$ we have $\beta|_U \sim_{rat} 0$
then by Lemma \ref{lemma-restrict-to-open} we see that $\beta$
is rationally equivalent to
a sum of $k$-cycles on $\mathbf{P}(\mathcal{E}|_{Z_i})$ for proper
closed subvarieties $Z_i \subset X$. By induction the result holds
for $\mathbf{P}(\mathcal{E}|_{Z_i}) \to Z_i$ and hence the
result holds for $\beta$. Hence we may replace $X$ by any nonempty
open of $X$.

\medskip\noindent
In particular we may assume that $\mathcal{E} \cong \mathcal{O}_X^{\oplus r}$.
In this case $\mathbf{P}(\mathcal{E}) = X \times \mathbf{P}^{r - 1}$.
Consider the open standard affine open
$\mathbf{A}^{r - 1} \subset \mathbf{P}^{r - 1}$ with complement
$\mathbf{P}^{r - 2} \to \mathbf{P}^{r - 1}$.
This gives an open subset $X \times \mathbf{A}^{r - 1}$ and a
closed immersion
$i : X \times \mathbf{P}^{r - 2} \to X \times \mathbf{P}^{r - 1}$.
The restriction of $\beta$ to the open subset is rationally equivalent
to a pullback of a cycle $\alpha_0$ from $X$. After replacing
$\beta$ by $\beta - \pi^*\alpha_0$ we see that $\beta$ restricts
to zero on $X \times \mathbf{A}^{r - 1}$. Hence $\beta$ is
of the form $i_*\beta'$ for some $\beta' \in A_k(X \times \mathbf{P}^{r - 2})$
by Lemma \ref{lemma-restrict-to-open}. By induction on $r$ we see that
$$
\beta' =
\pi^*\alpha'_0 +
c_1(\mathcal{O}_{\mathbf{P}^{r - 2}}(1)) \cap \pi^*\alpha'_1 + \ldots +
c_1(\mathcal{O}_{\mathbf{P}^{r - 2}}(1))^{r - 2} \cap \pi^*\alpha'_{r-2}
$$
The relation in Lemma \ref{lemma-relative-effective-cartier}
and the fact that
$\mathcal{O}_{\mathbf{P}^{r - 1}}(\mathbf{P}^{r - 2})
\cong \mathcal{O}_{\mathbf{P}^{r - 1}}(1)$ implies
that
$$
i_*\beta' =
c_1(\mathcal{O}_{\mathbf{P}^{r - 1}}(1)) \cap \pi^*\alpha'_0 +
c_1(\mathcal{O}_{\mathbf{P}^{r - 1}}(1))^2 \cap \pi^*\alpha'_1 + \ldots +
c_1(\mathcal{O}_{\mathbf{P}^{r - 1}}(1))^{r - 1} \cap \pi^*\alpha'_{r-2}
$$
as desired.
\end{proof}

\begin{lemma}
Let $X$ be a variety.
Let $\mathcal{E}$ be a finite locally free sheaf of rank $r$ on $X$.
Let
$$
p :
E = \underline{\text{Spec}}(\text{Sym}^*(\mathcal{E}))
\longrightarrow
X
$$
be the associated vector bundle over $X$.
Then $p^* : A_k(X) \to A_{k + r}(E)$ is an isomorphism for all $k$.
\end{lemma}

\begin{proof}
Consider the open immersion
$j : E \to \mathbf{P}(\mathcal{E} \oplus \mathcal{O}_X)$.
Note that 
$$
j^*\mathcal{O}_{\mathbf{P}(\mathcal{E} \oplus \mathcal{O}_X)}(1)
\cong \mathcal{O}_E.
$$
The map $j^*$ is surjective (Lemma \ref{lemma-exact-sequence-open}).
By the projective space bundle formula every
cycle on $\mathbf{P}(\mathcal{E} \oplus \mathcal{O}_X)$ is a
sum of terms of the form 
$$
c_1(\mathcal{O}_{\mathbf{P}(\mathcal{E} \oplus \mathcal{O}_X)}(1))^s
\cap
\pi^*\alpha_s
$$
By the displayed equality for the pullback of this invertible sheaf
to $E$ and Lemma \ref{lemma-flat-pullback-divisor-invertible-sheaf}
we see that all of these restrict to zero on $E$ except those where
$s = 0$. We conclude that $p^*$ is surjective.

\medskip\noindent
For injectivity we use that the kernel of
$$
j^* :
A_{k + r}(\mathbf{P}(\mathcal{E} \oplus \mathcal{O}_X))
\longrightarrow
A_{k + r}(E)
$$
are the cycles supported in the hyperplace $\mathbf{P}(\mathcal{E})$
at infinity and the fact that the cycles supported there map to cycles
whose expression (in the projective space bundle formula) is a sum of
terms as above with $s \geq 1$ (see proof of projective space bundle
formula). Details omitted.
\end{proof}








\section{The Chern classes of a vector bundle}
\label{section-chern-classes-vector-bundles}

\noindent
Let $X$ be a variety. Let $\mathcal{E}$ be a finite locally free sheaf
of rank $r$ as follows. We define the Chern classes of $\mathcal{E}$
as follows. By Lemma \ref{lemma-chow-ring-projective-bundle} there are
elements $c_i \in A_{\dim(X) - i}(X)$, $i=0, \ldots, r$
such that $c_0 = [X]$, and
\begin{equation}
\label{equation-chern-classes}
-(-1)^r c_1(\mathcal{O}_{\mathbf{P}(\mathcal{E})}(1))^r \cap
[\mathbf{P}(\mathcal{E})]
=
\sum\nolimits_{i = 0}^{r - 1}
(-1)^i c_1(\mathcal{O}_{\mathbf{P}(\mathcal{E})}(1))^i \cap \pi^*c_{r - i}.
\end{equation}
The reason for the signs will be explained later.

\begin{definition}
\label{definition-chern-classes}
With notation as above we set
$$
c_i(\mathcal{E}) \cap [X] = c_i
$$
as an element of $A_{\dim(X) - i}(X)$.
We call these the {\it chern classes of $\mathcal{E}$}.
The {\it total chern class} of 
${\mathcal E}$ is the element
$$
c({\mathcal E}) \cap [X] = 
c_0({\mathcal E}) \cap [X] + c_1({\mathcal E}) \cap [X] + \ldots
+ c_r({\mathcal E}) \cap [X].
$$
\end{definition}

\noindent
Note that the first chern class of an invertible
sheaf is unchanged, since if $\mathcal{E}$ has rank $1$ we have
$\mathbf{P}(\mathcal{E}) = X$ and Equation (\ref{equation-chern-classes})
reads
$$
-(-1)^1 c_1(\mathcal{E}) \cap [X] = (-1)^0 c_1
$$
whence $c_1 = c_1(\mathcal{E}) \cap [X]$.




\section{Intersecting with chern classes}
\label{section-intersecting-chern-classes}

\noindent
Let $X$ be a variety. Let $\mathcal{E}$ be a finite locally free sheaf on $X$
of rank $r$. We will define, for every integer $k$ and any
$0 \leq j \leq r$, an operation
$$
c_j(\mathcal{E}) \cap - :
Z_{k + j}(X) \to A_k(X)
$$
Namely, given a $(k + j)$-dimensional closed subvariety
$i : W \to X$ we may define
$$
c_j(\mathcal{E}) \cap [W] = i_*(c_j({i^*\mathcal{E}}) \cap [W])
$$
where $c_j({i^*\mathcal{E}}) \cap [W]$ is as defined in
Section \ref{section-chern-classes-vector-bundles}.
For a general $(k + j)$-cycle $\alpha = \sum n_i [W_i]$ we set
$$
c_j(\mathcal{E}) \cap \alpha = \sum n_i c_j(\mathcal{E}) \cap [W_i]
$$
Again, if $\mathcal{E}$ has rank $1$ then this agrees with our
previous definition.

\begin{lemma}
\label{lemma-determine-intersections}
Let $X$ be a variety.
Let $\mathcal{E}$ be a finite locally free sheaf of rank $r$ on $X$.
Let $\pi : \mathbf{P}(\mathcal{E}) \to X$ be the projective bundle
associated to $\mathcal{E}$.
For $\alpha \in Z_k(X)$ the elements
$c_j(\mathcal{E}) \cap \alpha$ are the unique elements
$\alpha_j$ of $A_{k - j}(X)$
such that
$$
-(-1)^r c_1(\mathcal{O}_{\mathbf{P}(\mathcal{E})}(1))^r \cap \pi^*\alpha
=
\sum\nolimits_{i = 0}^{r - 1}
(-1)^i c_1(\mathcal{O}_{\mathbf{P}(\mathcal{E})}(1))^i \cap
\pi^*(\alpha_{r - i})
$$
holds in the Chow group of $\mathbf{P}(\mathcal{E})$.
\end{lemma}

\begin{proof}
Omitted. Hint: this is true by definition for
$\alpha = [X]$ and follows by Lemma \ref{lemma-pushforward-cap-c1}
in general.
\end{proof}

\noindent
This characterization of chern classes allows us to prove many more
properties. In particular the lemma implies that
if $\alpha \sim_{rat} \beta$, then
$c_j(\mathcal{E}) \cap \alpha = c_j(\mathcal{E}) \cap \beta$
in $A_{k - j}(X)$. In other words capping with chern classes of
finite locally free sheaves factors through rational equivalence
to give maps
$$
c_j(\mathcal{E}) \cap - :
A_{k + j}(X) \to A_k(X).
$$

\begin{lemma}
\label{lemma-flat-pushback-cap-cj}
Let $p : X \to Y$ be a flat morphism of varieties of relative dimension $r$.
Let $\alpha$ be a $k$-cycle on $Y$.
Let $\mathcal{E}$ be a finite locally free sheaf on $Y$.
Then
$$
p_*(c_j(\mathcal{E}) \cap \alpha) = c_j(p^*\mathcal{E}) \cap p_*\alpha
$$
\end{lemma}

\begin{proof}
Omitted.
\end{proof}

\begin{lemma}
\label{lemma-pushforward-cap-cj}
Let $p : X \to Y$ be a proper morphism of varieties.
Let $\alpha$ be a $(k + j)$-cycle on $X$.
Let $\mathcal{E}$ be a finite locally free sheaf on $Y$.
Then
$$
p_*(c_j(p^*\mathcal{E}) \cap \alpha) = c_j(\mathcal{E}) \cap p_*\alpha
$$
\end{lemma}

\begin{proof}
Omitted.
\end{proof}

\begin{lemma}
\label{lemma-cap-commutative-chern}
Let $X$ be a variety.
Let $\mathcal{E}$, $\mathcal{F}$ be finite locally free sheaves on $X$
of ranks $r$ and $s$.
For any $\alpha \in A_{k + i + j}(X)$ we have
$$
c_i(\mathcal{E}) \cap c_j(\mathcal{F}) \cap \alpha
=
c_j(\mathcal{F}) \cap c_i(\mathcal{E}) \cap \alpha
$$
as elements of $A_k(X)$.
\end{lemma}

\begin{proof}
Consider
$$
\pi : \mathbf{P}(\mathcal{E}) \times_X \mathbf{P}(\mathcal{F})
\longrightarrow
X
$$
with invertible sheaves
$\mathcal{L} = \text{pr}_1^*\mathcal{O}_{\mathbf{P}(\mathcal{E})}(1)$
and
$\mathcal{N} = \text{pr}_2^*\mathcal{O}_{\mathbf{P}(\mathcal{F})}(1)$.
Write $\alpha_{i, j}$ for the left hand side and $\beta_{i, j}$
for the right hand side. Also write
$\alpha_j = c_j(\mathcal{F}) \cap \alpha$ and 
$\beta_i = c_i(\mathcal{E}) \cap \alpha$.
From Lemma \ref{lemma-determine-intersections}
(pulled back to $X$ for the first two)
and the fact that intersecting with $c_1(\mathcal{L})$ and
$c_1(\mathcal{N})$ commute (Lemma \ref{lemma-cap-commutative})
and Lemma \ref{lemma-flat-pullback-cap-c1}
we see successively that
\begin{align*}
-(-1)^s c_1(\mathcal{N})^s \cap \pi^*\alpha
& =
\sum (-1)^j c_1(\mathcal{N})^j \cap \pi^*\alpha_j \\
-(-1)^r c_1(\mathcal{L})^r \cap \pi^*\alpha
& =
\sum (-1)^i c_1(\mathcal{L})^i \cap \pi^*\beta_i \\
(-1)^{r + s}c_1(\mathcal{L})^r \cap c_1(\mathcal{N})^s \cap \pi^*\alpha
& =
\sum
(-1)^{i + j}
c_1(\mathcal{L})^i \cap c_1(\mathcal{N})^j \cap \pi^*\alpha_{i, j} \\
(-1)^{r + s}c_1(\mathcal{N})^s \cap c_1(\mathcal{L})^r \cap \pi^*\alpha
& =
\sum
(-1)^{i + j}
c_1(\mathcal{N})^j \cap c_1(\mathcal{L})^i \cap \pi^*\beta_{i, j}
\end{align*}
By the projective space bundle formula (applied twice) these representations
are unique. Whence the result.
\end{proof}









\section{Polynomial relations among chern classes}
\label{section-relations-chern-classes}

\noindent
Let $P(x_{i, j}) \in \mathbf{Z}[x_{i, j}]$ be a polynomial.
We may write $P$ as a finite sum
$$
\sum\nolimits_s
\sum\nolimits_{I = ((i_1, j_1), (i_2, j_2), \ldots, (i_s, j_s))}
a_I x_{i_1, j_1} \ldots x_{i_s, j_s}.
$$
Let $X$ be a variety. Let $\mathcal{E}_i$ be a finite collection
of polynomials. We say that $P(c_j(\mathcal{E}_i)) = 0$
if for any morphism of varieties $f : Y \to X$ the cycle
$$
\sum\nolimits_s
\sum\nolimits_{I = ((i_1, j_1), (i_2, j_2), \ldots, (i_s, j_s))}
a_I\ c_{j_1}(f^*\mathcal{E}_{i_1}) \cap \ldots
\cap c_{j_s}(f^*\mathcal{E}_{i_s}) \cap [Y]
$$
is zero in $A_*(Y)$. This is not an elegant definition but it will do
for now. It makes sense because we showed in
Lemma \ref{lemma-cap-commutative-chern} that
capping with chern classes of vector bundles is is commutative.
By our definitions and results above
this is equivalent with requiring all the
operations
$$
\sum\nolimits_s
\sum\nolimits_I
a_I\ c_{j_1}(f^*\mathcal{E}_{i_1}) \cap \ldots
\cap c_{j_s}(f^*\mathcal{E}_{i_s}) \cap - :
A_*(Y) \to A_*(Y)
$$
to be zero for all varieties $Y$.

\medskip\noindent
An example of such a relation is the relation
$$
c_1(\mathcal{L} \otimes_{\mathcal{O}_X} \mathcal{N})
=
c_1(\mathcal{L}) + c_1(\mathcal{N})
$$
proved in Lemma \ref{lemma-c1-cap-additive}. More generally,
suppose that ${\mathcal E}$ is a finite locally free sheaf of
rank $r$ and ${\mathcal L}$ is and invertible sheaf.
In this case there is a canonical morphsm
$$
\xymatrix{
{\bf P}({\mathcal E})\ar[rd] \ar[rr]_g
&
&
{\bf P}({\mathcal E} \otimes \mathcal{L})\ar[ld]
\cr
& X & 
\cr}
$$
such that $g^*{\mathcal O}_{{\bf P}({\mathcal E} \otimes \mathcal{L})}(1) 
= {\mathcal O}_{{\bf P}({\mathcal E})}(1) \otimes \pi^* {\mathcal L}$.
This means that we have
$$
-(-1)^r(\xi + x)^r \cap [\mathbf{P}(\mathcal{E})] =
\sum\nolimits_{i = 0}^{r - 1}
(-1)^i
(\xi + x)^i \cap \pi^*(c_{r - i}(\mathcal{E} \otimes \mathcal{L}) \cap [X])
$$
in $A_*(\mathbf{P}(\mathcal{E}))$, where $\xi$ represents
$c_1(\mathcal{O}_{\mathbf{P}(\mathcal{E})}(1))$ and $x$
represents $c_1(\pi^*\mathcal{L})$. By simple algebra this
is equivalent to
$$
-(-1)^r \xi^r \cap [\mathbf{P}(\mathcal{E})] =
\sum\nolimits_{i = 0}^{r - 1}
(-1)^i \xi^i \left(
\sum\nolimits_{j = i}^r
(-1)^{j - i}
\binom{j}{i}
x^{j - i} \cap
\pi^*(c_{r - j}(\mathcal{E} \otimes \mathcal{L}) \cap [X])
\right)
$$
Comparing with
Equation (\ref{equation-chern-classes}) it follows from this that
$$
c_{r - i}(\mathcal{E}) \cap [X] =
\sum\nolimits_{j = i}^r
\binom{j}{i}
(-c_1(\mathcal{L}))^{j - i} \cap
c_{r - j}(\mathcal{E} \otimes \mathcal{L}) \cap [X]
$$
This holds for any triple $(X, \mathcal{E}, \mathcal{L})$. Hence
it holds universally and gives a polynomial relation as above.
Reworking this (getting rid of minus signs, and renumbering) we see that
\begin{equation}
\label{equation-twist}
c_i({\mathcal E}\otimes {\mathcal L})
=
\sum\nolimits_{j = 0}^i
\binom{r - i + j}{j} c_{i - j}({\mathcal E}) c_1({\mathcal L})^j
\end{equation}
is a valid polynomial relation in the sense described above.
Some example cases are
\begin{align*}
c_1(\mathcal{E} \otimes \mathcal{L})
& =
c_1(\mathcal{E}) +
r c_1(\mathcal{L}) \\
c_2(\mathcal{E} \otimes \mathcal{L})
& =
c_2(\mathcal{E}) +
(r - 1) c_1(\mathcal{E}) c_1(\mathcal{L}) +
\binom{r}{2} c_1(\mathcal{L})^2 \\
c_3(\mathcal{E} \otimes \mathcal{L})
& =
c_3(\mathcal{E}) +
(r - 2) c_2(\mathcal{E})c_1(\mathcal{L}) + 
\binom{r - 1}{2} c_1(\mathcal{E})c_1(\mathcal{L})^2 +
\binom{r}{3} c_1(\mathcal{L})^3
\end{align*}








\section{The chern classes of the sum of invertible sheaves}
\label{section-chern-sum-linebundles}

\noindent
Let $X$ be a variety and let ${\mathcal L}_i$,
$i = 1, \ldots, r$ be invertible ${\mathcal O}_X$-modules on $X$. Set
$c_1({\mathcal L}_i) = x_i$. We claim that
$$
c({\mathcal L}_1 \oplus \ldots \oplus {\mathcal L}_r)
=
\prod\nolimits_{i = 1}^r (1 + x_i) 
$$
is a valid polynomial relation among chern classes in the sense of
Section \ref{section-relations-chern-classes}.

\medskip\noindent
By the polynomial relation (\ref{equation-twist}) and some easy
algebra we see that it suffices to prove the relation in case
$\mathcal{L}_r \cong \mathcal{O}_X$. In this case we have
$$
i : D = \mathbf{P}({\mathcal L}_1 \oplus \ldots \oplus {\mathcal L}_{r - 1})
\to
\mathbf{P}({\mathcal L}_1 \oplus \ldots \oplus {\mathcal L}_r)
$$
representing $c_1(\mathcal{O}(1))$. By
Lemma \ref{lemma-relative-effective-cartier} we see that
$$
-(-1)^r c_1(\mathcal{O}(1))^r \cap
[\mathbf{P}({\mathcal L}_1 \oplus \ldots \oplus {\mathcal L}_r)]
=
i_* (-1)^{r + 1} c_1(\mathcal{O}(1))^{r - 1} \cap
[\mathbf{P}({\mathcal L}_1 \oplus \ldots \oplus {\mathcal L}_{r - 1})]
$$
Hence it follows that
$$
c_j({\mathcal L}_1 \oplus \ldots \oplus
\mathcal{L}_{r - 1} \oplus {\mathcal O}_X)
=
c_j({\mathcal L}_1 \oplus \ldots \oplus {\mathcal L}_{r - 1})
$$
for all $j$. This argument works on all $Y$ mapping into $X$ in other
words we find the polynomial relation
We win by induction on $r$.



\section{Intersection multiplicities using Tor formula}
\label{section-tor-formula}

\noindent
Suppose that $X$ is a {\bf nonsingular} variety and that
$W,V \subset X$ are closed subvarieties with $\dim(W)=s$ and $\dim(V)=r$.
Assume that $\dim(W \cap V) \leq r + s - \dim(X)$. We say that $W$ and $V$
{\it intersect properly} if this holds. In this case the
sheaves $Tor_j^{{\mathcal O}_X}({\mathcal O}_W, {\mathcal O}_V)$ are coherent,
supported on $V\cap W$, and zero if $j < 0$ or $j > \dim(X)$.
In this situation we define
$$
W \cdot V =
\sum\nolimits_i (-1)^i
[Tor_i^{{\mathcal O}_X}({\mathcal O}_W, {\mathcal O}_V)]_{r+s-\dim(X)}.
$$
With this notation, the cycle $V \cdot W$ is a formal linear
combination $\sum e_i Z_i$ of the irreducible components $Z_i$
of the intersection $V \cap W$. The integers $e_i$ are called
the {\it intersection multiplicities}
$$
e_i = e(X, V \cdot W, Z_i).
$$
They satisfy many good properties, see [Serre].

\begin{lemma}
\label{lemma-formula-leading-coeff}
Notation as above.
Let $Z = Z_i$ be one of the irreducible components.
Let $A$ be the local ring of $X$ at the generic point
of $Z$. Suppose that the ideal of $V$ in $A$ is cut out by
a regular sequence $x_1,\ldots,x_c$ and suppose that the local
ring of $W$ at the generic point of $Z$ corresponds to the quotient
map  $A \to B$. In this case $e(X, V\cdot W, Z)$ is equal to
$c!$ times the leading coefficient in the Hilbert polynomial
$$
t \mapsto \hbox{length}_A(B/(x_1,\ldots,x_c)^tB), \ \ t>>0.
$$
\end{lemma}

\begin{proof}
Omitted.
\end{proof}

\begin{lemma}
\label{lemma-e-when-divisor}
Consider the case that $c=1$, i.e., $V$ is a(n effective) Cartier divisor.
Then $x_1$ is a nonzero divisor on $B$ by properness of intersection
of $V$ and $W$. Then
$$
e(X, V \cdot W, Z) = \hbox{length}_A(B/x_1B).
$$
\end{lemma}

\begin{proof}
Omitted.
\end{proof}


\section{Intersection product using Tor formula}
\label{section-tor-formula-intersection-product}

\noindent
Suppose that $X$ is a {\bf nonsingular} variety. Suppose
$\alpha = \sum n_i [W_i]$ is an $r$-cycle,
and $\beta = \sum_j m_j [V_j]$ is an $s$-cycle on $X$. We say
that $\alpha$ and $\beta$ {\it intersect properly} if 
$W_i$ and $V_j$ intersect properly for all $i$ and $j$.
In this case we define
$$
\alpha \cdot \beta = \sum_{i,j} n_i m_j W_i \cdot V_j.
$$
where $W_i \cdot V_j$ is as defined above using the Tor-formula.

\begin{lemma}
\label{lemma-tor-sheaves}
Situation as above.
Suppose ${\mathcal F}$ and ${\mathcal G}$ are coherent sheaves on $X$ with
$\dim(Supp({\mathcal F})) \leq s$, $\dim(Supp({\mathcal G})) \leq r$
and $\dim( Supp({\mathcal F}) \cap Supp({\mathcal G}) ) \leq r+s-\dim X$.
In this case
$$
[{\mathcal F}]_s \cdot [{\mathcal G}]_r = \sum (-1)^i 
[ Tor_i^{{\mathcal O}_X}({\mathcal F}, {\mathcal G}) ]_{r+s-\dim X}.
$$
\end{lemma}

\begin{proof}
See [Serre, Chapter V].
\end{proof}


\begin{lemma}
\label{lemma-associativity}
Let $X$ be a nonsingular variety.
Let $U, V, W$ be closed subvarieties. Assume that
$(U, V)$ and $([U] \cdot [V], W)$ intesect properly.
Then
$$
([U] \cdot [V]) \cdot [W] = [U] \cdot ( [V] \cdot [W] )
$$
\end{lemma}

\begin{proof}
Omitted.
\end{proof}



\section{Exterior Product}
\label{section-exterior-product}

\noindent
Let $X$ and $Y$ be nonsingular projective vareities.
Let $V$, resp.\ $W$ be a closed subvariety of $X$, resp.\ $Y$.
The product $V\times W$ is a closed subscheme of $X\times Y$.
It is a subvariety because the ground field is algebraically closed.
For a $k$-cycle $\alpha = \sum n_i [V_i]$ and a $l$-cycle
$\beta = \sum m_j [V_j]$ on $Y$ we define
$\alpha \times \beta = \sum n_i m_j [V_i \times W_j]$. 

\medskip\noindent
Consider the subvariety $X \subset X$ with class $[X]$.
Note that $pr_Y^*(\beta) = [X]\times \beta$.
Note that $\alpha \times [Y]$ and $[X]\times \beta$
intersect properly on $X\times Y$. With the definitions
above we have $\alpha \times \beta = 
(\alpha \times [Y])\cdot ([X]\times\beta) =
pr_Y^*(\alpha) \cdot pr_X^*(\beta)$.




\section{Reduction to the diagonal}
\label{section-reduction-diagonal}

\noindent
Let $X$ be a nonsingular projective
variety. Let $\Delta \subset X \times X$ denote the diagonal. 
We will identify $\Delta$ with $X$.
Let $\alpha$, resp.\ $\beta$ be $r$-cycles, resp.\ $s$-cycles
on $X$. Assume $\alpha$ and $\beta$ intersect properly. In this case
$\alpha \times \beta$ and $[\Delta]$ intersect properly. Note that
the cycle $\Delta \cdot \alpha\times\beta$ is supported on the diagonal
and hence we can think of it as a cycle on $X$. With this convention
we have $\alpha \cdot \beta = \Delta \cdot \alpha\times\beta$.
See [Serre, Chapter V]. 

\medskip\noindent
Perhaps a less confusing formulation would be that
$pr_{1,*}( \Delta \cdot \alpha\times\beta ) = \alpha\cdot\beta$,
where $pr_1 : X\times X \to X$ is the projection.




\section{Flat pullback and intersection products}
\label{section-flat-pullback-intersections}

\noindent
Suppose that $f : X \to Y$ is a flat morphism of nonsingular
projective varieties. Suppose that $\alpha$ is a $k$-cycle
on $Y$ and that $\beta$ is a $l$-cycle on $Y$. Assume that
$\alpha$ and $\beta$ intersect properly. Then $f^*\alpha$
and $f^*\beta$ intersect properly and $f^*( \alpha \cdot \beta )
= f^*\alpha \cdot f^*\beta$. This is not hard to see from
the material above.




\section{Projection formula for flat maps}
\label{section-projection-formula-flat}

\noindent
Let $f : X \to Y$ be a flat morphism
of relative dimension $r$ of nonsingular projective varieties.
Let $\alpha$ be an $k$-cycle on $X$ and let $\beta$ be a $l$-cycle
on $Y$. Assume that $f_*(\alpha)$ and $\beta$ intersect properly,
and that $\alpha$ and $f^*(\beta)$ intersect properly. The projection
formula says that $f_*(\alpha) \cdot \beta =
f_*( \alpha \cdot f^*\beta)$ in this case.
See [Serre, Chapter V, Section 7, formula (10)] for a more general formula. 

\medskip\noindent
We explain how to prove the projection formula in the flat case.
Let $W \subset X$ be a closed subvariety of dimension $k$.
Let $V \subset Y$ be a closed subvariety of dimension $l$,
so $f^{-1}(V)$ has pure dimension $l+r$.
Assume that $W$ and $[f^{-1}(V)]$ intersect properly. Note that
$f(W \cap f^{-1}(V)) = f(W) \cap V$. Hence it follows
that $f(W)$ and $V$ intersect properly as well. Let
$Z \subset f(W) \cap V$ be an irreducible component
of dimension $k+l-\dim Y$. Let $Z_i \subset W\cap f^{-1}(V)$
be the irreducible components of $W\cap f^{-1}(V)$
dominating $Z$. 
Let $A$ be the local ring of $X$ at the generic point of $Z$.
Let $A_i$ be the local ring of $Y$ at the generic point of $Z_i$.
Let $B$ be the local ring of $f(W)$ at the generic point of $Z$.
Let $B'$ be the stalk of $f_*({\mathcal O}_W)$ at the generic point
of $Z$. Then $B \to B'$ is finite, $B'$ is semi-local, and the
localizations $B'_i$ of $B'$ are the local rings of $W$ at the generic
point of the $Z_i$. Thus they are quotients $A_i \to B_i'$.
Let $C$ be the local ring of $V$ at the generic point of $Z$.
The multiplicity of $Z$ in $f_*([W])\cdot V$ is by definition
$$
(I) = [B' : B] \sum (-1)^j \hbox{length}_{A}(Tor^A_j(B,C)).
$$
Here $[B':B]$ is the rank of the $B$-module $B'$.
The multiplicity of $Z$ in $f_*( W \cdot f^*[V])$ is
by definition
$$
(II) = \sum\nolimits_{i,j} (-1)^j
\hbox{length}_{A_i}(Tor^{A_i}_j(B'_i, A_i\otimes_A C))
[\kappa(A_i):\kappa(A)]
$$
Here $\kappa(-)$ indicates the residue field.
The first thing is to note that
$\hbox{length}_A(M) = [\kappa(A_i):\kappa(A)]\hbox{length}_{A_i}(M)$
for a finite length $A_i$-module $M$.
We can compute all the Tor groups by choosing a free resolution
of $C$ as an $A$-module. Doing this it is easy to see that
(I) equals $\sum (-1)^j \hbox{length}_{A}(Tor^A_j(B', C))$.
Finally, note that, by definition, there is an $A$-module map
$B^{\oplus [B':B]} \to B'$ whose kernel and cokernel are
supported in a proper closed subset of $Spec(B)$. From the
additivity properties of the Tor-formula, see [Serre, Chapter V],
it follows that $\sum (-1)^j \hbox{length}_{A}(Tor^A_j(B', C)) =
[B':B] \sum (-1)^j \hbox{length}_{A}(Tor^A_j(B, C))$ as desired.

\section{Rational Equivalence Revisted}
\label{section-equivalence-revisited}

\noindent
Let $X$ be a nonsingular projective variety.
Let $\alpha = \sum n_i [W_i]$ be a $(k+1)$-cycle on $X \times {\bf P}^1$,
and let $a,b$ be two closed points of ${\bf P}^1$. Assume
that $X \times a$ and $\alpha$ intersect properly, and that
$X \times b$ and $\alpha$ intersect properly. This will be the
case if each $W_i$ dominates ${\bf P}^1$ for example. Let
$pr_X : X \times {\bf P}^1 \to X$ be the projection morphism.
A {\it cycle rationally equivalent to zero} is any cycle
of the form 
$$
pr_{X,*}( \alpha \cdot X\times a - \alpha \cdot X\times b).
$$
This is a $k$-cycle.
Note that these cycles are easy to compute in practice (given $\alpha$)
because they are obtained by proper intersection with Cartier divisors
(see formula above). It is a fact that the collection of $k$-cycles rationally
equivalent to zero is a additive subgroup of the group of $k$-cycles.
We say two $k$ cycles are {\it rationally equivalent}, notation
$\alpha \sim_{rat} \alpha'$ if $\alpha - \alpha'$ is a cycle rationally
equivalent to zero. See Chapter I of [Fulton].

\section{Pushforward and rational equivalence revisited}
\label{section-pushforward-rational-equivalence}

\noindent
Suppose that $f : X \to Y$ is a morphism
of projective smooth varieties. 
Let $\alpha \sim_{rat} 0$ be a $k$-cycle on
$X$ rationally equivalent to $0$. Then the {pushforward}
of $\alpha$ is rationally equivalent to zero:
$f_* \alpha \sim_{rat} 0$. See Chapter I of [Fulton].


\section{Pullback and rational equivalence}
\label{section-pullback-rational-equivalence}

\noindent
Suppose that $f : X \to Y$ is a flat morphism of relative dimension $r$
of projective smooth varieties. 
Let $\alpha \sim_{rat} 0$ be a $k$-cycle on
$Y$ rationally equivalent to $0$. Then the pullback
of $\alpha$ is rationally equivalent to zero:
$f^* \alpha \sim_{rat} 0$. See Chapter I of [Fulton].

\section{Moving Lemma}
\label{section-moving-lemma}

\noindent
The moving lemma states that given an $r$-cycle
$\alpha$ and a $s$ cycle $\beta$ there exists $\alpha'$, $\alpha' \sim_{rat}
\alpha$ such that $\alpha$ and $\beta$ intersect properly.
See [Samuel], [Chevalley], or [Fulton, Example 11,4,1].



\section{Intersection product and rational equivalence}
\label{section-intersection-product-rational-equivalence}

\noindent
With definitions as above we show that
the intersection product is well defined modulo rational
equivalence. Let $X$ be a nonsingular projective algebraic
variety. Let $\alpha$, resp.\ $\beta$ be a $s$, resp.\ $r$
cycle on $X$. Assume that $\alpha$ and $\beta$ intersect
properly so that $\alpha \cdot \beta$ is defined. Finally,
assume that $\alpha \sim_{rat} 0$. Goal: show that
$\alpha \cdot \beta \sim_{rat} 0$.

\medskip\noindent
After some formal arguments this amounts to showing the
following statement. Let $W \subset X \times {\bf P}^1$
be a $(s+1)$-dimensional subvariety dominating ${\bf P}^1$.
Let $W_a$, resp.\ $W_b$ be the fibre of $W \to {\bf P}^1$
over $a$, resp.\ $b$. Let $V$ be a $r$-dimensional
subvariety of $X$ such that $V$ intersects both $W_a$ and
$W_b$ properly. Then $V \cdot [W_a] \sim_{rat} V \cdot [W_b]$.

\medskip\noindent
In order to see this, note first that
$[W_a] = pr_{X,*}( W \cdot X\times a)$ and similar for $[W_b]$.
Thus we reduce to showing
$$
V \cdot pr_{X,*}( W \cdot X\times a) \sim_{rat} V \cdot 
pr_{X,*}( W \cdot X\times b).
$$
The projection formula -- which may be applied -- says
$V \cdot pr_{X,*}( W \cdot X\times a) =
pr_{X,*}(V \times {\bf P}^1 \cdot (W \cdot X\times a))$,
and similar for $b$. Thus we reduce to showing
$$
pr_{X,*}(V \times {\bf P}^1 \cdot (W \cdot X\times a))
\sim_{rat} 
pr_{X,*}(V \times {\bf P}^1 \cdot (W \cdot X\times b))
$$
Associativity for the intersection multiplicities (see [Serre,
Chapter V]) implies that $V \cdot (W \cdot X\times a) =
(V \times {\bf P}^1 \cdot W) \cdot X \times a$
and similar for $b$.  Thus we reduce to showing
$$
pr_{X,*}((V \times {\bf P}^1 \cdot W) \cdot X\times a)
\sim_{rat} 
pr_{X,*}((V \times {\bf P}^1 \cdot W) \cdot X\times b)
$$
which is true by definition of rational equivalence.

\medskip\noindent
{\bf Upshot: Chow rings.} Using the above, for any nonsingular
projective $X$ we set $A_k(X)$ equal to the group of $k$-cycles
on $X$ modulo rational equivalence. Since it is more convenient
we also use $A^c(X)=A_{\dim X -c}(X)$ to denote the group of
codimension $c$ cycles modulo rational equivalence. The intersection product
defines a product
$$
A^k(X) \times A^l(X) \longrightarrow A^{k+l}(X)
$$
defined as follows: for $a \in A^k(X)$ and $b \in A^l(X)$
we can find a codimension $k$ cycle $\alpha$ representing $\alpha$,
a codimension $l$ cycle $\beta$ representing $b$ such that $\alpha$
and $\beta$ intersect properly.
We define $a\cdot b$ to be the rational equivalence
class of $\alpha \cdot \beta$. End result: A commutative and associative
graded ring $A^*(X)$ with unit $1=[X]$.

\medskip\noindent
{\bf Pullback for a general morphism.} 
Let $X$ and $Y$ be nonsingular projective varieties,
and let $f : X \to Y$ be a morphism. We define 
$$
f^* : A_k(Y) \to A_{k+\dim X - \dim Y}(X)
$$
by the rule
$$
f^*(\alpha) = pr_{X,*}(\Gamma_f \cdot pr_{Y}^*(\alpha))
$$
where $\Gamma_f \subset X\times Y$ is the graph of $f$.
Note that it is defined only on cycle classes and not
on cylces. This pullback satisfies:
\begin{enumerate}
\item $f^* : A^*(Y) \to A^*(X)$ is a ring map,
\item $(f \circ g)^* = g^* \circ f^*$ for a composable
pair $f,g$,
\item the projection formula holds: $f_*(\alpha) \cdot \beta =
f_*( \alpha \cdot f^*\beta)$, and
\item if $f$ is flat then it agrees with the previous definition.
\end{enumerate}

\medskip\noindent
All of these follow easily from the above. For (1) you have to show that
$pr_{X,*}( \Gamma_f \cdot \alpha \cdot \beta) =
pr_{X,*}(\Gamma_f \cdot \alpha) \cdot
pr_{X,*}(\Gamma_f \cdot \beta)$. It is easy to see that
if $\alpha$ intersects $\Gamma_f$ properly, then
$\Gamma_f \cdot \alpha =  
\Gamma_f \cdot pr_X^*(pr_{X,*}(\Gamma_f \cdot \alpha))$ as cycles
because $\Gamma_f$ is a graph. Thus we get
\begin{align}
pr_{X,*}( \Gamma_f \cdot \alpha \cdot \beta) & =
pr_{X,*}( \Gamma_f \cdot
pr_X^*(pr_{X,*}(\Gamma_f \cdot \alpha)) \cdot \beta) \\
& = pr_{X,*}( pr_X^*(pr_{X,*}(\Gamma_f \cdot \alpha))
\cdot (\Gamma_f \cdot \beta)) \cr
& = pr_{X,*}(\Gamma_f \cdot \alpha) \cdot pr_{X,*}(\Gamma_f \cdot \beta)
\end{align}
the last step by the projection formula in the flat case. Properties
(2) and (3) are formal [for (3) use the flat projection formula twice].
Property (4) rests on identifying the intersection product
$\Gamma_f \cdot \alpha$ in the case $f$ is flat.

\medskip\noindent
{\bf Pullback of cycles.} Suppose that $X$ and $Y$ be nonsingular
projective varieties, and let $f : X \to Y$ be a morphism. 
Suppose that $Z \subset Y$ is a closed subvariety. Let $f^{-1}(Z)$
be the scheme theoretic inverse image:
$$
\begin{matrix}
f^{-1}(Z) & \rightarrow & Z \\
\downarrow & & \downarrow \\
X & \rightarrow & Y
\end{matrix}
$$
is a fibre product diagram of schemes. In particular $f^{-1}(Z) \subset X$
is a closed subscheme of $X$. In this case we always have
$$
\dim f^{-1}(Z) \geq \dim Z + \dim X - \dim Y.
$$
If equality holds in the formula above, then
$f^*[Z] = [f^{-1}(Z)]_{\dim Z + \dim X - \dim Y}$.
This follows by identifying $f^{-1}(Z)$ with the
scheme theoretic intersection of $\Gamma_f$ and $X\times Z$.


\medskip\noindent
{\bf Alterations and the projection formula.}
Suppose that $f : X \to Y$ is an alteration of smooth projective
varieties. In other words $f$ is dominant and generically finite.
Then for any $k$-cycle $\alpha$ on $Y$ we have
$$
f_*f^*\alpha = d\alpha
$$
where $d = [\mathbf{C}(X) : \mathbf{C}(Y)]$ is the degree of the
morphism $f$. This follows from the projection formula and the
(obvious) fact that $f_*f^*[Y] = d[Y]$.


\section{Chow groups of projective space}
\label{section-chow-projective}

\noindent
We claim that for every $n \geq 1$
there is a unique $h \in A^1({\bf P}^n)$ such that
\begin{enumerate}
\item $h$ is the class of a linear hypersurface in $\mathbf{P}^n$,
\item
for every linear map ${\bf P}^n \to {\bf P}^m$
the class $h$ pulls back to $h$, and
\item
the Segre map 
$
{\bf P}^n \times {\bf P}^m 
\longrightarrow
{\bf P}^{nm+n+m}
$
pulls $h$ back to $1\otimes h + h \otimes 1$.
\end{enumerate}

\medskip\noindent
{\bf The chow ring of ${\bf P}^n$.}
We have
$$
A^*(\mathbf{P}^n) = 
\mathbf{Z}[h]/(h^{n+1})
$$
where $h$ is as above.




\medskip\noindent
{\bf Splitting princliple.} 
For any finite locally free sheaf ${\mathcal E}$ on any nonsingular projective
variety $X$ over $\mathbf{C}$ there exists a smooth projective morphism
$f : Y \to X$ such that
\begin{enumerate}
\item $f^* : A^*(X) \to A^*(Y)$ is injective, and
\item $f^*{\mathcal E}$ has a filtration
with succesive quotients ${\mathcal L}_1, \ldots, {\mathcal L}_r$
for some invertible ${\mathcal O}_Y$-modules ${\mathcal L}_i$.
\end{enumerate}

\medskip\noindent
{\bf Additivity of chern classes.} Suppose that ${\mathcal E}$ sits in an
exact sequence
$$
0
\to
{\mathcal E}_1
\to 
{\mathcal E}
\to 
{\mathcal E}_2
\to 
0
$$
We claim that
$$
c({\mathcal E}) = c({\mathcal E}_1) \cup c({\mathcal E}_2)
$$

\medskip\noindent
{\bf Splitting principle revisited.} Using the above
we formally write
$$
c({\mathcal E}) = \prod\nolimits (1+x_i)
$$
and we call $x_i$ the {\it Chern roots} of ${\mathcal E}$.
Of course it really doesn't make sense in the cohomology 
of $X$, but it does make sense in some other variety
$Y$ such that the cohomology of $X$ injects into it.
As is customary, any symmetric polynomial in the $x_i$
then corresponds to a cohomology class on $X$ because
the chern classes of ${\mathcal E}$ are up to sign the 
elementary symmetric functions in the $x_i$.

\medskip\noindent
{\it Remark.} It is especially nice to work in the
cohomology of the Flag variety of ${\mathcal E}$ because
$ \bigoplus H^{2i}(Flag{\mathcal E})(i) =
\big(\bigoplus A^{i}(X)\big) [x_1,\ldots,x_r]/I $
where $I$ is smallest ideal such that the following
equation holds true:
$ \prod_{i=1}^r (T - x_i) =  T^r - c_1({\mathcal E}) T^{r-1}+ \ldots
+ (-1)^r c_r({\mathcal E})$. This can be proved by repeated application
of the projective space bundle formula. (And of course there
is a corresponding statement for the odd cohomology.)

\medskip\noindent
{\bf Chern classes and tensor product.}
We define the {\it Chern character} of a finite locally free
sheaf of rank $r$ to be the expression
$$
ch({\mathcal E}) := \sum\nolimits_{i=1}^r e^{x_i}
$$
if the $x_i$ are the chern roots of ${\mathcal E}$. By the above
we have, in case of an exact sequence
$
0
\to
{\mathcal E}_1
\to 
{\mathcal E}
\to 
{\mathcal E}_2
\to 
0
$
that $ch({\mathcal E}) = ch({\mathcal E}_1) + ch({\mathcal E}_2)$. 
Using the Chern character we can express the compatibility
of the chern classes and tensor product as follows:
$$
ch({\mathcal E}_1 \otimes_{{\mathcal O}_X} {\mathcal E}_2) =
ch({\mathcal E}_1) \cup ch({\mathcal E}_2)
$$
The proof follows directly from the splitting principle.

\medskip\noindent
{\bf Tiny bit of K-theory.} Let $X$ be a {\it smooth}
quasi-projective variety over $k$. We will use the
following facts:
\begin{enumerate}
\item{(1)} For any coherent ${\mathcal O}_X$-module ${\mathcal F}$
there exists a finite complex of finite locally free
${\mathcal O}_X$-modules ${\mathcal F}^*$ and a quasi-isomorphism
${\mathcal F}^* \to {\mathcal F}[0]$. We will call such a quasi-isomorphism
${\mathcal F}^* \to {\mathcal F}[0]$ a {\it resolution} of ${\mathcal F}$.
\item{(2)} For any short exact sequence
$
0
\to
{\mathcal F}_1
\to 
{\mathcal F}_2
\to 
{\mathcal F}_3
\to 
0
$
there exist resolutions ${\mathcal F}_i^* \to {\mathcal F}_i[0]$
and a short exact sequence 
$
0
\to
{\mathcal F}_1^*
\to 
{\mathcal F}_2^*
\to 
{\mathcal F}_3^*
\to 
0
$
of complexes that recovers the short exact sequence
upon taking cohomology sheaves.
\end{enumerate}
In these statements it is convenient to have complexes supported in
degrees $\leq 0$ but it is not necessary. 
Basically, (1) and (2) follow from the following two statements:
(a) for every coherent sheaf ${\mathcal F}$ on $X$ there exists a direct
sum $\oplus {\mathcal O}_X(-n)$ which surjects onto ${\mathcal F}$, and
(b) given an exact complex $0\to {\mathcal G} \to {\mathcal F}_{N}
\to \ldots \to {\mathcal F}_0 \to {\mathcal F} \to 0$ with ${\mathcal F}$
coherent, and ${\mathcal F}_i$ finite locally free, and $N \geq \dim X -1$
then ${\mathcal G}$ is locally free. The first is standard, see
Hartshorne, Chapter II, Corollary 5.18. The second is Serre's
$pd+depth=dim$, see Matsumura, Commutative Algebra, page 113.

\medskip\noindent
For any variety $X$ we define two abelian groups $K^0(X)$ and
$K_0(X)$. The group $K^0(X)$ is the free abelian group generated
by finite locally free ${\mathcal O}_X$-modules modulo the
relation that $[{\mathcal E}_2] = [{\mathcal E}_1] + [{\mathcal E}_3]$
whenever 
$
0
\to
{\mathcal E}_1
\to 
{\mathcal E}_2
\to 
{\mathcal E}_3
\to 
0
$
is a short exact sequence of finite locally free ${\mathcal O}_X$-modules.
The group $K_0(X)$ is the free abelian group generated
by coherent ${\mathcal O}_X$-modules modulo the
relation that $[{\mathcal F}_2] = [{\mathcal F}_1] + [{\mathcal F}_3]$
whenever 
$
0
\to
{\mathcal F}_1
\to 
{\mathcal F}_2
\to 
{\mathcal F}_3
\to 
0
$
is a short exact sequence of coherent ${\mathcal O}_X$-modules.
These relations signify that if ${\mathcal E}^*$ (resp.\ ${\mathcal F}^*$)
is a finite {\it exact} complex of finite locally free (resp.\ coherent)
${\mathcal O}_X$-modules then $\sum (-1)^i[{\mathcal E}_i] = 0$ (resp.\ 
$\sum (-1)^i[{\mathcal F}_i] = 0$) in $K^0(X)$ (resp.\ $K_0(X)$).
These groups satisfy the following functorialities
\begin{enumerate}
\item{} If $ f : X \to Y$ is any morphism then there is a
map $f^* : K^0(Y) \to K^0(X)$ defined by pullback.
\item{} If $f : X \to Y$ is a flat morphism of schemes of finite type
over $k$ there is a map $f^* : K_0(Y) \to K_0(X)$ defined by pullback.
\item{} If $f : X \to Y$ is a proper morphism of finite type
schemes over $k$ there is a map $f_* : K_0(X) \to K_0(Y)$
defined by the rule
$f_*([{\mathcal F}]) = \sum_i (-1)^i [R^if_*({\mathcal F})]$.
\end{enumerate}
Finally, there is an obvious homomorphism of abelian groups
$$
K^0(X) \longrightarrow K_0(X).
$$

\medskip\noindent
Claim: If $X$ is a {\it smooth} quasi-projective variety then
this homomorphism is an isomorphism. As an inverse
we map the class $[{\mathcal F}]$ in $K_0(X)$ to $I({\mathcal F}):=
\sum (-1)^i [{\mathcal F}^i]$ in $K^0(X)$, if ${\mathcal F}^* \to {\mathcal F}[0]$
is a resolution as above. The main problem is to show that
this is well defined. Let us say that a coherent ${\mathcal O}_X$-module
has property ${\mathcal P}$ if this is the case.
A finite locally free ${\mathcal O}_X$-module has property ${\mathcal P}$
as is clear from the definition of $K^0(X)$. Furthermore,
assertion (2) above says that if
$
0
\to
{\mathcal F}_1
\to 
{\mathcal F}_2
\to 
{\mathcal F}_3
\to 
0
$
is a short exact sequence of coherent ${\mathcal O}_X$-modules having
property ${\mathcal P}$, then $I({\mathcal F}_2) = I({\mathcal F}_1) + I({\mathcal F}_3)$.
We argue that every coherent ${\mathcal O}_X$-module has property
${\mathcal P}$ by induction on the projective dimension
of the sheaf ${\mathcal F}$. Suppose that
${\mathcal F}_1^* \to {\mathcal F}[0]$ and
${\mathcal F}_2^* \to {\mathcal F}[0]$ are two resolutions of a
coherent ${\mathcal O}_X$-module ${\mathcal F}$ with ${\mathcal F}_i^j=0$
for $j > 0$. Consider the sheaf ${\mathcal F}_{12}$ fitting
into the following exact diagram
$$
\xymatrix{
& & 0\ar[d] & 0\ar[d] &
\cr
& & {\mathcal K}_2 \ar@{=}[r]\ar[d] & {\mathcal K}_2\ar[d] &
\cr
0 \ar[r] & {\mathcal K}_1\ar@{=}[d]\ar[r] & {\mathcal F}_{12}\ar[r]\ar[d] & {\mathcal F}_2^0\ar[r]\ar[d] & 0
\cr
0 \ar[r] & {\mathcal K}_1\ar[r] & {\mathcal F}_{1}^0\ar[r]\ar[d] & {\mathcal F}\ar[d]\ar[r] & 0 \cr
& & 0 & 0 & \cr}
$$
where ${\mathcal K}_i$ is the kernel of ${\mathcal F}^0_i \to {\mathcal F}$.
Note that the projective dimensions of ${\mathcal K}_1$, ${\mathcal K}_2$,
and ${\mathcal F}_{12}$ are less than the projective dimension of
${\mathcal F}$ (unless ${\mathcal F}$ was locally free to begin with). 
Hence, our induction hypothesis applies to these sheaves. In particular
we see that $I({\mathcal K}_1) + I({\mathcal F}_2^0) =
I({\mathcal K}_2) + I({\mathcal F}_1^0)$. Since we may use the complex
${\mathcal F}_i^{\leq -1}[1]$ as a resolution of ${\mathcal K}_i$ 
we have $I({\mathcal K}_i) = \sum (-1)^{j+1}[{\mathcal F}_i^j]$.
The desired result $\sum_i (-1)^i[{\mathcal F}_1^i]
= \sum_i (-1)^i[{\mathcal F}_2^i]$ follows.


\medskip\noindent
{\bf The Chern character of a coherent sheaf.}
For any quasi-projective variety $X$ over $k$ we can extend the 
Chern character to a homomorphism
$$
ch^{dR} : K^0(X) \longrightarrow \bigoplus H^{2i}_{dR}(X)(i).
$$
In the {\it smooth} case we use the isomorphism $K^0(X) \to K_0(X)$
to define the {\it Chern character} of a coherent ${\mathcal O}_X$-module
${\mathcal F}$. The recipe is that we choose a resolution
${\mathcal F}^* \to {\mathcal F}$ as in (1) and we set
$$
ch^{dR}({\mathcal F})
:=
ch^{dR}({\mathcal F}^*)
:=
\sum\nolimits_i (-1)ch^{dR}({\mathcal F}^i).
$$
This is well defined because of the isomorphism $K^0(X) \to K_0(X)$
above. We have additivity for this chern character. Namely, suppose
that
$
0
\to
{\mathcal H}
\to 
{\mathcal G}
\to 
{\mathcal F}
\to 
0
$
is a short exact sequence of coherent sheaves on $X$. Then
$ch^{dR}({\mathcal G}) = ch^{dR}({\mathcal H}) + ch^{dR}({\mathcal F})$.
In general we no longer have $ch^{dR}({\mathcal F}\otimes_{{\mathcal O}_X}
{\mathcal G}) = ch^{dR}({\mathcal F}) \cup ch^{dR}({\mathcal G})$
for a pair of coherent sheaves ${\mathcal F}$, ${\mathcal G}$ on $X$.
The correct statement is
$$
ch^{dR}({\mathcal F}\otimes^{\bf L}_{{\mathcal O}_X}{\mathcal G})
=
\sum\nolimits_{i=0}^{\dim X}
(-1)^i
ch^{dR}(Tor_i^{{\mathcal O}_X}({\mathcal F},{\mathcal G}))
=
ch^{dR}({\mathcal F}) \cup ch^{dR}({\mathcal G}).
$$
The {\it derived} tensor product
${\mathcal F}\otimes^{\bf L}_{{\mathcal O}_X}{\mathcal G}$ can be
computed by taking a resolution ${\mathcal F}^* \to {\mathcal F}[0]$
and taking the complex ${\mathcal F}^*\otimes_{{\mathcal O}_X} {\mathcal G}$.
A final obvious fact is that if $f : X \to Y$ is a {\it flat}
morphism of smooth quasi-projective varieties over $k$ then
$f^*( ch^{dR}({\mathcal F}) ) = ch^{dR}(f^*{\mathcal F})$ for any
coherent ${\mathcal O}_Y$-module ${\mathcal F}$. In fact, it suffices
that $f$ is flat at all points of $f^{-1}(Supp({\mathcal F}))$.
The reason is that in this case a resolution will pull
back to a resolution.

\medskip\noindent
{\bf Example. The cohomology class of a skyscraper sheaf on
${\bf P}^n$.} Let $p\in {\bf P}^n(k)$ be a rational point
and denote $k_p$ the skyscraper sheaf at the point $p$.
There is a finite locally free resolution
$$
0
\to
{\mathcal O}(-n)
\to
{\mathcal O}(-n+1)^n
\to
\ldots
\to
{\mathcal O}(-i)^{\binom{n}{i}}
\to
\ldots
\to
{\mathcal O}(-1)^n
\to
{\mathcal O}
\to
k_p
\to
0.
$$
By definition we get
$$
ch^{dR}(k_p)
= 
\sum\nolimits_{i=0}^n
(-1)^i
ch^{dR}( {\mathcal O}(-i)^{\binom{n}{i}} )
=
\sum\nolimits_{i=0}^n
(-1)^i \binom{n}{i} e^{-i \xi}
=
(1 - e^{-\xi})^n 
=
\xi^n
$$
because $\xi^{n+1} = 0$.

\medskip\noindent
{\bf Chern character in flat families.} 
Suppose that $T$ is a smooth quasi-projective connected variety over $k$.
Let $X$ be a smooth projective variety over $k$. 
Suppose ${\mathcal F}$ is a coherent ${\mathcal O}_{X\times T}$-module
which is flat over $T$. Finally, let $a,b\in T(k)$ be rational
points and let ${\mathcal F}_a$ be the pullback of ${\mathcal F}$
via the morphism $X \to X\times a \to X\times T$, and
similarly for ${\mathcal F}_b$. We claim that
$$
ch^{dR}({\mathcal F}_a) = ch^{dR}({\mathcal F}_b).
$$
This would be completely clear if ${\mathcal F}$ were a locally
free ${\mathcal O}_{X\times T}$-module, by an argument similar 
to the one proving ``homotopy invariance''. In general,
flatness of ${\mathcal F}$ over $T$ implies that if
${\mathcal F}^* \to {\mathcal F}[0]$ is a resolution (as in (1) above)
then ${\mathcal F}^*_a \to {\mathcal F}_a$ is a resolution as well.
Detail left to the reader.

\medskip\noindent
{\bf The Chern character of a skyscraper sheaf.} 
Suppose that $x$, $x'$ are closed points in the smooth
quasi-projective variety $X$ over $k$. Set $d =\deg_k(x)$
and $d' = \deg_k(x')$. Denote $\kappa(x)$, $\kappa(x')$
the skyscraper sheaves (of residue fields). We claim that
$$
d'\ ch^{dR}( \kappa(x) ) = d\ ch^{dR}( \kappa(x') ).
$$
Of course this follows from the above because the family
of all zero cycles of a given degree form an irreducible
family, and we have invariance of Chern character in flat
families. The details are slightly harder than you would
think at first since we do not assume a variety is
geometrically irreducible. Observe that the structure
morphism $X \to Spec(k)$ factors as
$$
X \longrightarrow Spec( k' ) \longrightarrow Spec(k)
$$
where $k'$ is the algebraic closure of $k$ in the function
field $k(X)$. Recall that $k' \supset k$ is a finite extension.
Set $e=[k':k]$. Since the characteristic of $k$ is zero, since
$X$ is irreducible, and since $X$ is smooth over $k$ we conclude
that $X$ is geometrically irrreducible over $k'$. Hence we
can find a smooth curve $C \subset X$ with $x, x' \in C$ which
is geometrically irreducible over $k'$ as well. (Using Bertini
over $k$, see the excellent book ``Theoremes de Bertini et
Applications'' by Jouanolou.) In this case $x$ and $x'$ give
rise to effective Cartier divisors $(x')$ and $(x)$ on $C$
of degree $d/e$, respectively $d'/e$ over $k'$. Then $d'(x)$ and $d(x')$
correspond to $k'$-points $a', b'$ of the smooth geometrically irreducible
variety $Sym^{dd'/e}_{k'}(C)$ over $k'$. Consider the restriction
of scalars
$$
T := Res_{k'/k}\Big(Sym^{dd'/e}_{k'}(C)\Big).
$$
A $R$-point $t$ of $T$ is by definition a $R\otimes k'$ point
of $Sym^{dd'/e}_{k'}(C)$ for any $k$-algebra $R$. In particular
there is a canonical morphism $T \times_{Spec(k)} Spec(k') \to 
Sym^{dd'/e}_{k'}(C)$. From general properties of restrictions
of scalars we get that $T$ is smooth and geometrically irreducible
over $k$. Also, $a', b'$ correspond to $k$-points $a,b$ of $T$.
Let $D' \subset Sym^{dd'/e}_{k'}(C) \times_{Spec(k')} C$ be the
universal degree $dd'/e$ divisor over $k'$. Let
$D \subset T \times_{Spec(k)} C$ be the divisor which is the inverse
image of $D'$ under
$$
T \times_{Spec(k)} C
=
T \times_{Spec(k)} Spec(k') \times_{Spec(k')} C
\longrightarrow
Sym^{dd'/e}_{k'}(C) \times_{Spec(k')} C
$$
Then $D$ is a flat family of closed subschemes of $C$ of length
$dd'$ over $k$ and $D_a = d'(x)$ and $D_b = d(x')$ as closed
subschemes over $k$. Thus we conclude that
$ch^{dR}({\mathcal O}_{d'(x)}) = ch^{dR}({\mathcal O}_{d(x')})$ in the
cohomology of $X$. We leave it as an exercise to show that
$ch^{dR}({\mathcal O}_{d'(x)}) = d' ch^{dR}( \kappa(x) )$
and similarly for the other side. (Hint: Use additivity of
the Chern character.)

\medskip\noindent
{\bf The cohomology class of a point.}
The upshot is that the element
$$
u_X := \frac{ 1 }{ \deg_k(x) } ch^{dR}( \kappa(x) )
$$
for any closed point $x \in X$ is a canonical element
of $H^{2\dim X}(X)(\dim X)$. If $f : X \to Y$ is a
generically finite morphism of smooth
projective varieties of the {\it same} dimension
then $f^* u_Y = \deg(f) u_X$. Namely we can find a 
point $y \in Y$ such that $f^{-1}(y) = \{x_1,\ldots,x_t\}$
is finite and such that $f$ is flat and unramified in all points
of $f^{-1}(y)$ (uses characteristic zero).
Since in this case $\deg(f) \deg_k(y) = \sum \deg_k(x_i)$
we conclude.

\medskip\noindent
{\bf Nonvanishing.}
Sofar we have not argued that $u_X$ is nonzero. 
The reason that it is nonvanishing is that it is 
equal to $\xi^n \not=0$ in the case of ${\bf P}^n$,
and by the fact that choosing a finite surjective
morphism $\pi : X \to {\bf P}^{\dim X}$ gives
an {\it injection} $H^*_{dR}({\bf P}^{\dim X})
\to H^*_{dR}(X)$ (by the trace map for $\pi$, see above).



















\medskip\noindent
{\bf References}

\medskip\noindent
[Serre] = \cite{Serre_local_algebra}, or \cite{Serre_algebre_locale}.
\hfill\break
[Samuel] = \cite{Samuel}.
\hfill\break
[Chevalley] =  \cite{ChevalleyI} and \cite{ChevalleyII}.
\hfill\break
[Fulton] = \cite{F}.

\section{Other chapters}

\begin{multicols}{2}
\begin{enumerate}
\item \hyperref[introduction-section-phantom]{Introduction}
\item \hyperref[conventions-section-phantom]{Conventions}
\item \hyperref[sets-section-phantom]{Set Theory}
\item \hyperref[categories-section-phantom]{Categories}
\item \hyperref[topology-section-phantom]{Topology}
\item \hyperref[sheaves-section-phantom]{Sheaves on Spaces}
\item \hyperref[algebra-section-phantom]{Commutative Algebra}
\item \hyperref[sites-section-phantom]{Sites and Sheaves}
\item \hyperref[homology-section-phantom]{Homological Algebra}
\item \hyperref[derived-section-phantom]{Derived Categories}
\item \hyperref[more-algebra-section-phantom]{More Algebra}
\item \hyperref[simplicial-section-phantom]{Simplicial Methods}
\item \hyperref[modules-section-phantom]{Sheaves of Modules}
\item \hyperref[sites-modules-section-phantom]{Modules on Sites}
\item \hyperref[injectives-section-phantom]{Injectives}
\item \hyperref[cohomology-section-phantom]{Cohomology of Sheaves}
\item \hyperref[sites-cohomology-section-phantom]{Cohomology on Sites}
\item \hyperref[hypercovering-section-phantom]{Hypercoverings}
\item \hyperref[schemes-section-phantom]{Schemes}
\item \hyperref[constructions-section-phantom]{Constructions of Schemes}
\item \hyperref[properties-section-phantom]{Properties of Schemes}
\item \hyperref[morphisms-section-phantom]{Morphisms of Schemes}
\item \hyperref[coherent-section-phantom]{Coherent Cohomology}
\item \hyperref[divisors-section-phantom]{Divisors}
\item \hyperref[limits-section-phantom]{Limits of Schemes}
\item \hyperref[varieties-section-phantom]{Varieties}
\item \hyperref[chow-section-phantom]{Chow Homology}
\item \hyperref[topologies-section-phantom]{Topologies on Schemes}
\item \hyperref[descent-section-phantom]{Descent}
\item \hyperref[more-morphisms-section-phantom]{More on Morphisms}
\item \hyperref[flat-section-phantom]{More on Flatness}
\item \hyperref[groupoids-section-phantom]{Groupoid Schemes}
\item \hyperref[more-groupoids-section-phantom]{More on Groupoid Schemes}
\item \hyperref[etale-section-phantom]{\'Etale Morphisms of Schemes}
\item \hyperref[etale-cohomology-section-phantom]{\'Etale Cohomology}
\item \hyperref[spaces-section-phantom]{Algebraic Spaces}
\item \hyperref[spaces-properties-section-phantom]{Properties of Algebraic Spaces}
\item \hyperref[spaces-morphisms-section-phantom]{Morphisms of Algebraic Spaces}
\item \hyperref[spaces-topologies-section-phantom]{Topologies on Algebraic Spaces}
\item \hyperref[spaces-descent-section-phantom]{Descent and Algebraic Spaces}
\item \hyperref[spaces-more-morphisms-section-phantom]{More on Morphisms of Spaces}
\item \hyperref[quot-section-phantom]{Quot and Hilbert Spaces}
\item \hyperref[stacks-section-phantom]{Stacks}
\item \hyperref[spaces-groupoids-section-phantom]{Groupoids in Algebraic Spaces}
\item \hyperref[spaces-more-groupoids-section-phantom]{More on Groupoids in Spaces}
\item \hyperref[bootstrap-section-phantom]{Bootstrap}
\item \hyperref[examples-stacks-section-phantom]{Examples of Stacks}
\item \hyperref[groupoids-quotients-section-phantom]{Quotients of Groupoids}
\item \hyperref[algebraic-section-phantom]{Algebraic Stacks}
\item \hyperref[criteria-section-phantom]{Criteria for Representability}
\item \hyperref[stacks-properties-section-phantom]{Properties of Algebraic Stacks}
\item \hyperref[stacks-morphisms-section-phantom]{Morphisms of Algebraic Stacks}
\item \hyperref[examples-section-phantom]{Examples}
\item \hyperref[exercises-section-phantom]{Exercises}
\item \hyperref[guide-section-phantom]{Guide to Literature}
\item \hyperref[desirables-section-phantom]{Desirables}
\item \hyperref[coding-section-phantom]{Coding Style}
\item \hyperref[fdl-section-phantom]{GNU Free Documentation License}
\item \hyperref[index-section-phantom]{Auto Generated Index}
\end{enumerate}
\end{multicols}


\bibliography{my}
\bibliographystyle{alpha}

\end{document}
