\IfFileExists{stacks-project.cls}{%
\documentclass{stacks-project}
}{%
\documentclass{amsart}
}

% The following AMS packages are automatically loaded with
% the amsart documentclass:
%\usepackage{amsmath}
%\usepackage{amssymb}
%\usepackage{amsthm}

% For dealing with references we use the comment environment
\usepackage{verbatim}
\newenvironment{reference}{\comment}{\endcomment}
%\newenvironment{reference}{}{}
\newenvironment{slogan}{\comment}{\endcomment}
\newenvironment{history}{\comment}{\endcomment}

% For commutative diagrams you can use
% \usepackage{amscd}
\usepackage[all]{xy}

% We use 2cell for 2-commutative diagrams.
\xyoption{2cell}
\UseAllTwocells

% To put source file link in headers.
% Change "template.tex" to "this_filename.tex"
% \usepackage{fancyhdr}
% \pagestyle{fancy}
% \lhead{}
% \chead{}
% \rhead{Source file: \url{template.tex}}
% \lfoot{}
% \cfoot{\thepage}
% \rfoot{}
% \renewcommand{\headrulewidth}{0pt}
% \renewcommand{\footrulewidth}{0pt}
% \renewcommand{\headheight}{12pt}

\usepackage{multicol}

% For cross-file-references
\usepackage{xr-hyper}

% Package for hypertext links:
\usepackage{hyperref}

% For any local file, say "hello.tex" you want to link to please
% use \externaldocument[hello-]{hello}
\externaldocument[introduction-]{introduction}
\externaldocument[conventions-]{conventions}
\externaldocument[sets-]{sets}
\externaldocument[categories-]{categories}
\externaldocument[topology-]{topology}
\externaldocument[sheaves-]{sheaves}
\externaldocument[sites-]{sites}
\externaldocument[stacks-]{stacks}
\externaldocument[fields-]{fields}
\externaldocument[algebra-]{algebra}
\externaldocument[brauer-]{brauer}
\externaldocument[homology-]{homology}
\externaldocument[derived-]{derived}
\externaldocument[simplicial-]{simplicial}
\externaldocument[more-algebra-]{more-algebra}
\externaldocument[smoothing-]{smoothing}
\externaldocument[modules-]{modules}
\externaldocument[sites-modules-]{sites-modules}
\externaldocument[injectives-]{injectives}
\externaldocument[cohomology-]{cohomology}
\externaldocument[sites-cohomology-]{sites-cohomology}
\externaldocument[dga-]{dga}
\externaldocument[dpa-]{dpa}
\externaldocument[hypercovering-]{hypercovering}
\externaldocument[schemes-]{schemes}
\externaldocument[constructions-]{constructions}
\externaldocument[properties-]{properties}
\externaldocument[morphisms-]{morphisms}
\externaldocument[coherent-]{coherent}
\externaldocument[divisors-]{divisors}
\externaldocument[limits-]{limits}
\externaldocument[varieties-]{varieties}
\externaldocument[topologies-]{topologies}
\externaldocument[descent-]{descent}
\externaldocument[perfect-]{perfect}
\externaldocument[more-morphisms-]{more-morphisms}
\externaldocument[flat-]{flat}
\externaldocument[groupoids-]{groupoids}
\externaldocument[more-groupoids-]{more-groupoids}
\externaldocument[etale-]{etale}
\externaldocument[chow-]{chow}
\externaldocument[intersection-]{intersection}
\externaldocument[pic-]{pic}
\externaldocument[adequate-]{adequate}
\externaldocument[dualizing-]{dualizing}
\externaldocument[duality-]{duality}
\externaldocument[discriminant-]{discriminant}
\externaldocument[local-cohomology-]{local-cohomology}
\externaldocument[curves-]{curves}
\externaldocument[resolve-]{resolve}
\externaldocument[models-]{models}
\externaldocument[pione-]{pione}
\externaldocument[etale-cohomology-]{etale-cohomology}
\externaldocument[proetale-]{proetale}
\externaldocument[crystalline-]{crystalline}
\externaldocument[spaces-]{spaces}
\externaldocument[spaces-properties-]{spaces-properties}
\externaldocument[spaces-morphisms-]{spaces-morphisms}
\externaldocument[decent-spaces-]{decent-spaces}
\externaldocument[spaces-cohomology-]{spaces-cohomology}
\externaldocument[spaces-limits-]{spaces-limits}
\externaldocument[spaces-divisors-]{spaces-divisors}
\externaldocument[spaces-over-fields-]{spaces-over-fields}
\externaldocument[spaces-topologies-]{spaces-topologies}
\externaldocument[spaces-descent-]{spaces-descent}
\externaldocument[spaces-perfect-]{spaces-perfect}
\externaldocument[spaces-more-morphisms-]{spaces-more-morphisms}
\externaldocument[spaces-flat-]{spaces-flat}
\externaldocument[spaces-groupoids-]{spaces-groupoids}
\externaldocument[spaces-more-groupoids-]{spaces-more-groupoids}
\externaldocument[bootstrap-]{bootstrap}
\externaldocument[spaces-pushouts-]{spaces-pushouts}
\externaldocument[groupoids-quotients-]{groupoids-quotients}
\externaldocument[spaces-more-cohomology-]{spaces-more-cohomology}
\externaldocument[spaces-simplicial-]{spaces-simplicial}
\externaldocument[formal-spaces-]{formal-spaces}
\externaldocument[restricted-]{restricted}
\externaldocument[spaces-resolve-]{spaces-resolve}
\externaldocument[formal-defos-]{formal-defos}
\externaldocument[defos-]{defos}
\externaldocument[cotangent-]{cotangent}
\externaldocument[examples-defos-]{examples-defos}
\externaldocument[algebraic-]{algebraic}
\externaldocument[examples-stacks-]{examples-stacks}
\externaldocument[stacks-sheaves-]{stacks-sheaves}
\externaldocument[criteria-]{criteria}
\externaldocument[artin-]{artin}
\externaldocument[quot-]{quot}
\externaldocument[stacks-properties-]{stacks-properties}
\externaldocument[stacks-morphisms-]{stacks-morphisms}
\externaldocument[stacks-limits-]{stacks-limits}
\externaldocument[stacks-cohomology-]{stacks-cohomology}
\externaldocument[stacks-perfect-]{stacks-perfect}
\externaldocument[stacks-introduction-]{stacks-introduction}
\externaldocument[stacks-more-morphisms-]{stacks-more-morphisms}
\externaldocument[stacks-geometry-]{stacks-geometry}
\externaldocument[moduli-]{moduli}
\externaldocument[moduli-curves-]{moduli-curves}
\externaldocument[examples-]{examples}
\externaldocument[exercises-]{exercises}
\externaldocument[guide-]{guide}
\externaldocument[desirables-]{desirables}
\externaldocument[coding-]{coding}
\externaldocument[obsolete-]{obsolete}
\externaldocument[fdl-]{fdl}
\externaldocument[index-]{index}

% Theorem environments.
%
\theoremstyle{plain}
\newtheorem{theorem}[subsection]{Theorem}
\newtheorem{proposition}[subsection]{Proposition}
\newtheorem{lemma}[subsection]{Lemma}

\theoremstyle{definition}
\newtheorem{definition}[subsection]{Definition}
\newtheorem{example}[subsection]{Example}
\newtheorem{exercise}[subsection]{Exercise}
\newtheorem{situation}[subsection]{Situation}

\theoremstyle{remark}
\newtheorem{remark}[subsection]{Remark}
\newtheorem{remarks}[subsection]{Remarks}

\numberwithin{equation}{subsection}

% Macros
%
\def\lim{\mathop{\rm lim}\nolimits}
\def\colim{\mathop{\rm colim}\nolimits}
\def\Spec{\mathop{\rm Spec}}
\def\Hom{\mathop{\rm Hom}\nolimits}
\def\Ext{\mathop{\rm Ext}\nolimits}
\def\SheafHom{\mathop{\mathcal{H}\!{\it om}}\nolimits}
\def\SheafExt{\mathop{\mathcal{E}\!{\it xt}}\nolimits}
\def\Sch{\textit{Sch}}
\def\Mor{\mathop{\rm Mor}\nolimits}
\def\Ob{\mathop{\rm Ob}\nolimits}
\def\Sh{\mathop{\textit{Sh}}\nolimits}
\def\NL{\mathop{N\!L}\nolimits}
\def\proetale{{pro\text{-}\acute{e}tale}}
\def\etale{{\acute{e}tale}}
\def\QCoh{\textit{QCoh}}
\def\Ker{\mathop{\rm Ker}}
\def\Im{\mathop{\rm Im}}
\def\Coker{\mathop{\rm Coker}}
\def\Coim{\mathop{\rm Coim}}

%
% Macros for moduli stacks/spaces
%
\def\QCohstack{\mathcal{QC}\!{\it oh}}
\def\Cohstack{\mathcal{C}\!{\it oh}}
\def\Spacesstack{\mathcal{S}\!{\it paces}}
\def\Quotfunctor{{\rm Quot}}
\def\Hilbfunctor{{\rm Hilb}}
\def\Curvesstack{\mathcal{C}\!{\it urves}}
\def\Polarizedstack{\mathcal{P}\!{\it olarized}}
\def\Complexesstack{\mathcal{C}\!{\it omplexes}}
% \Pic is the operator that assigns to X its picard group, usage \Pic(X)
% \Picardstack_{X/B} denotes the Picard stack of X over B
% \Picardfunctor_{X/B} denotes the Picard functor of X over B
\def\Pic{\mathop{\rm Pic}\nolimits}
\def\Picardstack{\mathcal{P}\!{\it ic}}
\def\Picardfunctor{{\rm Pic}}
\def\Deformationcategory{\mathcal{D}\!{\it ef}}


% OK, start here.
%
\begin{document}

\title{Simplicial Methods}

%\begin{abstract}
%\end{abstract}

\maketitle

\tableofcontents

\section{Introduction}
\label{section-introduction}

\noindent
This is a minimal introduction to simplicial methods.
We just add here whenever somthing is needed later on.
A general reference to this material is perhaps \cite{SimpHom}.
An example of the things you can do is the paper
by Quillen on Homotopical Algebra, see \cite{Quillen}
or the paper on Etale Homotopy by Artin and Mazur, see \cite{ArtinMazur}.

\section{The category $\Delta$}
\label{section-Delta}

\noindent
The category $\Delta$ is the category with
\begin{enumerate}
\item objects $[0], [1], [2], \ldots$ with
$[n] = \{0, 1, 2, \ldots, n\}$ and
\item morphism $[n] \to [m]$ is the set of nondecreasing
maps of sets $\{0, 1, 2, \ldots, n\} \to \{0, 1, 2, \ldots, m\}$.
\end{enumerate}
Here {\it nondecreasing} for a map $\varphi : [n] \to [m]$
means by definition that $\varphi(i) \geq \varphi(j)$ if $i \geq j$.
In other words, $\Delta$ is a category equivalent to the
``big'' category of finite totally ordered sets and nondecreasing maps.
There are exactly $n + 1$ morphisms $[0] \to [n]$ and
there is exactly $1$ morphism $[n] \to [0]$. There are
exactly $(n + 1)(n + 2)/2$ morphisms $[1] \to [n]$ and there are
exactly $n + 2$ morphisms $[n] \to [1]$. And so on and so forth.

\begin{definition}
\label{definition-face-degeneracy}
For any integer $n\geq 1$, and any $0\leq j \leq n$ we let $\delta^n_j : [n-1]
\to [n]$ denote the injective order preserving map skipping $j$. For any
integer $n\geq 0$, and any $0\leq j \leq n$ we denote $\sigma^n_j : [n+1] 
\to [n]$ the surjective order preserving map with 
$(\sigma^n_j)^{-1}(\{j\}) = \{j, j+1\}$.
\end{definition}

\begin{lemma}
\label{lemma-face-degeneracy}
Any morphism in $\Delta$ can be written as a composition
of an identity morphism, and the morphisms $\delta^n_j$ and $\sigma^n_j$.
\end{lemma}

\begin{proof}
Let $\varphi : [n] \to [m]$ be a morphism of $\Delta$.
If $j \not \in \text{Im}(\varphi)$, then we can write
$\varphi$ as $\delta^m_j \circ \psi$ for some morphism
$\psi : [n] \to [m - 1]$. If $\varphi(j) = \varphi(j + 1)$
then we can write $\varphi$ as $\psi \circ \sigma^{n - 1}_j$
for some morphism $\psi : [n - 1] \to [m]$.
The result follows because each replacement
as above lowers $n + m$ and hence at some point
$\varphi$ is both injective and surjective, hence
an identity morphism.
\end{proof}

\begin{lemma}
\label{lemma-relations-face-degeneracy}
The morphisms $\delta^n_j$ and $\sigma^n_j$ satisfy the following relations.
\begin{enumerate}
\item If $0 \leq i < j \leq n + 1$, then
$\delta^{n + 1}_j \circ \delta^n_i =
\delta^{n + 1}_i \circ \delta^n_{j - 1}$.
In other words the diagram
$$
\xymatrix{
& [n] \ar[rd]^{\delta^{n + 1}_j} & \\
[n - 1] \ar[ru]^{\delta^n_i} \ar[rd]_{\delta^n_{j - 1}} & &
[n + 1] \\
& [n] \ar[ru]_{\delta^{n + 1}_i} & 
}
$$
commutes.
\item If $0 \leq i < j \leq n - 1$, then
$\sigma^{n - 1}_j \circ \delta^n_i =
\delta^{n - 1}_i \circ \sigma^{n - 2}_{j - 1}$.
In other words the diagram
$$
\xymatrix{
& [n] \ar[rd]^{\sigma^{n - 1}_j} & \\
[n - 1] \ar[ru]^{\delta^n_i} \ar[rd]_{\sigma^{n - 2}_{j - 1}} & &
[n - 1] \\
& [n - 2] \ar[ru]_{\delta^{n - 1}_i} & 
}
$$
commutes.
\item If $0 \leq j \leq n - 1$, then
$\sigma^{n - 1}_j \circ \delta^n_j = \text{id}_{[n - 1]}$
and
$\sigma^{n - 1}_j \circ \delta^n_{j + 1} = \text{id}_{[n - 1]}$.
In other words the diagram
$$
\xymatrix{
& [n] \ar[rd]^{\sigma^{n - 1}_j} & \\
[n - 1]
\ar[ru]^{\delta^n_j}
\ar[rd]_{\delta^n_{j + 1}}
\ar[rr]^{\text{id}_{[n - 1]}} & & [n - 1] \\
& [n] \ar[ru]_{\sigma^{n - 1}_j} &
}
$$
commutes.
\item If $0 < j + 1 < i \leq n$, then
$\sigma^{n - 1}_j \circ \delta^n_i =
\delta^{n - 1}_{i - 1} \circ \sigma^{n - 2}_j$.
In other words the diagram
$$
\xymatrix{
& [n] \ar[rd]^{\sigma^{n - 1}_j} & \\
[n - 1] \ar[ru]^{\delta^n_i} \ar[rd]_{\sigma^{n - 2}_j} & &
[n - 1] \\
& [n - 2] \ar[ru]_{\delta^{n - 1}_{i - 1}} & 
}
$$
commutes.
\item If $0 \leq i \leq j \leq n - 1$, then
$\sigma^{n - 1}_j \circ \sigma^n_i =
\sigma^{n - 1}_i \circ \sigma^n_{j + 1}$.
In other words the diagram
$$
\xymatrix{
& [n] \ar[rd]^{\sigma^{n - 1}_j} & \\
[n + 1] \ar[ru]^{\sigma^n_i} \ar[rd]_{\sigma^n_{j + 1}} & &
[n - 1] \\
& [n] \ar[ru]_{\sigma^{n - 1}_i} & 
}
$$
commutes.
\end{enumerate}
\end{lemma}

\begin{proof}
Omitted.
\end{proof}

\begin{lemma}
\label{lemma-face-degeneracy-category}
The category $\Delta$ is the universal category
with objects $[n]$, $n \geq 0$ and morphisms
$\delta^n_j$ and $\sigma^n_j$ such that (a) every morphism is
a composition of these morphisms, (b) the relations
listed in Lemma \ref{lemma-relations-face-degeneracy} are satisfied,
and (c) any relation among the morphisms is a consquence of
those relations.
\end{lemma}

\begin{proof}
Omitted.
\end{proof}







\section{Simplicial objects}
\label{section-simplicial-object}

\begin{definition}
\label{definition-simplicial-object}
Let $\mathcal{C}$ be a category.
\begin{enumerate}
\item A {\it simplicial object $U$ of $\mathcal{C}$}
is a contravariant functor $U$ from $\Delta$ to
$\mathcal{C}$, in a formula:
$$
U : \Delta^{opp} \longrightarrow \mathcal{C}
$$
\item If $\mathcal{C}$ is the category of sets, then we call
$U$ a {\it simplicial set}.
\item If $\mathcal{C}$ is the category of abelian groups,
then we call $U$ a {\it simplicial abelian group}.
\item A {\it morphism of simplicial objects $U \to U'$}
is a transformation of functors.
\item The {\it category of simplicial objects of $\mathcal{C}$}
is denoted $\text{Simp}(\mathcal{C})$.
\end{enumerate}
\end{definition}

\noindent
This means there are objects $U([0]), U([1]), U([2]), \ldots$
and morphisms $U(\varphi) : U([n]) \to U([m])$,
where $\varphi$ is any nondecreasing map $\varphi : [m] \to [n]$. 

\medskip\noindent
In particular there is a unique morphism $U([0]) \to U([n])$ and there are
exactly $n + 1$ morphisms $U([n]) \to U([0])$ corresponding to
the $n + 1$ maps $[0] \to [n]$. Obviously we need some more notation
to be able to talk 
intelligently about these simplicial objects. We do this by considering
the morphisms we singled out in Section \ref{section-Delta} above.

\begin{lemma}
\label{lemma-characterize-simplicial-object}
Let $\mathcal{C}$ be a category.
\begin{enumerate}
\item Given a simplicial object $U$ in $\mathcal{C}$
we obtain a sequence of objects $U_n = U([n])$ endowed
with the morphisms $d^n_j = U(\delta^n_j) : U_n \to U_{n-1}$ and
$s^n_j = U(\sigma^n_j) : U_n \to U_{n + 1}$. These morphisms
satisfy the opposites of the relations displayed in
Lemma \ref{lemma-relations-face-degeneracy}.
\item Conversely, given a sequence of objects $U_n$ and morphisms
$d^n_j$, $s^n_j$ satisfying these relations there exists a unique
simplicial object $U$ in $\mathcal{C}$ such that $U_n = U([n])$,
$d^n_j = U(\delta^n_j)$, and $s^n_j = U(\sigma^n_j)$.
\item A morphism between simplicial objects $U$ and $U'$
is given by a family of morphisms $U_n \to U'_n$ commuting
with the morphisms $d^n_j$ and $s^n_j$.
\end{enumerate}
\end{lemma}

\begin{proof}
This follows from Lemma \ref{lemma-face-degeneracy-category}.
\end{proof}

\begin{remark}
\label{remark-relations}
By abuse of notation we sometimes write $d_i : U_n \to U_{n - 1}$
instead of $d^n_i$, and similarly for $s_i : U_n \to U_{n + 1}$.
The relations among the morphisms $d^n_i$ and $s^n_i$
may be expressed as follows:
\begin{enumerate}
\item If $i < j$, then $d_i \circ d_j = d_{j - 1} \circ d_i$.
\item If $i < j$, then $d_i \circ s_j = s_{j - 1} \circ d_i$.
\item We have $\text{id} = d_j \circ s_j = d_{j + 1} \circ s_j$.
\item If $i > j + 1$, then $d_i \circ s_j = s_j \circ d_{i - 1}$.
\item If $i \leq j$, then $s_i \circ s_j = s_{j + 1} \circ s_i$.
\end{enumerate}
This means that whenever the compositions on both the left and the
right are defined then the corresponding equality should hold.
\end{remark}

\noindent
We get a unique morphism $s^0_0 = U(\sigma^0_0) : U_0 \to U_1$ and
two morphisms $d^1_0 = U(\delta^1_0)$, and
$d^1_1 = U(\delta^1_1)$ which are morphisms $U_1 \to U_0$.
There are two morphisms $s^1_0 = U(\sigma^1_0)$, $s^1_1 = U(\sigma^1_1)$
which are morphsms $U_1 \to U_2$. Three morphisms
$d^2_0 = U(\delta^2_0)$, $d^2_1 = U(\delta^2_1)$, $d^2_2 = U(\delta^2_2)$
which are morphisms $U_3 \to U_2$. And so on.

\medskip\noindent
Pictorially we think of $U$ as follows:
$$
\xymatrix{
U_2
\ar@<2ex>[r]
\ar@<0ex>[r]
\ar@<-2ex>[r]
&
U_1 
\ar@<1ex>[r]
\ar@<-1ex>[r]
\ar@<1ex>[l]
\ar@<-1ex>[l]
&
U_0
\ar@<0ex>[l]
}
$$
Here the $d$-morphisms are the arrows pointing right and the 
$s$-morphisms are the arrows pointing left.

\begin{example}
\label{example-constant-simplicial-object}
The simplest example is the {\it constant} simplicial object with
value $X \in \text{Ob}(\mathcal{C})$. In other words, $U_n=X$ and
all maps are $\text{id}_X$.
\end{example}

\begin{example}
\label{example-fibre-products-simplicial-object}
Suppose that $Y\to X$ is a morphism of $C$ such that all
the fibred products $Y\times_X Y \times_X \ldots \times_X Y$ exist.
Then we set $U_n$ equal to the $(n + 1)$-fold fibre product,
and we let $\varphi: [n] \to [m]$ correspond to the map
(on ``coordinates'')
$(y_0,\ldots, y_m) \mapsto (y_{\varphi(0)},\ldots, y_{\varphi(n)})$.
In other words, the map $U_0 = Y \to U_1 = Y\times_X Y$ is the
diagonal map. The two maps $U_1 = Y\times_X Y \to U_0 = Y$ are the
projection maps.
\end{example}

\noindent
Geometrically Example \ref{example-fibre-products-simplicial-object}
above is an important example. It tells us that it is a good
idea to think of the maps $d^n_j : U_{n + 1} \to U_n$
as projection maps (forgetting the $j$th component),
and to think of the maps $s^n_j : U_n \to U_{n + 1}$
as diagonal maps (repeating the $j$th coordinate).
We will return to this in the sections below.

\begin{lemma}
\label{lemma-si-injective}
Let $\mathcal{C}$ be a category.
Let $U$ be a simplicial object of $\mathcal{C}$.
Each of the morphisms $s^n_i : U_n \to U_{n + 1}$
has a left inverse. In particular $s^n_i$ is a monomorphism.
\end{lemma}

\begin{proof}
This is true because $d_i^{n + 1} \circ s^n_i = \text{id}_{U_n}$.
\end{proof}

\section{Simplicial objects as presheaves}
\label{section-simplicial-presheaves}

\noindent
Another observation is that we may think of a simplicial
object of $\mathcal{C}$ as a presheaf with values in $\mathcal{C}$
over $\Delta$. See
Sites, Definition \ref{sites-definition-presheaf}.
And in fact, if $U$, $U'$ are simplicial objects
of $\mathcal{C}$, then we have
\begin{equation}
\label{simplicial-set-presheaf}
\text{Mor}(U, U') = \text{Mor}_{\textit{PSh}(\Delta)}(U, U').
\end{equation}
Some of the material below could be replace by the more
general constructions in the chapter on sites.
However, it seems a clearer picture arises from the
arguments specific to simplicial objects.
















\section{Cosimplicial objects}
\label{section-cosimplicial-object}

\noindent
A cosimplicial object of a category $\mathcal{C}$ could
be defined simply as a simplicial object of the
opposite category $\mathcal{C}^{opp}$. This is not
really how the human brain works, so we introduce
them separately here and point out some simple
properties.

\begin{definition}
\label{definition-cosimplicial-object}
Let $\mathcal{C}$ be a category.
\begin{enumerate}
\item A {\it cosimplicial object $U$ of $\mathcal{C}$}
is a covariant functor $U$ from $\Delta$ to
$\mathcal{C}$, in a formula:
$$
U : \Delta \longrightarrow \mathcal{C}
$$
\item If $\mathcal{C}$ is the category of sets, then we call
$U$ a {\it cosimplicial set}.
\item If $\mathcal{C}$ is the category of abelian groups,
then we call $U$ a {\it cosimplicial abelian group}.
\item A {\it morphism of cosimplicial objects $U \to U'$}
is a transformation of functors.
\item The {\it category of cosimplicial objects of $\mathcal{C}$}
is denoted $\text{CoSimp}(\mathcal{C})$.
\end{enumerate}
\end{definition}

\noindent
This means there are objects $U([0]), U([1]), U([2]), \ldots$
and morphisms $U(\varphi) : U([m]) \to U([n])$,
where $\varphi$ is any nondecreasing map $\varphi : [m] \to [n]$. 

\medskip\noindent
In particular there is a unique morphism $U([n]) \to U([0])$ and there are
exactly $n + 1$ morphisms $U([0]) \to U([n])$ corresponding to
the $n + 1$ maps $[0] \to [n]$. Obviously we need some more notation
to be able to talk intelligently about these simplicial objects.
We do this by considering the morphisms we singled out in
Section \ref{section-Delta} above.

\begin{lemma}
\label{lemma-characterize-cosimplicial-object}
Let $\mathcal{C}$ be a category.
\begin{enumerate}
\item Given a cosimplicial object $U$ in $\mathcal{C}$
we obtain a sequence of objects $U_n = U([n])$ endowed
with the morphisms $\delta^n_j = U(\delta^n_j) : U_{n - 1} \to U_n$ and
$\sigma^n_j = U(\sigma^n_j) : U_{n + 1} \to U_n$. These morphisms
satisfy the relations displayed in
Lemma \ref{lemma-relations-face-degeneracy}.
\item Conversely, given a sequence of objects $U_n$ and morphisms
$\delta^n_j$, $\sigma^n_j$ satisfying these relations there exists a unique
cosimplicial object $U$ in $\mathcal{C}$ such that $U_n = U([n])$,
$\delta^n_j = U(\delta^n_j)$, and $\sigma^n_j = U(\sigma^n_j)$.
\item A morphism between simplicial objects $U$ and $U'$
is given by a family of morphisms $U_n \to U'_n$ commuting
with the morphisms $\delta^n_j$ and $\sigma^n_j$.
\end{enumerate}
\end{lemma}

\begin{proof}
This follows from Lemma \ref{lemma-face-degeneracy-category}.
\end{proof}

\begin{remark}
\label{remark-relations-cosimplicial}
By abuse of notation we sometimes write $\delta_i : U_{n - 1} \to U_n$
instead of $\delta^n_i$, and similarly for $\sigma_i : U_{n + 1} \to U_n$.
The relations among the morphisms $\delta^n_i$ and $\sigma^n_i$
may be expressed as follows:
\begin{enumerate}
\item If $i < j$, then
$\delta_j \circ \delta_i = \delta_i \circ \delta_{j - 1}$.
\item If $i < j$, then
$\sigma_j \circ \delta_i = \delta_i \circ \sigma_{j - 1}$.
\item We have
$\text{id} = \sigma_j \circ \delta_j = \sigma_j \circ \delta_{j + 1}$.
\item If $i > j + 1$, then
$\sigma_j \circ \delta_i = \delta_{i - 1} \circ \sigma_j$.
\item If $i \leq j$, then
$\sigma_j \circ \sigma_i = \sigma_i \circ \sigma_{j + 1}$.
\end{enumerate}
This means that whenever the compositions on both the left and the
right are defined then the corresponding equality should hold.
\end{remark}

\noindent
We get a unique morphism $\sigma^0_0 = U(\sigma^0_0) : U_1 \to U_0$ and
two morphisms $\delta^1_0 = U(\delta^1_0)$, and
$\delta^1_1 = U(\delta^1_1)$ which are morphisms $U_0 \to U_1$.
There are two morphisms
$\sigma^1_0 = U(\sigma^1_0)$, $\sigma^1_1 = U(\sigma^1_1)$
which are morphsms $U_2 \to U_1$. Three morphisms
$\delta^2_0 = U(\delta^2_0)$, $\delta^2_1 = U(\delta^2_1)$,
$\delta^2_2 = U(\delta^2_2)$
which are morphisms $U_2 \to U_3$. And so on.

\medskip\noindent
Pictorially we think of $U$ as follows:
$$
\xymatrix{
U_0
\ar@<1ex>[r]
\ar@<-1ex>[r]
&
U_1
\ar@<0ex>[l]
\ar@<2ex>[r]
\ar@<0ex>[r]
\ar@<-2ex>[r]
&
U_2
\ar@<1ex>[l]
\ar@<-1ex>[l]
}
$$
Here the $\delta$-morphisms are the arrows pointing right and the 
$\sigma$-morphisms are the arrows pointing left.

\begin{example}
\label{example-constant-cosimplicial-object}
The simplest example is the {\it constant} cosimplicial object with
value $X \in \text{Ob}(\mathcal{C})$. In other words, $U_n=X$ and
all maps are $\text{id}_X$.
\end{example}

\begin{example}
\label{example-push-outs-simplicial-object}
Suppose that $Y\to X$ is a morphism of $C$ such that all
the push outs $Y\coprod_X Y \coprod_X \ldots \coprod_X Y$ exist.
Then we set $U_n$ equal to the $(n + 1)$-fold push out,
and we let $\varphi: [n] \to [m]$ correspond to the map
$$
(y \text{ in }i\text{th component})
\mapsto
(y \text{ in }\varphi(i)\text{th component})
$$
on ``coordinates''.
In other words, the map $U_1 = Y \coprod_X Y \to U_0 = Y$ is the
identity on each component.
The two maps $U_0 = Y \to U_1 = Y \coprod_X Y$ are the two
natural maps.
\end{example}

\begin{lemma}
\label{lemma-di-injective}
Let $\mathcal{C}$ be a category.
Let $U$ be a cosimplicial object of $\mathcal{C}$.
Each of the morphisms $\delta^n_i : U_{n - 1} \to U_n$
has a left inverse. In particular $\delta^n_i$ is a monomorphism.
\end{lemma}

\begin{proof}
This is true because
$\sigma_i^{n - 1} \circ \delta^n_i = \text{id}_{U_n}$
for $j < n$.
\end{proof}


























\section{Products of simplicial objects}
\label{section-products}

\noindent
Of course we should define the product of simplicial objects
as the product in the category of simplicial objects. This
may lead to the potentially confusing situation where the product exists
but is not described as below. To avoid this we define the product
directly as follows.

\begin{definition}
\label{definition-product}
Let $\mathcal{C}$ be a category.
Let $U$ and $V$ be simplicial objects of $\mathcal{C}$.
Assume the products $U_n \times V_n$ exist in $\mathcal{C}$.
The {\it product of $U$ and $V$} is the simplicial object
$U\times V$ defined as follows:
\begin{enumerate}
\item $(U \times V)_n = U_n \times V_n$,
\item $d^n_i = (d^n_i, d^n_i)$, and
\item $s^n_i = (s^n_i, s^n_i)$.
\end{enumerate}
In other words, $U\times V$ is the product of the presheaves
$U$ and $V$ on $\Delta$.
\end{definition}

\begin{lemma}
\label{lemma-product}
If $U$ and $V$ are simplicial objects in the category $\mathcal{C}$,
and if $U\times V$ exists, then we have
$$
\text{Mor}(W, U\times V) = 
\text{Mor}(W, U) \times
\text{Mor}(W, V)
$$
for any third simplicial object $W$ of $\mathcal{C}$.
\end{lemma}

\begin{proof}
Omitted.
\end{proof}














\section{Products of cosimplicial objects}
\label{section-products-cosimplicial}

\noindent
Of course we should define the product of cosimplicial objects
as the product in the category of cosimplicial objects. This
may lead to the potentially confusing situation where the product exists
but is not described as below. To avoid this we define the product
directly as follows.

\begin{definition}
\label{definition-product-cosimplicial-objects}
Let $\mathcal{C}$ be a category.
Let $U$ and $V$ be cosimplicial objects of $\mathcal{C}$.
Assume the products $U_n \times V_n$ exist in $\mathcal{C}$.
The {\it product of $U$ and $V$} is the cosimplicial object
$U\times V$ defined as follows:
\begin{enumerate}
\item $(U \times V)_n = U_n \times V_n$,
\item for any $\varphi : [n] \to [m]$ the map
$(U \times V)(\varphi) : U_n \times V_n \to U_m \times V_m$
is the product $U(\varphi) \times V(\varphi)$.
\end{enumerate}
\end{definition}

\begin{lemma}
\label{lemma-product-cosimplicial-objects}
If $U$ and $V$ are cosimplicial objects in the category $\mathcal{C}$,
and if $U\times V$ exists, then we have
$$
\text{Mor}(W, U\times V) = 
\text{Mor}(W, U) \times
\text{Mor}(W, V)
$$
for any third cosimplicial object $W$ of $\mathcal{C}$.
\end{lemma}

\begin{proof}
Omitted.
\end{proof}


















\section{Simplicial sets}
\label{section-simplicial-set}

\noindent
Let $U$ be a simplical set. It is a good idea to think of
$U_0$ as the {\it $0$-simplices}, the set $U_1$ as the
{\it $1$-simplices},
the set $U_2$ as the {\it $2$-simplices}, and so on.

\medskip\noindent
We think of the maps $s^n_j : U_n \to U_{n + 1}$ as
the map that associates to an $n$-simplex $A$ the degenerate
$(n + 1)$-simplex $B$ whose $(j, j + 1)$-edge is collapsed
to the vertex $j$ of $A$. We think of the map $d^n_j : U_n \to U_{n - 1}$
as the map that associates to an $n$-simplex $A$ one of the
faces, namely the face that omits the vertex $j$.
In this way it become possible to visualize the relations
among the maps $s^n_j$ and $d^n_j$ geometrically.

\begin{definition}
\label{definition-terminology-simplicial-sets}
Let $U$ be a simplicial set. 
We say {\it $x$ is a $n$-simplex of $U$} to signify that
$x$ is an element of $U_n$. We say that {\it $y$ is the $j$the
face of $x$} to signify that $d^n_jx = y$. We say that
{\it $z$ is the $j$th degeneracy of $x$} if $z = s^n_jx$.
A simplex is called {\it degenerate} if it is the degeneracy
of another simplex.
\end{definition}

\noindent
Here are a few fundamental examples.

\begin{example}
\label{example-simplex-simplicial-set}
For every $n \geq 0$ we denote $\Delta[n]$ the simplicial set
\begin{align*}
\Delta^{opp} & \longrightarrow \textit{Sets} \\
[k] & \longmapsto \text{Mor}_{\Delta}([k], [n])
\end{align*}
We leave it to the reader to verify the following statements.
Every $m$-simplex of $\Delta[n]$ with $m > n$ is degenerate.
There is a unique nondegenerate $n$-simplex of $\Delta[n]$,
namely $\text{id}_{[n]}$.
\end{example}

\begin{lemma}
\label{lemma-simplex-map}
Let $U$ be a simplicial set. Let $n \geq 0$ be an integer.
There is a canonical bijection
$$
\text{Mor}(\Delta[n], U)
\longrightarrow
U_n
$$
which maps a morphism $\varphi$ to the value of $\varphi$
on the unique nondegenerate $n$-simplex of $\Delta[n]$.
\end{lemma}

\begin{proof}
Omitted.
\end{proof}

\begin{example}
\label{example-simplex-category}
Consider the category $\Delta/[n]$ of objects over $[n]$
in $\Delta$, see
Categories, Example \ref{categories-example-category-over-X}.
There is a functor $p : \Delta/[n] \to \Delta$.
The fibre category of $p$ over $[k]$, see
Categories, Section \ref{categories-section-fibred-groupoids},
has as objects the
set $\Delta[n]_k$ of $k$-simplices in $\Delta[n]$, and as
morphisms only identities. For every morphism
$\varphi : [k] \to [l]$ of $\Delta$, and every object $\psi : [l] \to [n]$
in the fibre category over $[l]$ there is a unique
object over $[k]$ with a morphism covering $\varphi$, namely
$\psi \circ \varphi : [k] \to [n]$. Thus $\Delta/[n]$
is fibred in sets over $\Delta$. In other words, we may
think of $\Delta/[n]$ as a presheaf of sets over $\Delta$.
See also, Categories,
Example \ref{categories-example-fibred-category-from-functor-of-points}.
And this presheaf of sets agrees with the simplicial set
$\Delta[n]$. In particular, from Equation
(\ref{simplicial-set-presheaf}) and 
Lemma \ref{lemma-simplex-map} above
we get the formula
$$
\text{Mor}_{\textit{PSh}(\Delta)}(\Delta/[n], U) = U_n
$$
for any simplicial set $U$.
\end{example}

\begin{lemma}
\label{lemma-product-degenerate}
Let $U$, $V$ be simplicial sets.
Let $a, b \geq 0$ be integers.
Assume every $n$-simplex of $U$ is degenerate if $n > a$.
Assume every $n$-simplex of $V$ is degenerate if $n > b$.
Then every $n$-simplex of $U \times V$ is degenerate
if $n > a + b$.
\end{lemma}

\begin{proof}
Suppose $n > a + b$. Let $(u,v) \in (U\times V)_n = U_n \times V_n$.
By assumption, there exists a $\alpha : [n] \to [a]$ and a
$u' \in U_a$ and a $\beta : [n] \to [b]$ and a $v' \in V_b$
such that $u = U(\alpha)(u')$ and $v = V(\beta)(v')$. Because
$n > a + b$, there exists an $0 \leq i \leq a + b$ such that
$\alpha(i) = \alpha(i + 1)$ and
$\beta(i) = \beta(i + 1)$. It follows immediately
that $(u,v)$ is in the image of $s^{n - 1}_i$.
\end{proof}



\section{Products with simplicial sets}
\label{section-product-with-simplicial-sets}

\noindent
Let $\mathcal{C}$ be a category.
Let $U$ be a simplicial set.
Let $V$ be a simplicial object of $\mathcal{C}$.
We can consider the covariant functor which associates
to a simplicial object $W$ of $\mathcal{C}$
the set
\begin{equation}
\label{equation-functor-product-with-simplicial-set}
\left\{
(f_{n, u} : V_n \to W_n)_{n \geq 0, u \in U_n}
\text{ such that }
\begin{matrix}
\forall \varphi : [m] \to [n] \\
f_{m, \varphi(u)} \circ V(\varphi) = W(\varphi) \circ f_{n, u}
\end{matrix}
\right\}
\end{equation}
If this functor is of the form
$\text{Mor}_{\text{Simp}(\mathcal{C})}(Q, -)$
then we can think of $Q$ as the product of $U$ with $V$.
Instead of formalizing this in this way we just directly
define the product as follows.

\begin{definition}
\label{definition-product-with-simplicial-set}
Let $\mathcal{C}$ be a category such that the coproduct of
any two objects of $\mathcal{C}$ exists. Let
$U$ be a simplicial set. Let $V$ be a simplicial
object of $\mathcal{C}$. Assume that each $U_n$ is
finite nonempty. In this case we define
{\it the product 
$
U \times V
$
of $U$ and $V$}
to be the simplicial object of $\mathcal{C}$ whose
$n$th term is the object
$$
(U \times V)_n = \coprod\nolimits_{u\in U_n} V_n
$$
with maps for $\varphi : [m] \to [n]$ given by the
morphism
$$
\coprod\nolimits_{u\in U_n} V_n
\longrightarrow
\coprod\nolimits_{u'\in U_m} V_m
$$
which maps the component $V_n$ corresponding to $u$ to the
component $V_m$ corresponding to $u' = U(\varphi)(u)$
via the morphism $V(\varphi)$.
More loosely, if all of the coproducts displayed above
exist (without assuming anything about $\mathcal{C}$)
we will say that the {\it product $U \times V$ exists}.
\end{definition}

\begin{lemma}
\label{lemma-check-product-with-simplicial-set}
Let $\mathcal{C}$ be a category such that the coproduct of
any two objects of $\mathcal{C}$ exists. Let
$U$ be a simplicial set. Let $V$ be a simplicial
object of $\mathcal{C}$. Assume that each $U_n$ is
finite nonempty. The functor 
$W \mapsto \text{Mor}_{\text{Simp}(\mathcal{C})}(U\times V, W)$
is canonically isomorphic to the functor which
maps $W$ to the set in
Equation (\ref{equation-functor-product-with-simplicial-set}).
\end{lemma}

\begin{proof}
Omitted.
\end{proof}

\begin{lemma}
\label{lemma-back-to-U}
Let $\mathcal{C}$ be a category such that the coproduct of
any two objects of $\mathcal{C}$ exists. Let us temporarily
denote $\textit{FSSets}$ the category of simplicial sets
all of whose components are finite nonempty.
\begin{enumerate}
\item The rule $(U, V) \mapsto U \times V$
defines a functor
$\textit{FSSets} \times \text{Simp}(\mathcal{C})
\to \text{Simp}(\mathcal{C})$.
\item For every $U$, $V$ as above
there is a canonical map of simplicial objects
$$
U \times V \longrightarrow V
$$
defined by taking the identity on each component of
$(U \times V)_n = \coprod_u V_n$.
\end{enumerate}
\end{lemma}

\begin{proof}
Omitted.
\end{proof}

\noindent
We briefly study a special case of the construction
above. Let $\mathcal{C}$ be a category.
Let $X$ be an object of $\mathcal{C}$.
Let $k \geq 0$ be an integer.
If all coproducts $X \coprod \ldots \coprod X$ exist
then according to the definition above the product
$$
X \times \Delta[k]
$$
exists, where we think of $X$ as the corresponding constant 
simplicial object.

\begin{lemma}
\label{lemma-morphism-from-coproduct}
With $X$ and $k$ as above.
For any simplicial object $V$ of
$\mathcal{C}$ we have the following
canonical bijection
$$
\text{Mor}_{\text{Simp}(\mathcal{C})}(X \times \Delta[k], V)
\longrightarrow
\text{Mor}_{\mathcal{C}}(X, V_k).
$$
wich maps $\gamma$ to the restriction of the
morphism $\gamma_k$ to the component corresponding
to $\text{id}_{[k]}$.
Similarly, for any $n \geq k$, if $W$ is an
$n$-truncated simplicial object
of $\mathcal{C}$, then we have
$$
\text{Mor}_{\text{Simp}_n(\mathcal{C})}(\text{sk}_n(X \times \Delta[k]), W)
=
\text{Mor}_{\mathcal{C}}(X, W_k).
$$
\end{lemma}

\begin{proof}
A morphism $\gamma : X \times \Delta[k] \to V$ is given by
a family of morphisms $\gamma_\alpha : X \to V_n$ where
$\alpha : [n] \to [k]$. The morphisms have to satisfy the
rules that for all $\varphi : [m] \to [n]$ the diagrams
$$
\xymatrix{
X \ar[r]^{\gamma_\alpha} \ar[d]^{\text{id}_X} & V_n \ar[d]^{V(\varphi)} \\
X \ar[r]^{\gamma_{\alpha \circ \varphi}} & V_m 
}
$$
commute. Taking $\alpha = \text{id}_{[k]}$, we see that
for any $\varphi : [m] \to [k]$ we have $\gamma_\varphi =
V(\varphi) \circ \gamma_{\text{id}_{[k]}}$. Thus the morphism
$\gamma$ is determined by the value of $\gamma$ on the
component corresponding to $\text{id}_{[k]}$. Conversely,
given such a morphism $f : X \to V_k$ we easily
construct a morphism $\gamma$ by putting
$\gamma_\alpha = V(\alpha) \circ f$.

\medskip\noindent
The truncated case is similar, and left to the reader.
\end{proof}

\noindent
A particular example of this is the case $k = 0$.
In this case the formula of the lemma just says
that
$$
\text{Mor}_\mathcal{C}(X, V_0) 
=
\text{Mor}_{\text{Simp}(\mathcal{C})}(X, V)
$$
where on the right hand side $X$ indicates the
constant simplicial object with value $X$. We will
use this formula without further mention in the
following.

\section{Internal Hom}
\label{section-internal-hom}

\noindent
Let $\mathcal{C}$ be a category with finite nonempty
products. Let $U$, $V$ be simplicial objects $\mathcal{C}$.
In some cases the functor
\begin{eqnarray*}
\text{Simp}(\mathcal{C})^{opp} & \longrightarrow & \textit{Sets} \\
W & \longmapsto & \text{Mor}_{\text{Simp}(\mathcal{C})}(W \times V, U)
\end{eqnarray*}
is representable. In this case we denote $\textit{Hom}(V, U)$
the resulting simplicial object of $\mathcal{C}$, and we say
that the {\it internal hom of $V$ into $U$ exists}. Moreover,
in this case we would have
\begin{eqnarray*}
\text{Mor}_{\mathcal{C}}(X, \textit{Hom}(V, U)_n)
& = &
\text{Mor}_{\text{Simp}(\mathcal{C})}(X \times \Delta[n], \textit{Hom}(V, U))
\\
& = &
\text{Mor}_{\text{Simp}(\mathcal{C})}(X \times \Delta[n]\times V, U) \\
& = &
\text{Mor}_{\text{Simp}(\mathcal{C})}(X, \textit{Hom}(\Delta[n] \times V, U))
\\
& = &
\text{Mor}_{\mathcal{C}}(X, \textit{Hom}(\Delta[n] \times V, U)_0)
\end{eqnarray*}
provided that $\textit{Hom}(\Delta[n] \times V, U)$
exists also. Here we have used the material from Section
\ref{section-product-with-simplicial-sets}.

\medskip\noindent
The lesson we learn from this is that, given $U$ and $V$,
if we want to construct the internal hom then we should try to  
construct the objects
$$
\textit{Hom}(\Delta[n] \times V, U)_0
$$
because these should be the $n$th term of $\textit{Hom}(V, U)$.
In the next section we study a construction of simplicial objects
``$\text{Hom}(\Delta[n], U)$''.


\section{Hom from simplicial sets}
\label{section-hom-from-simplicial-sets}

\noindent
Motivated by the discussion on internal hom we define
what should be the simplicial object classifying
morphisms from a simplicial set into a given
simplicial object of the category $\mathcal{C}$.

\begin{definition}
\label{definition-hom-from-simplicial-set}
Let $\mathcal{C}$ be a category such that the coproduct
of any two objects exists.
Let $U$ be a simplicial set, with $U_n$ finite nonempty
for all $n \geq 0$.
Let $V$ be a simplicial object of $\mathcal{C}$.
We denote $\text{Hom}(U, V)$ any simplicial object of
$\mathcal{C}$ such that
$$
\text{Mor}_{\text{Simp}(\mathcal{C})}(W, \text{Hom}(U, V))
=
\text{Mor}_{\text{Simp}(\mathcal{C})}(W \times U, V)
$$
functorially in the simplicial object $W$ of $\mathcal{C}$.
\end{definition}

\noindent
Of course $\text{Hom}(U, V)$ need not exist.
Also, by the discussion in Section \ref{section-internal-hom}
we expect that if it does exist, then
$\text{Hom}(U, V)_n = \text{Hom}(U \times \Delta[n], V)_0$.
We do not use the italic notation for these Hom objects
since $\text{Hom}(U, V)$ is not an internal hom.

\begin{lemma}
\label{lemma-exists-hom-0-from-simplicial-set}
Assume the category $\mathcal{C}$
has coproducts of any two objects and countable
limits. Let $U$ be a simplicial set, with $U_n$ finite nonempty
for all $n \geq 0$.
Let $V$ be a simplicial object of $\mathcal{C}$.
Then the functor
\begin{eqnarray*}
\mathcal{C}^{opp} & \longrightarrow & \textit{Sets} \\
X
& \longmapsto &
\text{Mor}_{\text{Simp}(\mathcal{C})}(X \times U, V)
\end{eqnarray*}
is representable.
\end{lemma}

\begin{proof}
A morphism from $X \times U$ into $V$ is given by a collection
of morphisms $f_u : X \to V_n$ with $n \geq 0$ and $u \in U_n$.
And such a collection actually defines a morphism if and only
if for all $\varphi : [m] \to [n]$ all the diagrams
$$
\xymatrix{
X \ar[r]^{f_u} \ar[d]_{\text{id}_X} & V_n \ar[d]^{V(\varphi)} \\
X \ar[r]^{f_{U(\varphi)(u)}} & V_m 
}
$$
commute. Thus it is natural to introduce a category
$\mathcal{U}$ and a functor
$\mathcal{V} : \mathcal{U}^{opp} \to \mathcal{C}$
as follows:
\begin{enumerate}
\item The set of objects of $\mathcal{U}$ is
$\coprod_{n \geq 0} U_n$,
\item a morphism from $u' \in U_m$ to $u \in U_n$
is a $\varphi : [m] \to [n]$ such that $U(\varphi)(u) = u'$
\item for $u \in U_n$ we set $\mathcal{V}(u) = V_n$, and
\item for $\varphi : [m] \to [n]$ such that $U(\varphi)(u) = u'$
we set $\mathcal{V}(\varphi) = V(\varphi) : V_n \to V_m$.
\end{enumerate}
At this point it is clear that our functor is nothing but the
functor defining
$$
\text{lim}_{\mathcal{U}^{opp}} \mathcal{V}
$$
Thus if $\mathcal{C}$ has countable limits then this limit
and hence an object representing the functor of the lemma
exist.
\end{proof}

\begin{lemma}
\label{lemma-exists-hom-0-from-simplicial-set-finite}
Assume the category $\mathcal{C}$
has coproducts of any two objects and finite
limits. Let $U$ be a simplicial set, with $U_n$ finite nonempty
for all $n \geq 0$. Assume that all $n$-simplices
of $U$ are degenerate for all $n \gg 0$.
Let $V$ be a simplicial object of $\mathcal{C}$.
Then the functor
\begin{eqnarray*}
\mathcal{C}^{opp} & \longrightarrow & \textit{Sets} \\
X
& \longmapsto &
\text{Mor}_{\text{Simp}(\mathcal{C})}(X \times U, V)
\end{eqnarray*}
is representable.
\end{lemma}

\begin{proof}
We have to show that the category $\mathcal{U}$ described
in the proof of Lemma \ref{lemma-exists-hom-0-from-simplicial-set}
has a finite subcategory $\mathcal{U}'$ such that the limit
of $\mathcal{V}$ over $\mathcal{U}'$ is the same as the
limit of $\mathcal{V}$ over $\mathcal{U}$. We will use
Categories, Lemma \ref{categories-lemma-limit-final-subcategory}.
For $m > 0$ let $\mathcal{U}_{\leq m}$ denote the full
subcategory with objects $\coprod_{0 \leq n \leq m} U_m$.
Let $m_0$ be an integer such that every $n$-simplex
of the simplicial set $U$ is degenerate if $n > m_0$.
For any $m \geq m_0$ large enough, the subcategory
$\mathcal{U}_{\leq m}$ satisfies property (1) of the lemma
cited above.

\medskip\noindent
Suppose that $u \in U_n$ and
$u' \in U_{n'}$ with $n, n' \leq m_0$ and suppose that
$\varphi : [k] \to [n]$, $\varphi' : [k] \to [n']$
are morphisms such that $U(\varphi)(u) = U(\varphi')(u')$.
A simple combinatorial argument shows that if $k > 2m_0$,
then there exists an index $0 \leq i \leq 2m_0$ such that
$\varphi(i) =\varphi(i + 1)$ and $\varphi'(i) = \varphi'(i + 1)$.
(The pidgeon hole principle would tell you this works if
$k > m_0^2$ which is good enough for the argument below
anyways.) Hence, if $k > 2m_0$, we may write
$\varphi = \psi \circ \sigma^{k - 1}_i$ and
$\varphi' = \psi' \circ \sigma^{k - 1}_i$ for some
$\psi : [k - 1] \to [n]$ and some $\psi' : [k - 1] \to [n']$.
Since $s^{k - 1}_i : U_{k - 1} \to U_k$ is injective,
see Lemma \ref{lemma-si-injective}, we conclude that
$U(\psi)(u) = U(\psi')(u')$ also. Continuing in this
fashion we conclude that given morphisms
$u \to z$ and $u' \to z$ of $\mathcal{U}$
with $u, u' \in \mathcal{U}_{\leq m_0}$, there exists
a commutative diagram
$$
\xymatrix{
u \ar[rd] \ar[rrd] & & \\
& a \ar[r] & z \\
u' \ar[ru] \ar[rru] 
}
$$
with $a \in \mathcal{U}_{\leq 2m_0}$.

\medskip\noindent
It is easy to deduce from this that the finite subcategory
$\mathcal{U}_{\leq 2m_0}$ works. Namely, suppose given
$x' \in U_n$ and $x'' \in U_{n'}$ with $n, n' \leq 2m_0$ as well as
morphisms $x' \to x$ and $x'' \to x$ of $\mathcal{U}$
with the same target. By our choice of $m_0$ we can
find objects $u, u'$ of $\mathcal{U}_{\leq m_0}$ and
morphisms $u \to x'$, $u' \to x''$.
By the above we can find $a \in \mathcal{U}_{\leq 2m_0}$
and morphisms $u \to a$, $u' \to a$ such that
$$
\xymatrix{
u \ar[rd] \ar[rrd] \ar[r] & x' \ar[rd] & \\
& a \ar[r] & x \\
u' \ar[ru] \ar[rru] \ar[r] & x'' \ar[ru] &
}
$$
is commutative. Turning this diagram 90 degrees clockwise
we get the desired diagram as in (2) of the
cited lemma.
\end{proof}

\begin{lemma}
\label{lemma-exists-hom-from-simplicial-set-finite}
Assume the category $\mathcal{C}$
has coproducts of any two objects and finite
limits. Let $U$ be a simplicial set, with $U_n$ finite nonempty
for all $n \geq 0$. Assume that all $n$-simplices
of $U$ are degenerate for all $n \gg 0$.
Let $V$ be a simplicial object of $\mathcal{C}$.
Then $\text{Hom}(U, V)$ exists, moreover
we have the expected equalities
$$
\text{Hom}(U, V)_n = \text{Hom}(U \times \Delta[n], V)_0.
$$
\end{lemma}

\begin{proof}
We construct this simplicial object as follows.
For $n \geq 0$ let $\text{Hom}(U, V)_n$ denote
the object of $\mathcal{C}$ representing the
functor
$$
X
\longmapsto
\text{Mor}_{\text{Simp}(\mathcal{C})}(X \times U \times \Delta[n], V)
$$
This exists by Lemma \ref{lemma-exists-hom-0-from-simplicial-set-finite} 
because $U \times \Delta[n]$ is a simplicial set with finite
sets of simplices and no nondegenerate simplices in high enough degree,
see Lemma \ref{lemma-product-degenerate}.
For $\varphi : [m] \to [n]$ we obtain an induced map of simplicial
sets $\varphi : \Delta[m] \to \Delta[n]$. Hence we obtain a morphism
$X \times U \times \Delta[m] \to X \times U \times \Delta[n]$
functorial in $X$, and hence a transformation of functors,
which in turn gives
$$
\text{Hom}(U, V)(\varphi) :
\text{Hom}(U, V)_n
\longrightarrow
\text{Hom}(U, V)_m.
$$
Clearly this defines a contravariant functor
$\text{Hom}(U, V)$ from
$\Delta$ into the category $\mathcal{C}$.
In other words, we have a simplicial object of $\mathcal{C}$.

\medskip\noindent
We have to show that $\text{Hom}(U, V)$ satisfies the desired
universal property
$$
\text{Mor}_{\text{Simp}(\mathcal{C})}(W, \text{Hom}(U, V))
=
\text{Mor}_{\text{Simp}(\mathcal{C})}(W \times U, V)
$$
To see this, let $f : W \to \text{Hom}(U, V)$ be given.
We want to construct the element $f' : W \times U \to V$
of the right hand side.
By construction, each $f_n : W_n \to \text{Hom}(U, V)_n$
corresponds to a morphism
$f_n : W_n \times U \times \Delta[n] \to V$. Further,
for every morphism $\varphi : [m] \to [n]$ the
diagram 
$$
\xymatrix{
W_n \times U \times \Delta[m]
\ar[rr]_{W(\varphi)\times \text{id} \times \text{id}}
\ar[d]_{\text{id} \times \text{id} \times \varphi} & &
W_m \times U \times \Delta[m] \ar[d]^{f_m} \\
W_n \times U \times \Delta[n] \ar[rr]^{f_n} & & V
}
$$
is commutative. For $\psi : [n] \to [k]$ in $(\Delta[n])_k$
we denote $(f_n)_{k, \psi} : W_n \times U_k \to V_k$
the component of $(f_n)_k$ corresponding to the element
$\psi$. We define $f'_n : W_n \times U_n \to V_n$
as $f'_n = (f_n)_{n, \text{id}}$, in other words, as 
the restriction of
$(f_n)_n : W_n \times U_n \times (\Delta[n])_n \to V_n$
to $W_n \times U_n \times \text{id}_{[n]}$.
To see that the collection $(f'_n)$ defines a
morphism of simplicial objects, we have to show
for any $\varphi : [m] \to [n]$ that
$V(\varphi) \circ f'_n =
f'_m \circ W(\varphi) \times U(\varphi)$.
The commutative diagram above says that
$(f_n)_{m, \varphi} : W_n \times U_m \to V_m$
is equal to
$(f_m)_{m, \text{id}} \circ W(\varphi) :
W_n \times U_m \to V_m$.
But then the fact that $f_n$ is a morphism of simplicial
objects implies that the diagram
$$
\xymatrix{
W_n \times U_n \times (\Delta[n])_n
\ar[r]_-{(f_n)_n}
\ar[d]_{\text{id}\times U(\varphi) \times \varphi}
& V_n \ar[d]^{V(\varphi)} \\
W_n \times U_m \times (\Delta[n])_m \ar[r]^-{(f_n)_m} & V_m
}
$$
is commutative. And this implies that
$(f_n)_{m, \varphi} \circ U(\varphi)$ is
equal to $V(\varphi) \circ (f_n)_{n, \text{id}}$.
Alltogether we obtain
$
V(\varphi) \circ (f_n)_{n, \text{id}}
=
(f_n)_{m, \varphi} \circ U(\varphi)
=
(f_m)_{m, \text{id}} \circ W(\varphi)\circ U(\varphi)
=
(f_m)_{m, \text{id}} \circ W(\varphi)\times U(\varphi)
$
as desired.

\medskip\noindent
On the other hand, given a morphism
$f' : W \times U \to V$ we define
a morphism $f : W \to \text{Hom}(U, V)$ 
as follows. By Lemma \ref{lemma-morphism-from-coproduct} the morphisms
$\text{id} : W_n \to W_n$ corresponds to a unique
morphism $c_n : W_n \times \Delta[n] \to W$.
Hence we can consider the composition
$$
W_n \times \Delta[n] \times U
\xrightarrow{c_n}
W \times U
\xrightarrow{f'}
V.
$$
By construction this corresponds to a unique morphism
$f_n : W_n \to \text{Hom}(U, V)_n$. We leave it to the reader
to see that these define a morphism of simplicial sets as
desired.

\medskip\noindent
We also leave it to the reader to see that
$f \mapsto f'$ and $f' \mapsto f$ are mutually inverse
operations.
\end{proof}

\noindent
We spell out the construction above in a special case.
Let $X$ be an object of a category $\mathcal{C}$.
Assume that self products $X \times \ldots \times X$ exist.
Let $k$ be an integer.
Consider the simplicial object $U$ with terms
$$
U_n = \prod\nolimits_{\alpha \in \text{Mor}([k], [n])} X
$$
and maps given $\varphi : [m] \to [n]$ 
\begin{eqnarray*}
U(\varphi) :
\prod\nolimits_{\alpha \in \text{Mor}([k], [n])} X
& \longrightarrow &
\prod\nolimits_{\alpha' \in \text{Mor}([k], [m])} X \\
(f_{\alpha})_{\alpha} & \longmapsto & 
(f_{\varphi \circ \alpha'})_{\alpha'}
\end{eqnarray*}
In terms of ``coordinates'', the element $(x_\alpha)_\alpha$
is mapped to the element $(x_{\varphi \circ \alpha'})_{\alpha'}$.
We claim this object is equal to
$$
\text{Hom}(\Delta[k], X)
$$
where we think of $X$ as the constant simplicial object $X$.

\begin{lemma}
\label{lemma-morphism-into-product}
With $X$, $k$ and $U$ as above.
\begin{enumerate}
\item For any simplicial object $V$ of
$\mathcal{C}$ we have the following
canonical bijection
$$
\text{Mor}_{\text{Simp}(\mathcal{C})}(V, U)
\longrightarrow
\text{Mor}_{\mathcal{C}}(V_k, X).
$$
wich maps $\gamma$ to the morphism $\gamma_k$ composed with
the projection onto the factor corresponding to $\text{id}_{[k]}$.
\item Similarly, if $W$ is an $k$-truncated simplicial object
of $\mathcal{C}$, then we have
$$
\text{Mor}_{\text{Simp}_k(\mathcal{C})}(W, \text{sk}_k U)
=
\text{Mor}_{\mathcal{C}}(W_k, X).
$$
\item The object $U$ constructed above is an
incarnation of $\text{Hom}(\Delta[k], X)$.
\end{enumerate}
\end{lemma}

\begin{proof}
We first prove (1).
Suppose that $\gamma : V \to U$ is a morphism.
This is given by a family of morphisms
$\gamma_{\alpha} : V_n \to X$ for $\gamma : [k] \to [n]$.
The morphisms have to satisfy the
rules that for all $\varphi : [m] \to [n]$ the diagrams
$$
\xymatrix{
X \ar[d]^{\text{id}_X} &
V_n \ar[d]^{V(\varphi)}
\ar[l]^{\gamma_{\varphi \circ \alpha'}} \\
X &
V_m \ar[l]_{\gamma_{\alpha'}}
}
$$
commute for all $\alpha' : [k] \to [m]$.
Taking $\alpha' = \text{id}_{[k]}$, we see that
for any $\varphi : [k] \to [n]$ we have $\gamma_\varphi =
\gamma_{\text{id}_{[k]}} \circ V(\varphi)$. Thus the morphism
$\gamma$ is determined by the component of $\gamma_k$
corresponding to $\text{id}_{[k]}$. Conversely,
given such a morphism $f : V_k \to X$ we easily
construct a morphism $\gamma$ by putting
$\gamma_\alpha = f \circ V(\alpha)$.

\medskip\noindent
The truncated case is similar, and left to the reader.

\medskip\noindent
To see (3) we argue as follows:
\begin{eqnarray*}
\text{Mor}(V, \text{Hom}(\Delta[k], X)) & = &
\text{Mor}(V \times \Delta[k], X) \\
& = & \{ (f_n : V_n \times \Delta[k]_n \to X) \mid
f_n \text{ compatible}\} \\
& = & \{ (f_n : V_n \to \prod\nolimits_{\Delta[k]_n} X) \mid
f_n \text{ compatible}\} \\
& = & \text{Mor}(V, U)
\end{eqnarray*}
Thus $U$ and $\text{Hom}(\Delta[k], X)$ define the same
functor on the category of simplicial objects and
hence are canonically isomorphic.
\end{proof}

















\section{Splitting simplicial objects}
\label{section-splitting}

\noindent
A subobject $N$ of an object $X$ of the category $\mathcal{C}$
is an object $N$ of $\mathcal{C}$ together with a monomorphism
$N \to X$. Of course we say (by abouse of notation) that
the subobjects $N$, $N'$ are equal if there exists an isomorphism
$N \to N'$ compatible with the morphisms to $X$. The collection
of subobjects forms a partially ordered set. (Because of our
conventions on categories; not true for category of spaces
up to homotopy for example.)

\begin{definition}
\label{definition-split}
Let $\mathcal{C}$ be a category which admits finite nonempty coproducts.
We say a simplicial object $U$ of $\mathcal{C}$ is split
if there exist subobjects $N(U_m)$ of $U_m$, $m \geq 0$
with the property that
\begin{equation}
\label{equation-splitting}
\coprod\nolimits_{\varphi : [n] \to [m]\text{ surjective}}
N(U_m)
\longrightarrow
U_n
\end{equation}
is an isomorphism for all $n \geq 0$.
\end{definition}

\noindent
If this is the case, then $N(U_0) = U_0$. Next, we have
$U_1 = U_0 \coprod N(U_1)$. Second we have
$$
U_2 = U_0 \coprod N(U_1) \coprod N(U_1) \coprod N(U_2).
$$
It turns out that in many categories $\mathcal{C}$
every simplicial object is split.

\begin{lemma}
\label{lemma-splitting-simplicial-sets}
Let $U$ be a simplicial set.
Then $U$ has a splitting
with $N(U_m)$ equal to the set of 
nondegenerate $m$-simplices.
\end{lemma}

\begin{proof}
Let $x \in U_n$. Suppose that
there are surjections $\varphi : [n] \to [k]$
and $\psi : [n] \to [l]$ and nondegenerate simplices
$y \in U_k$, $z \in U_l$ such that $x = U(\varphi)(y)$
and $x = U(\psi)(z)$. Choose a right inverse $\xi : [l] \to [n]$
of $\psi$, i.e., $\psi \circ \xi = \text{id}_{[l]}$.
Then $z = U(\xi)(x)$. Hence $z = U(\xi)(x) = U(\varphi \circ \xi)(y)$.
Since $z$ is nondegenerate we conclude that $\varphi \circ \xi :
[l] \to [k]$ is surjective, and hence $l \geq k$. Similarly
$k \geq l$. Hence we see that $\varphi \circ \xi : [l] \to [k]$
has to be the identity map for any choice of right inverse
$\xi$ of $\psi$. This easily implies that $\psi = \varphi$.
\end{proof}

\noindent
Of course it can happen that a map of simplicial sets
maps a nondegenerate $n$-simplex to a degenerate $n$-simplex.
Thus the splitting of Lemma \ref{lemma-splitting-simplicial-sets}
is not functorial. Here is a case where it is functorial.

\begin{lemma}
\label{lemma-injective-map-simplicial-sets}
Let $f : U \to V$ be a morphism of simplicial sets.
Suppose that (a) the image of every nondegenerate simplex of
$U$ is a nondegerate simplex of $V$ and (b)
no two nondegenerate simplices of $U$ are mapped
to the same simplex of $V$.
Then $f_n$ is injective for all $n$.
Same holds with ``injective'' replaced by
``surjective'' or ``bijective''.
\end{lemma}

\begin{proof}
Under hypothesis (a) we see that the map $f$ preserves
the disjoint union decompositions of the splitting
of Lemma \ref{lemma-splitting-simplicial-sets}, in other words
that we get commutative diagrams
$$
\xymatrix{
\coprod\nolimits_{\varphi : [n] \to [m]\text{ surjective}}
N(U_m)
\ar[r] \ar[d] &
U_n \ar[d] \\
\coprod\nolimits_{\varphi : [n] \to [m]\text{ surjective}}
N(V_m)
\ar[r] &
V_n.
}
$$
And then (b) clearly shows that the left vertical arrow is
injective (resp.\ surjective, resp.\ bijective).
\end{proof}

\begin{lemma}
\label{lemma-simplicial-set-n-skel-sub}
Let $U$ be a simplicial set.
Let $n \geq 0$ be an integer.
The rule
$$
U'_m = \bigcup\nolimits_{\varphi : [m] \to [i],\ i\leq n} \text{Im}(U(\varphi))
$$
defines a sub simplicial set $U' \subset U$ with
$U'_i = U_i$ for $i \leq n$.
Moreover, all $m$-simplices of $U'$ are degenerate for
all $m > n$.
\end{lemma}

\begin{proof}
If $x \in U_m$ and $x = U(\varphi)(y)$
for some $y \in U_i$, $i \leq n$ and some $\varphi : [m] \to [i]$
then any image $U(\psi)(x)$ for any $\psi : [m'] \to [m]$ is
equal to $U(\varphi \circ \psi)(y)$ and $\varphi \circ \psi :
[m'] \to [i]$. Hence $U'$ is a simplicial set. By construction
all simplices in dimension $n + 1$ and higher are degenerate.
\end{proof}

\begin{lemma}
\label{lemma-splitting-simplicial-groups}
Let $U$ be a simplicial abelian group.
Then $U$ has a splitting obtained by taking $N(U_0) = U_0$ and
for $m \geq 1$ taking
$$
N(U_m) = \bigcap\nolimits_{i = 0}^{m - 1} \text{Ker}(d^m_i).
$$
Moreover, this splitting is functorial on the category
of simplicial abelian groups.
\end{lemma}

\begin{proof}
By induction on $n$ we will show that the choice of $N(U_m)$
in the lemma garantees that (\ref{equation-splitting}) is
an isomorphism for $m \leq n$. This is clear for $n = 0$.
In the rest of this proof we are going to
drop the superscripts from the maps $d_i$ and $s_i$ in order
to improve readability. We will also repeatedly use the relations
from Remark \ref{remark-relations}.

\medskip\noindent
First we make a general remark.
For $0 \leq i \leq m$ and $z \in U_m$ we have
$d_i(s_i(z)) = z$. Hence we can write 
any $x \in U_{m + 1}$ uniquely as
$x = x' + x''$ with $d_i(x') = 0$
and $x'' \in \text{Im}(s_i)$
by taking $x' = (x - s_i(d_i(x)))$ and
$x'' = s_i(d_i(x))$. Moreover, the element
$z \in U_m$ such that $x'' = s_i(z)$
is unique because $s_i$ is injective.

\medskip\noindent
Here is a procedure for decomposing
any $x \in U_{n + 1}$.
First, write $x = x_0 + s_0(z_0)$ with $d_0(x_0) = 0$.
Next, write $x_0 = x_1 + s_1(z_1)$ with
$d_n(x_1) = 0$. Continue like this to get
\begin{eqnarray*}
x & = & x_0 + s_0(z_0), \\
x_0 & = & x_1 + s_1(z_1), \\
x_1 & = & x_2 + s_2(z_2), \\
\ldots & \ldots & \ldots \\
x_{n - 1} & = & x_n + s_n(z_n)
\end{eqnarray*}
where $d_i(x_i) = 0$ for all $i = n, \ldots, 0$.
By our general remark above all of the $x_i$
and $z_i$ are determined uniquely by $x$.
We claim that
$x_i \in
\text{Ker}(d_0) \cap
\text{Ker}(d_1) \cap
\ldots \cap
\text{Ker}(d_i)$
and
$z_i \in
\text{Ker}(d_0) \cap
\ldots \cap
\text{Ker}(d_{i - 1})$
for $i = n, \ldots, 0$.
Here and in the following
an empty intersection of kernels indicates
the whole space; i.e.,
the notation
$z_0 \in \text{Ker}(d_0) \cap
\ldots \cap
\text{Ker}(d_{i - 1})$
when $i = 0$ means $z_0 \in U_n$ with no restriction.

\medskip\noindent
We prove this by ascending induction on $i$.
It is clear for $i = 0$ by construction of $x_0$ and $z_0$.
Let us prove it for $0 < i \leq n$ assuming the result for $i - 1$.
First of all we have $d_i(x_i) = 0$ by construction.
So pick a $j$ with $0 \leq j < i$. We have
$d_j(x_{i - 1}) = 0$ by induction. Hence
$$
0 = d_j(x_{i - 1})
= d_j(x_i) + d_j(s_i(z_i))
= d_j(x_i) + s_{i - 1}(d_j(z_i)).
$$
The last equality by the relations of Remark \ref{remark-relations}.
These relations also imply that
$d_{i - 1}(d_j(x_i)) = d_j(d_i(x_i)) = 0$
because $d_i(x_i)= 0$ by construction.
Then the uniqueness in the general remark above shows the equality
$0 = x' + x'' = d_j(x_i) + s_{i - 1}(d_j(z_i))$
can only hold if both terms are zero. We conclude that
$d_j(x_i) = 0$ and by injectivity of $s_{i - 1}$ we also
conclude that $d_j(z_i) = 0$. This proves the claim.

\medskip\noindent
The claim implies we can uniquely write
$$
x = s_0(z_0) + s_1(z_1) + \ldots + s_n(z_n) + x_0
$$
with $x_0 \in N(U_{n + 1})$ and
$z_i \in \text{Ker}(d_0) \cap \ldots \cap \text{Ker}(d_{i - 1})$.
We can reformulate this as saying that we have found a direct
sum decomposition
$$
U_{n + 1}
=
N(U_{n + 1})
\oplus
\bigoplus\nolimits_{i = 0}^{i = n}
s_i\Big(\text{Ker}(d_0) \cap \ldots \cap \text{Ker}(d_{i - 1})\Big)
$$
with the property that
$$
\text{Ker}(d_0) \cap \ldots \cap \text{Ker}(d_j)
=
N(U_{n + 1}) \oplus
\bigoplus\nolimits_{i = j + 1}^{i = n}
s_i\Big(\text{Ker}(d_n) \cap \ldots \cap \text{Ker}(d_{i - 1})\Big)
$$
for $j = 0, \ldots, n$.
The result follows from this statement as follows.
Each of the $z_i$ in the expression for $x$
can be written uniquely as
$$
z_i = s_i(z'_{i, i}) + \ldots + s_{n - 1}(z'_{i, n - 1}) + z_{i, 0}
$$
with $z_{i, 0} \in N(U_n)$ and
$z'_{i, j} \in \text{Ker}(d_0) \cap \ldots \cap \text{Ker}(d_{j - 1})$.
The first few steps in the decomposition of $z_i$ are zero because
$z_i$ already is in the kernel of $d_0, \ldots, d_i$.
This in turn uniquely gives
$$
x = x_0 + s_0(z_{0, 0}) + s_1(z_{1, 0}) + \ldots + s_n(z_{n, 0}) +
\sum\nolimits_{0 \leq i \leq j \leq n - 1} s_i(s_j(z'_{i, j})).
$$
Continuing in this fashion we see that we in the end obtain
a decomposition of $x$ as a sum of terms
of the form
$$
s_{i_1} s_{i_2} \ldots s_{i_k} (z)
$$
with $0 \leq i_1 \leq i_2 \leq \ldots \leq i_k \leq n - k + 1$ and
$z \in N(U_{n + 1 - k})$. This is exactly the required
decomposition, because any surjective map $[n + 1] \to [n + 1 - k]$
can be uniquely expressed in the form
$$
\sigma^{n - k}_{i_k} \ldots \sigma^{n - 1}_{i_2} \sigma^n_{i_1}
$$
with $0 \leq i_1 \leq i_2 \leq \ldots \leq i_k \leq n - k + 1$.
\end{proof}

\begin{lemma}
\label{lemma-splitting-abelian-category}
Let $\mathcal{A}$ be an abelian category.
Let $U$ be a simplicial object in $\mathcal{A}$.
Then $U$ has a splitting obtained by taking $N(U_0) = U_0$ and
for $m \geq 1$ taking
$$
N(U_m) = \bigcap\nolimits_{i = 0}^{m - 1} \text{Ker}(d^m_i).
$$
Moreover, this splitting is functorial on the category of
simplicial objects of $\mathcal{A}$.
\end{lemma}

\begin{proof}
For any object $A$ of $\mathcal{A}$ we obtain
a simplicial abelian group $\text{Mor}_\mathcal{A}(A, U)$.
Each of these are canonically split by Lemma
\ref{lemma-splitting-simplicial-groups}. Moreover,
$$
N(\text{Mor}_\mathcal{A}(A, U_m)) =
\bigcap\nolimits_{i = 0}^{m - 1} \text{Ker}(d^m_i) =
\text{Mor}_\mathcal{A}(A, N(U_m)).
$$
Hence we see that the morphism (\ref{equation-splitting})
becomes an isomorphism after applying the functor
$\text{Mor}_\mathcal{A}(A, -)$ for any object of $\mathcal{A}$.
Hence it is an isomorphism by the Yoneda lemma.
\end{proof}

\begin{lemma}
\label{lemma-injective-map-simplicial-abelian}
Let $\mathcal{A}$ be an abelian category.
Let $f : U \to V$ be a morphism of
simplicial objects of $\mathcal{A}$.
If the induced morphisms $N(f)_i : N(U)_i \to N(V)_i$
are injective for all $i$, then $f_i$ is
injective for all $i$. Same holds with ``injective'' replaced
with ``surjective'', or ``isomorphism''.
\end{lemma}

\begin{proof}
This is clear from Lemma \ref{lemma-splitting-abelian-category}
and the definition of a splitting.
\end{proof}


\begin{lemma}
\label{lemma-N-d-in-N}
Let $\mathcal{A}$ be an abelian category.
Let $U$ be a simplicial object in $\mathcal{A}$.
Let $N(U_m)$ as in Lemma \ref{lemma-splitting-abelian-category} above.
Then $d^m_m(N(U_m)) \subset N(U_{m - 1})$.
\end{lemma}

\begin{proof}
For $j = 0, \ldots, m - 2$ we have
$d^{m - 1}_j d^m_m = d^{m - 1}_{m - 1} d^m_j$
by the relations in Remark \ref{remark-relations}.
The result follows.
\end{proof}

\begin{lemma}
\label{lemma-simplicial-abelian-n-skel-sub}
Let $\mathcal{A}$ be an abelian category.
Let $U$ be a simplicial object of $\mathcal{A}$.
Let $n \geq 0$ be an integer.
The rule
$$
U'_m = \sum\nolimits_{\varphi : [m] \to [i],\ i\leq n} \text{Im}(U(\varphi))
$$
defines a sub simplicial object $U' \subset U$ with $U'_i = U_i$
for $i \leq n$.
Moreover, $N(U'_m) = 0$ for all $m > n$.
\end{lemma}

\begin{proof}
Pick $m$, $i \leq n$ and some $\varphi : [m] \to [i]$.
The image under $U(\psi)$ of $\text{Im}(U(\varphi))$
for any $\psi : [m'] \to [m]$ is
equal to the image of $U(\varphi \circ \psi)$ and
$\varphi \circ \psi : [m'] \to [i]$.
Hence $U'$ is a simplicial object.
Pick $m > n$. We have to show $N(U'_m) = 0$.
By definition of $N(U_m)$ and $N(U'_m)$ we have
$N(U'_m) = U'_m \cap N(U_m)$ (intersection of subobjects).
Since $U$ is split by Lemma \ref{lemma-splitting-abelian-category},
it suffices to show that $U'_m$ is contained in the sum
$$
\sum\nolimits_{\varphi : [m] \to [m']\text{ surjective},\ m' < m}
\text{Im}(U(\varphi)|_{N(U_{m'})}).
$$
By the splitting each $U_{m'}$ is the sum of images of
$N(U_{m''})$ via $U(\psi)$ for surjective maps
$\psi : [m'] \to [m'']$. Hence the displayed sum above
is the same as
$$
\sum\nolimits_{\varphi : [m] \to [m']\text{ surjective},\ m' < m}
\text{Im}(U(\varphi)).
$$
Clearly $U'_m$ is contained in this by the simple fact that
any $\varphi : [m] \to [i]$, $i \leq n$ occuring in the definition
of $U'_m$ may be factored as
$[m] \to [m'] \to [i]$ with $[m] \to [m']$ surjective
and $m' < m$ as in the last displayed sum above.
\end{proof}


\section{Skelet and coskelet functors}
\label{section-skelet}

\noindent
Let $\Delta_{\leq n}$ denote the full subcategory of
$\Delta$ with objects $[0], [1], [2], \ldots, [n]$.
Let $\mathcal{C}$ be a category.

\begin{definition}
\label{definition-truncated-simplicial-object}
An {\it $n$-truncated simplicial object of $\mathcal{C}$} 
is a contravariant functor from $\Delta_{\leq n}$ to
$\mathcal{C}$. A {\it morphism of $n$-truncated
simplicial objects} is a transformation of functors.
We denote the category of $n$-truncated
simplicial objects of $\mathcal{C}$ by
the symbol $\text{Simp}_n(\mathcal{C})$.
\end{definition}

\noindent
Given a simplicial object $U$ of $\mathcal{C}$
the truncation $\text{sk}_n U$ is the restriction
of $U$ to the subcategory $\Delta_{\leq n}$.
This defines a {\it skelet functor}
$$
\text{sk}_n :
\text{Simp}(\mathcal{C}) \longrightarrow \text{Simp}_n(\mathcal{C})
$$
from the category of simplicial objects of $\mathcal{C}$
to the category of $n$-truncated simplicial objects of $\mathcal{C}$.
See Remark \ref{remark-sk-literature} to avoid possible confusion
with other functors in the literature.

\medskip\noindent
The {\it coskelet functor} (if it exists) is a functor
$$
\text{cosk}_n :
\text{Simp}(\mathcal{C}) \longrightarrow \text{Simp}_n(\mathcal{C})
$$
which is right adjoint to the skelet functor. In a formula
\begin{equation}
\label{equation-cosk}
\text{Mor}_{\text{Simp}(\mathcal{C})}(U, \text{cosk}_n V)
=
\text{Mor}_{\text{Simp}_n(\mathcal{C})}(\text{sk}_n U, V)
\end{equation}
Given a $n$-truncated simplicial object $V$ we 
say that {\it $\text{cosk}_nV$ exists} if there
exists a $\text{cosk}_nV \in \text{Ob}(\text{Simp}(\mathcal{C}))$
and a morphism $\text{sk}_n \text{cosk}_n V \to V$
such that the displayed formula holds, in other words
if the functor
$U \mapsto \text{Mor}_{\text{Simp}_n(\mathcal{C})}(\text{sk}_n U, V)$
is representable. If it exists it
is unique up to unique isomorphism by the Yoneda lemma.
See Categories, Section \ref{categories-section-opposite}.

\begin{example}
\label{example-cosk0}
Suppose the category $\mathcal{C}$ has finite nonempty self products.
A $0$-truncated simplicial object of $\mathcal{C}$ is the same
as an object $X$ of $\mathcal{C}$. In this case
we claim that $\text{cosk}_0(X)$ is the simplicial
object $U$ with $U_n = X^{n + 1}$ the $(n + 1)$-fold self
product of $X$, and structure of simplicial object
as in Example \ref{example-fibre-products-simplicial-object}.
Namely, a morphism $V \to U$ where $V$ is a simplicial
object is given by morphisms $V_n \to X^{n + 1}$, such
that all the diagrams
$$
\xymatrix{
V_n \ar[r] \ar[d]_{V([0] \to [n], 0 \mapsto i)} &
X^{n + 1} \ar[d]^{\text{pr}_i} \\
V_0 \ar[r] &
X
}
$$
commute. Clearly this means that the map determines and is determined
by a unique morphism $V_0 \to X$. This proves that formula
(\ref{equation-cosk}) holds.
\end{example}

\noindent
Recall the category $\Delta/[n]$, see Example \ref{example-simplex-category}.
We let $(\Delta/[n])_{\leq m}$ denote the full subcategory
of $\Delta/[n]$ consisting of objects $[k] \to [n]$
of $\Delta/[n]$ with $k \leq m$. In other words we have
the following commutative diagram of categories and functors
$$
\xymatrix{
(\Delta/[n])_{\leq m} \ar[r] \ar[d] &
\Delta/[n] \ar[d] \\
\Delta_{\leq m} \ar[r] &
\Delta
}
$$
Given a $m$-truncated
simplicial object $U$ of $\mathcal{C}$
we define a functor
$$
U(n) : (\Delta/[n])_{\leq m}^{opp} \longrightarrow \mathcal{C}
$$
by the rules
\begin{eqnarray*}
([k] \to [n]) & \longmapsto & U_k \\
(\psi : ([k'] \to [n]) \to ([k] \to [n])) &
\longmapsto &
U(\psi) : U_k \to U_{k'}
\end{eqnarray*}
For a given morphism $\varphi : [n] \to [n']$ of $\Delta$
we have an associated functor
$$
"\varphi" : (\Delta/[n])_{\leq m} \longrightarrow (\Delta/[n'])_{\leq m}
$$
which maps $\alpha : [k] \to [n]$ to
$\varphi \circ \alpha : [k] \to [n']$.
The composition $U(n') \circ "\varphi"$ is
equal to the functor $U(n)$.

\begin{lemma}
\label{lemma-existence-cosk}
If the category $\mathcal{C}$ has finite limits, then
$\text{cosk}_m$ functors exist for all $m$. Moreover,
for any $m$-truncated simplicial object $U$ the
simplicial object $\text{cosk}_mU$ is described
by the formula
$$
(\text{cosk}_mU)_n = \text{lim}_{(\Delta/[n])_{\leq m}^{opp}}\ U(n)
$$
and for $\varphi : [n] \to [n']$ the map
$\text{cosk}_mU(\varphi)$ comes from the
identification $U(n') \circ "\varphi" = U(n)$ above 
via Categories, Lemma \ref{categories-lemma-functorial-limit}.
\end{lemma}

\begin{proof}
During the proof of this lemma we denote $\text{cosk}_mU$ the
simplicial object with $(\text{cosk}_mU)_n$ equal to
$\text{lim}_{(\Delta/[n])_{\leq m}^{opp}}\ U(n)$.
We will conclude at the end of the proof that it does
satsify the required mapping property.

\medskip\noindent
Suppose that $V$ is a simplicial object.
A morphism $\gamma : V \to \text{cosk}_mU$ is given by a sequence
of morphisms $\gamma_n : V_n \to (\text{cosk}_mU)_n$.
By definition of a limit, this is given by a
collection of morphisms $\gamma(\alpha) : V_n \to U_k$
where $\alpha$ ranges over all $\alpha : [k] \to [n]$
with $k \leq m$. These morphisms then also satisfy
the rules that
$$
\xymatrix{
V_n \ar[r]_{\gamma(\alpha)} &  U_k \\
V_{n'} \ar[r]^{\gamma(\alpha')} \ar[u]^{V(\varphi)} & U_{k'} \ar[u]_{U(\psi)}
}
$$
are commutative, given any $0 \leq k, k' \leq m$, $0 \leq n, n'$
and any $\psi : [k] \to [k']$, $\varphi : [n] \to [n']$,
$\alpha : [k] \to [n]$ and $\alpha' : [k'] \to [n']$ in $\Delta$
such that $\varphi \circ \alpha = \alpha' \circ \psi$.
Taking $n = k$, $\varphi = \alpha'$, and $\alpha = \psi = \text{id}_{[k]}$
we deduce that $\gamma(\alpha') = \gamma(\text{id}_{[k]}) \circ V(\alpha')$.
In other words, the morphisms $\gamma(\text{id}_{[k]})$, $k \leq m$
determine the morphism $\gamma$. And it is easy to see that these
morphisms form a morphism $\text{sk}_m V \to U$.

\medskip\noindent
Conversely, given a morphism $\gamma : \text{sk}_m V \to U$, 
we obtain a family of morphsms $\gamma(\alpha)$
where $\alpha$ ranges over all $\alpha : [k] \to [n]$
with $k \leq m$ by setting $\gamma(\alpha) = 
\gamma(\text{id}_{[k]}) \circ V(\alpha)$. These morphisms
satisfy all the displayed commutativity restraints pictured
above, and hence give rise to a morphism $V \to \text{cosk}_m U$.
\end{proof}

\begin{lemma}
\label{lemma-trivial-cosk}
Let $\mathcal{C}$ be a category.
Let $U$ be an $m$-truncated simplicial object of $\mathcal{C}$.
For $n \leq m$ the limit $\text{lim}_{(\Delta/[n])_{\leq m}^{opp}}\ U(n)$
exists and is canonically isomorphic to $U_n$.
\end{lemma}

\begin{proof}
This is true because the category $(\Delta/[n])_{\leq m}$
has an final object in this case, namely the identity
map $[n] \to [n]$.
\end{proof}

\begin{lemma}
\label{lemma-recover-cosk}
Let $\mathcal{C}$ be a category with finite limits.
Let $U$ be an $n$-truncated simplicial object of $\mathcal{C}$.
The morphism $\text{sk}_n \text{cosk}_n U \to U$
is an isomorphism.
\end{lemma}

\begin{proof}
Combine Lemmas \ref{lemma-existence-cosk} and \ref{lemma-trivial-cosk}.
\end{proof}

\noindent
Let us describe a particular instance of the coskelet functor in more detail.
By abuse of notation we will denote $\text{sk}_n$
also the restriction functor
$\text{Simp}_{n'}(\mathcal{C}) \to \text{Simp}_n(\mathcal{C})$
for any $n' \geq n$. We are going to describe a right adjoint
of the functor
$\text{sk}_n : \text{Simp}_{n + 1}(\mathcal{C})
\to \text{Simp}_n(\mathcal{C})$.
For $n \geq 1$, $0 \leq i < j \leq n + 1$
define $\delta^{n + 1}_{i,j} : [n - 1] \to [n + 1]$
to be the increasing map omitting $i$ and $j$.
Note that
$\delta^{n + 1}_{i,j} =
\delta^{n + 1}_j \circ \delta^n_i =
\delta^{n + 1}_i \circ \delta^n_{j - 1}$, see
Lemma \ref{lemma-relations-face-degeneracy}. This motivates
the following lemma.

\begin{lemma}
\label{lemma-formula-limit}
Let $n$ be an integer $\geq 1$.
Let $U$ be a $n$-truncated simplicial object of $\mathcal{C}$.
Consider the contravariant functor from $\mathcal{C}$ to
$\textit{Sets}$ which associates to an object $T$ the set
$$
\{ (f_0,\ldots,f_{n + 1}) \in \text{Mor}_{\mathcal{C}}(T, U_n)
\mid
d^n_{j - 1} \circ f_i = d^n_i \circ f_j\ 
\forall\ 0\leq i < j\leq n + 1\}
$$
If this functor is representable by some object $U_{n + 1}$
of $\mathcal{C}$, then
$$
U_{n + 1} = \text{lim}_{(\Delta/[n + 1])_{\leq n}^{opp}}\ U(n)
$$
\end{lemma}

\begin{proof}
The limit, if it exists, represents the functor
that associates to an object $T$ the set
$$
\{
(f_\alpha)_{\alpha : [k] \to [n + 1], k \leq n}
\mid
f_{\alpha \circ \psi} = U(\psi) \circ f_\alpha\ \forall\ 
\psi : [k'] \to [k], \alpha : [k] \to [n + 1]
\}.
$$
In fact we will show this functor is isomorphic to the
one displayed in the lemma. The map in one direction
is given by the rule
$$
(f_\alpha)_{\alpha}
\longmapsto
(f_{\delta^{n + 1}_0}, \ldots, f_{\delta^{n + 1}_{n + 1}}).
$$
This satisfies the conditions of the lemma because
$$
d^n_{j - 1} \circ f_{\delta^{n + 1}_i} =
f_{\delta^{n + 1}_i \circ \delta^n_{j - 1}} =
f_{\delta^{n + 1}_j \circ \delta^n_i} =
d^n_i \circ f_{\delta^{n + 1}_j}
$$
by the relations we recalled above the lemma. To construct a map
in the other direction we have to associate to a system
$(f_0, \ldots, f_{n + 1})$ as in the displayed formula
of the lemma a system of maps $f_\alpha$. Let $\alpha : [k] \to [n + 1]$
be given. Since $k \leq n$ the map $\alpha$ is not surjective.
Hence we can write $\alpha = \delta^{n + 1}_i \circ \psi$
for some $0 \leq i \leq n + 1$ and some
$\psi : [k] \to [n]$. We have no choice but to define
$$
f_\alpha = U(\psi) \circ f_i.
$$
Of course we have to check that this is independent of the
choice of the pair $(i, \psi)$. First, observe that given $i$
there is a unique $\psi$ which works. Second, suppose that $(j, \phi)$ is
another pair. Then $i \not = j$ and we may assume $i < j$. Since
both $i, j$ are not in the image of $\alpha$ we may actually
write $\alpha = \delta^{n + 1}_{i, j} \circ \xi$ and then
we see that $\psi = \delta^n_{j - 1} \circ \xi$ and
$\phi = \delta^n_i \circ \xi$. Thus
\begin{eqnarray*}
U(\psi) \circ f_i & = & U(\delta^n_{j - 1} \circ \xi) \circ f_i \\
& = & U(\xi) \circ d^n_{j - 1} \circ f_i \\
& = & U(\xi) \circ d^n_i \circ f_j \\
& = & U(\delta^n_i \circ \xi) \circ f_j \\
& = & U(\phi) \circ f_j
\end{eqnarray*}
as desired. We still have to verify that the maps
$f_\alpha$ so defined satisfy the rules of a system
of maps $(f_\alpha)_\alpha$. To see this suppose that
$\psi : [k'] \to [k]$, $\alpha : [k] \to [n + 1]$ with
$k, k' \leq n$. Set $\alpha' = \alpha \circ \psi$.
Choose $i$ not in the image of $\alpha$. Then clearly
$i$ is not in the image of $\alpha'$ also. Write
$\alpha = \delta^n_i \circ \phi$ (we cannot use the letter $\psi$ here
because we've already used it). Then obviously
$\alpha' = \delta^n_i \circ \phi \circ \psi$. By construction above
we then have
$$
U(\psi) \circ f_\alpha = U(\psi) \circ U(\phi) \circ f_i
= U(\phi \circ \psi) \circ f_i = f_{\alpha \circ \psi} = f_{\alpha'}
$$
as desired. We leave to the reader the pleasant task of verifying
that our constructions are mutually inverse bijections, and are
functorial in $T$.
\end{proof}

\begin{lemma}
\label{lemma-work-out}
Let $n$ be an integer $\geq 1$. Let $U$ be a $n$-truncated
simplicial object of $\mathcal{C}$. Consider the
contravariant functor from $\mathcal{C}$ to $\textit{Sets}$
which associates to an object $T$ the set
$$
\{ (f_0,\ldots,f_{n + 1}) \in \text{Mor}_{\mathcal{C}}(T, U_n)
\mid
d^n_{j - 1} \circ f_i = d^n_i \circ f_j\ 
\forall\ 0\leq i < j\leq n + 1\}
$$
If this functor is representable by some object $U_{n + 1}$
of $\mathcal{C}$, then there exists an $(n + 1)$-truncated
simplicial object $\tilde U$, with $\text{sk}_n \tilde U = U$
and $\tilde U_{n + 1} = U_{n + 1}$ such that the following
adjointness holds
$$
\text{Mor}_{\text{Simp}_{n + 1}(\mathcal{C})}(V, \tilde U)
=
\text{Mor}_{\text{Simp}_n(\mathcal{C})}(\text{sk}_nV, U)
$$
\end{lemma}

\begin{proof}
By Lemma \ref{lemma-trivial-cosk} there are identifications
$$
U_i = \text{lim}_{(\Delta/[i])_{\leq n}^{opp}}\ U(i)
$$
for $0 \leq i \leq n$. By Lemma \ref{lemma-formula-limit}
we have
$$
U_{n + 1} = \text{lim}_{(\Delta/[n + 1])_{\leq n}^{opp}}\ U(n).
$$
Thus we may define for any $\varphi : [i] \to [j]$
with $i, j \leq n + 1$ the corresponding map
$\tilde U(\varphi) : \tilde U_j \to \tilde U_i$ exactly as
in Lemma \ref{lemma-existence-cosk}. This defines
an $(n + 1)$-truncated simplicial object $\tilde U$
with $\text{sk}_n \tilde U = U$.

\medskip\noindent
To see the adjointness we argue as follows. Given any element
$\gamma : \text{sk}_n V \to U$ of the right hand side of the formula
consider the morphisms
$f_i = \gamma_n \circ d^{n+1}_i : V_{n+1} \to V_n \to U_n$.
These clearly satisfy the relations $d^n_{j - 1} \circ f_i = d^n_i \circ f_j$
and hence define a unique morphism $V_{n + 1} \to U_{n + 1}$
by our choice of $U_{n + 1}$.
Conversely, given a morphsm $\gamma' : V \to \tilde U$
of the left hand side we can simply restrict to
$\Delta_{\leq n}$ to get an element of the right hand side.
We leave it to the reader to show these are mutually inverse
constructions. See also
Remark \ref{remark-cosk-simplicial-sets}.
\end{proof}

\begin{remark}
\label{remark-explicit-face-degeneracy}
Let $U$, and $U_{n + 1}$ be as in Lemma \ref{lemma-work-out}.
On $T$-valued points we can easily describe the face
and degeneracy maps of $\tilde U$.
Explicitly, the maps $d^{n + 1}_i : U_{n + 1} \to U_n$
are given by
$$
(f_0, \ldots, f_{n + 1}) \longmapsto f_i.
$$
And the maps $s^n_j : U_n \to U_{n + 1}$ are given by
\begin{eqnarray*}
f & \longmapsto & (
s^{n - 1}_{j - 1} \circ d^{n - 1}_0 \circ f,\\ 
& &
s^{n - 1}_{j - 1} \circ d^{n - 1}_1 \circ f,\\
& &
\ldots\\
& &
s^{n - 1}_{j - 1} \circ d^{n - 1}_{j - 1} \circ f, \\
& &
f,\\
& &
f,\\
& &
s^{n - 1}_j \circ d^{n - 1}_{j + 1} \circ f,\\
& &
s^{n - 1}_j \circ d^{n - 1}_{j + 2} \circ f,\\
& &
\ldots\\
& &
s^{n - 1}_j \circ d^{n - 1}_n \circ f
)
\end{eqnarray*}
where we leave it to the reader to verify that the RHS
is an element of the displayed set of Lemma \ref{lemma-work-out}.
For $n = 0$ there is one map, namely $f \mapsto (f, f)$.
For $n = 1$ there are two maps, namely
$f \mapsto (f, f, s_0d_1f)$ and
$f \mapsto (s_0d_0f, f, f)$.
For $n = 2$ there are three maps, namely
$f \mapsto (f, f, s_0d_1f, s_0d_2f)$,
$f \mapsto (s_0d_0f, f, f, s_1d_2f)$, and
$f \mapsto (s_1d_0f, s_1d_1f, f, f)$.
And so on and so forth.
\end{remark}

\begin{remark}
\label{remark-cosk-simplicial-sets}
The construction of Lemma \ref{lemma-work-out}
above in the case of simplicial
sets is the following. Given an $n$-truncated simplicial
set $U$, we make a canonical $(n + 1)$-truncated simplicial
set $\tilde U$ as follows. We add a set of $(n + 1)$-simplices
$U_{n + 1}$ by the formula of the lemma. Namely,
an element of $U_{n + 1}$ is a numbered collection of
$(f_0,\ldots,f_{n + 1})$ of $n$-simplices,
with the property that they glue
as they would in a $(n + 1)$-simplex. In other words,
the $i$th face of $f_j$ is the $(j-1)$st face of $f_i$
for $i < j$. Geometrically it is obvious how to define the
face and degeneracy maps for $\tilde U$.
If $V$ is an $(n + 1)$-truncated simplicial set,
then its $(n + 1)$-simplices give rise to compatible collections
of $n$-simplices $(f_0, \ldots, f_{n + 1})$ with $f_i \in V_n$.
Hence there is a natural map
$\text{Mor}(\text{sk}_nV, U) \to \text{Mor}(V, \tilde U)$
which is inverse to the canonical restriction mapping
the other way.

\medskip\noindent
Also, it is enough to do the combinatorics of the
construction in the case of truncated simplicial sets.
Namely, for any object $T$ of the category $\mathcal{C}$,
and any $n$-truncated simplicial object $U$ of $\mathcal{C}$
we can consider the $n$-truncated simplicial set
$\text{Mor}(T, U)$. We may apply the construction to this,
and take its set of $(n + 1)$-simplices, and require this to be
representable. This is a good way to think about
the result of Lemma \ref{lemma-work-out}.
\end{remark}

\begin{remark}
\label{remark-inductive-coskelet}
{\it Inductive construction of coskelets.}
Suppose that $\mathcal{C}$ is a category with
finite limits. Suppose that $U$ is an $m$-truncated
simplicial object in $\mathcal{C}$. Then we can
inductively construct $n$-truncated objects $U^n$ as
follows:
\begin{enumerate}
\item To start, set $U^m = U$.
\item Given $U^n$ for $n \geq m$ set $U^{n + 1} = \tilde U^n$,
where $\tilde U^n$ is constructed from $U^n$ as in Lemma
\ref{lemma-work-out}.
\end{enumerate}
Since the construction of Lemma \ref{lemma-work-out} has
the property that it leaves the $n$-skeleton of $U^n$
unchanged, we can then define $\text{cosk}_m U$ to be
the simplicial object with
$(\text{cosk}_m U)_n = U^n_n = U^{n + 1}_n = \ldots$.
And it follows formally from Lemma \ref{lemma-work-out}
that $U^n$ satisfies the formula
$$
\text{Mor}_{\text{Simp}_n(\mathcal{C})}(V, U^n) 
=
\text{Mor}_{\text{Simp}_m(\mathcal{C})}(\text{sk}_mV, U)
$$
for all $n \geq m$. It also then follows formally 
from this that
$$
\text{Mor}_{\text{Simp}(\mathcal{C})}(V, \text{cosk}_mU) 
=
\text{Mor}_{\text{Simp}_m(\mathcal{C})}(\text{sk}_mV, U)
$$
with $\text{cosk}_mU$ chosen as above.
\end{remark}

\begin{lemma}
\label{lemma-cosk-up}
Let $\mathcal{C}$ be a category which has finite limits.
\begin{enumerate}
\item For every $n$ the functor $\text{sk}_n : \text{Simp}(\mathcal{C})
\to \text{Simp}_n(\mathcal{C})$ has a right adjoint $\text{cosk}_n$.
\item For every $n' \geq n$ the functor
$\text{sk}_n : \text{Simp}_{n'}(\mathcal{C}) \to \text{Simp}_n(\mathcal{C})$
has a right adjoint, namely $\text{sk}_{n'}\text{cosk}_n$.
\item For every $m \geq n \geq 0$ and every $n$-truncated simplicial
object $U$ of $\mathcal{C}$ we have
$\text{cosk}_m \text{sk}_m \text{cosk}_n U = \text{cosk}_n U$.
\item If $U$ is a simplicial object of $\mathcal{C}$ such that
the canonical map
$U \to \text{cosk}_n \text{sk}_nU$
is an isomorphism for some $n \geq 0$, then the canonical map
$U \to \text{cosk}_m \text{sk}_mU$
is an isomorphism for all $m \geq n$.
\end{enumerate}
\end{lemma}

\begin{proof}
The existence in (1) follows from Lemma \ref{lemma-existence-cosk} above
and the equality in (2), and (3) follows from the discussion
in Remark \ref{remark-inductive-coskelet}. After this (4) is obvious.
\end{proof}

\begin{lemma}
\label{lemma-cosk-product}
Let $U$, $V$ be $n$-truncated simplicial objects of a
category $\mathcal{C}$. Then
$$
\text{cosk}_n (U \times V) = \text{cosk}_nU \times \text{cosk}_nV
$$
whenever the left and right hand sides exist.
\end{lemma}

\begin{proof}
Let $W$ be a simplicial object. We have
\begin{eqnarray*}
\text{Mor}(W, \text{cosk}_n (U \times V))
& = &
\text{Mor}(\text{sk}_n W, U \times V) \\
& = &
\text{Mor}(\text{sk}_n W, U)
\times
\text{Mor}(\text{sk}_nW, V) \\
& = &
\text{Mor}(W, \text{cosk}_n U)
\times
\text{Mor}(W, \text{cosk}_n V) \\
& = &
\text{Mor}(W, \text{cosk}_n U \times \text{cosk}_n V)
\end{eqnarray*}
The lemma follows.
\end{proof}

\begin{lemma}
\label{lemma-simplex-cosk}
The canonical map
$\Delta[n] \to \text{cosk}_1 \text{sk}_1 \Delta[n]$
is an isomorphism.
\end{lemma}

\begin{proof}
Consider a simplicial set $U$ and a morphism
$f : U \to \Delta[n]$. This is a rule that associates
to each $u \in U_i$ a map $f_u : [i] \to [n]$ in $\Delta$.
Furthermore, these maps should have the property that
$f_u \circ \varphi = f_{U(\varphi)(u)}$ for any 
$\varphi : [j] \to [i]$. Denote $\epsilon^i_j : [0] \to [i]$
the map which maps $0$ to $j$. Denote $F : U_0 \to [n]$
the map $u \mapsto f_u(0)$. Then we see that
$$
f_u(j) = F(\epsilon^i_j(u))
$$
for all $0 \leq j \leq i$ and $u \in U_i$.
In particular, if we know the function $F$
then we know the maps $f_u$ for all $u\in U_i$ all $i$.
Conversely, given a map $F : U_0 \to [n]$,
we can set for any $i$, and any $u \in U_i$ 
and any $0 \leq j \leq i$
$$
f_u(j) = F(\epsilon^i_j(u))
$$
This does not in general define a morphism $f$ of simplicial sets
as above. Namely, the condition is that all the maps $f_u$ are
nondecreasing. This clearly is equivalent to the condition
that $F(\epsilon^i_j(u)) \leq F(\epsilon^i_{j'}(u))$
whenever $0 \leq j \leq j' \leq i$ and $u \in U_i$. But in this
case the morphisms
$$
\epsilon^i_j, \epsilon^i_{j'} : [0] \to [i]
$$
both factor through the map
$\epsilon^i_{j, j'} : [1] \to [i]$ defined by the rules
$0 \mapsto j$, $1 \mapsto j'$.
In other words, it is enough to check the inequalities for
$i = 1$ and $u \in X_1$. In other words, we have
$$
\text{Mor}(U, \Delta[n])
=
\text{Mor}(\text{sk}_1 U, \text{sk}_1 \Delta[n])
$$
as desired.
\end{proof}








\section{Left adjoints to the skeleton functors}
\label{section-adjoint-left}

\noindent
In this section we construct a left adjoint $i_{m!}$
of the skeleton functor $\text{sk}_m$ in certain cases.
The adjointness formula is
$$
\text{Mor}_{\text{Simp}_m(\mathcal{C})}(U,\text{sk}_mV)
=
\text{Mor}_{\text{Simp}(\mathcal{C})}(i_{m!}U,V).
$$
It turns out that this left adjoint exists when
the category $\mathcal{C}$ has finite colimits.

\medskip\noindent
We use a similar construction as in Section \ref{section-skelet}.
Recall the category $[n]/\Delta$ of objects
under $[n]$, see
Categories, Example \ref{categories-example-category-under-X}.
Its objects are morphisms $\alpha : [n] \to [k]$
and its morphisms are commutative triangles.
We let $([n]/\Delta)_{\leq m}$ denote the full subcategory
of $[n]/\Delta$ consisting of objects $[n] \to [k]$
with $k \leq m$. Given a $m$-truncated
simplicial object $U$ of $\mathcal{C}$
we define a functor
$$
U(n) : ([n]/\Delta)_{\leq m}^{opp} \longrightarrow \mathcal{C}
$$
by the rules
\begin{eqnarray*}
([n] \to [k]) & \longmapsto & U_k \\
(\psi : ([n] \to [k']) \to ([n] \to [k]))
& \longmapsto &
U(\psi) : U_k \to U_{k'}
\end{eqnarray*}
For a given morphism $\varphi : [n] \to [n']$ of $\Delta$
we have an associated functor
$$
"\varphi" : ([n']/\Delta)_{\leq m} \longrightarrow ([n]/\Delta)_{\leq m}
$$
which maps $\alpha : [n'] \to [k]$ to
$\varphi \circ \alpha : [n] \to [k]$.
The composition $U(n) \circ "\varphi"$ is
equal to the functor $U(n')$.

\begin{lemma}
\label{lemma-left-adjoint-exists}
Let $\mathcal{C}$ be a category which has finite colimits.
The functors $i_{m!}$ exist for all $m$.
Let $U$ be an $m$-truncated simplicial object of $\mathcal{C}$.
The simplicial object $i_{m!}U$
is described by the formula
$$
(i_{m!}U)_n = \text{colim}_{([n]/\Delta)_{\leq m}^{opp}} U(n)
$$
and for $\varphi : [n] \to [n']$ the map
$i_{m!}U(\varphi)$ comes from the
identification $U(n) \circ "\varphi" = U(n')$ above 
via Categories, Lemma \ref{categories-lemma-functorial-colimit}.
\end{lemma}

\begin{proof}
In this proof we denote $i_{m!}U$ the simplicial object
whose $n$th term is given by the displayed formula of the 
lemma. We will show it satisfies the adjointness property.

\medskip\noindent
Let $V$ be a simplicial object of $\mathcal{C}$.
Let $\gamma : U \to \text{sk}_mV$ be given.
A morphism
$$
\text{colim}_{([n]/\Delta)_{\leq m}^{opp}}\ U(n) \to T
$$
is given by a compatible system of morphisms
$f_\alpha : U_k \to T$ where $\alpha : [n] \to [k]$
with $k \leq m$. Certainly, we have such a system of
morphisms by taking the compositions 
$$
U_k \xrightarrow{\gamma_k} V_k \xrightarrow{V(\alpha)} V_n.
$$
Hence we get an induced morphism $(i_{m!}U)_n \to V_n$.
We leave it to the reader to see that these form a 
morphism of simplicial objects $\gamma' : i_{m!}U \to V$.

\medskip\noindent
Coversely, given a morphism $\gamma' : i_{m!}U \to V$ we obtain
a morphism $\gamma : U \to \text{sk}_m V$ by setting
$\gamma_i : U_i \to V_i$ equal to the composition
$$
U_i
\xrightarrow{\text{id}_{[i]}}
\text{colim}_{([i]/\Delta)_{\leq m}^{opp}}\ U(i)
\xrightarrow{\gamma'_i}
V_i
$$
for $0 \leq i \leq n$. We leave it to the reader to see that
this is the inverse of the construction above.
\end{proof}

\begin{lemma}
\label{lemma-recovering-U}
Let $\mathcal{C}$ be a category.
Let $U$ be an $m$-truncated simplicial object of $\mathcal{C}$.
For any $n \leq m$ the colimit
$$
\text{colim}_{([n]/\Delta)_{\leq m}^{opp}} U(n)
$$
exists and is equal to $U_n$.
\end{lemma}

\begin{proof}
This is so because the category $([n]/\Delta)_{\leq m}$
has an initial object, namely $\text{id} : [n] \to [n]$.
\end{proof}

\begin{lemma}
\label{lemma-recovering-U-for-real}
Let $\mathcal{C}$ be a category which has finite colimits.
Let $U$ be an $m$-truncated simplicial object of $\mathcal{C}$.
The map $U \to \text{sk}_m i_{m!}U$ is an isomorphism.
\end{lemma}

\begin{proof}
Combine Lemmas \ref{lemma-left-adjoint-exists} and \ref{lemma-recovering-U}.
\end{proof}

\begin{lemma}
\label{lemma-imshriek-sets}
If $U$ is an $m$-truncated simplicial set and $n > m$
then all $n$-simplices of $i_{m!}U$ are degenerate.
\end{lemma}

\begin{proof}
This can be seen from the construction of
$i_{m!}U$ in Lemma \ref{lemma-left-adjoint-exists},
but we can also argue directly as follows.
Write $V = i_{m!}U$. Let $V' \subset V$ be the
simplicial subset with $V'_i = V_i$ for $i \leq m$
and all $i$ simplices degenerate for $i > m$,
see Lemma \ref{lemma-simplicial-set-n-skel-sub}.
By the adjunction formula,
since $\text{sk}_m V' = U$, there is an inverse to the
injection $V' \to V$. Hence $V' = V$.
\end{proof}

\begin{lemma}
\label{lemma-n-skeletion-sets}
Let $U$ be a simplicial set.
Let $n \geq 0$ be an integer.
The morphism $i_{n!} \text{sk}_n U \to U$ identifies
$i_{n!} \text{sk}_n U$ with the simplicial set
$U' \subset U$ defined in Lemma \ref{lemma-simplicial-set-n-skel-sub}.
\end{lemma}

\begin{proof}
By Lemma \ref{lemma-imshriek-sets} the only
nondegenerate simplices of $i_{n!} \text{sk}_n U$
are in degrees $\leq n$. The map
$i_{n!} \text{sk}_n U \to U$ is an isomorphism
in degrees $\leq n$. Combined we conclude
that the map $i_{n!} \text{sk}_n U \to U$ maps
nondegenerate simplices to nondegenerate simplices
and no two nondegenerate simplices have the same image.
Hence Lemma \ref{lemma-injective-map-simplicial-sets} applies.
Thus $i_{n!} \text{sk}_n U \to U$
is injective. The result follows easily from this.
\end{proof}

\begin{remark}
\label{remark-sk-literature}
In some texts the composite functor
$$
\text{Simp}(\mathcal{C})
\xrightarrow{\text{sk}_m}
\text{Simp}_m(\mathcal{C})
\xrightarrow{i_{m!}}
\text{Simp}(\mathcal{C})
$$
is denoted $\text{sk}_m$. This makes sense because
Lemma \ref{lemma-n-skeletion-sets} says
that $i_{m!} \text{sk}_m V$ is just the sub simplicial set
of $V$ consisting of all $i$-simplices of $V$, $i \leq m$
and their degeneracies. In those texts it is also customary
to denote the composition
$$
\text{Simp}(\mathcal{C})
\xrightarrow{\text{sk}_m}
\text{Simp}_m(\mathcal{C})
\xrightarrow{\text{cosk}_m}
\text{Simp}(\mathcal{C})
$$
by $\text{cosk}_m$.
\end{remark}

\begin{lemma}
\label{lemma-imshriek-abelian}
Let $\mathcal{A}$ be an abelian category
Let $U$ be an $m$-truncated simplicial object of
$\mathcal{A}$. For $n > m$ we have $N(i_{m!}U)_n = 0$.
\end{lemma}

\begin{proof}
Write $V = i_{m!}U$. Let $V' \subset V$ be the
simplicial subobject of $V$ with $V'_i = V_i$ for $i \leq m$
and $N(V'_i) = 0$ for $i > m$,
see Lemma \ref{lemma-simplicial-abelian-n-skel-sub}.
By the adjunction formula,
since $\text{sk}_m V' = U$, there is an inverse to the
injection $V' \to V$. Hence $V' = V$.
\end{proof}

\begin{lemma}
\label{lemma-n-skeletion-abelian}
Let $\mathcal{A}$ be an abelian category.
Let $U$ be a simplicial object of $\mathcal{A}$.
Let $n \geq 0$ be an integer.
The morphism $i_{n!} \text{sk}_n U \to U$ identifies
$i_{n!} \text{sk}_n U$ with the simplicial subobject
$U' \subset U$ defined in Lemma \ref{lemma-simplicial-abelian-n-skel-sub}.
\end{lemma}

\begin{proof}
By Lemma \ref{lemma-imshriek-abelian} 
we have $N(i_{n!} \text{sk}_n U)_i = 0$
for $i > n$. The map
$i_{n!} \text{sk}_n U \to U$ is an isomorphism
in degrees $\leq n$, see Lemma \ref{lemma-recovering-U-for-real}.
Combined we conclude that the map $i_{n!} \text{sk}_n U \to U$
induces injective maps $N(i_{n!} \text{sk}_n U)_i \to N(U)_i$
for all $i$. Hence Lemma \ref{lemma-injective-map-simplicial-abelian}
applies. Thus $i_{n!} \text{sk}_n U \to U$
is injective. The result follows easily from this.
\end{proof}







\section{Simplicial objects in abelian categories}
\label{section-abelian}

\noindent
Recall that an abelian category is defined
in Homology, Section \ref{homology-section-abelian-categories}.

\begin{lemma}
\label{lemma-abelian}
Let $\mathcal{A}$ be an abelian category.
\begin{enumerate}
\item The categories $\text{Simp}(\mathcal{A})$ and
$\text{CoSimp}(\mathcal{A})$ are abelian.
\item A morphism of (co)simplicial objects
$f : A \to B$ is injective
if and only if each $f_n : A_n \to B_n$ is injective.
\item A morphism of (co)simplicial objects
$f : A \to B$ is surjective
if and only if each $f_n : A_n \to B_n$ is surjective.
\item A sequence of (co)simplicial objects
$$
A \xrightarrow{f} B \xrightarrow{g} C
$$
is exact at $B$ if and only if each sequence
$$
A_i \xrightarrow{f_i} B_i \xrightarrow{g_i} C_i
$$
is exact at $B_i$.
\end{enumerate}
\end{lemma}

\begin{proof}
Pre-additivity is easy. A final object is
given by $U_n = 0$ in all degrees.
Existence of direct products we saw in
Lemmas \ref{lemma-product} and
\ref{lemma-product-cosimplicial-objects}.
Kernels and cokernels are obtained by taking
termwise kernels and cokernels.
\end{proof}

\noindent
For an object $A$ of $\mathcal{A}$ and an integer
$k$ consider the $k$-truncated simplicial object
$U$ with
\begin{enumerate}
\item $U_i = 0$ for $i < k$,
\item $U_k = A$,
\item all morphisms $U(\varphi)$ equal to zero,
except $U(\text{id}_{[k]}) = \text{id}_A$.
\end{enumerate}
Since $\mathcal{A}$ has both finite limits and finite colimits
we see that both $\text{cosk}_k U$ and $i_{k!}U$ exist.
We will describe both of these and the canonical
map $i_{k!}U \to \text{cosk}_kU$.

\begin{lemma}
\label{lemma-eilenberg-maclane-object}
With $A$, $k$ and $U$ as above, so $U_i = 0$, $i < k$ and $U_k = A$.
\begin{enumerate}
\item Given a $k$-truncated simplicial object $V$
we have 
$$
\text{Mor}(U, V)
=
\{ f : A \to V_k \mid d^k_i \circ f = 0,\ i = 0, \ldots, k \}
$$
and
$$
\text{Mor}(V, U)
=
\{ f : V_k \to A \mid f \circ s^{k - 1}_i = 0,\ i = 0, \ldots, k - 1 \}.
$$
\item The object $i_{k!} U$ has $n$th term equal to
$\bigoplus_\alpha A$ where $\alpha$ runs over all
surjective morphisms $\alpha : [n] \to [k]$.
\item For any $\varphi : [m] \to [n]$ the map
$i_{k!} U(\varphi)$ is described as the mapping
$\bigoplus_\alpha A \to \bigoplus_{\alpha'} A$
which maps to component corresponding to $\alpha : [n] \to [k]$
to zero if $\alpha \circ \varphi$ is not surjective and
by the identity to the component corresponding to
$\alpha \circ \varphi$ if it is surjective.
\item The object $\text{cosk}_k U$ has $n$th term equal to
$\bigoplus_\beta A$, where $\beta$ runs over all
injective morphisms $\beta : [k] \to [n]$.
\item For any $\varphi : [m] \to [n]$ the map
$\text{cosk}_k U(\varphi)$ is described as the mapping
$\bigoplus_\beta A \to \bigoplus_{\beta'} A$
which maps to component corresponding to $\beta : [k] \to [n]$
to zero if $\beta$ does not factor through $\varphi$ and
by the identity to each of the components corresponding to
$\beta'$ such that $\beta = \varphi \circ \beta'$
if it does.
\item The canonical map
$
c : i_{k !} U \to \text{cosk}_k U
$
in degree $n$ has $(\alpha, \beta)$ coefficient $A \to A$
equal to zero if $\alpha \circ \beta$ is not the identity
and equal to $\text{id}_A$ if it is.
\item The canonical map
$
c : i_{k !} U \to \text{cosk}_k U
$
is injective.
\end{enumerate}
\end{lemma}

\begin{proof}
The proof of (1) is left to the reader.

\medskip\noindent
Let us take the rules of (2) and (3)
as the definition of a simplicial object, call it $\tilde U$.
We will show that it is an incarnation of $i_{k!}U$.
This will prove (2), (3) at the same time. We have to show
that given a morphism $f : U \to \text{sk}_kV$
there exists a unique morphism $\tilde f : \tilde U \to V$
which recovers $f$ upon taking the $k$-skeleton.
From (1) we see that $f$ corresponds with a morphism
$f_k : A \to V_k$ which maps into the kernel of
$d^k_i$ for all $i$. For any surjective $\alpha : [n] \to [k]$
we set $\tilde f_\alpha : A \to V_n$ equal to the composition
$\tilde f_\alpha = V(\alpha) \circ f_k : A \to V_n$. We define
$\tilde f_n : \tilde U_n \to V_n$ as the sum of
the $\tilde f_\alpha$ over $\alpha : [n] \to [k]$ surjective.
Such a collection of $\tilde f_\alpha$ defines a morphism
of simplicial objects if and only if
for any $\varphi : [m] \to [n]$ the diagram
$$
\xymatrix{
\bigoplus_{\alpha : [n] \to [k]\text{ surjective}} A
\ar[r]_-{\tilde f_n}
\ar[d]_{(3)} &
V_n \ar[d]^{V(\varphi)} \\
\bigoplus_{\alpha' : [m] \to [k]\text{ surjective}} A
\ar[r]^-{\tilde f_m} &
V_m
}
$$
is commutative. Choosing $\varphi = \alpha$ shows our choice of
$\tilde f_\alpha$ is uniquely determined by $f_k$.
The commutativity in general may be checked for each summand
of the left upper corner separately. It is clear for the
summands corresponding to $\alpha$ where
$\alpha \circ \varphi$ is surjective, because those get
mapped by $\text{id}_A$ to the summand with
$\alpha' = \alpha \circ \varphi$, and we have
$\tilde f_{\alpha'} = V(\alpha') \circ f_k =
V(\alpha \circ \varphi) \circ f_k = V(\varphi) \circ \tilde f_\alpha$.
For those where $\alpha \circ \varphi$
is not surjective, we have to show that $V(\varphi) \circ \tilde f_\alpha = 0$.
By definition this is equal to
$V(\varphi) \circ V(\alpha) \circ f_k = V(\alpha \circ \varphi) \circ f_k$.
Since $\alpha \circ \varphi$ is not surjective we can write it
as $\delta^k_i \circ \psi$, and we deduce that
$V(\varphi) \circ V(\alpha) \circ f_k =
V(\psi) \circ d^k_i \circ f_k = 0$ see above.

\medskip\noindent
Let us take the rules of (4) and (5)
as the definition of a simplicial object, call it $\tilde U$.
We will show that it is an incarnation of $\text{cosk}_k U$.
This will prove (4), (5) at the same time. The argument is completely dual
to the proof of (2), (3) above, but we give it anyway.
We have to show
that given a morphism $f : \text{sk}_kV \to U$
there exists a unique morphism $\tilde f : V \to \tilde U$
which recovers $f$ upon taking the $k$-skeleton.
From (1) we see that $f$ corresponds with a morphism
$f_k : V_k \to A$ which is zero on the image of $s^{k - 1}_i$
for all $i$. For any injective $\beta : [k] \to [n]$
we set $\tilde f_\beta : V_n \to A$ equal to the composition
$\tilde f_\beta = f_k \circ V(\beta) : V_n \to A$. We define
$\tilde f_n : V_n \to \tilde U_n$ as the sum of
the $\tilde f_\beta$ over $\beta : [k] \to [n]$ injective.
Such a collection of $\tilde f_\beta$ defines a morphism
of simplicial objects if and only if
for any $\varphi : [m] \to [n]$ the diagram
$$
\xymatrix{
V_n
\ar[d]_{V(\varphi)}
\ar[r]_-{\tilde f_n}
&
\bigoplus_{\beta : [k] \to [n]\text{ injective}} A
\ar[d]^{(5)}
\\
V_m
\ar[r]^-{\tilde f_m}
& 
\bigoplus_{\beta' : [k] \to [m]\text{ injective}} A
}
$$
is commutative. Choosing $\varphi = \beta$ shows our choice of
$\tilde f_\beta$ is uniquely determined by $f_k$.
The commutativity in general may be checked for each summand
of the right lower corner separately. It is clear for the
summands corresponding to $\beta'$ where
$\varphi \circ \beta'$ is injective, because these summands
get mapped into by exactly the summand with
$\beta = \varphi \circ \beta'$ and we have in that case
$\tilde f_{\beta'} \circ V(\varphi) =
f_k \circ V(\beta') \circ V(\varphi) =
f_k \circ V(\beta) = \tilde f_\beta$. For those where
$\varphi \circ \beta'$ is not injective,
we have to show that $\tilde f_{\beta'} \circ V(\varphi) = 0$.
By definition this is equal to
$f_k \circ V(\beta') \circ V(\varphi) =
f_k \circ V(\varphi \circ \beta')$.
Since $\varphi \circ \beta'$ is not injective we can write it
as $\psi \circ \sigma^{k - 1}_i$, and we deduce that
$f_k \circ V(\beta') \circ V(\varphi) =
f_k \circ s^{k - 1}_i \circ V(\psi) = 0$ see above.

\medskip\noindent
The composition $i_{k!}U \to \text{cosk}_kU$ is the
unique map of simplicial objects which is
the identity on $A = U_k = (i_{k!}U)_k = (\text{cosk}_kU)_k$.
Hence it suffices to check that the proposed rule defines
a morphism of simplicial objects.
To see this we have to show that
for any $\varphi : [m] \to [n]$ the diagram
$$
\xymatrix{
\bigoplus_{\alpha : [n] \to [k]\text{ surjective}} A
\ar[d]_{(3)}
\ar[r]_{(6)}
&
\bigoplus_{\beta : [k] \to [n]\text{ injective}} A
\ar[d]^{(5)}
\\
\bigoplus_{\alpha' : [m] \to [k]\text{ surjective}} A
\ar[r]^{(6)}
& 
\bigoplus_{\beta' : [k] \to [m]\text{ injective}} A
}
$$
is commutative. Now we can think of this in terms of
matrices filled with only $0$'s and $1$'s as follows:
The matrix of (3) has a nonzero
$(\alpha', \alpha)$ entry if and only if
$\alpha' = \alpha \circ \varphi$. Likewise
the matrix of (5) has a nonzero
$(\beta', \beta)$ entry if and only if
$\beta = \varphi \circ \beta'$. The upper matrix
of (6) has a nonzero $(\alpha, \beta)$ entry if and only if
$\alpha \circ \beta = \text{id}_{[k]}$. Similarly for the
lower matrix of (6). The commutativity of the
diagram then comes down to computing the
$(\alpha, \beta')$ entry for both compositions
and seeing they are equal. This comes down to the
following equality
$$
\# \left\{
\beta
\mid
\beta = \varphi \circ \beta' \wedge \alpha \circ \beta = \text{id}_{[k]}
\right\}
=
\# \left\{
\alpha'
\mid
\alpha' = \alpha \circ \varphi \wedge \alpha' \circ \beta' = \text{id}_{[k]}
\right\}
$$
whose proof may safely be left to the reader.

\medskip\noindent
Finally, we prove (7). This follows directly from
Lemmas \ref{lemma-injective-map-simplicial-abelian},
\ref{lemma-recover-cosk}, \ref{lemma-recovering-U-for-real}
and \ref{lemma-imshriek-abelian}.
\end{proof}

\begin{definition}
\label{defintition-eilenberg-maclane}
Let $\mathcal{A}$ be an abelian category.
Let $A$ be an object of $\mathcal{A}$ and
let $k$ be an integer $\geq 0$.
The {\it Eilenberg-Maclane object $K(A, k)$}
is given by the object $K(A, k) = i_{k!}U$
which is described in
Lemma \ref{lemma-eilenberg-maclane-object} above.
\end{definition}


\begin{lemma}
\label{lemma-extension}
Let $\mathcal{A}$ be an abelian category.
Let $A$ be an object of $\mathcal{A}$ and
let $k$ be an integer $\geq 0$. Consider the
simplicial object $E$ defined by the following rules
\begin{enumerate}
\item $E_n = \bigoplus_\alpha A$, where the 
sum is over $\alpha : [n] \to [k + 1]$ whose
image is either $[k]$ or $[k + 1]$.
\item Given $\varphi : [m] \to [n]$ the map
$E_n \to E_m$ maps the summand corresponding
to $\alpha$ via $\txt{id}_A$ to the summand
corresponding to $\alpha \circ \varphi$,
provided $\text{Im}(\alpha \circ \varphi)$
is equal to $[k]$ or $[k + 1]$.
\end{enumerate}
Then there exists a short exact sequence
$$
0 \to K(A, k) \to E \to K(A, k + 1) \to 0
$$
which is term by term split exact.
\end{lemma}

\begin{proof}
The maps $K(A, k)_n \to E_n$ resp.\ $E_n \to K(A, k + 1)_n$
are given by the inclusion of direct sums, resp.\ projection
of direct sums which is obvious from the inclusions of index
sets. It is clear that these are maps of simplicial
objects.
\end{proof}

\begin{lemma}
\label{lemma-abelian-limit-skeleta}
Let $\mathcal{A}$ be an abelian category.
For any simplicial object $V$ of $\mathcal{A}$ we have
$$
V = \text{colim}_n\ i_{n!}\text{sk}_n V
$$
where all the transition maps are injections.
\end{lemma}

\begin{proof}
This is true simply because each $V_m$ is
equal to $(i_{n!}\text{sk}_n V)_m$ as
soon as $n \geq m$. See also Lemma \ref{lemma-n-skeletion-abelian}
for the transition maps.
\end{proof}

\section{Simplicial objects and chain complexes}
\label{section-complexes}

\noindent
Let $\mathcal{A}$ be an abelian category. See
Homology, Section \ref{homology-section-complexes}
for conventions and notation regarding chain
complexes.
Let $U$ be a simplicial object of $\mathcal{A}$.
The {\it associated chain complex $s(U)$ of $U$},
sometimes called the {\it Moore complex}, is
the chain complex
$$
\ldots \to U_2 \to U_1 \to U_0 \to 0 \to 0 \to \ldots
$$
with boundary maps $d_n : U_n \to U_{n - 1}$
given by the formula
$$
d_n = \sum\nolimits_{i = 0}^n (-1)^i d^n_i.
$$
This is a complex because, by the relations listed
in Remark \ref{remark-relations}, we have
\begin{eqnarray*}
d_n \circ d_{n + 1}
& = &
(\sum\nolimits_{i = 0}^n (-1)^i d^n_i) \circ
(\sum\nolimits_{j = 0}^{n + 1} (-1)^j d^{n + 1}_j) \\
& = &
\sum\nolimits_{0 \leq i < j \leq n + 1}
(-1)^{i + j} d^n_{j - 1} \circ d^{n + 1}_i
+
\sum\nolimits_{n \geq i \geq j \geq 0}
(-1)^{i + j} d^n_i \circ d^{n + 1}_j \\
& = & 0.
\end{eqnarray*}
The signs cancel! We denote the associated chain complex
$s(U)$. Clearly, the construction is functorial
and hence defines a functor
$$
s : \text{Simp}(\mathcal{A}) \longrightarrow \text{Ch}_{\geq 0}(\mathcal{A}).
$$
Thus we have the confusing but correct formula $s(U)_n = U_n$.

\begin{lemma}
\label{lemma-s-exact}
The functor $s$ is exact.
\end{lemma}

\begin{proof}
Clear from Lemma \ref{lemma-abelian}.
\end{proof}

\begin{lemma}
\label{lemma-homology-extension}
Let $\mathcal{A}$ be an abelian category.
Let $A$ be an object of $\mathcal{A}$ and
let $k$ be an integer. Let $E$ be the object
described in Lemma \ref{lemma-extension}.
Then the complex $s(E)$ is acyclic.
\end{lemma}

\begin{proof}
For a morphism $\alpha : [n] \to [k + 1]$
we define $\alpha' : [n + 1] \to [k + 1]$ to be
the map such that $\alpha'|_{[n]} = \alpha$ and
$\alpha'(n + 1) = k + 1$. Note that if the
image of $\alpha$ is $[k]$ or $[k + 1]$, then
the image of $\alpha'$ is $[k + 1]$.
Consider the family of
maps $h_n : E_n \to E_{n + 1}$ which maps
the summand corresponding to $\alpha$ to
the summand corresponding to $\alpha'$ via
the identity on $A$. 
Let us compute $d_{n + 1} \circ h_n - h_{n - 1} \circ d_n$.
We will first do this in case the category $\mathcal{A}$ is
the category of abelian groups.
Let us use the notation $x_\alpha$ to indicate
the element $x \in A$ in the summand of $E_n$ corresponding
to the map $\alpha$ occuring in the index set.
Let us also adopt the convention that
$x_\alpha$ designates the zero element of $E_n$
whenever $\text{Im}(\alpha)$ is not $[k]$ or $[k + 1]$.
With these conventions we see that
$$
d_{n + 1}(h_n(x_\alpha)) =
\sum\nolimits_{i = 0}^{n + 1} (-1)^i x_{\alpha' \circ \delta^{n + 1}_i}
$$
and 
$$
h_{n - 1}(d_n(x_\alpha)) =
\sum\nolimits_{i = 0}^n (-1)^i x_{(\alpha \circ \delta_i^n)'}
$$
It is easy to see that
$\alpha' \circ \delta^{n + 1}_i = (\alpha \circ \delta_i^n)'$
for $i = 0, \ldots, n$. It is also easy to see that
$\alpha' \circ \delta^{n + 1}_{n + 1} = \alpha$. Thus we
see that
$$
(d_{n + 1} \circ h_n - h_{n - 1} \circ d_n)(x_\alpha)
=
(-1)^{n + 1} x_\alpha
$$
These identities continue to hold if $\mathcal{A}$ is any abelian
category because they hold in the simplicial abelian group
$[n] \mapsto \text{Hom}(A, E_n)$; details left to the reader.
We conclude that the identity map on $E$ is
homotopic to zero, with homotopy given by the 
system of maps $h'_n = (-1)^{n + 1}h_n : E_n \to E_{n + 1}$.
Hence we see that $E$ is acyclic, for
example by Homology, Lemma \ref{homology-lemma-map-homology-homotopy}.
\end{proof}

\begin{lemma}
\label{lemma-homology-eilenberg-maclane}
Let $\mathcal{A}$ be an abelian category.
Let $A$ be an object of $\mathcal{A}$ and
let $k$ be an integer. We have
$H_i(s(K(A, k))) = A$ if $i = k$ and
$0$ else.
\end{lemma}

\begin{proof}
First, let us prove this if $k = 0$.
In this case we have $K(A, 0)_n = A$ for all $n$.
Furthermore, all the maps in this simplicial abelian
group are $\text{id}_A$, in other words $K(A, 0)$
is the constant simplicial object with value $A$.
The boundary maps $d_n = \sum_{i = 0}^n (-1)^i \text{id}_A
= 0$ if $n$ odd and $ = \text{id}_A$ if $n$ is even.
Thus $s(K(A, 0))$ looks like this
$$
\ldots \to A \xrightarrow{0} A \xrightarrow{1} A \xrightarrow{0} A \to 0
$$
and the result is clear.

\medskip\noindent
Next, we prove the result for all $k$ by induction.
Given the result for $k$ consider the short exact sequence
$$
0 \to K(A, k) \to E \to K(A, k + 1) \to 0
$$
from Lemma \ref{lemma-extension}.
By Lemma \ref{lemma-abelian} the associated sequence of
chain complexes is exact.
By Lemma \ref{lemma-homology-extension} we see that
$s(E)$ is acyclic. Hence the result for $k + 1$
follows from the long exact sequence of homlogy,
see Homology, Lemma \ref{homology-lemma-long-exact-sequence-chain}.
\end{proof}

\noindent
There is a second chain complex we can associate to
a simplicial object of $\mathcal{A}$. Recall that by
Lemma \ref{lemma-splitting-abelian-category}
any simplicial object $U$ of $\mathcal{A}$
is canonically split with
$N(U_m) = \bigcap_{i = 0}^{m - 1} \text{Ker}(d^m_i)$.
We define the {\it normalized chain complex $N(U)$}
to be the chain complex
$$
\ldots \to N(U_2) \to N(U_1) \to N(U_0) \to 0 \to 0 \to \ldots
$$
with boundary map $d_n : N(U_n) \to N(U_{n - 1})$ given
by the restriction of $(-1)^nd^n_n$ to the direct summand
$N(U_n)$ of $U_n$. Note that Lemma \ref{lemma-N-d-in-N}
implies that $d^n_n(N(U_n)) \subset N(U_{n - 1})$.
It is a complex because
$d^n_n \circ d^{n + 1}_{n + 1} = d^n_n \circ d^{n + 1}_n$
and $d^{n + 1}_n$ is zero on $N(U_{n + 1})$ by definition.
Thus we obtain a second functor
$$
N : \text{Simp}(\mathcal{A}) \longrightarrow \text{Ch}_{\geq 0}(\mathcal{A}).
$$
Here is the reason for the sign in the differential.

\begin{lemma}
\label{lemma-map-associated-complexes}
Let $\mathcal{A}$ be an abelian category.
Let $U$ be a simplicial object of $\mathcal{A}$.
The canonical map $N(U_n) \to U_n$ gives rise to
a morphism of complexes $N(U) \to s(U)$.
\end{lemma}

\begin{proof}
This is clear because the differential
on $s(U)_n = U_n$ is $\sum (-1)^i d^n_i$ and
the maps $d^n_i$, $i < n$ are zero on $N(U_n)$,
whereas the restriction of $(-1)^nd^n_n$ is the boundary
map of $N(U)$ by definition.
\end{proof}

\begin{lemma}
\label{lemma-N-K}
Let $\mathcal{A}$ be an abelian category.
Let $A$ be an object of $\mathcal{A}$ and
let $k$ be an integer. We have
$N(K(A, k))_i = A$ if $i = k$ and
$0$ else.
\end{lemma}

\begin{proof}
It is clear that $N(K(A, k))_i = 0$ when $i < k$
because $K(A, k)_i = 0$ in that case.
It is clear that $N(K(A, k))_k = A$ since
$K(A, k)_{k - 1} = 0$ and $K(A, k)_k = A$.
For $i > k$ we have $N(K(A, k))_i = 0$
by Lemma \ref{lemma-imshriek-abelian} and
the definition of $K(A, k)$, see Definition
\ref{defintition-eilenberg-maclane}.
\end{proof}

\begin{lemma}
\label{lemma-decompose-associated-complexes}
Let $\mathcal{A}$ be an abelian category.
Let $U$ be a simplicial object of $\mathcal{A}$.
The canonical morphism of chain complexes
$N(U) \to s(U)$ is split. In fact,
$$
S(U) = N(U) \oplus A(U)
$$
for some complex $A(U)$. The construction $U \mapsto A(U)$
is functorial.
\end{lemma}

\begin{proof}
Define $A(U)_n$ to be the image of
$$
\bigoplus\nolimits_{\varphi : [n] \to [m]\text{ surjective},\ m < n} N(U_m)
\xrightarrow{\bigoplus U(\varphi)}
U_n
$$
which is a subobject of $U_n$ complementary to
$N(U_n)$ according to Lemma \ref{lemma-splitting-abelian-category} and
Definition \ref{definition-split}. We show that
$A(U)$ is a subcomplex. Pick a surjective
map $\varphi : [n] \to [m]$ with $m < n$ and consider
the composition
$$
N(U_m) \xrightarrow{U(\varphi)} U_n \xrightarrow{d_n} U_{n - 1}.
$$
This composition is the sum of the maps
$$
N(U_m) \xrightarrow{U(\varphi \circ \delta^n_i)} U_{n - 1}
$$
with sign $(-1)^i$, $i = 0, \ldots, n$.

\medskip\noindent
First we will prove by ascending induction on $m$,
$0 \leq m < n - 1$ that all the maps $U(\varphi \circ \delta^n_i)$
map $N(U_m)$ into $A(U)_{n - 1}$. (The case $m = n - 1$ is treated below.)
Whenever the map $\varphi \circ \delta^n_i : [n - 1] \to [m]$
is surjective then the image of $N(U_m)$ under $U(\varphi \circ \delta^n_i)$
is contained in $A(U)_{n - 1}$ by definition.
If $\varphi \circ \delta^n_i : [n - 1] \to [m]$ is not surjective,
set $j = \varphi(i)$ and observe that $i$ is the unique
index whose image under $\varphi$ is $j$. We may write
$\varphi \circ \delta^n_i = \delta^m_j \circ \psi \circ \delta^n_i$
for some $\psi : [n - 1] \to [m - 1]$. Hence
$U(\varphi \circ \delta^n_i) = U(\psi \circ \delta^n_i) \circ d^m_j$ which
is zero on $N(U_m)$ unless $j = m$. If $j = m$, then
$d^m_m(N(U_m)) \subset N(U_{m - 1})$ and hence
$U(\varphi \circ \delta^n_i)(N(U_m)) \subset
U(\psi \circ \delta^n_i)(N(U_{m - 1}))$ and we win
by induction hypothesis.

\medskip\noindent
To finish proving that $A(U)$ is a subcomplex
we still have to deal with the composition
$$
N(U_m) \xrightarrow{U(\varphi)} U_n \xrightarrow{d_n} U_{n - 1}.
$$
in case $m =  n - 1$. In this case $\varphi = \sigma^{n - 1}_j$
for some $0 \leq j \leq n - 1$ and $U(\varphi) = s^{n - 1}_j$.
Thus the composition is given by the sum
$$
\sum (-1)^i d^n_i \circ s^{n - 1}_j
$$
Recall from Remark \ref{remark-relations} that
$d^n_j \circ s^{n - 1}_j = d^n_{j + 1} \circ s^{n - 1}_j = \text{id}$
and these drop out because the corresponding terms have opposite signs.
The map $d^n_n \circ s^{n - 1}_j$, if $j < n - 1$, is equal to
$s^{n - 2}_j \circ d^{n - 1}_{n - 1}$. Since 
$d^{n - 1}_{n - 1}$ maps $N(U_{n - 1})$ into $N(U_{n - 2})$,
we see that the image $d^n_n ( s^{n - 1}_j (N(U_{n - 1}))$
is contained in $s^{n - 2}_j(N(U_{n - 2}))$ which
is contained in $A(U_{n - 1})$ by definition. For all
other combinations of $(i, j)$ we have
either $d^n_i \circ s^{n - 1}_j = s^{n - 2}_{j - 1} \circ d^{n - 1}_i$
(if $i < j$), or
$d^n_i \circ s^{n - 1}_j = s^{n - 2}_j \circ d^{n - 1}_{i - 1}$
(if $n > i > j + 1$) and in these cases the map is zero because
of the definition of $N(U_{n - 1})$.
\end{proof}

\begin{lemma}
\label{lemma-N-exact}
The functor $N$ is exact.
\end{lemma}

\begin{proof}
By Lemma \ref{lemma-s-exact} and the functorial
decomposition of Lemma \ref{lemma-N-K}. 
\end{proof}

\begin{lemma}
\label{lemma-quasi-isomorphism}
Let $\mathcal{A}$ be an abelian category.
Let $V$ be a simplicial object of $\mathcal{A}$.
The canonical morphism of chain complexes
$N(V) \to s(V)$ is a quasi-isomorphism.
In other words, the complex $A(V)$ of Lemma
\ref{lemma-decompose-associated-complexes} is acyclic.
\end{lemma}

\begin{proof}
Note that the result holds for $K(A, k)$ for
any object $A$ and any $k \geq 0$, by Lemmas
\ref{lemma-homology-eilenberg-maclane} and \ref{lemma-N-K}.
Consider the hypothesis $IH_{n, m}$:
for all $V$ such that $V_j = 0$ for
$j \leq m$ and all $i \leq n$ the map
$N(V) \to s(V)$ induces an isomorphism
$H_i(N(V)) \to H_i(s(V))$.

\medskip\noindent
To start of the induction, note that $IH_{n, n}$
is trivially true, because in that case $N(V)_n = 0$
and $s(V)_n = 0$.

\medskip\noindent
Assume $IH_{n, m}$, with $m \leq n$.
Pick a simplicial object $V$ such that
$V_j = 0$ for $j < m$. By Lemma \ref{lemma-eilenberg-maclane-object}
and Definition \ref{defintition-eilenberg-maclane}
we have $K(V_m, m) = i_{m!} \text{sk}_mV$.
By Lemma \ref{lemma-n-skeletion-abelian} the natural morphism
$$
K(V_m, m) = i_{m!} \text{sk}_mV \to V
$$
is injective. Thus we get a short exact sequence
$$
0 \to K(V_m, m) \to V \to W \to 0
$$
for some $W$ with $W_i = 0$ for $i = 0, \ldots, m$. This short exact sequence
induces a morphism of short exact sequence of associated complexes 
$$
\xymatrix{
0 \ar[r] &
N(K(V_m, m)) \ar[r] \ar[d] &
N(V) \ar[r] \ar[d] &
N(W) \ar[r] \ar[d] &
0 \\
0 \ar[r] &
s(K(V_m, m)) \ar[r] &
s(V) \ar[r] &
s(W) \ar[r] &
0
}
$$
see Lemmas \ref{lemma-s-exact} and \ref{lemma-N-exact}.
Hence we deduce the result for $V$ from the result
on the ends.
\end{proof}


\section{Dold-Kan}
\label{section-dold-kan}

\begin{lemma}
\label{lemma-N-faithful}
Let $\mathcal{A}$ be an abelian category.
The functor $N$ is faithful, and reflects
isomorphisms, injections and surjections.
\end{lemma}

\begin{proof}
The faithfulness is immediate from the canonical
splitting of Lemma \ref{lemma-splitting-abelian-category}.
The statement on reflecting injections, surjections, and
isomorphisms follows from
Lemma \ref{lemma-injective-map-simplicial-abelian}.
\end{proof}

\begin{theorem}
\label{theorem-dold-kan}
Let $\mathcal{A}$ be an abelian category.
The functor $N$ induces an equivalence of
categories
$$
N :
\text{Simp}(\mathcal{A})
\longrightarrow
\text{Ch}_{\geq 0}(\mathcal{A})
$$
\end{theorem}

\begin{proof}
We will describe a functor in the reverse direction
inspired by the construction of Lemma \ref{lemma-extension}
(except that we throw in a sign to get the boundaries
right). Let $A_\bullet$ be a chain complex with boundary maps
$d_{A,n} : A_n \to A_{n - 1}$. For each $n \geq 0$ denote
$$
I_n =
\Big\{
\alpha : [n] \to \{0, 1, 2, \ldots\}
\mid
\text{Im}(\alpha) = [k]\text{ for some }k
\Big\}.
$$
For $\alpha \in I_n$ we denote $k(\alpha)$ the unique
integer such that $\text{Im}(\alpha) = [k]$.
We define a simplicial object $S(A_\bullet)$ as follows:
\begin{enumerate}
\item $S(A_\bullet)_n = \bigoplus_{\alpha \in I_n} A_{k(\alpha)}$, which
we will write as
$\bigoplus_{\alpha \in I_n} A_{k(\alpha)} \cdot \alpha$
to suggest thinking of ``$\alpha$'' as a basis vector for the
summand corresponding to it,
\item given $\varphi : [m] \to [n]$ we define
$S(A_\bullet)(\varphi)$ by its restriction to
the direct summand $A_{k(\alpha)} \cdot \alpha$
of $S(A_\bullet)_n$ as follows
\begin{enumerate}
\item $\alpha \circ \varphi \not \in I_m$ then we set it equal to zero,
\item $\alpha \circ \varphi \in I_m$ but $k(\alpha \circ \varphi)$
not equal to either $k(\alpha)$ or $k(\alpha) - 1$ then we set it
equal to zero as well,
\item if $\alpha \circ \varphi \in I_m$
and $k(\alpha \circ \varphi) = k(\alpha)$ then we use
the identity map to the summand
$A_{k(\alpha \circ \varphi)} \cdot (\alpha \circ \varphi)$
of $S(A_\bullet)_m$, and
\item if $\alpha \circ \varphi \in I_m$
and $k(\alpha \circ \varphi) = k(\alpha) - 1$
then we use the map $(-1)^{k(\alpha)} d_{A, k(\alpha)}$ to the summand 
$A_{k(\alpha \circ \varphi)}\cdot (\alpha \circ \varphi)$
of $S(A_\bullet)_m$.
\end{enumerate}
\end{enumerate}
It is a pleasant exercise to show that
this is a simplicial complex; one has to use in particular that
the compositions $d_{A, k} \circ d_{A, k - 1}$ are all zero. 

\medskip\noindent
Having verified this it is clear that $A_\bullet \mapsto
S(A_\bullet)$ is an exact functor from chain complexes
to simplicial objects. If $A_i = 0$ for $i = 0, \ldots, n$
then $S(A_\bullet)_i = 0$ for $i = 0, \ldots, n$.
The objects $K(A, k)$, see Definition \ref{defintition-eilenberg-maclane},
are equal to $S(A[-k])$ where $A[-k]$ is the chain complex
with $A$ in degree $k$ and zero elsewhere.

\medskip\noindent
Moreover, for each integer $k$ we get a sub simplicial
object $S_{\leq k}(A_\bullet)$ by considering
only those $\alpha$ with $k(\alpha) \leq k$.
In fact this is nothing but $S(\tau_{\leq k}A_\bullet)$,
where $\tau_{\leq k}A_\bullet$ is the ``stupid'' truncation
of $A_\bullet$ at $k$ (which simply replaces $A_i$ by
$0$ for $i > k$). Also, by Lemma \ref{lemma-n-skeletion-abelian}
we see that it is equal to $i_{k!}\text{sk}_k S(A_\bullet)$.
Clearly, the quotient 
$S_{\leq k}(A_\bullet)/S_{\leq k - 1}(A_\bullet) = K(A_k, k)$
and the quotient
$S(A_\bullet)/S_{\leq k}(A_\bullet) = S(A/\tau_{\leq k}A_\bullet)$
is a simplicial object whose $i$th term is zero for $i = 0, \ldots, k$.
Since $S_{\leq k - 1}(A_\bullet)$ is filtered with
subquotients $K(A_i, i)$, $i < k$ we see that
$N(S_{\leq k - 1}(A_\bullet))_k = 0$ by exactness
of the functor $N$, see Lemma \ref{lemma-N-exact}.
All in all we conclude that the maps
$$
N(S(A_\bullet))_k
\leftarrow
N(S_{\leq k}(A_\bullet))_k
\to
N(S(A_k[-k])) = N(K(A_k, k))_k = A_k
$$
are functorial isomorphisms.

\medskip\noindent
It is actually easy to identify the map $A_k \to N(S(A_\bullet))_k$.
Note that there is a unique map $A_k \to S(A_\bullet)_k$
corresponding to the summand $\alpha = \text{id}_{[k]}$.
Note that $\text{Im}(\text{id}_{[k]} \circ \delta^k_i)$
has cardinality $k - 1$ but does not
have image $[k - 1]$ unless $i = k$. Hence 
$d^k_i$ kills the summand $A_k \cdot \text{id}_{[k]}$
for $i = 0, \ldots, k - 1$. From the abstract computation
of $N(S(A_\bullet))_k$ above we conclude that
the summand $A_k \cdot \text{id}_{[k]}$ is equal to
$N(S(A_\bullet))_k$.

\medskip\noindent
In order to show that $N \circ S$ is the identity
functor on $\text{Ch}_{\geq 0}(\mathcal{A})$,
the last thing we have to verify
is that we recover the map $d_{A, k + 1} : A_{k + 1} \to A_k$
as the differential on the complex $N(S(A_\bullet))$
as follows
$$
A_{k + 1} = N(S(A_\bullet))_{k + 1} \to N(S(A_\bullet))_k = A_k
$$
By definition the map
$N(S(A_\bullet))_{k + 1} \to N(S(A_\bullet))_k$
corresponds to the restriction of $(-1)^{k + 1}d^{k + 1}_{k + 1}$
to $N(S(A_\bullet))$ which is
the summand $A_{k + 1} \cdot \text{id}_{[k + 1]}$.
And by the definition of $S(A_\bullet)$ above the 
map $d^{k + 1}_{k + 1}$ maps $A_{k + 1} \cdot \text{id}_{[k + 1]}$
into $A_k \cdot \text{id}_{[k]}$ by $(-1)^{k + 1}d_{A, k + 1}$.
The signs cancel and hence the desired equality.

\medskip\noindent
We know that $N$ is faithful, see Lemma \ref{lemma-N-faithful}.
If we can show that $S$ is essentially surjective, then
it will follow that $N$ is an equivalence,
see Homology, Lemma \ref{homology-lemma-S-N}.
Note that if $A_\bullet$ is a chain complex
then $S(A_\bullet) = \text{colim}_n\ S_{\leq n}(A_\bullet)
= \text{colim}_n\ S(\tau_{\leq n} A_\bullet) =
\text{colim}_n\ i_{n!} S(A_\bullet)$ by construction
of $S$.
By Lemma \ref{lemma-abelian-limit-skeleta} it suffices
to show that $i_{n!} V$ is in the essential image
for any $n$-truncated simplicial object $V$.
By induction on $n$ it suffices to show
that any extension
$$
0 \to S(A_\bullet) \to V \to K(A, n) \to 0
$$
where $A_i = 0$ for $i \geq n$ is in the essential
image of $S$. By Homology, Lemma \ref{homology-lemma-exact-functor-ext}
we have abelian group homomorphisms
$$
\xymatrix{
\text{Ext}_{\text{Simp}(\mathcal{A})}(K(A, n), S(A_\bullet))
\ar@<1ex>[r]^-{N} &
\text{Ext}_{\text{Ch}_{\geq 0}(\mathcal{A})}(A[-n], A_\bullet)
\ar@<1ex>[l]^-{S}
}
$$
between ext groups (see
Homology, Definition \ref{homology-definition-ext-group}).
We want to show that $S$ is surjective. We know that
$N \circ S = \text{id}$. Hence it suffices to show that
$\text{Ker}(N) = 0$. Clearly an extension
$$
\xymatrix{
&
0 \ar[r] &
0 \ar[r] \ar[d] &
A_{n - 1} \ar[r] \ar[d] &
A_{n - 2} \ar[r] \ar[d] &
\ldots \ar[r] &
A_0 \ar[r] \ar[d] &
0 \\
E : &
0 \ar[r] &
A \ar[r] \ar[d] &
A_{n - 1} \ar[r] \ar[d] &
A_{n - 2} \ar[r] \ar[d] &
\ldots \ar[r] &
A_0 \ar[r] \ar[d] &
0 \\
&
0 \ar[r] &
A \ar[r] &
0 \ar[r] &
0 \ar[r] &
\ldots \ar[r] &
0 \ar[r] &
0
}
$$
of $A_\bullet$ by $A[-n]$ in $\text{Ch}(\mathcal{A})$ is
zero if and only if the map $A \to A_{n - 1}$ is zero.
Thus we have to show that any extension
$$
0 \to S(A_\bullet) \to V \to K(A, n) \to 0
$$
such that $A = N(V)_n \to N(V)_{n - 1}$ is zero is split.
By Lemma \ref{lemma-eilenberg-maclane-object} we have
$$
\text{Mor}(K(A, n), V)
=
\left\{
f : A
\to
\bigcap\nolimits_{i = 0}^n \text{ker}(d^n_i : V_n \to V_{n - 1})
\right\}
$$
and if $A = N(V)_n \to N(V)_{n - 1}$ is zero, then
the intersection occuring in the formula above is
equal to $A$. Let $i : K(A, n) \to V$ be the morphism
corresponding to $\text{id}_A$ on the right hand side
of the displayed formula. Clearly this is a section
to the map $V \to K(A, n)$ and the extension is split
as desired.
\end{proof}







\section{Dold-Kan for cosimplicial objects}
\label{section-dold-kan-cosimplicial}

\noindent
Let $\mathcal{A}$ be an abelian category.
According to Homology, Lemma \ref{homology-lemma-abelian-opposite}
also $\mathcal{A}^{opp}$ is abelian. It follows
formally from the definitions that
$$
\text{CoSimp}(\mathcal{A}) = \text{Simp}(\mathcal{A}^{opp})^{opp}.
$$
Thus the Dold-Kan theorem \ref{theorem-dold-kan} implies
that $\text{CoSimp}(\mathcal{A})$ is equivalent to
the category
$\text{Ch}_{\geq 0}(\mathcal{A}^{opp})^{opp}$. And it
follows formally from the definitions that
$$
\text{CoCh}_{\geq 0}(\mathcal{A}) =
\text{Ch}_{\geq 0}(\mathcal{A}^{opp})^{opp}.
$$
Putting these arrows together we obtain an equivalence
$$
Q :
\text{CoSimp}(\mathcal{A})
\longrightarrow
\text{CoCh}_{\geq 0}(\mathcal{A}).
$$
In this section we describe $Q$.

\medskip\noindent
First we define the
{\it cochain complex $s(U)$ associated to a cosimplicial
object $U$}. It is the cochain complex with terms zero in
negative degrees, and $s(U)^i = U_i$ for $i \geq 0$.
As differentials we use the maps
$d^i : s(U)^i \to s(U)^{i + 1}$ defined by
$d^i = \sum_{i = 0}^{n + 1} (-1)^i \delta^{n + 1}_i$.
In other words the complex $s(U)$ looks like
$$
\xymatrix{
0 \ar[r] &
U_0 \ar[rr]^{\delta^1_0 - \delta^1_1} & &
U_1 \ar[rr]^{\delta^2_0 - \delta^2_1 + \delta^2_2} & &
U_2 \ar[r] &
\ldots
}
$$
This is sometimes also called the {\it Moore complex} associated
to $U$.

\medskip\noindent
On the other hand, given a
cosimplicial object $U$ of $\mathcal{A}$ set
$Q(U)^0 = U_0$ and 
$$
Q(U)^n = \text{Coker}(
\xymatrix{
\bigoplus_{i = 0}^{n - 1} U_{n - 1} \ar[r]^-{\delta^n_i} &
U_n
}).
$$
The differential $d^n : Q(U)^n \to Q(U)^{n + 1}$
is defined by fitting the morphism
$(-1)^{n + 1}\delta^{n + 1}_{n + 1}$ into a commutative
diagram
$$
\xymatrix{
U_n \ar[rr]_{(-1)^{n + 1}\delta^{n + 1}_{n + 1}} \ar[d] & &
U_{n + 1} \ar[d] \\
Q(U)^n \ar[rr]^{d_n} & &
Q(U)^{n + 1}.
}
$$
We leave it to the reader to show that this diagram makes
sense, i.e., that the image of $\delta^n_i$ maps into
the kernel of the right vertical arrow for $i = 0, \ldots, n - 1$.
(This is dual to Lemma \ref{lemma-N-d-in-N}.)
Thus our cochain complex $Q(U)$ looks like this
$$
0 \to Q(U)^0 \to Q(U)^1 \to Q(U)^2 \to \ldots
$$
This is called the {\it normalized cochain complex associated
to $U$}.
The dual to the Dold-Kan Theorem \ref{theorem-dold-kan} is the following.

\begin{lemma}
\label{lemma-dual-dold-kan}
Let $\mathcal{A}$ be an abelian category.
\begin{enumerate}
\item The functor
$s : \text{CoSimp}(\mathcal{A}) \to \text{CoCh}_{\geq 0}(\mathcal{A})$
is exact.
\item The maps $s(U)^n \to Q(U)^n$ define a morphism
of cochain complexes.
\item There exists a functorial direct sum decomposition
$s(U) = A(U) \oplus Q(U)$ in $\text{CoCh}_{\geq 0}(\mathcal{A})$.
\item The functor $Q$ is exact.
\item The morphism of complexes $s(U) \to Q(U)$ is a quasi-isomorphism.
\item The functor $U \mapsto Q(U)^\bullet$ defines
an equivalence of categories 
$\text{CoSimp}(\mathcal{A}) \to \text{CoCh}_{\geq 0}(\mathcal{A})$.
\end{enumerate}
\end{lemma}

\begin{proof}
Omitted. But the results are the exact dual statements to
Lemmas \ref{lemma-s-exact}, \ref{lemma-map-associated-complexes},
\ref{lemma-decompose-associated-complexes},
\ref{lemma-N-exact}, \ref{lemma-quasi-isomorphism}, and
Theorem \ref{theorem-dold-kan}.
\end{proof}

\section{Homotopies}
\label{section-homotopy}

\noindent
Consider the simplicial sets $\Delta[0]$ and $\Delta[1]$.
Recall that there are two morphisms
$$
e_i : \Delta[0] \longrightarrow \Delta[1],
$$
coming from the morphisms $[0] \to [1]$ mapping 
$0$ to $i \in \{0, 1\}$. Recall also that each
set $\Delta[1]_k$ is finite. Hence, if the category
$\mathcal{C}$ has finite coproducts, then we can
form the product
$$
U \times \Delta[1]
$$
for any simplicial object $U$ of $\mathcal{C}$, see
Definition \ref{definition-product-with-simplicial-set}.
Note that $\Delta[0]$ has the property that $\Delta[0]_k = \{*\}$
is a singleton for all $k \geq 0$. Hence $U \times \Delta[0]
= U$. Thus $e_i$ above gives rise to morphisms
$$
e_i : U \to U \times \Delta[1].
$$

\begin{definition}
\label{definition-homotopy}
Let $\mathcal{C}$ be a category having finite coproducts.
Suppose that $U$ and $V$ are two simplicial objects
of $\mathcal{C}$.
We say morphisms $a, b : U \to V$ are {\it homotopic}
if there exists a morphism
$$
h : U \times \Delta[1] \longrightarrow V
$$
such that $a = h \circ e_0$ and $b = h \circ e_1$.
In this case $h$ is called a {\it homotopy connecting
$a$ and $b$}.
\end{definition}

\noindent
It is possible to define this notion for pairs of
maps between simplicial objects in any category.
To do this you just work out what it means to
have the morphisms $h_n : (U \times \Delta[1])_n \to V_n$
in terms of the mapping property of coproducts.

\medskip\noindent
Let $\mathcal{C}$ be a category with finite coproducts.
Let $U$, $V$ be simplicial objects of $\mathcal{C}$.
Let $a, b : U \to V$ be morphisms. Further, suppose
that $h : U \times \Delta[1] \to V$ is a homotopy
connecting $a$ and $b$. For every $n \geq 0$
let us write
$$
\Delta[1]_n = \{\alpha^n_0, \ldots, \alpha^n_{n + 1}\}
$$
where $\alpha_i : [n] \to [1]$ is the map
such that
$$
\alpha^n_i(j)
=
\left\{
\begin{matrix}
0 & \text{if} & j < i\\
1 & \text{if} & j \geq i
\end{matrix}\right.
$$
Thus
$$
h_n : (U \times \Delta[1])_n = \coprod U_n \cdot \alpha^n_i
\longrightarrow
V_n
$$
has a component $h_{n,i} : U_n \to V_n$ which is the restriction
to the summand corresponding to $\alpha^n_i$ for all $i = 0, \ldots, n + 1$.

\begin{lemma}
\label{lemma-relations-homotopy}
In the situation above, we have the following relations:
\begin{enumerate}
\item We have $h_{n, 0} = b_n$ and $h_{n, n + 1} = a_n$.
\item We have $d^n_j \circ h_{n, i} = h_{n - 1, i - 1} \circ d^n_j$
for $i > j$.
\item We have $d^n_j \circ h_{n, i} = h_{n - 1, i} \circ d^n_j$
for $i \leq j$.
\item We have $s^n_j \circ h_{n, i} = h_{n + 1, i + 1} \circ s^n_j$
for $i > j$.
\item We have $s^n_j \circ h_{n, i} = h_{n + 1, i} \circ s^n_j$
for $i \leq j$.
\end{enumerate}
Conversely, given a system of maps $h_{n, i}$ satisfying the
properties listed above, then these define a morphisms
$h$ which is a homotopy between $a$ and $b$.
\end{lemma}

\begin{proof}
Omitted. You can prove the last statement using the fact, 
see Lemma \ref{lemma-face-degeneracy-category} that
to give a morphism of simplicial objects is the
same as giving a sequence of morphisms $h_n$ commuting
with all $d^n_j$ and $s^n_j$.
\end{proof}

\noindent
The following lemma says that $U \times \Delta[1]$
is ``homotopy equivalent'' to $U$.

\begin{lemma}
\label{lemma-contractible}
Let $\mathcal{C}$ be a category with finite coproducts.
Let $U$ be a simplicial object of $\mathcal{C}$.
Consider the maps $e_1, e_0 : U \to U\times \Delta[1]$,
and $\pi : U\times \Delta[1] \to U$, see
Lemma \ref{lemma-back-to-U}. 
\begin{enumerate}
\item We have $\pi \circ e_1 = \pi \circ e_0 = \text{id}_U$, and
\item The morphisms $\text{id}_{U \times \Delta[1]}$,
and $e_0 \circ \pi$ are homotopic.
\item The morphisms $\text{id}_{U \times \Delta[1]}$,
and $e_1 \circ \pi$ are homotopic.
\end{enumerate}
\end{lemma}

\begin{proof}
The first assertion is trivial.
For the second, consider the map
of simplicial sets
$\Delta[1] \times \Delta[1] \longrightarrow \Delta[1]$
which in degree $n$ assigns to a pair $(\beta_1, \beta_2)$,
$\beta_i : [n] \to [1]$ the morphism
$\beta : [n] \to [1]$ defined by the rule
$$
\beta(i) = \max\{\beta_1(i), \beta_2(i)\}.
$$
It is a morphism of simplicial sets, because the action
$\Delta[1](\varphi) : \Delta[1]_n \to \Delta[1]_m$
of $\varphi : [m] \to [n]$ is by precomposing.
Clearly, using notation from Section \ref{section-homotopy},
we have $\beta = \beta_1$ if $\beta_2 = \alpha^n_0$
and $\beta = \alpha^n_{n + 1}$ if $\beta_2 = \alpha^n_{n + 1}$.
This implies easily that the induced morphism
$$
U \times \Delta[1] \times \Delta[1]
\longrightarrow
U \times \Delta[1]
$$
of Lemma \ref{lemma-back-to-U}
is a homotopy between $\text{id}_{U \times \Delta[1]}$ and $e_0 \circ \pi$.
Similarly for $e_1 \circ \pi$ (use minimum instead of maximum).
\end{proof}


\section{Homotopies in abelian categories}
\label{section-homotopy-abelian}

\noindent
Let $\mathcal{A}$ be an abelian category.
Let $U$, $V$ be simplicial objects of $\mathcal{A}$.
Let $a, b : U \to V$ be morphisms. Further, suppose
that $h : U \times \Delta[1] \to V$ is a homotopy
connecting $a$ and $b$. Consider the two morphisms
of chain complexes
$$
s(a), s(b) : s(U) \longrightarrow s(V).
$$
Using the notation introduced above Lemma \ref{lemma-relations-homotopy}
we define
$$
s(h)_n : U_n \longrightarrow V_{n + 1}
$$
by the formula
$$
s(h)_n = \sum_{i = 0}^n (-1)^i h_{n + 1, i} \circ s^n_i.
$$
Let us compute $d_{n + 1} \circ s(h)_n + s(h)_{n - 1} \circ d_n$.
Here we go, where the first line is just the definitions
written out
$$
\begin{matrix}
\sum_{j = 0}^{n + 1} \sum_{i = 0}^n 
(-1)^{j + i} d^{n + 1}_j \circ h_{n + 1, i} \circ s^n_i
+ 
\sum_{i = 0}^{n - 1} \sum_{j = 0}^n
(-1)^{i + j} h_{n, i} \circ s^{n - 1}_i \circ d^n_j \\
= \\
\sum_{j = 0}^{n + 1}
\left(
\sum_{i = j + 1}^n
(-1)^{j + i} h_{n, i - 1} \circ d^{n + 1}_j \circ s^n_i
+
\sum_{i = 0}^j 
(-1)^{j + i} h_{n, i} \circ d^{n + 1}_j \circ s^n_i
\right) \\
+ \\
\sum_{i = 0}^{n - 1} \sum_{j = 0}^n
(-1)^{i + j} h_{n, i} \circ s^{n - 1}_i \circ d^n_j \\
= \\
\sum_{j = 0}^{n + 1}
\left(
\sum_{i = j + 1}^n
(-1)^{j + i} h_{n, i - 1} \circ s^{n - 1}_{i - 1} \circ d^n_j
+
\sum_{i = j - 1}^j 
(-1)^{j + i} h_{n, i} \right.\hfill\\
\hfill\left.
+
\sum_{i = 0}^{j - 2}
(-1)^{j + i} h_{n, i} \circ s^{n - 1}_i \circ d^n_{j - 1}
\right) \\
+ \\
\sum_{i = 0}^{n - 1} \sum_{j = 0}^n
(-1)^{i + j} h_{n, i} \circ s^{n - 1}_i \circ d^n_j \\
= \\
(-1)^{0 + 0} h_{n , 0} + (-1)^{n + 1 + n} h_{n, n} = b_n - a_n
\end{matrix}
$$
Thus we've proved part of the following lemma.

\begin{lemma}
\label{lemma-homotopy-s-N}
Let $\mathcal{A}$ be an abelian category.
Let $a, b : U \to V$ be morphisms of simplicial
objects of $\mathcal{A}$. If $a$, $b$ are homotopic,
then $s(a), s(b) : s(U) \to s(V)$, and
$N(a), N(b) : N(U) \to N(V)$ are homotopic maps
of chain complexes.
\end{lemma}

\begin{proof}
The part about $s(a)$ and $s(b)$ is clear from the calculation
above the lemma. On the other hand, if follows from
Lemma \ref{lemma-decompose-associated-complexes} that
$N(a)$, $N(b)$ are compositions
$$
N(U) \to s(U) \to s(V) \to N(V)
$$
where we use $s(a)$, $s(b)$ in the middle. Hence the assertion
follows from 
Homology, Lemma \ref{homology-lemma-compose-homotopy}.
\end{proof}

\begin{lemma}
\label{lemma-homotopy-backwards}
Let $\mathcal{A}$ be an abelian category.
Let $A, B \in \text{Ob}(\text{Ch}_{\geq 0}(\mathcal{A}))$.
Let $a, b : A \to B$ be morphisms of chain complexes
and let $\{h_n : A_n \to B_{n + 1}$ be a homotopy between
$a$ and $b$.
Let $U, V \in \text{Ob}(\text{Simp}(\mathcal{A}))$ be
such that $N(U) = A$ and $N(V) = B$.
Then there exists a morphism $U \times \Delta[1] \to V$
which gives rise to the homotopy $h$ connecting
$a$ and $b$ via the process of Lemma \ref{lemma-homotopy-s-N}.
\end{lemma}

\begin{proof}
Let $S$ be the functor
$\text{Ch}_{\geq 0}(\mathcal{A}) \to \text{Simp}(\mathcal{A})$
constructed in the proof of the Dold-Kan theorem \ref{theorem-dold-kan}.
We may assume that $U = S(A)$ and $V = S(B)$.
Our task is to construct a map
$$
S(h) : S(A) \times \Delta[1] \longrightarrow S(B)
$$
which has all the required properties. We have
$$
(S(A) \times \Delta[1])_n
=
\bigoplus\nolimits_{(\alpha, \beta)} A_{k(\alpha)} \times (\alpha, \beta)
\text{ and }
S(B)_n = \bigoplus\nolimits_{\alpha} B_{k(\alpha)} \times \alpha
$$
where $\alpha \in I_n$ and $\beta : [n] \to [1]$. We will also
use the notation, introduced in Section \ref{section-homotopy},
where $\text{Mor}_\Delta([n], [1]) = \{\alpha^n_0, \ldots, \alpha^n_{n + 1}\}$.
As another part of the notation let us use $d_{A, k}$, resp.\ 
$d_{B, k}$ as the differential on the chain complex
$A$, resp.\ $B$.
We define our morphism $S(h)$ by the following rules
\begin{enumerate}
\item For $\beta = \alpha^n_0$: We map the component
$A_{k(\alpha)} \times (\alpha, \alpha^n_0)$ to the component
$B_{k(\alpha)}$ via the morphism $b_{k(\alpha)} \cdot \alpha$.
\item For $\beta = \alpha^n_{n + 1}$: We map the component
$A_{k(\alpha)} \times (\alpha, \alpha^n_{n + 1})$ to the component
$B_{k(\alpha)} \cdot \alpha$ via the morphism $a_{k(\alpha)}$.
\item For $\beta = \alpha^n_i$ with $i = 1, \ldots, n$:
Consider the map $\alpha + \beta : [n] \to \{0, 1, \ldots\}$.
If $\alpha + \beta \not \in I_n$, or $\alpha + \beta \in I_n$
but $k(\alpha + \beta) \not = k(\alpha) + 1$, then we use
$$
A_{k(\alpha)} \cdot (\alpha, \beta)
\xrightarrow{b_{k(\alpha)}}
B_{k(\alpha)} \cdot \alpha
$$
If $\alpha + \beta \in I_n$ and $k(\alpha + \beta) = k(\alpha) + 1$
then we use
$$
A_{k(\alpha)} \times (\alpha, \beta)
\xrightarrow{b_{k(\alpha)}, h_{k(\alpha)}}
B_{k(\alpha)} \cdot \alpha \oplus
B_{k(\alpha + \beta)} \cdot (\alpha + \beta).
$$
\end{enumerate}
Let $\varphi : [m] \to [n]$. We check that the prescription
above does indeed commute with the action of $\varphi$.
Pick a pair $(\alpha, \beta)$. There are several cases to
consider:
\begin{enumerate}
\item The case $\beta = \alpha^n_0$. In this case
we are just checking the functoriality of the construction $S$,
but let us do it as a warm up anyways. Here we map
$A_{k(\alpha)} \cdot (\alpha , \beta)$ into $B_{k(\alpha)} \cdot \alpha$.
Here we also have $\beta \circ \varphi = \alpha^m_0$.
We further split this case into the following subcases
corresponding to the subcases in the definition of the maps
of the complex $S(A)$ and $S(B)$:
\begin{enumerate}
\item $\alpha \circ \varphi \not \in I_m$, or
$\alpha \circ \varphi \in I_n$ but
$k(\alpha \circ \varphi) \not = k(\alpha)$, or $k(\alpha) - 1$.
In this case the map $(S(A) \times \Delta[1])(\varphi)$
is zero on the component $A_{k(\alpha)} \cdot (\alpha , \beta)$
and the map $S(B)(\varphi)$ is zero on the component
$B_{k(\alpha)} \cdot \alpha$. Thus the commutativity is
clear.
\item $\alpha \circ \varphi \in I_n$ and
$k(\alpha \circ \varphi) = k(\alpha)$.
In this case the commutativity follows from the
commutativity of the diagram
$$
\xymatrix{
A_{k(\alpha)} \cdot (\alpha , \beta) \ar[r]^{b_{k(\alpha)}} \ar[d]^{\text{id}} &
B_{k(\alpha)} \cdot \alpha \ar[d]^{\text{id}} \\
A_{k(\alpha \circ \varphi)}
\cdot (\alpha \circ \varphi , \beta \circ \varphi) \ar[r]^{b_{k(\alpha)}} &
B_{k(\alpha \circ \varphi)} \cdot (\alpha \circ \varphi)
}
$$
\item $\alpha \circ \varphi \in I_n$ and
$k(\alpha \circ \varphi) = k(\alpha) - 1$.
In this case the commutativity follows from the
commutativity of the diagram
$$
\xymatrix{
A_{k(\alpha)} \cdot (\alpha , \beta) \ar[r]^{b_{k(\alpha)}}
\ar[d]^{d_{A, k(\alpha)}} &
B_{k(\alpha)} \cdot \alpha \ar[d]^{d_{B, k(\alpha)}} \\
A_{k(\alpha \circ \varphi)}
\cdot (\alpha \circ \varphi , \beta \circ \varphi) \ar[r]^{b_{k(\alpha)}} &
B_{k(\alpha \circ \varphi)} \cdot (\alpha \circ \varphi)
}
$$
\end{enumerate}
\item The case $\beta = \alpha^n_{n + 1}$. This is checked
in exactly the same way as the case $\beta = \alpha^n_0$.
\item The case $\beta \not = \alpha^n_0$ and $\beta \not = \alpha^n_{n + 1}$.
Here we map $A_{k(\alpha)} \cdot (\alpha , \beta)$ into
either
$B_{k(\alpha)} \cdot \alpha$, or
$B_{k(\alpha)} \cdot \alpha \oplus
B_{k(\alpha + \beta)} \cdot (\alpha + \beta)$
depending on the behaviour of $\alpha + \beta$.
We split this out into cases as follows.
\begin{enumerate}
\item $\alpha + \beta \not \in I_n$ and
$\alpha \circ \varphi + \beta \circ \varphi \not \in I_m$.
both $S(h)_n$ and $S(h)_m$ are zero
on the relevant summands and the commutativity follows trivially.
\item $\alpha + \beta \not \in I_n$ and
$(\alpha + \beta) \circ \varphi =
\alpha \circ \varphi + \beta \circ \varphi \in I_m$.
This means the diagram looks like this
$$
\xymatrix{
A_{k(\alpha)} \cdot (\alpha , \beta) \ar[r]
\ar[d] &
0 \ar[d] \\
A_{k(\alpha \circ \varphi)}
\cdot (\alpha \circ \varphi , \beta \circ \varphi) \ar[r] &
?
}
$$
If $k(\alpha \circ \varphi)$ is not $k(\alpha), k(\alpha) - 1$.
then the left vertical arrow is zero also and the diagram
commutes. How can this happen? Because $\alpha \in I_n$,
if $\alpha + \beta \not \in I_n$
then there exists an $0 \leq j < n$ such that
$\alpha(j) + 1 = \alpha(j + 1) \leq k(\alpha)$
and $\beta(j) = 0$, $\beta(j + 1) = 1$. In order to have
$(\alpha + \beta) \circ \varphi \in I_m$ we then see that
$\varphi$ cannot attain a value greater than $j$.
In particular $\beta \circ \varphi = \alpha^n_0$.
Thus the question mark in the diagram is
$B_{k(\alpha \circ \varphi)} \cdot (\alpha \circ \varphi)$.

Thus $k((\alpha + \beta) \circ \varphi) \leq k(\alpha) + \beta) - 2$.
This immediately forces the lower horizontal arrow to be zero,
and the diagram commutes.


\end{enumerate}
\end{enumerate}


\end{proof}










\section{A homotopy equivalence}
\label{section-homotopy-equivalence}

\noindent
Suppose that $A$, $B$ are sets, and that $f : A \to B$
is a map. Consider the associated map of
simplicial sets
$$
\xymatrix{
\text{cosk}_0(A) \ar@{=}[r] &
\Big(
\ldots
A\times A \times A
\ar[d]
\ar@<2ex>[r]
\ar@<0ex>[r]
\ar@<-2ex>[r]
&
A \times A
\ar[d]
\ar@<1ex>[r]
\ar@<-1ex>[r]
\ar@<1ex>[l]
\ar@<-1ex>[l]
&
A
\ar[d]
\ar@<0ex>[l]
\Big)
\\
\text{cosk}_0(B) \ar@{=}[r] &
\Big(
\ldots
B\times B \times B
\ar@<2ex>[r]
\ar@<0ex>[r]
\ar@<-2ex>[r]
&
B \times B
\ar@<1ex>[r]
\ar@<-1ex>[r]
\ar@<1ex>[l]
\ar@<-1ex>[l]
&
B
\ar@<0ex>[l]
\Big)
}
$$
See Example \ref{example-cosk0}.
The case $n = 0$ of the following lemma
says that this map of simplicial sets
has a section if $f$ is surjective.
The proof: choose a section of $f$.

\begin{lemma}
\label{lemma-section}
Let $f : V \to U$ be a morphism of simplicial sets.
Let $n \geq 0$ be an integer.
Assume
\begin{enumerate}
\item The map $f_i : V_i \to U_i$ is a bijection for $i < n$.
\item The map $f_n : V_n \to U_n$ is a surjection.
\item The canonical morphism $U \to \text{cosk}_n \text{sk}_n U$
is an isomorphism.
\item The canonical morphism $V \to \text{cosk}_n \text{sk}_n V$
is an isomorphism.
\end{enumerate}
Then there exists a morphism of simplicial sets $g : U \to V$
such that $f \circ g = \text{id}_U$.
\end{lemma}

\begin{proof}
By Lemma \ref{lemma-splitting-simplicial-sets}
both $U$ and $V$ have canonical splittings with $N(U_i)$
and $N(V_i)$ equal to the sets of nondegenerate simplices.
We have to find maps $g_m : U_m \to V_m$ for $m \geq 0$ such
that
\begin{eqnarray}
d^k_i \circ g_k & = & g_{k - 1} \circ d^k_i \label{cd}\\
s^k_i \circ g_k & = & g_{k + 1} \circ s^k_i \label{cs}
\end{eqnarray}
for all $k$. By induction on $m$ we will show that we can find maps
$g_0, \ldots, g_m$ such that (\ref{cd}) holds for
$1 \leq k \leq m$ and (\ref{cs}) holds for $0 \leq k \leq m - 1$.
We set $g_i$ equal to the inverse of $f_i$ for $i = 0, \ldots, n - 1$.
Clearly the induction hypothesis holds for $m = n - 1$.
We define $g_n : U_n \to V_n$ as follows.
Pick $u \in U_n$, then
\begin{enumerate}
\item if $u$ is degenerate, write  $u = U(\varphi)(u')$
for some nondegenerate $u' \in U_m$ and some
surjective $\varphi : [n] \to [m]$. We set
$g_n(u) = V(\varphi)(g_m(u'))$. This is well defined
as the pair $(\varphi, u')$ is unique.
\item if $u$ is nondegenerate, we choose any $v \in V_n$
mapping to $u$ and we set $g_n(u) = v$.
\end{enumerate}
This choice of $g_n$ garantees that the induction hypothesis
holds for $m = n$. Namely, we forced (\ref{cs}) with $k = n - 1$
by our choice of $g_n$ on degenerate simplices, and (\ref{cd})
with $k = n$ holds because the equality takes place in
$V_{n - 1} = U_{n - 1}$.

\medskip\noindent
One way to finish the proof at this point is to show
that the family of maps $g_0, \ldots, g_n$ defines
a morphism of $n$-truncated simplicial sets
$\text{sk}_n U \to \text{sk}_n V$ which is
a right inverse to $\text{sk}_nf$. Then since
$\text{cosk}_n$ is a functor and by the hypothesis
of the lemma we get $g$ as $\text{cosk}_n(g_0, \ldots, g_n)$.
But we can also see this directly as follows.

\medskip\noindent
Given the induction hypothesis for $m \geq n$
we inductively define $g_{m + 1}$ as follows.
Since $U \to \text{cosk}_n \text{sk}_n U$
is an isomorphism, we see that also
$U \to \text{cosk}_m \text{sk}_m U$ is an
isomorphism. Hence elements of $U_{m + 1}$
are $(m + 2)$-tuples $(u_0, \ldots, u_{m + 1})$ with
$u_i \in U_m$ satisfying the equalities
$d^m_{j - 1}(u_i) = d^m_i(u_j)\ \forall\ 0\leq i < j\leq m + 1$.
Similarly for $V_{m + 1}$.
Thus we may simply map the element
$(u_0, \ldots, u_{m + 1})$ to the element
$(g_m(u_0), \ldots, g_m(u_{m + 1}))$.
To verify the induction hypothesis for $m + 1$ with
this choice of $g_{m + 1}$ we will use the
explicit form of the maps $d_i$ and $s_i$
as given in Remark \ref{remark-explicit-face-degeneracy}.
This remark shows immediately that the commutation of
$g_0, \ldots, g_m$ with $d_i$ and $s_i$ implies the
desired commutation for $g_{m + 1}$.
\end{proof}

\noindent
Let $A, B$ be sets. Let $f^0, f^1 : A \to B$ be maps of sets.
Consider the induced maps $f^0, f^1 : \text{cosk}_0(A) \to \text{cosk}_0(B)$
abusively denoted by the same symbols. The following lemma for $n = 0$
says that $f_0$ is homotopic to $f_1$. In fact, the
homotopy is given by the map $h : \text{cosk}_0(A) \times
\Delta[1] \to \text{cosk}_0(A)$ with components
\begin{eqnarray*}
h_m : A \times \ldots \times A \times \text{Mor}_{\Delta}([m], [1])
& \longrightarrow &
A \times \ldots \times A, \\
(a_0, \ldots, a_m, \alpha) & \longmapsto &
(f^{\alpha(0)}(a_0), \ldots, f^{\alpha(m)}(a_m))
\end{eqnarray*}
To check that this works, note that for a map $\varphi : [k] \to [m]$
the induced maps are
$(a_0, \ldots, a_m) \mapsto (a_{\varphi(0)}, \ldots, a_{\varphi(k)})$
and $\alpha \mapsto \alpha \circ \varphi$. Thus $h = (h_m)_{m \geq 0}$
is clearly a map of simplicial sets as desired.

\begin{lemma}
\label{lemma-homotopy}
Let $f^0, f^1 : V \to U$ be maps of a simplicial sets.
Let $n \geq 0$ be an integer.
Assume
\begin{enumerate}
\item The maps $f^j_i : V_i \to V_i$, $j = 0, 1$ are equal for $i < n$.
\item The canonical morphism $U \to \text{cosk}_n \text{sk}_n U$
is an isomorphism.
\item The canonical morphism $V \to \text{cosk}_n \text{sk}_n V$
is an isomorphism.
\end{enumerate}
Then $f^0$ is homotopic to $f^1$.
\end{lemma}

\begin{proof}
We have to construct
a morphism of simplicial sets $h : V \times \Delta[1] \to U$
which recovers $f^i$ on composing with $e_i$.
The case $n = 0$ was dealt with above the lemma.
Thus we may assume that $n \geq 1$.
The map $\Delta[1] \to \text{cosk}_1 \text{sk}_1 \Delta[1]$
is an isomorphism, see Lemma \ref{lemma-simplex-cosk}.
Thus we see that $\Delta[1] \to \text{cosk}_n \text{sk}_n \Delta[1]$
is an isomorphism as $n \geq 1$, see
Lemma \ref{lemma-cosk-up}. And hence $V \times \Delta[1] \to
\text{cosk}_n \text{sk}_n (V \times \Delta[1])$
is an isomorphism too, see Lemma \ref{lemma-cosk-product}.
In other words, in order to construct the homotopy
it suffices to construct a suitable
morphism of $n$-truncated simplicial sets
$h : \text{sk}_n V \times \text{sk}_n \Delta[1] \to \text{sk}_n U$.

\medskip\noindent
For $k = 0, \ldots, n - 1$ we define $h_k$ by the
formula $h_k(v, \alpha) = f^0(v) = f^1(v)$.
The map $h_n : V_n \times \text{Mor}_{\Delta}([k], [1]) \to U_n$
is defined as follows. Pick $v \in V_n$ and $\alpha : [n] \to [1]$:
\begin{enumerate}
\item If $\text{Im}(\alpha) = \{0\}$, then we set $h_n(v, \alpha) = f^0(v)$.
\item If $\text{Im}(\alpha) = \{0, 1\}$, then we set $h_n(v, \alpha) = f^0(v)$.
\item If $\text{Im}(\alpha) = \{1\}$, then we set $h_n(v, \alpha) = f^1(v)$.
\end{enumerate}
Let $\varphi : [k] \to [l]$ be a morphism of $\Delta_{\leq n}$.
We will show that the diagram
$$
\xymatrix{
V_{[l]} \times \text{Mor}([l], [1]) \ar[r] \ar[d] &
U_{[l]} \ar[d] \\
V_{[k]} \times \text{Mor}([k], [1]) \ar[r] &
U_{[k]}
}
$$
commutes.
Pick $v \in V_{[l]}$ and $\alpha : [l] \to [1]$.
The commutativity means that
$$
h_k(V(\varphi)(v), \alpha \circ \varphi)
=
U(\varphi)(h_l(v, \alpha)).
$$
In almost every case this holds because
$h_k(V(\varphi)(v), \alpha \circ \varphi) = f^0(V(\varphi)(v))$
and $U(\varphi)(h_l(v, \alpha)) = U(\varphi)(f^0(v))$, combined
with the fact that $f^0$ is a morphism of simplicial sets.
The only cases where this does not hold is when
either (A) $\text{Im}(\alpha) = \{1\}$ and $l = n$
or (B) $\text{Im}(\alpha \circ \varphi) = \{1\}$ and $k = n$.
Observe moreover that necessarily $f^0(v) = f^1(v)$
for any degenerate $n$-simplex of $V$.
Thus we can narrow the cases above down even further
to the cases (A) $\text{Im}(\alpha) = \{1\}$, $l = n$
and $v$ nondegenerate, and (B)
$\text{Im}(\alpha \circ \varphi) = \{1\}$, $k = n$
and $V(\varphi)(v)$ nondegenerate.

\medskip\noindent
In case (A), we see that also $\text{Im}(\alpha \circ \varphi) = \{1\}$.
Hence we see that not only $h_l(v, \alpha) = f^1(v)$ but also
$h_k(V(\varphi)(v), \alpha \circ \varphi) = f^1(V(\varphi)(v))$.
Thus we see that the relation holds because $f^1$ is a morphism
of simplicial sets.

\medskip\noindent
In case (B) we conclude that $l = k = n$ and
$\varphi$ is bijective, since otherwise $V(\varphi)(v)$
is degenerate. Thus $\varphi = \text{id}_{[n]}$, which is a trivial case.
\end{proof}

\begin{lemma}
\label{lemma-equiv}
With assumptions and notation as in Lemma \ref{lemma-section}
above. The composition $g \circ f$ is homotopy equivalent
to the identity on $V$.
\end{lemma}

\begin{proof}
Immediate from Lemma \ref{lemma-homotopy} above.
\end{proof}























\section{Other chapters}

\begin{multicols}{2}
\begin{enumerate}
\item \hyperref[introduction-section-phantom]{Introduction}
\item \hyperref[conventions-section-phantom]{Conventions}
\item \hyperref[sets-section-phantom]{Set Theory}
\item \hyperref[categories-section-phantom]{Categories}
\item \hyperref[topology-section-phantom]{Topology}
\item \hyperref[sheaves-section-phantom]{Sheaves on Spaces}
\item \hyperref[algebra-section-phantom]{Commutative Algebra}
\item \hyperref[sites-section-phantom]{Sites and Sheaves}
\item \hyperref[homology-section-phantom]{Homological Algebra}
\item \hyperref[derived-section-phantom]{Derived Categories}
\item \hyperref[more-algebra-section-phantom]{More Algebra}
\item \hyperref[simplicial-section-phantom]{Simplicial Methods}
\item \hyperref[modules-section-phantom]{Sheaves of Modules}
\item \hyperref[sites-modules-section-phantom]{Modules on Sites}
\item \hyperref[injectives-section-phantom]{Injectives}
\item \hyperref[cohomology-section-phantom]{Cohomology of Sheaves}
\item \hyperref[sites-cohomology-section-phantom]{Cohomology on Sites}
\item \hyperref[hypercovering-section-phantom]{Hypercoverings}
\item \hyperref[schemes-section-phantom]{Schemes}
\item \hyperref[constructions-section-phantom]{Constructions of Schemes}
\item \hyperref[properties-section-phantom]{Properties of Schemes}
\item \hyperref[morphisms-section-phantom]{Morphisms of Schemes}
\item \hyperref[coherent-section-phantom]{Coherent Cohomology}
\item \hyperref[divisors-section-phantom]{Divisors}
\item \hyperref[limits-section-phantom]{Limits of Schemes}
\item \hyperref[varieties-section-phantom]{Varieties}
\item \hyperref[chow-section-phantom]{Chow Homology}
\item \hyperref[topologies-section-phantom]{Topologies on Schemes}
\item \hyperref[descent-section-phantom]{Descent}
\item \hyperref[more-morphisms-section-phantom]{More on Morphisms}
\item \hyperref[flat-section-phantom]{More on Flatness}
\item \hyperref[groupoids-section-phantom]{Groupoid Schemes}
\item \hyperref[more-groupoids-section-phantom]{More on Groupoid Schemes}
\item \hyperref[etale-section-phantom]{\'Etale Morphisms of Schemes}
\item \hyperref[etale-cohomology-section-phantom]{\'Etale Cohomology}
\item \hyperref[spaces-section-phantom]{Algebraic Spaces}
\item \hyperref[spaces-properties-section-phantom]{Properties of Algebraic Spaces}
\item \hyperref[spaces-morphisms-section-phantom]{Morphisms of Algebraic Spaces}
\item \hyperref[spaces-topologies-section-phantom]{Topologies on Algebraic Spaces}
\item \hyperref[spaces-descent-section-phantom]{Descent and Algebraic Spaces}
\item \hyperref[spaces-more-morphisms-section-phantom]{More on Morphisms of Spaces}
\item \hyperref[quot-section-phantom]{Quot and Hilbert Spaces}
\item \hyperref[stacks-section-phantom]{Stacks}
\item \hyperref[spaces-groupoids-section-phantom]{Groupoids in Algebraic Spaces}
\item \hyperref[spaces-more-groupoids-section-phantom]{More on Groupoids in Spaces}
\item \hyperref[bootstrap-section-phantom]{Bootstrap}
\item \hyperref[examples-stacks-section-phantom]{Examples of Stacks}
\item \hyperref[groupoids-quotients-section-phantom]{Quotients of Groupoids}
\item \hyperref[algebraic-section-phantom]{Algebraic Stacks}
\item \hyperref[criteria-section-phantom]{Criteria for Representability}
\item \hyperref[stacks-properties-section-phantom]{Properties of Algebraic Stacks}
\item \hyperref[stacks-morphisms-section-phantom]{Morphisms of Algebraic Stacks}
\item \hyperref[examples-section-phantom]{Examples}
\item \hyperref[exercises-section-phantom]{Exercises}
\item \hyperref[guide-section-phantom]{Guide to Literature}
\item \hyperref[desirables-section-phantom]{Desirables}
\item \hyperref[coding-section-phantom]{Coding Style}
\item \hyperref[fdl-section-phantom]{GNU Free Documentation License}
\item \hyperref[index-section-phantom]{Auto Generated Index}
\end{enumerate}
\end{multicols}


\bibliography{my}
\bibliographystyle{alpha}

\end{document}
