\IfFileExists{stacks-project.cls}{%
\documentclass{stacks-project}
}{%
\documentclass{amsart}
}

% The following AMS packages are automatically loaded with
% the amsart documentclass:
%\usepackage{amsmath}
%\usepackage{amssymb}
%\usepackage{amsthm}

% For dealing with references we use the comment environment
\usepackage{verbatim}
\newenvironment{reference}{\comment}{\endcomment}
%\newenvironment{reference}{}{}
\newenvironment{slogan}{\comment}{\endcomment}
\newenvironment{history}{\comment}{\endcomment}

% For commutative diagrams you can use
% \usepackage{amscd}
\usepackage[all]{xy}

% We use 2cell for 2-commutative diagrams.
\xyoption{2cell}
\UseAllTwocells

% To put source file link in headers.
% Change "template.tex" to "this_filename.tex"
% \usepackage{fancyhdr}
% \pagestyle{fancy}
% \lhead{}
% \chead{}
% \rhead{Source file: \url{template.tex}}
% \lfoot{}
% \cfoot{\thepage}
% \rfoot{}
% \renewcommand{\headrulewidth}{0pt}
% \renewcommand{\footrulewidth}{0pt}
% \renewcommand{\headheight}{12pt}

\usepackage{multicol}

% For cross-file-references
\usepackage{xr-hyper}

% Package for hypertext links:
\usepackage{hyperref}

% For any local file, say "hello.tex" you want to link to please
% use \externaldocument[hello-]{hello}
\externaldocument[introduction-]{introduction}
\externaldocument[conventions-]{conventions}
\externaldocument[sets-]{sets}
\externaldocument[categories-]{categories}
\externaldocument[topology-]{topology}
\externaldocument[sheaves-]{sheaves}
\externaldocument[sites-]{sites}
\externaldocument[stacks-]{stacks}
\externaldocument[fields-]{fields}
\externaldocument[algebra-]{algebra}
\externaldocument[brauer-]{brauer}
\externaldocument[homology-]{homology}
\externaldocument[derived-]{derived}
\externaldocument[simplicial-]{simplicial}
\externaldocument[more-algebra-]{more-algebra}
\externaldocument[smoothing-]{smoothing}
\externaldocument[modules-]{modules}
\externaldocument[sites-modules-]{sites-modules}
\externaldocument[injectives-]{injectives}
\externaldocument[cohomology-]{cohomology}
\externaldocument[sites-cohomology-]{sites-cohomology}
\externaldocument[dga-]{dga}
\externaldocument[dpa-]{dpa}
\externaldocument[hypercovering-]{hypercovering}
\externaldocument[schemes-]{schemes}
\externaldocument[constructions-]{constructions}
\externaldocument[properties-]{properties}
\externaldocument[morphisms-]{morphisms}
\externaldocument[coherent-]{coherent}
\externaldocument[divisors-]{divisors}
\externaldocument[limits-]{limits}
\externaldocument[varieties-]{varieties}
\externaldocument[topologies-]{topologies}
\externaldocument[descent-]{descent}
\externaldocument[perfect-]{perfect}
\externaldocument[more-morphisms-]{more-morphisms}
\externaldocument[flat-]{flat}
\externaldocument[groupoids-]{groupoids}
\externaldocument[more-groupoids-]{more-groupoids}
\externaldocument[etale-]{etale}
\externaldocument[chow-]{chow}
\externaldocument[intersection-]{intersection}
\externaldocument[pic-]{pic}
\externaldocument[adequate-]{adequate}
\externaldocument[dualizing-]{dualizing}
\externaldocument[duality-]{duality}
\externaldocument[discriminant-]{discriminant}
\externaldocument[local-cohomology-]{local-cohomology}
\externaldocument[curves-]{curves}
\externaldocument[resolve-]{resolve}
\externaldocument[models-]{models}
\externaldocument[pione-]{pione}
\externaldocument[etale-cohomology-]{etale-cohomology}
\externaldocument[proetale-]{proetale}
\externaldocument[crystalline-]{crystalline}
\externaldocument[spaces-]{spaces}
\externaldocument[spaces-properties-]{spaces-properties}
\externaldocument[spaces-morphisms-]{spaces-morphisms}
\externaldocument[decent-spaces-]{decent-spaces}
\externaldocument[spaces-cohomology-]{spaces-cohomology}
\externaldocument[spaces-limits-]{spaces-limits}
\externaldocument[spaces-divisors-]{spaces-divisors}
\externaldocument[spaces-over-fields-]{spaces-over-fields}
\externaldocument[spaces-topologies-]{spaces-topologies}
\externaldocument[spaces-descent-]{spaces-descent}
\externaldocument[spaces-perfect-]{spaces-perfect}
\externaldocument[spaces-more-morphisms-]{spaces-more-morphisms}
\externaldocument[spaces-flat-]{spaces-flat}
\externaldocument[spaces-groupoids-]{spaces-groupoids}
\externaldocument[spaces-more-groupoids-]{spaces-more-groupoids}
\externaldocument[bootstrap-]{bootstrap}
\externaldocument[spaces-pushouts-]{spaces-pushouts}
\externaldocument[groupoids-quotients-]{groupoids-quotients}
\externaldocument[spaces-more-cohomology-]{spaces-more-cohomology}
\externaldocument[spaces-simplicial-]{spaces-simplicial}
\externaldocument[formal-spaces-]{formal-spaces}
\externaldocument[restricted-]{restricted}
\externaldocument[spaces-resolve-]{spaces-resolve}
\externaldocument[formal-defos-]{formal-defos}
\externaldocument[defos-]{defos}
\externaldocument[cotangent-]{cotangent}
\externaldocument[examples-defos-]{examples-defos}
\externaldocument[algebraic-]{algebraic}
\externaldocument[examples-stacks-]{examples-stacks}
\externaldocument[stacks-sheaves-]{stacks-sheaves}
\externaldocument[criteria-]{criteria}
\externaldocument[artin-]{artin}
\externaldocument[quot-]{quot}
\externaldocument[stacks-properties-]{stacks-properties}
\externaldocument[stacks-morphisms-]{stacks-morphisms}
\externaldocument[stacks-limits-]{stacks-limits}
\externaldocument[stacks-cohomology-]{stacks-cohomology}
\externaldocument[stacks-perfect-]{stacks-perfect}
\externaldocument[stacks-introduction-]{stacks-introduction}
\externaldocument[stacks-more-morphisms-]{stacks-more-morphisms}
\externaldocument[stacks-geometry-]{stacks-geometry}
\externaldocument[moduli-]{moduli}
\externaldocument[moduli-curves-]{moduli-curves}
\externaldocument[examples-]{examples}
\externaldocument[exercises-]{exercises}
\externaldocument[guide-]{guide}
\externaldocument[desirables-]{desirables}
\externaldocument[coding-]{coding}
\externaldocument[obsolete-]{obsolete}
\externaldocument[fdl-]{fdl}
\externaldocument[index-]{index}

% Theorem environments.
%
\theoremstyle{plain}
\newtheorem{theorem}[subsection]{Theorem}
\newtheorem{proposition}[subsection]{Proposition}
\newtheorem{lemma}[subsection]{Lemma}

\theoremstyle{definition}
\newtheorem{definition}[subsection]{Definition}
\newtheorem{example}[subsection]{Example}
\newtheorem{exercise}[subsection]{Exercise}
\newtheorem{situation}[subsection]{Situation}

\theoremstyle{remark}
\newtheorem{remark}[subsection]{Remark}
\newtheorem{remarks}[subsection]{Remarks}

\numberwithin{equation}{subsection}

% Macros
%
\def\lim{\mathop{\rm lim}\nolimits}
\def\colim{\mathop{\rm colim}\nolimits}
\def\Spec{\mathop{\rm Spec}}
\def\Hom{\mathop{\rm Hom}\nolimits}
\def\Ext{\mathop{\rm Ext}\nolimits}
\def\SheafHom{\mathop{\mathcal{H}\!{\it om}}\nolimits}
\def\SheafExt{\mathop{\mathcal{E}\!{\it xt}}\nolimits}
\def\Sch{\textit{Sch}}
\def\Mor{\mathop{\rm Mor}\nolimits}
\def\Ob{\mathop{\rm Ob}\nolimits}
\def\Sh{\mathop{\textit{Sh}}\nolimits}
\def\NL{\mathop{N\!L}\nolimits}
\def\proetale{{pro\text{-}\acute{e}tale}}
\def\etale{{\acute{e}tale}}
\def\QCoh{\textit{QCoh}}
\def\Ker{\mathop{\rm Ker}}
\def\Im{\mathop{\rm Im}}
\def\Coker{\mathop{\rm Coker}}
\def\Coim{\mathop{\rm Coim}}

%
% Macros for moduli stacks/spaces
%
\def\QCohstack{\mathcal{QC}\!{\it oh}}
\def\Cohstack{\mathcal{C}\!{\it oh}}
\def\Spacesstack{\mathcal{S}\!{\it paces}}
\def\Quotfunctor{{\rm Quot}}
\def\Hilbfunctor{{\rm Hilb}}
\def\Curvesstack{\mathcal{C}\!{\it urves}}
\def\Polarizedstack{\mathcal{P}\!{\it olarized}}
\def\Complexesstack{\mathcal{C}\!{\it omplexes}}
% \Pic is the operator that assigns to X its picard group, usage \Pic(X)
% \Picardstack_{X/B} denotes the Picard stack of X over B
% \Picardfunctor_{X/B} denotes the Picard functor of X over B
\def\Pic{\mathop{\rm Pic}\nolimits}
\def\Picardstack{\mathcal{P}\!{\it ic}}
\def\Picardfunctor{{\rm Pic}}
\def\Deformationcategory{\mathcal{D}\!{\it ef}}


% OK, start here.
%
\begin{document}

\title{Cohomology on Sites}


\maketitle

\phantomsection
\label{section-phantom}

\tableofcontents

\section{Introduction}
\label{section-introduction}

\noindent
In this document we work out some topics on cohomology of sheaves.
We work out what happens for sheaves on sites,
allthough often we will simply duplicate the discussion,
the constructions, and the proofs from the topological
case in the case.
Basic references are \cite{SGA4}, \cite{Godement} and \cite{Iversen}.

\section{Topics}
\label{section-topics}

\noindent
Here are some topics that should be discussed in this chapter,
and have not yet been written.
\begin{enumerate}
\item Cohomology of a sheaf of modules on a site is the same
as the cohomology of the underlying abelian sheaf.
\item Hypercohomology on a site.
\item Ext-groups.
\item Ext sheaves.
\item Tor functors.
\item Higher direct images for a morphism of sites.
\item Derived pullback for morphisms between ringed sites.
\item Cup-product.
\item Group cohomology.
\item Comparison of group cohomology and cohomology on $\mathcal{T}_G$.
\item Cech cohomology on sites.
\item Cech to cohomology spectral sequence on sites.
\item Leray Spectral sequence for a morphism between ringed sites.
\item Etc, etc, etc.
\end{enumerate}



\section{Cohomology of sheaves}
\label{section-cohomology-sheaves}

\noindent
Let $\mathcal{C}$ be a site, see
Sites, Definition \ref{sites-definition-site}.
Let $\mathcal{F}$ be a abelian sheaf on $\mathcal{C}$.
We know that the category of abelian sheaves on $\mathcal{C}$
has enough injectives, see
Injectives, Theorem \ref{injectives-theorem-sheaves-injectives}.
Hence we can choose an injective resolution
$\mathcal{F}[0] \to \mathcal{I}^\bullet$.
For any object $X$ of the site $\mathcal{C}$ we define
$$
H^i(X, \mathcal{F}) = H^i(\Gamma(X, \mathcal{I}^\bullet))
$$
to be the {\it $i$th cohomology group of the abelian sheaf
$\mathcal{F}$ over the object $X$}.

\medskip\noindent
It sometimes happens that
the site $\mathcal{C}$ does not have a final object. In this
case we define the {\it global sections} of a presheaf
of sets $\mathcal{F}$ over $\mathcal{C}$ to be the set
$$
\Gamma(\mathcal{C}, \mathcal{F}) =
\text{Mor}_{\textit{PSh}(\mathcal{C})}(e, \mathcal{F})
$$
where $e$ is a final object in the category of presheaves on $\mathcal{C}$.
In this case, given an abelian sheaf $\mathcal{F}$ on $\mathcal{C}$,
we define the {\it $i$th cohomology group of $\mathcal{F}$
on $\mathcal{C}$} as follows
$$
H^i(\mathcal{C}, \mathcal{F}) = H^i(\Gamma(\mathcal{C}, \mathcal{I}^\bullet))
$$
in other words, it is the $i$th right derived functor of the
global sections functor.

\medskip\noindent
We want to generalize this discussion just a little bit.
Namely, suppose that $(\mathcal{C}, \mathcal{O})$
is a ringed site. (The case of abelian sheaves will be included
by thinking about the case where $\mathcal{O} = \underline{\mathbf{Z}}$
is the constant sheaf associated to $\mathbf{Z}$.)
By Homology, Remark \ref{homology-remark-big-abelian-category}
there exists a localization functor
$$
j = j_{(\mathcal{C}, \mathcal{O})} :
K^{+}(\textit{Mod}(\mathcal{O}))
\longrightarrow
D^{+}(\textit{Mod}(\mathcal{O})).
$$
To ease notation we are going to write
$$
K^{+}(\mathcal{O}) = K^{+}(\textit{Mod}(\mathcal{O}))
\quad
\text{and}
\quad
D^{+}(\mathcal{O}) = D^{+}(\textit{Mod}(\mathcal{O})).
$$
Given a ring $R$ we similarly write
$$
K^{+}(R) = K^{+}(\text{Mod}(R))
\quad
\text{and}
\quad
D^{+}(R) = D^{+}(\text{Mod}(R))
$$
and we denote $j_R : K^{+}(R) \to D^{+}(R)$ the localization functor.
For any left exact functor $F : \textit{Mod}(\mathcal{O}) \to \mathcal{B}$
into any abelian category $\mathcal{B}$ we will denote $RF$ the
right derived functor constructed in
Homology, Section \ref{homology-section-right-derived-functor}
using the localization functor $j$ just described.
Note that we may think of $RF$ as defined on $\textit{Mod}(\mathcal{O})$,
$\text{Comp}^{+}(\textit{Mod}(\mathcal{O}))$,
$K^{+}(\mathcal{O})$, or $D^{+}(\mathcal{O})$
depending on the situation. In addition, whenever possible
we actually think of $RF$ as a functor into $D^{+}(\mathcal{B})$, see
Homology, Remark \ref{homology-remark-right-derived-functor}.

\medskip\noindent
In particular, we have
$
\Gamma(\mathcal{C}, -) :
\textit{Mod}(\mathcal{O})
\longrightarrow
\text{Mod}(\Gamma(\mathcal{C}, \mathcal{O})
$
which gives rise to
$$
R\Gamma(\mathcal{C}, -) :
D^{+}(\mathcal{O})
\longrightarrow
D^{+}(\Gamma(\mathcal{C}, \mathcal{O}))
$$
and $H^i(\mathcal{C}, -) = R^i\Gamma(\mathcal{C}, -)$.
For any object $X$ of the site $\mathcal{C}$ we have
$
\Gamma(X, -) :
\textit{Mod}(\mathcal{O})
\longrightarrow
\text{Mod}(\mathcal{O}(X))
$
which gives rise to
$$
R\Gamma(X, -) :
D^{+}(\mathcal{O})
\longrightarrow
D^{+}(\mathcal{O}(X))
$$
and $H^i(X, -) = R^i\Gamma(X, -)$.







\section{First cohomology and torsors}
\label{section-h1-torsors}

\begin{definition}
\label{definition-torsor}
Let $\mathcal{C}$ be a site.
Let $\mathcal{G}$ be a sheaf of (possibly non-commutative)
groups on $\mathcal{C}$.
A {\it torsor}, or more precisely a {\it $\mathcal{G}$-torsor}, is a sheaf
of sets $\mathcal{F}$ on $\mathcal{C}$ endowed with an action
$\mathcal{G} \times \mathcal{F} \to \mathcal{F}$ such that
\begin{enumerate}
\item whenever $\mathcal{F}(U)$ is nonempty the action
$\mathcal{G}(U) \times \mathcal{F}(U) \to \mathcal{F}(U)$
is simply transitive, and
\item for every $U \in \text{Ob}(\mathcal{C})$
there exists a covering $\{U_i \to U\}_{i \in I}$ of $U$
such that $\mathcal{F}(U_i)$ is nonempty for all $i \in I$.
\end{enumerate}
A {\it morphism of $\mathcal{G}$-torsors} $\mathcal{F} \to \mathcal{F}'$
is simply a morphism of sheaves of sets compatible with the
$\mathcal{G}$-actions. The {\it trivial $\mathcal{G}$-torsor}
is the sheaf $\mathcal{G}$ endowed with the obvious left
$\mathcal{G}$-action.
\end{definition}

\noindent
It is clear that a morphism of torsors is automatically an isomorphism.

\begin{lemma}
\label{lemma-trivial-torsor}
Let $\mathcal{C}$ be a site.
Let $\mathcal{G}$ be a sheaf of (possibly non-commutative)
groups on $\mathcal{C}$.
A $\mathcal{G}$-torsor $\mathcal{F}$ is trivial if and only if
$\Gamma(\mathcal{C}, \mathcal{F}) \not = \emptyset$.
\end{lemma}

\begin{proof}
Omitted.
\end{proof}

\begin{lemma}
\label{lemma-torsors-h1}
Let $\mathcal{C}$ be a site.
Let $\mathcal{H}$ be an abelian sheaf on $\mathcal{C}$.
There is a canonical bijection between the set of isomorphism
classes of $\mathcal{H}$-torsors and $H^1(\mathcal{C}, \mathcal{H})$.
\end{lemma}

\begin{proof}
Let $\mathcal{F}$ be a $\mathcal{H}$-torsor.
Consider the free abelian sheaf $\mathbf{Z}[\mathcal{F}]$
on $\mathcal{F}$. It is the sheafification of the rule
which associates to $U \in \text{Ob}(\mathcal{C})$ the collection of finite
formal sums $\sum n_i[s_i]$ with $n_i \in \mathbf{Z}$
and $s_i \in \mathcal{F}(U)$. There is a natural map
$$
\sigma : \mathbf{Z}[\mathcal{F}] \longrightarrow \underline{\mathbf{Z}}
$$
which to a local section $\sum n_i[s_i]$ associates $\sum n_i$.
The kernel of $\sigma$ is generated by sections of the form
$[s] - [s']$. There is a canonical map
$a : \text{Ker}(\sigma) \to \mathcal{H}$
which maps $[s] - [s'] \mapsto h$ where $h$ is the local section of
$\mathcal{H}$ such that $h \cdot s = s'$. Consider the push out diagram
$$
\xymatrix{
0 \ar[r] &
\text{Ker}(\sigma) \ar[r] \ar[d]^a &
\mathbf{Z}[\mathcal{F}] \ar[r] \ar[d] &
\underline{\mathbf{Z}} \ar[r] \ar[d] &
0 \\
0 \ar[r] &
\mathcal{H} \ar[r] &
\mathcal{E} \ar[r] & 
\underline{\mathbf{Z}} \ar[r] &
0
}
$$
Here $\mathcal{E}$ is the extension obtained by push out.
From the long exact cohomology sequence associated to the lower
short exact sequence we obtain an element
$\xi = \xi_{\mathcal{F}} \in H^1(\mathcal{C}, \mathcal{H})$
by applying the boundary operator to
$1 \in H^0(\mathcal{C}, \underline{\mathbf{Z}})$.

\medskip\noindent
Conversely, given $\xi \in H^1(\mathcal{C}, \mathcal{H})$ we can associate to
$\xi$ a torsor as follows. Choose an embedding $\mathcal{H} \to \mathcal{I}$
of $\mathcal{H}$ into an injective abelian sheaf $\mathcal{I}$. We set
$\mathcal{Q} = \mathcal{I}/\mathcal{H}$ so that we have a short exact
sequence
$$
\xymatrix{
0 \ar[r] &
\mathcal{H} \ar[r] &
\mathcal{I} \ar[r] & 
\mathcal{Q} \ar[r] &
0
}
$$
The element $\xi$ is the image of a global section
$q \in H^0(\mathcal{C}, \mathcal{Q})$
because $H^1(\mathcal{C}, \mathcal{I}) = 0$ (see
Homology, Lemma \ref{homology-lemma-higher-derived-functors}).
Let $\mathcal{F} \subset \mathcal{I}$ be the subsheaf (of sets) of sections
that map to $q$ in the sheaf $\mathcal{Q}$. It is easy to verify that
$\mathcal{F}$ is a $\mathcal{H}$-torsor.

\medskip\noindent
We omit the verification that the two constructions given
above are mutually inverse.
\end{proof}












\section{Locality of cohomology}
\label{section-locality}

\noindent
Let $(\mathcal{C}, \mathcal{O})$ be a ringed site.
Let $\mathcal{F}$ be a sheaf of $\mathcal{O}$-modules.
Let $\varphi : U \to V$ be a morphism of $\mathcal{O}$.
Then there is a canonical {\it restriction mapping}
\begin{equation}
\label{equation-restriction-mapping}
H^n(V, \mathcal{F})
\longrightarrow
H^n(U, \mathcal{F}), \quad
\xi \longmapsto \xi|_U
\end{equation}
functorial in $\mathcal{F}$. Namely, choose any injective
resolution $\mathcal{F} \to \mathcal{I}^\bullet$. The restriction
mappings of the sheaves $\mathcal{I}^p$ give a morphism of complexes
$$
\Gamma(V, \mathcal{I}^\bullet)
\longrightarrow
\Gamma(U, \mathcal{I}^\bullet)
$$
The LHS is a complex representing $R\Gamma(V, \mathcal{F})$
and the RHS is a complex representing $R\Gamma(U, \mathcal{F})$.
We get the map on cohomology groups by applying the functor $H^n$.
As indicated we will use the notation $\xi \mapsto \xi|_U$ to denote this map.
Thus the rule $U \mapsto H^n(U, \mathcal{F})$ is a presheaf of
$\mathcal{O}$-modules. This presheaf is customarily denoted
$\underline{H}^n(\mathcal{F})$. We will give another interpretation 
of this presheaf in (insert future reference here).

\medskip\noindent
The following lemma says that it is possible to kill higher cohomology
classes by going to a covering.

\begin{lemma}
\label{lemma-kill-cohomology-class-on-covering}
Let $(\mathcal{C}, \mathcal{O})$ be a ringed site.
Let $\mathcal{F}$ be a sheaf of $\mathcal{O}$-modules.
Let $U$ be an object of $\mathcal{C}$.
Let $n > 0$ and let $\xi \in H^n(U, \mathcal{F})$.
Then there exists a covering $\{U_i \to U\}$ of $\mathcal{C}$
such that $\xi|_{U_i} = 0$ for all $i \in I$.
\end{lemma}

\begin{proof}
Let $\mathcal{F} \to \mathcal{I}^\bullet$ be an injective resolution.
Then
$$
H^n(U, \mathcal{F}) = 
\frac{\text{Ker}(\mathcal{I}^n(U) \to \mathcal{I}^{n + 1}(U))}
{\text{Im}(\mathcal{I}^{n - 1}(U) \to \mathcal{I}^n(U))}.
$$
Pick an element $\tilde \xi \in \mathcal{I}^n(U)$ representing the
cohomology class in the presentation above. Since $\mathcal{I}^\bullet$
is an injective resolution of $\mathcal{F}$ and $n > 0$ we see that
the complex $\mathcal{I}^\bullet$ is exact in degree $n$. Hence
$\text{Im}(\mathcal{I}^{n - 1} \to \mathcal{I}^n) =
\text{Ker}(\mathcal{I}^n \to \mathcal{I}^{n + 1})$ as sheaves.
Since $\tilde \xi$ is a section of the kernel sheaf over $U$
we conclude there exists a covering $\{U_i \to U\}$ of the site
such that $\tilde \xi|_{U_i}$ is the image under $d$ of a section
$\xi_i \in \mathcal{I}^{n - 1}(U_i)$. By our definition of the
restriction $\xi|_{U_i}$ as corresponding to the class of
$\tilde \xi|_{U_i}$ we conclude.
\end{proof}









\section{Other chapters}

\begin{multicols}{2}
\begin{enumerate}
\item \hyperref[introduction-section-phantom]{Introduction}
\item \hyperref[conventions-section-phantom]{Conventions}
\item \hyperref[sets-section-phantom]{Set Theory}
\item \hyperref[categories-section-phantom]{Categories}
\item \hyperref[topology-section-phantom]{Topology}
\item \hyperref[sheaves-section-phantom]{Sheaves on Spaces}
\item \hyperref[algebra-section-phantom]{Commutative Algebra}
\item \hyperref[sites-section-phantom]{Sites and Sheaves}
\item \hyperref[homology-section-phantom]{Homological Algebra}
\item \hyperref[derived-section-phantom]{Derived Categories}
\item \hyperref[more-algebra-section-phantom]{More Algebra}
\item \hyperref[simplicial-section-phantom]{Simplicial Methods}
\item \hyperref[modules-section-phantom]{Sheaves of Modules}
\item \hyperref[sites-modules-section-phantom]{Modules on Sites}
\item \hyperref[injectives-section-phantom]{Injectives}
\item \hyperref[cohomology-section-phantom]{Cohomology of Sheaves}
\item \hyperref[sites-cohomology-section-phantom]{Cohomology on Sites}
\item \hyperref[hypercovering-section-phantom]{Hypercoverings}
\item \hyperref[schemes-section-phantom]{Schemes}
\item \hyperref[constructions-section-phantom]{Constructions of Schemes}
\item \hyperref[properties-section-phantom]{Properties of Schemes}
\item \hyperref[morphisms-section-phantom]{Morphisms of Schemes}
\item \hyperref[coherent-section-phantom]{Coherent Cohomology}
\item \hyperref[divisors-section-phantom]{Divisors}
\item \hyperref[limits-section-phantom]{Limits of Schemes}
\item \hyperref[varieties-section-phantom]{Varieties}
\item \hyperref[chow-section-phantom]{Chow Homology}
\item \hyperref[topologies-section-phantom]{Topologies on Schemes}
\item \hyperref[descent-section-phantom]{Descent}
\item \hyperref[more-morphisms-section-phantom]{More on Morphisms}
\item \hyperref[flat-section-phantom]{More on Flatness}
\item \hyperref[groupoids-section-phantom]{Groupoid Schemes}
\item \hyperref[more-groupoids-section-phantom]{More on Groupoid Schemes}
\item \hyperref[etale-section-phantom]{\'Etale Morphisms of Schemes}
\item \hyperref[etale-cohomology-section-phantom]{\'Etale Cohomology}
\item \hyperref[spaces-section-phantom]{Algebraic Spaces}
\item \hyperref[spaces-properties-section-phantom]{Properties of Algebraic Spaces}
\item \hyperref[spaces-morphisms-section-phantom]{Morphisms of Algebraic Spaces}
\item \hyperref[spaces-topologies-section-phantom]{Topologies on Algebraic Spaces}
\item \hyperref[spaces-descent-section-phantom]{Descent and Algebraic Spaces}
\item \hyperref[spaces-more-morphisms-section-phantom]{More on Morphisms of Spaces}
\item \hyperref[quot-section-phantom]{Quot and Hilbert Spaces}
\item \hyperref[stacks-section-phantom]{Stacks}
\item \hyperref[spaces-groupoids-section-phantom]{Groupoids in Algebraic Spaces}
\item \hyperref[spaces-more-groupoids-section-phantom]{More on Groupoids in Spaces}
\item \hyperref[bootstrap-section-phantom]{Bootstrap}
\item \hyperref[examples-stacks-section-phantom]{Examples of Stacks}
\item \hyperref[groupoids-quotients-section-phantom]{Quotients of Groupoids}
\item \hyperref[algebraic-section-phantom]{Algebraic Stacks}
\item \hyperref[criteria-section-phantom]{Criteria for Representability}
\item \hyperref[stacks-properties-section-phantom]{Properties of Algebraic Stacks}
\item \hyperref[stacks-morphisms-section-phantom]{Morphisms of Algebraic Stacks}
\item \hyperref[examples-section-phantom]{Examples}
\item \hyperref[exercises-section-phantom]{Exercises}
\item \hyperref[guide-section-phantom]{Guide to Literature}
\item \hyperref[desirables-section-phantom]{Desirables}
\item \hyperref[coding-section-phantom]{Coding Style}
\item \hyperref[fdl-section-phantom]{GNU Free Documentation License}
\item \hyperref[index-section-phantom]{Auto Generated Index}
\end{enumerate}
\end{multicols}


\bibliography{my}
\bibliographystyle{alpha}

\end{document}
