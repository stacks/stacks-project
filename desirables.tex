\IfFileExists{stacks-project.cls}{%
\documentclass{stacks-project}
}{%
\documentclass{amsart}
}

% The following AMS packages are automatically loaded with
% the amsart documentclass:
%\usepackage{amsmath}
%\usepackage{amssymb}
%\usepackage{amsthm}

% For dealing with references we use the comment environment
\usepackage{verbatim}
\newenvironment{reference}{\comment}{\endcomment}
%\newenvironment{reference}{}{}
\newenvironment{slogan}{\comment}{\endcomment}
\newenvironment{history}{\comment}{\endcomment}

% For commutative diagrams you can use
% \usepackage{amscd}
\usepackage[all]{xy}

% We use 2cell for 2-commutative diagrams.
\xyoption{2cell}
\UseAllTwocells

% To put source file link in headers.
% Change "template.tex" to "this_filename.tex"
% \usepackage{fancyhdr}
% \pagestyle{fancy}
% \lhead{}
% \chead{}
% \rhead{Source file: \url{template.tex}}
% \lfoot{}
% \cfoot{\thepage}
% \rfoot{}
% \renewcommand{\headrulewidth}{0pt}
% \renewcommand{\footrulewidth}{0pt}
% \renewcommand{\headheight}{12pt}

\usepackage{multicol}

% For cross-file-references
\usepackage{xr-hyper}

% Package for hypertext links:
\usepackage{hyperref}

% For any local file, say "hello.tex" you want to link to please
% use \externaldocument[hello-]{hello}
\externaldocument[introduction-]{introduction}
\externaldocument[conventions-]{conventions}
\externaldocument[sets-]{sets}
\externaldocument[categories-]{categories}
\externaldocument[topology-]{topology}
\externaldocument[sheaves-]{sheaves}
\externaldocument[sites-]{sites}
\externaldocument[stacks-]{stacks}
\externaldocument[fields-]{fields}
\externaldocument[algebra-]{algebra}
\externaldocument[brauer-]{brauer}
\externaldocument[homology-]{homology}
\externaldocument[derived-]{derived}
\externaldocument[simplicial-]{simplicial}
\externaldocument[more-algebra-]{more-algebra}
\externaldocument[smoothing-]{smoothing}
\externaldocument[modules-]{modules}
\externaldocument[sites-modules-]{sites-modules}
\externaldocument[injectives-]{injectives}
\externaldocument[cohomology-]{cohomology}
\externaldocument[sites-cohomology-]{sites-cohomology}
\externaldocument[dga-]{dga}
\externaldocument[dpa-]{dpa}
\externaldocument[hypercovering-]{hypercovering}
\externaldocument[schemes-]{schemes}
\externaldocument[constructions-]{constructions}
\externaldocument[properties-]{properties}
\externaldocument[morphisms-]{morphisms}
\externaldocument[coherent-]{coherent}
\externaldocument[divisors-]{divisors}
\externaldocument[limits-]{limits}
\externaldocument[varieties-]{varieties}
\externaldocument[topologies-]{topologies}
\externaldocument[descent-]{descent}
\externaldocument[perfect-]{perfect}
\externaldocument[more-morphisms-]{more-morphisms}
\externaldocument[flat-]{flat}
\externaldocument[groupoids-]{groupoids}
\externaldocument[more-groupoids-]{more-groupoids}
\externaldocument[etale-]{etale}
\externaldocument[chow-]{chow}
\externaldocument[intersection-]{intersection}
\externaldocument[pic-]{pic}
\externaldocument[adequate-]{adequate}
\externaldocument[dualizing-]{dualizing}
\externaldocument[duality-]{duality}
\externaldocument[discriminant-]{discriminant}
\externaldocument[local-cohomology-]{local-cohomology}
\externaldocument[curves-]{curves}
\externaldocument[resolve-]{resolve}
\externaldocument[models-]{models}
\externaldocument[pione-]{pione}
\externaldocument[etale-cohomology-]{etale-cohomology}
\externaldocument[proetale-]{proetale}
\externaldocument[crystalline-]{crystalline}
\externaldocument[spaces-]{spaces}
\externaldocument[spaces-properties-]{spaces-properties}
\externaldocument[spaces-morphisms-]{spaces-morphisms}
\externaldocument[decent-spaces-]{decent-spaces}
\externaldocument[spaces-cohomology-]{spaces-cohomology}
\externaldocument[spaces-limits-]{spaces-limits}
\externaldocument[spaces-divisors-]{spaces-divisors}
\externaldocument[spaces-over-fields-]{spaces-over-fields}
\externaldocument[spaces-topologies-]{spaces-topologies}
\externaldocument[spaces-descent-]{spaces-descent}
\externaldocument[spaces-perfect-]{spaces-perfect}
\externaldocument[spaces-more-morphisms-]{spaces-more-morphisms}
\externaldocument[spaces-flat-]{spaces-flat}
\externaldocument[spaces-groupoids-]{spaces-groupoids}
\externaldocument[spaces-more-groupoids-]{spaces-more-groupoids}
\externaldocument[bootstrap-]{bootstrap}
\externaldocument[spaces-pushouts-]{spaces-pushouts}
\externaldocument[groupoids-quotients-]{groupoids-quotients}
\externaldocument[spaces-more-cohomology-]{spaces-more-cohomology}
\externaldocument[spaces-simplicial-]{spaces-simplicial}
\externaldocument[formal-spaces-]{formal-spaces}
\externaldocument[restricted-]{restricted}
\externaldocument[spaces-resolve-]{spaces-resolve}
\externaldocument[formal-defos-]{formal-defos}
\externaldocument[defos-]{defos}
\externaldocument[cotangent-]{cotangent}
\externaldocument[examples-defos-]{examples-defos}
\externaldocument[algebraic-]{algebraic}
\externaldocument[examples-stacks-]{examples-stacks}
\externaldocument[stacks-sheaves-]{stacks-sheaves}
\externaldocument[criteria-]{criteria}
\externaldocument[artin-]{artin}
\externaldocument[quot-]{quot}
\externaldocument[stacks-properties-]{stacks-properties}
\externaldocument[stacks-morphisms-]{stacks-morphisms}
\externaldocument[stacks-limits-]{stacks-limits}
\externaldocument[stacks-cohomology-]{stacks-cohomology}
\externaldocument[stacks-perfect-]{stacks-perfect}
\externaldocument[stacks-introduction-]{stacks-introduction}
\externaldocument[stacks-more-morphisms-]{stacks-more-morphisms}
\externaldocument[stacks-geometry-]{stacks-geometry}
\externaldocument[moduli-]{moduli}
\externaldocument[moduli-curves-]{moduli-curves}
\externaldocument[examples-]{examples}
\externaldocument[exercises-]{exercises}
\externaldocument[guide-]{guide}
\externaldocument[desirables-]{desirables}
\externaldocument[coding-]{coding}
\externaldocument[obsolete-]{obsolete}
\externaldocument[fdl-]{fdl}
\externaldocument[index-]{index}

% Theorem environments.
%
\theoremstyle{plain}
\newtheorem{theorem}[subsection]{Theorem}
\newtheorem{proposition}[subsection]{Proposition}
\newtheorem{lemma}[subsection]{Lemma}

\theoremstyle{definition}
\newtheorem{definition}[subsection]{Definition}
\newtheorem{example}[subsection]{Example}
\newtheorem{exercise}[subsection]{Exercise}
\newtheorem{situation}[subsection]{Situation}

\theoremstyle{remark}
\newtheorem{remark}[subsection]{Remark}
\newtheorem{remarks}[subsection]{Remarks}

\numberwithin{equation}{subsection}

% Macros
%
\def\lim{\mathop{\rm lim}\nolimits}
\def\colim{\mathop{\rm colim}\nolimits}
\def\Spec{\mathop{\rm Spec}}
\def\Hom{\mathop{\rm Hom}\nolimits}
\def\Ext{\mathop{\rm Ext}\nolimits}
\def\SheafHom{\mathop{\mathcal{H}\!{\it om}}\nolimits}
\def\SheafExt{\mathop{\mathcal{E}\!{\it xt}}\nolimits}
\def\Sch{\textit{Sch}}
\def\Mor{\mathop{\rm Mor}\nolimits}
\def\Ob{\mathop{\rm Ob}\nolimits}
\def\Sh{\mathop{\textit{Sh}}\nolimits}
\def\NL{\mathop{N\!L}\nolimits}
\def\proetale{{pro\text{-}\acute{e}tale}}
\def\etale{{\acute{e}tale}}
\def\QCoh{\textit{QCoh}}
\def\Ker{\mathop{\rm Ker}}
\def\Im{\mathop{\rm Im}}
\def\Coker{\mathop{\rm Coker}}
\def\Coim{\mathop{\rm Coim}}

%
% Macros for moduli stacks/spaces
%
\def\QCohstack{\mathcal{QC}\!{\it oh}}
\def\Cohstack{\mathcal{C}\!{\it oh}}
\def\Spacesstack{\mathcal{S}\!{\it paces}}
\def\Quotfunctor{{\rm Quot}}
\def\Hilbfunctor{{\rm Hilb}}
\def\Curvesstack{\mathcal{C}\!{\it urves}}
\def\Polarizedstack{\mathcal{P}\!{\it olarized}}
\def\Complexesstack{\mathcal{C}\!{\it omplexes}}
% \Pic is the operator that assigns to X its picard group, usage \Pic(X)
% \Picardstack_{X/B} denotes the Picard stack of X over B
% \Picardfunctor_{X/B} denotes the Picard functor of X over B
\def\Pic{\mathop{\rm Pic}\nolimits}
\def\Picardstack{\mathcal{P}\!{\it ic}}
\def\Picardfunctor{{\rm Pic}}
\def\Deformationcategory{\mathcal{D}\!{\it ef}}


% OK, start here.
%
\begin{document}

\title{Desirables}

\maketitle

\phantomsection
\label{section-phantom}


\tableofcontents

\section{Introduction}
\label{section-introduction}

\noindent
This is basically just a list of things that we want to put in the stacks
project. As we add material to the project continuously this is always
somewhat behind the current state of the project. In fact, it may have
been a mistake to try and list things we should add, because it seems
impossible to keep it up to date.

\medskip\noindent
Last updated: Thursday, May 9, 2013.


\section{Conventions}
\label{section-conventions}

\noindent
We should have a chapter with a short list of conventions used in the document.
This chapter already exists, see
Conventions, Section \ref{conventions-section-comments},
but a lot more could be added there. Especially useful would be to find
``hidden'' conventions and tacit assumptions and put those there.


\section{Sites and Topoi}
\label{section-sites}

\noindent
We have a chapter on sites and sheaves, see
Sites, Section \ref{sites-section-introduction}.
We have a chapter on ringed sites (and topoi) and modules on them, see
Modules on Sites, Section \ref{sites-modules-section-introduction}.
We have a chapter on cohomology in this setting, see
Cohomology on Sites, Section \ref{sites-cohomology-section-introduction}.
But a lot more could be added, especially in the chapter on cohomology.


\section{Stacks}
\label{section-stacks}

\noindent
We have a chapter on (abstract) stacks, see
Stacks, Section \ref{stacks-section-introduction}.
It would be nice if
\begin{enumerate}
\item improve the discussion on ``stackyfication'',
\item give examples of stackyfication,
\item more examples in general,
\item improve the discussion of gerbes.
\end{enumerate}

\medskip\noindent
Example result which has not been added yet: Given a sheaf of abelian
groups $\mathcal{F}$
over $\mathcal{C}$ the set of equivalence classes of gerbes with ``group''
$\mathcal{F}$ is bijective to $H^2(\mathcal{C}, \mathcal{F})$.


\section{Simplicial methods}
\label{section-simplicial}

\noindent
We have a chapter on simplicial methods, see
Simplicial, Section \ref{simplicial-section-introduction}.
This has to be reviewed and improved. The discussion of
the relationship between simplicial homotopy (also known as
combinatorial homotopy) and Kan complexes should be improved upon.
Moreover, there should be a
chapter on ``simplicial algebraic geometry'', where we discuss
simplicial schemes, and how to think of their geometry, cohomology,
etc. Then this should be tied into the chapter on hypercoverings
to ``explain'' the results of this chapter in the new language.


\section{Cohomology of schemes}
\label{section-schemes-cohomology}

\noindent
There is already a chapter on cohomology of quasi-coherent sheaves, see
Cohomology of Schemes, Section \ref{coherent-section-introduction}.
We also have chapters on \'etale cohomology of schemes,
crystalline cohomology of schemes, derived categories of schemes.
But most of the material is very basic and a lot more could be added here.


\section{Deformation theory \`a la Schlessinger}
\label{section-deformation-schlessinger}

\noindent
We have a chapter on this material, see
Formal Deformation Theory, Section \ref{formal-defos-section-introduction}.
What is needed is worked out examples of the general theory, for example
the case of representations of a fixed abstract group.


\section{Definition of algebraic stacks}
\label{section-definition-algebraic-stacks}

\noindent
An algebraic stack is a stack in groupoids over the category of schemes
with the fppf topoolgy that has a diagonal representable by algebraic
spaces and is the target of a surjective smooth morphism from a scheme.
The notion ``Deligne-Mumford stack'' will be reserved for a stack as in
\cite{DM}. We will reserve the term ``Artin stack'' for
a stack such as in the papers by Artin \cite{ArtinI}, and \cite{ArtinVersal}.
(See also \cite{conrad-dejong}.) In other words, and Artin stack will be an
algebraic stack with some reasonable finiteness and separatedness conditions.


\section{Examples of schemes, algebraic spaces, algebraic stacks}
\label{section-examples-stacks}

\noindent
It really is not that hard to show that $\mathcal{M}_g$ is an algebraic
stack for $g\geq 2$. We should really have a long list of moduli problems
here and prove they are all algebraic stacks. Some of them we can
prove are algebraic using Artin approximation. For example the Kontsevich
moduli space in characteristic $p > 0$.

\medskip\noindent
Here are some items for the list of moduli problems mentioned above.
\begin{enumerate}
\item $\mathcal{M}_g$, i.e., moduli of smooth projective curves of genus $g$,
\item $\overline{\mathcal{M}}_g$, i.e., moduli of stable genus $g$ curves,
\item $\mathcal{A}_g$,
i.e., principally polarized abelian schemes of genus $g$,
\item $\mathcal{M}_{1, 1}$, i.e.,
$1$-pointed smooth projective genus $1$ curves,
\item $\mathcal{M}_{g, n}$, i.e., smooth projective genus $g$-curves
with $n$ pairwise distinct labeled points,
\item $\overline{\mathcal{M}}_{g, n}$, i.e.,
stable $n$-pointed nodal projective genus $g$-curves,
\item $\SheafHom_S(\mathcal{X}, \mathcal{Y})$, moduli of morphisms
(with suitable conditions on the stacks $\mathcal{X}$, $\mathcal{Y}$
and the base scheme $S$),
\item $\textit{Bun}_G(X) = \SheafHom_S(X, BG)$, the stack of $G$-bundles
of the geometric Langlands programme (with suitable conditions on the scheme
$X$, the group scheme $G$, and the base scheme $S$),
\item $\textit{Pic}_{\mathcal{X}/S}$, i.e., the Picard stack associated
to an algebraic stack over a base scheme (or space).
\end{enumerate}

\medskip\noindent
How about the algebraic space you get from the deformation theory of
a general surface in $\mathbf{P}^3$ with a node? (I mean where you deform
it to a general smooth surface in $\mathbf{P}^3$.)
Perhaps we can talk about some small dimensional examples here too.
For example the stack where you have $\mathbf{A}^1$ with a $B(\mathbf{Z}/2)$
sitting at $0$. Bugeyed covers. And so on.


\section{Properties of algebraic stacks}
\label{section-stacks-properties}

\noindent
This is perhaps one of the easier projects to work on, as most of the
basic theory is there now. An interesting project is discussing the
various ways of defining what a proper algebraic stack is.
Of course these things are really properties of morphisms of stacks.
We can define singularities (up to smooth factors) etc. Prove that a
connected normal stack is irreducible, etc.


\section{Lisse \'etale site of an algebraic stack}
\label{section-lisse-etale}

\noindent
This has been introduced in
Cohomology of Stacks, Section \ref{stacks-cohomology-section-lisse-etale}.
An example to show that it is not functorial with respect to $1$-morphisms
of algebraic stacks is discussed in
Examples, Section \ref{examples-section-lisse-etale-not-functorial}.
Of course a lot more could be said about this, but it turns out
to be very useful to prove things using the ``big'' \'etale site
as much as possible.



\section{Things you always wanted to know but were afraid to ask}
\label{section-stacks-fun-lemmas}

\noindent
There are going to be lots of lemmas that you use over and over again
that are useful but aren't really mentioned specifically in the literature,
or it isn't easy to find references for. Bag of tricks.

\medskip\noindent
Example: Given two groupoids in schemes $R\Rightarrow U$ and
$R' \Rightarrow U'$ what does it mean to have a $1$-morphism
$[U/R] \to [U'/R']$ purely in terms of groupoids in schemes.



\section{Quasi-coherent sheaves on stacks}
\label{section-quasi-coherent}

\noindent
These are defined and discussed in the chapter
Cohomology of Stacks, Section \ref{stacks-cohomology-section-introduction}.
Derived categories of modules are discussed in the chapter
Derived Categories of Stacks, Section \ref{stacks-perfect-section-introduction}.
A lot more could be added to these chapters.



\section{Flat and smooth}
\label{section-flat-smooth}

\noindent
Artin's theorem that having a flat surjection from a scheme is a replacement
for the smooth surjective condition. This is now available as
Criteria for Representability, Theorem \ref{criteria-theorem-bootstrap}.


\section{Artin's representability theorem}
\label{section-representability}

\noindent
This is discussed in the chapter
Artin's Axioms, Section \ref{artin-section-introduction}.
We also have an application, see
Quot, Theorem \ref{quot-theorem-coherent-algebraic}.
There should be a lot more applications and the chapter
itself has to be cleaned up as well.


\section{DM stacks are finitely covered by schemes}
\label{section-dm-finite-cover}

\noindent
This all begins with Gabber's lemma I think. Somewhere in Asterisque about
Faltings proof of Mordell?


\section{Martin Olson's paper on properness}
\label{section-proper-parametrization}

\noindent
This proves two notions of proper are the same. We can also discuss Faltings
result that it suffices to use DVR's in certain cases.


\section{Proper pushforward of coherent sheaves}
\label{section-proper-pushforward}

\noindent
No comments yet.


\section{Keel and Mori}
\label{section-keel-mori}

\noindent
See \cite{K-M}. This material has been incorporated throughout the
Stacks project. See for example
More on Groupoids, Section \ref{more-groupoids-section-etale-localize}
and
More on Groupoids in Spaces, Section
\ref{spaces-more-groupoids-section-etale-localize}.


\section{Add more here}
\label{section-add-more}

\noindent
Please.


\section{Other chapters}

\begin{multicols}{2}
\begin{enumerate}
\item \hyperref[introduction-section-phantom]{Introduction}
\item \hyperref[conventions-section-phantom]{Conventions}
\item \hyperref[sets-section-phantom]{Set Theory}
\item \hyperref[categories-section-phantom]{Categories}
\item \hyperref[topology-section-phantom]{Topology}
\item \hyperref[sheaves-section-phantom]{Sheaves on Spaces}
\item \hyperref[algebra-section-phantom]{Commutative Algebra}
\item \hyperref[sites-section-phantom]{Sites and Sheaves}
\item \hyperref[homology-section-phantom]{Homological Algebra}
\item \hyperref[derived-section-phantom]{Derived Categories}
\item \hyperref[more-algebra-section-phantom]{More Algebra}
\item \hyperref[simplicial-section-phantom]{Simplicial Methods}
\item \hyperref[modules-section-phantom]{Sheaves of Modules}
\item \hyperref[sites-modules-section-phantom]{Modules on Sites}
\item \hyperref[injectives-section-phantom]{Injectives}
\item \hyperref[cohomology-section-phantom]{Cohomology of Sheaves}
\item \hyperref[sites-cohomology-section-phantom]{Cohomology on Sites}
\item \hyperref[hypercovering-section-phantom]{Hypercoverings}
\item \hyperref[schemes-section-phantom]{Schemes}
\item \hyperref[constructions-section-phantom]{Constructions of Schemes}
\item \hyperref[properties-section-phantom]{Properties of Schemes}
\item \hyperref[morphisms-section-phantom]{Morphisms of Schemes}
\item \hyperref[coherent-section-phantom]{Coherent Cohomology}
\item \hyperref[divisors-section-phantom]{Divisors}
\item \hyperref[limits-section-phantom]{Limits of Schemes}
\item \hyperref[varieties-section-phantom]{Varieties}
\item \hyperref[chow-section-phantom]{Chow Homology}
\item \hyperref[topologies-section-phantom]{Topologies on Schemes}
\item \hyperref[descent-section-phantom]{Descent}
\item \hyperref[more-morphisms-section-phantom]{More on Morphisms}
\item \hyperref[flat-section-phantom]{More on Flatness}
\item \hyperref[groupoids-section-phantom]{Groupoid Schemes}
\item \hyperref[more-groupoids-section-phantom]{More on Groupoid Schemes}
\item \hyperref[etale-section-phantom]{\'Etale Morphisms of Schemes}
\item \hyperref[etale-cohomology-section-phantom]{\'Etale Cohomology}
\item \hyperref[spaces-section-phantom]{Algebraic Spaces}
\item \hyperref[spaces-properties-section-phantom]{Properties of Algebraic Spaces}
\item \hyperref[spaces-morphisms-section-phantom]{Morphisms of Algebraic Spaces}
\item \hyperref[spaces-topologies-section-phantom]{Topologies on Algebraic Spaces}
\item \hyperref[spaces-descent-section-phantom]{Descent and Algebraic Spaces}
\item \hyperref[spaces-more-morphisms-section-phantom]{More on Morphisms of Spaces}
\item \hyperref[quot-section-phantom]{Quot and Hilbert Spaces}
\item \hyperref[stacks-section-phantom]{Stacks}
\item \hyperref[spaces-groupoids-section-phantom]{Groupoids in Algebraic Spaces}
\item \hyperref[spaces-more-groupoids-section-phantom]{More on Groupoids in Spaces}
\item \hyperref[bootstrap-section-phantom]{Bootstrap}
\item \hyperref[examples-stacks-section-phantom]{Examples of Stacks}
\item \hyperref[groupoids-quotients-section-phantom]{Quotients of Groupoids}
\item \hyperref[algebraic-section-phantom]{Algebraic Stacks}
\item \hyperref[criteria-section-phantom]{Criteria for Representability}
\item \hyperref[stacks-properties-section-phantom]{Properties of Algebraic Stacks}
\item \hyperref[stacks-morphisms-section-phantom]{Morphisms of Algebraic Stacks}
\item \hyperref[examples-section-phantom]{Examples}
\item \hyperref[exercises-section-phantom]{Exercises}
\item \hyperref[guide-section-phantom]{Guide to Literature}
\item \hyperref[desirables-section-phantom]{Desirables}
\item \hyperref[coding-section-phantom]{Coding Style}
\item \hyperref[fdl-section-phantom]{GNU Free Documentation License}
\item \hyperref[index-section-phantom]{Auto Generated Index}
\end{enumerate}
\end{multicols}



\bibliography{my}
\bibliographystyle{amsalpha}

\end{document}
