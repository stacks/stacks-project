\IfFileExists{stacks-project.cls}{%
\documentclass{stacks-project}
}{%
\documentclass{amsart}
}

% The following AMS packages are automatically loaded with
% the amsart documentclass:
%\usepackage{amsmath}
%\usepackage{amssymb}
%\usepackage{amsthm}

% For dealing with references we use the comment environment
\usepackage{verbatim}
\newenvironment{reference}{\comment}{\endcomment}
%\newenvironment{reference}{}{}
\newenvironment{slogan}{\comment}{\endcomment}
\newenvironment{history}{\comment}{\endcomment}

% For commutative diagrams you can use
% \usepackage{amscd}
\usepackage[all]{xy}

% We use 2cell for 2-commutative diagrams.
\xyoption{2cell}
\UseAllTwocells

% To put source file link in headers.
% Change "template.tex" to "this_filename.tex"
% \usepackage{fancyhdr}
% \pagestyle{fancy}
% \lhead{}
% \chead{}
% \rhead{Source file: \url{template.tex}}
% \lfoot{}
% \cfoot{\thepage}
% \rfoot{}
% \renewcommand{\headrulewidth}{0pt}
% \renewcommand{\footrulewidth}{0pt}
% \renewcommand{\headheight}{12pt}

\usepackage{multicol}

% For cross-file-references
\usepackage{xr-hyper}

% Package for hypertext links:
\usepackage{hyperref}

% For any local file, say "hello.tex" you want to link to please
% use \externaldocument[hello-]{hello}
\externaldocument[introduction-]{introduction}
\externaldocument[conventions-]{conventions}
\externaldocument[sets-]{sets}
\externaldocument[categories-]{categories}
\externaldocument[topology-]{topology}
\externaldocument[sheaves-]{sheaves}
\externaldocument[sites-]{sites}
\externaldocument[stacks-]{stacks}
\externaldocument[fields-]{fields}
\externaldocument[algebra-]{algebra}
\externaldocument[brauer-]{brauer}
\externaldocument[homology-]{homology}
\externaldocument[derived-]{derived}
\externaldocument[simplicial-]{simplicial}
\externaldocument[more-algebra-]{more-algebra}
\externaldocument[smoothing-]{smoothing}
\externaldocument[modules-]{modules}
\externaldocument[sites-modules-]{sites-modules}
\externaldocument[injectives-]{injectives}
\externaldocument[cohomology-]{cohomology}
\externaldocument[sites-cohomology-]{sites-cohomology}
\externaldocument[dga-]{dga}
\externaldocument[dpa-]{dpa}
\externaldocument[hypercovering-]{hypercovering}
\externaldocument[schemes-]{schemes}
\externaldocument[constructions-]{constructions}
\externaldocument[properties-]{properties}
\externaldocument[morphisms-]{morphisms}
\externaldocument[coherent-]{coherent}
\externaldocument[divisors-]{divisors}
\externaldocument[limits-]{limits}
\externaldocument[varieties-]{varieties}
\externaldocument[topologies-]{topologies}
\externaldocument[descent-]{descent}
\externaldocument[perfect-]{perfect}
\externaldocument[more-morphisms-]{more-morphisms}
\externaldocument[flat-]{flat}
\externaldocument[groupoids-]{groupoids}
\externaldocument[more-groupoids-]{more-groupoids}
\externaldocument[etale-]{etale}
\externaldocument[chow-]{chow}
\externaldocument[intersection-]{intersection}
\externaldocument[pic-]{pic}
\externaldocument[adequate-]{adequate}
\externaldocument[dualizing-]{dualizing}
\externaldocument[duality-]{duality}
\externaldocument[discriminant-]{discriminant}
\externaldocument[local-cohomology-]{local-cohomology}
\externaldocument[curves-]{curves}
\externaldocument[resolve-]{resolve}
\externaldocument[models-]{models}
\externaldocument[pione-]{pione}
\externaldocument[etale-cohomology-]{etale-cohomology}
\externaldocument[proetale-]{proetale}
\externaldocument[crystalline-]{crystalline}
\externaldocument[spaces-]{spaces}
\externaldocument[spaces-properties-]{spaces-properties}
\externaldocument[spaces-morphisms-]{spaces-morphisms}
\externaldocument[decent-spaces-]{decent-spaces}
\externaldocument[spaces-cohomology-]{spaces-cohomology}
\externaldocument[spaces-limits-]{spaces-limits}
\externaldocument[spaces-divisors-]{spaces-divisors}
\externaldocument[spaces-over-fields-]{spaces-over-fields}
\externaldocument[spaces-topologies-]{spaces-topologies}
\externaldocument[spaces-descent-]{spaces-descent}
\externaldocument[spaces-perfect-]{spaces-perfect}
\externaldocument[spaces-more-morphisms-]{spaces-more-morphisms}
\externaldocument[spaces-flat-]{spaces-flat}
\externaldocument[spaces-groupoids-]{spaces-groupoids}
\externaldocument[spaces-more-groupoids-]{spaces-more-groupoids}
\externaldocument[bootstrap-]{bootstrap}
\externaldocument[spaces-pushouts-]{spaces-pushouts}
\externaldocument[groupoids-quotients-]{groupoids-quotients}
\externaldocument[spaces-more-cohomology-]{spaces-more-cohomology}
\externaldocument[spaces-simplicial-]{spaces-simplicial}
\externaldocument[formal-spaces-]{formal-spaces}
\externaldocument[restricted-]{restricted}
\externaldocument[spaces-resolve-]{spaces-resolve}
\externaldocument[formal-defos-]{formal-defos}
\externaldocument[defos-]{defos}
\externaldocument[cotangent-]{cotangent}
\externaldocument[examples-defos-]{examples-defos}
\externaldocument[algebraic-]{algebraic}
\externaldocument[examples-stacks-]{examples-stacks}
\externaldocument[stacks-sheaves-]{stacks-sheaves}
\externaldocument[criteria-]{criteria}
\externaldocument[artin-]{artin}
\externaldocument[quot-]{quot}
\externaldocument[stacks-properties-]{stacks-properties}
\externaldocument[stacks-morphisms-]{stacks-morphisms}
\externaldocument[stacks-limits-]{stacks-limits}
\externaldocument[stacks-cohomology-]{stacks-cohomology}
\externaldocument[stacks-perfect-]{stacks-perfect}
\externaldocument[stacks-introduction-]{stacks-introduction}
\externaldocument[stacks-more-morphisms-]{stacks-more-morphisms}
\externaldocument[stacks-geometry-]{stacks-geometry}
\externaldocument[moduli-]{moduli}
\externaldocument[moduli-curves-]{moduli-curves}
\externaldocument[examples-]{examples}
\externaldocument[exercises-]{exercises}
\externaldocument[guide-]{guide}
\externaldocument[desirables-]{desirables}
\externaldocument[coding-]{coding}
\externaldocument[obsolete-]{obsolete}
\externaldocument[fdl-]{fdl}
\externaldocument[index-]{index}

% Theorem environments.
%
\theoremstyle{plain}
\newtheorem{theorem}[subsection]{Theorem}
\newtheorem{proposition}[subsection]{Proposition}
\newtheorem{lemma}[subsection]{Lemma}

\theoremstyle{definition}
\newtheorem{definition}[subsection]{Definition}
\newtheorem{example}[subsection]{Example}
\newtheorem{exercise}[subsection]{Exercise}
\newtheorem{situation}[subsection]{Situation}

\theoremstyle{remark}
\newtheorem{remark}[subsection]{Remark}
\newtheorem{remarks}[subsection]{Remarks}

\numberwithin{equation}{subsection}

% Macros
%
\def\lim{\mathop{\rm lim}\nolimits}
\def\colim{\mathop{\rm colim}\nolimits}
\def\Spec{\mathop{\rm Spec}}
\def\Hom{\mathop{\rm Hom}\nolimits}
\def\Ext{\mathop{\rm Ext}\nolimits}
\def\SheafHom{\mathop{\mathcal{H}\!{\it om}}\nolimits}
\def\SheafExt{\mathop{\mathcal{E}\!{\it xt}}\nolimits}
\def\Sch{\textit{Sch}}
\def\Mor{\mathop{\rm Mor}\nolimits}
\def\Ob{\mathop{\rm Ob}\nolimits}
\def\Sh{\mathop{\textit{Sh}}\nolimits}
\def\NL{\mathop{N\!L}\nolimits}
\def\proetale{{pro\text{-}\acute{e}tale}}
\def\etale{{\acute{e}tale}}
\def\QCoh{\textit{QCoh}}
\def\Ker{\mathop{\rm Ker}}
\def\Im{\mathop{\rm Im}}
\def\Coker{\mathop{\rm Coker}}
\def\Coim{\mathop{\rm Coim}}

%
% Macros for moduli stacks/spaces
%
\def\QCohstack{\mathcal{QC}\!{\it oh}}
\def\Cohstack{\mathcal{C}\!{\it oh}}
\def\Spacesstack{\mathcal{S}\!{\it paces}}
\def\Quotfunctor{{\rm Quot}}
\def\Hilbfunctor{{\rm Hilb}}
\def\Curvesstack{\mathcal{C}\!{\it urves}}
\def\Polarizedstack{\mathcal{P}\!{\it olarized}}
\def\Complexesstack{\mathcal{C}\!{\it omplexes}}
% \Pic is the operator that assigns to X its picard group, usage \Pic(X)
% \Picardstack_{X/B} denotes the Picard stack of X over B
% \Picardfunctor_{X/B} denotes the Picard functor of X over B
\def\Pic{\mathop{\rm Pic}\nolimits}
\def\Picardstack{\mathcal{P}\!{\it ic}}
\def\Picardfunctor{{\rm Pic}}
\def\Deformationcategory{\mathcal{D}\!{\it ef}}


% OK, start here.
%
\begin{document}

\title{Algebraic Stacks Desirables}

\begin{abstract}
OK, here are some ideas about how to write a text about 
stacks, what should go into it, and fixing ideas for some
of the definitions. Please email comments to the
\href{http://www.math.columbia.edu/mailman/listinfo/algebraic_geometry}%
{mailing list}.
\end{abstract}

\maketitle

\tableofcontents

\section{Introduction}
\label{section-introduction}

\noindent
This is basically just a list of things that we want to put in.

\section{Foundational and prerequisites}
\label{section-foundational}

\noindent
The next section is more interesting. Everything that comes from outside 
of the project is mentioned in here (?). Note that subsections may actually
be documents all by themselves in the project.

\subsection{Introduction}
\label{subsection-introduction}

\noindent
Introduction -- introduction -- introduction.

\subsection{Conventions}
\label{subsection-conventions}

\noindent
Short list of conventions used in the documents.

\subsection{Set Theory}
\label{subsection-set-theory}

\noindent 
We use Zermelo-Fraenkel set theory with the axiom of choice. We do not 
use classes or large sets such a universes (different from SGA4). We do
not stress set-theoretic issues but we make sure everything is correct 
(of course) and so we do not ignore them either.

\subsection{Categories}
\label{subsection-categories}

\noindent
Categories are sets. This means the category of rings (or of affine 
schemes) will mean all rings belonging to some (largish) set. We will 
perhaps have to introduce some notation for this, such as 
$\text{Rings}_\alpha$, where $\alpha$ is some ordinal. This will 
obviously create some problems and difficulties later. Perhaps we
will have contributors who will now and then go through the project and 
fix them up.

\smallskip\noindent
Define limits in this section; discuss suitable conditions on the
category (or partially ordered set) over which to take the limit.

\smallskip\noindent
We define ``catergories fibred in groupoids'' here. Mention that they are
always equivalent to others that are ``split'' (where you have pullback 
functors that compose on the nose). So in other words functors from the base
category to the category of groupoids.

\subsection{Sites and Topoi}
\label{subsection-sites}

\noindent
Do a little bit of theory here. Talk about sheaves, morphisms of sites.
The category of sheaves on a site now means all sheaves with values in
$V_\alpha = \text{Sets}_\alpha$ where $\alpha$ is suitably large (relative
to the site).

Introduce the notion of topos and morphism of topoi. The notion of 
simplicial and strictly simplicial topos.

\smallskip\noindent
Ringed sites, quasi-coherent sheaves of modules. Ringed topos and 
morphism of ringed topoi.

\smallskip\noindent
Some generalities about cohomology goes in here as well. (Just with
injective resolutions, nothing fancy. Allthough it might be nice to
have hypercoverings here for later use.) Injective resolutions of a
sheaf in a (strictly) simplicial topos.  The fundamental spectral 
sequence relating cohomology of the individual pieces to global
cohomology. The simplicial topos arising from a covering of the final
object in a topos and comparison of cohomologies.

\smallskip\noindent
Some basic facts about cohomological descent (at least enough to deal
with a flat hypercover for quasi--coherent sheaves and a proper
hypercover for \'etale sheaves).

\subsection{Stacks}
\label{stacks}

\noindent
Stacks are stacks in groupoids $p : \mathcal{S} \to \mathcal{C}$, where 
$\mathcal{C}$ is a site. We talk about $1$-morphisms between stacks and 
$2$-morphisms between $1$-morphisms. If we want to talk about a 
2-category of stacks then we choose a (large) ordinal $\alpha$ as before.

\smallskip\noindent
Introduce representable stacks. Introduce $2$-fibre products. Define 
representable $1$-morphisms of stacks. (Let's not identify an object 
with the functor it represents, or its associated stack.) Define the
inertia stack of a stack. Define gerbes.

\smallskip\noindent
Talk about properties of morphisms in $\mathcal{C}$ and properties of
$1$-morphisms of stacks: Suppose that $P$ is a property of morphisms
in $\mathcal{C}$ that is ``local on the target''. Then there is a 
corresponding property of representable $1$-morphisms. Etc.

\smallskip\noindent
Example result: Given a sheaf of abelian groups $\mathcal{F}$ 
over $\mathcal{C}$ the set of equivalence classes of gerbes with ``group'' 
$\mathcal{F}$ is bijective to $H^2(\mathcal{C}, \mathcal{F})$.
In particular enlarging $\alpha$ above will not matter.

\subsection{Algebra}
\label{subsection-algebra}

\noindent
Mainly commutative algebra here. In particular we can define what it means
for a morphism of rings to be flat, \'etale, smooth, finite type, etc.

\smallskip\noindent
We should have the nice short argument (Grothendieck's I think) proving 
descent of modules through flaithfully flat ring maps here so that we can 
have a complete proof of descent for polarized projective schemes later on.

\subsection{Schemes}
\label{subsection-schemes}

\noindent
Some results and definitions about schemes go here. This is a little backward
because schemes will be defined later. Make sure we have a list of properties
of morphisms of schemes here mirroring the list of properites of ring maps
in \autoref{subsection-algebra}. Also, projective and proper morphisms.

\smallskip\noindent
Define simplicial schemes.

\subsection{Cohomology of schemes}
\label{subsection-schemes-cohomology}

\noindent
Talk a little about specifically \'etale cohomology and flat cohomology
of schemes, Galois cohomology etc.

\subsection{Deformation theory a la Schlessinger}
\label{subsection-deformation-schlessinger}

\noindent
Maybe this should come later but we could have a short discussion of 
Schlessinger's paper and also discussing a tiny bit what happens if 
you have automorphisms (functor in groupoids case -- this is somewhere
in the literature SGA??).

\section{Algebraic Stacks, algebraic spaces and schemes}
\label{section-algebraic-stacks}

\noindent
Here we introduce algebraic stacks and say what they are. The basic setup
here is that schemes, algebraic spaces and algebraic stacks are all
stacks over the category of affine schemes $\textbf{Aff}_\alpha$ with
the fppf topology. This means we do not assume that we know what a scheme
is. Also, as suggested by Martin Olsson, it makes sense not to impose 
stronger separation conditions then strictly necessary.

\smallskip\noindent
A technicality will be that a $1$-morphism between stacks over the category
of affines being representable means that it is ``in reality'' representable
by affine schemes. In particular, if you think of a scheme as a stack (as we
will) then the diagonal $X \to X\times X$ is in general not representable as
a $1$-morphism of stacks on the category of affine schemes. Of course it is
representable if $X$ is separated, so perhaps if makes sense to define 
separated schemes first? Yes, this is ugly! (See 
\autoref{subsubsection-schemes-alternative} below for an alternative.)

\subsection{Definition of schemes}
\label{subsection-definition-schemes}

\noindent
A separated scheme is a stack $\mathcal{S}$ over the category of affine 
schemes $\mathbf{Aff}_\alpha$ such that: (a) all fibre categories are 
equivalent to sets, (b) there is a set $I$ and $1$-morphisms
$\pi_i : X_i \to \mathcal{S}$, $i\in I$ such that (c) each $X_i$ is
representable, (d) each $\pi_i$ is representable by open immersions,
(e) $\coprod X_i \to \mathcal{S}$ is a covering for the Zariski topology, 
and (f) the $1$-morphism $\mathcal{S} \to \mathcal{S}\times \mathcal{S}$ is
representable by closed immersions. 

\smallskip\noindent
A scheme is a stack with properties (a), (b), (c), (d)', (e), where
(d)' $=$ each $\pi_i$ representable by separated schemes and is an open 
immersion. 

\smallskip\noindent
In this subsection we explain how this is exactly the same as the notion of
schemes in EGA or Hartshorne.

\subsubsection{Alternative}
\label{subsubsection-schemes-alternative}

\noindent 
Define what it means for a $1$-morphism of stacks over $\textbf{Aff}_\alpha$
to be representable by quasi-affine schemes and use that in (d) to get the
definition of a scheme in 1 step.

\subsection{Definition of algebraic spaces}
\label{subsection-definition-algebraic-spaces}

\noindent
Here we use (a), (b), (c), (d)'' and (e)', where
(d)'' $=$ each $\pi_i$ is representable by schemes and \'etale, and
in (e)' we say it is an \'etale covering. There is also some weak
separation property (FIXME).

\subsection{Definition of algebraic stacks}
\label{subsection-definition-algebraic-stacks}

\noindent
An algebraic stack is a stack that has a diagonal representable by algebraic
spaces, that is the target of a surjective smooth morphism from a scheme,
and that has some weak separation condition (FIXME).

\smallskip\noindent
The notion ``Deligne-Mumford stack'' will be reserved for a stack as in 
\cite{DM}. Perhaps it makes sense to reserve the term ``Artin stack'' for
a stack such as in the papers by Artin \cite{Artin}, and \cite{ArtinVersal}.
(See also \cite{ConradeJong}.) In other words, and Artin stack will be an
algebraic stack with some reasonable finiteness and separatedness conditions.

\subsection{Examples of schemes, algebraic spaces, algebraic stacks}
\label{subsection-examples-stacks}

\noindent
It really is not that hard to show that $\mathcal{M}_g$ is an algebraic
stack for $g\geq 2$. We should have $[X/G]$ here. We should really
have a long list of moduli problems here and prove they are all algebraic
stacks. (Some of them we can postpone the proof until after Artin
approximation.) For example the Kontsevich moduli space in characteristic 
$p > 0$.

\smallskip\noindent
How about the algebraic space you get from the deformation theory of
a general surface in $\mathbf{P}^3$ with a node? (I mean where you deform
it to a general smooth surface in $\mathbf{P}^3$.)

\smallskip\noindent
Perhaps we can talk about some small dimensional examples here too. 
For example the stack where you have $\mathbf{A}^1$ with a $B(\mathbf{Z}/2)$ 
sitting at $0$. Bugeyed covers. You name it.

\subsection{Properties of algebraic stacks}
\label{subsection-stacks-properties}

\noindent
Such as the various ways of defining what a proper algebraic stack is.
Of course these things are really properties of morphisms of stacks.

\smallskip\noindent
We can define singularities (up to smooth factors) etc. Prove that a 
connected normal stack is irreducible, etc.

\subsection{Lisse etale site of an algebraic stack}
\label{subsection-lisse-etale}

\noindent
This has to be explained. Explain it is not functorial with respect to
$1$-morphisms of algebraic stacks. Define etale cohomology of an algebraic 
stack.

\subsection{Things you always wanted to know but were afraid to ask}
\label{stacks-fun-lemmas}

\noindent
There are going to be lots of lemmas that you use over and over again
that are usefull but aren't really mentioned specifically in the literature,
or it isn't easy to find references for. Bag of tricks.

\smallskip\noindent
Example: Given two groupoids in schemes $R\Rightarrow U$ and
$R' \Rightarrow U'$ whatdoes it mean to have a $1$-morphism
$[U/R] \to [U'/R']$ purely in terms of groupoids in schemes.
(This is bad because surely this is in the lit somewhere.) 
More anybody?

\subsection{Quasi-coherent sheaves on stacks}
\label{subsection-quasi-coherent}

\noindent
Define them and explain how you get them. You can define them as living on the 
lisse-etale site or on all of the stack and show the two notions are 
equivalent. Cohomology of quasi-coherent sheaves.

\section{Fundamental results}
\label{section-results-fundamental}

\noindent
Here is a list of results that could be explained. (From memory and in 
random order.) Actually, perhaps in stead of having the goal to fully explain
these results we should look in the relevant papers for the basic tricks 
used in them and explain those and put them in 
\autoref{section-algebraic-stacks}.

\subsection{Flat and smooth} 
\label{subsection-flat-smooth}

\noindent
Artin's theorem that having a flat surjection from a scheme is a replacement 
for the smooth surjective condition.

\subsection{Artin's representability theorem} 
\label{subsection-representability}

\noindent
Title is clear enough. Perhaps we can reformulate the condition of having a
deformation theory a little to adapt it more to the examples we know about,
especially those where there is a perfect obstruction theory (discussions 
with Jason)?

\subsection{DM stacks are finitely covered by schemes}
\label{subsection-dm-finite-cover}

\noindent
This all begins with Gabber's lemma I think. Somewhere in Asterisque about
Faltings proof of Mordell?

\subsection{Martin Olson's paper on properness}
\label{subsection-proper-parametrization}

\noindent
This proves two notions of proper are the same. We can also discuss Faltings
result that it suffices to use DVR's in certain cases.

\subsection{Proper pushforward of coherent sheaves}
\label{subsection-proper-pushforward}

\noindent
No comments yet.

\subsection{Keel and Mori} 
\label{subsection-keel-mori}

\noindent
See \cite{K-M}. The steps in this article also give a good way of looking at
what an algebraic stack locally looks like.

\subsection{Add more here} 
\label{subsection-add-more}

\noindent
Please.


\section{Other chapters}

\begin{multicols}{2}
\begin{enumerate}
\item \hyperref[introduction-section-phantom]{Introduction}
\item \hyperref[conventions-section-phantom]{Conventions}
\item \hyperref[sets-section-phantom]{Set Theory}
\item \hyperref[categories-section-phantom]{Categories}
\item \hyperref[topology-section-phantom]{Topology}
\item \hyperref[sheaves-section-phantom]{Sheaves on Spaces}
\item \hyperref[algebra-section-phantom]{Commutative Algebra}
\item \hyperref[sites-section-phantom]{Sites and Sheaves}
\item \hyperref[homology-section-phantom]{Homological Algebra}
\item \hyperref[derived-section-phantom]{Derived Categories}
\item \hyperref[more-algebra-section-phantom]{More Algebra}
\item \hyperref[simplicial-section-phantom]{Simplicial Methods}
\item \hyperref[modules-section-phantom]{Sheaves of Modules}
\item \hyperref[sites-modules-section-phantom]{Modules on Sites}
\item \hyperref[injectives-section-phantom]{Injectives}
\item \hyperref[cohomology-section-phantom]{Cohomology of Sheaves}
\item \hyperref[sites-cohomology-section-phantom]{Cohomology on Sites}
\item \hyperref[hypercovering-section-phantom]{Hypercoverings}
\item \hyperref[schemes-section-phantom]{Schemes}
\item \hyperref[constructions-section-phantom]{Constructions of Schemes}
\item \hyperref[properties-section-phantom]{Properties of Schemes}
\item \hyperref[morphisms-section-phantom]{Morphisms of Schemes}
\item \hyperref[coherent-section-phantom]{Coherent Cohomology}
\item \hyperref[divisors-section-phantom]{Divisors}
\item \hyperref[limits-section-phantom]{Limits of Schemes}
\item \hyperref[varieties-section-phantom]{Varieties}
\item \hyperref[chow-section-phantom]{Chow Homology}
\item \hyperref[topologies-section-phantom]{Topologies on Schemes}
\item \hyperref[descent-section-phantom]{Descent}
\item \hyperref[more-morphisms-section-phantom]{More on Morphisms}
\item \hyperref[flat-section-phantom]{More on Flatness}
\item \hyperref[groupoids-section-phantom]{Groupoid Schemes}
\item \hyperref[more-groupoids-section-phantom]{More on Groupoid Schemes}
\item \hyperref[etale-section-phantom]{\'Etale Morphisms of Schemes}
\item \hyperref[etale-cohomology-section-phantom]{\'Etale Cohomology}
\item \hyperref[spaces-section-phantom]{Algebraic Spaces}
\item \hyperref[spaces-properties-section-phantom]{Properties of Algebraic Spaces}
\item \hyperref[spaces-morphisms-section-phantom]{Morphisms of Algebraic Spaces}
\item \hyperref[spaces-topologies-section-phantom]{Topologies on Algebraic Spaces}
\item \hyperref[spaces-descent-section-phantom]{Descent and Algebraic Spaces}
\item \hyperref[spaces-more-morphisms-section-phantom]{More on Morphisms of Spaces}
\item \hyperref[quot-section-phantom]{Quot and Hilbert Spaces}
\item \hyperref[stacks-section-phantom]{Stacks}
\item \hyperref[spaces-groupoids-section-phantom]{Groupoids in Algebraic Spaces}
\item \hyperref[spaces-more-groupoids-section-phantom]{More on Groupoids in Spaces}
\item \hyperref[bootstrap-section-phantom]{Bootstrap}
\item \hyperref[examples-stacks-section-phantom]{Examples of Stacks}
\item \hyperref[groupoids-quotients-section-phantom]{Quotients of Groupoids}
\item \hyperref[algebraic-section-phantom]{Algebraic Stacks}
\item \hyperref[criteria-section-phantom]{Criteria for Representability}
\item \hyperref[stacks-properties-section-phantom]{Properties of Algebraic Stacks}
\item \hyperref[stacks-morphisms-section-phantom]{Morphisms of Algebraic Stacks}
\item \hyperref[examples-section-phantom]{Examples}
\item \hyperref[exercises-section-phantom]{Exercises}
\item \hyperref[guide-section-phantom]{Guide to Literature}
\item \hyperref[desirables-section-phantom]{Desirables}
\item \hyperref[coding-section-phantom]{Coding Style}
\item \hyperref[fdl-section-phantom]{GNU Free Documentation License}
\item \hyperref[index-section-phantom]{Auto Generated Index}
\end{enumerate}
\end{multicols}



\bibliography{my}
\bibliographystyle{alpha}

\end{document}
