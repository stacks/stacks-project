\IfFileExists{stacks-project.cls}{%
\documentclass{stacks-project}
}{%
\documentclass{amsart}
}

% The following AMS packages are automatically loaded with
% the amsart documentclass:
%\usepackage{amsmath}
%\usepackage{amssymb}
%\usepackage{amsthm}

% For dealing with references we use the comment environment
\usepackage{verbatim}
\newenvironment{reference}{\comment}{\endcomment}
%\newenvironment{reference}{}{}
\newenvironment{slogan}{\comment}{\endcomment}
\newenvironment{history}{\comment}{\endcomment}

% For commutative diagrams you can use
% \usepackage{amscd}
\usepackage[all]{xy}

% We use 2cell for 2-commutative diagrams.
\xyoption{2cell}
\UseAllTwocells

% To put source file link in headers.
% Change "template.tex" to "this_filename.tex"
% \usepackage{fancyhdr}
% \pagestyle{fancy}
% \lhead{}
% \chead{}
% \rhead{Source file: \url{template.tex}}
% \lfoot{}
% \cfoot{\thepage}
% \rfoot{}
% \renewcommand{\headrulewidth}{0pt}
% \renewcommand{\footrulewidth}{0pt}
% \renewcommand{\headheight}{12pt}

\usepackage{multicol}

% For cross-file-references
\usepackage{xr-hyper}

% Package for hypertext links:
\usepackage{hyperref}

% For any local file, say "hello.tex" you want to link to please
% use \externaldocument[hello-]{hello}
\externaldocument[introduction-]{introduction}
\externaldocument[conventions-]{conventions}
\externaldocument[sets-]{sets}
\externaldocument[categories-]{categories}
\externaldocument[topology-]{topology}
\externaldocument[sheaves-]{sheaves}
\externaldocument[sites-]{sites}
\externaldocument[stacks-]{stacks}
\externaldocument[fields-]{fields}
\externaldocument[algebra-]{algebra}
\externaldocument[brauer-]{brauer}
\externaldocument[homology-]{homology}
\externaldocument[derived-]{derived}
\externaldocument[simplicial-]{simplicial}
\externaldocument[more-algebra-]{more-algebra}
\externaldocument[smoothing-]{smoothing}
\externaldocument[modules-]{modules}
\externaldocument[sites-modules-]{sites-modules}
\externaldocument[injectives-]{injectives}
\externaldocument[cohomology-]{cohomology}
\externaldocument[sites-cohomology-]{sites-cohomology}
\externaldocument[dga-]{dga}
\externaldocument[dpa-]{dpa}
\externaldocument[hypercovering-]{hypercovering}
\externaldocument[schemes-]{schemes}
\externaldocument[constructions-]{constructions}
\externaldocument[properties-]{properties}
\externaldocument[morphisms-]{morphisms}
\externaldocument[coherent-]{coherent}
\externaldocument[divisors-]{divisors}
\externaldocument[limits-]{limits}
\externaldocument[varieties-]{varieties}
\externaldocument[topologies-]{topologies}
\externaldocument[descent-]{descent}
\externaldocument[perfect-]{perfect}
\externaldocument[more-morphisms-]{more-morphisms}
\externaldocument[flat-]{flat}
\externaldocument[groupoids-]{groupoids}
\externaldocument[more-groupoids-]{more-groupoids}
\externaldocument[etale-]{etale}
\externaldocument[chow-]{chow}
\externaldocument[intersection-]{intersection}
\externaldocument[pic-]{pic}
\externaldocument[adequate-]{adequate}
\externaldocument[dualizing-]{dualizing}
\externaldocument[duality-]{duality}
\externaldocument[discriminant-]{discriminant}
\externaldocument[local-cohomology-]{local-cohomology}
\externaldocument[curves-]{curves}
\externaldocument[resolve-]{resolve}
\externaldocument[models-]{models}
\externaldocument[pione-]{pione}
\externaldocument[etale-cohomology-]{etale-cohomology}
\externaldocument[proetale-]{proetale}
\externaldocument[crystalline-]{crystalline}
\externaldocument[spaces-]{spaces}
\externaldocument[spaces-properties-]{spaces-properties}
\externaldocument[spaces-morphisms-]{spaces-morphisms}
\externaldocument[decent-spaces-]{decent-spaces}
\externaldocument[spaces-cohomology-]{spaces-cohomology}
\externaldocument[spaces-limits-]{spaces-limits}
\externaldocument[spaces-divisors-]{spaces-divisors}
\externaldocument[spaces-over-fields-]{spaces-over-fields}
\externaldocument[spaces-topologies-]{spaces-topologies}
\externaldocument[spaces-descent-]{spaces-descent}
\externaldocument[spaces-perfect-]{spaces-perfect}
\externaldocument[spaces-more-morphisms-]{spaces-more-morphisms}
\externaldocument[spaces-flat-]{spaces-flat}
\externaldocument[spaces-groupoids-]{spaces-groupoids}
\externaldocument[spaces-more-groupoids-]{spaces-more-groupoids}
\externaldocument[bootstrap-]{bootstrap}
\externaldocument[spaces-pushouts-]{spaces-pushouts}
\externaldocument[groupoids-quotients-]{groupoids-quotients}
\externaldocument[spaces-more-cohomology-]{spaces-more-cohomology}
\externaldocument[spaces-simplicial-]{spaces-simplicial}
\externaldocument[formal-spaces-]{formal-spaces}
\externaldocument[restricted-]{restricted}
\externaldocument[spaces-resolve-]{spaces-resolve}
\externaldocument[formal-defos-]{formal-defos}
\externaldocument[defos-]{defos}
\externaldocument[cotangent-]{cotangent}
\externaldocument[examples-defos-]{examples-defos}
\externaldocument[algebraic-]{algebraic}
\externaldocument[examples-stacks-]{examples-stacks}
\externaldocument[stacks-sheaves-]{stacks-sheaves}
\externaldocument[criteria-]{criteria}
\externaldocument[artin-]{artin}
\externaldocument[quot-]{quot}
\externaldocument[stacks-properties-]{stacks-properties}
\externaldocument[stacks-morphisms-]{stacks-morphisms}
\externaldocument[stacks-limits-]{stacks-limits}
\externaldocument[stacks-cohomology-]{stacks-cohomology}
\externaldocument[stacks-perfect-]{stacks-perfect}
\externaldocument[stacks-introduction-]{stacks-introduction}
\externaldocument[stacks-more-morphisms-]{stacks-more-morphisms}
\externaldocument[stacks-geometry-]{stacks-geometry}
\externaldocument[moduli-]{moduli}
\externaldocument[moduli-curves-]{moduli-curves}
\externaldocument[examples-]{examples}
\externaldocument[exercises-]{exercises}
\externaldocument[guide-]{guide}
\externaldocument[desirables-]{desirables}
\externaldocument[coding-]{coding}
\externaldocument[obsolete-]{obsolete}
\externaldocument[fdl-]{fdl}
\externaldocument[index-]{index}

% Theorem environments.
%
\theoremstyle{plain}
\newtheorem{theorem}[subsection]{Theorem}
\newtheorem{proposition}[subsection]{Proposition}
\newtheorem{lemma}[subsection]{Lemma}

\theoremstyle{definition}
\newtheorem{definition}[subsection]{Definition}
\newtheorem{example}[subsection]{Example}
\newtheorem{exercise}[subsection]{Exercise}
\newtheorem{situation}[subsection]{Situation}

\theoremstyle{remark}
\newtheorem{remark}[subsection]{Remark}
\newtheorem{remarks}[subsection]{Remarks}

\numberwithin{equation}{subsection}

% Macros
%
\def\lim{\mathop{\rm lim}\nolimits}
\def\colim{\mathop{\rm colim}\nolimits}
\def\Spec{\mathop{\rm Spec}}
\def\Hom{\mathop{\rm Hom}\nolimits}
\def\Ext{\mathop{\rm Ext}\nolimits}
\def\SheafHom{\mathop{\mathcal{H}\!{\it om}}\nolimits}
\def\SheafExt{\mathop{\mathcal{E}\!{\it xt}}\nolimits}
\def\Sch{\textit{Sch}}
\def\Mor{\mathop{\rm Mor}\nolimits}
\def\Ob{\mathop{\rm Ob}\nolimits}
\def\Sh{\mathop{\textit{Sh}}\nolimits}
\def\NL{\mathop{N\!L}\nolimits}
\def\proetale{{pro\text{-}\acute{e}tale}}
\def\etale{{\acute{e}tale}}
\def\QCoh{\textit{QCoh}}
\def\Ker{\mathop{\rm Ker}}
\def\Im{\mathop{\rm Im}}
\def\Coker{\mathop{\rm Coker}}
\def\Coim{\mathop{\rm Coim}}

%
% Macros for moduli stacks/spaces
%
\def\QCohstack{\mathcal{QC}\!{\it oh}}
\def\Cohstack{\mathcal{C}\!{\it oh}}
\def\Spacesstack{\mathcal{S}\!{\it paces}}
\def\Quotfunctor{{\rm Quot}}
\def\Hilbfunctor{{\rm Hilb}}
\def\Curvesstack{\mathcal{C}\!{\it urves}}
\def\Polarizedstack{\mathcal{P}\!{\it olarized}}
\def\Complexesstack{\mathcal{C}\!{\it omplexes}}
% \Pic is the operator that assigns to X its picard group, usage \Pic(X)
% \Picardstack_{X/B} denotes the Picard stack of X over B
% \Picardfunctor_{X/B} denotes the Picard functor of X over B
\def\Pic{\mathop{\rm Pic}\nolimits}
\def\Picardstack{\mathcal{P}\!{\it ic}}
\def\Picardfunctor{{\rm Pic}}
\def\Deformationcategory{\mathcal{D}\!{\it ef}}


% OK, start here.
%
\begin{document}

\title{Hypercoverings}

%\begin{abstract}
%\end{abstract}

\maketitle

\tableofcontents

\section{Introduction}
\label{section-introduction}

\noindent
Hypercoverings can be used to compute cohomology of abelian sheaves on sites
without recourse to injective resolutions. See \cite[Expose V, Sec. 7]{SGA4}.
A nice manuscript on cohomological descent is the text by Brian Conrad,
see \url{http://www.math.lsa.umich.edu/~bdconrad/papers/hypercover.pdf}.
Probably it is useless to try to improve on Brian's article, so we look
at the question a little differently (more naively).

\section{Definitions}
\label{section-definitions}

\noindent
Let $\mathcal{C}$ be a category. Let $\Delta$ be the category of finite 
ordered sets with objects $[0]=\{0\}, [1]=\{0,1\}, [2]=\{0,1,2\},\ldots$ 
and order preserving maps. A simplicial object $U_\bullet$ of $\mathcal{C}$ 
is a contravariant functor $U_\bullet : \Delta \to \mathcal{C}$. This means 
there are objects $U_0,U_1,U_2,\ldots$ and morphisms $U_\bullet(\varphi) : 
U_n \to U_m$, where $\varphi$ is any order preserving map 
$\varphi : [m] \to [n]$.

\smallskip\noindent
In particular there is a unique morphism $U_0 \to U_n$ and there are
exactly $n+1$ morphisms $U_n \to U_0$ corresponding to the $n+1$ maps
$[0] \to [n]$. Obviously we need some more notation to be able to talk 
intelligently about these simplicial objects.

\begin{definition}
\label{definition-face-degeneracy}
For any integer $n\geq 1$, and any $0\leq j \leq n$ we let $d^n_j : [n-1]
\to [n]$ denote the injective order preserving map skipping $j$. For any
integer $n\geq 0$, and any $0\leq j \leq n$ we denote $s^n_j : [n+1] 
\to [n]$ the surjective order preserving map with 
$(s^n_j)^{-1}(\{j\}) = \{j, j+1\}$.
\end{definition}

\noindent
We get a unique morphism $U_\bullet(s^0_0) : U_0 \to U_1$ and
two morphisms $U_\bullet(d^1_0), U_\bullet(d^1_1) : U_1 \to U_0$.
There are two morphisms $U_\bullet(s^1_0), U_\bullet(s^1_1) :
U_1 \to U_2$ and three morphisms $U_\bullet(d^2_0), 
U_\bullet(d^2_1), U_\bullet(d^2_2) : U_3 \to U_2$. And so on.
FIXME: This notation...

\smallskip\noindent
FIXME: Much more.

\begin{example}
\label{example-simplicial-products}
(1) The simplest example is the {\it constant} simplicial object with
value $X \in \text{Ob}(\mathcal{C})$. In other words, $U_n=X$ and
all maps are $\text{id}_X$. \\
(2) Suppose that $Y\to X$ is a morphism of $C$ such that all
the fibred products $Y_{/X}^nY = \times_X Y \times_X \ldots Y$ exist.
Then we set $U_n = Y^{n+1}_{/X}$, and we let $s: [n] \to [m]$
correspond to the map (on ``coordinates'') $(y_0,\ldots, y_m) 
\mapsto (y_{s(0)},\ldots, y_{s(n)})$.
\end{example}

\subsection{Goals}

\noindent
Assume that $\mathcal{C}$ is a site with the property
that the set of coverings consisting of $1$ morphism is cofinal.
Let $\mathcal{F}$ be a sheaf of abelian groups on
the site $\mathcal{C}$ which is assumed to have the property
that the set of coverings consisting of $1$ morphism is cofinal.
Choose an injective resolution $\mathcal{F} \to \mathcal{J}^\bullet$
(for example a canonical one, see 
Injectives, Subsection \ref{injectives-subsection-injectives-sheaves}).
Let $X$ be an object of $\mathcal{C}$. We want to compute 
$R\Gamma(X, \mathcal{F}) = \Gamma(X, \mathcal{J}^\bullet)$
or at least the cohomology groups $H^j(X, \mathcal{F})$.
The idea is to construct simplicial objects $U_\bullet$ 
augmented towards $X$, so $U_\bullet \to X$, such that 
$$
R\Gamma(X, \mathcal{F}) 
= \text{Tot}(R\Gamma(U_\bullet, \mathcal{J}^\bullet))
\leqno{(*)}
$$
is a quasi-isomorphism (for any $\mathcal{F}$). On the right hand 
side this is the total complex associated to the double complex. 
(The maps are always canonical since we have the resolution over 
all of $\mathcal{C}$.)
The complex $\Gamma(U_\bullet, \mathcal{F})$ maps into the
complex on the right. We will show that for any
element $\eta \in H^j(X, \mathcal{F})$ there exists a choice
of $U_\bullet \to X$ such that $\eta$ comes from an element
in $H^j(U_\bullet, \mathcal{F})$. This is a first step and
it already allows us to define cup products for example.
The starting point is the following.

\begin{lemma}
\label{lemma-product-hypercovering}
Suppose that $\{Y \to X\}$ is a covering in the topology of
$\mathcal{C}$. Let $U_n = Y^n_{/X}$ be the simplicial
object defined in Example \ref{example-simplicial-products}.
The augmentation $U_\bullet \to X$ has the property
that $(*)$ is a quasi-isomorphism for all $\mathcal{F}$.
\end{lemma}

\begin{proof}
FIXME.
\end{proof}

\subsection{Making simplicial objects}
\label{subsection-making-simplicial}

\noindent
Suppose that $U_\bullet$ is a simplicial object of $\mathcal{C}$. Now let
$n\geq 0$ and let $V \to U_n$ be a representable morphism of 
$\mathcal{C}$. This means that the fibre products $V \times_{U_n} W$ 
exist for all morphisms $W \to U_n$.

\smallskip\noindent
For any $m$ consider the fibre product (over $U_m$)
$$
U'_m = \prod\nolimits_{\varphi \in \text{Mor}_\Delta([n],[m])}
V\times_{U_n, U_\bullet(\varphi)} U_m.
$$
By our assumption on the morphism $V \to U_n$ this fibre product
exists. For any $\psi : [m1] \to [m2]$ there is a canonical morphism
$U'_{m2} \to U'_{m1}$ coming from the map $\text{Mor}_\Delta([n],[m1])
\to \text{Mor}_\Delta([n],[m2]), \varphi \mapsto \varphi \circ \psi$,
the identity map on $V$ and the canonical map $U_\bullet(\psi) : 
U_{m2} \to U_{m1}$

\smallskip\noindent
Clearly, these data give rise to a simplicial object $U'_\bullet$ in
$\mathcal{C}$. The natural morphisms $U'_m \to U_m$ give rise to a
morphism of simplicial objects $U'_\bullet \to U_\bullet$. Note that
the morphism $U'_n \to U_n$ factors throught the morphism $V \to U_n$
by projection onto the factor corresponding to $\varphi=\text{id}_{[n]}$.
Also, note that if $\mathcal{C}$ is a site and if 
$\{V \to U_n\}$ is a covering in the site then for any $m$ it is true
that $\{U'_m \to U_m\}$ is a covering. This proves the following lemma.

\begin{lemma}
\label{lemma-construct-new-covers}
Suppose that $U_\bullet$ and $V\to U_n$ are as above such that
$\{V \to U_n\}$ is a covering for the topology on the site
$\mathcal{C}$. The morphism of simplicial objects 
$U'_\bullet \to U_\bullet$ constructed above has the following 
properties:
(1) The morphism $U'_n \to U_n$ factors trough $V \to U_n$.
(2) For any $m$ the set $\{U'_m \to U_m\}$ is a covering
in the topology of $\mathcal{C}$.
\end{lemma}

\subsection{Doubly simplicial stuff}
\label{subsection-doubly-simplicial}

\noindent
A doubly simplicial object of $\mathcal{C}$ is a functor
$U_{\bullet,\bullet} : (\Delta\times\Delta)^\circ \to \mathcal{C}$.
By subdividing we can make this into a simplicial object 
$W(U_{\bullet,\bullet})$ with the same cohomology. FIXME: Explain this.

\noindent
Suppose that $U'_\bullet \to U_\bullet$ is a morphism of simplicial
objects of $\mathcal{C}$ such that each of the morphisms $U'_n \to
U_n$ is representable. Then we can construct a doubly-simplicial
object $U'_{\bullet,\bullet}$ by setting $U'_{n,0}= U'_n$,
$$
U'_{n,1} = U'_n \times_{U_n} U'_n,
$$
etc. Compare Example \ref{example-simplicial-products}. 
Out of this object we can construct a single simplicial object
$W(U'_{\bullet,\bullet})$ as explained above. Construct the 
natural morphism of simplicial objects 
$W(U'_{\bullet,\bullet}) \to U_\bullet$.

\begin{lemma}
Suppose that every $\{U'_n \to U_n\}$ is a covering for the topology
of $\mathcal{C}$. Suppose that $\mathcal{F}$ is a sheaf on 
$\mathcal{C}$. Then there is a natural
morphism of complexes
$$
R\Gamma(U_\bullet, \mathcal{F}) \to 
R\Gamma(W(U_{\bullet,\bullet}), \mathcal{F})
$$
which is a quasi-isomorphism. FIXME: Something like this in any case.
\end{lemma}

\section{The general case}

\noindent
Mention how things work more generally, for example if $\mathcal{C}$
does not have the property that coverings consisting of a single map
are cofinal. State the theorem in the correct generality.

\section{Other chapters}

\begin{multicols}{2}
\begin{enumerate}
\item \hyperref[introduction-section-phantom]{Introduction}
\item \hyperref[conventions-section-phantom]{Conventions}
\item \hyperref[sets-section-phantom]{Set Theory}
\item \hyperref[categories-section-phantom]{Categories}
\item \hyperref[topology-section-phantom]{Topology}
\item \hyperref[sheaves-section-phantom]{Sheaves on Spaces}
\item \hyperref[algebra-section-phantom]{Commutative Algebra}
\item \hyperref[sites-section-phantom]{Sites and Sheaves}
\item \hyperref[homology-section-phantom]{Homological Algebra}
\item \hyperref[derived-section-phantom]{Derived Categories}
\item \hyperref[more-algebra-section-phantom]{More Algebra}
\item \hyperref[simplicial-section-phantom]{Simplicial Methods}
\item \hyperref[modules-section-phantom]{Sheaves of Modules}
\item \hyperref[sites-modules-section-phantom]{Modules on Sites}
\item \hyperref[injectives-section-phantom]{Injectives}
\item \hyperref[cohomology-section-phantom]{Cohomology of Sheaves}
\item \hyperref[sites-cohomology-section-phantom]{Cohomology on Sites}
\item \hyperref[hypercovering-section-phantom]{Hypercoverings}
\item \hyperref[schemes-section-phantom]{Schemes}
\item \hyperref[constructions-section-phantom]{Constructions of Schemes}
\item \hyperref[properties-section-phantom]{Properties of Schemes}
\item \hyperref[morphisms-section-phantom]{Morphisms of Schemes}
\item \hyperref[coherent-section-phantom]{Coherent Cohomology}
\item \hyperref[divisors-section-phantom]{Divisors}
\item \hyperref[limits-section-phantom]{Limits of Schemes}
\item \hyperref[varieties-section-phantom]{Varieties}
\item \hyperref[chow-section-phantom]{Chow Homology}
\item \hyperref[topologies-section-phantom]{Topologies on Schemes}
\item \hyperref[descent-section-phantom]{Descent}
\item \hyperref[more-morphisms-section-phantom]{More on Morphisms}
\item \hyperref[flat-section-phantom]{More on Flatness}
\item \hyperref[groupoids-section-phantom]{Groupoid Schemes}
\item \hyperref[more-groupoids-section-phantom]{More on Groupoid Schemes}
\item \hyperref[etale-section-phantom]{\'Etale Morphisms of Schemes}
\item \hyperref[etale-cohomology-section-phantom]{\'Etale Cohomology}
\item \hyperref[spaces-section-phantom]{Algebraic Spaces}
\item \hyperref[spaces-properties-section-phantom]{Properties of Algebraic Spaces}
\item \hyperref[spaces-morphisms-section-phantom]{Morphisms of Algebraic Spaces}
\item \hyperref[spaces-topologies-section-phantom]{Topologies on Algebraic Spaces}
\item \hyperref[spaces-descent-section-phantom]{Descent and Algebraic Spaces}
\item \hyperref[spaces-more-morphisms-section-phantom]{More on Morphisms of Spaces}
\item \hyperref[quot-section-phantom]{Quot and Hilbert Spaces}
\item \hyperref[stacks-section-phantom]{Stacks}
\item \hyperref[spaces-groupoids-section-phantom]{Groupoids in Algebraic Spaces}
\item \hyperref[spaces-more-groupoids-section-phantom]{More on Groupoids in Spaces}
\item \hyperref[bootstrap-section-phantom]{Bootstrap}
\item \hyperref[examples-stacks-section-phantom]{Examples of Stacks}
\item \hyperref[groupoids-quotients-section-phantom]{Quotients of Groupoids}
\item \hyperref[algebraic-section-phantom]{Algebraic Stacks}
\item \hyperref[criteria-section-phantom]{Criteria for Representability}
\item \hyperref[stacks-properties-section-phantom]{Properties of Algebraic Stacks}
\item \hyperref[stacks-morphisms-section-phantom]{Morphisms of Algebraic Stacks}
\item \hyperref[examples-section-phantom]{Examples}
\item \hyperref[exercises-section-phantom]{Exercises}
\item \hyperref[guide-section-phantom]{Guide to Literature}
\item \hyperref[desirables-section-phantom]{Desirables}
\item \hyperref[coding-section-phantom]{Coding Style}
\item \hyperref[fdl-section-phantom]{GNU Free Documentation License}
\item \hyperref[index-section-phantom]{Auto Generated Index}
\end{enumerate}
\end{multicols}


\bibliography{my}
\bibliographystyle{alpha}

\end{document}
