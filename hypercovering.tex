\IfFileExists{stacks-project.cls}{%
\documentclass{stacks-project}
}{%
\documentclass{amsart}
}

% The following AMS packages are automatically loaded with
% the amsart documentclass:
%\usepackage{amsmath}
%\usepackage{amssymb}
%\usepackage{amsthm}

% For dealing with references we use the comment environment
\usepackage{verbatim}
\newenvironment{reference}{\comment}{\endcomment}
%\newenvironment{reference}{}{}
\newenvironment{slogan}{\comment}{\endcomment}
\newenvironment{history}{\comment}{\endcomment}

% For commutative diagrams you can use
% \usepackage{amscd}
\usepackage[all]{xy}

% We use 2cell for 2-commutative diagrams.
\xyoption{2cell}
\UseAllTwocells

% To put source file link in headers.
% Change "template.tex" to "this_filename.tex"
% \usepackage{fancyhdr}
% \pagestyle{fancy}
% \lhead{}
% \chead{}
% \rhead{Source file: \url{template.tex}}
% \lfoot{}
% \cfoot{\thepage}
% \rfoot{}
% \renewcommand{\headrulewidth}{0pt}
% \renewcommand{\footrulewidth}{0pt}
% \renewcommand{\headheight}{12pt}

\usepackage{multicol}

% For cross-file-references
\usepackage{xr-hyper}

% Package for hypertext links:
\usepackage{hyperref}

% For any local file, say "hello.tex" you want to link to please
% use \externaldocument[hello-]{hello}
\externaldocument[introduction-]{introduction}
\externaldocument[conventions-]{conventions}
\externaldocument[sets-]{sets}
\externaldocument[categories-]{categories}
\externaldocument[topology-]{topology}
\externaldocument[sheaves-]{sheaves}
\externaldocument[sites-]{sites}
\externaldocument[stacks-]{stacks}
\externaldocument[fields-]{fields}
\externaldocument[algebra-]{algebra}
\externaldocument[brauer-]{brauer}
\externaldocument[homology-]{homology}
\externaldocument[derived-]{derived}
\externaldocument[simplicial-]{simplicial}
\externaldocument[more-algebra-]{more-algebra}
\externaldocument[smoothing-]{smoothing}
\externaldocument[modules-]{modules}
\externaldocument[sites-modules-]{sites-modules}
\externaldocument[injectives-]{injectives}
\externaldocument[cohomology-]{cohomology}
\externaldocument[sites-cohomology-]{sites-cohomology}
\externaldocument[dga-]{dga}
\externaldocument[dpa-]{dpa}
\externaldocument[hypercovering-]{hypercovering}
\externaldocument[schemes-]{schemes}
\externaldocument[constructions-]{constructions}
\externaldocument[properties-]{properties}
\externaldocument[morphisms-]{morphisms}
\externaldocument[coherent-]{coherent}
\externaldocument[divisors-]{divisors}
\externaldocument[limits-]{limits}
\externaldocument[varieties-]{varieties}
\externaldocument[topologies-]{topologies}
\externaldocument[descent-]{descent}
\externaldocument[perfect-]{perfect}
\externaldocument[more-morphisms-]{more-morphisms}
\externaldocument[flat-]{flat}
\externaldocument[groupoids-]{groupoids}
\externaldocument[more-groupoids-]{more-groupoids}
\externaldocument[etale-]{etale}
\externaldocument[chow-]{chow}
\externaldocument[intersection-]{intersection}
\externaldocument[pic-]{pic}
\externaldocument[adequate-]{adequate}
\externaldocument[dualizing-]{dualizing}
\externaldocument[duality-]{duality}
\externaldocument[discriminant-]{discriminant}
\externaldocument[local-cohomology-]{local-cohomology}
\externaldocument[curves-]{curves}
\externaldocument[resolve-]{resolve}
\externaldocument[models-]{models}
\externaldocument[pione-]{pione}
\externaldocument[etale-cohomology-]{etale-cohomology}
\externaldocument[proetale-]{proetale}
\externaldocument[crystalline-]{crystalline}
\externaldocument[spaces-]{spaces}
\externaldocument[spaces-properties-]{spaces-properties}
\externaldocument[spaces-morphisms-]{spaces-morphisms}
\externaldocument[decent-spaces-]{decent-spaces}
\externaldocument[spaces-cohomology-]{spaces-cohomology}
\externaldocument[spaces-limits-]{spaces-limits}
\externaldocument[spaces-divisors-]{spaces-divisors}
\externaldocument[spaces-over-fields-]{spaces-over-fields}
\externaldocument[spaces-topologies-]{spaces-topologies}
\externaldocument[spaces-descent-]{spaces-descent}
\externaldocument[spaces-perfect-]{spaces-perfect}
\externaldocument[spaces-more-morphisms-]{spaces-more-morphisms}
\externaldocument[spaces-flat-]{spaces-flat}
\externaldocument[spaces-groupoids-]{spaces-groupoids}
\externaldocument[spaces-more-groupoids-]{spaces-more-groupoids}
\externaldocument[bootstrap-]{bootstrap}
\externaldocument[spaces-pushouts-]{spaces-pushouts}
\externaldocument[groupoids-quotients-]{groupoids-quotients}
\externaldocument[spaces-more-cohomology-]{spaces-more-cohomology}
\externaldocument[spaces-simplicial-]{spaces-simplicial}
\externaldocument[formal-spaces-]{formal-spaces}
\externaldocument[restricted-]{restricted}
\externaldocument[spaces-resolve-]{spaces-resolve}
\externaldocument[formal-defos-]{formal-defos}
\externaldocument[defos-]{defos}
\externaldocument[cotangent-]{cotangent}
\externaldocument[examples-defos-]{examples-defos}
\externaldocument[algebraic-]{algebraic}
\externaldocument[examples-stacks-]{examples-stacks}
\externaldocument[stacks-sheaves-]{stacks-sheaves}
\externaldocument[criteria-]{criteria}
\externaldocument[artin-]{artin}
\externaldocument[quot-]{quot}
\externaldocument[stacks-properties-]{stacks-properties}
\externaldocument[stacks-morphisms-]{stacks-morphisms}
\externaldocument[stacks-limits-]{stacks-limits}
\externaldocument[stacks-cohomology-]{stacks-cohomology}
\externaldocument[stacks-perfect-]{stacks-perfect}
\externaldocument[stacks-introduction-]{stacks-introduction}
\externaldocument[stacks-more-morphisms-]{stacks-more-morphisms}
\externaldocument[stacks-geometry-]{stacks-geometry}
\externaldocument[moduli-]{moduli}
\externaldocument[moduli-curves-]{moduli-curves}
\externaldocument[examples-]{examples}
\externaldocument[exercises-]{exercises}
\externaldocument[guide-]{guide}
\externaldocument[desirables-]{desirables}
\externaldocument[coding-]{coding}
\externaldocument[obsolete-]{obsolete}
\externaldocument[fdl-]{fdl}
\externaldocument[index-]{index}

% Theorem environments.
%
\theoremstyle{plain}
\newtheorem{theorem}[subsection]{Theorem}
\newtheorem{proposition}[subsection]{Proposition}
\newtheorem{lemma}[subsection]{Lemma}

\theoremstyle{definition}
\newtheorem{definition}[subsection]{Definition}
\newtheorem{example}[subsection]{Example}
\newtheorem{exercise}[subsection]{Exercise}
\newtheorem{situation}[subsection]{Situation}

\theoremstyle{remark}
\newtheorem{remark}[subsection]{Remark}
\newtheorem{remarks}[subsection]{Remarks}

\numberwithin{equation}{subsection}

% Macros
%
\def\lim{\mathop{\rm lim}\nolimits}
\def\colim{\mathop{\rm colim}\nolimits}
\def\Spec{\mathop{\rm Spec}}
\def\Hom{\mathop{\rm Hom}\nolimits}
\def\Ext{\mathop{\rm Ext}\nolimits}
\def\SheafHom{\mathop{\mathcal{H}\!{\it om}}\nolimits}
\def\SheafExt{\mathop{\mathcal{E}\!{\it xt}}\nolimits}
\def\Sch{\textit{Sch}}
\def\Mor{\mathop{\rm Mor}\nolimits}
\def\Ob{\mathop{\rm Ob}\nolimits}
\def\Sh{\mathop{\textit{Sh}}\nolimits}
\def\NL{\mathop{N\!L}\nolimits}
\def\proetale{{pro\text{-}\acute{e}tale}}
\def\etale{{\acute{e}tale}}
\def\QCoh{\textit{QCoh}}
\def\Ker{\mathop{\rm Ker}}
\def\Im{\mathop{\rm Im}}
\def\Coker{\mathop{\rm Coker}}
\def\Coim{\mathop{\rm Coim}}

%
% Macros for moduli stacks/spaces
%
\def\QCohstack{\mathcal{QC}\!{\it oh}}
\def\Cohstack{\mathcal{C}\!{\it oh}}
\def\Spacesstack{\mathcal{S}\!{\it paces}}
\def\Quotfunctor{{\rm Quot}}
\def\Hilbfunctor{{\rm Hilb}}
\def\Curvesstack{\mathcal{C}\!{\it urves}}
\def\Polarizedstack{\mathcal{P}\!{\it olarized}}
\def\Complexesstack{\mathcal{C}\!{\it omplexes}}
% \Pic is the operator that assigns to X its picard group, usage \Pic(X)
% \Picardstack_{X/B} denotes the Picard stack of X over B
% \Picardfunctor_{X/B} denotes the Picard functor of X over B
\def\Pic{\mathop{\rm Pic}\nolimits}
\def\Picardstack{\mathcal{P}\!{\it ic}}
\def\Picardfunctor{{\rm Pic}}
\def\Deformationcategory{\mathcal{D}\!{\it ef}}


% OK, start here.
%
\begin{document}

\title{Hypercoverings}


\maketitle

\phantomsection
\label{section-phantom}

\tableofcontents

\section{Introduction}
\label{section-introduction}

\noindent
Let $\mathcal{C}$ be a site, see Sites, Definition \ref{sites-definition-site}.
Let $X$ be an object of $\mathcal{C}$.
Given an abelian sheaf $\mathcal{F}$
on $\mathcal{C}$ we would like to compute
its cohomology groups
$$
H^i(X, \mathcal{F}).
$$
According to our general definitions
(insert future reference here)
this cohomology group is computed by
choosing an injective resolution
$$
0 \to \mathcal{F} \to \mathcal{I}^0 \to \mathcal{I}^1 \to \ldots
$$
and setting
$$
H^i(X, \mathcal{F})
=
H^i(
\Gamma(X, \mathcal{I}^0) \to
\Gamma(X, \mathcal{I}^1) \to
\Gamma(X, \mathcal{I}^2)\to \ldots)
$$
We will have to do quite a bit of work to prove that we
may also compute these cohomology groups without
choosing an injective resolution. Also, we will only do this
in case the site $\mathcal{C}$ has fibre products.

\medskip\noindent
A hypercovering in a site is a generalization of a covering.
See \cite[Expos\'e V, Sec. 7]{SGA4}. A hypercovering is a special
case of a simplicial augmentation where one has cohomological
descent, see \cite[Expos\'e Vbis]{SGA4}. A nice manuscript on
cohomological descent is the text by Brian Conrad, see
\url{http://math.stanford.edu/~conrad/papers/hypercover.pdf}.
Brian's text follows the exposition in \cite[Expos\'e Vbis]{SGA4}, and in
particular discusses a more general kind of hypercoverings, such as
proper hypercoverings of schemes used to compute \'etale cohomology
for example. A proper hypercovering can be seen as a hypercovering
in the category of schemes endowed with a different topology than
the \'etale topology, but still they can be used to compute the \'etale
cohomology.





























\section{Hypercoverings}
\label{section-hypercoverings}

\noindent
In order to start we make the following definition.
The letters ``SR'' stand for Semi-Representable.

\begin{definition}
\label{definition-SR}
Let $\mathcal{C}$ be a site with fibre products.
Let $X \in \text{Ob}(\mathcal{C})$ be an object of $\mathcal{C}$.
We denote $\text{SR}(\mathcal{C}, X)$ the category of
{\it semi-representable objects} defined as follows
\begin{enumerate}
\item objects are families of morphisms
$\{U_i \to X\}_{i \in I}$, and
\item morphisms $\{U_i \to X\}_{i \in I} \to
\{V_j \to X\}_{j \in J}$ are given by
a map $\alpha : I \to J$ and for each $i \in I$
a morphism $f_i : U_i \to V_{\alpha(i)}$ over $X$.
\end{enumerate}
\end{definition}

\noindent
This definition is different from the one in
\cite[Expos\'e V, Sec. 7]{SGA4}, but it seems flexible
enough to do all the required arguments.
Note that this is a ``big'' category. We will later
``bound'' the size of the index sets $I$ that we need
and we can then redefine $\text{SR}(\mathcal{C}, X)$
to become a category.

\begin{definition}
\label{definition-SR-F}
Let $\mathcal{C}$ be a site with fibre products.
Let $X \in \text{Ob}(\mathcal{C})$ be an object of $\mathcal{C}$.
We denote $F$ the functor {\it which associates a sheaf to a
semi-representable object}. In a formula
\begin{eqnarray*}
F : \text{SR}(\mathcal{C}, X) & \longrightarrow & \textit{PSh}(\mathcal{C}) \\
\{U_i \to X\}_{i \in I} & \longmapsto & \amalg_{i\in I} h_{U_i}
\end{eqnarray*}
where $h_U$ denotes the representable presheaf associated to
the object $U$.
\end{definition}

\noindent
Given a morphism $U \to X$ we obtain a morphism
$h_U \to h_X$ of representable presheaves.
Thus it makes more sense to think of $F$ as a functor
into the category of presheaves of sets over $h_X$,
namely $\textit{PSh}(\mathcal{C})/h_X$.

\begin{lemma}
\label{lemma-coprod-prod-SR}
Let $\mathcal{C}$ be a site with fibre products.
Let $X \in \text{Ob}(\mathcal{C})$ be an object of $\mathcal{C}$.
The category $\text{SR}(\mathcal{C}, X)$ has
coproducts and finite limits. Moreover, the functor $F$ commutes
with coproducts and fibre products, and transforms products
into fibre products over $h_X$. In other words, it commutes
with finite limits as a functor into $\textit{PSh}(\mathcal{C})/h_X$.
\end{lemma}

\begin{proof}
It is clear that the coproduct of
$\{U_i \to X\}_{i \in I}$ and $\{V_j \to X\}_{j \in J}$
is $\{U_i \to X\}_{i \in I} \amalg \{V_j \to X\}_{j \in J}$
and similarly for coproducts of
families of families of morphisms with target $X$.
The object $\{X \to X\}$ is a final
object of $\text{SR}(\mathcal{C}, X)$.
Suppose given a morphism
$(\alpha, f_i) : \{U_i \to X\}_{i \in I} \to \{V_j \to X\}_{j \in J}$
and a morphism
$(\beta, g_k) : \{W_k \to X\}_{k \in K} \to \{V_j \to X\}_{j \in J}$.
The fibred product of these morphisms is given by
$$
\{ U_i \times_{f_i, V_j, g_k} W_k \to X \}_{(i, j, k) \in I\times J\times K
\text{ such that } k = \alpha(i) = \beta(j)}
$$
The fibre products exist by the assumption that
$\mathcal{C}$ has fibre products.
Thus $\text{SR}(\mathcal{C}, X)$ has finite limits,
see Categories, Lemma \ref{categories-lemma-finite-limits-exist}.
The statements on the functor $F$ are clear from the constructions
above.
\end{proof}

\begin{definition}
\label{definition-covering-SR}
Let $\mathcal{C}$ be a site with fibred products.
Let $X$ be an object of $\mathcal{C}$.
Let $f = (\alpha, f_i) : \{U_i \to X\}_{i \in I} \to \{V_j \to X\}_{j \in J}$
be a morphism in the category $\text{SR}(\mathcal{C}, X)$.
We say that $f$ is a {\it covering} if for every $j \in J$ the
family of morphisms $\{U_i \to V_j\}_{i \in I, \alpha(i) = j}$
is a covering for the site $\mathcal{C}$.
\end{definition}

\begin{lemma}
\label{lemma-covering-permanence}
Let $\mathcal{C}$ be a site with fibred products.
Let $X \in \text{Ob}(\mathcal{C})$.
\begin{enumerate}
\item A composition of coverings in $\text{SR}(\mathcal{C}, X)$
is a covering.
\item A base change of coverings is a covering.
\item If $A \to B$ and $K \to L$ are coverings,
then $A \times K \to B \times L$ is a covering.
\end{enumerate}
\end{lemma}

\begin{proof}
Immediate from the axioms of a site.
(Number (3) is the composition $A \times K \to B \times K \to B \times L$
and hence a composition of basechanges of coverings.)
\end{proof}

\noindent
According to the results in the chapter on simplicial methods the
coskelet of a truncated simplicial object of
$\text{SR}(\mathcal{C}, X)$ exists. Hence the following
definition makes sense.

\begin{definition}
\label{definition-hypercovering}
Let $\mathcal{C}$ be a site.
Let $X \in \text{Ob}(\mathcal{C})$ be an object of $\mathcal{C}$.
A {\it hypercovering} of $X$ is a simplicial object
$K$ in the category $\text{SR}(\mathcal{C}, X)$ such that
\begin{enumerate}
\item The object $K_0$ is a covering of $X$ for the site $\mathcal{C}$.
\item For every $n \geq 0$ the canonical morphism
$$
K_{n + 1} \longrightarrow (\text{cosk}_n \text{sk}_n K)_{n + 1}
$$
is a covering in the sense defined above.
\end{enumerate}
\end{definition}

\noindent
Condition (1) makes sense since each object of
$\text{SR}(\mathcal{C}, X)$ is after all a family
of morphisms with target $X$. It could also be
formulated as saying that the morphism of $K_0$ to
the final object of $\text{SR}(\mathcal{C}, X)$
is a covering.

\begin{example}
\label{example-cech}
Let $\{U_i \to X\}_{i \in I}$ be a covering of the site $\mathcal{C}$.
Set $K_0 = \{U_i \to X\}_{i \in I}$.
Then $K_0$ is a $0$-truncated simplicial object of
$\text{SR}(\mathcal{C}, X)$. Hence we may form
$$
K = \text{cosk}_0 K_0.
$$
Clearly $K$ passes condition (1) of Definition \ref{definition-hypercovering}.
Since all the morphisms $K_{n + 1} \to (\text{cosk}_n \text{sk}_n K)_{n + 1}$
are isomorphisms it also passes condition (2). Note that
the terms $K_n$ are the usual
$$
K_n = \{
U_{i_0} \times_X U_{i_1} \times_X \ldots \times_X U_{i_n} \to X
\}_{(i_0, i_1, \ldots, i_n) \in I^{n + 1}}
$$
\end{example}

\begin{lemma}
\label{lemma-hypercoverings-set}
Let $\mathcal{C}$ be a site with fibre products.
Let $X \in \text{Ob}(\mathcal{C})$ be an object of $\mathcal{C}$.
The collection of all hypercoverings of $X$ forms a set.
\end{lemma}

\begin{proof}
Since $\mathcal{C}$ is a site, the set of all coverings of
$S$ forms a set. Thus we see that the collection
of possible $K_0$ forms a set. Suppose we have shown that
the collection of all possible $K_0, \ldots, K_n$ form
a set. Then it is enough to show that given
$K_0, \ldots, K_n$ the collection of all possible
$K_{n + 1}$ forms a set. And this is clearly true since
we have to choose $K_{n + 1}$ among all possible coverings
of $(\text{cosk}_n \text{sk}_n K)_{n + 1}$.
\end{proof}

\begin{remark}
\label{remark-hypercoverings-really-set}
The lemma does not just say that there is a cofinal
system of choices of hypercoverings that is a set,
but that really the hypercoverings form a set.
\end{remark}

\noindent
The category of presheaves on $\mathcal{C}$ has
finite (co)limits. Hence the functors $\text{cosk}_n$
exists for presheaves of sets.

\begin{lemma}
\label{lemma-hypercovering-F}
Let $\mathcal{C}$ be a site with fibre products.
Let $X \in \text{Ob}(\mathcal{C})$ be an object of $\mathcal{C}$.
Let $K$ be a hypercovering of $X$.
Consider the simplicial object $F(K)$ of $\textit{PSh}(\mathcal{C})$,
endowed with its augmentation to the constant simplicial presheaf $h_X$.
\begin{enumerate}
\item The morphism of presheaves $F(K)_0 \to h_X$ becomes
a surjection after sheafification.
\item The morphism
$$
(d^1_0, d^1_1) :
F(K)_1
\longrightarrow
F(K)_0 \times_{h_X} F(K)_0
$$
becomes a surjection after sheafification.
\item For every $n \geq 1$ the morphism
$$
F(K)_{n + 1} \longrightarrow (\text{cosk}_n \text{sk}_n F(K))_{n + 1}
$$
turns into a surjection after sheafification.
\end{enumerate}
\end{lemma}

\begin{proof}
We will use the fact that if
$\{U_i \to U\}_{i \in I}$ is a covering of the site
$\mathcal{C}$, then the morphism
$$
\amalg_{i \in I} h_{U_i} \to h_U
$$
becomes surjective after sheafification, see
Sites, Lemma \ref{sites-lemma-covering-surjective-after-sheafification}.
Thus the first assertion follows immediately.

\medskip\noindent
For the second assertion, note that according to
Simplicial, Example \ref{simplicial-example-cosk0}
the simplicial object $\text{cosk}_0 \text{sk}_0 K$
has terms $K_0 \times \ldots \times K_0$. Thus
according to the definition of a hypercovering we
see that $(d^1_0, d^1_1) : K_1 \to K_0 \times K_0$ is a
covering. Hence (2) follows from the claim above
and the fact that $F$ transforms products into fibred
products over $h_X$.

\medskip\noindent
For the third, we claim that
$\text{cosk}_n \text{sk}_n F(K) =
F(\text{cosk}_n \text{sk}_n K)$ for $n \geq 1$.
To prove this, denote temporarily $F'$ the functor
$\text{SR}(\mathcal{C}, X) \to \textit{PSh}(\mathcal{C})/h_X$.
By Lemma \ref{lemma-coprod-prod-SR} the functor
$F'$ commutes with finite limits.
By our description of the $\text{cosk}_n$ functor in
Simplicial, Section \ref{simplicial-section-skelet}
we see that $\text{cosk}_n \text{sk}_n F'(K) =
F'(\text{cosk}_n \text{sk}_n K)$.
Recall that the category used in the description of
$(\text{cosk}_n U)_m$ in
Simplicial, Lemma \ref{simplicial-lemma-existence-cosk}
is the category $(\Delta/[m])^{opp}_{\leq n}$. It is an
amusing exercise to show that $(\Delta/[m])_{\leq n}$ is
a nonempty connected category (see
Categories, Definition \ref{categories-definition-category-connected})
as soon as $n \geq 1$. Hence,
Categories, Lemma \ref{categories-lemma-connected-limit-over-X}
shows that $\text{cosk}_n \text{sk}_n F'(K) =
\text{cosk}_n \text{sk}_n F(K)$. Whence the claim.
Property (2) follows from this, because now we see that
the morphism in (2) is the result of applying the
functor $F$ to a covering as in Definition \ref{definition-covering-SR},
and the result follows from the first fact mentioned
in this proof.
\end{proof}



\section{Acyclicity}
\label{section-acyclicity}

\noindent
Let $\mathcal{C}$ be a site.
For a presheaf of sets $\mathcal{F}$ we denote $\mathbf{Z}_\mathcal{F}$
the presheaf of abelian groups defined by the rule
$$
\mathbf{Z}_\mathcal{F}(U) = \text{free abelian group on }\mathcal{F}(U).
$$
We will sometimes call this the {\it free abelian presheaf on $\mathcal{F}$}.
Of course the construction $\mathcal{F} \mapsto \mathbf{Z}_\mathcal{F}$
is a functor and it is left adjoint to the forgetful functor
$\textit{PAb}(\mathcal{C}) \to \textit{PSh}(\mathcal{C})$.
Of course the sheafification $\mathbf{Z}_\mathcal{F}^\#$ is
a sheaf of abelian groups, and the functor
$\mathcal{F} \mapsto \mathbf{Z}_\mathcal{F}^\#$ is a
left adjoint as well. We sometimes call $\mathbf{Z}_\mathcal{F}^\#$
the {\it free abelian sheaf on $\mathcal{F}$}.

\medskip\noindent
For an object $X$ of the site $\mathcal{C}$ we denote
$\mathbf{Z}_X$ the free abelian presheaf on $h_X$, and
we denote $\mathbf{Z}_X^\#$ its sheafification.

\begin{definition}
\label{definition-homology}
Let $\mathcal{C}$ be a site.
Let $K$ be a simplicial object of $\textit{PSh}(\mathcal{C})$.
By the above we get a simplicial object $\mathbf{Z}_K^\#$ of
$\textit{Ab}(\mathcal{C})$. We can take its associated
complex of abelian presheaves $s(\mathbf{Z}_K^\#)$, see
Simplicial, Section \ref{simplicial-section-complexes}.
The {\it homology of $K$} is the homology of the
complex of abelian sheaves $s(\mathbf{Z}_K^\#)$.
\end{definition}

\noindent
In other words, the {\it $i$th homology $H_i(K)$ of $K$}
is the sheaf of abelian groups $H_i(K) = H_i(s(\mathbf{Z}_K^\#))$.
In this section we worry about the homology in case $K$
is a hypercovering of an object $X$ of $\mathcal{C}$.

\begin{lemma}
\label{lemma-compare-cosk0}
Let $\mathcal{C}$ be a site.
Let $\mathcal{F} \to \mathcal{G}$ be a morphism
of presheaves of sets. Denote $K$ the simplicial
object of $\textit{PSh}(\mathcal{C})$ whose $n$th
term is the $(n + 1)$st fibre product of $\mathcal{F}$
over $\mathcal{G}$, see
Simplicial, Example \ref{simplicial-example-fibre-products-simplicial-object}.
Then, if $\mathcal{F} \to \mathcal{G}$ is surjective after
sheafification, we have
$$
H_i(K) =
\left\{
\begin{matrix}
0 & \text{if} & i > 0\\
\mathbf{Z}_\mathcal{G}^\# & \text{if} & i = 0
\end{matrix}
\right.
$$
The isomorphism in degree $0$ is given by the
morphsm $H_0(K) \to \mathbf{Z}_\mathcal{G}^\#$
coming from the map $(\mathbf{Z}_K^\#)_0 =
\mathbf{Z}_\mathcal{F}^\# \to \mathbf{Z}_\mathcal{G}^\#$.
\end{lemma}

\begin{proof}
Let $\mathcal{G}' \subset \mathcal{G}$ be the image of
the morphism $\mathcal{F} \to \mathcal{G}$.
Let $U \in \text{Ob}(\mathcal{C})$. Set
$A = \mathcal{F}(U)$ and $B = \mathcal{G}'(U)$.
Then the simplicial set $K(U)$ is equal to the simplicial
set with $n$-simplices given by
$$
A \times_B A \times_B \ldots \times_B A\ (n + 1 \text{ factors)}.
$$
By Simplicial, Lemma \ref{simplicial-lemma-cosk-minus-one-equivalence}
the morphism $K(U) \to B$ is a homotopy equivalence. Hence
applying the functor ``free abelian group on'' to this
we deduce that
$$
\mathbf{Z}_K(U) \longrightarrow \mathbf{Z}_B
$$
is a homotopy equivalence. Note that $s(\mathbf{Z}_B)$ is
the complex
$$
\ldots \to
\bigoplus\nolimits_{b \in B}\mathbf{Z} \xrightarrow{0}
\bigoplus\nolimits_{b \in B}\mathbf{Z} \xrightarrow{1}
\bigoplus\nolimits_{b \in B}\mathbf{Z} \xrightarrow{0}
\bigoplus\nolimits_{b \in B}\mathbf{Z} \to 0
$$
see Simplicial, Lemma \ref{simplicial-lemma-homology-eilenberg-maclane}.
Thus we see that
$H_i(s(\mathbf{Z}_K(U))) = 0$ for $i > 0$, and
$H_0(s(\mathbf{Z}_K(U))) = \bigoplus_{b \in B}\mathbf{Z}
= \bigoplus_{s \in \mathcal{G}'(U)} \mathbf{Z}$.
These identifications are compatible with restriction
maps.

\medskip\noindent
We conclude that $H_i(s(\mathbf{Z}_K)) = 0$ for $i > 0$ and
$H_0(s(\mathbf{Z}_K)) = \mathbf{Z}_{\mathcal{G}'}$, where here
we compute homology groups in $\textit{PAb}(\mathcal{C})$. Since
sheafification is an exact functor we deduce the result
of the lemma. Namely, the exactness implies
that $H_0(s(\mathbf{Z}_K))^\# = H_0(s(\mathbf{Z}_K^\#))$,
and similarly for other indices.
\end{proof}

\begin{lemma}
\label{lemma-acyclicity}
Let $\mathcal{C}$ be a site.
Let $f : L \to K$ be a morphism of
simplicial objects of $\textit{PSh}(\mathcal{C})$.
Let $n \geq 0$ be an integer.
Assume that
\begin{enumerate}
\item For $i < n$ the morphism $L_i \to K_i$ is an isomorphism.
\item The morphism $L_n \to K_n$ is surjective after sheafification.
\item The canonical map $L \to \text{cosk}_n \text{sk}_n L$ is an isomorphism.
\item The canonical map $K \to \text{cosk}_n \text{sk}_n K$ is an isomorphism.
\end{enumerate}
Then $H_i(f) : H_i(L) \to H_i(K)$ is an isomorphism.
\end{lemma}

\begin{proof}
This proof is exactly the same as the proof of
Lemma \ref{lemma-compare-cosk0} above. Namely,
we first let $K_n' \subset K_n$ be the sub presheaf
which is the image of the map $L_n \to K_n$. Assumption
(2) means that the sheafification of $K_n'$ is equal to
the sheafification of $K_n$. Moreover, since $L_i = K_i$
for all $i < n$ we see that get an $n$-truncated
simplicial presheaf $U$ by taking
$U_0 = L_0 = K_0, \ldots, U_{n - 1} = L_{n - 1} = K_{n - 1}, U_n = K'_n$.
Denote $K' = \text{cosk}_n U$, a simplicial presheaf.
Because we can construct $K'_m$ as a finite limit, and
since sheafification is exact, we see that
$(K'_m)^\# = K_m$. In other words, $(K')^\# = K^\#$.
We conclude, by exactness of sheafification once more,
that $H_i(K) = H_i(K')$. Thus it suffices to prove the lemma
for the morphism $L \to K'$, in other words, we may
assume that $L_n \to K_n$ is a surjective morphism
of {\it presheaves}!

\medskip\noindent
In this case, for any object $U$ of $\mathcal{C}$ we
see that the morphism of simplicial sets
$$
L(U) \longrightarrow K(U)
$$
satisfies all the assumptions of
Simplicial, Lemma \ref{simplicial-lemma-equiv}.
Hence it is a homotopy equivalence, and
thus
$$
\mathbf{Z}_L(U) \longrightarrow \mathbf{Z}_K(U)
$$
is a homotopy equivalence too. This for all $U$.
The result follows.
\end{proof}

\begin{lemma}
\label{lemma-acyclic-hypercover-sheaves}
Let $\mathcal{C}$ be a site.
Let $K$ be a simplicial presheaf.
Let $\mathcal{G}$ be a presheaf.
Let $K \to \mathcal{G}$ be an augmentation of $K$
towards $\mathcal{G}$. Assume that
\begin{enumerate}
\item The morphism of presheaves $K_0 \to \mathcal{G}$ becomes
a surjection after sheafification.
\item The morphism
$$
(d^1_0, d^1_1) :
K_1
\longrightarrow
K_0 \times_{\mathcal{G}} K_0
$$
becomes a surjection after sheafification.
\item For every $n \geq 1$ the morphism
$$
K_{n + 1} \longrightarrow (\text{cosk}_n \text{sk}_n K)_{n + 1}
$$
turns into a surjection after sheafification.
\end{enumerate}
Then $H_i(K) = 0$ for $i > 0$ and
$H_0(K) = \mathbf{Z}_\mathcal{G}^\#$.
\end{lemma}

\begin{proof}
Denote $K^n = \text{cosk}_n \text{sk}_n K$ for $n \geq 1$.
Define $K^0$ as the simplicial object with terms
$(K^0)_n$ equal to the $(n + 1)$-fold fibred product
$K_0 \times_{\mathcal{G}} \ldots \times_{\mathcal{G}} K_0$,
see Simplicial,
Example \ref{simplicial-example-fibre-products-simplicial-object}.
We have morphisms
$$
K \longrightarrow \ldots \to K^n \to K^{n - 1} \to \ldots \to K^1 \to K^0.
$$
The morphisms $K \to K^i$, $K^j \to K^i$ for $j \geq i \geq 1$ come
from the universal properties of the $\text{cosk}_n$ functors.
The morphism $K^1 \to K^0$ is the canonical morphism
from
Simplicial, Remark \ref{simplicial-remark-augmentation}.
We also recall that $K^0 \to \text{cosk}_1 \text{sk}_1 K^0$
is an isomorphism, see Simplicial, Lemma \ref{simplicial-lemma-cosk-minus-one}.

\medskip\noindent
By Lemma \ref{lemma-compare-cosk0} we see that
$H_i(K^0) = 0$ for $i > 0$ and $H_0(K^0) = \mathbf{Z}_\mathcal{G}^\#$.

\medskip\noindent
Pick $n \geq 1$. Consider the morphism $K^n \to K^{n - 1}$.
It is an isomorphism on terms of degree $< n$.
Note that $K^n \to \text{cosk}_n \text{sk}_n K^n$ and
$K^{n - 1} \to \text{cosk}_n \text{sk}_n K^{n - 1}$
are isomorphisms. Note that $(K^n)_n = K_n$ and
that $(K^{n - 1})_n = (\text{cosk}_{n - 1} \text{sk}_{n - 1} K)_n$.
Hence by assumption, we have that $(K^n)_n \to (K^{n - 1})_n$
is a morphism of presheaves which becomes surjective after
sheafification. By Lemma \ref{lemma-acyclicity} we conclude that
$H_i(K^n) = H_i(K^{n - 1})$.
Combined with the above this proves the lemma.
\end{proof}

\begin{lemma}
\label{lemma-hypercovering-acyclic}
Let $\mathcal{C}$ be a site with fibre products.
Let $X$ be an object of of $\mathcal{C}$.
Let $K$ be a hypercovering of $X$.
The homology of the simplicial presheaf $F(K)$ is
$0$ in degrees $> 0$ and equal to $\mathbf{Z}_X^\#$
in degree $0$.
\end{lemma}

\begin{proof}
Combine Lemmas \ref{lemma-acyclic-hypercover-sheaves}
and \ref{lemma-hypercovering-F}.
\end{proof}







\section{Covering hypercoverings}
\label{section-covering}

\noindent
Here are some ways to construct hypercoverings.
We note that since the category
$\text{SR}(\mathcal{C}, X)$ has fibre products
the category of simplicial objects
of $\text{SR}(\mathcal{C}, X)$ has fibre products
as well, see Simplicial, Lemma \ref{simplicial-lemma-fibre-product}.

\begin{lemma}
\label{lemma-funny-gamma}
Let $\mathcal{C}$ be a site with fibre products.
Let $X$ be an object of $\mathcal{C}$.
Let $K, L, M$ be simplicial objects of $\text{SR}(\mathcal{C}, X)$.
Let $a : K \to L$, $b : M \to L$ be morphisms.
Assume
\begin{enumerate}
\item $K$ is a hypercovering of $X$,
\item the morphism $M_0 \to L_0$ is a covering, and
\item for all $n \geq 0$ in the diagram
$$
\xymatrix{
M_{n + 1} \ar[dd] \ar[rr] \ar[rd]^\gamma &
&
(\text{cosk}_n \text{sk}_n M)_{n + 1} \ar[dd] \\
&
L_{n + 1}
\times_{(\text{cosk}_n \text{sk}_n L)_{n + 1}}
(\text{cosk}_n \text{sk}_n M)_{n + 1}
\ar[ld] \ar[ru]
& \\
L_{n + 1} \ar[rr] & & (\text{cosk}_n \text{sk}_n L)_{n + 1}
}
$$
the arrow $\gamma$ is a covering.
\end{enumerate}
Then the fibre product $K \times_L M$ is a hypercovering of $X$.
\end{lemma}

\begin{proof}
The morphism $(K \times_L M)_0 = K_0 \times_{L_0} M_0 \to K_0$
is a base change of a covering by (2), hence a covering, see
Lemma \ref{lemma-covering-permanence}. And $K_0 \to \{X \to X\}$
is a covering by (1). Thus $(K \times_L M)_0 \to \{X \to X\}$
is a covering by Lemma \ref{lemma-covering-permanence}. Hence
$K\times_L M$ satisfies the first condition of Definition
\ref{definition-hypercovering}.

\medskip\noindent
We still have to check that
$$
K_{n + 1} \times_{L_{n + 1}} M_{n + 1} = (K \times_L M)_{n + 1}
\longrightarrow
(\text{cosk}_n \text{sk}_n (K\times_L M))_{n + 1}
$$
is a covering for all $n \geq 0$. We abbreviate as follows:
$A = (\text{cosk}_n \text{sk}_n K)_{n + 1}$,
$B = (\text{cosk}_n \text{sk}_n L)_{n + 1}$, and
$C = (\text{cosk}_n \text{sk}_n M)_{n + 1}$.
The functor $\text{cosk}_n \text{sk}_n$ commutes with fibre products,
see Simplicial, Lemma \ref{simplicial-lemma-cosk-fibre-product}.
Thus the right hand side above is equal to $A \times_B C$.
Consider the following commutative diagram
$$
\xymatrix{
K_{n + 1} \times_{L_{n + 1}} M_{n + 1} \ar[r] \ar[d] &
M_{n + 1} \ar[d] \ar[rd]_\gamma \ar[rrd] &
& \\
K_{n + 1} \ar[r] \ar[rd] &
L_{n + 1} \ar[rrd] &
L_{n + 1} \times_B C \ar[l] \ar[r] &
C \ar[d] \\
&
A \ar[rr] &
&
B
}
$$
This diagram shows that
$$
K_{n + 1} \times_{L_{n + 1}} M_{n + 1}
=
(K_{n + 1} \times_B C)
\times_{(L_{n + 1} \times_B C), \gamma}
M_{n + 1}
$$
Now, $K_{n + 1} \times_B C \to A\times_B C$
is a base change of the covering $K_{n + 1} \to A$
via the morphism $A \times_B C \to A$, hence is a
covering. By assumption (3) the morphism $\gamma$ is a covering.
Hence the morphism
$$
(K_{n + 1} \times_B C)
\times_{(L_{n + 1} \times_B C), \gamma}
M_{n + 1}
\longrightarrow
K_{n + 1} \times_B C
$$
is a covering as a base change of a covering.
The lemma follows as a composition of coverings
is a covering.
\end{proof}

\begin{lemma}
\label{lemma-product-hypercoverings}
Let $\mathcal{C}$ be a site with fibre products.
Let $X$ be an object of $\mathcal{C}$.
If $K, L$ are hypercoverings of $X$, then
$K \times L$ is a hypercovering of $X$.
\end{lemma}

\begin{proof}
You can either verify this directly, or use
Lemma \ref{lemma-funny-gamma} above and check that $L \to \{X \to X\}$
has property (3).
\end{proof}


\noindent
Let $\mathcal{C}$ be a site with fibre products.
Let $X$ be an object of $\mathcal{C}$.
Since the category $\text{SR}(\mathcal{C}, X)$ has coproducts and
finite limits, it is permissible to speak about the objects
$U \times K$ and $\Hom(U, K)$ for certain simplicial sets $U$
(for example those with finitely many nondegenerate simplices)
and any simplicial object $K$ of $\text{SR}(\mathcal{C}, X)$.
See Simplicial, Sections
\ref{simplicial-section-product-with-simplicial-sets} and
\ref{simplicial-section-hom-from-simplicial-sets}.

\begin{lemma}
\label{lemma-covering}
Let $\mathcal{C}$ be a site with fibre products.
Let $X$ be an object of $\mathcal{C}$.
Let $K$ be a hypercovering of $X$.
Let $k \geq 0$ be an integer.
Let $u : Z \to K_k$ be a covering in
in $\text{SR}(\mathcal{C}, X)$.
Then there exists a morphism of hypercoverings
$f: L \to K$ such that $L_k \to K_k$
factors through $u$.
\end{lemma}

\begin{proof}
Denote $Y = K_k$. There is a canonical morphism
$K \to \Hom(\Delta[k], Y)$ corresponding to
$\text{id}_Y$ via
Simplicial, Lemma \ref{simplicial-lemma-morphism-into-product}.
We will use the description of $\Hom(\Delta[k], Y)$
and $\Hom(\Delta[k], Z)$ given in that lemma. In particular
there is a morphism $\Hom(\Delta[k], Y) \to \Hom(\Delta[k], Z)$
which on degree $n$ terms is the morphism
$$
\prod\nolimits_{\alpha : [k] \to [n]} Y
\longrightarrow
\prod\nolimits_{\alpha : [k] \to [n]} Z.
$$
Set
$$
L =
K
\times_{\Hom(\Delta[n], Y)}
\Hom(\Delta[n], Z).
$$
The morphism $L_k \to K_k$ sits in to a commutative diagram
$$
\xymatrix{
L_k \ar[r] \ar[d] &
\prod_{\alpha : [k] \to [n]} Y \ar[r]^-{\text{pr}_{\text{id}_{[k]}}} \ar[d] &
Y \ar[d] \\
K_k \ar[r] &
\prod_{\alpha : [k] \to [n]} Z \ar[r]^-{\text{pr}_{\text{id}_{[k]}}} &
Z
}
$$
Since the composition of the two bottom arrows is the identity
we conclude that we have the desired factorization.

\medskip\noindent
We still have to show that $L$ is a hypercovering of $X$.
To see this we will use Lemma \ref{lemma-funny-gamma}.
Condition (1) is satisfied by assumption.
For (2), the morphism
$$
\Hom(\Delta[k], Y)_0 \to \Hom(\Delta[k], Z)_0
$$
is a covering because it is a product of coverings,
see Lemma \ref{lemma-covering-permanence}. For (3)
suppose first that $n \geq 1$. In this case by
Simplicial, Lemma \ref{simplicial-lemma-cosk-hom-deltak}
we have
$\Hom(\Delta[k], Y) =
\text{cosk}_n \text{sk}_n \Hom(\Delta[k], Y)$
and similarly for $Z$. Thus condition (3) for $n > 0$
is clear. For $n = 0$, the diagram of condition
(3) of Lemma \ref{lemma-funny-gamma} is,
according to Simplicial, Lemma \ref{simplicial-lemma-cosk0-hom-deltak},
the diagram
$$
\xymatrix{
\prod\nolimits_{\alpha : [k] \to [1]} Z \ar[r] \ar[d] &
Z \times Z \ar[d] \\
\prod\nolimits_{\alpha : [k] \to [1]} Y \ar[r] &
Y \times Y
}
$$
with obvious horizontal arrows. Thus the morphism $\gamma$
is the morphism
$$
\prod\nolimits_{\alpha : [k] \to [1]} Z
\longrightarrow
\prod\nolimits_{\alpha : [k] \to [1]\text{ not onto}} Z
\times
\prod\nolimits_{\alpha : [k] \to [1]\text{ onto}} Y
$$
which is a product of coverings and hence a covering
according to Lemma \ref{lemma-funny-gamma} once again.
\end{proof}

\begin{lemma}
\label{lemma-covering-sheaf}
Let $\mathcal{C}$ be a site with fibre products.
Let $X$ be an object of $\mathcal{C}$.
Let $K$ be a hypercovering of $X$.
Let $n \geq 0$ be an integer.
Let $u : \mathcal{F} \to F(K_n)$ be a morphism
of presheaves which becomes surjective
on sheafification.
Then there exists a morphism of hypercoverings
$f: L \to K$ such that $F(f_n) : F(L_n) \to F(K_n)$
factors through $u$.
\end{lemma}

\begin{proof}
Write $K_n = \{U_i \to X\}_{i \in I}$.
Thus the map $u$ is a morphism of presheaves of sets
$u : \mathcal{F} \to \amalg h_{u_i}$.
The assumption on $u$ means that for every
$i \in I$ there exists a covering $\{U_{ij} \to U_i\}_{j \in I_i}$
of the site $\mathcal{C}$ and a morphism of presheaves
$t_{ij} : h_{U_{ij}} \to \mathcal{F}$ such that
$u \circ t_{ij}$ is the map $h_{U_{ij}} \to h_{U_i}$
coming from the morphism $U_{ij} \to U_i$.
Set $J = \amalg_{i \in I} I_i$, and let
$\alpha : J \to I$ be the obvious map.
For $j \in J$ denote $V_j = U_{\alpha(j)j}$. Set
$Z = \{V_j \to X\}_{j \in J}$.
Finally, consider the morphism
$u' : Z \to K_n$ given by $\alpha : J \to I$
and the morphisms $V_j = U_{\alpha(j)j} \to U_{\alpha(j)}$
above. Clearly, this is a covering in the
category $\text{SR}(\mathcal{C}, X)$, and by
construction $F(u') : F(Z) \to F(K_n)$ factors through $u$.
Thus the result follows from Lemma \ref{lemma-covering} above.
\end{proof}


\section{Adding simplices}
\label{section-adding-simplices}

\noindent
In this section we prove some technical lemmas which we will need later.
Let $\mathcal{C}$ be a site with fibre products.
Let $X$ be an object of $\mathcal{C}$.
As we pointed out in Section \ref{section-covering} above,
the objects $U \times K$ and $\Hom(U, K)$
for certain simplicial sets $U$
and any simplicial object $K$ of $\text{SR}(\mathcal{C}, X)$
are defined. See Simplicial, Sections
\ref{simplicial-section-product-with-simplicial-sets} and
\ref{simplicial-section-hom-from-simplicial-sets}.

\begin{lemma}
\label{lemma-one-more-simplex}
Let $\mathcal{C}$ be a site with fibre products.
Let $X$ be an object of $\mathcal{C}$.
Let $K$ be a hypercovering of $X$.
Let $U \subset V$ be simplicial sets, with $U_n, V_n$
finite nonempty for all $n$.
Assume that $U$ has finitely many nondegenerate simplices.
Suppose $n \geq 0$ and $x \in V_n$,
$x \not \in U_n$ are such that
\begin{enumerate}
\item $V_i = U_i$ for $i < n$,
\item $V_n = U_n \cup \{x\}$,
\item any $z \in V_j$, $z \not \in U_j$ for $j > n$
is degenerate.
\end{enumerate}
Then the morphism
$$
\Hom(V, K)_0
\longrightarrow
\Hom(U, K)_0
$$
of $\text{SR}(\mathcal{C}, X)$ is a covering.
\end{lemma}

\begin{proof}
If $n = 0$, then it follows easily that $V = U \amalg \Delta[0]$
(see below). In this case $\Hom(V, K)_0 =
\Hom(U, K)_0 \times K_0$. The result, in this case, then follows
from Lemma \ref{lemma-covering-permanence}.

\medskip\noindent
Let $a : \Delta[n] \to V$ be the morphism associated to $x$
as in Simplicial, Lemma \ref{simplicial-lemma-simplex-map}.
Let us write $\partial \Delta[n] = i_{(n-1)!} \text{sk}_{n - 1} \Delta[n]$
for the $(n - 1)$-skeleton of $\Delta[n]$.
Let $b : \partial \Delta[n] \to U$ be the restriction
of $a$ to the $(n - 1)$ skeleton of $\Delta[n]$.
By
Simplicial, Lemma
\ref{simplicial-lemma-glue-simlex}
we have $V = U \amalg_{\partial \Delta[n]} \Delta[n]$. By
Simplicial, Lemma
\ref{simplicial-lemma-hom-from-coprod}
we get that
$$
\xymatrix{
\Hom(V, K)_0 \ar[r] \ar[d] &
\Hom(U, K)_0 \ar[d] \\
\Hom(\Delta[n], K)_0 \ar[r] &
\Hom(\partial \Delta[n], K)_0
}
$$
is a fibre product square. Thus it suffices to show that
the bottom horizontal arrow is a covering. By
Simplicial, Lemma \ref{simplicial-lemma-cosk-shriek}
this arrow is identified with
$$
K_n \to (\text{cosk}_{n - 1} \text{sk}_{n - 1} K)_n
$$
and hence is a covering by definition of a hypercovering.
\end{proof}

\begin{lemma}
\label{lemma-add-simplices}
Let $\mathcal{C}$ be a site with fibre products.
Let $X$ be an object of $\mathcal{C}$.
Let $K$ be a hypercovering of $X$.
Let $U \subset V$ be simplicial sets, with $U_n, V_n$
finite nonempty for all $n$.
Assume that $U$ and $V$ have finitely many nondegenerate simplices.
Then the morphism
$$
\Hom(V, K)_0
\longrightarrow
\Hom(U, K)_0
$$
of $\text{SR}(\mathcal{C}, X)$ is a covering.
\end{lemma}

\begin{proof}
By Lemma \ref{lemma-one-more-simplex}
above, it suffices to prove a simple lemma
about inclusions of simplicial sets $U \subset V$ as in the
lemma. And this is exactly the result of
Simplicial, Lemma \ref{simplicial-lemma-add-simplices}.
\end{proof}




\section{Homotopies}
\label{section-homotopies}

\noindent
Let $\mathcal{C}$ be a site with fibre products.
Let $X$ be an object of $\mathcal{C}$.
Let $L$ be a simplicial object of $\text{SR}(\mathcal{C}, X)$.
According to
Simplicial, Lemma \ref{simplicial-lemma-exists-hom-from-simplicial-set-finite}
there exists an object $\Hom(\Delta[1], L)$
in the category $\text{Simp}(\text{SR}(\mathcal{C}, X))$ which represents the
functor
$$
T
\longmapsto
\Mor_{\text{Simp}(\text{SR}(\mathcal{C}, X))}(\Delta[1] \times T, L)
$$
There is a canonical morphism
$$
\Hom(\Delta[1], L) \to L \times L
$$
coming from $e_i : \Delta[0] \to \Delta[1]$ and the identification
$\Hom(\Delta[0], L) = L$.

\begin{lemma}
\label{lemma-hom-hypercovering}
Let $\mathcal{C}$ be a site with fibre products.
Let $X$ be an object of $\mathcal{C}$.
Let $L$ be a simplicial object of $\text{SR}(\mathcal{C}, X)$.
Let $n \geq 0$. Consider the commutative diagram
\begin{equation}
\label{equation-diagram}
\xymatrix{
\Hom(\Delta[1], L)_{n + 1} \ar[r] \ar[d] &
(\text{cosk}_n \text{sk}_n \Hom(\Delta[1], L))_{n + 1} \ar[d] \\
(L \times L)_{n + 1} \ar[r] &
(\text{cosk}_n \text{sk}_n (L \times L))_{n + 1}
}
\end{equation}
coming from the morphism defined above.
We can identify the terms in this diagram as follows,
where
$\partial \Delta[n + 1] = i_{n!}\text{sk}_n \Delta[n + 1]$
is the $n$-skeleton of the $(n + 1)$-simplex:
\begin{eqnarray*}
\Hom(\Delta[1], L)_{n + 1}
& = &
\Hom(\Delta[1] \times \Delta[n + 1], L)_0 \\
(\text{cosk}_n \text{sk}_n \Hom(\Delta[1], L))_{n + 1}
& = &
\Hom(\Delta[1] \times \partial \Delta[n + 1], L)_0 \\
(L \times L)_{n + 1}
& = &
\Hom(
(\Delta[n + 1] \amalg \Delta[n + 1], L)_0 \\
(\text{cosk}_n \text{sk}_n (L \times L))_{n + 1}
& = &
\Hom(
\partial \Delta[n + 1]
\amalg
\partial \Delta[n + 1], L)_0
\end{eqnarray*}
and the morphism between these objects of $\text{SR}(\mathcal{C}, X)$
come from the commutative diagram of simplicial sets
\begin{equation}
\label{equation-dual-diagram}
\xymatrix{
\Delta[1] \times \Delta[n + 1] &
\Delta[1] \times \partial\Delta[n + 1] \ar[l] \\
\Delta[n + 1] \amalg \Delta[n + 1] \ar[u] &
\partial\Delta[n + 1] \amalg \partial\Delta[n + 1]
\ar[l] \ar[u]
}
\end{equation}
Moreover the fibre product of the bottom arrow and the
right arrow in (\ref{equation-diagram}) is equal to
$$
\Hom(U, L)_0
$$
where $U \subset \Delta[1] \times \Delta[n + 1]$
is the smallest simplicial subset such that both
$\Delta[n + 1] \amalg \Delta[n + 1]$ and
$\Delta[1] \times \partial\Delta[n + 1]$ map into it.
\end{lemma}

\begin{proof}
The first and third equalities are
Simplicial, Lemma \ref{simplicial-lemma-exists-hom-from-simplicial-set-finite}.
The second and fourth follow from the cited lemma combined with
Simplicial, Lemma \ref{simplicial-lemma-cosk-shriek}.
The last assertion follows from the fact that
$U$ is the push-out of the bottom and right arrow of the
diagram (\ref{equation-dual-diagram}), via
Simplicial, Lemma \ref{simplicial-lemma-hom-from-coprod}.
To see that $U$ is equal to this push-out it suffices
to see that the intersection of
$\Delta[n + 1] \amalg \Delta[n + 1]$ and
$\Delta[1] \times \partial\Delta[n + 1]$
in $\Delta[1] \times \Delta[n + 1]$ is equal to
$\partial\Delta[n + 1] \amalg \partial\Delta[n + 1]$.
This we leave to the reader.
\end{proof}

\begin{lemma}
\label{lemma-homotopy}
Let $\mathcal{C}$ be a site with fibre products.
Let $X$ be an object of $\mathcal{C}$.
Let $K, L$ be hypercoverings of $X$.
Let $a, b : K \to L$ be morphisms of hypercoverings.
There exists a morphism of hypercoverings
$c : K' \to K$ such that $a \circ c$ is homotopic
to $b \circ c$.
\end{lemma}

\begin{proof}
Consider the following commutative diagram
$$
\xymatrix{
K' \ar@{=}[r]^-{def} \ar[rd]_c &
K \times_{(L \times L)} \Hom(\Delta[1], L)
\ar[r] \ar[d] & \Hom(\Delta[1], L) \ar[d] \\
& K \ar[r]^{(a, b)} & L \times L
}
$$
By the functorial property of $\Hom(\Delta[1], L)$
the composition of the horizontal morphisms
corresponds to a morphism $K' \Delta[1] \to L$ which
defines a homotopy between $c \circ a$ and $c \circ b$.
Thus if we can show that $K'$ is a
hypercovering of $X$, then we obtain the lemma.
To see this we will apply Lemma \ref{lemma-funny-gamma}
to the pair of morphisms $K \to L \times L$
and $\Hom(\Delta[1], L) \to L\times L$.
Condition (1) of Lemma \ref{lemma-funny-gamma} is statisfied.
Condition (2) of Lemma \ref{lemma-funny-gamma} is true because
$\Hom(\Delta[1], L)_0 = L_1$, and the morphism
$(d^1_0, d^1_1) : L_1 \to L_0 \times L_0$ is a
covering of $\text{SR}(\mathcal{C}, X)$ by our
assumption that $L$ is a hypercovering.
To prove condition (3) of Lemma \ref{lemma-funny-gamma}
we use Lemma \ref{lemma-hom-hypercovering} above. According
to this lemma the morphism $\gamma$ of condition (3) of Lemma
\ref{lemma-funny-gamma} is the morphism
$$
\Hom(\Delta[1] \times \Delta[n + 1], L)_0
\longrightarrow
\Hom(U, L)_0
$$
where $U \subset \Delta[1] \times \Delta[n + 1]$.
According to Lemma \ref{lemma-add-simplices}
this is a covering and hence the claim has been proven.
\end{proof}

\begin{remark}
\label{remark-contractible-category}
Note that the crux of the proof is to use
Lemma \ref{lemma-add-simplices}. This lemma
is completely general and does not care about the
exact shape of the simplicial sets (as long as they
have only finitely many nondegenerate simplices).
It seems altogether reasonable to expect a result
of the following kind:
Given any morphism $a : K \times \partial \Delta[k]
\to L$, with $K$ and $L$ hypercoverings, there
exists a morphism of hypercoverings $c : K' \to K$
and a morphism  $g : K' \times \Delta[k] \to L$
such that
$g|_{K' \times \partial \Delta[k]} =
a \circ (c \times \text{id}_{\partial \Delta[k]})$.
In other words, the category of hypercoverings is in
a suitable sense contractible.
\end{remark}










\section{Cech cohomology associated to hypercoverings}
\label{section-hyper-cech}

\noindent
Let $\mathcal{C}$ be a site with fibre products.
Let $X$ be an object of $\mathcal{C}$.
Consider a presheaf of abelian groups
$\mathcal{F}$ on the site $\mathcal{C}$.
It defines a functor
\begin{eqnarray*}
\mathcal{F} : \text{SR}(\mathcal{C}, X)^{opp}
& \longrightarrow &
\textit{Ab} \\
\{U_i \to X\}_{i \in I} &
\longmapsto &
\prod\nolimits_{i \in I} \mathcal{F}(U_i)
\end{eqnarray*}
Thus a simplicial object $K$ of $\text{SR}(\mathcal{C}, X)$
is turned into a cosimplicial object $\mathcal{F}(K)$ of $\textit{Ab}$.
In this situation we define
$$
\check{H}^i(K, \mathcal{F})
=
H^i(s(\mathcal{F}(K))).
$$
Recall that $s(\mathcal{F}(K))$ is the cochain
complex associated to the cosimplicial abelian
group $\mathcal{F}(K)$, see Simplicial, Section
\ref{simplicial-section-dold-kan-cosimplicial}.
In this section we prove analogues of some of the results for
Cech cohomology of open coverings proved in
Cohomology, Sections \ref{cohomology-section-cech},
\ref{cohomology-section-cech-functor} and
\ref{cohomology-section-cech-cohomology-cohomology}.

\begin{lemma}
\label{lemma-h0-cech}
Let $\mathcal{C}$ be a site with fibre products.
Let $X$ be an object of $\mathcal{C}$.
Let $K$ be a hypercovering of $X$.
Let $\mathcal{F}$ be a sheaf of abelian groups on $\mathcal{C}$.
Then $\check{H}^0(K, \mathcal{F}) = \mathcal{F}(X)$.
\end{lemma}

\begin{proof}
We have
$$
\check{H}^0(K, \mathcal{F})
=
\text{Ker}(\mathcal{F}(K_0) \longrightarrow \mathcal{F}(K_1))
$$
Write $K_0 = \{U_i \to X\}$. It is a covering in the site
$\mathcal{C}$. As well, we have that $K_1 \to K_0 \times K_0$
is a covering in $\text{SR}(\mathcal{C}, X)$. Hence we may
write $K_1 = \amalg_{i_0, i_1 \in I} \{V_{i_0i_1j} \to X\}$
so that the morphism $K_1 \to K_0 \times K_0$ is given
by coverings $\{V_{i_0i_1j} \to U_{i_0} \times_X U_{i_1}\}$
of the site $\mathcal{C}$. Thus we can further identify
$$
\check{H}^0(K, \mathcal{F})
=
\text{Ker}(
\prod\nolimits_{i} \mathcal{F}(U_i)
\longrightarrow
\prod\nolimits_{i_0i_1 j} \mathcal{F}(V_{i_0i_1j})
)
$$
with obvious map. The sheaf property of $\mathcal{F}$
implies that $\check{H}^0(K, \mathcal{F}) = H^0(X, \mathcal{F})$.
\end{proof}

\noindent
In fact this property characterizes the abelian sheaves among all
abelian presheaves on $\mathcal{C}$ of course.
The analogue of Cohomology, Lemma \ref{lemma-injective-trivial-cech}
in this case is the following.

\begin{lemma}
\label{lemma-injective-trivial-cech}
Let $\mathcal{C}$ be a site with fibre products.
Let $X$ be an object of $\mathcal{C}$.
Let $K$ be a hypercovering of $X$.
Let $\mathcal{I}$ be an injective sheaf of abelian groups on $\mathcal{C}$.
Then
$$
\check{H}^p(K, \mathcal{I}) =
\left\{
\begin{matrix}
\mathcal{I}(X) & \text{if} & p = 0 \\
0 & \text{if} & p > 0
\end{matrix}
\right.
$$
\end{lemma}

\begin{proof}
Observe that for any object $Z = \{U_i \to X\}$ of
$\text{SR}(\mathcal{C}, X)$ and any abelian sheaf
$\mathcal{F}$ on $\mathcal{C}$ we have
\begin{eqnarray*}
\mathcal{F}(Z)
& = &
\prod \mathcal{F}(U_i) \\
& = &
\prod \Mor_{\textit{PSh}(\mathcal{C})}(h_{U_i}, \mathcal{F})\\
& = &
\Mor_{\textit{PSh}(\mathcal{C})}(F(Z), \mathcal{F})\\
& = &
\Mor_{\textit{PAb}(\mathcal{C})}(\mathbf{Z}_{F(Z)}, \mathcal{F}) \\
& = &
\Mor_{\textit{Ab}(\mathcal{C})}(\mathbf{Z}_{F(Z)}^\#, \mathcal{F})
\end{eqnarray*}
Thus we see, for any simplicial object $K$ of
$\text{SR}(\mathcal{C}, X)$ that we have
\begin{equation}
\label{equation-identify-cech}
s(\mathcal{F}(K))
=
\Hom_{\textit{Ab}(\mathcal{C})}(s(\mathbf{Z}_K^\#), \mathcal{F})
\end{equation}
see Definition \ref{definition-homology} for notation.
Now, we know that $s(\mathbf{Z}_K^\#)$ is quasi-isomorphic
to $\mathbf{Z}_X^\#$ if $K$ is a hypercovering, see
Lemma \ref{lemma-hypercovering-acyclic}. We conclude
that if $\mathcal{I}$ is an injective abelian sheaf, and
$K$ a hypercovering, then the complex $s(\mathcal{I}(K))$
is acyclic except possibly in degree $0$.
In other words, we have
$$
\check{H}^i(K, \mathcal{I}) = 0
$$
for $i > 0$. Combined with Lemma \ref{lemma-h0-cech} the lemma is proved.
\end{proof}

\noindent
Next we come to the analogue of
Cohomology, Lemma \ref{lemma-cech-spectral-sequence}.
To state it we need to introduce a little more notation.
Let $\mathcal{C}$ be a site with fibre products.
Let $\mathcal{F}$ be a sheaf of abelian groups on $\mathcal{C}$.
The symbol $\underline{H}^i(\mathcal{F})$ indicates the presheaf
of abelian groups on $\mathcal{C}$ which is defined by the
rule
$$
\underline{H}^i(\mathcal{F}) : U \longmapsto H^i(U, \mathcal{F})
$$
where $U$ ranges over the objects of $\mathcal{C}$.

\begin{lemma}
\label{lemma-cech-spectral-sequence}
Let $\mathcal{C}$ be a site with fibre products.
Let $X$ be an object of $\mathcal{C}$.
Let $K$ be a hypercovering of $X$.
Let $\mathcal{F}$ be a sheaf of abelian groups on $\mathcal{C}$.
There is a map
$$
s(\mathcal{F}(K))
\longrightarrow
R\Gamma(X, \mathcal{F})
$$
in $D^{+}(\textit{Ab})$ functorial in $\mathcal{F}$, which induces
natural transformations
$$
\check{H}^i(K, -) \longrightarrow H^i(X, -)
$$
as functors $\textit{Ab}(\mathcal{C}) \to \textit{Ab}$. Moreover,
there is a spectral sequence $(E_r, d_r)_{r \geq 0}$ with
$$
E_2^{p, q} = \check{H}^p(K, \underline{H}^q(\mathcal{F}))
$$
converging to $H^{p + q}(X, \mathcal{F})$.
This spectral sequence is functorial in $\mathcal{F}$ and
in the hypercovering $K$.
\end{lemma}

\begin{proof}
We could prove this by the same method as employed in the corresponding
lemma in the chapter on cohomology. Instead let us prove this by a
double complex argument.

\medskip\noindent
Choose an injective resolution $\mathcal{F} \to \mathcal{I}^\bullet$
in the category of abelian sheaves on $\mathcal{C}$. Consider the
double complex $A^{\bullet, \bullet}$ with terms
$$
A^{p, q} = \mathcal{I}^q(K_p)
$$
where the differential $d_1^{p, q} : A^{p, q} \to A^{p + 1, q}$
is the one coming from the differential $\mathcal{I}^p \to \mathcal{I}^{p + 1}$
and the differential $d_2^{p, q} : A^{p, q} \to A^{p, q + 1}$ is the
one coming from the differential on the complex
$s(\mathcal{I}^p(K))$ associated to the cosimplicial abelian group
$\mathcal{I}^p(K)$ as explained above.
As usual we denote $sA^\bullet$ the simple complex associated to
the double complex $A^{\bullet, \bullet}$.
We will use the two spectral
sequences $({}'E_r, {}'d_r)$ and $({}''E_r, {}''d_r)$
associated to this double complex, see
Homology, Section \ref{homology-section-double-complex}.

\medskip\noindent
By Lemma \ref{lemma-injective-trivial-cech}
the complexes $s(\mathcal{I}^p(K))$ are acyclic in
positive degrees and have $H^0$ equal to $\mathcal{I}^p(X)$.
Hence by
Homology, Lemma \ref{homology-lemma-double-complex-gives-resolution}
and its proof the spectral sequence $({}'E_r, {}'d_r)$ degenerates,
and the natural map
$$
\mathcal{I}^\bullet(X) \longrightarrow sA^\bullet
$$
is a quasi-isomorphism of complexes of abelian groups. In particular
we conclude that $H^n(sA^\bullet) = H^n(X, \mathcal{F})$.

\medskip\noindent
The map $s(\mathcal{F}(K)) \longrightarrow R\Gamma(X, \mathcal{F})$ of
the lemma is the composition of the natural map
$s(\mathcal{F}(K)) \to sA^\bullet$ followed by the inverse
of the displayed quasi-isomorphism above. This works because
$\mathcal{I}^\bullet(X)$ is a representative of $R\Gamma(X, \mathcal{F})$.

\medskip\noindent
Consider the spectral sequence $({}''E_r, {}''d_r)_{r \geq 0}$. By
Homology, Lemma \ref{homology-lemma-ss-double-complex}
we see that
$$
{}''E_2^{p, q} = H^p_{II}(H^q_{I}(A^{\bullet, \bullet}))
$$
In other words, we first take cohomology with respect to
$d_1$ which gives the groups
${}''E_1^{p, q} = \underline{H}^p(\mathcal{F})(K_q)$.
Hence it is indeed the case (by the description of the differential
${}''d_1$) that
${}''E_2^{p, q} = \check{H}^p(K, \underline{H}^q(\mathcal{F}))$. 
And by the other spectral sequence above we see that this one
converges to $H^n(X, \mathcal{F})$ as desired.

\medskip\noindent
We omit the proof of the statements regarding the functoriality of
the above constructions in the abelian sheaf $\mathcal{F}$ and the
hypercovering $K$.
\end{proof}


























\section{Cohomology and hypercoverings}
\label{section-cohomology}

\noindent
Let $\mathcal{C}$ be a site with fibre products.
Let $X$ be an object of $\mathcal{C}$.
Let $\mathcal{F}$ be a sheaf of abelian groups on $\mathcal{C}$.
Let $K, L$ be hypercoverings of $X$.
If $a, b : K \to L$ are homotopic maps,
then $\mathcal{F}(a), \mathcal{F}(b) : \mathcal{F}(K) \to \mathcal{F}(L)$
are homotopic maps, see
Simplicial, Lemma \ref{simplicial-lemma-functorial-homotopy}.
Hence have the same effect on cohomology groups of the associated
cochain complexes, see
Simplicial, Lemma \ref{simplicial-lemma-homotopy-s-Q}.
We are going to use this to define the colimit over all
hypercoverings.

\medskip\noindent
Let us temporarily denote $\text{HC}(\mathcal{C}, X)$
the category of hypercoverings of $X$. We have seen that
this is a category and not a ``big'' category,
see Lemma \ref{lemma-hypercoverings-set}.
This will be the index category for our diagram, see
Categories, Section \ref{categories-section-limits} for notation.
Consider the diagram
$$
\check{H}^i(-, \mathcal{F}) :
\text{HC}(\mathcal{C}, X)
\longrightarrow
\textit{Ab}.
$$
By Lemma \ref{lemma-product-hypercoverings} and
Lemma \ref{lemma-homotopy}, and the remark on homotopies above,
this diagram is directed, see
Categories, Definition \ref{categories-definition-directed}.
Thus the colimit
$$
\check{H}^i_{\text{HC}}(X, \mathcal{F})
=
\colim_{K \in \text{HC}(\mathcal{C}, X)}
\check{H}^i(K, \mathcal{F})
$$
has a particularly simple discription (see location cited).

\begin{theorem}
\label{theorem-cohomology-hypercoverings}
Let $\mathcal{C}$ be a site with fibre products.
Let $X$ be an object of $\mathcal{C}$. Let $i \geq 0$.
The functors
\begin{eqnarray*}
\textit{Ab}(\mathcal{C}) & \longrightarrow & \textit{Ab} \\
\mathcal{F} & \longmapsto & H^i(X, \mathcal{F}) \\
\mathcal{F} & \longmapsto & \check{H}^i_{\text{HC}}(X, \mathcal{F})
\end{eqnarray*}
are canonically isomorphic.
\end{theorem}

\begin{proof}[Proof using spectral sequences.]
Suppose that $\xi \in H^p(X, \mathcal{F})$ for some $p \geq 0$.
Let us show that $\xi$ is in the image of the map
$\check{H}^p(X, \mathcal{F}) \to H^p(X, \mathcal{F})$ of
Lemma \ref{lemma-cech-spectral-sequence}
for some hypercovering $K$ of $X$.

\medskip\noindent
This is true if $p = 0$ by Lemma \ref{lemma-h0-cech}.
If $p = 1$, choose a Cech hypercovering $K$ of $X$ as in
Example \ref{example-cech} starting with a covering
$K_0 = \{U_i \to X\}$ in the site $\mathcal{C}$ such that
$\xi|_{U_i} = 0$, see
Cohomology on Sites,
Lemma \ref{sites-cohomology-lemma-kill-cohomology-class-on-covering}.
It follows immediately from the spectral sequence
in Lemma \ref{lemma-cech-spectral-sequence} that $\xi$ comes
from an element of $\check{H}^1(K, \mathcal{F})$ in this case.
In general, choose any hypercovering $K$ of $X$ such
that $\xi$ maps to zero in $\underline{H}^p(\mathcal{F})(K_0)$
(using Example \ref{example-cech} and
Cohomology on Sites,
Lemma \ref{sites-cohomology-lemma-kill-cohomology-class-on-covering}
again).
By the spectral sequence of Lemma \ref{lemma-cech-spectral-sequence}
the obstruction for $\xi$ to come from an element of
$\check{H}^p(K, \mathcal{F})$ is a sequence of elements
$\xi_1, \ldots, \xi_{p - 1}$ with
$\xi_q \in \check{H}^{p - q}(K, \underline{H}^q(\mathcal{F}))$
(more precisely the images of the $\xi_q$ in certain subquotients
of these groups).

\medskip\noindent
We can inductively replace the hypercovering $K$ by refinements
such that the obstructions $\xi_1, \ldots, \xi_{p - 1}$ restrict to zero
(and not just the images
in the subquotients -- so no subtlety here). Indeed, suppose we have
already managed to reach the situation where
$\xi_{q + 1}, \ldots, \xi_{p - 1}$ are zero.
Note that $\xi_q \in \check{H}^{p - q}(K, \underline{H}^q(\mathcal{F}))$
is the class of some element
$$
\tilde \xi_q \in
\underline{H}^q(\mathcal{F})(K_{p - q}) =
\prod H^q(U_i, \mathcal{F})
$$
if $K_{p - q} = \{U_i \to X\}_{i \in I}$. Let $\xi_{q, i}$
be the component of $\tilde \xi_q$ in $H^q(U_i, \mathcal{F})$.
As $q \geq 1$ we can use
Cohomology on Sites,
Lemma \ref{sites-cohomology-lemma-kill-cohomology-class-on-covering}
yet again to choose coverings $\{U_{i, j} \to U_i\}$
of the site such that each restriction $\xi_{q, i}|_{U_{i, j}} = 0$.
Consider the object $Z = \{U_{i, j} \to X\}$ of the category
$\text{SR}(\mathcal{C}, X)$ and its obvious morphism
$u : Z \to K_{p - q}$. It is clear that $u$ is a covering, see
Definition \ref{definition-covering-SR}. By
Lemma \ref{lemma-covering} there
exists a morphism $L \to K$ of hypercoverings of $X$ such that
$L_{p - q} \to K_{p - q}$ factors through $u$. Then clearly the
image of $\xi_q$ in $\underline{H}^q(\mathcal{F})(L_{p - q})$.
is zero. Since the spectral sequence of
Lemma \ref{lemma-cech-spectral-sequence}
is functorial this means that after replacing $K$ by $L$ we reach the
situation where $\xi_q, \ldots, \xi_{p - 1}$ are all zero.
Continuing like this we end up with a hypercovering where they are all
zero and hence $\xi$ is in the image of the map
$\check{H}^p(X, \mathcal{F}) \to H^p(X, \mathcal{F})$.

\medskip\noindent
Suppose that $K$ is a hypercovering of $X$, that
$\xi \in \check{H}^p(K, \mathcal{F})$ and that the image of
$\xi$ under the map 
$\check{H}^p(X, \mathcal{F}) \to H^p(X, \mathcal{F})$ of
Lemma \ref{lemma-cech-spectral-sequence}
is zero. To finish the proof of the theorem we have to show that
there exists a morphism of hypercoverings $L \to K$ such that
$\xi$ restricts to zero in $\check{H}^p(L, \mathcal{F})$.
By the spectral sequence of Lemma \ref{lemma-cech-spectral-sequence}
the vanishing of the image of $\xi$ in $H^p(X, \mathcal{F})$
means that there exist elements $\xi_1, \ldots, \xi_{p - 2}$
with $\xi_q \in \check{H}^{p - 1 - q}(K, \underline{H}^q(\mathcal{F}))$
(more precisely the images of these in certain subquotients)
such that the images $d_{q + 1}^{p - 1 - q, q}\xi_q$ (in the spectral
sequence) add up to $\xi$. Hence by exacly the same mechanism as above
we can find a morphism of hypercoverings $L \to K$ such that
the restrictions of the elements $\xi_q$, $q = 1, \ldots, p - 2$
in $\check{H}^{p - 1 - q}(L, \underline{H}^q(\mathcal{F}))$ are zero.
Then it follows that $\xi$ is zero since the morphism $L \to K$
induces a morphism of spectral sequences according to
Lemma \ref{lemma-cech-spectral-sequence}.
\end{proof}

\begin{proof}[Proof without using spectral sequences.]
We have seen the result for $i = 0$, see Lemma \ref{lemma-h0-cech}.
We know that the functors $H^i(X, -)$ form a universal $\delta$-functor, see
Derived Categories, Lemma \ref{derived-lemma-higher-derived-functors}.
In order to prove the theorem it suffices to show that
the sequence of functors $\check{H}^i_{HC}(X, -)$ forms a
$\delta$-functor. Namely we know that Cech cohomology
is zero on injective sheaves (Lemma \ref{lemma-injective-trivial-cech})
and then we can apply
Homology, Lemma \ref{homology-lemma-efface-implies-universal}.

\medskip\noindent
Let
$$
0 \to \mathcal{F} \to \mathcal{G} \to \mathcal{H} \to 0
$$
be a short exact sequence of abelian sheaves on $\mathcal{C}$.
Let $\xi \in \check{H}^p_{HC}(X, \mathcal{H})$. Choose a hypercovering
$K$ of $X$ and an element $\sigma \in \mathcal{H}(K_p)$ representing
$\xi$ in cohomology. There is a corresponding exact sequence of
complexes
$$
0 \to s(\mathcal{F}(K)) \to s(\mathcal{G}(K)) \to s(\mathcal{H}(K))
$$
but we are not assured that there is a zero on the right also and this
is the only thing that
prevents us from defining $\delta(\xi)$ by a simple application of the
snake lemma. Recall that
$$
\mathcal{H}(K_p) = \prod \mathcal{H}(U_i)
$$
if $K_p = \{U_i \to X\}$. Let $\sigma =\prod \sigma_i$ with
$\sigma_i \in \mathcal{H}(U_i)$. Since $\mathcal{G} \to \mathcal{H}$ is
a surjection of sheaves we see that there exist coverings
$\{U_{i, j} \to U_i\}$ such that $\sigma_i|_{U_{i, j}}$ is the
image of some element $\tau_{i, j} \in \mathcal{G}(U_{i, j})$.
Consider the object $Z = \{U_{i, j} \to X\}$ of the category
$\text{SR}(\mathcal{C}, X)$ and its obvious morphism
$u : Z \to K_p$. It is clear that $u$ is a covering, see
Definition \ref{definition-covering-SR}. By
Lemma \ref{lemma-covering} there
exists a morphism $L \to K$ of hypercoverings of $X$ such that
$L_p \to K_p$ factors through $u$. After replacing $K$ by $L$
we may therefore assume that $\sigma$ is the image of an
element $\tau \in \mathcal{G}(K_p)$. Note that $d(\sigma) = 0$,
but not necessarily $d(\tau) = 0$. Thus $d(\tau) \in \mathcal{F}(K_{p + 1})$
is a cocycle. In this situation we define
$\delta(\xi)$ as the class of the cocycle $d(\tau)$ in
$\check{H}^{p + 1}_{HC}(X, \mathcal{F})$.

\medskip\noindent
At this point there are several things to verify:
(a) $\delta(\xi)$ does not depend on the choice of $\tau$,
(b) $\delta(\xi)$ does not depend on the choice of the hypercovering
$L \to K$ such that $\sigma$ lifts, and
(c) $\delta(\xi)$ does not depend on the initial hypercovering and
$\sigma$ chosen to represent $\xi$. We omit the verification of
(a), (b), and (c); the independence of the choices of the hypercoverings
really comes down to Lemmas \ref{lemma-product-hypercoverings}
and \ref{lemma-homotopy}. We also omit the verification that
$\delta$ is functorial with respect to morphisms of short exact
sequences of abelian sheaves on $\mathcal{C}$.

\medskip\noindent
Finally, we have to verify that with this definition of $\delta$
our short exact sequence of abelian sheaves above leads to a
long exact sequence of Cech cohomology groups.
First we show that if $\delta(\xi) = 0$ (with $\xi$ as above) then
$\xi$ is the image of some element
$\xi' \in \check{H}^p_{HC}(X, \mathcal{G})$.
Namely, if $\delta(\xi) = 0$, then, with notation as above, we
see that the class of $d(\tau)$ is zero in
$\check{H}^{p + 1}_{HC}(X, \mathcal{F})$. Hence there exists
a morphism of hypercoverings $L \to K$ such that the restriction
of $d(\tau)$ to an element of $\mathcal{F}(L_{p + 1})$ is
equal to $d(\upsilon)$ for some $\upsilon \in \mathcal{F}(L_p)$.
This implies that $\tau|_{L_p} + \upsilon$ form a
cocycle, and determine a class $\xi' \in \check{H}^p(L, \mathcal{G})$
which maps to $\xi$ as desired.

\medskip\noindent
We omit the proof that if $\xi' \in \check{H}^{p + 1}_{HC}(X, \mathcal{F})$
maps to zero in $\check{H}^{p + 1}_{HC}(X, \mathcal{G})$, then it is
equal to $\delta(\xi)$ for some $\xi \in \check{H}^p_{HC}(X, \mathcal{H})$.
\end{proof}







\section{Hypercoverings of spaces}
\label{section-hypercoverings-spaces}

\noindent
The theory above is mildly interesting even in the case of topological
spaces. In this case we can work out what is a hypercovering and see
what the result actually says.

\medskip\noindent
Let $X$ be a topological space. Consider the site $\mathcal{T}_X$
of Sites, Example \ref{sites-example-site-topological}. Recall that
an object of $\mathcal{T}_X$ is simply an open of $X$ and that morphisms
of $\mathcal{T}_X$ correspond simply to inclusions. So what is a
hypercovering of $X$ for the site $\mathcal{T}_X$?

\medskip\noindent
Let us first unwind Definition \ref{definition-SR}.
An object of $\text{SR}(\mathcal{C}, X)$ is simply given by a set
$I$ and for each $i \in I$ an open $U_i \subset X$.
Let us denote this by $\{U_i\}_{i \in I}$ since there can be no
confusion about the morphism $U_i \to X$.
A morphism $\{U_i\}_{i \in I} \to \{V_j\}_{j \in J}$
between two such objects is given by a map of sets
$\alpha : I \to J$ such that $U_i \subset V_{\alpha(i)}$ for all
$i \in I$. When is such a morphism a covering? This is the case
if and only if for every $j \in J$ we have
$V_j = \bigcup_{i\in I,\ \alpha(i) = j} U_i$ (and is
a covering in the site $\mathcal{T}_X$).

\medskip\noindent
Using the above we get the following description of a hypercovering
in the site $\mathcal{T}_X$. A hypercovering of $X$ in $\mathcal{T}_X$
is given by the following data
\begin{enumerate}
\item a simplicial set $I$ (see
Simplicial, Section \ref{simplicial-section-simplicial-set}), and
\item for each $n \geq 0$ and every $i \in I_n$ an open set $U_i \subset X$.
\end{enumerate}
We will denote such a collection of data by the notation $(I, \{U_i\})$.
In order for this to be a hypercovering of $X$ we require
the following properties
\begin{itemize}
\item for $i \in I_n$ and $0 \leq a \leq n + 1$
we have $U_i \subset U_{d^n_a(i)}$,
\item for $i \in I_n$ and $0 \leq a \leq n$ we have $U_i = U_{s^n_a(i)}$,
\item we have
\begin{equation}
\label{equation-covering-X}
X = \bigcup\nolimits_{i \in I_0} U_i,
\end{equation}
\item for every $i_0, i_1 \in I_0$, we have
\begin{equation}
\label{equation-covering-two}
U_{i_0} \cap U_{i_1} =
\bigcup\nolimits_{i \in I_1,\ d^1_0(i) = i_0,\ d^1_1(i) = i_1} U_i,
\end{equation}
\item for every $n \geq 1$ and every
$(i_0, \ldots, i_{n + 1}) \in (I_n)^{n + 2}$ such that
$d^n_{b - 1}(i_a) = d^n_a(i_b)$ for all $0\leq a < b\leq n + 1$
we have
\begin{equation}
\label{equation-covering-general}
U_{i_0} \cap \ldots \cap U_{i_{n + 1}} =
\bigcup\nolimits_{i \in I_{n + 1},
\ d^{n + 1}_a(i) = i_a,\ a = 0, \ldots, n + 1} U_i,
\end{equation}
\item each of the open coverings (\ref{equation-covering-X}),
(\ref{equation-covering-two}), and (\ref{equation-covering-general})
is an element of $\text{Cov}(\mathcal{T}_X)$
(this is a set theoretic condition, bounding
the size of the index sets of the coverings).
\end{itemize}
Condititions (\ref{equation-covering-X}) and
(\ref{equation-covering-two}) should be familiar from the
chapter on sheaves on spaces for example, and condition
(\ref{equation-covering-general}) is the natural generalization.

\begin{remark}
\label{remark-not-covering-set}
One feature of this description is that if one of the multiple
intersections $U_{i_0} \cap \ldots \cap U_{i_{n + 1}}$ is empty then
the covering on the right hand side may be the empty covering.
Thus it is not automatically the case that the maps
$I_{n + 1} \to (\text{cosk}_n\text{sk}_n I)_{n + 1}$ are surjective.
This means that the geometric realization of $I$ may be an interesting
(non-contractible) space.

\medskip\noindent
In fact, let $I'_n \subset I_n$ be the subset
consisting of those simplices $i \in I_n$ such that
$U_i \not = \emptyset$. It is easy to see that $I' \subset I$
is a subsimplicial set, and that $(I', \{U_i\})$ is a hypercovering.
Hence we can always refine a hypercovering to a hypercovering where
none of the opens $U_i$ is empty.
\end{remark}

\begin{remark}
\label{remark-repackage-into-simplicial-space}
Let us repackage this information in yet another way.
Namely, suppose that $(I, \{U_i\})$ is a hypercovering of
the topological space $X$. Given this data we can construct
a simplicial toplogical space $U_\bullet$ by setting
$$
U_n = \coprod\nolimits_{i \in I_n} U_i,
$$
and where for given $\varphi : [n] \to [m]$ we let
morphisms $U(\varphi) : U_n \to U_m$ be the morphism
coming from the inclusions $U_i \subset U_{\varphi(i)}$
for $i \in I_n$. This simplicial topological space comes
with an augmentation $\epsilon : U_\bullet \to X$.
With this morphism the simplicial space $U_\bullet$ becomes
a hypercovering of $X$ along which one has cohomological descent
in the sense of \cite[Expos\'e Vbis]{SGA4}.
In other words, $H^n(U_\bullet, \epsilon^*\mathcal{F}) = H^n(X, \mathcal{F})$.
(Insert future reference here to cohomology over simplicial
spaces and cohomological descent formulated in those terms.)
Suppose that $\mathcal{F}$ is an abelian sheaf on $X$.
In this case the spectral sequence of Lemma \ref{lemma-cech-spectral-sequence}
becomes the spectral sequence with $E_1$-term
$$
E_1^{p, q} = H^q(U_p, \epsilon_q^*\mathcal{F})
\Rightarrow
H^{p + q}(U_\bullet, \epsilon^*\mathcal{F}) = H^{p + q}(X, \mathcal{F})
$$
comparing the total cohomology of $\epsilon^*\mathcal{F}$
to the cohomology groups of $\mathcal{F}$ over the pieces
of $U_\bullet$. (Insert future reference to this spectral sequence
here.)
\end{remark}

\noindent
In topology we often want to find hypercoverings of $X$ which
have the property that all the $U_i$ come from a given basis for the topology
of $X$ and that all the coverings 
(\ref{equation-covering-two}) and (\ref{equation-covering-general})
are from a given cofinal collection of coverings.
Here are two example lemmas.

\begin{lemma}
\label{lemma-basis-hypercovering}
Let $X$ be a topological space.
Let $\mathcal{B}$ be a basis for the topology of $X$.
There exists a hypercovering $(I, \{U_i\})$ of $X$
such that each $U_i$ is an element of $\mathcal{B}$.
\end{lemma}

\begin{proof}
Let $n \geq 0$.
Let us say that an {\it $n$-truncated hypercovering} of $X$ is
given by an $n$-truncated simplicial set $I$ and for each
$i \in I_a$, $0 \leq a \leq n$ an open $U_i$ of $X$ such that
the conditions defining a hypercovering hold whenever they make sense.
In other words we require the inclusion relations and covering
conditions only when all simplices that occur in them
are $a$-simplices with $a \leq n$. The lemma follows if we can prove
that given a $n$-truncated hypercovering $(I, \{U_i\})$ with
all $U_i \in \mathcal{B}$ we can extend it to an $(n + 1)$-truncated
hypercovering without adding any $a$-simplices for $a \leq n$.
This we do as follows. First we consider the $(n + 1)$-truncated
simplicial set $I'$ defined by
$I' = \text{sk}_{n + 1}(\text{cosk}_n I)$.
Recall that
$$
I'_{n + 1} =
\left\{
\begin{matrix}
(i_0, \ldots, i_{n + 1}) \in (I_n)^{n + 2} \text{ such that}\\
d^n_{b - 1}(i_a) = d^n_a(i_b) \text{ for all }0\leq a < b\leq n + 1
\end{matrix}
\right\}
$$
If $i' \in I'_{n + 1}$ is degenerate, say $i' = s^n_a(i)$ then we set
$U_{i'} = U_i$ (this is forced on us anyway by the second condition).
We also set $J_{i'} = \{i'\}$ in this case
If $i' \in I'_{n + 1}$ is nondegerate, and say
$i' = (i_0, \ldots, i_{n + 1})$, then we choose a set
$J_{i'}$ and an open covering
$$
U_{i_0} \cap \ldots \cap U_{i_{n + 1}} =
\bigcup\nolimits_{i \in J_{i'}} U_i,
$$
with $U_i \in \mathcal{B}$ for $i \in J_{i'}$.
Set
$$
I_{n + 1} = \coprod\nolimits_{i' \in I'_{n + 1}} J_{i'}
$$
There is a canonical map $\pi : I_{n + 1} \to I'_{n + 1}$ which is
a bijection over the set of degenerate simplices in $I'_{n + 1}$ by
construction.
For $i \in I_{n + 1}$ we define $d^{n + 1}_a(i) = d^{n + 1}_a(\pi(i))$.
For $i \in I_n$ we define $s^n_a(i) \in I_{n + 1}$ as the unique
simplex lying over the degenerate simplex $s^n_a(i) \in I'_{n + 1}$.
We omit the verification that this defines an $(n + 1)$-truncated
hypercovering of $X$.
\end{proof}

\begin{lemma}
\label{lemma-quasi-separated-quasi-compact-hypercovering}
Let $X$ be a topological space.
Let $\mathcal{B}$ be a basis for the topology of $X$.
Assume that
\begin{enumerate}
\item $X$ is quasi-compact,
\item each $U \in \mathcal{B}$ is quasi-compact open, and
\item the intersection of any two quasi-compact opens in
$X$ is quasi-compact.
\end{enumerate}
Then there exists a hypercovering $(I, \{U_i\})$ of $X$ with the
following properties
\begin{enumerate}
\item each $U_i$ is an element of the basis $\mathcal{B}$,
\item each of the $I_n$ is a finite set, and in particular
\item each of the coverings  (\ref{equation-covering-X}),
(\ref{equation-covering-two}), and (\ref{equation-covering-general})
is finite.
\end{enumerate}
\end{lemma}

\begin{proof}
This follows directly from the construction in the proof of
Lemma \ref{lemma-basis-hypercovering} if we choose at each stage
finite coverings by elements of $\mathcal{B}$. Details omitted.
\end{proof}



\section{Other chapters}

\begin{multicols}{2}
\begin{enumerate}
\item \hyperref[introduction-section-phantom]{Introduction}
\item \hyperref[conventions-section-phantom]{Conventions}
\item \hyperref[sets-section-phantom]{Set Theory}
\item \hyperref[categories-section-phantom]{Categories}
\item \hyperref[topology-section-phantom]{Topology}
\item \hyperref[sheaves-section-phantom]{Sheaves on Spaces}
\item \hyperref[algebra-section-phantom]{Commutative Algebra}
\item \hyperref[sites-section-phantom]{Sites and Sheaves}
\item \hyperref[homology-section-phantom]{Homological Algebra}
\item \hyperref[derived-section-phantom]{Derived Categories}
\item \hyperref[more-algebra-section-phantom]{More Algebra}
\item \hyperref[simplicial-section-phantom]{Simplicial Methods}
\item \hyperref[modules-section-phantom]{Sheaves of Modules}
\item \hyperref[sites-modules-section-phantom]{Modules on Sites}
\item \hyperref[injectives-section-phantom]{Injectives}
\item \hyperref[cohomology-section-phantom]{Cohomology of Sheaves}
\item \hyperref[sites-cohomology-section-phantom]{Cohomology on Sites}
\item \hyperref[hypercovering-section-phantom]{Hypercoverings}
\item \hyperref[schemes-section-phantom]{Schemes}
\item \hyperref[constructions-section-phantom]{Constructions of Schemes}
\item \hyperref[properties-section-phantom]{Properties of Schemes}
\item \hyperref[morphisms-section-phantom]{Morphisms of Schemes}
\item \hyperref[coherent-section-phantom]{Coherent Cohomology}
\item \hyperref[divisors-section-phantom]{Divisors}
\item \hyperref[limits-section-phantom]{Limits of Schemes}
\item \hyperref[varieties-section-phantom]{Varieties}
\item \hyperref[chow-section-phantom]{Chow Homology}
\item \hyperref[topologies-section-phantom]{Topologies on Schemes}
\item \hyperref[descent-section-phantom]{Descent}
\item \hyperref[more-morphisms-section-phantom]{More on Morphisms}
\item \hyperref[flat-section-phantom]{More on Flatness}
\item \hyperref[groupoids-section-phantom]{Groupoid Schemes}
\item \hyperref[more-groupoids-section-phantom]{More on Groupoid Schemes}
\item \hyperref[etale-section-phantom]{\'Etale Morphisms of Schemes}
\item \hyperref[etale-cohomology-section-phantom]{\'Etale Cohomology}
\item \hyperref[spaces-section-phantom]{Algebraic Spaces}
\item \hyperref[spaces-properties-section-phantom]{Properties of Algebraic Spaces}
\item \hyperref[spaces-morphisms-section-phantom]{Morphisms of Algebraic Spaces}
\item \hyperref[spaces-topologies-section-phantom]{Topologies on Algebraic Spaces}
\item \hyperref[spaces-descent-section-phantom]{Descent and Algebraic Spaces}
\item \hyperref[spaces-more-morphisms-section-phantom]{More on Morphisms of Spaces}
\item \hyperref[quot-section-phantom]{Quot and Hilbert Spaces}
\item \hyperref[stacks-section-phantom]{Stacks}
\item \hyperref[spaces-groupoids-section-phantom]{Groupoids in Algebraic Spaces}
\item \hyperref[spaces-more-groupoids-section-phantom]{More on Groupoids in Spaces}
\item \hyperref[bootstrap-section-phantom]{Bootstrap}
\item \hyperref[examples-stacks-section-phantom]{Examples of Stacks}
\item \hyperref[groupoids-quotients-section-phantom]{Quotients of Groupoids}
\item \hyperref[algebraic-section-phantom]{Algebraic Stacks}
\item \hyperref[criteria-section-phantom]{Criteria for Representability}
\item \hyperref[stacks-properties-section-phantom]{Properties of Algebraic Stacks}
\item \hyperref[stacks-morphisms-section-phantom]{Morphisms of Algebraic Stacks}
\item \hyperref[examples-section-phantom]{Examples}
\item \hyperref[exercises-section-phantom]{Exercises}
\item \hyperref[guide-section-phantom]{Guide to Literature}
\item \hyperref[desirables-section-phantom]{Desirables}
\item \hyperref[coding-section-phantom]{Coding Style}
\item \hyperref[fdl-section-phantom]{GNU Free Documentation License}
\item \hyperref[index-section-phantom]{Auto Generated Index}
\end{enumerate}
\end{multicols}


\bibliography{my}
\bibliographystyle{amsalpha}

\end{document}
