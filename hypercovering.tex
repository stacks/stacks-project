\IfFileExists{stacks-project.cls}{%
\documentclass{stacks-project}
}{%
\documentclass{amsart}
}

% The following AMS packages are automatically loaded with
% the amsart documentclass:
%\usepackage{amsmath}
%\usepackage{amssymb}
%\usepackage{amsthm}

% For dealing with references we use the comment environment
\usepackage{verbatim}
\newenvironment{reference}{\comment}{\endcomment}
%\newenvironment{reference}{}{}
\newenvironment{slogan}{\comment}{\endcomment}
\newenvironment{history}{\comment}{\endcomment}

% For commutative diagrams you can use
% \usepackage{amscd}
\usepackage[all]{xy}

% We use 2cell for 2-commutative diagrams.
\xyoption{2cell}
\UseAllTwocells

% To put source file link in headers.
% Change "template.tex" to "this_filename.tex"
% \usepackage{fancyhdr}
% \pagestyle{fancy}
% \lhead{}
% \chead{}
% \rhead{Source file: \url{template.tex}}
% \lfoot{}
% \cfoot{\thepage}
% \rfoot{}
% \renewcommand{\headrulewidth}{0pt}
% \renewcommand{\footrulewidth}{0pt}
% \renewcommand{\headheight}{12pt}

\usepackage{multicol}

% For cross-file-references
\usepackage{xr-hyper}

% Package for hypertext links:
\usepackage{hyperref}

% For any local file, say "hello.tex" you want to link to please
% use \externaldocument[hello-]{hello}
\externaldocument[introduction-]{introduction}
\externaldocument[conventions-]{conventions}
\externaldocument[sets-]{sets}
\externaldocument[categories-]{categories}
\externaldocument[topology-]{topology}
\externaldocument[sheaves-]{sheaves}
\externaldocument[sites-]{sites}
\externaldocument[stacks-]{stacks}
\externaldocument[fields-]{fields}
\externaldocument[algebra-]{algebra}
\externaldocument[brauer-]{brauer}
\externaldocument[homology-]{homology}
\externaldocument[derived-]{derived}
\externaldocument[simplicial-]{simplicial}
\externaldocument[more-algebra-]{more-algebra}
\externaldocument[smoothing-]{smoothing}
\externaldocument[modules-]{modules}
\externaldocument[sites-modules-]{sites-modules}
\externaldocument[injectives-]{injectives}
\externaldocument[cohomology-]{cohomology}
\externaldocument[sites-cohomology-]{sites-cohomology}
\externaldocument[dga-]{dga}
\externaldocument[dpa-]{dpa}
\externaldocument[hypercovering-]{hypercovering}
\externaldocument[schemes-]{schemes}
\externaldocument[constructions-]{constructions}
\externaldocument[properties-]{properties}
\externaldocument[morphisms-]{morphisms}
\externaldocument[coherent-]{coherent}
\externaldocument[divisors-]{divisors}
\externaldocument[limits-]{limits}
\externaldocument[varieties-]{varieties}
\externaldocument[topologies-]{topologies}
\externaldocument[descent-]{descent}
\externaldocument[perfect-]{perfect}
\externaldocument[more-morphisms-]{more-morphisms}
\externaldocument[flat-]{flat}
\externaldocument[groupoids-]{groupoids}
\externaldocument[more-groupoids-]{more-groupoids}
\externaldocument[etale-]{etale}
\externaldocument[chow-]{chow}
\externaldocument[intersection-]{intersection}
\externaldocument[pic-]{pic}
\externaldocument[adequate-]{adequate}
\externaldocument[dualizing-]{dualizing}
\externaldocument[duality-]{duality}
\externaldocument[discriminant-]{discriminant}
\externaldocument[local-cohomology-]{local-cohomology}
\externaldocument[curves-]{curves}
\externaldocument[resolve-]{resolve}
\externaldocument[models-]{models}
\externaldocument[pione-]{pione}
\externaldocument[etale-cohomology-]{etale-cohomology}
\externaldocument[proetale-]{proetale}
\externaldocument[crystalline-]{crystalline}
\externaldocument[spaces-]{spaces}
\externaldocument[spaces-properties-]{spaces-properties}
\externaldocument[spaces-morphisms-]{spaces-morphisms}
\externaldocument[decent-spaces-]{decent-spaces}
\externaldocument[spaces-cohomology-]{spaces-cohomology}
\externaldocument[spaces-limits-]{spaces-limits}
\externaldocument[spaces-divisors-]{spaces-divisors}
\externaldocument[spaces-over-fields-]{spaces-over-fields}
\externaldocument[spaces-topologies-]{spaces-topologies}
\externaldocument[spaces-descent-]{spaces-descent}
\externaldocument[spaces-perfect-]{spaces-perfect}
\externaldocument[spaces-more-morphisms-]{spaces-more-morphisms}
\externaldocument[spaces-flat-]{spaces-flat}
\externaldocument[spaces-groupoids-]{spaces-groupoids}
\externaldocument[spaces-more-groupoids-]{spaces-more-groupoids}
\externaldocument[bootstrap-]{bootstrap}
\externaldocument[spaces-pushouts-]{spaces-pushouts}
\externaldocument[groupoids-quotients-]{groupoids-quotients}
\externaldocument[spaces-more-cohomology-]{spaces-more-cohomology}
\externaldocument[spaces-simplicial-]{spaces-simplicial}
\externaldocument[formal-spaces-]{formal-spaces}
\externaldocument[restricted-]{restricted}
\externaldocument[spaces-resolve-]{spaces-resolve}
\externaldocument[formal-defos-]{formal-defos}
\externaldocument[defos-]{defos}
\externaldocument[cotangent-]{cotangent}
\externaldocument[examples-defos-]{examples-defos}
\externaldocument[algebraic-]{algebraic}
\externaldocument[examples-stacks-]{examples-stacks}
\externaldocument[stacks-sheaves-]{stacks-sheaves}
\externaldocument[criteria-]{criteria}
\externaldocument[artin-]{artin}
\externaldocument[quot-]{quot}
\externaldocument[stacks-properties-]{stacks-properties}
\externaldocument[stacks-morphisms-]{stacks-morphisms}
\externaldocument[stacks-limits-]{stacks-limits}
\externaldocument[stacks-cohomology-]{stacks-cohomology}
\externaldocument[stacks-perfect-]{stacks-perfect}
\externaldocument[stacks-introduction-]{stacks-introduction}
\externaldocument[stacks-more-morphisms-]{stacks-more-morphisms}
\externaldocument[stacks-geometry-]{stacks-geometry}
\externaldocument[moduli-]{moduli}
\externaldocument[moduli-curves-]{moduli-curves}
\externaldocument[examples-]{examples}
\externaldocument[exercises-]{exercises}
\externaldocument[guide-]{guide}
\externaldocument[desirables-]{desirables}
\externaldocument[coding-]{coding}
\externaldocument[obsolete-]{obsolete}
\externaldocument[fdl-]{fdl}
\externaldocument[index-]{index}

% Theorem environments.
%
\theoremstyle{plain}
\newtheorem{theorem}[subsection]{Theorem}
\newtheorem{proposition}[subsection]{Proposition}
\newtheorem{lemma}[subsection]{Lemma}

\theoremstyle{definition}
\newtheorem{definition}[subsection]{Definition}
\newtheorem{example}[subsection]{Example}
\newtheorem{exercise}[subsection]{Exercise}
\newtheorem{situation}[subsection]{Situation}

\theoremstyle{remark}
\newtheorem{remark}[subsection]{Remark}
\newtheorem{remarks}[subsection]{Remarks}

\numberwithin{equation}{subsection}

% Macros
%
\def\lim{\mathop{\rm lim}\nolimits}
\def\colim{\mathop{\rm colim}\nolimits}
\def\Spec{\mathop{\rm Spec}}
\def\Hom{\mathop{\rm Hom}\nolimits}
\def\Ext{\mathop{\rm Ext}\nolimits}
\def\SheafHom{\mathop{\mathcal{H}\!{\it om}}\nolimits}
\def\SheafExt{\mathop{\mathcal{E}\!{\it xt}}\nolimits}
\def\Sch{\textit{Sch}}
\def\Mor{\mathop{\rm Mor}\nolimits}
\def\Ob{\mathop{\rm Ob}\nolimits}
\def\Sh{\mathop{\textit{Sh}}\nolimits}
\def\NL{\mathop{N\!L}\nolimits}
\def\proetale{{pro\text{-}\acute{e}tale}}
\def\etale{{\acute{e}tale}}
\def\QCoh{\textit{QCoh}}
\def\Ker{\mathop{\rm Ker}}
\def\Im{\mathop{\rm Im}}
\def\Coker{\mathop{\rm Coker}}
\def\Coim{\mathop{\rm Coim}}

%
% Macros for moduli stacks/spaces
%
\def\QCohstack{\mathcal{QC}\!{\it oh}}
\def\Cohstack{\mathcal{C}\!{\it oh}}
\def\Spacesstack{\mathcal{S}\!{\it paces}}
\def\Quotfunctor{{\rm Quot}}
\def\Hilbfunctor{{\rm Hilb}}
\def\Curvesstack{\mathcal{C}\!{\it urves}}
\def\Polarizedstack{\mathcal{P}\!{\it olarized}}
\def\Complexesstack{\mathcal{C}\!{\it omplexes}}
% \Pic is the operator that assigns to X its picard group, usage \Pic(X)
% \Picardstack_{X/B} denotes the Picard stack of X over B
% \Picardfunctor_{X/B} denotes the Picard functor of X over B
\def\Pic{\mathop{\rm Pic}\nolimits}
\def\Picardstack{\mathcal{P}\!{\it ic}}
\def\Picardfunctor{{\rm Pic}}
\def\Deformationcategory{\mathcal{D}\!{\it ef}}


% OK, start here.
%
\begin{document}

\title{Hypercoverings}

%\begin{abstract}
%\end{abstract}

\maketitle

\tableofcontents

\section{Introduction}
\label{section-introduction}

\noindent
Hypercoverings can be used to compute cohomology of abelian sheaves on sites
without recourse to injective resolutions. See \cite[Expose V, Sec. 7]{SGA4}.
A nice manuscript on cohomological descent is the text by Brian Conrad,
see \url{http://www.math.lsa.umich.edu/~bdconrad/papers/hypercover.pdf}.
Probably it is useless to try to improve on Brian's article, so we look
at the question a little differently (more naively).

\section{The category $\Delta$}
\label{section-Delta}

\noindent
The category $\Delta$ is the category with
\begin{enumerate}
\item objects $[0], [1], [2], \ldots$ with
$[n] = \{0, 1, 2, \ldots, n\}$ and
\item morphism $[n] \to [m]$ is the set of nondecreasing
maps of sets $\{0, 1, 2, \ldots, n\} \to \{0, 1, 2, \ldots, m\}$.
\end{enumerate}
Here {\it nondecreasing} for a map $\varphi : [n] \to [m]$
means by definition that $\varphi(i) \geq \varphi(j)$ if $i \geq j$.
In other words, $\Delta$ is a category equivalent to the
``big'' category of finite totally ordered sets and nondecreasing maps.
There are exactly $n + 1$ morphisms $[0] \to [n]$ and
there is exactly $1$ morphism $[n] \to [0]$. There are
exactly $(n + 1)(n + 2)/2$ morphisms $[1] \to [n]$ and there are
exactly $n + 2$ morphisms $[n] \to [1]$. And so on and so forth.

\begin{definition}
\label{definition-face-degeneracy}
For any integer $n\geq 1$, and any $0\leq j \leq n$ we let $\delta^n_j : [n-1]
\to [n]$ denote the injective order preserving map skipping $j$. For any
integer $n\geq 0$, and any $0\leq j \leq n$ we denote $\sigma^n_j : [n+1] 
\to [n]$ the surjective order preserving map with 
$(\sigma^n_j)^{-1}(\{j\}) = \{j, j+1\}$.
\end{definition}

\begin{lemma}
\label{lemma-face-degeneracy}
Any morphism in $\Delta$ can be written as a composition
of an identity morphism, and the morphisms $\delta^n_j$ and $\sigma^n_j$.
\end{lemma}

\begin{proof}
Let $\varphi : [n] \to [m]$ be a morphism of $\Delta$.
If $j \not \in \text{Im}(\varphi)$, then we can write
$\varphi$ as $\delta^m_j \circ \psi$ for some morphism
$\psi : [n] \to [m - 1]$. If $\varphi(j) = \varphi(j + 1)$
then we can write $\varphi$ as $\psi \circ \sigma^{n - 1}_j$
for some morphism $\psi : [n - 1] \to [m]$.
The result follows because each replacement
as above lowers $n + m$ and hence at some point
$\varphi$ is both injective and surjective, hence
an identity morphism.
\end{proof}

\begin{lemma}
\label{lemma-relations-face-degeneracy}
The morphisms $\delta^n_j$ and $\sigma^n_j$ satisfy the following relations.
\begin{enumerate}
\item $\delta^{n + 1}_i \circ \delta^n_j =
\delta^{n + 1}_j \circ \delta^n_{i - 1}$
for $0 \leq j < i \leq n + 1$.
In other words the diagram
$$
\xymatrix{
& [n] \ar[rd]^{\delta^{n + 1}_i} & \\
[n - 1] \ar[ru]^{\delta^n_j} \ar[rd]_{\delta^n_{i - 1}} & &
[n + 1] \\
& [n] \ar[ru]_{\delta^{n + 1}_j} & 
}
$$
commutes.
\item $\sigma^{n - 1}_i \circ \delta^n_j =
\delta^{n - 1}_{j - 1} \circ \sigma^{n - 2}_i$
for $0 < i + 1 < j \leq n$.
In other words the diagram
$$
\xymatrix{
& [n] \ar[rd]^{\sigma^{n - 1}_i} & \\
[n - 1] \ar[ru]^{\delta^n_j} \ar[rd]_{\sigma^{n - 2}_i} & &
[n - 1] \\
& [n - 2] \ar[ru]_{\delta^{n - 1}_{j - 1}} & 
}
$$
commutes.
\item $\sigma^{n - 1}_j \circ \delta^n_j =
\text{id}_{[n - 1]}$
for $0 \leq j \leq n - 1$.
In other words the diagram
$$
\xymatrix{
& [n] \ar[rd]^{\sigma^{n - 1}_j} & \\
[n - 1] \ar[ru]^{\delta^n_j} \ar[rr]^{\text{id}_{[n - 1]}} & & [n - 1]
}
$$
commutes.
\item $\sigma^{n - 1}_{j - 1} \circ \delta^n_j =
\text{id}_{[n - 1]}$
for $1 \leq j \leq n$.
In other words the diagram
$$
\xymatrix{
& [n] \ar[rd]^{\sigma^{n - 1}_{j - 1}} & \\
[n - 1] \ar[ru]^{\delta^n_j} \ar[rr]_{\text{id}_{[n - 1]}} & & [n - 1]}
$$
commutes.
\item $\sigma^{n - 1}_i \circ \delta^n_j =
\delta^{n - 1}_j \circ \sigma^{n - 2}_{i + 1}$
for $i > j \geq 0$.
In other words the diagram
$$
\xymatrix{
& [n] \ar[rd]^{\sigma^{n - 1}_i} & \\
[n - 1] \ar[ru]^{\delta^n_j} \ar[rd]_{\sigma^{n - 2}_{i + 1}} & &
[n - 1] \\
& [n - 2] \ar[ru]_{\delta^{n - 1}_j} & 
}
$$
commutes.
\item $\sigma^{n - 1}_i \circ \sigma^n_j =
\sigma^{n - 1}_{j - 1} \circ \sigma^n_i$
for $0 \leq i < j \leq n$.
In other words the diagram
$$
\xymatrix{
& [n] \ar[rd]^{\sigma^{n - 1}_i} & \\
[n + 1] \ar[ru]^{\sigma^n_j} \ar[rd]_{\sigma^n_i} & &
[n - 1] \\
& [n] \ar[ru]_{\sigma^{n - 1}_{j - 1}} & 
}
$$
commutes.
\end{enumerate}
\end{lemma}

\begin{proof}
Omitted.
\end{proof}

\begin{lemma}
\label{lemma-face-degeneracy-category}
The category $\Delta$ is the universal category
with objects $[n]$, $n \geq 0$ and morphisms
$\delta^n_j$ and $\sigma^n_j$ such that (a) every morphism is
a composition of these morphisms, (b) the relations
listed in Lemma \ref{lemma-relations-face-degeneracy} are satisfied,
and (c) any relation among the morphisms is a consquence of
those relations.
\end{lemma}







\section{Simplicial objects}
\label{section-simplicial-object}

\begin{definition}
\label{definition-simplicial-object}
Let $\mathcal{C}$ be a category.
\begin{enumerate}
\item A {\it simplicial object $U$ of $\mathcal{C}$}
is a contravariant functor $U$ from $\Delta$ to
$\mathcal{C}$, in a formula:
$$
U : \Delta^{opp} \longrightarrow \mathcal{C}
$$
\item If $\mathcal{C}$ is the category of sets, then we call
$U$ a {\it simplicial set}.
\item If $\mathcal{C}$ is the category of abelian groups,
then we call $U$ a {\it simplicial abelian group}.
\item A {\it morphism of simplicial objects $U \to U'$}
is a transformation of functors.
\end{enumerate}
\end{definition}

\noindent
This means there are objects $U([0]), U([1]), U([2]), \ldots$
and morphisms $U(\varphi) : U([n]) \to U([m])$,
where $\varphi$ is any nondecreasing map $\varphi : [m] \to [n]$.

\medskip\noindent
In particular there is a unique morphism $U([0]) \to U([n])$ and there are
exactly $n + 1$ morphisms $U([n]) \to U([0])$ corresponding to
the $n + 1$ maps $[0] \to [n]$. Obviously we need some more notation
to be able to talk 
intelligently about these simplicial objects. We do this by considering
the morphisms we singled out in Section \ref{section-Delta} above.

\begin{lemma}
\label{lemma-characterize-simplicial-object}
Let $\mathcal{C}$ be a category.
\begin{enumerate}
\item Given a simplicial object $U$ in $\mathcal{C}$
we obtain a sequence of objects $U_n = U([n])$ endowed
with the morphisms $d^n_j = U(\delta^n_j) : U_n \to U_{n-1}$ and
$s^n_j = U(\sigma^n_j) : U_n \to U_{n + 1}$. These morphisms
satisfy the opposites of the relations displayed in
Lemma \ref{lemma-relations-face-degeneracy}.
\item Conversely, given a sequence of objects $U_n$ and morphisms
$d^n_j$, $s^n_j$ satisfying these relations there exists a unique
a simplicial object $U$ in $\mathcal{C}$ such that $U_n = U([n])$,
$d^n_j = U(\delta^n_j)$, and $s^n_j = U(\sigma^n_j)$.
\item A morphism between simplicial objects $U$ and $U'$
is given by a family of morphisms $U_n \to U'_n$ commuting
with the morphisms $d^n_j$ and $s^n_j$.
\end{enumerate}
\end{lemma}

\begin{proof}
This follows from Lemma \ref{lemma-face-degeneracy-category}.
\end{proof}

\noindent
We get a unique morphism $s^0_0 = U(\sigma^0_0) : U_0 \to U_1$ and
two morphisms $d^1_0 = U(\delta^1_0)$, and
$d^1_1 = U(\delta^1_1)$ which are morphisms $U_1 \to U_0$.
There are two morphisms $s^1_0 = U(\sigma^1_0)$, $s^1_1 = U(\sigma^1_1)$
which are morphsms $U_1 \to U_2$. Three morphisms
$d^2_0 = U(\delta^2_0)$, $d^2_1 = U(\delta^2_1)$, $d^2_2 = U(\delta^2_2)$
which are morphisms $U_3 \to U_2$. And so on.

\medskip\noindent
Pictorially we think of $U$ as follows:
$$
\xymatrix{
U_2
\ar@<2ex>[r]
\ar@<0ex>[r]
\ar@<-2ex>[r]
&
U_1 
\ar@<1ex>[r]
\ar@<-1ex>[r]
\ar@<1ex>[l]
\ar@<-1ex>[l]
&
U_0
\ar@<0ex>[l]
}
$$

\begin{example}
\label{example-constant-simplicial-object}
The simplest example is the {\it constant} simplicial object with
value $X \in \text{Ob}(\mathcal{C})$. In other words, $U_n=X$ and
all maps are $\text{id}_X$.
\end{example}

\begin{example}
\label{example-fibre-products-simplicial-object}
Suppose that $Y\to X$ is a morphism of $C$ such that all
the fibred products $Y\times_X Y \times_X \ldots \times_X Y$ exist.
Then we set $U_n$ equal to the $(n + 1)$-fold fibre product,
and we let $\varphi: [n] \to [m]$ correspond to the map
(on ``coordinates'')
$(y_0,\ldots, y_m) \mapsto (y_{\varphi(0)},\ldots, y_{\varphi(n)})$.
In other words, the map $U_0 = Y \to U_1 = Y\times_X Y$ is the
diagonal map. The two maps $U_1 = Y\times_X Y \to U_0 = Y$ are the
projection maps.
\end{example}

\noindent
Geometrically Example \ref{example-fibre-products-simplicial-object}
above is an important example. It tells us that it is a good
idea to think of the maps $d^n_j : U_{n + 1} \to U_n$
as projection maps (forgetting the $j$th component),
and to think of the maps $s^n_j : U_n \to U_{n + 1}$
as diagonal maps (repeating the $j$th coordinate).
We will return to this in the next section.



\section{Simplicial sets}
\label{section-simplicial-set}

\noindent
Let $U$ be a simplical set. It is a good idea to think of
$U_0$ as the {\it $0$-simplices}, the set $U_1$ as the
{\it $1$-simplices},
the set $U_2$ as the {\it $2$-simplices}, and so on.

\medskip\noindent
We think of the maps $s^n_j : U_n \to U_{n + 1}$ as
the map that associates to an $n$-simplex $A$ the degenerate
$(n + 1)$-simplex $B$ whose $(j, j + 1)$-edge is collapsed
to the vertex $j$ of $A$. We think of the map $d^n_j : U_n \to U_{n - 1}$
as the map that associates to an $n$-simplex $A$ one of the
faces, namely the face that omits the vertex $j$.
In this way it become possible to visualize the relations
among the maps $s^n_j$ and $d^n_j$ geometrically.

\begin{definition}
\label{definition-terminology-simplicial-sets}
Let $U$ be a simplicial set. 
We say {\it $x$ is a $n$-simplex of $U$} to signify that
$x$ is an element of $U_n$. We say that {\it $y$ is the $j$the
face of $x$} to signify that $d^n_jx = y$. We say that
{\it $z$ is the $j$th degeneracy of $x$} if $z = s^n_jx$.
A simplex is called {\it degenerate} if it is the degeneracy
of another simplex.
\end{definition}

\noindent
Here are a few fundamental examples.

\begin{example}
\label{example-simplex-simplicial-set}
For every $n \geq 0$ we denote $\Delta[n]$ the simplicial set
\begin{align*}
\Delta^{opp} & \longrightarrow \textit{Sets} \\
[k] & \longmapsto \text{Mor}_{\Delta}([k], [n])
\end{align*}
We leave it to the reader to verify the following statements.
Every $m$-simplex of $\Delta[n]$ with $m > n$ is degenerate.
There is a unique nondegenerate $n$-simplex of $\Delta[n]$,
namely $\text{id}_{[n]}$.
\end{example}

\begin{example-simplex-category}
Consider the category $\Delta/[n]$ of objects over $[n]$
in $\Delta$, see
Categories, Example \ref{categories-example-comma-category}.
There is a functor $\Delta/[n] \to \Delta$.
Note that the objects of $\Delta/[n]$ for the disjoint
union of the sets of $k$-simplices in $\Delta[n]$.
\end{example}




\begin{lemma}
\label{lemma-simplex-map}
Let $U$ be a simplicial set. Let $n \geq 0$ be an integer.
There is a canonical bijection
$$
\text{Mor}(\Delta[n], U)
\longrightarrow
U_n
$$
which maps a morphism $\varphi$ to the value of $\varphi$
on the unique nondegenerate $n$-simplex of $\Delta[n]$.
\end{lemma}

\begin{proof}
Omitted.
\end{proof}




\section{Skelet and coskelet functors}
\label{section-skelet}

\noindent
Let $\Delta_{\leq n}$ denote the full subcategory of
$\Delta$ with objects $[0], [1], [2], \ldots, [n]$.
Let $\mathcal{C}$ be a category.
An {\it $n$-truncated simplicial object of $\mathcal{C}$} 
is a contravariant functor from $\Delta_{\leq n}$ to
$\mathcal{C}$.

\medskip\noindent
Given a simplicial object $U$ of $\mathcal{C}$
the truncation $\text{sk}_n U$ is the restriction
of $U$ to the subcategory $\Delta_{\leq n}$.
This defines a {\it skelet functor}
$$
\text{sk}_n :
\text{Simp}(\mathcal{C}) \longrightarrow \text{Simp}_n(\mathcal{C})
$$
from the category of simplicial objects of $\mathcal{C}$
to the category of $n$-truncated simplicial objects of $\mathcal{C}$.

\medskip\noindent
The {\it coskelet functor} (if it exists) is a functor
$$
\text{cosk}_n :
\text{Simp}(\mathcal{C}) \longrightarrow \text{Simp}_n(\mathcal{C})
$$
which is right adjoint to the skelet functor. In a formula
$$
\text{Mor}_{\text{Simp}(\mathcal{C})}(U, \text{cosk}_n V)
=
\text{Mor}_{\text{Simp}_n(\mathcal{C})}(\text{sk}_n U, V)
$$
Given a $n$-truncated simplicial object we 
say that {\it $\text{cosk}_nV$ exists} if there
exists an object $\text{cosk}_nV \in \text{Simp}(\mathcal{C})$
such that the formula above is valid. If it exists it
is unique up to unique isomorphism by the Yoneda lemma.
See Categories, Section \ref{categories-section-opposite}.

\begin{lemma}
\label{lemma-existence-cosk}

\end{lemma}

\begin{proof}
Let 


\end{proof}





\section{Hypercoverings}

\noindent
Assume that $\mathcal{C}$ is a site with the property
that the set of coverings consisting of $1$ morphism is cofinal.
Let $\mathcal{F}$ be a sheaf of abelian groups on
the site $\mathcal{C}$ which is assumed to have the property
that the set of coverings consisting of $1$ morphism is cofinal.
Choose an injective resolution $\mathcal{F} \to \mathcal{J}^\bullet$
(for example a canonical one, see 
Injectives, Section \ref{injectives-section-injectives-sheaves}).
Let $X$ be an object of $\mathcal{C}$. We want to compute 
$R\Gamma(X, \mathcal{F}) = \Gamma(X, \mathcal{J}^\bullet)$
or at least the cohomology groups $H^j(X, \mathcal{F})$.
The idea is to construct simplicial objects $U_\bullet$ 
augmented towards $X$, so $U_\bullet \to X$, such that 
$$
R\Gamma(X, \mathcal{F}) 
= \text{Tot}(R\Gamma(U_\bullet, \mathcal{J}^\bullet))
\leqno{(*)}
$$
is a quasi-isomorphism (for any $\mathcal{F}$). On the right hand 
side this is the total complex associated to the double complex. 
(The maps are always canonical since we have the resolution over 
all of $\mathcal{C}$.)
The complex $\Gamma(U_\bullet, \mathcal{F})$ maps into the
complex on the right. We will show that for any
element $\eta \in H^j(X, \mathcal{F})$ there exists a choice
of $U_\bullet \to X$ such that $\eta$ comes from an element
in $H^j(U_\bullet, \mathcal{F})$. This is a first step and
it already allows us to define cup products for example.
The starting point is the following.

\begin{lemma}
\label{lemma-product-hypercovering}
Suppose that $\{Y \to X\}$ is a covering in the topology of
$\mathcal{C}$. Let $U_n = Y^n_{/X}$ be the simplicial
object defined in Example \ref{example-fibre-products-simplicial-object}.
The augmentation $U_\bullet \to X$ has the property
that $(*)$ is a quasi-isomorphism for all $\mathcal{F}$.
\end{lemma}

\begin{proof}
FIXME.
\end{proof}

\section{Making simplicial objects}
\label{section-making-simplicial}

\noindent
Suppose that $U_\bullet$ is a simplicial object of $\mathcal{C}$. Now let
$n\geq 0$ and let $V \to U_n$ be a representable morphism of 
$\mathcal{C}$. This means that the fibre products $V \times_{U_n} W$ 
exist for all morphisms $W \to U_n$.

\medskip\noindent
For any $m$ consider the fibre product (over $U_m$)
$$
U'_m = \prod\nolimits_{\varphi \in \text{Mor}_\Delta([n],[m])}
V\times_{U_n, U_\bullet(\varphi)} U_m.
$$
By our assumption on the morphism $V \to U_n$ this fibre product
exists. For any $\psi : [m1] \to [m2]$ there is a canonical morphism
$U'_{m2} \to U'_{m1}$ coming from the map $\text{Mor}_\Delta([n],[m1])
\to \text{Mor}_\Delta([n],[m2]), \varphi \mapsto \varphi \circ \psi$,
the identity map on $V$ and the canonical map $U_\bullet(\psi) : 
U_{m2} \to U_{m1}$

\medskip\noindent
Clearly, these data give rise to a simplicial object $U'_\bullet$ in
$\mathcal{C}$. The natural morphisms $U'_m \to U_m$ give rise to a
morphism of simplicial objects $U'_\bullet \to U_\bullet$. Note that
the morphism $U'_n \to U_n$ factors throught the morphism $V \to U_n$
by projection onto the factor corresponding to $\varphi=\text{id}_{[n]}$.
Also, note that if $\mathcal{C}$ is a site and if 
$\{V \to U_n\}$ is a covering in the site then for any $m$ it is true
that $\{U'_m \to U_m\}$ is a covering. This proves the following lemma.

\begin{lemma}
\label{lemma-construct-new-covers}
Suppose that $U_\bullet$ and $V\to U_n$ are as above such that
$\{V \to U_n\}$ is a covering for the topology on the site
$\mathcal{C}$. The morphism of simplicial objects 
$U'_\bullet \to U_\bullet$ constructed above has the following 
properties:
(1) The morphism $U'_n \to U_n$ factors trough $V \to U_n$.
(2) For any $m$ the set $\{U'_m \to U_m\}$ is a covering
in the topology of $\mathcal{C}$.
\end{lemma}

\section{Doubly simplicial stuff}
\label{section-doubly-simplicial}

\noindent
A doubly simplicial object of $\mathcal{C}$ is a functor
$U_{\bullet,\bullet} : (\Delta\times\Delta)^\circ \to \mathcal{C}$.
By subdividing we can make this into a simplicial object 
$W(U_{\bullet,\bullet})$ with the same cohomology. FIXME: Explain this.

\noindent
Suppose that $U'_\bullet \to U_\bullet$ is a morphism of simplicial
objects of $\mathcal{C}$ such that each of the morphisms $U'_n \to
U_n$ is representable. Then we can construct a doubly-simplicial
object $U'_{\bullet,\bullet}$ by setting $U'_{n,0}= U'_n$,
$$
U'_{n,1} = U'_n \times_{U_n} U'_n,
$$
etc. Compare Example \ref{example-fibre-products-simplicial-object}. 
Out of this object we can construct a single simplicial object
$W(U'_{\bullet,\bullet})$ as explained above. Construct the 
natural morphism of simplicial objects 
$W(U'_{\bullet,\bullet}) \to U_\bullet$.

\begin{lemma}
Suppose that every $\{U'_n \to U_n\}$ is a covering for the topology
of $\mathcal{C}$. Suppose that $\mathcal{F}$ is a sheaf on 
$\mathcal{C}$. Then there is a natural
morphism of complexes
$$
R\Gamma(U_\bullet, \mathcal{F}) \to 
R\Gamma(W(U_{\bullet,\bullet}), \mathcal{F})
$$
which is a quasi-isomorphism. FIXME: Something like this in any case.
\end{lemma}

\section{The general case}

\noindent
Mention how things work more generally, for example if $\mathcal{C}$
does not have the property that coverings consisting of a single map
are cofinal. State the theorem in the correct generality.

\section{Other chapters}

\begin{multicols}{2}
\begin{enumerate}
\item \hyperref[introduction-section-phantom]{Introduction}
\item \hyperref[conventions-section-phantom]{Conventions}
\item \hyperref[sets-section-phantom]{Set Theory}
\item \hyperref[categories-section-phantom]{Categories}
\item \hyperref[topology-section-phantom]{Topology}
\item \hyperref[sheaves-section-phantom]{Sheaves on Spaces}
\item \hyperref[algebra-section-phantom]{Commutative Algebra}
\item \hyperref[sites-section-phantom]{Sites and Sheaves}
\item \hyperref[homology-section-phantom]{Homological Algebra}
\item \hyperref[derived-section-phantom]{Derived Categories}
\item \hyperref[more-algebra-section-phantom]{More Algebra}
\item \hyperref[simplicial-section-phantom]{Simplicial Methods}
\item \hyperref[modules-section-phantom]{Sheaves of Modules}
\item \hyperref[sites-modules-section-phantom]{Modules on Sites}
\item \hyperref[injectives-section-phantom]{Injectives}
\item \hyperref[cohomology-section-phantom]{Cohomology of Sheaves}
\item \hyperref[sites-cohomology-section-phantom]{Cohomology on Sites}
\item \hyperref[hypercovering-section-phantom]{Hypercoverings}
\item \hyperref[schemes-section-phantom]{Schemes}
\item \hyperref[constructions-section-phantom]{Constructions of Schemes}
\item \hyperref[properties-section-phantom]{Properties of Schemes}
\item \hyperref[morphisms-section-phantom]{Morphisms of Schemes}
\item \hyperref[coherent-section-phantom]{Coherent Cohomology}
\item \hyperref[divisors-section-phantom]{Divisors}
\item \hyperref[limits-section-phantom]{Limits of Schemes}
\item \hyperref[varieties-section-phantom]{Varieties}
\item \hyperref[chow-section-phantom]{Chow Homology}
\item \hyperref[topologies-section-phantom]{Topologies on Schemes}
\item \hyperref[descent-section-phantom]{Descent}
\item \hyperref[more-morphisms-section-phantom]{More on Morphisms}
\item \hyperref[flat-section-phantom]{More on Flatness}
\item \hyperref[groupoids-section-phantom]{Groupoid Schemes}
\item \hyperref[more-groupoids-section-phantom]{More on Groupoid Schemes}
\item \hyperref[etale-section-phantom]{\'Etale Morphisms of Schemes}
\item \hyperref[etale-cohomology-section-phantom]{\'Etale Cohomology}
\item \hyperref[spaces-section-phantom]{Algebraic Spaces}
\item \hyperref[spaces-properties-section-phantom]{Properties of Algebraic Spaces}
\item \hyperref[spaces-morphisms-section-phantom]{Morphisms of Algebraic Spaces}
\item \hyperref[spaces-topologies-section-phantom]{Topologies on Algebraic Spaces}
\item \hyperref[spaces-descent-section-phantom]{Descent and Algebraic Spaces}
\item \hyperref[spaces-more-morphisms-section-phantom]{More on Morphisms of Spaces}
\item \hyperref[quot-section-phantom]{Quot and Hilbert Spaces}
\item \hyperref[stacks-section-phantom]{Stacks}
\item \hyperref[spaces-groupoids-section-phantom]{Groupoids in Algebraic Spaces}
\item \hyperref[spaces-more-groupoids-section-phantom]{More on Groupoids in Spaces}
\item \hyperref[bootstrap-section-phantom]{Bootstrap}
\item \hyperref[examples-stacks-section-phantom]{Examples of Stacks}
\item \hyperref[groupoids-quotients-section-phantom]{Quotients of Groupoids}
\item \hyperref[algebraic-section-phantom]{Algebraic Stacks}
\item \hyperref[criteria-section-phantom]{Criteria for Representability}
\item \hyperref[stacks-properties-section-phantom]{Properties of Algebraic Stacks}
\item \hyperref[stacks-morphisms-section-phantom]{Morphisms of Algebraic Stacks}
\item \hyperref[examples-section-phantom]{Examples}
\item \hyperref[exercises-section-phantom]{Exercises}
\item \hyperref[guide-section-phantom]{Guide to Literature}
\item \hyperref[desirables-section-phantom]{Desirables}
\item \hyperref[coding-section-phantom]{Coding Style}
\item \hyperref[fdl-section-phantom]{GNU Free Documentation License}
\item \hyperref[index-section-phantom]{Auto Generated Index}
\end{enumerate}
\end{multicols}


\bibliography{my}
\bibliographystyle{alpha}

\end{document}
