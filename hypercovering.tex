\IfFileExists{stacks-project.cls}{%
\documentclass{stacks-project}
}{%
\documentclass{amsart}
}

% The following AMS packages are automatically loaded with
% the amsart documentclass:
%\usepackage{amsmath}
%\usepackage{amssymb}
%\usepackage{amsthm}

% For dealing with references we use the comment environment
\usepackage{verbatim}
\newenvironment{reference}{\comment}{\endcomment}
%\newenvironment{reference}{}{}
\newenvironment{slogan}{\comment}{\endcomment}
\newenvironment{history}{\comment}{\endcomment}

% For commutative diagrams you can use
% \usepackage{amscd}
\usepackage[all]{xy}

% We use 2cell for 2-commutative diagrams.
\xyoption{2cell}
\UseAllTwocells

% To put source file link in headers.
% Change "template.tex" to "this_filename.tex"
% \usepackage{fancyhdr}
% \pagestyle{fancy}
% \lhead{}
% \chead{}
% \rhead{Source file: \url{template.tex}}
% \lfoot{}
% \cfoot{\thepage}
% \rfoot{}
% \renewcommand{\headrulewidth}{0pt}
% \renewcommand{\footrulewidth}{0pt}
% \renewcommand{\headheight}{12pt}

\usepackage{multicol}

% For cross-file-references
\usepackage{xr-hyper}

% Package for hypertext links:
\usepackage{hyperref}

% For any local file, say "hello.tex" you want to link to please
% use \externaldocument[hello-]{hello}
\externaldocument[introduction-]{introduction}
\externaldocument[conventions-]{conventions}
\externaldocument[sets-]{sets}
\externaldocument[categories-]{categories}
\externaldocument[topology-]{topology}
\externaldocument[sheaves-]{sheaves}
\externaldocument[sites-]{sites}
\externaldocument[stacks-]{stacks}
\externaldocument[fields-]{fields}
\externaldocument[algebra-]{algebra}
\externaldocument[brauer-]{brauer}
\externaldocument[homology-]{homology}
\externaldocument[derived-]{derived}
\externaldocument[simplicial-]{simplicial}
\externaldocument[more-algebra-]{more-algebra}
\externaldocument[smoothing-]{smoothing}
\externaldocument[modules-]{modules}
\externaldocument[sites-modules-]{sites-modules}
\externaldocument[injectives-]{injectives}
\externaldocument[cohomology-]{cohomology}
\externaldocument[sites-cohomology-]{sites-cohomology}
\externaldocument[dga-]{dga}
\externaldocument[dpa-]{dpa}
\externaldocument[hypercovering-]{hypercovering}
\externaldocument[schemes-]{schemes}
\externaldocument[constructions-]{constructions}
\externaldocument[properties-]{properties}
\externaldocument[morphisms-]{morphisms}
\externaldocument[coherent-]{coherent}
\externaldocument[divisors-]{divisors}
\externaldocument[limits-]{limits}
\externaldocument[varieties-]{varieties}
\externaldocument[topologies-]{topologies}
\externaldocument[descent-]{descent}
\externaldocument[perfect-]{perfect}
\externaldocument[more-morphisms-]{more-morphisms}
\externaldocument[flat-]{flat}
\externaldocument[groupoids-]{groupoids}
\externaldocument[more-groupoids-]{more-groupoids}
\externaldocument[etale-]{etale}
\externaldocument[chow-]{chow}
\externaldocument[intersection-]{intersection}
\externaldocument[pic-]{pic}
\externaldocument[adequate-]{adequate}
\externaldocument[dualizing-]{dualizing}
\externaldocument[duality-]{duality}
\externaldocument[discriminant-]{discriminant}
\externaldocument[local-cohomology-]{local-cohomology}
\externaldocument[curves-]{curves}
\externaldocument[resolve-]{resolve}
\externaldocument[models-]{models}
\externaldocument[pione-]{pione}
\externaldocument[etale-cohomology-]{etale-cohomology}
\externaldocument[proetale-]{proetale}
\externaldocument[crystalline-]{crystalline}
\externaldocument[spaces-]{spaces}
\externaldocument[spaces-properties-]{spaces-properties}
\externaldocument[spaces-morphisms-]{spaces-morphisms}
\externaldocument[decent-spaces-]{decent-spaces}
\externaldocument[spaces-cohomology-]{spaces-cohomology}
\externaldocument[spaces-limits-]{spaces-limits}
\externaldocument[spaces-divisors-]{spaces-divisors}
\externaldocument[spaces-over-fields-]{spaces-over-fields}
\externaldocument[spaces-topologies-]{spaces-topologies}
\externaldocument[spaces-descent-]{spaces-descent}
\externaldocument[spaces-perfect-]{spaces-perfect}
\externaldocument[spaces-more-morphisms-]{spaces-more-morphisms}
\externaldocument[spaces-flat-]{spaces-flat}
\externaldocument[spaces-groupoids-]{spaces-groupoids}
\externaldocument[spaces-more-groupoids-]{spaces-more-groupoids}
\externaldocument[bootstrap-]{bootstrap}
\externaldocument[spaces-pushouts-]{spaces-pushouts}
\externaldocument[groupoids-quotients-]{groupoids-quotients}
\externaldocument[spaces-more-cohomology-]{spaces-more-cohomology}
\externaldocument[spaces-simplicial-]{spaces-simplicial}
\externaldocument[formal-spaces-]{formal-spaces}
\externaldocument[restricted-]{restricted}
\externaldocument[spaces-resolve-]{spaces-resolve}
\externaldocument[formal-defos-]{formal-defos}
\externaldocument[defos-]{defos}
\externaldocument[cotangent-]{cotangent}
\externaldocument[examples-defos-]{examples-defos}
\externaldocument[algebraic-]{algebraic}
\externaldocument[examples-stacks-]{examples-stacks}
\externaldocument[stacks-sheaves-]{stacks-sheaves}
\externaldocument[criteria-]{criteria}
\externaldocument[artin-]{artin}
\externaldocument[quot-]{quot}
\externaldocument[stacks-properties-]{stacks-properties}
\externaldocument[stacks-morphisms-]{stacks-morphisms}
\externaldocument[stacks-limits-]{stacks-limits}
\externaldocument[stacks-cohomology-]{stacks-cohomology}
\externaldocument[stacks-perfect-]{stacks-perfect}
\externaldocument[stacks-introduction-]{stacks-introduction}
\externaldocument[stacks-more-morphisms-]{stacks-more-morphisms}
\externaldocument[stacks-geometry-]{stacks-geometry}
\externaldocument[moduli-]{moduli}
\externaldocument[moduli-curves-]{moduli-curves}
\externaldocument[examples-]{examples}
\externaldocument[exercises-]{exercises}
\externaldocument[guide-]{guide}
\externaldocument[desirables-]{desirables}
\externaldocument[coding-]{coding}
\externaldocument[obsolete-]{obsolete}
\externaldocument[fdl-]{fdl}
\externaldocument[index-]{index}

% Theorem environments.
%
\theoremstyle{plain}
\newtheorem{theorem}[subsection]{Theorem}
\newtheorem{proposition}[subsection]{Proposition}
\newtheorem{lemma}[subsection]{Lemma}

\theoremstyle{definition}
\newtheorem{definition}[subsection]{Definition}
\newtheorem{example}[subsection]{Example}
\newtheorem{exercise}[subsection]{Exercise}
\newtheorem{situation}[subsection]{Situation}

\theoremstyle{remark}
\newtheorem{remark}[subsection]{Remark}
\newtheorem{remarks}[subsection]{Remarks}

\numberwithin{equation}{subsection}

% Macros
%
\def\lim{\mathop{\rm lim}\nolimits}
\def\colim{\mathop{\rm colim}\nolimits}
\def\Spec{\mathop{\rm Spec}}
\def\Hom{\mathop{\rm Hom}\nolimits}
\def\Ext{\mathop{\rm Ext}\nolimits}
\def\SheafHom{\mathop{\mathcal{H}\!{\it om}}\nolimits}
\def\SheafExt{\mathop{\mathcal{E}\!{\it xt}}\nolimits}
\def\Sch{\textit{Sch}}
\def\Mor{\mathop{\rm Mor}\nolimits}
\def\Ob{\mathop{\rm Ob}\nolimits}
\def\Sh{\mathop{\textit{Sh}}\nolimits}
\def\NL{\mathop{N\!L}\nolimits}
\def\proetale{{pro\text{-}\acute{e}tale}}
\def\etale{{\acute{e}tale}}
\def\QCoh{\textit{QCoh}}
\def\Ker{\mathop{\rm Ker}}
\def\Im{\mathop{\rm Im}}
\def\Coker{\mathop{\rm Coker}}
\def\Coim{\mathop{\rm Coim}}

%
% Macros for moduli stacks/spaces
%
\def\QCohstack{\mathcal{QC}\!{\it oh}}
\def\Cohstack{\mathcal{C}\!{\it oh}}
\def\Spacesstack{\mathcal{S}\!{\it paces}}
\def\Quotfunctor{{\rm Quot}}
\def\Hilbfunctor{{\rm Hilb}}
\def\Curvesstack{\mathcal{C}\!{\it urves}}
\def\Polarizedstack{\mathcal{P}\!{\it olarized}}
\def\Complexesstack{\mathcal{C}\!{\it omplexes}}
% \Pic is the operator that assigns to X its picard group, usage \Pic(X)
% \Picardstack_{X/B} denotes the Picard stack of X over B
% \Picardfunctor_{X/B} denotes the Picard functor of X over B
\def\Pic{\mathop{\rm Pic}\nolimits}
\def\Picardstack{\mathcal{P}\!{\it ic}}
\def\Picardfunctor{{\rm Pic}}
\def\Deformationcategory{\mathcal{D}\!{\it ef}}


% OK, start here.
%
\begin{document}

\title{Hypercoverings}

%\begin{abstract}
%\end{abstract}

\maketitle

\tableofcontents

\section{Introduction}
\label{section-introduction}

\noindent
Hypercoverings can be used to compute cohomology of abelian sheaves on sites
without recourse to injective resolutions. See \cite[Expose V, Sec. 7]{SGA4}.
A nice manuscript on cohomological descent is the text by Brian Conrad,
see \url{http://www.math.lsa.umich.edu/~bdconrad/papers/hypercover.pdf}.
This text follows the exposition in \cite[Expose Vbis]{SGA4}, and in
particular discusses a more general kind of hypercoverings, such as
proper hypercoverings. Probably it is useless to try to improve on
Brian's article, so we discuss the particular case of hypercoverings
in a site.

\section{The category $\Delta$}
\label{section-Delta}

\noindent
The category $\Delta$ is the category with
\begin{enumerate}
\item objects $[0], [1], [2], \ldots$ with
$[n] = \{0, 1, 2, \ldots, n\}$ and
\item morphism $[n] \to [m]$ is the set of nondecreasing
maps of sets $\{0, 1, 2, \ldots, n\} \to \{0, 1, 2, \ldots, m\}$.
\end{enumerate}
Here {\it nondecreasing} for a map $\varphi : [n] \to [m]$
means by definition that $\varphi(i) \geq \varphi(j)$ if $i \geq j$.
In other words, $\Delta$ is a category equivalent to the
``big'' category of finite totally ordered sets and nondecreasing maps.
There are exactly $n + 1$ morphisms $[0] \to [n]$ and
there is exactly $1$ morphism $[n] \to [0]$. There are
exactly $(n + 1)(n + 2)/2$ morphisms $[1] \to [n]$ and there are
exactly $n + 2$ morphisms $[n] \to [1]$. And so on and so forth.

\begin{definition}
\label{definition-face-degeneracy}
For any integer $n\geq 1$, and any $0\leq j \leq n$ we let $\delta^n_j : [n-1]
\to [n]$ denote the injective order preserving map skipping $j$. For any
integer $n\geq 0$, and any $0\leq j \leq n$ we denote $\sigma^n_j : [n+1] 
\to [n]$ the surjective order preserving map with 
$(\sigma^n_j)^{-1}(\{j\}) = \{j, j+1\}$.
\end{definition}

\begin{lemma}
\label{lemma-face-degeneracy}
Any morphism in $\Delta$ can be written as a composition
of an identity morphism, and the morphisms $\delta^n_j$ and $\sigma^n_j$.
\end{lemma}

\begin{proof}
Let $\varphi : [n] \to [m]$ be a morphism of $\Delta$.
If $j \not \in \text{Im}(\varphi)$, then we can write
$\varphi$ as $\delta^m_j \circ \psi$ for some morphism
$\psi : [n] \to [m - 1]$. If $\varphi(j) = \varphi(j + 1)$
then we can write $\varphi$ as $\psi \circ \sigma^{n - 1}_j$
for some morphism $\psi : [n - 1] \to [m]$.
The result follows because each replacement
as above lowers $n + m$ and hence at some point
$\varphi$ is both injective and surjective, hence
an identity morphism.
\end{proof}

\begin{lemma}
\label{lemma-relations-face-degeneracy}
The morphisms $\delta^n_j$ and $\sigma^n_j$ satisfy the following relations.
\begin{enumerate}
\item If $0 \leq i < j \leq n + 1$, then
$\delta^{n + 1}_j \circ \delta^n_i =
\delta^{n + 1}_i \circ \delta^n_{j - 1}$.
In other words the diagram
$$
\xymatrix{
& [n] \ar[rd]^{\delta^{n + 1}_j} & \\
[n - 1] \ar[ru]^{\delta^n_i} \ar[rd]_{\delta^n_{j - 1}} & &
[n + 1] \\
& [n] \ar[ru]_{\delta^{n + 1}_i} & 
}
$$
commutes.
\item If $0 \leq i < j \leq n - 1$, then
$\sigma^{n - 1}_j \circ \delta^n_i =
\delta^{n - 1}_i \circ \sigma^{n - 2}_{j - 1}$.
In other words the diagram
$$
\xymatrix{
& [n] \ar[rd]^{\sigma^{n - 1}_j} & \\
[n - 1] \ar[ru]^{\delta^n_i} \ar[rd]_{\sigma^{n - 2}_{j - 1}} & &
[n - 1] \\
& [n - 2] \ar[ru]_{\delta^{n - 1}_i} & 
}
$$
commutes.
\item If $0 \leq j \leq n - 1$, then
$\sigma^{n - 1}_j \circ \delta^n_j = \text{id}_{[n - 1]}$
and
$\sigma^{n - 1}_j \circ \delta^n_{j + 1} = \text{id}_{[n - 1]}$.
In other words the diagram
$$
\xymatrix{
& [n] \ar[rd]^{\sigma^{n - 1}_j} & \\
[n - 1]
\ar[ru]^{\delta^n_j}
\ar[rd]_{\delta^n_{j + 1}}
\ar[rr]^{\text{id}_{[n - 1]}} & & [n - 1] \\
& [n] \ar[ru]_{\sigma^{n - 1}_j} &
}
$$
commutes.
\item If $0 < j + 1 < i \leq n$, then
$\sigma^{n - 1}_j \circ \delta^n_i =
\delta^{n - 1}_{i - 1} \circ \sigma^{n - 2}_j$.
In other words the diagram
$$
\xymatrix{
& [n] \ar[rd]^{\sigma^{n - 1}_j} & \\
[n - 1] \ar[ru]^{\delta^n_i} \ar[rd]_{\sigma^{n - 2}_j} & &
[n - 1] \\
& [n - 2] \ar[ru]_{\delta^{n - 1}_{i - 1}} & 
}
$$
commutes.
\item If $0 \leq i \leq j \leq n - 1$, then
$\sigma^{n - 1}_j \circ \sigma^n_i =
\sigma^{n - 1}_i \circ \sigma^n_{j + 1}$.
In other words the diagram
$$
\xymatrix{
& [n] \ar[rd]^{\sigma^{n - 1}_j} & \\
[n + 1] \ar[ru]^{\sigma^n_i} \ar[rd]_{\sigma^n_{j + 1}} & &
[n - 1] \\
& [n] \ar[ru]_{\sigma^{n - 1}_i} & 
}
$$
commutes.
\end{enumerate}
\end{lemma}

\begin{proof}
Omitted.
\end{proof}

\begin{lemma}
\label{lemma-face-degeneracy-category}
The category $\Delta$ is the universal category
with objects $[n]$, $n \geq 0$ and morphisms
$\delta^n_j$ and $\sigma^n_j$ such that (a) every morphism is
a composition of these morphisms, (b) the relations
listed in Lemma \ref{lemma-relations-face-degeneracy} are satisfied,
and (c) any relation among the morphisms is a consquence of
those relations.
\end{lemma}

\begin{proof}
Omitted.
\end{proof}







\section{Simplicial objects}
\label{section-simplicial-object}

\begin{definition}
\label{definition-simplicial-object}
Let $\mathcal{C}$ be a category.
\begin{enumerate}
\item A {\it simplicial object $U$ of $\mathcal{C}$}
is a contravariant functor $U$ from $\Delta$ to
$\mathcal{C}$, in a formula:
$$
U : \Delta^{opp} \longrightarrow \mathcal{C}
$$
\item If $\mathcal{C}$ is the category of sets, then we call
$U$ a {\it simplicial set}.
\item If $\mathcal{C}$ is the category of abelian groups,
then we call $U$ a {\it simplicial abelian group}.
\item A {\it morphism of simplicial objects $U \to U'$}
is a transformation of functors.
\item The {\it category of simplicial objects of $\mathcal{C}$}
is denoted $\text{Simp}(\mathcal{C})$.
\end{enumerate}
\end{definition}

\noindent
This means there are objects $U([0]), U([1]), U([2]), \ldots$
and morphisms $U(\varphi) : U([n]) \to U([m])$,
where $\varphi$ is any nondecreasing map $\varphi : [m] \to [n]$. 

\medskip\noindent
In particular there is a unique morphism $U([0]) \to U([n])$ and there are
exactly $n + 1$ morphisms $U([n]) \to U([0])$ corresponding to
the $n + 1$ maps $[0] \to [n]$. Obviously we need some more notation
to be able to talk 
intelligently about these simplicial objects. We do this by considering
the morphisms we singled out in Section \ref{section-Delta} above.

\begin{lemma}
\label{lemma-characterize-simplicial-object}
Let $\mathcal{C}$ be a category.
\begin{enumerate}
\item Given a simplicial object $U$ in $\mathcal{C}$
we obtain a sequence of objects $U_n = U([n])$ endowed
with the morphisms $d^n_j = U(\delta^n_j) : U_n \to U_{n-1}$ and
$s^n_j = U(\sigma^n_j) : U_n \to U_{n + 1}$. These morphisms
satisfy the opposites of the relations displayed in
Lemma \ref{lemma-relations-face-degeneracy}.
\item Conversely, given a sequence of objects $U_n$ and morphisms
$d^n_j$, $s^n_j$ satisfying these relations there exists a unique
simplicial object $U$ in $\mathcal{C}$ such that $U_n = U([n])$,
$d^n_j = U(\delta^n_j)$, and $s^n_j = U(\sigma^n_j)$.
\item A morphism between simplicial objects $U$ and $U'$
is given by a family of morphisms $U_n \to U'_n$ commuting
with the morphisms $d^n_j$ and $s^n_j$.
\end{enumerate}
\end{lemma}

\begin{proof}
This follows from Lemma \ref{lemma-face-degeneracy-category}.
\end{proof}

\begin{remark}
By abuse of notation we sometimes write $d_i : U_n \to U_{n - 1}$
instead of $d^n_i$, and similarly for $s_i : U_n \to U_{n + 1}$.
The relations among the morphisms $d^n_i$ and $s^n_i$
may be expressed as follows:
\begin{enumerate}
\item If $i < j$, then $d_i \circ d_j = d_{j - 1} \circ d_i$.
\item If $i < j$, then $d_i \circ s_j = s_{j - 1} \circ d_i$.
\item We have $\text{id} = d_j \circ s_j = d_{j + 1} \circ s_j$.
\item If $i > j + 1$, then $d_i \circ s_j = s_j \circ d_{i - 1}$.
\item If $i \leq j$, then $s_i \circ s_j = s_{j + 1} \circ s_i$.
\end{enumerate}
This means that whenever the compositions on both the left and the
right are defined then the corresponding equality should hold.
\end{remark}

\noindent
We get a unique morphism $s^0_0 = U(\sigma^0_0) : U_0 \to U_1$ and
two morphisms $d^1_0 = U(\delta^1_0)$, and
$d^1_1 = U(\delta^1_1)$ which are morphisms $U_1 \to U_0$.
There are two morphisms $s^1_0 = U(\sigma^1_0)$, $s^1_1 = U(\sigma^1_1)$
which are morphsms $U_1 \to U_2$. Three morphisms
$d^2_0 = U(\delta^2_0)$, $d^2_1 = U(\delta^2_1)$, $d^2_2 = U(\delta^2_2)$
which are morphisms $U_3 \to U_2$. And so on.

\medskip\noindent
Pictorially we think of $U$ as follows:
$$
\xymatrix{
U_2
\ar@<2ex>[r]
\ar@<0ex>[r]
\ar@<-2ex>[r]
&
U_1 
\ar@<1ex>[r]
\ar@<-1ex>[r]
\ar@<1ex>[l]
\ar@<-1ex>[l]
&
U_0
\ar@<0ex>[l]
}
$$
Here the $d$-morphisms are the arrows pointing right and the 
$s$-morphisms are the arrows pointing left.

\begin{example}
\label{example-constant-simplicial-object}
The simplest example is the {\it constant} simplicial object with
value $X \in \text{Ob}(\mathcal{C})$. In other words, $U_n=X$ and
all maps are $\text{id}_X$.
\end{example}

\begin{example}
\label{example-fibre-products-simplicial-object}
Suppose that $Y\to X$ is a morphism of $C$ such that all
the fibred products $Y\times_X Y \times_X \ldots \times_X Y$ exist.
Then we set $U_n$ equal to the $(n + 1)$-fold fibre product,
and we let $\varphi: [n] \to [m]$ correspond to the map
(on ``coordinates'')
$(y_0,\ldots, y_m) \mapsto (y_{\varphi(0)},\ldots, y_{\varphi(n)})$.
In other words, the map $U_0 = Y \to U_1 = Y\times_X Y$ is the
diagonal map. The two maps $U_1 = Y\times_X Y \to U_0 = Y$ are the
projection maps.
\end{example}

\noindent
Geometrically Example \ref{example-fibre-products-simplicial-object}
above is an important example. It tells us that it is a good
idea to think of the maps $d^n_j : U_{n + 1} \to U_n$
as projection maps (forgetting the $j$th component),
and to think of the maps $s^n_j : U_n \to U_{n + 1}$
as diagonal maps (repeating the $j$th coordinate).
We will return to this in the sections below.

\begin{lemma}
\label{lemma-si-injective}
Let $\mathcal{C}$ be a category.
Let $U$ be a simplicial object of $\mathcal{C}$.
Each of the morphisms $s^n_i : U_n \to U_{n + 1}$
has a left inverse. In particular $s^n_i$ is a monomorphism.
\end{lemma}

\begin{proof}
This is true because $d_i^{n + 1} \circ s^n_i = \text{id}_{U_n}$.
\end{proof}

\section{Simplicial objects as presheaves}
\label{section-simplicial-presheaves}

\noindent
Another observation is that we may think of a simplicial
object of $\mathcal{C}$ as a presheaf with values in $\mathcal{C}$
over $\Delta$. See
Sites, Definition \ref{sites-definition-presheaf}.
And in fact, if $U$, $U'$ are simplicial objects
of $\mathcal{C}$, then we have
\begin{equation}
\label{simplicial-set-presheaf}
\text{Mor}(U, U') = \text{Mor}_{\textit{PSh}(\Delta)}(U, U').
\end{equation}
Some of the material below could be replace by the more
general constructions in the chapter on sites.
However, it seems a clearer picture arises from the
arguments specific to simplicial objects.
















\section{Cosimplicial objects}
\label{section-cosimplicial-object}

\noindent
A cosimplicial object of a category $\mathcal{C}$ could
be defined simply as a simplicial object of the
opposite category $\mathcal{C}^{opp}$. This is not
really how the human brain works, so we introduce
them separately here and point out some simple
properties.

\begin{definition}
\label{definition-cosimplicial-object}
Let $\mathcal{C}$ be a category.
\begin{enumerate}
\item A {\it cosimplicial object $U$ of $\mathcal{C}$}
is a covariant functor $U$ from $\Delta$ to
$\mathcal{C}$, in a formula:
$$
U : \Delta \longrightarrow \mathcal{C}
$$
\item If $\mathcal{C}$ is the category of sets, then we call
$U$ a {\it cosimplicial set}.
\item If $\mathcal{C}$ is the category of abelian groups,
then we call $U$ a {\it cosimplicial abelian group}.
\item A {\it morphism of cosimplicial objects $U \to U'$}
is a transformation of functors.
\item The {\it category of cosimplicial objects of $\mathcal{C}$}
is denoted $\text{Cosimp}(\mathcal{C})$.
\end{enumerate}
\end{definition}

\noindent
This means there are objects $U([0]), U([1]), U([2]), \ldots$
and morphisms $U(\varphi) : U([m]) \to U([n])$,
where $\varphi$ is any nondecreasing map $\varphi : [m] \to [n]$. 

\medskip\noindent
In particular there is a unique morphism $U([n]) \to U([0])$ and there are
exactly $n + 1$ morphisms $U([0]) \to U([n])$ corresponding to
the $n + 1$ maps $[0] \to [n]$. Obviously we need some more notation
to be able to talk intelligently about these simplicial objects.
We do this by considering the morphisms we singled out in
Section \ref{section-Delta} above.

\begin{lemma}
\label{lemma-characterize-cosimplicial-object}
Let $\mathcal{C}$ be a category.
\begin{enumerate}
\item Given a cosimplicial object $U$ in $\mathcal{C}$
we obtain a sequence of objects $U_n = U([n])$ endowed
with the morphisms $\delta^n_j = U(\delta^n_j) : U_{n - 1} \to U_n$ and
$\sigma^n_j = U(\sigma^n_j) : U_{n + 1} \to U_n$. These morphisms
satisfy the relations displayed in
Lemma \ref{lemma-relations-face-degeneracy}.
\item Conversely, given a sequence of objects $U_n$ and morphisms
$\delta^n_j$, $\sigma^n_j$ satisfying these relations there exists a unique
cosimplicial object $U$ in $\mathcal{C}$ such that $U_n = U([n])$,
$\delta^n_j = U(\delta^n_j)$, and $\sigma^n_j = U(\sigma^n_j)$.
\item A morphism between simplicial objects $U$ and $U'$
is given by a family of morphisms $U_n \to U'_n$ commuting
with the morphisms $\delta^n_j$ and $\sigma^n_j$.
\end{enumerate}
\end{lemma}

\begin{proof}
This follows from Lemma \ref{lemma-face-degeneracy-category}.
\end{proof}

\begin{remark}
By abuse of notation we sometimes write $\delta_i : U_{n - 1} \to U_n$
instead of $\delta^n_i$, and similarly for $\sigma_i : U_{n + 1} \to U_n$.
The relations among the morphisms $\delta^n_i$ and $\sigma^n_i$
may be expressed as follows:
\begin{enumerate}
\item If $i < j$, then
$\delta_j \circ \delta_i = \delta_i \circ \delta_{j - 1}$.
\item If $i < j$, then
$\sigma_j \circ \delta_i = \delta_i \circ \sigma_{j - 1}$.
\item We have
$\text{id} = \sigma_j \circ \delta_j = \sigma_j \circ \delta_{j + 1}$.
\item If $i > j + 1$, then
$\sigma_j \circ \delta_i = \delta_{i - 1} \circ \sigma_j$.
\item If $i \leq j$, then
$\sigma_j \circ \sigma_i = \sigma_i \circ \sigma_{j + 1}$.
\end{enumerate}
This means that whenever the compositions on both the left and the
right are defined then the corresponding equality should hold.
\end{remark}

\noindent
We get a unique morphism $\sigma^0_0 = U(\sigma^0_0) : U_1 \to U_0$ and
two morphisms $\delta^1_0 = U(\delta^1_0)$, and
$\delta^1_1 = U(\delta^1_1)$ which are morphisms $U_0 \to U_1$.
There are two morphisms
$\sigma^1_0 = U(\sigma^1_0)$, $\sigma^1_1 = U(\sigma^1_1)$
which are morphsms $U_2 \to U_1$. Three morphisms
$\delta^2_0 = U(\delta^2_0)$, $\delta^2_1 = U(\delta^2_1)$,
$\delta^2_2 = U(\delta^2_2)$
which are morphisms $U_2 \to U_3$. And so on.

\medskip\noindent
Pictorially we think of $U$ as follows:
$$
\xymatrix{
U_0
\ar@<1ex>[r]
\ar@<-1ex>[r]
&
U_1
\ar@<0ex>[l]
\ar@<2ex>[r]
\ar@<0ex>[r]
\ar@<-2ex>[r]
&
U_2
\ar@<1ex>[l]
\ar@<-1ex>[l]
}
$$
Here the $\delta$-morphisms are the arrows pointing right and the 
$\sigma$-morphisms are the arrows pointing left.

\begin{example}
\label{example-constant-cosimplicial-object}
The simplest example is the {\it constant} cosimplicial object with
value $X \in \text{Ob}(\mathcal{C})$. In other words, $U_n=X$ and
all maps are $\text{id}_X$.
\end{example}

\begin{example}
\label{example-push-outs-simplicial-object}
Suppose that $Y\to X$ is a morphism of $C$ such that all
the push outs $Y\coprod_X Y \coprod_X \ldots \coprod_X Y$ exist.
Then we set $U_n$ equal to the $(n + 1)$-fold push out,
and we let $\varphi: [n] \to [m]$ correspond to the map
$$
(y \text{ in }i\text{th component})
\mapsto
(y \text{ in }\varphi(i)\text{th component})
$$
on ``coordinates''.
In other words, the map $U_1 = Y \coprod_X Y \to U_0 = Y$ is the
identity on each component.
The two maps $U_0 = Y \to U_1 = Y \coprod_X Y$ are the two
natural maps.
\end{example}

\begin{lemma}
\label{lemma-di-injective}
Let $\mathcal{C}$ be a category.
Let $U$ be a cosimplicial object of $\mathcal{C}$.
Each of the morphisms $\delta^n_i : U_{n - 1} \to U_n$
has a left inverse. In particular $\delta^n_i$ is a monomorphism.
\end{lemma}

\begin{proof}
This is true because
$\sigma_i^{n - 1} \circ \delta^n_i = \text{id}_{U_n}$
for $j < n$.
\end{proof}


























\section{Products of simplicial objects}
\label{section-products}

\noindent
Of course we should define the product of simplicial objects
as the product in the category of simplicial objects. This
may lead to the potentially confusing situation where the product exists
but is not described as below. To avoid this we define the product
directly as follows.

\begin{definition}
\label{definition-product}
Let $\mathcal{C}$ be a category.
Let $U$ and $V$ be simplicial objects of $\mathcal{C}$.
Assume the products $U_n \times V_n$ exist in $\mathcal{C}$.
The {\it product of $U$ and $V$} is the simplicial object
$U\times V$ defined as follows:
\begin{enumerate}
\item $(U \times V)_n = U_n \times V_n$,
\item $d^n_i = (d^n_i, d^n_i)$, and
\item $s^n_i = (s^n_i, s^n_i)$.
\end{enumerate}
In other words, $U\times V$ is the product of the presheaves
$U$ and $V$ on $\Delta$.
\end{definition}

\begin{lemma}
\label{lemma-product}
If $U$ and $V$ are simplicial objects in the category $\mathcal{C}$,
and if $U\times V$ exists, then we have
$$
\text{Mor}(W, U\times V) = 
\text{Mor}(W, U) \times
\text{Mor}(W, V)
$$
for any third simplicial object $W$ of $\mathcal{C}$.
\end{lemma}

\begin{proof}
Omitted.
\end{proof}














\section{Products of cosimplicial objects}
\label{section-products}

\noindent
Of course we should define the product of cosimplicial objects
as the product in the category of cosimplicial objects. This
may lead to the potentially confusing situation where the product exists
but is not described as below. To avoid this we define the product
directly as follows.

\begin{definition}
\label{definition-product-cosimplicial-objects}
Let $\mathcal{C}$ be a category.
Let $U$ and $V$ be cosimplicial objects of $\mathcal{C}$.
Assume the products $U_n \times V_n$ exist in $\mathcal{C}$.
The {\it product of $U$ and $V$} is the cosimplicial object
$U\times V$ defined as follows:
\begin{enumerate}
\item $(U \times V)_n = U_n \times V_n$,
\item for any $\varphi : [n] \to [m]$ the map
$(U \times V)(\varphi) : U_n \times V_n \to U_m \times V_m$
is the product $U(\varphi) \times V(\varphi)$.
\end{enumerate}
\end{definition}

\begin{lemma}
\label{lemma-product-cosimplicial-objects}
If $U$ and $V$ are cosimplicial objects in the category $\mathcal{C}$,
and if $U\times V$ exists, then we have
$$
\text{Mor}(W, U\times V) = 
\text{Mor}(W, U) \times
\text{Mor}(W, V)
$$
for any third cosimplicial object $W$ of $\mathcal{C}$.
\end{lemma}

\begin{proof}
Omitted.
\end{proof}


















\section{Simplicial sets}
\label{section-simplicial-set}

\noindent
Let $U$ be a simplical set. It is a good idea to think of
$U_0$ as the {\it $0$-simplices}, the set $U_1$ as the
{\it $1$-simplices},
the set $U_2$ as the {\it $2$-simplices}, and so on.

\medskip\noindent
We think of the maps $s^n_j : U_n \to U_{n + 1}$ as
the map that associates to an $n$-simplex $A$ the degenerate
$(n + 1)$-simplex $B$ whose $(j, j + 1)$-edge is collapsed
to the vertex $j$ of $A$. We think of the map $d^n_j : U_n \to U_{n - 1}$
as the map that associates to an $n$-simplex $A$ one of the
faces, namely the face that omits the vertex $j$.
In this way it become possible to visualize the relations
among the maps $s^n_j$ and $d^n_j$ geometrically.

\begin{definition}
\label{definition-terminology-simplicial-sets}
Let $U$ be a simplicial set. 
We say {\it $x$ is a $n$-simplex of $U$} to signify that
$x$ is an element of $U_n$. We say that {\it $y$ is the $j$the
face of $x$} to signify that $d^n_jx = y$. We say that
{\it $z$ is the $j$th degeneracy of $x$} if $z = s^n_jx$.
A simplex is called {\it degenerate} if it is the degeneracy
of another simplex.
\end{definition}

\noindent
Here are a few fundamental examples.

\begin{example}
\label{example-simplex-simplicial-set}
For every $n \geq 0$ we denote $\Delta[n]$ the simplicial set
\begin{align*}
\Delta^{opp} & \longrightarrow \textit{Sets} \\
[k] & \longmapsto \text{Mor}_{\Delta}([k], [n])
\end{align*}
We leave it to the reader to verify the following statements.
Every $m$-simplex of $\Delta[n]$ with $m > n$ is degenerate.
There is a unique nondegenerate $n$-simplex of $\Delta[n]$,
namely $\text{id}_{[n]}$.
\end{example}

\begin{lemma}
\label{lemma-simplex-map}
Let $U$ be a simplicial set. Let $n \geq 0$ be an integer.
There is a canonical bijection
$$
\text{Mor}(\Delta[n], U)
\longrightarrow
U_n
$$
which maps a morphism $\varphi$ to the value of $\varphi$
on the unique nondegenerate $n$-simplex of $\Delta[n]$.
\end{lemma}

\begin{proof}
Omitted.
\end{proof}

\begin{example}
\label{example-simplex-category}
Consider the category $\Delta/[n]$ of objects over $[n]$
in $\Delta$, see
Categories, Example \ref{categories-example-category-over-X}.
There is a functor $p : \Delta/[n] \to \Delta$.
The fibre category of $p$ over $[k]$, see
Categories, Section \ref{categories-section-fibred-groupoids},
has as objects the
set $\Delta[n]_k$ of $k$-simplices in $\Delta[n]$, and as
morphisms only identities. For every morphism
$\varphi : [k] \to [l]$ of $\Delta$, and every object $\psi : [l] \to [n]$
in the fibre category over $[l]$ there is a unique
object over $[k]$ with a morphism covering $\varphi$, namely
$\psi \circ \varphi : [k] \to [n]$. Thus $\Delta/[n]$
is fibred in sets over $\Delta$. In other words, we may
think of $\Delta/[n]$ as a presheaf of sets over $\Delta$.
See also, Categories,
Example \ref{categories-example-fibred-category-from-functor-of-points}.
And this presheaf of sets agrees with the simplicial set
$\Delta[n]$. In particular, from Equation
(\ref{simplicial-set-presheaf}) and 
Lemma \ref{lemma-simplex-map} above
we get the formula
$$
\text{Mor}_{\textit{PSh}(\Delta)}(\Delta/[n], U) = U_n
$$
for any simplicial set $U$.
\end{example}

\begin{lemma}
\label{lemma-product-degenerate}
Let $U$, $V$ be simplicial sets.
Let $a, b \geq 0$ be integers.
Assume every $n$-simplex of $U$ is degenerate if $n > a$.
Assume every $n$-simplex of $V$ is degenerate if $n > b$.
Then every $n$-simplex of $U \times V$ is degenerate
if $n > a + b$.
\end{lemma}

\begin{proof}
Suppose $n > a + b$. Let $(u,v) \in (U\times V)_n = U_n \times V_n$.
By assumption, there exists a $\alpha : [n] \to [a]$ and a
$u' \in U_a$ and a $\beta : [n] \to [b]$ and a $v' \in V_b$
such that $u = U(\alpha)(u')$ and $v = V(\beta)(v')$. Because
$n > a + b$, there exists an $0 \leq i \leq a + b$ such that
$\alpha(i) = \alpha(i + 1)$ and
$\beta(i) = \beta(i + 1)$. It follows immediately
that $(u,v)$ is in the image of $s^{n - 1}_i$.
\end{proof}



\section{Products with simplicial sets}
\label{section-product-with-simplicial-sets}

\begin{definition}
\label{definition-product-with-simplicial-set}
Let $\mathcal{C}$ be a category such that the coproduct of
any two objects of $\mathcal{C}$ exists. Let
$U$ be a simplicial set. Let $V$ be a simplicial
object of $\mathcal{C}$. Assume that each $U_n$ is
finite nonempty. In this case we define
{\it the product 
$
U \times V
$
of $U$ and $V$}
to be the simplicial object of $\mathcal{C}$ whose
$n$th term is the object
$$
(U \times V)_n = \coprod\nolimits_{u\in U_n} V_n
$$
with maps for $\varphi : [m] \to [n]$ given by the
morphism
$$
\coprod\nolimits_{u\in U_n} V_n
\longrightarrow
\coprod\nolimits_{u'\in U_m} V_m
$$
which maps the component $V_n$ corresponding to $u$ to the
component $V_m$ corresponding to $u' = U(\varphi)(u)$
via the morphism $V(\varphi)$.
More loosely, if all of the coproducts displayed above
exist (without assuming anything about $\mathcal{C}$)
we will say that the {\it product $U \times V$ exists}.
\end{definition}

\noindent
Let $\mathcal{C}$ be a category.
Let $X$ be an object of $\mathcal{C}$.
Let $k \geq 0$ be an integer.
If all coproducts $X \coprod \ldots \coprod X$ exist
then according to the definition above the product
$$
X \times \Delta[k]
$$
exists, where we think of $X$ as the corresponding constant 
simplicial object.

\begin{lemma}
\label{lemma-morphism-from-coproduct}
With $X$ and $k$ as above.
For any simplicial object $V$ of
$\mathcal{C}$ we have the following
canonical bijection
$$
\text{Mor}_{\text{Simp}(\mathcal{C})}(X \times \Delta[k], V)
\longrightarrow
\text{Mor}_{\mathcal{C}}(X, V_k).
$$
wich maps $\gamma$ to the restriction of the
morphism $\gamma_k$ to the component corresponding
to $\text{id}_{[k]}$.
Similarly, for any $n \geq k$, if $W$ is an
$n$-truncated simplicial object
of $\mathcal{C}$, then we have
$$
\text{Mor}_{\text{Simp}_n(\mathcal{C})}(\text{sk}_n(X \times \Delta[k]), W)
=
\text{Mor}_{\mathcal{C}}(X, W_k).
$$
\end{lemma}

\begin{proof}
A morphism $\gamma : X \times \Delta[k] \to V$ is given by
a family of morphisms $\gamma_\alpha : X \to V_n$ where
$\alpha : [n] \to [k]$. The morphisms have to satisfy the
rules that for all $\varphi : [m] \to [n]$ the diagrams
$$
\xymatrix{
X \ar[r]^{\gamma_\alpha} \ar[d]^{\text{id}_X} & V_n \ar[d]^{V(\varphi)} \\
X \ar[r]^{\gamma_{\alpha \circ \varphi}} & V_m 
}
$$
commute. Taking $\alpha = \text{id}_{[k]}$, we see that
for any $\varphi : [m] \to [k]$ we have $\gamma_\varphi =
V(\varphi) \circ \gamma_{\text{id}_{[k]}}$. Thus the morphism
$\gamma$ is determined by the value of $\gamma$ on the
component corresponding to $\text{id}_{[k]}$. Conversely,
given such a morphism $f : X \to V_k$ we easily
construct a morphism $\gamma$ by putting
$\gamma_\alpha = V(\alpha) \circ f$.

\medskip\noindent
The truncated case is similar, and left to the reader.
\end{proof}

\noindent
A particular example of this is the case $k = 0$.
In this case the formula of the lemma just says
that
$$
\text{Mor}_\mathcal{C}(X, V_0) 
=
\text{Mor}_{\text{Simp}(\mathcal{C})}(X, V)
$$
where on the right hand side $X$ indicates the
constant simplicial object with value $X$. We will
use this formula without further mention in the
following.

\section{Internal Hom}
\label{section-internal-hom}

\noindent
Let $\mathcal{C}$ be a category with finite nonempty
products. Let $U$, $V$ be simplicial objects $\mathcal{C}$.
In some cases the functor
\begin{eqnarray*}
\text{Simp}(\mathcal{C})^{opp} & \longrightarrow & \textit{Sets} \\
W & \longmapsto & \text{Mor}_{\text{Simp}(\mathcal{C})}(W \times V, U)
\end{eqnarray*}
is representable. In this case we denote $\textit{Hom}(V, U)$
the resulting simplicial object of $\mathcal{C}$, and we say
that the {\it internal hom of $V$ into $U$ exists}. Moreover,
in this case we would have
\begin{eqnarray*}
\text{Mor}_{\mathcal{C}}(X, \textit{Hom}(V, U)_n)
& = &
\text{Mor}_{\text{Simp}(\mathcal{C})}(X \times \Delta[n], \textit{Hom}(V, U))
\\
& = &
\text{Mor}_{\text{Simp}(\mathcal{C})}(X \times \Delta[n]\times V, U) \\
& = &
\text{Mor}_{\text{Simp}(\mathcal{C})}(X, \textit{Hom}(\Delta[n] \times V, U))
\\
& = &
\text{Mor}_{\mathcal{C}}(X, \textit{Hom}(\Delta[n] \times V, U)_0)
\end{eqnarray*}
provided that $\textit{Hom}(\Delta[n] \times V, U)$
exists also. Here we have used the material from Section
\ref{section-product-with-simplicial-sets}.

\medskip\noindent
The lesson we learn from this is that, given $U$ and $V$,
if we want to construct the internal hom then we should try to  
construct the objects
$$
\textit{Hom}(\Delta[n] \times V, U)_0
$$
because these should be the $n$th term of $\textit{Hom}(V, U)$.
In the next section we study a construction of simplicial objects
``$\text{Hom}(\Delta[n], U)$''.


\section{Hom from simplicial sets}
\label{section-hom-from-simplicial-sets}

\noindent
Motivated by the discussion on internal hom we define
what should be the simplicial object classifying
morphisms from a simplicial set into a given
simplicial object of the category $\mathcal{C}$.

\begin{definition}
\label{definition-hom-from-simplicial-set}
Let $\mathcal{C}$ be a category such that the coproduct
of any two objects exists.
Let $U$ be a simplicial set, with $U_n$ finite nonempty
for all $n \geq 0$.
Let $V$ be a simplicial object of $\mathcal{C}$.
We denote $\text{Hom}(U, V)$ any simplicial object of
$\mathcal{C}$ such that
$$
\text{Mor}_{\text{Simp}(\mathcal{C})}(W, \text{Hom}(U, V))
=
\text{Mor}_{\text{Simp}(\mathcal{C})}(W \times U, V)
$$
functorially in the simplicial object $W$ of $\mathcal{C}$.
\end{definition}

\noindent
Of course $\text{Hom}(U, V)$ need not exist.
Also, by the discussion in Section \ref{section-internal-hom}
we expect that if it does exist, then
$\text{Hom}(U, V)_n = \text{Hom}(U \times \Delta[n], V)_0$.
We do not use the italic notation for these Hom objects
since $\text{Hom}(U, V)$ is not an internal hom.

\begin{lemma}
\label{lemma-exists-hom-0-from-simplicial-set}
Assume the category $\mathcal{C}$
has coproducts of any two objects and countable
limits. Let $U$ be a simplicial set, with $U_n$ finite nonempty
for all $n \geq 0$.
Let $V$ be a simplicial object of $\mathcal{C}$.
Then the functor
\begin{eqnarray*}
\mathcal{C}^{opp} & \longrightarrow & \textit{Sets} \\
X
& \longmapsto &
\text{Mor}_{\text{Simp}(\mathcal{C})}(X \times U, V)
\end{eqnarray*}
is representable.
\end{lemma}

\begin{proof}
A morphism from $X \times U$ into $V$ is given by a collection
of morphisms $f_u : X \to V_n$ with $n \geq 0$ and $u \in U_n$.
And such a collection actually defines a morphism if and only
if for all $\varphi : [m] \to [n]$ all the diagrams
$$
\xymatrix{
X \ar[r]^{f_u} \ar[d]_{\text{id}_X} & V_n \ar[d]^{V(\varphi)} \\
X \ar[r]^{f_{U(\varphi)(u)}} & V_m 
}
$$
commute. Thus it is natural to introduce a category
$\mathcal{U}$ and a functor
$\mathcal{V} : \mathcal{U}^{opp} \to \mathcal{C}$
as follows:
\begin{enumerate}
\item The set of objects of $\mathcal{U}$ is
$\coprod_{n \geq 0} U_n$,
\item a morphism from $u' \in U_m$ to $u \in U_n$
is a $\varphi : [m] \to [n]$ such that $U(\varphi)(u) = u'$
\item for $u \in U_n$ we set $\mathcal{V}(u) = V_n$, and
\item for $\varphi : [m] \to [n]$ such that $U(\varphi)(u) = u'$
we set $\mathcal{V}(\varphi) = V(\varphi) : V_n \to V_m$.
\end{enumerate}
At this point it is clear that our functor is nothing but the
functor defining
$$
\text{lim}_{\mathcal{U}^{opp}} \mathcal{V}
$$
Thus if $\mathcal{C}$ has countable limits then this limit
and hence an object representing the functor of the lemma
exist.
\end{proof}

\begin{lemma}
\label{lemma-exists-hom-0-from-simplicial-set-finite}
Assume the category $\mathcal{C}$
has coproducts of any two objects and finite
limits. Let $U$ be a simplicial set, with $U_n$ finite nonempty
for all $n \geq 0$. Assume that all $n$-simplices
of $U$ are degenerate for all $n \gg 0$.
Let $V$ be a simplicial object of $\mathcal{C}$.
Then the functor
\begin{eqnarray*}
\mathcal{C}^{opp} & \longrightarrow & \textit{Sets} \\
X
& \longmapsto &
\text{Mor}_{\text{Simp}(\mathcal{C})}(X \times U, V)
\end{eqnarray*}
is representable.
\end{lemma}

\begin{proof}
We have to show that the category $\mathcal{U}$ described
in the proof of Lemma \ref{lemma-exists-hom-0-from-simplicial-set}
has a finite subcategory $\mathcal{U}'$ such that the limit
of $\mathcal{V}$ over $\mathcal{U}'$ is the same as the
limit of $\mathcal{V}$ over $\mathcal{U}$. We will use
Categories, Lemma \ref{categories-lemma-limit-final-subcategory}.
For $m > 0$ let $\mathcal{U}_{\leq m}$ denote the full
subcategory with objects $\coprod_{0 \leq n \leq m} U_m$.
Let $m_0$ be an integer such that every $n$-simplex
of the simplicial set $U$ is degenerate if $n > m_0$.
For any $m \geq m_0$ large enough, the subcategory
$\mathcal{U}_{\leq m}$ satisfies property (1) of the lemma
cited above.

\medskip\noindent
Suppose that $u \in U_n$ and
$u' \in U_{n'}$ with $n, n' \leq m_0$ and suppose that
$\varphi : [k] \to [n]$, $\varphi' : [k] \to [n']$
are morphisms such that $U(\varphi)(u) = U(\varphi')(u')$.
A simple combinatorial argument shows that if $k > 2m_0$,
then there exists an index $0 \leq i \leq 2m_0$ such that
$\varphi(i) =\varphi(i + 1)$ and $\varphi'(i) = \varphi'(i + 1)$.
(The pidgeon hole principle would tell you this works if
$k > m_0^2$ which is good enough for the argument below
anyways.) Hence, if $k > 2m_0$, we may write
$\varphi = \psi \circ \sigma^{k - 1}_i$ and
$\varphi' = \psi' \circ \sigma^{k - 1}_i$ for some
$\psi : [k - 1] \to [n]$ and some $\psi' : [k - 1] \to [n']$.
Since $s^{k - 1}_i : U_{k - 1} \to U_k$ is injective,
see Lemma \ref{lemma-si-injective}, we conclude that
$U(\psi)(u) = U(\psi')(u')$ also. Continuing in this
fashion we conclude that given morphisms
$u \to z$ and $u' \to z$ of $\mathcal{U}$
with $u, u' \in \mathcal{U}_{\leq m_0}$, there exists
a commutative diagram
$$
\xymatrix{
u \ar[rd] \ar[rrd] & & \\
& a \ar[r] & z \\
u' \ar[ru] \ar[rru] 
}
$$
with $a \in \mathcal{U}_{\leq 2m_0}$.

\medskip\noindent
It is easy to deduce from this that the finite subcategory
$\mathcal{U}_{\leq 2m_0}$ works. Namely, suppose given
$x' \in U_n$ and $x'' \in U_{n'}$ with $n, n' \leq 2m_0$ as well as
morphisms $x' \to x$ and $x'' \to x$ of $\mathcal{U}$
with the same target. By our choice of $m_0$ we can
find objects $u, u'$ of $\mathcal{U}_{\leq m_0}$ and
morphisms $u \to x'$, $u' \to x''$.
By the above we can find $a \in \mathcal{U}_{\leq 2m_0}$
and morphisms $u \to a$, $u' \to a$ such that
$$
\xymatrix{
u \ar[rd] \ar[rrd] \ar[r] & x' \ar[rd] & \\
& a \ar[r] & x \\
u' \ar[ru] \ar[rru] \ar[r] & x'' \ar[ru] &
}
$$
is commutative. Turning this diagram 90 degrees clockwise
we get the desired diagram as in (2) of the
cited lemma.
\end{proof}

\begin{lemma}
\label{lemma-exists-hom-from-simplicial-set-finite}
Assume the category $\mathcal{C}$
has coproducts of any two objects and finite
limits. Let $U$ be a simplicial set, with $U_n$ finite nonempty
for all $n \geq 0$. Assume that all $n$-simplices
of $U$ are degenerate for all $n \gg 0$.
Let $V$ be a simplicial object of $\mathcal{C}$.
Then $\text{Hom}(U, V)$ exists, moreover
we have the expected equalities
$$
\text{Hom}(U, V)_n = \text{Hom}(U \times \Delta[n], V)_0.
$$
\end{lemma}

\begin{proof}
We construct this simplicial object as follows.
For $n \geq 0$ let $\text{Hom}(U, V)_n$ denote
the object of $\mathcal{C}$ representing the
functor
$$
X
\longmapsto
\text{Mor}_{\text{Simp}(\mathcal{C})}(X \times U \times \Delta[n], V)
$$
This exists by Lemma \ref{lemma-exists-hom-0-from-simplicial-set-finite} 
because $U \times \Delta[n]$ is a simplicial set with finite
sets of simplices and no nondegenerate simplices in high enough degree,
see Lemma \ref{lemma-product-degenerate}.
For $\varphi : [m] \to [n]$ we obtain an induced map of simplicial
sets $\varphi : \Delta[m] \to \Delta[n]$. Hence we obtain a morphism
$X \times U \times \Delta[m] \to X \times U \times \Delta[n]$
functorial in $X$, and hence a transformation of functors,
which in turn gives
$$
\text{Hom}(U, V)(\varphi) :
\text{Hom}(U, V)_n
\longrightarrow
\text{Hom}(U, V)_m.
$$
Clearly this defines a contravariant functor
$\text{Hom}(U, V)$ from
$\Delta$ into the category $\mathcal{C}$.
In other words, we have a simplicial object of $\mathcal{C}$.

\medskip\noindent
We have to show that $\text{Hom}(U, V)$ satisfies the desired
universal property
$$
\text{Mor}_{\text{Simp}(\mathcal{C})}(W, \text{Hom}(U, V))
=
\text{Mor}_{\text{Simp}(\mathcal{C})}(W \times U, V)
$$
To see this, let $f : W \to \text{Hom}(U, V)$ be given.
We want to construct the element $f' : W \times U \to V$
of the right hand side.
By construction, each $f_n : W_n \to \text{Hom}(U, V)_n$
corresponds to a morphism
$f_n : W_n \times U \times \Delta[n] \to V$. Further,
for every morphism $\varphi : [m] \to [n]$ the
diagram 
$$
\xymatrix{
W_n \times U \times \Delta[m]
\ar[rr]_{W(\varphi)\times \text{id} \times \text{id}}
\ar[d]_{\text{id} \times \text{id} \times \varphi} & &
W_m \times U \times \Delta[m] \ar[d]^{f_m} \\
W_n \times U \times \Delta[n] \ar[rr]^{f_n} & & V
}
$$
is commutative. For $\psi : [n] \to [k]$ in $(\Delta[n])_k$
we denote $(f_n)_{k, \psi} : W_n \times U_k \to V_k$
the component of $(f_n)_k$ corresponding to the element
$\psi$. We define $f'_n : W_n \times U_n \to V_n$
as $f'_n = (f_n)_{n, \text{id}}$, in other words, as 
the restriction of
$(f_n)_n : W_n \times U_n \times (\Delta[n])_n \to V_n$
to $W_n \times U_n \times \text{id}_{[n]}$.
To see that the collection $(f'_n)$ defines a
morphism of simplicial objects, we have to show
for any $\varphi : [m] \to [n]$ that
$V(\varphi) \circ f'_n =
f'_m \circ W(\varphi) \times U(\varphi)$.
The commutative diagram above says that
$(f_n)_{m, \varphi} : W_n \times U_m \to V_m$
is equal to
$(f_m)_{m, \text{id}} \circ W(\varphi) :
W_n \times U_m \to V_m$.
But then the fact that $f_n$ is a morphism of simplicial
objects implies that the diagram
$$
\xymatrix{
W_n \times U_n \times (\Delta[n])_n
\ar[r]_-{(f_n)_n}
\ar[d]_{\text{id}\times U(\varphi) \times \varphi}
& V_n \ar[d]^{V(\varphi)} \\
W_n \times U_m \times (\Delta[n])_m \ar[r]^-{(f_n)_m} & V_m
}
$$
is commutative. And this implies that
$(f_n)_{m, \varphi} \circ U(\varphi)$ is
equal to $V(\varphi) \circ (f_n)_{n, \text{id}}$.
Alltogether we obtain
$
V(\varphi) \circ (f_n)_{n, \text{id}}
=
(f_n)_{m, \varphi} \circ U(\varphi)
=
(f_m)_{m, \text{id}} \circ W(\varphi)\circ U(\varphi)
=
(f_m)_{m, \text{id}} \circ W(\varphi)\times U(\varphi)
$
as desired.

\medskip\noindent
On the other hand, given a morphism
$f' : W \times U \to V$ we define
a morphism $f : W \to \text{Hom}(U, V)$ 
as follows. By Lemma \ref{lemma-morphism-from-coproduct} the morphisms
$\text{id} : W_n \to W_n$ corresponds to a unique
morphism $c_n : W_n \times \Delta[n] \to W$.
Hence we can consider the composition
$$
W_n \times \Delta[n] \times U
\xrightarrow{c_n}
W \times U
\xrightarrow{f'}
V.
$$
By construction this corresponds to a unique morphism
$f_n : W_n \to \text{Hom}(U, V)_n$. We leave it to the reader
to see that these define a morphism of simplicial sets as
desired.

\medskip\noindent
We also leave it to the reader to see that
$f \mapsto f'$ and $f' \mapsto f$ are mutually inverse
operations.
\end{proof}

\noindent
We spell out the construction above in a special case.
Let $X$ be an object of a category $\mathcal{C}$.
Assume that self products $X \times \ldots \times X$ exist.
Let $k$ be an integer.
Consider the simplicial object $U$ with terms
$$
U_n = \prod\nolimits_{\alpha \in \text{Mor}([k], [n])} X
$$
and maps given $\varphi : [m] \to [n]$ 
\begin{eqnarray*}
U(\varphi) :
\prod\nolimits_{\alpha \in \text{Mor}([k], [n])} X
& \longrightarrow &
\prod\nolimits_{\alpha' \in \text{Mor}([k], [m])} X \\
(f_{\alpha})_{\alpha} & \longmapsto & 
(f_{\varphi \circ \alpha'})_{\alpha'}
\end{eqnarray*}
In terms of ``coordinates'', the element $(x_\alpha)_\alpha$
is mapped to the element $(x_{\varphi \circ \alpha'})_{\alpha'}$.
We claim this object is equal to
$$
\text{Hom}(\Delta[k], X)
$$
where we think of $X$ as the constant simplicial object $X$.

\begin{lemma}
\label{lemma-morphism-into-product}
With $X$, $k$ and $U$ as above.
\begin{enumerate}
\item For any simplicial object $V$ of
$\mathcal{C}$ we have the following
canonical bijection
$$
\text{Mor}_{\text{Simp}(\mathcal{C})}(V, U)
\longrightarrow
\text{Mor}_{\mathcal{C}}(V_k, X).
$$
wich maps $\gamma$ to the morphism $\gamma_k$ composed with
the projection onto the factor corresponding to $\text{id}_{[k]}$.
\item Similarly, if $W$ is an $k$-truncated simplicial object
of $\mathcal{C}$, then we have
$$
\text{Mor}_{\text{Simp}_k(\mathcal{C})}(W, \text{sk}_k U)
=
\text{Mor}_{\mathcal{C}}(W_k, X).
$$
\item The object $U$ constructed above is an
incarnation of $\text{Hom}(\Delta[k], X)$.
\end{enumerate}
\end{lemma}

\begin{proof}
We first prove (1).
Suppose that $\gamma : V \to U$ is a morphism.
This is given by a family of morphisms
$\gamma_{\alpha} : V_n \to X$ for $\gamma : [k] \to [n]$.
The morphisms have to satisfy the
rules that for all $\varphi : [m] \to [n]$ the diagrams
$$
\xymatrix{
X \ar[d]^{\text{id}_X} &
V_n \ar[d]^{V(\varphi)}
\ar[l]^{\gamma_{\varphi \circ \alpha'}} \\
X &
V_m \ar[l]_{\gamma_{\alpha'}}
}
$$
commute for all $\alpha' : [k] \to [m]$.
Taking $\alpha' = \text{id}_{[k]}$, we see that
for any $\varphi : [k] \to [n]$ we have $\gamma_\varphi =
\gamma_{\text{id}_{[k]}} \circ V(\varphi)$. Thus the morphism
$\gamma$ is determined by the component of $\gamma_k$
corresponding to $\text{id}_{[k]}$. Conversely,
given such a morphism $f : V_k \to X$ we easily
construct a morphism $\gamma$ by putting
$\gamma_\alpha = f \circ V(\alpha)$.

\medskip\noindent
The truncated case is similar, and left to the reader.

\medskip\noindent
To see (3) we argue as follows:
\begin{eqnarray*}
\text{Mor}(V, \text{Hom}(\Delta[k], X)) & = &
\text{Mor}(V \times \Delta[k], X) \\
& = & \{ (f_n : V_n \times \Delta[k]_n \to X) \mid f_n \text{ compatible}\} \\
& = & \{ (f_n : V_n \to \prod\nolimits_{\Delta[k]_n} X) \mid f_n \text{ compatible}\} \\
& = & \text{Mor}(V, U)
\end{eqnarray*}
Thus $U$ and $\text{Hom}(\Delta[k], X)$ define the same
functor on the category of simplicial objects and
hence are canonically isomorphic.
\end{proof}

















\section{Splitting simplicial objects}
\label{section-splitting}

\noindent
A subobject $N$ of an object $X$ of the category $\mathcal{C}$
is an object $N$ of $\mathcal{C}$ together with a monomorphism
$N \to X$. Of course we say (by abouse of notation) that
the subobjects $N$, $N'$ are equal if there exists an isomorphism
$N \to N'$ compatible with the morphisms to $X$. The collection
of subobjects forms a partially ordered set. (Because of our
conventions on categories; not true for category of spaces
up to homotopy for example.)

\begin{definition}
\label{definition-split}
Let $\mathcal{C}$ be a category which admits finite nonempty coproducts.
We say a simplicial object $U$ of $\mathcal{C}$ is split
if there exist subobjects $N(U_m)$ of $U_m$, $m \geq 0$
with the property that
\begin{equation}
\label{equation-splitting}
\coprod\nolimits_{\varphi : [n] \to [m]\text{ surjective}}
N(U_m)
\longrightarrow
U_n
\end{equation}
is an isomorphism for all $n \geq 0$.
\end{definition}

\noindent
If this is the case, then $N(U_0) = U_0$. Next, we have
$U_1 = U_0 \coprod N(U_1)$. Second we have
$$
U_2 = U_0 \coprod N(U_1) \coprod N(U_1) \coprod N(U_2).
$$
It turns out that in many categories $\mathcal{C}$
every simplicial object is split.

\begin{lemma}
\label{lemma-splitting-simplicial-sets}
Let $U$ be a simplicial set.
Then $U$ has a canonical splitting
with $N(U_m)$ equal to the set of 
nondegenerate $m$-simplices.
\end{lemma}

\begin{proof}
Let $x \in U_n$. Suppose that
there are surjections $\varphi : [n] \to [k]$
and $\psi : [n] \to [l]$ and nondegenerate simplices
$y \in U_k$, $z \in U_l$ such that $x = U(\varphi)(y)$
and $x = U(\psi)(z)$. Choose a right inverse $\xi : [l] \to [n]$
of $\psi$, i.e., $\psi \circ \xi = \text{id}_{[l]}$.
Then $z = U(\xi)(x)$. Hence $z = U(\xi)(x) = U(\varphi \circ \xi)(y)$.
Since $z$ is nondegenerate we conclude that $\varphi \circ \xi :
[l] \to [k]$ is surjective, and hence $l \geq k$. Similarly
$k \geq l$. Hence we see that $\varphi \circ \xi : [l] \to [k]$
has to be the identity map for any choice of right inverse
$\xi$ of $\psi$. This easily implies that $\psi = \varphi$.
\end{proof}

\begin{lemma}
\label{lemma-splitting-simplicial-groups}
Let $U$ be a simplicial abelian group.
Then $U$ has a splitting obtained by taking $N(U_0) = U_0$ and
for $m \geq 1$ taking
$$
N(U_m) = \bigcap\nolimits_{i = 1}^m \text{Ker}(d^m_i).
$$
\end{lemma}

\begin{proof}
By induction on $n$ we will show that the choice of $N(U_m)$
in the lemma garantees that (\ref{equation-splitting}) is
an isomorphism for $m \leq n$. This is clear for $n = 0$.
In the rest of this proof we are going to
drop the superscripts from the maps $d_i$ and $s_i$ in order
to improve readability.

\medskip\noindent
First we make a general remark.
For $1 \leq i \leq m + 1$ and $z \in U_m$ we have
$d_i(s_{i - 1}(z)) = z$. Hence we can write 
any $x \in U_{m + 1}$ uniquely as
$x = x_i + x'_i$ with $d_i(x_i) = 0$
and $x'_i \in \text{Im}(s_{i - 1})$
by taking $x_i = (x - s_{i - 1}(d_i(x)))$ and
$x'_i = s_{i - 1}(d_i(x))$.

\medskip\noindent
We invent a procedure for decomposing
any $x \in U_{n + 1}$ as follows.
First write $x = x_n + s_n(z_n)$ with $d_{n + 1}(x_n) = 0$.
Next, write $x_n = x_{n - 1} + s_{n - 1}(z_{n - 1})$ with
$d_n(x_{n - 1}) = 0$. Then
$0 = d_{n + 1}(x_n) =
d_{n + 1}(x_{n - 1}) + d_{n + 1}(s_{n - 1}(z_{n - 1})) =
d_{n + 1}(x_{n - 1}) + s_{n - 1}(d_n(z_{n - 1}))$.
Now the commutation relations among the $d_b$ show that
$d_n(d_{n + 1}(x_{n - 1})) = d_n(d_n(x_{n - 1})) = 0$
because $d_n(x_{n - 1})= 0$ by construction.
The uniqueness above shows the equality
$0 = d_{n + 1}(x_{n - 1}) + s_{n - 1}(d_n(z_{n - 1}))$
can only hold if both terms are zero. In this way we conclude that
$x_{n - 1} \in \text{Ker}(d_{n + 1}) \cap \text{Ker}(d_n)$,
and $z_{n - 1} \in \text{Ker}(d_n)$. We continue in this way
to get
\begin{eqnarray*}
x & = & x_n + s_n(z_n), \\
x_n & = & x_{n - 1} + s_{n - 1}(z_{n - 1}), \\
x_{n - 1} & = & x_{n - 2} + s_{n - 2}(z_{n - 2}), \\
\ldots & \ldots & \ldots \\
x_1 & = & x_0 + s_0(z_0)
\end{eqnarray*}
such that
$x_i \in
\text{Ker}(d_{n + 1}) \cap \text{Ker}(d_n)
\cap \ldots \cap \text{Ker}(d_{i + 1})$
and
$z_i \in \text{Ker}(d_n) \cap \ldots \cap \text{Ker}(d_{i + 1})$.
In other words, we can uniquely write
$$
x = s_n(z_n) + s_{n - 1}(z_{n - 1}) + \ldots + s_0(z_0) + x_0
$$
with $x_0 \in N(U_{n + 1})$ and
$z_i \in \text{Ker}(d_n) \cap \ldots \cap \text{Ker}(d_{i + 1})$.
We can reformulate this as saying that we have found a direct
sum decomposition
$$
U_{n + 1}
=
N(U_{n + 1})
\oplus
\bigoplus\nolimits_{i = 0}^{i = n}
s_i\Big(\text{Ker}(d_n) \cap \ldots \cap \text{Ker}(d_{i + 1})\Big)
$$
with the property that
$$
\text{Ker}(d_{n + 1}) \cap \ldots \cap \text{Ker}(d_j)
=
N(U_{n + 1}) \oplus
\bigoplus\nolimits_{i = 0}^{i = j - 2}
s_i\Big(\text{Ker}(d_n) \cap \ldots \cap \text{Ker}(d_{i + 1})\Big)
$$
for $j = n + 1, \ldots, 1$.
The result follows from this statement as follows.
Each of the $z_i$ in the expression for $x$
can be written uniquely as
$$
z_i = s_{i - 1}(z'_{i, i - 1}) + \ldots + s_0(z'_{i, 0}) + z_{i, 0}
$$
with $z_{i, 0} \in N(U_n)$ and
$z'_{i, j} \in \text{Ker}(d_{n - 1}) \cap \ldots \cap \text{Ker}(d_{j + 1})$.
The first few steps in the decomposition of $z_i$ are zero because
$z_i$ already is in the kernel of $d_n, \ldots, d_{i + 1}$.
This in turn uniquely gives
$$
x = x_0 + s_n(z_{n, 0}) + s_{n - 1}(z_{n - 1, 0}) + \ldots + s_0(z_{0, 0}) +
\sum\nolimits_{0 \leq j < i \leq n} s_i(s_j(z'_{i, j})).
$$
Continuing in this fashion we see that we in the end obtain
a decomposition of $x$ as a sum of terms
of the form
$$
s_{i_1} s_{i_2} \ldots s_{i_k} (z)
$$
with $n \geq i_1 > i_2 > \ldots > i_k \geq 0$ and
$z \in N(U_{n + 1 - k})$. This is exactly the required
decomposition, because any surjective map $[n + 1] \to [n + 1 - k]$
can be uniquely expressed in the form $s_{i_1} s_{i_2} \ldots s_{i_k}$
with $n \geq i_1 > i_2 > \ldots > i_k \geq 0$.
\end{proof}





\section{Skelet and coskelet functors}
\label{section-skelet}

\noindent
Let $\Delta_{\leq n}$ denote the full subcategory of
$\Delta$ with objects $[0], [1], [2], \ldots, [n]$.
Let $\mathcal{C}$ be a category.

\begin{definition}
\label{definition-truncated-simplicial-object}
An {\it $n$-truncated simplicial object of $\mathcal{C}$} 
is a contravariant functor from $\Delta_{\leq n}$ to
$\mathcal{C}$. A {\it morphism of $n$-truncated
simplicial objects} is a transformation of functors.
We denote the category of $n$-truncated
simplicial objects of $\mathcal{C}$ by
the symbol $\text{Simp}_n(\mathcal{C})$.
\end{definition}

\noindent
Given a simplicial object $U$ of $\mathcal{C}$
the truncation $\text{sk}_n U$ is the restriction
of $U$ to the subcategory $\Delta_{\leq n}$.
This defines a {\it skelet functor}
$$
\text{sk}_n :
\text{Simp}(\mathcal{C}) \longrightarrow \text{Simp}_n(\mathcal{C})
$$
from the category of simplicial objects of $\mathcal{C}$
to the category of $n$-truncated simplicial objects of $\mathcal{C}$.
See Remark \ref{remark-sk-literature} to avoid possible confusion
with other functors in the literature.

\medskip\noindent
The {\it coskelet functor} (if it exists) is a functor
$$
\text{cosk}_n :
\text{Simp}(\mathcal{C}) \longrightarrow \text{Simp}_n(\mathcal{C})
$$
which is right adjoint to the skelet functor. In a formula
\begin{equation}
\label{equation-cosk}
\text{Mor}_{\text{Simp}(\mathcal{C})}(U, \text{cosk}_n V)
=
\text{Mor}_{\text{Simp}_n(\mathcal{C})}(\text{sk}_n U, V)
\end{equation}
Given a $n$-truncated simplicial object $V$ we 
say that {\it $\text{cosk}_nV$ exists} if there
exists a $\text{cosk}_nV \in \text{Ob}(\text{Simp}(\mathcal{C}))$
and a morphism $\text{sk}_n \text{cosk}_n V \to V$
such that the displayed formula holds, in other words
if the functor
$U \mapsto \text{Mor}_{\text{Simp}_n(\mathcal{C})}(\text{sk}_n U, V)$
is representable. If it exists it
is unique up to unique isomorphism by the Yoneda lemma.
See Categories, Section \ref{categories-section-opposite}.

\begin{example}
\label{example-cosk0}
Suppose the category $\mathcal{C}$ has finite nonempty self products.
A $0$-truncated simplicial object of $\mathcal{C}$ is the same
as an object $X$ of $\mathcal{C}$. In this case
we claim that $\text{cosk}_0(X)$ is the simplicial
object $U$ with $U_n = X^{n + 1}$ the $(n + 1)$-fold self
product of $X$, and structure of simplicial object
as in Example \ref{example-fibre-products-simplicial-object}.
Namely, a morphism $V \to U$ where $V$ is a simplicial
object is given by morphisms $V_n \to X^{n + 1}$, such
that all the diagrams
$$
\xymatrix{
V_n \ar[r] \ar[d]_{V([0] \to [n], 0 \mapsto i)} &
X^{n + 1} \ar[d]^{\text{pr}_i} \\
V_0 \ar[r] &
X
}
$$
commute. Clearly this means that the map determines and is determined
by a unique morphism $V_0 \to X$. This proves that formula
(\ref{equation-cosk}) holds.
\end{example}

\noindent
Recall the category $\Delta/[n]$, see Example \ref{example-simplex-category}.
We let $(\Delta/[n])_{\leq m}$ denote the full subcategory
of $\Delta/[n]$ consisting of objects $[k] \to [n]$
of $\Delta/[n]$ with $k \leq m$. In other words we have
the following commutative diagram of categories and functors
$$
\xymatrix{
(\Delta/[n])_{\leq m} \ar[r] \ar[d] &
\Delta/[n] \ar[d] \\
\Delta_{\leq m} \ar[r] &
\Delta
}
$$
Given a $m$-truncated
simplicial object $U$ of $\mathcal{C}$
we define a functor
$$
U(n) : (\Delta/[n])_{\leq m}^{opp} \longrightarrow \mathcal{C}
$$
by the rules
\begin{eqnarray*}
([k] \to [n]) & \longmapsto & U_k \\
(\psi : ([k'] \to [n]) \to ([k] \to [n])) &
\longmapsto &
U(\psi) : U_k \to U_{k'}
\end{eqnarray*}
For a given morphism $\varphi : [n] \to [n']$ of $\Delta$
we have an associated functor
$$
"\varphi" : (\Delta/[n])_{\leq m} \longrightarrow (\Delta/[n'])_{\leq m}
$$
which maps $\alpha : [k] \to [n]$ to
$\varphi \circ \alpha : [k] \to [n']$.
The composition $U(n') \circ "\varphi"$ is
equal to the functor $U(n)$.

\begin{lemma}
\label{lemma-existence-cosk}
If the category $\mathcal{C}$ has finite limits, then
$\text{cosk}_m$ functors exist for all $m$. Moreover,
for any $m$-truncated simplicial object $U$ the
simplicial object $\text{cosk}_mU$ is described
by the formula
$$
(\text{cosk}_mU)_n = \text{lim}_{(\Delta/[n])_{\leq m}^{opp}}\ U(n)
$$
and for $\varphi : [n] \to [n']$ the map
$\text{cosk}_mU(\varphi)$ comes from the
identification $U(n') \circ "\varphi" = U(n)$ above 
via Categories, Lemma \ref{categories-lemma-functorial-limit}.
\end{lemma}

\begin{proof}
During the proof of this lemma we denote $\text{cosk}_mU$ the
simplicial object with $(\text{cosk}_mU)_n$ equal to
$\text{lim}_{(\Delta/[n])_{\leq m}^{opp}}\ U(n)$.
We will conclude at the end of the proof that it does
satsify the required mapping property.

\medskip\noindent
Suppose that $V$ is a simplicial object.
A morphism $\gamma : V \to \text{cosk}_mU$ is given by a sequence
of morphisms $\gamma_n : V_n \to (\text{cosk}_mU)_n$.
By definition of a limit, this is given by a
collection of morphisms $\gamma(\alpha) : V_n \to U_k$
where $\alpha$ ranges over all $\alpha : [k] \to [n]$
with $k \leq m$. These morphisms then also satisfy
the rules that
$$
\xymatrix{
V_n \ar[r]_{\gamma(\alpha)} &  U_k \\
V_{n'} \ar[r]^{\gamma(\alpha')} \ar[u]^{V(\varphi)} & U_{k'} \ar[u]_{U(\psi)}
}
$$
are commutative, given any $0 \leq k, k' \leq m$, $0 \leq n, n'$
and any $\psi : [k] \to [k']$, $\varphi : [n] \to [n']$,
$\alpha : [k] \to [n]$ and $\alpha' : [k'] \to [n']$ in $\Delta$
such that $\varphi \circ \alpha = \alpha' \circ \psi$.
Taking $n = k$, $\varphi = \alpha'$, and $\alpha = \psi = \text{id}_{[k]}$
we deduce that $\gamma(\alpha') = \gamma(\text{id}_{[k]}) \circ V(\alpha')$.
In other words, the morphisms $\gamma(\text{id}_{[k]})$, $k \leq m$
determine the morphism $\gamma$. And it is easy to see that these
morphisms form a morphism $\text{sk}_m V \to U$.

\medskip\noindent
Conversely, given a morphism $\gamma : \text{sk}_m V \to U$, 
we obtain a family of morphsms $\gamma(\alpha)$
where $\alpha$ ranges over all $\alpha : [k] \to [n]$
with $k \leq m$ by setting $\gamma(\alpha) = 
\gamma(\text{id}_{[k]}) \circ V(\alpha)$. These morphisms
satisfy all the displayed commutativity restraints pictured
above, and hence give rise to a morphism $V \to \text{cosk}_m U$.
\end{proof}

\begin{lemma}
\label{lemma-trivial-cosk}
Let $\mathcal{C}$ be a category.
Let $U$ be an $m$-truncated simplicial object of $\mathcal{C}$.
For $n \leq m$ the limit $\text{lim}_{(\Delta/[n])_{\leq m}^{opp}}\ U(n)$
exists and is canonically isomorphic to $U_n$.
\end{lemma}

\begin{proof}
This is true because the category $(\Delta/[n])_{\leq m}$
has an final object in this case, namely the identity
map $[n] \to [n]$.
\end{proof}

\noindent
Let us describe a particular instance of the coskelet functor in more detail.
By abuse of notation we will denote $\text{sk}_n$
also the restriction functor
$\text{Simp}_{n'}(\mathcal{C}) \to \text{Simp}_n(\mathcal{C})$
for any $n' \geq n$. We are going to describe a right adjoint
of the functor
$\text{sk}_n : \text{Simp}_{n + 1}(\mathcal{C})
\to \text{Simp}_n(\mathcal{C})$.
For $n \geq 1$, $0 \leq i < j \leq n + 1$
define $\delta^{n + 1}_{i,j} : [n - 1] \to [n + 1]$
to be the increasing map omitting $i$ and $j$.
Note that
$\delta^{n + 1}_{i,j} =
\delta^{n + 1}_j \circ \delta^n_i =
\delta^{n + 1}_i \circ \delta^n_{j - 1}$, see
Lemma \ref{lemma-relations-face-degeneracy}. This motivates
the following lemma.

\begin{lemma}
\label{lemma-formula-limit}
Let $n$ be an integer $\geq 1$.
Let $U$ be a $n$-truncated simplicial object of $\mathcal{C}$.
Consider the contravariant functor from $\mathcal{C}$ to
$\textit{Sets}$ which associates to an object $T$ the set
$$
\{ (f_0,\ldots,f_{n + 1}) \in \text{Mor}_{\mathcal{C}}(T, U_n)
\mid
d^n_{j - 1} \circ f_i = d^n_i \circ f_j\ 
\forall\ 0\leq i < j\leq n + 1\}
$$
If this functor is representable by some object $U_{n + 1}$
of $\mathcal{C}$, then
$$
U_{n + 1} = \text{lim}_{(\Delta/[n + 1])_{\leq n}^{opp}}\ U(n)
$$
\end{lemma}

\begin{proof}
The limit, if it exists, represents the functor
that associates to an object $T$ the set
$$
\{
(f_\alpha)_{\alpha : [k] \to [n + 1], k \leq n}
\mid
f_{\alpha \circ \psi} = U(\psi) \circ f_\alpha\ \forall\ 
\psi : [k'] \to [k], \alpha : [k] \to [n + 1]
\}.
$$
In fact we will show this functor is isomorphic to the
one displayed in the lemma. The map in one direction
is given by the rule
$$
(f_\alpha)_{\alpha}
\longmapsto
(f_{\delta^{n + 1}_0}, \ldots, f_{\delta^{n + 1}_{n + 1}}).
$$
This satisfies the conditions of the lemma because
$$
d^n_{j - 1} \circ f_{\delta^{n + 1}_i} =
f_{\delta^{n + 1}_i \circ \delta^n_{j - 1}} =
f_{\delta^{n + 1}_j \circ \delta^n_i} =
d^n_i \circ f_{\delta^{n + 1}_j}
$$
by the relations we recalled above the lemma. To construct a map
in the other direction we have to associate to a system
$(f_0, \ldots, f_{n + 1})$ as in the displayed formula
of the lemma a system of maps $f_\alpha$. Let $\alpha : [k] \to [n + 1]$
be given. Since $k \leq n$ the map $\alpha$ is not surjective.
Hence we can write $\alpha = \delta^{n + 1}_i \circ \psi$
for some $0 \leq i \leq n + 1$ and some
$\psi : [k] \to [n]$. We have no choice but to define
$$
f_\alpha = U(\psi) \circ f_i.
$$
Of course we have to check that this is independent of the
choice of the pair $(i, \psi)$. First, observe that given $i$
there is a unique $\psi$ which works. Second, suppose that $(j, \phi)$ is
another pair. Then $i \not = j$ and we may assume $i < j$. Since
both $i, j$ are not in the image of $\alpha$ we may actually
write $\alpha = \delta^{n + 1}_{i, j} \circ \xi$ and then
we see that $\psi = \delta^n_{j - 1} \circ \xi$ and
$\phi = \delta^n_i \circ \xi$. Thus
\begin{eqnarray*}
U(\psi) \circ f_i & = & U(\delta^n_{j - 1} \circ \xi) \circ f_i \\
& = & U(\xi) \circ d^n_{j - 1} \circ f_i \\
& = & U(\xi) \circ d^n_i \circ f_j \\
& = & U(\delta^n_i \circ \xi) \circ f_j \\
& = & U(\phi) \circ f_j
\end{eqnarray*}
as desired. We still have to verify that the maps
$f_\alpha$ so defined satisfy the rules of a system
of maps $(f_\alpha)_\alpha$. To see this suppose that
$\psi : [k'] \to [k]$, $\alpha : [k] \to [n + 1]$ with
$k, k' \leq n$. Set $\alpha' = \alpha \circ \psi$.
Choose $i$ not in the image of $\alpha$. Then clearly
$i$ is not in the image of $\alpha'$ also. Write
$\alpha = \delta^n_i \circ \phi$ (we cannot use the letter $\psi$ here
because we've already used it). Then obviously
$\alpha' = \delta^n_i \circ \phi \circ \psi$. By construction above
we then have
$$
U(\psi) \circ f_\alpha = U(\psi) \circ U(\phi) \circ f_i
= U(\phi \circ \psi) \circ f_i = f_{\alpha \circ \psi} = f_{\alpha'}
$$
as desired. We leave to the reader the pleasant task of verifying
that our constructions are mutually inverse bijections, and are
functorial in $T$.
\end{proof}

\begin{lemma}
\label{lemma-work-out}
Let $n$ be an integer $\geq 1$. Let $U$ be a $n$-truncated
simplicial object of $\mathcal{C}$. Consider the
contravariant functor from $\mathcal{C}$ to $\textit{Sets}$
which associates to an object $T$ the set
$$
\{ (f_0,\ldots,f_{n + 1}) \in \text{Mor}_{\mathcal{C}}(T, U_n)
\mid
d^n_{j - 1} \circ f_i = d^n_i \circ f_j\ 
\forall\ 0\leq i < j\leq n + 1\}
$$
If this functor is representable by some object $U_{n + 1}$
of $\mathcal{C}$, then there exists an $(n + 1)$-truncated
simplicial object $\tilde U$, with $\text{sk}_n \tilde U = U$
and $\tilde U_{n + 1} = U_{n + 1}$ such that the following
adjointness holds
$$
\text{Mor}_{\text{Simp}_{n + 1}(\mathcal{C})}(V, \tilde U)
=
\text{Mor}_{\text{Simp}_n(\mathcal{C})}(\text{sk}_nV, U)
$$
\end{lemma}

\begin{proof}
By Lemma \ref{lemma-trivial-cosk} there are identifications
$$
U_i = \text{lim}_{(\Delta/[i])_{\leq n}^{opp}}\ U(i)
$$
for $0 \leq i \leq n$. By Lemma \ref{lemma-formula-limit}
we have
$$
U_{n + 1} = \text{lim}_{(\Delta/[n + 1])_{\leq n}^{opp}}\ U(n).
$$
Thus we may define for any $\varphi : [i] \to [j]$
with $i, j \leq n + 1$ the corresponding map
$\tilde U(\varphi) : \tilde U_j \to \tilde U_i$ exactly as
in Lemma \ref{lemma-existence-cosk}. This defines
an $(n + 1)$-truncated simplicial object $\tilde U$
with $\text{sk}_n \tilde U = U$.

\medskip\noindent
To see the adjointness we argue as follows. Given any element
$\gamma : \text{sk}_n V \to U$ of the right hand side of the formula
consider the morphisms
$f_i = \gamma_n \circ d^{n+1}_i : V_{n+1} \to V_n \to U_n$.
These clearly satisfy the relations $d^n_{j - 1} \circ f_i = d^n_i \circ f_j$
and hence define a unique morphism $V_{n + 1} \to U_{n + 1}$
by our choice of $U_{n + 1}$.
Conversely, given a morphsm $\gamma' : V \to \tilde U$
of the left hand side we can simply restrict to
$\Delta_{\leq n}$ to get an element of the right hand side.
We leave it to the reader to show these are mutually inverse
constructions. See also
Remark \ref{remark-cosk-simplicial-sets}.
\end{proof}

\begin{remark}
\label{remark-explicit-face-degeneracy}
Let $U$, and $U_{n + 1}$ be as in Lemma \ref{lemma-work-out}.
On $T$-valued points we can easily describe the face
and degeneracy maps of $\tilde U$.
Explicitly, the maps $d^{n + 1}_i : U_{n + 1} \to U_n$
are given by
$$
(f_0, \ldots, f_{n + 1}) \longmapsto f_i.
$$
And the maps $s^n_j : U_n \to U_{n + 1}$ are given by
\begin{eqnarray*}
f & \longmapsto & (
s^{n - 1}_{j - 1} \circ d^{n - 1}_0 \circ f,\\ 
& &
s^{n - 1}_{j - 1} \circ d^{n - 1}_1 \circ f,\\
& &
\ldots\\
& &
s^{n - 1}_{j - 1} \circ d^{n - 1}_{j - 1} \circ f, \\
& &
f,\\
& &
f,\\
& &
s^{n - 1}_j \circ d^{n - 1}_{j + 1} \circ f,\\
& &
s^{n - 1}_j \circ d^{n - 1}_{j + 2} \circ f,\\
& &
\ldots\\
& &
s^{n - 1}_j \circ d^{n - 1}_n \circ f
)
\end{eqnarray*}
where we leave it to the reader to verify that the RHS
is an element of the displayed set of Lemma \ref{lemma-work-out}.
For $n = 0$ there is one map, namely $f \mapsto (f, f)$.
For $n = 1$ there are two maps, namely
$f \mapsto (f, f, s_0d_1f)$ and
$f \mapsto (s_0d_0f, f, f)$.
For $n = 2$ there are three maps, namely
$f \mapsto (f, f, s_0d_1f, s_0d_2f)$,
$f \mapsto (s_0d_0f, f, f, s_1d_2f)$, and
$f \mapsto (s_1d_0f, s_1d_1f, f, f)$.
And so on and so forth.
\end{remark}

\begin{remark}
\label{remark-cosk-simplicial-sets}
The construction of Lemma \ref{lemma-work-out}
above in the case of simplicial
sets is the following. Given an $n$-truncated simplicial
set $U$, we make a canonical $(n + 1)$-truncated simplicial
set $\tilde U$ as follows. We add a set of $(n + 1)$-simplices
$U_{n + 1}$ by the formula of the lemma. Namely,
an element of $U_{n + 1}$ is a numbered collection of
$(f_0,\ldots,f_{n + 1})$ of $n$-simplices,
with the property that they glue
as they would in a $(n + 1)$-simplex. In other words,
the $i$th face of $f_j$ is the $(j-1)$st face of $f_i$
for $i < j$. Geometrically it is obvious how to define the
face and degeneracy maps for $\tilde U$.
If $V$ is an $(n + 1)$-truncated simplicial set,
then its $(n + 1)$-simplices give rise to compatible collections
of $n$-simplices $(f_0, \ldots, f_{n + 1})$ with $f_i \in V_n$.
Hence there is a natural map
$\text{Mor}(\text{sk}_nV, U) \to \text{Mor}(V, \tilde U)$
which is inverse to the canonical restriction mapping
the other way.

\medskip\noindent
Also, it is enough to do the combinatorics of the
construction in the case of truncated simplicial sets.
Namely, for any object $T$ of the category $\mathcal{C}$,
and any $n$-truncated simplicial object $U$ of $\mathcal{C}$
we can consider the $n$-truncated simplicial set
$\text{Mor}(T, U)$. We may apply the construction to this,
and take its set of $(n + 1)$-simplices, and require this to be
representable. This is a good way to think about
the result of Lemma \ref{lemma-work-out}.
\end{remark}

\begin{remark}
\label{remark-inductive-coskelet}
{\it Inductive construction of coskelets.}
Suppose that $\mathcal{C}$ is a category with
finite limits. Suppose that $U$ is an $m$-truncated
simplicial object in $\mathcal{C}$. Then we can
inductively construct $n$-truncated objects $U^n$ as
follows:
\begin{enumerate}
\item To start, set $U^m = U$.
\item Given $U^n$ for $n \geq m$ set $U^{n + 1} = \tilde U^n$,
where $\tilde U^n$ is constructed from $U^n$ as in Lemma
\ref{lemma-work-out}.
\end{enumerate}
Since the construction of Lemma \ref{lemma-work-out} has
the property that it leaves the $n$-skeleton of $U^n$
unchanged, we can then define $\text{cosk}_m U$ to be
the simplicial object with
$(\text{cosk}_m U)_n = U^n_n = U^{n + 1}_n = \ldots$.
And it follows formally from Lemma \ref{lemma-work-out}
that $U^n$ satisfies the formula
$$
\text{Mor}_{\text{Simp}_n(\mathcal{C})}(V, U^n) 
=
\text{Mor}_{\text{Simp}_m(\mathcal{C})}(\text{sk}_mV, U)
$$
for all $n \geq m$. It also then follows formally 
from this that
$$
\text{Mor}_{\text{Simp}(\mathcal{C})}(V, \text{cosk}_mU) 
=
\text{Mor}_{\text{Simp}_m(\mathcal{C})}(\text{sk}_mV, U)
$$
with $\text{cosk}_mU$ chosen as above.
\end{remark}

\begin{lemma}
\label{lemma-cosk-up}
Let $\mathcal{C}$ be a category which has finite limits.
\begin{enumerate}
\item For every $n$ the functor $\text{sk}_n : \text{Simp}(\mathcal{C})
\to \text{Simp}_n(\mathcal{C})$ has a right adjoint $\text{cosk}_n$.
\item For every $n' \geq n$ the functor
$\text{sk}_n : \text{Simp}_{n'}(\mathcal{C}) \to \text{Simp}_n(\mathcal{C})$
has a right adjoint, namely $\text{sk}_{n'}\text{cosk}_n$.
\item For every $m \geq n \geq 0$ and every $n$-truncated simplicial
object $U$ of $\mathcal{C}$ we have
$\text{cosk}_m \text{sk}_m \text{cosk}_n U = \text{cosk}_n U$.
\item If $U$ is a simplicial object of $\mathcal{C}$ such that
the canonical map
$U \to \text{cosk}_n \text{sk}_nU$
is an isomorphism for some $n \geq 0$, then the canonical map
$U \to \text{cosk}_m \text{sk}_mU$
is an isomorphism for all $m \geq n$.
\end{enumerate}
\end{lemma}

\begin{proof}
The existence in (1) follows from Lemma \ref{lemma-existence-cosk} above
and the equality in (2), and (3) follows from the discussion
in Remark \ref{remark-inductive-coskelet}. After this (4) is obvious.
\end{proof}

\begin{lemma}
\label{lemma-cosk-product}
Let $U$, $V$ be $n$-truncated simplicial objects of a
category $\mathcal{C}$. Then
$$
\text{cosk}_n (U \times V) = \text{cosk}_nU \times \text{cosk}_nV
$$
whenever the left and right hand sides exist.
\end{lemma}

\begin{proof}
Let $W$ be a simplicial object. We have
\begin{eqnarray*}
\text{Mor}(W, \text{cosk}_n (U \times V))
& = &
\text{Mor}(\text{sk}_n W, U \times V) \\
& = &
\text{Mor}(\text{sk}_n W, U)
\times
\text{Mor}(\text{sk}_nW, V) \\
& = &
\text{Mor}(W, \text{cosk}_n U)
\times
\text{Mor}(W, \text{cosk}_n V) \\
& = &
\text{Mor}(W, \text{cosk}_n U \times \text{cosk}_n V)
\end{eqnarray*}
The lemma follows.
\end{proof}

\begin{lemma}
\label{lemma-simplex-cosk}
The canonical map
$\Delta[n] \to \text{cosk}_1 \text{sk}_1 \Delta[n]$
is an isomorphism.
\end{lemma}

\begin{proof}
Consider a simplicial set $U$ and a morphism
$f : U \to \Delta[n]$. This is a rule that associates
to each $u \in U_i$ a map $f_u : [i] \to [n]$ in $\Delta$.
Furthermore, these maps should have the property that
$f_u \circ \varphi = f_{U(\varphi)(u)}$ for any 
$\varphi : [j] \to [i]$. Denote $\epsilon^i_j : [0] \to [i]$
the map which maps $0$ to $j$. Denote $F : U_0 \to [n]$
the map $u \mapsto f_u(0)$. Then we see that
$$
f_u(j) = F(\epsilon^i_j(u))
$$
for all $0 \leq j \leq i$ and $u \in U_i$.
In particular, if we know the function $F$
then we know the maps $f_u$ for all $u\in U_i$ all $i$.
Conversely, given a map $F : U_0 \to [n]$,
we can set for any $i$, and any $u \in U_i$ 
and any $0 \leq j \leq i$
$$
f_u(j) = F(\epsilon^i_j(u))
$$
This does not in general define a morphism $f$ of simplicial sets
as above. Namely, the condition is that all the maps $f_u$ are
nondecreasing. This clearly is equivalent to the condition
that $F(\epsilon^i_j(u)) \leq F(\epsilon^i_{j'}(u))$
whenever $0 \leq j \leq j' \leq i$ and $u \in U_i$. But in this
case the morphisms
$$
\epsilon^i_j, \epsilon^i_{j'} : [0] \to [i]
$$
both factor through the map
$\epsilon^i_{j, j'} : [1] \to [i]$ defined by the rules
$0 \mapsto j$, $1 \mapsto j'$.
In other words, it is enough to check the inequalities for
$i = 1$ and $u \in X_1$. In other words, we have
$$
\text{Mor}(U, \Delta[n])
=
\text{Mor}(\text{sk}_1 U, \text{sk}_1 \Delta[n])
$$
as desired.
\end{proof}








\section{Left adjoints to the skeleton functors}
\label{section-adjoint-left}

\noindent
In this section we construct a left adjoint $i_{m, !}$
of the skeleton functor $\text{sk}_m$ in certain cases.
The adjointness formula is
$$
\text{Mor}_{\text{Simp}_m(\mathcal{C})}(U,\text{sk}_mV)
=
\text{Mor}_{\text{Simp}(\mathcal{C})}(i_{m, !}U,V).
$$
It turns out that this left adjoint exists when
the category $\mathcal{C}$ has finite colimits.

\medskip\noindent
We use a similar construction as in Section \ref{section-skelet}.
Recall the category $[n]/\Delta$ of objects
under $[n]$, see
Categories, Example \ref{categories-example-category-under-X}.
Its objects are morphisms $\alpha : [n] \to [k]$
and its morphisms are commutative triangles.
We let $([n]/\Delta)_{\leq m}$ denote the full subcategory
of $[n]/\Delta$ consisting of objects $[n] \to [k]$
with $k \leq m$. Given a $m$-truncated
simplicial object $U$ of $\mathcal{C}$
we define a functor
$$
U(n) : ([n]/\Delta)_{\leq m}^{opp} \longrightarrow \mathcal{C}
$$
by the rules
\begin{eqnarray*}
([n] \to [k]) & \longmapsto & U_k \\
(\psi : ([n] \to [k']) \to ([n] \to [k]))
& \longmapsto &
U(\psi) : U_k \to U_{k'}
\end{eqnarray*}
For a given morphism $\varphi : [n] \to [n']$ of $\Delta$
we have an associated functor
$$
"\varphi" : ([n']/\Delta)_{\leq m} \longrightarrow ([n]/\Delta)_{\leq m}
$$
which maps $\alpha : [n'] \to [k]$ to
$\varphi \circ \alpha : [n] \to [k]$.
The composition $U(n) \circ "\varphi"$ is
equal to the functor $U(n')$.

\begin{lemma}
\label{lemma-left-adjoint-exists}
Let $\mathcal{C}$ be a category which has finite colimits.
The functors $i_{m, !}$ exist for all $m$.
Let $U$ be an $m$-truncated simplicial object of $\mathcal{C}$.
The simplicial object $i_{m, !}U$
is described by the formula
$$
(i_{m, !}U)_n = \text{colim}_{([n]/\Delta)_{\leq m}^{opp}} U(n)
$$
and for $\varphi : [n] \to [n']$ the map
$i_{m, !}U(\varphi)$ comes from the
identification $U(n) \circ "\varphi" = U(n')$ above 
via Categories, Lemma \ref{categories-lemma-functorial-colimit}.
\end{lemma}

\begin{proof}
In this proof we denote $i_{m, !}U$ the simplicial object
whose $n$th term is given by the displayed formula of the 
lemma. We will show it satisfies the adjointness property.

\medskip\noindent
Let $V$ be a simplicial object of $\mathcal{C}$.
Let $\gamma : U \to \text{sk}_mV$ be given.
A morphism
$$
\text{colim}_{([n]/\Delta)_{\leq m}^{opp}}\ U(n) \to T
$$
is given by a compatible system of morphisms
$f_\alpha : U_k \to T$ where $\alpha : [n] \to [k]$
with $k \leq m$. Certainly, we have such a system of
morphisms by taking the compositions 
$$
U_k \xrightarrow{\gamma_k} V_k \xrightarrow{V(\alpha)} V_n.
$$
Hence we get an induced morphism $(i_{m, !}U)_n \to V_n$.
We leave it to the reader to see that these form a 
morphism of simplicial objects $\gamma' : i_{m, !}U \to V$.

\medskip\noindent
Coversely, given a morphism $\gamma' : i_{m, !}U \to V$ we obtain
a morphism $\gamma : U \to \text{sk}_m V$ by setting
$\gamma_i : U_i \to V_i$ equal to the composition
$$
U_i
\xrightarrow{\text{id}_{[i]}}
\text{colim}_{([i]/\Delta)_{\leq m}^{opp}}\ U(i)
\xrightarrow{\gamma'_i}
V_i
$$
for $0 \leq i \leq n$. We leave it to the reader to see that
this is the inverse of the construction above.
\end{proof}

\noindent
The following lemma implies that
$\text{sk}_mi_{m, !}U = U$ in the situation
of the lemma above.

\begin{lemma}
\label{lemma-recovering-U}
Let $\mathcal{C}$ be a category.
Let $U$ be an $m$-truncated simplicial object of $\mathcal{C}$.
For any $n \leq m$ the colimit
$$
\text{colim}_{([n]/\Delta)_{\leq m}^{opp}} U(n)
$$
exists and is equal to $U_n$.
\end{lemma}

\begin{proof}
This is so because the category $([n]/\Delta)_{\leq m}$
has an initial object, namely $\text{id} : [n] \to [n]$.
\end{proof}

\begin{lemma}
\label{lemma-imshriek-sets}
If $U$ is an $m$-truncated simplicial set and $n > m$
then all $n$-simplices of $i_{m, !}U$ are degenerate.
\end{lemma}

\begin{proof}
This can be seen directly from the construction of
$i_{m, !}U$ in Lemma \ref{lemma-left-adjoint-exists},
but we can also argue directly as follows.
Write $V = i_{m, !}U$. For $i \leq m$ set $V'_i = V_i$
and for $i > m$ set $V'_i$ equal to the set of simplices
that are in the image of $U_j \to U_i$ for some
$j \leq m$. It is straightforward to see that $V'$
is a sub-simplicial set. By the adjunction formula,
since $\text{sk}_m V' = U$, there is an inverse to the
injection $V' \to V$. Hence $V' = V$.
\end{proof}

\begin{remark}
\label{remark-sk-literature}
In some texts the composite functor
$$
\text{Simp}(\mathcal{C})
\xrightarrow{\text{sk}_m}
\text{Simp}_m(\mathcal{C})
\xrightarrow{i_{m,!}}
\text{Simp}(\mathcal{C})
$$
is denoted $\text{sk}_m$. This makes sense because in the case
of simplicial sets we see from Lemma \ref{lemma-imshriek-sets}
that $i_{m,!} \text{sk}_m V$ is just the sub simplicial set
of $V$ consisting of all $i$-simplices of $V$, $i \leq m$
and their degeneracies. In those texts it is also customary
to denote the composition
$$
\text{Simp}(\mathcal{C})
\xrightarrow{\text{sk}_m}
\text{Simp}_m(\mathcal{C})
\xrightarrow{\text{cosk}_m}
\text{Simp}(\mathcal{C})
$$
by $\text{cosk}_m$.
\end{remark}




\section{Homotopies}
\label{section-homotopy}

\noindent
Consider the simplicial sets $\Delta[0]$ and $\Delta[1]$.
Recall that there are two morphisms
$$
e_i : \Delta[0] \longrightarrow \Delta[1],
$$
coming from the morphisms $[0] \to [1]$ mapping 
$0$ to $i \in \{0, 1\}$. Recall also that each
set $\Delta[1]_k$ is finite. Hence, if the category
$\mathcal{C}$ has finite coproducts, then we can
form the product
$$
U \times \Delta[1]
$$
for any simplicial object $U$ of $\mathcal{C}$, exactly
as in Definition \ref{definition-product}. (More generally
this works if finite coproducts $\coprod_{i=1}^N U_n$
of the objects $U_n$ exist.)
Note that $\Delta[0]$ has the property that $\Delta[0]_k = \{*\}$
is a singleton for all $k \geq 0$. Hence $U \times \Delta[0]
= U$. Thus $e_i$ above gives rise to morphisms
$$
e_i : U \to U \times \Delta[1].
$$

\begin{definition}
\label{definition-homotopy}
Suppose that $U$ and $V$ are two simplicial objects
of $\mathcal{C}$. Assume that $U \times \Delta[1]$ exists.
We say morphisms $a, b : U \to V$ are {\it homotopic}
if there exists a morphism
$$
h : U \times \Delta[1] \longrightarrow V
$$
such that $a = h \circ e_0$ and $b = h \circ e_1$.
In this case $h$ is called a {\it homotopy connecting
$a$ and $b$}.
\end{definition}








\section{A homotopy equivalence}
\label{section-homotopy-equivalence}

\noindent
Suppose that $A$, $B$ are sets, and that $f : A \to B$
is a map. Consider the associated map of
simplicial sets
$$
\xymatrix{
\text{cosk}_0(A) \ar@{=}[r] &
\Big(
\ldots
A\times A \times A
\ar[d]
\ar@<2ex>[r]
\ar@<0ex>[r]
\ar@<-2ex>[r]
&
A \times A
\ar[d]
\ar@<1ex>[r]
\ar@<-1ex>[r]
\ar@<1ex>[l]
\ar@<-1ex>[l]
&
A
\ar[d]
\ar@<0ex>[l]
\Big)
\\
\text{cosk}_0(B) \ar@{=}[r] &
\Big(
\ldots
B\times B \times B
\ar@<2ex>[r]
\ar@<0ex>[r]
\ar@<-2ex>[r]
&
B \times B
\ar@<1ex>[r]
\ar@<-1ex>[r]
\ar@<1ex>[l]
\ar@<-1ex>[l]
&
B
\ar@<0ex>[l]
\Big)
}
$$
See Example \ref{example-cosk0}.
The case $n = 0$ of the following lemma
says that this map of simplicial sets
has a section if $f$ is surjective.
The proof: choose a section of $f$.

\begin{lemma}
\label{lemma-section}
Let $f : V \to U$ be a morphism of simplicial sets.
Let $n \geq 0$ be an integer.
Assume
\begin{enumerate}
\item The map $f_i : V_i \to U_i$ is a bijection for $i < n$.
\item The map $f_n : V_n \to U_n$ is a surjection.
\item The canonical morphism $U \to \text{cosk}_n \text{sk}_n U$
is an isomorphism.
\item The canonical morphism $V \to \text{cosk}_n \text{sk}_n V$
is an isomorphism.
\end{enumerate}
Then there exists a morphism of simplicial sets $g : U \to V$
such that $f \circ g = \text{id}_U$.
\end{lemma}

\begin{proof}
By Lemma \ref{lemma-splitting-simplicial-sets}
both $U$ and $V$ have canonical splittings with $N(U_i)$
and $N(V_i)$ equal to the sets of nondegenerate simplices.
We have to find maps $g_m : U_m \to V_m$ for $m \geq 0$ such
that
\begin{eqnarray}
d^k_i \circ g_k & = & g_{k - 1} \circ d^k_i \label{cd}\\
s^k_i \circ g_k & = & g_{k + 1} \circ s^k_i \label{cs}
\end{eqnarray}
for all $k$. By induction on $m$ we will show that we can find maps
$g_0, \ldots, g_m$ such that (\ref{cd}) holds for
$1 \leq k \leq m$ and (\ref{cs}) holds for $0 \leq k \leq m - 1$.
We set $g_i$ equal to the inverse of $f_i$ for $i = 0, \ldots, n - 1$.
Clearly the induction hypothesis holds for $m = n - 1$.
We define $g_n : U_n \to V_n$ as follows.
Pick $u \in U_n$, then
\begin{enumerate}
\item if $u$ is degenerate, write  $u = U(\varphi)(u')$
for some nondegenerate $u' \in U_m$ and some
surjective $\varphi : [n] \to [m]$. We set
$g_n(u) = V(\varphi)(g_m(u'))$. This is well defined
as the pair $(\varphi, u')$ is unique.
\item if $u$ is nondegenerate, we choose any $v \in V_n$
mapping to $u$ and we set $g_n(u) = v$.
\end{enumerate}
This choice of $g_n$ garantees that the induction hypothesis
holds for $m = n$. Namely, we forced (\ref{cs}) with $k = n - 1$
by our choice of $g_n$ on degenerate simplices, and (\ref{cd})
with $k = n$ holds because the equality takes place in
$V_{n - 1} = U_{n - 1}$.

\medskip\noindent
One way to finish the proof at this point is to show
that the family of maps $g_0, \ldots, g_n$ defines
a morphism of $n$-truncated simplicial sets
$\text{sk}_n U \to \text{sk}_n V$ which is
a right inverse to $\text{sk}_nf$. Then since
$\text{cosk}_n$ is a functor and by the hypothesis
of the lemma we get $g$ as $\text{cosk}_n(g_0, \ldots, g_n)$.
But we can also see this directly as follows.

\medskip\noindent
Given the induction hypothesis for $m \geq n$
we inductively define $g_{m + 1}$ as follows.
Since $U \to \text{cosk}_n \text{sk}_n U$
is an isomorphism, we see that also
$U \to \text{cosk}_m \text{sk}_m U$ is an
isomorphism. Hence elements of $U_{m + 1}$
are $(m + 2)$-tuples $(u_0, \ldots, u_{m + 1})$ with
$u_i \in U_m$ satisfying the equalities
$d^m_{j - 1}(u_i) = d^m_i(u_j)\ \forall\ 0\leq i < j\leq m + 1$.
Similarly for $V_{m + 1}$.
Thus we may simply map the element
$(u_0, \ldots, u_{m + 1})$ to the element
$(g_m(u_0), \ldots, g_m(u_{m + 1}))$.
To verify the induction hypothesis for $m + 1$ with
this choice of $g_{m + 1}$ we will use the
explicit form of the maps $d_i$ and $s_i$
as given in Remark \ref{remark-explicit-face-degeneracy}.
This remark shows immediately that the commutation of
$g_0, \ldots, g_m$ with $d_i$ and $s_i$ implies the
desired commutation for $g_{m + 1}$.
\end{proof}

\noindent
Let $A, B$ be sets. Let $f^0, f^1 : A \to B$ be maps of sets.
Consider the induced maps $f^0, f^1 : \text{cosk}_0(A) \to \text{cosk}_0(B)$
abusively denoted by the same symbols. The following lemma for $n = 0$
says that $f_0$ is homotopic to $f_1$. In fact, the
homotopy is given by the map $h : \text{cosk}_0(A) \times
\Delta[1] \to \text{cosk}_0(A)$ with components
\begin{eqnarray*}
h_m : A \times \ldots \times A \times \text{Mor}_{\Delta}([m], [1])
& \longrightarrow &
A \times \ldots \times A, \\
(a_0, \ldots, a_m, \alpha) & \longmapsto &
(f^{\alpha(0)}(a_0), \ldots, f^{\alpha(m)}(a_m))
\end{eqnarray*}
To check that this works, note that for a map $\varphi : [k] \to [m]$
the induced maps are
$(a_0, \ldots, a_m) \mapsto (a_{\varphi(0)}, \ldots, a_{\varphi(k)})$
and $\alpha \mapsto \alpha \circ \varphi$. Thus $h = (h_m)_{m \geq 0}$
is clearly a map of simplicial sets as desired.

\begin{lemma}
\label{lemma-homotopy}
Let $f^0, f^1 : V \to U$ be maps of a simplicial sets.
Let $n \geq 0$ be an integer.
Assume
\begin{enumerate}
\item The maps $f^j_i : V_i \to V_i$, $j = 0, 1$ are equal for $i < n$.
\item The canonical morphism $U \to \text{cosk}_n \text{sk}_n U$
is an isomorphism.
\item The canonical morphism $V \to \text{cosk}_n \text{sk}_n V$
is an isomorphism.
\end{enumerate}
Then $f^0$ is homotopic to $f^1$.
\end{lemma}

\begin{proof}
We have to construct
a morphism of simplicial sets $h : V \times \Delta[1] \to U$
which recovers $f^i$ on composing with $e_i$.
The case $n = 0$ was dealt with above the lemma.
Thus we may assume that $n \geq 1$.
The map $\Delta[1] \to \text{cosk}_1 \text{sk}_1 \Delta[1]$
is an isomorphism, see Lemma \ref{lemma-simplex-cosk}.
Thus we see that $\Delta[1] \to \text{cosk}_n \text{sk}_n \Delta[1]$
is an isomorphism as $n \geq 1$, see
Lemma \ref{lemma-cosk-up}. And hence $V \times \Delta[1] \to
\text{cosk}_n \text{sk}_n (V \times \Delta[1])$
is an isomorphism too, see Lemma \ref{lemma-cosk-product}.
In other words, in order to construct the homotopy
it suffices to construct a suitable
morphism of $n$-truncated simplicial sets
$h : \text{sk}_n V \times \text{sk}_n \Delta[1] \to \text{sk}_n U$.

\medskip\noindent
For $k = 0, \ldots, n - 1$ we define $h_k$ by the
formula $h_k(v, \alpha) = f^0(v) = f^1(v)$.
The map $h_n : V_n \times \text{Mor}_{\Delta}([k], [1]) \to U_n$
is defined as follows. Pick $v \in V_n$ and $\alpha : [n] \to [1]$:
\begin{enumerate}
\item If $\text{Im}(\alpha) = \{0\}$, then we set $h_n(v, \alpha) = f^0(v)$.
\item If $\text{Im}(\alpha) = \{0, 1\}$, then we set $h_n(v, \alpha) = f^0(v)$.
\item If $\text{Im}(\alpha) = \{1\}$, then we set $h_n(v, \alpha) = f^1(v)$.
\end{enumerate}
Let $\varphi : [k] \to [l]$ be a morphism of $\Delta_{\leq n}$.
We will show that the diagram
$$
\xymatrix{
V_{[l]} \times \text{Mor}([l], [1]) \ar[r] \ar[d] &
U_{[l]} \ar[d] \\
V_{[k]} \times \text{Mor}([k], [1]) \ar[r] &
U_{[k]}
}
$$
commutes.
Pick $v \in V_{[l]}$ and $\alpha : [l] \to [1]$.
The commutativity means that
$$
h_k(V(\varphi)(v), \alpha \circ \varphi)
=
U(\varphi)(h_l(v, \alpha)).
$$
In almost every case this holds because
$h_k(V(\varphi)(v), \alpha \circ \varphi) = f^0(V(\varphi)(v))$
and $U(\varphi)(h_l(v, \alpha)) = U(\varphi)(f^0(v))$, combined
with the fact that $f^0$ is a morphism of simplicial sets.
The only cases where this does not hold is when
either (A) $\text{Im}(\alpha) = \{1\}$ and $l = n$
or (B) $\text{Im}(\alpha \circ \varphi) = \{1\}$ and $k = n$.
Observe moreover that necessarily $f^0(v) = f^1(v)$
for any degenerate $n$-simplex of $V$.
Thus we can narrow the cases above down even further
to the cases (A) $\text{Im}(\alpha) = \{1\}$, $l = n$
and $v$ nondegenerate, and (B)
$\text{Im}(\alpha \circ \varphi) = \{1\}$, $k = n$
and $V(\varphi)(v)$ nondegenerate.

\medskip\noindent
In case (A), we see that also $\text{Im}(\alpha \circ \varphi) = \{1\}$.
Hence we see that not only $h_l(v, \alpha) = f^1(v)$ but also
$h_k(V(\varphi)(v), \alpha \circ \varphi) = f^1(V(\varphi)(v))$.
Thus we see that the relation holds because $f^1$ is a morphism
of simplicial sets.

\medskip\noindent
In case (B) we conclude that $l = k = n$ and
$\varphi$ is bijective, since otherwise $V(\varphi)(v)$
is degenerate. Thus $\varphi = \text{id}_{[n]}$, which is a trivial case.
\end{proof}

\begin{lemma}
\label{lemma-equiv}
With assumptions and notation as in Lemma \ref{lemma-section}
above. The composition $g \circ f$ is homotopy equivalent
to the identity on $V$.
\end{lemma}

\begin{proof}
Immediate from Lemma \ref{lemma-homotopy} above.
\end{proof}















\section{Covering simplicial objects}
\label{section-making-simplicial}

\noindent
Let $\mathcal{C}$ be a category.
Let $U$ be a simplicial object of $\mathcal{C}$.
Suppose $n\geq 0$, and suppose $\pi : V \to U_n$ is
a representable morphism of $\mathcal{C}$. This
means that the fibre products $V \times_{U_n} W$
exist for all morphisms $W \to U_n$.

\medskip\noindent
For any $k \geq 0$ consider the fibre product over $U_k$
$$
U'_k = \prod\nolimits_{\varphi \in \text{Mor}_\Delta([n],[k])}
V\times_{U_n, U(\varphi)} U_k.
$$
By our assumption on the morphism $V \to U_n$ this fibre product
exists. For any $T \in \text{Ob}(\mathcal{C})$ the set of morphism
$T \to U'_k$ is given by the following formula
$$ 
\{
(f : T \to U_k, (f_\varphi : T \to V)_{\varphi \in \text{Mor}_\Delta([n],[k])})
\mid
\pi \circ f_{\varphi} = U(\varphi) \circ f\ \forall \varphi
\}
$$
For any $\psi : [l] \to [k]$ there is a canonical morphism
$U'_k \to U'_l$ coming from the map $\text{Mor}_\Delta([n],[l])
\to \text{Mor}_\Delta([n],[k]), \varphi \mapsto \varphi \circ \psi$,
the identity map on $V$ and the morphism
$U(\psi) : U_k \to U_l$. In terms of the $T$-valued points
it maps $(f, (f_\varphi)_\varphi)$ to the collection
$(U(\psi) \circ f, (f_{\varphi \circ \psi})_{\varphi})$.

\medskip\noindent
Clearly, this gives rise to a simplicial object $U'$ of
$\mathcal{C}$. The natural morphisms $U'_m \to U_m$ give rise to a
morphism of simplicial objects $U' \to U$. Note that
the morphism $U'_n \to U_n$ factors throught the morphism $V \to U_n$
by projection onto the factor corresponding to $\varphi=\text{id}_{[n]}$.

\begin{lemma}
\label{lemma-construct-new-covers}
Suppose that $U$ and $V\to U_n$ are as above.
The morphism of simplicial objects 
$U' \to U$ constructed above has the following 
properties:
(1) The morphism $U'_n \to U_n$ factors trough $V \to U_n$.
(2) For any $m$ the morphism $U'_m \to U_m$
is a fibre product over $U_m$ of base changes
of the morphism $V \to U_n$.
\end{lemma}























\section{Hypercoverings}
\label{section-hypercoverings}

\noindent
Let $\mathcal{C}$ be a site, see
Sites Definition \ref{sites-definition-site}.
Let $X$ be an object of $\mathcal{C}$.
Given an abelian sheaf $\mathcal{F}$
on $\mathcal{C}$ we would like to compute
its cohomology groups
$$
H^i(X, \mathcal{F}).
$$
According to our general definitions
(insert future reference here)
this cohomology group is computed by
choosing an injective resolution
$$
0 \to \mathcal{F} \to \mathcal{I}^0 \to \mathcal{I}^1 \to \ldots
$$
and setting
$$
H^i(X, \mathcal{F})
=
H^i(\ \Gamma(X, \mathcal{I}^0) \to \Gamma(X, \mathcal{I}^1) \to \ldots\ )
$$
We will have to do quite a bit of work to prove that we
may also compute these cohomology groups without
choosing an injective resolution. In order to do
so we make the following definition.

\begin{definition}
\label{definition-SR}
Let $\mathcal{C}$ be a site.
Let $X \in \text{Ob}(\mathcal{C})$ be an object of $\mathcal{C}$.
We denote $\text{SR}(\mathcal{C}, X)$ the category with
\begin{enumerate}
\item objects are families of morphisms
$\{U_i \to X\}_{i \in I}$ where each $U_i \to X$ is
representable, and
\item morphisms $\{U_i \to X\}_{i \in I} \to
\{V_j \to X\}_{j \in J}$ are given by
a map $\alpha : I \to J$ and for each $i \in I$
a representable morphism $f_i : U_i \to V_{\alpha(i)}$ over $X$.
\end{enumerate}
\end{definition}

\noindent
This is a category because the composition of
representable morphisms is representable, see
Categories, Lemma \ref{categories-lemma-composition-representable}.
This definition is different from the one in
\cite[Expose V, Sec. 7]{SGA4}, but it seems flexible
enough to do all the required arguments.
Note that this is a ``big'' category. We will later
``bound'' the size of the index sets $I$ that we need
and we can then redefine $\text{SR}(\mathcal{C}, X)$ 
to become a category.

\begin{lemma}
\label{lemma-coprod-prod-SR}
Let $\mathcal{C}$ be a site.
Let $X \in \text{Ob}(\mathcal{C})$ be an object of $\mathcal{C}$.
The category $\text{SR}(\mathcal{C}, X)$ has
coproducts and finite limits.
\end{lemma}

\begin{proof}
It is clear that the coproduct of
$\{U_i \to X\}_{i \in I}$ and $\{V_j \to X\}_{j \in J}$
is $\{U_i \to X\}_{i \in I} \coprod \{V_j \to X\}_{j \in J}$
and similarly for coproducts of
families of families of representable morphisms with target $X$.
The object $\{X \to X\}$ is a final
object of $\text{SR}(\mathcal{C}, X)$.
Suppose given a morphism
$(\alpha, f_i) : \{U_i \to X\}_{i \in I} \to \{V_j \to X\}_{j \in J}$
and a morphism
$(\beta, g_k) : \{W_k \to X\}_{k \in K} \to \{V_j \to X\}_{j \in J}$.
The fibred product of these morphisms is given by
$$
\{ U_i \times_{f_i, V_j, g_k} W_k \to X \}_{(i,j,k) \in I\times J\times K
\text{ such that } k = \alpha(i) = \beta(j)}
$$
The fibre products exist by the assumption that the morphisms
$f_i$ and $g_j$ are representable. In addition, it follows
from Categories, Lemmas
\ref{categories-lemma-composition-representable} and
\ref{categories-lemma-base-change-representable}
that $U_i \times_{f_i, V_j, g_k} W_k \to X$ is representable.
Thus $\text{SR}(\mathcal{C}, X)$ has finite limits,
see Categories, Lemma \ref{categories-lemma-finite-limits-exist}.
\end{proof}

\begin{definition}
\label{definition-covering-SR}
Let $\mathcal{C}$ be a site.
Let $X \in \text{Ob}(\mathcal{C})$ be an object of $\mathcal{C}$.
Let $f = (\alpha, f_i) : \{U_i \to X\}_{i \in I} \to \{V_j \to X\}_{j \in J}$
be a morphism in the category $\text{SR}(\mathcal{C}, X)$.
We say that $f$ is a {\it covering} if for every $j \in J$ the
family of morphisms $\{U_i \to V_j\}_{i \in I, \alpha(i) = j}$
is a covering for the site $\mathcal{C}$.
\end{definition}

\noindent
According to the results earlier in this chapter the
coskelet of a truncated simplicial object of
$\text{SR}(\mathcal{C}, X)$ exists. Hence the following
definition makes sense.

\begin{definition}
\label{definition-hypercovering}
Let $\mathcal{C}$ be a site.
Let $X \in \text{Ob}(\mathcal{C})$ be an object of $\mathcal{C}$.
A {\it hypercovering} of $X$ is a simplicial object
$U$ in the category $\text{SR}(\mathcal{C}, X)$ such that
\begin{enumerate}
\item The object $U_0$ is a covering of $X$ for the site $\mathcal{C}$.
\item
For every $n \geq 0$ the canonical morphism
$$
U_{n + 1} \longrightarrow (\text{cosk}_n \text{sk}_n U)_{n + 1}
$$
is a covering in the sense defined above.
\end{enumerate}
\end{definition}

\noindent
Condition (1) makes sense since each object of
$\text{SR}(\mathcal{C}, X)$ is after all a family
of morphisms with target $X$.

\begin{example}
\label{example-cech}
Let $\{U_i \to X\}_{i \in I}$ be a covering of the site $\mathcal{C}$.
Set $U_0 = \{U_i \to X\}_{i \in I}$. Note that
$U_0 \in \text{Ob}(\text{SR}(\mathcal{C}, X))$ since every
$U_i \to X$ is representable by the axioms of a site.
Then $U_0$ is a $0$-truncated simplicial object of
$\text{SR}(\mathcal{C}, X)$. Hence we may form
$$
U = \text{cosk}_0 U_0.
$$
Clearly $U$ passes condition (1) of Definition \ref{definition-hypercovering}.
Since all the morphisms $U_{n + 1} \to (\text{cosk}_n \text{sk}_n U)_{n + 1}$
are isomorphisms it also passes condition (2). Note that
the terms $U_n$ are the usual
$$
U_n = \{
U_{i_0} \times_X U_{i_1} \times_X \ldots \times_X U_{i_n} \to X
\}_{(i_0, i_1, \ldots, i_n) \in I^{n + 1}}
$$
\end{example}

\begin{lemma}
\label{lemma-hypercoverings-set}
Let $\mathcal{C}$ be a site.
Let $X \in \text{Ob}(\mathcal{C})$ be an object of $\mathcal{C}$.
The collection of all hypercoverings of $X$ forms a set.
\end{lemma}

\begin{proof}
Since $\mathcal{C}$ is a site, we see that the collection
of possible $U_0$ forms a set. Suppose we have shown that
the collection of all possible $U_0, \ldots, U_n$ form
a set. Then it is enough to show that given
$U_0, \ldots, U_n$ the collection of all possible
$U_{n + 1}$ forms a set. And this is clearly true since
we have to choose $U_{n + 1}$ among all possible coverings
of $(\text{cosk}_n \text{sk}_n U)_{n + 1}$.
\end{proof}

\begin{remark}
\label{remark-hypercoverings-really-set}
The lemma does not just say that there is a cofinal
system of choices of hypercoverings that is a set,
but that really the hypercoverings form a set.
\end{remark}

\section{Acyclicity}
\label{section-acyclicity}

\noindent





\section{The general case}

\noindent
Mention how things work more generally, for example if $\mathcal{C}$
does not have the property that coverings consisting of a single map
are cofinal. State the theorem in the correct generality.

\section{Other chapters}

\begin{multicols}{2}
\begin{enumerate}
\item \hyperref[introduction-section-phantom]{Introduction}
\item \hyperref[conventions-section-phantom]{Conventions}
\item \hyperref[sets-section-phantom]{Set Theory}
\item \hyperref[categories-section-phantom]{Categories}
\item \hyperref[topology-section-phantom]{Topology}
\item \hyperref[sheaves-section-phantom]{Sheaves on Spaces}
\item \hyperref[algebra-section-phantom]{Commutative Algebra}
\item \hyperref[sites-section-phantom]{Sites and Sheaves}
\item \hyperref[homology-section-phantom]{Homological Algebra}
\item \hyperref[derived-section-phantom]{Derived Categories}
\item \hyperref[more-algebra-section-phantom]{More Algebra}
\item \hyperref[simplicial-section-phantom]{Simplicial Methods}
\item \hyperref[modules-section-phantom]{Sheaves of Modules}
\item \hyperref[sites-modules-section-phantom]{Modules on Sites}
\item \hyperref[injectives-section-phantom]{Injectives}
\item \hyperref[cohomology-section-phantom]{Cohomology of Sheaves}
\item \hyperref[sites-cohomology-section-phantom]{Cohomology on Sites}
\item \hyperref[hypercovering-section-phantom]{Hypercoverings}
\item \hyperref[schemes-section-phantom]{Schemes}
\item \hyperref[constructions-section-phantom]{Constructions of Schemes}
\item \hyperref[properties-section-phantom]{Properties of Schemes}
\item \hyperref[morphisms-section-phantom]{Morphisms of Schemes}
\item \hyperref[coherent-section-phantom]{Coherent Cohomology}
\item \hyperref[divisors-section-phantom]{Divisors}
\item \hyperref[limits-section-phantom]{Limits of Schemes}
\item \hyperref[varieties-section-phantom]{Varieties}
\item \hyperref[chow-section-phantom]{Chow Homology}
\item \hyperref[topologies-section-phantom]{Topologies on Schemes}
\item \hyperref[descent-section-phantom]{Descent}
\item \hyperref[more-morphisms-section-phantom]{More on Morphisms}
\item \hyperref[flat-section-phantom]{More on Flatness}
\item \hyperref[groupoids-section-phantom]{Groupoid Schemes}
\item \hyperref[more-groupoids-section-phantom]{More on Groupoid Schemes}
\item \hyperref[etale-section-phantom]{\'Etale Morphisms of Schemes}
\item \hyperref[etale-cohomology-section-phantom]{\'Etale Cohomology}
\item \hyperref[spaces-section-phantom]{Algebraic Spaces}
\item \hyperref[spaces-properties-section-phantom]{Properties of Algebraic Spaces}
\item \hyperref[spaces-morphisms-section-phantom]{Morphisms of Algebraic Spaces}
\item \hyperref[spaces-topologies-section-phantom]{Topologies on Algebraic Spaces}
\item \hyperref[spaces-descent-section-phantom]{Descent and Algebraic Spaces}
\item \hyperref[spaces-more-morphisms-section-phantom]{More on Morphisms of Spaces}
\item \hyperref[quot-section-phantom]{Quot and Hilbert Spaces}
\item \hyperref[stacks-section-phantom]{Stacks}
\item \hyperref[spaces-groupoids-section-phantom]{Groupoids in Algebraic Spaces}
\item \hyperref[spaces-more-groupoids-section-phantom]{More on Groupoids in Spaces}
\item \hyperref[bootstrap-section-phantom]{Bootstrap}
\item \hyperref[examples-stacks-section-phantom]{Examples of Stacks}
\item \hyperref[groupoids-quotients-section-phantom]{Quotients of Groupoids}
\item \hyperref[algebraic-section-phantom]{Algebraic Stacks}
\item \hyperref[criteria-section-phantom]{Criteria for Representability}
\item \hyperref[stacks-properties-section-phantom]{Properties of Algebraic Stacks}
\item \hyperref[stacks-morphisms-section-phantom]{Morphisms of Algebraic Stacks}
\item \hyperref[examples-section-phantom]{Examples}
\item \hyperref[exercises-section-phantom]{Exercises}
\item \hyperref[guide-section-phantom]{Guide to Literature}
\item \hyperref[desirables-section-phantom]{Desirables}
\item \hyperref[coding-section-phantom]{Coding Style}
\item \hyperref[fdl-section-phantom]{GNU Free Documentation License}
\item \hyperref[index-section-phantom]{Auto Generated Index}
\end{enumerate}
\end{multicols}


\bibliography{my}
\bibliographystyle{alpha}

\end{document}
