\IfFileExists{stacks-project.cls}{%
\documentclass{stacks-project}
}{%
\documentclass{amsart}
}

% The following AMS packages are automatically loaded with
% the amsart documentclass:
%\usepackage{amsmath}
%\usepackage{amssymb}
%\usepackage{amsthm}

% For dealing with references we use the comment environment
\usepackage{verbatim}
\newenvironment{reference}{\comment}{\endcomment}
%\newenvironment{reference}{}{}
\newenvironment{slogan}{\comment}{\endcomment}
\newenvironment{history}{\comment}{\endcomment}

% For commutative diagrams you can use
% \usepackage{amscd}
\usepackage[all]{xy}

% We use 2cell for 2-commutative diagrams.
\xyoption{2cell}
\UseAllTwocells

% To put source file link in headers.
% Change "template.tex" to "this_filename.tex"
% \usepackage{fancyhdr}
% \pagestyle{fancy}
% \lhead{}
% \chead{}
% \rhead{Source file: \url{template.tex}}
% \lfoot{}
% \cfoot{\thepage}
% \rfoot{}
% \renewcommand{\headrulewidth}{0pt}
% \renewcommand{\footrulewidth}{0pt}
% \renewcommand{\headheight}{12pt}

\usepackage{multicol}

% For cross-file-references
\usepackage{xr-hyper}

% Package for hypertext links:
\usepackage{hyperref}

% For any local file, say "hello.tex" you want to link to please
% use \externaldocument[hello-]{hello}
\externaldocument[introduction-]{introduction}
\externaldocument[conventions-]{conventions}
\externaldocument[sets-]{sets}
\externaldocument[categories-]{categories}
\externaldocument[topology-]{topology}
\externaldocument[sheaves-]{sheaves}
\externaldocument[sites-]{sites}
\externaldocument[stacks-]{stacks}
\externaldocument[fields-]{fields}
\externaldocument[algebra-]{algebra}
\externaldocument[brauer-]{brauer}
\externaldocument[homology-]{homology}
\externaldocument[derived-]{derived}
\externaldocument[simplicial-]{simplicial}
\externaldocument[more-algebra-]{more-algebra}
\externaldocument[smoothing-]{smoothing}
\externaldocument[modules-]{modules}
\externaldocument[sites-modules-]{sites-modules}
\externaldocument[injectives-]{injectives}
\externaldocument[cohomology-]{cohomology}
\externaldocument[sites-cohomology-]{sites-cohomology}
\externaldocument[dga-]{dga}
\externaldocument[dpa-]{dpa}
\externaldocument[hypercovering-]{hypercovering}
\externaldocument[schemes-]{schemes}
\externaldocument[constructions-]{constructions}
\externaldocument[properties-]{properties}
\externaldocument[morphisms-]{morphisms}
\externaldocument[coherent-]{coherent}
\externaldocument[divisors-]{divisors}
\externaldocument[limits-]{limits}
\externaldocument[varieties-]{varieties}
\externaldocument[topologies-]{topologies}
\externaldocument[descent-]{descent}
\externaldocument[perfect-]{perfect}
\externaldocument[more-morphisms-]{more-morphisms}
\externaldocument[flat-]{flat}
\externaldocument[groupoids-]{groupoids}
\externaldocument[more-groupoids-]{more-groupoids}
\externaldocument[etale-]{etale}
\externaldocument[chow-]{chow}
\externaldocument[intersection-]{intersection}
\externaldocument[pic-]{pic}
\externaldocument[adequate-]{adequate}
\externaldocument[dualizing-]{dualizing}
\externaldocument[duality-]{duality}
\externaldocument[discriminant-]{discriminant}
\externaldocument[local-cohomology-]{local-cohomology}
\externaldocument[curves-]{curves}
\externaldocument[resolve-]{resolve}
\externaldocument[models-]{models}
\externaldocument[pione-]{pione}
\externaldocument[etale-cohomology-]{etale-cohomology}
\externaldocument[proetale-]{proetale}
\externaldocument[crystalline-]{crystalline}
\externaldocument[spaces-]{spaces}
\externaldocument[spaces-properties-]{spaces-properties}
\externaldocument[spaces-morphisms-]{spaces-morphisms}
\externaldocument[decent-spaces-]{decent-spaces}
\externaldocument[spaces-cohomology-]{spaces-cohomology}
\externaldocument[spaces-limits-]{spaces-limits}
\externaldocument[spaces-divisors-]{spaces-divisors}
\externaldocument[spaces-over-fields-]{spaces-over-fields}
\externaldocument[spaces-topologies-]{spaces-topologies}
\externaldocument[spaces-descent-]{spaces-descent}
\externaldocument[spaces-perfect-]{spaces-perfect}
\externaldocument[spaces-more-morphisms-]{spaces-more-morphisms}
\externaldocument[spaces-flat-]{spaces-flat}
\externaldocument[spaces-groupoids-]{spaces-groupoids}
\externaldocument[spaces-more-groupoids-]{spaces-more-groupoids}
\externaldocument[bootstrap-]{bootstrap}
\externaldocument[spaces-pushouts-]{spaces-pushouts}
\externaldocument[groupoids-quotients-]{groupoids-quotients}
\externaldocument[spaces-more-cohomology-]{spaces-more-cohomology}
\externaldocument[spaces-simplicial-]{spaces-simplicial}
\externaldocument[spaces-duality-]{spaces-duality}
\externaldocument[formal-spaces-]{formal-spaces}
\externaldocument[restricted-]{restricted}
\externaldocument[spaces-resolve-]{spaces-resolve}
\externaldocument[formal-defos-]{formal-defos}
\externaldocument[defos-]{defos}
\externaldocument[cotangent-]{cotangent}
\externaldocument[examples-defos-]{examples-defos}
\externaldocument[algebraic-]{algebraic}
\externaldocument[examples-stacks-]{examples-stacks}
\externaldocument[stacks-sheaves-]{stacks-sheaves}
\externaldocument[criteria-]{criteria}
\externaldocument[artin-]{artin}
\externaldocument[quot-]{quot}
\externaldocument[stacks-properties-]{stacks-properties}
\externaldocument[stacks-morphisms-]{stacks-morphisms}
\externaldocument[stacks-limits-]{stacks-limits}
\externaldocument[stacks-cohomology-]{stacks-cohomology}
\externaldocument[stacks-perfect-]{stacks-perfect}
\externaldocument[stacks-introduction-]{stacks-introduction}
\externaldocument[stacks-more-morphisms-]{stacks-more-morphisms}
\externaldocument[stacks-geometry-]{stacks-geometry}
\externaldocument[moduli-]{moduli}
\externaldocument[moduli-curves-]{moduli-curves}
\externaldocument[examples-]{examples}
\externaldocument[exercises-]{exercises}
\externaldocument[guide-]{guide}
\externaldocument[desirables-]{desirables}
\externaldocument[coding-]{coding}
\externaldocument[obsolete-]{obsolete}
\externaldocument[fdl-]{fdl}
\externaldocument[index-]{index}

% Theorem environments.
%
\theoremstyle{plain}
\newtheorem{theorem}[subsection]{Theorem}
\newtheorem{proposition}[subsection]{Proposition}
\newtheorem{lemma}[subsection]{Lemma}

\theoremstyle{definition}
\newtheorem{definition}[subsection]{Definition}
\newtheorem{example}[subsection]{Example}
\newtheorem{exercise}[subsection]{Exercise}
\newtheorem{situation}[subsection]{Situation}

\theoremstyle{remark}
\newtheorem{remark}[subsection]{Remark}
\newtheorem{remarks}[subsection]{Remarks}

\numberwithin{equation}{subsection}

% Macros
%
\def\lim{\mathop{\mathrm{lim}}\nolimits}
\def\colim{\mathop{\mathrm{colim}}\nolimits}
\def\Spec{\mathop{\mathrm{Spec}}}
\def\Hom{\mathop{\mathrm{Hom}}\nolimits}
\def\Ext{\mathop{\mathrm{Ext}}\nolimits}
\def\SheafHom{\mathop{\mathcal{H}\!\mathit{om}}\nolimits}
\def\SheafExt{\mathop{\mathcal{E}\!\mathit{xt}}\nolimits}
\def\Sch{\mathit{Sch}}
\def\Mor{\operatorname{Mor}\nolimits}
\def\Ob{\mathop{\mathrm{Ob}}\nolimits}
\def\Sh{\mathop{\mathit{Sh}}\nolimits}
\def\NL{\mathop{N\!L}\nolimits}
\def\proetale{{pro\text{-}\acute{e}tale}}
\def\etale{{\acute{e}tale}}
\def\QCoh{\mathit{QCoh}}
\def\Ker{\mathop{\mathrm{Ker}}}
\def\Im{\mathop{\mathrm{Im}}}
\def\Coker{\mathop{\mathrm{Coker}}}
\def\Coim{\mathop{\mathrm{Coim}}}

%
% Macros for moduli stacks/spaces
%
\def\QCohstack{\mathcal{QC}\!\mathit{oh}}
\def\Cohstack{\mathcal{C}\!\mathit{oh}}
\def\Spacesstack{\mathcal{S}\!\mathit{paces}}
\def\Quotfunctor{\mathrm{Quot}}
\def\Hilbfunctor{\mathrm{Hilb}}
\def\Curvesstack{\mathcal{C}\!\mathit{urves}}
\def\Polarizedstack{\mathcal{P}\!\mathit{olarized}}
\def\Complexesstack{\mathcal{C}\!\mathit{omplexes}}
% \Pic is the operator that assigns to X its picard group, usage \Pic(X)
% \Picardstack_{X/B} denotes the Picard stack of X over B
% \Picardfunctor_{X/B} denotes the Picard functor of X over B
\def\Pic{\mathop{\mathrm{Pic}}\nolimits}
\def\Picardstack{\mathcal{P}\!\mathit{ic}}
\def\Picardfunctor{\mathrm{Pic}}
\def\Deformationcategory{\mathcal{D}\!\mathit{ef}}


% OK, start here.
%
\begin{document}

\title{Simplicial Spaces}


\maketitle

\phantomsection
\label{section-phantom}

\tableofcontents

\section{Introduction}
\label{section-introduction}

\noindent
This chapter develops some theory concerning simplicial topological spaces,
simplicial ringed spaces, simplicial schemes, and simplicial algebraic spaces.
The theory of simplicial spaces sometimes allows one to prove local to global
principles which appear difficult to prove in other ways.
Some example applications can be found in the papers
\cite{faltings_finiteness}, \cite{Kiehl}, and \cite{HodgeIII}.

\medskip\noindent
We assume throughout that the reader is familiar with the basic concepts
and results of the chapter Simplicial Methods, see
Simplicial, Section \ref{simplicial-section-introduction}.
In particular, we continue to write $X$ and not $X_\bullet$
for a simplicial object.








\section{Simplicial topological spaces}
\label{section-simplicial-top}

\noindent
A {\it simplicial space} is a simplicial object in the category of
topological spaces where morphisms are continuous maps of topological
spaces. (We will use ``simplicial algebraic space'' to refer to simplicial
objects in the category of algebraic spaces.)
We may picture a simplicial space $X$ as follows
$$
\xymatrix{
X_2
\ar@<2ex>[r]
\ar@<0ex>[r]
\ar@<-2ex>[r]
&
X_1
\ar@<1ex>[r]
\ar@<-1ex>[r]
\ar@<1ex>[l]
\ar@<-1ex>[l]
&
X_0
\ar@<0ex>[l]
}
$$
Here there are two morphisms $d^1_0, d^1_1 : X_1 \to X_0$
and a single morphism $s^0_0 : X_0 \to X_1$, etc.
It is important to keep in mind that $d^n_i : X_n \to X_{n - 1}$
should be thought of as a ``projection forgetting the
$i$th coordinate'' and $s^n_j : X_n \to X_{n + 1}$ as the diagonal
map repeating the $j$th coordinate.

\medskip\noindent
Let $X$ be a simplicial space. We associate a site
$X_{Zar}$\footnote{This notation is similar to the notation in
Sites, Example \ref{sites-example-site-topological}
and
Topologies, Definition \ref{topologies-definition-big-small-Zariski}.}
to $X$ as follows.
\begin{enumerate}
\item An object of $X_{Zar}$ is an open $U$ of $X_n$ for some $n$,
\item a morphism $U \to V$ of $X_{Zar}$ is given by a
$\varphi : [m] \to [n]$ where $n, m$ are such that
$U \subset X_n$, $V \subset X_m$ and $\varphi$ is such that
$X(\varphi)(U) \subset V$, and
\item a covering $\{U_i \to U\}$ in $X_{Zar}$ means
that $U, U_i \subset X_n$ are open, the maps $U_i \to U$ are
given by $\text{id} : [n] \to [n]$, and $U = \bigcup U_i$.
\end{enumerate}
Note that in particular, if $U \to V$ is a morphism of $X_{Zar}$
given by $\varphi$, then $X(\varphi) : X_n \to X_m$ does in fact
induce a continuous map $U \to V$ of topological spaces.

\noindent
It is clear that the above is a special case of a construction that
associates to any diagram of topological spaces a site. We formulate
the obligatory lemma.

\begin{lemma}
\label{lemma-simplicial-site}
Let $X$ be a simplicial space. Then $X_{Zar}$
as defined above is a site.
\end{lemma}

\begin{proof}
Omitted.
\end{proof}

\noindent
Let $X$ be a simplicial space. Let $\mathcal{F}$ be a sheaf on $X_{Zar}$.
It is clear from the definition of coverings, that the restriction
of $\mathcal{F}$ to the opens of $X_n$ defines a sheaf $\mathcal{F}_n$
on the topological space $X_n$. For every $\varphi : [m] \to [n]$ the
restriction maps of $\mathcal{F}$ for pairs $U \subset X_n$, $V \subset X_m$
with $X(\varphi)(U) \subset V$, define an $X(\varphi)$-map
$\mathcal{F}(\varphi) : \mathcal{F}_m \to \mathcal{F}_n$, see
Sheaves, Definition \ref{sheaves-definition-f-map}.
Moreover, given $\varphi : [m] \to [n]$ and $\psi : [l] \to [m]$
we have
$$
\mathcal{F}(\varphi) \circ \mathcal{F}(\psi) =
\mathcal{F}(\varphi \circ \psi)
$$
(LHS uses composition of $f$-maps, see
Sheaves, Definition \ref{sheaves-definition-composition-f-maps}).
Clearly, the converse is true as well: if we have a system
$(\{\mathcal{F}_n\}_{n \geq 0},
\{\mathcal{F}(\varphi)\}_{\varphi \in \text{Arrows}(\Delta)})$
as above, satisfying the displayed equalities,
then we obtain a sheaf on $X_{Zar}$.

\begin{lemma}
\label{lemma-describe-sheaves-simplicial-site}
Let $X$ be a simplicial space. There is an equivalence of
categories between
\begin{enumerate}
\item $\Sh(X_{Zar})$, and
\item category of systems $(\mathcal{F}_n, \mathcal{F}(\varphi))$
described above.
\end{enumerate}
\end{lemma}

\begin{proof}
See discussion above.
\end{proof}

\begin{lemma}
\label{lemma-simplicial-space-site-functorial}
Let $f : Y \to X$ be a morphism of simplicial spaces.
Then the functor $u : X_{Zar} \to Y_{Zar}$
which associates to the open $U \subset X_n$ the open
$f_n^{-1}(U) \subset Y_n$ defines a morphism of sites
$f_{Zar} : Y_{Zar} \to X_{Zar}$.
\end{lemma}

\begin{proof}
It is clear that $u$ is a continuous functor. Hence we obtain functors
$f_{Zar, *} = u^s$ and $f_{Zar}^{-1} = u_s$, see
Sites, Section \ref{sites-section-morphism-sites}.
To see that we obtain a morphism of sites we have to show
that $u_s$ is exact. We will use
Sites, Lemma \ref{sites-lemma-directed-morphism} to see this.
Let $V \subset Y_n$ be an open subset. The category
$\mathcal{I}_V^u$ (see Sites, Section \ref{sites-section-functoriality-PSh})
consists of pairs $(U, \varphi)$ where
$\varphi : [m] \to [n]$ and $U \subset X_m$ open such that
$Y(\varphi)(V) \subset f_m^{-1}(U)$. Moreover, a morphism
$(U, \varphi) \to (U', \varphi')$ is given by a
$\psi : [m'] \to [m]$ such that $X(\psi)(U) \subset U'$
and $\varphi \circ \psi = \varphi'$.
It is our task to show that $\mathcal{I}_V^u$ is cofiltered.

\medskip\noindent
We verify the conditions of
Categories, Definition \ref{categories-definition-codirected}.
Condition (1) holds because $(X_n, \text{id}_{[n]})$ is an object.
Let $(U, \varphi)$ be an object. The condition
$Y(\varphi)(V) \subset f_m^{-1}(U)$ is equivalent to
$V \subset f_n^{-1}(X(\varphi)^{-1}(U))$. Hence we obtain a morphism
$(X(\varphi)^{-1}(U), \text{id}_{[n]}) \to (U, \varphi)$ given
by setting $\psi = \varphi$. Moreover, given a pair of objects
of the form $(U, \text{id}_{[n]})$ and $(U', \text{id}_{[n]})$
we see there exists an object, namely $(U \cap U', \text{id}_{[n]})$,
which maps to both of them. Thus condition (2) holds.
To verify condition (3) suppose given two morphisms
$a, a': (U, \varphi) \to (U', \varphi')$ given by $\psi, \psi' : [m'] \to [m]$.
Then precomposing with the morphism
$(X(\varphi)^{-1}(U), \text{id}_{[n]}) \to (U, \varphi)$ given
by $\varphi$ equalizes $a, a'$ because
$\varphi \circ \psi = \varphi' = \varphi \circ \psi'$.
This finishes the proof.
\end{proof}

\begin{lemma}
\label{lemma-describe-functoriality}
Let $f : Y \to X$ be a morphism of simplicial spaces. In terms of the
description of sheaves in
Lemma \ref{lemma-describe-sheaves-simplicial-site} the
morphism $f_{Zar}$ of Lemma \ref{lemma-simplicial-space-site-functorial}
can be described as follows.
\begin{enumerate}
\item If $\mathcal{G}$ is a sheaf on $Y$, then
$(f_{Zar, *}\mathcal{G})_n = f_{n, *}\mathcal{G}_n$.
\item If $\mathcal{F}$ is a sheaf on $X$, then
$(f_{Zar}^{-1}\mathcal{F})_n = f_n^{-1}\mathcal{F}_n$.
\end{enumerate}
\end{lemma}

\begin{proof}
The first part is immediate from the definitions. For the second part, note
that in the proof of
Lemma \ref{lemma-simplicial-space-site-functorial}
we have shown that for a $V \subset Y_n$ open the category
$(\mathcal{I}_V^u)^{opp}$ contains as a cofinal subcategory
the category of opens $U \subset X_n$ with $f_n^{-1}(U) \supset V$
and morphisms given by inclusions. Hence we see that the restriction
of $u_p\mathcal{F}$ to opens of $Y_n$ is the presheaf
$f_{n, p}\mathcal{F}_n$ as defined in
Sheaves, Lemma \ref{sheaves-lemma-pullback-presheaves}.
Since $f_{Zar}^{-1}\mathcal{F} = u_s\mathcal{F}$ is the sheafification
of $u_p\mathcal{F}$ and since sheafification uses only coverings and
since coverings in $Y_{Zar}$ use only inclusions between opens on the
same $Y_n$, the result follows from the fact that $f_n^{-1}\mathcal{F}_n$
is (correspondingly) the sheafification of $f_{n, p}\mathcal{F}_n$, see
Sheaves, Section \ref{sheaves-section-presheaves-functorial}.
\end{proof}

\noindent
Let $X$ be a topological space. In
Sites, Example \ref{sites-example-site-topological}
we denoted $X_{Zar}$ the site consisting of opens of $X$
with inclusions as morphisms and coverings given by open coverings.
We identify the topos $\Sh(X_{Zar})$ with the category
of sheaves on $X$.

\begin{lemma}
\label{lemma-restriction-to-components}
Let $X$ be a simplicial space. The functor
$X_{n, Zar} \to X_{Zar}$, $U \mapsto U$ is continuous
and cocontinuous. The associated morphism of
topoi $g_n : \Sh(X_n) \to \Sh(X_{Zar})$ satisfies
\begin{enumerate}
\item $g_n^{-1}$ associates to the sheaf $\mathcal{F}$ on $X$
the sheaf $\mathcal{F}_n$ on $X_n$,
\item $g_n^{-1} : \Sh(X_{Zar}) \to \Sh(X_n)$ has a left adjoint $g^{Sh}_{n!}$,
\item $g^{Sh}_{n!}$ commutes with finite connected limits,
\item $g_n^{-1} : \textit{Ab}(X_{Zar}) \to \textit{Ab}(X_n)$
has a left adjoint $g_{n!}$, and
\item $g_{n!}$ is exact.
\end{enumerate}
\end{lemma}

\begin{proof}
Besides the properties of our functor mentioned in the statement,
the category $X_{n, Zar}$ has fibre products and equalizers
and the functor commutes with them (beware that $X_{Zar}$ does not
have all fibre products). Hence the lemma follows from the discussion in
Sites, Sections \ref{sites-section-cocontinuous-functors} and
\ref{sites-section-cocontinuous-morphism-topoi}
and
Modules on Sites, Section \ref{sites-modules-section-exactness-lower-shriek}.
More precisely,
Sites, Lemmas \ref{sites-lemma-cocontinuous-morphism-topoi},
\ref{sites-lemma-when-shriek}, and
\ref{sites-lemma-preserve-equalizers}
and
Modules on Sites, Lemmas
\ref{sites-modules-lemma-g-shriek-adjoint} and
\ref{sites-modules-lemma-exactness-lower-shriek}.
\end{proof}

\begin{lemma}
\label{lemma-restriction-injective-to-component}
Let $X$ be a simplicial space. If $\mathcal{I}$ is an injective abelian
sheaf on $X_{Zar}$, then $\mathcal{I}_n$ is an injective abelian sheaf
on $X_n$.
\end{lemma}

\begin{proof}
This follows from
Homology, Lemma \ref{homology-lemma-adjoint-preserve-injectives}
and
Lemma \ref{lemma-restriction-to-components}.
\end{proof}

\begin{lemma}
\label{lemma-restriction-to-components-functorial}
Let $f : Y \to X$ be a morphism of simplicial spaces. Then
$$
\xymatrix{
\Sh(Y_n) \ar[d] \ar[r]_{f_n} & \Sh(X_n) \ar[d] \\
\Sh(Y_{Zar}) \ar[r]^{f_{Zar}} & \Sh(X_{Zar})
}
$$
is a commutative diagram of topoi.
\end{lemma}

\begin{proof}
Direct from the description of pullback functors in
Lemmas \ref{lemma-describe-functoriality} and
\ref{lemma-restriction-to-components}.
\end{proof}

\begin{lemma}
\label{lemma-augmentation}
Let $Y$ be a simplicial space and let $a : Y \to X$ be an augmentation
(Simplicial, Definition \ref{simplicial-definition-augmentation}).
Let $a_n : Y_n \to X$ be the corresponding morphisms of topological spaces.
There is a canonical morphism of topoi
$$
a : \Sh(Y_{Zar}) \to \Sh(X)
$$
with the following properties:
\begin{enumerate}
\item $a^{-1}\mathcal{F}$ is the sheaf restricting to $a_n^{-1}\mathcal{F}$
on $Y_n$,
\item $a_m \circ Y(\varphi) = a_n$ for all $\varphi : [m] \to [n]$,
\item $a \circ g_n = a_n$ as morphisms of topoi with
$g_n$ as in Lemma \ref{lemma-restriction-to-components},
\item $a_*\mathcal{G}$ for $\mathcal{G} \in \Sh(Y_{Zar})$
is the equalizer of the two maps
$a_{0, *}\mathcal{G}_0 \to a_{1, *}\mathcal{G}_1$.
\end{enumerate}
\end{lemma}

\begin{proof}
Part (2) holds for augmentations of simplicial objects in any category.
Thus $Y(\varphi)^{-1} a_m^{-1} \mathcal{F} = a_n^{-1}\mathcal{F}$
which defines an $Y(\varphi)$-map from $a_m^{-1}\mathcal{F}$
to $a_n^{-1}\mathcal{F}$.
Thus we can use (1) as the definition of $a^{-1}\mathcal{F}$ (using
Lemma \ref{lemma-describe-sheaves-simplicial-site}) and
(4) as the definition of $a_*$. If this defines a morphism of topoi
then part (3) follows because we'll have $g_n^{-1} \circ a^{-1} = a_n^{-1}$
by construction. To check $a$ is a morphism of topoi we have to show
that $a^{-1}$ is left adjoint to $a_*$ and we have to show that
$a^{-1}$ is exact. The last fact is immediate from the exactness of
the functors $a_n^{-1}$.

\medskip\noindent
Let $\mathcal{F}$ be an object of $\Sh(X)$ and let $\mathcal{G}$
be an object of $\Sh(Y_{Zar})$. Given
$\beta : a^{-1}\mathcal{F} \to \mathcal{G}$ we can look at the
components $\beta_n : a_n^{-1}\mathcal{F} \to \mathcal{G}_n$.
These maps are adjoint to maps
$\beta_n : \mathcal{F} \to a_{n, *}\mathcal{G}_n$.
Compatibility with the simplicial structure shows that
$\beta_0$ maps into $a_*\mathcal{G}$.
Conversely, suppose given a map $\alpha : \mathcal{F} \to a_*\mathcal{G}$.
For any $n$ choose a $\varphi : [0] \to [n]$. Then we can look at
the composition
$$
\mathcal{F} \xrightarrow{\alpha} a_*\mathcal{G}
\to a_{0, *}\mathcal{G}_0 \xrightarrow{\mathcal{G}(\varphi)}
a_{n, *}\mathcal{G}_n
$$
These are adjoint to maps $a_n^{-1}\mathcal{F} \to \mathcal{G}_n$
which define a morphism of sheaves $a^{-1}\mathcal{F} \to \mathcal{G}$.
We omit the proof that the constructions given above define
mutually inverse bijections
$$
\Mor_{\Sh(Y_{Zar})}(a^{-1}\mathcal{F}, \mathcal{G}) =
\Mor_{\Sh(X)}(\mathcal{F}, a_*\mathcal{G})
$$
This finishes the proof. An interesting observation is here that
this morphism of topoi does not correspond to any obvious geometric
functor between the sites defining the topoi.
\end{proof}

\begin{lemma}
\label{lemma-simplicial-resolution-Z}
Let $X$ be a simplicial topological space. The complex of
abelian presheaves on $X_{Zar}$
$$
\ldots \to \mathbf{Z}_{X_2} \to \mathbf{Z}_{X_1} \to \mathbf{Z}_{X_0}
$$
with boundary $\sum (-1)^i d^n_i$ is a resolution
of the constant presheaf $\mathbf{Z}$.
\end{lemma}

\begin{proof}
Let $U \subset X_m$ be an object of $X_{Zar}$. Then the value of
the complex above on $U$ is the complex of abelian groups
$$
\ldots \to
\mathbf{Z}[\Mor_\Delta([2], [m])] \to
\mathbf{Z}[\Mor_\Delta([1], [m])] \to
\mathbf{Z}[\Mor_\Delta([0], [m])]
$$
In other words, this is the complex associated to the
free abelian group on the simplicial set $\Delta[m]$, see
Simplicial, Example \ref{simplicial-example-simplex-simplicial-set}.
Since $\Delta[m]$ is homotopy equivalent to $\Delta[0]$, see
Simplicial, Example \ref{simplicial-example-simplex-contractible},
and since ``taking free abelian groups'' is a functor,
we see that the complex above is homotopy equivalent to
the free abelian group on $\Delta[0]$
(Simplicial, Remark \ref{simplicial-remark-homotopy-better} and
Lemma \ref{simplicial-lemma-homotopy-equivalence-s-N}).
This complex is acyclic in positive degrees
and equal to $\mathbf{Z}$ in degree $0$.
\end{proof}

\begin{lemma}
\label{lemma-simplicial-sheaf-cohomology}
Let $X$ be a simplicial topological space. Let $\mathcal{F}$ be an abelian
sheaf on $X$. There is a spectral sequence $(E_r, d_r)_{r \geq 0}$ with
$$
E_1^{p, q} = H^q(X_p, \mathcal{F}_p)
$$
converging to $H^{p + q}(X_{Zar}, \mathcal{F})$.
This spectral sequence is functorial in $\mathcal{F}$.
\end{lemma}

\begin{proof}
Let $\mathcal{F} \to \mathcal{I}^\bullet$ be an injective resolution.
Consider the double complex with terms
$$
A^{p, q} = \mathcal{I}^q(X_p)
$$
and first differential given by the alternating sum along the maps
$d^{p + 1}_i$-maps $\mathcal{I}_p^q \to \mathcal{I}_{p + 1}^q$, see
Lemma \ref{lemma-describe-sheaves-simplicial-site}. Note that
$$
A^{p, q} = \Gamma(X_p, \mathcal{I}_p^q) =
\Mor_{\textit{PSh}}(h_{X_p}, \mathcal{I}^q) =
\Mor_{\textit{PAb}}(\mathbf{Z}_{X_p}, \mathcal{I}^q)
$$
Hence it follows from Lemma \ref{lemma-simplicial-resolution-Z} and
Cohomology on Sites, Lemma
\ref{sites-cohomology-lemma-injective-abelian-sheaf-injective-presheaf}
that the rows of the double complex are exact in positive degrees and
evaluate to $\Gamma(X_{Zar}, \mathcal{I}^q)$ in degree $0$.
On the other hand, since restriction is exact
(Lemma \ref{lemma-restriction-to-components})
the map
$$
\mathcal{F}_p \to \mathcal{I}_p^\bullet
$$
is a resolution. The sheaves $\mathcal{I}_p^q$ are injective
abelian sheaves on $X_p$
(Lemma \ref{lemma-restriction-injective-to-component}).
Hence the cohomology of the columns computes the groups
$H^q(X_p, \mathcal{F}_p)$. We conclude by applying
Homology, Lemmas \ref{homology-lemma-first-quadrant-ss} and
\ref{homology-lemma-double-complex-gives-resolution}.
\end{proof}

\begin{lemma}
\label{lemma-augmentation-pushforward-higher-direct-image}
Let $X$ be a simplicial space and let $a : X \to Y$
be an augmentation. Let $\mathcal{F}$ be an abelian sheaf
on $X_{Zar}$. Then $R^na_*\mathcal{F}$ is the sheaf associated
to the presheaf
$$
V \longmapsto H^n((X \times_Y V)_{Zar}, \mathcal{F}|_{(X \times_Y V)_{Zar}})
$$
\end{lemma}

\begin{proof}
This is the analogue of
Cohomology, Lemma \ref{cohomology-lemma-describe-higher-direct-images} or of
Cohomology on Sites, Lemma \ref{sites-cohomology-lemma-higher-direct-images}
and we strongly encourge the reader to skip the proof.
Choosing an injective resolution of $\mathcal{F}$ on
$X_{Zar}$ and using the definitions we see that it suffices to show:
(1) the restriction of an injective abelian
sheaf on $X_{Zar}$ to $(X \times_Y V)_{Zar}$ is an injective abelian sheaf and
(2) $a_*\mathcal{F}$ is equal to the rule
$$
V \longmapsto H^0((X \times_Y V)_{Zar}, \mathcal{F}|_{(X \times_Y V)_{Zar}})
$$
Part (2) follows from the following facts
\begin{enumerate}
\item[(2a)] $a_*\mathcal{F}$ is the equalizer of the two maps
$a_{0, *}\mathcal{F}_0 \to a_{1, *}\mathcal{F}_1$
by Lemma \ref{lemma-augmentation},
\item[(2b)] $a_{0, *}\mathcal{F}_0(V) =
H^0(a_0^{-1}(V), \mathcal{F}_0)$ and
$a_{1, *}\mathcal{F}_1(V) = H^0(a_1^{-1}(V), \mathcal{F}_1)$,
\item[(2c)] $X_0 \times_Y V = a_0^{-1}(V)$ and $X_1 \times_Y V = a_1^{-1}(V)$,
\item[(2d)] $H^0((X \times_Y V)_{Zar}, \mathcal{F}|_{(X \times_Y V)_{Zar}})$
is the equalizer of the two maps
$H^0(X_0 \times_Y V, \mathcal{F}_0) \to H^0(X_1 \times_Y V, \mathcal{F}_1)$
for example by Lemma \ref{lemma-simplicial-sheaf-cohomology}.
\end{enumerate}
Part (1) follows after one defines an exact left adjoint
$j_! : \textit{Ab}((X \times_Y V)_{Zar}) \to \textit{Ab}(X_{Zar})$
(extension by zero) to restriction
$\textit{Ab}(X_{Zar}) \to \textit{Ab}((X \times_Y V)_{Zar})$
and using Homology, Lemma \ref{homology-lemma-adjoint-preserve-injectives}.
\end{proof}

\noindent
Let $X$ be a topological space. Denote $X_\bullet$ the constant simplicial
topological space with value $X$. By
Lemma \ref{lemma-describe-sheaves-simplicial-site}
a sheaf on $X_{\bullet, Zar}$ is the same
thing as a cosimplicial object in the category of sheaves on $X$.

\begin{lemma}
\label{lemma-constant-simplicial-space}
Let $X$ be a topological space. Let $X_\bullet$ be the constant
simplicial topological space with value $X$. The functor
$$
X_{\bullet, Zar} \longrightarrow X_{Zar},\quad
U \longmapsto U
$$
is continuous and cocontinuous and defines a morphism of
topoi $g : \Sh(X_{\bullet, Zar}) \to \Sh(X)$ as well as a left adjoint
$g_!$ to $g^{-1}$. We have
\begin{enumerate}
\item $g^{-1}$ associates to a sheaf on $X$ the constant cosimplicial
sheaf on $X$,
\item $g_!$ associates to a sheaf $\mathcal{F}$ on $X_{\bullet, Zar}$ the
sheaf $\mathcal{F}_0$, and
\item $g_*$ associates to a sheaf $\mathcal{F}$ on $X_{\bullet, Zar}$ the
equalizer of the two maps $\mathcal{F}_0 \to \mathcal{F}_1$.
\end{enumerate}
\end{lemma}

\begin{proof}
The statements about the functor are straightforward to verify.
The existence of $g$ and $g_!$ follow from
Sites, Lemmas \ref{sites-lemma-cocontinuous-morphism-topoi} and
\ref{sites-lemma-when-shriek}. The description of
$g^{-1}$ is immediate from Sites, Lemma \ref{sites-lemma-when-shriek}.
The description of $g_*$ and $g_!$ follows as the functors given are
right and left adjoint to $g^{-1}$.
\end{proof}








\section{Simplicial sites and topoi}
\label{section-simplicial-sites}

\noindent
It seems natural to define a {\it simplicial site} as a simplicial
object in the (big) category whose objects are sites
and whose morphisms are morphisms of sites.
See Sites, Definitions \ref{sites-definition-site} and
\ref{sites-definition-morphism-sites}
with composition of morphisms as in 
Sites, Lemma \ref{sites-lemma-composition-morphisms-sites}.
But here are some variants one might want to consider:
(a) we could work with cocontinuous functors
(see Sites, Sections \ref{sites-section-cocontinuous-functors} and
\ref{sites-section-cocontinuous-morphism-topoi}) between sites instead,
(b) we could work in a suitable $2$-category of sites where one introduces
the notion of a $2$-morphism between morphisms of sites,
(c) we could work in a $2$-category constructed out of cocontinuous
functors. Instead of picking one of these variants as a definition
we will simply develop theory as needed.

\medskip\noindent
Certainly a {\it simplicial topos} should probably be defined as a
pseudo-functor from $\Delta^{opp}$ into the $2$-category of topoi.
See Categories, Definition \ref{categories-definition-functor-into-2-category}
and Sites, Section \ref{sites-section-topoi} and
\ref{sites-section-2-category}. We will try to avoid working with such
a beast if possible.

\medskip\noindent
{\bf Case A.}
Let $\mathcal{C}$ be a simplicial object in the category whose objects
are sites and whose morphisms are morphisms of sites. This means that
for every morphism $\varphi : [m] \to [n]$ of $\Delta$ we have a morphism
of sites $f_\varphi : \mathcal{C}_n \to \mathcal{C}_m$. This morphism is
given by a continuous functor in the opposite direction which we will denote
$u_\varphi : \mathcal{C}_m \to \mathcal{C}_n$.

\begin{lemma}
\label{lemma-simplicial-site-site}
Let $\mathcal{C}$ be a simplicial object in the category of sites.
With notation as above we construct a site $\mathcal{C}_{total}$ as follows.
\begin{enumerate}
\item An object of $\mathcal{C}_{total}$ is an object $U$ of
$\mathcal{C}_n$ for some $n$,
\item a morphism $(\varphi, f) : U \to V$ of $\mathcal{C}_{total}$
is given by a map $\varphi : [m] \to [n]$ with
$U \in \Ob(\mathcal{C}_n)$, $V \in \Ob(\mathcal{C}_m)$
and a morphism $f : U \to u_\varphi(V)$ of $\mathcal{C}_n$, and
\item a covering $\{(\text{id}, f_i) :  U_i \to U\}$ in $\mathcal{C}_{total}$
is given by an $n$ and a covering $\{f_i : U_i \to U\}$
of $\mathcal{C}_n$.
\end{enumerate}
\end{lemma}

\begin{proof}
Composition of $(\varphi, f) : U \to V$ with $(\psi, g) : V \to W$
is given by $(\varphi \circ \psi, u_\varphi(g) \circ f)$.
This uses that $u_\varphi \circ u_\psi = u_{\varphi \circ \psi}$.

\medskip\noindent
Let $\{(\text{id}, f_i) :  U_i \to U\}$ be a covering as in (3)
and let $(\varphi, g) : W \to U$ be a morphism with
$W \in \Ob(\mathcal{C}_m)$. We claim that
$$
W \times_{(\varphi, g), U, (\text{id}, f_i)} U_i =
W \times_{g, u_\varphi(U), u_\varphi(f_i)} u_\varphi(U_i)
$$
in the category $\mathcal{C}_{total}$. This makes sense as by our
definition of morphisms of sites, the required fibre products
in $\mathcal{C}_m$ exist since $u_\varphi$ transforms coverings into
coverings. The same reasoning implies the claim (details omitted).
Thus we see that the collection of coverings is stable under base
change. The other axioms of a site are immediate.
\end{proof}

\noindent
{\bf Case B.}
Let $\mathcal{C}$ be a simplicial object in the category whose objects are
sites and whose morphisms are cocontinuous functors. This means that for
every morphism $\varphi : [m] \to [n]$ of $\Delta$ we have a cocontinuous
functor denoted $u_\varphi : \mathcal{C}_n \to \mathcal{C}_m$. The associated
morphism of topoi is denoted
$f_\varphi : \Sh(\mathcal{C}_n) \to \Sh(\mathcal{C}_m)$.

\begin{lemma}
\label{lemma-simplicial-cocontinuous-site}
Let $\mathcal{C}$ be a simplicial object in the category whose objects are
sites and whose morphisms are cocontinuous functors. With notation as above,
assume the functors $u_\varphi : \mathcal{C}_n \to \mathcal{C}_m$
have property $P$ of Sites, Remark \ref{sites-remark-cartesian-cocontinuous}.
Then we can construct a site $\mathcal{C}_{total}$ as follows.
\begin{enumerate}
\item An object of $\mathcal{C}_{total}$ is an object $U$ of
$\mathcal{C}_n$ for some $n$,
\item a morphism $(\varphi, f) : U \to V$ of $\mathcal{C}_{total}$
is given by a map $\varphi : [m] \to [n]$ with
$U \in \Ob(\mathcal{C}_n)$, $V \in \Ob(\mathcal{C}_m)$
and a morphism $f : u_\varphi(U) \to V$ of $\mathcal{C}_m$, and
\item a covering $\{(\text{id}, f_i) :  U_i \to U\}$ in $\mathcal{C}_{total}$
is given by an $n$ and a covering $\{f_i : U_i \to U\}$
of $\mathcal{C}_n$.
\end{enumerate}
\end{lemma}

\begin{proof}
Composition of $(\varphi, f) : U \to V$ with $(\psi, g) : V \to W$
is given by $(\varphi \circ \psi, g \circ u_\psi(f))$.
This uses that $u_\psi \circ u_\varphi = u_{\varphi \circ \psi}$.

\medskip\noindent
Let $\{(\text{id}, f_i) :  U_i \to U\}$ be a covering as in (3)
and let $(\varphi, g) : W \to U$ be a morphism with
$W \in \Ob(\mathcal{C}_m)$. We claim that
$$
W \times_{(\varphi, g), U, (\text{id}, f_i)} U_i =
W \times_{g, U, f_i} U_i
$$
in the category $\mathcal{C}_{total}$ where the right hand side
is the object of $\mathcal{C}_m$ defined in
Sites, Remark \ref{sites-remark-cartesian-cocontinuous}
which exists by property $P$. Compatibility of this type of fibre product
with compositions of functors implies the claim (details omitted).
Since the family $\{W \times_{g, U, f_i} U_i \to W\}$ is a
covering of $\mathcal{C}_m$ by property $P$ we see that
the collection of coverings is stable under base
change. The other axioms of a site are immediate.
\end{proof}

\begin{situation}
\label{situation-simplicial-site}
Here we have one of the following two cases:
\begin{enumerate}
\item[(A)] $\mathcal{C}$ is a simplicial object in the category whose
objects are sites and whose morphisms are morphisms of sites. For every
morphism $\varphi : [m] \to [n]$ of $\Delta$ we have a morphism of sites
$f_\varphi : \mathcal{C}_n \to \mathcal{C}_m$ given by a continuous
functor $u_\varphi : \mathcal{C}_m \to \mathcal{C}_n$.
\item[(B)] $\mathcal{C}$ is a simplicial object in the category whose
objects are sites and whose morphisms are cocontinuous functors having
property $P$ of Sites, Remark \ref{sites-remark-cartesian-cocontinuous}.
For every morphism $\varphi : [m] \to [n]$ of $\Delta$ we have a cocontinuous
functor $u_\varphi : \mathcal{C}_n \to \mathcal{C}_m$ which induces a
morphism of topoi $f_\varphi : \Sh(\mathcal{C}_n) \to \Sh(\mathcal{C}_m)$.
\end{enumerate}
As usual we will denote $f_\varphi^{-1}$ and $f_{\varphi, *}$ the
pullback and pushforward. We let $\mathcal{C}_{total}$ denote the
site defined in
Lemma \ref{lemma-simplicial-site-site} (case A) or
Lemma \ref{lemma-simplicial-cocontinuous-site} (case B).
\end{situation}

\noindent
Let $\mathcal{C}$ be as in Situation \ref{situation-simplicial-site}.
Let $\mathcal{F}$ be a sheaf on $\mathcal{C}_{total}$.
It is clear from the definition of coverings, that the restriction
of $\mathcal{F}$ to the objects of $\mathcal{C}_n$ defines a sheaf
$\mathcal{F}_n$ on the site $\mathcal{C}_n$. For every
$\varphi : [m] \to [n]$ the restriction maps of $\mathcal{F}$
along the morphisms $(\varphi, f) : U \to V$ with 
$U \in \Ob(\mathcal{C}_n)$ and $V \in \Ob(\mathcal{C}_m)$
define an element $\mathcal{F}(\varphi)$ of
$$
\Mor_{\Sh(\mathcal{C}_m)}(\mathcal{F}_m, f_{\varphi, *}\mathcal{F}_n) =
\Mor_{\Sh(\mathcal{C}_n)}(f_\varphi^{-1}\mathcal{F}_m, \mathcal{F}_n)
$$
Moreover, given $\varphi : [m] \to [n]$ and $\psi : [l] \to [m]$
the diagrams
$$
\vcenter{
\xymatrix{
\mathcal{F}_l \ar[rr]_{\mathcal{F}(\varphi \circ \psi)}
\ar[rd]_{\mathcal{F}(\psi)}
& & f_{\varphi \circ \psi, *} \mathcal{F}_n \\
& f_{\psi, *}\mathcal{F}_m \ar[ur]_{f_{\psi, *}\mathcal{F}(\varphi)}
}
}
\quad\text{and}\quad
\vcenter{
\xymatrix{
f_{\varphi \circ \psi}^{-1}\mathcal{F}_l
\ar[rr]_{\mathcal{F}(\varphi \circ \psi)}
\ar[rd]_{f_\varphi^{-1}\mathcal{F}(\psi)}
& & \mathcal{F}_n \\
& f_\varphi^{-1}\mathcal{F}_m \ar[ur]_{\mathcal{F}(\varphi)}
}
}
$$
commute. Clearly, the converse statement is true as well: if we have a system
$(\{\mathcal{F}_n\}_{n \geq 0},
\{\mathcal{F}(\varphi)\}_{\varphi \in \text{Arrows}(\Delta)})$
satisfying the commutativity constraints above,
then we obtain a sheaf on $\mathcal{C}_{total}$.

\begin{lemma}
\label{lemma-describe-sheaves-simplicial-site-site}
In Situation \ref{situation-simplicial-site} there is an equivalence of
categories between
\begin{enumerate}
\item $\Sh(\mathcal{C}_{total})$, and
\item the category of systems $(\mathcal{F}_n, \mathcal{F}(\varphi))$
described above.
\end{enumerate}
In particular, the topos $\Sh(\mathcal{C}_{total})$ only depends on
the topoi $\Sh(\mathcal{C}_n)$ and the morphisms of topoi $f_\varphi$.
\end{lemma}

\begin{proof}
See discussion above.
\end{proof}

\begin{lemma}
\label{lemma-restriction-to-components-site}
In Situation \ref{situation-simplicial-site} the functor
$\mathcal{C}_n \to \mathcal{C}_{total}$, $U \mapsto U$ is continuous
and cocontinuous. The associated morphism of
topoi $g_n : \Sh(\mathcal{C}_n) \to \Sh(\mathcal{C}_{total})$ satisfies
\begin{enumerate}
\item $g_n^{-1}$ associates to the sheaf $\mathcal{F}$ on $\mathcal{C}_{total}$
the sheaf $\mathcal{F}_n$ on $\mathcal{C}_n$,
\item $g_n^{-1} : \Sh(\mathcal{C}_{total}) \to \Sh(\mathcal{C}_n)$
has a left adjoint $g^{Sh}_{n!}$,
\item for $\mathcal{G}$ in $\Sh(\mathcal{C}_n)$ the restriction of
$g_{n!}^{Sh}\mathcal{G}$ to $\mathcal{C}_m$ is
$\coprod\nolimits_{\varphi : [n] \to [m]} f_\varphi^{-1}\mathcal{G}$,
\item $g_{n!}^{Sh}$ commutes with finite connected limits,
\item $g_n^{-1} : \textit{Ab}(\mathcal{C}_{total}) \to
\textit{Ab}(\mathcal{C}_n)$ has a left adjoint $g_{n!}$,
\item for $\mathcal{G}$ in $\textit{Ab}(\mathcal{C}_n)$ the restriction of
$g_{n!}\mathcal{G}$ to $\mathcal{C}_m$ is
$\bigoplus\nolimits_{\varphi : [n] \to [m]} f_\varphi^{-1}\mathcal{G}$, and
\item $g_{n!}$ is exact.
\end{enumerate}
\end{lemma}

\begin{proof}
Case A. If $\{U_i \to U\}_{i \in I}$ is a covering in $\mathcal{C}_n$
then the image $\{U_i \to U\}_{i \in I}$ is a covering in $\mathcal{C}_{total}$
by definition (Lemma \ref{lemma-simplicial-site-site}). For a morphism
$V \to U$ of $\mathcal{C}_n$, the fibre product
$V \times_U U_i$ in $\mathcal{C}_n$ is also
the fibre product in $\mathcal{C}_{total}$ (by the claim in the
proof of Lemma \ref{lemma-simplicial-site-site}).
Therefore our functor is continuous. On the other hand, our functor
defines a bijection between coverings of $U$ in $\mathcal{C}_n$
and coverings of $U$ in $\mathcal{C}_{total}$. Therefore it is
certainly the case that our functor is cocontinuous.

\medskip\noindent
Case B. If $\{U_i \to U\}_{i \in I}$ is a covering in $\mathcal{C}_n$
then the image $\{U_i \to U\}_{i \in I}$ is a covering in $\mathcal{C}_{total}$
by definition (Lemma \ref{lemma-simplicial-cocontinuous-site}). For a morphism
$V \to U$ of $\mathcal{C}_n$, the fibre product
$V \times_U U_i$ in $\mathcal{C}_n$ is also
the fibre product in $\mathcal{C}_{total}$ (by the claim in the
proof of Lemma \ref{lemma-simplicial-cocontinuous-site}).
Therefore our functor is continuous. On the other hand, our functor
defines a bijection between coverings of $U$ in $\mathcal{C}_n$
and coverings of $U$ in $\mathcal{C}_{total}$. Therefore it is
certainly the case that our functor is cocontinuous.

\medskip\noindent
At this point part (1) and the existence of $g^{Sh}_{n!}$ and $g_{n!}$
in cases A and B follows from
Sites, Lemmas \ref{sites-lemma-cocontinuous-morphism-topoi} and
\ref{sites-lemma-when-shriek}
and
Modules on Sites, Lemmas \ref{sites-modules-lemma-g-shriek-adjoint} and
\ref{sites-modules-lemma-back-and-forth}.

\medskip\noindent
Proof of (3). Let $\mathcal{G}$ be a sheaf on $\mathcal{C}_n$.
Consider the sheaf $\mathcal{H}$ on $\mathcal{C}_{total}$
whose degree $m$ part is the sheaf
$$
\mathcal{H}_m = \coprod\nolimits_{\varphi : [n] \to [m]}
f_\varphi^{-1}\mathcal{G}
$$
given in part (3) of the statement of the lemma.
Given a map $\psi : [m] \to [m']$ the map
$\mathcal{H}(\psi) : f_\psi^{-1}\mathcal{H}_m \to \mathcal{H}_{m'}$
is given on components by the identifications
$$
f_\psi^{-1} f_\varphi^{-1} \mathcal{G} \to
f_{\psi \circ \varphi}^{-1}\mathcal{G}
$$
Observe that given a map $\alpha : \mathcal{H} \to \mathcal{F}$
of sheaves on $\mathcal{C}_{total}$ we obtain a map
$\mathcal{G} \to \mathcal{F}_n$
corresponding to the restriction of $\alpha_n$ to the component
$\mathcal{G}$ in $\mathcal{H}_n$. Conversely, given a map
$\beta : \mathcal{G} \to \mathcal{F}_n$ of sheaves on $\mathcal{C}_n$
we can define
$\alpha : \mathcal{H} \to \mathcal{F}$ by letting $\alpha_m$
be the map which on components
$$
f_\varphi^{-1}\mathcal{G} \to \mathcal{F}_m
$$
uses the maps adjoint to $\mathcal{F}(\varphi) \circ f_\varphi^{-1}\beta$.
We omit the arguments showing these two constructions give
mutually inverse maps
$$
\Mor_{\Sh(\mathcal{C}_n)}(\mathcal{G}, \mathcal{F}_n) =
\Mor_{\Sh(\mathcal{C}_{total})}(\mathcal{H}, \mathcal{F})
$$
Thus $\mathcal{H} = g^{Sh}_{n!}\mathcal{G}$ as desired.

\medskip\noindent
Proof of (4). If $\mathcal{G}$ is an abelian sheaf on $\mathcal{C}_n$,
then we proceed in exactly the same ammner as above, except that
we define $\mathcal{H}$ is the abelian sheaf on $\mathcal{C}_{total}$
whose degree $m$ part is the sheaf
$$
\bigoplus\nolimits_{\varphi : [n] \to [m]} f_\varphi^{-1}\mathcal{G}
$$
with transition maps defined exactly as above. The bijection
$$
\Mor_{\textit{Ab}(\mathcal{C}_n)}(\mathcal{G}, \mathcal{F}_n) =
\Mor_{\textit{Ab}(\mathcal{C}_{total})}(\mathcal{H}, \mathcal{F})
$$
is proved exactly as above.
Thus $\mathcal{H} = g_{n!}\mathcal{G}$ as desired.

\medskip\noindent
The exactness properties of $g^{Sh}_{n!}$ and $g_{n!}$ follow
from formulas given for these functors.
\end{proof}

\begin{lemma}
\label{lemma-restriction-injective-to-component-site}
\begin{slogan}
An injective abelian sheaf on a simplicial site is injective on each component
\end{slogan}
In Situation \ref{situation-simplicial-site}.
If $\mathcal{I}$ is injective in $\textit{Ab}(\mathcal{C}_{total})$,
then $\mathcal{I}_n$ is injective in $\textit{Ab}(\mathcal{C}_n)$.
If $\mathcal{I}^\bullet$ is a K-injective complex in
$\textit{Ab}(\mathcal{C}_{total})$,
then $\mathcal{I}_n^\bullet$ is K-injective in $\textit{Ab}(\mathcal{C}_n)$.
\end{lemma}

\begin{proof}
The first statement follows from
Homology, Lemma \ref{homology-lemma-adjoint-preserve-injectives}
and
Lemma \ref{lemma-restriction-to-components-site}.
The second statement from
Derived Categories, Lemma \ref{derived-lemma-adjoint-preserve-K-injectives}
and
Lemma \ref{lemma-restriction-to-components-site}.
\end{proof}







\section{Augmentations of simplicial sites}
\label{section-augmentation-simplicial-sites}

\noindent
We continue in the fashion described in
Section \ref{section-simplicial-sites}
working out the meaning of augmentations in cases A and B
treated in that section.

\begin{remark}
\label{remark-augmentation-site}
In Situation \ref{situation-simplicial-site} an
{\it augmentation $a_0$ towards a site $\mathcal{D}$} will mean
\begin{enumerate}
\item[(A)] $a_0 : \mathcal{C}_0 \to \mathcal{D}$ is a morphism of sites
given by a continuous functor $u_0 : \mathcal{D} \to \mathcal{C}_0$
such that for all $\varphi, \psi : [0] \to [n]$ we have
$u_\varphi \circ u_0 = u_\psi \circ u_0$.
\item[(B)] $a_0 : \Sh(\mathcal{C}_0) \to \Sh(\mathcal{D})$ is a morphism
of topoi given by a cocontinuous functor $u_0 : \mathcal{C}_0 \to \mathcal{D}$
such that for all $\varphi, \psi : [0] \to [n]$ we have
$u_0 \circ u_\varphi = u_0 \circ u_\psi$.
\end{enumerate}
\end{remark}

\begin{lemma}
\label{lemma-augmentation-site}
In Situation \ref{situation-simplicial-site} let $a_0$ be an
augmentation towards a site $\mathcal{D}$ as in
Remark \ref{remark-augmentation-site}. Then $a_0$ induces
\begin{enumerate}
\item a morphism of topoi $a_n : \Sh(\mathcal{C}_n) \to \Sh(\mathcal{D})$
for all $n \geq 0$,
\item a morphism of topoi $a : \Sh(\mathcal{C}_{total}) \to \Sh(\mathcal{D})$
\end{enumerate}
such that
\begin{enumerate}
\item for all $\varphi : [m] \to [n]$ we have $a_m \circ f_\varphi = a_n$,
\item if $g_n : \Sh(\mathcal{C}_n) \to \Sh(\mathcal{C}_{total})$
is as in Lemma \ref{lemma-restriction-to-components-site}, then
$a \circ g_n = a_n$, and
\item $a_*\mathcal{F}$ for $\mathcal{F} \in \Sh(\mathcal{C}_{total})$
is the equalizer of the two maps
$a_{0, *}\mathcal{F}_0 \to a_{1, *}\mathcal{F}_1$.
\end{enumerate}
\end{lemma}

\begin{proof}
Case A. Let $u_n : \mathcal{D} \to \mathcal{C}_n$ be the common
value of the functors $u_\varphi \circ u_0$ for $\varphi : [0] \to [n]$.
Then $u_n$ corresponds to a morphism of sites
$a_n : \mathcal{C}_n \to \mathcal{D}$, see
Sites, Lemma \ref{sites-lemma-composition-morphisms-sites}.
The same lemma shows that for all $\varphi : [m] \to [n]$ we have
$a_m \circ f_\varphi = a_n$.

\medskip\noindent
Case B. Let $u_n : \mathcal{C}_n \to \mathcal{D}$ be the common
value of the functors $u_0 \circ u_\varphi$ for $\varphi : [0] \to [n]$.
Then $u_n$ is cocontinuous and hence defines a morphism of topoi
$a_n : \Sh(\mathcal{C}_n) \to \Sh(\mathcal{D)}$, see
Sites, Lemma \ref{sites-lemma-composition-cocontinuous}.
The same lemma shows that for all $\varphi : [m] \to [n]$ we have
$a_m \circ f_\varphi = a_n$.

\medskip\noindent
Consider the functor $a^{-1} : \Sh(\mathcal{D}) \to \Sh(\mathcal{C}_{total})$
which to a sheaf of sets $\mathcal{G}$ associates the sheaf
$\mathcal{F} = a^{-1}\mathcal{G}$ whose components are $a_n^{-1}\mathcal{G}$
and whose transition maps $\mathcal{F}(\varphi)$ are the identifications
$$
f_\varphi^{-1}\mathcal{F}_m =
f_\varphi^{-1} a_m^{-1}\mathcal{G} =
a_n^{-1}\mathcal{G} =
\mathcal{F}_n
$$
for $\varphi : [m] \to [n]$, see the description of
$\Sh(\mathcal{C}_{total})$ in
Lemma \ref{lemma-describe-sheaves-simplicial-site-site}.
Since the functors $a_n^{-1}$ are exact, $a^{-1}$ is an exact functor.
Finally, for $a_* : \Sh(\mathcal{C}_{total}) \to \Sh(\mathcal{D})$
we take the functor which to a sheaf $\mathcal{F}$ on $\Sh(\mathcal{D})$
associates
$$
\xymatrix{
a_*\mathcal{F} \ar@{=}[r] &
\text{Equalizer}(a_{0, *}\mathcal{F}_0
\ar@<1ex>[r] \ar@<-1ex>[r] &
a_{1, *}\mathcal{F}_1)
}
$$
Here the two maps come from the two maps $\varphi : [0] \to [1]$
via
$$
a_{0, *}\mathcal{F}_0 \to
a_{0, *}f_{\varphi, *} f_\varphi^{-1}\mathcal{F}_0
\xrightarrow{\mathcal{F}(\varphi)}
a_{0, *}f_{\varphi, *} \mathcal{F}_0 = a_{1, *}\mathcal{F}_1
$$
where the first arrow comes from $1 \to f_{\varphi, *} f_\varphi^{-1}$.
Let $\mathcal{G}_\bullet$ denote the constant simplicial sheaf
with value $\mathcal{G}$ and let $a_{\bullet, *}\mathcal{F}$
denote the simplicial sheaf having $a_{n, *}\mathcal{F}_n$ in degree $n$.
By the usual adjuntion for the morphisms of topoi $a_n$ we see that
a map $a^{-1}\mathcal{G} \to \mathcal{F}$
is the same thing as a map
$$
\mathcal{G}_\bullet \longrightarrow a_{\bullet, *}\mathcal{F}
$$
of simplicial sheaves.
By Simplicial, Lemma \ref{simplicial-lemma-augmentation-howto}
this is the same thing as a map $\mathcal{G} \to a_*\mathcal{F}$.
Thus $a^{-1}$ and $a_*$ are adjoint functors and we obtain
our morphism of topoi $a$\footnote{In case B the morphism $a$
corresponds to the cocontinuous functor
$\mathcal{C}_{total} \to \mathcal{D}$ sending
$U$ in $\mathcal{C}_n$ to $u_n(U)$.}. The equalities
$a \circ g_n = f_n$ follow immediately from the definitions.
\end{proof}



\section{Morphisms of simplicial sites}
\label{section-morphism-simplicial-sites}

\noindent
We continue in the fashion described in
Section \ref{section-simplicial-sites}
working out the meaning of morphisms of simplicial sites
in cases A and B treated in that section.

\begin{remark}
\label{remark-morphism-simplicial-sites}
Let $\mathcal{C}_n, f_\varphi, u_\varphi$ and
$\mathcal{C}'_n, f'_\varphi, u'_\varphi$ be as in
Situation \ref{situation-simplicial-site}. A
{\it morphism $h$ between simplicial sites} will mean
\begin{enumerate}
\item[(A)] Morphisms of sites
$h_n : \mathcal{C}_n \to \mathcal{C}'_n$
such that $f'_\varphi \circ h_n = h_m \circ f_\varphi$
as morphisms of sites for all $\varphi : [m] \to [n]$.
\item[(B)] Cocontinuous functors
$v_n : \mathcal{C}_n \to \mathcal{C}'_n$
inducing morphisms of topoi $h_n : \Sh(\mathcal{C}_n) \to \Sh(\mathcal{C}'_n)$
such that $u'_\varphi \circ v_n = v_m \circ u_\varphi$
as functors for all $\varphi : [m] \to [n]$.
\end{enumerate}
In both cases we have
$f'_\varphi \circ h_n = h_m \circ f_\varphi$
as morphisms of topoi, see
Sites, Lemma \ref{sites-lemma-composition-cocontinuous}
for case B and Sites,
Definition \ref{sites-definition-composition-morphisms-sites}
for case A.
\end{remark}

\begin{lemma}
\label{lemma-morphism-simplicial-sites}
Let $\mathcal{C}_n, f_\varphi, u_\varphi$ and
$\mathcal{C}'_n, f'_\varphi, u'_\varphi$ be as in
Situation \ref{situation-simplicial-site}.
Let $h$ be a morphism between simplicial sites as in
Remark \ref{remark-morphism-simplicial-sites}.
Then we obtain a morphism of topoi
$$
h_{total} : \Sh(\mathcal{C}_{total}) \to \Sh(\mathcal{C}'_{total})
$$
and commutative diagrams
$$
\xymatrix{
\Sh(\mathcal{C}_n) \ar[d]_{g_n} \ar[r]_{h_n} &
\Sh(\mathcal{C}'_n) \ar[d]^{g'_n} \\
\Sh(\mathcal{C}_{total}) \ar[r]^{h_{total}} &
\Sh(\mathcal{C}'_{total})
}
$$
Moreover, we have $(g'_n)^{-1} \circ h_{total, *} = h_{n, *} \circ g_n^{-1}$.
\end{lemma}

\begin{proof}
Case A. Say $h_n$ corresponds to the continuous functor
$v_n : \mathcal{C}'_n \to \mathcal{C}_n$. Then we can define
a functor $v_{total} : \mathcal{C}'_{total} \to \mathcal{C}_{total}$
by using $v_n$ in degree $n$. This is clearly a continuous functor
(see definition of coverings in Lemma \ref{lemma-simplicial-site-site}).
Let
$h_{total}^{-1} = v_{total, s} :
\Sh(\mathcal{C}'_{total}) \to \Sh(\mathcal{C}_{total})$ and
$h_{total, *} = v_{total}^s = v_{total}^p :
\Sh(\mathcal{C}_{total}) \to \Sh(\mathcal{C}'_{total})$
be the adjoint pair of functors constructed and studied in
Sites, Sections \ref{sites-section-continuous-functors} and
\ref{sites-section-morphism-sites}.
To see that $h_{total}$ is a morphism of topoi
we still have to verify that $h_{total}^{-1}$ is exact.
We first observe that
$(g'_n)^{-1} \circ h_{total, *} = h_{n, *} \circ g_n^{-1}$;
this is immediate by computing sections over an object $U$
of $\mathcal{C}'_n$. Thus, if we think of a sheaf $\mathcal{F}$
on $\mathcal{C}_{total}$ as a system $(\mathcal{F}_n, \mathcal{F}(\varphi))$
as in Lemma \ref{lemma-describe-sheaves-simplicial-site-site}, then
$h_{total, *}\mathcal{F}$ corresponds to
the system $(h_{n, *}\mathcal{F}_n, h_{n, *}\mathcal{F}(\varphi))$.
Clearly, the functor
$(\mathcal{F}'_n, \mathcal{F}'(\varphi)) \to
(h_n^{-1}\mathcal{F}'_n, h_n^{-1}\mathcal{F}'(\varphi))$
is its left adjoint. By uniqueness of adjoints, we conclude that
$h_{total}^{-1}$ is given by this rule on systems. In particular,
$h_{total}^{-1}$ is exact (by the description of sheaves on
$\mathcal{C}_{total}$ given in the lemma and the exactness of
the functors $h_n^{-1}$) and we have our morphism of topoi.
Finally, we obtain $g_n^{-1} \circ h_{total}^{-1} =
h_n^{-1} \circ (g'_n)^{-1}$ as well, which proves that the
displayed diagram of the lemma commutes.

\medskip\noindent
Case B. Here we have a functor
$v_{total} : \mathcal{C}_{total} \to \mathcal{C}'_{total}$
by using $v_n$ in degree $n$. This is clearly a cocontinuous functor
(see definition of coverings in Lemma \ref{lemma-simplicial-cocontinuous-site}).
Let $h_{total}$ be the morphism of topoi associated to $v_{total}$.
The commutativity of the displayed diagram of the lemma follows
immediately from Sites, Lemma \ref{sites-lemma-composition-cocontinuous}.
Taking left adjoints the final equality of the lemma becomes
$$
h_{total}^{-1} \circ (g'_n)^{Sh}_! = g^{Sh}_{n!} \circ h_n^{-1}
$$
This follows immediately from the explicit description of the functors
$(g'_n)^{Sh}_!$ and $g^{Sh}_{n!}$ in
Lemma \ref{lemma-restriction-to-components-site},
the fact that $h_n^{-1} \circ (f'_\varphi)^{-1} =
f_\varphi^{-1} \circ h_m^{-1}$ for $\varphi : [m] \to [n]$, and
the fact that we already know $h_{total}^{-1}$ commutes
with restrictions to the degree $n$ parts of the simplicial sites.
\end{proof}

\begin{lemma}
\label{lemma-direct-image-morphism-simplicial-sites}
With notation and hypotheses as in Lemma \ref{lemma-morphism-simplicial-sites}.
For $K \in D(\mathcal{C}_{total})$ we have
$(g'_n)^{-1}Rh_{total, *}K = Rh_{n, *}g_n^{-1}K$.
\end{lemma}

\begin{proof}
Let $\mathcal{I}^\bullet$ be a K-injective complex on $\mathcal{C}_{total}$
representing $K$. Then $g_n^{-1}K$ is represented by
$g_n^{-1}\mathcal{I}^\bullet = \mathcal{I}_n^\bullet$
which is K-injective by
Lemma \ref{lemma-restriction-injective-to-component-site}.
We have $(g'_n)^{-1}h_{total, *}\mathcal{I}^\bullet =
h_{n, *}g_n^{-1}\mathcal{I}_n^\bullet$ by
Lemma \ref{lemma-morphism-simplicial-sites}
which gives the desired equality.
\end{proof}

\begin{remark}
\label{remark-morphism-augmentation-simplicial-sites}
Let $\mathcal{C}_n, f_\varphi, u_\varphi$ and
$\mathcal{C}'_n, f'_\varphi, u'_\varphi$ be as in
Situation \ref{situation-simplicial-site}.
Let $a_0$, resp.\ $a'_0$ be an augmentation
towards a site $\mathcal{D}$, resp.\ $\mathcal{D}'$
as in Remark \ref{remark-augmentation-site}.
Let $h$ be a morphism between simplicial sites as in
Remark \ref{remark-morphism-simplicial-sites}.
We say a morphism of topoi $h_{-1} : \Sh(\mathcal{D}) \to \Sh(\mathcal{D}')$
is {\it compatible with $h$, $a_0$, $a'_0$} if
\begin{enumerate}
\item[(A)] $h_{-1}$ comes from a morphism of sites
$h_{-1} : \mathcal{D} \to \mathcal{D}'$
such that $a'_0 \circ h_0 = h_{-1} \circ a_0$
as morphisms of sites.
\item[(B)] $h_{-1}$ comes from a cocontinuous functor
$v_{-1} : \mathcal{D} \to \mathcal{D}'$
such that $u'_0 \circ v_0 = v_{-1} \circ u_0$
as functors.
\end{enumerate}
In both cases we have $a'_0 \circ h_0 = h_{-1} \circ a_0$
as morphisms of topoi, see
Sites, Lemma \ref{sites-lemma-composition-cocontinuous}
for case B and Sites,
Definition \ref{sites-definition-composition-morphisms-sites}
for case A.
\end{remark}

\begin{lemma}
\label{lemma-morphism-augmentation-simplicial-sites}
Let $\mathcal{C}_n, f_\varphi, u_\varphi, \mathcal{D}, a_0$,
$\mathcal{C}'_n, f'_\varphi, u'_\varphi, \mathcal{D}', a'_0$, and
$h_n$, $n \geq -1$ be as in
Remark \ref{remark-morphism-augmentation-simplicial-sites}.
Then we obtain a commutative diagram
$$
\xymatrix{
\Sh(\mathcal{C}_{total}) \ar[d]_a \ar[r]_{h_{total}} &
\Sh(\mathcal{C}'_{total}) \ar[d]^{a'} \\
\Sh(\mathcal{D}) \ar[r]^{h_{-1}} &
\Sh(\mathcal{D}')
}
$$
\end{lemma}

\begin{proof}
The morphism $h$ is defined in Lemma \ref{lemma-morphism-simplicial-sites}.
The morphisms $a$ and $a'$ are defined in Lemma \ref{lemma-augmentation-site}.
Thus the only thing is to prove the commutativity of the diagram.
To do this, we prove that
$a^{-1} \circ h_{-1}^{-1} = h_{total}^{-1} \circ (a')^{-1}$.
By the commutative diagrams of
Lemma \ref{lemma-morphism-simplicial-sites}
and the description of $\Sh(\mathcal{C}_{total})$
and $\Sh(\mathcal{C}'_{total})$ in terms of components
in Lemma \ref{lemma-describe-sheaves-simplicial-site-site},
it suffices to show that
$$
\xymatrix{
\Sh(\mathcal{C}_n) \ar[d]_{a_n} \ar[r]_{h_n} &
\Sh(\mathcal{C}'_n) \ar[d]^{a'_n} \\
\Sh(\mathcal{D}) \ar[r]^{h_{-1}} &
\Sh(\mathcal{D}')
}
$$
commutes for all $n$. This follows from the case for $n = 0$
(which is an assumption in
Remark \ref{remark-morphism-augmentation-simplicial-sites})
and for $n > 0$ we pick $\varphi : [0] \to [n]$
and then the required commutativity follows from the case $n = 0$
and the relations $a_n = a_0 \circ f_\varphi$
and $a'_n = a'_0 \circ f'_\varphi$
as well as the commutation relations
$f'_\varphi \circ h_n = h_0 \circ f_\varphi$.
\end{proof}




\section{Ringed simplicial sites}
\label{section-simplicial-sites-modules}

\noindent
Let us endow our simplicial topos with a sheaf of rings.

\begin{lemma}
\label{lemma-restriction-module-to-components-site}
In Situation \ref{situation-simplicial-site}. Let $\mathcal{O}$
be a sheaf of rings on $\mathcal{C}_{total}$.
There is a canonical morphism of ringed topoi
$g_n : (\Sh(\mathcal{C}_n), \mathcal{O}_n) \to
(\Sh(\mathcal{C}_{total}), \mathcal{O})$
agreeing with the morphism $g_n$ of
Lemma \ref{lemma-restriction-to-components-site} on underlying topoi.
The functor
$g_n^* : \textit{Mod}(\mathcal{O}) \to \textit{Mod}(\mathcal{O}_n)$
has a left adjoint $g_{n!}$.
For $\mathcal{G}$ in $\textit{Mod}(\mathcal{O}_n)$-modules the
restriction of $g_{n!}\mathcal{G}$ to $\mathcal{C}_m$ is
$$
\bigoplus\nolimits_{\varphi : [n] \to [m]} f_\varphi^*\mathcal{G}
$$
where $f_\varphi : (\Sh(\mathcal{C}_m), \mathcal{O}_m) \to
(\Sh(\mathcal{C}_n), \mathcal{O}_n)$ is the morphism of ringed topoi
agreeing with the previously defined $f_\varphi$ on topoi and
using the map
$\mathcal{O}(\varphi) : f_\varphi^{-1}\mathcal{O}_n \to \mathcal{O}_m$
on sheaves of rings.
\end{lemma}

\begin{proof}
By Lemma \ref{lemma-restriction-to-components-site} we have
$g_n^{-1}\mathcal{O} = \mathcal{O}_n$ and hence we obtain our
morphism of ringed topoi. By Modules on Sites, Lemma
\ref{sites-modules-lemma-lower-shriek-modules}
we obtain the adjoint $g_{n!}$. To prove the formula for $g_{n!}$
we first define a sheaf of $\mathcal{O}$-modules $\mathcal{H}$
on $\mathcal{C}_{total}$ with degree $m$ component
the $\mathcal{O}_m$-module
$$
\mathcal{H}_m =
\bigoplus\nolimits_{\varphi : [n] \to [m]} f_\varphi^*\mathcal{G}
$$
Given a map $\psi : [m] \to [m']$ the map
$\mathcal{H}(\psi) : f_\psi^{-1}\mathcal{H}_m \to \mathcal{H}_{m'}$
is given on components by
$$
f_\psi^{-1} f_\varphi^*\mathcal{G} \to
f_\psi^* f_\varphi^*\mathcal{G} \to
f_{\psi \circ \varphi}^*\mathcal{G}
$$
Since this map $f_\psi^{-1}\mathcal{H}_m \to \mathcal{H}_{m'}$ is
$\mathcal{O}(\psi) : f_\psi^{-1}\mathcal{O}_m \to \mathcal{O}_{m'}$-semi-linear,
this indeed does define an $\mathcal{O}$-module
(use Lemma \ref{lemma-describe-sheaves-simplicial-site-site}).
Then one proves directly that
$$
\Mor_{\mathcal{O}_n}(\mathcal{G}, \mathcal{F}_n) =
\Mor_{\mathcal{O}}(\mathcal{H}, \mathcal{F})
$$
proceeding as in the proof of Lemma \ref{lemma-restriction-to-components-site}.
Thus $\mathcal{H} = g_{n!}\mathcal{G}$ as desired.
\end{proof}

\begin{lemma}
\label{lemma-restriction-injective-to-component-limp}
In Situation \ref{situation-simplicial-site}.
Let $\mathcal{O}$ be a sheaf of rings on $\mathcal{C}_{total}$.
If $\mathcal{I}$ is injective in $\textit{Mod}(\mathcal{O})$, then
$\mathcal{I}_n$ is a limp sheaf on $\mathcal{C}_n$.
\end{lemma}

\begin{proof}
This follows from
Cohomology on Sites, Lemma \ref{sites-cohomology-lemma-pullback-injective-limp}
applied to the inclusion functor $\mathcal{C}_n \to \mathcal{C}_{total}$
and its properties proven in Lemma \ref{lemma-restriction-to-components-site}.
\end{proof}

\begin{lemma}
\label{lemma-exactness-g-shriek-modules}
With assumptions as in
Lemma \ref{lemma-restriction-module-to-components-site} the functor
$g_{n!} : \textit{Mod}(\mathcal{O}_n) \to \textit{Mod}(\mathcal{O})$
is exact if the maps $f_\varphi^{-1}\mathcal{O}_n \to \mathcal{O}_m$
are flat for all $\varphi : [n] \to [m]$.
\end{lemma}

\begin{proof}
Recall that $g_{n!}\mathcal{G}$ is the $\mathcal{O}$-module
whose degree $m$ part is the $\mathcal{O}_m$-module
$$
\bigoplus\nolimits_{\varphi : [n] \to [m]} f_\varphi^*\mathcal{G}
$$
Here the morphism of ringed topoi
$f_\varphi : (\Sh(\mathcal{C}_m), \mathcal{O}_m) \to
(\Sh(\mathcal{C}_n), \mathcal{O}_n)$ uses the map
$f_\varphi^{-1}\mathcal{O}_n \to \mathcal{O}_m$ of the
statement of the lemma. If these maps are flat, then
$f_\varphi^*$ is exact
(Modules on Sites, Lemma \ref{sites-modules-lemma-flat-pullback-exact}).
By definition of the site $\mathcal{C}_{total}$ we see that these
functors have the desired exactness properties and we conclude.
\end{proof}

\begin{lemma}
\label{lemma-restriction-injective-to-component-site-module}
In Situation \ref{situation-simplicial-site}.
Let $\mathcal{O}$ be a sheaf of rings on $\mathcal{C}_{total}$
such that $f_\varphi^{-1}\mathcal{O}_n \to \mathcal{O}_m$
is flat for all $\varphi : [n] \to [m]$.
If $\mathcal{I}$ is injective in $\textit{Mod}(\mathcal{O})$, then
$\mathcal{I}_n$ is injective in $\textit{Mod}(\mathcal{O}_n)$.
\end{lemma}

\begin{proof}
This follows from
Homology, Lemma \ref{homology-lemma-adjoint-preserve-injectives}
and
Lemma \ref{lemma-exactness-g-shriek-modules}.
\end{proof}







\section{Morphisms of ringed simplicial sites}
\label{section-morphism-simplicial-sites-modules}

\noindent
We continue the discussion of Section \ref{section-morphism-simplicial-sites}.

\begin{remark}
\label{remark-morphism-simplicial-sites-modules}
Let $\mathcal{C}_n, f_\varphi, u_\varphi$ and
$\mathcal{C}'_n, f'_\varphi, u'_\varphi$ be as in
Situation \ref{situation-simplicial-site}.
Let $\mathcal{O}$ and $\mathcal{O}'$
be a sheaf of rings on $\mathcal{C}_{total}$ and $\mathcal{C}'_{total}$.
We will say that $(h, h^\sharp)$ is a
{\it morphism between ringed simplicial sites}
if $h$ is a morphism between simplicial sites as in
Remark \ref{remark-morphism-simplicial-sites}
and $h^\sharp : h_{total}^{-1}\mathcal{O}' \to \mathcal{O}$
or equivalently $h^\sharp : \mathcal{O}' \to h_{total, *}\mathcal{O}$
is a homomorphism of sheaves of rings.
\end{remark}

\begin{lemma}
\label{lemma-morphism-simplicial-sites-modules}
Let $\mathcal{C}_n, f_\varphi, u_\varphi$ and
$\mathcal{C}'_n, f'_\varphi, u'_\varphi$ be as in
Situation \ref{situation-simplicial-site}.
Let $\mathcal{O}$ and $\mathcal{O}'$
be a sheaf of rings on $\mathcal{C}_{total}$ and $\mathcal{C}'_{total}$.
Let $(h, h^\sharp)$ be a morphism between simplicial sites as in
Remark \ref{remark-morphism-simplicial-sites-modules}.
Then we obtain a morphism of ringed topoi
$$
h_{total} :
(\Sh(\mathcal{C}_{total}, \mathcal{O})
\to
(\Sh(\mathcal{C}'_{total}), \mathcal{O}')
$$
and commutative diagrams
$$
\xymatrix{
(\Sh(\mathcal{C}_n), \mathcal{O}_n) \ar[d]_{g_n} \ar[r]_{h_n} &
(\Sh(\mathcal{C}'_n), \mathcal{O}'_n) \ar[d]^{g'_n} \\
(\Sh(\mathcal{C}_{total}), \mathcal{O}) \ar[r]^{h_{total}} &
(\Sh(\mathcal{C}'_{total}), \mathcal{O}')
}
$$
of ringed topoi where $g_n$ and $g'_n$ are as in
Lemma \ref{lemma-restriction-module-to-components-site}.
Moreover, we have
$(g'_n)^* \circ h_{total, *} = h_{n, *} \circ g_n^*$
as functor $\textit{Mod}(\mathcal{O}) \to \textit{Mod}(\mathcal{O}'_n)$.
\end{lemma}

\begin{proof}
Follows from
Lemma \ref{lemma-morphism-simplicial-sites} and
\ref{lemma-restriction-module-to-components-site}
by keeping track of the sheaves of rings.
A small point is that in order to define $h_n$ as a morphism
of ringed topoi we set
$h_n^\sharp = g_n^{-1}h^\sharp :
g_n^{-1}h_{total}^{-1}\mathcal{O}' \to g_n^{-1}\mathcal{O}$
which makes sense because
$g_n^{-1}h_{total}^{-1}\mathcal{O}' = h_n^{-1}(g'_n)^{-1}\mathcal{O}' =
h_n^{-1}\mathcal{O}'_n$ and $g_n^{-1}\mathcal{O} = \mathcal{O}_n$.
Note that $g_n^*\mathcal{F} = g_n^{-1}\mathcal{F}$
for a sheaf of $\mathcal{O}$-modules $\mathcal{F}$
and similarly for $g'_n$ and this helps explain why
$(g'_n)^* \circ h_{total, *} = h_{n, *} \circ g_n^*$
follows from the corresponding statement of
Lemma \ref{lemma-morphism-simplicial-sites}.
\end{proof}

\begin{lemma}
\label{lemma-direct-image-morphism-simplicial-sites-modules}
With notation and hypotheses as in
Lemma \ref{lemma-morphism-simplicial-sites-modules}.
For $K \in D(\mathcal{O})$ we have
$(g'_n)^*Rh_{total, *}K = Rh_{n, *}g_n^*K$.
\end{lemma}

\begin{proof}
Recall that $g_n^* = g_n^{-1}$ because $g_n^{-1}\mathcal{O} = \mathcal{O}_n$
by the construction in Lemma \ref{lemma-restriction-module-to-components-site}.
In particular $g_n^*$ is exact and $Lg_n^*$ is given by applying $g_n^*$
to any representative complex of modules. Similarly for $g'_n$.
There is a canonical base change map
$(g'_n)^*Rh_{total, *}K \to Rh_{n, *}g_n^*K$, see
Cohomology on Sites, Remark \ref{sites-cohomology-remark-base-change}.
By Cohomology on Sites, Lemma
\ref{sites-cohomology-lemma-modules-abelian-unbounded}
the image of this in $D(\mathcal{C}'_n)$ is the map
$(g'_n)^{-1}Rh_{total, *}K_{ab} \to Rh_{n, *}g_n^{-1}K_{ab}$
where $K_{ab}$ is the image of $K$ in $D(\mathcal{C}_{total})$.
This we proved to be an isomorphism in
Lemma \ref{lemma-direct-image-morphism-simplicial-sites}
and the result follows.
\end{proof}








\section{Cohomology on simplicial sites}
\label{section-cohomology-simplicial-sites}

\noindent
Let $\mathcal{C}$ be as in Situation \ref{situation-simplicial-site}.
In statement of the following lemmas we will let
$g_n : \Sh(\mathcal{C}_n) \to \Sh(\mathcal{C}_{total})$ be the
morphism of topoi of
Lemma \ref{lemma-restriction-to-components-site}. If $\varphi : [m] \to [n]$
is a morphism of $\Delta$, then the diagram of topoi
$$
\xymatrix{
\Sh(\mathcal{C}_n) \ar[rd]_{g_n} \ar[rr]_{f_\varphi} & &
\Sh(\mathcal{C}_m) \ar[ld]^{g_m} \\
& \Sh(\mathcal{C}_{total})
}
$$
is not commutative, but there is a $2$-morphism $g_n \to g_m \circ f_\varphi$
coming from the maps
$\mathcal{F}(\varphi) : f_\varphi^{-1}\mathcal{F}_m \to \mathcal{F}_n$.
See Sites, Section \ref{sites-section-2-category}.

\begin{lemma}
\label{lemma-simplicial-resolution-Z-site}
In Situation \ref{situation-simplicial-site} and with notation as above
there is a complex
$$
\ldots \to g_{2!}\mathbf{Z} \to g_{1!}\mathbf{Z} \to g_{0!}\mathbf{Z}
$$
of abelian sheaves on $\mathcal{C}_{total}$ which forms a resolution of
the constant sheaf with value $\mathbf{Z}$ on $\mathcal{C}_{total}$.
\end{lemma}

\begin{proof}
We will use the description of the functors $g_{n!}$ in
Lemma \ref{lemma-restriction-to-components-site} without further mention.
As maps of the complex we take $\sum (-1)^i d^n_i$ where
$d^n_i : g_{n!}\mathbf{Z} \to g_{n - 1!}\mathbf{Z}$ is the
adjoint to the map $\mathbf{Z} \to
\bigoplus_{[n - 1] \to [n]} \mathbf{Z} = g_n^{-1}g_{n - 1!}\mathbf{Z}$
corresponding to the factor labeled with $\delta^n_i : [n - 1] \to [n]$.
Then $g_m^{-1}$ applied to the complex gives the complex
$$
\ldots \to
\bigoplus\nolimits_{\alpha \in \Mor_\Delta([2], [m])]} \mathbf{Z} \to
\bigoplus\nolimits_{\alpha \in \Mor_\Delta([1], [m])]} \mathbf{Z} \to
\bigoplus\nolimits_{\alpha \in \Mor_\Delta([0], [m])]} \mathbf{Z}
$$
on $\mathcal{C}_m$.
In other words, this is the complex associated to the
free abelian sheaf on the simplicial set $\Delta[m]$, see
Simplicial, Example \ref{simplicial-example-simplex-simplicial-set}.
Since $\Delta[m]$ is homotopy equivalent to $\Delta[0]$, see
Simplicial, Example \ref{simplicial-example-simplex-contractible},
and since ``taking free abelian sheaf on'' is a functor,
we see that the complex above is homotopy equivalent to
the free abelian sheaf on $\Delta[0]$
(Simplicial, Remark \ref{simplicial-remark-homotopy-better} and
Lemma \ref{simplicial-lemma-homotopy-equivalence-s-N}).
This complex is acyclic in positive degrees
and equal to $\mathbf{Z}$ in degree $0$.
\end{proof}

\begin{lemma}
\label{lemma-cech-complex}
In Situation \ref{situation-simplicial-site}. Let $\mathcal{F}$ be an abelian
sheaf on $\mathcal{C}_{total}$ there is a canonical complex
$$
0 \to \Gamma(\mathcal{C}_{total}, \mathcal{F}) \to
\Gamma(\mathcal{C}_0, \mathcal{F}_0) \to
\Gamma(\mathcal{C}_1, \mathcal{F}_1) \to
\Gamma(\mathcal{C}_2, \mathcal{F}_2) \to \ldots
$$
which is exact in degrees $-1, 0$ and exact everywhere
if $\mathcal{F}$ is injective.
\end{lemma}

\begin{proof}
Observe that
$\Hom(\mathbf{Z}, \mathcal{F}) = \Gamma(\mathcal{C}_{total}, \mathcal{F})$
and
$\Hom(g_{n!}\mathbf{Z}, \mathcal{F}) = \Gamma(\mathcal{C}_n, \mathcal{F}_n)$.
Hence this lemma is an immediate consequence of
Lemma \ref{lemma-simplicial-resolution-Z-site}
and the fact that $\Hom(-, \mathcal{F})$ is exact if
$\mathcal{F}$ is injective.
\end{proof}

\begin{lemma}
\label{lemma-simplicial-sheaf-cohomology-site}
In Situation \ref{situation-simplicial-site}. For $K$ in
$D^+(\mathcal{C}_{total})$ there is a spectral sequence
$(E_r, d_r)_{r \geq 0}$ with
$$
E_1^{p, q} = H^q(\mathcal{C}_p, K_p),\quad
d_1^{p, q} : E_1^{p, q} \to E_1^{p + 1, q}
$$
converging to $H^{p + q}(\mathcal{C}_{total}, K)$.
This spectral sequence is functorial in $K$.
\end{lemma}

\begin{proof}
Let $\mathcal{I}^\bullet$ be a bounded below complex of injectives
representing $K$. Consider the double complex with terms
$$
A^{p, q} = \Gamma(\mathcal{C}_p, \mathcal{I}^q_p)
$$
where the horizontal arrows come from Lemma \ref{lemma-cech-complex}
and the vertical arrows from the differentials of the
complex $\mathcal{I}^\bullet$. The rows of the double complex are exact
in positive degrees and evaluate to
$\Gamma(\mathcal{C}_{total}, \mathcal{I}^q)$ in degree $0$.
On the other hand, since restriction to $\mathcal{C}_p$ is exact
(Lemma \ref{lemma-restriction-to-components-site})
the complex $\mathcal{I}_p^\bullet$ represents $K_p$ in
$D(\mathcal{C}_p)$. The sheaves $\mathcal{I}_p^q$ are injective
abelian sheaves on $\mathcal{C}_p$
(Lemma \ref{lemma-restriction-injective-to-component-site}).
Hence the cohomology of the columns computes the groups
$H^q(\mathcal{C}_p, K_p)$. We conclude by applying
Homology, Lemmas \ref{homology-lemma-first-quadrant-ss} and
\ref{homology-lemma-double-complex-gives-resolution}.
\end{proof}

\begin{lemma}
\label{lemma-sanity-check}
Let $\mathcal{C}$ be as in Situation \ref{situation-simplicial-site}.
Let $U \in \Ob(\mathcal{C}_n)$. Let
$\mathcal{F} \in \textit{Ab}(\mathcal{C}_{total})$.
Then $H^p(U, \mathcal{F}) = H^p(U, g_n^{-1}\mathcal{F})$
where on the left hand side $U$ is viewed as an object of $\mathcal{C}_{total}$.
\end{lemma}

\begin{proof}
Observe that ``$U$ viewed as object of $\mathcal{C}_{total}$''
is explained by the construction of $\mathcal{C}_{total}$ in
Lemma \ref{lemma-simplicial-site-site} in case (A) and
Lemma \ref{lemma-simplicial-cocontinuous-site} in case (B).
The equality then follows from
Lemma \ref{lemma-restriction-injective-to-component-site}
and the definition of cohomology.
\end{proof}






\section{Cohomology and augmentations of simplicial sites}
\label{section-cohomology-augmentation-simplicial-sites}

\noindent
Consider a simplicial site $\mathcal{C}$ as in
Situation \ref{situation-simplicial-site}.
Let $a_0$ be an augmentation towards a site $\mathcal{D}$ as in
Remark \ref{remark-augmentation-site}.
By Lemma \ref{lemma-augmentation-site} we obtain a morphism of topoi
$$
a : \Sh(\mathcal{C}_{total}) \longrightarrow \Sh(\mathcal{D})
$$
and morphisms of topoi
$g_n : \Sh(\mathcal{C}_n) \to \Sh(\mathcal{C}_{total})$
as in Lemma \ref{lemma-restriction-to-components-site}.
The compositions $a \circ g_n$ are denoted
$a_n : \Sh(\mathcal{C}_n) \to \Sh(\mathcal{D})$.
Furthermore, the simplicial structure gives
morphisms of topoi
$f_\varphi : \Sh(\mathcal{C}_n) \to \Sh(\mathcal{C}_m)$
such that $a_n \circ f_\varphi = a_m$ for all $\varphi : [m] \to [n]$.

\begin{lemma}
\label{lemma-simplicial-resolution-augmentation}
In Situation \ref{situation-simplicial-site} let
$a_0$ be an augmentation towards a site $\mathcal{D}$
as in Remark \ref{remark-augmentation-site}.
For any abelian sheaf $\mathcal{G}$ on $\mathcal{D}$ 
there is an exact complex
$$
\ldots \to
g_{2!}(a_2^{-1}\mathcal{G}) \to
g_{1!}(a_1^{-1}\mathcal{G}) \to
g_{0!}(a_0^{-1}\mathcal{G}) \to
a^{-1}\mathcal{G} \to 0
$$
of abelian sheaves on $\mathcal{C}_{total}$.
\end{lemma}

\begin{proof}
We encourage the reader to read the proof of
Lemma \ref{lemma-simplicial-resolution-Z-site} first.
We will use Lemma \ref{lemma-augmentation-site} and
the description of the functors $g_{n!}$ in
Lemma \ref{lemma-restriction-to-components-site} without further mention.
In particular $g_{n!}(a_n^{-1}\mathcal{G})$ is the
sheaf on $\mathcal{C}_{total}$ whose restriction to $\mathcal{C}_m$
is the sheaf
$$
\bigoplus\nolimits_{\varphi : [n] \to [m]} f_\varphi^{-1}a_n^{-1}\mathcal{G} =
\bigoplus\nolimits_{\varphi : [n] \to [m]} a_m^{-1}\mathcal{G}
$$
As maps of the complex we take $\sum (-1)^i d^n_i$ where
$d^n_i : g_{n!}(a_n^{-1}\mathcal{G}) \to g_{n - 1!}(a_{n - 1}^{-1}\mathcal{G})$
is the adjoint to the map
$a_n^{-1}\mathcal{G} \to \bigoplus_{[n - 1] \to [n]} a_n^{-1}\mathcal{G} =
g_n^{-1}g_{n - 1!}(a_{n - 1}^{-1}\mathcal{G})$
corresponding to the factor labeled with $\delta^n_i : [n - 1] \to [n]$.
The map $g_{0!}(a_0^{-1}\mathcal{G}) \to a^{-1}\mathcal{G}$ is adjoint
to the identity map of $a_0^{-1}\mathcal{G}$.
Then $g_m^{-1}$ applied to the chain complex in degrees
$\ldots, 2, 1, 0$ gives the complex
$$
\ldots \to
\bigoplus\nolimits_{\alpha \in \Mor_\Delta([2], [m])]} a_m^{-1}\mathcal{G} \to
\bigoplus\nolimits_{\alpha \in \Mor_\Delta([1], [m])]} a_m^{-1}\mathcal{G} \to
\bigoplus\nolimits_{\alpha \in \Mor_\Delta([0], [m])]} a_m^{-1}\mathcal{G}
$$
on $\mathcal{C}_m$. This is equal to $a_m^{-1}\mathcal{G}$
tensored over the constant sheaf $\mathbf{Z}$ with the complex
$$
\ldots \to
\bigoplus\nolimits_{\alpha \in \Mor_\Delta([2], [m])]} \mathbf{Z} \to
\bigoplus\nolimits_{\alpha \in \Mor_\Delta([1], [m])]} \mathbf{Z} \to
\bigoplus\nolimits_{\alpha \in \Mor_\Delta([0], [m])]} \mathbf{Z}
$$
discussed in the proof of Lemma \ref{lemma-simplicial-resolution-Z-site}.
There we have seen that this complex is homotopy equivalent to
$\mathbf{Z}$ placed in degree $0$ which finishes the proof.
\end{proof}

\begin{lemma}
\label{lemma-augmentation-cech-complex}
In Situation \ref{situation-simplicial-site} let
$a_0$ be an augmentation towards a site $\mathcal{D}$
as in Remark \ref{remark-augmentation-site}.
For an abelian sheaf $\mathcal{F}$ on $\mathcal{C}_{total}$
there is a canonical complex
$$
0 \to a_*\mathcal{F} \to a_{0, *}\mathcal{F}_0 \to a_{1, *}\mathcal{F}_1 \to
a_{2, *}\mathcal{F}_2 \to \ldots
$$
on $\mathcal{D}$ which is exact in degrees $-1, 0$ and
exact everywhere if $\mathcal{F}$ is injective.
\end{lemma}

\begin{proof}
To construct the complex, by the Yoneda lemma, it suffices for any
abelian sheaf $\mathcal{G}$ on $\mathcal{D}$ to construct a complex
$$
0 \to \Hom(\mathcal{G}, a_*\mathcal{F}) \to
\Hom(\mathcal{G}, a_{0, *}\mathcal{F}_0) \to
\Hom(\mathcal{G}, a_{1, *}\mathcal{F}_1) \to \ldots
$$
functorially in $\mathcal{G}$. To do this apply $\Hom(-, \mathcal{F})$
to the exact complex of Lemma \ref{lemma-simplicial-resolution-augmentation}
and use adjointness of pullback and pushforward.
The exactness properties in degrees $-1, 0$ follow from
the construction as $\Hom(-, \mathcal{F})$ is left exact.
If $\mathcal{F}$ is an injective abelian sheaf, then the
complex is exact because $\Hom(-, \mathcal{F})$ is exact.
\end{proof}

\begin{lemma}
\label{lemma-augmentation-spectral-sequence}
In Situation \ref{situation-simplicial-site} let
$a_0$ be an augmentation towards a site $\mathcal{D}$
as in Remark \ref{remark-augmentation-site}.
For any $K$ in $D^+(\mathcal{C}_{total})$ there is a spectral
sequence 
$(E_r, d_r)_{r \geq 0}$ with
$$
E_1^{p, q} = R^qa_{p, *} K_p,\quad
d_1^{p, q} : E_1^{p, q} \to E_1^{p + 1, q}
$$
converging to $R^{p + q}a_*K$. This spectral sequence is functorial in $K$.
\end{lemma}

\begin{proof}
Let $\mathcal{I}^\bullet$ be a bounded below complex of injectives
representing $K$. Consider the double complex with terms
$$
A^{p, q} = a_{p, *}\mathcal{I}^q_p
$$
where the horizontal arrows come from
Lemma \ref{lemma-augmentation-cech-complex}
and the vertical arrows from the differentials of the
complex $\mathcal{I}^\bullet$. The rows of the double complex are exact
in positive degrees and evaluate to $a_*\mathcal{I}^q$ in degree $0$.
On the other hand, since restriction to $\mathcal{C}_p$ is exact
(Lemma \ref{lemma-restriction-to-components-site})
the complex $\mathcal{I}_p^\bullet$ represents $K_p$ in
$D(\mathcal{C}_p)$. The sheaves $\mathcal{I}_p^q$ are injective
abelian sheaves on $\mathcal{C}_p$
(Lemma \ref{lemma-restriction-injective-to-component-site}).
Hence the cohomology of the columns computes $R^qa_{p, *}K_p$.
We conclude by applying
Homology, Lemmas \ref{homology-lemma-first-quadrant-ss} and
\ref{homology-lemma-double-complex-gives-resolution}.
\end{proof}





\section{Cohomology on ringed simplicial sites}
\label{section-cohomology-simplicial-sites-modules}

\noindent
This section is the analogue of
Section \ref{section-cohomology-simplicial-sites}
for sheaves of modules.

\medskip\noindent
In Situation \ref{situation-simplicial-site} let $\mathcal{O}$
be a sheaf of rings on $\mathcal{C}_{total}$.
In statement of the following lemmas we will let
$g_n : (\Sh(\mathcal{C}_n), \mathcal{O}_n) \to
(\Sh(\mathcal{C}_{total}), \mathcal{O})$
be the morphism of ringed topoi of
Lemma \ref{lemma-restriction-module-to-components-site}.
If $\varphi : [m] \to [n]$ is a morphism of $\Delta$, then the diagram
of ringed topoi
$$
\xymatrix{
(\Sh(\mathcal{C}_n), \mathcal{O}_n) \ar[rd]_{g_n} \ar[rr]_{f_\varphi} & &
(\Sh(\mathcal{C}_m), \mathcal{O}_m) \ar[ld]^{g_m} \\
& (\Sh(\mathcal{C}_{total}), \mathcal{O})
}
$$
is not commutative, but there is a $2$-morphism $g_n \to g_m \circ f_\varphi$
coming from the maps
$\mathcal{F}(\varphi) : f_\varphi^{-1}\mathcal{F}_m \to \mathcal{F}_n$.
See Sites, Section \ref{sites-section-2-category}.

\begin{lemma}
\label{lemma-simplicial-resolution-ringed}
In Situation \ref{situation-simplicial-site} let $\mathcal{O}$
be a sheaf of rings on $\mathcal{C}_{total}$. There is a complex
$$
\ldots \to g_{2!}\mathcal{O}_2 \to g_{1!}\mathcal{O}_1 \to g_{0!}\mathcal{O}_0
$$
of $\mathcal{O}$-modules which forms a resolution of
$\mathcal{O}$.
Here $g_{n!}$ is as in Lemma \ref{lemma-restriction-module-to-components-site}.
\end{lemma}

\begin{proof}
We will use the description of $g_{n!}$ given in
Lemma \ref{lemma-restriction-to-components-site}.
As maps of the complex we take $\sum (-1)^i d^n_i$ where
$d^n_i : g_{n!}\mathcal{O}_n \to g_{n - 1!}\mathcal{O}_{n - 1}$
is the adjoint to the map
$\mathcal{O}_n \to \bigoplus_{[n - 1] \to [n]} \mathcal{O}_n =
g_n^*g_{n - 1!}\mathcal{O}_{n - 1}$
corresponding to the factor labeled with $\delta^n_i : [n - 1] \to [n]$.
Then $g_m^{-1}$ applied to the complex gives the complex
$$
\ldots \to
\bigoplus\nolimits_{\alpha \in \Mor_\Delta([2], [m])]} \mathcal{O}_m \to
\bigoplus\nolimits_{\alpha \in \Mor_\Delta([1], [m])]} \mathcal{O}_m \to
\bigoplus\nolimits_{\alpha \in \Mor_\Delta([0], [m])]} \mathcal{O}_m
$$
on $\mathcal{C}_m$.
In other words, this is the complex associated to the
free $\mathcal{O}_m$-module on the simplicial set $\Delta[m]$, see
Simplicial, Example \ref{simplicial-example-simplex-simplicial-set}.
Since $\Delta[m]$ is homotopy equivalent to $\Delta[0]$, see
Simplicial, Example \ref{simplicial-example-simplex-contractible},
and since ``taking free abelian sheaf on'' is a functor,
we see that the complex above is homotopy equivalent to
the free abelian sheaf on $\Delta[0]$
(Simplicial, Remark \ref{simplicial-remark-homotopy-better} and
Lemma \ref{simplicial-lemma-homotopy-equivalence-s-N}).
This complex is acyclic in positive degrees
and equal to $\mathcal{O}_m$ in degree $0$.
\end{proof}

\begin{lemma}
\label{lemma-cech-complex-modules}
In Situation \ref{situation-simplicial-site} let $\mathcal{O}$
be a sheaf of rings. Let $\mathcal{F}$ be a
sheaf of $\mathcal{O}$-modules. There is a canonical complex
$$
0 \to \Gamma(\mathcal{C}_{total}, \mathcal{F}) \to
\Gamma(\mathcal{C}_0, \mathcal{F}_0) \to
\Gamma(\mathcal{C}_1, \mathcal{F}_1) \to
\Gamma(\mathcal{C}_2, \mathcal{F}_2) \to \ldots
$$
which is exact in degrees $-1, 0$ and exact everywhere
if $\mathcal{F}$ is an injective $\mathcal{O}$-module.
\end{lemma}

\begin{proof}
Observe that
$\Hom(\mathcal{O}, \mathcal{F}) = \Gamma(\mathcal{C}_{total}, \mathcal{F})$
and
$\Hom(g_{n!}\mathcal{O}_n, \mathcal{F}) = \Gamma(\mathcal{C}_n, \mathcal{F}_n)$.
Hence this lemma is an immediate consequence of
Lemma \ref{lemma-simplicial-resolution-ringed}
and the fact that $\Hom(-, \mathcal{F})$ is exact if
$\mathcal{F}$ is injective.
\end{proof}

\begin{lemma}
\label{lemma-simplicial-module-cohomology-site}
In Situation \ref{situation-simplicial-site} let $\mathcal{O}$
be a sheaf of rings. For $K$ in $D^+(\mathcal{O})$
there is a spectral sequence $(E_r, d_r)_{r \geq 0}$ with
$$
E_1^{p, q} = H^q(\mathcal{C}_p, K_p),\quad
d_1^{p, q} : E_1^{p, q} \to E_1^{p + 1, q}
$$
converging to $H^{p + q}(\mathcal{C}_{total}, K)$.
This spectral sequence is functorial in $K$.
\end{lemma}

\begin{proof}
Let $\mathcal{I}^\bullet$ be a bounded below complex of injective
$\mathcal{O}$-modules representing $K$. Consider the double complex with terms
$$
A^{p, q} = \Gamma(\mathcal{C}_p, \mathcal{I}^q_p)
$$
where the horizontal arrows come from
Lemma \ref{lemma-cech-complex-modules}
and the vertical arrows from the differentials of the
complex $\mathcal{I}^\bullet$. Observe that
$\Gamma(\mathcal{D}, -) =
\Hom_{\mathcal{O}_\mathcal{D}}(\mathcal{O}_\mathcal{D}, -)$
on $\textit{Mod}(\mathcal{O}_\mathcal{D})$. Hence the lemma
says rows of the double complex are exact
in positive degrees and evaluate to
$\Gamma(\mathcal{C}_{total}, \mathcal{I}^q)$ in degree $0$.
Thus the total complex associated to the double complex
computes $R\Gamma(\mathcal{C}_{total}, K)$ by
Homology, Lemma \ref{homology-lemma-double-complex-gives-resolution}.
On the other hand, since restriction to $\mathcal{C}_p$ is exact
(Lemma \ref{lemma-restriction-to-components-site})
the complex $\mathcal{I}_p^\bullet$ represents $K_p$ in
$D(\mathcal{C}_p)$. The sheaves $\mathcal{I}_p^q$
are limp on $\mathcal{C}_p$
(Lemma \ref{lemma-restriction-injective-to-component-limp}).
Hence the cohomology of the columns computes the groups
$H^q(\mathcal{C}_p, K_p)$ by Leray's acyclicity lemma
(Derived Categories, Lemma \ref{derived-lemma-leray-acyclicity})
and
Cohomology on Sites, Lemma \ref{sites-cohomology-lemma-limp-acyclic}.
We conclude by applying
Homology, Lemma \ref{homology-lemma-first-quadrant-ss}.
\end{proof}

\begin{lemma}
\label{lemma-sanity-check-modules}
In Situation \ref{situation-simplicial-site} let $\mathcal{O}$
be a sheaf of rings. Let $U \in \Ob(\mathcal{C}_n)$. Let
$\mathcal{F} \in \textit{Mod}(\mathcal{O})$.
Then $H^p(U, \mathcal{F}) = H^p(U, g_n^*\mathcal{F})$
where on the left hand side $U$ is viewed as an object of
$\mathcal{C}_{total}$.
\end{lemma}

\begin{proof}
Observe that ``$U$ viewed as object of $\mathcal{C}_{total}$''
is explained by the construction of $\mathcal{C}_{total}$ in
Lemma \ref{lemma-simplicial-site-site} in case (A) and
Lemma \ref{lemma-simplicial-cocontinuous-site} in case (B).
In both cases the functor $\mathcal{C}_n \to \mathcal{C}$
is continuous and cocontinuous, see
Lemma \ref{lemma-restriction-to-components-site}, and
$g_n^{-1}\mathcal{O} = \mathcal{O}_n$ by definition.
Hence the result is a special case of
Cohomology on Sites, Lemma
\ref{sites-cohomology-lemma-pullback-same-cohomology}.
\end{proof}






\section{Cohomology and augmentations of ringed simplicial sites}
\label{section-cohomology-augmentation-ringed-simplicial-sites}

\noindent
This section is the analogue of
Section \ref{section-cohomology-augmentation-simplicial-sites}
for sheaves of modules.

\medskip\noindent
Consider a simplicial site $\mathcal{C}$ as in
Situation \ref{situation-simplicial-site}.
Let $a_0$ be an augmentation towards a site $\mathcal{D}$ as in
Remark \ref{remark-augmentation-site}.
Let $\mathcal{O}$ be a sheaf of rings on $\mathcal{C}_{total}$.
Let $\mathcal{O}_\mathcal{D}$ be a sheaf of rings on $\mathcal{D}$.
Suppose we are given a morphism
$$
a^\sharp : \mathcal{O}_\mathcal{D} \longrightarrow a_*\mathcal{O}
$$
where $a$ is as in Lemma \ref{lemma-augmentation-site}.
Consequently, we obtain a morphism of ringed topoi
$$
a :
(\Sh(\mathcal{C}_{total}), \mathcal{O})
\longrightarrow
(\Sh(\mathcal{D}), \mathcal{O}_\mathcal{D})
$$
We will think of $g_n : (\Sh(\mathcal{C}_n), \mathcal{O}_n) \to
(\Sh(\mathcal{C}_{total}), \mathcal{O})$ as a morphism of ringed topoi
as in
Lemma \ref{lemma-restriction-module-to-components-site}, then
taking the composition $a_n = a \circ g_n$
(Lemma \ref{lemma-augmentation-site})
as morphisms of ringed topoi we obtain
$$
a_n :
(\Sh(\mathcal{C}_n), \mathcal{O}_n)
\longrightarrow
(\Sh(\mathcal{D}), \mathcal{O}_\mathcal{D})
$$
Using the transition maps $f_\varphi^{-1}\mathcal{O}_m \to \mathcal{O}_n$
we obtain morphisms of ringed topoi
$$
f_\varphi : (\Sh(\mathcal{C}_n), \mathcal{O}_n) \to
(\Sh(\mathcal{C}_m), \mathcal{O}_m)
$$
such that $a_n \circ f_\varphi = a_m$ as morphisms of
ringed topoi for all $\varphi : [m] \to [n]$.

\begin{lemma}
\label{lemma-flat-augmentation-modules}
With notation as above. The morphism
$a : (\Sh(\mathcal{C}_{total}), \mathcal{O}) \to
(\Sh(\mathcal{D}), \mathcal{O}_\mathcal{D})$
is flat if and only if
$a_n : (\Sh(\mathcal{C}_n), \mathcal{O}_n) \to
(\Sh(\mathcal{D}), \mathcal{O}_\mathcal{D})$
is flat for $n \geq 0$.
\end{lemma}

\begin{proof}
Since $g_n : (\Sh(\mathcal{C}_n), \mathcal{O}_n) \to
(\Sh(\mathcal{C}_{total}), \mathcal{O})$ is flat, we see
that if $a$ is flat, then $a_n = a \circ g_n$ is flat as
a composition. Conversely, suppose that $a_n$ is flat for all $n$.
We have to check that $\mathcal{O}$ is flat as a sheaf of
$a^{-1}\mathcal{O}_\mathcal{D}$-modules. Let $\mathcal{F} \to \mathcal{G}$
be an injective map of $a^{-1}\mathcal{O}_\mathcal{D}$-modules.
We have to show that
$$
\mathcal{F} \otimes_{a^{-1}\mathcal{O}_\mathcal{D}} \mathcal{O}
\to
\mathcal{G} \otimes_{a^{-1}\mathcal{O}_\mathcal{D}} \mathcal{O}
$$
is injective. We can check this on $\mathcal{C}_n$, i.e., after
applying $g_n^{-1}$. Since $g_n^* = g_n^{-1}$ because
$g_n^{-1}\mathcal{O} = \mathcal{O}_n$ we obtain
$$
g_n^{-1}\mathcal{F} \otimes_{g_n^{-1}a^{-1}\mathcal{O}_\mathcal{D}}
\mathcal{O}_n
\to
g_n^{-1}\mathcal{G} \otimes_{g_n^{-1}a^{-1}\mathcal{O}_\mathcal{D}}
\mathcal{O}_n
$$
which is injective because
$g_n^{-1}a^{-1}\mathcal{O}_\mathcal{D} = a_n^{-1}\mathcal{O}_\mathcal{D}$
and we assume $a_n$ was flat.
\end{proof}

\begin{lemma}
\label{lemma-simplicial-resolution-augmentation-modules}
With notation as above. For a $\mathcal{O}_\mathcal{D}$-module $\mathcal{G}$
there is an exact complex
$$
\ldots \to
g_{2!}(a_2^*\mathcal{G}) \to
g_{1!}(a_1^*\mathcal{G}) \to
g_{0!}(a_0^*\mathcal{G}) \to
a^*\mathcal{G} \to 0
$$
of sheaves of $\mathcal{O}$-modules on $\mathcal{C}_{total}$.
Here $g_{n!}$ is as in Lemma \ref{lemma-restriction-module-to-components-site}.
\end{lemma}

\begin{proof}
Observe that $a^*\mathcal{G}$ is the $\mathcal{O}$-module on
$\mathcal{C}_{total}$ whose restriction to $\mathcal{C}_m$
is the $\mathcal{O}_m$-module $a_m^*\mathcal{G}$.
The description of the functors $g_{n!}$ on modules
in Lemma \ref{lemma-restriction-module-to-components-site}
shows that $g_{n!}(a_n^*\mathcal{G})$ is the
$\mathcal{O}$-module on $\mathcal{C}_{total}$
whose restriction to $\mathcal{C}_m$ is the $\mathcal{O}_m$-module
$$
\bigoplus\nolimits_{\varphi : [n] \to [m]} f_\varphi^*a_n^*\mathcal{G} =
\bigoplus\nolimits_{\varphi : [n] \to [m]} a_m^*\mathcal{G}
$$
The rest of the proof is exactly the same as the proof of
Lemma \ref{lemma-simplicial-resolution-augmentation},
replacing $a_m^{-1}\mathcal{G}$ by $a_m^*\mathcal{G}$.
\end{proof}

\begin{lemma}
\label{lemma-augmentation-cech-complex-modules}
With notation as above.
For an $\mathcal{O}$-module $\mathcal{F}$ on $\mathcal{C}_{total}$
there is a canonical complex
$$
0 \to a_*\mathcal{F} \to a_{0, *}\mathcal{F}_0 \to a_{1, *}\mathcal{F}_1 \to
a_{2, *}\mathcal{F}_2 \to \ldots
$$
of $\mathcal{O}_\mathcal{D}$-modules which is exact in degrees $-1, 0$.
If $\mathcal{F}$ is an injective $\mathcal{O}$-module, then the complex
is exact in all degrees and remains exact on applying the functor
$\Hom_{\mathcal{O}_\mathcal{D}}(\mathcal{G}, -)$ for any
$\mathcal{O}_\mathcal{D}$-module $\mathcal{G}$.
\end{lemma}

\begin{proof}
To construct the complex, by the Yoneda lemma, it suffices for any
$\mathcal{O}_\mathcal{D}$-modules $\mathcal{G}$ on $\mathcal{D}$
to construct a complex
$$
0 \to \Hom_{\mathcal{O}_\mathcal{D}}(\mathcal{G}, a_*\mathcal{F}) \to
\Hom_{\mathcal{O}_\mathcal{D}}(\mathcal{G}, a_{0, *}\mathcal{F}_0) \to
\Hom_{\mathcal{O}_\mathcal{D}}(\mathcal{G}, a_{1, *}\mathcal{F}_1) \to \ldots
$$
functorially in $\mathcal{G}$. To do this apply
$\Hom_\mathcal{O}(-, \mathcal{F})$
to the exact complex of
Lemma \ref{lemma-simplicial-resolution-augmentation-modules}
and use adjointness of pullback and pushforward.
The exactness properties in degrees $-1, 0$ follow from
the construction as $\Hom_\mathcal{O}(-, \mathcal{F})$ is left exact.
If $\mathcal{F}$ is an injective $\mathcal{O}$-module, then the
complex is exact because $\Hom_\mathcal{O}(-, \mathcal{F})$ is exact.
\end{proof}

\begin{lemma}
\label{lemma-augmentation-spectral-sequence-modules}
With notation as above for any $K$ in $D^+(\mathcal{O})$ there is a spectral
sequence $(E_r, d_r)_{r \geq 0}$ in $\textit{Mod}(\mathcal{O}_\mathcal{D})$
with
$$
E_1^{p, q} = R^qa_{p, *} K_p
$$
converging to $R^{p + q}a_*K$. This spectral sequence is functorial in $K$.
\end{lemma}

\begin{proof}
Let $\mathcal{I}^\bullet$ be a bounded below complex of injective
$\mathcal{O}$-modules representing $K$. Consider the double complex with terms
$$
A^{p, q} = a_{p, *}\mathcal{I}^q_p
$$
where the horizontal arrows come from
Lemma \ref{lemma-augmentation-cech-complex-modules}
and the vertical arrows from the differentials of the
complex $\mathcal{I}^\bullet$. The lemma
says rows of the double complex are exact
in positive degrees and evaluate to
$a_*\mathcal{I}^q$ in degree $0$.
Thus the total complex associated to the double complex
computes $Ra_*K$ by
Homology, Lemma \ref{homology-lemma-double-complex-gives-resolution}.
On the other hand, since restriction to $\mathcal{C}_p$ is exact
(Lemma \ref{lemma-restriction-to-components-site})
the complex $\mathcal{I}_p^\bullet$ represents $K_p$ in
$D(\mathcal{C}_p)$. The sheaves $\mathcal{I}_p^q$
are limp on $\mathcal{C}_p$
(Lemma \ref{lemma-restriction-injective-to-component-limp}).
Hence the cohomology of the columns are the sheaves
$R^qa_{p, *}K_p$ by Leray's acyclicity lemma
(Derived Categories, Lemma \ref{derived-lemma-leray-acyclicity})
and
Cohomology on Sites, Lemma \ref{sites-cohomology-lemma-limp-acyclic}.
We conclude by applying
Homology, Lemma \ref{homology-lemma-first-quadrant-ss}.
\end{proof}





\section{Cartesian sheaves and modules}
\label{section-cartesian}

\noindent
Here is the definition.

\begin{definition}
\label{definition-cartesian-sheaf}
In Situation \ref{situation-simplicial-site}.
\begin{enumerate}
\item A sheaf $\mathcal{F}$ of sets or of abelian groups on
$\mathcal{C}$ is {\it cartesian} if the maps
$\mathcal{F}(\varphi) : f_\varphi^{-1}\mathcal{F}_m \to \mathcal{F}_n$
are isomorphisms for all $\varphi : [m] \to [n]$.
\item If $\mathcal{O}$ is a sheaf of rings on $\mathcal{C}_{total}$,
then a sheaf $\mathcal{F}$ of $\mathcal{O}$-modules is
{\it cartesian} if  the maps $f_\varphi^*\mathcal{F}_m \to \mathcal{F}_n$
are isomorphisms for all $\varphi : [m] \to [n]$.
\item An object $K$ of $D(\mathcal{C}_{total})$ is {\it cartesian} if the maps
$f_\varphi^{-1}K_m \to K_n$
are isomorphisms for all $\varphi : [m] \to [n]$.
\item If $\mathcal{O}$ is a sheaf of rings on $\mathcal{C}_{total}$, then
an object $K$ of $D(\mathcal{O})$ is {\it cartesian} if the maps
$Lf_\varphi^*K_m \to K_n$
are isomorphisms for all $\varphi : [m] \to [n]$.
\end{enumerate}
\end{definition}

\noindent
Of course there is a general notion of a cartesian section of a
fibred category and the above are merely examples of this.
The property on pullbacks needs only be checked for the degeneracies.

\begin{lemma}
\label{lemma-check-cartesian-module}
In Situation \ref{situation-simplicial-site}.
\begin{enumerate}
\item A sheaf $\mathcal{F}$ of sets or abelian groups is cartesian
if and only if the maps
$(f_{\delta^n_j})^{-1}\mathcal{F}_{n - 1} \to \mathcal{F}_n$
are isomorphisms.
\item An object $K$ of $D(\mathcal{C}_{total})$ is cartesian
if and only if the maps
$(f_{\delta^n_j})^{-1}K_{n - 1} \to K_n$
are isomorphisms.
\item If $\mathcal{O}$ is a sheaf of rings on $\mathcal{C}_{total}$
a sheaf $\mathcal{F}$ of $\mathcal{O}$-modules is cartesian
if and only if the maps
$(f_{\delta^n_j})^*\mathcal{F}_{n - 1} \to \mathcal{F}_n$
are isomorphisms.
\item If $\mathcal{O}$ is a sheaf of rings on $\mathcal{C}_{total}$
an object $K$ of $D(\mathcal{O})$ is cartesian
if and only if the maps
$L(f_{\delta^n_j})^*K_{n - 1} \to K_n$
are isomorphisms.
\item Add more here.
\end{enumerate}
\end{lemma}

\begin{proof}
In each case the key is that the pullback functors
compose to pullback functor; for part (4) see
Cohomology on Sites, Lemma
\ref{sites-cohomology-lemma-derived-pullback-composition}.
We show how the argument works in case (1) and omit the proof
in the other cases.
The category $\Delta$ is generated by the morphisms
the morphisms $\delta^n_j$ and $\sigma^n_j$, see
Simplicial, Lemma \ref{simplicial-lemma-face-degeneracy}.
Hence we only need to check the maps
$(f_{\delta^n_j})^{-1}\mathcal{F}_{n - 1} \to \mathcal{F}_n$
and $(f_{\sigma^n_j})^{-1}\mathcal{F}_{n + 1} \to \mathcal{F}_n$ are
isomorphisms, see
Simplicial, Lemma \ref{simplicial-lemma-characterize-simplicial-object}
for notation. Since $\sigma^n_j \circ \delta_j^{n + 1} = \text{id}_{[n]}$
the composition
$$
\mathcal{F}_n =
(f_{\sigma^n_j})^{-1}
(f_{\delta_j^{n + 1}})^{-1}
\mathcal{F}_n \to
(f_{\sigma^n_j})^{-1}
\mathcal{F}_{n + 1} \to
\mathcal{F}_n
$$
is the identity. Thus the result for $\delta^{n + 1}_j$ implies the result
for $\sigma^n_j$.
\end{proof}

\begin{lemma}
\label{lemma-augmentation-cartesian-module}
In Situation \ref{situation-simplicial-site} let
$a_0$ be an augmentation towards a site $\mathcal{D}$ as in
Remark \ref{remark-augmentation-site}.
\begin{enumerate}
\item The pullback $a^{-1}\mathcal{G}$ of a sheaf of sets or abelian groups
on $\mathcal{D}$ is cartesian.
\item The pullback $a^{-1}K$ of an object $K$ of $D(\mathcal{D})$
is cartesian.
\end{enumerate}
Let $\mathcal{O}$ be a sheaf of rings on $\mathcal{C}_{total}$ and
$\mathcal{O}_\mathcal{D}$ a sheaf of rings on $\mathcal{D}$
and $a^\sharp : \mathcal{O}_\mathcal{D} \to a_*\mathcal{O}$ a
morphism as in
Section \ref{section-cohomology-augmentation-ringed-simplicial-sites}.
\begin{enumerate}
\item[(3)] The pullback $a^*\mathcal{F}$ of a sheaf of
$\mathcal{O}_\mathcal{D}$-modules is cartesian.
\item[(4)] The derived pullback $La^*K$ of an object
$K$ of $D(\mathcal{O}_\mathcal{D})$ is cartesian.
\end{enumerate}
\end{lemma}

\begin{proof}
This follows immediately from the identities
$a_m \circ f_\varphi = a_n$ for all $\varphi : [m] \to [n]$.
See Lemma \ref{lemma-augmentation-site} and the discussion in
Section \ref{section-cohomology-augmentation-ringed-simplicial-sites}.
\end{proof}

\begin{lemma}
\label{lemma-characterize-cartesian}
In Situation \ref{situation-simplicial-site}.
The category of cartesian sheaves of sets (resp.\ abelian groups)
is equivalent to the category of pairs $(\mathcal{F}, \alpha)$
where $\mathcal{F}$ is a sheaf of sets (resp.\ abelian groups)
on $\mathcal{C}_0$ and
$$
\alpha :
(f_{\delta_1^1})^{-1}\mathcal{F}
\longrightarrow (f_{\delta_0^1})^{-1}\mathcal{F}
$$
is an isomorphism of sheaves of sets (resp.\ abelian groups)
on $\mathcal{C}_1$ such that
$(f_{\delta^2_1})^{-1}\alpha =
(f_{\delta^2_0})^{-1}\alpha \circ (f_{\delta^2_2})^{-1}\alpha$
as maps of sheaves on $\mathcal{C}_2$.
\end{lemma}

\begin{proof}
We abbreviate
$d^n_j = f_{\delta^n_j} : \Sh(\mathcal{C}_n) \to \Sh(\mathcal{C}_{n - 1})$.
The condition on $\alpha$ in the statement of the lemma makes sense because
$$
d^1_1 \circ d^2_2 = d^1_1 \circ d^2_1, \quad
d^1_1 \circ d^2_0 = d^1_0 \circ d^2_2, \quad
d^1_0 \circ d^2_0 = d^1_0 \circ d^2_1
$$
as morphisms of topoi $\Sh(\mathcal{C}_2) \to \Sh(\mathcal{C}_0)$, see
Simplicial, Remark \ref{simplicial-remark-relations}. Hence we
can picture these maps as follows
$$
\xymatrix{
& (d^2_0)^{-1}(d^1_1)^{-1}\mathcal{F} \ar[r]_-{(d^2_0)^{-1}\alpha} &
(d^2_0)^{-1}(d^1_0)^{-1}\mathcal{F} \ar@{=}[rd] & \\
(d^2_2)^{-1}(d^1_0)^{-1}\mathcal{F} \ar@{=}[ru] & & &
(d^2_1)^{-1}(d^1_0)^{-1}\mathcal{F} \\
& (d^2_2)^{-1}(d^1_1)^{-1}\mathcal{F} \ar[lu]^{(d^2_2)^{-1}\alpha} \ar@{=}[r] &
(d^2_1)^{-1}(d^1_1)^{-1}\mathcal{F} \ar[ru]_{(d^2_1)^{-1}\alpha}
}
$$
and the condition signifies the diagram is commutative. It is clear that
given a cartesian sheaf $\mathcal{G}$ of sets (resp.\ abelian groups)
on $\mathcal{C}_{total}$
we can set $\mathcal{F} = \mathcal{G}_0$ and $\alpha$ equal to the composition
$$
(d_1^1)^{-1}\mathcal{G}_0 \to \mathcal{G}_1
\leftarrow (d_1^0)^{-1}\mathcal{G}_0
$$
where the arrows are invertible as $\mathcal{G}$ is cartesian.
To prove this functor
is an equivalence we construct a quasi-inverse. The construction of
the quasi-inverse is analogous to the construction discussed in
Descent, Section \ref{descent-section-descent-modules} from which we borrow
the notation $\tau^n_i : [0] \to [n]$, $0 \mapsto i$ and
$\tau^n_{ij} : [1] \to [n]$, $0 \mapsto i$, $1 \mapsto j$.
Namely, given a pair $(\mathcal{F}, \alpha)$
as in the lemma we set $\mathcal{G}_n = (f_{\tau^n_n})^{-1}\mathcal{F}$.
Given $\varphi : [n] \to [m]$ we define
$\mathcal{G}(\varphi) : (f_\varphi)^{-1}\mathcal{G}_n \to \mathcal{G}_m$
using
$$
\xymatrix{
(f_\varphi)^{-1}\mathcal{G}_n \ar@{=}[r] &
(f_\varphi)^{-1}(f_{\tau^n_n})^{-1}\mathcal{F} \ar@{=}[r] &
(f_{\tau^m_{\varphi(n)}})^{-1}\mathcal{F} \ar@{=}[r] &
(f_{\tau^m_{\varphi(n)m}})^{-1}(d^1_1)^{-1}\mathcal{F}
\ar[d]^{(f_{\tau^m_{\varphi(n)m}})^{-1}\alpha} \\
&
\mathcal{G}_m \ar@{=}[r] &
(f_{\tau^m_m})^{-1}\mathcal{F} \ar@{=}[r] &
(f_{\tau^m_{\varphi(n)m}})^{-1}(d^1_0)^{-1}\mathcal{F}
}
$$
We omit the verification that the commutativity of the displayed diagram
above implies the maps compose correctly and hence give rise to a
sheaf on $\mathcal{C}_{total}$, see
Lemma \ref{lemma-describe-sheaves-simplicial-site-site}.
We also omit the verification
that the two functors are quasi-inverse to each other.
\end{proof}

\begin{lemma}
\label{lemma-characterize-cartesian-modules}
In Situation \ref{situation-simplicial-site}
let $\mathcal{O}$ be a sheaf of rings on $\mathcal{C}_{total}$.
The category of cartesian $\mathcal{O}$-modules
is equivalent to the category of pairs $(\mathcal{F}, \alpha)$
where $\mathcal{F}$ is a $\mathcal{O}_0$-module
and
$$
\alpha :
(f_{\delta_1^1})^*\mathcal{F}
\longrightarrow (f_{\delta_0^1})^*\mathcal{F}
$$
is an isomorphism of $\mathcal{O}_1$-modules such that
$(f_{\delta^2_1})^*\alpha =
(f_{\delta^2_0})^*\alpha \circ (f_{\delta^2_2})^*\alpha$
as $\mathcal{O}_2$-module maps.
\end{lemma}

\begin{proof}
The proof is identical to the proof of
Lemma \ref{lemma-characterize-cartesian}
with pullback of sheaves of abelian groups replaced
by pullback of modules.
\end{proof}

\begin{lemma}
\label{lemma-Serre-subcat-cartesian-modules}
In Situation \ref{situation-simplicial-site}.
\begin{enumerate}
\item The full subcategory of cartesian abelian sheaves forms a
weak Serre subcategory of $\textit{Ab}(\mathcal{C}_{total})$.
Colimits of systems of cartesian abelian sheaves are cartesian.
\item Let $\mathcal{O}$ be a sheaf of rings on $\mathcal{C}_{total}$
such that the morphisms
$$
f_{\delta^n_j} : (\Sh(\mathcal{C}_n), \mathcal{O}_n)
\to (\Sh(\mathcal{C}_{n - 1}), \mathcal{O}_{n - 1})
$$
are flat. The full subcategory of cartesian $\mathcal{O}$-modules forms a
weak Serre subcategory of $\textit{Mod}(\mathcal{O})$.
Colimits of systems of cartesian $\mathcal{O}$-modules are cartesian.
\end{enumerate}
\end{lemma}

\begin{proof}
To see we obtain a weak Serre subcategory in (1)
we check the conditions listed in
Homology, Lemma \ref{homology-lemma-characterize-weak-serre-subcategory}.
First, if $\varphi : \mathcal{F} \to \mathcal{G}$ is a map
between cartesian abelian sheaves, then
$\Ker(\varphi)$ and $\Coker(\varphi)$ are cartesian too
because the restriction functors
$\Sh(\mathcal{C}_{total}) \to \Sh(\mathcal{C}_n)$
and the functors $f_\varphi^{-1}$ are exact.
Similarly, if
$$
0 \to \mathcal{F} \to \mathcal{H} \to \mathcal{G} \to 0
$$
is a short exact sequence of abelian sheaves on $\mathcal{C}_{total}$
with $\mathcal{F}$ and $\mathcal{G}$ cartesian, then it follows that
$\mathcal{H}$ is cartesian from the 5-lemma. To see the property of
colimits, use that colimits commute with pullback as pullback is a
left adjoint. In the case of modules
we argue in the same manner, using the exactness of flat pullback
(Modules on Sites, Lemma \ref{sites-modules-lemma-flat-pullback-exact})
and the fact that it suffices to check the condition
for $f_{\delta^n_j}$, see Lemma \ref{lemma-check-cartesian-module}.
\end{proof}

\begin{remark}[Warning]
\label{remark-warning-cartesian-modules}
Lemma \ref{lemma-Serre-subcat-cartesian-modules} notwithstanding, it
can happen that the category of cartesian $\mathcal{O}$-modules is
abelian without being a Serre subcategory of $\textit{Mod}(\mathcal{O})$.
Namely, suppose that we only know that
$f_{\delta_1^1}$ and $f_{\delta_0^1}$ are flat.
Then it follows easily from
Lemma \ref{lemma-characterize-cartesian-modules}
that the category of cartesian $\mathcal{O}$-modules is abelian.
But if $f_{\delta_0^2}$ is not flat (for example),
there is no reason for the inclusion functor
from the category of cartesian $\mathcal{O}$-modules
to all $\mathcal{O}$-modules to be exact.
\end{remark}

\begin{lemma}
\label{lemma-derived-cartesian-modules}
In Situation \ref{situation-simplicial-site}.
\begin{enumerate}
\item An object $K$ of $D(\mathcal{C}_{total})$ is cartesian if and only
if $H^q(K)$ is a cartesian abelian sheaf for all $q$.
\item Let $\mathcal{O}$ be a sheaf
of rings on $\mathcal{C}_{total}$ such that the morphisms
$f_{\delta^n_j} : (\Sh(\mathcal{C}_n), \mathcal{O}_n)
\to (\Sh(\mathcal{C}_{n - 1}), \mathcal{O}_{n - 1})$ are flat.
Then an object $K$ of $D(\mathcal{O})$ is cartesian if and only
if $H^q(K)$ is a cartesian $\mathcal{O}$-module for all $q$.
\end{enumerate}
\end{lemma}

\begin{proof}
Part (1) is true because the pullback functors $(f_\varphi)^{-1}$
are exact. Part (2) follows from the characterization in
Lemma \ref{lemma-check-cartesian-module}
and the fact that $L(f_{\delta^n_j})^* = (f_{\delta^n_j})^*$
by flatness.
\end{proof}

\begin{lemma}
\label{lemma-derived-cartesian-shriek}
In Situation \ref{situation-simplicial-site}.
\begin{enumerate}
\item An object $K$ of $D(\mathcal{C}_{total})$ is cartesian if and only
the canonical map
$$
g_{n!}K_n \longrightarrow
g_{n!}\mathbf{Z} \otimes^\mathbf{L}_\mathbf{Z} K
$$
is an isomorphism for all $n$.
\item Let $\mathcal{O}$ be a sheaf of rings on $\mathcal{C}_{total}$
such that the morphisms $f_\varphi^{-1}\mathcal{O}_n \to \mathcal{O}_m$
are flat for all $\varphi : [n] \to [m]$. Then an object $K$ of
$D(\mathcal{O})$ is cartesian if and only if the canonical map
$$
g_{n!}K_n \longrightarrow
g_{n!}\mathcal{O}_n \otimes^\mathbf{L}_\mathcal{O} K
$$
is an isomorphism for all $n$.
\end{enumerate}
\end{lemma}

\begin{proof}
Proof of (1). Since $g_{n!}$ is exact, it induces a functor
on derived categories adjoint to $g_n^{-1}$.
The map is the adjoint of the map
$K_n \to (g_n^{-1}g_{n!}\mathbf{Z}) \otimes^\mathbf{L}_\mathbf{Z} K_n$
corresponding to $\mathbf{Z} \to g_n^{-1}g_{n!}\mathbf{Z}$
which in turn is adjoint to
$\text{id} : g_{n!}\mathbf{Z} \to g_{n!}\mathbf{Z}$.
Using the description of $g_{n!}$
given in Lemma \ref{lemma-restriction-to-components-site}
we see that the restriction to $\mathcal{C}_m$ of this map
is
$$
\bigoplus\nolimits_{\varphi : [n] \to [m]} f_\varphi^{-1}K_n
\longrightarrow
\bigoplus\nolimits_{\varphi : [n] \to [m]} K_m
$$
Thus the statement is clear.

\medskip\noindent
Proof of (2). Since $g_{n!}$ is exact
(Lemma \ref{lemma-exactness-g-shriek-modules}), it induces a functor
on derived categories adjoint to $g_n^*$ (also exact).
The map is the adjoint of the map
$K_n \to (g_n^*g_{n!}\mathcal{O}_n) \otimes^\mathbf{L}_{\mathcal{O}_n} K_n$
corresponding to $\mathcal{O}_n \to g_n^*g_{n!}\mathcal{O}_n$
which in turn is adjoint to
$\text{id} : g_{n!}\mathcal{O}_n \to g_{n!}\mathcal{O}_n$.
Using the description of $g_{n!}$
given in Lemma \ref{lemma-restriction-module-to-components-site}
we see that the restriction to $\mathcal{C}_m$ of this map
is
$$
\bigoplus\nolimits_{\varphi : [n] \to [m]} f_\varphi^*K_n
\longrightarrow
\bigoplus\nolimits_{\varphi : [n] \to [m]}
f_\varphi^*\mathcal{O}_n \otimes_{\mathcal{O}_m} K_m =
\bigoplus\nolimits_{\varphi : [n] \to [m]} K_m
$$
Thus the statement is clear.
\end{proof}

\begin{lemma}
\label{lemma-quasi-coherent-sheaf}
In Situation \ref{situation-simplicial-site}
let $\mathcal{O}$ be a sheaf of rings on $\mathcal{C}_{total}$.
Let $\mathcal{F}$ be a sheaf of $\mathcal{O}$-modules.
Then $\mathcal{F}$ is quasi-coherent in the sense of
Modules on Sites, Definition \ref{sites-modules-definition-site-local}
if and only if $\mathcal{F}$ is cartesian
and $\mathcal{F}_n$ is a quasi-coherent $\mathcal{O}_n$-module for all $n$.
\end{lemma}

\begin{proof}
Assume $\mathcal{F}$ is quasi-coherent. Since pullbacks of
quasi-coherent modules are quasi-coherent
(Modules on Sites, Lemma \ref{sites-modules-lemma-local-pullback})
we see that $\mathcal{F}_n$ is a quasi-coherent $\mathcal{O}_n$-module
for all $n$. To show that $\mathcal{F}$ is cartesian, let $U$
be an object of $\mathcal{C}_n$ for some $n$. Let us view $U$
as an object of $\mathcal{C}_{total}$. Because $\mathcal{F}$
is quasi-coherent there exists a covering $\{U_i \to U\}$
and for each $i$ a presentation
$$
\bigoplus\nolimits_{j \in J_i} \mathcal{O}_{\mathcal{C}_{total}/U_i} \to
\bigoplus\nolimits_{k \in K_i} \mathcal{O}_{\mathcal{C}_{total}/U_i} \to
\mathcal{F}|_{\mathcal{C}_{total}/U_i} \to 0
$$
Observe that $\{U_i \to U\}$ is a covering of $\mathcal{C}_n$ by
the construction of the site $\mathcal{C}_{total}$.
Next, let $V$ be an object of $\mathcal{C}_m$ for some $m$ and let
$V \to U$ be a morphism of $\mathcal{C}_{total}$ lying over
$\varphi : [n] \to [m]$. The fibre products $V_i = V \times_U U_i$
exist and we get an induced covering $\{V_i \to V\}$ in $\mathcal{C}_m$.
Restricting the presentation above to the sites
$\mathcal{C}_n/U_i$ and $\mathcal{C}_m/V_i$ we obtain
presentations
$$
\bigoplus\nolimits_{j \in J_i} \mathcal{O}_{\mathcal{C}_m/U_i} \to
\bigoplus\nolimits_{k \in K_i} \mathcal{O}_{\mathcal{C}_m/U_i} \to
\mathcal{F}_n|_{\mathcal{C}_n/U_i} \to 0
$$
and
$$
\bigoplus\nolimits_{j \in J_i} \mathcal{O}_{\mathcal{C}_m/V_i} \to
\bigoplus\nolimits_{k \in K_i} \mathcal{O}_{\mathcal{C}_m/V_i} \to
\mathcal{F}_m|_{\mathcal{C}_m/V_i} \to 0
$$
These presentations are compatible with the map
$\mathcal{F}(\varphi) : f_\varphi^*\mathcal{F}_n \to \mathcal{F}_m$
(as this map is defined using the restriction maps of $\mathcal{F}$
along morphisms of $\mathcal{C}_{total}$ lying over $\varphi$).
We conclude that $\mathcal{F}(\varphi)|_{\mathcal{C}_m/V_i}$
is an isomorphism. As $\{V_i \to V\}$ is a covering we conclude
$\mathcal{F}(\varphi)|_{\mathcal{C}_m/V}$ is an isomorphism.
Since $V$ and $U$ were arbitrary this proves that $\mathcal{F}$ is cartesian.
(In case A use Sites, Lemma \ref{sites-lemma-morphism-of-sites-covering}.)

\medskip\noindent
Conversely, assume $\mathcal{F}_n$ is quasi-coherent
for all $n$ and that $\mathcal{F}$ is cartesian.
Then for any $n$ and object $U$ of $\mathcal{C}_n$ we
can choose a covering $\{U_i \to U\}$ of $\mathcal{C}_n$
and for each $i$ a presentation
$$
\bigoplus\nolimits_{j \in J_i} \mathcal{O}_{\mathcal{C}_m/U_i} \to
\bigoplus\nolimits_{k \in K_i} \mathcal{O}_{\mathcal{C}_m/U_i} \to
\mathcal{F}_n|_{\mathcal{C}_n/U_i} \to 0
$$
Pulling back to $\mathcal{C}_{total}/U_i$ we obtain complexes
$$
\bigoplus\nolimits_{j \in J_i} \mathcal{O}_{\mathcal{C}_{total}/U_i} \to
\bigoplus\nolimits_{k \in K_i} \mathcal{O}_{\mathcal{C}_{total}/U_i} \to
\mathcal{F}|_{\mathcal{C}_{total}/U_i} \to 0
$$
of modules on $\mathcal{C}_{total}/U_i$. Then the property that
$\mathcal{F}$ is cartesian implies that this is exact.
We omit the details.
\end{proof}











\section{Simplicial systems of the derived category}
\label{section-glueing}

\noindent
In this section we are going to prove a special case of
\cite[Proposition 3.2.9]{BBD} in the setting of derived
categories of abelian sheaves. The case of modules
is discussed in Section \ref{section-glueing-modules}.

\begin{definition}
\label{definition-cartesian-derived}
In Situation \ref{situation-simplicial-site}. A
{\it simplicial system of the derived category}
consists of the following data
\begin{enumerate}
\item for every $n$ an object $K_n$ of $D(\mathcal{C}_n)$,
\item for every $\varphi : [m] \to [n]$ a map
$K_\varphi : f_\varphi^{-1}K_m \to K_n$ in $D(\mathcal{C}_n)$
\end{enumerate}
subject to the condition that
$$
K_{\varphi \circ \psi} = K_\varphi \circ f_\varphi^{-1}K_\psi :
f_{\varphi \circ \psi}^{-1}K_l = f_\varphi^{-1} f_\psi^{-1}K_l
\longrightarrow
K_n
$$
for any morphisms $\varphi : [m] \to [n]$ and $\psi : [l] \to [m]$ of $\Delta$.
We say the simplicial system is {\it cartesian} if the maps $K_\varphi$
are isomorphisms for all $\varphi$.
Given two simplicial systems of the derived category
there is an obvious notion of a
{\it morphism of simplicial systems of the derived category}.
\end{definition}

\noindent
We have given this notion a ridiculously long name intentionally.
The goal is to show that a simplicial system of the derived category
comes from an object of $D(\mathcal{C}_{total})$ under certain
hypotheses.

\begin{lemma}
\label{lemma-cartesian-objects-derived}
In Situation \ref{situation-simplicial-site}.
If $K \in D(\mathcal{C}_{total})$ is an object,
then $(K_n, K(\varphi))$ is a simplicial system of the derived category.
If $K$ is cartesian, so is the system.
\end{lemma}

\begin{proof}
This is obvious.
\end{proof}

\begin{lemma}
\label{lemma-abelian-postnikov}
In Situation \ref{situation-simplicial-site}. Let $K$ be
an object of $D(\mathcal{C}_{total})$. Set
$$
X_n = (g_{n!}\mathbf{Z})
\otimes^\mathbf{L}_\mathbf{Z} K
\quad\text{and}\quad
Y_n =
(g_{n!}\mathbf{Z} \to \ldots \to g_{0!}\mathbf{Z})[-n]
\otimes^\mathbf{L}_\mathbf{Z} K
$$
as objects of $D(\mathcal{C}_{total})$ where the maps are
as in Lemma \ref{lemma-simplicial-resolution-Z-site}.
With the evident canonical maps $Y_n \to X_n$ and
$Y_0 \to Y_1[1] \to Y_2[2] \to \ldots$ we have
\begin{enumerate}
\item the distinguished triangles $Y_n \to X_n \to Y_{n - 1} \to Y_n[1]$
define a Postnikov system
(Derived Categories, Definition \ref{derived-definition-postnikov-system})
for $\ldots \to X_2 \to X_1 \to X_0$,
\item $K = \text{hocolim} Y_n[n]$ in $D(\mathcal{C}_{total})$.
\end{enumerate}
\end{lemma}

\begin{proof}
First, if $K = \mathbf{Z}$, then this is the construction of
Derived Categories, Example \ref{derived-example-key-postnikov}
applied to the complex
$$
\ldots \to
g_{2!}\mathbf{Z} \to
g_{1!}\mathbf{Z} \to
g_{0!}\mathbf{Z}
$$
in $\textit{Ab}(\mathcal{C}_{total})$ combined with the fact that
this complex represents $K = \mathbf{Z}$ in $D(\mathcal{C}_{total})$
by Lemma \ref{lemma-simplicial-resolution-Z-site}.
The general case follows from this, the fact that the exact functor
$- \otimes^\mathbf{L}_\mathbf{Z} K$ sends Postnikov systems to
Postnikov systems, and
that $- \otimes^\mathbf{L}_\mathbf{Z} K$ commutes with homotopy colimits.
\end{proof}

\begin{lemma}
\label{lemma-nullity-cartesian-objects-derived}
In Situation \ref{situation-simplicial-site}.
If $K, K' \in D(\mathcal{C}_{total})$.
Assume
\begin{enumerate}
\item $K$ is cartesian,
\item $\Hom(K_i[i], K'_i) = 0$ for $i > 0$, and
\item $\Hom(K_i[i + 1], K'_i) = 0$ for $i \geq 0$.
\end{enumerate}
Then any map $K \to K'$ which induces the zero map $K_0 \to K'_0$ is zero.
\end{lemma}

\begin{proof}
Consider the objects $X_n$ and the Postnikov system $Y_n$
associated to $K$ in Lemma \ref{lemma-abelian-postnikov}.
As $K = \text{hocolim} Y_n[n]$ the map $K \to K'$ induces
a compatible family of morphisms $Y_n[n] \to K'$.
By (1) and Lemma \ref{lemma-derived-cartesian-shriek} we have
$X_n = g_{n!}K_n$. Since $Y_0 = X_0$ we find that
$K_0 \to K'_0$ being zero implies $Y_0 \to K'$ is zero.
Suppose we've shown that the map $Y_n[n] \to K'$ is zero
for some $n \geq 0$. From the distinguished triangle
$$
Y_n[n] \to Y_{n + 1}[n + 1] \to X_{n + 1}[n + 1] \to Y_n[n + 1]
$$
we get an exact sequence
$$
\Hom(X_{n + 1}[n + 1], K') \to
\Hom(Y_{n + 1}[n + 1], K') \to
\Hom(Y_n[n], K')
$$
As $X_{n + 1}[n + 1] = g_{n + 1!}K_{n + 1}[n + 1]$ the first group is equal to
$$
\Hom(K_{n + 1}[n + 1], K'_{n + 1})
$$
which is zero by assumption (2). By induction we conclude all the maps
$Y_n[n] \to K'$ are zero. Consider the defining distinguished triangle
$$
\bigoplus Y_n[n] \to
\bigoplus Y_n[n] \to
K \to
(\bigoplus Y_n[n])[1]
$$
for the homotopy colimit. Arguing as above, we find that it suffices
to show that
$$
\Hom((\bigoplus Y_n[n])[1], K') = \prod \Hom(Y_n[n + 1], K')
$$
is zero for all $n \geq 0$. To see this, arguing as above,
it suffices to show that
$$
\Hom(K_n[n + 1], K'_n)  = 0
$$
for all $n \geq 0$ which follows from condition (3).
\end{proof}

\begin{lemma}
\label{lemma-hom-cartesian-objects-derived}
In Situation \ref{situation-simplicial-site}.
If $K, K' \in D(\mathcal{C}_{total})$.
Assume
\begin{enumerate}
\item $K$ is cartesian,
\item $\Hom(K_i[i - 1], K'_i) = 0$ for $i > 1$.
\end{enumerate}
Then any map $\{K_n \to K'_n\}$ between the associated simplicial systems 
of $K$ and $K'$ comes from a map $K \to K'$ in $D(\mathcal{C}_{total})$.
\end{lemma}

\begin{proof}
Let $\{K_n \to K'_n\}_{n \geq 0}$
be a morphism of simplicial systems of the derived category.
Consider the objects $X_n$ and Postnikov system $Y_n$
associated to $K$ of Lemma \ref{lemma-abelian-postnikov}.
By (1) and Lemma \ref{lemma-derived-cartesian-shriek} we have
$X_n = g_{n!}K_n$. In particular, the map $K_0 \to K'_0$
induces a morphism $X_0 \to K'$. Since $\{K_n \to K'_n\}$
is a morphism of systems, a computation (omitted) shows that
the composition
$$
X_1 \to X_0 \to K'
$$
is zero. As $Y_0 = X_0$ and as $Y_1$ fits into a distinguished
triangle
$$
Y_1 \to X_1 \to Y_0 \to Y_1[1]
$$
we conclude that there exists a morphism $Y_1[1] \to K'$ whose
composition with $X_0 = Y_0 \to Y_1[1]$ is the morphism $X_0 \to K'$
given above. Suppose given a map $Y_n[n] \to K'$ for $n \geq 1$.
From the distinguished triangle
$$
X_{n + 1}[n] \to Y_n[n] \to Y_{n + 1}[n + 1] \to X_{n + 1}[n + 1]
$$
we get an exact sequence
$$
\Hom(Y_{n + 1}[n + 1], K') \to
\Hom(Y_n[n], K') \to
\Hom(X_{n + 1}[n], K')
$$
As $X_{n + 1}[n] = g_{n + 1!}K_{n + 1}[n]$ the last group is equal to
$$
\Hom(K_{n + 1}[n], K'_{n + 1})
$$
which is zero by assumption (2). By induction we get a system of
maps $Y_n[n] \to K'$ compatible with transition maps and reducing
to the given map on $Y_0$. This produces a map
$$
\gamma :
K = \text{hocolim} Y_n[n]
\longrightarrow
K'
$$
This map in any case has the property that the diagram
$$
\xymatrix{
X_0 \ar[rd] \ar[r] &
K \ar[d]^\gamma \\
& K'
}
$$
is commutative. Restricting to
$\mathcal{C}_0$ we deduce that the map $\gamma_0 : K_0 \to K'_0$
is the same as the first map $K_0 \to K'_0$ of the morphism
of simplicial systems. Since $K$ is cartesian, this easily gives that
$\{\gamma_n\}$ is the map of simplicial systems we started out with.
\end{proof}

\begin{lemma}
\label{lemma-cartesian-object-derived-from-simplicial}
In Situation \ref{situation-simplicial-site}. Let
$(K_n, K_\varphi)$ be a simplicial system of the derived category.
Assume
\begin{enumerate}
\item $(K_n, K_\varphi)$ is cartesian,
\item $\Hom(K_i[t], K_i) = 0$ for $i \geq 0$ and $t > 0$.
\end{enumerate}
Then there exists a cartesian object $K$ of $D(\mathcal{C}_{total})$
whose associated simplicial system is isomorphic to $(K_n, K_\varphi)$.
\end{lemma}

\begin{proof}
Set $X_n = g_{n!}K_n$ in $D(\mathcal{C}_{total})$. For each $n \geq 1$
we have
$$
\Hom(X_n, X_{n - 1}) =
\Hom(K_n, g_n^{-1}g_{n - 1!}K_{n - 1}) =
\bigoplus\nolimits_{\varphi : [n - 1] \to [n]}
\Hom(K_n, f_\varphi^{-1}K_{n - 1})
$$
Thus we get a map $X_n \to X_{n - 1}$ corresponding to the
alternating sum of the maps
$K_\varphi^{-1} : K_n \to f_\varphi^{-1}K_{n - 1}$
where $\varphi$ runs over $\delta^n_0, \ldots, \delta^n_n$.
We can do this because $K_\varphi$ is invertible by assumption (1).
Please observe the similarity with the definition of the maps
in the proof of Lemma \ref{lemma-simplicial-resolution-Z-site}.
We obtain a complex
$$
\ldots \to X_2 \to X_1 \to X_0
$$
in $D(\mathcal{C}_{total})$. We omit the computation which shows
that the compositions are zero. By
Derived Categories, Lemma \ref{derived-lemma-existence-postnikov-system}
if we have
$$
\Hom(X_i[i - j - 2], X_j) = 0\text{ for }i > j + 2
$$
then we can extend this complex to a Postnikov system.
The group is equal to
$$
\Hom(K_i[i - j - 2], g_i^{-1}g_{j!}K_j)
$$
Again using that $(K_n, K_\varphi)$ is cartesian we see that
$g_i^{-1}g_{j!}K_j$ is isomorphic to a finite direct sum of copies of
$K_i$. Hence the group vanishes by assumption (2).
Let the Postnikov system be given by $Y_0 = X_0$ and distinguished
sequences $Y_n \to X_n \to Y_{n - 1} \to Y_n[1]$ for $n \geq 1$.
We set
$$
K = \text{hocolim} Y_n[n]
$$
To finish the proof we have to show that $g_m^{-1}K$ is isomorphic
to $K_m$ for all $m$ compatible with the maps $K_\varphi$. Observe that
$$
g_m^{-1} K = \text{hocolim} g_m^{-1}Y_n[n]
$$
and that $g_m^{-1}Y_n[n]$ is a Postnikov system for $g_m^{-1}X_n$.
Consider the isomorphisms
$$
g_m^{-1}X_n =
\bigoplus\nolimits_{\varphi : [n] \to [m]} f_\varphi^{-1}K_n
\xrightarrow{\bigoplus K_\varphi}
\bigoplus\nolimits_{\varphi : [n] \to [m]} K_m
$$
These maps define an isomorphism of complexes
$$
\xymatrix{
\ldots \ar[r] &
g_m^{-1}X_2 \ar[r] \ar[d] &
g_m^{-1}X_1 \ar[r] \ar[d] &
g_m^{-1}X_0 \ar[d] \\
\ldots \ar[r] &
\bigoplus\nolimits_{\varphi : [2] \to [m]} K_m \ar[r] &
\bigoplus\nolimits_{\varphi : [1] \to [m]} K_m \ar[r] &
\bigoplus\nolimits_{\varphi : [0] \to [m]} K_m
}
$$
in $D(\mathcal{C}_m)$ where the arrows in the bottom row are as
in the proof of Lemma \ref{lemma-simplicial-resolution-Z-site}.
The squares commute by our choice of the arrows of the complex
$\ldots \to X_2 \to X_1 \to X_0$; we omit the computation.
The bottom row complex has a postnikov tower given by
$$
Y'_{m, n} =
\left(\bigoplus\nolimits_{\varphi : [n] \to [m]} \mathbf{Z} \to
\ldots \to
\bigoplus\nolimits_{\varphi : [0] \to [m]} \mathbf{Z}\right)[-n]
\otimes^\mathbf{L}_\mathbf{Z} K_m
$$
and $\text{hocolim} Y'_{m, n} = K_m$
(please compare with the proof of Lemma \ref{lemma-abelian-postnikov}
and Derived Categories, Example \ref{derived-example-key-postnikov}).
Applying the second part of
Derived Categories, Lemma \ref{derived-lemma-existence-postnikov-system}
the vertical maps in the big diagram extend to an isomorphism
of Postnikov systems provided we have
$$
\Hom(g_m^{-1}X_i[i - j - 1], \bigoplus\nolimits_{\varphi : [j] \to [m]} K_m)
= 0\text{ for }i > j + 1
$$
The is true if $\Hom(K_m[i - j - 1], K_m) = 0$ for $i > j + 1$
which holds by assumption (2). Choose an isomorphism given
by $\gamma_{m, n} : g_m^{-1}Y_n \to Y'_{m, n}$ of Postnikov systems
in $D(\mathcal{C}_m)$. By uniqueness of homotopy colimits,
we can find an isomorphism
$$
g_m^{-1} K = \text{hocolim} g_m^{-1}Y_n[n]
\xrightarrow{\gamma_m}
\text{hocolim} Y'_{m, n} = K_m
$$
compatible with $\gamma_{m, n}$.

\medskip\noindent
We still have to prove that the maps $\gamma_m$ fit into commutative diagrams
$$
\xymatrix{
f_\varphi^{-1}g_m^{-1}K \ar[d]_{f_\varphi^{-1}\gamma_m} \ar[r]_{K(\varphi)} &
g_n^{-1}K \ar[d]^{\gamma_n} \\
f_\varphi^{-1}K_m \ar[r]^{K_\varphi} &
K_n
}
$$
for every $\varphi : [m] \to [n]$. Consider the diagram
$$
\xymatrix{
f_\varphi^{-1}(\bigoplus_{\psi : [0] \to [m]} f_\psi^{-1}K_0)
\ar@{=}[r] \ar[d]_{f_\varphi^{-1}(\bigoplus K_\psi)} &
f_\varphi^{-1}g_m^{-1}X_0 \ar[d] \ar[r]_{X_0(\varphi)} &
g_n^{-1}X_0 \ar[d] &
\bigoplus_{\chi : [0] \to [n]} f_\chi^{-1}K_0
\ar@{=}[l] \ar[d]^{\bigoplus K_\chi} \\
f_\varphi^{-1}(\bigoplus_{\psi : [0] \to [m]} K_m) \ar@{=}[d] &
f_\varphi^{-1}g_m^{-1}K \ar[d]_{f_\varphi^{-1}\gamma_m} \ar[r]_{K(\varphi)} &
g_n^{-1}K \ar[d]^{\gamma_n} &
\bigoplus_{\chi : [0] \to [n]} K_n \ar@{=}[d] \\
f_\varphi^{-1}Y'_{0, m} \ar[r] &
f_\varphi^{-1}K_m \ar[r]^{K_\varphi} &
K_n &
Y'_{0, n} \ar[l]
}
$$
The top middle square is commutative as $X_0 \to K$ is a morphism
of simplicial objects. The left, resp.\ the right rectangles are
commutative as $\gamma_m$, resp.\ $\gamma_n$ is compatible with
$\gamma_{0, m}$, resp.\ $\gamma_{0, n}$ which are the arrows
$\bigoplus K_\psi$ and $\bigoplus K_\chi$ in the diagram.
Going around the outer rectangle of the diagram
is commutative as $(K_n, K_\varphi)$ is a simplical system
and the map $X_0(\varphi)$ is given by the obvious identifications
$f_\varphi^{-1}f_\psi^{-1}K_0 = f_{\varphi \circ \psi}^{-1}K_0$.
Note that the arrow $\bigoplus_\psi K_m \to Y'_{0, m} \to K_m$
induces an isomorphism on any of the direct summands
(because of our explicit construction of the Postnikov
systems $Y'_{i, j}$ above).
Hence, if we take a direct summand of
the upper left and corner, then this maps isomorphically to
$f_\varphi^{-1}g_m^{-1}K$ as $\gamma_m$ is an isomorphism.
Working out what the above says,
but looking only at this direct summand we conclude the lower
middle square commutes as we well. This concludes the proof.
\end{proof}









\section{Simplicial systems of the derived category: modules}
\label{section-glueing-modules}

\noindent
In this section we are going to prove a special case of
\cite[Proposition 3.2.9]{BBD} in the setting of derived
categories of $\mathcal{O}$-modules. The (slightly) easier
case of abelian sheaves is discussed in Section \ref{section-glueing}.

\begin{definition}
\label{definition-cartesian-derived-modules}
In Situation \ref{situation-simplicial-site}. Let $\mathcal{O}$
be a sheaf of rings on $\mathcal{C}_{total}$. A
{\it simplicial system of the derived category of modules}
consists of the following data
\begin{enumerate}
\item for every $n$ an object $K_n$ of $D(\mathcal{O}_n)$,
\item for every $\varphi : [m] \to [n]$ a map
$K_\varphi : Lf_\varphi^*K_m \to K_n$ in $D(\mathcal{O}_n)$
\end{enumerate}
subject to the condition that
$$
K_{\varphi \circ \psi} = K_\varphi \circ Lf_\varphi^*K_\psi :
Lf_{\varphi \circ \psi}^*K_l = Lf_\varphi^* Lf_\psi^*K_l
\longrightarrow
K_n
$$
for any morphisms $\varphi : [m] \to [n]$ and $\psi : [l] \to [m]$ of $\Delta$.
We say the simplicial system is {\it cartesian} if the maps $K_\varphi$
are isomorphisms for all $\varphi$.
Given two simplicial systems of the derived category
there is an obvious notion of a
{\it morphism of simplicial systems of the derived category of modules}.
\end{definition}

\noindent
We have given this notion a ridiculously long name intentionally.
The goal is to show that a simplicial system of the derived category
of modules comes from an object of $D(\mathcal{O})$ under certain
hypotheses.

\begin{lemma}
\label{lemma-cartesian-objects-derived-modules}
In Situation \ref{situation-simplicial-site} let $\mathcal{O}$ be a
sheaf of rings on $\mathcal{C}_{total}$.
If $K \in D(\mathcal{O})$ is an object, then $(K_n, K(\varphi))$
is a simplicial system of the derived category of modules.
If $K$ is cartesian, so is the system.
\end{lemma}

\begin{proof}
This is immediate from the definitions.
\end{proof}

\begin{lemma}
\label{lemma-modules-postnikov}
In Situation \ref{situation-simplicial-site} let $\mathcal{O}$
be a sheaf of rings on $\mathcal{C}_{total}$. Let $K$ be
an object of $D(\mathcal{C}_{total})$. Set
$$
X_n = (g_{n!}\mathcal{O}_n)
\otimes^\mathbf{L}_\mathcal{O} K
\quad\text{and}\quad
Y_n =
(g_{n!}\mathcal{O}_n \to \ldots \to g_{0!}\mathcal{O}_0)[-n]
\otimes^\mathbf{L}_\mathcal{O} K
$$
as objects of $D(\mathcal{O})$ where the maps are
as in Lemma \ref{lemma-simplicial-resolution-Z-site}.
With the evident canonical maps $Y_n \to X_n$ and
$Y_0 \to Y_1[1] \to Y_2[2] \to \ldots$ we have
\begin{enumerate}
\item the distinguished triangles $Y_n \to X_n \to Y_{n - 1} \to Y_n[1]$
define a Postnikov system
(Derived Categories, Definition \ref{derived-definition-postnikov-system})
for $\ldots \to X_2 \to X_1 \to X_0$,
\item $K = \text{hocolim} Y_n[n]$ in $D(\mathcal{O})$.
\end{enumerate}
\end{lemma}

\begin{proof}
First, if $K = \mathcal{O}$, then this is the construction of
Derived Categories, Example \ref{derived-example-key-postnikov}
applied to the complex
$$
\ldots \to
g_{2!}\mathcal{O}_2 \to
g_{1!}\mathcal{O}_1 \to
g_{0!}\mathcal{O}_0
$$
in $\textit{Ab}(\mathcal{C}_{total})$ combined with the fact that
this complex represents $K = \mathcal{O}$ in $D(\mathcal{C}_{total})$
by Lemma \ref{lemma-simplicial-resolution-ringed}.
The general case follows from this, the fact that the exact functor
$- \otimes^\mathbf{L}_\mathcal{O} K$ sends Postnikov systems to
Postnikov systems, and
that $- \otimes^\mathbf{L}_\mathcal{O} K$ commutes with homotopy colimits.
\end{proof}

\begin{lemma}
\label{lemma-nullity-cartesian-modules-derived}
In Situation \ref{situation-simplicial-site} let $\mathcal{O}$ be
a sheaf of rings on $\mathcal{C}_{total}$.
If $K, K' \in D(\mathcal{O})$.
Assume
\begin{enumerate}
\item $f_\varphi^{-1}\mathcal{O}_n \to \mathcal{O}_m$ is flat for
$\varphi : [m] \to [n]$,
\item $K$ is cartesian,
\item $\Hom(K_i[i], K'_i) = 0$ for $i > 0$, and
\item $\Hom(K_i[i + 1], K'_i) = 0$ for $i \geq 0$.
\end{enumerate}
Then any map $K \to K'$ which induces the zero map $K_0 \to K'_0$ is zero.
\end{lemma}

\begin{proof}
The proof is exactly the same as the proof of
Lemma \ref{lemma-nullity-cartesian-objects-derived} except using
Lemma \ref{lemma-modules-postnikov} instead of
Lemma \ref{lemma-abelian-postnikov}.
\end{proof}

\begin{lemma}
\label{lemma-hom-cartesian-modules-derived}
In Situation \ref{situation-simplicial-site} let $\mathcal{O}$ be
a sheaf of rings on $\mathcal{C}_{total}$.
If $K, K' \in D(\mathcal{O})$.
Assume
\begin{enumerate}
\item $f_\varphi^{-1}\mathcal{O}_n \to \mathcal{O}_m$ is flat for
$\varphi : [m] \to [n]$,
\item $K$ is cartesian,
\item $\Hom(K_i[i - 1], K'_i) = 0$ for $i > 1$.
\end{enumerate}
Then any map $\{K_n \to K'_n\}$ between the associated simplicial systems 
of $K$ and $K'$ comes from a map $K \to K'$ in $D(\mathcal{O})$.
\end{lemma}

\begin{proof}
The proof is exactly the same as the proof of
Lemma \ref{lemma-hom-cartesian-objects-derived} except using
Lemma \ref{lemma-modules-postnikov} instead of
Lemma \ref{lemma-abelian-postnikov}.
\end{proof}

\begin{lemma}
\label{lemma-cartesian-module-derived-from-simplicial}
In Situation \ref{situation-simplicial-site} let $\mathcal{O}$ be
a sheaf of rings on $\mathcal{C}_{total}$. Let
$(K_n, K_\varphi)$ be a simplicial system of the derived category
of modules. Assume
\begin{enumerate}
\item $f_\varphi^{-1}\mathcal{O}_n \to \mathcal{O}_m$ is flat for
$\varphi : [m] \to [n]$,
\item $(K_n, K_\varphi)$ is cartesian,
\item $\Hom(K_i[t], K_i) = 0$ for $i \geq 0$ and $t > 0$.
\end{enumerate}
Then there exists a cartesian object $K$ of $D(\mathcal{O})$
whose associated simplicial system is isomorphic to $(K_n, K_\varphi)$.
\end{lemma}

\begin{proof}
The proof is exactly the same as the proof of
Lemma \ref{lemma-cartesian-object-derived-from-simplicial}
with the following changes
\begin{enumerate}
\item use $g_n^* = Lg_n^*$ everywhere instead of $g_n^{-1}$,
\item use $f_\varphi^* = Lf_\varphi^*$ everywhere instead of $f_\varphi^{-1}$,
\item refer to Lemma \ref{lemma-simplicial-resolution-ringed}
instead of Lemma \ref{lemma-simplicial-resolution-Z-site},
\item in the construction of $Y'_{m, n}$ use
$\mathcal{O}_m$ instead of $\mathbf{Z}$,
\item compare with the proof of Lemma \ref{lemma-modules-postnikov}
rather than the proof of Lemma \ref{lemma-abelian-postnikov}.
\end{enumerate}
This ends the proof.
\end{proof}







\section{The site associated to a semi-representable object}
\label{section-semi-representable}

\noindent
Let $\mathcal{C}$ be a site. Recall that a {\it semi-representable object}
of $\mathcal{C}$ is simply a family $\{U_i\}_{i \in I}$
of objects of $\mathcal{C}$. A
{\it morphism $\{U_i\}_{i \in I} \to \{V_j\}_{j \in J}$ of
semi-representable objects} is given by a map $\alpha : I \to J$
and for every $i \in I$ a morphism $f_i : U_i \to V_{\alpha(i)}$
of $\mathcal{C}$.
The category of semi-representable objects of $\mathcal{C}$
is denoted $\text{SR}(\mathcal{C})$.
See Hypercoverings, Definition \ref{hypercovering-definition-SR}
and the enclosing section for more information.

\medskip\noindent
For a semi-representable object $K = \{U_i\}_{i \in I}$ of $\mathcal{C}$
we let
$$
\mathcal{C}/K = \coprod\nolimits_{i \in I} \mathcal{C}/U_i
$$
be the disjoint union of the localizations of $\mathcal{C}$ at $U_i$.
There is a natural structure of a site on this category, with
coverings inherited from the localizations $\mathcal{C}/U_i$.
The site $\mathcal{C}/K$ is called the
{\it localization of $\mathcal{C}$ at $K$}.
Observe that a sheaf on $\mathcal{C}/K$ is the same thing as
a family of sheaves $\mathcal{F}_i$ on $\mathcal{C}/U_i$, i.e.,
$$
\Sh(\mathcal{C}/K) = \prod\nolimits_{i \in I} \Sh(\mathcal{C}/U_i)
$$
This is occasionally useful to understand what is going on.

\medskip\noindent
Let $\mathcal{C}$ be a site. Let $K = \{U_i\}_{i \in I}$ be an object of
$\text{SR}(\mathcal{C})$. There is a continuous and cocontinuous
localization functor $j : \mathcal{C}/K \to \mathcal{C}$ which is
the product of the localization functors
$j_i : \mathcal{C}/V_i \to \mathcal{C}$.
We obtain functors $j_!$, $j^{-1}$, $j_*$ exactly
as in Sites, Section \ref{sites-section-localize}.
In terms of the product decomposition
$\Sh(\mathcal{C}/K) = \prod\nolimits_{i \in I} \Sh(\mathcal{C}/U_i)$
we have
$$
\begin{matrix}
j_! & : &
(\mathcal{F}_i)_{i \in I} &
\longmapsto &
\coprod j_{i, !}\mathcal{F}_i \\
j^{-1} & : &
\mathcal{G} &
\longmapsto &
(j_i^{-1}\mathcal{G})_{i \in I} \\
j_* & : &
(\mathcal{F}_i)_{i \in I} &
\longmapsto &
\prod j_{i, *}\mathcal{F}_i
\end{matrix}
$$
as the reader easily verifies.

\medskip\noindent
Let $f : K \to L$ be a morphism of $\text{SR}(\mathcal{C})$.
Then we obtain a continuous and cocontinuous functor
$$
v : \mathcal{C}/K \longrightarrow \mathcal{C}/L
$$
by applying the construction of Sites, Lemma \ref{sites-lemma-relocalize}
to the components. More precisely, suppose $f = (\alpha, f_i)$
where $K = \{U_i\}_{i \in I}$, $L = \{V_j\}_{j \in J}$, $\alpha : I \to J$,
and $f_i : U_i \to V_{\alpha(i)}$. Then the functor $v$ maps the component
$\mathcal{C}/U_i$ into the component $\mathcal{C}/V_{\alpha(i)}$
via the construction of the aforementioned lemma. In particular
we obtain a morphism
$$
f : \Sh(\mathcal{C}/K) \to \Sh(\mathcal{C}/L)
$$
of topoi. In terms of the product decompositions
$\Sh(\mathcal{C}/K) = \prod\nolimits_{i \in I} \Sh(\mathcal{C}/U_i)$ and
$\Sh(\mathcal{C}/L) = \prod\nolimits_{j \in J} \Sh(\mathcal{C}/V_j)$
the reader verifies that
$$
\begin{matrix}
f_! & : &
(\mathcal{F}_i)_{i \in I} &
\longmapsto &
(\coprod\nolimits_{i \in I, \alpha(i) = j} f_{i, !}\mathcal{F}_i)_{j \in J} \\
f^{-1} & : &
(\mathcal{G}_j)_{j \in J} &
\longmapsto &
(f_i^{-1}\mathcal{G}_{\alpha(i)})_{i \in I} \\
f_* & : &
(\mathcal{F}_i)_{i \in I} &
\longmapsto &
(\prod\nolimits_{i \in I, \alpha(i) = j} f_{i, *}\mathcal{F}_i)_{j \in J}
\end{matrix}
$$
where $f_i : \Sh(\mathcal{C}/U_i) \to \Sh(\mathcal{C}/V_{\alpha(i)})$
is the morphism associated to the localization functor
$\mathcal{C}/U_i \to \mathcal{C}/V_{\alpha(i)}$ corresponding to
$f_i : U_i \to V_{\alpha(i)}$.

\begin{lemma}
\label{lemma-has-P}
Let $\mathcal{C}$ be a site.
\begin{enumerate}
\item For $K$ in $\text{SR}(\mathcal{C})$ the functor
$j : \mathcal{C}/K \to \mathcal{C}$ is continuous,
cocontinuous, and has property P of
Sites, Remark \ref{sites-remark-cartesian-cocontinuous}.
\item For $f : K \to L$ in $\text{SR}(\mathcal{C})$
the functor $v : \mathcal{C}/K \to \mathcal{C}/L$ (see above)
is continuous, cocontinuous, and has property P of
Sites, Remark \ref{sites-remark-cartesian-cocontinuous}.
\end{enumerate}
\end{lemma}

\begin{proof}
Proof of (2). In the notation of the discussion preceding the lemma,
the localization functors $\mathcal{C}/U_i \to \mathcal{C}/V_{\alpha(i)}$
are continuous and cocontinuous by
Sites, Section \ref{sites-section-localize}
and satisfy $P$ by
Sites, Remark \ref{sites-remark-localization-cartesian-cocontinuous}.
It is formal to deduce $v$ is continuous and cocontinuous and has $P$.
We omit the details. We also omit the proof of (1).
\end{proof}

\begin{lemma}
\label{lemma-push-pull-localization}
Let $\mathcal{C}$ be a site and $K$ in $\text{SR}(\mathcal{C})$.
For $\mathcal{F}$ in $\Sh(\mathcal{C})$ we have
$$
j_*j^{-1}\mathcal{F} = \SheafHom(F(K)^\#, \mathcal{F})
$$
where $F$ is as in
Hypercoverings, Definition \ref{hypercovering-definition-SR-F}.
\end{lemma}

\begin{proof}
Say $K = \{U_i\}_{i \in I}$.
Using the description of the functors $j^{-1}$ and $j_*$
given above we see that
$$
j_*j^{-1}\mathcal{F} =
\prod\nolimits_{i \in I} j_{i, *}(\mathcal{F}|_{\mathcal{C}/U_i}) =
\prod\nolimits_{i \in I} \SheafHom(h_{U_i}^\#, \mathcal{F})
$$
The second equality by Sites, Lemma \ref{sites-lemma-hom-sheaf-hU}.
Since $F(K) = \coprod h_{U_i}$ in $\textit{PSh}(\mathcal{C}$,
we have $F(K)^\# = \coprod h_{U_i}^\#$ in $\Sh(\mathcal{C})$
and since $\SheafHom(-, \mathcal{F})$ turns coproducts into
products (immediate from the construction in
Sites, Section \ref{sites-section-glueing-sheaves}), we conclude.
\end{proof}

\begin{lemma}
\label{lemma-localize-compare}
Let $\mathcal{C}$ be a site.
\begin{enumerate}
\item For $K$ in $\text{SR}(\mathcal{C})$ the functor $j_!$
gives an equivalence $\Sh(\mathcal{C}/K) \to \Sh(\mathcal{C})/F(K)^\#$
where $F$ is as in
Hypercoverings, Definition \ref{hypercovering-definition-SR-F}.
\item The functor $j^{-1} : \Sh(\mathcal{C}) \to \Sh(\mathcal{C}/K)$
corresponds via the identification of (1) with
$\mathcal{F} \mapsto (\mathcal{F} \times F(K)^\# \to F(K)^\#)$.
\item For $f : K \to L$ in $\text{SR}(\mathcal{C})$ the functor
$f^{-1}$ corresponds via the identifications of (1) to the functor
$\Sh(\mathcal{C})/F(L)^\# \to \Sh(\mathcal{C})/F(K)^\#$,
$(\mathcal{G} \to F(L)^\#) \mapsto
(\mathcal{G} \times_{F(L)^\#} F(K)^\# \to F(K)^\#)$.
\end{enumerate}
\end{lemma}

\begin{proof}
Observe that if $K = \{U_i\}_{i \in I}$ then the category
$\Sh(\mathcal{C}/K)$ decomposes as the product of the categories
$\Sh(\mathcal{C}/U_i)$. Observe that
$F(K)^\# = \coprod_{i \in I} h_{U_i}^\#$ (coproduct in sheaves).
Hence $\Sh(\mathcal{C})/F(K)^\#$ is the product of the
categories $\Sh(\mathcal{C})/h_{U_i}^\#$.
Thus (1) and (2) follow from the corresponding
statements for each $i$, see
Sites, Lemmas \ref{sites-lemma-essential-image-j-shriek} and
\ref{sites-lemma-compute-j-shriek-restrict}.
Similarly, if $L = \{V_j\}_{j \in J}$ and $f$ is given
by $\alpha : I \to J$ and $f_i : U_i \to V_{\alpha(i)}$,
then we can apply
Sites, Lemma \ref{sites-lemma-relocalize-explicit}
to each of the re-localization morphisms
$\mathcal{C}/U_i \to \mathcal{C}/V_{\alpha(i)}$
to get (3).
\end{proof}

\begin{lemma}
\label{lemma-localize-injective}
Let $\mathcal{C}$ be a site. For $K$ in $\text{SR}(\mathcal{C})$
the functor $j^{-1}$ sends injective abelian sheaves to injective
abelian sheaves. Similarly, the functor $j^{-1}$ sends K-injective
complexes of abelian sheaves to K-injective complexes of
abelian sheaves.
\end{lemma}

\begin{proof}
The first statement is the natural generalization of
Cohomology on Sites, Lemma
\ref{sites-cohomology-lemma-cohomology-of-open}
to semi-representable objects.
In fact, it follows from this lemma
by the product decomposition of $\Sh(\mathcal{C}/K)$
and the description of the functor $j^{-1}$ given above.
The second statement is the natural generalization of
Cohomology on Sites, Lemma
\ref{sites-cohomology-lemma-restrict-K-injective-to-open}
and follows from it by the product decomposition of the topos.

\medskip\noindent
Alternative: since $j$ induces a localization of topoi by
Lemma \ref{lemma-localize-compare} part (1)
it also follows immediately from
Cohomology on Sites, Lemmas \ref{sites-cohomology-lemma-cohomology-of-open}
and \ref{sites-cohomology-lemma-restrict-K-injective-to-open}
by enlarging the site; compare with the proof of
Cohomology on Sites, Lemma
\ref{sites-cohomology-lemma-cohomology-on-sheaf-sets}
in the case of injective sheaves.
\end{proof}

\begin{remark}[Variant for over an object]
\label{remark-semi-representable-over-object}
Let $\mathcal{C}$ be a site. Let $X \in \Ob(\mathcal{C})$.
The category $\text{SR}(\mathcal{C}, X)$
of {\it semi-representable objects over $X$}
is defined by the formula
$\text{SR}(\mathcal{C}, X) = \text{SR}(\mathcal{C}/X)$.
See Hypercoverings, Definition \ref{hypercovering-definition-SR}.
Thus we may apply the above discussion to the site
$\mathcal{C}/X$. Briefly, the constructions above give
\begin{enumerate}
\item a site $\mathcal{C}/K$ for $K$ in $\text{SR}(\mathcal{C}, X)$,
\item a decomposition
$\Sh(\mathcal{C}/K) = \prod \Sh(\mathcal{C}/U_i)$ if $K = \{U_i/X\}$,
\item a localization functor $j : \mathcal{C}/K \to \mathcal{C}/X$,
\item a morphism $f : \Sh(\mathcal{C}/K) \to \Sh(\mathcal{C}/L)$
for $f : K \to L$ in $\text{SR}(\mathcal{C}, X)$.
\end{enumerate}
All results of this section hold in this situation by replacing
$\mathcal{C}$ everywhere by $\mathcal{C}/X$.
\end{remark}

\begin{remark}[Ringed variant]
\label{remark-semi-representable-ringed}
Let $\mathcal{C}$ be a site. Let $\mathcal{O}_\mathcal{C}$
be a sheaf of rings on $\mathcal{C}$. In this case, for any
semi-representable object $K$ of $\mathcal{C}$ the site
$\mathcal{C}/K$ is a ringed site with sheaf
of rings $\mathcal{O}_K = j^{-1}\mathcal{O}_\mathcal{C}$.
The constructions above give
\begin{enumerate}
\item a ringed site $(\mathcal{C}/K, \mathcal{O}_K)$
for $K$ in $\text{SR}(\mathcal{C})$,
\item a decomposition
$\textit{Mod}(\mathcal{O}_K) =
\prod \textit{Mod}(\mathcal{O}_{U_i})$ if $K = \{U_i\}$,
\item a localization morphism
$j : (\Sh(\mathcal{C}/K), \mathcal{O}_K) \to
(\Sh(\mathcal{C}), \mathcal{O}_\mathcal{C})$
of ringed topoi,
\item a morphism
$f : (\Sh(\mathcal{C}/K), \mathcal{O}_K) \to
(\Sh(\mathcal{C}/L), \mathcal{O}_L)$ of ringed topoi
for $f : K \to L$ in $\text{SR}(\mathcal{C})$.
\end{enumerate}
Many of the results above hold in this setting. For example, the
functor $j^*$ has an exact left adjoint
$$
j_! : \textit{Mod}(\mathcal{O}_K) \to \textit{Mod}(\mathcal{O}_\mathcal{C}),
$$
which in terms of the product decomposition given in (2) sends
$(\mathcal{F}_i)_{i \in I}$ to $\bigoplus j_{i, !}\mathcal{F}_i$.
Similarly, given $f : K \to L$ as above, the functor $f^*$ has
an exact left adjoint
$f_! : \textit{Mod}(\mathcal{O}_K) \to \textit{Mod}(\mathcal{O}_L)$.
Thus the functors $j^*$ and $f^*$ are exact, i.e.,
$j$ and $f$ are flat morphisms of ringed topoi (also follows
from the equalities $\mathcal{O}_K = j^{-1}\mathcal{O}_\mathcal{C}$
and $\mathcal{O}_K = f^{-1}\mathcal{O}_L$).
\end{remark}

\begin{remark}[Ringed variant over an object]
\label{remark-semi-representable-ringed-over-object}
Let $\mathcal{C}$ be a site. Let $\mathcal{O}_\mathcal{C}$
be a sheaf of rings on $\mathcal{C}$. Let $X \in \Ob(\mathcal{C})$
and denote $\mathcal{O}_X = \mathcal{O}_\mathcal{C}|_{\mathcal{C}/U}$.
Then we can combine the constructions given in
Remarks \ref{remark-semi-representable-over-object}
and \ref{remark-semi-representable-ringed} to get
\begin{enumerate}
\item a ringed site $(\mathcal{C}/K, \mathcal{O}_K)$
for $K$ in $\text{SR}(\mathcal{C}, X)$,
\item a decomposition
$\textit{Mod}(\mathcal{O}_K) =
\prod \textit{Mod}(\mathcal{O}_{U_i})$ if $K = \{U_i\}$,
\item a localization morphism
$j : (\Sh(\mathcal{C}/K), \mathcal{O}_K) \to
(\Sh(\mathcal{C}/X), \mathcal{O}_X)$
of ringed topoi,
\item a morphism
$f : (\Sh(\mathcal{C}/K), \mathcal{O}_K) \to
(\Sh(\mathcal{C}/L), \mathcal{O}_L)$ of ringed topoi
for $f : K \to L$ in $\text{SR}(\mathcal{C}, X)$.
\end{enumerate}
Of course all of the results mentioned in
Remark \ref{remark-semi-representable-ringed}
hold in this setting as well.
\end{remark}





\section{The site associate to a simplicial semi-representable object}
\label{section-simplicial-semi-representable}

\noindent
Let $\mathcal{C}$ be a site. Let $K$ be a simplicial object of
$\text{SR}(\mathcal{C})$. As usual, set $K_n = K([n])$ and denote
$K(\varphi) : K_n \to K_m$ the morphism associated to $\varphi : [m] \to [n]$.
By the construction in
Section \ref{section-semi-representable} we obtain a simplicial object
$n \mapsto \mathcal{C}/K_n$ in the category whose objects are sites and
whose morphisms are cocontinuous functors. In other words, we get
a gadget as in Case B of Section \ref{section-simplicial-sites}.
The functors satisfy property P by Lemma \ref{lemma-has-P}.
Hence we may apply
Lemma \ref{lemma-simplicial-cocontinuous-site}
to obtain a site $(\mathcal{C}/K)_{total}$.

\medskip\noindent
We can describe the site $(\mathcal{C}/K)_{total}$ explicitly as follows.
Say $K_n = \{U_{n, i}\}_{i \in I_n}$. For $\varphi : [m] \to [n]$
the morphism $K(\varphi) : K_n \to K_m$ is given by a map
$\alpha(\varphi) : I_n \to I_m$ and morphisms
$f_{\varphi, i} : U_{n, i} \to U_{m, \alpha(\varphi)(i)}$ for $i \in I_n$.
Then we have
\begin{enumerate}
\item an object of $(\mathcal{C}/K)_{total}$
corresponds to an object $(U/U_{n, i})$
of $\mathcal{C}/U_{n, i}$ for some $n$ and some $i \in I_n$,
\item a morphism between $U/U_{n, i}$ and $V/U_{m, j}$
is a pair $(\varphi, f)$ where $\varphi : [m] \to [n]$,
$j = \alpha(\varphi)(i)$, and $f : U \to V$ is a morphism of
$\mathcal{C}$ such that
$$
\vcenter{
\xymatrix{
U \ar[r]_f \ar[d] & V \ar[d] \\
U_{n, i} \ar[r]^-{f_{\varphi, i}} &
U_{m, j}
}
}
$$
is commutative, and
\item coverings of the object $U/U_{n, i}$ are constructed
by starting with a covering $\{f_j : U_j \to U\}$ in $\mathcal{C}$
and letting $\{(\text{id}, f_j) : U_j/U_{n, i} \to U/U_{n, i}\}$
be a covering in $(\mathcal{C}/K)_{total}$.
\end{enumerate}
All of our general theory developed for simplicial sites applies to
$(\mathcal{C}/K)_{total}$. Observe that the obvious forgetful functor
$$
j_{total} : (\mathcal{C}/K)_{total} \longrightarrow \mathcal{C}
$$
is continuous and cocontinuous. It turns out that the associated
morphism of topoi comes from an (obvious) augmentation.

\begin{lemma}
\label{lemma-augmentation-simplicial-semi-representable}
Let $\mathcal{C}$ be a site. Let $K$ be a simplicial object of
$\text{SR}(\mathcal{C})$. The localization functor
$j_0 : \mathcal{C}/K_0 \to \mathcal{C}$ defines an augmentation
$a_0 : \Sh(\mathcal{C}/K_0) \to \Sh(\mathcal{C})$, as in case (B) of
Remark \ref{remark-augmentation-site}.
The corresponding morphisms of topoi
$$
a_n : \Sh(\mathcal{C}/K_n) \longrightarrow \Sh(\mathcal{C}),\quad
a : \Sh((\mathcal{C}/K)_{total}) \longrightarrow \Sh(\mathcal{C})
$$
of Lemma \ref{lemma-augmentation-site}
are equal to the morphisms of topoi associated to the
continuous and cocontinuous localization functors
$j_n : \mathcal{C}/K_n \to \mathcal{C}$ and
$j_{total} : (\mathcal{C}/K)_{total} \to \mathcal{C}$.
\end{lemma}

\begin{proof}
This is immediate from working through the definitions.
See in particular the footnote in the proof of
Lemma \ref{lemma-augmentation-site}
for the relationship between $a$ and $j_{total}$.
\end{proof}

\begin{lemma}
\label{lemma-comparison}
With assumption and notation as in
Lemma \ref{lemma-augmentation-simplicial-semi-representable}
we have the following properties:
\begin{enumerate}
\item there is a functor
$a^{Sh}_! : \Sh((\mathcal{C}/K)_{total}) \to \Sh(\mathcal{C})$
left adjoint to $a^{-1} : \Sh(\mathcal{C}) \to \Sh((\mathcal{C}/K)_{total})$,
\item there is a functor
$a_! : \textit{Ab}((\mathcal{C}/K)_{total}) \to \textit{Ab}(\mathcal{C})$
left adjoint to
$a^{-1} : \textit{Ab}(\mathcal{C}) \to \textit{Ab}((\mathcal{C}/K)_{total})$,
\item the functor $a^{-1}$ associates to
$\mathcal{F}$ in $\Sh(\mathcal{C})$ the sheaf on $(\mathcal{C}/K)_{total}$
wich in degree $n$ is equal to $a_n^{-1}\mathcal{F}$,
\item the functor $a_*$ associates to $\mathcal{G}$ in
$\textit{Ab}((\mathcal{C}/K)_{total})$ the equalizer of the two maps
$j_{0, *}\mathcal{G}_0 \to j_{1, *}\mathcal{G}_1$,
\end{enumerate}
\end{lemma}

\begin{proof}
Parts (3) and (4) hold for any augmentation of a
simplicial site, see Lemma \ref{lemma-augmentation-site}.
Parts (1) and (2) follow as $j_{total}$ is continuous and cocontinuous.
The functor $a^{Sh}_!$ is constructed in
Sites, Lemma \ref{sites-lemma-when-shriek}
and the functor $a_!$ is constructed in
Modules on Sites, Lemma
\ref{sites-modules-lemma-g-shriek-adjoint}.
\end{proof}

\begin{lemma}
\label{lemma-sanity-check-simplicial-semi-representable}
Let $\mathcal{C}$ be a site. Let $K$ be a simplicial object of
$\text{SR}(\mathcal{C})$. Let $U/U_{n, i}$ be an object of
$\mathcal{C}/K_n$. Let
$\mathcal{F} \in \textit{Ab}((\mathcal{C}/K)_{total})$.
Then
$$
H^p(U, \mathcal{F}) = H^p(U, \mathcal{F}_{n, i})
$$
where
\begin{enumerate}
\item on the left hand side $U$ is viewed as an object of
$\mathcal{C}_{total}$, and
\item on the right hand side $\mathcal{F}_{n, i}$ is the $i$th
component of the sheaf $\mathcal{F}_n$ on $\mathcal{C}/K_n$
in the decomposition $\Sh(\mathcal{C}/K_n) = \prod \Sh(\mathcal{C}/U_{n, i})$
of Section \ref{section-semi-representable}.
\end{enumerate}
\end{lemma}

\begin{proof}
This follows immediately from Lemma \ref{lemma-sanity-check}
and the product decompositions of Section \ref{section-semi-representable}.
\end{proof}

\begin{remark}[Variant for over an object]
\label{remark-augmentation-over-object}
Let $\mathcal{C}$ be a site. Let $X \in \Ob(\mathcal{C})$.
Recall that we have a category
$\text{SR}(\mathcal{C}, X) = \text{SR}(\mathcal{C}/X)$
of semi-representable objects over $X$,
see Remark \ref{remark-semi-representable-over-object}.
We may apply the above discussion to the site
$\mathcal{C}/X$. Briefly, the constructions above give
\begin{enumerate}
\item a site $(\mathcal{C}/K)_{total}$ for a simplicial $K$ object
of $\text{SR}(\mathcal{C}, X)$,
\item a localization functor
$j_{total} : (\mathcal{C}/K)_{total} \to \mathcal{C}/X$,
\item localization functors $j_n : \mathcal{C}/K_n \to \mathcal{C}/X$,
\item a morphism of topoi
$a : \Sh((\mathcal{C}/K)_{total}) \to \Sh(\mathcal{C}/X)$,
\item morphisms of topoi
$a_n : \Sh(\mathcal{C}/K_n) \to \Sh(\mathcal{C}/X)$,
\item a functor
$a^{Sh}_! : \Sh((\mathcal{C}/K)_{total}) \to \Sh(\mathcal{C}/X)$
left adjoint to $a^{-1}$, and
\item a functor
$a_! : \textit{Ab}((\mathcal{C}/K)_{total}) \to \textit{Ab}(\mathcal{C}/X)$
left adjoint to $a^{-1}$.
\end{enumerate}
All of the results of this section hold in this setting.
To prove this one replaces
the site $\mathcal{C}$ everywhere by $\mathcal{C}/X$.
\end{remark}

\begin{remark}[Ringed variant]
\label{remark-augmentation-ringed}
Let $\mathcal{C}$ be a site. Let $\mathcal{O}_\mathcal{C}$ be a sheaf of rings.
Given a simplicial semi-representable object $K$ of $\mathcal{C}$
we set $\mathcal{O} = a^{-1}\mathcal{O}_\mathcal{C}$, where $a$
is as in Lemmas \ref{lemma-augmentation-simplicial-semi-representable} and
\ref{lemma-comparison}.
The constructions above, keeping track of the sheaves of rings
as in Remark \ref{remark-semi-representable-ringed}, give
\begin{enumerate}
\item a ringed site $((\mathcal{C}/K)_{total}, \mathcal{O})$
for a simplicial $K$ object of $\text{SR}(\mathcal{C})$,
\item a morphism of ringed topoi
$a : (\Sh((\mathcal{C}/K)_{total}), \mathcal{O}) \to
(\Sh(\mathcal{C}), \mathcal{O}_\mathcal{C})$,
\item morphisms of ringed topoi
$a_n : (\Sh(\mathcal{C}/K_n), \mathcal{O}_n) \to
(\Sh(\mathcal{C}), \mathcal{O}_\mathcal{C})$,
\item a functor
$a_! : \textit{Mod}(\mathcal{O}) \to \textit{Mod}(\mathcal{O}_\mathcal{C})$
left adjoint to $a^*$.
\end{enumerate}
The functor $a_!$ exists (but in general is not exact)
because $a^{-1}\mathcal{O}_\mathcal{C} = \mathcal{O}$
and we can replace the use of
Modules on Sites, Lemma \ref{sites-modules-lemma-g-shriek-adjoint}
in the proof of Lemma \ref{lemma-comparison}
by Modules on Sites, Lemma \ref{sites-modules-lemma-lower-shriek-modules}.
As discussed in Remark \ref{remark-semi-representable-ringed}
there are exact functors
$a_{n!} : \textit{Mod}(\mathcal{O}_n) \to
\textit{Mod}(\mathcal{O}_\mathcal{C})$
left adjoint to $a_n^*$. Consequently, the morphisms $a$ and $a_n$ are flat.
Remark \ref{remark-semi-representable-ringed}
implies the morphism of ringed topoi
$f_\varphi : (\Sh(\mathcal{C}/K_n), \mathcal{O}_n) \to
(\Sh(\mathcal{C}/K_m), \mathcal{O}_m)$
for $\varphi : [m] \to [n]$ is flat and there exists an exact functor
$f_{\varphi !} : \textit{Mod}(\mathcal{O}_n) \to \textit{Mod}(\mathcal{O}_m)$
left adjoint to $f_\varphi^*$. This in turn implies that for
the flat morphism of ringed topoi
$g_n : (\Sh(\mathcal{C}/K_n), \mathcal{O}_n) \to
(\Sh((\mathcal{C}/K)_{total}), \mathcal{O})$
the functor $g_{n!} : \textit{Mod}(\mathcal{O}_n) \to
\textit{Mod}(\mathcal{O})$ left adjoint to $g_n^*$ is exact, see
Lemma \ref{lemma-exactness-g-shriek-modules}.
\end{remark}

\begin{remark}[Ringed variant over an object]
\label{remark-augmentation-ringed-over-object}
Let $\mathcal{C}$ be a site. Let $\mathcal{O}_\mathcal{C}$ be a sheaf of rings.
Let $X \in \Ob(\mathcal{C})$ and denote
$\mathcal{O}_X = \mathcal{O}_\mathcal{C}|_{\mathcal{C}/X}$.
Then we can combine the constructions given in
Remarks \ref{remark-augmentation-over-object} and
\ref{remark-augmentation-ringed}
to get
\begin{enumerate}
\item a ringed site $((\mathcal{C}/K)_{total}, \mathcal{O})$
for a simplicial $K$ object of $\text{SR}(\mathcal{C}, X)$,
\item a morphism of ringed topoi
$a : (\Sh((\mathcal{C}/K)_{total}), \mathcal{O}) \to
(\Sh(\mathcal{C}/X), \mathcal{O}_X)$,
\item morphisms of ringed topoi
$a_n : (\Sh(\mathcal{C}/K_n), \mathcal{O}_n) \to
(\Sh(\mathcal{C}/X), \mathcal{O}_X)$,
\item a functor
$a_! : \textit{Mod}(\mathcal{O}) \to \textit{Mod}(\mathcal{O}_X)$
left adjoint to $a^*$.
\end{enumerate}
Of course, all the results mentioned in
Remark \ref{remark-augmentation-ringed}
hold in this setting as well.
\end{remark}






\section{Cohomological descent for hypercoverings}
\label{section-cohomological-descent-hypercoverings}

\noindent
Let $\mathcal{C}$ be a site. In this section we assume $\mathcal{C}$
has equalizers and fibre products. We let $K$ be a hypercovering
as defined in Hypercoverings, Definition
\ref{hypercovering-definition-hypercovering-variant}. We will study
the augmentation
$$
a : \Sh((\mathcal{C}/K)_{total}) \longrightarrow \Sh(\mathcal{C})
$$
of Section \ref{section-simplicial-semi-representable}.

\begin{lemma}
\label{lemma-hypercovering-descent-sheaves}
Let $\mathcal{C}$ be a site with equalizers and fibre products.
Let $K$ be a hypercovering. Then
\begin{enumerate}
\item $a^{-1} : \Sh(\mathcal{C}) \to \Sh((\mathcal{C}/K)_{total})$
is fully faithful with essential image the cartesian sheaves of sets,
\item $a^{-1} : \textit{Ab}(\mathcal{C}) \to
\textit{Ab}((\mathcal{C}/K)_{total})$
is fully faithful with essential image the cartesian sheaves
of abelian groups.
\end{enumerate}
In both cases $a_*$ provides the quasi-inverse functor.
\end{lemma}

\begin{proof}
The case of abelian sheaves follows immediately from the case
of sheaves of sets as the functor $a^{-1}$ commutes with products.
In the rest of the proof we work with sheaves of sets.
Observe that $a^{-1}\mathcal{F}$ is cartesian for
$\mathcal{F}$ in $\Sh(\mathcal{C})$ by
Lemma \ref{lemma-augmentation-cartesian-module}.
It suffices to show that the adjunction map
$\mathcal{F} \to a_*a^{-1}\mathcal{F}$
is an isomorphism $\mathcal{F}$ in $\Sh(\mathcal{C})$
and that for a cartesian sheaf
$\mathcal{G}$ on $(\mathcal{C}/K)_{total}$
the adjunction map
$a^{-1}a_*\mathcal{G} \to \mathcal{G}$ is an isomorphism.

\medskip\noindent
Let $\mathcal{F}$ be a sheaf on $\mathcal{C}$.
Recall that $a_*a^{-1}\mathcal{F}$ is the equalizer
of the two maps $a_{0, *}a_0^{-1}\mathcal{F} \to a_{1, *}a_1^{-1}\mathcal{F}$,
see Lemma \ref{lemma-comparison}.
By Lemma \ref{lemma-push-pull-localization}
$$
a_{0, *}a_0^{-1}\mathcal{F} = \SheafHom(F(K_0)^\#, \mathcal{F})
\quad\text{and}\quad
a_{1, *}a_1^{-1}\mathcal{F} = \SheafHom(F(K_1)^\#, \mathcal{F})
$$
On the other hand, we know that
$$
\xymatrix{
F(K_1)^\# \ar@<1ex>[r] \ar@<-1ex>[r] &
F(K_0)^\# \ar[r] & \text{final object }*\text{ of }\Sh(\mathcal{C})
}
$$
is a coequalizer diagram in sheaves of sets by definition of
a hypercovering. Thus it suffices to prove
that $\SheafHom(-, \mathcal{F})$ transforms coequalizers
into equalizers which is immediate from the construction
in Sites, Section \ref{sites-section-glueing-sheaves}.

\medskip\noindent
Let $\mathcal{G}$ be a cartesian sheaf on $(\mathcal{C}/K)_{total}$.
We will show that $\mathcal{G} = a^{-1}\mathcal{F}$ for some sheaf
$\mathcal{F}$ on $\mathcal{C}$. This will finish the proof because
then $a^{-1}a_*\mathcal{G} = a^{-1}a_*a^{-1}\mathcal{F} =
a^{-1}\mathcal{F} = \mathcal{G}$ by the result of the previous paragraph.
Set $\mathcal{K}_n = F(K_n)^\#$ for $n \geq 0$. Then we have maps of sheaves
$$
\xymatrix{
\mathcal{K}_2
\ar@<1ex>[r]
\ar@<0ex>[r]
\ar@<-1ex>[r]
&
\mathcal{K}_1
\ar@<0.5ex>[r]
\ar@<-0.5ex>[r]
&
\mathcal{K}_0
}
$$
coming from the fact that $K$ is a simplicial semi-representable object.
The fact that $K$ is a hypercovering means that
$$
\mathcal{K}_1 \to \mathcal{K}_0 \times \mathcal{K}_0
\quad\text{and}\quad
\mathcal{K}_2 \to
\left(\text{cosq}_1(
\xymatrix{
\mathcal{K}_1
\ar@<0.5ex>[r]
\ar@<-0.5ex>[r]
&
\mathcal{K}_0 \ar[l]
})\right)_2
$$
are surjective maps of sheaves. Using the description of cartesian sheaves on
$(\mathcal{C}/K)_{total}$ given in Lemma \ref{lemma-characterize-cartesian}
and using the description of $\Sh(\mathcal{C}/K_n)$ in
Lemma \ref{lemma-localize-compare}
we find that our problem can be entirely formulated\footnote{Even though it
does not matter what the precise formulation is, we spell it out:
the problem is to show that given an object
$\mathcal{G}_0/\mathcal{K}_0$ of $\Sh(\mathcal{C})/\mathcal{K}_0$
and an isomorphism
$$
\alpha :
\mathcal{G}_0 \times_{\mathcal{K}_0, \mathcal{K}(\delta^1_1)} \mathcal{K}_1 \to
\mathcal{G}_0 \times_{\mathcal{K}_0, \mathcal{K}(\delta^1_0)} \mathcal{K}_1
$$
over $\mathcal{K}_1$ satisfying a cocycle condtion in
$\Sh(\mathcal{C})/\mathcal{K}_2$, there exists
$\mathcal{F}$ in $\Sh(\mathcal{C})$ and an isomorphism
$\mathcal{F} \times \mathcal{K}_0 \to \mathcal{G}_0$ over $\mathcal{K}_0$
compatible with $\alpha$.} in terms of
\begin{enumerate}
\item the topos $\Sh(\mathcal{C})$, and
\item the simplicial object $\mathcal{K}$ in $\Sh(\mathcal{C})$
whose terms are $\mathcal{K}_n$.
\end{enumerate}
Thus, after replacing $\mathcal{C}$ by a different site $\mathcal{C}'$
as in Sites, Lemma \ref{sites-lemma-topos-good-site}, we may assume
$\mathcal{C}$ has all finite limits,
the topology on $\mathcal{C}$ is subcanonical,
a family $\{V_j \to V\}$ of morphisms of $\mathcal{C}$
is a covering if and only if $\coprod h_{V_j} \to V$ is surjective, and
there exists a simplicial object $U$ of $\mathcal{C}$
such that $\mathcal{K}_n = h_{U_n}$ as simplicial sheaves.
Working backwards through the equivalences we may assume
$K_n = \{U_n\}$ for all $n$.

\medskip\noindent
Let $X$ be the final object of $\mathcal{C}$.
Then $\{U_0 \to X\}$ is a covering,
$\{U_1 \to U_0 \times U_0\}$ is a covering, and
$\{U_2 \to (\text{cosq}_1 \text{sk}_1 U)_2\}$ is a covering.
Let us use $d^n_i : U_n \to U_{n - 1}$ and
$s^n_j : U_n \to U_{n + 1}$ the morphisms corresponding
to $\delta^n_i$ and $\sigma^n_j$ as in
Simplicial, Definition \ref{simplicial-definition-face-degeneracy}.
By abuse of notation, given a morphism
$c : V \to W$ of $\mathcal{C}$ we denote the morphism of topoi
$c : \Sh(\mathcal{C}/V) \to \Sh(\mathcal{C}/W)$ by the same letter.
Now $\mathcal{G}$ is given by a sheaf $\mathcal{G}_0$
on $\mathcal{C}/U_0$ and an isomorphism
$\alpha : (d^1_1)^{-1}\mathcal{G}_0 \to (d^1_0)^{-1}\mathcal{G}_0$
satisfying the cocycle condition on $\mathcal{C}/U_2$
formulated in Lemma \ref{lemma-characterize-cartesian}.
Since $\{U_2 \to (\text{cosq}_1 \text{sk}_1 U)_2\}$
is a covering, the corresponding pullback functor
on sheaves is faithful (small detail omitted).
Hence we may replace $U$ by $\text{cosk}_1 \text{sk}_1 U$, because
this replaces $U_2$ by $(\text{cosq}_1 \text{sk}_1 U)_2$ and leaves
$U_1$ and $U_0$ unchanged. Then
$$
(d^2_0, d^2_1, d^2_2) : U_2 \to U_1 \times U_1 \times U_1
$$
is a monomorphism whose its image on $T$-valued points is
described in Simplicial, Lemma \ref{simplicial-lemma-work-out}.
In particular, there is a morphism $c$ fitting into a commutative diagram
$$
\xymatrix{
U_1 \times_{(d^1_1, d^1_0), U_0 \times U_0, (d^1_1, d^1_0)} U_1
\ar[d] \ar[rr]_c & & U_2 \ar[d] \\
U_1 \times U_1
\ar[rr]^{(\text{pr}_1, \text{pr}_2, s^0_0 \circ d^1_1 \circ \text{pr}_1)} & &
U_1 \times U_1 \times U_1
}
$$
as going around the other way defines a point of $U_2$.
Pulling back the cocycle condition for $\alpha$ on $U_2$
translates into the condition that the pullbacks of $\alpha$
via the projections to
$U_1 \times_{(d^1_1, d^1_0), U_0 \times U_0, (d^1_1, d^1_0)} U_1$
are the same as the pullback of $\alpha$ via
$s^0_0 \circ d^1_1 \circ \text{pr}_1$ is the identity map
(namely, the pullback of $\alpha$ by $s^0_0$ is the identity).
By Sites, Lemma \ref{sites-lemma-glue-maps}
this means that $\alpha$ comes from an isomorphism
$$
\alpha' : \text{pr}_1^{-1}\mathcal{G}_0 \to \text{pr}_2^{-1}\mathcal{G}_0
$$
of sheaves on $\mathcal{C}/U_0 \times U_0$.
Then finally, the morphism $U_2 \to U_0 \times U_0 \times U_0$
is surjective on associated sheaves as is easily seen using the
surjectivity of $U_1 \to U_0 \times U_0$
and the description of $U_2$ given above. Therefore $\alpha'$
satisfies the cocycle condition on $U_0 \times U_0 \times U_0$.
The proof is finished by an application of
Sites, Lemma \ref{sites-lemma-mapping-property-glue}
to the covering $\{U_0 \to X\}$.
\end{proof}

\begin{lemma}
\label{lemma-hypercovering-cech-complex}
Let $\mathcal{C}$ be a site with equalizers and fibre products.
Let $K$ be a hypercovering. The {\v C}ech complex
of Lemma \ref{lemma-augmentation-cech-complex} associated to
$a^{-1}\mathcal{F}$
$$
a_{0, *}a_0^{-1}\mathcal{F} \to a_{1, *}a_1^{-1}\mathcal{F} \to
a_{2, *}a_2^{-1}\mathcal{F} \to \ldots
$$
is equal to the complex $\SheafHom(s(\mathbf{Z}_{F(K)}^\#), \mathcal{F})$.
Here $s(\mathbf{Z}_{F(K)}^\#)$ is as in
Hypercoverings, Definition \ref{hypercovering-definition-homology}.
\end{lemma}

\begin{proof}
By Lemma \ref{lemma-push-pull-localization} we have
$$
a_{n, *}a_n^{-1}\mathcal{F} = \SheafHom'(F(K_n)^\#, \mathcal{F})
$$
where $\SheafHom'$ is as in Sites, Section \ref{sites-section-glueing-sheaves}.
The boundary maps in the complex of
Lemma \ref{lemma-augmentation-cech-complex}
come from the simplicial structure.
Thus the equality of complexes comes 
from the canonical identifications
$\SheafHom'(\mathcal{G}, \mathcal{F}) =
\SheafHom(\mathbf{Z}_\mathcal{G}, \mathcal{F})$ for
$\mathcal{G}$ in $\Sh(\mathcal{C})$.
\end{proof}

\begin{lemma}
\label{lemma-hypercovering-descent-bounded-abelian}
Let $\mathcal{C}$ be a site with equalizers and fibre products.
Let $K$ be a hypercovering. For
$E \in D(\mathcal{C})$ the map
$$
E \longrightarrow Ra_*a^{-1}E
$$
is an isomorphism.
\end{lemma}

\begin{proof}
First, let $\mathcal{I}$ be an injective abelian sheaf on $\mathcal{C}$.
Then the spectral sequence of
Lemma \ref{lemma-augmentation-spectral-sequence}
for the sheaf $a^{-1}\mathcal{I}$ degenerates as
$(a^{-1}\mathcal{I})_p = a_p^{-1}\mathcal{I}$
is injective by Lemma \ref{lemma-localize-injective}.
Thus the complex
$$
a_{0, *}a_0^{-1}\mathcal{I} \to
a_{1, *}a_1^{-1}\mathcal{I} \to
a_{2, *}a_2^{-1}\mathcal{I} \to\ldots
$$
computes $Ra_*a^{-1}\mathcal{I}$. By
Lemma \ref{lemma-hypercovering-cech-complex}
this is equal to the complex
$\SheafHom(s(\mathbf{Z}_{F(K)}^\#), \mathcal{I})$.
Because $K$ is a hypercovering, we see that
$s(\mathbf{Z}_{F(K)}^\#)$ is exact in degrees $> 0$ by
Hypercoverings, Lemma \ref{hypercovering-lemma-acyclic-hypercover-sheaves}
applied to the simplicial presheaf $F(K)$.
Since $\mathcal{I}$ is injective, the functor $\SheafHom(-, \mathcal{I})$
is exact and we conclude that
$\SheafHom(s(\mathbf{Z}_{F(K)}^\#), \mathcal{I})$
is exact in positive degrees. We conclude that
$R^pa_*a^{-1}\mathcal{I} = 0$ for $p > 0$.
On the other hand, we have $\mathcal{I} = a_*a^{-1}\mathcal{I}$
by Lemma \ref{lemma-hypercovering-descent-sheaves}.

\medskip\noindent
Bounded case. Let $E \in D^+(\mathcal{C})$.
Choose a bounded below complex $\mathcal{I}^\bullet$ of injectives
representing $E$. By the result of the first paragraph and
Leray's acyclicity lemma
(Derived Categories, Lemma \ref{derived-lemma-leray-acyclicity})
$Ra_*a^{-1}\mathcal{I}^\bullet$
is computed by the complex
$a_*a^{-1}\mathcal{I}^\bullet = \mathcal{I}^\bullet$
and we conclude the lemma is true in this case.

\medskip\noindent
Unbounded case. We urge the reader to skip this, since the argument
is the same as above, except that we use explicit representation
by double complexes to get around convergence issues.
Let $E \in D(\mathcal{C})$.
To show the map $E \to Ra_*a^{-1}E$ is an isomorphism,
it suffices to show for every object $U$ of $\mathcal{C}$ that
$$
R\Gamma(U, E) = R\Gamma(U, Ra_*a^{-1}E)
$$
We will compute both sides and show the map $E \to Ra_*a^{-1}E$
induces an isomorphism. Choose a K-injective
complex $\mathcal{I}^\bullet$ representing $E$. Choose a quasi-isomorphism
$a^{-1}\mathcal{I}^\bullet \to \mathcal{J}^\bullet$
for some K-injective complex $\mathcal{J}^\bullet$ on
$(\mathcal{C}/K)_{total}$. We have
$$
R\Gamma(U, E) = R\Hom(\mathbf{Z}_U^\#, E)
$$
and
$$
R\Gamma(U, Ra_*a^{-1}E) = R\Hom(\mathbf{Z}_U^\#, Ra_*a^{-1}E) =
R\Hom(a^{-1}\mathbf{Z}_U^\#, a^{-1}E)
$$
By Lemma \ref{lemma-simplicial-resolution-augmentation}
we have a quasi-isomorphism
$$
\Big(\ldots \to
g_{2!}(a_2^{-1}\mathbf{Z}_U^\#) \to
g_{1!}(a_1^{-1}\mathbf{Z}_U^\#) \to
g_{0!}(a_0^{-1}\mathbf{Z}_U^\#)\Big)
\longrightarrow
a^{-1}\mathbf{Z}_U^\#
$$
Hence $R\Hom(a^{-1}\mathbf{Z}_U^\#, a^{-1}E)$ is equal to
$$
R\Gamma((\mathcal{C}/K)_{total},
R\SheafHom(
\ldots \to
g_{2!}(a_2^{-1}\mathbf{Z}_U^\#) \to
g_{1!}(a_1^{-1}\mathbf{Z}_U^\#) \to
g_{0!}(a_0^{-1}\mathbf{Z}_U^\#),
\mathcal{J}^\bullet))
$$
By the construction in Cohomology on Sites, Section
\ref{sites-cohomology-section-internal-hom}
and since $\mathcal{J}^\bullet$ is K-injective, we see that
this is represented by the complex of abelian groups with terms
$$
\prod\nolimits_{p + q = n}
\Hom(g_{p!}(a_p^{-1}\mathbf{Z}_U^\#), \mathcal{J}^q) =
\prod\nolimits_{p + q = n}
\Hom(a_p^{-1}\mathbf{Z}_U^\#, g_p^{-1}\mathcal{J}^q)
$$
See Cohomology on Sites, Lemmas
\ref{sites-cohomology-lemma-RHom-into-K-injective} and
\ref{sites-cohomology-lemma-section-RHom-over-U} for more information.
Thus we find that $R\Gamma(U, Ra_*a^{-1}E)$ is computed by
the product total complex $\text{Tot}_\pi(B^{\bullet, \bullet})$
with $B^{p, q} = \Hom(a_p^{-1}\mathbf{Z}_U^\#, g_p^{-1}\mathcal{J}^q)$.
For the other side we argue similarly. First we note that
$$
s(\mathbf{Z}_{F(K)}^\#) \longrightarrow \mathbf{Z}
$$
is a quasi-isomorphism of complexes on $\mathcal{C}$
by Hypercoverings, Lemma \ref{hypercovering-lemma-acyclic-hypercover-sheaves}.
Since $\mathbf{Z}_U^\#$ is a flat sheaf of $\mathbf{Z}$-modules
we see that
$$
s(\mathbf{Z}_{F(K)}^\#) \otimes_\mathbf{Z} \mathbf{Z}_U^\#
\longrightarrow
\mathbf{Z}_U^\#
$$
is a quasi-isomorphism. Therefore
$R\Hom(\mathbf{Z}_U^\#, E)$ is equal to
$$
R\Gamma(\mathcal{C}, R\SheafHom(
s(\mathbf{Z}_{F(K)}^\#) \otimes_\mathbf{Z} \mathbf{Z}_U^\#,
\mathcal{I}^\bullet))
$$
By the construction of $R\SheafHom$ and since $\mathcal{I}^\bullet$
is K-injective, this is represented by the complex of abelian groups
with terms
$$
\prod\nolimits_{p + q = n}
\Hom(\mathbf{Z}^\#_{K_p} \otimes_\mathbf{Z} \mathbf{Z}_U^\#, \mathcal{I}^q)
=
\prod\nolimits_{p + q = n}
\Hom(a_p^{-1}\mathbf{Z}_U^\#, a_p^{-1}\mathcal{I}^q)
$$
The equality of terms follows from the fact that
$\mathbf{Z}^\#_{K_p} \otimes_\mathbf{Z} \mathbf{Z}_U^\# =
a_{p!}a_p^{-1}\mathbf{Z}_U^\#$ by Modules on Sites,
Remark \ref{sites-modules-remark-j-shriek-tensor}.
Thus we find that $R\Gamma(U, E)$ is computed by
the product total complex $\text{Tot}_\pi(A^{\bullet, \bullet})$
with $A^{p, q} = \Hom(a_p^{-1}\mathbf{Z}_U^\#, a_p^{-1}\mathcal{I}^q)$.

\medskip\noindent
Since $\mathcal{I}^\bullet$ is K-injective we see that
$a_p^{-1}\mathcal{I}^\bullet$ is K-injective, see
Lemma \ref{lemma-localize-injective}.
Since $\mathcal{J}^\bullet$ is K-injective we see that
$g_p^{-1}\mathcal{J}^\bullet$ is K-injective, see
Lemma \ref{lemma-restriction-injective-to-component-site}.
Both represent the object $a_p^{-1}E$.
Hence for every $p \geq 0$ the map of complexes
$$
A^{p, \bullet} = \Hom(a_p^{-1}\mathbf{Z}_U^\#, a_p^{-1}\mathcal{I}^\bullet)
\longrightarrow
\Hom(a_p^{-1}\mathbf{Z}_U^\#, g_p^{-1}\mathcal{J}^\bullet) = B^{p, \bullet}
$$
induced by $g_p^{-1}$ applied to the given map
$a^{-1}\mathcal{I}^\bullet \to \mathcal{J}^\bullet$
is a quasi-isomorphisms as these complexes both compute
$$
R\Hom(a_p^{-1}\mathbf{Z}_U^\#, a_p^{-1}E)
$$
By More on Algebra, Lemma \ref{more-algebra-lemma-prod-qis-gives-qis}
we conclude that the right vertical arrow in the commutative diagram
$$
\xymatrix{
R\Gamma(U, E) \ar[r] \ar[d] &
\text{Tot}_\pi(A^{\bullet, \bullet}) \ar[d] \\
R\Gamma(U, Ra_*a^{-1}E) \ar[r] &
\text{Tot}_\pi(B^{\bullet, \bullet})
}
$$
is a quasi-isomorphism. Since we saw above that the horizontal arrows
are quasi-isomorphisms, so is the left vertical arrow.
\end{proof}

\begin{lemma}
\label{lemma-compare-cohomology-hypercovering}
Let $\mathcal{C}$ be a site with equalizers and fibre products.
Let $K$ be a hypercovering.
Then we have a canonical isomorphism
$$
R\Gamma(\mathcal{C}, E) =
R\Gamma((\mathcal{C}/K)_{total}, a^{-1}E)
$$
for $E \in D(\mathcal{C})$.
\end{lemma}

\begin{proof}
This follows from Lemma \ref{lemma-hypercovering-descent-bounded-abelian}
because $R\Gamma((\mathcal{C}/K)_{total}, -) =
R\Gamma(\mathcal{C}, -) \circ Ra_*$ by
Cohomology on Sites, Remark \ref{sites-cohomology-remark-before-Leray}.
\end{proof}

\begin{lemma}
\label{lemma-hypercovering-equivalence-bounded}
Let $\mathcal{C}$ be a site with equalizers and fibre products.
Let $K$ be a hypercovering.
Let $\mathcal{A} \subset \textit{Ab}((\mathcal{C}/K)_{total})$
denote the weak Serre subcategory of cartesian abelian sheaves.
Then the functor $a^{-1}$ defines an equivalence
$$
D^+(\mathcal{C}) \longrightarrow D_\mathcal{A}^+((\mathcal{C}/K)_{total})
$$
with quasi-inverse $Ra_*$.
\end{lemma}

\begin{proof}
Observe that $\mathcal{A}$ is a weak Serre subcategory by
Lemma \ref{lemma-Serre-subcat-cartesian-modules}.
The equivalence is a
formal consequence of the results obtained so far. Use
Lemmas \ref{lemma-hypercovering-descent-sheaves} and
\ref{lemma-hypercovering-descent-bounded-abelian} and
Cohomology on Sites, Lemma \ref{sites-cohomology-lemma-equivalence-bounded}
\end{proof}

\noindent
We urge the reader to skip the following remark.

\begin{remark}
\label{remark-compare-cohomology-hypercovering-presheaf}
Let $\mathcal{C}$ be a site. Let $\mathcal{G}$ be a presheaf of sets on
$\mathcal{C}$. If $\mathcal{C}$ has equalizers and fibre products, then
we've defined the notion of a hypercovering of $\mathcal{G}$ in
Hypercoverings, Definition \ref{hypercovering-definition-hypercovering-variant}.
We claim that all the results in this section have a
valid counterpart in this setting.
To see this,
define the localization $\mathcal{C}/\mathcal{G}$
of $\mathcal{C}$ at $\mathcal{G}$ exactly as in
Sites, Lemma \ref{sites-lemma-localize-topos-site}
(which is stated only for sheaves; the topos
$\Sh(\mathcal{C}/\mathcal{G})$ is equal to the localization
of the topos $\Sh(\mathcal{C})$ at the sheaf $\mathcal{G}^\#$).
Then the reader easily shows that the site
$\mathcal{C}/\mathcal{G}$ has fibre products and equalizers
and that a hypercovering of $\mathcal{G}$ in $\mathcal{C}$
is the same thing as a hypercovering for the site $\mathcal{C}/\mathcal{G}$.
Hence replacing the site $\mathcal{C}$ by $\mathcal{C}/\mathcal{G}$
in the lemmas on hypercoverings above we obtain proofs of the
corresponding results for hypercoverings of $\mathcal{G}$.
Example: for a hypercovering $K$ of $\mathcal{G}$ we have
$$
R\Gamma(\mathcal{C}/\mathcal{G}, E) =
R\Gamma((\mathcal{C}/K)_{total}, a^{-1}E)
$$
for $E \in D^+(\mathcal{C}/\mathcal{G})$ where
$a : \Sh((\mathcal{C}/K)_{total}) \to \Sh(\mathcal{C}/\mathcal{G})$
is the canonical augmentation. This is
Lemma \ref{lemma-compare-cohomology-hypercovering}.
Let $R\Gamma(\mathcal{G}, -) : D(\mathcal{C}) \to D(\textit{Ab})$
be defined as the derived functor of the functor
$H^0(\mathcal{G}, -) = H^0(\mathcal{G}^\#, -)$
discussed in Hypercoverings, Section
\ref{hypercovering-section-hypercoverings-verdier} and
Cohomology on Sites, Section \ref{sites-cohomology-section-limp}.
We have
$$
R\Gamma(\mathcal{G}, E) = R\Gamma(\mathcal{C}/\mathcal{G}, j^{-1}E)
$$
by the analogue of Cohomology on Sites, Lemma
\ref{sites-cohomology-lemma-cohomology-of-open}
for the localization fuctor $j : \mathcal{C}/\mathcal{G} \to \mathcal{C}$.
Putting everything together we obtain
$$
R\Gamma(\mathcal{G}, E) =
R\Gamma((\mathcal{C}/K)_{total}, a^{-1}j^{-1}E) =
R\Gamma((\mathcal{C}/K)_{total}, g^{-1}E)
$$
for $E \in D^+(\mathcal{C})$ where
$g : \Sh((\mathcal{C}/K)_{total}) \to \Sh(\mathcal{C})$
is the composition of $a$ and $j$.
\end{remark}







\section{Cohomological descent for hypercoverings: modules}
\label{section-cohomological-descent-hypercoverings-modules}

\noindent
Let $\mathcal{C}$ be a site. Let $\mathcal{O}_\mathcal{C}$
be a sheaf of rings. Assume $\mathcal{C}$
has equalizers and fibre products and let $K$ be a hypercovering
as defined in Hypercoverings, Definition
\ref{hypercovering-definition-hypercovering-variant}. We will study
cohomological descent for the augmentation
$$
a :
(\Sh((\mathcal{C}/K)_{total}), \mathcal{O})
\longrightarrow
(\Sh(\mathcal{C}), \mathcal{O}_\mathcal{C})
$$
of Remark \ref{remark-augmentation-ringed}.

\begin{lemma}
\label{lemma-hypercovering-descent-modules}
Let $\mathcal{C}$ be a site with equalizers and fibre products.
Let $\mathcal{O}_\mathcal{C}$ be a sheaf of rings.
Let $K$ be a hypercovering. With notation as above
$$
a^* : \textit{Mod}(\mathcal{O}_\mathcal{C}) \to \textit{Mod}(\mathcal{O})
$$
is fully faithful with essential image the cartesian $\mathcal{O}$-modules.
The functor $a_*$ provides the quasi-inverse.
\end{lemma}

\begin{proof}
Since $a^{-1}\mathcal{O}_\mathcal{C} = \mathcal{O}$ we have
$a^* = a^{-1}$. Hence the lemma follows
immediately from Lemma \ref{lemma-hypercovering-descent-sheaves}.
\end{proof}

\begin{lemma}
\label{lemma-hypercovering-descent-bounded-modules}
Let $\mathcal{C}$ be a site with equalizers and fibre products.
Let $\mathcal{O}_\mathcal{C}$ be a sheaf of rings.
Let $K$ be a hypercovering. For
$E \in D(\mathcal{O}_\mathcal{C})$ the map
$$
E \longrightarrow Ra_*La^*E
$$
is an isomorphism.
\end{lemma}

\begin{proof}
Since $a^{-1}\mathcal{O}_\mathcal{C} = \mathcal{O}$ we have
$La^* = a^* = a^{-1}$. Moreover $Ra_*$ agrees with
$Ra_*$ on abelian sheaves, see
Cohomology on Sites, Lemma
\ref{sites-cohomology-lemma-modules-abelian-unbounded}.
Hence the lemma follows
immediately from Lemma \ref{lemma-hypercovering-descent-bounded-abelian}.
\end{proof}

\begin{lemma}
\label{lemma-compare-cohomology-hypercovering-modules}
Let $\mathcal{C}$ be a site with equalizers and fibre products.
Let $\mathcal{O}_\mathcal{C}$ be a sheaf of rings.
Let $K$ be a hypercovering.
Then we have a canonical isomorphism
$$
R\Gamma(\mathcal{C}, E) =
R\Gamma((\mathcal{C}/K)_{total}, La^*E)
$$
for $E \in D(\mathcal{O}_\mathcal{C})$.
\end{lemma}

\begin{proof}
This follows from Lemma \ref{lemma-hypercovering-descent-bounded-modules}
because $R\Gamma((\mathcal{C}/K)_{total}, -) =
R\Gamma(\mathcal{C}, -) \circ Ra_*$ by
Cohomology on Sites, Remark \ref{sites-cohomology-remark-before-Leray}
or by
Cohomology on Sites, Lemma \ref{sites-cohomology-lemma-Leray-unbounded}.
\end{proof}

\begin{lemma}
\label{lemma-hypercovering-equivalence-bounded-modules}
Let $\mathcal{C}$ be a site with equalizers and fibre products.
Let $\mathcal{O}_\mathcal{C}$ be a sheaf of rings.
Let $K$ be a hypercovering.
Let $\mathcal{A} \subset \textit{Mod}(\mathcal{O})$
denote the weak Serre subcategory of cartesian $\mathcal{O}$-modules.
Then the functor $La^*$ defines an equivalence
$$
D^+(\mathcal{O}_\mathcal{C}) \longrightarrow D_\mathcal{A}^+(\mathcal{O})
$$
with quasi-inverse $Ra_*$.
\end{lemma}

\begin{proof}
Observe that $\mathcal{A}$ is a weak Serre subcategory by
Lemma \ref{lemma-Serre-subcat-cartesian-modules}
(the required hypotheses hold by the discussion in
Remark \ref{remark-augmentation-ringed}).
The equivalence is a
formal consequence of the results obtained so far. Use
Lemmas \ref{lemma-hypercovering-descent-modules} and
\ref{lemma-hypercovering-descent-bounded-modules} and
Cohomology on Sites, Lemma \ref{sites-cohomology-lemma-equivalence-bounded}.
\end{proof}






\section{Cohomological descent for hypercoverings of an object}
\label{section-cohomological-descent-hypercoverings-X}

\noindent
In this section we assume $\mathcal{C}$ has fibre products
and $X \in \Ob(\mathcal{C})$. We let $K$ be a hypercovering of $X$
as defined in
Hypercoverings, Definition \ref{hypercovering-definition-hypercovering}.
We will study the augmentation
$$
a : \Sh((\mathcal{C}/K)_{total}) \longrightarrow \Sh(\mathcal{C}/X)
$$
of Remark \ref{remark-augmentation-over-object}.
Observe that $\mathcal{C}/X$ is a site which has equalizers
and fibre products and that $K$ is a
hypercovering for the site $\mathcal{C}/X$\footnote{The converse may not
be the case, i.e., if $K$ is a simplicial object of
$\text{SR}(\mathcal{C}, X) = \text{SR}(\mathcal{C}/X)$
which defines a hypercovering for the site $\mathcal{C}/X$ as in
Hypercoverings, Definition \ref{hypercovering-definition-hypercovering-variant},
then it may not be true that $K$ defines a hypercovering of $X$.
For example, if $K_0 = \{U_{0, i}\}_{i \in I_0}$
then the latter condition guarantees
$\{U_{0, i} \to X\}$ is a covering of $\mathcal{C}$
whereas the former condition only requires
$\coprod h_{U_{0, i}}^\# \to h_X^\#$ to be a surjective map
of sheaves.} by Hypercoverings, Lemma
\ref{hypercovering-lemma-hypercovering-F}.
This means that every single result proved for hypercoverings
in Section \ref{section-cohomological-descent-hypercoverings}
has an immediate analogue in the situation in this section.

\begin{lemma}
\label{lemma-hypercovering-X-descent-sheaves}
Let $\mathcal{C}$ be a site with fibre products and $X \in \Ob(\mathcal{C})$.
Let $K$ be a hypercovering of $X$. Then
\begin{enumerate}
\item $a^{-1} : \Sh(\mathcal{C}/X) \to \Sh((\mathcal{C}/K)_{total})$
is fully faithful with essential image the cartesian sheaves of sets,
\item $a^{-1} : \textit{Ab}(\mathcal{C}/X) \to
\textit{Ab}((\mathcal{C}/K)_{total})$
is fully faithful with essential image the cartesian sheaves
of abelian groups.
\end{enumerate}
In both cases $a_*$ provides the quasi-inverse functor.
\end{lemma}

\begin{proof}
Via Remarks \ref{remark-semi-representable-over-object} and
\ref{remark-augmentation-over-object} and the discussion in
the introduction to this section
this follows from Lemma \ref{lemma-hypercovering-descent-sheaves}.
\end{proof}

\begin{lemma}
\label{lemma-hypercovering-X-descent-bounded-abelian}
Let $\mathcal{C}$ be a site with fibre product and $X \in \Ob(\mathcal{C})$.
Let $K$ be a hypercovering of $X$. For
$E \in D(\mathcal{C}/X)$ the map
$$
E \longrightarrow Ra_*a^{-1}E
$$
is an isomorphism.
\end{lemma}

\begin{proof}
Via Remarks \ref{remark-semi-representable-over-object} and
\ref{remark-augmentation-over-object} and the discussion in
the introduction to this section
this follows from Lemma \ref{lemma-hypercovering-descent-bounded-abelian}.
\end{proof}

\begin{lemma}
\label{lemma-compare-cohomology-hypercovering-X}
Let $\mathcal{C}$ be a site with fibre products and $X \in \Ob(\mathcal{C})$.
Let $K$ be a hypercovering of $X$.
Then we have a canonical isomorphism
$$
R\Gamma(X, E) = R\Gamma((\mathcal{C}/K)_{total}, a^{-1}E)
$$
for $E \in D(\mathcal{C}/X)$.
\end{lemma}

\begin{proof}
Via Remarks \ref{remark-semi-representable-over-object} and
\ref{remark-augmentation-over-object}
this follows from Lemma \ref{lemma-compare-cohomology-hypercovering}.
\end{proof}

\begin{lemma}
\label{lemma-hypercovering-X-equivalence-bounded}
Let $\mathcal{C}$ be a site with fibre products and $X \in \Ob(\mathcal{C})$.
Let $K$ be a hypercovering of $X$.
Let $\mathcal{A} \subset \textit{Ab}((\mathcal{C}/K)_{total})$
denote the weak Serre subcategory of cartesian abelian sheaves.
Then the functor $a^{-1}$ defines an equivalence
$$
D^+(\mathcal{C}/X) \longrightarrow D_\mathcal{A}^+((\mathcal{C}/K)_{total})
$$
with quasi-inverse $Ra_*$.
\end{lemma}

\begin{proof}
Via Remarks \ref{remark-semi-representable-over-object} and
\ref{remark-augmentation-over-object}
this follows from Lemma \ref{lemma-hypercovering-equivalence-bounded}.
\end{proof}








\section{Cohomological descent for hypercoverings of an object: modules}
\label{section-cohomological-descent-hypercoverings-X-modules}

\noindent
In this section we assume $\mathcal{C}$ has fibre products
and $X \in \Ob(\mathcal{C})$. We let $K$ be a hypercovering of $X$
as defined in
Hypercoverings, Definition \ref{hypercovering-definition-hypercovering}.
Let $\mathcal{O}_\mathcal{C}$ be a sheaf of rings on $\mathcal{C}$.
Set $\mathcal{O}_X = \mathcal{O}_\mathcal{C}|_{\mathcal{C}/X}$.
We will study the augmentation
$$
a :
(\Sh((\mathcal{C}/K)_{total}), \mathcal{O})
\longrightarrow
(\Sh(\mathcal{C}/X), \mathcal{O}_X)
$$
of Remark \ref{remark-augmentation-ringed-over-object}.
Observe that $\mathcal{C}/X$ is a site which has equalizers
and fibre products and that $K$ is a
hypercovering for the site $\mathcal{C}/X$.
Therefore the results in this section are immediate consequences
of the corresponding results in
Section \ref{section-cohomological-descent-hypercoverings-modules}.

\begin{lemma}
\label{lemma-hypercovering-X-descent-modules}
Let $\mathcal{C}$ be a site with fibre products and $X \in \Ob(\mathcal{C})$.
Let $\mathcal{O}_\mathcal{C}$ be a sheaf of rings.
Let $K$ be a hypercovering of $X$. With notation as above
$$
a^* : \textit{Mod}(\mathcal{O}_X) \to \textit{Mod}(\mathcal{O})
$$
is fully faithful with essential image the cartesian $\mathcal{O}$-modules.
The functor $a_*$ provides the quasi-inverse.
\end{lemma}

\begin{proof}
Via Remarks \ref{remark-semi-representable-ringed-over-object} and
\ref{remark-augmentation-ringed-over-object} and the discussion in
the introduction to this section
this follows from Lemma \ref{lemma-hypercovering-descent-modules}.
\end{proof}

\begin{lemma}
\label{lemma-hypercovering-X-descent-bounded-modules}
Let $\mathcal{C}$ be a site with fibre products and $X \in \Ob(\mathcal{C})$.
Let $\mathcal{O}_\mathcal{C}$ be a sheaf of rings.
Let $K$ be a hypercovering of $X$. For
$E \in D(\mathcal{O}_X)$ the map
$$
E \longrightarrow Ra_*La^*E
$$
is an isomorphism.
\end{lemma}

\begin{proof}
Via Remarks \ref{remark-semi-representable-ringed-over-object} and
\ref{remark-augmentation-ringed-over-object} and the discussion in
the introduction to this section
this follows from Lemma \ref{lemma-hypercovering-descent-bounded-modules}.
\end{proof}

\begin{lemma}
\label{lemma-compare-cohomology-hypercovering-X-modules}
Let $\mathcal{C}$ be a site with fibre products and $X \in \Ob(\mathcal{C})$.
Let $\mathcal{O}_\mathcal{C}$ be a sheaf of rings.
Let $K$ be a hypercovering of $X$.
Then we have a canonical isomorphism
$$
R\Gamma(X, E) = R\Gamma((\mathcal{C}/K)_{total}, La^*E)
$$
for $E \in D(\mathcal{O}_\mathcal{C})$.
\end{lemma}

\begin{proof}
Via Remarks \ref{remark-semi-representable-ringed-over-object} and
\ref{remark-augmentation-ringed-over-object} and the discussion in
the introduction to this section
this follows from Lemma \ref{lemma-compare-cohomology-hypercovering-modules}.
\end{proof}

\begin{lemma}
\label{lemma-hypercovering-X-equivalence-bounded-modules}
Let $\mathcal{C}$ be a site with fibre products and $X \in \Ob(\mathcal{C})$.
Let $\mathcal{O}_\mathcal{C}$ be a sheaf of rings.
Let $K$ be a hypercovering of $X$.
Let $\mathcal{A} \subset \textit{Mod}(\mathcal{O})$
denote the weak Serre subcategory of cartesian $\mathcal{O}$-modules.
Then the functor $La^*$ defines an equivalence
$$
D^+(\mathcal{O}_X) \longrightarrow D_\mathcal{A}^+(\mathcal{O})
$$
with quasi-inverse $Ra_*$.
\end{lemma}

\begin{proof}
Via Remarks \ref{remark-semi-representable-ringed-over-object} and
\ref{remark-augmentation-ringed-over-object} and the discussion in
the introduction to this section
this follows from Lemma \ref{lemma-hypercovering-equivalence-bounded-modules}.
\end{proof}










\section{Hypercovering by a simplicial object of the site}
\label{section-hypercovering}

\noindent
Let $\mathcal{C}$ be a site with fibre products and
let $X \in \Ob(\mathcal{C})$.
In this section we elucidate the results of
Section \ref{section-cohomological-descent-hypercoverings-X}
in the case that our hypercovering is given by
a simplicial object of the site.
Let $U$ be a simplicial object of $\mathcal{C}$.
As usual we denote $U_n = U([n])$ and $f_\varphi : U_n \to U_m$
the morphism $f_\varphi = U(\varphi)$ corresponding to
$\varphi : [m] \to [n]$.
Assume we have an augmentation
$$
a : U \to X
$$
From this we obtain a simplicial site $(\mathcal{C}/U)_{total}$
and an augmentation morphism
$$
a : \Sh((\mathcal{C}/U)_{total}) \longrightarrow \Sh(\mathcal{C}/X)
$$
by thinking of $U$ as a simiplical semi-representable
object of $\mathcal{C}/X$ whose degree $n$ part is the singleton
element $\{U_n/X\}$ and applying the constructions in
Remark \ref{remark-augmentation-over-object}.

\medskip\noindent
An object of the site $(\mathcal{C}/U)_{total}$ is given by
a $V/U_n$ and a morphism $(\varphi, f) : V/U_n \to W/U_m$ is given
by a morphism $\varphi : [m] \to [n]$ in $\Delta$ and a morphism
$f : V \to W$ such that the diagram
$$
\xymatrix{
V \ar[r]_f \ar[d] & W \ar[d] \\
U_n \ar[r]^{f_\varphi} & U_m
}
$$
is commutative. The morphism of topoi $a$ is given by the cocontinuous
functor $V/U_n \mapsto V/X$. That's all folks!

\medskip\noindent
Let us say that the augmentation $a : U \to X$ is a
{\it hypercovering of $X$ in $\mathcal{C}$}
if the following hold
\begin{enumerate}
\item $\{U_0 \to X\}$ is a covering of $\mathcal{C}$,
\item $\{U_1 \to U_0 \times_X U_0\}$ is a covering of $\mathcal{C}$,
\item $\{U_{n + 1} \to (\text{cosk}_n\text{sk}_n U)_{n + 1}\}$
is a covering of $\mathcal{C}$ for $n \geq 1$.
\end{enumerate}
The category $\mathcal{C}/X$ has all connected finite limits, hence the
coskeleta used in the formulation above exist. Of course, we see
that $U$ is a hypercovering of $X$ in $\mathcal{C}$ if and only if
the simplicial semi-representable object $\{U_n\}$ is a hypercovering of $X$
in the sense of Section \ref{section-cohomological-descent-hypercoverings-X}.

\begin{lemma}
\label{lemma-hypercovering-X-simple-descent-sheaves}
Let $\mathcal{C}$ be a site with fibre product and $X \in \Ob(\mathcal{C})$.
Let $a : U \to X$ be a hypercovering of $X$ in $\mathcal{C}$ as defined above.
Then
\begin{enumerate}
\item $a^{-1} : \Sh(\mathcal{C}/X) \to \Sh((\mathcal{C}/U)_{total})$
is fully faithful with essential image the cartesian sheaves of sets,
\item $a^{-1} : \textit{Ab}(\mathcal{C}/X) \to
\textit{Ab}((\mathcal{C}/U)_{total})$
is fully faithful with essential image the cartesian sheaves
of abelian groups.
\end{enumerate}
In both cases $a_*$ provides the quasi-inverse functor.
\end{lemma}

\begin{proof}
This is a special case of
Lemma \ref{lemma-hypercovering-X-descent-sheaves}.
\end{proof}

\begin{lemma}
\label{lemma-hypercovering-X-simple-descent-bounded-abelian}
Let $\mathcal{C}$ be a site with fibre product and $X \in \Ob(\mathcal{C})$.
Let $a : U \to X$ be a hypercovering of $X$ in $\mathcal{C}$ as defined above.
For $E \in D(\mathcal{C}/X)$ the map
$$
E \longrightarrow Ra_*a^{-1}E
$$
is an isomorphism.
\end{lemma}

\begin{proof}
This is a special case of
Lemma \ref{lemma-hypercovering-X-descent-bounded-abelian}.
\end{proof}

\begin{lemma}
\label{lemma-compare-cohomology-hypercovering-X-simple}
Let $\mathcal{C}$ be a site with fibre products and $X \in \Ob(\mathcal{C})$.
Let $a : U \to X$ be a hypercovering of $X$ in $\mathcal{C}$ as defined above.
Then we have a canonical isomorphism
$$
R\Gamma(X, E) = R\Gamma((\mathcal{C}/U)_{total}, a^{-1}E)
$$
for $E \in D(\mathcal{C}/X)$.
\end{lemma}

\begin{proof}
This is a special case of
Lemma \ref{lemma-compare-cohomology-hypercovering-X}.
\end{proof}

\begin{lemma}
\label{lemma-hypercovering-X-simple-equivalence-bounded}
Let $\mathcal{C}$ be a site with fibre product and $X \in \Ob(\mathcal{C})$.
Let $a : U \to X$ be a hypercovering of $X$ in $\mathcal{C}$ as defined above.
Let $\mathcal{A} \subset \textit{Ab}((\mathcal{C}/U)_{total})$
denote the weak Serre subcategory of cartesian abelian sheaves.
Then the functor $a^{-1}$ defines an equivalence
$$
D^+(\mathcal{C}/X) \longrightarrow D_\mathcal{A}^+((\mathcal{C}/U)_{total})
$$
with quasi-inverse $Ra_*$.
\end{lemma}

\begin{proof}
This is a special case of
Lemma \ref{lemma-hypercovering-X-equivalence-bounded}
\end{proof}

\begin{lemma}
\label{lemma-sr-when-fibre-products}
Let $U$ be a simplicial object of a site $\mathcal{C}$
with fibre products.
\begin{enumerate}
\item $\mathcal{C}/U$ has the structure of a simplicial object
in the category whose objects are sites and
whose morphisms are morphisms of sites,
\item the construction of Lemma \ref{lemma-simplicial-site-site}
applied to the structure in (1)
reproduces the site $(\mathcal{C}/U)_{total}$ above,
\item if $a : U \to X$ is an augmentation, then
$a_0 : \mathcal{C}/U_0 \to \mathcal{C}/X$ is an augmentation
as in Remark \ref{remark-augmentation-site} part (A) and gives the
same morphism of topoi
$a : \Sh((\mathcal{C}/U)_{total}) \to \Sh(\mathcal{C}/X)$
as the one above.
\end{enumerate}
\end{lemma}

\begin{proof}
Given a morphism of objects $V \to W$ of $\mathcal{C}$ the localization
morphism $j : \mathcal{C}/V \to \mathcal{C}/W$ is a left adjoint to
the base change functor $\mathcal{C}/W \to \mathcal{C}/V$.
The base change functor is continuous and induces the same morphism of
topoi as $j$. See
Sites, Lemma \ref{sites-lemma-relocalize-given-fibre-products}.
This proves (1).

\medskip\noindent
Part (2) holds because a morphism $V/U_n \to W/U_m$
of the category constructed
in Lemma \ref{lemma-simplicial-site-site}
is a morphism $V \to W \times_{U_m, f_\varphi} U_n$ over $U_n$
which is the same thing as a morphism $f : V \to W$
over the morphism $f_\varphi : U_n \to U_m$, i.e.,
the same thing as a morphism in the category $(\mathcal{C}/U)_{total}$
defined above. Equality of sets of coverings is
immediate from the definition.

\medskip\noindent
We omit the proof of (3).
\end{proof}







\section{Hypercovering by a simplicial object of the site: modules}
\label{section-hypercovering-modules}

\noindent
Let $\mathcal{C}$ be a site with fibre products and $X \in \Ob(\mathcal{C})$.
Let $\mathcal{O}_\mathcal{C}$ be a sheaf of rings on $\mathcal{C}$.
Let $U \to X$ be a hypercovering of $X$ in $\mathcal{C}$ as defined
in Section \ref{section-hypercovering}. In this section we study the
augmentation
$$
a :
(\Sh((\mathcal{C}/U)_{total}), \mathcal{O})
\longrightarrow
(\Sh(\mathcal{C}/X), \mathcal{O}_X)
$$
we obtain by thinking of $U$ as a simiplical semi-representable
object of $\mathcal{C}/X$ whose degree $n$ part is the singleton
element $\{U_n/X\}$ and applying the constructions in
Remark \ref{remark-augmentation-ringed-over-object}.
Thus all the results in this section are immediate consequences
of the corresponding results in
Section \ref{section-cohomological-descent-hypercoverings-X-modules}.

\begin{lemma}
\label{lemma-hypercovering-X-simple-descent-modules}
Let $\mathcal{C}$ be a site with fibre products and $X \in \Ob(\mathcal{C})$.
Let $\mathcal{O}_\mathcal{C}$ be a sheaf of rings.
Let $U$ be a hypercovering of $X$ in $\mathcal{C}$. With notation as above
$$
a^* : \textit{Mod}(\mathcal{O}_X) \to \textit{Mod}(\mathcal{O})
$$
is fully faithful with essential image the cartesian $\mathcal{O}$-modules.
The functor $a_*$ provides the quasi-inverse.
\end{lemma}

\begin{proof}
This is a special case of
Lemma \ref{lemma-hypercovering-X-descent-modules}.
\end{proof}

\begin{lemma}
\label{lemma-hypercovering-X-simple-descent-bounded-modules}
Let $\mathcal{C}$ be a site with fibre products and $X \in \Ob(\mathcal{C})$.
Let $\mathcal{O}_\mathcal{C}$ be a sheaf of rings.
Let $U$ be a hypercovering of $X$ in $\mathcal{C}$. For
$E \in D(\mathcal{O}_X)$ the map
$$
E \longrightarrow Ra_*La^*E
$$
is an isomorphism.
\end{lemma}

\begin{proof}
This is a special case of
Lemma \ref{lemma-hypercovering-X-descent-bounded-modules}.
\end{proof}

\begin{lemma}
\label{lemma-compare-cohomology-hypercovering-X-simple-modules}
Let $\mathcal{C}$ be a site with fibre products and $X \in \Ob(\mathcal{C})$.
Let $\mathcal{O}_\mathcal{C}$ be a sheaf of rings.
Let $U$ be a hypercovering of $X$ in $\mathcal{C}$.
Then we have a canonical isomorphism
$$
R\Gamma(X, E) = R\Gamma((\mathcal{C}/U)_{total}, La^*E)
$$
for $E \in D(\mathcal{O}_\mathcal{C})$.
\end{lemma}

\begin{proof}
This is a special case of
Lemma \ref{lemma-compare-cohomology-hypercovering-X-modules}.
\end{proof}

\begin{lemma}
\label{lemma-hypercovering-X-simple-equivalence-bounded-modules}
Let $\mathcal{C}$ be a site with fibre products and $X \in \Ob(\mathcal{C})$.
Let $\mathcal{O}_\mathcal{C}$ be a sheaf of rings.
Let $U$ be a hypercovering of $X$ in $\mathcal{C}$.
Let $\mathcal{A} \subset \textit{Mod}(\mathcal{O})$
denote the weak Serre subcategory of cartesian $\mathcal{O}$-modules.
Then the functor $La^*$ defines an equivalence
$$
D^+(\mathcal{O}_X) \longrightarrow D_\mathcal{A}^+(\mathcal{O})
$$
with quasi-inverse $Ra_*$.
\end{lemma}

\begin{proof}
This is a special case of
Lemma \ref{lemma-hypercovering-X-equivalence-bounded-modules}.
\end{proof}







\section{Unbounded cohomological descent for hypercoverings}
\label{section-unbounded-cohomological-descent}

\noindent
In this section we discuss unbounded cohomological descent.
The results themselves will be immediate consequences of
our results on bounded cohomological descent in the previous
sections and Cohomology on Sites, Lemmas
\ref{sites-cohomology-lemma-equivalence-unbounded-one} and/or
\ref{sites-cohomology-lemma-equivalence-unbounded-two}; the real work lies
in setting up notation and choosing appropriate assumptions.
Our discussion is motivated by the discussion in \cite{six-I}
although the details are a good bit different.

\medskip\noindent
Let $(\mathcal{C}, \mathcal{O}_\mathcal{C})$ be a ringed site.
Assume given for every object $U$ of $\mathcal{C}$
a weak Serre subcategory $\mathcal{A}_U \subset \textit{Mod}(\mathcal{O}_U)$
satisfying the following properties
\begin{enumerate}
\item
\label{item-restriction}
given a morphism $U \to V$ of $\mathcal{C}$ the restriction
functor $\textit{Mod}(\mathcal{O}_V) \to \textit{Mod}(\mathcal{O}_U)$
sends $\mathcal{A}_V$ into $\mathcal{A}_U$,
\item
\label{item-local}
given a covering $\{U_i \to U\}_{i \in I}$ of $\mathcal{C}$
an object $\mathcal{F}$ of $\textit{Mod}(\mathcal{O}_U)$
is in $\mathcal{A}_U$ if and only if the restriction of
$\mathcal{F}$ to $\mathcal{C}/U_i$ is in $\mathcal{A}_{U_i}$
for all $i \in I$.
\item
\label{item-bounded-dimension}
there exists a subset $\mathcal{B} \subset \Ob(\mathcal{C})$
such that
\begin{enumerate}
\item every object of $\mathcal{C}$ has a covering whose
members are in $\mathcal{B}$, and
\item for every $V \in \mathcal{B}$ there exists an integer $d_V$
and a cofinal system $\text{Cov}_V$ of coverings of $V$ such
that
$$
H^p(V_i, \mathcal{F}) = 0 \text{ for }
\{V_i \to V\} \in \text{Cov}_V,\ p > d_V, \text{ and }
\mathcal{F} \in \Ob(\mathcal{A}_V)
$$
\end{enumerate}
\end{enumerate}
Note that we require this to be true for $\mathcal{F}$ in
$\mathcal{A}_V$ and not just for ``global'' objects
(and thus it is stronger than the condition imposed in
Cohomology on Sites, Situation \ref{sites-cohomology-situation-olsson-laszlo}).
In this situation, there is a weak Serre subcategory
$\mathcal{A} \subset \textit{Mod}(\mathcal{O}_\mathcal{C})$
consisting of objects whose restriction to $\mathcal{C}/U$
is in $\mathcal{A}_U$ for all $U \in \Ob(\mathcal{C})$.
Moreover, there are derived categories
$D_\mathcal{A}(\mathcal{O}_\mathcal{C})$ and
$D_{\mathcal{A}_U}(\mathcal{O}_U)$ and the restriction
functors send these into each other.

\begin{example}
\label{example-quasi-coherent-spaces-etale}
Let $S$ be a scheme and let $X$ be an algebraic space over $S$.
Let $\mathcal{C} = X_{spaces, \etale}$ be the \'etale site
on the category of algebraic spaces \'etale over $X$, see
Properties of Spaces, Definition
\ref{spaces-properties-definition-spaces-etale-site}.
Denote $\mathcal{O}_\mathcal{C}$ the structure sheaf, i.e., the
sheaf given by the rule $U \mapsto \Gamma(U, \mathcal{O}_U)$.
Denote $\mathcal{A}_U$ the category of quasi-coherent $\mathcal{O}_U$-modules.
Let $\mathcal{B} = \Ob(\mathcal{C})$ and for $V \in \mathcal{B}$
set $d_V = 0$ and let $\text{Cov}_V$ denote
the coverings $\{V_i \to V\}$ with $V_i$ affine for all $i$.
Then the assumptions (1), (2), (3) are satisfied.
See Properties of Spaces, Lemmas
\ref{spaces-properties-lemma-pullback-quasi-coherent} and
\ref{spaces-properties-lemma-properties-quasi-coherent}
for properties (1) and (2) and the vanishing in (3) follows from
Cohomology of Schemes, Lemma
\ref{coherent-lemma-quasi-coherent-affine-cohomology-zero}
and the discussion in Cohomology of Spaces, Section
\ref{spaces-cohomology-section-higher-direct-image}.
\end{example}

\begin{example}
\label{example-etale}
Let $S$ be one of the following types of schemes
\begin{enumerate}
\item the spectrum of a finite field,
\item the spectrum of a separably closed field,
\item the spectrum of a strictly henselian Noetherian local ring,
\item the spectrum of a henselian Noetherian local ring with
finite residue field,
\item add more here.
\end{enumerate}
Let $\Lambda$ be a finite ring whose order is invertible on $S$.
Let $\mathcal{C} \subset (\Sch/S)_\etale$
be the full subcategory consisting of schemes locally of finite
type over $S$ endowed with the \'etale topology.
Let $\mathcal{O}_\mathcal{C} = \underline{\Lambda}$ be the
constant sheaf. Set $\mathcal{A}_U = \textit{Mod}(\mathcal{O}_U)$,
in other words, we consider all \'etale sheaves of $\Lambda$-modules.
Let $\mathcal{B} \subset \Ob(\mathcal{C})$
be the set of quasi-compact objects. For $V \in \mathcal{B}$ set
$$
d_V = 1 + 2\dim(S) +
\sup\nolimits_{v \in V}(\text{trdeg}_{\kappa(s)}(\kappa(v)) +
2 \dim \mathcal{O}_{V, v})
$$
and let $\text{Cov}_V$ denote the \'etale coverings $\{V_i \to V\}$
with $V_i$ quasi-compact for all $i$.
Our choice of bound $d_V$ comes from Gabber's theorem
on cohomological dimension. To see that condition (3)
holds with this choice, use 
\cite[Expos\'e VIII-A, Corollary 1.2 and Lemma 2.2]{Traveaux}
plus elementary arguments on cohomological dimensions of fields.
We add $1$ to the formula because our list contains cases where we allow $S$
to have finite residue field.
We will come back to this example later (insert future reference).
\end{example}

\noindent
Let $(\mathcal{C}, \mathcal{O}_\mathcal{C})$ be a ringed site.
Assume given weak Serre subcategories
$\mathcal{A}_U \subset \textit{Mod}(\mathcal{O}_U)$
satisfying condition (\ref{item-restriction}).
Then
\begin{enumerate}
\item given a semi-representable object $K = \{U_i\}_{i \in I}$
we get a weak Serre subcategory
$\mathcal{A}_K \subset \textit{Mod}(\mathcal{O}_K)$
by taking $\prod \mathcal{A}_{U_i} \subset 
\prod \textit{Mod}(\mathcal{O}_{U_i}) = \textit{Mod}(\mathcal{O}_K)$, and
\item given a morphism of semi-representable objects
$f : K \to L$ the pullback map
$f^* : \textit{Mod}(\mathcal{O}_L) \to \textit{Mod}(\mathcal{O}_L)$
sends $\mathcal{A}_L$ into $\mathcal{A}_K$.
\end{enumerate}
See Remark \ref{remark-semi-representable-ringed} for notation and
explanation. In particular, given a simplicial semi-representable object $K$
it is unambiguous to say what it means for an object $\mathcal{F}$ of
$\textit{Mod}(\mathcal{O})$ as in Remark \ref{remark-augmentation-ringed}
to have restrictions $\mathcal{F}_n$ in
$\mathcal{A}_{K_n}$ for all $n$.

\begin{lemma}
\label{lemma-hypercovering-equivalence-modules}
Let $(\mathcal{C}, \mathcal{O}_\mathcal{C})$ be a ringed site.
Assume given weak Serre subcategories
$\mathcal{A}_U \subset \textit{Mod}(\mathcal{O}_U)$
satisfying conditions (\ref{item-restriction}),
(\ref{item-local}), and (\ref{item-bounded-dimension}) above.
Assume $\mathcal{C}$ has equalizers and fibre products and
let $K$ be a hypercovering.
Let $((\mathcal{C}/K)_{total}, \mathcal{O})$ be as in
Remark \ref{remark-augmentation-ringed}.
Let $\mathcal{A}_{total} \subset \textit{Mod}(\mathcal{O})$
denote the weak Serre subcategory of cartesian $\mathcal{O}$-modules
$\mathcal{F}$ whose restriction $\mathcal{F}_n$ is in
$\mathcal{A}_{K_n}$ for all $n$ (as defined above).
Then the functor $La^*$ defines an equivalence
$$
D_\mathcal{A}(\mathcal{O}_\mathcal{C})
\longrightarrow
D_{\mathcal{A}_{total}}(\mathcal{O})
$$
with quasi-inverse $Ra_*$.
\end{lemma}

\begin{proof}
The cartesian $\mathcal{O}$-modules form a weak Serre subcategory by
Lemma \ref{lemma-Serre-subcat-cartesian-modules}
(the required hypotheses hold by the discussion in
Remark \ref{remark-augmentation-ringed}).
Since the restriction functor
$g_n^* : \textit{Mod}(\mathcal{O}) \to \textit{Mod}(\mathcal{O}_n)$
are exact, it follows that $\mathcal{A}_{total}$ is a weak Serre
subcategory.

\medskip\noindent
Let us show that $a^* : \mathcal{A} \to \mathcal{A}_{total}$
is an equivalence of categories with inverse given by $La_*$.
We already know that $La_*a^*\mathcal{F} = \mathcal{F}$ by the
bounded version
(Lemma \ref{lemma-hypercovering-equivalence-bounded-modules}).
It is clear that $a^*\mathcal{F}$ is in $\mathcal{A}_{total}$
for $\mathcal{F}$ in $\mathcal{A}$. Conversely, assume that
$\mathcal{G} \in \mathcal{A}_{total}$. Because $\mathcal{G}$
is cartesian we see that $\mathcal{G} = a^*\mathcal{F}$
for some $\mathcal{O}_\mathcal{C}$-module $\mathcal{F}$ by
Lemma \ref{lemma-hypercovering-descent-modules}.
We want to show that $\mathcal{F}$ is in $\mathcal{A}$.
Take $U \in \Ob(\mathcal{C})$. We have to show that the
restriction of $\mathcal{F}$ to $\mathcal{C}/U$ is in $\mathcal{A}_U$.
As usual, write $K_0 = \{U_{0, i}\}_{i \in I_0}$.
Since $K$ is a hypercovering, the map $\coprod_{i \in I_0} h_{U_{0, i}} \to *$
becomes surjective after sheafification. This implies there is
a covering $\{U_j \to U\}_{j \in J}$ and a map $\tau : J \to I_0$
and for each $j \in J$ a morphism $\varphi_j : U_j \to U_{0, \tau(j)}$.
Since $\mathcal{G}_0 = a_0^*\mathcal{F}$ we find
that the restriction of $\mathcal{F}$ to $\mathcal{C}/U_j$
is equal to the restriction of the $\tau(j)$th component of
$\mathcal{G}_0$ to $\mathcal{C}/U_j$ via the morphism
$\varphi_j : U_j \to U_{0, \tau(i)}$. Hence by
(\ref{item-restriction}) we find that $\mathcal{F}|_{\mathcal{C}/U_j}$
is in $\mathcal{A}_{U_j}$ and in turn by
(\ref{item-local}) we find that $\mathcal{F}|_{\mathcal{C}/U}$
is in $\mathcal{A}_U$.

\medskip\noindent
In particular the statement of the lemma makes sense.
The lemma now follows from Cohomology on Sites,
Lemma \ref{sites-cohomology-lemma-equivalence-unbounded-one}.
Assumption (1) is clear (see Remark \ref{remark-augmentation-ringed}).
Assumptions (2) and (3) we proved in the preceding paragraph.
Assumption (4) is immediate from (\ref{item-bounded-dimension}).
For assumption (5) let $\mathcal{B}_{total}$ be the set of
objects $U/U_{n, i}$ of the site $(\mathcal{C}/K)_{total}$
such that $U \in \mathcal{B}$ where $\mathcal{B}$ is as in
(\ref{item-bounded-dimension}). Here we use the description of
the site $(\mathcal{C}/K)_{total}$ given in
Section \ref{section-simplicial-semi-representable}.
Moreover, we set $\text{Cov}_{U/U_{n, i}}$ equal to $\text{Cov}_U$
and $d_{U/U_{n, i}}$ equal $d_U$ where $\text{Cov}_U$ and $d_U$
are given to us by (\ref{item-bounded-dimension}).
Then we claim that condition (5) holds with these choices.
This follows immediately from
Lemma \ref{lemma-sanity-check-simplicial-semi-representable}
and the fact that $\mathcal{F} \in \mathcal{A}_{total}$
implies $\mathcal{F}_n \in \mathcal{A}_{K_n}$ and hence
$\mathcal{F}_{n, i} \in \mathcal{A}_{U_{n, i}}$.
(The reader who worries about the difference between
cohomology of abelian sheaves versus cohomology
of sheaves of modules may consult Cohomology on Sites, Lemma
\ref{sites-cohomology-lemma-cohomology-modules-abelian-agree}.)
\end{proof}










\section{Glueing complexes}
\label{section-glueing-complexes}

\noindent
This section is the continuation of
Cohomology, Section \ref{cohomology-section-glueing-complexes}.
The goal is to prove a slight generalization of \cite[Theorem 3.2.4]{BBD}.
Our method will be a tiny bit different in that we use
the material from Sections \ref{section-glueing} and
\ref{section-glueing-modules}. We will also reprove the
unbounded version as it is proved in \cite{six-I}.

\medskip\noindent
Here is the situation we are interested in.

\begin{situation}
\label{situation-locally-given}
Let $(\mathcal{C}, \mathcal{O}_\mathcal{C})$ be a ringed site. We are given
\begin{enumerate}
\item a category $\mathcal{B}$ and a functor
$u : \mathcal{B} \to \mathcal{C}$,
\item an object $E_U$ in $D(\mathcal{O}_{u(U)})$ for $U \in \Ob(\mathcal{B})$,
\item an isomorphism $\rho_a : E_U|_{\mathcal{C}/u(V)} \to E_V$ in
$D(\mathcal{O}_{u(V)})$ for $a : V \to U$ in $\mathcal{B}$
\end{enumerate}
such that whenever we have composable arrows
$b : W \to V$ and $a : V \to U$ of $\mathcal{B}$, then
$\rho_{a \circ b} = \rho_b \circ \rho_a|_{\mathcal{C}/u(W)}$.
\end{situation}

\noindent
We won't be able to prove anything about this without making more
assumptions. An interesting case is where $\mathcal{B}$ is a full
subcategory such that every object of $\mathcal{C}$ has a covering
whose members are objects of $\mathcal{B}$ (this is the case considered
in \cite{BBD}). For us it is important to allow cases where this is not
the case; the main alternative case is where we have a morphism
of sites $f : \mathcal{C} \to \mathcal{D}$ and $\mathcal{B}$
is a full subcategory of $\mathcal{D}$ such that every object of
$\mathcal{D}$ has a covering whose members are objects of $\mathcal{B}$.

\medskip\noindent
In Situation \ref{situation-locally-given} a {\it solution}
will be a pair $(E, \rho_U)$ where $E$ is an object of
$D(\mathcal{O}_\mathcal{C})$
and $\rho_U : E|_{\mathcal{C}/u(U)} \to E_U$
for $U \in \Ob(\mathcal{B})$
are isomorphisms such that
we have $\rho_a \circ \rho_U|_{\mathcal{C}/u(V)} = \rho_V$
for $a : V \to U$ in $\mathcal{B}$.

\begin{lemma}
\label{lemma-prepare-bbd-glueing}
In Situation \ref{situation-locally-given}.
Assume negative self-exts of $E_U$ in $D(\mathcal{O}_{u(U)})$ are zero.
Let $L$ be a simplicial object of $\text{SR}(\mathcal{B})$.
Consider the simplicial object $K = u(L)$ of $\text{SR}(\mathcal{C})$
and let $((\mathcal{C}/K)_{total}, \mathcal{O})$ be as in
Remark \ref{remark-augmentation-ringed}.
There exists a cartesian object $E$ of $D(\mathcal{O})$
such that writing $L_n = \{U_{n, i}\}_{i \in I_n}$
the restriction of $E$ to $D(\mathcal{O}_{\mathcal{C}/u(U_{n, i})})$
is $E_{U_{n, i}}$ compatibly (see proof for details).
Moreover, $E$ is unique up to unique isomorphism.
\end{lemma}

\begin{proof}
Recall that
$\Sh(\mathcal{C}/K_n) = \prod_{i \in I_n} \Sh(\mathcal{C}/u(U_{n, i}))$
and similarly for the categories of modules. This product decomposition
is also inherited by the derived categories of sheaves of modules.
Moreover, this product decomposition is compatible with
the morphisms in the simplicial semi-representable object $K$.
See Section \ref{section-semi-representable}.
Hence we can set $E_n = \prod_{i \in I_n} E_{U_{n, i}}$
(``formal'' product) in $D(\mathcal{O}_n)$.
Taking (formal) products of the maps $\rho_a$ of
Situation \ref{situation-locally-given}
we obtain isomorphisms $E_\varphi : f_\varphi^*E_n \to E_m$.
The assumption about compostions of the maps $\rho_a$
immediately implies that $(E_n, E_\varphi)$
defines a simplicial system of the derived category of modules
as in Definition \ref{definition-cartesian-derived-modules}.
The vanishing of negative exts assumed in the lemma implies that
$\Hom(E_n[t], E_n) = 0$ for $n \geq 0$ and $t > 0$.
Thus by
Lemma \ref{lemma-cartesian-module-derived-from-simplicial}
we obtain $E$.
Uniqueness up to unique isomorphism follows from
Lemmas \ref{lemma-nullity-cartesian-modules-derived} and
\ref{lemma-hom-cartesian-modules-derived}.
\end{proof}

\begin{lemma}[BBD glueing lemma]
\label{lemma-bbd-glueing}
In Situation \ref{situation-locally-given}. Assume
\begin{enumerate}
\item $\mathcal{C}$ has equalizers and fibre products,
\item there is a morphism of sites $f : \mathcal{C} \to \mathcal{D}$
given by a continuous functor $u : \mathcal{D} \to \mathcal{C}$
such that
\begin{enumerate}
\item $\mathcal{D}$ has equalizers and fibre products and $u$
commutes with them,
\item $\mathcal{B}$ is a full subcategory of $\mathcal{D}$
and $u : \mathcal{B} \to \mathcal{C}$ is the restriction of $u$,
\item every object of $\mathcal{D}$ has a covering whose members
are objects of $\mathcal{B}$,
\end{enumerate}
\item all negative self-exts of $E_U$ in $D(\mathcal{O}_{u(U)})$ are zero, and
\item there exists a $t \in \mathbf{Z}$ such that $H^i(E_U) = 0$ for $i < t$
and $U \in \Ob(\mathcal{B})$.
\end{enumerate}
Then there exists a solution unique up to unique isomorphism.
\end{lemma}

\begin{proof}
By Hypercoverings, Lemma \ref{hypercovering-lemma-hypercovering-site}
there exists a hypercovering $L$ for the site $\mathcal{D}$ such that
$L_n = \{U_{n, i}\}_{i \in I_n}$ with $U_{i, n} \in \Ob(\mathcal{B})$.
Set $K = u(L)$. Apply Lemma \ref{lemma-prepare-bbd-glueing}
to get a cartesian object $E$ of $D(\mathcal{O})$ on the site
$(\mathcal{C}/K)_{total}$ restricting to $E_{U_{n, i}}$ on
$\mathcal{C}/u(U_{n, i})$ compatibly.
The assumption on $t$ implies that $E \in D^+(\mathcal{O})$.
By Hypercoverings, Lemma \ref{hypercovering-lemma-hypercovering-morphism-sites}
we see that $K$ is a hypercovering too.
By Lemma \ref{lemma-hypercovering-equivalence-bounded-modules}
we find that $E = a^*F$ for some $F$ in $D^+(\mathcal{O}_\mathcal{C})$.

\medskip\noindent
To prove that $F$ is a solution we will use the construction of
$L_0$ and $L_1$ given in the proof of
Hypercoverings, Lemma \ref{hypercovering-lemma-hypercovering-site}.
(This is a bit inelegant but there does not seem to be a completely
straightforward way around it.)

\medskip\noindent
Namely, we have $I_0 = \Ob(\mathcal{B})$ and so
$L_0 = \{U\}_{U \in \Ob(\mathcal{B})}$.
Hence the isomorphism $a^*F \to E$ restricted to the components
$\mathcal{C}/u(U)$ of $\mathcal{C}/K_0$ defines isomorphisms
$\rho_U : F|_{\mathcal{C}/u(U)} \to E_U$ for $U \in \Ob(\mathcal{B})$
by our choice of $E$.

\medskip\noindent
To prove that $\rho_U$ satisfy the requirement of compatibility
with the maps $\rho_a$ of Situation \ref{situation-locally-given}
we use that $I_1$ contains the set
$$
\Omega =
\{(U, V, W, a, b) \mid U, V, W \in \mathcal{B}, a : U \to V, b : U \to W\}
$$
and that for $i = (U, V, W, a, b)$ in $\Omega$ we have
$U_{1, i} = U$. Moreover, the component maps $f_{\delta^1_0, i}$ and
$f_{\delta^1_1, i}$ of the two morphisms $K_1 \to K_0$ are the morphisms
$$
a : U \to V \quad\text{and}\quad b : U \to V
$$
Hence the compatibility mentioned in
Lemma \ref{lemma-prepare-bbd-glueing} gives that
$$
\rho_a \circ \rho_V|_{\mathcal{C}/u(U)} = \rho_U
\quad\text{and}\quad
\rho_b \circ \rho_W|_{\mathcal{C}/u(U)} = \rho_U
$$
Taking $i = (U, V, U, a, \text{id}_U) \in \Omega$ for example, we find
that we have the desired compatibility. The uniqueness of $F$ follows
from the uniqueness of $E$ in the previous lemma (small detail omitted).
\end{proof}

\begin{lemma}[Unbounded BBD glueing lemma]
\label{lemma-bbd-unbounded-glueing}
In Situation \ref{situation-locally-given}. Assume
\begin{enumerate}
\item $\mathcal{C}$ has equalizers and fibre products,
\item there is a morphism of sites $f : \mathcal{C} \to \mathcal{D}$
given by a continuous functor $u : \mathcal{D} \to \mathcal{C}$
such that
\begin{enumerate}
\item $\mathcal{D}$ has equalizers and fibre products and $u$
commutes with them,
\item $\mathcal{B}$ is a full subcategory of $\mathcal{D}$
and $u : \mathcal{B} \to \mathcal{C}$ is the restriction of $u$,
\item every object of $\mathcal{D}$ has a covering whose members
are objects of $\mathcal{B}$,
\end{enumerate}
\item all negative self-exts of $E_U$ in $D(\mathcal{O}_{u(U)})$ are zero, and
\item there exist weak Serre subcategories
$\mathcal{A}_U \subset \textit{Mod}(\mathcal{O}_U)$ for all
$U \in \Ob(\mathcal{C})$ satisfying conditions (\ref{item-restriction}),
(\ref{item-local}), and (\ref{item-bounded-dimension}),
\item $E_U \in D_{\mathcal{A}_U}(\mathcal{O}_U)$.
\end{enumerate}
Then there exists a solution unique up to unique isomorphism.
\end{lemma}

\begin{proof}
The proof is {\bf exactly} the same as the proof of
Lemma \ref{lemma-bbd-glueing}. The only change is that
$E$ is an object of $D_{\mathcal{A}_{total}}(\mathcal{O})$
and hence we use Lemma \ref{lemma-hypercovering-equivalence-modules}
to obtain $F$ with $E = a^*F$
instead of Lemma \ref{lemma-hypercovering-equivalence-bounded-modules}.
\end{proof}





\section{Proper hypercoverings in topology}
\label{section-proper-hypercovering}

\noindent
Let's work in the category $\textit{LC}$ of Hausdorff and locally
quasi-compact topological spaces and continuous maps, see
Cohomology on Sites, Section \ref{sites-cohomology-section-cohomology-LC}.
Let $X$ be an object of $\textit{LC}$ and let $U$ be a simplicial
object of $\textit{LC}$. Assume we have an augmentation
$$
a : U \to X
$$
We say that $U$ is a {\it proper hypercovering} of $X$ if
\begin{enumerate}
\item $U_0 \to X$ is a proper surjective map,
\item $U_1 \to U_0 \times_X U_0$ is a proper surjective map,
\item $U_{n + 1} \to (\text{cosk}_n\text{sk}_n U)_{n + 1}$
is a proper surjective map for $n \geq 1$.
\end{enumerate}
The category $\textit{LC}$ has all finite limits, hence the
coskeleta used in the formulation above exist.
$$
\fbox{Principle: Proper hypercoverings can be used to compute cohomology.}
$$
A key idea behind the proof of the principle is to find a topology
on $\textit{LC}$ which is stronger than the usual one such that
(a) a surjective proper map defines a covering, and
(b) cohomology of usual sheaves with respect to this stronger
topology agrees with the usual cohomology.
Properties (a) and (b) hold for the qc topology, see
Cohomology on Sites, Section \ref{sites-cohomology-section-cohomology-LC}.
Once we have (a) and (b) we deduce the principle via
the earlier work done in this chapter.

\begin{lemma}
\label{lemma-compare-simplicial-objects}
Let $U$ be a simplicial object of $\textit{LC}$ and let $a : U \to X$
be an augmentation. There is a commutative diagram
$$
\xymatrix{
\Sh((\textit{LC}_{qc}/U)_{total}) \ar[r]_-h \ar[d]_{a_{qc}} &
\Sh(U_{Zar}) \ar[d]^a \\
\Sh(\textit{LC}_{qc}/X) \ar[r]^-{h_{-1}} &
\Sh(X)
}
$$
where the left vertical arrow is defined in
Section \ref{section-hypercovering}
and the right vertical arrow is defined in
Lemma \ref{lemma-augmentation}.
\end{lemma}

\begin{proof}
Write $\Sh(X) = \Sh(X_{Zar})$. Observe that both
$(\textit{LC}_{qc}/U)_{total}$ and $U_{Zar}$ fall
into case A of Situation \ref{situation-simplicial-site}.
This is immediate from the construction of
$U_{Zar}$ in Section \ref{section-simplicial-top}
and it follows from Lemma \ref{lemma-sr-when-fibre-products}
for $(\textit{LC}_{qc}/U)_{total}$.
Next, consider the functors
$U_{n, Zar} \to \textit{LC}_{qc}/U_n$, $U \mapsto U/U_n$
and
$X_{Zar} \to \textit{LC}_{qc}/X$, $U \mapsto U/X$.
We have seen that these define morphisms of sites
in Cohomology on Sites, Section \ref{sites-cohomology-section-cohomology-LC}.
Thus we obtain a morphism of simplicial sites compatible with
augmentations as in Remark \ref{remark-morphism-augmentation-simplicial-sites}
and we may apply
Lemma \ref{lemma-morphism-augmentation-simplicial-sites} to conclude.
\end{proof}

\begin{lemma}
\label{lemma-descent-sheaves-for-proper-hypercovering}
Let $U$ be a simplicial object of $\textit{LC}$ and let $a : U \to X$
be an augmentation. If $a : U \to X$ gives a proper hypercovering of $X$,
then
$$
a^{-1} : \Sh(X) \to \Sh(U_{Zar})
\quad\text{and}\quad
a^{-1} : \textit{Ab}(X) \to \textit{Ab}(U_{Zar})
$$
are fully faithful with essential image the cartesian sheaves and
quasi-inverse given by $a_*$. Here $a : \Sh(U_{Zar}) \to \Sh(X)$ is as in
Lemma \ref{lemma-augmentation}.
\end{lemma}

\begin{proof}
We will prove the statement for sheaves of sets. It will be an
almost formal consequence of results already established.
Consider the diagram of Lemma \ref{lemma-compare-simplicial-objects}.
By Cohomology on Sites, Lemma \ref{sites-cohomology-lemma-describe-pullback-pi}
the functor $(h_{-1})^{-1}$ is fully faithful with quasi-inverse $h_{-1, *}$.
The same holds true for the components $h_n$ of $h$.
By the description of the functors $h^{-1}$ and $h_*$ of
Lemma \ref{lemma-morphism-simplicial-sites}
we conclude that $h^{-1}$ is fully faithful with quasi-inverse $h_*$.
Observe that $U$ is a hypercovering of $X$ in $\textit{LC}_{qc}$
(as defined in Section \ref{section-hypercovering}) by
Cohomology on Sites, Lemma
\ref{sites-cohomology-lemma-proper-surjective-is-qc-covering}.
By Lemma \ref{lemma-hypercovering-X-simple-descent-sheaves}
we see that $a_{qc}^{-1}$ is fully faithful with quasi-inverse $a_{qc, *}$
and with essential image the cartesian sheaves on
$(\textit{LC}_{qc}/U)_{total}$.
A formal argument (chasing around the diagram) now shows that
$a^{-1}$ is fully faithful.

\medskip\noindent
Finally, suppose that $\mathcal{G}$ is a cartesian sheaf on $U_{Zar}$.
Then $h^{-1}\mathcal{G}$ is a cartesian sheaf on $\textit{LC}_{qc}/U$.
Hence $h^{-1}\mathcal{G} = a_{qc}^{-1}\mathcal{H}$ for some sheaf
$\mathcal{H}$ on $\textit{LC}_{qc}/X$.
We compute
\begin{align*}
(h_{-1})^{-1}(a_*\mathcal{G})
& =
(h_{-1})^{-1}
\text{Eq}(
\xymatrix{
a_{0, *}\mathcal{G}_0
\ar@<1ex>[r] \ar@<-1ex>[r] &
a_{1, *}\mathcal{G}_1
}
) \\
& =
\text{Eq}(
\xymatrix{
(h_{-1})^{-1}a_{0, *}\mathcal{G}_0
\ar@<1ex>[r] \ar@<-1ex>[r] &
(h_{-1})^{-1}a_{1, *}\mathcal{G}_1
}
) \\
& =
\text{Eq}(
\xymatrix{
a_{qc, 0, *}h_0^{-1}\mathcal{G}_0
\ar@<1ex>[r] \ar@<-1ex>[r] &
a_{qc, 1, *}h_1^{-1}\mathcal{G}_1
}
) \\
& =
\text{Eq}(
\xymatrix{
a_{qc, 0, *}a_{qc, 0}^{-1}\mathcal{H}
\ar@<1ex>[r] \ar@<-1ex>[r] &
a_{qc, 1, *}a_{qc, 1}^{-1}\mathcal{H}
}
) \\
& =
a_{qc, *}a_{qc}^{-1}\mathcal{H} \\
& =
\mathcal{H}
\end{align*}
Here the first equality follows from Lemma \ref{lemma-augmentation},
the second equality follows as $(h_{-1})^{-1}$ is an exact functor,
the third equality follows from
Cohomology on Sites, Lemma \ref{sites-cohomology-lemma-push-pull-LC}
(here we use that $a_0 : U_0 \to X$ and $a_1: U_1 \to X$ are proper),
the fourth follows from $a_{qc}^{-1}\mathcal{H} = h^{-1}\mathcal{G}$,
the fifth from Lemma \ref{lemma-augmentation-site}, and the
sixth we've seen above. Since $a_{qc}^{-1}\mathcal{H} = h^{-1}\mathcal{G}$
we deduce that $h^{-1}\mathcal{G} \cong h^{-1}a^{-1}a_*\mathcal{G}$
which ends the proof by fully faithfulness of $h^{-1}$.
\end{proof}

\begin{lemma}
\label{lemma-cohomological-descent-for-proper-hypercovering}
Let $U$ be a simplicial object of $\textit{LC}$ and let $a : U \to X$
be an augmentation. If $a : U \to X$ gives a proper hypercovering of $X$,
then for $K \in D^+(X)$
$$
K \to Ra_*(a^{-1}K)
$$
is an isomorphism where $a : \Sh(U_{Zar}) \to \Sh(X)$ is as in
Lemma \ref{lemma-augmentation}.
\end{lemma}

\begin{proof}
Consider the diagram of Lemma \ref{lemma-compare-simplicial-objects}.
Observe that $Rh_{n, *}h_n^{-1}$ is the identity functor
on $D^+(U_n)$ by Cohomology on Sites, Lemma
\ref{sites-cohomology-lemma-cohomological-descent-LC}.
Hence $Rh_*h^{-1}$ is the identity functor on
$D^+(U_{Zar})$ by
Lemma \ref{lemma-direct-image-morphism-simplicial-sites}.
We have
\begin{align*}
Ra_*(a^{-1}K)
& =
Ra_*Rh_*h^{-1}a^{-1}K \\
& =
Rh_{-1, *}Ra_{qc, *}a_{qc}^{-1}(h_{-1})^{-1}K \\
& =
Rh_{-1, *}(h_{-1})^{-1}K \\
& =
K
\end{align*}
The first equality by the discussion above, the second equality
because of the commutativity of the diagram in
Lemma \ref{lemma-compare-simplicial-objects}, the third equality by
Lemma \ref{lemma-hypercovering-X-simple-descent-bounded-abelian}
($U$ is a hypercovering of $X$ in $\textit{LC}_{qc}$ by
Cohomology on Sites, Lemma
\ref{sites-cohomology-lemma-proper-surjective-is-qc-covering}),
and the last equality by the already used Cohomology on Sites, Lemma
\ref{sites-cohomology-lemma-cohomological-descent-LC}.
\end{proof}

\begin{lemma}
\label{lemma-compute-via-proper-hypercovering}
Let $U$ be a simplicial object of $\textit{LC}$ and let $a : U \to X$
be an augmentation. If $U$ is a proper hypercovering of $X$, then
$$
R\Gamma(X, K) = R\Gamma(U_{Zar}, a^{-1}K)
$$
for $K \in D^+(X)$ where $a : \Sh(U_{Zar}) \to \Sh(X)$
is as in Lemma \ref{lemma-augmentation}.
\end{lemma}

\begin{proof}
This follows from
Lemma \ref{lemma-cohomological-descent-for-proper-hypercovering}
because $R\Gamma(U_{Zar}, -) = R\Gamma(X, -) \circ Ra_*$ by
Cohomology on Sites, Remark \ref{sites-cohomology-remark-before-Leray}.
\end{proof}

\begin{lemma}
\label{lemma-proper-hypercovering-equivalence-bounded}
Let $U$ be a simplicial object of $\textit{LC}$ and let $a : U \to X$
be an augmentation.
Let $\mathcal{A} \subset \textit{Ab}(U_{Zar})$
denote the weak Serre subcategory of cartesian abelian sheaves.
If $U$ is a proper hypercovering of $X$, then
the functor $a^{-1}$ defines an equivalence
$$
D^+(X) \longrightarrow D_\mathcal{A}^+(U_{Zar})
$$
with quasi-inverse $Ra_*$ where $a : \Sh(U_{Zar}) \to \Sh(X)$
is as in Lemma \ref{lemma-augmentation}.
\end{lemma}

\begin{proof}
Observe that $\mathcal{A}$ is a weak Serre subcategory by
Lemma \ref{lemma-Serre-subcat-cartesian-modules}.
The equivalence is a
formal consequence of the results obtained so far. Use
Lemmas \ref{lemma-descent-sheaves-for-proper-hypercovering} and
\ref{lemma-cohomological-descent-for-proper-hypercovering} and
Cohomology on Sites, Lemma \ref{sites-cohomology-lemma-equivalence-bounded}.
\end{proof}

\begin{lemma}
\label{lemma-spectral-sequence-proper-hypercovering}
Let $U$ be a simplicial object of $\textit{LC}$ and let
$a : U \to X$ be an augmentation. Let $\mathcal{F}$ be an abelian sheaf
on $X$. Let $\mathcal{F}_n$ be the pullback to $U_n$.
If $U$ is a proper hypercovering of $X$, then
there exists a canonical spectral sequence
$$
E_1^{p, q} = H^q(U_p, \mathcal{F}_p)
$$
converging to $H^{p + q}(X, \mathcal{F})$.
\end{lemma}

\begin{proof}
Immediate consequence of Lemmas \ref{lemma-compute-via-proper-hypercovering}
and \ref{lemma-simplicial-sheaf-cohomology}.
\end{proof}




\section{Simplicial schemes}
\label{section-simplicial}

\noindent
A {\it simplicial scheme} is a simplicial object in the category of schemes,
see Simplicial, Definition \ref{simplicial-definition-simplicial-object}.
Recall that a simplicial scheme looks like
$$
\xymatrix{
X_2
\ar@<2ex>[r]
\ar@<0ex>[r]
\ar@<-2ex>[r]
&
X_1
\ar@<1ex>[r]
\ar@<-1ex>[r]
\ar@<1ex>[l]
\ar@<-1ex>[l]
&
X_0
\ar@<0ex>[l]
}
$$
Here there are two morphisms $d^1_0, d^1_1 : X_1 \to X_0$
and a single morphism $s^0_0 : X_0 \to X_1$, etc.
These morphisms satisfy some required relations such as
$d^1_0 \circ s^0_0 = \text{id}_{X_0} = d^1_1 \circ s^0_0$, see
Simplicial, Lemma \ref{simplicial-lemma-characterize-simplicial-object}.
It is useful to think of $d^n_i : X_n \to X_{n - 1}$
as the ``projection forgetting the $i$th coordinate'' and
to think of $s^n_j : X_n \to X_{n + 1}$ as the ``diagonal map repeating
the $j$th coordinate''.

\medskip\noindent
A {\it morphism of simplicial schemes} $h : X \to Y$ is the same
thing as a morphism of simplicial objects in the category of schemes,
see Simplicial, Definition \ref{simplicial-definition-simplicial-object}.
Thus $h$ consists of morphisms of schemes $h_n : X_n \to Y_n$
such that $h_{n - 1} \circ d^n_j = d^n_j \circ h_n$ and
$h_{n + 1} \circ s^n_j = s^n_j \circ h_n$ whenever this makes sense.

\medskip\noindent
An {\it augmentation} of a simplicial scheme $X$ is a morphism
of schemes $a_0 : X_0 \to S$ such that $a_0 \circ d^1_0 = a_0 \circ d^1_1$.
See Simplicial, Section \ref{simplicial-section-augmentation}.

\medskip\noindent
Let $X$ be a simplicial scheme. The construction of
Section \ref{section-simplicial-top} applied to the underlying
simplicial topological space gives a site $X_{Zar}$.
On the other hand, for every $n$ we have the small Zariski site
$X_{n, Zar}$ (Topologies, Definition
\ref{topologies-definition-big-small-Zariski})
and for every morphism $\varphi : [m] \to [n]$
we have a morphism of sites
$f_\varphi = X(\varphi)_{small} : X_{n, Zar} \to X_{m, Zar}$,
associated to the morphism of schemes
$X(\varphi) : X_n \to X_m$ (Topologies, Lemma
\ref{topologies-lemma-morphism-big-small}).
This gives a simplicial object $\mathcal{C}$ in the category of sites.
In Lemma \ref{lemma-simplicial-site-site} we constructed an associated
site $\mathcal{C}_{total}$. Assigning to an open immersion its image
defines an equivalence $\mathcal{C}_{total} \to X_{Zar}$ which
identifies sheaves, i.e., $\Sh(\mathcal{C}_{total}) = \Sh(X_{Zar})$.
The difference between $\mathcal{C}_{total}$ and $X_{Zar}$
is similar to the difference between the small Zariski site $S_{Zar}$
and the underlying topological space of $S$.
We will silently identify these sites in what follows.

\medskip\noindent
Let $X_{Zar}$ be the site associated to a simplicial scheme $X$.
There is a sheaf of rings $\mathcal{O}$ on $X_{Zar}$ whose restriction
to $X_n$ is the structure sheaf $\mathcal{O}_{X_n}$. This follows
from Lemma \ref{lemma-describe-sheaves-simplicial-site} or from
Lemma \ref{lemma-describe-sheaves-simplicial-site-site}. We will say
{\it $\mathcal{O}$ is the structure sheaf of the simplicial scheme $X$}.
At this point all the material developed for simplicial (ringed) sites
applies, see Sections \ref{section-simplicial-sites},
\ref{section-augmentation-simplicial-sites},
\ref{section-morphism-simplicial-sites},
\ref{section-simplicial-sites-modules},
\ref{section-cohomology-simplicial-sites},
\ref{section-cohomology-augmentation-simplicial-sites},
\ref{section-cohomology-simplicial-sites-modules},
\ref{section-cohomology-augmentation-ringed-simplicial-sites},
\ref{section-cartesian},
\ref{section-glueing}, and
\ref{section-glueing-modules}.

\medskip\noindent
Let $X$ be a simplicial scheme with structure sheaf $\mathcal{O}$.
As on any ringed topos, there is a notion
of a {\it quasi-coherent $\mathcal{O}$-module on $X_{Zar}$}, see
Modules on Sites, Definition \ref{sites-modules-definition-site-local}.
However, a quasi-coherent $\mathcal{O}$-module on $X_{Zar}$ is
just a cartesian $\mathcal{O}$-module $\mathcal{F}$ whose restrictions
$\mathcal{F}_n$ are quasi-coherent on $X_n$, see
Lemma \ref{lemma-quasi-coherent-sheaf}.

\medskip\noindent
Let $h : X \to Y$ be a morphism of simplicial schemes. Either by
Lemma \ref{lemma-simplicial-space-site-functorial} or by
(the proof of) Lemma \ref{lemma-morphism-simplicial-sites}
we obtain a morphism of sites $h_{Zar} : X_{Zar} \to Y_{Zar}$.
Recall that $h_{Zar}^{-1}$ and $h_{Zar, *}$ have a simple
description in terms of the components, see
Lemma \ref{lemma-describe-functoriality} or
Lemma \ref{lemma-morphism-simplicial-sites}.
Let $\mathcal{O}_X$, resp.\ $\mathcal{O}_Y$ denote the structure
sheaf of $X$, resp.\ $Y$. We define
$h_{Zar}^\sharp : h_{Zar, *}\mathcal{O}_X \to \mathcal{O}_Y$
to be the map of sheaves of rings on $Y_{Zar}$ given by
$h_n^\sharp : h_{n, *}\mathcal{O}_{X_n} \to \mathcal{O}_{Y_n}$ on $Y_n$.
We obtain a morphism of ringed sites
$$
h_{Zar} : (X_{Zar}, \mathcal{O}_X) \longrightarrow (Y_{Zar}, \mathcal{O}_Y)
$$

\medskip\noindent
Let $X$ be a simplicial scheme with structure sheaf $\mathcal{O}$.
Let $S$ be a scheme and let $a_0 : X_0 \to S$ be an augmentation of $X$.
Either by
Lemma \ref{lemma-augmentation} or by
Lemma \ref{lemma-augmentation-site}
we obtain a corresponding morphism of topoi $a : \Sh(X_{Zar}) \to \Sh(S)$.
Observe that $a^{-1}\mathcal{G}$ is the sheaf on $X_{Zar}$ with components
$a_n^{-1}\mathcal{G}$. Hence we can use the maps
$a_n^\sharp : a_n^{-1}\mathcal{O}_S \to \mathcal{O}_{X_n}$ to define
a map $a^\sharp : a^{-1}\mathcal{O}_S \to \mathcal{O}$, or equivalently
by adjunction a map $a^\sharp : \mathcal{O}_S \to a_*\mathcal{O}$
(which as usual has the same name). This puts us in the situation
discussed in
Section \ref{section-cohomology-augmentation-ringed-simplicial-sites}.
Therefore we obtain a morphism of ringed topoi
$$
a : (\Sh(X_{Zar}), \mathcal{O}) \longrightarrow (\Sh(S), \mathcal{O}_S)
$$

\medskip\noindent
A final observation is the following. Suppose we are given a morphism
$h : X \to Y$ of simplicial schemes $X$ and $Y$ with structure sheaves
$\mathcal{O}_X$, $\mathcal{O}_Y$, augmentations
$a_0 : X_0 \to X_{-1}$, $b_0 : Y_0 \to Y_{-1}$ and a morphism
$h_{-1} : X_{-1} \to Y_{-1}$ such that
$$
\xymatrix{
X_0 \ar[r]_{h_0} \ar[d]_{a_0} & Y_0 \ar[d]^{b_0} \\
X_{-1} \ar[r]^{h_{-1}} & Y_{-1}
}
$$
commutes. Then from the constructions elucidated above
we obtain a commutative diagram of morphisms of ringed topoi as follows
$$
\xymatrix{
(\Sh(X_{Zar}), \mathcal{O}_X) \ar[r]_{h_{Zar}} \ar[d]_a &
(\Sh(Y_{Zar}), \mathcal{O}_Y) \ar[d]^b \\
(\Sh(X_{-1}), \mathcal{O}_{X_{-1}}) \ar[r]^{h_{-1}} &
(\Sh(Y_{-1}), \mathcal{O}_{Y_{-1}})
}
$$








\section{Descent in terms of simplicial schemes}
\label{section-simplicial-descent}

\noindent
Cartesian morphisms are defined as follows.

\begin{definition}
\label{definition-cartesian-morphism}
Let $a : Y \to X$ be a morphism of simplicial schemes.
We say $a$ is {\it cartesian}, or that {\it $Y$ is cartesian over $X$},
if for every morphism $\varphi : [n] \to [m]$ of $\Delta$ the corresponding
diagram
$$
\xymatrix{
Y_m \ar[r]_a \ar[d]_{Y(\varphi)} & X_m \ar[d]^{X(\varphi)}\\
Y_n \ar[r]^{a} & X_n
}
$$
is a fibre square in the category of schemes.
\end{definition}

\noindent
Cartesian morphisms are related to descent data. First we prove a general
lemma describing the category of cartesian simplicial schemes over a
fixed simplicial scheme. In this lemma we denote $f^* : \Sch/X \to \Sch/Y$
the base change functor associated to a morphism of schemes $f :Y \to X$.

\begin{lemma}
\label{lemma-characterize-cartesian-schemes}
Let $X$ be a simplicial scheme. The category of simplicial schemes cartesian
over $X$ is equivalent to the category of pairs $(V, \varphi)$
where $V$ is a scheme over $X_0$ and
$$
\varphi :
V \times_{X_0, d^1_1} X_1
\longrightarrow
X_1 \times_{d^1_0, X_0} V
$$
is an isomorphism over $X_1$ such that
$(s_0^0)^*\varphi = \text{id}_V$ and such that
$$
(d^2_1)^*\varphi = (d^2_0)^*\varphi \circ (d^2_2)^*\varphi
$$
as morphisms of schemes over $X_2$.
\end{lemma}

\begin{proof}
The statement of the displayed equality makes sense because
$d^1_1 \circ d^2_2 = d^1_1 \circ d^2_1$,
$d^1_1 \circ d^2_0 = d^1_0 \circ d^2_2$, and
$d^1_0 \circ d^2_0 = d^1_0 \circ d^2_1$ as morphisms $X_2 \to X_0$, see
Simplicial, Remark \ref{simplicial-remark-relations} hence we
can picture these maps as follows
$$
\xymatrix{
&
X_2 \times_{d^1_1 \circ d^2_0, X_0} V
\ar[r]_-{(d^2_0)^*\varphi} &
X_2 \times_{d^1_0 \circ d^2_0, X_0} V
\ar@{=}[rd] & \\
X_2 \times_{d^1_0 \circ d^2_2, X_0} V
\ar@{=}[ru] & & &
X_2 \times_{d^1_0 \circ d^2_1, X_0} V \\
&
X_2 \times_{d^1_1 \circ d^2_2, X_0} V
\ar[lu]^{(d^2_2)^*\varphi} \ar@{=}[r] &
X_2 \times_{d^1_1 \circ d^2_1, X_0} V
\ar[ru]_{(d^2_1)^*\varphi}
}
$$
and the condition signifies the diagram is commutative. It is clear that
given a simplicial scheme $Y$ cartesian over $X$ we can
set $V = Y_0$ and $\varphi$ equal to the composition
$$
V \times_{X_0, d^1_1} X_1 =
Y_0 \times_{X_0, d^1_1} X_1 = Y_1 =
X_1 \times_{X_0, d^1_0} Y_0 =
X_1 \times_{X_0, d^1_0} V
$$
of identifications given by the cartesian structure. To prove this functor
is an equivalence we construct a quasi-inverse. The construction of
the quasi-inverse is analogous to the construction discussed in
Descent, Section \ref{descent-section-descent-modules} from which we borrow
the notation $\tau^n_i : [0] \to [n]$, $0 \mapsto i$ and
$\tau^n_{ij} : [1] \to [n]$, $0 \mapsto i$, $1 \mapsto j$.
Namely, given a pair $(V, \varphi)$
as in the lemma we set $Y_n = X_n \times_{X(\tau^n_n), X_0} V$.
Then given $\beta : [n] \to [m]$ we define
$V(\beta) : Y_m \to Y_n$ as the pullback by $X(\tau^m_{\beta(n)m})$
of the map $\varphi$ postcomposed by the projection
$X_m \times_{X(\beta), X_n} Y_n \to Y_n$. This makes sense because
$$
X_m \times_{X(\tau^m_{\beta(n)m}), X_1} X_1 \times_{d^1_1, X_0} V
=
X_m \times_{X(\tau^m_m), X_0} V = Y_m
$$
and
$$
X_m \times_{X(\tau^m_{\beta(n)m}), X_1} X_1 \times_{d^1_0, X_0} V =
X_m \times_{X(\tau^m_{\beta(n)}), X_0} V =
X_m \times_{X(\beta), X_n} Y_n.
$$
We omit the verification that the commutativity
of the displayed diagram
above implies the maps compose correctly. We also omit the verification
that the two functors are quasi-inverse to each other.
\end{proof}

\begin{definition}
\label{definition-fibre-products-simplicial-scheme}
Let $f : X \to S$ be a morphism of schemes. The {\it simplicial scheme
associated to $f$}, denoted $(X/S)_\bullet$, is the functor
$\Delta^{opp} \to \Sch$, $[n] \mapsto X \times_S \ldots \times_S X$
described in
Simplicial, Example \ref{simplicial-example-fibre-products-simplicial-object}.
\end{definition}

\noindent
Thus $(X/S)_n$ is the $(n + 1)$-fold fibre product of $X$ over $S$.
The morphism $d^1_0 : X \times_S X \to X$ is the map
$(x_0, x_1) \mapsto x_1$ and the morphism $d^1_1$ is the other
projection. The morphism $s^0_0$ is the diagonal morphism
$X \to X \times_S X$.

\begin{lemma}
\label{lemma-cartesian-over}
Let $f : X \to S$ be a morphism of schemes.
Let $\pi : Y \to (X/S)_\bullet$ be a cartesian morphism
of simplicial schemes.
Set $V = Y_0$ considered as a scheme over $X$.
The morphisms $d^1_0, d^1_1 : Y_1 \to Y_0$ and the morphism
$\pi_1 : Y_1 \to X \times_S X$ induce isomorphisms
$$
\xymatrix{
V \times_S X & &
Y_1 \ar[ll]_-{(d^1_1, \text{pr}_1 \circ \pi_1)}
\ar[rr]^-{(\text{pr}_0 \circ \pi_1, d^1_0)} & &
X \times_S V.
}
$$
Denote $\varphi : V \times_S X \to X \times_S V$ the
resulting isomorphism.
Then the pair $(V, \varphi)$ is a descent datum relative
to $X \to S$.
\end{lemma}

\begin{proof}
This is a special case of (part of)
Lemma \ref{lemma-characterize-cartesian-schemes}
as the displayed equation of that lemma is
equivalent to the cocycle condition of
Descent, Definition \ref{descent-definition-descent-datum}.
\end{proof}

\begin{lemma}
\label{lemma-cartesian-equivalent-descent-datum}
Let $f : X \to S$ be a morphism of schemes. The construction
$$
\begin{matrix}
\text{category of cartesian } \\
\text{schemes over } (X/S)_\bullet
\end{matrix}
\longrightarrow
\begin{matrix}
\text{ category of descent data} \\
\text{ relative to } X/S
\end{matrix}
$$
of Lemma \ref{lemma-cartesian-over}
is an equivalence of categories.
\end{lemma}

\begin{proof}
The functor from left to right is given in
Lemma \ref{lemma-cartesian-over}.
Hence this is a special case of
Lemma \ref{lemma-characterize-cartesian-schemes}.
\end{proof}

\noindent
We may reinterpret the pullback of
Descent, Lemma \ref{descent-lemma-pullback} as follows.
Suppose given a morphism of simplicial schemes $f : X' \to X$ and a
cartesian morphism of simplicial schemes $Y \to X$. Then
the fibre product (viewed as a ``pullback'')
$$
f^*Y = Y \times_X X'
$$
of simplicial schemes is a simplicial scheme cartesian over $X'$.
Suppose given a commutative diagram of morphisms of schemes
$$
\xymatrix{
X' \ar[r]_f \ar[d] & X \ar[d] \\
S' \ar[r] & S.
}
$$
This gives rise to a morphism of simplicial schemes
$$
f_\bullet : (X'/S')_\bullet \longrightarrow (X/S)_\bullet.
$$
We claim that the ``pullback'' $f_\bullet^*$ along the morphism
$f_\bullet : (X'/S')_\bullet \to (X/S)_\bullet$ corresponds via
Lemma \ref{lemma-cartesian-equivalent-descent-datum}
with the pullback defined in terms of descent data in
the aforementioned
Descent, Lemma \ref{descent-lemma-pullback}.







\section{Quasi-coherent modules on simplicial schemes}
\label{section-modules-simplicial}

\begin{lemma}
\label{lemma-pullback-cartesian-module}
Let $f : V \to U$ be a morphism of simplicial schemes. Given a
quasi-coherent module $\mathcal{F}$ on $U_{Zar}$ the pullback
$f^*\mathcal{F}$ is a quasi-coherent module on $V_{Zar}$.
\end{lemma}

\begin{proof}
Recall that $\mathcal{F}$ is cartesian with $\mathcal{F}_n$
quasi-coherent, see Lemma \ref{lemma-quasi-coherent-sheaf}.
By Lemma \ref{lemma-describe-functoriality} we see that
$(f^*\mathcal{F})_n = f_n^*\mathcal{F}_n$ (some details omitted).
Hence $(f^*\mathcal{F})_n$ is quasi-coherent.
The same fact and the cartesian property for $\mathcal{F}$
imply the cartesian property for $f^*\mathcal{F}$.
Thus $\mathcal{F}$ is quasi-coherent by
Lemma \ref{lemma-quasi-coherent-sheaf} again.
\end{proof}

\begin{lemma}
\label{lemma-pushforward-cartesian-module}
Let $f : V \to U$ be a cartesian morphism of simplicial schemes.
Assume the morphisms $d^n_j : U_n \to U_{n - 1}$ are
flat and the morphisms $V_n \to U_n$ are quasi-compact and quasi-separated.
For a quasi-coherent module $\mathcal{G}$ on $V_{Zar}$
the pushforward $f_*\mathcal{G}$ is a quasi-coherent module on $U_{Zar}$.
\end{lemma}

\begin{proof}
If $\mathcal{F} = f_* \mathcal{G}$, then
$\mathcal{F}_n = f_{n , *}\mathcal{G}_n$ by
Lemma \ref{lemma-describe-functoriality}.
The maps $\mathcal{F}(\varphi)$ are defined using the base change maps, see
Cohomology, Section \ref{cohomology-section-base-change-map}.
The sheaves $\mathcal{F}_n$ are quasi-coherent by
Schemes, Lemma \ref{schemes-lemma-push-forward-quasi-coherent}
and the fact that $\mathcal{G}_n$ is quasi-coherent
by Lemma \ref{lemma-quasi-coherent-sheaf}.
The base change maps along the degeneracies
$d^n_j$ are isomorphisms by Cohomology of Schemes, Lemma
\ref{coherent-lemma-flat-base-change-cohomology}
and the fact that $\mathcal{G}$ is cartesian
by Lemma \ref{lemma-quasi-coherent-sheaf}.
Hence $\mathcal{F}$ is cartesian by
Lemma \ref{lemma-check-cartesian-module}.
Thus $\mathcal{F}$ is quasi-coherent by
Lemma \ref{lemma-quasi-coherent-sheaf}.
\end{proof}

\begin{lemma}
\label{lemma-adjoint-functors-cartesian-modules}
Let $f : V \to U$ be a cartesian morphism of
simplicial schemes. Assume the morphisms $d^n_j : U_n \to U_{n - 1}$ are
flat and the morphisms $V_n \to U_n$ are quasi-compact and quasi-separated.
Then $f^*$ and $f_*$ form an adjoint pair of functors
between the categories of quasi-coherent modules on $U_{Zar}$ and $V_{Zar}$.
\end{lemma}

\begin{proof}
We have seen in Lemmas \ref{lemma-pullback-cartesian-module} and
\ref{lemma-pushforward-cartesian-module}
that the statement makes sense. The adjointness property follows
immediately from the fact that each $f_n^*$ is adjoint to $f_{n, *}$.
\end{proof}

\begin{lemma}
\label{lemma-cartesian-modules-with-section}
Let $f : X \to S$ be a morphism of schemes which has a
section\footnote{In fact, it would be enough to assume that $f$
has fpqc locally on $S$ a section, since we have descent of
quasi-coherent modules by Descent,
Section \ref{descent-section-fpqc-descent-quasi-coherent}.}.
Let $(X/S)_\bullet$ be the simplicial
scheme associated to $X \to S$, see
Definition \ref{definition-fibre-products-simplicial-scheme}.
Then pullback defines an equivalence between the category of
quasi-coherent $\mathcal{O}_S$-modules and the category of
quasi-coherent modules on $((X/S)_\bullet)_{Zar}$.
\end{lemma}

\begin{proof}
Let $\sigma : S \to X$ be a section of $f$. Let $(\mathcal{F}, \alpha)$
be a pair as in Lemma \ref{lemma-characterize-cartesian-modules}.
Set $\mathcal{G} = \sigma^*\mathcal{F}$. Consider the diagram
$$
\xymatrix{
X \ar[r]_-{(\sigma \circ f, 1)} \ar[d]_f &
X \times_S X \ar[d]^{\text{pr}_0} \ar[r]_-{\text{pr}_1} & X \\
S \ar[r]^\sigma & X
}
$$
Note that $\text{pr}_0 = d^1_1$ and $\text{pr}_1 = d^1_0$. Hence we
see that $(\sigma \circ f, 1)^*\alpha$ defines an isomorphism
$$
f^*\mathcal{G} = (\sigma \circ f, 1)^*\text{pr}_0^*\mathcal{F}
\longrightarrow
(\sigma \circ f, 1)^*\text{pr}_1^*\mathcal{F} = \mathcal{F}
$$
We omit the verification that this isomorphism is compatible
with $\alpha$ and the canonical isomorphism
$\text{pr}_0^*f^*\mathcal{G} \to \text{pr}_1^*f^*\mathcal{G}$.
\end{proof}






\section{Groupoids and simplicial schemes}
\label{section-groupoids-simplicial}

\noindent
Given a groupoid in schemes we can build a simplicial scheme.
It will turn out that the category of quasi-coherent sheaves on a
groupoid is equivalent to the category of cartesian quasi-coherent
sheaves on the associated simplicial scheme.

\begin{lemma}
\label{lemma-groupoid-simplicial}
Let $(U, R, s, t, c, e, i)$ be a groupoid scheme over $S$.
There exists a simplicial scheme $X$ over $S$
with the following properties
\begin{enumerate}
\item $X_0 = U$, $X_1 = R$, $X_2 = R \times_{s, U, t} R$,
\item $s_0^0 = e : X_0 \to X_1$,
\item $d^1_0 = s : X_1 \to X_0$, $d^1_1 = t : X_1 \to X_0$,
\item $s_0^1 = (e \circ t, 1) : X_1 \to X_2$,
$s_1^1 = (1, e \circ t) : X_1 \to X_2$,
\item $d^2_0 = \text{pr}_1 : X_2 \to X_1$,
$d^2_1 = c : X_2 \to X_1$,
$d^2_2 = \text{pr}_0$, and
\item $X = \text{cosk}_2 \text{sk}_2 X$.
\end{enumerate}
For all $n$ we have $X_n = R \times_{s, U, t} \ldots \times_{s, U, t} R$
with $n$ factors. The map $d^n_j : X_n \to X_{n - 1}$ is given on
functors of points by
$$
(r_1, \ldots, r_n) \longmapsto (r_1, \ldots, c(r_j, r_{j + 1}), \ldots, r_n)
$$
for $1 \leq j \leq n - 1$ whereas
$d^n_0(r_1, \ldots, r_n) = (r_2, \ldots, r_n)$ and
$d^n_n(r_1, \ldots, r_n) = (r_1, \ldots, r_{n - 1})$.
\end{lemma}

\begin{proof}
We only have to verify that the rules prescribed in (1), (2), (3), (4), (5)
define a $2$-truncated simplicial scheme $U'$ over $S$, since then (6)
allows us to set $X = \text{cosk}_2 U'$, see
Simplicial, Lemma \ref{simplicial-lemma-existence-cosk}.
Using the functor of points approach, all we have to verify is that
if $(\text{Ob}, \text{Arrows}, s, t, c, e, i)$ is a groupoid, then
$$
\xymatrix{
\text{Arrows} \times_{s, \text{Ob}, t} \text{Arrows}
\ar@<8ex>[d]^{\text{pr}_0}
\ar@<0ex>[d]_c
\ar@<-8ex>[d]_{\text{pr}_1}
\\
\text{Arrows}
\ar@<4ex>[d]^t
\ar@<-4ex>[d]_s
\ar@<4ex>[u]^{1, e}
\ar@<-4ex>[u]_{e, 1}
\\
\text{Ob}
\ar@<0ex>[u]_e
}
$$
is a $2$-truncated simplicial set. We omit the details.

\medskip\noindent
Finally, the description of $X_n$ for $n > 2$ follows by induction from
the description of $X_0$, $X_1$, $X_2$, and
Simplicial, Remark \ref{simplicial-remark-inductive-coskeleton} and
Lemma \ref{simplicial-lemma-work-out}. Alternately, one shows that
$\text{cosk}_2$ applied to the $2$-truncated simplicial set displayed above
gives a simplicial set whose $n$th term equals
$\text{Arrows} \times_{s, \text{Ob}, t} \ldots \times_{s, \text{Ob}, t}
\text{Arrows}$ with $n$ factors and degeneracy maps as given in the lemma.
Some details omitted.
\end{proof}

\begin{lemma}
\label{lemma-quasi-coherent-groupoid-simplicial}
Let $S$ be a scheme. Let $(U, R, s, t, c)$ be a groupoid scheme
over $S$. Let $X$ be the simplicial scheme over $S$ constructed
in Lemma \ref{lemma-groupoid-simplicial}.
Then the category of quasi-coherent modules on $(U, R, s, t, c)$
is equivalent to the category of quasi-coherent modules on $X_{Zar}$.
\end{lemma}

\begin{proof}
This is clear from Lemmas
\ref{lemma-quasi-coherent-sheaf} and
\ref{lemma-characterize-cartesian-modules}
and Groupoids, Definition \ref{groupoids-definition-groupoid-module}.
\end{proof}

\noindent
In the following lemma we will use the concept of a cartesian
morphism $V \to U$ of simplicial schemes as defined in
Definition \ref{definition-cartesian-morphism}.

\begin{lemma}
\label{lemma-quasi-coherent-groupoid-R-cartesian}
Let $(U, R, s, t, c)$ be a groupoid scheme over a scheme $S$.
Let $X$ be the simplicial scheme over $S$ constructed
in Lemma \ref{lemma-groupoid-simplicial}.
Let $(R/U)_\bullet$ be the simplicial
scheme associated to $s : R \to U$, see
Definition \ref{definition-fibre-products-simplicial-scheme}.
There exists a cartesian morphism $t_\bullet : (R/U)_\bullet \to X$
of simplicial schemes with low degree morphisms given by
$$
\xymatrix{
R \times_{s, U, s} R \times_{s, U, s} R
\ar@<3ex>[r]_-{\text{pr}_{12}}
\ar@<0ex>[r]_-{\text{pr}_{02}}
\ar@<-3ex>[r]_-{\text{pr}_{01}}
\ar[dd]_{(r_0, r_1, r_2) \mapsto (r_0 \circ r_1^{-1}, r_1 \circ r_2^{-1})} &
R \times_{s, U, s} R
\ar@<1ex>[r]_-{\text{pr}_1} \ar@<-2ex>[r]_-{\text{pr}_0}
\ar[dd]_{(r_0, r_1) \mapsto r_0 \circ r_1^{-1}} &
R \ar[dd]^t
\\
\\
R \times_{s, U, t} R
\ar@<3ex>[r]_{\text{pr}_1}
\ar@<0ex>[r]_c
\ar@<-3ex>[r]_{\text{pr}_0} &
R \ar@<1ex>[r]_s \ar@<-2ex>[r]_t &
U
}
$$
\end{lemma}

\begin{proof}
For arbitrary $n$ we define $(R/U)_\bullet \to X_n$ by the rule
$$
(r_0, \ldots, r_n)
\longrightarrow
(r_0 \circ r_1^{-1}, \ldots, r_{n - 1} \circ r_n^{-1})
$$
Compatibility with degeneracy maps is clear from the description of the
degeneracies in Lemma \ref{lemma-groupoid-simplicial}.
We omit the verification that the maps respect the morphisms $s^n_j$.
Groupoids, Lemma \ref{groupoids-lemma-diagram-pull}
(with the roles of $s$ and $t$ reversed)
shows that the two right squares are cartesian. In exactly the same manner
one shows all the other squares are cartesian too. Hence
the morphism is cartesian.
\end{proof}




\section{Descent data give equivalence relations}
\label{section-equivalence-relation}

\noindent
In Section \ref{section-simplicial-descent} we saw how descent data relative to
$X \to S$ can be formulated in terms of cartesian simplicial
schemes over $(X/S)_\bullet$. Here we link this to equivalence
relations as follows.

\begin{lemma}
\label{lemma-equivalence-relation}
Let $f : X \to S$ be a morphism of schemes.
Let $\pi : Y \to (X/S)_\bullet$ be a cartesian morphism of simplicial
schemes, see Definitions \ref{definition-cartesian-morphism} and
\ref{definition-fibre-products-simplicial-scheme}.
Then the morphism
$$
j = (d^1_1, d^1_0) : Y_1 \to Y_0 \times_S Y_0
$$
defines an equivalence relation on $Y_0$ over $S$,
see Groupoids, Definition \ref{groupoids-definition-equivalence-relation}.
\end{lemma}

\begin{proof}
Note that $j$ is a monomorphism. Namely the
composition $Y_1 \to Y_0 \times_S Y_0 \to Y_0 \times_S X$
is an isomorphism as $\pi$ is cartesian.

\medskip\noindent
Consider the morphism
$$
(d^2_2, d^2_0) : Y_2 \to Y_1 \times_{d^1_0, Y_0, d^1_1} Y_1.
$$
This works because $d_0 \circ d_2 = d_1 \circ d_0$,
see Simplicial, Remark \ref{simplicial-remark-relations}.
Also, it is a morphism over $(X/S)_2$. It is an isomorphism
because $Y \to (X/S)_\bullet$ is cartesian. Note for example that the
right hand side is isomorphic to
$Y_0 \times_{\pi_0, X, \text{pr}_1} (X \times_S X \times_S X) =
X \times_S Y_0 \times_S X$
because $\pi$ is cartesian. Details omitted.

\medskip\noindent
As in Groupoids, Definition \ref{groupoids-definition-equivalence-relation}
we denote $t = \text{pr}_0 \circ j = d^1_1$ and
$s = \text{pr}_1 \circ j = d^1_0$.
The isomorphism above, combined with the morphism
$d^2_1 : Y_2 \to Y_1$ give us a composition morphism
$$
c : Y_1 \times_{s, Y_0, t} Y_1 \longrightarrow Y_1
$$
over $Y_0 \times_S Y_0$. This immediately implies
that for any scheme $T/S$ the relation
$Y_1(T) \subset Y_0(T) \times Y_0(T)$ is transitive.

\medskip\noindent
Reflexivity follows from the fact that the
restriction of the morphism $j$ to the diagonal
$\Delta : X \to X \times_S X$ is an isomorphism
(again use the cartesian property of $\pi$).

\medskip\noindent
To see symmetry we consider the morphism
$$
(d^2_2, d^2_1) : Y_2 \to Y_1 \times_{d^1_1, Y_0, d^1_1} Y_1.
$$
This works because $d_1 \circ d_2 = d_1 \circ d_1$,
see Simplicial, Remark \ref{simplicial-remark-relations}.
It is an isomorphism
because $Y \to (X/S)_\bullet$ is cartesian.
Note for example that the
right hand side is isomorphic to
$Y_0 \times_{\pi_0, X, \text{pr}_0} (X \times_S X \times_S X) =
Y_0 \times_S X \times_S X$
because $\pi$ is cartesian. Details omitted.

\medskip\noindent
Let $T/S$ be a scheme. Let $a \sim b$ for $a, b \in Y_0(T)$
be synonymous with $(a, b) \in Y_1(T)$.
The isomorphism $(d^2_2, d^2_1)$ above
implies that if $a \sim b$ and $a \sim c$, then $b \sim c$.
Combined with reflexivity this shows that $\sim$ is
an equivalence relation.
\end{proof}







\section{An example case}
\label{section-example}

\noindent
In this section we show that disjoint unions of spectra
of Artinian rings can be descended along a quasi-compact
surjective flat morphism of schemes.

\begin{lemma}
\label{lemma-equivalence-classes-points}
Let $X \to S$ be a morphism of schemes. Suppose $Y \to (X/S)_\bullet$
is a cartesian morphism of simplicial schemes. For $y \in Y_0$ a point define
$$
T_y = \{y' \in Y_0 \mid \exists\ y_1 \in Y_1:
d^1_1(y_1) = y, d^1_0(y_1) = y'\}
$$
as a subset of $Y_0$. Then $y \in T_y$ and
$T_y \cap T_{y'} \not = \emptyset \Rightarrow T_y = T_{y'}$.
\end{lemma}

\begin{proof}
Combine Lemma \ref{lemma-equivalence-relation} and
Groupoids, Lemma
\ref{groupoids-lemma-pre-equivalence-equivalence-relation-points}.
\end{proof}

\begin{lemma}
\label{lemma-quasi-compact}
Let $X \to S$ be a morphism of schemes.
Suppose $Y \to (X/S)_\bullet$ is a cartesian morphism of simplicial schemes.
Let $y \in Y_0$ be a point. If $X \to S$ is quasi-compact, then
$$
T_y = \{y' \in Y_0 \mid \exists\ y_1 \in Y_1:
d^1_1(y_1) = y, d^1_0(y_1) = y'\}
$$
is a quasi-compact subset of $Y_0$.
\end{lemma}

\begin{proof}
Let $F_y$ be the scheme theoretic fibre of $d^1_1 : Y_1 \to Y_0$
at $y$. Then we see that $T_y$ is the image of the morphism
$$
\xymatrix{
F_y \ar[r] \ar[d] &
Y_1 \ar[r]^{d^1_0} \ar[d]^{d^1_1} &
Y_0 \\
y \ar[r] &
Y_0 &
}
$$
Note that $F_y$ is quasi-compact. This proves the lemma.
\end{proof}

\begin{lemma}
\label{lemma-descent-disjoint-union-Artinian-along-fields}
Let $X \to S$ be a quasi-compact flat surjective morphism.
Let $(V, \varphi)$ be a descent datum relative
to $X \to S$. If $V$ is a disjoint union of
spectra of Artinian rings, then $(V, \varphi)$ is effective.
\end{lemma}

\begin{proof}
Let $Y \to (X/S)_\bullet$ be the cartesian morphism of simplicial
schemes corresponding to $(V, \varphi)$ by
Lemma \ref{lemma-cartesian-equivalent-descent-datum}.
Observe that $Y_0 = V$.
Write $V = \coprod_{i \in I} \Spec(A_i)$ with each $A_i$ local
Artinian. Moreover, let $v_i \in V$ be the unique closed point of
$\Spec(A_i)$ for all $i \in I$. Write $i \sim j$ if and only if
$v_i \in T_{v_j}$ with notation as in
Lemma \ref{lemma-equivalence-classes-points} above.
By Lemmas \ref{lemma-equivalence-classes-points} and \ref{lemma-quasi-compact}
this is an equivalence relation with finite equivalence
classes. Let $\overline{I} = I/\sim$. Then we can write
$V = \coprod_{\overline{i} \in \overline{I}} V_{\overline{i}}$
with
$V_{\overline{i}} = \coprod_{i \in \overline{i}} \Spec(A_i)$.
By construction we see that
$\varphi : V \times_S X \to X \times_S V$ maps
the open and closed subspaces $V_{\overline{i}} \times_S X$
into the open and closed subspaces $X \times_S V_{\overline{i}}$.
In other words, we get descent data
$(V_{\overline{i}}, \varphi_{\overline{i}})$, and
$(V, \varphi)$ is the coproduct of them in the category of
descent data.
Since each of the $V_{\overline{i}}$ is a finite union of
spectra of Artinian local rings the morphism $V_{\overline{i}} \to X$
is affine, see Morphisms, Lemma \ref{morphisms-lemma-Artinian-affine}.
Since $\{X \to S\}$ is an fpqc covering we see that all
the descent data $(V_{\overline{i}}, \varphi_{\overline{i}})$ are effective
by Descent, Lemma \ref{descent-lemma-affine}.
\end{proof}

\noindent
To be sure, the lemma above has very limited applicability!









\section{Simplicial algebraic spaces}
\label{section-simplicial-algebraic-spaces}

\noindent
Let $S$ be a scheme. A {\it simplicial algebraic space}
is a simplicial object in the category of algebraic spaces over $S$,
see Simplicial, Definition \ref{simplicial-definition-simplicial-object}.
Recall that a simplicial algebraic space looks like
$$
\xymatrix{
X_2
\ar@<2ex>[r]
\ar@<0ex>[r]
\ar@<-2ex>[r]
&
X_1
\ar@<1ex>[r]
\ar@<-1ex>[r]
\ar@<1ex>[l]
\ar@<-1ex>[l]
&
X_0
\ar@<0ex>[l]
}
$$
Here there are two morphisms $d^1_0, d^1_1 : X_1 \to X_0$
and a single morphism $s^0_0 : X_0 \to X_1$, etc.
These morphisms satisfy some required relations such as
$d^1_0 \circ s^0_0 = \text{id}_{X_0} = d^1_1 \circ s^0_0$, see
Simplicial, Lemma \ref{simplicial-lemma-characterize-simplicial-object}.
It is useful to think of $d^n_i : X_n \to X_{n - 1}$
as the ``projection forgetting the $i$th coordinate'' and
to think of $s^n_j : X_n \to X_{n + 1}$ as the ``diagonal map repeating
the $j$th coordinate''.

\medskip\noindent
A {\it morphism of simplicial algebraic spaces} $h : X \to Y$ is the same
thing as a morphism of simplicial objects in
the category of algebraic spaces over $S$,
see Simplicial, Definition \ref{simplicial-definition-simplicial-object}.
Thus $h$ consists of morphisms of algebraic spaces $h_n : X_n \to Y_n$
such that $h_{n - 1} \circ d^n_j = d^n_j \circ h_n$ and
$h_{n + 1} \circ s^n_j = s^n_j \circ h_n$ whenever this makes sense.

\medskip\noindent
An {\it augmentation} $a : X \to X_{-1}$
of a simplicial algebraic space $X$ is given by a morphism
of algebraic spaces $a_0 : X_0 \to X_{-1}$
such that $a_0 \circ d^1_0 = a_0 \circ d^1_1$.
See Simplicial, Section \ref{simplicial-section-augmentation}.
In this situation we always indicate $a_n : X_n \to X_{-1}$ the induced
morphisms for $n \geq 0$.

\medskip\noindent
Let $X$ be a simplicial algebraic space. For every $n$ we have the
site $X_{n, spaces, \etale}$ (Properties of Spaces, Definition
\ref{spaces-properties-definition-spaces-etale-site})
and for every morphism $\varphi : [m] \to [n]$ we have a morphism of sites
$$
f_\varphi = X(\varphi)_{spaces, \etale} :
X_{n, spaces, \etale} \to X_{m, spaces, \etale},
$$
associated to the morphism of algebraic spaces
$X(\varphi) : X_n \to X_m$ (Properties of Spaces, Lemma
\ref{spaces-properties-lemma-functoriality-etale-site}).
This gives a simplicial object in the category of sites.
In Lemma \ref{lemma-simplicial-site-site} we constructed an associated
site which we denote $X_{spaces, \etale}$.
An object of the site $X_{spaces, \etale}$ is a
an algebraic space $U$ \'etale over $X_n$ for some $n$
and a morphism $(\varphi, f) : U/X_n \to V/X_m$ is given
by a morphism $\varphi : [m] \to [n]$ in $\Delta$ and a morphism
$f : U \to V$ of algebraic spaces such that the diagram
$$
\xymatrix{
U \ar[r]_f \ar[d] & V \ar[d] \\
X_n \ar[r]^{f_\varphi} & X_m
}
$$
is commutative. Consider the full subcategories
$$
X_{affine, \etale} \subset X_\etale \subset X_{spaces, \etale}
$$
whose objects are $U/X_n$ with $U$ affine, respectively a scheme.
Endowing these categories with their natural topologies
(see
Properties of Spaces, Lemma \ref{spaces-properties-lemma-alternative},
Definition \ref{spaces-properties-definition-etale-site}, and
Lemma \ref{spaces-properties-lemma-compare-etale-sites})
these inclusion functors define equivalences of topoi
$$
\Sh(X_{affine, \etale}) = \Sh(X_\etale) = \Sh(X_{spaces, \etale})
$$
In the following we will silently identify these topoi.
We will say that $X_\etale$ is the {\it small \'etale site of $X$}
and its topos is the {\it small \'etale topos of $X$}.

\medskip\noindent
Let $X_\etale$ be the small \'etale site of a simplicial algebraic space $X$.
There is a sheaf of rings $\mathcal{O}$ on $X_\etale$ whose restriction
to $X_n$ is the structure sheaf $\mathcal{O}_{X_n}$. This follows from
Lemma \ref{lemma-describe-sheaves-simplicial-site-site}. We will say
{\it $\mathcal{O}$ is the structure sheaf of the
simplicial algebraic space $X$}.
At this point all the material developed for simplicial (ringed) sites
applies, see Sections \ref{section-simplicial-sites},
\ref{section-augmentation-simplicial-sites},
\ref{section-morphism-simplicial-sites},
\ref{section-simplicial-sites-modules},
\ref{section-cohomology-simplicial-sites},
\ref{section-cohomology-augmentation-simplicial-sites},
\ref{section-cohomology-simplicial-sites-modules},
\ref{section-cohomology-augmentation-ringed-simplicial-sites},
\ref{section-cartesian},
\ref{section-glueing}, and
\ref{section-glueing-modules}.

\medskip\noindent
Let $X$ be a simplicial algebraic space with structure sheaf $\mathcal{O}$.
As on any ringed topos, there is a notion
of a {\it quasi-coherent $\mathcal{O}$-module on $X_\etale$}, see
Modules on Sites, Definition \ref{sites-modules-definition-site-local}.
However, a quasi-coherent $\mathcal{O}$-module on $X_\etale$ is
just a cartesian $\mathcal{O}$-module $\mathcal{F}$ whose restrictions
$\mathcal{F}_n$ are quasi-coherent on $X_n$, see
Lemma \ref{lemma-quasi-coherent-sheaf}.

\medskip\noindent
Let $h : X \to Y$ be a morphism of simplicial algebraic spaces over $S$.
By Lemma \ref{lemma-morphism-simplicial-sites} applied to the morphisms
of sites $(h_n)_{spaces, \etale} : X_{spaces, \etale} \to Y_{spaces, \etale}$
(Properties of Spaces, Lemma
\ref{spaces-properties-lemma-functoriality-etale-site})
we obtain a morphism of small \'etale topoi
$h_\etale : \Sh(X_\etale) \to \Sh(Y_\etale)$.
Recall that $h_\etale^{-1}$ and $h_{\etale, *}$ have a simple
description in terms of the components, see
Lemma \ref{lemma-morphism-simplicial-sites}.
Let $\mathcal{O}_X$, resp.\ $\mathcal{O}_Y$ denote the structure
sheaf of $X$, resp.\ $Y$. We define
$h_\etale^\sharp : h_{\etale, *}\mathcal{O}_X \to \mathcal{O}_Y$
to be the map of sheaves of rings on $Y_\etale$ given by
$h_n^\sharp : h_{n, *}\mathcal{O}_{X_n} \to \mathcal{O}_{Y_n}$ on $Y_n$.
We obtain a morphism of ringed topoi
$$
h_\etale :
(\Sh(X_\etale), \mathcal{O}_X)
\longrightarrow
(\Sh(Y_\etale), \mathcal{O}_Y)
$$

\medskip\noindent
Let $X$ be a simplicial algebraic space with structure sheaf $\mathcal{O}$.
Let $X_{-1}$ be an algebraic space over $S$ and let $a_0 : X_0 \to X_{-1}$
be an augmentation of $X$. By
Lemma \ref{lemma-augmentation-site}
applied to the morphism of sites
$(a_0)_{spaces, \etale} :
X_{0, spaces, \etale} \to X_{-1, spaces, \etale}$
we obtain a corresponding morphism of topoi
$a : \Sh(X_\etale) \to \Sh(X_{-1, \etale})$.
Observe that $a^{-1}\mathcal{G}$ is the sheaf on
$X_\etale$ with components $a_n^{-1}\mathcal{G}$. Hence we can use the maps
$a_n^\sharp : a_n^{-1}\mathcal{O}_{X_{-1}} \to \mathcal{O}_{X_n}$ to define
a map $a^\sharp : a^{-1}\mathcal{O}_{X_{-1}} \to \mathcal{O}$, or equivalently
by adjunction a map $a^\sharp : \mathcal{O}_{X_{-1}} \to a_*\mathcal{O}$
(which as usual has the same name). This puts us in the situation
discussed in
Section \ref{section-cohomology-augmentation-ringed-simplicial-sites}.
Therefore we obtain a morphism of ringed topoi
$$
a :
(\Sh(X_\etale), \mathcal{O})
\longrightarrow
(\Sh(X_{-1}), \mathcal{O}_{X_{-1}})
$$

\medskip\noindent
A final observation is the following. Suppose we are given a morphism
$h : X \to Y$ of simplicial algebraic spaces $X$ and $Y$ with structure sheaves
$\mathcal{O}_X$, $\mathcal{O}_Y$, augmentations
$a_0 : X_0 \to X_{-1}$, $b_0 : Y_0 \to Y_{-1}$ and a morphism
$h_{-1} : X_{-1} \to Y_{-1}$ such that
$$
\xymatrix{
X_0 \ar[r]_{h_0} \ar[d]_{a_0} & Y_0 \ar[d]^{b_0} \\
X_{-1} \ar[r]^{h_{-1}} & Y_{-1}
}
$$
commutes. Then from the constructions elucidated above
we obtain a commutative diagram of morphisms of ringed topoi as follows
$$
\xymatrix{
(\Sh(X_\etale), \mathcal{O}_X) \ar[r]_{h_\etale} \ar[d]_a &
(\Sh(Y_\etale), \mathcal{O}_Y) \ar[d]^b \\
(\Sh(X_{-1}), \mathcal{O}_{X_{-1}}) \ar[r]^{h_{-1}} &
(\Sh(Y_{-1}), \mathcal{O}_{Y_{-1}})
}
$$




\section{Fppf hypercoverings of algebraic spaces}
\label{section-fppf-hypercovering}

\noindent
This section is the analogue of Section \ref{section-proper-hypercovering}
for the case of algebraic spaces and fppf hypercoverings.
The reader who wishes to do so, can replace ``algebraic space''
everywhere with ``scheme'' and get equally valid results.
This has the advantage of replacing the references to
More on Cohomology of Spaces, Section
\ref{spaces-more-cohomology-section-fppf-etale}
with references to
\'Etale Cohomology, Section \ref{etale-cohomology-section-fppf-etale}.

\medskip\noindent
We fix a base scheme $S$.
Let $X$ be an algebraic space over $S$ and let $U$ be a simplicial
algebraic space over $S$. Assume we have an augmentation
$$
a : U \to X
$$
See Section \ref{section-simplicial-algebraic-spaces}.
We say that $U$ is an {\it fppf hypercovering} of $X$ if
\begin{enumerate}
\item $U_0 \to X$ is flat, locally of finite presentation, and surjective,
\item $U_1 \to U_0 \times_X U_0$ is flat, locally of finite presentation, and
surjective,
\item $U_{n + 1} \to (\text{cosk}_n\text{sk}_n U)_{n + 1}$
is flat, locally of finite presentation, and surjective for $n \geq 1$.
\end{enumerate}
The category of algebraic spaces over $S$ has all finite limits, hence the
coskeleta used in the formulation above exist.
$$
\fbox{Principle: Fppf hypercoverings can be used to compute \'etale cohomology.}
$$
The key idea behind the proof of the principle is to compare the
fppf and \'etale topologies on the category $\textit{Spaces}/S$.
Namely, the fppf topology is stronger than the \'etale topology and we have
(a) a flat, locally finitely presented, surjective map defines
an fppf covering, and
(b) fppf cohomology of sheaves pulled back from the small \'etale site
agrees with \'etale cohomology as we have seen in
More on Cohomology of Spaces, Section
\ref{spaces-more-cohomology-section-fppf-etale}.

\begin{lemma}
\label{lemma-compare-simplicial-objects-fppf-etale}
Let $S$ be a scheme. Let $X$ be an algebraic space over $S$.
Let $U$ be a simplicial algebraic space over $S$.
Let $a : U \to X$ be an augmentation. There is a commutative diagram
$$
\xymatrix{
\Sh((\textit{Spaces}/U)_{fppf, total}) \ar[r]_-h \ar[d]_{a_{fppf}} &
\Sh(U_\etale) \ar[d]^a \\
\Sh((\textit{Spaces}/X)_{fppf}) \ar[r]^-{h_{-1}} &
\Sh(X_\etale)
}
$$
where the left vertical arrow is defined in
Section \ref{section-hypercovering}
and the right vertical arrow is defined in
Section \ref{section-simplicial-algebraic-spaces}.
\end{lemma}

\begin{proof}
The notation $(\textit{Spaces}/U)_{fppf, total}$ indicates that
we are using the construction of
Section \ref{section-hypercovering}
for the site $(\textit{Spaces}/S)_{fppf}$ and the
simplicial object $U$ of this site\footnote{We could also
use the \'etale topology and this would be denoted
$(\textit{Spaces}/U)_{\etale, total}$.}.
We will use the sites $X_{spaces, \etale}$ and $U_{spaces, \etale}$
for the topoi on the right hand side; this is permissible
see discussion in Section \ref{section-simplicial-algebraic-spaces}.

\medskip\noindent
Observe that both $(\textit{Spaces}/U)_{fppf, total}$ and
$U_{spaces, \etale}$
fall into case A of Situation \ref{situation-simplicial-site}.
This is immediate from the construction of
$U_\etale$ in Section \ref{section-simplicial-algebraic-spaces}
and it follows from Lemma \ref{lemma-sr-when-fibre-products}
for $(\textit{Spaces}/U)_{fppf, total}$.
Next, consider the functors
$U_{n, spaces, \etale} \to (\textit{Spaces}/U_n)_{fppf}$, $U \mapsto U/U_n$
and
$X_{spaces, \etale} \to (\textit{Spaces}/X)_{fppf}$, $U \mapsto U/X$.
We have seen that these define morphisms of sites in
More on Cohomology of Spaces, Section
\ref{spaces-more-cohomology-section-fppf-etale}
where these were denoted $a_{U_n} = \epsilon_{U_n} \circ \pi_{u_n}$
and $a_X = \epsilon_X \circ \pi_X$.
Thus we obtain a morphism of simplicial sites compatible with
augmentations as in Remark \ref{remark-morphism-augmentation-simplicial-sites}
and we may apply
Lemma \ref{lemma-morphism-augmentation-simplicial-sites} to conclude.
\end{proof}

\begin{lemma}
\label{lemma-descent-sheaves-for-fppf-hypercovering}
Let $S$ be a scheme. Let $X$ be an algebraic space over $S$.
Let $U$ be a simplicial algebraic space over $S$. Let $a : U \to X$
be an augmentation. If $a : U \to X$ is an fppf hypercovering of $X$,
then
$$
a^{-1} : \Sh(X_\etale) \to \Sh(U_\etale)
\quad\text{and}\quad
a^{-1} : \textit{Ab}(X_\etale) \to \textit{Ab}(U_\etale)
$$
are fully faithful with essential image the cartesian sheaves and
quasi-inverse given by $a_*$. Here $a : \Sh(U_\etale) \to \Sh(X_\etale)$
is as in Section \ref{section-simplicial-algebraic-spaces}.
\end{lemma}

\begin{proof}
We will prove the statement for sheaves of sets. It will be an
almost formal consequence of results already established.
Consider the diagram of
Lemma \ref{lemma-compare-simplicial-objects-fppf-etale}.
In the proof of this lemma we have seen that
$h_{-1}$ is the morphism $a_X$ of
More on Cohomology of Spaces, Section
\ref{spaces-more-cohomology-section-fppf-etale}.
Thus it follows from
More on Cohomology of Spaces, Lemma
\ref{spaces-more-cohomology-lemma-comparison-fppf-etale}
that $(h_{-1})^{-1}$ is fully faithful with quasi-inverse $h_{-1, *}$.
The same holds true for the components $h_n$ of $h$.
By the description of the functors $h^{-1}$ and $h_*$ of
Lemma \ref{lemma-morphism-simplicial-sites}
we conclude that $h^{-1}$ is fully faithful with quasi-inverse $h_*$.
Observe that $U$ is a hypercovering of $X$ in $(\textit{Spaces}/S)_{fppf}$
as defined in Section \ref{section-hypercovering}.
By Lemma \ref{lemma-hypercovering-X-simple-descent-sheaves}
we see that $a_{fppf}^{-1}$ is fully faithful with quasi-inverse
$a_{fppf, *}$ and with essential image the cartesian sheaves
on $(\textit{Spaces}/U)_{fppf, total}$.
A formal argument (chasing around the diagram) now shows that
$a^{-1}$ is fully faithful.

\medskip\noindent
Finally, suppose that $\mathcal{G}$ is a cartesian sheaf on $U_\etale$.
Then $h^{-1}\mathcal{G}$ is a cartesian sheaf on
$(\textit{Spaces}/U)_{fppf, total}$. Hence
$h^{-1}\mathcal{G} = a_{fppf}^{-1}\mathcal{H}$ for some sheaf
$\mathcal{H}$ on $(\textit{Spaces}/X)_{fppf}$.
In particular we find that
$h_0^{-1}\mathcal{G}_0 = (a_{0, big, fppf})^{-1}\mathcal{H}$.
Recalling that $h_0 = a_{U_0}$ and that $U_0 \to X$ is
flat, locally of finite presentation, and surjective, we
find from More on Cohomology of Spaces,
Lemma \ref{spaces-more-cohomology-lemma-descent-sheaf-fppf-etale}
that there exists a sheaf $\mathcal{F}$ on $X_\etale$ and isomorphism
$\mathcal{H} = (h_{-1})^{-1}\mathcal{F}$.
Since $a_{fppf}^{-1}\mathcal{H} = h^{-1}\mathcal{G}$
we deduce that $h^{-1}\mathcal{G} \cong h^{-1}a^{-1}\mathcal{F}$.
By fully faithfulness of $h^{-1}$ we conclude that
$a^{-1}\mathcal{F} \cong \mathcal{G}$.

\medskip\noindent
Fix an isomorphism $\theta : a^{-1}\mathcal{F} \to \mathcal{G}$.
To finish the proof we have to show $\mathcal{G} = a^{-1}a_*\mathcal{G}$
(in order to show that the quasi-inverse is given by $a_*$; everything
else has been proven above).
Because $a^{-1}$ is fully faithful we have $\text{id} \cong a_*a^{-1}$ by
Categories, Lemma \ref{categories-lemma-adjoint-fully-faithful}.
Thus $\mathcal{F} \cong a_*a^{-1}\mathcal{F}$ and
$a_*\theta : a_*a^{-1}\mathcal{F} \to a_*\mathcal{G}$
combine to an isomorphism $\mathcal{F} \to a_*\mathcal{G}$.
Pulling back by $a$ and precomposing by $\theta^{-1}$
we find the desired isomorphism.
\end{proof}

\begin{lemma}
\label{lemma-cohomological-descent-for-fppf-hypercovering}
Let $S$ be a scheme. Let $X$ be an algebraic space over $S$.
Let $U$ be a simplicial algebraic space over $S$. Let $a : U \to X$
be an augmentation. If $a : U \to X$ is an fppf hypercovering of $X$,
then for $K \in D^+(X_\etale)$
$$
K \to Ra_*(a^{-1}K)
$$
is an isomorphism. Here $a : \Sh(U_\etale) \to \Sh(X_\etale)$
is as in Section \ref{section-simplicial-algebraic-spaces}.
\end{lemma}

\begin{proof}
Consider the diagram of Lemma \ref{lemma-compare-simplicial-objects-fppf-etale}.
Observe that $Rh_{n, *}h_n^{-1}$ is the identity functor
on $D^+(U_{n, \etale})$ by
More on Cohomology of Spaces, Lemma
\ref{spaces-more-cohomology-lemma-cohomological-descent-etale-fppf}.
Hence $Rh_*h^{-1}$ is the identity functor on
$D^+(U_\etale)$ by
Lemma \ref{lemma-direct-image-morphism-simplicial-sites}.
We have
\begin{align*}
Ra_*(a^{-1}K)
& =
Ra_*Rh_*h^{-1}a^{-1}K \\
& =
Rh_{-1, *}Ra_{fppf, *}a_{fppf}^{-1}(h_{-1})^{-1}K \\
& =
Rh_{-1, *}(h_{-1})^{-1}K \\
& =
K
\end{align*}
The first equality by the discussion above, the second equality
because of the commutativity of the diagram in
Lemma \ref{lemma-compare-simplicial-objects}, the third equality by
Lemma \ref{lemma-hypercovering-X-simple-descent-bounded-abelian}
as $U$ is a hypercovering of $X$ in $(\textit{Spaces}/S)_{fppf}$,
and the last equality by the already used
More on Cohomology of Spaces, Lemma
\ref{spaces-more-cohomology-lemma-cohomological-descent-etale-fppf}.
\end{proof}

\begin{lemma}
\label{lemma-compute-via-fppf-hypercovering}
Let $S$ be a scheme. Let $X$ be an algebraic space over $S$.
Let $U$ be a simplicial algebraic space over $S$. Let $a : U \to X$
be an augmentation. If $a : U \to X$ is an fppf hypercovering of $X$, then
$$
R\Gamma(X_\etale, K) = R\Gamma(U_\etale, a^{-1}K)
$$
for $K \in D^+(X_\etale)$. Here $a : \Sh(U_\etale) \to \Sh(X_\etale)$
is as in Section \ref{section-simplicial-algebraic-spaces}.
\end{lemma}

\begin{proof}
This follows from
Lemma \ref{lemma-cohomological-descent-for-fppf-hypercovering}
because $R\Gamma(U_\etale, -) = R\Gamma(X_\etale, -) \circ Ra_*$ by
Cohomology on Sites, Remark \ref{sites-cohomology-remark-before-Leray}.
\end{proof}

\begin{lemma}
\label{lemma-fppf-hypercovering-equivalence-bounded}
Let $S$ be a scheme. Let $X$ be an algebraic space over $S$.
Let $U$ be a simplicial algebraic space over $S$. Let $a : U \to X$
be an augmentation.
Let $\mathcal{A} \subset \textit{Ab}(U_\etale)$
denote the weak Serre subcategory of cartesian abelian sheaves.
If $U$ is an fppf hypercovering of $X$, then
the functor $a^{-1}$ defines an equivalence
$$
D^+(X_\etale) \longrightarrow D_\mathcal{A}^+(U_\etale)
$$
with quasi-inverse $Ra_*$. Here $a : \Sh(U_\etale) \to \Sh(X_\etale)$
is as in Section \ref{section-simplicial-algebraic-spaces}.
\end{lemma}

\begin{proof}
Observe that $\mathcal{A}$ is a weak Serre subcategory by
Lemma \ref{lemma-Serre-subcat-cartesian-modules}.
The equivalence is a
formal consequence of the results obtained so far. Use
Lemmas \ref{lemma-descent-sheaves-for-fppf-hypercovering} and
\ref{lemma-cohomological-descent-for-fppf-hypercovering} and
Cohomology on Sites, Lemma \ref{sites-cohomology-lemma-equivalence-bounded}.
\end{proof}

\begin{lemma}
\label{lemma-spectral-sequence-fppf-hypercovering}
Let $S$ be a scheme. Let $X$ be an algebraic space over $S$.
Let $U$ be a simplicial algebraic space over $S$. Let $a : U \to X$
be an augmentation. Let $\mathcal{F}$ be an abelian sheaf
on $X_\etale$. Let $\mathcal{F}_n$ be the pullback to $U_{n, \etale}$.
If $U$ is an fppf hypercovering of $X$, then
there exists a canonical spectral sequence
$$
E_1^{p, q} = H^q_\etale(U_p, \mathcal{F}_p)
$$
converging to $H^{p + q}_\etale(X, \mathcal{F})$.
\end{lemma}

\begin{proof}
Immediate consequence of Lemmas \ref{lemma-compute-via-fppf-hypercovering}
and \ref{lemma-simplicial-sheaf-cohomology-site}.
\end{proof}








\section{Fppf hypercoverings of algebraic spaces: modules}
\label{section-fppf-hypercovering-modules}

\noindent
We continue the discussion of (cohomological) descent for fppf hypercoverings
started in Section \ref{section-fppf-hypercovering}
but in this section we discuss what happens for sheaves of modules.
We mainly discuss quasi-coherent modules and it turns out that
we can do unbounded cohomological descent for those.

\begin{lemma}
\label{lemma-compare-simplicial-objects-fppf-etale-modules}
Let $S$ be a scheme. Let $X$ be an algebraic space over $S$.
Let $U$ be a simplicial algebraic space over $S$.
Let $a : U \to X$ be an augmentation. There is a commutative diagram
$$
\xymatrix{
(\Sh((\textit{Spaces}/U)_{fppf, total}), \mathcal{O}_{big, total})
\ar[r]_-h \ar[d]_{a_{fppf}} &
(\Sh(U_\etale), \mathcal{O}_U) \ar[d]^a \\
(\Sh((\textit{Spaces}/X)_{fppf}), \mathcal{O}_{big}) \ar[r]^-{h_{-1}} &
(\Sh(X_\etale), \mathcal{O}_X)
}
$$
of ringed topoi where the left vertical arrow is defined in
Section \ref{section-hypercovering-modules}
and the right vertical arrow is defined in
Section \ref{section-simplicial-algebraic-spaces}.
\end{lemma}

\begin{proof}
For the underlying diagram of topoi we refer to the discussion in
the proof of Lemma \ref{lemma-compare-simplicial-objects-fppf-etale}.
The sheaf $\mathcal{O}_U$ is the structure sheaf of the
simplicial algebraic space $U$ as defined in
Section \ref{section-simplicial-algebraic-spaces}.
The sheaf $\mathcal{O}_X$ is the usual structure sheaf of the algebraic
space $X$. The sheaves of rings $\mathcal{O}_{big, total}$ and
$\mathcal{O}_{big}$ come from the structure sheaf on
$(\textit{Spaces}/S)_{fppf}$ in the manner explained in
Section \ref{section-hypercovering-modules}
which also constructs $a_{fppf}$ as a morphism of ringed topoi.
The component morphisms $h_n = a_{U_n}$ and $h_{-1} = a_X$
are morphisms of ringed topoi by
More on Cohomology of Spaces, Section
\ref{spaces-more-cohomology-section-fppf-etale-modules}.
Finally, since the continuous functor
$u : U_{spaces, \etale} \to (\textit{Spaces}/U)_{fppf, total}$
used to define $h$\footnote{This happened in the proof of
Lemma \ref{lemma-compare-simplicial-objects-fppf-etale}
via an application of
Lemma \ref{lemma-morphism-augmentation-simplicial-sites}.}
is given by $V/U_n \mapsto V/U_n$
we see that $h_*\mathcal{O}_{big, total} = \mathcal{O}_U$
which is how we endow $h$ with the structure of a morphism
of ringed simplicial sites as in
Remark \ref{remark-morphism-simplicial-sites-modules}.
Then we obtain $h$ as a morphism of ringed topoi
by Lemma \ref{lemma-morphism-simplicial-sites-modules}.
Please observe that the morphisms $h_n$ indeed agree
with the morphisms $a_{U_n}$ described above.
We omit the verification
that the diagram is commutative (as a diagram of
ringed topoi -- we already know it is commutative
as a diagram of topoi).
\end{proof}

\begin{lemma}
\label{lemma-descent-qcoh-for-fppf-hypercovering}
Let $S$ be a scheme. Let $X$ be an algebraic space over $S$.
Let $U$ be a simplicial algebraic space over $S$. Let $a : U \to X$
be an augmentation. If $a : U \to X$ is an fppf hypercovering of $X$,
then
$$
a^* : \QCoh(\mathcal{O}_X) \to \QCoh(\mathcal{O}_U)
$$
is an equivalence fully faithful with quasi-inverse given by $a_*$.
Here $a : \Sh(U_\etale) \to \Sh(X_\etale)$
is as in Section \ref{section-simplicial-algebraic-spaces}.
\end{lemma}

\begin{proof}
Consider the diagram of
Lemma \ref{lemma-compare-simplicial-objects-fppf-etale-modules}.
In the proof of this lemma we have seen that
$h_{-1}$ is the morphism $a_X$ of
More on Cohomology of Spaces, Section
\ref{spaces-more-cohomology-section-fppf-etale-modules}.
Thus it follows from
More on Cohomology of Spaces, Lemma
\ref{spaces-more-cohomology-lemma-review-quasi-coherent}
that
$$
(h_{-1})^* :
\QCoh(\mathcal{O}_X)
\longrightarrow
\QCoh(\mathcal{O}_{big})
$$
is an equivalence with quasi-inverse $h_{-1, *}$.
The same holds true for the components $h_n$ of $h$.
Recall that $\QCoh(\mathcal{O}_U)$ and $\QCoh(\mathcal{O}_{big, total})$
consist of cartesian modules whose components are quasi-coherent, see
Lemma \ref{lemma-quasi-coherent-sheaf}.
Since the functors $h^*$ and $h_*$ of
Lemma \ref{lemma-morphism-simplicial-sites-modules}
agree with the functors $h_n^*$ and $h_{n, *}$ on components
we conclude that
$$
h^* :
\QCoh(\mathcal{O}_U)
\longrightarrow 
\QCoh(\mathcal{O}_{big, total})
$$
is an equivalence with quasi-inverse $h_*$.
Observe that $U$ is a hypercovering of $X$ in $(\textit{Spaces}/S)_{fppf}$
as defined in Section \ref{section-hypercovering}.
By Lemma \ref{lemma-hypercovering-X-simple-descent-modules}
we see that $a_{fppf}^*$ is fully faithful with quasi-inverse
$a_{fppf, *}$ and with essential image the cartesian sheaves
of $\mathcal{O}_{fppf, total}$-modules.
Thus, by the description of $\QCoh(\mathcal{O}_{big})$ and
$\QCoh(\mathcal{O}_{big, total})$ of Lemma \ref{lemma-quasi-coherent-sheaf},
we get an equivalence
$$
a_{fppf}^* :
\QCoh(\mathcal{O}_{big})
\longrightarrow
\QCoh(\mathcal{O}_{big, total})
$$
with quasi-inverse given by $a_{fppf, *}$.
A formal argument (chasing around the diagram) now shows that
$a^*$ is fully faithful on $\QCoh(\mathcal{O}_X)$ and has
image contained in $\QCoh(\mathcal{O}_U)$.

\medskip\noindent
Finally, suppose that $\mathcal{G}$ is in $\QCoh(\mathcal{O}_U)$.
Then $h^*\mathcal{G}$ is in $\QCoh(\mathcal{O}_{big, total})$.
Hence $h^*\mathcal{G} = a_{fppf}^*\mathcal{H}$ with
$\mathcal{H} = a_{fppf, *}h^*\mathcal{G}$
in $\QCoh(\mathcal{O}_{big})$ (see above).
In turn we see that $\mathcal{H} = (h_{-1})^*\mathcal{F}$
with $\mathcal{F} = h_{-1, *}\mathcal{H}$ in $\QCoh(\mathcal{O}_X)$.
Going around the diagram we deduce that
$h^*\mathcal{G} \cong h^*a^*\mathcal{F}$.
By fully faithfulness of $h^*$ we conclude that
$a^*\mathcal{F} \cong \mathcal{G}$.
Since $\mathcal{F} = h_{-1, *}a_{fppf, *}h^*\mathcal{G} =
a_*h_*h^*\mathcal{G} = a_*\mathcal{G}$ we also obtain
the statement that the quasi-inverse is given by $a_*$.
\end{proof}

\begin{lemma}
\label{lemma-cohomological-descent-qcoh-for-fppf-hypercovering}
Let $S$ be a scheme. Let $X$ be an algebraic space over $S$.
Let $U$ be a simplicial algebraic space over $S$. Let $a : U \to X$
be an augmentation. If $a : U \to X$ is an fppf hypercovering of $X$,
then for $\mathcal{F}$ a quasi-coherent $\mathcal{O}_X$-module
the map
$$
\mathcal{F} \to Ra_*(a^*\mathcal{F})
$$
is an isomorphism. Here $a : \Sh(U_\etale) \to \Sh(X_\etale)$
is as in Section \ref{section-simplicial-algebraic-spaces}.
\end{lemma}

\begin{proof}
Consider the diagram of Lemma \ref{lemma-compare-simplicial-objects-fppf-etale}.
Let $\mathcal{F}_n = a_n^*\mathcal{F}$ be the $n$th component of
$a^*\mathcal{F}$. This is a quasi-coherent $\mathcal{O}_{U_n}$-module.
Then $\mathcal{F}_n = Rh_{n, *}h_n^*\mathcal{F}_n$
by More on Cohomology of Spaces, Lemma
\ref{spaces-more-cohomology-lemma-cohomological-descent-etale-fppf-modules}.
Hence $a^*\mathcal{F} = Rh_*h^*a^*\mathcal{F}$ by
Lemma \ref{lemma-direct-image-morphism-simplicial-sites-modules}.
We have
\begin{align*}
Ra_*(a^*\mathcal{F})
& =
Ra_*Rh_*h^*a^*\mathcal{F} \\
& =
Rh_{-1, *}Ra_{fppf, *}a_{fppf}^*(h_{-1})^*\mathcal{F} \\
& =
Rh_{-1, *}(h_{-1})^*\mathcal{F} \\
& =
\mathcal{F}
\end{align*}
The first equality by the discussion above, the second equality
because of the commutativity of the diagram in
Lemma \ref{lemma-compare-simplicial-objects}, the third equality by
Lemma \ref{lemma-hypercovering-X-simple-descent-bounded-modules}
as $U$ is a hypercovering of $X$ in $(\textit{Spaces}/S)_{fppf}$
and $La_{fppf}^* = a_{fppf}^*$ as $a_{fppf}$ is flat
(namely $a_{fppf}^{-1}\mathcal{O}_{big} = \mathcal{O}_{big, total}$,
see Remark \ref{remark-augmentation-ringed}), and
the last equality by the already used
More on Cohomology of Spaces, Lemma
\ref{spaces-more-cohomology-lemma-cohomological-descent-etale-fppf-modules}.
\end{proof}

\begin{lemma}
\label{lemma-coh-descent-qcoh-unbounded-for-fppf-hypercovering}
Let $S$ be a scheme. Let $X$ be an algebraic space over $S$.
Let $U$ be a simplicial algebraic space over $S$. Let $a : U \to X$
be an augmentation. Assume $a : U \to X$ is an fppf hypercovering of $X$.
Then $\QCoh(\mathcal{O}_U)$ is a weak Serre subcategory of
$\textit{Mod}(\mathcal{O}_U)$ and
$$
a^* : D_\QCoh(\mathcal{O}_X) \longrightarrow D_\QCoh(\mathcal{O}_U)
$$
is an equivalence of categories with quasi-inverse given by
$Ra_*$. Here $a : \Sh(U_\etale) \to \Sh(X_\etale)$
is as in Section \ref{section-simplicial-algebraic-spaces}.
\end{lemma}

\begin{proof}
First observe that the maps $a_n : U_n \to X$ and $d^n_i : U_n \to U_{n - 1}$
are flat, locally of finite presentation, and surjective by
Hypercoverings, Remark \ref{hypercovering-remark-P-covering}.

\medskip\noindent
Recall that an $\mathcal{O}_U$-module $\mathcal{F}$ is quasi-coherent if and
only if it is cartesian and $\mathcal{F}_n$ is quasi-coherent for all $n$.
See Lemma \ref{lemma-quasi-coherent-sheaf}.
By Lemma \ref{lemma-Serre-subcat-cartesian-modules}
(and flatness of the maps $d^n_i : U_n \to U_{n - 1}$ shown above)
the cartesian modules for a weak Serre subcategory of
$\textit{Mod}(\mathcal{O}_U)$. On the other hand
$\QCoh(\mathcal{O}_{U_n}) \subset \textit{Mod}(\mathcal{O}_{U_n})$
is a weak Serre subcategory for each $n$
(Properties of Spaces, Lemma
\ref{spaces-properties-lemma-properties-quasi-coherent}).
Combined we see that
$\QCoh(\mathcal{O}_U) \subset \textit{Mod}(\mathcal{O}_U)$
is a weak Serre subcategory.

\medskip\noindent
To finish the proof we check the conditions (1) -- (5) of
Cohomology on Sites, Lemma
\ref{sites-cohomology-lemma-equivalence-unbounded-one} one by one.

\medskip\noindent
Ad (1). This holds since $a_n$ flat (seen above) implies $a$ is flat
by Lemma \ref{lemma-flat-augmentation-modules}.

\medskip\noindent
Ad (2). This is the content of
Lemma \ref{lemma-descent-qcoh-for-fppf-hypercovering}.

\medskip\noindent
Ad (3). This is the content of
Lemma \ref{lemma-cohomological-descent-qcoh-for-fppf-hypercovering}.

\medskip\noindent
Ad (4). Recall that we can use either the site $U_\etale$ or
$U_{spaces, \etale}$ to define the small \'etale topos
$\Sh(U_\etale)$, see Section \ref{section-simplicial-algebraic-spaces}.
The assumption of
Cohomology on Sites, Situation \ref{sites-cohomology-situation-olsson-laszlo}
holds for the triple
$(U_{spaces, \etale}, \mathcal{O}_U, \QCoh(\mathcal{O}_U))$
and by the same reasoning for the triple
$(U_\etale, \mathcal{O}_U, \QCoh(\mathcal{O}_U))$.
Namely, take
$$
\mathcal{B} \subset \Ob(U_\etale) \subset \Ob(U_{spaces, \etale})
$$
to be the set of affine objects. For $V/U_n \in \mathcal{B}$
take $d_{V/U_n} = 0$ and take $\text{Cov}_{V/U_n}$ to be the set of
\'etale coverings $\{V_i \to V\}$ with $V_i$ affine.
Then we get the desired vanishing because for
$\mathcal{F} \in \QCoh(\mathcal{O}_U)$
and any $V/U_n \in \mathcal{B}$ we have
$$
H^p(V/U_n, \mathcal{F}) = H^p(V, \mathcal{F}_n)
$$
by Lemma \ref{lemma-sanity-check-modules}. Here on the
right hand side we have the cohomology of the quasi-coherent
sheaf $\mathcal{F}_n$ on $U_n$ over the affine obect $V$
of $U_{n, \etale}$. This vanishes for $p > 0$ by the discussion in
Cohomology of Spaces, Section
\ref{spaces-cohomology-section-higher-direct-image} and
Cohomology of Schemes, Lemma
\ref{coherent-lemma-quasi-coherent-affine-cohomology-zero}.

\medskip\noindent
Ad (5). Follows by taking $\mathcal{B} \subset \Ob(X_{spaces, \etale})$
the set of affine objects and the references given above.
\end{proof}

\begin{lemma}
\label{lemma-compute-via-fppf-hypercovering-modules}
Let $S$ be a scheme. Let $X$ be an algebraic space over $S$.
Let $U$ be a simplicial algebraic space over $S$. Let $a : U \to X$
be an augmentation. If $a : U \to X$ is an fppf hypercovering of $X$, then
$$
R\Gamma(X_\etale, K) = R\Gamma(U_\etale, a^*K)
$$
for $K \in D_\QCoh(\mathcal{O}_X)$. Here $a : \Sh(U_\etale) \to \Sh(X_\etale)$
is as in Section \ref{section-simplicial-algebraic-spaces}.
\end{lemma}

\begin{proof}
This follows from
Lemma \ref{lemma-coh-descent-qcoh-unbounded-for-fppf-hypercovering}
because $R\Gamma(U_\etale, -) = R\Gamma(X_\etale, -) \circ Ra_*$ by
Cohomology on Sites, Remark \ref{sites-cohomology-remark-before-Leray}.
\end{proof}

\begin{lemma}
\label{lemma-spectral-sequence-fppf-hypercovering-modules}
Let $S$ be a scheme. Let $X$ be an algebraic space over $S$.
Let $U$ be a simplicial algebraic space over $S$. Let $a : U \to X$
be an augmentation. Let $\mathcal{F}$ be quasi-coherent
$\mathcal{O}_X$-module. Let $\mathcal{F}_n$ be the pullback to
$U_{n, \etale}$. If $U$ is an fppf hypercovering of $X$, then
there exists a canonical spectral sequence
$$
E_1^{p, q} = H^q_\etale(U_p, \mathcal{F}_p)
$$
converging to $H^{p + q}_\etale(X, \mathcal{F})$.
\end{lemma}

\begin{proof}
Immediate consequence of
Lemmas \ref{lemma-compute-via-fppf-hypercovering-modules}
and \ref{lemma-simplicial-module-cohomology-site}.
\end{proof}







\section{Fppf descent of complexes}
\label{section-fppf-descent-derived}

\noindent
In this section we pull some of the previously shown
results together for fppf coverings of algebraic spaces
and derived categories of quasi-coherent modules.

\begin{lemma}
\label{lemma-fppf-neg-ext-zero-hom}
Let $X$ be an algebraic space over a scheme $S$.
Let $K, E \in D_\QCoh(\mathcal{O}_X)$.
Let $a : U \to X$ be an fppf hypercovering.
Assume that for all $n \geq 0$ we have
$$
\Ext_{\mathcal{O}_{U_n}}^i(La_n^*K, La_n^*E) = 0
\text{ for } i < 0
$$
Then we have
\begin{enumerate}
\item $\Ext_{\mathcal{O}_X}^i(K, E) = 0$ for $i < 0$, and
\item there is an exact sequence
$$
0
\to
\Hom_{\mathcal{O}_X}(K, E)
\to
\Hom_{\mathcal{O}_{U_0}}(La_0^*K, La_0^*E)
\to
\Hom_{\mathcal{O}_{U_1}}(La_1^*K, La_1^*E)
$$
\end{enumerate}
\end{lemma}

\begin{proof}
Write $K_n = La_n^*K$ and $E_n = La_n^*E$. Then these are the
simplicial systems of the derived category of modules
(Definition \ref{definition-cartesian-derived-modules})
associated to $La^*K$ and $La^*E$
(Lemma \ref{lemma-cartesian-objects-derived-modules})
where $a : U_\etale \to X_\etale$ is as in
Section \ref{section-simplicial-algebraic-spaces}.
Let us prove (2) first. By
Lemma \ref{lemma-coh-descent-qcoh-unbounded-for-fppf-hypercovering}
we have
$$
\Hom_{\mathcal{O}_X}(K, E) =
\Hom_{\mathcal{O}_U}(La^*K, La^*E)
$$
Thus the sequence looks like this:
$$
0
\to
\Hom_{\mathcal{O}_U}(La^*K, La^*E)
\to
\Hom_{\mathcal{O}_{U_0}}(K_0, E_0)
\to
\Hom_{\mathcal{O}_{U_1}}(K_1, E_1)
$$
The first arrow  is injective by
Lemma \ref{lemma-nullity-cartesian-modules-derived}.
The image of this arrow is the kernel of the second
by Lemma \ref{lemma-hom-cartesian-modules-derived}.
This finishes the proof of (2).
Part (1) follows by applying part (2) with
$K[i]$ and $E$ for $i > 0$.
\end{proof}

\begin{lemma}
\label{lemma-fppf-glue-neg-ext-zero}
Let $X$ be an algebraic space over a scheme $S$.
Let $a : U \to X$ be an fppf hypercovering.
Suppose given $K_0 \in D_\QCoh(U_0)$ and an isomorphism
$$
\alpha :
L(f_{\delta_1^1})^*K_0
\longrightarrow
L(f_{\delta_0^1})^*K_0
$$
satisfying the cocycle condition on $U_1$. Set
$\tau^n_i : [0] \to [n]$, $0 \mapsto i$ and
set $K_n = Lf_{\tau^n_n}^*K_0$.
Assume $\Ext^i_{\mathcal{O}_{U_n}}(K_n, K_n) = 0$ for $i < 0$.
Then there exists an object $K \in D_\QCoh(\mathcal{O}_X)$
and an isomorphism $La_0^*K \to K$ compatible with $\alpha$.
\end{lemma}

\begin{proof}
We claim that the objects $K_n$ form the members of a 
simplicial system of the derived category of modules
(Definition \ref{definition-cartesian-derived-modules})
of the ringed simplicial site $U_\etale$ of
Section \ref{section-simplicial-algebraic-spaces}.
The construction is analogous to the construction discussed in
Descent, Section \ref{descent-section-descent-modules} from which we borrow
the notation $\tau^n_i : [0] \to [n]$, $0 \mapsto i$ and
$\tau^n_{ij} : [1] \to [n]$, $0 \mapsto i$, $1 \mapsto j$.
Given $\varphi : [n] \to [m]$ we define
$K_\varphi : L(f_\varphi)^*K_n \to K_m$
using
$$
\xymatrix{
L(f_\varphi)^*K_n \ar@{=}[r] &
L(f_\varphi)^* L(f_{\tau^n_n})^*K_0 \ar@{=}[r] &
L(f_{\tau^m_{\varphi(n)}})^*K_0 \ar@{=}[r] &
L(f_{\tau^m_{\varphi(n)m}})^* L(f_{\delta^1_1})^*K_0
\ar[d]_{L(f_{\tau^m_{\varphi(n)m}})^*\alpha} \\
&
K_m \ar@{=}[r] &
L(f_{\tau^m_m})^*K_0 \ar@{=}[r] &
L(f_{\tau^m_{\varphi(n)m}})^* L(f_{\delta^1_0})^*K_0
}
$$
We omit the verification that the cocycle condition
implies the maps compose correctly (in their respective
derived categories) and hence give rise to a
simplicial systems of the derived category of modules\footnote{This
verification is the same as that done in the proof
of Lemma \ref{lemma-characterize-cartesian}
as well as in the chapter on descent referenced
above. We should probably write this as a general lemma about
fibred and cofibred categories over $\Delta$.}.
Once this is verified, we obtain an object
$K' \in D_\QCoh(\mathcal{O}_{U_\etale})$
such that $(K_n, K_\varphi)$ is the system deduced from $K'$, see
Lemma \ref{lemma-cartesian-module-derived-from-simplicial}.
Finally, we apply
Lemma \ref{lemma-coh-descent-qcoh-unbounded-for-fppf-hypercovering}
to see that $K' = La^*K$ for some $K \in D_\QCoh(\mathcal{O}_X)$
as desired.
\end{proof}











\section{Proper hypercoverings of algebraic spaces}
\label{section-proper-hypercovering-spaces}

\noindent
This section is the analogue of Section \ref{section-proper-hypercovering}
for the case of algebraic spaces.
The reader who wishes to do so, can replace ``algebraic space''
everywhere with ``scheme'' and get equally valid results.
This has the advantage of replacing the references to
More on Cohomology of Spaces, Section
\ref{spaces-more-cohomology-section-ph-etale}
with references to
\'Etale Cohomology, Section \ref{etale-cohomology-section-ph-etale}.

\medskip\noindent
We fix a base scheme $S$.
Let $X$ be an algebraic space over $S$ and let $U$ be a simplicial
algebraic space over $S$. Assume we have an augmentation
$$
a : U \to X
$$
See Section \ref{section-simplicial-algebraic-spaces}.
We say that $U$ is a {\it proper hypercovering} of $X$ if
\begin{enumerate}
\item $U_0 \to X$ is proper and surjective,
\item $U_1 \to U_0 \times_X U_0$ is proper and surjective,
\item $U_{n + 1} \to (\text{cosk}_n\text{sk}_n U)_{n + 1}$
is proper and surjective for $n \geq 1$.
\end{enumerate}
The category of algebraic spaces over $S$ has all finite limits, hence the
coskeleta used in the formulation above exist.
$$
\fbox{Principle: Proper hypercoverings can be
used to compute \'etale cohomology.}
$$
The key idea behind the proof of the principle is to compare the
ph and \'etale topologies on the category $\textit{Spaces}/S$.
Namely, the ph topology is stronger than the \'etale topology and we have
(a) a proper surjective map defines a ph covering, and
(b) ph cohomology of sheaves pulled back from the small \'etale site
agrees with \'etale cohomology as we have seen in
More on Cohomology of Spaces, Section
\ref{spaces-more-cohomology-section-ph-etale}.

\medskip\noindent
All results in this section generalize to the case
where $U \to X$ is merely a ``ph hypercovering'', meaning a
hypercovering of $X$ in the site $(\textit{Spaces}/S)_{ph}$
as defined in Section \ref{section-hypercovering}. If we ever need
this, we will precisely formulate and prove this here.

\begin{lemma}
\label{lemma-compare-simplicial-objects-ph-etale}
Let $S$ be a scheme. Let $X$ be an algebraic space over $S$.
Let $U$ be a simplicial algebraic space over $S$.
Let $a : U \to X$ be an augmentation. There is a commutative diagram
$$
\xymatrix{
\Sh((\textit{Spaces}/U)_{ph, total}) \ar[r]_-h \ar[d]_{a_{ph}} &
\Sh(U_\etale) \ar[d]^a \\
\Sh((\textit{Spaces}/X)_{ph}) \ar[r]^-{h_{-1}} &
\Sh(X_\etale)
}
$$
where the left vertical arrow is defined in
Section \ref{section-hypercovering}
and the right vertical arrow is defined in
Section \ref{section-simplicial-algebraic-spaces}.
\end{lemma}

\begin{proof}
The notation $(\textit{Spaces}/U)_{ph, total}$ indicates that
we are using the construction of
Section \ref{section-hypercovering}
for the site $(\textit{Spaces}/S)_{ph}$ and the
simplicial object $U$ of this site\footnote{To distinguish from
$(\textit{Spaces}/U)_{fppf, total}$ defined using the fppf
topology in Section \ref{section-fppf-hypercovering}.}.
We will use the sites $X_{spaces, \etale}$ and $U_{spaces, \etale}$
for the topoi on the right hand side; this is permissible
see discussion in Section \ref{section-simplicial-algebraic-spaces}.

\medskip\noindent
Observe that both $(\textit{Spaces}/U)_{ph, total}$ and
$U_{spaces, \etale}$
fall into case A of Situation \ref{situation-simplicial-site}.
This is immediate from the construction of
$U_\etale$ in Section \ref{section-simplicial-algebraic-spaces}
and it follows from Lemma \ref{lemma-sr-when-fibre-products}
for $(\textit{Spaces}/U)_{ph, total}$.
Next, consider the functors
$U_{n, spaces, \etale} \to (\textit{Spaces}/U_n)_{ph}$, $U \mapsto U/U_n$
and
$X_{spaces, \etale} \to (\textit{Spaces}/X)_{ph}$, $U \mapsto U/X$.
We have seen that these define morphisms of sites in
More on Cohomology of Spaces, Section
\ref{spaces-more-cohomology-section-ph-etale}
where these were denoted $a_{U_n} = \epsilon_{U_n} \circ \pi_{u_n}$
and $a_X = \epsilon_X \circ \pi_X$.
Thus we obtain a morphism of simplicial sites compatible with
augmentations as in Remark \ref{remark-morphism-augmentation-simplicial-sites}
and we may apply
Lemma \ref{lemma-morphism-augmentation-simplicial-sites} to conclude.
\end{proof}

\begin{lemma}
\label{lemma-descent-sheaves-for-ph-hypercovering}
Let $S$ be a scheme. Let $X$ be an algebraic space over $S$.
Let $U$ be a simplicial algebraic space over $S$. Let $a : U \to X$
be an augmentation. If $a : U \to X$ is a proper hypercovering of $X$,
then
$$
a^{-1} : \Sh(X_\etale) \to \Sh(U_\etale)
\quad\text{and}\quad
a^{-1} : \textit{Ab}(X_\etale) \to \textit{Ab}(U_\etale)
$$
are fully faithful with essential image the cartesian sheaves and
quasi-inverse given by $a_*$. Here $a : \Sh(U_\etale) \to \Sh(X_\etale)$
is as in Section \ref{section-simplicial-algebraic-spaces}.
\end{lemma}

\begin{proof}
We will prove the statement for sheaves of sets. It will be an
almost formal consequence of results already established.
Consider the diagram of
Lemma \ref{lemma-compare-simplicial-objects-ph-etale}.
In the proof of this lemma we have seen that
$h_{-1}$ is the morphism $a_X$ of
More on Cohomology of Spaces, Section
\ref{spaces-more-cohomology-section-ph-etale}.
Thus it follows from
More on Cohomology of Spaces, Lemma
\ref{spaces-more-cohomology-lemma-comparison-ph-etale}
that $(h_{-1})^{-1}$ is fully faithful with quasi-inverse $h_{-1, *}$.
The same holds true for the components $h_n$ of $h$.
By the description of the functors $h^{-1}$ and $h_*$ of
Lemma \ref{lemma-morphism-simplicial-sites}
we conclude that $h^{-1}$ is fully faithful with quasi-inverse $h_*$.
Observe that $U$ is a hypercovering of $X$ in $(\textit{Spaces}/S)_{ph}$
as defined in Section \ref{section-hypercovering} since a surjective
proper morphism gives a ph covering by Topologies on Spaces, Lemma
\ref{spaces-topologies-lemma-surjective-proper-ph}.
By Lemma \ref{lemma-hypercovering-X-simple-descent-sheaves}
we see that $a_{ph}^{-1}$ is fully faithful with quasi-inverse
$a_{ph, *}$ and with essential image the cartesian sheaves
on $(\textit{Spaces}/U)_{ph, total}$.
A formal argument (chasing around the diagram) now shows that
$a^{-1}$ is fully faithful.

\medskip\noindent
Finally, suppose that $\mathcal{G}$ is a cartesian sheaf on $U_\etale$.
Then $h^{-1}\mathcal{G}$ is a cartesian sheaf on
$(\textit{Spaces}/U)_{ph, total}$.
Hence $h^{-1}\mathcal{G} = a_{ph}^{-1}\mathcal{H}$ for some sheaf
$\mathcal{H}$ on $(\textit{Spaces}/X)_{ph}$.
We compute using somewhat pedantic notation
\begin{align*}
(h_{-1})^{-1}(a_*\mathcal{G})
& =
(h_{-1})^{-1}
\text{Eq}(
\xymatrix{
a_{0, small, *}\mathcal{G}_0
\ar@<1ex>[r] \ar@<-1ex>[r] &
a_{1, small, *}\mathcal{G}_1
}
) \\
& =
\text{Eq}(
\xymatrix{
(h_{-1})^{-1}a_{0, small, *}\mathcal{G}_0
\ar@<1ex>[r] \ar@<-1ex>[r] &
(h_{-1})^{-1}a_{1, small, *}\mathcal{G}_1
}
) \\
& =
\text{Eq}(
\xymatrix{
a_{0, big, ph, *}h_0^{-1}\mathcal{G}_0
\ar@<1ex>[r] \ar@<-1ex>[r] &
a_{1, big, ph, *}h_1^{-1}\mathcal{G}_1
}
) \\
& =
\text{Eq}(
\xymatrix{
a_{0, big, ph, *}(a_{0, big, ph})^{-1}\mathcal{H}
\ar@<1ex>[r] \ar@<-1ex>[r] &
a_{1, big, ph, *}(a_{1, big, ph})^{-1}\mathcal{H}
}
) \\
& =
a_{ph, *}a_{ph}^{-1}\mathcal{H} \\
& =
\mathcal{H}
\end{align*}
Here the first equality follows from Lemma \ref{lemma-augmentation-site},
the second equality follows as $(h_{-1})^{-1}$ is an exact functor,
the third equality follows from
More on Cohomology of Spaces, Lemma
\ref{spaces-more-cohomology-lemma-proper-push-pull-ph-etale}
(here we use that $a_0 : U_0 \to X$ and $a_1: U_1 \to X$ are proper),
the fourth follows from $a_{ph}^{-1}\mathcal{H} = h^{-1}\mathcal{G}$,
the fifth from Lemma \ref{lemma-augmentation-site}, and the
sixth we've seen above. Since $a_{ph}^{-1}\mathcal{H} = h^{-1}\mathcal{G}$
we deduce that $h^{-1}\mathcal{G} \cong h^{-1}a^{-1}a_*\mathcal{G}$
which ends the proof by fully faithfulness of $h^{-1}$.
\end{proof}

\begin{lemma}
\label{lemma-cohomological-descent-for-ph-hypercovering}
Let $S$ be a scheme. Let $X$ be an algebraic space over $S$.
Let $U$ be a simplicial algebraic space over $S$. Let $a : U \to X$
be an augmentation. If $a : U \to X$ is a proper hypercovering of $X$,
then for $K \in D^+(X_\etale)$
$$
K \to Ra_*(a^{-1}K)
$$
is an isomorphism. Here $a : \Sh(U_\etale) \to \Sh(X_\etale)$
is as in Section \ref{section-simplicial-algebraic-spaces}.
\end{lemma}

\begin{proof}
Consider the diagram of Lemma \ref{lemma-compare-simplicial-objects-ph-etale}.
Observe that $Rh_{n, *}h_n^{-1}$ is the identity functor
on $D^+(U_{n, \etale})$ by
More on Cohomology of Spaces, Lemma
\ref{spaces-more-cohomology-lemma-cohomological-descent-etale-ph}.
Hence $Rh_*h^{-1}$ is the identity functor on
$D^+(U_\etale)$ by
Lemma \ref{lemma-direct-image-morphism-simplicial-sites}.
We have
\begin{align*}
Ra_*(a^{-1}K)
& =
Ra_*Rh_*h^{-1}a^{-1}K \\
& =
Rh_{-1, *}Ra_{ph, *}a_{ph}^{-1}(h_{-1})^{-1}K \\
& =
Rh_{-1, *}(h_{-1})^{-1}K \\
& =
K
\end{align*}
The first equality by the discussion above, the second equality
because of the commutativity of the diagram in
Lemma \ref{lemma-compare-simplicial-objects}, the third equality by
Lemma \ref{lemma-hypercovering-X-simple-descent-bounded-abelian}
as $U$ is a hypercovering of $X$ in $(\textit{Spaces}/S)_{ph}$
by Topologies on Spaces, Lemma
\ref{spaces-topologies-lemma-surjective-proper-ph},
and the last equality by the already used
More on Cohomology of Spaces, Lemma
\ref{spaces-more-cohomology-lemma-cohomological-descent-etale-ph}.
\end{proof}

\begin{lemma}
\label{lemma-compute-via-ph-hypercovering}
Let $S$ be a scheme. Let $X$ be an algebraic space over $S$.
Let $U$ be a simplicial algebraic space over $S$. Let $a : U \to X$
be an augmentation. If $a : U \to X$ is a proper hypercovering of $X$, then
$$
R\Gamma(X_\etale, K) = R\Gamma(U_\etale, a^{-1}K)
$$
for $K \in D^+(X_\etale)$. Here $a : \Sh(U_\etale) \to \Sh(X_\etale)$
is as in Section \ref{section-simplicial-algebraic-spaces}.
\end{lemma}

\begin{proof}
This follows from
Lemma \ref{lemma-cohomological-descent-for-ph-hypercovering}
because $R\Gamma(U_\etale, -) = R\Gamma(X_\etale, -) \circ Ra_*$ by
Cohomology on Sites, Remark \ref{sites-cohomology-remark-before-Leray}.
\end{proof}

\begin{lemma}
\label{lemma-ph-hypercovering-equivalence-bounded}
Let $S$ be a scheme. Let $X$ be an algebraic space over $S$.
Let $U$ be a simplicial algebraic space over $S$. Let $a : U \to X$
be an augmentation.
Let $\mathcal{A} \subset \textit{Ab}(U_\etale)$
denote the weak Serre subcategory of cartesian abelian sheaves.
If $U$ is a proper hypercovering of $X$, then
the functor $a^{-1}$ defines an equivalence
$$
D^+(X_\etale) \longrightarrow D_\mathcal{A}^+(U_\etale)
$$
with quasi-inverse $Ra_*$. Here $a : \Sh(U_\etale) \to \Sh(X_\etale)$
is as in Section \ref{section-simplicial-algebraic-spaces}.
\end{lemma}

\begin{proof}
Observe that $\mathcal{A}$ is a weak Serre subcategory by
Lemma \ref{lemma-Serre-subcat-cartesian-modules}.
The equivalence is a
formal consequence of the results obtained so far. Use
Lemmas \ref{lemma-descent-sheaves-for-ph-hypercovering} and
\ref{lemma-cohomological-descent-for-ph-hypercovering} and
Cohomology on Sites, Lemma \ref{sites-cohomology-lemma-equivalence-bounded}.
\end{proof}

\begin{lemma}
\label{lemma-spectral-sequence-ph-hypercovering}
Let $S$ be a scheme. Let $X$ be an algebraic space over $S$.
Let $U$ be a simplicial algebraic space over $S$. Let $a : U \to X$
be an augmentation. Let $\mathcal{F}$ be an abelian sheaf
on $X_\etale$. Let $\mathcal{F}_n$ be the pullback to $U_{n, \etale}$.
If $U$ is a ph hypercovering of $X$, then
there exists a canonical spectral sequence
$$
E_1^{p, q} = H^q_\etale(U_p, \mathcal{F}_p)
$$
converging to $H^{p + q}_\etale(X, \mathcal{F})$.
\end{lemma}

\begin{proof}
Immediate consequence of Lemmas \ref{lemma-compute-via-ph-hypercovering}
and \ref{lemma-simplicial-sheaf-cohomology-site}.
\end{proof}










\begin{multicols}{2}[\section{Other chapters}]
\noindent
Preliminaries
\begin{enumerate}
\item \hyperref[introduction-section-phantom]{Introduction}
\item \hyperref[conventions-section-phantom]{Conventions}
\item \hyperref[sets-section-phantom]{Set Theory}
\item \hyperref[categories-section-phantom]{Categories}
\item \hyperref[topology-section-phantom]{Topology}
\item \hyperref[sheaves-section-phantom]{Sheaves on Spaces}
\item \hyperref[sites-section-phantom]{Sites and Sheaves}
\item \hyperref[stacks-section-phantom]{Stacks}
\item \hyperref[fields-section-phantom]{Fields}
\item \hyperref[algebra-section-phantom]{Commutative Algebra}
\item \hyperref[brauer-section-phantom]{Brauer Groups}
\item \hyperref[homology-section-phantom]{Homological Algebra}
\item \hyperref[derived-section-phantom]{Derived Categories}
\item \hyperref[simplicial-section-phantom]{Simplicial Methods}
\item \hyperref[more-algebra-section-phantom]{More on Algebra}
\item \hyperref[smoothing-section-phantom]{Smoothing Ring Maps}
\item \hyperref[modules-section-phantom]{Sheaves of Modules}
\item \hyperref[sites-modules-section-phantom]{Modules on Sites}
\item \hyperref[injectives-section-phantom]{Injectives}
\item \hyperref[cohomology-section-phantom]{Cohomology of Sheaves}
\item \hyperref[sites-cohomology-section-phantom]{Cohomology on Sites}
\item \hyperref[dga-section-phantom]{Differential Graded Algebra}
\item \hyperref[dpa-section-phantom]{Divided Power Algebra}
\item \hyperref[hypercovering-section-phantom]{Hypercoverings}
\end{enumerate}
Schemes
\begin{enumerate}
\setcounter{enumi}{24}
\item \hyperref[schemes-section-phantom]{Schemes}
\item \hyperref[constructions-section-phantom]{Constructions of Schemes}
\item \hyperref[properties-section-phantom]{Properties of Schemes}
\item \hyperref[morphisms-section-phantom]{Morphisms of Schemes}
\item \hyperref[coherent-section-phantom]{Cohomology of Schemes}
\item \hyperref[divisors-section-phantom]{Divisors}
\item \hyperref[limits-section-phantom]{Limits of Schemes}
\item \hyperref[varieties-section-phantom]{Varieties}
\item \hyperref[topologies-section-phantom]{Topologies on Schemes}
\item \hyperref[descent-section-phantom]{Descent}
\item \hyperref[perfect-section-phantom]{Derived Categories of Schemes}
\item \hyperref[more-morphisms-section-phantom]{More on Morphisms}
\item \hyperref[flat-section-phantom]{More on Flatness}
\item \hyperref[groupoids-section-phantom]{Groupoid Schemes}
\item \hyperref[more-groupoids-section-phantom]{More on Groupoid Schemes}
\item \hyperref[etale-section-phantom]{\'Etale Morphisms of Schemes}
\end{enumerate}
Topics in Scheme Theory
\begin{enumerate}
\setcounter{enumi}{40}
\item \hyperref[chow-section-phantom]{Chow Homology}
\item \hyperref[intersection-section-phantom]{Intersection Theory}
\item \hyperref[weil-section-phantom]{Weil Cohomology Theories}
\item \hyperref[pic-section-phantom]{Picard Schemes of Curves}
\item \hyperref[adequate-section-phantom]{Adequate Modules}
\item \hyperref[dualizing-section-phantom]{Dualizing Complexes}
\item \hyperref[duality-section-phantom]{Duality for Schemes}
\item \hyperref[discriminant-section-phantom]{Discriminants and Differents}
\item \hyperref[local-cohomology-section-phantom]{Local Cohomology}
\item \hyperref[algebraization-section-phantom]{Algebraic and Formal Geometry}
\item \hyperref[curves-section-phantom]{Algebraic Curves}
\item \hyperref[resolve-section-phantom]{Resolution of Surfaces}
\item \hyperref[models-section-phantom]{Semistable Reduction}
\item \hyperref[pione-section-phantom]{Fundamental Groups of Schemes}
\item \hyperref[etale-cohomology-section-phantom]{\'Etale Cohomology}
\item \hyperref[crystalline-section-phantom]{Crystalline Cohomology}
\item \hyperref[proetale-section-phantom]{Pro-\'etale Cohomology}
\item \hyperref[more-etale-section-phantom]{More \'Etale Cohomology}
\item \hyperref[trace-section-phantom]{The Trace Formula}
\end{enumerate}
Algebraic Spaces
\begin{enumerate}
\setcounter{enumi}{59}
\item \hyperref[spaces-section-phantom]{Algebraic Spaces}
\item \hyperref[spaces-properties-section-phantom]{Properties of Algebraic Spaces}
\item \hyperref[spaces-morphisms-section-phantom]{Morphisms of Algebraic Spaces}
\item \hyperref[decent-spaces-section-phantom]{Decent Algebraic Spaces}
\item \hyperref[spaces-cohomology-section-phantom]{Cohomology of Algebraic Spaces}
\item \hyperref[spaces-limits-section-phantom]{Limits of Algebraic Spaces}
\item \hyperref[spaces-divisors-section-phantom]{Divisors on Algebraic Spaces}
\item \hyperref[spaces-over-fields-section-phantom]{Algebraic Spaces over Fields}
\item \hyperref[spaces-topologies-section-phantom]{Topologies on Algebraic Spaces}
\item \hyperref[spaces-descent-section-phantom]{Descent and Algebraic Spaces}
\item \hyperref[spaces-perfect-section-phantom]{Derived Categories of Spaces}
\item \hyperref[spaces-more-morphisms-section-phantom]{More on Morphisms of Spaces}
\item \hyperref[spaces-flat-section-phantom]{Flatness on Algebraic Spaces}
\item \hyperref[spaces-groupoids-section-phantom]{Groupoids in Algebraic Spaces}
\item \hyperref[spaces-more-groupoids-section-phantom]{More on Groupoids in Spaces}
\item \hyperref[bootstrap-section-phantom]{Bootstrap}
\item \hyperref[spaces-pushouts-section-phantom]{Pushouts of Algebraic Spaces}
\end{enumerate}
Topics in Geometry
\begin{enumerate}
\setcounter{enumi}{76}
\item \hyperref[spaces-chow-section-phantom]{Chow Groups of Spaces}
\item \hyperref[groupoids-quotients-section-phantom]{Quotients of Groupoids}
\item \hyperref[spaces-more-cohomology-section-phantom]{More on Cohomology of Spaces}
\item \hyperref[spaces-simplicial-section-phantom]{Simplicial Spaces}
\item \hyperref[spaces-duality-section-phantom]{Duality for Spaces}
\item \hyperref[formal-spaces-section-phantom]{Formal Algebraic Spaces}
\item \hyperref[restricted-section-phantom]{Restricted Power Series}
\item \hyperref[spaces-resolve-section-phantom]{Resolution of Surfaces Revisited}
\end{enumerate}
Deformation Theory
\begin{enumerate}
\setcounter{enumi}{84}
\item \hyperref[formal-defos-section-phantom]{Formal Deformation Theory}
\item \hyperref[defos-section-phantom]{Deformation Theory}
\item \hyperref[cotangent-section-phantom]{The Cotangent Complex}
\item \hyperref[examples-defos-section-phantom]{Deformation Problems}
\end{enumerate}
Algebraic Stacks
\begin{enumerate}
\setcounter{enumi}{88}
\item \hyperref[algebraic-section-phantom]{Algebraic Stacks}
\item \hyperref[examples-stacks-section-phantom]{Examples of Stacks}
\item \hyperref[stacks-sheaves-section-phantom]{Sheaves on Algebraic Stacks}
\item \hyperref[criteria-section-phantom]{Criteria for Representability}
\item \hyperref[artin-section-phantom]{Artin's Axioms}
\item \hyperref[quot-section-phantom]{Quot and Hilbert Spaces}
\item \hyperref[stacks-properties-section-phantom]{Properties of Algebraic Stacks}
\item \hyperref[stacks-morphisms-section-phantom]{Morphisms of Algebraic Stacks}
\item \hyperref[stacks-limits-section-phantom]{Limits of Algebraic Stacks}
\item \hyperref[stacks-cohomology-section-phantom]{Cohomology of Algebraic Stacks}
\item \hyperref[stacks-perfect-section-phantom]{Derived Categories of Stacks}
\item \hyperref[stacks-introduction-section-phantom]{Introducing Algebraic Stacks}
\item \hyperref[stacks-more-morphisms-section-phantom]{More on Morphisms of Stacks}
\item \hyperref[stacks-geometry-section-phantom]{The Geometry of Stacks}
\end{enumerate}
Topics in Moduli Theory
\begin{enumerate}
\setcounter{enumi}{102}
\item \hyperref[moduli-section-phantom]{Moduli Stacks}
\item \hyperref[moduli-curves-section-phantom]{Moduli of Curves}
\end{enumerate}
Miscellany
\begin{enumerate}
\setcounter{enumi}{104}
\item \hyperref[examples-section-phantom]{Examples}
\item \hyperref[exercises-section-phantom]{Exercises}
\item \hyperref[guide-section-phantom]{Guide to Literature}
\item \hyperref[desirables-section-phantom]{Desirables}
\item \hyperref[coding-section-phantom]{Coding Style}
\item \hyperref[obsolete-section-phantom]{Obsolete}
\item \hyperref[fdl-section-phantom]{GNU Free Documentation License}
\item \hyperref[index-section-phantom]{Auto Generated Index}
\end{enumerate}
\end{multicols}


\bibliography{my}
\bibliographystyle{amsalpha}

\end{document}
