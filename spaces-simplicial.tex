\IfFileExists{stacks-project.cls}{%
\documentclass{stacks-project}
}{%
\documentclass{amsart}
}

% The following AMS packages are automatically loaded with
% the amsart documentclass:
%\usepackage{amsmath}
%\usepackage{amssymb}
%\usepackage{amsthm}

% For dealing with references we use the comment environment
\usepackage{verbatim}
\newenvironment{reference}{\comment}{\endcomment}
%\newenvironment{reference}{}{}
\newenvironment{slogan}{\comment}{\endcomment}
\newenvironment{history}{\comment}{\endcomment}

% For commutative diagrams you can use
% \usepackage{amscd}
\usepackage[all]{xy}

% We use 2cell for 2-commutative diagrams.
\xyoption{2cell}
\UseAllTwocells

% To put source file link in headers.
% Change "template.tex" to "this_filename.tex"
% \usepackage{fancyhdr}
% \pagestyle{fancy}
% \lhead{}
% \chead{}
% \rhead{Source file: \url{template.tex}}
% \lfoot{}
% \cfoot{\thepage}
% \rfoot{}
% \renewcommand{\headrulewidth}{0pt}
% \renewcommand{\footrulewidth}{0pt}
% \renewcommand{\headheight}{12pt}

\usepackage{multicol}

% For cross-file-references
\usepackage{xr-hyper}

% Package for hypertext links:
\usepackage{hyperref}

% For any local file, say "hello.tex" you want to link to please
% use \externaldocument[hello-]{hello}
\externaldocument[introduction-]{introduction}
\externaldocument[conventions-]{conventions}
\externaldocument[sets-]{sets}
\externaldocument[categories-]{categories}
\externaldocument[topology-]{topology}
\externaldocument[sheaves-]{sheaves}
\externaldocument[sites-]{sites}
\externaldocument[stacks-]{stacks}
\externaldocument[fields-]{fields}
\externaldocument[algebra-]{algebra}
\externaldocument[brauer-]{brauer}
\externaldocument[homology-]{homology}
\externaldocument[derived-]{derived}
\externaldocument[simplicial-]{simplicial}
\externaldocument[more-algebra-]{more-algebra}
\externaldocument[smoothing-]{smoothing}
\externaldocument[modules-]{modules}
\externaldocument[sites-modules-]{sites-modules}
\externaldocument[injectives-]{injectives}
\externaldocument[cohomology-]{cohomology}
\externaldocument[sites-cohomology-]{sites-cohomology}
\externaldocument[dga-]{dga}
\externaldocument[dpa-]{dpa}
\externaldocument[hypercovering-]{hypercovering}
\externaldocument[schemes-]{schemes}
\externaldocument[constructions-]{constructions}
\externaldocument[properties-]{properties}
\externaldocument[morphisms-]{morphisms}
\externaldocument[coherent-]{coherent}
\externaldocument[divisors-]{divisors}
\externaldocument[limits-]{limits}
\externaldocument[varieties-]{varieties}
\externaldocument[topologies-]{topologies}
\externaldocument[descent-]{descent}
\externaldocument[perfect-]{perfect}
\externaldocument[more-morphisms-]{more-morphisms}
\externaldocument[flat-]{flat}
\externaldocument[groupoids-]{groupoids}
\externaldocument[more-groupoids-]{more-groupoids}
\externaldocument[etale-]{etale}
\externaldocument[chow-]{chow}
\externaldocument[intersection-]{intersection}
\externaldocument[pic-]{pic}
\externaldocument[adequate-]{adequate}
\externaldocument[dualizing-]{dualizing}
\externaldocument[duality-]{duality}
\externaldocument[discriminant-]{discriminant}
\externaldocument[local-cohomology-]{local-cohomology}
\externaldocument[curves-]{curves}
\externaldocument[resolve-]{resolve}
\externaldocument[models-]{models}
\externaldocument[pione-]{pione}
\externaldocument[etale-cohomology-]{etale-cohomology}
\externaldocument[proetale-]{proetale}
\externaldocument[crystalline-]{crystalline}
\externaldocument[spaces-]{spaces}
\externaldocument[spaces-properties-]{spaces-properties}
\externaldocument[spaces-morphisms-]{spaces-morphisms}
\externaldocument[decent-spaces-]{decent-spaces}
\externaldocument[spaces-cohomology-]{spaces-cohomology}
\externaldocument[spaces-limits-]{spaces-limits}
\externaldocument[spaces-divisors-]{spaces-divisors}
\externaldocument[spaces-over-fields-]{spaces-over-fields}
\externaldocument[spaces-topologies-]{spaces-topologies}
\externaldocument[spaces-descent-]{spaces-descent}
\externaldocument[spaces-perfect-]{spaces-perfect}
\externaldocument[spaces-more-morphisms-]{spaces-more-morphisms}
\externaldocument[spaces-flat-]{spaces-flat}
\externaldocument[spaces-groupoids-]{spaces-groupoids}
\externaldocument[spaces-more-groupoids-]{spaces-more-groupoids}
\externaldocument[bootstrap-]{bootstrap}
\externaldocument[spaces-pushouts-]{spaces-pushouts}
\externaldocument[groupoids-quotients-]{groupoids-quotients}
\externaldocument[spaces-more-cohomology-]{spaces-more-cohomology}
\externaldocument[spaces-simplicial-]{spaces-simplicial}
\externaldocument[formal-spaces-]{formal-spaces}
\externaldocument[restricted-]{restricted}
\externaldocument[spaces-resolve-]{spaces-resolve}
\externaldocument[formal-defos-]{formal-defos}
\externaldocument[defos-]{defos}
\externaldocument[cotangent-]{cotangent}
\externaldocument[examples-defos-]{examples-defos}
\externaldocument[algebraic-]{algebraic}
\externaldocument[examples-stacks-]{examples-stacks}
\externaldocument[stacks-sheaves-]{stacks-sheaves}
\externaldocument[criteria-]{criteria}
\externaldocument[artin-]{artin}
\externaldocument[quot-]{quot}
\externaldocument[stacks-properties-]{stacks-properties}
\externaldocument[stacks-morphisms-]{stacks-morphisms}
\externaldocument[stacks-limits-]{stacks-limits}
\externaldocument[stacks-cohomology-]{stacks-cohomology}
\externaldocument[stacks-perfect-]{stacks-perfect}
\externaldocument[stacks-introduction-]{stacks-introduction}
\externaldocument[stacks-more-morphisms-]{stacks-more-morphisms}
\externaldocument[stacks-geometry-]{stacks-geometry}
\externaldocument[moduli-]{moduli}
\externaldocument[moduli-curves-]{moduli-curves}
\externaldocument[examples-]{examples}
\externaldocument[exercises-]{exercises}
\externaldocument[guide-]{guide}
\externaldocument[desirables-]{desirables}
\externaldocument[coding-]{coding}
\externaldocument[obsolete-]{obsolete}
\externaldocument[fdl-]{fdl}
\externaldocument[index-]{index}

% Theorem environments.
%
\theoremstyle{plain}
\newtheorem{theorem}[subsection]{Theorem}
\newtheorem{proposition}[subsection]{Proposition}
\newtheorem{lemma}[subsection]{Lemma}

\theoremstyle{definition}
\newtheorem{definition}[subsection]{Definition}
\newtheorem{example}[subsection]{Example}
\newtheorem{exercise}[subsection]{Exercise}
\newtheorem{situation}[subsection]{Situation}

\theoremstyle{remark}
\newtheorem{remark}[subsection]{Remark}
\newtheorem{remarks}[subsection]{Remarks}

\numberwithin{equation}{subsection}

% Macros
%
\def\lim{\mathop{\rm lim}\nolimits}
\def\colim{\mathop{\rm colim}\nolimits}
\def\Spec{\mathop{\rm Spec}}
\def\Hom{\mathop{\rm Hom}\nolimits}
\def\Ext{\mathop{\rm Ext}\nolimits}
\def\SheafHom{\mathop{\mathcal{H}\!{\it om}}\nolimits}
\def\SheafExt{\mathop{\mathcal{E}\!{\it xt}}\nolimits}
\def\Sch{\textit{Sch}}
\def\Mor{\mathop{\rm Mor}\nolimits}
\def\Ob{\mathop{\rm Ob}\nolimits}
\def\Sh{\mathop{\textit{Sh}}\nolimits}
\def\NL{\mathop{N\!L}\nolimits}
\def\proetale{{pro\text{-}\acute{e}tale}}
\def\etale{{\acute{e}tale}}
\def\QCoh{\textit{QCoh}}
\def\Ker{\mathop{\rm Ker}}
\def\Im{\mathop{\rm Im}}
\def\Coker{\mathop{\rm Coker}}
\def\Coim{\mathop{\rm Coim}}

%
% Macros for moduli stacks/spaces
%
\def\QCohstack{\mathcal{QC}\!{\it oh}}
\def\Cohstack{\mathcal{C}\!{\it oh}}
\def\Spacesstack{\mathcal{S}\!{\it paces}}
\def\Quotfunctor{{\rm Quot}}
\def\Hilbfunctor{{\rm Hilb}}
\def\Curvesstack{\mathcal{C}\!{\it urves}}
\def\Polarizedstack{\mathcal{P}\!{\it olarized}}
\def\Complexesstack{\mathcal{C}\!{\it omplexes}}
% \Pic is the operator that assigns to X its picard group, usage \Pic(X)
% \Picardstack_{X/B} denotes the Picard stack of X over B
% \Picardfunctor_{X/B} denotes the Picard functor of X over B
\def\Pic{\mathop{\rm Pic}\nolimits}
\def\Picardstack{\mathcal{P}\!{\it ic}}
\def\Picardfunctor{{\rm Pic}}
\def\Deformationcategory{\mathcal{D}\!{\it ef}}


% OK, start here.
%
\begin{document}

\title{Simplicial Spaces}


\maketitle

\phantomsection
\label{section-phantom}

\tableofcontents

\section{Introduction}
\label{section-introduction}

\noindent
This chapter develops some theory concerning simplicial topological spaces,
simplicial ringed spaces, simplicial schemes, and simplicial algebraic spaces.
The theory of simplicial spaces sometimes allows one to prove local to global
principles which appear difficult to prove in other ways.
Some example applications can be found in the papers
\cite{faltings_finiteness}, \cite{Kiehl}, and \cite{HodgeIII}.



\section{Conventions}
\label{section-conventions}

\noindent
The standing assumption is that all schemes are contained in
a big fppf site $\Sch_{fppf}$. And all rings $A$ considered
have the property that $\Spec(A)$ is (isomorphic) to an
object of this big site.

\medskip\noindent
Let $S$ be a scheme and let $X$ be an algebraic space over $S$.
In this chapter and the following we will write $X \times_S X$
for the product of $X$ with itself (in the category of algebraic
spaces over $S$), instead of $X \times X$.

\medskip\noindent
We continue our convention to label projection maps starting with
index $0$, so we have $\text{pr}_0 : X \times_S Y \to X$ and
$\text{pr}_1 : X \times_S Y \to Y$.




\section{Descent in terms of simplicial schemes}
\label{section-simplicial}

\noindent
A {\it simplicial scheme} is a simplicial object in the category of schemes,
see Simplicial, Definition \ref{simplicial-definition-simplicial-object}.
In this chapter we will use a subscript $\bullet$ to denote simplicial
objects. Recall that a simplicial scheme looks like
$$
\xymatrix{
X_2
\ar@<2ex>[r]
\ar@<0ex>[r]
\ar@<-2ex>[r]
&
X_1
\ar@<1ex>[r]
\ar@<-1ex>[r]
\ar@<1ex>[l]
\ar@<-1ex>[l]
&
X_0
\ar@<0ex>[l]
}
$$
Here there are two morphisms $d^1_0, d^1_1 : X_1 \to X_0$
and a single morphism $s^0_0 : X_0 \to X_1$, etc.
It is important to keep in mind that $d^n_i : X_n \to X_{n - 1}$
should be thought of as a ``projection forgetting the
$i$th coordinate'' and $s^n_j : X_n \to X_{n + 1}$ as the diagonal
map repeating the $j$th coordinate.

\begin{definition}
\label{definition-cartesian-morphism}
Let $a : V_\bullet \to X_\bullet$ be a morphism of simplicial schemes.
We say $a$ is {\it cartesian}, or that {\it $V_\bullet$ is cartesian over
$X_\bullet$}, if for every morphism
$\varphi : [n] \to [m]$ of $\Delta$ the corresponding diagram
$$
\xymatrix{
V_m \ar[r]_a \ar[d]_{V_\bullet(\varphi)} & X_m \ar[d]^{X_\bullet(\varphi)}\\
V_n \ar[r]^{a} & X_n
}
$$
is a fibre square in the category of schemes.
\end{definition}

\noindent
Cartesian morphisms are related to descent data. First we prove a general
lemma describing the category of cartesian simplicial schemes over a
fixed simplicial scheme. In this lemma we denote $f^* : \Sch/X \to \Sch/Y$
the base change functor associated to a morphism of schemes $Y \to X$.

\begin{lemma}
\label{lemma-characterize-cartesian-schemes}
Let $X_\bullet$ be a simplicial scheme.
The category of simplicial schemes cartesian over $X_\bullet$
is equivalent to the category of pairs $(V, \varphi)$
where $V$ is a scheme over $X_0$ and
$$
\varphi :
V \times_{X_0, d^1_1} X_1
\longrightarrow
X_1 \times_{d^1_0, X_0} V
$$
is an isomorphism over $X_1$ such that
$(s_0^0)^*\varphi = \text{id}_V$ and such that
$$
(d^2_1)^*\varphi = (d^2_0)^*\varphi \circ (d^2_2)^*\varphi
$$
as morphisms of schemes over $X_2$.
\end{lemma}

\begin{proof}
The statement of the displayed equality makes sense because
$d^1_1 \circ d^2_2 = d^1_1 \circ d^2_1$,
$d^1_1 \circ d^2_0 = d^1_0 \circ d^2_2$, and
$d^1_0 \circ d^2_0 = d^1_0 \circ d^2_1$ as morphisms $X_2 \to X_0$, see
Simplicial, Remark \ref{simplicial-remark-relations} hence we
can picture these maps as follows
$$
\xymatrix{
&
X_2 \times_{d^1_1 \circ d^2_0, X_0} V
\ar[r]_-{(d^2_0)^*\varphi} &
X_2 \times_{d^1_0 \circ d^2_0, X_0} V
\ar@{=}[rd] & \\
X_2 \times_{d^1_0 \circ d^2_2, X_0} V
\ar@{=}[ru] & & &
X_2 \times_{d^1_0 \circ d^2_1, X_0} V \\
&
X_2 \times_{d^1_1 \circ d^2_2, X_0} V
\ar[lu]^{(d^2_2)^*\varphi} \ar@{=}[r] &
X_2 \times_{d^1_1 \circ d^2_1, X_0} V
\ar[ru]_{(d^2_1)^*\varphi}
}
$$
and the condition signifies the diagram is commutative. It is clear that
given a simplicial scheme $V_\bullet$ cartesian over $X_\bullet$ we can
set $V = V_0$ and $\varphi$ equal to the composition
$$
V \times_{X_0, d^1_1} X_1 = V_1 = X_1 \times_{X_0, d^1_0} V
$$
of identifications given by the cartesian structure. To prove this functor
is an equivalence we construct a quasi-inverse. The construction of
the quasi-inverse is analogous to the construction discussed in
Descent, Section \ref{descent-section-descent-modules} from which we borrow
the notation $\tau^n_i : [0] \to [n]$, $0 \mapsto i$ and
$\tau^n_{ij} : [1] \to [n]$, $0 \mapsto i$, $1 \mapsto j$.
Namely, given a pair $(V, \varphi)$
as in the lemma we set $V_n = X_n \times_{X(\tau^n_n), X_0} V$.
Then given $\beta : [n] \to [m]$ we define
$V(\beta) : V_m \to V_n$ as the pullback by $X(\tau^m_{\beta(n)m})$
of the map $\varphi$ postcomposed by the projection
$X_m \times_{X(\beta), X_n} V_n \to V_n$. This makes sense because
$$
X_m \times_{X(\tau^m_{\beta(n)m}), X_1} X_1 \times_{d^1_1, X_0} V
=
X_m \times_{X(\tau^m_m), X_0} V = V_m
$$
and
$$
X_m \times_{X(\tau^m_{\beta(n)m}), X_1} X_1 \times_{d^1_0, X_0} V =
X_m \times_{X(\tau^m_{\beta(n)}), X_0} V =
X_m \times_{X(\beta), X_n} V_n.
$$
We omit the verification that the commutativity
of the displayed diagram
above implies the maps compose correctly. We also omit the verification
that the two functors are quasi-inverse to each other.
\end{proof}

\begin{definition}
\label{definition-fibre-products-simplicial-scheme}
Let $f : X \to S$ be a morphism of schemes.
The {\it simplicial scheme associated to $f$}, denoted $(X/S)_\bullet$,
is the functor $\Delta^{opp} \to \Sch$,
$[n] \mapsto X \times_S \ldots \times_S X$
described in
Simplicial, Example \ref{simplicial-example-fibre-products-simplicial-object}.
\end{definition}

\noindent
Thus $(X/S)_n$ is the $(n + 1)$-fold fibre product of $X$ over $S$.
The morphism $d^1_0 : X \times_S X \to X$ is the map
$(x_0, x_1) \mapsto x_1$ and the morphism $d^1_1$ is the other
projection. The morphism $s^0_0$ is the diagonal morphism
$X \to X \times_S X$.

\begin{lemma}
\label{lemma-cartesian-over}
Let $f : X \to S$ be a morphism of schemes.
Let $\pi : V_\bullet \to (X/S)_\bullet$ be a cartesian morphism.
Set $V = V_0$ considered as a scheme over $X$.
The morphisms $d^1_0, d^1_1 : V_1 \to V_0$ and the morphism
$\pi_1 : V_1 \to X \times_S X$ induce isomorphisms
$$
\xymatrix{
V \times_S X & &
V_1 \ar[ll]_-{(d^1_1, \text{pr}_1 \circ \pi_1)}
\ar[rr]^-{(\text{pr}_0 \circ \pi_1, d^1_0)} & &
X \times_S V.
}
$$
Denote $\varphi : V \times_S X \to X \times_S V$ the
resulting isomorphism.
Then the pair $(V, \varphi)$ is a descent datum relative
to $X \to S$.
\end{lemma}

\begin{proof}
This is a special case of (part of)
Lemma \ref{lemma-characterize-cartesian-schemes}
as the displayed equation of that lemma is
equivalent to the cocycle condition of
Descent, Definition \ref{descent-definition-descent-datum}.
\end{proof}

\begin{lemma}
\label{lemma-cartesian-equivalent-descent-datum}
Let $f : X \to S$ be a morphism of schemes. The construction
$$
\begin{matrix}
\text{category of cartesian } \\
\text{schemes over } (X/S)_\bullet
\end{matrix}
\longrightarrow
\begin{matrix}
\text{ category of descent data} \\
\text{ relative to } X/S
\end{matrix}
$$
of Lemma \ref{lemma-cartesian-over}
is an equivalence of categories.
\end{lemma}

\begin{proof}
The functor from left to right is given in
Lemma \ref{lemma-cartesian-over}.
Hence this is a special case of
Lemma \ref{lemma-characterize-cartesian-schemes}.
\end{proof}

\noindent
We may reinterpret the pullback of
Groupoids, Lemma \ref{groupoids-lemma-pullback} as follows.
Suppose given a commutative diagram of morphisms of schemes
$$
\xymatrix{
X' \ar[r]_f \ar[d] & X \ar[d] \\
S' \ar[r] & S.
}
$$
This gives rise to a morphism of simplicial schemes
$$
f_\bullet : (X'/S')_\bullet \longrightarrow (X/S)_\bullet.
$$
It is a pleasant exercise to check that given any morphism
of simplicial schemes $f_\bullet : Y_\bullet \to X_\bullet$ and a
cartesian simplicial scheme $V_\bullet \to X_\bullet$
the fibre product
$$
f_\bullet^*V_\bullet = Y_\bullet \times_{X_\bullet} V_\bullet
$$
is a cartesian simplicial scheme over $Y_\bullet$. We omit
the verification that this applied to the morphism
$(X'/S')_\bullet \to (X/S)_\bullet$ corresponds via
Lemma \ref{lemma-cartesian-equivalent-descent-datum}
with the pullback defined in terms of descent data.








\section{Quasi-coherent modules on simplicial schemes}
\label{section-modules-simplicial}

\noindent
In the following definition we make use of the notion of an
$f$-map between sheaves of modules which was introduced in
Sheaves, Section \ref{sheaves-section-ringed-spaces-functoriality-modules}.

\begin{definition}
\label{definition-cartesian-sheaf}
Let $S$ be a scheme. Let $U_\bullet$ be a simplicial scheme over $S$.
\begin{enumerate}
\item A {\it quasi-coherent sheaf} on $U_\bullet$ is given by
quasi-coherent modules $\mathcal{F}_n$ on $U_n$ and for every
$\varphi : [n] \to [m]$ an $U_\bullet(\varphi)$-map
$\mathcal{F}(\varphi) : \mathcal{F}_n \to \mathcal{F}_m$ such that
$\mathcal{F}(\varphi) \circ \mathcal{F}(\psi) =
\mathcal{F}(\varphi \circ \psi)$ for any pair of composable
morphisms of $\Delta$.
\item A quasi-coherent sheaf $\mathcal{F}_\bullet$ on $U_\bullet$
is {\it cartesian} if and only if all the maps
$\mathcal{F}(\varphi) : \mathcal{F}_n \to \mathcal{F}_m$
induce isomorphisms $U_\bullet(\varphi)^*\mathcal{F}_n \to \mathcal{F}_m$.
\end{enumerate}
\end{definition}

\noindent
The property on pullbacks needs only be checked for the degeneracies.

\begin{lemma}
\label{lemma-check-cartesian-module}
Let $S$ be a scheme. Let $U_\bullet$ be a simplicial scheme over $S$.
Let $\mathcal{F}_\bullet$ be a quasi-coherent module on $U_\bullet$.
Then $\mathcal{F}_\bullet$ is cartesian if and only if the induced
maps $(d^n_j)^*\mathcal{F}_{n - 1} \to \mathcal{F}_n$ are
isomorphisms.
\end{lemma}

\begin{proof}
The category $\Delta$ is generated by the morphisms
the morphisms $\delta^n_j$ and $\sigma^n_j$, see
Simplicial, Lemma \ref{simplicial-lemma-face-degeneracy}.
Hence we only need to check the maps
$(d^n_j)^*\mathcal{F}_{n - 1} \to \mathcal{F}_n$
and $(s^n_j)^*\mathcal{F}_{n + 1} \to \mathcal{F}_n$ are
isomorphisms, see
Simplicial, Lemma \ref{simplicial-lemma-characterize-simplicial-object}
for notation. But $d_j^{n + 1} \circ s^n_j = \text{id}_{U_n}$
so it the result for $d^{n + 1}_j$ implies the result
for $s^n_j$.
\end{proof}

\begin{lemma}
\label{lemma-characterize-cartesian-modules}
Let $S$ be a scheme. Let $U_\bullet$ be a simplicial scheme over $S$.
The category of cartesian quasi-coherent modules over $U_\bullet$
is equivalent to the category of pairs $(\mathcal{F}, \alpha)$
where $\mathcal{F}$ is a quasi-coherent module over $U_0$
and
$$
\alpha : (d_1^1)^*\mathcal{F} \longrightarrow (d_0^1)^*\mathcal{F}
$$
is an isomorphism such that $(s_0^0)^*\alpha = \text{id}_\mathcal{F}$
and such that
$$
(d^2_1)^*\alpha = (d^2_0)^*\alpha \circ (d^2_2)^*\alpha
$$
on $X_2$.
\end{lemma}

\begin{proof}
The statement of the displayed equality makes sense because
$d^1_1 \circ d^2_2 = d^1_1 \circ d^2_1$,
$d^1_1 \circ d^2_0 = d^1_0 \circ d^2_2$, and
$d^1_0 \circ d^2_0 = d^1_0 \circ d^2_1$ as morphisms $X_2 \to X_0$, see
Simplicial, Remark \ref{simplicial-remark-relations} hence we
can picture these maps as follows
$$
\xymatrix{
& (d^2_0)^*(d^1_1)^*\mathcal{F} \ar[r]_-{(d^2_0)^*\alpha} &
(d^2_0)^*(d^1_0)^*\mathcal{F} \ar@{=}[rd] & \\
(d^2_2)^*(d^1_0)^*\mathcal{F} \ar@{=}[ru] & & &
(d^2_1)^*(d^1_0)^*\mathcal{F} \\
& (d^2_2)^*(d^1_1)^*\mathcal{F} \ar[lu]^{(d^2_2)^*\alpha} \ar@{=}[r] &
(d^2_1)^*(d^1_1)^*\mathcal{F} \ar[ru]_{(d^2_1)^*\alpha}
}
$$
and the condition signifies the diagram is commutative. It is clear that
given a cartesian quasi-coherent sheaf $\mathcal{F}_\bullet$ we can
set $\mathcal{F} = \mathcal{F}_0$ and $\alpha$ equal to the composition
$$
(d_1^0)^*\mathcal{F}_0 = \mathcal{F}_1 = (d_0^0)^*\mathcal{F}_0
$$
of identifications given by the cartesian structure. To prove this functor
is an equivalence we construct a quasi-inverse. The construction of
the quasi-inverse is analogous to the construction discussed in
Descent, Section \ref{descent-section-descent-modules} from which we borrow
the notation $\tau^n_i : [0] \to [n]$, $0 \mapsto i$ and
$\tau^n_{ij} : [1] \to [n]$, $0 \mapsto i$, $1 \mapsto j$.
Namely, given a pair $(\mathcal{F}, \alpha)$
as in the lemma we set $\mathcal{F}_n = X(\tau^n_n)^*\mathcal{F}$.
Then given $\beta : [n] \to [m]$ we define
$\mathcal{F}(\beta) : \mathcal{F}_n \to \mathcal{F}_m$ as the
pullback by $X(\tau^m_{\beta(n)m})$ of the map $\alpha$ precomposed
with the canonical $X(\beta)$-map $\mathcal{F}_n \to X(\beta)^*\mathcal{F}_n$.
We omit the verification that the commutativity of the displayed diagram
above implies the maps compose correctly. We also omit the verification
that the two functors are quasi-inverse to each other.
\end{proof}

\begin{lemma}
\label{lemma-pullback-cartesian-module}
Let $f_\bullet : V_\bullet \to U_\bullet$ be a morphism of
simplicial schemes. Given a cartesian quasi-coherent
module $\mathcal{F}_\bullet$ on $U_\bullet$ the pullback
$f_\bullet^* \mathcal{F}_\bullet$ is a cartesian quasi-coherent module
on $V_\bullet$.
\end{lemma}

\begin{proof}
This is immediate from the definitions.
\end{proof}

\begin{lemma}
\label{lemma-pushforward-cartesian-module}
Let $f_\bullet : V_\bullet \to U_\bullet$ be a cartesian morphism of
simplicial schemes. Assume the morphisms $d^n_j : U_n \to U_{n - 1}$ are
flat and the morphisms $V_n \to U_n$ are quasi-compact and quasi-separated.
For a cartesian quasi-coherent module $\mathcal{G}_\bullet$ on $V_\bullet$
the pushforward $f_{\bullet, *} \mathcal{G}_\bullet$ is a cartesian
quasi-coherent module on $U_\bullet$.
\end{lemma}

\begin{proof}
If $\mathcal{F}_\bullet = f_{\bullet, *} \mathcal{G}_\bullet$, then
$\mathcal{F}_n = f_{n , *}\mathcal{G}_n$ and the maps $\mathcal{F}(\varphi)$
are defined using the base change maps, see
Cohomology, Section \ref{cohomology-section-base-change-map}.
The sheaves $\mathcal{F}_n$ are quasi-coherent by
Schemes, Lemma \ref{schemes-lemma-push-forward-quasi-coherent}.
The base change maps along the degeneracies $d^n_j$ are isomorphisms
by Cohomology of Schemes, Lemma
\ref{coherent-lemma-flat-base-change-cohomology}.
Hence we are done by Lemma \ref{lemma-check-cartesian-module}.
\end{proof}

\begin{lemma}
\label{lemma-adjoint-functors-cartesian-modules}
Let $f_\bullet : V_\bullet \to U_\bullet$ be a cartesian morphism of
simplicial schemes. Assume the morphisms $d^n_j : U_n \to U_{n - 1}$ are
flat and the morphisms $V_n \to U_n$ are quasi-compact and quasi-separated.
Then $f_\bullet^*$ and $f_{\bullet, *}$ form an adjoint pair of functors
between the categories of cartesian quasi-coherent modules on
$U_\bullet$ and $V_\bullet$.
\end{lemma}

\begin{proof}
We have seen in Lemmas \ref{lemma-pullback-cartesian-module} and
\ref{lemma-pushforward-cartesian-module}
that the statement makes sense. The adjointness property follows
immediately from the fact that each $f_n^*$ is adjoint to $f_{n, *}$.
\end{proof}

\begin{lemma}
\label{lemma-cartesian-modules-with-section}
Let $f : X \to S$ be a morphism of schemes which has a
section\footnote{In fact, it would be enough to assume that $f$
has fpqc locally on $S$ a section, since we have descent of
quasi-coherent modules by Descent,
Section \ref{descent-section-fpqc-descent-quasi-coherent}.}.
Let $(X/S)_\bullet$ be the simplicial
scheme associated to $X \to S$, see
Definition \ref{definition-fibre-products-simplicial-scheme}.
Then pullback defines an equivalence between the category of
quasi-coherent $\mathcal{O}_S$-modules and the category of
cartesian quasi-coherent modules on $(X/S)_\bullet$.
\end{lemma}

\begin{proof}
Let $\sigma : S \to X$ be a section of $f$. Let $(\mathcal{F}, \alpha)$
be a pair as in Lemma \ref{lemma-characterize-cartesian-modules}.
Set $\mathcal{G} = \sigma^*\mathcal{F}$. Consider the diagram
$$
\xymatrix{
X \ar[r]_-{(\sigma \circ f, 1)} \ar[d]_f &
X \times_S X \ar[d]^{\text{pr}_0} \ar[r]_-{\text{pr}_1} & X \\
S \ar[r]^\sigma & X
}
$$
Note that $\text{pr}_0 = d^1_1$ and $\text{pr}_1 = d^1_0$. Hence we
see that $(\sigma \circ f, 1)^*\alpha$ defines an isomorphism
$$
f^*\mathcal{G} = (\sigma \circ f, 1)^*\text{pr}_0^*\mathcal{F}
\longrightarrow
(\sigma \circ f, 1)^*\text{pr}_1^*\mathcal{F} = \mathcal{F}
$$
We omit the verification that this isomorphism is compatible
with $\alpha$ and the canonical isomorphism
$\text{pr}_0^*f^*\mathcal{G} \to \text{pr}_1^*f^*\mathcal{G}$.
\end{proof}

\section{Groupoids and simplicial schemes}
\label{section-groupoids-simplicial}

\noindent
Given a groupoid in schemes we can build a simplicial scheme.
It will turn out that the category of quasi-coherent sheaves on a
groupoid is equivalent to the category of cartesian quasi-coherent
sheaves on the associated simplicial scheme.

\begin{lemma}
\label{lemma-groupoid-simplicial}
Let $(U, R, s, t, c, e, i)$ be a groupoid scheme over $S$.
There exists a simplicial scheme $U_\bullet$ over $S$
with the following properties
\begin{enumerate}
\item $U_0 = U$, $U_1 = R$, $U_2 = R \times_{s, U, t} R$,
\item $s_0^0 = e : U_0 \to U_1$,
\item $d^1_0 = s : U_1 \to U_0$, $d^1_1 = t : U_1 \to U_0$,
\item $s_0^1 = (e \circ t, 1) : U_1 \to U_2$,
$s_1^1 = (1, e \circ t) : U_1 \to U_2$,
\item $d^2_0 = \text{pr}_1 : U_2 \to U_1$,
$d^2_1 = c : U_2 \to U_1$,
$d^2_2 = \text{pr}_0$, and
\item $U_\bullet = \text{cosk}_2 \text{sk}_2 U_\bullet$.
\end{enumerate}
For all $n$ we have $U_n = R \times_{s, U, t} \ldots \times_{s, U, t} R$
with $n$ factors. The map $d^n_j : U_n \to U_{n - 1}$ is given on
functors of points by
$$
(r_1, \ldots, r_n) \longmapsto (r_1, \ldots, c(r_j, r_{j + 1}), \ldots, r_n)
$$
for $1 \leq j \leq n - 1$ whereas
$d^n_0(r_1, \ldots, r_n) = (r_2, \ldots, r_n)$ and
$d^n_n(r_1, \ldots, r_n) = (r_1, \ldots, r_{n - 1})$.
\end{lemma}

\begin{proof}
We only have to verify that the rules prescribed in (1), (2), (3), (4), (5)
define a $2$-truncated simplicial scheme $U'$ over $S$, since then (6)
allows us to set $U_\bullet = \text{cosk}_2 U'$, see
Simplicial, Lemma \ref{simplicial-lemma-existence-cosk}.
Using the functor of points approach, all we have to verify is that
if $(\text{Ob}, \text{Arrows}, s, t, c, e, i)$ is a groupoid, then
$$
\xymatrix{
\text{Arrows} \times_{s, \text{Ob}, t} \text{Arrows}
\ar@<8ex>[d]^{\text{pr}_0}
\ar@<0ex>[d]_c
\ar@<-8ex>[d]_{\text{pr}_1}
\\
\text{Arrows}
\ar@<4ex>[d]^t
\ar@<-4ex>[d]_s
\ar@<4ex>[u]^{1, e}
\ar@<-4ex>[u]_{e, 1}
\\
\text{Ob}
\ar@<0ex>[u]_e
}
$$
is a $2$-truncated simplicial set. We omit the details.

\medskip\noindent
Finally, the description of $U_n$ for $n > 2$ follows by induction from
the description of $U_0$, $U_1$, $U_2$, and
Simplicial, Remark \ref{simplicial-remark-inductive-coskelet} and
Lemma \ref{simplicial-lemma-work-out}. Alternately, one shows that
$\text{cosk}_2$ applied to the $2$-truncated simplicial set displayed above
gives a simplicial set whose $n$th term equals
$\text{Arrows} \times_{s, \text{Ob}, t} \ldots \times_{s, \text{Ob}, t}
\text{Arrows}$ with $n$ factors and degeneracy maps as given in the lemma.
Some details omitted.
\end{proof}

\begin{lemma}
\label{lemma-quasi-coherent-groupoid-simplicial}
Let $S$ be a scheme. Let $(U, R, s, t, c)$ be a groupoid scheme
over $S$. Let $U_\bullet$ be the simplicial scheme over $S$ constructed
in Lemma \ref{lemma-groupoid-simplicial}.
Then the category of quasi-coherent modules on $(U, R, s, t, c)$
is equivalent to the category of cartesian quasi-coherent modules
on $U_\bullet$.
\end{lemma}

\begin{proof}
This is clear from Lemma \ref{lemma-characterize-cartesian-modules}
and the definitions.
\end{proof}

\noindent
In the following lemma we will use the concept of a cartesian
morphism $V_\bullet \to U_\bullet$ of simplicial schemes as defined in
Definition \ref{definition-cartesian-morphism}.

\begin{lemma}
\label{lemma-quasi-coherent-groupoid-R-cartesian}
Let $(U, R, s, t, c)$ be a groupoid scheme over a scheme $S$.
Let $U_\bullet$ be the simplicial scheme over $S$ constructed
in Lemma \ref{lemma-groupoid-simplicial}.
Let $R_\bullet = (R/U)_\bullet$ be the simplicial
scheme associated to $s : R \to U$, see
Definition \ref{definition-fibre-products-simplicial-scheme}.
There exists a cartesian morphism $t_\bullet : R_\bullet \to U_\bullet$
of simplicial schemes with low degree morphisms given by
$$
\xymatrix{
R \times_{s, U, s} R \times_{s, U, s} R
\ar@<3ex>[r]_-{\text{pr}_{12}}
\ar@<0ex>[r]_-{\text{pr}_{02}}
\ar@<-3ex>[r]_-{\text{pr}_{01}}
\ar[dd]_{(r_0, r_1, r_2) \mapsto (r_0 \circ r_1^{-1}, r_1 \circ r_2^{-1})} &
R \times_{s, U, s} R
\ar@<1ex>[r]_-{\text{pr}_1} \ar@<-2ex>[r]_-{\text{pr}_0}
\ar[dd]_{(r_0, r_1) \mapsto r_0 \circ r_1^{-1}} &
R \ar[dd]^t
\\
\\
R \times_{s, U, t} R
\ar@<3ex>[r]_{\text{pr}_1}
\ar@<0ex>[r]_c
\ar@<-3ex>[r]_{\text{pr}_0} &
R \ar@<1ex>[r]_s \ar@<-2ex>[r]_t &
U
}
$$
\end{lemma}

\begin{proof}
For arbitrary $n$ we define $R_n \to U_n$ by the rule
$$
(r_0, \ldots, r_n)
\longrightarrow
(r_0 \circ r_1^{-1}, \ldots, r_{n - 1} \circ r_n^{-1})
$$
Compatibility with degeneracy maps is clear from the description of the
degeneracies in Lemma \ref{lemma-groupoid-simplicial}.
We omit the verification that the maps respect the morphisms $s^n_j$.
Groupoids, Lemma \ref{groupoids-lemma-diagram-pull}
(with the roles of $s$ and $t$ reversed)
shows that the two right squares are cartesian. In exactly the same manner
one shows all the other squares are cartesian too. Hence
the morphism is cartesian.
\end{proof}




\section{Descent data give equivalence relations}
\label{section-equivalence-relation}

\noindent
In Section \ref{section-simplicial} we saw how descent
data relative to $X \to S$ can be formulated in terms of cartesian simplicial
schemes over $(X/S)_\bullet$. Here we link this to equivalence
relations as follows.

\begin{lemma}
\label{lemma-equivalence-relation}
Let $f : X \to S$ be a morphism of schemes.
Let $\pi : V_\bullet \to (X/S)_\bullet$ be a cartesian morphism,
see Definition \ref{definition-cartesian-morphism}.
Then the morphism
$$
j = (d^1_1, d^1_0) : V_1 \to V_0 \times_S V_0
$$
defines an equivalence relation on $V_0$ over $S$,
see Groupoids, Definition \ref{groupoids-definition-equivalence-relation}.
\end{lemma}

\begin{proof}
Note that $j$ is a monomorphism. Namely the
composition $V_1 \to V_0 \times_S V_0 \to V_0 \times_S X$
is an isomorphism as $\pi$ is cartesian.

\medskip\noindent
Consider the morphism
$$
(d^2_2, d^2_0) : V_2 \to V_1 \times_{d^1_0, V_0, d^1_1} V_1.
$$
This works because $d_0 \circ d_2 = d_1 \circ d_0$,
see Simplicial, Remark \ref{simplicial-remark-relations}.
Also, it is a morphism over $(X/S)_2$. It is an isomorphism
because $V_\bullet \to (X/S)_\bullet$ is cartesian.
Note for example that the
right hand side is isomorphic to
$V_0 \times_{\pi_0, X, \text{pr}_1} (X \times_S X \times_S X) =
X \times_S V_0 \times_S X$
because $\pi$ is cartesian. Details omitted.

\medskip\noindent
As in Groupoids, Definition \ref{groupoids-definition-equivalence-relation}
we denote $t = \text{pr}_0 \circ j = d^1_1$ and
$s = \text{pr}_1 \circ j = d^1_0$.
The isomorphism above, combined with the morphism
$d^2_1 : V_2 \to V_1$ give us a composition morphism
$$
c : V_1 \times_{s, V_0, t} V_1 \longrightarrow V_1
$$
over $V_0 \times_S V_0$. This immediately implies
that for any scheme $T/S$ the relation
$V_1(T) \subset V_0(T) \times V_0(T)$ is transitive.

\medskip\noindent
Reflexivity follows from the fact that the
restriction of the morphism $j$ to the diagonal
$\Delta : X \to X \times_S X$ is an isomorphism
(again use the cartesian property of $\pi$).

\medskip\noindent
To see symmetry we consider the morphism
$$
(d^2_2, d^2_1) : V_2 \to V_1 \times_{d^1_1, V_0, d^1_1} V_1.
$$
This works because $d_1 \circ d_2 = d_1 \circ d_1$,
see Simplicial, Remark \ref{simplicial-remark-relations}.
It is an isomorphism
because $V_\bullet \to (X/S)_\bullet$ is cartesian.
Note for example that the
right hand side is isomorphic to
$V_0 \times_{\pi_0, X, \text{pr}_0} (X \times_S X \times_S X) =
V_0 \times_S X \times_S X$
because $\pi$ is cartesian. Details omitted.

\medskip\noindent
Let $T/S$ be a scheme. Let $a \sim b$ for $a, b \in V_0(T)$
be synonymous with $(a, b) \in V_1(T)$.
The isomorphism $(d^2_2, d^2_1)$ above
implies that if $a \sim b$ and $a \sim c$, then $b \sim c$.
Combined with reflexivity this shows that $\sim$ is
an equivalence relation.
\end{proof}







\section{An example case}
\label{section-example}

\noindent
In this section we show that disjoint unions of spectra
of Artinian rings can be descended along a quasi-compact
surjective flat morphism of schemes.

\begin{lemma}
\label{lemma-equivalence-classes-points}
Let $X \to S$ be a morphism of schemes.
Suppose $V_\bullet \to (X/S)_\bullet$ is cartesian.
For $v \in V_0$ a point define
$$
T_v = \{v' \in V_0 \mid \exists\ v_1 \in V_1:
d^1_1(v_1) = v, d^1_0(v_1) = v'\}
$$
as a subset of $V_0$. Then $v \in T_v$ and
$T_v \cap T_{v'} \not = \emptyset \Rightarrow T_v = T_{v'}$.
\end{lemma}

\begin{proof}
Combine Lemma \ref{lemma-equivalence-relation} and
Groupoids, Lemma
\ref{groupoids-lemma-pre-equivalence-equivalence-relation-points}.
\end{proof}

\begin{lemma}
\label{lemma-quasi-compact}
Let $X \to S$ be a morphism of schemes.
Suppose $V_\bullet \to (X/S)_\bullet$ is cartesian.
Let $v \in V_0$ be a point. If $X \to S$ is quasi-compact, then
$$
T_v = \{v' \in V_0 \mid \exists\ v_1 \in V_1:
d^1_1(v_1) = v, d^1_0(v_1) = v'\}
$$
is a quasi-compact subset of $V_0$.
\end{lemma}

\begin{proof}
Let $F_v$ be the scheme theoretic fibre of $d^1_1 : V_1 \to V_0$
at $v$. Then we see that $T_v$ is the image of the morphism
$$
\xymatrix{
F_v \ar[r] \ar[d] &
V_1 \ar[r]^{d^1_0} \ar[d]^{d^1_1} &
V_0 \\
v \ar[r] &
V_0 &
}
$$
Note that $F_v$ is quasi-compact. This proves the lemma.
\end{proof}

\begin{lemma}
\label{lemma-descent-disjoint-union-Artinian-along-fields}
Let $X \to S$ be a quasi-compact flat surjective morphism.
Let $(V, \varphi)$ be a descent datum relative
to $X \to S$. If $V$ is a disjoint union of
spectra of Artinian rings, then $(V, \varphi)$
is effective.
\end{lemma}

\begin{proof}
We may write $V = \coprod_{i \in I} \Spec(A_i)$
with each $A_i$ local Artinian. Moreover, let
$v_i \in V$ be the unique closed point of $\Spec(A_i)$
for all $i \in I$. Write $i \sim j$ if and only if
$v_i \in T_{v_j}$ with notation as in
Lemma \ref{lemma-equivalence-classes-points} above.
By Lemmas \ref{lemma-equivalence-classes-points} and \ref{lemma-quasi-compact}
this is an equivalence relation with finite equivalence
classes. Let $\overline{I} = I/\sim$. Then we can write
$V = \coprod_{\overline{i} \in \overline{I}} V_{\overline{i}}$
with
$V_{\overline{i}} = \coprod_{i \in \overline{i}} \Spec(A_i)$.
By construction we see that
$\varphi : V \times_S X \to X \times_S V$ maps
the open and closed subspaces $V_{\overline{i}} \times_S X$
into the open and closed subspaces $X \times_S V_{\overline{i}}$.
In other words, we get descent data
$(V_{\overline{i}}, \varphi_{\overline{i}})$, and
$(V, \varphi)$ is the coproduct of them in the category of
descent data.
Since each of the $V_{\overline{i}}$ is a finite union of
spectra of Artinian local rings the morphism $V_{\overline{i}} \to X$
is affine, see Morphisms, Lemma \ref{morphisms-lemma-Artinian-affine}.
Since $\{X \to S\}$ is an fpqc covering we see that all
the descent data $(V_{\overline{i}}, \varphi_{\overline{i}})$ are effective
by Descent, Lemma \ref{descent-lemma-affine}.
Hence we win.
\end{proof}

\noindent
To be sure, the lemma above has very limited applicability!





\section{Other chapters}

\begin{multicols}{2}
\begin{enumerate}
\item \hyperref[introduction-section-phantom]{Introduction}
\item \hyperref[conventions-section-phantom]{Conventions}
\item \hyperref[sets-section-phantom]{Set Theory}
\item \hyperref[categories-section-phantom]{Categories}
\item \hyperref[topology-section-phantom]{Topology}
\item \hyperref[sheaves-section-phantom]{Sheaves on Spaces}
\item \hyperref[algebra-section-phantom]{Commutative Algebra}
\item \hyperref[sites-section-phantom]{Sites and Sheaves}
\item \hyperref[homology-section-phantom]{Homological Algebra}
\item \hyperref[derived-section-phantom]{Derived Categories}
\item \hyperref[more-algebra-section-phantom]{More Algebra}
\item \hyperref[simplicial-section-phantom]{Simplicial Methods}
\item \hyperref[modules-section-phantom]{Sheaves of Modules}
\item \hyperref[sites-modules-section-phantom]{Modules on Sites}
\item \hyperref[injectives-section-phantom]{Injectives}
\item \hyperref[cohomology-section-phantom]{Cohomology of Sheaves}
\item \hyperref[sites-cohomology-section-phantom]{Cohomology on Sites}
\item \hyperref[hypercovering-section-phantom]{Hypercoverings}
\item \hyperref[schemes-section-phantom]{Schemes}
\item \hyperref[constructions-section-phantom]{Constructions of Schemes}
\item \hyperref[properties-section-phantom]{Properties of Schemes}
\item \hyperref[morphisms-section-phantom]{Morphisms of Schemes}
\item \hyperref[coherent-section-phantom]{Coherent Cohomology}
\item \hyperref[divisors-section-phantom]{Divisors}
\item \hyperref[limits-section-phantom]{Limits of Schemes}
\item \hyperref[varieties-section-phantom]{Varieties}
\item \hyperref[chow-section-phantom]{Chow Homology}
\item \hyperref[topologies-section-phantom]{Topologies on Schemes}
\item \hyperref[descent-section-phantom]{Descent}
\item \hyperref[more-morphisms-section-phantom]{More on Morphisms}
\item \hyperref[flat-section-phantom]{More on Flatness}
\item \hyperref[groupoids-section-phantom]{Groupoid Schemes}
\item \hyperref[more-groupoids-section-phantom]{More on Groupoid Schemes}
\item \hyperref[etale-section-phantom]{\'Etale Morphisms of Schemes}
\item \hyperref[etale-cohomology-section-phantom]{\'Etale Cohomology}
\item \hyperref[spaces-section-phantom]{Algebraic Spaces}
\item \hyperref[spaces-properties-section-phantom]{Properties of Algebraic Spaces}
\item \hyperref[spaces-morphisms-section-phantom]{Morphisms of Algebraic Spaces}
\item \hyperref[spaces-topologies-section-phantom]{Topologies on Algebraic Spaces}
\item \hyperref[spaces-descent-section-phantom]{Descent and Algebraic Spaces}
\item \hyperref[spaces-more-morphisms-section-phantom]{More on Morphisms of Spaces}
\item \hyperref[quot-section-phantom]{Quot and Hilbert Spaces}
\item \hyperref[stacks-section-phantom]{Stacks}
\item \hyperref[spaces-groupoids-section-phantom]{Groupoids in Algebraic Spaces}
\item \hyperref[spaces-more-groupoids-section-phantom]{More on Groupoids in Spaces}
\item \hyperref[bootstrap-section-phantom]{Bootstrap}
\item \hyperref[examples-stacks-section-phantom]{Examples of Stacks}
\item \hyperref[groupoids-quotients-section-phantom]{Quotients of Groupoids}
\item \hyperref[algebraic-section-phantom]{Algebraic Stacks}
\item \hyperref[criteria-section-phantom]{Criteria for Representability}
\item \hyperref[stacks-properties-section-phantom]{Properties of Algebraic Stacks}
\item \hyperref[stacks-morphisms-section-phantom]{Morphisms of Algebraic Stacks}
\item \hyperref[examples-section-phantom]{Examples}
\item \hyperref[exercises-section-phantom]{Exercises}
\item \hyperref[guide-section-phantom]{Guide to Literature}
\item \hyperref[desirables-section-phantom]{Desirables}
\item \hyperref[coding-section-phantom]{Coding Style}
\item \hyperref[fdl-section-phantom]{GNU Free Documentation License}
\item \hyperref[index-section-phantom]{Auto Generated Index}
\end{enumerate}
\end{multicols}


\bibliography{my}
\bibliographystyle{amsalpha}

\end{document}
