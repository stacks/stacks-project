\IfFileExists{stacks-project.cls}{%
\documentclass{stacks-project}
}{%
\documentclass{amsart}
}

% The following AMS packages are automatically loaded with
% the amsart documentclass:
%\usepackage{amsmath}
%\usepackage{amssymb}
%\usepackage{amsthm}

% For dealing with references we use the comment environment
\usepackage{verbatim}
\newenvironment{reference}{\comment}{\endcomment}
%\newenvironment{reference}{}{}
\newenvironment{slogan}{\comment}{\endcomment}
\newenvironment{history}{\comment}{\endcomment}

% For commutative diagrams you can use
% \usepackage{amscd}
\usepackage[all]{xy}

% We use 2cell for 2-commutative diagrams.
\xyoption{2cell}
\UseAllTwocells

% To put source file link in headers.
% Change "template.tex" to "this_filename.tex"
% \usepackage{fancyhdr}
% \pagestyle{fancy}
% \lhead{}
% \chead{}
% \rhead{Source file: \url{template.tex}}
% \lfoot{}
% \cfoot{\thepage}
% \rfoot{}
% \renewcommand{\headrulewidth}{0pt}
% \renewcommand{\footrulewidth}{0pt}
% \renewcommand{\headheight}{12pt}

\usepackage{multicol}

% For cross-file-references
\usepackage{xr-hyper}

% Package for hypertext links:
\usepackage{hyperref}

% For any local file, say "hello.tex" you want to link to please
% use \externaldocument[hello-]{hello}
\externaldocument[introduction-]{introduction}
\externaldocument[conventions-]{conventions}
\externaldocument[sets-]{sets}
\externaldocument[categories-]{categories}
\externaldocument[topology-]{topology}
\externaldocument[sheaves-]{sheaves}
\externaldocument[sites-]{sites}
\externaldocument[stacks-]{stacks}
\externaldocument[fields-]{fields}
\externaldocument[algebra-]{algebra}
\externaldocument[brauer-]{brauer}
\externaldocument[homology-]{homology}
\externaldocument[derived-]{derived}
\externaldocument[simplicial-]{simplicial}
\externaldocument[more-algebra-]{more-algebra}
\externaldocument[smoothing-]{smoothing}
\externaldocument[modules-]{modules}
\externaldocument[sites-modules-]{sites-modules}
\externaldocument[injectives-]{injectives}
\externaldocument[cohomology-]{cohomology}
\externaldocument[sites-cohomology-]{sites-cohomology}
\externaldocument[dga-]{dga}
\externaldocument[dpa-]{dpa}
\externaldocument[hypercovering-]{hypercovering}
\externaldocument[schemes-]{schemes}
\externaldocument[constructions-]{constructions}
\externaldocument[properties-]{properties}
\externaldocument[morphisms-]{morphisms}
\externaldocument[coherent-]{coherent}
\externaldocument[divisors-]{divisors}
\externaldocument[limits-]{limits}
\externaldocument[varieties-]{varieties}
\externaldocument[topologies-]{topologies}
\externaldocument[descent-]{descent}
\externaldocument[perfect-]{perfect}
\externaldocument[more-morphisms-]{more-morphisms}
\externaldocument[flat-]{flat}
\externaldocument[groupoids-]{groupoids}
\externaldocument[more-groupoids-]{more-groupoids}
\externaldocument[etale-]{etale}
\externaldocument[chow-]{chow}
\externaldocument[intersection-]{intersection}
\externaldocument[pic-]{pic}
\externaldocument[adequate-]{adequate}
\externaldocument[dualizing-]{dualizing}
\externaldocument[duality-]{duality}
\externaldocument[discriminant-]{discriminant}
\externaldocument[local-cohomology-]{local-cohomology}
\externaldocument[curves-]{curves}
\externaldocument[resolve-]{resolve}
\externaldocument[models-]{models}
\externaldocument[pione-]{pione}
\externaldocument[etale-cohomology-]{etale-cohomology}
\externaldocument[proetale-]{proetale}
\externaldocument[crystalline-]{crystalline}
\externaldocument[spaces-]{spaces}
\externaldocument[spaces-properties-]{spaces-properties}
\externaldocument[spaces-morphisms-]{spaces-morphisms}
\externaldocument[decent-spaces-]{decent-spaces}
\externaldocument[spaces-cohomology-]{spaces-cohomology}
\externaldocument[spaces-limits-]{spaces-limits}
\externaldocument[spaces-divisors-]{spaces-divisors}
\externaldocument[spaces-over-fields-]{spaces-over-fields}
\externaldocument[spaces-topologies-]{spaces-topologies}
\externaldocument[spaces-descent-]{spaces-descent}
\externaldocument[spaces-perfect-]{spaces-perfect}
\externaldocument[spaces-more-morphisms-]{spaces-more-morphisms}
\externaldocument[spaces-flat-]{spaces-flat}
\externaldocument[spaces-groupoids-]{spaces-groupoids}
\externaldocument[spaces-more-groupoids-]{spaces-more-groupoids}
\externaldocument[bootstrap-]{bootstrap}
\externaldocument[spaces-pushouts-]{spaces-pushouts}
\externaldocument[groupoids-quotients-]{groupoids-quotients}
\externaldocument[spaces-more-cohomology-]{spaces-more-cohomology}
\externaldocument[spaces-simplicial-]{spaces-simplicial}
\externaldocument[spaces-duality-]{spaces-duality}
\externaldocument[formal-spaces-]{formal-spaces}
\externaldocument[restricted-]{restricted}
\externaldocument[spaces-resolve-]{spaces-resolve}
\externaldocument[formal-defos-]{formal-defos}
\externaldocument[defos-]{defos}
\externaldocument[cotangent-]{cotangent}
\externaldocument[examples-defos-]{examples-defos}
\externaldocument[algebraic-]{algebraic}
\externaldocument[examples-stacks-]{examples-stacks}
\externaldocument[stacks-sheaves-]{stacks-sheaves}
\externaldocument[criteria-]{criteria}
\externaldocument[artin-]{artin}
\externaldocument[quot-]{quot}
\externaldocument[stacks-properties-]{stacks-properties}
\externaldocument[stacks-morphisms-]{stacks-morphisms}
\externaldocument[stacks-limits-]{stacks-limits}
\externaldocument[stacks-cohomology-]{stacks-cohomology}
\externaldocument[stacks-perfect-]{stacks-perfect}
\externaldocument[stacks-introduction-]{stacks-introduction}
\externaldocument[stacks-more-morphisms-]{stacks-more-morphisms}
\externaldocument[stacks-geometry-]{stacks-geometry}
\externaldocument[moduli-]{moduli}
\externaldocument[moduli-curves-]{moduli-curves}
\externaldocument[examples-]{examples}
\externaldocument[exercises-]{exercises}
\externaldocument[guide-]{guide}
\externaldocument[desirables-]{desirables}
\externaldocument[coding-]{coding}
\externaldocument[obsolete-]{obsolete}
\externaldocument[fdl-]{fdl}
\externaldocument[index-]{index}

% Theorem environments.
%
\theoremstyle{plain}
\newtheorem{theorem}[subsection]{Theorem}
\newtheorem{proposition}[subsection]{Proposition}
\newtheorem{lemma}[subsection]{Lemma}

\theoremstyle{definition}
\newtheorem{definition}[subsection]{Definition}
\newtheorem{example}[subsection]{Example}
\newtheorem{exercise}[subsection]{Exercise}
\newtheorem{situation}[subsection]{Situation}

\theoremstyle{remark}
\newtheorem{remark}[subsection]{Remark}
\newtheorem{remarks}[subsection]{Remarks}

\numberwithin{equation}{subsection}

% Macros
%
\def\lim{\mathop{\mathrm{lim}}\nolimits}
\def\colim{\mathop{\mathrm{colim}}\nolimits}
\def\Spec{\mathop{\mathrm{Spec}}}
\def\Hom{\mathop{\mathrm{Hom}}\nolimits}
\def\Ext{\mathop{\mathrm{Ext}}\nolimits}
\def\SheafHom{\mathop{\mathcal{H}\!\mathit{om}}\nolimits}
\def\SheafExt{\mathop{\mathcal{E}\!\mathit{xt}}\nolimits}
\def\Sch{\mathit{Sch}}
\def\Mor{\operatorname{Mor}\nolimits}
\def\Ob{\mathop{\mathrm{Ob}}\nolimits}
\def\Sh{\mathop{\mathit{Sh}}\nolimits}
\def\NL{\mathop{N\!L}\nolimits}
\def\proetale{{pro\text{-}\acute{e}tale}}
\def\etale{{\acute{e}tale}}
\def\QCoh{\mathit{QCoh}}
\def\Ker{\mathop{\mathrm{Ker}}}
\def\Im{\mathop{\mathrm{Im}}}
\def\Coker{\mathop{\mathrm{Coker}}}
\def\Coim{\mathop{\mathrm{Coim}}}

%
% Macros for moduli stacks/spaces
%
\def\QCohstack{\mathcal{QC}\!\mathit{oh}}
\def\Cohstack{\mathcal{C}\!\mathit{oh}}
\def\Spacesstack{\mathcal{S}\!\mathit{paces}}
\def\Quotfunctor{\mathrm{Quot}}
\def\Hilbfunctor{\mathrm{Hilb}}
\def\Curvesstack{\mathcal{C}\!\mathit{urves}}
\def\Polarizedstack{\mathcal{P}\!\mathit{olarized}}
\def\Complexesstack{\mathcal{C}\!\mathit{omplexes}}
% \Pic is the operator that assigns to X its picard group, usage \Pic(X)
% \Picardstack_{X/B} denotes the Picard stack of X over B
% \Picardfunctor_{X/B} denotes the Picard functor of X over B
\def\Pic{\mathop{\mathrm{Pic}}\nolimits}
\def\Picardstack{\mathcal{P}\!\mathit{ic}}
\def\Picardfunctor{\mathrm{Pic}}
\def\Deformationcategory{\mathcal{D}\!\mathit{ef}}


% OK, start here.
%
\begin{document}

\title{Simplicial Spaces}


\maketitle

\phantomsection
\label{section-phantom}

\tableofcontents

\section{Introduction}
\label{section-introduction}

\noindent
This chapter develops some theory concerning simplicial topological spaces,
simplicial ringed spaces, simplicial schemes, and simplicial algebraic spaces.
The theory of simplicial spaces sometimes allows one to prove local to global
principles which appear difficult to prove in other ways.
Some example applications can be found in the papers
\cite{faltings_finiteness}, \cite{Kiehl}, and \cite{HodgeIII}.

\medskip\noindent
We assume throughout that the reader is familiar with the basic concepts
and results of the chapter Simplical Methods, see
Simplicial, Section \ref{simplicial-section-introduction}.
In particular, we continue to write $X$ and not $X_\bullet$
for a simplicial object.








\section{Simplicial topological spaces}
\label{section-simplicial-top}

\noindent
A {\it simplicial space} is a simplicial object in the category of
topological spaces where morphisms are continuous maps of topological
spaces. (We will use ``simplicial algebraic space'' to refer to simplicial
objects in the category of algebraic spaces.)
We may picture a simplicial space $X$ as follows
$$
\xymatrix{
X_2
\ar@<2ex>[r]
\ar@<0ex>[r]
\ar@<-2ex>[r]
&
X_1
\ar@<1ex>[r]
\ar@<-1ex>[r]
\ar@<1ex>[l]
\ar@<-1ex>[l]
&
X_0
\ar@<0ex>[l]
}
$$
Here there are two morphisms $d^1_0, d^1_1 : X_1 \to X_0$
and a single morphism $s^0_0 : X_0 \to X_1$, etc.
It is important to keep in mind that $d^n_i : X_n \to X_{n - 1}$
should be thought of as a ``projection forgetting the
$i$th coordinate'' and $s^n_j : X_n \to X_{n + 1}$ as the diagonal
map repeating the $j$th coordinate.

\medskip\noindent
Let $X$ be a simplicial space. We associate a site
$X_{Zar}$\footnote{This notation is similar to the notation in
Sites, Example \ref{sites-example-site-topological}
and
Topologies, Definition \ref{topologies-definition-big-small-Zariski}.}
to $X$ as follows.
\begin{enumerate}
\item An object of $X_{Zar}$ is an open $U$ of $X_n$ for some $n$,
\item a morphism $U \to V$ of $X_{Zar}$ is given by a
$\varphi : [m] \to [n]$ where $n, m$ are such that
$U \subset X_n$, $V \subset X_m$ and $\varphi$ is such that
$X(\varphi)(U) \subset V$, and
\item a covering $\{U_i \to U\}$ in $X_{Zar}$ means
that $U, U_i \subset X_n$ are open, the maps $U_i \to U$ are
given by $\text{id} : [n] \to [n]$, and $U = \bigcup U_i$.
\end{enumerate}
Note that in particular, if $U \to V$ is a morphism of $X_{Zar}$
given by $\varphi$, then $X(\varphi) : X_n \to X_m$ does in fact
induce a continuous map $U \to V$ of topological spaces.

\noindent
It is clear that the above is a special case of a construction that
associates to any diagram of topological spaces a site. We formulate
the obligatory lemma.

\begin{lemma}
\label{lemma-simplicial-site}
Let $X$ be a simplicial space. Then $X_{Zar}$
as defined above is a site.
\end{lemma}

\begin{proof}
Omitted.
\end{proof}

\noindent
Let $X$ be a simplicial space. Let $\mathcal{F}$ be a sheaf on $X_{Zar}$.
It is clear from the definition of coverings, that the restriction
of $\mathcal{F}$ to the opens of $X_n$ defines a sheaf $\mathcal{F}_n$
on the topological space $X_n$. For every $\varphi : [m] \to [n]$ the
restriction maps of $\mathcal{F}$ for pairs $U \subset X_n$, $V \subset X_m$
with $X(\varphi)(U) \subset V$, define an $X(\varphi)$-map
$\mathcal{F}(\varphi) : \mathcal{F}_m \to \mathcal{F}_n$, see
Sheaves, Definition \ref{sheaves-definition-f-map}.
Moreover, given $\varphi : [m] \to [n]$ and $\psi : [l] \to [m]$
we have
$$
\mathcal{F}(\psi) \circ \mathcal{F}(\varphi) =
\mathcal{F}(\varphi \circ \psi)
$$
(LHS uses composition of $f$-maps, see
Sheaves, Definition \ref{sheaves-definition-composition-f-maps}).
Clearly, the converse is true as well: if we have a system
$(\{\mathcal{F}_n\}_{n \geq 0},
\{\mathcal{F}(\varphi)\}_{\varphi \in \text{Arrows}(\Delta)})$
as above, satisfying the displayed equalities,
then we obtain a sheaf on $X_{Zar}$.

\begin{lemma}
\label{lemma-describe-sheaves-simplicial-site}
Let $X$ be a simplicial space. There is an equivalence of
categories between
\begin{enumerate}
\item $\Sh(X_{Zar})$, and
\item category of systems $(\mathcal{F}_n, \mathcal{F}(\varphi))$
described above.
\end{enumerate}
\end{lemma}

\begin{proof}
See discussion above.
\end{proof}

\begin{lemma}
\label{lemma-simplicial-space-site-functorial}
Let $f : Y \to X$ be a morphism of simplicial spaces.
Then the functor $u : X_{Zar} \to Y_{Zar}$
which associates to the open $U \subset X_n$ the open
$f_n^{-1}(U) \subset Y_n$ defines a morphism of sites
$f_{Zar} : Y_{Zar} \to X_{Zar}$.
\end{lemma}

\begin{proof}
It is clear that $u$ is a continuous functor. Hence we obtain functors
$f_{Zar, *} = u^s$ and $f_{Zar}^{-1} = u_s$, see
Sites, Section \ref{sites-section-morphism-sites}.
To see that we obtain a morphism of sites we have to show
that $u^s$ is exact. We will use
Sites, Lemma \ref{sites-lemma-directed-morphism} to see this.
Let $V \subset Y_n$ be an open subset. The category
$\mathcal{I}_V^u$ (see Sites, Section \ref{sites-section-functoriality-PSh})
consists of pairs $(U, \varphi)$ where
$\varphi : [m] \to [n]$ and $U \subset X_m$ open such that
$Y(\varphi)(V) \subset f_m^{-1}(U)$. Moreover, a morphism
$(U, \varphi) \to (U', \varphi')$ is given by a
$\psi : [m'] \to [m]$ such that $X(\psi)(U) \subset U'$
and $\varphi \circ \psi = \varphi'$.
It is our task to show that $\mathcal{I}_V^u$ is cofiltered.

\medskip\noindent
We verify the conditions of
Categories, Definition \ref{categories-definition-codirected}.
Condition (1) holds because $(X_n, \text{id}_{[n]})$ is an object.
Let $(U, \varphi)$ be an object. The condition
$Y(\varphi)(V) \subset f_m^{-1}(U)$ is equivalent to
$V \subset f_n^{-1}(X(\varphi)^{-1}(U))$. Hence we obtain a morphism
$(X(\varphi)^{-1}(U), \text{id}_{[n]}) \to (U, \varphi)$ given
by setting $\psi = \varphi$. Moreover, given a pair of objects
of the form $(U, \text{id}_{[n]})$ and $(U', \text{id}_{[n]})$
we see there exists an object, namely $(U \cap U', \text{id}_{[n]})$,
which maps to both of them. Thus condition (2) holds.
To verify condition (3) suppose given two morphisms
$a, a': (U, \varphi) \to (U', \varphi')$ given by $\psi, \psi' : [m'] \to [m]$.
Then precomposing with the morphism
$(X(\varphi)^{-1}(U), \text{id}_{[n]}) \to (U, \varphi)$ given
by $\varphi$ equalizes $a, a'$ because
$\varphi \circ \psi = \varphi' = \varphi \circ \psi'$.
This finishes the proof.
\end{proof}

\begin{lemma}
\label{lemma-describe-functoriality}
Let $f : Y \to X$ be a morphism of simplicial spaces. In terms of the
description of sheaves in
Lemma \ref{lemma-describe-sheaves-simplicial-site} the
morphism $f_{Zar}$ of Lemma \ref{lemma-simplicial-space-site-functorial}
can be described as follows.
\begin{enumerate}
\item If $\mathcal{G}$ is a sheaf on $Y$, then
$(f_{Zar, *}\mathcal{G})_n = f_{n, *}\mathcal{G}_n$.
\item If $\mathcal{F}$ is a sheaf on $X$, then
$(f_{Zar}^{-1}\mathcal{F})_n = f_n^{-1}\mathcal{F}_n$.
\end{enumerate}
\end{lemma}

\begin{proof}
The first part is immediate from the definitions. For the second part, note
that in the proof of
Lemma \ref{lemma-simplicial-space-site-functorial}
we have shown that for a $V \subset Y_n$ open the category
$(\mathcal{I}_V^u)^{opp}$ contains as a cofinal subcategory
the category of opens $U \subset X_n$ with $f_n^{-1}(U) \supset V$
and morphisms given by inclusions. Hence we see that the restriction
of $u_p\mathcal{F}$ to opens of $Y_n$ is the presheaf
$f_{n, p}\mathcal{F}_n$ as defined in
Sheaves, Lemma \ref{sheaves-lemma-pullback-presheaves}.
Since $f_{Zar}^{-1}\mathcal{F} = u_s\mathcal{F}$ is the sheafification
of $u_p\mathcal{F}$ and since sheafification uses only coverings and
since coverings in $Y_{Zar}$ use only inclusions between opens on the
same $Y_n$, the result follows from the fact that $f_n^{-1}\mathcal{F}_n$
is (correspondingly) the sheafification of $f_{n, p}\mathcal{F}_n$, see
Sheaves, Section \ref{sheaves-section-presheaves-functorial}.
\end{proof}

\noindent
Let $X$ be a topological space. In
Sites, Example \ref{sites-example-site-topological}
we denoted $X_{Zar}$ the site consisting of opens of $X$
with inclusions as morphisms and coverings given by open coverings.
We identify the topos $\Sh(X_{Zar})$ with the category
of sheaves on $X$.

\begin{lemma}
\label{lemma-restriction-to-components}
Let $X$ be a simplicial space. The functor
$X_{n, Zar} \to X_{Zar}$, $U \mapsto U$ is continuous
and cocontinuous. The associated morphism of
topoi $g : \Sh(X_n) \to \Sh(X_{Zar})$ satisfies
\begin{enumerate}
\item $g^{-1}$ associates to the sheaf $\mathcal{F}$ on $X$
the sheaf $\mathcal{F}_n$ on $X_n$,
\item $g^{-1}$ has a left adjoint $g^{Sh}_!$ which commutes
with finite connected limits,
\item $g^{-1} : \textit{Ab}(X_{Zar}) \to \textit{Ab}(X_n)$
has a left adjoint $g_! : \textit{Ab}(X_n) \to \textit{Ab}(X_{Zar})$
which is exact.
\end{enumerate}
\end{lemma}

\begin{proof}
Besides the properties of our functor mentioned in the statement,
the category $X_{n, Zar}$ has fibre products and equalizers
and the functor commutes with them (beware that $X_{Zar}$ does not
have all fibre products). Hence the lemma follows from the discussion in
Sites, Sections \ref{sites-section-cocontinuous-functors} and
\ref{sites-section-cocontinuous-morphism-topoi}
and
Modules on Sites, Section \ref{sites-modules-section-exactness-lower-shriek}.
More precisely,
Sites, Lemmas \ref{sites-lemma-cocontinuous-morphism-topoi},
\ref{sites-lemma-when-shriek}, and
\ref{sites-lemma-preserve-equalizers}
and
Modules on Sites, Lemmas
\ref{sites-modules-lemma-g-shriek-adjoint} and
\ref{sites-modules-lemma-exactness-lower-shriek}.
\end{proof}

\begin{lemma}
\label{lemma-restriction-injective-to-component}
Let $X$ be a simplicial space. If $\mathcal{I}$ is an injective abelian
sheaf on $X_{Zar}$, then $\mathcal{I}_n$ is an injective abelian sheaf
on $X_n$.
\end{lemma}

\begin{proof}
This follows from
Homology, Lemma \ref{homology-lemma-adjoint-preserve-injectives}
and
Lemma \ref{lemma-restriction-to-components}.
\end{proof}

\begin{lemma}
\label{lemma-restriction-to-components-functorial}
Let $f : Y \to X$ be a morphism of simplicial spaces. Then
$$
\xymatrix{
\Sh(Y_n) \ar[d] \ar[r]_{f_n} & \Sh(X_n) \ar[d] \\
\Sh(Y_{Zar}) \ar[r]^{f_{Zar}} & \Sh(X_{Zar})
}
$$
is a commutative diagram of topoi.
\end{lemma}

\begin{proof}
Direct from the description of pullback functors in
Lemmas \ref{lemma-describe-functoriality} and
\ref{lemma-restriction-to-components}.
\end{proof}

\noindent
Let $X$ be a topological space. Denote $X_\bullet$ the constant simplicial
topological space with value $X$. By
Lemma \ref{lemma-describe-sheaves-simplicial-site}
a sheaf on $X_{\bullet, Zar}$ is the same
thing as a cosimplicial object in the category of sheaves on $X$.

\begin{lemma}
\label{lemma-constant-simplicial-space}
Let $X$ be a topological space. Let $X_\bullet$ be the constant
simplical topological space with value $X$. The functor
$$
X_{\bullet, Zar} \longrightarrow X_{Zar},\quad
U \longmapsto U
$$
is continuous and cocontinuous and defines a morphism of
topoi $g : \Sh(X_{\bullet, Zar}) \to \Sh(X)$ as well as a left adjoint
$g_!$ to $g^{-1}$. We have
\begin{enumerate}
\item $g^{-1}$ associates to a sheaf on $X$ the constant cosimplicial
sheaf on $X$,
\item $g_!$ associates to a sheaf $\mathcal{F}$ on $X_{\bullet, Zar}$ the
sheaf $\mathcal{F}_0$, and
\item $g_*$ associates to a sheaf $\mathcal{F}$ on $X_{\bullet, Zar}$ the
equalizer of the two maps $\mathcal{F}_0 \to \mathcal{F}_1$.
\end{enumerate}
\end{lemma}

\begin{proof}
The statements about the functor are straightforward to verify.
The existence of $g$ and $g_!$ follow from
Sites, Lemmas \ref{sites-lemma-cocontinuous-morphism-topoi} and
\ref{sites-lemma-when-shriek}. The description of
$g^{-1}$ is immediate from Sites, Lemma \ref{sites-lemma-when-shriek}.
The description of $g_*$ and $g_!$ follows as the functors given are
right and left adjoint to $g^{-1}$.
\end{proof}

\begin{lemma}
\label{lemma-augmentation}
Let $Y$ be a simplicial space and $X$ a topological space.
Let $a : Y \to X$ be an augmentation
(Simplicial, Definition \ref{simplicial-definition-augmentation}).
There is a canonical morphism of topoi
$$
a : \Sh(Y_{Zar}) \to \Sh(X)
$$
which comes from composing the morphism
$a_{Zar} : \Sh(Y_{Zar}) \to \Sh(X_{\bullet, Zar})$ of
Lemma \ref{lemma-simplicial-space-site-functorial}
with the morphism $g : \Sh(X_{\bullet, Zar}) \to \Sh(X)$ of
Lemma \ref{lemma-constant-simplicial-space}.
\end{lemma}

\begin{proof}
This lemma proves itself.
\end{proof}

\begin{lemma}
\label{lemma-simplicial-resolution-Z}
Let $X$ be a simplicial topological space. The complex of
abelian presheaves on $X_{Zar}$
$$
\ldots \to \mathbf{Z}_{X_2} \to \mathbf{Z}_{X_1} \to \mathbf{Z}_{X_0}
$$
with boundary $\sum (-1)^i d^n_i$ is a resolution
of the constant presheaf $\mathbf{Z}$.
\end{lemma}

\begin{proof}
Let $U \subset X_m$ be an object of $X_{Zar}$. Then the value of
the complex above on $U$ is the complex of abelian groups
$$
\ldots \to
\mathbf{Z}[\Mor_\Delta([2], [m])] \to
\mathbf{Z}[\Mor_\Delta([1], [m])] \to
\mathbf{Z}[\Mor_\Delta([0], [m])]
$$
In other words, this is the complex associated to the
free abelian group on the simplicial set $\Delta[m]$, see
Simplicial, Example \ref{simplicial-example-simplex-simplicial-set}.
Since $\Delta[m]$ is homotopy equivalent to $\Delta[0]$, see
Simplicial, Example \ref{simplicial-example-simplex-contractible},
and since ``taking free abelian groups'' is a functor,
we see that the complex above is homotopy equivalent to
the free abelian group on $\Delta[0]$
(Simplicial, Remark \ref{simplicial-remark-homotopy-better} and
Lemma \ref{simplicial-lemma-homotopy-equivalence-s-N}).
This complex is acyclic in positive degrees
and equal to $\mathbf{Z}$ in degree $0$.
\end{proof}

\begin{lemma}
\label{lemma-simplicial-sheaf-cohomology}
Let $X$ be a simplicial topological space. Let $\mathcal{F}$ be an abelian
sheaf on $X$. There is a spectral sequence $(E_r, d_r)_{r \geq 0}$ with
$$
E_1^{p, q} = H^q(X_p, \mathcal{F}_p)
$$
converging to $H^{p + q}(X_{Zar}, \mathcal{F})$.
This spectral sequence is functorial in $\mathcal{F}$.
\end{lemma}

\begin{proof}
Let $\mathcal{F} \to \mathcal{I}^\bullet$ be an injective resolution.
Consider the double complex with terms
$$
A^{p, q} = \mathcal{I}^q(X_p)
$$
and first differential given by the alternating sum along the maps
$d^{p + 1}_i$-maps $\mathcal{I}_p^q \to \mathcal{I}_{p + 1}^q$, see
Lemma \ref{lemma-describe-sheaves-simplicial-site}. Note that
$$
A^{p, q} = \Gamma(X_p, \mathcal{I}_p^q) =
\Mor_{\textit{PSh}}(h_{X_p}, \mathcal{I}^q) =
\Mor_{\textit{PAb}}(\mathbf{Z}_{X_p}, \mathcal{I}^q)
$$
Hence it follows from Lemma \ref{lemma-simplicial-resolution-Z} and
Cohomology on Sites, Lemma
\ref{sites-cohomology-lemma-injective-abelian-sheaf-injective-presheaf}
that the rows of the double complex are exact in positive degrees and
evaluate to $\Gamma(X_{Zar}, \mathcal{I}^q)$ in degree $0$.
On the other hand, since restriction is exact
(Lemma \ref{lemma-restriction-to-components})
the map
$$
\mathcal{F}_p \to \mathcal{I}_p^\bullet
$$
is a resolution. The sheaves $\mathcal{I}_p^q$ are injective
abelian sheaves on $X_p$
(Lemma \ref{lemma-restriction-injective-to-component}).
Hence the cohomology of the columns computes the groups
$H^q(X_p, \mathcal{F}_p)$. We conclude by applying
Homology, Lemmas \ref{homology-lemma-first-quadrant-ss} and
\ref{homology-lemma-double-complex-gives-resolution}.
\end{proof}








\section{Simplicial sites and topoi}
\label{section-simplicial-sites}

\noindent
It seems natural to define a {\it simplicial site} as a simplicial
object in the (big) category whose objects are sites
and whose morphisms are morphisms of sites.
See Sites, Definitions \ref{sites-definition-site} and
\ref{sites-definition-morphism-sites}
with composition of morphisms as in 
Sites, Lemma \ref{sites-lemma-composition-morphisms-sites}.
But here are some variants one might want to consider:
(a) we could work with cocontinuous functors
(see Sites, Sections \ref{sites-section-cocontinuous-functors} and
\ref{sites-section-cocontinuous-morphism-topoi}) between sites instead,
(b) we could work in a suitable $2$-category of sites where one introduces
the notion of a $2$-morphism between morphisms of sites,
(c) we could work in a $2$-category constructed out of cocontinuous
functors. Instead of picking one of these variants as a definition
we will simply develop theory as needed.

\medskip\noindent
Certainly a {\it simplicial topos} should probably be defined as a
pseudo-functor from $\Delta^{opp}$ into the $2$-category of topoi.
See Categories, Definition \ref{categories-definition-functor-into-2-category}
and Sites, Section \ref{sites-section-topoi} and
\ref{sites-section-2-category}. We will try to avoid working with such
a beast if possible.

\medskip\noindent
Let $\mathcal{C}$ be a simplicial object in the category whose objects
are sites and whose morphisms are morphisms of sites. This means that
for every morphism $\varphi : [m] \to [n]$ of $\Delta$ we have a morphism
of sites $f_\varphi : \mathcal{C}_n \to \mathcal{C}_m$. This morphism is
given by a continuous functor in the opposite direction which we will denote
$u_\varphi : \mathcal{C}_m \to \mathcal{C}_n$.

\begin{lemma}
\label{lemma-simplicial-site-site}
Let $\mathcal{C}$ be a simplicial object in the category of sites.
With notation as above we construct a site $\mathcal{C}_{site}$ as follows.
\begin{enumerate}
\item An object of $\mathcal{C}_{site}$ is an object $U$ of
$\mathcal{C}_n$ for some $n$,
\item a morphism $(\varphi, f) : U \to V$ of $\mathcal{C}_{site}$
is given by a map $\varphi : [m] \to [n]$ with
$U \in \Ob(\mathcal{C}_n)$, $V \in \Ob(\mathcal{C}_m)$
and a morphism $f : U \to u_\varphi(V)$ of $\mathcal{C}_n$, and
\item a covering $\{(\text{id}, f_i) :  U_i \to U\}$ in $\mathcal{C}_{site}$
is given by an $n$ and a covering $\{f_i : U_i \to U\}$
of $\mathcal{C}_n$.
\end{enumerate}
\end{lemma}

\begin{proof}
Composition of $(\varphi, f) : U \to V$ with $(\psi, g) : V \to W$
is given by $(\varphi \circ \psi, u_\varphi(g) \circ f)$.
This uses that $u_\varphi \circ u_\psi = u_{\varphi \circ \psi}$.

\medskip\noindent
Let $\{(\text{id}, f_i) :  U_i \to U\}$ be a covering as in (3)
and let $(\varphi, g) : W \to U$ be a morphism with
$W \in \Ob(\mathcal{C}_m)$. We claim that
$$
W \times_{(\varphi, g), U, (\text{id}, f_i)} U_i =
W \times_{g, u_\varphi(U), u_\varphi(f_i)} u_\varphi(U_i)
$$
in the category $\mathcal{C}_{site}$. This makes sense as by our
definition of morphisms of sites, the required fibre products
in $\mathcal{C}_m$ exist since $u_\varphi$ transforms coverings into
coverings. The same reasoning implies the claim (details omitted).
Thus we see that the collection of coverings is stable under base
change. The other axioms of a site are immediate.
\end{proof}

\noindent
Let $\mathcal{C}$ be a simplicial object in the category whose objects are
sites and whose morphisms are cocontinuous functors. This means that for
every morphism $\varphi : [m] \to [n]$ of $\Delta$ we have a cocontinuous
functor denoted $u_\varphi : \mathcal{C}_n \to \mathcal{C}_m$.

\begin{lemma}
\label{lemma-simplicial-cocontinuous-site}
Let $\mathcal{C}$ be a simplicial object in the category whose objects are
sites and whose morphisms are cocontinuous functors. With notation as above,
assume the functors $u_\varphi : \mathcal{C}_n \to \mathcal{C}_m$
have property $P$ of Sites, Remark \ref{sites-remark-cartesian-cocontinuous}.
Then we can construct a site $\mathcal{C}_{site}$ as follows.
\begin{enumerate}
\item An object of $\mathcal{C}_{site}$ is an object $U$ of
$\mathcal{C}_n$ for some $n$,
\item a morphism $(\varphi, f) : U \to V$ of $\mathcal{C}_{site}$
is given by a map $\varphi : [m] \to [n]$ with
$U \in \Ob(\mathcal{C}_n)$, $V \in \Ob(\mathcal{C}_m)$
and a morphism $f : u_\varphi(U) \to V$ of $\mathcal{C}_m$, and
\item a covering $\{(\text{id}, f_i) :  U_i \to U\}$ in $\mathcal{C}_{site}$
is given by an $n$ and a covering $\{f_i : U_i \to U\}$
of $\mathcal{C}_n$.
\end{enumerate}
\end{lemma}

\begin{proof}
Composition of $(\varphi, f) : U \to V$ with $(\psi, g) : V \to W$
is given by $(\varphi \circ \psi, g \circ u_\psi(f))$.
This uses that $u_\psi \circ u_\varphi = u_{\varphi \circ \psi}$.

\medskip\noindent
Let $\{(\text{id}, f_i) :  U_i \to U\}$ be a covering as in (3)
and let $(\varphi, g) : W \to U$ be a morphism with
$W \in \Ob(\mathcal{C}_m)$. We claim that
$$
W \times_{(\varphi, g), U, (\text{id}, f_i)} U_i =
W \times_{g, U, f_i} U_i
$$
in the category $\mathcal{C}_{site}$ where the right hand side
is the object of $\mathcal{C}_m$ defined in
Sites, Remark \ref{sites-remark-cartesian-cocontinuous}
which exists by property $P$. Compatibility of this type of fibre product
with compositions of functors implies the claim (details omitted).
Since the family $\{W \times_{g, U, f_i} U_i \to W\}$ is a
covering of $\mathcal{C}_m$ by property $P$ we see that
the collection of coverings is stable under base
change. The other axioms of a site are immediate.
\end{proof}

\begin{situation}
\label{situation-simplicial-site}
Here we have one of the following two cases:
\begin{enumerate}
\item[(A)] $\mathcal{C}$ is a simplicial object in the category whose
objects are sites and whose morphisms are morphisms of sites. For every
morphism $\varphi : [m] \to [n]$ of $\Delta$ we have a morphism of sites
$f_\varphi : \mathcal{C}_n \to \mathcal{C}_m$ given by a continuous
functor $u_\varphi : \mathcal{C}_m \to \mathcal{C}_n$.
\item[(B)] $\mathcal{C}$ is a simplicial object in the category whose
objects are sites and whose morphisms are cocontinuous functors having
property $P$ of Sites, Remark \ref{sites-remark-cartesian-cocontinuous}.
For every morphism $\varphi : [m] \to [n]$ of $\Delta$ we have a cocontinuous
functor $u_\varphi : \mathcal{C}_n \to \mathcal{C}_m$ which induces a
morphism of topoi $f_\varphi : \Sh(\mathcal{C}_n) \to \Sh(\mathcal{C}_m)$.
\end{enumerate}
As usual we will denote $f_\varphi^{-1}$ and $f_{\varphi, *}$ the
pullback and pushforward. We let $\mathcal{C}_{site}$ denote the
site defined in
Lemma \ref{lemma-simplicial-site-site} (case A) or
Lemma \ref{lemma-simplicial-cocontinuous-site} (case B).
\end{situation}

\noindent
Let $\mathcal{C}$ be as in Situation \ref{situation-simplicial-site}.
Let $\mathcal{F}$ be a sheaf on $\mathcal{C}_{site}$.
It is clear from the definition of coverings, that the restriction
of $\mathcal{F}$ to the objects of $\mathcal{C}_n$ defines a sheaf
$\mathcal{F}_n$ on the site $\mathcal{C}_n$. For every
$\varphi : [m] \to [n]$ the restriction maps of $\mathcal{F}$
along the morphisms $(\varphi, f) : U \to V$ with 
$U \in \Ob(\mathcal{C}_n)$ and $V \in \Ob(\mathcal{C}_m)$
define an element $\mathcal{F}(\varphi)$ of
$$
\Mor_{\Sh(\mathcal{C}_m)}(\mathcal{F}_m, f_{\varphi, *}\mathcal{F}_n) =
\Mor_{\Sh(\mathcal{C}_n)}(f_\varphi^{-1}\mathcal{F}_m, \mathcal{F}_n)
$$
Moreover, given $\varphi : [m] \to [n]$ and $\psi : [l] \to [m]$
we have
$$
f_\varphi^{-1}\mathcal{F}(\psi) \circ \mathcal{F}(\varphi) =
\mathcal{F}(\varphi \circ \psi)
$$
Clearly, the converse is true as well: if we have a system
$(\{\mathcal{F}_n\}_{n \geq 0},
\{\mathcal{F}(\varphi)\}_{\varphi \in \text{Arrows}(\Delta)})$
as above, satisfying the displayed equalities,
then we obtain a sheaf on $\mathcal{C}_{site}$.

\begin{lemma}
\label{lemma-describe-sheaves-simplicial-site-site}
In Situation \ref{situation-simplicial-site} there is an equivalence of
categories between
\begin{enumerate}
\item $\Sh(\mathcal{C}_{site})$, and
\item category of systems $(\mathcal{F}_n, \mathcal{F}(\varphi))$
described above.
\end{enumerate}
In particular, the topos $\Sh(\mathcal{C}_{site})$ only depends on
the topoi $\Sh(\mathcal{C}_n)$ and the morphisms of topoi $f_\varphi$.
\end{lemma}

\begin{proof}
See discussion above.
\end{proof}

\begin{lemma}
\label{lemma-restriction-to-components-site}
In Situation \ref{situation-simplicial-site} the functor
$\mathcal{C}_n \to \mathcal{C}_{site}$, $U \mapsto U$ is continuous
and cocontinuous. The associated morphism of
topoi $g : \Sh(\mathcal{C}_n) \to \Sh(\mathcal{C}_{site})$ satisfies
\begin{enumerate}
\item $g^{-1}$ associates to the sheaf $\mathcal{F}$ on $\mathcal{C}_{site}$
the sheaf $\mathcal{F}_n$ on $\mathcal{C}_n$,
\item $g^{-1}$ has a left adjoint $g^{Sh}_!$ which commutes
with finite connected limits, and
\item $g^{-1} : \textit{Ab}(\mathcal{C}_{site}) \to \textit{Ab}(\mathcal{C}_n)$
has a left adjoint
$g_! : \textit{Ab}(\mathcal{C}_n) \to \textit{Ab}(\mathcal{C}_{site})$
which is exact.
\end{enumerate}
\end{lemma}

\begin{proof}
It is clear that functor $\mathcal{C}_n \to \mathcal{C}_{site}$ is
continuous and cocontinuous. Hence part (1) and the existence
of $g^{Sh}_!$ and $g_!$ follows from
Sites, Lemmas \ref{sites-lemma-cocontinuous-morphism-topoi} and
\ref{sites-lemma-when-shriek}
and
Modules on Sites, Lemmas \ref{sites-modules-lemma-g-shriek-adjoint} and
\ref{sites-modules-lemma-back-and-forth}.

\medskip\noindent
Next, we come to the exactness properties of $g^{Sh}_!$ and $g_!$.
Perhaps the most straightforward way to prove this is to give a formula
for these functors. If $\mathcal{G}$ is a sheaf on $\mathcal{C}_n$,
then we claim $\mathcal{H} = g^{Sh}_!\mathcal{G}$ is the sheaf on
$\mathcal{C}_{site}$ whose degree $m$ part is the sheaf
$$
\mathcal{H}_m = \coprod\nolimits_{\varphi : [n] \to [m]}
f_\varphi^{-1}\mathcal{G}
$$
Given a map $\psi : [m] \to [m']$ the map
$\mathcal{H}(\psi) : f_\psi^{-1}\mathcal{H}_m \to \mathcal{H}_{m'}$
is given on components by the identifications
$$
f_\psi^{-1} f_\varphi^{-1} \mathcal{G} \to
f_{\psi \circ \varphi}^{-1}\mathcal{G}
$$
Observe that given a map $a : \mathcal{H} \to \mathcal{F}$ of sheaves on
$\mathcal{C}_{site}$ we obtain a map $\mathcal{G} \to \mathcal{F}_n$
corresponding to the restriction of $a_n$ to the component
$\mathcal{G}$ in $\mathcal{H}_n$. Conversely, given
$b : \mathcal{G} \to \mathcal{H}_n$ we can define
$a : \mathcal{H} \to \mathcal{F}$ by letting $a_m$ be the map which
on components
$$
f_\varphi^{-1}\mathcal{G} \to \mathcal{F}_m
$$
uses the maps adjoint to $\mathcal{F}(\varphi) \circ f_\varphi^{-1}b$.
We omit the arguments showing these two constructions give
mutually inverse maps
$$
\Mor_{\Sh(\mathcal{C}_n)}(\mathcal{G}, \mathcal{F}_n) =
\Mor_{\Sh(\mathcal{C}_{site})}(\mathcal{H}, \mathcal{F})
$$
thus verifying the claim above.
If $\mathcal{G}$ is an abelian sheaf on $\mathcal{C}_n$,
then $g_!\mathcal{G}$ is the abelian sheaf on $\mathcal{C}_{site}$
whose degree $m$ part is the sheaf
$$
\bigoplus\nolimits_{\varphi : [n] \to [m]} f_\varphi^{-1}\mathcal{G}
$$
with transition maps defined exactly as above. By definition of the
site $\mathcal{C}_{site}$ we see that these functors have the desired
exactness properties and we conclude.
\end{proof}

\begin{lemma}
\label{lemma-restriction-injective-to-component-site}
\begin{slogan}
An injective abelian sheaf on a simplicial site is injective on each component
\end{slogan}
In Situation \ref{situation-simplicial-site}. If $\mathcal{I}$ is an
injective abelian sheaf on $\mathcal{C}_{site}$, then $\mathcal{I}_n$ is an
injective abelian sheaf on $\mathcal{C}_n$.
\end{lemma}

\begin{proof}
This follows from
Homology, Lemma \ref{homology-lemma-adjoint-preserve-injectives}
and
Lemma \ref{lemma-restriction-to-components-site}.
\end{proof}

\noindent
Let $\mathcal{C}$ be as in Situation \ref{situation-simplicial-site}.
In statement of the following lemmas we will let
$g_n : \mathcal{C}_n \to \mathcal{C}_{site}$ be the functor of
Lemma \ref{lemma-restriction-to-components-site}. If $\varphi : [m] \to [n]$
is a morphism of $\Delta$, then the diagram of topoi
$$
\xymatrix{
\Sh(\mathcal{C}_n) \ar[rd]_{g_n} \ar[rr]_{f_\varphi} & &
\Sh(\mathcal{C}_m) \ar[ld]^{g_m} \\
& \Sh(\mathcal{C}_{site})
}
$$
is not commutative, but there is a $2$-morphism $g_n \to g_m \circ f_\varphi$
coming from the maps
$\mathcal{F}(\varphi) : f_\varphi^{-1}\mathcal{F}_m \to \mathcal{F}_n$.
See Sites, Section \ref{sites-section-2-category}.

\begin{lemma}
\label{lemma-simplicial-resolution-Z-site}
In Situation \ref{situation-simplicial-site} and with notation as above
there is a complex
$$
\ldots \to g_{2!}\mathbf{Z} \to g_{1!}\mathbf{Z} \to g_{0!}\mathbf{Z}
$$
of abelian sheaves on $\mathcal{C}_{site}$ which forms a resolution of
the constant sheaf with value $\mathbf{Z}$ on $\mathcal{C}_{site}$.
\end{lemma}

\begin{proof}
We will use the description of the functors $g_{n!}$ in the proof of
Lemma \ref{lemma-restriction-to-components-site} without further mention.
As maps of the complex we take $\sum (-1)^i d^n_i$ where
$d^n_i : g_{n!}\mathbf{Z} \to g_{n - 1!}\mathbf{Z}$ is the
adjoint to the map $\mathbf{Z} \to
\bigoplus_{[n - 1] \to [n]} \mathbf{Z} = g_n^{-1}g_{n - 1!}\mathbf{Z}$
corresponding to the factor labeled with $\delta^n_i : [n - 1] \to [n]$.
Then $g_m^{-1}$ applied to the complex gives the complex
$$
\ldots \to
\bigoplus\nolimits_{\alpha \in \Mor_\Delta([2], [m])]} \mathbf{Z} \to
\bigoplus\nolimits_{\alpha \in \Mor_\Delta([1], [m])]} \mathbf{Z} \to
\bigoplus\nolimits_{\alpha \in \Mor_\Delta([0], [m])]} \mathbf{Z}
$$
on $\mathcal{C}_m$.
In other words, this is the complex associated to the
free abelian sheaf on the simplicial set $\Delta[m]$, see
Simplicial, Example \ref{simplicial-example-simplex-simplicial-set}.
Since $\Delta[m]$ is homotopy equivalent to $\Delta[0]$, see
Simplicial, Example \ref{simplicial-example-simplex-contractible},
and since ``taking free abelian sheaf on'' is a functor,
we see that the complex above is homotopy equivalent to
the free abelian sheaf on $\Delta[0]$
(Simplicial, Remark \ref{simplicial-remark-homotopy-better} and
Lemma \ref{simplicial-lemma-homotopy-equivalence-s-N}).
This complex is acyclic in positive degrees
and equal to $\mathbf{Z}$ in degree $0$.
\end{proof}

\begin{lemma}
\label{lemma-simplicial-sheaf-cohomology-site}
In Situation \ref{situation-simplicial-site}. Let $\mathcal{F}$ be an abelian
sheaf on $\mathcal{C}_{site}$. There is a spectral sequence
$(E_r, d_r)_{r \geq 0}$ with
$$
E_1^{p, q} = H^q(\mathcal{C}_p, \mathcal{F}_p)
$$
converging to $H^{p + q}(\mathcal{C}_{site}, \mathcal{F})$.
This spectral sequence is functorial in $\mathcal{F}$.
\end{lemma}

\begin{proof}
Let $\mathcal{F} \to \mathcal{I}^\bullet$ be an injective resolution.
Consider the double complex with terms
$$
A^{p, q} = \Gamma(\mathcal{C}_p, \mathcal{I}^q_p)
$$
and first differential given by the alternating sum along the maps
$d^{p + 1}_i$-maps $\mathcal{I}_p^q \to \mathcal{I}_{p + 1}^q$, see
Lemma \ref{lemma-describe-sheaves-simplicial-site-site}. Note that
$$
A^{p, q} = \Gamma(\mathcal{C}_p, \mathcal{I}_p^q) =
\Mor_{\textit{Ab}(\mathcal{C}_{site})}(g_{p!}\mathbf{Z}, \mathcal{I}^q)
$$
Hence it follows from Lemma \ref{lemma-simplicial-resolution-Z-site}
that the rows of the double complex are exact in positive degrees
and evaluate to $\Gamma(\mathcal{C}_{site}, \mathcal{I}^q)$ in degree $0$.
On the other hand, since restriction is exact
(Lemma \ref{lemma-restriction-to-components-site})
the map
$$
\mathcal{F}_p \to \mathcal{I}_p^\bullet
$$
is a resolution. The sheaves $\mathcal{I}_p^q$ are injective
abelian sheaves on $\mathcal{C}_p$
(Lemma \ref{lemma-restriction-injective-to-component-site}).
Hence the cohomology of the columns computes the groups
$H^q(\mathcal{C}_p, \mathcal{F}_p)$. We conclude by applying
Homology, Lemmas \ref{homology-lemma-first-quadrant-ss} and
\ref{homology-lemma-double-complex-gives-resolution}.
\end{proof}







\section{Simplicial semi-representable objects}
\label{section-semi-representable}

\noindent
Let $\mathcal{C}$ be a site. Recall that $\text{SR}(\mathcal{C})$
denotes the category of semi-representable objects of $\mathcal{C}$.
See Hypercoverings, Definition \ref{hypercovering-definition-SR}.
For an object $K = \{U_i\}_{i \in I}$ of $\text{SR}(\mathcal{C})$
we will use the notation
$$
\mathcal{C}/K = \coprod\nolimits_{i \in I} \mathcal{C}/U_i
$$
and we will call it the {\it localization of $\mathcal{C}$ at $K$}.
There is a natural structure of a site on this category, with
coverings inherited from the localizations $\mathcal{C}/U_i$
(and whence from $\mathcal{C}$). If $f : K \to L$ is a morphism of
$\text{SR}(\mathcal{C})$, then we obtain a cocontinuous functor
$$
f : \mathcal{C}/K \longrightarrow \mathcal{C}/L
$$
by applying the construction of Sites, Lemma \ref{sites-lemma-relocalize}
to the components. More precisely, if $f = (\alpha, f_i)$
where $K = \{U_i\}_{i \in I}$, $L = \{V_j\}_{j \in J}$, $\alpha : I \to J$,
and $f_i : U_i \to V_{\alpha(i)}$ then $f$ maps the component
$\mathcal{C}/U_i$ into the component $\mathcal{C}/V_{\alpha(i)}$
via the construction of the aforementioned lemma.
 
\medskip\noindent
Let $K$ be a simplicial object of $\text{SR}(\mathcal{C})$.
By the construction above we obtain a simplicial object
$n \mapsto \mathcal{C}/K_n$ in the category
whose objects are sites and whose morphisms are cocontinuous
functors of sites. Since these localization functors satisfy the assumption
of Lemma \ref{lemma-simplicial-cocontinuous-site} by
Sites, Remark \ref{sites-remark-localization-cartesian-cocontinuous}
we obtain a site $(\mathcal{C}/K)_{site}$.

\medskip\noindent
We can describe this site explicitly as follows. Say
$K_n = \{U_{n, i}\}_{i \in I_n}$ and that for $\varphi : [m] \to [n]$
the morphism $K(\varphi) : K_n \to K_m$ is given by
$a(\varphi) : I_n \to I_m$ and
$f_{\varphi, i} : U_{n, i} \to U_{m, a(\varphi)(i)}$ for $i \in I_n$.
Then we have
\begin{enumerate}
\item an object of $\mathcal{C}/K$ corresponds to an object $(U/U_{n, i})$
of $\mathcal{C}/U_{n, i}$ for some $n$ and some $i \in I_n$,
\item a morphism between $U$ and $V$ is a pair $(\varphi, f)$
where $\varphi : [m] \to [n]$ with $U/U_{n, i}$ and
$V/U_{m, a(\varphi)(i)}$ and $f : U \to V$ is a morphism of $\mathcal{C}$
such that
$$
\xymatrix{
U \ar[r]_f \ar[d] & V \ar[d] \\
U_{n, i} \ar[r]^-{f_{\varphi, i}} & U_{m, a(\varphi)(i)}
}
$$
is commutative, and
\item a covering $\{(\text{id}, f_j) : U_j \to U\}$ is given by
an $n$ and $i \in I_n$ and objects $U/U_{n, i}$, $U_j/U_{n, i}$
such that $\{f_j  : U_j \to U\}$ is a covering of $\mathcal{C}$.
\end{enumerate}

\begin{lemma}
\label{lemma-sr-when-fibre-products}
Let $\mathcal{C}$ be a site. Let $K$ be a simplicial object of
$\text{SR}(\mathcal{C})$. If $\mathcal{C}$ has fibre products,
then $\mathcal{C}/K$ can also be viewed as a simplicial object
in the category whose objects are sites and whose morphisms are
morphisms of sites. The construction of
Lemma \ref{lemma-simplicial-site-site}
then produces the same site as the construction above.
\end{lemma}

\begin{proof}
Given a morphism of objects $U \to V$ of $\mathcal{C}$ the localization
morphism $j : \mathcal{C}/U \to \mathcal{C}/U$ is a left adjoint to
the base change functor $\mathcal{C}/V \to \mathcal{C}/U$.
The base change functor is continuous and induces the same morphism of
topoi as $j$. See
Sites, Lemma \ref{sites-lemma-relocalize-given-fibre-products}.
Argueing as above we can use this to define a morphism of sites
$\mathcal{C}/A \to \mathcal{C}/B$ given any morphism $A \to B$
of $\text{SR}(\mathcal{C})$. Applying this to the morphisms of
the simplicial object $K$ we obtain simplicial object
$(\mathcal{C}/K)'$ in the category of sites with morphisms of sites.
Let $(\mathcal{C}/K)'_{site}$ be the site constructed in
Lemma \ref{lemma-simplicial-site-site}.
Since the base change functors are adjoint to the localization
functors, we find that $(\mathcal{C}/K)'_{site}$ is the same
as the category $(\mathcal{C}/K)_{site}$. Equality of the
sets of coverings is immediate from the definitions.
\end{proof}

\noindent
Let $\mathcal{C}$ be a site. Let $L = \{V_i\}$ be an object of
$\text{SR}(\mathcal{C})$. There is a continuous and cocontinuous
localization functor $j : \mathcal{C}/K \to \mathcal{C}$ which is
the product of the localization functors $\mathcal{C}/V_i \to \mathcal{C}$.
We obtain functors $j^{-1}$, $j_*$, $j^{Sh}_!$, and $j_!$ exactly
as in Sites, Section \ref{sites-section-localize} and
Modules on Sites, Section \ref{sites-modules-section-localize}.
Given a simplicial
object $K$ of $\text{SR}(\mathcal{C})$ we obtain a family
of localization functors $j_n : \mathcal{C}/K_n \to \mathcal{C}$.

\begin{lemma}
\label{lemma-comparison}
Let $\mathcal{C}$ be a site. Let $K$ be a simplicial object of
$\text{SR}(\mathcal{C})$. The forgetful functor
$(\mathcal{C}/K)_{site} \to \mathcal{C}$ is continuous and cocontinuous
and induces a morphism of topoi
$$
g : \Sh((\mathcal{C}/K)_{site}) \longrightarrow \Sh(\mathcal{C})
$$
as well as functors $g^{Sh}_!$ and $g_!$ left adjoint to $g^{-1}$
on sheaves of sets and abelian groups with the following properties:
\begin{enumerate}
\item the functor $g^{-1}$ associates to a sheaf $\mathcal{F}$ on
$\mathcal{C}$ the sheaf on $(\mathcal{C}/K)_{site}$ wich in degree $n$
is equal to $j_n^{-1}\mathcal{F}$,
\item the functor $g_*$ associates to a sheaf $\mathcal{G}$ on
$(\mathcal{C}/K)_{site}$ the equalizer of the two maps
$j_{0, *}\mathcal{G}_0 \to j_{1, *}\mathcal{G}_1$,
\end{enumerate}
\end{lemma}

\begin{proof}
The functor is continuous and cocontinuous by our choice of coverings and
our description of (certain) fibre products in $(\mathcal{C}/K)_{site}$
in the proof of Lemma \ref{lemma-simplicial-cocontinuous-site}. Details omitted.
Thus we obtain a morphism of topoi and functors $g^{Sh}_!$ and $g_!$, see
Sites, Section \ref{sites-section-cocontinuous-morphism-topoi} and
Modules on Sites, Section
\ref{sites-modules-section-exactness-lower-shriek}.
The description of $g^{-1}$ is immediate from the definition as the
compostion $\mathcal{C}/K_n \to \mathcal{C}/K \to \mathcal{C}$ is
the localization morphism $j_n$.

\medskip\noindent
Proof of (2). Let $\mathcal{F}$ be a sheaf on $\mathcal{C}$ and let
$\mathcal{G}$ be a sheaf on $(\mathcal{C}/K)_{site}$. A map
$a : g^{-1}\mathcal{F} \to \mathcal{G}$ corresponds to a system of maps
$a_n : j_n^{-1}\mathcal{F} \to \mathcal{G}_n$ on $\mathcal{C}/K_n$
by Lemma \ref{lemma-describe-sheaves-simplicial-site-site}.
Taking $n = 0$ we get a map $j_0^{-1}\mathcal{F} \to \mathcal{G}_0$
which is adjoint to a map $a_0 : \mathcal{F} \to j_{0, *}\mathcal{G}_0$.
Since $a_0$ is compatible with $a_1$ via the two maps
$j_{0, *}\mathcal{G}_0 \to j_{1, *}\mathcal{G}_1$ we see that
$a_0$ maps into the equalizer. Conversely, given a map
$a_0 : \mathcal{F} \to j_{0, *}\mathcal{G}_0$ into the equalizer
we can pick, for any $n$, one of the maps
$j_{0, *}\mathcal{G}_0 \to j_{n, *}\mathcal{G}_n$ and compose
to get a well defined map $a_n : \mathcal{F} \to j_{n, *}\mathcal{G}_n$.
These fit together to define a map of sheaves
$g^{-1}\mathcal{F} \to \mathcal{G}$.
\end{proof}

\begin{lemma}
\label{lemma-compare-cohomology-hypercovering}
Let $\mathcal{C}$ be a site with equalizers and fibre products.
Let $\mathcal{G}$ be a presheaf of sets on
$\mathcal{C}$. Let $K$ be a hypercovering of $\mathcal{G}$, see
Hypercoverings, Definition
\ref{hypercovering-definition-hypercovering-variant}.
Then we have a canonical isomorphism
$$
R\Gamma(\mathcal{G}, E) =
R\Gamma((\mathcal{C}/K)_{site}, g^{-1}E)
$$
for $E \in D^+(\mathcal{C})$. If $K$ is a
hypercovering, then
$R\Gamma(E) = R\Gamma((\mathcal{C}/K)_{site}, g^{-1}E)$.
\end{lemma}

\begin{proof}
First, let $\mathcal{I}$ be an injective abelian sheaf on $\mathcal{C}$.
Then the spectral sequence of
Lemma \ref{lemma-simplicial-sheaf-cohomology-site}
for the sheaf $g^{-1}\mathcal{I}$ degenerates as
$(g^{-1}\mathcal{I})_p$ is the restriction of $\mathcal{I}$
to $\mathcal{C}/K_p$ which is injective by
Cohomology on Sites, Lemma \ref{sites-cohomology-lemma-cohomology-of-open}
(extended in the obvious manner to localization at
semi-representable objects of $\mathcal{C}$).
Thus we see that the complex
$$
\mathcal{I}(K_0) \to \mathcal{I}(K_1) \to \mathcal{I}(K_2) \to \ldots
$$
computes $R\Gamma((\mathcal{C}/K)_{site}, g^{-1}\mathcal{I})$.
This is exactly the {\v C}ech complex of $\mathcal{I}$ with respect
to the simplicial object $K$ of $\text{SR}(\mathcal{C})$ as defined in
Hypercoverings, Section \ref{hypercovering-section-hyper-cech}.
Thus
Hypercoverings, Lemma \ref{hypercovering-lemma-injective-trivial-cech-variant}
shows that this complex computes $R\Gamma(\mathcal{G}, \mathcal{I})$
(which has zero higher cohomology groups as $\mathcal{I}$ is injective).
In other words, we have
$H^0(\mathcal{G}, \mathcal{I}) = H^0((\mathcal{C}/K)_{site}, \mathcal{I})$
and
$H^p(\mathcal{G}, \mathcal{I}) = H^p((\mathcal{C}/K)_{site}, \mathcal{I}) = 0$
for all $p > 0$.

\medskip\noindent
The lemma now follows formally. Namely, let $A \in D^+(\mathcal{C})$
be arbitrary. We can represent $A$ by a bounded below complex
$\mathcal{I}^\bullet$ of injective abelian sheaves. By Leray's acyclicity
lemma (Derived Categories, Lemma \ref{derived-lemma-leray-acyclicity})
$R\Gamma((\mathcal{C}/K)_{site}, A)$
is computed by the complex
$\Gamma((\mathcal{C}/K)_{site}, g^{-1}\mathcal{I}^\bullet)$
and $R\Gamma(\mathcal{G}, A)$ is computed by
$\Gamma(\mathcal{G}, \mathcal{I}^\bullet)$.
Since these complexes are the same we obtain the conclusion.

\medskip\noindent
The final statement refers to the special case where $\mathcal{G} = *$
is the final object in the category of presheaves on $\mathcal{C}$.
\end{proof}

\begin{lemma}
\label{lemma-compare-cohomology-hypercovering-X}
Let $\mathcal{C}$ be a site with fibre products. Let $X$ be an object of
$\mathcal{C}$. Let $K$ be a hypercovering of $X$, see
Hypercoverings, Definition
\ref{hypercovering-definition-hypercovering}.
Then we have a canonical isomorphism
$$
R\Gamma(X, E) =
R\Gamma((\mathcal{C}/K)_{site}, g^{-1}E)
$$
for $E \in D^+(\mathcal{C})$.
\end{lemma}

\begin{proof}
If $\mathcal{C}$ also has equalizers, then this is a special case of
Lemma \ref{lemma-compare-cohomology-hypercovering}
because a hypercovering of $X$ is a hypercovering of $h_X$ by
Hypercoverings, Lemma \ref{hypercovering-lemma-hypercovering-F}.
This also uses that
$H^q(h_X, \mathcal{F}) = H^q(h_X^\#, \mathcal{F}) = H^q(X, \mathcal{F})$,
see discussion in
Hypercoverings, Section \ref{hypercovering-section-hypercoverings-verdier}
and
Cohomology on Sites, Section \ref{sites-cohomology-section-limp}.
In general (when $\mathcal{C}$ does not have equalizers) one proves
this using {\it exactly} the same argument as in the proof of
Lemma \ref{lemma-compare-cohomology-hypercovering}
but substituting
Hypercoverings, Lemma \ref{hypercovering-lemma-injective-trivial-cech}
for
Hypercoverings, Lemma \ref{hypercovering-lemma-injective-trivial-cech-variant}.
\end{proof}






\section{Hypercovering in a site}
\label{section-hypercovering}

\noindent
In the previous section we worked out, in great generality, how
hypercoverings give rise to simplicial sites and how cohomology
of (say) constant sheaves on this site computes the cohomology
of the object the hypercovering is augmented towards. In this section
we explain what this means in a special case.

\medskip\noindent
Let $\mathcal{C}$ be a site with fibre products. Let $X$ be an object of
$\mathcal{C}$ and let $X_\bullet$ be a simplicial object of $\mathcal{C}$.
Assume we have an augmentation
$$
a : X_\bullet \to X
$$
The discussion above turns this into a morphism of topoi
$$
g : (\mathcal{C}/X_\bullet)_{site} \longrightarrow \mathcal{C}/X
$$
Here an object of the site $(\mathcal{C}/X_\bullet)_{site}$ is given by
a $U/X_n$ and a morphism $(\varphi, f) : U/X_n \to V/X_m$ is given
by a morphism $\varphi : [m] \to [n]$ in $\Delta$ and a morphism
$f : U \to V$ such that the diagram
$$
\xymatrix{
U \ar[r]_f \ar[d] & V \ar[d] \\
X_n \ar[r]^\varphi & X_m
}
$$
is commutative. The morphism of topoi $g$ is given by the cocontinuous
functor $U/X_n \mapsto U/X$. That's all folks!

\medskip\noindent
Thus we may translate some of the results above to this setting.
For example, let us say that the augmentation is a {\it hypercovering}
if the following hold
\begin{enumerate}
\item $\{X_0 \to X\}$ is a covering of $\mathcal{C}$,
\item $\{X_1 \to X_0 \times_X X_0\}$ is a covering of $\mathcal{C}$,
\item $\{X_{n + 1} \to (\text{cosk}_n\text{sk}_n X_\bullet)_{n + 1}\}$
is a covering of $\mathcal{C}$ for $n \geq 1$.
\end{enumerate}
The category $\mathcal{C}/X$ has all finite limits, hence the
coskeleta used in the formulation above exist.

\begin{lemma}
\label{lemma-compare-cohomology-hypercovering-X-simple}
In the situation above assume that $X_\bullet$ is a hypercovering of $X$.
Then we have a canonical isomorphism
$$
R\Gamma(X, E) = R\Gamma((\mathcal{C}/X_\bullet)_{site}, g^{-1}E)
$$
for $E \in D^+(\mathcal{C}/X)$.
\end{lemma}

\begin{proof}
This is a special case of
Lemma \ref{lemma-compare-cohomology-hypercovering-X}.
\end{proof}








\section{Proper hypercoverings in topology}
\label{section-proper-hypercovering}

\noindent
Let's work in the category $\textit{LC}$ of Hausdorff and locally
quasi-compact topological spaces and continuous maps, see
Cohomology on Sites, Section \ref{sites-cohomology-section-cohomology-LC}.
Let $X$ be an object of $\textit{LC}$ and let $X_\bullet$ be a simplicial
object of $\textit{LC}$. Assume we have an augmentation
$$
a : X_\bullet \to X
$$
We say that $X_\bullet$ is a {\it proper hypercovering} of $X$ if
\begin{enumerate}
\item $X_0 \to X$ is a proper surjective map,
\item $X_1 \to X_0 \times_X X_0$ is a proper surjective map,
\item $X_{n + 1} \to (\text{cosk}_n\text{sk}_n X_\bullet)_{n + 1}$
is a proper surjective map for $n \geq 1$.
\end{enumerate}
The category $\textit{LC}$ has all finite limits, hence the
coskeleta used in the formulation above exist.
$$
\fbox{Principle: Proper hypercoverings can be used to compute cohomology.}
$$
A key idea behind the proof of the principle is to find a topology
on $\textit{LC}$ which is stronger than the usual one such that
(A) a surjective proper map defines a covering, and
(B) cohomology of usual sheaves with respect to this stronger
topology agrees with the usual cohomology.
Properties (A) and (B) hold for the qc topology, see
Cohomology on Sites, Section \ref{sites-cohomology-section-cohomology-LC}.
Once we have (A) and (B) we deduce the principle via
a combination of the spectral sequences of
Hypercoverings, Lemma \ref{hypercovering-lemma-cech-spectral-sequence}
and
Lemma \ref{lemma-simplicial-sheaf-cohomology}.
The following lemma is just a first step.

\begin{lemma}
\label{lemma-spectral-sequence-proper-hypercovering}
In the situation above, let $\mathcal{F}$ be an abelian sheaf
on $X$. Let $\mathcal{F}_n$ be the pullback to $X_n$.
If $X_\bullet$ is a proper hypercovering of $X$, then
there exists a canonical spectral sequence
$$
E_1^{p, q} = H^q(X_p, \mathcal{F}_p)
$$
converging to $H^{p + q}(X, \mathcal{F})$.
\end{lemma}

\begin{proof}
By Cohomology on Sites, Lemma
\ref{sites-cohomology-lemma-compare-cohomology-LC}
we have
$$
H^*(X, \mathcal{F}) =
H^*(\textit{LC}_{qc}/X, \epsilon^{-1}\pi^{-1}\mathcal{F}).
$$
Since a proper surjective map defines a qc covering
(Cohomology on Sites, Lemma
\ref{sites-cohomology-lemma-proper-surjective-is-qc-covering})
we see that $X_\bullet \to X$ is a hypercovering in the site
$\textit{LC}_{qc}$ as in
Section \ref{section-hypercovering}.
Thus we have
$$
R\Gamma(X, \mathcal{F}) =
R\Gamma(\textit{LC}_{qc}/X, \epsilon^{-1}\pi^{-1}\mathcal{F}) =
R\Gamma((\textit{LC}/X_\bullet)_{site}, g^{-1}\epsilon^{-1}\pi^{-1}\mathcal{F})
$$
by Lemma \ref{lemma-compare-cohomology-hypercovering-X-simple}.
By Lemma \ref{lemma-simplicial-sheaf-cohomology-site}
there is a spectral sequence with
$$
E_1^{p, q} =
H^q(\textit{LC}_{qc}/X_p, (g^{-1}\epsilon^{-1}\pi^{-1}\mathcal{F})_p)
$$
converging to the cohomology of $g^{-1}\epsilon^{-1}\pi^{-1}\mathcal{F}$.
Finally, the restriction $(g^{-1}\epsilon^{-1}\pi^{-1}\mathcal{F})_p$
is just the restriction to $\textit{LC}_{qc}/X_p$ of
$\epsilon^{-1}\pi^{-1}\mathcal{F}$ which by
Cohomology on Sites, Lemma \ref{sites-cohomology-lemma-describe-pullback-pi}
is the pullback of $\mathcal{F}_p$ to $\textit{LC}_{qc}/X_p$.
By Cohomology on Sites, Lemma
\ref{sites-cohomology-lemma-compare-cohomology-LC}
again we conclude that
$$
H^q(\textit{LC}_{qc}/X_p, (g^{-1}\epsilon^{-1}\pi^{-1}\mathcal{F})_p)
= H^q(X_p, \mathcal{F}_p)
$$
and the proof is finished.
\end{proof}

\begin{lemma}
\label{lemma-compute-via-proper-hypercovering}
In the situation above, let $\mathcal{F}$ be an abelian sheaf on $X$.
Let $\mathcal{F}_\bullet$ be the pullback of $\mathcal{F}$ via
$a : X_\bullet \to X$. If $X_\bullet$ is a proper hypercovering
of $X$, then
$$
H^*(X, \mathcal{F}) = H^*((X_\bullet)_{Zar}, \mathcal{F}_\bullet)
$$
\end{lemma}

\begin{proof}
Consider the continuous functor
$$
(X_\bullet)_{Zar} \longrightarrow (\textit{LC}_{qc}/X_\bullet)_{site},\quad
U \longmapsto U
$$
We obtain a commutative diagram of topoi
$$
\xymatrix{
\Sh((\textit{LC}_{qc}/X_\bullet)_{site}) \ar[d]_g \ar[r] &
\Sh((X_\bullet)_{Zar}) \ar[d]_g \\
\Sh(\textit{LC}_{qc}/X) \ar[r]^{\pi \circ \epsilon} & \Sh(X_{Zar})
}
$$
Thus our sheaf $\mathcal{F}$ gives rise to a compatible collection
of abelian sheaves in each topos. In the proof of
Lemma \ref{lemma-spectral-sequence-proper-hypercovering}
we have seen that the sheaf $\mathcal{F}$ has the same cohomology as the sheaf
$\epsilon^{-1}\pi^{-1}\mathcal{F}$ and
$g^{-1}\epsilon^{-1}\pi^{-1}\mathcal{F}$.
On the other hand, the terms of the spectral sequence of
Lemma \ref{lemma-simplicial-sheaf-cohomology}
for $\mathcal{F}_\bullet$ are the same as those in the statement and
proof of Lemma \ref{lemma-spectral-sequence-proper-hypercovering}.
A simple argument with spectral sequences then shows that the map
$$
R\Gamma((X_\bullet)_{Zar}, \mathcal{F}_\bullet)
\longrightarrow
R\Gamma((\textit{LC}_{qc}/X_\bullet)_{site},
g^{-1}\epsilon^{-1}\pi^{-1}\mathcal{F})
$$
is an isomorphism. Some details omitted.
\end{proof}

\begin{lemma}
\label{lemma-cohomological-descent-for-proper-hypercovering}
In the situation above, assume $a : X_\bullet \to X$
gives a proper hypercovering of $X$. Then for all $K \in D^+(X)$
$$
K \to Ra_*(a^{-1}K)
$$
is an isomorphism where
$a : \Sh((X_\bullet)_{Zar}) \to \Sh(X)$ is as in
Lemma \ref{lemma-augmentation}.
\end{lemma}

\begin{proof}
Observe that for any abelian sheaf
$\mathcal{F}$ on $X$ the sheaf $R^qa_*(a^{-1}\mathcal{F})$ is the sheaf
associated to the presheaf
$$
U \mapsto H^q((U_\bullet)_{Zar}, a^{-1}\mathcal{F}) = H^q(U, \mathcal{F})
$$
where $U_\bullet = a^{-1}(U)$. The last equality holds by
Lemma \ref{lemma-compute-via-proper-hypercovering}.
Thus $R^qa_*(a^{-1}\mathcal{F})$ is zero for $q > 0$ and equal
to $\mathcal{F}$ for $q = 0$. This proves the result in case
$K$ consists of a single abelian sheaf in a single degree.
The general case follows from this immediately.
\end{proof}



\section{Simplicial schemes}
\label{section-simplicial}

\noindent
A {\it simplicial scheme} is a simplicial object in the category of schemes,
see Simplicial, Definition \ref{simplicial-definition-simplicial-object}.
Recall that a simplicial scheme looks like
$$
\xymatrix{
X_2
\ar@<2ex>[r]
\ar@<0ex>[r]
\ar@<-2ex>[r]
&
X_1
\ar@<1ex>[r]
\ar@<-1ex>[r]
\ar@<1ex>[l]
\ar@<-1ex>[l]
&
X_0
\ar@<0ex>[l]
}
$$
Here there are two morphisms $d^1_0, d^1_1 : X_1 \to X_0$
and a single morphism $s^0_0 : X_0 \to X_1$, etc.
It is important to keep in mind that $d^n_i : X_n \to X_{n - 1}$
should be thought of as a ``projection forgetting the
$i$th coordinate'' and $s^n_j : X_n \to X_{n + 1}$ as the diagonal
map repeating the $j$th coordinate.





\section{Descent in terms of simplicial schemes}
\label{section-simplicial-descent}

\noindent
Cartesian morphisms are defined as follows.

\begin{definition}
\label{definition-cartesian-morphism}
Let $a : Y \to X$ be a morphism of simplicial schemes.
We say $a$ is {\it cartesian}, or that {\it $Y$ is cartesian over $X$},
if for every morphism $\varphi : [n] \to [m]$ of $\Delta$ the corresponding
diagram
$$
\xymatrix{
Y_m \ar[r]_a \ar[d]_{Y(\varphi)} & X_m \ar[d]^{X(\varphi)}\\
Y_n \ar[r]^{a} & X_n
}
$$
is a fibre square in the category of schemes.
\end{definition}

\noindent
Cartesian morphisms are related to descent data. First we prove a general
lemma describing the category of cartesian simplicial schemes over a
fixed simplicial scheme. In this lemma we denote $f^* : \Sch/X \to \Sch/Y$
the base change functor associated to a morphism of schemes $f :Y \to X$.

\begin{lemma}
\label{lemma-characterize-cartesian-schemes}
Let $X$ be a simplicial scheme. The category of simplicial schemes cartesian
over $X$ is equivalent to the category of pairs $(V, \varphi)$
where $V$ is a scheme over $X_0$ and
$$
\varphi :
V \times_{X_0, d^1_1} X_1
\longrightarrow
X_1 \times_{d^1_0, X_0} V
$$
is an isomorphism over $X_1$ such that
$(s_0^0)^*\varphi = \text{id}_V$ and such that
$$
(d^2_1)^*\varphi = (d^2_0)^*\varphi \circ (d^2_2)^*\varphi
$$
as morphisms of schemes over $X_2$.
\end{lemma}

\begin{proof}
The statement of the displayed equality makes sense because
$d^1_1 \circ d^2_2 = d^1_1 \circ d^2_1$,
$d^1_1 \circ d^2_0 = d^1_0 \circ d^2_2$, and
$d^1_0 \circ d^2_0 = d^1_0 \circ d^2_1$ as morphisms $X_2 \to X_0$, see
Simplicial, Remark \ref{simplicial-remark-relations} hence we
can picture these maps as follows
$$
\xymatrix{
&
X_2 \times_{d^1_1 \circ d^2_0, X_0} V
\ar[r]_-{(d^2_0)^*\varphi} &
X_2 \times_{d^1_0 \circ d^2_0, X_0} V
\ar@{=}[rd] & \\
X_2 \times_{d^1_0 \circ d^2_2, X_0} V
\ar@{=}[ru] & & &
X_2 \times_{d^1_0 \circ d^2_1, X_0} V \\
&
X_2 \times_{d^1_1 \circ d^2_2, X_0} V
\ar[lu]^{(d^2_2)^*\varphi} \ar@{=}[r] &
X_2 \times_{d^1_1 \circ d^2_1, X_0} V
\ar[ru]_{(d^2_1)^*\varphi}
}
$$
and the condition signifies the diagram is commutative. It is clear that
given a simplicial scheme $Y$ cartesian over $X$ we can
set $V = Y_0$ and $\varphi$ equal to the composition
$$
V \times_{X_0, d^1_1} X_1 =
Y_0 \times_{X_0, d^1_1} X_1 = Y_1 =
X_1 \times_{X_0, d^1_0} Y_0 =
X_1 \times_{X_0, d^1_0} V
$$
of identifications given by the cartesian structure. To prove this functor
is an equivalence we construct a quasi-inverse. The construction of
the quasi-inverse is analogous to the construction discussed in
Descent, Section \ref{descent-section-descent-modules} from which we borrow
the notation $\tau^n_i : [0] \to [n]$, $0 \mapsto i$ and
$\tau^n_{ij} : [1] \to [n]$, $0 \mapsto i$, $1 \mapsto j$.
Namely, given a pair $(V, \varphi)$
as in the lemma we set $Y_n = X_n \times_{X(\tau^n_n), X_0} V$.
Then given $\beta : [n] \to [m]$ we define
$V(\beta) : Y_m \to Y_n$ as the pullback by $X(\tau^m_{\beta(n)m})$
of the map $\varphi$ postcomposed by the projection
$X_m \times_{X(\beta), X_n} Y_n \to Y_n$. This makes sense because
$$
X_m \times_{X(\tau^m_{\beta(n)m}), X_1} X_1 \times_{d^1_1, X_0} V
=
X_m \times_{X(\tau^m_m), X_0} V = Y_m
$$
and
$$
X_m \times_{X(\tau^m_{\beta(n)m}), X_1} X_1 \times_{d^1_0, X_0} V =
X_m \times_{X(\tau^m_{\beta(n)}), X_0} V =
X_m \times_{X(\beta), X_n} Y_n.
$$
We omit the verification that the commutativity
of the displayed diagram
above implies the maps compose correctly. We also omit the verification
that the two functors are quasi-inverse to each other.
\end{proof}

\begin{definition}
\label{definition-fibre-products-simplicial-scheme}
Let $f : X \to S$ be a morphism of schemes. The {\it simplicial scheme
associated to $f$}, denoted $(X/S)_\bullet$, is the functor
$\Delta^{opp} \to \Sch$, $[n] \mapsto X \times_S \ldots \times_S X$
described in
Simplicial, Example \ref{simplicial-example-fibre-products-simplicial-object}.
\end{definition}

\noindent
Thus $(X/S)_n$ is the $(n + 1)$-fold fibre product of $X$ over $S$.
The morphism $d^1_0 : X \times_S X \to X$ is the map
$(x_0, x_1) \mapsto x_1$ and the morphism $d^1_1$ is the other
projection. The morphism $s^0_0$ is the diagonal morphism
$X \to X \times_S X$.

\begin{lemma}
\label{lemma-cartesian-over}
Let $f : X \to S$ be a morphism of schemes.
Let $\pi : Y \to (X/S)_\bullet$ be a cartesian morphism
of simplicial schemes.
Set $V = Y_0$ considered as a scheme over $X$.
The morphisms $d^1_0, d^1_1 : Y_1 \to Y_0$ and the morphism
$\pi_1 : Y_1 \to X \times_S X$ induce isomorphisms
$$
\xymatrix{
V \times_S X & &
Y_1 \ar[ll]_-{(d^1_1, \text{pr}_1 \circ \pi_1)}
\ar[rr]^-{(\text{pr}_0 \circ \pi_1, d^1_0)} & &
X \times_S V.
}
$$
Denote $\varphi : V \times_S X \to X \times_S V$ the
resulting isomorphism.
Then the pair $(V, \varphi)$ is a descent datum relative
to $X \to S$.
\end{lemma}

\begin{proof}
This is a special case of (part of)
Lemma \ref{lemma-characterize-cartesian-schemes}
as the displayed equation of that lemma is
equivalent to the cocycle condition of
Descent, Definition \ref{descent-definition-descent-datum}.
\end{proof}

\begin{lemma}
\label{lemma-cartesian-equivalent-descent-datum}
Let $f : X \to S$ be a morphism of schemes. The construction
$$
\begin{matrix}
\text{category of cartesian } \\
\text{schemes over } (X/S)_\bullet
\end{matrix}
\longrightarrow
\begin{matrix}
\text{ category of descent data} \\
\text{ relative to } X/S
\end{matrix}
$$
of Lemma \ref{lemma-cartesian-over}
is an equivalence of categories.
\end{lemma}

\begin{proof}
The functor from left to right is given in
Lemma \ref{lemma-cartesian-over}.
Hence this is a special case of
Lemma \ref{lemma-characterize-cartesian-schemes}.
\end{proof}

\noindent
We may reinterpret the pullback of
Descent, Lemma \ref{descent-lemma-pullback} as follows.
Suppose given a morphism of simplicial schemes $f : X' \to X$ and a
cartesian morphism of simplicial schemes $Y \to X$. Then
the fibre product (viewed as a ``pullback'')
$$
f^*Y = Y \times_X X'
$$
of simplicial schemes is a simplicial scheme cartesian over $X'$.
Suppose given a commutative diagram of morphisms of schemes
$$
\xymatrix{
X' \ar[r]_f \ar[d] & X \ar[d] \\
S' \ar[r] & S.
}
$$
This gives rise to a morphism of simplicial schemes
$$
f_\bullet : (X'/S')_\bullet \longrightarrow (X/S)_\bullet.
$$
We claim that the ``pullback'' $f_\bullet^*$ along the morphism
$f_\bullet : (X'/S')_\bullet \to (X/S)_\bullet$ corresponds via
Lemma \ref{lemma-cartesian-equivalent-descent-datum}
with the pullback defined in terms of descent data in
the aforementioned
Descent, Lemma \ref{descent-lemma-pullback}.







\section{Quasi-coherent modules on simplicial schemes}
\label{section-modules-simplicial}

\noindent
In the following definition we make use of the description of
sheaves on a simplicial space given in
Lemma \ref{lemma-describe-sheaves-simplicial-site}.

\begin{definition}
\label{definition-cartesian-sheaf}
Let $S$ be a scheme. Let $U$ be a simplicial scheme over $S$.
\begin{enumerate}
\item A {\it quasi-coherent sheaf} on $U$ is given by
a sheaf of $\mathcal{O}_U$-modules $\mathcal{F}$ such that
$\mathcal{F}_n$ is quasi-coherent for all $n \geq 0$.
\item A quasi-coherent sheaf $\mathcal{F}$ on $U$ is {\it cartesian}
if and only if all the maps
$\mathcal{F}(\varphi) : \mathcal{F}_n \to \mathcal{F}_m$
induce isomorphisms
$U(\varphi)^*\mathcal{F}_n \to \mathcal{F}_m$.
\end{enumerate}
\end{definition}

\noindent
The property on pullbacks needs only be checked for the degeneracies.

\begin{lemma}
\label{lemma-check-cartesian-module}
Let $S$ be a scheme. Let $U$ be a simplicial scheme over $S$.
Let $\mathcal{F}$ be a quasi-coherent module on $U$.
Then $\mathcal{F}$ is cartesian if and only if the induced
maps $(d^n_j)^*\mathcal{F}_{n - 1} \to \mathcal{F}_n$ are
isomorphisms.
\end{lemma}

\begin{proof}
The category $\Delta$ is generated by the morphisms
the morphisms $\delta^n_j$ and $\sigma^n_j$, see
Simplicial, Lemma \ref{simplicial-lemma-face-degeneracy}.
Hence we only need to check the maps
$(d^n_j)^*\mathcal{F}_{n - 1} \to \mathcal{F}_n$
and $(s^n_j)^*\mathcal{F}_{n + 1} \to \mathcal{F}_n$ are
isomorphisms, see
Simplicial, Lemma \ref{simplicial-lemma-characterize-simplicial-object}
for notation. But $d_j^{n + 1} \circ s^n_j = \text{id}_{U_n}$
so it the result for $d^{n + 1}_j$ implies the result
for $s^n_j$.
\end{proof}

\begin{lemma}
\label{lemma-characterize-cartesian-modules}
Let $S$ be a scheme. Let $U$ be a simplicial scheme over $S$.
The category of cartesian quasi-coherent modules over $U$
is equivalent to the category of pairs $(\mathcal{F}, \alpha)$
where $\mathcal{F}$ is a quasi-coherent module over $U_0$
and
$$
\alpha : (d_1^1)^*\mathcal{F} \longrightarrow (d_0^1)^*\mathcal{F}
$$
is an isomorphism such that $(s_0^0)^*\alpha = \text{id}_\mathcal{F}$
and such that
$$
(d^2_1)^*\alpha = (d^2_0)^*\alpha \circ (d^2_2)^*\alpha
$$
on $X_2$.
\end{lemma}

\begin{proof}
The statement of the displayed equality makes sense because
$d^1_1 \circ d^2_2 = d^1_1 \circ d^2_1$,
$d^1_1 \circ d^2_0 = d^1_0 \circ d^2_2$, and
$d^1_0 \circ d^2_0 = d^1_0 \circ d^2_1$ as morphisms $X_2 \to X_0$, see
Simplicial, Remark \ref{simplicial-remark-relations} hence we
can picture these maps as follows
$$
\xymatrix{
& (d^2_0)^*(d^1_1)^*\mathcal{F} \ar[r]_-{(d^2_0)^*\alpha} &
(d^2_0)^*(d^1_0)^*\mathcal{F} \ar@{=}[rd] & \\
(d^2_2)^*(d^1_0)^*\mathcal{F} \ar@{=}[ru] & & &
(d^2_1)^*(d^1_0)^*\mathcal{F} \\
& (d^2_2)^*(d^1_1)^*\mathcal{F} \ar[lu]^{(d^2_2)^*\alpha} \ar@{=}[r] &
(d^2_1)^*(d^1_1)^*\mathcal{F} \ar[ru]_{(d^2_1)^*\alpha}
}
$$
and the condition signifies the diagram is commutative. It is clear that
given a cartesian quasi-coherent sheaf $\mathcal{F}$ we can
set $\mathcal{F} = \mathcal{F}_0$ and $\alpha$ equal to the composition
$$
(d_1^0)^*\mathcal{F}_0 = \mathcal{F}_1 = (d_0^0)^*\mathcal{F}_0
$$
of identifications given by the cartesian structure. To prove this functor
is an equivalence we construct a quasi-inverse. The construction of
the quasi-inverse is analogous to the construction discussed in
Descent, Section \ref{descent-section-descent-modules} from which we borrow
the notation $\tau^n_i : [0] \to [n]$, $0 \mapsto i$ and
$\tau^n_{ij} : [1] \to [n]$, $0 \mapsto i$, $1 \mapsto j$.
Namely, given a pair $(\mathcal{F}, \alpha)$
as in the lemma we set $\mathcal{F}_n = X(\tau^n_n)^*\mathcal{F}$.
Then given $\beta : [n] \to [m]$ we define
$\mathcal{F}(\beta) : \mathcal{F}_n \to \mathcal{F}_m$ as the
pullback by $X(\tau^m_{\beta(n)m})$ of the map $\alpha$ precomposed
with the canonical $X(\beta)$-map $\mathcal{F}_n \to X(\beta)^*\mathcal{F}_n$.
We omit the verification that the commutativity of the displayed diagram
above implies the maps compose correctly. We also omit the verification
that the two functors are quasi-inverse to each other.
\end{proof}

\begin{lemma}
\label{lemma-pullback-cartesian-module}
Let $f : V \to U$ be a morphism of simplicial schemes. Given a cartesian
quasi-coherent module $\mathcal{F}$ on $U$ the pullback
$f^*\mathcal{F}$ is a cartesian quasi-coherent module on $V$.
\end{lemma}

\begin{proof}
This is immediate from the definitions.
\end{proof}

\begin{lemma}
\label{lemma-pushforward-cartesian-module}
Let $f : V \to U$ be a cartesian morphism of simplicial schemes.
Assume the morphisms $d^n_j : U_n \to U_{n - 1}$ are
flat and the morphisms $V_n \to U_n$ are quasi-compact and quasi-separated.
For a cartesian quasi-coherent module $\mathcal{G}$ on $V$
the pushforward $f_* \mathcal{G}$ is a cartesian
quasi-coherent module on $U$.
\end{lemma}

\begin{proof}
If $\mathcal{F} = f_* \mathcal{G}$, then
$\mathcal{F}_n = f_{n , *}\mathcal{G}_n$ and the maps $\mathcal{F}(\varphi)$
are defined using the base change maps, see
Cohomology, Section \ref{cohomology-section-base-change-map}.
The sheaves $\mathcal{F}_n$ are quasi-coherent by
Schemes, Lemma \ref{schemes-lemma-push-forward-quasi-coherent}.
The base change maps along the degeneracies $d^n_j$ are isomorphisms
by Cohomology of Schemes, Lemma
\ref{coherent-lemma-flat-base-change-cohomology}.
Hence we are done by Lemma \ref{lemma-check-cartesian-module}.
\end{proof}

\begin{lemma}
\label{lemma-adjoint-functors-cartesian-modules}
Let $f : V \to U$ be a cartesian morphism of
simplicial schemes. Assume the morphisms $d^n_j : U_n \to U_{n - 1}$ are
flat and the morphisms $V_n \to U_n$ are quasi-compact and quasi-separated.
Then $f^*$ and $f_*$ form an adjoint pair of functors
between the categories of cartesian quasi-coherent modules on $U$ and $V$.
\end{lemma}

\begin{proof}
We have seen in Lemmas \ref{lemma-pullback-cartesian-module} and
\ref{lemma-pushforward-cartesian-module}
that the statement makes sense. The adjointness property follows
immediately from the fact that each $f_n^*$ is adjoint to $f_{n, *}$.
\end{proof}

\begin{lemma}
\label{lemma-cartesian-modules-with-section}
Let $f : X \to S$ be a morphism of schemes which has a
section\footnote{In fact, it would be enough to assume that $f$
has fpqc locally on $S$ a section, since we have descent of
quasi-coherent modules by Descent,
Section \ref{descent-section-fpqc-descent-quasi-coherent}.}.
Let $(X/S)_\bullet$ be the simplicial
scheme associated to $X \to S$, see
Definition \ref{definition-fibre-products-simplicial-scheme}.
Then pullback defines an equivalence between the category of
quasi-coherent $\mathcal{O}_S$-modules and the category of
cartesian quasi-coherent modules on $(X/S)_\bullet$.
\end{lemma}

\begin{proof}
Let $\sigma : S \to X$ be a section of $f$. Let $(\mathcal{F}, \alpha)$
be a pair as in Lemma \ref{lemma-characterize-cartesian-modules}.
Set $\mathcal{G} = \sigma^*\mathcal{F}$. Consider the diagram
$$
\xymatrix{
X \ar[r]_-{(\sigma \circ f, 1)} \ar[d]_f &
X \times_S X \ar[d]^{\text{pr}_0} \ar[r]_-{\text{pr}_1} & X \\
S \ar[r]^\sigma & X
}
$$
Note that $\text{pr}_0 = d^1_1$ and $\text{pr}_1 = d^1_0$. Hence we
see that $(\sigma \circ f, 1)^*\alpha$ defines an isomorphism
$$
f^*\mathcal{G} = (\sigma \circ f, 1)^*\text{pr}_0^*\mathcal{F}
\longrightarrow
(\sigma \circ f, 1)^*\text{pr}_1^*\mathcal{F} = \mathcal{F}
$$
We omit the verification that this isomorphism is compatible
with $\alpha$ and the canonical isomorphism
$\text{pr}_0^*f^*\mathcal{G} \to \text{pr}_1^*f^*\mathcal{G}$.
\end{proof}






\section{Groupoids and simplicial schemes}
\label{section-groupoids-simplicial}

\noindent
Given a groupoid in schemes we can build a simplicial scheme.
It will turn out that the category of quasi-coherent sheaves on a
groupoid is equivalent to the category of cartesian quasi-coherent
sheaves on the associated simplicial scheme.

\begin{lemma}
\label{lemma-groupoid-simplicial}
Let $(U, R, s, t, c, e, i)$ be a groupoid scheme over $S$.
There exists a simplicial scheme $X$ over $S$
with the following properties
\begin{enumerate}
\item $X_0 = U$, $X_1 = R$, $X_2 = R \times_{s, U, t} R$,
\item $s_0^0 = e : X_0 \to X_1$,
\item $d^1_0 = s : X_1 \to X_0$, $d^1_1 = t : X_1 \to X_0$,
\item $s_0^1 = (e \circ t, 1) : X_1 \to X_2$,
$s_1^1 = (1, e \circ t) : X_1 \to X_2$,
\item $d^2_0 = \text{pr}_1 : X_2 \to X_1$,
$d^2_1 = c : X_2 \to X_1$,
$d^2_2 = \text{pr}_0$, and
\item $X = \text{cosk}_2 \text{sk}_2 X$.
\end{enumerate}
For all $n$ we have $X_n = R \times_{s, U, t} \ldots \times_{s, U, t} R$
with $n$ factors. The map $d^n_j : X_n \to X_{n - 1}$ is given on
functors of points by
$$
(r_1, \ldots, r_n) \longmapsto (r_1, \ldots, c(r_j, r_{j + 1}), \ldots, r_n)
$$
for $1 \leq j \leq n - 1$ whereas
$d^n_0(r_1, \ldots, r_n) = (r_2, \ldots, r_n)$ and
$d^n_n(r_1, \ldots, r_n) = (r_1, \ldots, r_{n - 1})$.
\end{lemma}

\begin{proof}
We only have to verify that the rules prescribed in (1), (2), (3), (4), (5)
define a $2$-truncated simplicial scheme $U'$ over $S$, since then (6)
allows us to set $X = \text{cosk}_2 U'$, see
Simplicial, Lemma \ref{simplicial-lemma-existence-cosk}.
Using the functor of points approach, all we have to verify is that
if $(\text{Ob}, \text{Arrows}, s, t, c, e, i)$ is a groupoid, then
$$
\xymatrix{
\text{Arrows} \times_{s, \text{Ob}, t} \text{Arrows}
\ar@<8ex>[d]^{\text{pr}_0}
\ar@<0ex>[d]_c
\ar@<-8ex>[d]_{\text{pr}_1}
\\
\text{Arrows}
\ar@<4ex>[d]^t
\ar@<-4ex>[d]_s
\ar@<4ex>[u]^{1, e}
\ar@<-4ex>[u]_{e, 1}
\\
\text{Ob}
\ar@<0ex>[u]_e
}
$$
is a $2$-truncated simplicial set. We omit the details.

\medskip\noindent
Finally, the description of $X_n$ for $n > 2$ follows by induction from
the description of $X_0$, $X_1$, $X_2$, and
Simplicial, Remark \ref{simplicial-remark-inductive-coskeleton} and
Lemma \ref{simplicial-lemma-work-out}. Alternately, one shows that
$\text{cosk}_2$ applied to the $2$-truncated simplicial set displayed above
gives a simplicial set whose $n$th term equals
$\text{Arrows} \times_{s, \text{Ob}, t} \ldots \times_{s, \text{Ob}, t}
\text{Arrows}$ with $n$ factors and degeneracy maps as given in the lemma.
Some details omitted.
\end{proof}

\begin{lemma}
\label{lemma-quasi-coherent-groupoid-simplicial}
Let $S$ be a scheme. Let $(U, R, s, t, c)$ be a groupoid scheme
over $S$. Let $X$ be the simplicial scheme over $S$ constructed
in Lemma \ref{lemma-groupoid-simplicial}.
Then the category of quasi-coherent modules on $(U, R, s, t, c)$
is equivalent to the category of cartesian quasi-coherent modules
on $X$.
\end{lemma}

\begin{proof}
This is clear from Lemma \ref{lemma-characterize-cartesian-modules}
and Groupoids, Definition \ref{groupoids-definition-groupoid-module}.
\end{proof}

\noindent
In the following lemma we will use the concept of a cartesian
morphism $V \to U$ of simplicial schemes as defined in
Definition \ref{definition-cartesian-morphism}.

\begin{lemma}
\label{lemma-quasi-coherent-groupoid-R-cartesian}
Let $(U, R, s, t, c)$ be a groupoid scheme over a scheme $S$.
Let $X$ be the simplicial scheme over $S$ constructed
in Lemma \ref{lemma-groupoid-simplicial}.
Let $(R/U)_\bullet$ be the simplicial
scheme associated to $s : R \to U$, see
Definition \ref{definition-fibre-products-simplicial-scheme}.
There exists a cartesian morphism $t_\bullet : (R/U)_\bullet \to X$
of simplicial schemes with low degree morphisms given by
$$
\xymatrix{
R \times_{s, U, s} R \times_{s, U, s} R
\ar@<3ex>[r]_-{\text{pr}_{12}}
\ar@<0ex>[r]_-{\text{pr}_{02}}
\ar@<-3ex>[r]_-{\text{pr}_{01}}
\ar[dd]_{(r_0, r_1, r_2) \mapsto (r_0 \circ r_1^{-1}, r_1 \circ r_2^{-1})} &
R \times_{s, U, s} R
\ar@<1ex>[r]_-{\text{pr}_1} \ar@<-2ex>[r]_-{\text{pr}_0}
\ar[dd]_{(r_0, r_1) \mapsto r_0 \circ r_1^{-1}} &
R \ar[dd]^t
\\
\\
R \times_{s, U, t} R
\ar@<3ex>[r]_{\text{pr}_1}
\ar@<0ex>[r]_c
\ar@<-3ex>[r]_{\text{pr}_0} &
R \ar@<1ex>[r]_s \ar@<-2ex>[r]_t &
U
}
$$
\end{lemma}

\begin{proof}
For arbitrary $n$ we define $(R/U)_\bullet \to X_n$ by the rule
$$
(r_0, \ldots, r_n)
\longrightarrow
(r_0 \circ r_1^{-1}, \ldots, r_{n - 1} \circ r_n^{-1})
$$
Compatibility with degeneracy maps is clear from the description of the
degeneracies in Lemma \ref{lemma-groupoid-simplicial}.
We omit the verification that the maps respect the morphisms $s^n_j$.
Groupoids, Lemma \ref{groupoids-lemma-diagram-pull}
(with the roles of $s$ and $t$ reversed)
shows that the two right squares are cartesian. In exactly the same manner
one shows all the other squares are cartesian too. Hence
the morphism is cartesian.
\end{proof}




\section{Descent data give equivalence relations}
\label{section-equivalence-relation}

\noindent
In Section \ref{section-simplicial-descent} we saw how descent data relative to
$X \to S$ can be formulated in terms of cartesian simplicial
schemes over $(X/S)_\bullet$. Here we link this to equivalence
relations as follows.

\begin{lemma}
\label{lemma-equivalence-relation}
Let $f : X \to S$ be a morphism of schemes.
Let $\pi : Y \to (X/S)_\bullet$ be a cartesian morphism of simplicial
schemes, see Definitions \ref{definition-cartesian-morphism} and
\ref{definition-fibre-products-simplicial-scheme}.
Then the morphism
$$
j = (d^1_1, d^1_0) : Y_1 \to Y_0 \times_S Y_0
$$
defines an equivalence relation on $Y_0$ over $S$,
see Groupoids, Definition \ref{groupoids-definition-equivalence-relation}.
\end{lemma}

\begin{proof}
Note that $j$ is a monomorphism. Namely the
composition $Y_1 \to Y_0 \times_S Y_0 \to Y_0 \times_S X$
is an isomorphism as $\pi$ is cartesian.

\medskip\noindent
Consider the morphism
$$
(d^2_2, d^2_0) : Y_2 \to Y_1 \times_{d^1_0, Y_0, d^1_1} Y_1.
$$
This works because $d_0 \circ d_2 = d_1 \circ d_0$,
see Simplicial, Remark \ref{simplicial-remark-relations}.
Also, it is a morphism over $(X/S)_2$. It is an isomorphism
because $Y \to (X/S)_\bullet$ is cartesian. Note for example that the
right hand side is isomorphic to
$Y_0 \times_{\pi_0, X, \text{pr}_1} (X \times_S X \times_S X) =
X \times_S Y_0 \times_S X$
because $\pi$ is cartesian. Details omitted.

\medskip\noindent
As in Groupoids, Definition \ref{groupoids-definition-equivalence-relation}
we denote $t = \text{pr}_0 \circ j = d^1_1$ and
$s = \text{pr}_1 \circ j = d^1_0$.
The isomorphism above, combined with the morphism
$d^2_1 : Y_2 \to Y_1$ give us a composition morphism
$$
c : Y_1 \times_{s, Y_0, t} Y_1 \longrightarrow Y_1
$$
over $Y_0 \times_S Y_0$. This immediately implies
that for any scheme $T/S$ the relation
$Y_1(T) \subset Y_0(T) \times Y_0(T)$ is transitive.

\medskip\noindent
Reflexivity follows from the fact that the
restriction of the morphism $j$ to the diagonal
$\Delta : X \to X \times_S X$ is an isomorphism
(again use the cartesian property of $\pi$).

\medskip\noindent
To see symmetry we consider the morphism
$$
(d^2_2, d^2_1) : Y_2 \to Y_1 \times_{d^1_1, Y_0, d^1_1} Y_1.
$$
This works because $d_1 \circ d_2 = d_1 \circ d_1$,
see Simplicial, Remark \ref{simplicial-remark-relations}.
It is an isomorphism
because $Y \to (X/S)_\bullet$ is cartesian.
Note for example that the
right hand side is isomorphic to
$Y_0 \times_{\pi_0, X, \text{pr}_0} (X \times_S X \times_S X) =
Y_0 \times_S X \times_S X$
because $\pi$ is cartesian. Details omitted.

\medskip\noindent
Let $T/S$ be a scheme. Let $a \sim b$ for $a, b \in Y_0(T)$
be synonymous with $(a, b) \in Y_1(T)$.
The isomorphism $(d^2_2, d^2_1)$ above
implies that if $a \sim b$ and $a \sim c$, then $b \sim c$.
Combined with reflexivity this shows that $\sim$ is
an equivalence relation.
\end{proof}







\section{An example case}
\label{section-example}

\noindent
In this section we show that disjoint unions of spectra
of Artinian rings can be descended along a quasi-compact
surjective flat morphism of schemes.

\begin{lemma}
\label{lemma-equivalence-classes-points}
Let $X \to S$ be a morphism of schemes. Suppose $Y \to (X/S)_\bullet$
is a cartesian morphism of simplicial schemes. For $y \in Y_0$ a point define
$$
T_y = \{y' \in Y_0 \mid \exists\ y_1 \in Y_1:
d^1_1(y_1) = y, d^1_0(y_1) = y'\}
$$
as a subset of $Y_0$. Then $y \in T_y$ and
$T_y \cap T_{y'} \not = \emptyset \Rightarrow T_y = T_{y'}$.
\end{lemma}

\begin{proof}
Combine Lemma \ref{lemma-equivalence-relation} and
Groupoids, Lemma
\ref{groupoids-lemma-pre-equivalence-equivalence-relation-points}.
\end{proof}

\begin{lemma}
\label{lemma-quasi-compact}
Let $X \to S$ be a morphism of schemes.
Suppose $Y \to (X/S)_\bullet$ is a cartesian morphism of simplicial schemes.
Let $y \in Y_0$ be a point. If $X \to S$ is quasi-compact, then
$$
T_y = \{y' \in Y_0 \mid \exists\ y_1 \in Y_1:
d^1_1(y_1) = y, d^1_0(y_1) = y'\}
$$
is a quasi-compact subset of $Y_0$.
\end{lemma}

\begin{proof}
Let $F_y$ be the scheme theoretic fibre of $d^1_1 : Y_1 \to Y_0$
at $y$. Then we see that $T_y$ is the image of the morphism
$$
\xymatrix{
F_y \ar[r] \ar[d] &
Y_1 \ar[r]^{d^1_0} \ar[d]^{d^1_1} &
Y_0 \\
y \ar[r] &
Y_0 &
}
$$
Note that $F_y$ is quasi-compact. This proves the lemma.
\end{proof}

\begin{lemma}
\label{lemma-descent-disjoint-union-Artinian-along-fields}
Let $X \to S$ be a quasi-compact flat surjective morphism.
Let $(V, \varphi)$ be a descent datum relative
to $X \to S$. If $V$ is a disjoint union of
spectra of Artinian rings, then $(V, \varphi)$ is effective.
\end{lemma}

\begin{proof}
Let $Y \to (X/S)_\bullet$ be the cartesian morphism of simplicial
schemes corresponding to $(V, \varphi)$ by
Lemma \ref{lemma-cartesian-equivalent-descent-datum}.
Observe that $Y_0 = V$.
Write $V = \coprod_{i \in I} \Spec(A_i)$ with each $A_i$ local
Artinian. Moreover, let $v_i \in V$ be the unique closed point of
$\Spec(A_i)$ for all $i \in I$. Write $i \sim j$ if and only if
$v_i \in T_{v_j}$ with notation as in
Lemma \ref{lemma-equivalence-classes-points} above.
By Lemmas \ref{lemma-equivalence-classes-points} and \ref{lemma-quasi-compact}
this is an equivalence relation with finite equivalence
classes. Let $\overline{I} = I/\sim$. Then we can write
$V = \coprod_{\overline{i} \in \overline{I}} V_{\overline{i}}$
with
$V_{\overline{i}} = \coprod_{i \in \overline{i}} \Spec(A_i)$.
By construction we see that
$\varphi : V \times_S X \to X \times_S V$ maps
the open and closed subspaces $V_{\overline{i}} \times_S X$
into the open and closed subspaces $X \times_S V_{\overline{i}}$.
In other words, we get descent data
$(V_{\overline{i}}, \varphi_{\overline{i}})$, and
$(V, \varphi)$ is the coproduct of them in the category of
descent data.
Since each of the $V_{\overline{i}}$ is a finite union of
spectra of Artinian local rings the morphism $V_{\overline{i}} \to X$
is affine, see Morphisms, Lemma \ref{morphisms-lemma-Artinian-affine}.
Since $\{X \to S\}$ is an fpqc covering we see that all
the descent data $(V_{\overline{i}}, \varphi_{\overline{i}})$ are effective
by Descent, Lemma \ref{descent-lemma-affine}.
\end{proof}

\noindent
To be sure, the lemma above has very limited applicability!





\begin{multicols}{2}[\section{Other chapters}]
\noindent
Preliminaries
\begin{enumerate}
\item \hyperref[introduction-section-phantom]{Introduction}
\item \hyperref[conventions-section-phantom]{Conventions}
\item \hyperref[sets-section-phantom]{Set Theory}
\item \hyperref[categories-section-phantom]{Categories}
\item \hyperref[topology-section-phantom]{Topology}
\item \hyperref[sheaves-section-phantom]{Sheaves on Spaces}
\item \hyperref[sites-section-phantom]{Sites and Sheaves}
\item \hyperref[stacks-section-phantom]{Stacks}
\item \hyperref[fields-section-phantom]{Fields}
\item \hyperref[algebra-section-phantom]{Commutative Algebra}
\item \hyperref[brauer-section-phantom]{Brauer Groups}
\item \hyperref[homology-section-phantom]{Homological Algebra}
\item \hyperref[derived-section-phantom]{Derived Categories}
\item \hyperref[simplicial-section-phantom]{Simplicial Methods}
\item \hyperref[more-algebra-section-phantom]{More on Algebra}
\item \hyperref[smoothing-section-phantom]{Smoothing Ring Maps}
\item \hyperref[modules-section-phantom]{Sheaves of Modules}
\item \hyperref[sites-modules-section-phantom]{Modules on Sites}
\item \hyperref[injectives-section-phantom]{Injectives}
\item \hyperref[cohomology-section-phantom]{Cohomology of Sheaves}
\item \hyperref[sites-cohomology-section-phantom]{Cohomology on Sites}
\item \hyperref[dga-section-phantom]{Differential Graded Algebra}
\item \hyperref[dpa-section-phantom]{Divided Power Algebra}
\item \hyperref[hypercovering-section-phantom]{Hypercoverings}
\end{enumerate}
Schemes
\begin{enumerate}
\setcounter{enumi}{24}
\item \hyperref[schemes-section-phantom]{Schemes}
\item \hyperref[constructions-section-phantom]{Constructions of Schemes}
\item \hyperref[properties-section-phantom]{Properties of Schemes}
\item \hyperref[morphisms-section-phantom]{Morphisms of Schemes}
\item \hyperref[coherent-section-phantom]{Cohomology of Schemes}
\item \hyperref[divisors-section-phantom]{Divisors}
\item \hyperref[limits-section-phantom]{Limits of Schemes}
\item \hyperref[varieties-section-phantom]{Varieties}
\item \hyperref[topologies-section-phantom]{Topologies on Schemes}
\item \hyperref[descent-section-phantom]{Descent}
\item \hyperref[perfect-section-phantom]{Derived Categories of Schemes}
\item \hyperref[more-morphisms-section-phantom]{More on Morphisms}
\item \hyperref[flat-section-phantom]{More on Flatness}
\item \hyperref[groupoids-section-phantom]{Groupoid Schemes}
\item \hyperref[more-groupoids-section-phantom]{More on Groupoid Schemes}
\item \hyperref[etale-section-phantom]{\'Etale Morphisms of Schemes}
\end{enumerate}
Topics in Scheme Theory
\begin{enumerate}
\setcounter{enumi}{40}
\item \hyperref[chow-section-phantom]{Chow Homology}
\item \hyperref[intersection-section-phantom]{Intersection Theory}
\item \hyperref[weil-section-phantom]{Weil Cohomology Theories}
\item \hyperref[pic-section-phantom]{Picard Schemes of Curves}
\item \hyperref[adequate-section-phantom]{Adequate Modules}
\item \hyperref[dualizing-section-phantom]{Dualizing Complexes}
\item \hyperref[duality-section-phantom]{Duality for Schemes}
\item \hyperref[discriminant-section-phantom]{Discriminants and Differents}
\item \hyperref[local-cohomology-section-phantom]{Local Cohomology}
\item \hyperref[algebraization-section-phantom]{Algebraic and Formal Geometry}
\item \hyperref[curves-section-phantom]{Algebraic Curves}
\item \hyperref[resolve-section-phantom]{Resolution of Surfaces}
\item \hyperref[models-section-phantom]{Semistable Reduction}
\item \hyperref[pione-section-phantom]{Fundamental Groups of Schemes}
\item \hyperref[etale-cohomology-section-phantom]{\'Etale Cohomology}
\item \hyperref[crystalline-section-phantom]{Crystalline Cohomology}
\item \hyperref[proetale-section-phantom]{Pro-\'etale Cohomology}
\item \hyperref[more-etale-section-phantom]{More \'Etale Cohomology}
\item \hyperref[trace-section-phantom]{The Trace Formula}
\end{enumerate}
Algebraic Spaces
\begin{enumerate}
\setcounter{enumi}{59}
\item \hyperref[spaces-section-phantom]{Algebraic Spaces}
\item \hyperref[spaces-properties-section-phantom]{Properties of Algebraic Spaces}
\item \hyperref[spaces-morphisms-section-phantom]{Morphisms of Algebraic Spaces}
\item \hyperref[decent-spaces-section-phantom]{Decent Algebraic Spaces}
\item \hyperref[spaces-cohomology-section-phantom]{Cohomology of Algebraic Spaces}
\item \hyperref[spaces-limits-section-phantom]{Limits of Algebraic Spaces}
\item \hyperref[spaces-divisors-section-phantom]{Divisors on Algebraic Spaces}
\item \hyperref[spaces-over-fields-section-phantom]{Algebraic Spaces over Fields}
\item \hyperref[spaces-topologies-section-phantom]{Topologies on Algebraic Spaces}
\item \hyperref[spaces-descent-section-phantom]{Descent and Algebraic Spaces}
\item \hyperref[spaces-perfect-section-phantom]{Derived Categories of Spaces}
\item \hyperref[spaces-more-morphisms-section-phantom]{More on Morphisms of Spaces}
\item \hyperref[spaces-flat-section-phantom]{Flatness on Algebraic Spaces}
\item \hyperref[spaces-groupoids-section-phantom]{Groupoids in Algebraic Spaces}
\item \hyperref[spaces-more-groupoids-section-phantom]{More on Groupoids in Spaces}
\item \hyperref[bootstrap-section-phantom]{Bootstrap}
\item \hyperref[spaces-pushouts-section-phantom]{Pushouts of Algebraic Spaces}
\end{enumerate}
Topics in Geometry
\begin{enumerate}
\setcounter{enumi}{76}
\item \hyperref[spaces-chow-section-phantom]{Chow Groups of Spaces}
\item \hyperref[groupoids-quotients-section-phantom]{Quotients of Groupoids}
\item \hyperref[spaces-more-cohomology-section-phantom]{More on Cohomology of Spaces}
\item \hyperref[spaces-simplicial-section-phantom]{Simplicial Spaces}
\item \hyperref[spaces-duality-section-phantom]{Duality for Spaces}
\item \hyperref[formal-spaces-section-phantom]{Formal Algebraic Spaces}
\item \hyperref[restricted-section-phantom]{Restricted Power Series}
\item \hyperref[spaces-resolve-section-phantom]{Resolution of Surfaces Revisited}
\end{enumerate}
Deformation Theory
\begin{enumerate}
\setcounter{enumi}{84}
\item \hyperref[formal-defos-section-phantom]{Formal Deformation Theory}
\item \hyperref[defos-section-phantom]{Deformation Theory}
\item \hyperref[cotangent-section-phantom]{The Cotangent Complex}
\item \hyperref[examples-defos-section-phantom]{Deformation Problems}
\end{enumerate}
Algebraic Stacks
\begin{enumerate}
\setcounter{enumi}{88}
\item \hyperref[algebraic-section-phantom]{Algebraic Stacks}
\item \hyperref[examples-stacks-section-phantom]{Examples of Stacks}
\item \hyperref[stacks-sheaves-section-phantom]{Sheaves on Algebraic Stacks}
\item \hyperref[criteria-section-phantom]{Criteria for Representability}
\item \hyperref[artin-section-phantom]{Artin's Axioms}
\item \hyperref[quot-section-phantom]{Quot and Hilbert Spaces}
\item \hyperref[stacks-properties-section-phantom]{Properties of Algebraic Stacks}
\item \hyperref[stacks-morphisms-section-phantom]{Morphisms of Algebraic Stacks}
\item \hyperref[stacks-limits-section-phantom]{Limits of Algebraic Stacks}
\item \hyperref[stacks-cohomology-section-phantom]{Cohomology of Algebraic Stacks}
\item \hyperref[stacks-perfect-section-phantom]{Derived Categories of Stacks}
\item \hyperref[stacks-introduction-section-phantom]{Introducing Algebraic Stacks}
\item \hyperref[stacks-more-morphisms-section-phantom]{More on Morphisms of Stacks}
\item \hyperref[stacks-geometry-section-phantom]{The Geometry of Stacks}
\end{enumerate}
Topics in Moduli Theory
\begin{enumerate}
\setcounter{enumi}{102}
\item \hyperref[moduli-section-phantom]{Moduli Stacks}
\item \hyperref[moduli-curves-section-phantom]{Moduli of Curves}
\end{enumerate}
Miscellany
\begin{enumerate}
\setcounter{enumi}{104}
\item \hyperref[examples-section-phantom]{Examples}
\item \hyperref[exercises-section-phantom]{Exercises}
\item \hyperref[guide-section-phantom]{Guide to Literature}
\item \hyperref[desirables-section-phantom]{Desirables}
\item \hyperref[coding-section-phantom]{Coding Style}
\item \hyperref[obsolete-section-phantom]{Obsolete}
\item \hyperref[fdl-section-phantom]{GNU Free Documentation License}
\item \hyperref[index-section-phantom]{Auto Generated Index}
\end{enumerate}
\end{multicols}


\bibliography{my}
\bibliographystyle{amsalpha}

\end{document}
