\IfFileExists{stacks-project.cls}{%
\documentclass{stacks-project}
}{%
\documentclass{amsart}
}

% The following AMS packages are automatically loaded with
% the amsart documentclass:
%\usepackage{amsmath}
%\usepackage{amssymb}
%\usepackage{amsthm}

% For dealing with references we use the comment environment
\usepackage{verbatim}
\newenvironment{reference}{\comment}{\endcomment}
%\newenvironment{reference}{}{}
\newenvironment{slogan}{\comment}{\endcomment}
\newenvironment{history}{\comment}{\endcomment}

% For commutative diagrams you can use
% \usepackage{amscd}
\usepackage[all]{xy}

% We use 2cell for 2-commutative diagrams.
\xyoption{2cell}
\UseAllTwocells

% To put source file link in headers.
% Change "template.tex" to "this_filename.tex"
% \usepackage{fancyhdr}
% \pagestyle{fancy}
% \lhead{}
% \chead{}
% \rhead{Source file: \url{template.tex}}
% \lfoot{}
% \cfoot{\thepage}
% \rfoot{}
% \renewcommand{\headrulewidth}{0pt}
% \renewcommand{\footrulewidth}{0pt}
% \renewcommand{\headheight}{12pt}

\usepackage{multicol}

% For cross-file-references
\usepackage{xr-hyper}

% Package for hypertext links:
\usepackage{hyperref}

% For any local file, say "hello.tex" you want to link to please
% use \externaldocument[hello-]{hello}
\externaldocument[introduction-]{introduction}
\externaldocument[conventions-]{conventions}
\externaldocument[sets-]{sets}
\externaldocument[categories-]{categories}
\externaldocument[topology-]{topology}
\externaldocument[sheaves-]{sheaves}
\externaldocument[sites-]{sites}
\externaldocument[stacks-]{stacks}
\externaldocument[fields-]{fields}
\externaldocument[algebra-]{algebra}
\externaldocument[brauer-]{brauer}
\externaldocument[homology-]{homology}
\externaldocument[derived-]{derived}
\externaldocument[simplicial-]{simplicial}
\externaldocument[more-algebra-]{more-algebra}
\externaldocument[smoothing-]{smoothing}
\externaldocument[modules-]{modules}
\externaldocument[sites-modules-]{sites-modules}
\externaldocument[injectives-]{injectives}
\externaldocument[cohomology-]{cohomology}
\externaldocument[sites-cohomology-]{sites-cohomology}
\externaldocument[dga-]{dga}
\externaldocument[dpa-]{dpa}
\externaldocument[hypercovering-]{hypercovering}
\externaldocument[schemes-]{schemes}
\externaldocument[constructions-]{constructions}
\externaldocument[properties-]{properties}
\externaldocument[morphisms-]{morphisms}
\externaldocument[coherent-]{coherent}
\externaldocument[divisors-]{divisors}
\externaldocument[limits-]{limits}
\externaldocument[varieties-]{varieties}
\externaldocument[topologies-]{topologies}
\externaldocument[descent-]{descent}
\externaldocument[perfect-]{perfect}
\externaldocument[more-morphisms-]{more-morphisms}
\externaldocument[flat-]{flat}
\externaldocument[groupoids-]{groupoids}
\externaldocument[more-groupoids-]{more-groupoids}
\externaldocument[etale-]{etale}
\externaldocument[chow-]{chow}
\externaldocument[intersection-]{intersection}
\externaldocument[pic-]{pic}
\externaldocument[adequate-]{adequate}
\externaldocument[dualizing-]{dualizing}
\externaldocument[duality-]{duality}
\externaldocument[discriminant-]{discriminant}
\externaldocument[local-cohomology-]{local-cohomology}
\externaldocument[curves-]{curves}
\externaldocument[resolve-]{resolve}
\externaldocument[models-]{models}
\externaldocument[pione-]{pione}
\externaldocument[etale-cohomology-]{etale-cohomology}
\externaldocument[proetale-]{proetale}
\externaldocument[crystalline-]{crystalline}
\externaldocument[spaces-]{spaces}
\externaldocument[spaces-properties-]{spaces-properties}
\externaldocument[spaces-morphisms-]{spaces-morphisms}
\externaldocument[decent-spaces-]{decent-spaces}
\externaldocument[spaces-cohomology-]{spaces-cohomology}
\externaldocument[spaces-limits-]{spaces-limits}
\externaldocument[spaces-divisors-]{spaces-divisors}
\externaldocument[spaces-over-fields-]{spaces-over-fields}
\externaldocument[spaces-topologies-]{spaces-topologies}
\externaldocument[spaces-descent-]{spaces-descent}
\externaldocument[spaces-perfect-]{spaces-perfect}
\externaldocument[spaces-more-morphisms-]{spaces-more-morphisms}
\externaldocument[spaces-flat-]{spaces-flat}
\externaldocument[spaces-groupoids-]{spaces-groupoids}
\externaldocument[spaces-more-groupoids-]{spaces-more-groupoids}
\externaldocument[bootstrap-]{bootstrap}
\externaldocument[spaces-pushouts-]{spaces-pushouts}
\externaldocument[groupoids-quotients-]{groupoids-quotients}
\externaldocument[spaces-more-cohomology-]{spaces-more-cohomology}
\externaldocument[spaces-simplicial-]{spaces-simplicial}
\externaldocument[formal-spaces-]{formal-spaces}
\externaldocument[restricted-]{restricted}
\externaldocument[spaces-resolve-]{spaces-resolve}
\externaldocument[formal-defos-]{formal-defos}
\externaldocument[defos-]{defos}
\externaldocument[cotangent-]{cotangent}
\externaldocument[examples-defos-]{examples-defos}
\externaldocument[algebraic-]{algebraic}
\externaldocument[examples-stacks-]{examples-stacks}
\externaldocument[stacks-sheaves-]{stacks-sheaves}
\externaldocument[criteria-]{criteria}
\externaldocument[artin-]{artin}
\externaldocument[quot-]{quot}
\externaldocument[stacks-properties-]{stacks-properties}
\externaldocument[stacks-morphisms-]{stacks-morphisms}
\externaldocument[stacks-limits-]{stacks-limits}
\externaldocument[stacks-cohomology-]{stacks-cohomology}
\externaldocument[stacks-perfect-]{stacks-perfect}
\externaldocument[stacks-introduction-]{stacks-introduction}
\externaldocument[stacks-more-morphisms-]{stacks-more-morphisms}
\externaldocument[stacks-geometry-]{stacks-geometry}
\externaldocument[moduli-]{moduli}
\externaldocument[moduli-curves-]{moduli-curves}
\externaldocument[examples-]{examples}
\externaldocument[exercises-]{exercises}
\externaldocument[guide-]{guide}
\externaldocument[desirables-]{desirables}
\externaldocument[coding-]{coding}
\externaldocument[obsolete-]{obsolete}
\externaldocument[fdl-]{fdl}
\externaldocument[index-]{index}

% Theorem environments.
%
\theoremstyle{plain}
\newtheorem{theorem}[subsection]{Theorem}
\newtheorem{proposition}[subsection]{Proposition}
\newtheorem{lemma}[subsection]{Lemma}

\theoremstyle{definition}
\newtheorem{definition}[subsection]{Definition}
\newtheorem{example}[subsection]{Example}
\newtheorem{exercise}[subsection]{Exercise}
\newtheorem{situation}[subsection]{Situation}

\theoremstyle{remark}
\newtheorem{remark}[subsection]{Remark}
\newtheorem{remarks}[subsection]{Remarks}

\numberwithin{equation}{subsection}

% Macros
%
\def\lim{\mathop{\rm lim}\nolimits}
\def\colim{\mathop{\rm colim}\nolimits}
\def\Spec{\mathop{\rm Spec}}
\def\Hom{\mathop{\rm Hom}\nolimits}
\def\Ext{\mathop{\rm Ext}\nolimits}
\def\SheafHom{\mathop{\mathcal{H}\!{\it om}}\nolimits}
\def\SheafExt{\mathop{\mathcal{E}\!{\it xt}}\nolimits}
\def\Sch{\textit{Sch}}
\def\Mor{\mathop{\rm Mor}\nolimits}
\def\Ob{\mathop{\rm Ob}\nolimits}
\def\Sh{\mathop{\textit{Sh}}\nolimits}
\def\NL{\mathop{N\!L}\nolimits}
\def\proetale{{pro\text{-}\acute{e}tale}}
\def\etale{{\acute{e}tale}}
\def\QCoh{\textit{QCoh}}
\def\Ker{\mathop{\rm Ker}}
\def\Im{\mathop{\rm Im}}
\def\Coker{\mathop{\rm Coker}}
\def\Coim{\mathop{\rm Coim}}

%
% Macros for moduli stacks/spaces
%
\def\QCohstack{\mathcal{QC}\!{\it oh}}
\def\Cohstack{\mathcal{C}\!{\it oh}}
\def\Spacesstack{\mathcal{S}\!{\it paces}}
\def\Quotfunctor{{\rm Quot}}
\def\Hilbfunctor{{\rm Hilb}}
\def\Curvesstack{\mathcal{C}\!{\it urves}}
\def\Polarizedstack{\mathcal{P}\!{\it olarized}}
\def\Complexesstack{\mathcal{C}\!{\it omplexes}}
% \Pic is the operator that assigns to X its picard group, usage \Pic(X)
% \Picardstack_{X/B} denotes the Picard stack of X over B
% \Picardfunctor_{X/B} denotes the Picard functor of X over B
\def\Pic{\mathop{\rm Pic}\nolimits}
\def\Picardstack{\mathcal{P}\!{\it ic}}
\def\Picardfunctor{{\rm Pic}}
\def\Deformationcategory{\mathcal{D}\!{\it ef}}


% OK, start here.
%
\begin{document}

\title{Varieties}


\maketitle

\phantomsection
\label{section-phantom}

\tableofcontents

\section{Introduction}
\label{section-introduction}

\noindent
In this chapter we start studying varieties and more generally
schemes over a field. A fundamental reference is \cite{EGA}.








\section{Notation}
\label{section-notation}

\noindent
Throughout this chapter we use the letter $k$ to denote the ground field.










\section{Varieties}
\label{section-varieties}

\begin{definition}
\label{definition-variety}
Let $k$ be a field. A {\it variety} is a scheme $X$ over $k$
such that $X$ is integral and the structure morphism
$X \to \text{Spec}(k)$ is separated and of finite type.
\end{definition}

\noindent
This definition has the following drawback. Suppose that
$k \subset k'$ is an extension of fields. Suppose that $X$
is a variety over $k$. Then the base change
$X_{k'} = X \times_{\text{Spec}(k)} \text{Spec}(k')$ is
not necessarily a variety over $k'$. This will be discussed
in some detail in the following sections.

\medskip\noindent
In fact, even the product of two varieties need not be a variety
(this is really the same phenomenon). Here is an example.

\begin{example}
\label{example-product-not-a-variety}
Let $k = \mathbf{Q}$. Let $X = \text{Spec}(\mathbf{Q}(i))$
and $Y = \text{Spec}(\mathbf{Q}(i))$. Then the product
$X \times_{\text{Spec}(k)} Y$ of the varieties $X$ and $Y$
is not a variety, since it is reducible. (It is isomorphic
to the disjoint union of two copies of $X$.)
\end{example}







\section{Geometrically reduced schemes}
\label{section-geometrically-reduced}

\noindent
If $X$ is a reduced scheme over a field, then it can happen that $X$
becomes nonreduced after extending the ground field. This does not happen
for geometrically reduced schemes.

\begin{definition}
\label{definition-geometrically-reduced}
Let $k$ be a field.
Let $X$ be a scheme over $k$.
Let $x \in X$ be a point.
\begin{enumerate}
\item Let $x \in X$ be a point.
We say $X$ is {\it geometrically reduced at $x$}
if for any field extension $k \subset k'$
and any point $x' \in X_{k'}$ lying over $x$
the local ring $\mathcal{O}_{X_{k'}, x'}$ is reduced.
\item We say $X$ is {\it geometrically reduced} over $k$
if $X$ is geometrically reduced at every point of $X$.
\end{enumerate}
\end{definition}

\noindent
This may seem a little mysterious at first, but it is
really the same thing as the notion discussed in the algebra chapter.
Here are some basic results explaining the connection.

\begin{lemma}
\label{lemma-geometrically-reduced-at-point}
Let $k$ be a field.
Let $X$ be a scheme over $k$.
Let $x \in X$.
The following are equivalent
\begin{enumerate}
\item $X$ is geometrically reduced at $x$, and
\item the ring $\mathcal{O}_{X, x}$ is geometrically
reduced over $k$ (see
Algebra, Definition \ref{algebra-definition-geometrically-reduced}).
\end{enumerate}
\end{lemma}

\begin{proof}
Assume (1). This in particular implies that $\mathcal{O}_{X, x}$
is reduced. Let $k \subset k'$ be a finite purely inseparable field
extension. Consider the ring $\mathcal{O}_{X, x} \otimes_k k'$.
By Algebra, Lemma \ref{algebra-lemma-p-ring-map}
its spectrum is the same as the spectrum of $\mathcal{O}_{X, x}$.
Hence it is a local ring also
(Algebra, Lemma \ref{algebra-lemma-characterize-local-ring}).
Therefore there is a unique point $x' \in X_{k'}$ lying over $x$
and $\mathcal{O}_{X_{k'}, x'} \cong \mathcal{O}_{X, x} \otimes_k k'$.
By assumption this is a reduced ring. Hence we deduce (2) by
Algebra, Lemma
\ref{algebra-lemma-geometrically-reduced-finite-purely-inseparable-extension}.

\medskip\noindent
Assume (2). Let $k \subset k'$ be a field extension. Since
$\text{Spec}(k') \to \text{Spec}(k)$ is surjective, also
$X_{k'} \to X$ is surjective
(Morphisms, Lemma \ref{morphisms-lemma-base-change-surjective}).
Let $x' \in X_{k'}$ be any point lying over $x$.
The local ring $\mathcal{O}_{X_{k'}, x'}$
is a localization of the ring $\mathcal{O}_{X, x} \otimes_k k'$.
Hence it is reduced by assumption and (1) is proved.
\end{proof}

\begin{lemma}
\label{lemma-geometrically-reduced}
Let $k$ be a field.
Let $X$ be a scheme over $k$.
The following are equivalent
\begin{enumerate}
\item $X$ is geometrically reduced,
\item for every field extension $k \subset k'$ the scheme $X_{k'}$
is reduced, and
\item for every affine open $U \subset X$ the ring $\mathcal{O}_X(U)$
is geometrically reduced (see
Algebra, Definition \ref{algebra-definition-geometrically-reduced}).
\end{enumerate}
\end{lemma}

\begin{proof}
Assume (1). Then for every field extension $k \subset k'$ and
every point $x' \in X_{k'}$ the local ring of $X_{k'}$ at $x'$
is reduced. In other words $X_{k'}$ is reduced. Hence (2).

\medskip\noindent
Assume (2). Let $U \subset X$ be an affine open. Then for
every field extension $k \subset k'$ the scheme $X_{k'}$ is reduced, hence
$U_{k'} = \text{Spec}(\mathcal{O}(U)\otimes_k k')$ is reduced,
hence $\mathcal{O}(U)\otimes_k k'$ is reduced (see Properties,
Section \ref{properties-section-integral}). In other words
$\mathcal{O}(U)$ is geometrically reduced, so (3) holds.

\medskip\noindent
Assume (3). For any field extension $k \subset k'$ the base
change $X_{k'}$ is gotten by gluing the spectra of the
rings $\mathcal{O}_X(U) \otimes_k k'$ where $U$ is affine open
in $X$ (see Schemes, Section \ref{schemes-section-fibre-products}).
Hence $X_{k'}$ is reduced. So (1) holds.
\end{proof}

\begin{lemma}
\label{lemma-check-only-finite-inseparable-extensions}
Let $k$ be a field. Let $X$ be a scheme over $k$.
\begin{enumerate}
\item Let $x \in X$, and assume $\mathcal{O}_{X, x}$ reduced.
To check that $X$ is geometrically reduced at $x$ it suffices to check
$\mathcal{O}_{X_{k'}, x'}$ is reduced for every
finite purely inseparable field extension $k'$ of $k$ and
$x' \in X_{k'}$ over $x$.
\item Assume $X$ reduced.
To see that $X$ is geometrically reduced it suffices to check
$X_{k'}$ is reduced for every finite purely inseparable field
extension $k'/k$.
\end{enumerate}
\end{lemma}

\begin{proof}
This follows from Lemmas \ref{lemma-geometrically-reduced-at-point}
and \ref{lemma-geometrically-reduced} above and Algebra, Lemma
\ref{algebra-lemma-geometrically-reduced-finite-purely-inseparable-extension}.
Details omitted.
\end{proof}

\begin{lemma}
\label{lemma-perfect-reduced}
Let $X$ be a scheme over a perfect field $k$.
If $X$ is reduced, then $X$ is geometrically reduced over $k$.
\end{lemma}

\begin{proof}
Follows immediately from
Lemma \ref{lemma-check-only-finite-inseparable-extensions}
and the definition of a perfect field
(Algebra, Definition \ref{algebra-definition-perfect}).
\end{proof}

\begin{lemma}
\label{lemma-geometrically-reduced-upstairs}
Let $k$ be a field.
Let $X$ be a scheme over $k$.
Let $k'/k$ be a field extension.
Let $x \in X$ be a point, and let $x' \in X_{k'}$ be a point lying over $x$.
The following are equivalent
\begin{enumerate}
\item $X$ is geometrically reduced at $x$,
\item $X_{k'}$ is geometrically reduced at $x'$.
\end{enumerate}
In particular, $X$ is geometrically reduced over $k$ if and only if
$X_{k'}$ is geometrically reduced over $k'$.
\end{lemma}

\begin{proof}
It is clear that (1) implies (2). Assume (2).
Let $k \subset k''$ be a finite purely inseparable field extension
and let $x'' \in X_{k''}$ be a point lying over $x$ (actually it is
unique). We can find a common field extension $k \subset k'''$
(i.e.\ with both $k' \subset k'''$ and $k'' \subset k'''$) and a point
$x''' \in X_{k'''}$ lying over both $x'$ and $x''$.
Consider the map of local rings
$$
\mathcal{O}_{X_{k''}, x''} \longrightarrow \mathcal{O}_{X_{k'''}, x''''}.
$$
This is a flat local ring homomorphism and hence faithfully flat.
By (2) we see that the local ring on the right is reduced.
Thus by Algebra, Lemma \ref{algebra-lemma-descent-reduced}
we conlude that $\mathcal{O}_{X_{k''}, x''}$ is reduced.
Thus by Lemma \ref{lemma-check-only-finite-inseparable-extensions}
we conclude that $X$ is geometrically reduced at $x$.
\end{proof}

\begin{lemma}
\label{lemma-geometrically-reduced-any-base-change}
Let $k$ be a field.
Let $X$, $Y$ be schemes over $k$.
\begin{enumerate}
\item If $X$ is geometrically reduced at $x$, and $Y$ reduced,
then $X \times_k Y$ is reduced at every point lying over $x$.
\item If $X$ geometrically reduced over $k$ and $Y$ reduced.
Then $X \times_k Y$ is reduced.
\end{enumerate}
\end{lemma}

\begin{proof}
Combine, Lemmas \ref{lemma-geometrically-reduced-at-point}
and \ref{lemma-geometrically-reduced} and Algebra,
Lemma \ref{algebra-lemma-geometrically-reduced-any-reduced-base-change}.
\end{proof}

\begin{lemma}
\label{lemma-generic-points-geometrically-reduced}
Let $k$ be a field.
Let $X$ be a scheme over $k$.
\begin{enumerate}
\item If $x' \leadsto x$ is a specialization and $X$ is geometrically
reduced at $x$, then $X$ is geometrically reduced at $x'$.
\item If $x \in X$ such that (a) $\mathcal{O}_{X, x}$
is reduced, and (b) for each specialization $x' \leadsto x$ where
$x'$ is a generic point of an irreducible component of $X$ the
scheme $X$ is geometrically reduced at $x'$, then $X$ is geometrically
reduced at $x$.
\item If $X$ is reduced and geometrically reduced at all generic
points of irreducible components of $X$, then $X$ is geometrically
reduced.
\end{enumerate}
\end{lemma}

\begin{proof}
Part (1) follows from
Lemma \ref{lemma-geometrically-reduced-at-point}
and the fact that if $A$ is a geometrically reduced
$k$-algebra, then $S^{-1}A$ is a geometrically reduced $k$-algebra for
any multiplicative subset $S$ of $A$, see
Algebra, Lemma \ref{algebra-lemma-geometrically-reduced-permanence}.

\medskip\noindent
Let $A = \mathcal{O}_{X, x}$. The assumptions (a) and (b) of (2) imply
that $A$ is reduced, and that $A_{\mathfrak q}$ is geometrically
reduced over $k$ for every minimal prime $\mathfrak q$ of $A$.
Here we have used
Lemma \ref{lemma-geometrically-reduced-at-point}.
By the same token, if we show $A \otimes_k k'$ is reduced
for every finite purely inseparable extension $k \subset k'$,
then (2) holds (use
Algebra, Lemma
\ref{algebra-lemma-geometrically-reduced-finite-purely-inseparable-extension}).
Since $A$ is reduced the map
$A \to \prod_{\mathfrak q\text{ minial}} A_{\mathfrak q}$
is injective, see
Algebra, Lemma \ref{algebra-lemma-reduced-ring-sub-product-fields}.
Since $k \subset k'$ is finite free we have
$(\prod_{\mathfrak q} A_{\mathfrak q}) \otimes_k k'
=\prod_{\mathfrak q} A_{\mathfrak q} \otimes_k k'$
and since $k \to k'$ is flat we have
$A \otimes_k k' \subset (\prod_{\mathfrak q} A_{\mathfrak q}) \otimes_k k'$.
Thus we see that $A \otimes_k k'$ is reduced, i.e., (2) holds.

\medskip\noindent
Part (3) follows trivially from part (2).
\end{proof}

\begin{lemma}
\label{lemma-Noetherian-geometrically-reduced-at-point}
Let $k$ be a field.
Let $X$ be a scheme over $k$.
Let $x \in X$.
Assume $X$ locally Noetherian and geometrically reduced at $x$.
Then there exists an open neighbourhood $U \subset X$ of $x$
which is geometrically reduced over $k$.
\end{lemma}

\begin{proof}
Let $R$ be a Noetherian $k$-algebra.
Let $\mathfrak p \subset R$ be a prime.
Let $I = \text{Ker}(R \to R_{\mathfrak p}$.
Since $IR_{\mathfrak p} = R_{\mathfrak p}$ and $I$ is finitely generated
there exists an $f \in R$, $f \not \in \mathfrak p$ such that $fI = 0$. 
Hence $R_f \subset R_{\mathfrak p}$.

\medskip\noindent
Assume $X$ locally Noetherian and geometrically reduced at $x$.
If we apply the above to $R = \mathcal{O}_X(U)$ for some affine
open neighbourhood of $x$, and $\mathfrak p \subset R$ the prime
corresponding to $x$, then we see that after shrinking $U$ we may
assume $R \subset R_{\mathfrak p}$. By
Lemma \ref{lemma-geometrically-reduced-at-point} the assumption
means that $R_{\mathfrak p}$ is geometrically reduced over $k$.
By Algebra, Lemma \ref{algebra-lemma-subalgebra-separable}
this implies that $R$ is geometrically reduced over $k$, which
in turn implies that $U$ is geometrically reduced.
\end{proof}

\begin{example}
\label{example-not-geometrically-reduced}
Let $k = \mathbf{F}_p(s, t)$, i.e., a purely transcendental extension
of the prime field. Consider the variety
$X = \text{Spec}(k[x, y]/(1 + sx^p + ty^p))$.
Let $k \subset k'$ be any extension such that
both $s$ and $t$ have a $p$th root in $k'$.
Then the base change $X_{k'}$ is not reduced.
Namely, the ring $k'[x, y]/(1 + s x^p + ty^p)$ contains the element
$1 + s^{1/p}x + t^{1/p}y$ whose $p$th power is zero but
which is not zero (since the ideal $(1 + sx^p + ty^p)$ certainly
does not contain any nonzero element of degree $< p$).
\end{example}

\begin{lemma}
\label{lemma-finite-extension-geometrically-reduced}
Let $k$ be a field.
Let $X \to \text{Spec}(k)$ be locally of finite type.
Assume $X$ has finitely many irreducible components.
Then there exists a finite purely inseparable extension $k \subset k'$
such that $(X_{k'})_{red}$ is geometrically reduced over $k'$.
\end{lemma}

\begin{proof}
To prove this lemma we may replace $X$ by its reduction $X_{red}$.
Hence we may assume that $X$ is reduced and locally of finite type
over $k$.
Let $x_1, \ldots, x_n \in X$ be the generic points of the irreducible
components of $X$.
Note that for every purely inseparable algebraic extension $k \subset k'$
the morphism $(X_{k'})_{red} \to X$ is a homeomorphism, see
Algebra, Lemma \ref{algebra-lemma-p-ring-map}. Hence the points
$x'_1, \ldots, x'_n$ lying over $x_1, \ldots, x_n$ are the generic
points of the irreducible components of $(X_{k'})_{red}$.
As $X$ is reduced the local rings $K_i = \mathcal{O}_{X, x_i}$ are fields, see
Algebra, Lemma \ref{algebra-lemma-minimal-prime-reduced-ring}.
As $X$ is locally of finite type over $k$ the field extensions
$k \subset K_i$ are finitely generated field extensions.
Finally, the local rings $\mathcal{O}_{X_{k'}, x'_i}$ are the
fields $(K_i \otimes_k k')_{red}$. By
Algebra, Lemma \ref{algebra-lemma-make-separable}
we can find a finite purely inseparable extension $k \subset k'$
such that $(K_i \otimes_k k')_{red}$ are separable field
extensions of $k'$. In particular each $(K_i \otimes_k k')_{red}$
is geometrically reduced over $k'$ by
Algebra, Lemma \ref{algebra-lemma-characterize-separable-field-extensions}.
At this point
Lemma \ref{lemma-generic-points-geometrically-reduced} part (3)
implies that $(X_{k'})_{red}$ is geometrically reduced.
\end{proof}






\section{Geometrically connected schemes}
\label{section-geometrically-connected}

\noindent
If $X$ is a connected scheme over a field, then it can happen that $X$
becomes disconnected after extending the ground field. This does not happen
for geometrically connected schemes.

\begin{definition}
\label{definition-geometrically-connected}
Let $X$ be a scheme over the field $k$.
We say $X$ is {\it geometrically connected} over $k$
if the scheme $X_{k'}$ is connected\footnote{An empty scheme is connected.}
for every field extension $k'$ of $k$.
\end{definition}

\noindent
Here is an example of a variety which is not geometrically connected.

\begin{example}
\label{example-not-geometrically-irreducible}
Let $k = \mathbf{Q}$. The scheme
$X = \text{Spec}(\mathbf{Q}(i))$ is a variety over $\text{Spec}(\mathbf{Q})$.
But the base change $X_{\mathbf{C}}$ is the spectrum of
$\mathbf{C} \otimes_{\mathbf{Q}} \mathbf{Q}(i) \cong
\mathbf{C} \times \mathbf{C}$ which is the disjoint union of
two copies of $\text{Spec}(\mathbf{C})$. So in fact, this is an
example of a non-geometrically connected variety.
\end{example}

\begin{lemma}
\label{lemma-bijection-connected-components}
Let $k$ be a field.
Let $X$, $Y$ be schemes over $k$.
Assume $X$ is geometrically connected over $k$.
Then the projection morphism
$$
p : X \times_k Y \longrightarrow Y
$$
induces a bijection between connected components.
\end{lemma}

\begin{proof}
The scheme theoretic fibres of $p$ are connected
and nonempty, since they
are base changes of the geometrically connected scheme $X$ by
field extensions. Moreover the scheme theoretic fibres are
homeomorphic to the set theoretic fibres, see
Schemes, Lemma \ref{schemes-lemma-fibre-topological}.
By Morphisms, Lemma \ref{morphisms-lemma-scheme-over-field-universally-open}
the map $p$ is open.
Thus we may apply Topology,
Lemma \ref{topology-lemma-connected-fibres-connected-components}
to conclude.
\end{proof}

\begin{lemma}
\label{lemma-affine-geometrically-connected}
Let $k$ be a field.
Let $A$ be a $k$-algebra.
Then $X = \text{Spec}(A)$ is geometrically connected over $k$
if and only if $A$ is geometrically connected over $k$ (see
Algebra, Definition \ref{algebra-definition-geometrically-connected}).
\end{lemma}

\begin{proof}
Immediate from the definitions.
\end{proof}

\begin{lemma}
\label{lemma-separably-closed-field-connected-components}
Let $k \subset k'$ be an extension of fields.
Let $X$ be a scheme over $k$. Set $X' = X_{k'}$.
Assume $k$ separably algebraically closed.
Then the morphism $X' \to X$ induces a bijection of connected
components.
\end{lemma}

\begin{proof}
Since $k$ is separably algebraically closed we see that
$k'$ is geometrically connected over $k$, see Algebra,
Lemma \ref{algebra-lemma-separably-closed-connected-implies-geometric}.
Hence $Z = \text{Spec}(k')$ is geometrically connected over $k$.
by Lemma \ref{lemma-affine-geometrically-connected} above.
Since $X' = Z \times_k X$ the result is a special case
of Lemma \ref{lemma-bijection-connected-components}.
\end{proof}

\begin{lemma}
\label{lemma-characterize-geometrically-connected}
Let $k$ be a field.
Let $X$ be a scheme over $k$.
Let $\overline{k}$ be a separable algebraic closure of $k$.
Then $X$ is geometrically connected if and only if the base change
$X_{\overline{k}}$ is connected.
\end{lemma}

\begin{proof}
Assume $X_{\overline{k}}$ is connected.
Let $k \subset k'$ be a field extension.
There exists a field extension $\overline{k} \subset \overline{k}'$
such that $k'$ embeds into $\overline{k}'$ as an extension of $k$.
By Lemma \ref{lemma-separably-closed-field-connected-components}
we see that $X_{\overline{k}'}$ is connected.
Since $X_{\overline{k}'} \to X_{k'}$ is surjective we conclude
that $X_{k'}$ is connected as desired.
\end{proof}

\begin{lemma}
\label{lemma-descend-open}
Let $k$ be a field.
Let $X$ be a scheme over $k$.
Let $A$ be a $k$-algebra.
Assume $X$ is quasi-compact.
Let $V \subset X_A$ be a quasi-compact open.
Then there exists a finitely generated $k$-subalgebra $A' \subset A$
and a quasi-compact open $V' \subset X_{A'}$
such that $V = V'_A$.
\end{lemma}

\begin{proof}
We remark that if $X$ is also quasi-separated this follows from
Limits, Lemma \ref{limits-lemma-descend-opens}. Let
$X = U_1 \cup \ldots \cup U_n$ be a finite affine open
covering. Say $U_i = \text{Spec}(R_i)$.
Since $V$ is quasi-compact we can find finitely many
$f_{ij} \in R_i \otimes_k A$, $j = 1, \ldots, n_i$
such that $V = \bigcup_i \bigcup_{j = 1, \ldots, n_i} D(f_{ij})$
where $D(f_{ij}) \subset U_{i, A}$ is the corresponding standard
open. (We do not claim that $V \cap U_{i, A}$ is the union
of the $D(f_{ij})$, $j = 1, \ldots, n_i$.)
It is clear that we can find a finitely generated $k$-subalgebra
$A' \subset A$ such that $f_{ij}$ is the image of some
$f_{ij}' \in R_i \otimes_k A'$.
Set $V' = \bigcup D(f_{ij}')$ which is a quasi-compact open of $X_{A'}$.
Denote $\pi : X_A \to X_{A'}$ the canonical morphism.
We have $\pi(V) \subset V'$ as $\pi(D(f_{ij})) \subset D(f_{ij}')$.
If $x \in X_A$ with $\pi(x) \in V'$, then $\pi(x) \in D(f_{ij}')$
for some $i, j$ and we see that $x \in D(f_{ij})$ as $f_{ij}'$
maps to $f_{ij}$. Thus we see that $V = \pi^{-1}(V')$ as desired.
\end{proof}

\begin{lemma}
\label{lemma-characterize-geometrically-connected-quasi-compact}
Let $k$ be a field. Let $X$ be a scheme over $k$.
Assume $X$ is quasi-compact. The following are
equivalent:
\begin{enumerate}
\item $X$ is geometrically connected over $k$, and
\item for every finite separable field extension $k \subset k'$
the scheme $X_{k'}$ is connected.
\end{enumerate}
\end{lemma}

\begin{proof}
Let $k \subset \overline{k}$ be a separable algebraic closure of $k$.
By Lemma \ref{lemma-characterize-geometrically-connected} it suffices
to prove the following statement:
If $X_{\overline{k}}$ is disconnected, then also $X_{k'}$ is
disconnected for some finite subextension $k \subset k' \subset \overline{k}$.
We will give two proofs of this fact.

\medskip\noindent
First proof.
Let $X = U_1 \cup \ldots \cup U_n$ be a finite affine open
covering. Say $U_i = \text{Spec}(A_i)$. If $X_{\overline{k}}$ is disconnected
then there exist an open and closed subscheme
$\overline{V} \subset X_{\overline{k}}$ which is neither empty
nor equal to $X_{\overline{k}}$. For each $i$ we see that
$\overline{V} \cap U_{i, \overline{k}}$ corresponds to an
idempotent $e_i \in A_i \otimes_k \overline{k}$, see
Algebra, Lemma \ref{algebra-lemma-disjoint-decomposition}. Moreover, at
least for one $i$ the idempotent $e_i$ is a nontrivial idempotent.
Clearly, we can find a finite subextension $k \subset k' \subset \overline{k}$
such that each $e_i$ is actually an element of $A_i \otimes_k k'$.
Write $A_{ij} = \Gamma(U_i \cap U_j, \mathcal{O}_X)$.
We remark that
$$
A_{ij} \otimes_k k'
\longrightarrow
\Gamma((U_i \cap U_j)_{k'}, \mathcal{O}_{X_{k'}})
=
\Gamma(U_{i, k'} \cap U_{j, k'}, \mathcal{O}_{X_{k'}})
$$
is injective for all field extensions $k \subset k'$ (verification omitted).
Thus the $e_i \in A_i \otimes_k k'$ glue to a global idempotent
$e' \in \Gamma(X_{k'}, \mathcal{O})$.
Then it is clear that $\overline{U}$ comes from the open and closed
subscheme $U' \subset X_{k'}$ corresponding to $e'$.

\medskip\noindent
Second proof.
Write $X_{\overline{k}} = U \coprod V$ with both $U$ and $V$ open and
closed (in particular quasi-compact).
By Lemma \ref{lemma-descend-open} there exists a finite subextension
$k \subset k' \subset \overline{k}$ and opens $U', V' \subset X_{k'}$
which pull back to $U$ and $V$. Since $X_{\overline{k}} \to X_{k'}$
is surjective we coclude that $X_{k'} = U' \coprod V'$ as desired.
\end{proof}

\noindent
Let $k$ be a field. Let $k \subset \overline{k}$ be a (possibly infinite)
Galois extension. For example $\overline{k}$ could be the
separable algebraic closure of $k$.
For any $\sigma \in \text{Gal}(\overline{k}/k)$ we get a corresponding
automorphism
$
\text{Spec}(\sigma) :
\text{Spec}(\overline{k})
\longrightarrow
\text{Spec}(\overline{k})
$.
Note that
$\text{Spec}(\sigma) \circ \text{Spec}(\tau) = \text{Spec}(\tau \circ \sigma)$.
Hence we get an action
$$
\text{Gal}(\overline{k}/k)^{opp} \times \text{Spec}(\overline{k})
\longrightarrow
\text{Spec}(\overline{k})
$$
of the opposite group on the scheme $\text{Spec}(\overline{k})$.
Let $X$ be a scheme over $k$. Since
$X_{\overline{k}} =
\text{Spec}(\overline{k}) \times_{\text{Spec}(k)} X$
by definition we see that the action above induces a canonical action
\begin{equation}
\label{equation-galois-action-base-change-kbar}
\text{Gal}(\overline{k}/k)^{opp} \times X_{\overline{k}}
\longrightarrow
X_{\overline{k}}.
\end{equation}

\begin{lemma}
\label{lemma-closed-fixed-by-Galois}
Let $k$ be a field. Let $k \subset \overline{k}$ be a (possibly infinite)
Galois extension. Let $X$ be a scheme over $k$. Let
$\overline{T} \subset X_{\overline{k}}$ have the following properties
\begin{enumerate}
\item $\overline{T}$ is a closed subset of $X_{\overline{k}}$,
\item for every $\sigma \in \text{Gal}(\overline{k}/k)$
we have $\sigma(\overline{T}) = \overline{T}$.
\end{enumerate}
Then there exists a closed subset $T \subset X$ whose inverse image
in $X_{k'}$ is $\overline{T}$.
\end{lemma}

\begin{proof}
This lemma immediately reduces to the case where $X = \text{Spec}(A)$
is affine. In this case, let $\overline{I} \subset A \otimes_k \overline{k}$
be the radical ideal corresponding to $\overline{T}$.
Assumption (2) implies that $\sigma(\overline{I}) = \overline{I}$
for all $\sigma \in \text{Gal}(\overline{k}/k)$.
Pick $x \in \overline{I}$. There exists a finite Galois extension
$k \subset k'$ contained in $\overline{k}$ such that $x \in A \otimes_k k'$.
Set $G = \text{Gal}(k'/k)$. Set
$$
P(T) = \prod\nolimits_{\sigma \in G} (T - \sigma(x)) \in (A \otimes_k k')[T]
$$
It is clear that $P(T)$ is monic and is actually an element of
$(A \otimes_k k')^G[T] = A[T]$ (by basic Galois theory).
Moreover, if we write $P(T) = T^d + a_1T^{d - 1} + \ldots + a_0$
the we see that $a_i \in I := A \cap \overline{I}$. By
Algebra, Lemma \ref{algebra-lemma-polynomials-divide}
we see that $x$ is contained in the radical of $I(A \otimes_k \overline{k})$.
Hence $\overline{I}$ is the radical of $I(A \otimes_k \overline{k})$ and
setting $T = V(I)$ is a solution.
\end{proof}

\begin{lemma}
\label{lemma-tricky}
Let $k$ be a field. Let $k \subset \overline{k}$ be a (possibly infinite)
Galois extension. Let $X$ be a scheme over $k$. Let $T \subset X$ be
a nonempty closed subset. Assume $T_{\overline{k}}$ connected, and
assume $X_{\overline{k}}$ disconnected. Then $X$ is disconnected.
\end{lemma}

\begin{proof}
Write $X_{\overline{k}} = \overline{U} \coprod \overline{V}$
with $\overline{U}$ and $\overline{V}$ open and closed.
Since $T_{\overline{k}}$ is connected we see that
$T_{\overline{k}}$ is contained in either $\overline{U}$ or $\overline{V}$.
Say $T_{\overline{k}} \subset \overline{U}$.

\medskip\noindent
Fix a quasi-compact open $W \subset X$. There exists a
finite Galois subextension $k \subset k' \subset \overline{k}$
such that $\overline{U} \cap W_{\overline{k}}$ and
$\overline{V} \cap W_{\overline{k}}$ come from quasi-compact
opens $U', V' \subset W_{k'}$. Then also $W_{k'} = U' \coprod V'$.
Consider
$$
U'' = \bigcap\nolimits_{\sigma \in \text{Gal}(k'/k)} \sigma(U'),
\quad
V'' = \bigcup\nolimits_{\sigma \in \text{Gal}(k'/k)} \sigma(V').
$$
These are Galois invariant, open and closed, and
$W_{k'} = U'' \coprod V''$.
By Lemma \ref{lemma-closed-fixed-by-Galois} we get open and closed subsets
$U_W, V_W \subset W$ such that
$U'' = (U_W)_{k'}$, $V'' = (V_W)_{k'}$ and
$W = U_W \coprod V_W$.

\medskip\noindent
We claim that if $W \subset W' \subset X$ are quasi-compact
open, then $W \cap U_{W'} = U_W$ and $W \cap V_{W'} = V_W$.
Verification omitted.
Hence we see that upon defining $U = \bigcup_{W \subset X} U_W$
and $V = \bigcup_{W \subset X} V_W$ we obtain $X = U \coprod V$.
It is clear that $V$ is nonempty as it is constructed by taking
unions (locally). On the other hand, $U$ is nonempty since it contains
$T$ by construction.
\end{proof}

\begin{lemma}
\label{lemma-geometrically-connected-if-connected-and-point}
Let $k$ be a field. Let $X$ be a scheme over $k$.
Assume that $X$ has a $k$-rational point and that $X$ is connected.
Then $X$ is geometrically connected.
\end{lemma}

\begin{proof}
Let $x \in X$ be a point with $\kappa(x) = k$. Then $x$ is a
closed point of $X$, see
Morphisms, Lemma
\ref{morphisms-lemma-algebraic-residue-field-extension-closed-point-fibre}.
Moreover, if $T = \text{Spec}(\kappa(x)) \subset X$ denotes the corresponding
closed subscheme, then it is clear that $T_{\overline{k}}$ is connected, where
$\overline{k}$ is a separable algebraic closure of $k$. Hence by
Lemma \ref{lemma-tricky}
we see that $X_{\overline{k}}$ is connected. By
Lemma \ref{lemma-characterize-geometrically-connected}
we conclude that $X$ is geometrically connected.
\end{proof}

\noindent
Let $X$ be a scheme. We denote $\pi_0(X)$ the set of connected
components of $X$.

\begin{lemma}
\label{lemma-galois-action-connected-components}
Let $k$ be a field, with separable algebraic closure $\overline{k}$.
Let $X$ be a scheme over $k$.
There is an action
$$
\text{Gal}(\overline{k}/k)^{opp} \times \pi_0(X_{\overline{k}})
\longrightarrow
\pi_0(X_{\overline{k}})
$$
with the following properties:
\begin{enumerate}
\item An element $\overline{T} \in \pi_0(X_{\overline{k}})$
is fixed by the action if and only if there exists a connected component
$T \subset X$, which is geometrically connected over $k$,
such that $T_{\overline{k}} = \overline{T}$.
\item For any field extension $k \subset k'$ with separable
algebraic closure $\overline{k}'$ the diagram
$$
\xymatrix{
\text{Gal}(\overline{k}'/k') \times \pi_0(X_{\overline{k}'})
\ar[r] \ar[d] &
\pi_0(X_{\overline{k}'}) \ar[d] \\
\text{Gal}(\overline{k}/k) \times \pi_0(X_{\overline{k}})
\ar[r] &
\pi_0(X_{\overline{k}})
}
$$
is commutative (where the right vertical arrow is a bijection
according to Lemma \ref{lemma-separably-closed-field-connected-components}).
\end{enumerate}
\end{lemma}

\begin{proof}
The action (\ref{equation-galois-action-base-change-kbar})
of $\text{Gal}(\overline{k}/k)$ on $X_{\overline{k}}$
induces an action on its connected components.
Connected components are always closed
(Topology, Lemma \ref{topology-lemma-connected-components}).
Hence if $\overline{T}$ is as in (1), then by
Lemma \ref{lemma-closed-fixed-by-Galois} there exists a closed
subset $T \subset X$ such that $\overline{T} = T_{\overline{k}}$.
Note that $T$ is geometrically connected over $k$, see
Lemma \ref{lemma-characterize-geometrically-connected}.
To see that $T$ is a connected component of $X$, suppose that
$T \subset T'$, $T \not = T'$ where $T'$ is a connected component of $X$.
In this case $T'_{k'}$ strictly contains $\overline{T}$ and hence is
disconnnected. By Lemma \ref{lemma-tricky} this means that $T'$ is
disconnected! Contradiction.

\medskip\noindent
We omit the proof of the functoriality in (2).
\end{proof}

\begin{lemma}
\label{lemma-galois-action-connected-components-continuous}
Let $k$ be a field, with separable algebraic closure $\overline{k}$.
Let $X$ be a scheme over $k$.
Assume
\begin{enumerate}
\item $X$ is quasi-compact, and
\item the connected components of $X_{\overline{k}}$ are open.
\end{enumerate}
Then
\begin{enumerate}
\item[(a)] $\pi_0(X_{\overline{k}})$ is finite, and
\item[(b)] the action of $\text{Gal}(\overline{k}/k)$ on
$\pi_0(X_{\overline{k}})$ is continuous.
\end{enumerate}
Moreover, assumptions (1) and (2) are satisfied when $X$ is
of finite type over $k$.
\end{lemma}

\begin{proof}
Since the connected components are open, cover $X_{\overline{k}}$
(Topology, Lemma \ref{topology-lemma-connected-components}) and
$X_{\overline{k}}$ is quasi-compact, we conclude that there are only
finitely many of them. Thus (a) holds.
By Lemma \ref{lemma-descend-open} these connected components
are each defined over a finite subextension of $k \subset \overline{k}$
and we get (b).
If $X$ is of finite type over $k$, then $X_{\overline{k}}$ is of finite
type over $\overline{k}$
(Morphisms, Lemma \ref{morphisms-lemma-base-change-finite-type}).
Hence $X_{\overline{k}}$ is a Noetherian scheme
(Morphisms, Lemma \ref{morphisms-lemma-finite-type-noetherian}) and has
an underlying Noetherian topological space
(Properties, Lemma \ref{properties-lemma-Noetherian-topology}).
Thus $X_{\overline{k}}$ has finitely many irreducible components
(Topology, Lemma \ref{topology-lemma-Noetherian})
and a fortiori finitely many connected components (which are
therefore open).
\end{proof}









\section{Geometrically irreducible schemes}
\label{section-geometrically-irreducible}

\noindent
If $X$ is an irreducible scheme over a field, then it can happen that $X$
becomes reducible after extending the ground field. This does not happen
for geometrically irreducible schemes.

\begin{definition}
\label{definition-geometrically-irreducible}
Let $X$ be a scheme over the field $k$.
We say $X$ is {\it geometrically irreducible} over $k$ if the scheme
$X_{k'}$ is irreducible\footnote{An irreducible space is nonempty.}
for any field extension $k'$ of $k$.
\end{definition}

\begin{lemma}
\label{lemma-separably-closed-irreducible}
Let $X$ be a scheme over a separably closed field $k$.
If $X$ is irreducible, then $X_K$ is irreducible for any
field extension $k \subset K$. I.e., $X$ is geometrically
irreducible over $k$.
\end{lemma}

\begin{proof}
Use Properties, Lemma \ref{properties-lemma-characterize-irreducible}
and Algebra, Lemma \ref{algebra-lemma-separably-closed-irreducible}.
\end{proof}

\begin{lemma}
\label{lemma-bijection-irreducible-components}
Let $k$ be a field.
Let $X$, $Y$ be schemes over $k$.
Assume $X$ is geometrically irreducible over $k$.
Then the projection morphism
$$
p : X \times_k Y \longrightarrow Y
$$
induces a bijection between irreducible components.
\end{lemma}

\begin{proof}
First, note that the scheme theoretic fibres of $p$ are irreducible,
since they are base changes of the geometrically irreducible scheme $X$
by field extensions. Moreover the scheme theoretic fibres are
homeomorphic to the set theoretic fibres, see
Schemes, Lemma \ref{schemes-lemma-fibre-topological}.
By Morphisms, Lemma \ref{morphisms-lemma-scheme-over-field-universally-open}
the map $p$ is open.
Thus we may apply Topology,
Lemma \ref{topology-lemma-irreducible-fibres-irreducible-components}
to conclude.
\end{proof}

\begin{lemma}
\label{lemma-geometrically-irreducible-local}
Let $k$ be a field. Let $X$ be a scheme over $k$.
The following are equivalent
\begin{enumerate}
\item $X$ is geometrically irreducible over $k$,
\item for every affine open $U$ the $k$-algebra $\mathcal{O}_X(U)$
is geometrically irreducible over $k$ (see
Algebra, Definition \ref{algebra-definition-geometrically-irreducible}),
\item $X$ is irreducible and there exists an affine open covering
$X = \bigcup U_i$ such that each $k$-algebra $\mathcal{O}_X(U_i)$ is
geometrically irreducible, and
\item there exists an open covering $X = \bigcup_{i \in I} X_i$ such
that $X_i$ is geometrically irreducible for each $i$ and such that
$X_i \cap X_j \not = \emptyset$ for all $i, j \in I$.
\end{enumerate}
Moreover, if $X$ is geometrically irreducible so is every
open subscheme of $X$.
\end{lemma}

\begin{proof}
An affine scheme $\text{Spec}(A)$ over $k$ is geometrically
irreducible if and only if $A$ is geometrically irreducible over $k$;
this is immediate from the definitions.
Recall that if a scheme is irreducible so is every nonempty
open subscheme of $X$, any two nonempty open subsets have
a nonempty intersection. Also, if every affine open is irreducible
then the scheme is irreducible, see Properties,
Lemma \ref{properties-lemma-characterize-irreducible}.
Hence the final statement of the lemma
is clear, as well as the implications (1) $\Rightarrow$ (2),
(2) $\Rightarrow$ (3), and (3) $\Rightarrow$ (4). If (4) holds,
then for any field extension $k'/k$ the scheme $X_{k'}$
has a covering by irreducible opens which pairwise intersect.
Hence $X_{k'}$ is irreducible. Hence (4) implies (1).
\end{proof}

\begin{lemma}
\label{lemma-separably-closed-field-irreducible-components}
Let $k \subset k'$ be an extension of fields.
Let $X$ be a scheme over $k$. Set $X' = X_{k'}$.
Assume $k$ separably algebraically closed.
Then the morphism $X' \to X$ induces a bijection of irreducible components.
\end{lemma}

\begin{proof}
Since $k$ is separably algebraically closed we see that
$k'$ is geometrically irreducible over $k$, see Algebra,
Lemma \ref{algebra-lemma-separably-closed-irreducible-implies-geometric}.
Hence $Z = \text{Spec}(k')$ is geometrically irreducible over $k$.
by Lemma \ref{lemma-geometrically-irreducible-local} above.
Since $X' = Z \times_k X$ the result is a special case
of Lemma \ref{lemma-bijection-irreducible-components}.
\end{proof}

\begin{lemma}
\label{lemma-characterize-geometrically-irreducible}
Let $k$ be a field. Let $X$ be a scheme over $k$.
Assume $X$ is quasi-compact. The following are equivalent:
\begin{enumerate}
\item $X$ is geometrically irreducible over $k$,
\item for every finite separable field extension $k \subset k'$
the scheme $X_{k'}$ is irreducible, and
\item $X_{\overline{k}}$ is irreducible, where $k \subset \overline{k}$
is a separable algebraic closure of $k$.
\end{enumerate}
\end{lemma}

\begin{proof}
Assume $X_{\overline{k}}$ is irreducible, i.e., assume (3).
Let $k \subset k'$ be a field extension.
There exists a field extension $\overline{k} \subset \overline{k}'$
such that $k'$ embeds into $\overline{k}'$ as an extension of $k$.
By Lemma \ref{lemma-separably-closed-field-irreducible-components}
we see that $X_{\overline{k}'}$ is irreducible.
Since $X_{\overline{k}'} \to X_{k'}$ is surjective we conclude
that $X_{k'}$ is irreducible. Hence (1) holds.

\medskip\noindent
Let $k \subset \overline{k}$ be a separable algebraic closure of $k$.
Assume not (3), i.e., assume $X_{\overline{k}}$ is reducible.
Our goal is to show that also $X_{k'}$ is
reducible for some finite subextension
$k \subset k' \subset \overline{k}$.
Let $X = \bigcup_{i \in I} U_i$ be an affine open covering
with $U_i$ not empty. If for some $i$ the scheme
$U_i$ is reducible, or if for some pair $i \not = j$ the
intersection $U_i \cap U_j$ is empty, then $X$ is reducible
(Properties, Lemma \ref{properties-lemma-characterize-irreducible})
and we are done.
In particular we may assume that
$U_{i, \overline{k}} \cap U_{j, \overline{k}}$ for all $i, j \in I$
is nonempty and we conclude that $U_{i, \overline{k}}$ has
to be reducible for some $i$. According to
Algebra, Lemma \ref{algebra-lemma-geometrically-irreducible}
this means that $U_{i, k'}$ is reducible for some
finite separable field extension $k \subset k'$.
Hence also $X_{k'}$ is reducible. Thus we see that
(2) implies (3).

\medskip\noindent
The implication (1) $\Rightarrow$ (2) is immediate.
This proves the lemma.
\end{proof}

\begin{lemma}
\label{lemma-inverse-image-irreducible}
Let $k \subset K$ be an extension of fields.
Let $X$ be a scheme over $k$.
For every irreducible component $T$ of $X$ the inverse image
$T_K \subset X_K$ is a union of irreducible components of $X_K$.
\end{lemma}

\begin{proof}
Let $T \subset X$ be an irreducible component of $X$.
The morphism $T_K \to T$ is flat, so generalizations lift
along $T_K \to T$. Hence every $\xi \in T_K$
which is a generic point of an irreducible component of $T_K$
maps to the generic point $\eta$ of $T$. If $\xi' \leadsto \xi$ is
a specialization in $X_K$ then $\xi'$ maps to $\eta$ since there
are no points specializing to $\eta$ in $X$. Hence $\xi' \in T_K$
and we conclude that $\xi = \xi'$. In other words $\xi$ is the
generic point of an irreducible component of $X_K$. This
means that the irreducible components of $T_K$ are all irreducible
components of $X_K$.
\end{proof}

\noindent
For a scheme $X$ we denote $\text{IrredComp}(X)$ the set of
irreducible components of $X$.

\begin{lemma}
\label{lemma-image-irreducible}
Let $k \subset K$ be an extension of fields.
Let $X$ be a scheme over $k$.
For every irreducible component $\overline{T} \subset X_K$
the image of $\overline{T}$ in $X$ is an irreducible component in $X$.
This defines a canonical map
$$
\text{IrredComp}(X_K)
\longrightarrow
\text{IrredComp}(X)
$$
which is surjective.
\end{lemma}

\begin{proof}
Consider the diagram
$$
\xymatrix{
X_K \ar[d] & X_{\overline{K}} \ar[d] \ar[l] \\
X & X_{\overline{k}} \ar[l]
}
$$
where $\overline{K}$ is the separable algebraic closure of $K$, and
where $\overline{k}$ is the separable algebraic closure of $k$. By
Lemma \ref{lemma-separably-closed-field-irreducible-components}
the morphism $X_{\overline{K}} \to X_{\overline{k}}$ induces
a bijection between irreducible components. Hence it suffices
to show the lemma for the morphisms
$X_{\overline{k}} \to X$ and $X_{\overline{K}} \to X_K$.
In other words we may assume that $K = \overline{k}$.

\medskip\noindent
The morphism $p : X_{\overline{k}} \to X$ is integral, flat and surjective.
Flatness implies that generalizations lift along $p$, see
Morphisms, Lemma \ref{morphisms-lemma-generalizations-lift-flat}.
Hence generic points of irreducible components of $X_{\overline{k}}$
map to generic points of irreducible components of $X$.
Integrality implies that $p$ is universally closed, see
Morphisms, Lemma \ref{morphisms-lemma-integral-universally-closed}.
Hence we conclude that the image $p(\overline{T})$ of an irreducible component
is a closed irreducible subset which contains a generic point of an
irreducible component of $X$, hence $p(\overline{T})$
is an irreducible component of $X$. This proves the first assertion.
If $T \subset X$ is an irreducible component, then $p^{-1}(T) =T_K$
is a nonempty union of irreducible components, see
Lemma \ref{lemma-inverse-image-irreducible}.
Each of these necessarily maps onto $T$ by the first part.
Hence the map is surjective.
\end{proof}

\begin{lemma}
\label{lemma-galois-action-irreducible-components}
Let $k$ be a field, with separable algebraic closure $\overline{k}$.
Let $X$ be a scheme over $k$.
There is an action
$$
\text{Gal}(\overline{k}/k)^{opp} \times \text{IrredComp}(X_{\overline{k}})
\longrightarrow
\text{IrredComp}(X_{\overline{k}})
$$
with the following properties:
\begin{enumerate}
\item An element $\overline{T} \in \text{IrredComp}(X_{\overline{k}})$
is fixed by the action if and only if there exists an irreducible
component $T \subset X$, which is geometrically irreducible over $k$,
such that $T_{\overline{k}} = \overline{T}$.
\item For any field extension $k \subset k'$ with separable
algebraic closure $\overline{k}'$ the diagram
$$
\xymatrix{
\text{Gal}(\overline{k}'/k') \times \text{IrredComp}(X_{\overline{k}'})
\ar[r] \ar[d] &
\text{IrredComp}(X_{\overline{k}'}) \ar[d] \\
\text{Gal}(\overline{k}/k) \times \text{IrredComp}(X_{\overline{k}})
\ar[r] &
\text{IrredComp}(X_{\overline{k}})
}
$$
is commutative (where the right vertical arrow is a bijection
according to Lemma \ref{lemma-separably-closed-field-irreducible-components}).
\end{enumerate}
\end{lemma}

\begin{proof}
The action (\ref{equation-galois-action-base-change-kbar})
of $\text{Gal}(\overline{k}/k)$ on $X_{\overline{k}}$
induces an action on its irreducible components.
Irreducible components are always closed
(Topology, Lemma \ref{topology-lemma-connected-components}).
Hence if $\overline{T}$ is as in (1), then by
Lemma \ref{lemma-closed-fixed-by-Galois} there exists a closed
subset $T \subset X$ such that $\overline{T} = T_{\overline{k}}$.
Note that $T$ is geometrically irreducible over $k$, see
Lemma \ref{lemma-characterize-geometrically-irreducible}.
To see that $T$ is an irreducible component of $X$, suppose that
$T \subset T'$, $T \not = T'$ where $T'$ is an irreducible
component of $X$. Let $\overline{\eta}$ be the generic point of
$\overline{T}$. It maps to the generic point $\eta$ of $T$.
Then the generic point $\xi \in T'$ specializes to $\eta$.
As $X_{\overline{k}} \to X$ is flat there exists a point
$\overline{\xi} \in X_{\overline{k}}$ which maps to $\xi$ and
specializes to $\overline{\eta}$. It follows that
the closure of the singleton $\{\overline{\xi}\}$ is an
irreducible closed subset of $X_{\overline{\xi}}$ which
strictly contains $\overline{T}$. This is the desired contradiction.

\medskip\noindent
We omit the proof of the functoriality in (2).
\end{proof}

\begin{lemma}
\label{lemma-orbit-irreducible-components}
Let $k$ be a field, with separable algebraic closure $\overline{k}$.
Let $X$ be a scheme over $k$.
The fibres of the map
$$
\text{IrredComp}(X_{\overline{k}})
\longrightarrow
\text{IrredComp}(X)
$$
of
Lemma \ref{lemma-image-irreducible}
are exactly the orbits of $\text{Gal}(\overline{k}/k)$ under the action of
Lemma \ref{lemma-galois-action-irreducible-components}.
\end{lemma}

\begin{proof}
Let $T \subset X$ be an irreducible component of $X$.
Let $\eta \in T$ be its generic point. By
Lemmas \ref{lemma-inverse-image-irreducible} and
\ref{lemma-image-irreducible}
the generic points of irreducible components $\overline{T}$ which map into $T$
map to $\eta$. By
Algebra, Lemma \ref{algebra-lemma-Galois-orbit}
the Galois group acts transitively on
all of the points of $X_{\overline{k}}$ mapping to $\eta$.
Hence the lemma follows.
\end{proof}

\begin{lemma}
\label{lemma-galois-action-irreducible-components-locally-finite-type}
Let $k$ be a field.
Assume $X \to \text{Spec}(k)$ locally of finite type.
In this case
\begin{enumerate}
\item the action
$$
\text{Gal}(\overline{k}/k)^{opp} \times \text{IrredComp}(X_{\overline{k}})
\longrightarrow
\text{IrredComp}(X_{\overline{k}})
$$
is continuous if we give $\text{IrredComp}(X_{\overline{k}})$ the discrete
topology,
\item every irreducible component of $X_{\overline{k}}$
can be defined over a finite extension of $k$, and
\item given any irreducible component $T \subset X$ the scheme
$T_{\overline{k}}$ is a finite union of irreducible components of
$X_{\overline{k}}$ which are all in the same
$\text{Gal}(\overline{k}/k)$-orbit.
\end{enumerate}
\end{lemma}

\begin{proof}
Let $\overline{T}$ be an irreducible component of $X_{\overline{k}}$.
We may choose an affine open $U \subset X$ such that
$\overline{T} \cap U_{\overline{k}}$ is not empty.
Write $U = \text{Spec}(A)$, so $A$ is a finite type $k$-algebra, see
Morphisms, Lemma \ref{morphisms-lemma-locally-finite-type-characterize}.
Hence $A_{\overline{k}}$ is a finite type $\overline{k}$-algebra,
and in particular Noetherian. Let $\mathfrak p = (f_1, \ldots, f_n)$
be the prime ideal corresponding to $\overline{T} \cap U_{\overline{k}}$.
Since $A_{\overline{k}} = A \otimes_k \overline{k}$
we see that there exists a finite subextension
$k \subset k' \subset \overline{k}$ such that each $f_i \in A_{k'}$.
It is clear that $\text{Gal}(\overline{k}/k')$
fixes $\overline{T}$, which proves (1). Part (2) follows as
by (1) applied to the situation over $k'$ the irreducible component
$\overline{T}$ is of the form $T'_{\overline{k}}$ for some
irreducible $T' \subset X_{k'}$.
To prove (3), let $T \subset X$ be an irreducible component.
Choose an irreducible component $\overline{T} \subset X_{\overline{k}}$
which maps to $T$, see
Lemma \ref{lemma-image-irreducible}.
By the above the orbit of $\overline{T}$ is finite, say it is
$\overline{T}_1, \ldots, \overline{T}_n$. Then
$\overline{T}_1 \cup \ldots \cup \overline{T}_n$
is a $\text{Gal}(\overline{k}/k)$-invariant closed subset of $X_{\overline{k}}$
hence of the form $W_{\overline{k}}$ for some $W \subset X$ closed by
Lemma \ref{lemma-closed-fixed-by-Galois}.
Clearly $W = T$ and we win.
\end{proof}











\section{Geometrically integral schemes}
\label{section-geometrically-integral}

\noindent
If $X$ is an irreducible scheme over a field, then it can happen that $X$
becomes reducible after extending the ground field. This does not happen
for geometrically irreducible schemes.

\begin{definition}
\label{definition-geometrically-integral}
Let $X$ be a scheme over the field $k$.
\begin{enumerate}
\item Let $x \in X$. We say $X$ is
{\it geometrically pointwise integral at $x$} if for every
field extension $k \subset k'$ and every $x' \in X_{k'}$ lying over $x$
the local ring $\mathcal{O}_{X_{k'}, x'}$ is integral.
\item We say $X$ is {\it geometrically pointwise integral} if $X$
is geometrically pointwise integral at every point.
\item We say $X$ is {\it geometrically integral} over $k$ if the scheme
$X_{k'}$ is integral for every field extension $k'$ of $k$.
\end{enumerate}
\end{definition}

\noindent
The distinction between notions (2) and (3) is necessary.
For example if $k = \mathbf{R}$ and $X = \text{Spec}(\mathbf{C}[x])$,
then $X$ is geometrically pointwise integral over $\mathbf{R}$ but
of course not geometrically integral.

\begin{lemma}
\label{lemma-geometrically-integral}
Let $k$ be a field.
Let $X$ be a scheme over $k$.
Then $X$ is geometrically integral over $k$ if and only if
$X$ is both geometrically reduced and geometrically irreducible
over $k$.
\end{lemma}

\begin{proof}
See Properties, Lemma \ref{properties-lemma-characterize-integral}.
\end{proof}





\section{Geometrically normal schemes}
\label{section-geometrically-normal}

\noindent
In Properties, Definition \ref{properties-definition-normal}
we have defined the notion of a normal scheme.
This notion is defined even for non-Noetherian
schemes. Hence, contrary to our discussion of
``geometrically regular'' schemes we consider all
field extensions of the ground field.

\begin{definition}
\label{definition-geometrically-normal}
Let $X$ be a scheme over the field $k$.
\begin{enumerate}
\item Let $x \in X$. We say $X$ is
{\it geometrically normal at $x$} if for every
field extension $k \subset k'$ and every $x' \in X_{k'}$ lying over $x$
the local ring $\mathcal{O}_{X_{k'}, x'}$ is normal.
\item We say $X$ is {\it geometrically normal} over $k$ if $X$
is geometrically normal at every $x \in X$.
\end{enumerate}
\end{definition}

\begin{lemma}
\label{lemma-geometrically-normal-at-point}
Let $k$ be a field.
Let $X$ be a scheme over $k$.
Let $x \in X$.
The following are equivalent
\begin{enumerate}
\item $X$ is geometrically normal at $x$,
\item for every finite purely inseparable field extension $k'$ of $k$
and $x' \in X_{k'}$ lying over over $x$ the local ring
$\mathcal{O}_{X_{k'}, x'}$ is normal, and
\item the ring $\mathcal{O}_{X, x}$ is geometrically
normal over $k$ (see
Algebra, Definition \ref{algebra-definition-geometrically-normal}).
\end{enumerate}
\end{lemma}

\begin{proof}
It is clear that (1) implies (2). Assume (2). Let $k \subset k'$ be a finite
purely inseparable field extension (for example $k = k'$). Consider the ring
$\mathcal{O}_{X, x} \otimes_k k'$.
By Algebra, Lemma \ref{algebra-lemma-p-ring-map}
its spectrum is the same as the spectrum of $\mathcal{O}_{X, x}$.
Hence it is a local ring also
(Algebra, Lemma \ref{algebra-lemma-characterize-local-ring}).
Therefore there is a unique point $x' \in X_{k'}$ lying over $x$
and $\mathcal{O}_{X_{k'}, x'} \cong \mathcal{O}_{X, x} \otimes_k k'$.
By assumption this is a normal ring. Hence we deduce (3) by
Algebra, Lemma
\ref{algebra-lemma-geometrically-normal}.

\medskip\noindent
Assume (3). Let $k \subset k'$ be a field extension. Since
$\text{Spec}(k') \to \text{Spec}(k)$ is surjective, also
$X_{k'} \to X$ is surjective
(Morphisms, Lemma \ref{morphisms-lemma-base-change-surjective}).
Let $x' \in X_{k'}$ be any point lying over $x$.
The local ring $\mathcal{O}_{X_{k'}, x'}$
is a localization of the ring $\mathcal{O}_{X, x} \otimes_k k'$.
Hence it is normal by assumption and (1) is proved.
\end{proof}

\begin{lemma}
\label{lemma-geometrically-normal}
Let $k$ be a field.
Let $X$ be a scheme over $k$.
The following are equivalent
\begin{enumerate}
\item $X$ is geometrically normal,
\item $X_{k'}$ is a normal scheme for every field extension $k \subset k'$,
\item $X_{k'}$ is a normal scheme for every finitely generated field
extension $k \subset k'$,
\item $X_{k'}$ is a normal scheme for every finite purely inseparable
field extension $k \subset k'$, and
\item for every affine open $U \subset X$ the ring $\mathcal{O}_X(U)$
is geometrically normal (see
Algebra, Definition \ref{algebra-definition-geometrically-normal}).
\end{enumerate}
\end{lemma}

\begin{proof}
Assume (1). Then for every field extension $k \subset k'$ and
every point $x' \in X_{k'}$ the local ring of $X_{k'}$ at $x'$
is normal. By definition this means that $X_{k'}$ is normal.
Hence (2).

\medskip\noindent
It is clear that (2) implies (3) implies (4).

\medskip\noindent
Assume (4) and let $U \subset X$ be an affine open subscheme.
Then $U_{k'}$ is a normal scheme for any finite purely inseparable
extension $k \subset k'$ (including $k = k'$). This means that
$k' \otimes_k \mathcal{O}(U)$ is a normal ring for all
finite purely inseparable extensions $k \subset k'$. Hence
$\mathcal{O}(U)$ is a geometrically normal $k$-algebra by definition.

\medskip\noindent
Assume (5). For any field extension $k \subset k'$ the base
change $X_{k'}$ is gotten by gluing the spectra of the
rings $\mathcal{O}_X(U) \otimes_k k'$ where $U$ is affine open
in $X$ (see Schemes, Section \ref{schemes-section-fibre-products}).
Hence $X_{k'}$ is normal. So (1) holds.
\end{proof}

\begin{lemma}
\label{lemma-geometrically-normal-upstairs}
Let $k$ be a field.
Let $X$ be a scheme over $k$.
Let $k'/k$ be a field extension.
Let $x \in X$ be a point, and let $x' \in X_{k'}$ be a point lying over $x$.
The following are equivalent
\begin{enumerate}
\item $X$ is geometrically normal at $x$,
\item $X_{k'}$ is geometrically normal at $x'$.
\end{enumerate}
In particular, $X$ is geometrically normal over $k$ if and only if
$X_{k'}$ is geometrically normal over $k'$.
\end{lemma}

\begin{proof}
It is clear that (1) implies (2). Assume (2).
Let $k \subset k''$ be a finite purely inseparable field extension
and let $x'' \in X_{k''}$ be a point lying over $x$ (actually it is
unique). We can find a common field extension $k \subset k'''$
(i.e.\ with both $k' \subset k'''$ and $k'' \subset k'''$) and a point
$x''' \in X_{k'''}$ lying over both $x'$ and $x''$.
Consider the map of local rings
$$
\mathcal{O}_{X_{k''}, x''} \longrightarrow \mathcal{O}_{X_{k'''}, x''''}.
$$
This is a flat local ring homomorphism and hence faithfully flat.
By (2) we see that the local ring on the right is normal.
Thus by Algebra, Lemma \ref{algebra-lemma-descent-normal}
we conlude that $\mathcal{O}_{X_{k''}, x''}$ is normal.
By Lemma \ref{lemma-geometrically-normal-at-point} we see that $X$
is geometrically normal at $x$.
\end{proof}










\section{Change of fields and locally Noetherian schemes}
\label{section-locally-Noetherian}

\noindent
Let $X$ a locally Noetherian scheme over a field $k$.
It is not always that case that $X_{k'}$ is locally Noetherian too.
For example if $X = \text{Spec}(\overline{\mathbf{Q}})$ and
$k = \mathbf{Q}$, then $X_{\overline{\mathbf{Q}}}$ is the spectrum
of $\overline{\mathbf{Q}} \otimes_{\mathbf{Q}} \overline{\mathbf{Q}}$
which is not Noetherian. (Hint: It has too many idempotents).
But if we only base change using finitely generated field extensions
then the Noetherian property is preserved. (Or if $X$ is locally of finite
type over $k$, since this proprety is preserved under base change.)

\begin{lemma}
\label{lemma-locally-Noetherian-base-change}
Let $k$ be a field.
Let $X$ be a scheme over $k$.
Let $k \subset k'$ be a finitely generated field extension.
Then $X$ is locally Noetherian if and only if $X_{k'}$ is locally
Noetherian.
\end{lemma}

\begin{proof}
Using Properties, Lemma \ref{properties-lemma-locally-Noetherian}
we reduce to the case where $X$ is
affine, say $X = \text{Spec}(A)$. In this case we have to prove that
$A$ is Noetherian if and only if $A_{k'}$ is Noetherian.
Since $A \to A_{k'} = k' \otimes_k A$ is faithfully flat, we see
that if $A_{k'}$ is Noetherian, then so is $A$, by
Algebra, Lemma \ref{algebra-lemma-descent-Noetherian}.
Conversely, if $A$ is Noetherian then $A_{k'}$ is Noetherian by
Algebra, Lemma \ref{algebra-lemma-Noetherian-field-extension}.
\end{proof}







\section{Geometrically regular schemes}
\label{section-geometrically-regular}

\noindent
A geometrically regular scheme over a field $k$ is a locally Noetherian
scheme over $k$ which remains regular upon suitable changes of base field.
A finite type scheme over $k$ is geometrically regular if and only
if it is smooth over $k$ (see Lemma \ref{lemma-geometrically-regular-smooth}).
The notion of geometric regularity is most interesting in situations
where smoothness cannot be used such as formal fibres (insert future
reference here).

\medskip\noindent
In the following definition we restrict ourselves to locally Noetherian
schemes, since the property of being a regular local ring is only
defined for Noetherian local rings. By Lemma \ref{lemma-geometrically-normal}
above, if we restrict ourselves to finitely generated field extensions then
this property is preserved under change of base field. This comment will be
used without further reference in this section. In particular the following
definition makes sense.

\begin{definition}
\label{definition-geometrically-regular}
Let $k$ be a field. Let $X$ be a locally Noetherian scheme over $k$.
\begin{enumerate}
\item Let $x \in X$. We say $X$ is {\it geometrically regular at $x$}
over $k$ if for every finitely generated field extension $k \subset k'$
and any $x' \in X_{k'}$ lying over $x$ the local ring
$\mathcal{O}_{X_{k'}, x'}$ is regular.
\item We say $X$ is {\it geometrically regular over $k$} if
$X$ is geometrically regular at all of its points.
\end{enumerate}
\end{definition}

\noindent
A similar definition works to define geometrically
Cohen-Macaulay, $(R_k)$, and $(S_k)$ schemes over a field.
We will add a section for these separately as needed.

\begin{lemma}
\label{lemma-geometrically-regular-at-point}
Let $k$ be a field.
Let $X$ be a locally Noetherian scheme over $k$.
Let $x \in X$.
The following are equivalent
\begin{enumerate}
\item $X$ is geometrically regular at $x$,
\item for every finite purely inseparable field extension $k'$ of $k$
and $x' \in X_{k'}$ lying over over $x$ the local ring
$\mathcal{O}_{X_{k'}, x'}$ is regular, and
\item the ring $\mathcal{O}_{X, x}$ is geometrically
regular over $k$ (see
Algebra, Definition \ref{algebra-definition-geometrically-regular}).
\end{enumerate}
\end{lemma}

\begin{proof}
It is clear that (1) implies (2).
Assume (2). This in particular implies that $\mathcal{O}_{X, x}$
is a regular local ring. Let $k \subset k'$ be a finite purely inseparable
field extension. Consider the ring $\mathcal{O}_{X, x} \otimes_k k'$.
By Algebra, Lemma \ref{algebra-lemma-p-ring-map}
its spectrum is the same as the spectrum of $\mathcal{O}_{X, x}$.
Hence it is a local ring also
(Algebra, Lemma \ref{algebra-lemma-characterize-local-ring}).
Therefore there is a unique point $x' \in X_{k'}$ lying over $x$
and $\mathcal{O}_{X_{k'}, x'} \cong \mathcal{O}_{X, x} \otimes_k k'$.
By assumption this is a regular ring. Hence we deduce (3)
from the definition of a geometrically regular ring.

\medskip\noindent
Assume (3). Let $k \subset k'$ be a field extension. Since
$\text{Spec}(k') \to \text{Spec}(k)$ is surjective, also
$X_{k'} \to X$ is surjective
(Morphisms, Lemma \ref{morphisms-lemma-base-change-surjective}).
Let $x' \in X_{k'}$ be any point lying over $x$.
The local ring $\mathcal{O}_{X_{k'}, x'}$
is a localization of the ring $\mathcal{O}_{X, x} \otimes_k k'$.
Hence it is regular by assumption and (1) is proved.
\end{proof}

\begin{lemma}
\label{lemma-geometrically-regular}
Let $k$ be a field.
Let $X$ be a locally Noetherian scheme over $k$.
The following are equivalent
\begin{enumerate}
\item $X$ is geometrically regular,
\item $X_{k'}$ is a regular scheme for every finitely generated field
extension $k \subset k'$,
\item $X_{k'}$ is a regular scheme for every finite purely inseparable
field extension $k \subset k'$, and
\item for every affine open $U \subset X$ the ring $\mathcal{O}_X(U)$
is geometrically regular (see
Algebra, Definition \ref{algebra-definition-geometrically-regular}).
\end{enumerate}
\end{lemma}

\begin{proof}
Assume (1). Then for every finitely generated field extension
$k \subset k'$ and
every point $x' \in X_{k'}$ the local ring of $X_{k'}$ at $x'$
is regular. By Properties, Lemma \ref{properties-lemma-characterize-regular}
this means that $X_{k'}$ is regular. Hence (2).

\medskip\noindent
It is clear that (2) implies (3).

\medskip\noindent
Assume (3) and let $U \subset X$ be an affine open subscheme.
Then $U_{k'}$ is a regular scheme for any finite purely inseparable
extension $k \subset k'$ (including $k = k'$). This means that
$k' \otimes_k \mathcal{O}(U)$ is a regular ring for all
finite purely inseparable extensions $k \subset k'$. Hence
$\mathcal{O}(U)$ is a geometrically regular $k$-algebra by definition.

\medskip\noindent
Assume (5). For any field extension $k \subset k'$ the base
change $X_{k'}$ is gotten by gluing the spectra of the
rings $\mathcal{O}_X(U) \otimes_k k'$ where $U$ is affine open
in $X$ (see Schemes, Section \ref{schemes-section-fibre-products}).
Hence $X_{k'}$ is regular. So (1) holds.
\end{proof}

\begin{lemma}
\label{lemma-geometrically-regular-upstairs}
Let $k$ be a field.
Let $X$ be a scheme over $k$.
Let $k'/k$ be a finitely generated field extension.
Let $x \in X$ be a point, and let $x' \in X_{k'}$ be a point lying over $x$.
The following are equivalent
\begin{enumerate}
\item $X$ is geometrically regular at $x$,
\item $X_{k'}$ is geometrically regular at $x'$.
\end{enumerate}
In particular, $X$ is geometrically regular over $k$ if and only if
$X_{k'}$ is geometrically regular over $k'$.
\end{lemma}

\begin{proof}
It is clear that (1) implies (2). Assume (2).
Let $k \subset k''$ be a finite purely inseparable field extension
and let $x'' \in X_{k''}$ be a point lying over $x$ (actually it is
unique). We can find a common, finitely generated, field extension
$k \subset k'''$ (i.e.\ with both $k' \subset k'''$ and $k'' \subset k'''$)
and a point $x''' \in X_{k'''}$ lying over both $x'$ and $x''$.
Consider the map of local rings
$$
\mathcal{O}_{X_{k''}, x''} \longrightarrow \mathcal{O}_{X_{k'''}, x''''}.
$$
This is a flat local ring homomorphism of Noetherian local rings
and hence faithfully flat.
By (2) we see that the local ring on the right is regular.
Thus by Algebra, Lemma \ref{algebra-lemma-flat-under-regular}
we conlude that $\mathcal{O}_{X_{k''}, x''}$ is regular.
By Lemma \ref{lemma-geometrically-regular-at-point} we see that $X$
is geometrically regular at $x$.
\end{proof}

\begin{lemma}
\label{lemma-geometrically-regular-smooth}
Let $k$ be a field.
Let $X$ be a scheme of finite type over $k$.
Let $x \in X$.
Then $X$ is geometrically regular at $x$ if and only if $X \to \text{Spec}(k)$
is smooth at $x$ (Morphisms, Definition \ref{morphisms-definition-smooth}).
\end{lemma}

\begin{proof}
The question is local around $x$,
hence we may assume that $X = \text{Spec}(A)$
for some finite type $k$-algebra.
Let $x$ correspond to the prime $\mathfrak p$.

\medskip\noindent
If $A$ is smooth over $k$ at $\mathfrak p$, then we may localize $A$
and assume that $A$ is smooth over $k$. In this case $k' \otimes_k A$
is smooth over $k'$ for all extension fields $k'/k$, and each of
these Noetherian rings is regular by
Algebra, Lemma \ref{algebra-lemma-characterize-smooth-over-field}.

\medskip\noindent
Assume $X$ is geometrically regular at $x$.
Consider the residue field $K := \kappa(x) = \kappa(\mathfrak p)$ of $x$.
It is a finitely generated extension of $k$.
By Algebra, Lemma \ref{algebra-lemma-make-separable}
there exists a finite purely inseparable
extension $k \subset k'$ such that the compositum
$k'K$ is a separable field extension of $k'$.
Let $\mathfrak p' \subset A' = k' \otimes_k A$ be a prime ideal
lying over $\mathfrak p$. It is the unique prime lying over $\mathfrak p$, see
Algebra, Lemma \ref{algebra-lemma-p-ring-map}.
Hence the residue field $K' := \kappa(\mathfrak p')$
is the compositum $k'K$. By assumption the local ring
$(A')_{\mathfrak p'}$ is regular. Hence by
Algebra, Lemma \ref{algebra-lemma-separable-smooth}
we see that $k' \to A'$ is smooth at $\mathfrak p'$.
This in turn implies that $k \to A$ is smooth at $\mathfrak p$ by
Algebra, Lemma \ref{algebra-lemma-smooth-field-change-local}.
The lemma is proved.
\end{proof}


\begin{example}
\label{example-geometrically-reduced-not-normal}
Let $k =\mathbf{F}_p(t)$. It is quite easy to give an example of a regular
variety $V$ over $k$ which is not geometrically reduced. For example we
can take $\text{Spec}(k[x]/(x^p - t))$. In fact, there exists an
example of a regular variety $V$ which is geometrically reduced, but
not even geometrically normal. Namely, take for $p > 2$ the scheme
$V = \text{Spec}(k[x, y]/(y^2 - x^p + t))$. This is a variety as the
polynomial $y^2 - x^p + t \in k[x, y]$ is irreducible.
The morphism $V \to \text{Spec}(k)$ is smooth at all points
except at the point $v_0 \in V$ corresponding to the maximal ideal
$(y, x^p - t)$ (because $2y$ is invertible). In particular we see that
$V$ is (geometrically) regular at all points, except possibly $v_0$.
The local ring
$$
\mathcal{O}_{V, v_0} = \left(k[x, y]/(y^2 - x^p + t)\right)_{(y, x^p - t)}
$$
is a domain of dimension $1$. Its maximal ideal is generated by $1$ element,
namely $x^p - t$. Hence it is a discrete valuation ring and regular.
Let $k' = k[t^{1/p}]$. Denote $t' = t^{1/p} \in k'$,
$V' = V_{k'}$, $v'_0 \in V'$ the unique point lying over $v_0$.
Over $k'$ we can write $x^p - t = (x - t')^p$, but the polynomial
$y^2 - (x - t')^p$ is still irreducible and $V'$ is still a variety.
But the element
$$
\frac{y}{x - t'} \in f.f.(\mathcal{O}_{V', v'_0})
$$
is integral over $\mathcal{O}_{V', v'_0}$ (just compute its square)
and not contained in it, so $V'$ is not normal at $v'_0$. This concludes
the example.
\end{example}







\section{Change of fields and the Cohen-Macaulay property}
\label{section-CM}

\noindent
The following lemma says that it does not make sense to define
geometrically Cohen-Macaulay schemes, since these would be the
same as Cohen-Macaulay schemes.

\begin{lemma}
\label{lemma-CM-base-change}
Let $X$ be a locally Noetherian scheme over the field $k$.
Let $k \subset k'$ be a finitely generated field extension.
Let $x \in X$ be a point, and let $x' \in X_{k'}$ be a point lying
over $x$. Then we have
$$
\mathcal{O}_{X, x}\text{ is Cohen-Macaulay}
\Leftrightarrow
\mathcal{O}_{X_{k'}, x'}\text{ is Cohen-Macaulay}
$$
If $X$ is locally of finite type over $k$, the same holds for any
field extension $k \subset k'$.
\end{lemma}

\begin{proof}
The first case of the lemma follows from
Algebra, Lemma \ref{algebra-lemma-CM-geometrically-CM}.
The second case of the lemma is equivalent to
Algebra, Lemma \ref{algebra-lemma-extend-field-CM-locus}. 
\end{proof}







\section{Change of fields and the Jacobson property}
\label{section-overfield}

\noindent
A scheme locally of finite type over a field has plenty of closed
points, namely it is Jacobson. Moreover, the residue fields are
finite extensions of the ground field.

\begin{lemma}
\label{lemma-locally-finite-type-Jacobson}
Let $X$ be a scheme which is locally of finite type over $k$.
Then
\begin{enumerate}
\item for any closed point $x \in X$ the extension $k \subset \kappa(x)$
is algebraic, and
\item $X$ is a Jacobson scheme
(Properties, Definition \ref{properties-definition-jacobson}).
\end{enumerate}
\end{lemma}

\begin{proof}
A scheme is Jacobson if and only if it has an affine open covering
by Jacobson schemes, see
Properties, Lemma \ref{properties-lemma-locally-jacobson}.
The property on residue fields at closed points is also local on $X$.
Hence we may assume that $X$ is affine. In this case the result
is a consequence of the Hilbert Nullstellenstaz, see
Algebra, Theorem \ref{algebra-theorem-nullstellensatz}.
It also follows from a combination of
Morphisms, Lemmas \ref{morphisms-lemma-jacobson-finite-type-points},
\ref{morphisms-lemma-Jacobson-universally-Jacobson}, and
\ref{morphisms-lemma-ubiquity-Jacobson-schemes}.
\end{proof}

\noindent
It turns out that if $X$ is not locally of finite type, then we can
achieve the same result after making a suitably large base field extension.

\begin{lemma}
\label{lemma-make-Jacobson}
Let $X$ be a scheme over a field $k$.
For any field extension $k \subset K$ whose cardinality is large enough
we have
\begin{enumerate}
\item for any closed point $x \in X_K$ the extension $K \subset \kappa(x)$
is algebraic, and
\item $X_K$ is a Jacobson scheme
(Properties, Definition \ref{properties-definition-jacobson}).
\end{enumerate}
\end{lemma}

\begin{proof}
Choose an affine open covering $X = \bigcup U_i$.
By
Algebra, Lemma \ref{algebra-lemma-base-change-Jacobson}
and
Properties, Lemma \ref{properties-lemma-affine-jacobson}
there exist cardinals $\kappa_i$ such that $U_{i, K}$ has
the desired properties over $K$ if $\#(K) \geq \kappa_i$.
Set $\kappa = \max\{\kappa_i\}$. Then if the cardinality of
$K$ is larger than $\kappa$ we see that each $U_{i, K}$ satisfies
the conclusions of the lemma. Hence $X_K$ is Jacobson by
Properties, Lemma \ref{properties-lemma-locally-jacobson}.
The statement on residue fields at closed points of $X_K$
follows from the corresponding
statements for residue fields of closed points of the $U_{i, K}$.
\end{proof}





\section{Miscellany}
\label{section-miscellany}

\begin{lemma}
\label{lemma-closure-of-product}
Let $k$ be a field.
Let $X$, $Y$ be schemes over $k$, and let
$A \subset X$, $B \subset Y$ be subsets.
Set
$$
AB =
\{z \in X \times_k Y \mid
\text{pr}_X(\gamma) \in A,\ \text{pr}_Y(\gamma) \in B\}
\subset X \times_k Y
$$
Then set theoretically we have
$$
\overline{A} \times_k \overline{B} = \overline{AB}
$$
\end{lemma}

\begin{proof}
The inclusion $\overline{AB} \subset \overline{A} \times_k \overline{B}$
is immediate.
We may replace $X$ and $Y$ by the reduced closed subschemes $\overline{A}$
and $\overline{B}$.
Let $W \subset X \times_k Y$ be a nonempty open subset. By
Morphisms, Lemma \ref{morphisms-lemma-scheme-over-field-universally-open}
the subset $U = \text{pr}_X(W)$ is nonempty open in $X$. 
Hence $A \cap U$ is nonempty. Pick $a \in A \cap U$.
Denote $Y_{\kappa(a)} = \{a\} \times_k Y$
the fibre of $\text{pr}_X : X \times_k Y \to X$ over $a$. By
Morphisms, Lemma \ref{morphisms-lemma-scheme-over-field-universally-open}
again the morphism $Y_a \to Y$ is open as
$\text{Spec}(\kappa(a)) \to \text{Spec}(k)$ is universally open.
Hence the nonempty open
subset $W_a = W \times_{X \times_k Y} Y_a$
maps to a nonempty open subset of $Y$.
We conclude there exists a $b \in B$ in the image.
Hence $AB \cap W \not = \emptyset$ as desired.
\end{proof}





\section{Types of varieties}
\label{section-types}

\noindent
Short section discussion some elementary global properties of varieties.

\begin{definition}
\label{definition-variety-type}
Let $k$ be a field. Let $X$ be a variety over $k$.
\begin{enumerate}
\item We say $X$ is an {\it affine variety} if $X$ is an affine scheme.
This is equivalent to requiring $X$ it be isomorphic to a closed
subscheme of $\mathbf{A}^n_k$ for some $n$.
\item We say $X$ is a {\it projective variety} if the
structure morphism $X \to \text{Spec}(k)$ is projective. By
Morphisms, Lemma \ref{morphisms-lemma-characterize-locally-projective}
this is true if and only if $X$ is isomorphic to a closed
subscheme of $\mathbf{P}^n_k$ for some $n$.
\item We say $X$ is a {\it quasi-projective variety} if
the structure morphism $X \to \text{Spec}(k)$ is quasi-projective. By
Morphisms, Lemma \ref{morphisms-lemma-characterize-locally-quasi-projective}
this is true if and only if $X$ is isomorphic to a
locally closed subscheme of $\mathbf{P}^n_k$ for some $n$.
\item A {\it proper variety} is a variety such that the
morphism $X \to \text{Spec}(k)$ is proper.
\end{enumerate}
\end{definition}

\noindent
Note that a projective variety is a proper variety, see
Morphisms, Lemma \ref{morphisms-lemma-locally-projective-proper}.
Also, an affine variety is quasi-projective as $\mathbf{A}^n_k$
is isomorphic to an open subscheme of $\mathbf{P}^n_k$, see
Constructions,
Lemma \ref{constructions-lemma-standard-covering-projective-space}.

\begin{lemma}
\label{lemma-regular-functions-proper-variety}
Let $X$ be a proper variety over $k$.
Then $\Gamma(X, \mathcal{O}_X)$ is a field which is
a finite extension of the field $k$.
\end{lemma}

\begin{proof}
By
Coherent, Lemma \ref{coherent-lemma-proper-pushforward-coherent}
we see that $\Gamma(X, \mathcal{O}_X)$ is a finite dimensional
$k$-vector space. It is also a $k$-algebra without zero-divisors.
Hence it is a field, see
Algebra, Lemma \ref{algebra-lemma-integral-over-field}.
\end{proof}




\section{Groups of invertible functions}
\label{section-units}

\noindent
It is often (but not always) the case that $\mathcal{O}^*(X)/k^*$
is a finitely generated abelian group if $X$ is a variety over $k$.
We show this by a series of lemmas.
Everything rests on the following special case.

\begin{lemma}
\label{lemma-open-in-normal-proper}
Let $k$ be an algebraically closed field.
Let $\overline{X}$ be a proper variety over $k$.
Let $X \subset \overline{X}$ be an open subscheme.
Assume $X$ is normal.
Then $\mathcal{O}^*(X)/k^*$ is a finitely generated abelian group.
\end{lemma}

\begin{proof}
We will use without further mention that for any affine open $U$ of
$\overline{X}$ the ring $\mathcal{O}(U)$ is a finitely generated
$k$-algebra, which is Noetherian, a domain and normal, see
Algebra, Lemma \ref{algebra-lemma-Noetherian-permanence},
Properties, Definition \ref{properties-definition-integral},
Properties, Lemmas \ref{properties-lemma-locally-Noetherian} and
\ref{properties-lemma-locally-normal},
Morphisms, Lemma \ref{morphisms-lemma-locally-finite-type-characterize}.

\medskip\noindent
Let $\xi_1, \ldots, \xi_r$ be the generic points of the complement of $X$
in $\overline{X}$. There are finitely many since $\overline{X}$ has a
Noetherian underlying topological space (see
Morphisms, Lemma \ref{morphisms-lemma-finite-type-noetherian},
Properties, Lemma \ref{properties-lemma-Noetherian-topology}, and
Topology, Lemma \ref{topology-lemma-Noetherian}).
For each $i$ the local ring $\mathcal{O}_i = \mathcal{O}_{X, \xi_i}$
is a normal Noetherian local domain (as a localization of a
Noetherian normal domain). Let $J \subset \{1, \ldots, r\}$ be the set of
indices $i$ such that $\dim(\mathcal{O}_i) = 1$. For $j \in J$ the
local ring $\mathcal{O}_j$ is a discrete valuation ring, see
Algebra, Lemma \ref{algebra-lemma-characterize-dvr}.
Hence we obtain a valuation
$$
v_j : k(\overline{X})^* \longrightarrow \mathbf{Z}
$$
with the property that $v_j(f) \geq 0 \Leftrightarrow f \in \mathcal{O}_j$.

\medskip\noindent
Think of $\mathcal{O}(X)$ as a sub $k$-algebra of $k(X) = k(\overline{X})$.
We claim that the kernel of the map
$$
\mathcal{O}(X)^* \longrightarrow
\prod\nolimits_{j \in J} \mathbf{Z},
\quad
f \longmapsto \prod v_j(f)
$$
is $k^*$. It is clear that this claim proves the lemma.
Namely, suppose that $f \in \mathcal{O}(X)$ is an element of the kernel.
Let $U = \text{Spec}(B) \subset \overline{X}$ be any affine open.
Then $B$ is a Noetherian normal domain.
For every height one prime $\mathfrak q \subset B$ with corresponding
point $\xi \in X$ we see that either $\xi = \xi_j$ for some $j \in J$
or that $\xi \in X$. The reason is that
$\text{codim}(\overline{\{\xi\}}, \overline{X}) = 1$ by
Properties, Lemma \ref{properties-lemma-codimension-local-ring}
and hence if $\xi \in \overline{X} \setminus X$ it must be a
generic point of $\overline{X} \setminus X$, hence equal to some
$\xi_j$, $j \in J$.
We conclude that $f \in \mathcal{O}_{X, \xi} = B_{\mathfrak q}$
in either case as $f$ is in the kernel of the map. Thus
$f \in \bigcap_{\text{ht}(\mathfrak q) = 1} B_{\mathfrak q} = B$, see
Algebra, Lemma
\ref{algebra-lemma-normal-domain-intersection-localizations-height-1}.
In other words, we see that $f \in \Gamma(X, \mathcal{O}_X)$.
But since $k$ is algebraically closed we conclude that
$f \in k$ by
Lemma \ref{lemma-regular-functions-proper-variety}.
\end{proof}

\noindent
Next, we generalize the case above by some elementary arguments, still
keeping the field algebraically closed.

\begin{lemma}
\label{lemma-units-integral-finite-type-algebraically-closed}
Let $k$ be an algebraically closed field.
Let $X$ be an integral scheme locally of finite type over $k$.
Then $\mathcal{O}^*(X)/k^*$ is a finitely generated abelian group.
\end{lemma}

\begin{proof}
As $X$ is integral the restriction mapping
$\mathcal{O}(X) \to \mathcal{O}(U)$ is injective for any
nonempty open subscheme $U \subset X$. Hence we may assume
that $X$ is affine. Choose a closed immersion
$X \to \mathbf{A}^n_k$
and denote $\overline{X}$ the closure of $X$ in $\mathbf{P}^n_k$
via the usual immersion $\mathbf{A}^n_k \to \mathbf{P}^n_k$.
Thus we may assume that $X$ is an affine open of a projective
variety $\overline{X}$.

\medskip\noindent
Let $\nu : \overline{X}^\nu \to \overline{X}$ be the normalization
morphism, see
Morphisms, Definition \ref{morphisms-definition-normalization}.
We know that $\nu$ is finite, dominant, and that $\overline{X}^\nu$
is a normal irreducible scheme, see
Morphisms, Lemmas \ref{morphisms-lemma-normalization-normal},
\ref{morphisms-lemma-Japanese-normalization}, and
\ref{morphisms-lemma-ubiquity-nagata}.
It follows that $\overline{X}^\nu$ is a proper variety,
because $\overline{X} \to \text{Spec}(k)$ is proper as a composition
of a finite and a proper morphism (see results in
Morphisms, Sections \ref{morphisms-section-proper} and
\ref{morphisms-section-integral}).
It also follows that $\nu$ is a surjective morphism, because
the image of $\nu$ is closed and contains the generic point of $\overline{X}$.
Hence setting $X^\nu = \nu^{-1}(X)$ we see that it suffices to prove the
result for $X^\nu$. In other words, we may assume that $X$ is a nonempty
open of a normal proper variety $\overline{X}$. This case is handled by
Lemma \ref{lemma-open-in-normal-proper}.
\end{proof}

\noindent
The preceding lemma implies the following slight generalization.

\begin{lemma}
\label{lemma-units-general-algebraically-closed}
Let $k$ be an algebraically closed field.
Let $X$ be a connected reduced scheme which is locally of finite type
over $k$ with finitely many irreducible components.
Then $\mathcal{O}^*(X)/k^*$ is a finitely generated abelian group.
\end{lemma}

\begin{proof}
Let $X = \bigcup X_i$ be the irreducible components. By
Lemma \ref{lemma-units-integral-finite-type-algebraically-closed}
we see that $\mathcal{O}(X_i)^*/k^*$ is a finitely generated
abelian group. Let $f \in \mathcal{O}(X)^*$ be in the kernel
of the map
$$
\mathcal{O}(X)^* \longrightarrow \prod \mathcal{O}(X_i)^*/k^*.
$$
Then for each $i$ there exists an element $\lambda_i \in k$
such that $f|_{X_i} = \lambda_i$.
By restricting to $X_i \cap X_j$ we conclude that
$\lambda_i = \lambda_j$ if $X_i \cap X_j \not = \emptyset$.
Since $X$ is connected we conclude that all $\lambda_i$ agree
and hence that $f \in k^*$. This proves that
$$
\mathcal{O}(X)^*/k^* \subset \prod \mathcal{O}(X_i)^*/k^*
$$
and the lemma follows as on the right we have a product of finitely
many finitely generated abelian groups.
\end{proof}

\begin{lemma}
\label{lemma-units-general-rational-point}
Let $k$ be a field.
Let $X$ be locally of finite type over $k$, reduced, with finitely
many irreducible components and with a $k$-rational point.
Then $\mathcal{O}(X)^*/k^*$ is a finitely generated abelian group.
\end{lemma}

\begin{proof}
Let $e : \text{Spec}(k) \to X$ be the $k$-rational point.
Let $\overline{k}$ be an algebraic closure of $k$.
Denote $Y = (X_{\overline{k}})_{red}$ the reduction of the base
change of $X$ to $\overline{k}$. Note that $Y \to X$ is a surjective
morphism and that $Y$ is locally of finite type over $\overline{k}$.
Denote
$\overline{e} : \text{Spec}(\overline{k}) \to Y$
the base change of $e$. Then we have the following commutive diagram
$$
\xymatrix{
X \ar[d] & Y \ar[l] \ar[d] \\
\text{Spec}(k) \ar@/^1pc/[u]^e &
\text{Spec}(\overline{k}) \ar[l] \ar@/_1pc/[u]_{\overline{e}}
}
$$
Now by
Lemma \ref{lemma-galois-action-irreducible-components-locally-finite-type}
we see that the scheme $Y$ has finitely many irreducible components,
and by
Lemma \ref{lemma-geometrically-connected-if-connected-and-point}
we see that $Y$ is connected.
Hence
Lemma \ref{lemma-units-general-algebraically-closed}
applies to $Y$. Thus
$\mathcal{O}(Y)^*/\overline{k}^*$ is a finitely
generated abelian group. Suppose that
$f \in \mathcal{O}(X)^*$ maps to the trivial element of
$\mathcal{O}(Y)^*/\overline{k}^*$. Then we see that
$e(f) = \overline{e}(f|_Y)$ and hence we see that $f \in k^*$.
In other words, we have shown that $\mathcal{O}(X)^*/k^*$
maps injectively into the group $\mathcal{O}(Y)^*/\overline{k}^*$
and we win.
\end{proof}

\begin{lemma}
\label{lemma-units-variety}
Let $k$ be a field.
Let $X$ be a variety over $k$.
In each of the following cases the group $\mathcal{O}(X)^*/k^*$
is a finitely generated abelian group:
\begin{enumerate}
\item If $k$ is algebraically closed in $k(X)$,
\item if $X$ is geometrically integral over $k$, and
\item if $k$ is the ``intersection'' of the field extensions
$k \subset \kappa(x)$ where $x$ runs over the closed points of $x$.
\end{enumerate}
\end{lemma}

\begin{proof}
Omitted. Hint: let $k \subset \overline{k}$ be an algebraic closure.
Let $Y$ be an irreducible component of $X_{\overline{k}}$ seen as
a variety over $\overline{k}$. Note that $Y \to X$ is surjective, see
Lemma \ref{lemma-image-irreducible}.
Consider the map
$\mathcal{O}(X)^*/k^* \to \mathcal{O}(Y)^*/\overline{k}^*$.
Since $\mathcal{O}(X) \subset \mathcal{O}(Y)$ we see that the
only ``problem'' is the elements of $\mathcal{O}(X)$ which
are not in $k$ but end up in $\overline{k} \subset \mathcal{O}(Y)$.
\end{proof}



\section{Other chapters}

\begin{multicols}{2}
\begin{enumerate}
\item \hyperref[introduction-section-phantom]{Introduction}
\item \hyperref[conventions-section-phantom]{Conventions}
\item \hyperref[sets-section-phantom]{Set Theory}
\item \hyperref[categories-section-phantom]{Categories}
\item \hyperref[topology-section-phantom]{Topology}
\item \hyperref[sheaves-section-phantom]{Sheaves on Spaces}
\item \hyperref[algebra-section-phantom]{Commutative Algebra}
\item \hyperref[sites-section-phantom]{Sites and Sheaves}
\item \hyperref[homology-section-phantom]{Homological Algebra}
\item \hyperref[derived-section-phantom]{Derived Categories}
\item \hyperref[more-algebra-section-phantom]{More Algebra}
\item \hyperref[simplicial-section-phantom]{Simplicial Methods}
\item \hyperref[modules-section-phantom]{Sheaves of Modules}
\item \hyperref[sites-modules-section-phantom]{Modules on Sites}
\item \hyperref[injectives-section-phantom]{Injectives}
\item \hyperref[cohomology-section-phantom]{Cohomology of Sheaves}
\item \hyperref[sites-cohomology-section-phantom]{Cohomology on Sites}
\item \hyperref[hypercovering-section-phantom]{Hypercoverings}
\item \hyperref[schemes-section-phantom]{Schemes}
\item \hyperref[constructions-section-phantom]{Constructions of Schemes}
\item \hyperref[properties-section-phantom]{Properties of Schemes}
\item \hyperref[morphisms-section-phantom]{Morphisms of Schemes}
\item \hyperref[coherent-section-phantom]{Coherent Cohomology}
\item \hyperref[divisors-section-phantom]{Divisors}
\item \hyperref[limits-section-phantom]{Limits of Schemes}
\item \hyperref[varieties-section-phantom]{Varieties}
\item \hyperref[chow-section-phantom]{Chow Homology}
\item \hyperref[topologies-section-phantom]{Topologies on Schemes}
\item \hyperref[descent-section-phantom]{Descent}
\item \hyperref[more-morphisms-section-phantom]{More on Morphisms}
\item \hyperref[flat-section-phantom]{More on Flatness}
\item \hyperref[groupoids-section-phantom]{Groupoid Schemes}
\item \hyperref[more-groupoids-section-phantom]{More on Groupoid Schemes}
\item \hyperref[etale-section-phantom]{\'Etale Morphisms of Schemes}
\item \hyperref[etale-cohomology-section-phantom]{\'Etale Cohomology}
\item \hyperref[spaces-section-phantom]{Algebraic Spaces}
\item \hyperref[spaces-properties-section-phantom]{Properties of Algebraic Spaces}
\item \hyperref[spaces-morphisms-section-phantom]{Morphisms of Algebraic Spaces}
\item \hyperref[spaces-topologies-section-phantom]{Topologies on Algebraic Spaces}
\item \hyperref[spaces-descent-section-phantom]{Descent and Algebraic Spaces}
\item \hyperref[spaces-more-morphisms-section-phantom]{More on Morphisms of Spaces}
\item \hyperref[quot-section-phantom]{Quot and Hilbert Spaces}
\item \hyperref[stacks-section-phantom]{Stacks}
\item \hyperref[spaces-groupoids-section-phantom]{Groupoids in Algebraic Spaces}
\item \hyperref[spaces-more-groupoids-section-phantom]{More on Groupoids in Spaces}
\item \hyperref[bootstrap-section-phantom]{Bootstrap}
\item \hyperref[examples-stacks-section-phantom]{Examples of Stacks}
\item \hyperref[groupoids-quotients-section-phantom]{Quotients of Groupoids}
\item \hyperref[algebraic-section-phantom]{Algebraic Stacks}
\item \hyperref[criteria-section-phantom]{Criteria for Representability}
\item \hyperref[stacks-properties-section-phantom]{Properties of Algebraic Stacks}
\item \hyperref[stacks-morphisms-section-phantom]{Morphisms of Algebraic Stacks}
\item \hyperref[examples-section-phantom]{Examples}
\item \hyperref[exercises-section-phantom]{Exercises}
\item \hyperref[guide-section-phantom]{Guide to Literature}
\item \hyperref[desirables-section-phantom]{Desirables}
\item \hyperref[coding-section-phantom]{Coding Style}
\item \hyperref[fdl-section-phantom]{GNU Free Documentation License}
\item \hyperref[index-section-phantom]{Auto Generated Index}
\end{enumerate}
\end{multicols}


\bibliography{my}
\bibliographystyle{amsalpha}

\end{document}
