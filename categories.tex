\IfFileExists{stacks-project.cls}{%
\documentclass{stacks-project}
}{%
\documentclass{amsart}
}

% The following AMS packages are automatically loaded with
% the amsart documentclass:
%\usepackage{amsmath}
%\usepackage{amssymb}
%\usepackage{amsthm}

% For dealing with references we use the comment environment
\usepackage{verbatim}
\newenvironment{reference}{\comment}{\endcomment}
%\newenvironment{reference}{}{}
\newenvironment{slogan}{\comment}{\endcomment}
\newenvironment{history}{\comment}{\endcomment}

% For commutative diagrams you can use
% \usepackage{amscd}
\usepackage[all]{xy}

% We use 2cell for 2-commutative diagrams.
\xyoption{2cell}
\UseAllTwocells

% To put source file link in headers.
% Change "template.tex" to "this_filename.tex"
% \usepackage{fancyhdr}
% \pagestyle{fancy}
% \lhead{}
% \chead{}
% \rhead{Source file: \url{template.tex}}
% \lfoot{}
% \cfoot{\thepage}
% \rfoot{}
% \renewcommand{\headrulewidth}{0pt}
% \renewcommand{\footrulewidth}{0pt}
% \renewcommand{\headheight}{12pt}

\usepackage{multicol}

% For cross-file-references
\usepackage{xr-hyper}

% Package for hypertext links:
\usepackage{hyperref}

% For any local file, say "hello.tex" you want to link to please
% use \externaldocument[hello-]{hello}
\externaldocument[introduction-]{introduction}
\externaldocument[conventions-]{conventions}
\externaldocument[sets-]{sets}
\externaldocument[categories-]{categories}
\externaldocument[topology-]{topology}
\externaldocument[sheaves-]{sheaves}
\externaldocument[sites-]{sites}
\externaldocument[stacks-]{stacks}
\externaldocument[fields-]{fields}
\externaldocument[algebra-]{algebra}
\externaldocument[brauer-]{brauer}
\externaldocument[homology-]{homology}
\externaldocument[derived-]{derived}
\externaldocument[simplicial-]{simplicial}
\externaldocument[more-algebra-]{more-algebra}
\externaldocument[smoothing-]{smoothing}
\externaldocument[modules-]{modules}
\externaldocument[sites-modules-]{sites-modules}
\externaldocument[injectives-]{injectives}
\externaldocument[cohomology-]{cohomology}
\externaldocument[sites-cohomology-]{sites-cohomology}
\externaldocument[dga-]{dga}
\externaldocument[dpa-]{dpa}
\externaldocument[hypercovering-]{hypercovering}
\externaldocument[schemes-]{schemes}
\externaldocument[constructions-]{constructions}
\externaldocument[properties-]{properties}
\externaldocument[morphisms-]{morphisms}
\externaldocument[coherent-]{coherent}
\externaldocument[divisors-]{divisors}
\externaldocument[limits-]{limits}
\externaldocument[varieties-]{varieties}
\externaldocument[topologies-]{topologies}
\externaldocument[descent-]{descent}
\externaldocument[perfect-]{perfect}
\externaldocument[more-morphisms-]{more-morphisms}
\externaldocument[flat-]{flat}
\externaldocument[groupoids-]{groupoids}
\externaldocument[more-groupoids-]{more-groupoids}
\externaldocument[etale-]{etale}
\externaldocument[chow-]{chow}
\externaldocument[intersection-]{intersection}
\externaldocument[pic-]{pic}
\externaldocument[adequate-]{adequate}
\externaldocument[dualizing-]{dualizing}
\externaldocument[duality-]{duality}
\externaldocument[discriminant-]{discriminant}
\externaldocument[local-cohomology-]{local-cohomology}
\externaldocument[curves-]{curves}
\externaldocument[resolve-]{resolve}
\externaldocument[models-]{models}
\externaldocument[pione-]{pione}
\externaldocument[etale-cohomology-]{etale-cohomology}
\externaldocument[proetale-]{proetale}
\externaldocument[crystalline-]{crystalline}
\externaldocument[spaces-]{spaces}
\externaldocument[spaces-properties-]{spaces-properties}
\externaldocument[spaces-morphisms-]{spaces-morphisms}
\externaldocument[decent-spaces-]{decent-spaces}
\externaldocument[spaces-cohomology-]{spaces-cohomology}
\externaldocument[spaces-limits-]{spaces-limits}
\externaldocument[spaces-divisors-]{spaces-divisors}
\externaldocument[spaces-over-fields-]{spaces-over-fields}
\externaldocument[spaces-topologies-]{spaces-topologies}
\externaldocument[spaces-descent-]{spaces-descent}
\externaldocument[spaces-perfect-]{spaces-perfect}
\externaldocument[spaces-more-morphisms-]{spaces-more-morphisms}
\externaldocument[spaces-flat-]{spaces-flat}
\externaldocument[spaces-groupoids-]{spaces-groupoids}
\externaldocument[spaces-more-groupoids-]{spaces-more-groupoids}
\externaldocument[bootstrap-]{bootstrap}
\externaldocument[spaces-pushouts-]{spaces-pushouts}
\externaldocument[groupoids-quotients-]{groupoids-quotients}
\externaldocument[spaces-more-cohomology-]{spaces-more-cohomology}
\externaldocument[spaces-simplicial-]{spaces-simplicial}
\externaldocument[spaces-duality-]{spaces-duality}
\externaldocument[formal-spaces-]{formal-spaces}
\externaldocument[restricted-]{restricted}
\externaldocument[spaces-resolve-]{spaces-resolve}
\externaldocument[formal-defos-]{formal-defos}
\externaldocument[defos-]{defos}
\externaldocument[cotangent-]{cotangent}
\externaldocument[examples-defos-]{examples-defos}
\externaldocument[algebraic-]{algebraic}
\externaldocument[examples-stacks-]{examples-stacks}
\externaldocument[stacks-sheaves-]{stacks-sheaves}
\externaldocument[criteria-]{criteria}
\externaldocument[artin-]{artin}
\externaldocument[quot-]{quot}
\externaldocument[stacks-properties-]{stacks-properties}
\externaldocument[stacks-morphisms-]{stacks-morphisms}
\externaldocument[stacks-limits-]{stacks-limits}
\externaldocument[stacks-cohomology-]{stacks-cohomology}
\externaldocument[stacks-perfect-]{stacks-perfect}
\externaldocument[stacks-introduction-]{stacks-introduction}
\externaldocument[stacks-more-morphisms-]{stacks-more-morphisms}
\externaldocument[stacks-geometry-]{stacks-geometry}
\externaldocument[moduli-]{moduli}
\externaldocument[moduli-curves-]{moduli-curves}
\externaldocument[examples-]{examples}
\externaldocument[exercises-]{exercises}
\externaldocument[guide-]{guide}
\externaldocument[desirables-]{desirables}
\externaldocument[coding-]{coding}
\externaldocument[obsolete-]{obsolete}
\externaldocument[fdl-]{fdl}
\externaldocument[index-]{index}

% Theorem environments.
%
\theoremstyle{plain}
\newtheorem{theorem}[subsection]{Theorem}
\newtheorem{proposition}[subsection]{Proposition}
\newtheorem{lemma}[subsection]{Lemma}

\theoremstyle{definition}
\newtheorem{definition}[subsection]{Definition}
\newtheorem{example}[subsection]{Example}
\newtheorem{exercise}[subsection]{Exercise}
\newtheorem{situation}[subsection]{Situation}

\theoremstyle{remark}
\newtheorem{remark}[subsection]{Remark}
\newtheorem{remarks}[subsection]{Remarks}

\numberwithin{equation}{subsection}

% Macros
%
\def\lim{\mathop{\mathrm{lim}}\nolimits}
\def\colim{\mathop{\mathrm{colim}}\nolimits}
\def\Spec{\mathop{\mathrm{Spec}}}
\def\Hom{\mathop{\mathrm{Hom}}\nolimits}
\def\Ext{\mathop{\mathrm{Ext}}\nolimits}
\def\SheafHom{\mathop{\mathcal{H}\!\mathit{om}}\nolimits}
\def\SheafExt{\mathop{\mathcal{E}\!\mathit{xt}}\nolimits}
\def\Sch{\mathit{Sch}}
\def\Mor{\operatorname{Mor}\nolimits}
\def\Ob{\mathop{\mathrm{Ob}}\nolimits}
\def\Sh{\mathop{\mathit{Sh}}\nolimits}
\def\NL{\mathop{N\!L}\nolimits}
\def\proetale{{pro\text{-}\acute{e}tale}}
\def\etale{{\acute{e}tale}}
\def\QCoh{\mathit{QCoh}}
\def\Ker{\mathop{\mathrm{Ker}}}
\def\Im{\mathop{\mathrm{Im}}}
\def\Coker{\mathop{\mathrm{Coker}}}
\def\Coim{\mathop{\mathrm{Coim}}}

%
% Macros for moduli stacks/spaces
%
\def\QCohstack{\mathcal{QC}\!\mathit{oh}}
\def\Cohstack{\mathcal{C}\!\mathit{oh}}
\def\Spacesstack{\mathcal{S}\!\mathit{paces}}
\def\Quotfunctor{\mathrm{Quot}}
\def\Hilbfunctor{\mathrm{Hilb}}
\def\Curvesstack{\mathcal{C}\!\mathit{urves}}
\def\Polarizedstack{\mathcal{P}\!\mathit{olarized}}
\def\Complexesstack{\mathcal{C}\!\mathit{omplexes}}
% \Pic is the operator that assigns to X its picard group, usage \Pic(X)
% \Picardstack_{X/B} denotes the Picard stack of X over B
% \Picardfunctor_{X/B} denotes the Picard functor of X over B
\def\Pic{\mathop{\mathrm{Pic}}\nolimits}
\def\Picardstack{\mathcal{P}\!\mathit{ic}}
\def\Picardfunctor{\mathrm{Pic}}
\def\Deformationcategory{\mathcal{D}\!\mathit{ef}}



% OK, start here.
%
\begin{document}

\title{Categories}


\maketitle

\phantomsection
\label{section-phantom}

\tableofcontents

\section{Introduction}
\label{section-introduction}

\noindent
Categories were first introduced in \cite{GenEqui}.
The category of categories (which is a proper class)
is a $2$-category. Similarly, the category of stacks
forms a $2$-category. If you already know
about categories, but not about $2$-categories you
should read
Section \ref{section-formal-cat-cat}
as an introduction to the formal definitions later on.

\section{Definitions}
\label{section-definition-categories}

\noindent
We recall the definitions, partly to fix notation.

\begin{definition}
\label{definition-category}
A {\it category} $\mathcal{C}$ consists of the following data:
\begin{enumerate}
\item A set of objects $\Ob(\mathcal{C})$.
\item For each pair $x, y \in \Ob(\mathcal{C})$ a set of morphisms
$\Mor_\mathcal{C}(x, y)$.
\item For each triple $x, y, z\in \Ob(\mathcal{C})$ a composition
map $ \Mor_\mathcal{C}(y, z) \times \Mor_\mathcal{C}(x, y)
\to \Mor_\mathcal{C}(x, z) $, denoted $(\phi, \psi) \mapsto
\phi \circ \psi$.
\end{enumerate}
These data are to satisfy the following rules:
\begin{enumerate}
\item For every element $x\in \Ob(\mathcal{C})$ there exists a
morphism $\text{id}_x\in \Mor_\mathcal{C}(x, x)$ such that
$\text{id}_x \circ \phi = \phi$ and $\psi \circ \text{id}_x = \psi $ whenever
these compositions make sense.
\item Composition is associative, i.e., $(\phi \circ \psi) \circ \chi =
\phi \circ ( \psi \circ \chi)$ whenever these compositions make sense.
\end{enumerate}
\end{definition}

\noindent
It is customary to require all the morphism sets
$\Mor_\mathcal{C}(x, y)$ to be disjoint.
In this way a morphism $\phi : x \to y$ has a unique {\it source} $x$
and a unique {\it target} $y$. This is not strictly necessary,
although care has to be taken in formulating condition (2) above
if it is not the case. It is convenient and we will often assume
this is the case. In this case we say that $\phi$ and $\psi$ are
{\it composable} if the source of $\phi$ is equal to the
target of $\psi$, in which case $\phi \circ \psi$ is defined.
An equivalent definition would be to define a category
as a quintuple $(\text{Ob}, \text{Arrows}, s, t, \circ)$
consisting of a set of objects, a set of morphisms (arrows),
source, target and composition subject to a long list of axioms.
We will occasionally use this point of view.

\begin{remark}
\label{remark-big-categories}
Big categories. In some texts a category is allowed to have a proper
class of objects. We will allow this as well in these notes but only
in the following list of cases (to be updated as we go along).
In particular, when we say: ``Let $\mathcal{C}$ be a category''
then it is understood that $\Ob(\mathcal{C})$ is a set.
\begin{enumerate}
\item The category $\textit{Sets}$ of sets.
\item The category $\textit{Ab}$ of abelian groups.
\item The category $\textit{Groups}$ of groups.
\item Given a group $G$ the category $G\textit{-Sets}$ of
sets with a left $G$-action.
\item Given a ring $R$ the category $\text{Mod}_R$ of $R$-modules.
\item Given a field $k$ the category of vector spaces over $k$.
\item The category of rings.
\item The category of schemes.
\item The category $\textit{Top}$ of topological spaces.
\item Given a topological space $X$ the category
$\textit{PSh}(X)$ of presheaves of sets over $X$.
\item Given a topological space $X$ the category
$\Sh(X)$ of sheaves of sets over $X$.
\item Given a topological space $X$ the category
$\textit{PAb}(X)$ of presheaves of abelian groups over $X$.
\item Given a topological space $X$ the category
$\textit{Ab}(X)$ of sheaves of abelian groups over $X$.
\item Given a small category $\mathcal{C}$ the category of functors
from $\mathcal{C}$ to $\textit{Sets}$.
\item Given a category $\mathcal{C}$ the category of presheaves of sets
over $\mathcal{C}$.
\item Given a site $\mathcal{C}$ the category of sheaves
of sets over $\mathcal{C}$.
\end{enumerate}
One of the reason to enumerate these here is to try and avoid
working with something like the ``collection'' of ``big'' categories
which would be like working with the collection of all classes
which I think definitively is a meta-mathematical object.
\end{remark}

\begin{remark}
\label{remark-unique-identity}
It follows directly from the definition that any two identity morphisms
of an object $x$ of $\mathcal{A}$ are the same. Thus we may and will
speak of {\it the} identity morphism $\text{id}_x$ of $x$.
\end{remark}

\begin{definition}
\label{definition-isomorphism}
A morphism $\phi : x \to y$ is an {\it isomorphism} of the category
$\mathcal{C}$ if there exists a morphism $\psi : y \to x$
such that $\phi \circ \psi = \text{id}_y$ and
$\psi \circ \phi = \text{id}_x$.
\end{definition}

\noindent
An isomorphism $\phi$ is also sometimes called an {\it invertible}
morphism, and the morphism $\psi$ of the definition is called the
{\it inverse} and denoted $\phi^{-1}$. It is unique if it exists. Note that
given an object $x$ of a category $\mathcal{A}$ the set of invertible
elements $\text{Aut}_\mathcal{A}(x)$
of $\Mor_\mathcal{A}(x, x)$ forms a group under composition.
This group is called the {\it automorphism} group of $x$ in $\mathcal{A}$.

\begin{definition}
\label{definition-groupoid}
A {\it groupoid} is a category where every morphism is an isomorphism.
\end{definition}

\begin{example}
\label{example-group-groupoid}
A group $G$ gives rise to a groupoid with a single object $x$
and morphisms $\Mor(x, x) = G$, with the composition rule
given by the group law in $G$. Every groupoid with a single
object is of this form.
\end{example}

\begin{example}
\label{example-set-groupoid}
A set $C$ gives rise to a groupoid $\mathcal{C}$ defined as follows:
As objects we take $\Ob(\mathcal{C}) := C$ and for morphisms
we take $\Mor(x, y)$ empty if $x\neq y$ and equal to
$\{\text{id}_x\}$ if $x = y$.
\end{example}

\begin{definition}
\label{definition-functor}
A {\it functor} $F : \mathcal{A} \to \mathcal{B}$
between two categories $\mathcal{A}, \mathcal{B}$ is given by the
following data:
\begin{enumerate}
\item A map $F : \Ob(\mathcal{A}) \to \Ob(\mathcal{B})$.
\item For every $x, y \in \Ob(\mathcal{A})$ a map
$F : \Mor_\mathcal{A}(x, y) \to \Mor_\mathcal{B}(F(x), F(y))$,
denoted $\phi \mapsto F(\phi)$.
\end{enumerate}
These data should be compatible with composition and identity morphisms
in the following manner: $F(\phi \circ \psi) =
F(\phi) \circ F(\psi)$ for a composable pair $(\phi, \psi)$ of
morphisms of $\mathcal{A}$ and $F(\text{id}_x) = \text{id}_{F(x)}$.
\end{definition}

\noindent
Note that every category $\mathcal{A}$ has an
{\it identity} functor $\text{id}_\mathcal{A}$.
In addition, given a functor $G : \mathcal{B} \to \mathcal{C}$
and a functor $F : \mathcal{A} \to \mathcal{B}$ there is
a {\it composition} functor $G \circ F : \mathcal{A} \to \mathcal{C}$
defined in an obvious manner.

\begin{definition}
\label{definition-faithful}
Let $F : \mathcal{A} \to \mathcal{B}$ be a functor.
\begin{enumerate}
\item We say $F$ is {\it faithful} if
for any objects $x, y \in \Ob(\mathcal{A})$ the map
$$
F : \Mor_\mathcal{A}(x, y) \to \Mor_\mathcal{B}(F(x), F(y))
$$
is injective.
\item If these maps are all bijective then $F$ is called
{\it fully faithful}.
\item
The functor $F$ is called {\it essentially surjective} if for any
object $y \in \Ob(\mathcal{B})$ there exists an object
$x \in \Ob(\mathcal{A})$ such that $F(x)$ is isomorphic to $y$ in
$\mathcal{B}$.
\end{enumerate}
\end{definition}

\begin{definition}
\label{definition-subcategory}
A {\it subcategory} of a category $\mathcal{B}$ is
a category $\mathcal{A}$ whose objects and arrows
form subsets of the objects and arrows
of $\mathcal{B}$ and such that source, target
and composition in $\mathcal{A}$ agree with those
of $\mathcal{B}$. We say $\mathcal{A}$ is a
{\it full subcategory} of $\mathcal{B}$ if $\Mor_\mathcal{A}(x, y)
= \Mor_\mathcal{B}(x, y)$ for all $x, y \in \Ob(\mathcal{A})$.
We say $\mathcal{A}$ is a {\it strictly full} subcategory of $\mathcal{B}$
if it is a full subcategory and given $x \in \Ob(\mathcal{A})$ any
object of $\mathcal{B}$ which is isomorphic to $x$ is also in $\mathcal{A}$.
\end{definition}

\noindent
If $\mathcal{A} \subset \mathcal{B}$ is a subcategory then the
identity map is a functor from $\mathcal{A}$ to $\mathcal{B}$.
Furthermore a subcategory $\mathcal{A} \subset \mathcal{B}$
is full if and only if the inclusion functor is fully faithful.
Note that given a category $\mathcal{B}$ the set of full subcategories
of $\mathcal{B}$ is the same as the set of subsets of
$\Ob(\mathcal{B})$.

\begin{remark}
\label{remark-functor-into-sets}
Suppose that $\mathcal{A}$ is a category.
A functor $F$ from $\mathcal{A}$ to $\textit{Sets}$
is a mathematical object (i.e., it is a set not a class or a formula
of set theory, see
Sets, Section \ref{sets-section-sets-everything})
even though the category of sets is ``big''.
Namely, the range of $F$ on objects will be
a set $F(\Ob(\mathcal{A}))$ and then we
may think of $F$ as a functor between
$\mathcal{A}$ and the full subcategory
of the category of sets whose
objects are elements of $F(\Ob(\mathcal{A}))$.
\end{remark}

\begin{example}
\label{example-group-homomorphism-functor}
A homomorphism $p : G\to H$ of groups gives rise to a functor
between the associated groupoids in Example \ref{example-group-groupoid}. It is
faithful (resp.\ fully faithful) if and only if $p$ is injective (resp.\ an
isomorphism).
\end{example}

\begin{example}
\label{example-category-over-X}
Given a category $\mathcal{C}$ and an object $X\in \Ob(\mathcal{C})$
we define the {\it category of objects over $X$},
denoted $\mathcal{C}/X$ as follows.
The objects of $\mathcal{C}/X$ are morphisms $Y\to X$ for
some $Y\in \Ob(\mathcal{C})$. Morphisms between objects
$Y\to X$ and $Y'\to X$ are morphisms $Y\to Y'$ in $\mathcal{C}$ that
make the obvious diagram commute.  Note that there is a functor
$p_X : \mathcal{C}/X\to \mathcal{C}$ which simply forgets the
morphism.  Moreover given a morphism $f : X'\to X$ in
$\mathcal{C}$ there is an induced functor
$F : \mathcal{C}/X' \to \mathcal{C}/X$ obtained by composition with $f$,
and $p_X\circ F = p_{X'}$.
\end{example}

\begin{example}
\label{example-category-under-X}
Given a category $\mathcal{C}$ and an object $X\in \Ob(\mathcal{C})$
we define the {\it category of objects under $X$},
denoted $X/\mathcal{C}$ as follows.
The objects of $X/\mathcal{C}$ are morphisms $X\to Y$ for
some $Y\in \Ob(\mathcal{C})$. Morphisms between objects
$X\to Y$ and $X\to Y'$ are morphisms $Y\to Y'$ in $\mathcal{C}$ that
make the obvious diagram commute.  Note that there is a functor
$p_X : X/\mathcal{C}\to \mathcal{C}$ which simply forgets the
morphism.  Moreover given a morphism $f : X'\to X$ in
$\mathcal{C}$ there is an induced functor
$F : X/\mathcal{C} \to X'/\mathcal{C}$
obtained by composition with $f$,
and $p_{X'}\circ F = p_X$.
\end{example}




\begin{definition}
\label{definition-transformation-functors}
Let $F, G : \mathcal{A} \to \mathcal{B}$ be functors.
A {\it natural transformation}, or a {\it morphism of functors}
$t : F \to G$, is a collection $\{t_x\}_{x\in \Ob(\mathcal{A})}$
such that
\begin{enumerate}
\item $t_x : F(x) \to G(x)$ is a morphism in the category $\mathcal{B}$, and
\item for every morphism $\phi : x \to y$ of $\mathcal{A}$ the following
diagram is commutative
$$
\xymatrix{
F(x) \ar[r]^{t_x} \ar[d]_{F(\phi)} & G(x) \ar[d]^{G(\phi)} \\
F(y) \ar[r]^{t_y} & G(y) }
$$
\end{enumerate}
\end{definition}

\noindent
Sometimes we use the diagram
$$
\xymatrix{
\mathcal{A}
\rtwocell^F_G{t}
&
\mathcal{B}
}
$$
to indicate that $t$ is a morphism from $F$ to $G$.

\medskip\noindent
Note that every functor $F$ comes with the {\it identity} transformation
$\text{id}_F : F \to F$. In addition, given a morphism of
functors $t : F \to G$ and a morphism of functors $s : E \to F$
then the {\it composition} $t \circ s$ is defined by the rule
$$
(t \circ s)_x = t_x \circ s_x : E(x) \to G(x)
$$
for $x \in \Ob(\mathcal{A})$.
It is easy to verify that this is indeed a morphism of functors
from $E$ to $G$.
In this way, given categories
$\mathcal{A}$ and $\mathcal{B}$ we obtain a new category,
namely the category of functors between $\mathcal{A}$ and
$\mathcal{B}$.

\begin{remark}
\label{remark-functors-sets-sets}
This is one instance where the same thing does not hold if
$\mathcal{A}$ is a ``big'' category. For example consider
functors $\textit{Sets} \to \textit{Sets}$. As we have currently
defined it such a functor is a class and not a set. In other
words, it is given by a formula in set theory (with some variables
equal to specified sets)! It is not a good idea to try to consider
all possible formulae of set theory as part of the definition of
a mathematical object. The same problem presents itself when
considering sheaves on the category of schemes for example.
We will come back to this point later.
\end{remark}

\begin{definition}
\label{definition-equivalence-categories}
An {\it equivalence of categories}
$F : \mathcal{A} \to \mathcal{B}$ is a functor such that there
exists a functor $G : \mathcal{B} \to \mathcal{A}$ such that
the compositions $F \circ G$ and $G \circ F$ are isomorphic to the
identity functors $\text{id}_\mathcal{B}$,
respectively $\text{id}_\mathcal{A}$.
In this case we say that $G$ is a {\it quasi-inverse} to $F$.
\end{definition}

\begin{lemma}
\label{lemma-construct-quasi-inverse}
Let $F : \mathcal{A} \to \mathcal{B}$ be a fully faithful functor.
Suppose for every $X \in \Ob(\mathcal{B})$ given an
object $j(X)$ of $\mathcal{A}$ and an isomorphism $i_X : X \to F(j(X))$.
Then there is a unique functor $j : \mathcal{B} \to \mathcal{A}$
such that $j$ extends the rule on objects, and the isomorphisms
$i_X$ define an isomorphism of functors
$\text{id}_\mathcal{B} \to F \circ j$. Moreover, $j$ and $F$
are quasi-inverse equivalences of categories.
\end{lemma}

\begin{proof}
This lemma proves itself.
\end{proof}

\begin{lemma}
\label{lemma-equivalence-categories}
A functor is an equivalence of categories if and only if it is both fully
faithful and essentially surjective.
\end{lemma}

\begin{proof}
Let $F : \mathcal{A} \to \mathcal{B}$ be essentially surjective and fully
faithful. As by convention all categories are small and as $F$ is essentially
surjective we can, using the axiom of choice, choose for every
$X \in \Ob(\mathcal{B})$ an object $j(X)$ of $\mathcal{A}$ and an
isomorphism $i_X : X \to F(j(X))$. Then we apply
Lemma \ref{lemma-construct-quasi-inverse}
using that $F$ is fully faithful.
\end{proof}

\begin{definition}
\label{definition-product-category}
Let $\mathcal{A}$, $\mathcal{B}$ be categories.
We define the {\it product category}
$\mathcal{A} \times \mathcal{B}$ to be the category with
objects
$\Ob(\mathcal{A} \times \mathcal{B}) =
\Ob(\mathcal{A}) \times \Ob(\mathcal{B})$
and
$$
\Mor_{\mathcal{A} \times \mathcal{B}}((x, y), (x', y'))
:=
\Mor_\mathcal{A}(x, x')\times
\Mor_\mathcal{B}(y, y').
$$
Composition is defined componentwise.
\end{definition}


\section{Opposite Categories and the Yoneda Lemma}
\label{section-opposite}

\begin{definition}
\label{definition-opposite}
Given a category $\mathcal{C}$ the {\it opposite category}
$\mathcal{C}^{opp}$ is the category with the same objects
as $\mathcal{C}$ but all morphisms reversed.
\end{definition}

\noindent
In other words
$\Mor_{\mathcal{C}^{opp}}(x, y) = \Mor_\mathcal{C}(y, x)$.
Composition in $\mathcal{C}^{opp}$ is the same as in $\mathcal{C}$
except backwards: if $\phi : y \to z$ and $\psi : x \to y$
are morphisms in $\mathcal{C}^{opp}$, in other words arrows
$z \to y$ and $y \to x$ in $\mathcal{C}$,
then $\phi \circ^{opp} \psi$ is the morphism $x \to z$
of $\mathcal{C}^{opp}$ which corresponds to the composition
$z \to y \to x$ in $\mathcal{C}$.

\begin{definition}
\label{definition-contravariant}
Let $\mathcal{C}$, $\mathcal{S}$ be categories.
A {\it contravariant} functor $F$
from $\mathcal{C}$ to $\mathcal{S}$
is a functor $\mathcal{C}^{opp}\to \mathcal{S}$.
\end{definition}

\noindent
Concretely, a contravariant functor $F$ is given
by a map $F : \Ob(\mathcal{C}) \to
\Ob(\mathcal{S})$ and for every morphism
$\psi : x \to y$ in $\mathcal{C}$ a morphism
$F(\psi) : F(y) \to F(x)$. These should satisfy the property
that, given another morphism
$\phi : y \to z$, we have $F(\phi \circ \psi)
= F(\psi) \circ F(\phi)$ as morphisms $F(z) \to F(x)$.
(Note the reverse of order.)

\begin{definition}
\label{definition-presheaf}
Let $\mathcal{C}$ be a category.
\begin{enumerate}
\item A {\it presheaf of sets on $\mathcal{C}$}
or simply a {\it presheaf} is a contravariant functor
$F$ from $\mathcal{C}$ to $\textit{Sets}$.
\item The category of presheaves is denoted $\textit{PSh}(\mathcal{C})$.
\end{enumerate}
\end{definition}

\noindent
Of course the category of presheaves is a proper class.

\begin{example}
\label{example-hom-functor}
Functor of points.
For any $U\in \Ob(\mathcal{C})$ there is a contravariant
functor
$$
\begin{matrix}
h_U & : & \mathcal{C}
&
\longrightarrow
&
\textit{Sets} \\
& &
X
&
\longmapsto
&
\Mor_\mathcal{C}(X, U)
\end{matrix}
$$
which takes an object $X$ to the set
$\Mor_\mathcal{C}(X, U)$. In other words $h_U$ is a presheaf.
Given a morphism $f : X\to Y$ the corresponding map
$h_U(f) :  \Mor_\mathcal{C}(Y, U)\to \Mor_\mathcal{C}(X, U)$
takes $\phi$ to $\phi\circ f$. We will always denote
this presheaf $h_U : \mathcal{C}^{opp} \to \textit{Sets}$.
It is called the {\it representable presheaf} associated to $U$.
If $\mathcal{C}$ is the category of schemes this functor is
sometimes referred to as the
\emph{functor of points} of $U$.
\end{example}

\noindent
Note that given a morphism $\phi : U \to V$ in $\mathcal{C}$ we get a
corresponding natural transformation of functors $h(\phi) : h_U \to h_V$
defined by composing with the morphism $U \to V$. This turns
composition of morphisms in $\mathcal{C}$ into composition of
transformations of functors. In other words we get a functor
$$
h :
\mathcal{C}
\longrightarrow
\textit{PSh}(\mathcal{C})
$$
Note that the target is a ``big'' category, see
Remark \ref{remark-big-categories}. On the other hand,
$h$ is an actual mathematical object (i.e.\ a set), compare Remark
\ref{remark-functor-into-sets}.

\begin{lemma}[Yoneda lemma]
\label{lemma-yoneda}
\begin{reference}
Appeared in some form in \cite{Yoneda-homology}. Used by Grothendieck in a
generalized form in \cite{Gr-II}.
\end{reference}
Let $U, V \in \Ob(\mathcal{C})$.
Given any morphism of functors $s : h_U \to h_V$
there is a unique morphism $\phi : U \to V$
such that $h(\phi) = s$. In other words the
functor $h$ is fully faithful. More generally,
given any contravariant functor $F$ and any object
$U$ of $\mathcal{C}$ we have a natural bijection
$$
\Mor_{\textit{PSh}(\mathcal{C})}(h_U, F) \longrightarrow F(U),
\quad
s \longmapsto s_U(\text{id}_U).
$$
\end{lemma}

\begin{proof}
For the first statement, just take
$\phi = s_U(\text{id}_U) \in \Mor_\mathcal{C}(U, V)$.
For the second statement, given $\xi \in F(U)$ define
$s$ by $s_V : h_U(V) \to F(V)$ by sending the element $f : V \to U$
of $h_U(V) = \Mor_\mathcal{C}(V, U)$ to $F(f)(\xi)$.
\end{proof}

\begin{definition}
\label{definition-representable-functor}
A contravariant functor $F : \mathcal{C}\to \textit{Sets}$ is said
to be {\it representable} if it is isomorphic to the functor of
points $h_U$ for some object $U$ of $\mathcal{C}$.
\end{definition}

\noindent
Let $\mathcal{C}$ be a category and let
$F : \mathcal{C}^{opp} \to \textit{Sets}$ be a representable functor.
Choose an object $U$ of $\mathcal{C}$ and an isomorphism $s : h_U \to F$.
The Yoneda lemma guarantees that the pair $(U, s)$
is unique up to unique isomorphism. The object
$U$ is called an object {\it representing} $F$.
By the Yoneda lemma the transformation $s$ corresponds to a unique
element $\xi \in F(U)$. This element is called the {\it universal object}.
It has the property that for $V \in \Ob(\mathcal{C})$ the map
$$
\Mor_\mathcal{C}(V, U) \longrightarrow F(V),\quad
(f : V \to U) \longmapsto F(f)(\xi)
$$
is a bijection. Thus $\xi$ is universal in the sense that every element
of $F(V)$ is equal to the image of $\xi$ via $F(f)$ for a unique morphism
$f : V \to U$ in $\mathcal{C}$.






\section{Products of pairs}
\label{section-products-pairs}

\begin{definition}
\label{definition-products}
Let $x, y\in \Ob(\mathcal{C})$.
A {\it product} of $x$ and $y$ is
an object $x \times y \in \Ob(\mathcal{C})$
together with morphisms
$p\in \Mor_{\mathcal C}(x \times y, x)$ and
$q\in\Mor_{\mathcal C}(x \times y, y)$ such
that the following universal property holds: for
any $w\in \Ob(\mathcal{C})$ and morphisms
$\alpha \in \Mor_{\mathcal C}(w, x)$ and
$\beta \in \Mor_\mathcal{C}(w, y)$
there is a unique
$\gamma\in \Mor_{\mathcal C}(w, x \times y)$ making
the diagram
$$
\xymatrix{
w \ar[rrrd]^\beta \ar@{-->}[rrd]_\gamma \ar[rrdd]_\alpha & & \\
& & x \times y \ar[d]_p \ar[r]_q & y \\
& & x &
}
$$
commute.
\end{definition}

\noindent
If a product exists it is unique up to unique
isomorphism. This follows from the Yoneda lemma as
the definition requires $x \times y$ to be an object
of $\mathcal{C}$ such that
$$
h_{x \times y}(w) = h_x(w) \times h_y(w)
$$
functorially in $w$. In other words the product $x \times y$
is an object representing the functor
$w \mapsto h_x(w) \times h_y(w)$.

\begin{definition}
\label{definition-has-products-of-pairs}
We say the category $\mathcal{C}$ {\it has products of pairs
of objects} if a product $x \times y$
exists for any $x, y \in \Ob(\mathcal{C})$.
\end{definition}

\noindent
We use this terminology to distinguish this notion from the notion
of ``having products'' or ``having finite products'' which usually means
something else (in particular it always implies there exists a
final object).





\section{Coproducts of pairs}
\label{section-coproducts-pairs}

\begin{definition}
\label{definition-coproducts}
Let $x, y \in \Ob(\mathcal{C})$.
A {\it coproduct}, or {\it amalgamated sum} of $x$ and $y$ is
an object $x \amalg y \in \Ob(\mathcal{C})$
together with morphisms
$i \in \Mor_{\mathcal C}(x, x \amalg y)$ and
$j \in \Mor_{\mathcal C}(y, x \amalg y)$ such
that the following universal property holds: for
any $w \in \Ob(\mathcal{C})$ and morphisms
$\alpha \in \Mor_{\mathcal C}(x, w)$ and
$\beta \in \Mor_\mathcal{C}(y, w)$
there is a unique
$\gamma \in \Mor_{\mathcal C}(x \amalg y, w)$ making
the diagram
$$
\xymatrix{
& y \ar[d]^j \ar[rrdd]^\beta \\
x \ar[r]^i \ar[rrrd]_\alpha & x \amalg y \ar@{-->}[rrd]^\gamma \\
& & & w
}
$$
commute.
\end{definition}

\noindent
If a coproduct exists it is unique up to unique
isomorphism. This follows from the Yoneda lemma (applied to the
opposite category) as
the definition requires $x \amalg y$ to be an object
of $\mathcal{C}$ such that
$$
\Mor_\mathcal{C}(x \amalg y, w) =
\Mor_\mathcal{C}(x, w) \times \Mor_\mathcal{C}(y, w)
$$
functorially in $w$.

\begin{definition}
\label{definition-has-coproducts-of-pairs}
We say the category $\mathcal{C}$ {\it has coproducts of pairs
of objects} if a coproduct $x \amalg y$
exists for any $x, y \in \Ob(\mathcal{C})$.
\end{definition}

\noindent
We use this terminology to distinguish this notion from the notion
of ``having coproducts'' or ``having finite coproducts'' which usually means
something else (in particular it always implies there exists an
initial object in $\mathcal{C}$).





\section{Fibre products}
\label{section-fibre-products}

\begin{definition}
\label{definition-fibre-products}
Let $x, y, z\in \Ob(\mathcal{C})$,
$f\in \Mor_\mathcal{C}(x, y)$
and $g\in \Mor_{\mathcal C}(z, y)$.
A {\it fibre product} of $f$ and $g$ is
an object $x \times_y z\in \Ob(\mathcal{C})$
together with morphisms
$p \in \Mor_{\mathcal C}(x \times_y z, x)$ and
$q \in \Mor_{\mathcal C}(x \times_y z, z)$ making the diagram
$$
\xymatrix{
x \times_y z \ar[r]_q \ar[d]_p & z \ar[d]^g \\
x \ar[r]^f & y
}
$$
commute, and such that the following universal property holds: for
any $w\in \Ob(\mathcal{C})$ and morphisms
$\alpha \in \Mor_{\mathcal C}(w, x)$ and
$\beta \in \Mor_\mathcal{C}(w, z)$ with
$f \circ \alpha = g \circ \beta$
there is a unique
$\gamma \in \Mor_{\mathcal C}(w, x \times_y z)$ making
the diagram
$$
\xymatrix{
w \ar[rrrd]^\beta \ar@{-->}[rrd]_\gamma \ar[rrdd]_\alpha & & \\
& & x \times_y z \ar[d]^p \ar[r]_q & z \ar[d]^g \\
& & x \ar[r]^f & y
}
$$
commute.
\end{definition}

\noindent
If a fibre product exists it is unique up to unique
isomorphism. This follows from the Yoneda lemma as
the definition requires $x \times_y z$ to be an object
of $\mathcal{C}$ such that
$$
h_{x \times_y z}(w) = h_x(w) \times_{h_y(w)} h_z(w)
$$
functorially in $w$. In other words the fibre product $x \times_y z$
is an object representing the functor
$w \mapsto h_x(w) \times_{h_y(w)} h_z(w)$.

\begin{definition}
\label{definition-cartesian}
We say a commutative diagram
$$
\xymatrix{
w \ar[r] \ar[d] &
z \ar[d] \\
x \ar[r] &
y
}
$$
in a category is {\it cartesian} if $w$ and the morphisms $w \to x$ and
$w \to z$ form a fibre product of the morphisms $x \to y$ and $z \to y$.
\end{definition}

\begin{definition}
\label{definition-has-fibre-products}
We say the category $\mathcal{C}$ {\it has fibre products} if
the fibre product exists for any $f\in \Mor_{\mathcal C}(x, y)$
and $g\in \Mor_{\mathcal C}(z, y)$.
\end{definition}

\begin{definition}
\label{definition-representable-morphism}
A morphism $f : x \to y$ of a category $\mathcal{C}$ is said to be
{\it representable} if for every morphism $z \to y$
in $\mathcal{C}$ the fibre product $x \times_y z$ exists.
\end{definition}

\begin{lemma}
\label{lemma-composition-representable}
Let $\mathcal{C}$ be a category.
Let $f : x \to y$, and $g : y \to z$ be representable.
Then $g \circ f : x \to z$ is representable.
\end{lemma}

\begin{proof}
Omitted.
\end{proof}

\begin{lemma}
\label{lemma-base-change-representable}
Let $\mathcal{C}$ be a category.
Let $f : x \to y$ be representable.
Let $y' \to y$ be a morphism of $\mathcal{C}$.
Then the morphism $x' := x \times_y y' \to y'$ is representable also.
\end{lemma}

\begin{proof}
Let $z \to y'$ be a morphism. The fibre product
$x' \times_{y'} z$ is supposed to represent the
functor
\begin{eqnarray*}
w & \mapsto & h_{x'}(w)\times_{h_{y'}(w)} h_z(w) \\
& = & (h_x(w) \times_{h_y(w)} h_{y'}(w)) \times_{h_{y'}(w)} h_z(w) \\
& = & h_x(w) \times_{h_y(w)} h_z(w)
\end{eqnarray*}
which is representable by assumption.
\end{proof}

\section{Examples of fibre products}
\label{section-example-fibre-products}

\noindent
In this section we list examples of fibre products and
we describe them.

\medskip\noindent
As a really trivial first example we observe
that the category of sets has fibre products and hence every
morphism is representable. Namely, if $f : X \to Y$
and $g : Z \to Y$ are maps of sets then we define
$X \times_Y Z$ as the subset of $X \times Z$ consisting
of pairs $(x, z)$ such that $f(x) = g(z)$. The morphisms
$p : X \times_Y Z \to X$ and $q : X \times_Y Z \to Z$ are
the projection maps $(x, z) \mapsto x$, and $(x, z) \mapsto z$.
Finally, if $\alpha : W \to X$ and $\beta : W \to Z$
are morphisms such that $f \circ \alpha = g \circ \beta$
then the map $W \to X \times Z$, $w\mapsto (\alpha(w), \beta(w))$
obviously ends up in $X \times_Y Z$ as desired.

\medskip\noindent
In many categories whose objects are sets endowed with certain types of
algebraic structures the fibre product of the underlying sets also
provides the fibre product in the category. For example, suppose
that $X$, $Y$ and $Z$ above are groups and that $f$, $g$ are
homomorphisms of groups. Then the set-theoretic fibre product
$X \times_Y Z$ inherits the structure of a group, simply by
defining the product of two pairs by the formula
$(x, z) \cdot (x', z') = (xx', zz')$. Here we list those categories
for which a similar reasoning works.
\begin{enumerate}
\item The category $\textit{Groups}$ of groups.
\item The category $G\textit{-Sets}$ of sets
endowed with a left $G$-action for some fixed group $G$.
\item The category of rings.
\item The category of $R$-modules given a ring $R$.
\end{enumerate}




\section{Fibre products and representability}
\label{section-representable-map-presheaves}

\noindent
In this section we work out fibre products in the
category of contravariant functors from a category
to the category of sets. This will later be superseded
during the discussion of sites, presheaves, sheaves. Of some
interest is the notion of a ``representable morphism'' between
such functors.

\begin{lemma}
\label{lemma-fibre-product-presheaves}
Let $\mathcal{C}$ be a category.
Let $F, G, H : \mathcal{C}^{opp} \to \textit{Sets}$
be functors. Let $a : F \to G$ and $b : H \to G$ be
transformations of functors. Then the fibre product
$F \times_{a, G, b} H$ in the category
$\textit{PSh}(\mathcal{C})$
exists and is given by the formula
$$
(F \times_{a, G, b} H)(X) =
F(X) \times_{a_X, G(X), b_X} H(X)
$$
for any object $X$ of $\mathcal{C}$.
\end{lemma}

\begin{proof}
Omitted.
\end{proof}

\noindent
As a special case suppose we have a morphism
$a : F \to G$, an object $U \in \Ob(\mathcal{C})$
and an element $\xi \in G(U)$. According to the Yoneda
Lemma \ref{lemma-yoneda} this gives a transformation
$\xi : h_U \to G$. The fibre product in this case
is described by the rule
$$
(h_U \times_{\xi, G, a} F)(X) =
\{ (f, \xi') \mid f : X \to U, \ \xi' \in F(X), \ G(f)(\xi) = a_X(\xi')\}
$$
If $F$, $G$ are also representable, then this is the functor representing the
fibre product, if it exists, see Section \ref{section-fibre-products}.
The analogy with Definition \ref{definition-representable-morphism}
prompts us to define a notion
of representable transformations.

\begin{definition}
\label{definition-representable-map-presheaves}
Let $\mathcal{C}$ be a category.
Let $F, G : \mathcal{C}^{opp} \to \textit{Sets}$
be functors. We say a morphism $a : F \to G$ is
{\it representable}, or that {\it $F$ is relatively representable
over $G$}, if for every $U \in \Ob(\mathcal{C})$
and any $\xi \in G(U)$ the functor
$h_U \times_G F$ is representable.
\end{definition}

\begin{lemma}
\label{lemma-representable-over-representable}
Let $\mathcal{C}$ be a category.
Let $a : F \to G$ be a morphism of contravariant functors
from $\mathcal{C}$ to $\textit{Sets}$. If $a$ is representable,
and $G$ is a representable functor, then $F$ is representable.
\end{lemma}

\begin{proof}
Omitted.
\end{proof}

\begin{lemma}
\label{lemma-representable-diagonal}
Let $\mathcal{C}$ be a category.
Let $F : \mathcal{C}^{opp} \to \textit{Sets}$ be a functor.
Assume $\mathcal{C}$ has products of pairs of objects and fibre products.
The following are equivalent:
\begin{enumerate}
\item the diagonal $\Delta : F \to F \times F$ is representable,
\item for every $U$ in $\mathcal{C}$,
and any $\xi \in F(U)$ the map $\xi : h_U \to F$ is representable,
\item for every pair $U, V$ in $\mathcal{C}$
and any $\xi \in F(U)$, $\xi' \in F(V)$ the fibre product
$h_U \times_{\xi, F, \xi'} h_V$ is representable.
\end{enumerate}
\end{lemma}

\begin{proof}
We will continue to use the Yoneda lemma to identify $F(U)$
with transformations $h_U \to F$ of functors.

\medskip\noindent
Equivalence of (2) and (3). Let $U, \xi, V, \xi'$ be as in (3).
Both (2) and (3) tell us exactly that $h_U \times_{\xi, F, \xi'} h_V$
is representable; the only difference is that the statement
(3) is symmetric in $U$ and $V$ whereas (2) is not.

\medskip\noindent
Assume condition (1). Let $U, \xi, V, \xi'$
be as in (3). Note that $h_U \times h_V = h_{U \times V}$ is representable.
Denote $\eta : h_{U \times V} \to F \times F$ the map
corresponding to the product $\xi \times \xi' : h_U \times h_V \to F \times F$.
Then the fibre product $F \times_{\Delta, F \times F, \eta} h_{U \times V}$
is representable by assumption. This means there exist
$W \in \Ob(\mathcal{C})$, morphisms
$W \to U$, $W \to V$ and $h_W \to F$ such that
$$
\xymatrix{
h_W \ar[d] \ar[r] & h_U \times h_V \ar[d]^{\xi \times \xi'} \\
F \ar[r] & F \times F
}
$$
is cartesian. Using the explicit description of fibre products
in Lemma \ref{lemma-fibre-product-presheaves} the reader sees that this
implies that $h_W = h_U \times_{\xi, F, \xi'} h_V$ as desired.

\medskip\noindent
Assume the equivalent conditions (2) and (3). Let $U$ be an object
of $\mathcal{C}$ and let $(\xi, \xi') \in (F \times F)(U)$.
By (3) the fibre product $h_U \times_{\xi, F, \xi'} h_U$ is
representable. Choose an object $W$ and an isomorphism
$h_W \to h_U \times_{\xi, F, \xi'} h_U$. The two projections
$\text{pr}_i : h_U \times_{\xi, F, \xi'} h_U \to h_U$
correspond to morphisms $p_i : W \to U$ by Yoneda. Consider
$W' = W \times_{(p_1, p_2), U \times U} U$. It is formal
to show that $W'$ represents $F \times_{\Delta, F \times F} h_U$
because
$$
h_{W'} =  h_W \times_{h_U \times h_U} h_U
= (h_U \times_{\xi, F, \xi'} h_U) \times_{h_U \times h_U} h_U
= F \times_{F \times F} h_U.
$$
Thus $\Delta$ is representable and this finishes the proof.
\end{proof}









\section{Pushouts}
\label{section-pushouts}

\noindent
The dual notion to fibre products is that of pushouts.

\begin{definition}
\label{definition-pushouts}
Let $x, y, z\in \Ob(\mathcal{C})$,
$f\in \Mor_\mathcal{C}(y, x)$
and $g\in \Mor_{\mathcal C}(y, z)$.
A {\it pushout} of $f$ and $g$ is
an object $x\amalg_y z\in \Ob(\mathcal{C})$
together with morphisms
$p\in \Mor_{\mathcal C}(x, x\amalg_y z)$ and
$q\in\Mor_{\mathcal C}(z, x\amalg_y z)$ making the diagram
$$
\xymatrix{
y \ar[r]_g \ar[d]_f & z \ar[d]^q \\
x \ar[r]^p & x\amalg_y z
}
$$
commute, and such that the following universal property holds:
For any $w\in \Ob(\mathcal{C})$ and morphisms
$\alpha \in \Mor_{\mathcal C}(x, w)$ and
$\beta \in \Mor_\mathcal{C}(z, w)$ with
$\alpha \circ f = \beta \circ g$ there is a unique
$\gamma\in \Mor_{\mathcal C}(x\amalg_y z, w)$ making
the diagram
$$
\xymatrix{
y \ar[r]_g \ar[d]_f & z \ar[d]^q \ar[rrdd]^\beta & & \\
x \ar[r]^p \ar[rrrd]^\alpha & x \amalg_y z  \ar@{-->}[rrd]^\gamma & & \\
& & & w
}
$$
commute.
\end{definition}

\noindent
It is possible and straightforward to prove the uniqueness of the triple
$(x\amalg_y z, p, q)$ up to unique isomorphism (if it exists) by direct
arguments. Another possibility is to think of the pushout as the
fibre product in the opposite category, thereby getting this uniqueness for
free from the discussion in Section \ref{section-fibre-products}.

\begin{definition}
\label{definition-cocartesian}
We say a commutative diagram
$$
\xymatrix{
y \ar[r] \ar[d] & z \ar[d] \\
x \ar[r] & w
}
$$
in a category is {\it cocartesian} if $w$ and the morphisms $x \to w$ and
$z \to w$ form a pushout of the morphisms $y \to x$ and $y \to z$.
\end{definition}


\section{Equalizers}
\label{section-equalizers}

\begin{definition}
\label{definition-equalizers}
Suppose that $X$, $Y$ are objects of a category $\mathcal{C}$
and that $a, b : X \to Y$ are morphisms. We say a morphism
$e : Z \to X$ is an {\it equalizer} for the pair $(a, b)$ if
$a \circ e = b \circ e$ and if $(Z, e)$ satisfies the following
universal property: For every morphism $t : W \to X$
in $\mathcal{C}$ such that $a \circ t = b \circ t$ there exists
a unique morphism $s : W \to Z$ such that $t = e \circ s$.
\end{definition}

\noindent
As in the case of the fibre products above, equalizers when
they exist are unique up to unique isomorphism. There is a
straightforward generalization of this definition to the
case where we have more than $2$ morphisms.

\section{Coequalizers}
\label{section-coequalizers}

\begin{definition}
\label{definition-coequalizers}
Suppose that $X$, $Y$ are objects of a category $\mathcal{C}$
and that $a, b : X \to Y$ are morphisms. We say a morphism
$c : Y \to Z$ is a {\it coequalizer} for the pair $(a, b)$ if
$c \circ a = c \circ b$ and if $(Z, c)$ satisfies the following
universal property: For every morphism $t : Y \to W$
in $\mathcal{C}$ such that $t \circ a = t \circ b$ there exists
a unique morphism $s : Z \to W$ such that $t = s \circ c$.
\end{definition}

\noindent
As in the case of the pushouts above, coequalizers when
they exist are unique up to unique isomorphism, and this follows
from the uniqueness of equalizers upon considering the opposite
category. There is a straightforward generalization of this definition
to the case where we have more than $2$ morphisms.

\section{Initial and final objects}
\label{section-initial-final}

\begin{definition}
\label{definition-initial-final}
Let $\mathcal{C}$ be a category.
\begin{enumerate}
\item An object $x$ of the category $\mathcal{C}$ is called
an {\it initial} object if for every object $y$ of $\mathcal{C}$
there is exactly one morphism $x \to y$.
\item An object $x$ of the category $\mathcal{C}$ is called
a {\it final} object if for every object $y$ of $\mathcal{C}$
there is exactly one morphism $y \to x$.
\end{enumerate}
\end{definition}

\noindent
In the category of sets the empty set $\emptyset$ is an
initial object, and in fact the only initial object.
Also, any {\it singleton}, i.e., a set with one element,
is a final object (so it is not unique).




\section{Monomorphisms and Epimorphisms}
\label{section-mono-epi}

\begin{definition}
\label{definition-mono-epi}
Let $\mathcal{C}$ be a category and let $f : X \to Y$ be
a morphism of $\mathcal{C}$.
\begin{enumerate}
\item We say that $f$ is a {\it monomorphism} if for every object
$W$ and every pair of morphisms $a, b : W \to X$ such that
$f \circ a = f \circ b$ we have $a = b$.
\item We say that $f$ is an {\it epimorphism} if for every object
$W$ and every pair of morphisms $a, b : Y \to W$ such that
$a \circ f = b \circ f$ we have $a = b$.
\end{enumerate}
\end{definition}

\begin{example}
\label{example-mono-epi-sets}
In the category of sets the monomorphisms correspond to injective
maps and the epimorphisms correspond to surjective maps.
\end{example}

\begin{lemma}
\label{lemma-characterize-mono-epi}
Let $\mathcal{C}$ be a category, and let $f : X \to Y$ be
a morphism of $\mathcal{C}$. Then
\begin{enumerate}
\item $f$ is a monomorphism if and only if $X$ is the fibre
product $X \times_Y X$, and
\item $f$ is an epimorphism if and only if $Y$ is the pushout
$Y \amalg_X Y$.
\end{enumerate}
\end{lemma}

\begin{proof}
Omitted.
\end{proof}





\section{Limits and colimits}
\label{section-limits}

\noindent
Let $\mathcal{C}$ be a category. A {\it diagram} in $\mathcal{C}$ is
simply a functor $M : \mathcal{I} \to \mathcal{C}$. We say that
$\mathcal{I}$ is the {\it index category} or that $M$ is an
$\mathcal{I}$-diagram. We will use the notation $M_i$ to denote the
image of the object
$i$ of $\mathcal{I}$. Hence for $\phi : i \to i'$ a morphism
in $\mathcal{I}$ we have $M(\phi) : M_i \to M_{i'}$.

\begin{definition}
\label{definition-limit}
A {\it limit} of the $\mathcal{I}$-diagram $M$ in the category
$\mathcal{C}$ is given by an object $\lim_\mathcal{I} M$ in $\mathcal{C}$
together with morphisms $p_i : \lim_\mathcal{I} M \to M_i$ such that
\begin{enumerate}
\item for $\phi : i \to i'$ a morphism
in $\mathcal{I}$ we have $p_{i'} =  M(\phi) \circ p_i$, and
\item for any object $W$ in $\mathcal{C}$ and any family of
morphisms $q_i : W \to M_i$ (indexed by $i \in \Ob(\mathcal{I})$)
such that for all $\phi : i \to i'$
in $\mathcal{I}$ we have $q_{i'} = M(\phi) \circ q_i$ there
exists a unique morphism $q : W \to \lim_\mathcal{I} M$ such that
$q_i = p_i \circ q$ for every object $i$ of $\mathcal{I}$.
\end{enumerate}
\end{definition}

\noindent
Limits $(\lim_\mathcal{I} M, (p_i)_{i\in \Ob(\mathcal{I})})$ are
(if they exist)
unique up to unique isomorphism by the uniqueness requirement
in the definition. Products of pairs, fibre products, and equalizers are
examples of limits. The limit over the empty diagram is a final object
of $\mathcal{C}$.
In the category of sets all limits exist.
The dual notion is that of colimits.

\begin{definition}
\label{definition-colimit}
A {\it colimit} of the $\mathcal{I}$-diagram $M$ in the category
$\mathcal{C}$ is given by an object $\colim_\mathcal{I} M$ in $\mathcal{C}$
together with morphisms $s_i : M_i \to \colim_\mathcal{I} M$ such that
\begin{enumerate}
\item for $\phi : i \to i'$ a morphism
in $\mathcal{I}$ we have $s_i = s_{i'} \circ M(\phi)$, and
\item for any object $W$ in $\mathcal{C}$ and any family of
morphisms $t_i : M_i \to W$ (indexed by $i \in \Ob(\mathcal{I})$)
such that for all $\phi : i \to i'$
in $\mathcal{I}$ we have $t_i = t_{i'} \circ M(\phi)$ there
exists a unique morphism $t : \colim_\mathcal{I} M \to W$ such that
$t_i = t \circ s_i$ for every object $i$ of $\mathcal{I}$.
\end{enumerate}
\end{definition}

\noindent
Colimits $(\colim_\mathcal{I} M, (s_i)_{i\in \Ob(\mathcal{I})})$ are
(if they exist) unique up to unique isomorphism by the uniqueness requirement
in the definition. Coproducts of pairs, pushouts, and coequalizers are
examples of colimits. The colimit over an empty diagram is an initial object
of $\mathcal{C}$. In the category of sets all colimits exist.

\begin{remark}
\label{remark-diagram-small}
The index category of a (co)limit will never be allowed to have
a proper class of objects. In this project it means that
it cannot be one of the categories listed in
Remark \ref{remark-big-categories}
\end{remark}

\begin{remark}
\label{remark-limit-colim}
We often write $\lim_i M_i$, $\colim_i M_i$,
$\lim_{i\in \mathcal{I}} M_i$, or $\colim_{i\in \mathcal{I}} M_i$
instead of the versions indexed by $\mathcal{I}$.
Using this notation, and using the description of
limits and colimits of sets in Section \ref{section-limit-sets}
below, we can say the following.
Let $M : \mathcal{I} \to \mathcal{C}$ be a diagram.
\begin{enumerate}
\item The object $\lim_i M_i$ if it exists satisfies the following property
$$
\Mor_\mathcal{C}(W, \lim_i M_i)
=
\lim_i \Mor_\mathcal{C}(W, M_i)
$$
where the limit on the right takes place in the category of sets.
\item The object $\colim_i M_i$ if it
exists satisfies the following property
$$
\Mor_\mathcal{C}(\colim_i M_i, W)
=
\lim_{i\in \mathcal{I}^\text{opp}} \Mor_\mathcal{C}(M_i, W)
$$
where on the right we have the limit over the opposite category
with value in the category of sets.
\end{enumerate}
By the Yoneda lemma (and its dual) this formula completely determines the
limit, respectively the colimit.
\end{remark}

\begin{remark}
\label{remark-cones-and-cocones}
Let $M : \mathcal{I} \to \mathcal{C}$ be a diagram. In this setting a
{\it cone} for $M$ is given by an object $W$ and a family of morphisms
$q_i : W \to M_i$, $i \in \Ob(\mathcal{I})$ such that for all morphisms
$\phi : i \to i'$ of $\mathcal{I}$ the diagram
$$
\xymatrix{
& W \ar[dl]_{q_i} \ar[dr]^{q_{i'}} \\
M_i \ar[rr]^{M(\phi)} & & M_{i'}
}
$$
is commutative. The collection of cones forms a category with an obvious
notion of morphisms. Clearly, the limit of $M$, if it exists, is a final
object in the category of cones. Dually, a {\it cocone} for $M$ is given
by an object $W$ and a family of morphisms $t_i : M_i \to W$ such that for
all morphisms $\phi : i \to i'$ in $\mathcal{I}$ the diagram
$$
\xymatrix{
M_i \ar[rr]^{M(\phi)} \ar[dr]_{t_i} & & M_{i'} \ar[dl]^{t_{i'}} \\
& W
}
$$
commutes. The collection of cocones forms a category with an obvious notion
of morphisms. Similarly to the above the colimit of $M$ exists
if and only if the category of cocones has an initial object.
\end{remark}

\noindent
As an application of the notions of limits and colimits
we define products and coproducts.

\begin{definition}
\label{definition-product}
Suppose that $I$ is a set, and suppose given for every $i \in I$ an
object $M_i$ of the category $\mathcal{C}$. A {\it product}
$\prod_{i\in I} M_i$ is by definition $\lim_\mathcal{I} M$
(if it exists)
where $\mathcal{I}$ is the category having only identities as
morphisms and having the elements of $I$ as objects.
\end{definition}

\noindent
An important special case is where $I = \emptyset$ in which case the
product is a final object of the category.
The morphisms $p_i : \prod M_i \to M_i$ are called the
{\it projection morphisms}.

\begin{definition}
\label{definition-coproduct}
Suppose that $I$ is a set, and suppose given for every $i \in I$ an
object $M_i$ of the category $\mathcal{C}$. A {\it coproduct}
$\coprod_{i\in I} M_i$ is by definition $\colim_\mathcal{I} M$
(if it exists) where $\mathcal{I}$ is the category having only
identities as morphisms and having the elements of $I$ as objects.
\end{definition}

\noindent
An important special case is where $I = \emptyset$ in which case the
coproduct is an initial object of the category.
Note that the coproduct comes equipped with morphisms
$M_i \to \coprod M_i$. These are sometimes called the
{\it coprojections}.

\begin{lemma}
\label{lemma-functorial-colimit}
Suppose that $M : \mathcal{I} \to \mathcal{C}$,
and $N : \mathcal{J} \to \mathcal{C}$ are diagrams
whose colimits exist. Suppose
$H : \mathcal{I} \to \mathcal{J}$ is
a functor, and suppose $t : M \to N \circ H$
is a transformation of functors.
Then there is a unique morphism
$$
\theta :
\colim_\mathcal{I} M
\longrightarrow
\colim_\mathcal{J} N
$$
such that all the diagrams
$$
\xymatrix{
M_i \ar[d]_{t_i} \ar[r]
&
\colim_\mathcal{I} M \ar[d]^{\theta}
\\
N_{H(i)} \ar[r]
&
\colim_\mathcal{J} N
}
$$
commute.
\end{lemma}

\begin{proof}
Omitted.
\end{proof}

\begin{lemma}
\label{lemma-functorial-limit}
Suppose that $M : \mathcal{I} \to \mathcal{C}$,
and $N : \mathcal{J} \to \mathcal{C}$ are diagrams
whose limits exist. Suppose $H : \mathcal{I} \to \mathcal{J}$ is
a functor, and suppose $t : N \circ H \to M$
is a transformation of functors.
Then there is a unique morphism
$$
\theta :
\lim_\mathcal{J} N
\longrightarrow
\lim_\mathcal{I} M
$$
such that all the diagrams
$$
\xymatrix{
\lim_\mathcal{J} N \ar[d]^{\theta} \ar[r]
&
N_{H(i)} \ar[d]_{t_i}
\\
\lim_\mathcal{I} M \ar[r]
&
M_i
}
$$
commute.
\end{lemma}

\begin{proof}
Omitted.
\end{proof}


\begin{lemma}
\label{lemma-colimits-commute}
Let $\mathcal{I}$, $\mathcal{J}$ be index categories.
Let $M : \mathcal{I} \times \mathcal{J} \to \mathcal{C}$ be a functor.
We have
$$
\colim_i \colim_j M_{i, j}
=
\colim_{i, j} M_{i, j}
=
\colim_j \colim_i M_{i, j}
$$
provided all the indicated colimits exist. Similar for limits.
\end{lemma}

\begin{proof}
Omitted.
\end{proof}

\begin{lemma}
\label{lemma-limits-products-equalizers}
Let $M : \mathcal{I} \to \mathcal{C}$ be a diagram.
Write $I = \Ob(\mathcal{I})$ and $A = \text{Arrows}(\mathcal{I})$.
Denote $s, t : A \to I$ the source and target maps.
Suppose that $\prod_{i \in I} M_i$ and $\prod_{a \in A} M_{t(a)}$
exist. Suppose that the equalizer of
$$
\xymatrix{
\prod_{i \in I} M_i
\ar@<1ex>[r]^\phi \ar@<-1ex>[r]_\psi
&
\prod_{a \in A} M_{t(a)}
}
$$
exists, where the morphisms are determined by their components
as follows: $p_a \circ \psi = M(a) \circ p_{s(a)}$
and $p_a \circ \phi = p_{t(a)}$. Then this equalizer is the
limit of the diagram.
\end{lemma}

\begin{proof}
Omitted.
\end{proof}


\begin{lemma}
\label{lemma-colimits-coproducts-coequalizers}
\begin{slogan}
If all coproducts and coequalizers exist, all colimits exist.
\end{slogan}
Let $M : \mathcal{I} \to \mathcal{C}$ be a diagram.
Write $I = \Ob(\mathcal{I})$ and $A = \text{Arrows}(\mathcal{I})$.
Denote $s, t : A \to I$ the source and target maps.
Suppose that $\coprod_{i \in I} M_i$ and $\coprod_{a \in A} M_{s(a)}$
exist. Suppose that the coequalizer of
$$
\xymatrix{
\coprod_{a \in A} M_{s(a)}
\ar@<1ex>[r]^\phi \ar@<-1ex>[r]_\psi
&
\coprod_{i \in I} M_i
}
$$
exists, where the morphisms are determined by their components
as follows: The component $M_{s(a)}$ maps via $\psi$
to the component $M_{t(a)}$ via the morphism $a$.
The component $M_{s(a)}$ maps via $\phi$ to the component
$M_{s(a)}$ by the identity morphism. Then this coequalizer is the
colimit of the diagram.
\end{lemma}

\begin{proof}
Omitted.
\end{proof}







\section{Limits and colimits in the category of sets}
\label{section-limit-sets}

\noindent
Not only do limits and colimits exist in $\textit{Sets}$
but they are also easy to describe. Namely, let $M : \mathcal{I}
\to \textit{Sets}$, $i \mapsto M_i$ be a diagram of sets.
Denote $I = \Ob(\mathcal{I})$.
The limit is described as
$$
\lim_\mathcal{I} M
=
\{
(m_i)_{i\in I} \in \prod\nolimits_{i\in I} M_i
\mid
\forall \phi : i \to i' \text{ in }\mathcal{I},
M(\phi)(m_i) = m_{i'}
\}.
$$
So we think of an element of the limit as a compatible system of elements
of all the sets $M_i$.

\medskip\noindent
On the other hand, the colimit is
$$
\colim_\mathcal{I} M
=
(\coprod\nolimits_{i\in I} M_i)/\sim
$$
where the equivalence relation $\sim$ is the equivalence relation
generated by setting $m_i \sim m_{i'}$ if $m_i \in M_i$,
$m_{i'} \in M_{i'}$ and $M(\phi)(m_i) = m_{i'}$ for some
$\phi : i \to i'$. In other words, $m_i \in M_i$
and $m_{i'} \in M_{i'}$ are equivalent if there are a
chain of morphisms in $\mathcal{I}$
$$
\xymatrix{
&
i_1 \ar[ld] \ar[rd] & &
i_3 \ar[ld] & &
i_{2n-1} \ar[rd] & \\
i = i_0 & &
i_2 & &
\ldots & &
i_{2n} = i'
}
$$
and elements $m_{i_j} \in M_{i_j}$ mapping to each other under
the maps $M_{i_{2k-1}} \to M_{i_{2k-2}}$ and $M_{i_{2k-1}}
\to M_{i_{2k}}$ induced from the maps in $\mathcal{I}$ above.

\medskip\noindent
This is not a very pleasant type of object to work with.
But if the diagram is filtered then it is much easier to
describe. We will explain this in Section \ref{section-directed-colimits}.



\section{Connected limits}
\label{section-connected-limits}

\noindent
A (co)limit is called connected if its index category is connected.

\begin{definition}
\label{definition-category-connected}
We say that a category $\mathcal{I}$ is {\it connected}
if the equivalence relation generated by
$x \sim y \Leftrightarrow \Mor_\mathcal{I}(x, y) \not = \emptyset$
has exactly one equivalence class.
\end{definition}

\noindent
Here we follow the convention of
Topology, Definition \ref{topology-definition-connected-components}
that connected spaces are nonempty.
The following in some vague sense characterizes connected limits.

\begin{lemma}
\label{lemma-connected-limit-over-X}
Let $\mathcal{C}$ be a category.
Let $X$ be an object of $\mathcal{C}$.
Let $M : \mathcal{I} \to \mathcal{C}/X$ be a diagram
in the category of objects over $X$.
If the index category $\mathcal{I}$ is connected
and the limit of $M$ exists in $\mathcal{C}/X$,
then the limit of the composition
$\mathcal{I} \to \mathcal{C}/X \to \mathcal{C}$
exists and is the same.
\end{lemma}

\begin{proof}
Let $L \to X$ be an object representing the limit in $\mathcal{C}/X$.
Consider the functor
$$
W \longmapsto \lim_i \Mor_\mathcal{C}(W, M_i).
$$
Let $(\varphi_i)$ be an element of the set on the right.
Since each $M_i$ comes equipped with a morphism $s_i : M_i \to X$ we
get morphisms $f_i = s_i \circ \varphi_i : W \to X$. But as $\mathcal{I}$
is connected we see that all $f_i$ are equal. Since $\mathcal{I}$
is nonempty there is at least one $f_i$.
Hence this common value $W \to X$ defines the structure of an
object of $W$ in $\mathcal{C}/X$ and $(\varphi_i)$ defines an
element of $\lim_i \Mor_{\mathcal{C}/X}(W, M_i)$.
Thus we obtain a unique morphism $\phi : W \to L$ such that
$\varphi_i$ is the composition of $\phi$ with $L \to M_i$ as desired.
\end{proof}

\begin{lemma}
\label{lemma-connected-colimit-under-X}
Let $\mathcal{C}$ be a category.
Let $X$ be an object of $\mathcal{C}$.
Let $M : \mathcal{I} \to X/\mathcal{C}$ be a diagram
in the category of objects under $X$.
If the index category $\mathcal{I}$ is connected
and the colimit of $M$ exists in $X/\mathcal{C}$,
then the colimit of the composition
$\mathcal{I} \to X/\mathcal{C} \to \mathcal{C}$
exists and is the same.
\end{lemma}

\begin{proof}
Omitted. Hint: This lemma is dual to Lemma \ref{lemma-connected-limit-over-X}.
\end{proof}




\section{Cofinal and initial categories}
\label{section-cofinal}

\noindent
In the literature sometimes the word ``final'' is used instead of cofinal
in the following definition.

\begin{definition}
\label{definition-cofinal}
Let $H : \mathcal{I} \to \mathcal{J}$ be a functor between categories.
We say {\it $\mathcal{I}$ is cofinal in $\mathcal{J}$} or that
$H$ is {\it cofinal} if
\begin{enumerate}
\item for all $y \in \Ob(\mathcal{J})$ there exist an
$x \in \Ob(\mathcal{I})$ and a morphism $y \to H(x)$, and
\item given $y \in \Ob(\mathcal{J})$, $x, x' \in \Ob(\mathcal{I})$
and morphisms $y \to H(x)$ and $y \to H(x')$ there exist a sequence
of morphisms
$$
x = x_0 \leftarrow x_1 \rightarrow x_2 \leftarrow x_3 \rightarrow \ldots
\rightarrow x_{2n} = x'
$$
in $\mathcal{I}$ and morphisms $y \to H(x_i)$ in $\mathcal{J}$
such that the diagrams
$$
\xymatrix{
& y \ar[ld] \ar[d] \ar[rd] \\
H(x_{2k}) & H(x_{2k + 1}) \ar[l] \ar[r] & H(x_{2k + 2})
}
$$
commute for $k = 0, \ldots, n - 1$.
\end{enumerate}
\end{definition}

\begin{lemma}
\label{lemma-cofinal}
Let $H : \mathcal{I} \to \mathcal{J}$ be a functor of categories. Assume
$\mathcal{I}$ is cofinal in $\mathcal{J}$. Then for every diagram
$M : \mathcal{J} \to \mathcal{C}$ we have a canonical isomorphism
$$
\colim_\mathcal{I} M \circ H
=
\colim_\mathcal{J} M
$$
if either side exists.
\end{lemma}

\begin{proof}
Omitted.
\end{proof}

\begin{definition}
\label{definition-initial}
Let $H : \mathcal{I} \to \mathcal{J}$ be a functor between categories.
We say {\it $\mathcal{I}$ is initial in $\mathcal{J}$} or that
$H$ is {\it initial} if
\begin{enumerate}
\item for all $y \in \Ob(\mathcal{J})$ there exist an
$x \in \Ob(\mathcal{I})$ and a morphism $H(x) \to y$,
\item for any $y \in \Ob(\mathcal{J})$, $x , x' \in \Ob(\mathcal{I})$ and
morphisms $H(x) \to y$, $H(x') \to y$ in $\mathcal{J}$
there exist a sequence of morphisms
$$
x = x_0 \leftarrow x_1 \rightarrow x_2 \leftarrow x_3 \rightarrow \ldots
\rightarrow x_{2n} = x'
$$
in $\mathcal{I}$ and morphisms $H(x_i) \to y$ in $\mathcal{J}$
such that the diagrams
$$
\xymatrix{
H(x_{2k}) \ar[rd] &
H(x_{2k + 1}) \ar[l] \ar[r] \ar[d] &
H(x_{2k + 2}) \ar[ld] \\
& y
}
$$
commute for $k = 0, \ldots, n - 1$.
\end{enumerate}
\end{definition}

\noindent
This is just the dual notion to ``cofinal'' functors.

\begin{lemma}
\label{lemma-initial}
Let $H : \mathcal{I} \to \mathcal{J}$ be a functor of categories.
Assume $\mathcal{I}$ is initial in $\mathcal{J}$.
Then for every diagram $M : \mathcal{J} \to \mathcal{C}$ we
have a canonical isomorphism
$$
\lim_\mathcal{I} M \circ H = \lim_\mathcal{J} M
$$
if either side exists.
\end{lemma}

\begin{proof}
Omitted.
\end{proof}

\begin{lemma}
\label{lemma-colimit-constant-connected-fibers}
Let $F : \mathcal{I} \to \mathcal{I}'$ be a functor.
Assume
\begin{enumerate}
\item the fibre categories (see
Definition \ref{definition-fibre-category})
of $\mathcal{I}$ over $\mathcal{I}'$ are all connected, and
\item for every morphism $\alpha' : x' \to y'$ in $\mathcal{I}'$ there
exists a morphism $\alpha : x \to y$ in $\mathcal{I}$ such that
$F(\alpha) = \alpha'$.
\end{enumerate}
Then for every diagram $M : \mathcal{I}' \to \mathcal{C}$
the colimit $\colim_\mathcal{I} M \circ F$ exists if and only
if $\colim_{\mathcal{I}'} M$ exists and if so these colimits
agree.
\end{lemma}

\begin{proof}
One can prove this by showing that $\mathcal{I}$ is cofinal in
$\mathcal{I}'$ and applying Lemma \ref{lemma-cofinal}.
But we can also prove it directly as follows.
It suffices to show that for any object $T$ of $\mathcal{C}$ we have
$$
\lim_{\mathcal{I}^{opp}} \Mor_\mathcal{C}(M_{F(i)}, T)
=
\lim_{(\mathcal{I}')^{opp}} \Mor_\mathcal{C}(M_{i'}, T)
$$
If $(g_{i'})_{i' \in \Ob(\mathcal{I}')}$ is an element of
the right hand side, then setting $f_i = g_{F(i)}$ we obtain an
element $(f_i)_{i \in \Ob(\mathcal{I})}$ of the left hand side.
Conversely, let $(f_i)_{i \in \Ob(\mathcal{I})}$ be an element of the
left hand side. Note that on each (connected)
fibre category $\mathcal{I}_{i'}$ the functor $M \circ F$
is constant with value $M_{i'}$. Hence the morphisms
$f_i$ for $i \in \Ob(\mathcal{I})$ with $F(i) = i'$
are all the same and determine a well defined morphism
$g_{i'} : M_{i'} \to T$. By assumption (2) the collection
$(g_{i'})_{i' \in \Ob(\mathcal{I}')}$ defines an element
of the right hand side.
\end{proof}

\begin{lemma}
\label{lemma-product-with-connected}
Let $\mathcal{I}$ and $\mathcal{J}$ be a categories and denote
$p : \mathcal{I} \times \mathcal{J} \to \mathcal{J}$ the projection.
If $\mathcal{I}$ is connected, then for a diagram
$M : \mathcal{J} \to \mathcal{C}$ the colimit $\colim_\mathcal{J} M$ exists
if and only if $\colim_{\mathcal{I} \times \mathcal{J}} M \circ p$ exists and
if so these colimits are equal.
\end{lemma}

\begin{proof}
This is a special case of Lemma \ref{lemma-colimit-constant-connected-fibers}.
\end{proof}






\section{Finite limits and colimits}
\label{section-finite-limits}

\noindent
A {\it finite} (co)limit is a (co)limit whose diagram category is finite,
i.e., the diagram category has finitely many objects and finitely many
morphisms. A (co)limit is called {\it nonempty} if the index category is
nonempty. A (co)limit is called {\it connected} if the index category is
connected, see
Definition \ref{definition-category-connected}.
It turns out that there are ``enough'' finite diagram categories.

\begin{lemma}
\label{lemma-finite-diagram-category}
Let $\mathcal{I}$ be a category with
\begin{enumerate}
\item $\Ob(\mathcal{I})$ is finite, and
\item there exist finitely many morphisms
$f_1, \ldots, f_m \in \text{Arrows}(\mathcal{I})$ such
that every morphism of $\mathcal{I}$ is a composition
$f_{j_1} \circ f_{j_2} \circ \ldots \circ f_{j_k}$.
\end{enumerate}
Then there exists a functor $F : \mathcal{J} \to \mathcal{I}$
such that
\begin{enumerate}
\item[(a)] $\mathcal{J}$ is a finite category, and
\item[(b)] for any diagram $M : \mathcal{I} \to \mathcal{C}$ the
(co)limit of $M$ over $\mathcal{I}$ exists if and only if
the (co)limit of $M \circ F$ over $\mathcal{J}$ exists and in this case
the (co)limits are canonically isomorphic.
\end{enumerate}
Moreover, $\mathcal{J}$ is connected (resp.\ nonempty) if and only if
$\mathcal{I}$ is so.
\end{lemma}

\begin{proof}
Say $\Ob(\mathcal{I}) = \{x_1, \ldots, x_n\}$.
Denote $s, t : \{1, \ldots, m\} \to \{1, \ldots, n\}$ the functions
such that $f_j : x_{s(j)} \to x_{t(j)}$.
We set $\Ob(\mathcal{J}) = \{y_1, \ldots, y_n, z_1, \ldots, z_n\}$
Besides the identity morphisms we introduce morphisms
$g_j : y_{s(j)} \to z_{t(j)}$, $j = 1, \ldots, m$ and morphisms
$h_i : y_i \to z_i$, $i = 1, \ldots, n$. Since all of the nonidentity
morphisms in $\mathcal{J}$ go from a $y$ to a $z$ there are no
compositions to define and no associativities to check.
Set $F(y_i) = F(z_i) = x_i$. Set $F(g_j) = f_j$ and $F(h_i) = \text{id}_{x_i}$.
It is clear that $F$ is a functor.
It is clear that $\mathcal{J}$ is finite.
It is clear that $\mathcal{J}$ is connected, resp.\ nonempty
if and only if $\mathcal{I}$ is so.

\medskip\noindent
Let $M : \mathcal{I} \to \mathcal{C}$ be a diagram.
Consider an object $W$ of $\mathcal{C}$ and morphisms
$q_i : W \to M(x_i)$ as in
Definition \ref{definition-limit}.
Then by taking $q_i : W \to M(F(y_i)) = M(F(z_i)) = M(x_i)$ we obtain
a family of maps as in
Definition \ref{definition-limit}
for the diagram $M \circ F$.
Conversely, suppose we are given maps
$qy_i : W \to M(F(y_i))$ and $qz_i : W \to M(F(z_i))$
as in
Definition \ref{definition-limit}
for the diagram $M \circ F$. Since
$$
M(F(h_i)) = \text{id} : M(F(y_i)) = M(x_i) \longrightarrow M(x_i) = M(F(z_i))
$$
we conclude that $qy_i = qz_i$ for all $i$. Set $q_i$ equal to this common
value. The compatibility of
$q_{s(j)} = qy_{s(j)}$ and $q_{t(j)} = qz_{t(j)}$ with the morphism
$M(f_j)$ guarantees that the family $q_i$ is compatible with all morphisms
in $\mathcal{I}$ as by assumption every such morphism is a composition
of the morphisms $f_j$. Thus we have found a canonical bijection
$$
\lim_{B \in \Ob(\mathcal{J})} \Mor_\mathcal{C}(W, M(F(B)))
=
\lim_{A \in \Ob(\mathcal{I})} \Mor_\mathcal{C}(W, M(A))
$$
which implies the statement on limits in the lemma. The statement on colimits
is proved in the same way (proof omitted).
\end{proof}

\begin{lemma}
\label{lemma-fibre-products-equalizers-exist}
Let $\mathcal{C}$ be a category.
The following are equivalent:
\begin{enumerate}
\item Connected finite limits exist in $\mathcal{C}$.
\item Equalizers and fibre products exist in $\mathcal{C}$.
\end{enumerate}
\end{lemma}

\begin{proof}
Since equalizers and fibre products are finite connected
limits we see that (1) implies (2). For the converse, let $\mathcal{I}$
be a finite connected diagram category. Let
$F : \mathcal{J} \to \mathcal{I}$
be the functor of diagram categories constructed in the proof of
Lemma \ref{lemma-finite-diagram-category}.
Then we see that we may replace $\mathcal{I}$ by $\mathcal{J}$.
The result is that we may assume that
$\Ob(\mathcal{I}) = \{x_1, \ldots, x_n\} \amalg \{y_1, \ldots, y_m\}$
with $n, m \geq 1$ such that all nonidentity morphisms in $\mathcal{I}$
are morphisms $f : x_i \to y_j$ for some $i$ and $j$.

\medskip\noindent
Suppose that $n > 1$. Since $\mathcal{I}$ is connected there
exist indices $i_1, i_2$ and $j_0$ and morphisms $a : x_{i_1} \to y_{j_0}$
and $b : x_{i_2} \to y_{j_0}$. Consider the category
$$
\mathcal{I}' =
\{x\} \amalg \{x_1, \ldots, \hat x_{i_1}, \ldots, \hat x_{i_2}, \ldots x_n\}
\amalg \{y_1, \ldots, y_m\}
$$
with
$$
\Mor_{\mathcal{I}'}(x, y_j) = \Mor_\mathcal{I}(x_{i_1}, y_j)
\amalg \Mor_\mathcal{I}(x_{i_2}, y_j)
$$
and all other morphism sets the same as in $\mathcal{I}$. For any functor
$M : \mathcal{I} \to \mathcal{C}$ we can construct a functor
$M' : \mathcal{I}' \to \mathcal{C}$ by setting
$$
M'(x) = M(x_{i_1}) \times_{M(a), M(y_{j_0}), M(b)} M(x_{i_2})
$$
and for a morphism $f' : x \to y_j$ corresponding to, say,
$f : x_{i_1} \to y_j$ we set $M'(f) = M(f) \circ \text{pr}_1$.
Then the functor $M$ has a limit if and only if the functor $M'$ has
a limit (proof omitted). Hence by induction we reduce to the case $n = 1$.

\medskip\noindent
If $n = 1$, then the limit of any $M : \mathcal{I} \to \mathcal{C}$ is
the successive equalizer of pairs of maps $x_1 \to y_j$ hence
exists by assumption.
\end{proof}

\begin{lemma}
\label{lemma-almost-finite-limits-exist}
Let $\mathcal{C}$ be a category.
The following are equivalent:
\begin{enumerate}
\item Nonempty finite limits exist in $\mathcal{C}$.
\item Products of pairs and equalizers exist in $\mathcal{C}$.
\item Products of pairs and fibre products exist in $\mathcal{C}$.
\end{enumerate}
\end{lemma}

\begin{proof}
Since products of pairs, fibre products, and equalizers are limits with
nonempty index categories we see that (1) implies both (2) and (3).
Assume (2). Then finite nonempty products and equalizers exist. Hence by
Lemma \ref{lemma-limits-products-equalizers}
we see that finite nonempty limits exist, i.e., (1) holds. Assume (3).
If $a, b : A \to B$ are morphisms of $\mathcal{C}$, then the
equalizer of $a, b$ is
$$
(A \times_{a, B, b} A)\times_{(\text{pr}_1, \text{pr}_2), A \times A, \Delta} A.
$$
Thus (3) implies (2), and the lemma is proved.
\end{proof}

\begin{lemma}
\label{lemma-finite-limits-exist}
Let $\mathcal{C}$ be a category.
The following are equivalent:
\begin{enumerate}
\item Finite limits exist in $\mathcal{C}$.
\item Finite products and equalizers exist.
\item The category has a final object and fibre products exist.
\end{enumerate}
\end{lemma}

\begin{proof}
Since products of pairs, fibre products, equalizers, and final objects
are limits over finite index categories we see that (1) implies both (2)
and (3). By
Lemma \ref{lemma-limits-products-equalizers}
above we see that (2) implies (1). Assume (3).
Note that the product $A \times B$ is the fibre product over the
final object. If $a, b : A \to B$ are morphisms of $\mathcal{C}$, then the
equalizer of $a, b$ is
$$
(A \times_{a, B, b} A)\times_{(\text{pr}_1, \text{pr}_2), A \times A, \Delta} A.
$$
Thus (3) implies (2) and the lemma is proved.
\end{proof}

\begin{lemma}
\label{lemma-push-outs-coequalizers-exist}
Let $\mathcal{C}$ be a category.
The following are equivalent:
\begin{enumerate}
\item Connected finite colimits exist in $\mathcal{C}$.
\item Coequalizers and pushouts exist in $\mathcal{C}$.
\end{enumerate}
\end{lemma}

\begin{proof}
Omitted. Hint: This is dual to
Lemma \ref{lemma-fibre-products-equalizers-exist}.
\end{proof}

\begin{lemma}
\label{lemma-almost-finite-colimits-exist}
Let $\mathcal{C}$ be a category.
The following are equivalent:
\begin{enumerate}
\item Nonempty finite colimits exist in $\mathcal{C}$.
\item Coproducts of pairs and coequalizers exist in $\mathcal{C}$.
\item Coproducts of pairs and pushouts exist in $\mathcal{C}$.
\end{enumerate}
\end{lemma}

\begin{proof}
Omitted. Hint: This is the dual of
Lemma \ref{lemma-almost-finite-limits-exist}.
\end{proof}

\begin{lemma}
\label{lemma-colimits-exist}
Let $\mathcal{C}$ be a category.
The following are equivalent:
\begin{enumerate}
\item Finite colimits exist in $\mathcal{C}$,
\item Finite coproducts and coequalizers exist in $\mathcal{C}$, and
\item The category has an initial object and pushouts exist.
\end{enumerate}
\end{lemma}

\begin{proof}
Omitted. Hint: This is dual to Lemma \ref{lemma-finite-limits-exist}.
\end{proof}





\section{Filtered colimits}
\label{section-directed-colimits}

\noindent
Colimits are easier to compute or describe when they
are over a filtered diagram. Here is the definition.

\begin{definition}
\label{definition-directed}
We say that a diagram $M : \mathcal{I} \to \mathcal{C}$ is {\it directed},
or {\it filtered} if the following conditions hold:
\begin{enumerate}
\item the category $\mathcal{I}$ has at least one object,
\item for every pair of objects $x, y$ of $\mathcal{I}$
there exist an object $z$ and morphisms $x \to z$,
$y \to z$, and
\item for every pair of objects $x, y$ of $\mathcal{I}$
and every pair of morphisms $a, b : x \to y$ of $\mathcal{I}$
there exists a morphism $c : y \to z$ of $\mathcal{I}$
such that $M(c \circ a) = M(c \circ b)$ as morphisms in $\mathcal{C}$.
\end{enumerate}
We say that an index category $\mathcal{I}$ is {\it directed}, or
{\it filtered} if $\text{id} : \mathcal{I} \to \mathcal{I}$ is filtered
(in other words you erase the $M$ in part (3) above).
\end{definition}

\noindent
We observe that any diagram with filtered index category is filtered,
and this is how filtered colimits usually come about. In fact, if
$M : \mathcal{I} \to \mathcal{C}$ is a filtered diagram, then we
can factor $M$ as $\mathcal{I} \to \mathcal{I}' \to \mathcal{C}$
where $\mathcal{I}'$ is a filtered index category\footnote{Namely, let
$\mathcal{I}'$ have the same objects as $\mathcal{I}$ but
where $\Mor_{\mathcal{I}'}(x, y)$ is the quotient of $\Mor_\mathcal{I}(x, y)$
by the equivalence relation which identifies
$a, b : x \to y$ if $M(a) = M(b)$.}
such that $\colim_\mathcal{I} M$ exists if and only if
$\colim_{\mathcal{I}'} M'$ exists in which case the colimits are
canonically isomorphic.

\medskip\noindent
Suppose that $M : \mathcal{I} \to \textit{Sets}$ is a filtered diagram. In
this case we may describe the equivalence relation in the formula
$$
\colim_\mathcal{I} M
=
(\coprod\nolimits_{i\in I} M_i)/\sim
$$
simply as follows
$$
m_i \sim m_{i'}
\Leftrightarrow
\exists i'', \phi : i \to i'', \phi': i' \to i'',
M(\phi)(m_i) = M(\phi')(m_{i'}).
$$
In other words, two elements are equal in the colimit if and only if
they ``eventually become equal''.

\begin{lemma}
\label{lemma-directed-commutes}
Let $\mathcal{I}$ and $\mathcal{J}$ be index categories.
Assume that $\mathcal{I}$ is filtered and $\mathcal{J}$ is finite.
Let $M : \mathcal{I} \times \mathcal{J} \to \textit{Sets}$,
$(i, j) \mapsto M_{i, j}$ be a diagram of diagrams of sets.
In this case
$$
\colim_i \lim_j M_{i, j}
=
\lim_j \colim_i M_{i, j}.
$$
In particular, colimits over $\mathcal{I}$ commute with finite products,
fibre products, and equalizers of sets.
\end{lemma}

\begin{proof}
Omitted. In fact, it is a fun exercise to prove that a category is
filtered if and only if colimits over the category commute with finite
limits (into the category of sets).
\end{proof}

\noindent
We give a counter example to the lemma in
the case where $\mathcal{J}$ is infinite. Namely, let
$\mathcal{I}$ consist of $\mathbf{N} = \{1, 2, 3, \ldots\}$
with a unique morphism $i \to i'$ whenever $i \leq i'$.
Let $\mathcal{J}$ consist of the discrete category
$\mathbf{N} = \{1, 2, 3, \ldots\}$ (only morphisms are identities).
Let $M_{i, j} = \{1, 2, \ldots, i\}$ with obvious inclusion maps
$M_{i, j} \to M_{i', j}$ when $i \leq i'$. In this case
$\colim_i M_{i, j} = \mathbf{N}$ and hence
$$
\lim_j \colim_i M_{i, j}
=
\prod\nolimits_j \mathbf{N}
=
\mathbf{N}^\mathbf{N}
$$
On the other hand $\lim_j M_{i, j} = \prod\nolimits_j M_{i, j}$ and
hence
$$
\colim_i \lim_j M_{i, j}
=
\bigcup\nolimits_i \{1, 2, \ldots, i\}^{\mathbf{N}}
$$
which is smaller than the other limit.

\begin{lemma}
\label{lemma-cofinal-in-filtered}
Let $\mathcal{I}$ be a category. Let $\mathcal{J}$ be a full subcategory.
Assume that $\mathcal{I}$ is filtered. Assume also that for any object
$i$ of $\mathcal{I}$, there exists a morphism $i \to j$
to some object $j$ of $\mathcal{J}$. Then $\mathcal{J}$
is filtered and cofinal in $\mathcal{I}$.
\end{lemma}

\begin{proof}
Omitted. Pleasant exercise of the notions involved.
\end{proof}

\noindent
It turns out we sometimes need a more finegrained control over the
possible conditions one can impose on index categories. Thus we add
some lemmas on the possible things one can require.

\begin{lemma}
\label{lemma-preserve-products}
Let $\mathcal{I}$ be an index category, i.e., a category. Assume
that for every pair of objects $x, y$ of $\mathcal{I}$
there exist an object $z$ and morphisms $x \to z$ and $y \to z$.
Then
\begin{enumerate}
\item If $M$ and $N$ are diagrams of sets over $\mathcal{I}$,
then $\colim (M_i \times N_i) \to \colim M_i \times \colim N_i$
is surjective,
\item in general colimits of diagrams of sets over $\mathcal{I}$
do not commute with finite nonempty products.
\end{enumerate}
\end{lemma}

\begin{proof}
Proof of (1). Let $(\overline{m}, \overline{n})$
be an element of $\colim M_i \times \colim N_i$.
Then we can find $m \in M_x$ and $n \in N_y$ for some
$x, y \in \Ob(\mathcal{I})$ such that $m$ mapsto
$\overline{m}$ and $n$ mapsto $\overline{n}$. See
Section \ref{section-limit-sets}.
Choose $a : x \to z$ and $b : y \to z$
in $\mathcal{I}$. Then $(M(a)(m), N(b)(n))$ is an element of
$(M \times N)_z$ whose image in $\colim (M_i \times N_i)$
maps to $(\overline{m}, \overline{n})$ as desired.

\medskip\noindent
Proof of (2). Let $G$ be a non-trivial group and
let $\mathcal{I}$ be the one-object category with endomorphism monoid $G$.
Then $\mathcal{I}$ trivially satisfies the condition stated in the lemma.
Now let $G$ act on itself by translation and view the $G$-set $G$
as a set-valued $\mathcal{I}$-diagram. Then
$$
\colim_\mathcal{I} G \times \colim_\mathcal{I} G \cong G/G \times G/G
$$
is not isomorphic to
$$
\colim_\mathcal{I} (G \times G) \cong (G \times G)/G
$$
This example indicates that you cannot just drop the additional
condition Lemma \ref{lemma-directed-commutes}
even if you only care about finite products.
\end{proof}

\begin{lemma}
\label{lemma-colimits-abelian-as-sets}
Let $\mathcal{I}$ be an index category, i.e., a category. Assume
that for every pair of objects $x, y$ of $\mathcal{I}$
there exist an object $z$ and morphisms $x \to z$ and $y \to z$.
Let $M : \mathcal{I} \to \textit{Ab}$ be a diagram of abelian
groups over $\mathcal{I}$. Then the colimit of $M$ in the category
of sets surjects onto the colimit of $M$ in the category of
abelian groups.
\end{lemma}

\begin{proof}
Recall that the colimit in the category of sets is the quotient of
the disjoint union $\coprod M_i$ by relation, see
Section \ref{section-limit-sets}.
Similarly, the colimit in the category of abelian groups is a quotient
of the direct sum $\bigoplus M_i$.
The assumption of the lemma means that given $i, j \in \Ob(\mathcal{I})$
and $m \in M_i$ and $n \in M_j$, then we can find an object
$k$ and morphisms $a : i \to k$ and $b : j \to k$.
Thus $m + n$ is represented in the colimit by the element
$M(a)(m) + M(b)(n)$ of $M_k$. Thus the $\coprod M_i$
surjects onto the colimit.
\end{proof}

\begin{lemma}
\label{lemma-split-into-connected}
Let $\mathcal{I}$ be an index category, i.e., a category. Assume
that for every solid diagram
$$
\xymatrix{
x \ar[d] \ar[r] & y \ar@{..>}[d] \\
z \ar@{..>}[r] & w
}
$$
in $\mathcal{I}$ there exist an object $w$ and dotted arrows
making the diagram commute. Then $\mathcal{I}$ is either empty
or a nonempty disjoint union of connected categories having
the same property.
\end{lemma}

\begin{proof}
If $\mathcal{I}$ is the empty category, then the lemma is true.
Otherwise, we define a relation on objects of $\mathcal{I}$ by
saying that $x \sim y$ if there exist a $z$ and
morphisms $x \to z$ and $y \to z$. This is an equivalence
relation by the assumption of the lemma. Hence $\Ob(\mathcal{I})$
is a disjoint union of equivalence classes. Let $\mathcal{I}_j$
be the full subcategories corresponding to these equivalence classes.
Then $\mathcal{I} = \coprod \mathcal{I}_j$ with $\mathcal{I}_j$
nonempty as desired.
\end{proof}

\begin{lemma}
\label{lemma-preserve-injective-maps}
Let $\mathcal{I}$ be an index category, i.e., a category. Assume
that for every solid diagram
$$
\xymatrix{
x \ar[d] \ar[r] & y \ar@{..>}[d] \\
z \ar@{..>}[r] & w
}
$$
in $\mathcal{I}$ there exist an object $w$ and dotted arrows
making the diagram commute. Then
\begin{enumerate}
\item an injective morphism $M \to N$ of diagrams of sets over
$\mathcal{I}$ gives rise to an injective map $\colim M_i \to \colim N_i$
of sets,
\item in general the same is not the case for diagrams of abelian
groups and their colimits.
\end{enumerate}
\end{lemma}

\begin{proof}
If $\mathcal{I}$ is the empty category, then the lemma is true.
Thus we may assume $\mathcal{I}$ is nonempty. In this case
we can write $\mathcal{I} = \coprod \mathcal{I}_j$ where each
$\mathcal{I}_j$ is nonempty and satisfies the same property, see
Lemma \ref{lemma-split-into-connected}. Since
$\colim_\mathcal{I} M = \coprod_j \colim_{\mathcal{I}_j} M|_{\mathcal{I}_j}$
this reduces the proof of (1) to the connected case.

\medskip\noindent
Assume $\mathcal{I}$ is connected and $M \to N$ is injective, i.e.,
all the maps $M_i \to N_i$ are injective.
We identify $M_i$ with the image of $M_i \to N_i$, i.e.,
we will think of $M_i$ as a subset of $N_i$.
We will use the description of the colimits given in
Section \ref{section-limit-sets} without further mention.
Let $s, s' \in \colim M_i$ map to the same element of $\colim N_i$.
Say $s$ comes from an element $m$ of $M_i$ and $s'$ comes from an
element $m'$ of $M_{i'}$. Then we can find a sequence
$i = i_0, i_1, \ldots, i_n = i'$ of objects of $\mathcal{I}$
and morphisms
$$
\xymatrix{
&
i_1 \ar[ld] \ar[rd] & &
i_3 \ar[ld] & &
i_{2n-1} \ar[rd] & \\
i = i_0 & &
i_2 & &
\ldots & &
i_{2n} = i'
}
$$
and elements $n_{i_j} \in N_{i_j}$ mapping to each other under
the maps $N_{i_{2k-1}} \to N_{i_{2k-2}}$ and $N_{i_{2k-1}}
\to N_{i_{2k}}$ induced from the maps in $\mathcal{I}$ above
with $n_{i_0} = m$ and $n_{i_{2n}} = m'$. We will prove by induction
on $n$ that this implies $s = s'$. The base case $n = 0$ is trivial.
Assume $n \geq 1$. Using the assumption on $\mathcal{I}$
we find a commutative diagram
$$
\xymatrix{
& i_1 \ar[ld] \ar[rd] \\
i_0 \ar[rd] & & i_2 \ar[ld] \\
& w
}
$$
We conclude that $m$ and $n_{i_2}$ map to the same element of $N_w$
because both are the image of the element $n_{i_1}$.
In particular, this element is an element $m'' \in M_w$ which
gives rise to the same element as $s$ in $\colim M_i$.
Then we find the chain
$$
\xymatrix{
&
i_3 \ar[ld] \ar[rd] & &
i_5 \ar[ld] & &
i_{2n-1} \ar[rd] & \\
w & &
i_4 & &
\ldots & &
i_{2n} = i'
}
$$
and the elements $n_{i_j}$ for $j \geq 3$ which has a smaller length
than the chain we started with. This proves the induction step and the
proof of (1) is complete.

\medskip\noindent
Let $G$ be a group and let $\mathcal{I}$ be the one-object category with
endomorphism monoid $G$. Then $\mathcal{I}$ satisfies the condition stated
in the lemma because given $g_1, g_2 \in G$ we can find $h_1, h_2 \in G$
with $h_1 g_1 = h_2 g_2$. An diagram $M$ over $\mathcal{I}$ in
$\textit{Ab}$ is the same thing as an abelian group $M$ with $G$-action
and $\colim_\mathcal{I} M$ is the coinvariants $M_G$ of $M$.
Take $G$ the group of order $2$ acting trivially on $M = \mathbf{Z}/2\mathbf{Z}$
mapping into the first summand of
$N = \mathbf{Z}/2\mathbf{Z} \times \mathbf{Z}/2\mathbf{Z}$
where the nontrivial element of $G$ acts by
$(x, y) \mapsto (x + y, y)$. Then $M_G \to N_G$ is zero.
\end{proof}

\begin{lemma}
\label{lemma-split-into-directed}
Let $\mathcal{I}$ be an index category, i.e., a category.
Assume
\begin{enumerate}
\item for every pair of morphisms $a : w \to x$ and $b : w \to y$
in $\mathcal{I}$ there exist an object $z$ and morphisms $c : x \to z$
and $d : y \to z$ such that $c \circ a = d \circ b$, and
\item for every pair of morphisms $a, b : x \to y$ there exists
a morphism $c : y \to z$ such that $c \circ a = c \circ b$.
\end{enumerate}
Then $\mathcal{I}$ is a (possibly empty) union
of disjoint filtered index categories $\mathcal{I}_j$.
\end{lemma}

\begin{proof}
If $\mathcal{I}$ is the empty category, then the lemma is true.
Otherwise, we define a relation on objects of $\mathcal{I}$ by
saying that $x \sim y$ if there exist a $z$ and
morphisms $x \to z$ and $y \to z$. This is an equivalence
relation by the first assumption of the lemma. Hence $\Ob(\mathcal{I})$
is a disjoint union of equivalence classes. Let $\mathcal{I}_j$
be the full subcategories corresponding to these equivalence classes.
The rest is clear from the definitions.
\end{proof}

\begin{lemma}
\label{lemma-almost-directed-commutes-equalizers}
Let $\mathcal{I}$ be an index category satisfying the hypotheses of
Lemma \ref{lemma-split-into-directed} above. Then colimits over $\mathcal{I}$
commute with fibre products and equalizers in sets (and more generally
with finite connected limits).
\end{lemma}

\begin{proof}
By
Lemma \ref{lemma-split-into-directed}
we may write $\mathcal{I} = \coprod \mathcal{I}_j$ with each $\mathcal{I}_j$
filtered. By
Lemma \ref{lemma-directed-commutes}
we see that colimits of $\mathcal{I}_j$ commute with equalizers and
fibre products. Thus it suffices to show that equalizers and fibre products
commute with coproducts in the category of sets (including empty coproducts).
In other words, given a set $J$ and sets $A_j, B_j, C_j$ and set maps
$A_j \to B_j$, $C_j \to B_j$ for $j \in J$ we have to show that
$$
(\coprod\nolimits_{j \in J} A_j)
\times_{(\coprod\nolimits_{j \in J} B_j)}
(\coprod\nolimits_{j \in J} C_j)
=
\coprod\nolimits_{j \in J} A_j \times_{B_j} C_j
$$
and given $a_j, a'_j : A_j \to B_j$ that
$$
\text{Equalizer}(
\coprod\nolimits_{j \in J} a_j,
\coprod\nolimits_{j \in J} a'_j)
=
\coprod\nolimits_{j \in J}
\text{Equalizer}(a_j, a'_j)
$$
This is true even if $J = \emptyset$. Details omitted.
\end{proof}



\section{Cofiltered limits}
\label{section-codirected-limits}

\noindent
Limits are easier to compute or describe when they
are over a cofiltered diagram. Here is the definition.

\begin{definition}
\label{definition-codirected}
We say that a diagram $M : \mathcal{I} \to \mathcal{C}$ is {\it codirected}
or {\it cofiltered} if the following conditions hold:
\begin{enumerate}
\item the category $\mathcal{I}$ has at least one object,
\item for every pair of objects $x, y$ of $\mathcal{I}$
there exist an object $z$ and morphisms $z \to x$,
$z \to y$, and
\item for every pair of objects $x, y$ of $\mathcal{I}$
and every pair of morphisms $a, b : x \to y$ of $\mathcal{I}$
there exists a morphism $c : w \to x$ of $\mathcal{I}$
such that $M(a \circ c) = M(b \circ c)$ as morphisms in $\mathcal{C}$.
\end{enumerate}
We say that an index category $\mathcal{I}$ is {\it codirected}, or
{\it cofiltered} if $\text{id} : \mathcal{I} \to \mathcal{I}$ is
cofiltered (in other words you erase the $M$ in part (3) above).
\end{definition}

\noindent
We observe that any diagram with cofiltered index category is cofiltered,
and this is how this situation usually occurs.

\medskip\noindent
As an example of why cofiltered limits of sets are ``easier'' than
general ones, we mention the fact that a cofiltered diagram of finite
nonempty sets has nonempty limit (Lemma \ref{lemma-nonempty-limit}).
This result does not hold for a general limit of finite
nonempty sets.












\section{Limits and colimits over preordered sets}
\label{section-posets-limits}

\noindent
A special case of diagrams is given by systems over preordered sets.

\begin{definition}
\label{definition-directed-set}
Let $I$ be a set and let $\leq$ be a binary relation on $I$.
\begin{enumerate}
\item We say $\leq$ is a {\it preorder} if it is
transitive (if $i \leq j$ and $j \leq k$ then $i \leq k$) and
reflexive ($i \leq i$ for all $i \in I$).
\item A {\it preordered set} is a set endowed with a preorder.
\item A {\it directed set} is a preordered set $(I, \leq)$
such that $I$ is not empty and such that $\forall i, j \in I$,
there exists $k \in I$ with $i \leq k, j \leq k$.
\item We say $\leq$ is a {\it partial order} if it is a preorder
which is antisymmetric (if $i \leq j$ and $j \leq i$, then $i = j$).
\item A {\it partially ordered set} is a set endowed with a partial order.
\item A {\it directed partially ordered set} is a directed set
whose ordering is a partial order.
\end{enumerate}
\end{definition}

\noindent
It is customary to drop the $\leq$ from the notation when talking
about preordered sets, that is, one speaks of
the preordered set $I$ rather than of the preordered set $(I, \leq)$.
Given a preordered set $I$ the symbol $\geq$ is defined by
the rule $i \geq j \Leftrightarrow j \leq i$ for all $i, j \in I$.
The phrase ``partially ordered set'' is sometimes abbreviated to ``poset''.

\medskip\noindent
Given a preordered set $I$ we can construct a category: the objects are
the elements of $I$, there is exactly one morphism $i \to i'$
if $i \leq i'$, and otherwise none. Conversely, given a category $\mathcal{C}$
with at most one arrow between any two objects, the set
$\Ob(\mathcal{C})$ is endowed with a preorder defined by the rule
$x \leq y \Leftrightarrow \Mor_\mathcal{C}(x, y) \not = \emptyset$.

\begin{definition}
\label{definition-system-over-poset}
Let $(I, \leq)$ be a preordered set. Let $\mathcal{C}$ be a category.
\begin{enumerate}
\item A {\it system over $I$ in $\mathcal{C}$}, sometimes called a
{\it inductive system over $I$ in $\mathcal{C}$} is given by
objects $M_i$ of $\mathcal{C}$ and for every $i \leq i'$ a
morphism $f_{ii'} : M_i \to M_{i'}$ such that $f_{ii}
= \text{id}$ and such that $f_{ii''} = f_{i'i''} \circ f_{i i'}$
whenever $i \leq i' \leq i''$.
\item An {\it inverse system over $I$ in $\mathcal{C}$},
sometimes called a {\it projective system over $I$ in $\mathcal{C}$}
is given by objects $M_i$ of $\mathcal{C}$ and for every $i' \leq i$ a
morphism $f_{ii'} : M_i \to M_{i'}$ such that $f_{ii}
= \text{id}$ and such that $f_{ii''} = f_{i'i''} \circ f_{i i'}$
whenever $i'' \leq i' \leq i$. (Note reversal of inequalities.)
\end{enumerate}
We will say $(M_i, f_{ii'})$ is a (inverse) system over $I$ to
denote this. The maps $f_{ii'}$ are sometimes
called the {\it transition maps}.
\end{definition}

\noindent
In other words a system over $I$ is just a diagram
$M : \mathcal{I} \to \mathcal{C}$ where $\mathcal{I}$ is the category
we associated to $I$ above: objects are elements of $I$ and
there is a unique arrow $i \to i'$ in $\mathcal{I}$ if and only if $i \leq i'$.
An inverse system is a diagram $M : \mathcal{I}^{opp} \to \mathcal{C}$.
From this point of view we could take (co)limits of any (inverse)
system over $I$. However, it is customary to take
{\it only colimits of systems over $I$} and
{\it only limits of inverse systems over $I$}.
More precisely: Given a system $(M_i, f_{ii'})$
over $I$ the colimit of the system
$(M_i, f_{ii'})$ is defined as
$$
\colim_{i \in I} M_i = \colim_\mathcal{I} M,
$$
i.e., as the colimit of the corresponding diagram.
Given a inverse system $(M_i, f_{ii'})$ over $I$ the limit
of the inverse system $(M_i, f_{ii'})$ is defined as
$$
\lim_{i \in I} M_i = \lim_{\mathcal{I}^{opp}} M,
$$
i.e., as the limit of the corresponding diagram.

\begin{remark}
\label{remark-preorder-versus-partial-order}
Let $I$ be a preordered set. From $I$ we can construct a canonical
partially ordered set $\overline{I}$ and an order preserving map
$\pi : I \to \overline{I}$. Namely, we can define an equivalence
relation $\sim$ on $I$ by the rule
$$
i \sim j \Leftrightarrow (i \leq j\text{ and }j \leq i).
$$
We set $\overline{I} = I/\sim$ and we let $\pi : I \to \overline{I}$
be the quotient map. Finally, $\overline{I}$ comes with a unique
partial ordering such that
$\pi(i) \leq \pi(j) \Leftrightarrow i \leq j$.
Observe that if $I$ is a directed set, then $\overline{I}$
is a directed partially ordered set.
Given an (inverse) system $N$ over $\overline{I}$ we obtain an
(inverse) system $M$ over $I$ by setting $M_i = N_{\pi(i)}$.
This construction defines a functor between the category
of inverse systems over $I$ and $\overline{I}$.
In fact, this is an equivalence.
The reason is that if $i \sim j$, then for any system
$M$ over $I$ the maps $M_i \to M_j$ and $M_j \to M_i$ are
mutually inverse isomorphisms. More precisely, choosing
a section $s : \overline{I} \to I$ of $\pi$ a quasi-inverse
of the functor above sends $M$ to $N$ with
$N_{\overline{i}} = M_{s(\overline{i})}$.
Finally, this correspondence is compatible with colimits of systems:
if $M$ and $N$ are related as above and
if either $\colim_{\overline{I}} N$ or $\colim_I M$ exists
then so does the other and
$\colim_{\overline{I}} N = \colim_I M$.
Similar results hold for inverse systems and limits of inverse systems.
\end{remark}

\noindent
The upshot of Remark \ref{remark-preorder-versus-partial-order}
is that while computing a colimit of a system or a limit of
an inverse system, we may always assume the preorder is a partial order.

\begin{definition}
\label{definition-directed-system}
Let $I$ be a preordered set. We say a system (resp.\ inverse system)
$(M_i, f_{ii'})$ is a
{\it directed system} (resp.\ {\it directed inverse system})
if $I$ is a directed set
(Definition \ref{definition-directed-set}): $I$ is nonempty and
for all $i_1, i_2 \in I$ there exists $i\in I$ such that
$i_1 \leq i$ and $i_2 \leq i$.
\end{definition}

\noindent
In this case the colimit is sometimes (unfortunately)
called the ``direct limit''. We will not use this last
terminology. It turns out that diagrams over a filtered
category are no more general than directed systems in the
following sense.

\begin{lemma}
\label{lemma-directed-category-system}
Let $\mathcal{I}$ be a filtered index category.
There exist a directed set $I$
and a system $(x_i, \varphi_{ii'})$ over $I$ in $\mathcal{I}$
with the following properties:
\begin{enumerate}
\item For every category $\mathcal{C}$ and every diagram
$M : \mathcal{I} \to \mathcal{C}$ with values in $\mathcal{C}$,
denote $(M(x_i), M(\varphi_{ii'}))$
the corresponding system over $I$. If
$\colim_{i \in I} M(x_i)$ exists then so does
$\colim_\mathcal{I} M$ and the transformation
$$
\theta :
\colim_{i \in I} M(x_i)
\longrightarrow
\colim_\mathcal{I} M
$$
of Lemma \ref{lemma-functorial-colimit} is an isomorphism.
\item For every category $\mathcal{C}$ and every diagram
$M : \mathcal{I}^{opp} \to \mathcal{C}$ in $\mathcal{C}$, denote
$(M(x_i), M(\varphi_{ii'}))$ the corresponding inverse system
over $I$. If $\lim_{i \in I} M(x_i)$ exists then so does
$\lim_\mathcal{I} M$ and the transformation
$$
\theta :
\lim_{\mathcal{I}^{opp}} M
\longrightarrow
\lim_{i \in I} M(x_i)
$$
of Lemma \ref{lemma-functorial-limit} is an isomorphism.
\end{enumerate}
\end{lemma}

\begin{proof}
As explained in the text following
Definition \ref{definition-system-over-poset}, we may view
preordered sets as categories and systems as functors.
Throughout the proof, we will freely shift between these two points of view.
We prove the first statement by constructing a category
$\mathcal{I}_0$, corresponding to a directed set\footnote{In fact,
our construction will produce a directed partially ordered set.}, and a cofinal
functor $M_0 : \mathcal{I}_0 \to \mathcal{I}$. Then, by
Lemma \ref{lemma-cofinal}, the colimit of a diagram
$M : \mathcal{I} \to \mathcal{C}$ coincides with the
colimit of the diagram $M \circ M_0 | \mathcal{I}_0 \to \mathcal{C}$,
from which the statement follows. The second statement is dual to the
first and may be proved by interpreting a limit in $\mathcal{C}$ as
a colimit in $\mathcal{C}^{opp}$. We omit the details.

\medskip\noindent
A category $\mathcal{F}$ is called {\em finitely generated} if
there exists a finite set $F$ of arrows in $\mathcal{F}$, such that
each arrow in $\mathcal{F}$ may be obtained by composing
arrows from $F$. In particular, this implies that $\mathcal{F}$ has
finitely many objects. We start the proof by reducing to the case
when $\mathcal{I}$ has the property that every finitely generated
subcategory of $\mathcal{I}$ may be extended to a finitely
generated subcategory with a unique final object.

\medskip\noindent
Let $\omega$ denote the directed set of finite ordinals, which
we view as a filtered category. It is easy to verify that the
product category $\mathcal{I}\times \omega$ is also filtered,
and the projection
$\Pi : \mathcal{I} \times \omega \to \mathcal{I}$
is cofinal.

\medskip\noindent
Now let $\mathcal{F}$ be any finitely generated
subcategory of $\mathcal{I}\times \omega$.
By using the axioms of a filtered category and a simple induction
argument on a finite set of generators of $\mathcal{F}$,
we may construct a cocone $(\{f_i\}, i_\infty)$ in $\mathcal{I}$
for the diagram $\mathcal{F} \to \mathcal{I}$. That is, a morphism
$f_i : i \to i_\infty$ for every object $i$ in $\mathcal{F}$
such that for each arrow $f : i \to i'$ in $\mathcal{F}$
we have $f_i = f\circ f_{i'}$. We can also choose $i_\infty$ such
that there are no arrows from $i_\infty$ to an object in $\mathcal{F}$.
This is possible since
we may always post-compose the arrows $f_i$ with an arrow
which is the identity on the $\mathcal{I}$-component and
strictly increasing on the $\omega$-component.
Now let $\mathcal{F}^+$ denote the category consisting of all
objects and arrows in $\mathcal{F}$
together with the object $i_\infty$, the identity
arrow $\text{id}_{i_\infty}$ and the arrows $f_i$.
Since there are no arrows from $i_\infty$ in $\mathcal{F}^+$
to any object of $\mathcal{F}$, the arrow set in $\mathcal{F}^+$
is closed under composition, so $\mathcal{F}^+$ is indeed
a category. By construction, it is a finitely
generated subcategory of $\mathcal{I}$ which has $i_\infty$ as
unique final object. Since, by Lemma \ref{lemma-cofinal},
the colimit of  any diagram $M : \mathcal{I} \to \mathcal{C}$
coincides with the colimit of $M\circ\Pi$ , this gives the desired
reduction.

\medskip\noindent
The set of all finitely generated subcategories of $\mathcal{I}$
with a unique final object is naturally ordered by inclusion.
We take $\mathcal{I}_0$ to be the category corresponding
to this set. We also have a functor
$M_0 : \mathcal{I}_0 \to \mathcal{I}$, which takes an
arrow $\mathcal{F} \subset \mathcal{F'}$ in
$\mathcal{I}_0$ to the unique map from the final object of
$\mathcal{F}$ to the final object of $\mathcal{F}'$.
Given any two finitely generated subcategories of
$\mathcal{I}$, the category generated by these two categories is
also finitely generated. By our assumption on $\mathcal{I}$, it is
also contained in a finitely generated subcategory of $\mathcal{I}$
with a unique final object. This shows that $\mathcal{I}_0$ is directed.

\medskip\noindent
Finally, we verify that $M_0$ is cofinal. Since any
object of $\mathcal{I}$ is the final object in the subcategory
consisting of only that object and its identity arrow, the functor
$M_0$ is surjective on objects. In particular, Condition (1) of
Definition \ref{definition-cofinal} is satisfied. Given
an object $i$ of $\mathcal{I}$, $\mathcal{F}_1, \mathcal{F}_2$ in
$\mathcal{I}_0$ and maps $\varphi_1 : i \to M_0(\mathcal{F}_1)$
and $\varphi_2 : i \to M_0(\mathcal{F}_2)$ in
$\mathcal{I}$, we can take $\mathcal{F}_{12}$ to be a finitely
generated category with a unique final object containing
$\mathcal{F}_1$, $\mathcal{F}_2$ and the morphisms $\varphi_1, \varphi_2$.
The resulting diagram commutes
$$
\xymatrix{
& M_0(\mathcal{F}_{12}) & \\
M_0(\mathcal{F}_{1}) \ar[ru] & & M_0(\mathcal{F}_{2}) \ar[lu] \\
& i \ar[lu] \ar[ru]
}
$$
since it lives in the category $\mathcal{F}_{12}$ and
$M_0(\mathcal{F}_{12})$ is final in
this category. Hence also Condition (2) is satisfied, which concludes
the proof.
\end{proof}

\begin{remark}
\label{remark-trick-needed}
Note that a finite directed set $(I, \geq)$ always has a greatest object
$i_\infty$. Hence any colimit of a system $(M_i, f_{ii'})$ over such a set
is trivial in the sense that the colimit equals $M_{i_\infty}$. In contrast,
a colimit indexed by a finite filtered category need not
be trivial. For instance, let $\mathcal{I}$ be the category with a single object
$i$ and a single non-trivial morphism $e$ satisfying $e = e \circ e$. The
colimit of a diagram $M : \mathcal{I} \to Sets$ is the image of the
idempotent $M(e)$. This illustrates that something like the trick of passing
to $\mathcal{I}\times \omega$ in the proof of
Lemma \ref{lemma-directed-category-system} is essential.
\end{remark}

\begin{lemma}
\label{lemma-nonempty-limit}
If $S : \mathcal{I} \to \textit{Sets}$ is a cofiltered diagram of sets
and all the $S_i$ are finite nonempty, then $\lim_i S_i$ is nonempty.
In other words, the limit of a directed inverse system of finite nonempty sets
is nonempty.
\end{lemma}

\begin{proof}
The two statements are equivalent by
Lemma \ref{lemma-directed-category-system}.
Let $I$ be a directed set and let $(S_i)_{i \in I}$
be an inverse system of finite nonempty sets over $I$.
Let us say that a {\it subsystem} $T$ is a family $T = (T_i)_{i \in I}$
of nonempty subsets $T_i \subset S_i$ such that $T_{i'}$ is mapped
into $T_i$ by the transition map $S_{i'} \to S_i$ for all $i' \geq i$.
Denote $\mathcal{T}$ the set of subsystems. We order $\mathcal{T}$
by inclusion. Suppose $T_\alpha$, $\alpha \in A$ is a totally ordered family
of elements of $\mathcal{T}$. Say $T_\alpha = (T_{\alpha, i})_{i \in I}$.
Then we can find a lower bound $T = (T_i)_{i \in I}$ by setting
$T_i = \bigcap_{\alpha \in A} T_{\alpha, i}$ which is manifestly a
finite nonempty subset of $S_i$ as all the $T_{\alpha, i}$ are nonempty
and as the $T_\alpha$ form a totally ordered family. Thus we may
apply Zorn's lemma to see that $\mathcal{T}$ has minimal elements.

\medskip\noindent
Let's analyze what a minimal element $T \in \mathcal{T}$ looks like.
First observe that the maps $T_{i'} \to T_i$ are all surjective.
Namely, as $I$ is a directed set and $T_i$ is finite,
the intersection $T'_i = \bigcap_{i' \geq i} \Im(T_{i'} \to T_i)$
is nonempty. Thus $T' = (T'_i)$ is a subsystem contained in $T$ and
by minimality $T' = T$. Finally, we claim that $T_i$ is a singleton
for each $i$. Namely, if $x \in T_i$, then we can define
$T'_{i'} = (T_{i'} \to T_i)^{-1}(\{x\})$ for $i' \geq i$ and
$T'_j = T_j$ if $j \not \geq i$. This is another subsystem as we've seen
above that the transition maps of the subsystem $T$ are surjective.
By minimality we see that $T = T'$ which indeed implies that $T_i$
is a singleton. This holds for every $i \in I$, hence we see that
$T_i = \{x_i\}$ for some $x_i \in S_i$ with $x_{i'} \mapsto x_i$
under the map $S_{i'} \to S_i$ for every $i' \geq i$. In other words,
$(x_i) \in \lim S_i$ and the lemma is proved.
\end{proof}






\section{Essentially constant systems}
\label{section-essentially-constant}

\noindent
Let $M : \mathcal{I} \to \mathcal{C}$ be a diagram in a category $\mathcal{C}$.
Assume the index category $\mathcal{I}$ is filtered. In this case
there are three successively stronger notions which pick out an object
$X$ of $\mathcal{C}$. The first is just
$$
X = \colim_{i \in \mathcal{I}} M_i.
$$
Then $X$ comes equipped with the coprojections $M_i \to X$.
A stronger condition would be to require that $X$ is the colimit and
that there exist an $i \in \mathcal{I}$ and a morphism $X \to M_i$ such
that the composition $X \to M_i \to X$ is $\text{id}_X$. An even
stronger condition is the following.

\begin{definition}
\label{definition-essentially-constant-diagram}
Let $M : \mathcal{I} \to \mathcal{C}$ be a diagram in a category
$\mathcal{C}$.
\begin{enumerate}
\item Assume the index category $\mathcal{I}$ is filtered and
let $(X, \{M_i \to X\}_i)$ be a cocone for $M$, see
Remark \ref{remark-cones-and-cocones}. We say $M$ is
{\it essentially constant} with {\it value} $X$ if there exist an
$i \in \mathcal{I}$ and a morphism $X \to M_i$ such that
\begin{enumerate}
\item $X \to M_i \to X$ is $\text{id}_X$, and
\item for all $j$ there exist $k$ and morphisms $i \to k$ and $j \to k$
such that the morphism $M_j \to M_k$ equals the composition
$M_j \to X \to M_i \to M_k$.
\end{enumerate}
\item Assume the index category $\mathcal{I}$ is cofiltered and let
$(X, \{X \to M_i\}_i)$ be a cone for $M$, see
Remark \ref{remark-cones-and-cocones}. We say
$M$ is {\it essentially constant} with {\it value} $X$ if
there exist an $i \in \mathcal{I}$
and a morphism $M_i \to X$ such that
\begin{enumerate}
\item $X \to M_i \to X$ is $\text{id}_X$, and
\item for all $j$ there exist $k$ and morphisms $k \to i$ and $k \to j$
such that the morphism $M_k \to M_j$ equals the composition
$M_k \to M_i \to X \to M_j$.
\end{enumerate}
\end{enumerate}
Please keep in mind Lemma \ref{lemma-essentially-constant-is-limit-colimit}
when using this definition.
\end{definition}

\noindent
Which of the two versions is meant will be clear from context. If there is
any confusion we will distinguish between these by saying that the first
version means $M$ is essentially constant as an {\it ind-object}, and in
the second case we will say it is essentially constant as a {\it pro-object}.
This terminology is further explained in
Remarks \ref{remark-ind-category} and \ref{remark-pro-category}.
In fact we will often use the terminology ``essentially constant system''
which formally speaking is only defined for systems over directed sets.

\begin{definition}
\label{definition-essentially-constant-system}
Let $\mathcal{C}$ be a category. A directed system
$(M_i, f_{ii'})$ is an {\it essentially constant system}
if $M$ viewed as a functor $I \to \mathcal{C}$
defines an essentially constant diagram. A directed inverse system
$(M_i, f_{ii'})$ is an {\it essentially constant inverse system} if
$M$ viewed as a functor $I^{opp} \to \mathcal{C}$ defines an
essentially constant inverse diagram.
\end{definition}

\noindent
If $(M_i, f_{ii'})$ is an essentially constant system and the morphisms
$f_{ii'}$ are monomorphisms, then for all $i \leq i'$ sufficiently large the
morphisms $f_{ii'}$ are isomorphisms. In general this need not be the
case however. An example is the system
$$
\mathbf{Z}^2 \to \mathbf{Z}^2 \to \mathbf{Z}^2 \to \ldots
$$
with maps given by $(a, b) \mapsto (a + b, 0)$. This system is essentially
constant with value $\mathbf{Z}$. A non-example is to let
$M = \bigoplus_{n \geq 0} \mathbf{Z}$ and to let $S : M \to M$ be the
shift operator $(a_0, a_1, \ldots) \mapsto (a_1, a_2, \ldots)$. In this
case the system $M \to M \to M \to \ldots$ with transition maps $S$
has colimit $0$ and the composition $0 \to M \to 0$ is the identity,
but the system is not essentially constant.

\medskip\noindent
The following lemma is a sanity check.

\begin{lemma}
\label{lemma-essentially-constant-is-limit-colimit}
Let $M : \mathcal{I} \to \mathcal{C}$ be a diagram.
If $\mathcal{I}$ is filtered and $M$ is essentially
constant as an ind-object, then $X = \colim M_i$ exists and $M$
is essentially constant with value $X$.
If $\mathcal{I}$ is cofiltered and $M$ is essentially
constant as a pro-object, then $X = \lim M_i$ exists and $M$ is
essentially constant with value $X$.
\end{lemma}

\begin{proof}
Omitted. This is a good excercise in the definitions.
\end{proof}

\begin{remark}
\label{remark-ind-category}
Let $\mathcal{C}$ be a category. There exists a big category
$\text{Ind-}\mathcal{C}$ of {\it ind-objects of} $\mathcal{C}$.
Namely, if $F : \mathcal{I} \to \mathcal{C}$ and
$G : \mathcal{J} \to \mathcal{C}$ are filtered diagrams in $\mathcal{C}$,
then we can define
$$
\Mor_{\text{Ind-}\mathcal{C}}(F, G) =
\lim_i \colim_j \Mor_\mathcal{C}(F(i), G(j)).
$$
There is a canonical functor $\mathcal{C} \to \text{Ind-}\mathcal{C}$
which maps $X$ to the {\it constant system} on $X$. This is a fully
faithful embedding. In this language one sees that a diagram $F$ is
essentially constant if and only if $F$ is isomorphic to a constant system.
If we ever need this material, then we will formulate this into a lemma
and prove it here.
\end{remark}

\begin{remark}
\label{remark-pro-category}
Let $\mathcal{C}$ be a category. There exists a big category
$\text{Pro-}\mathcal{C}$ of {\it pro-objects} of $\mathcal{C}$.
Namely, if $F : \mathcal{I} \to \mathcal{C}$ and
$G : \mathcal{J} \to \mathcal{C}$ are cofiltered diagrams in $\mathcal{C}$,
then we can define
$$
\Mor_{\text{Pro-}\mathcal{C}}(F, G) =
\lim_j \colim_i \Mor_\mathcal{C}(F(i), G(j)).
$$
There is a canonical functor $\mathcal{C} \to \text{Pro-}\mathcal{C}$
which maps $X$ to the {\it constant system} on $X$. This is a fully
faithful embedding. In this language one sees that a diagram $F$ is
essentially constant if and only if $F$ is isomorphic to a constant system.
If we ever need this material, then we will formulate this into a lemma
and prove it here.
\end{remark}

\begin{example}
\label{example-pro-morphism-inverse-systems}
Let $\mathcal{C}$ be a category. Let $(X_n)$ and $(Y_n)$ be inverse
systems in $\mathcal{C}$ over $\mathbf{N}$ with the usual ordering.
Picture:
$$
\ldots \to X_3 \to X_2 \to X_1
\quad\text{and}\quad
\ldots \to Y_3 \to Y_2 \to Y_1
$$
Let $a : (X_n) \to (Y_n)$ be a morphism of pro-objects of $\mathcal{C}$.
What does $a$ amount to? Well, for each $n \in \mathbf{N}$ there should
exist an $m(n)$ and a morphism $a_n : X_{m(n)} \to Y_n$. These morphisms
ought to agree in the following sense: for all $n' \geq n$ there exists an
$m(n', n) \geq m(n'), m(n)$ such that the diagram
$$
\xymatrix{
X_{m(n, n')} \ar[rr] \ar[d] & & X_{m(n)} \ar[d]^{a_n} \\
X_{m(n')} \ar[r]^{a_{n'}} & Y_{n'} \ar[r] & Y_n
}
$$
commutes. After replacing $m(n)$ by $\max_{k, l \leq n}\{m(n, k), m(k, l)\}$
we see that we obtain $\ldots \geq m(3) \geq m(2) \geq m(1)$ and a commutative
diagram
$$
\xymatrix{
\ldots \ar[r] &
X_{m(3)} \ar[d]^{a_3} \ar[r] &
X_{m(2)} \ar[d]^{a_2} \ar[r] &
X_{m(1)} \ar[d]^{a_1} \\
\ldots \ar[r] &
Y_3 \ar[r] &
Y_2 \ar[r] &
Y_1
}
$$
Given an increasing map $m' : \mathbf{N} \to \mathbf{N}$ with $m' \geq m$
and setting $a'_i : X_{m'(i)} \to X_{m(i)} \to Y_i$ the pair
$(m', a')$ defines the same morphism of pro-systems. Conversely, given
two pairs $(m_1, a_1)$ and $(m_1, a_2)$ as above then these define the same
morphism of pro-objects if and only if we can find $m' \geq m_1, m_2$
such that $a'_1 = a'_2$.
\end{example}

\begin{remark}
\label{remark-pro-category-copresheaves}
Let $\mathcal{C}$ be a category. Let $F : \mathcal{I} \to \mathcal{C}$ and
$G : \mathcal{J} \to \mathcal{C}$ be cofiltered diagrams in $\mathcal{C}$.
Consider the functors $A, B : \mathcal{C} \to \textit{Sets}$ defined by
$$
A(X) = \colim_i \Mor_\mathcal{C}(F(i), X)
\quad\text{and}\quad
B(X) = \colim_j \Mor_\mathcal{C}(G(j), X)
$$
We claim that a morphism of pro-systems from $F$ to $G$ is the same thing
as a transformation of functors $t : B \to A$. Namely, given $t$
we can apply $t$ to the class of $\text{id}_{G(j)}$ in $B(G(j))$
to get a compatible system of elements
$\xi_j \in A(G(j)) = \colim_i \Mor_\mathcal{C}(F(i), G(j))$
which is exactly our definition of a morphism in $\text{Pro-}\mathcal{C}$ in
Remark \ref{remark-pro-category}. We omit the construction of a
transformation $B \to A$ given a morphism of pro-objects from $F$ to $G$.
\end{remark}

\begin{lemma}
\label{lemma-image-essentially-constant}
Let $\mathcal{C}$ be a category. Let $M : \mathcal{I} \to \mathcal{C}$
be a diagram with filtered (resp.\ cofiltered) index category $\mathcal{I}$.
Let $F : \mathcal{C} \to \mathcal{D}$ be a functor.
If $M$ is essentially constant as an ind-object (resp.\ pro-object),
then so is $F \circ M : \mathcal{I} \to \mathcal{D}$.
\end{lemma}

\begin{proof}
If $X$ is a value for $M$, then it follows immediately from the
definition that $F(X)$ is a value for $F \circ M$.
\end{proof}

\begin{lemma}
\label{lemma-characterize-essentially-constant-ind}
Let $\mathcal{C}$ be a category. Let $M : \mathcal{I} \to \mathcal{C}$
be a diagram with filtered index category $\mathcal{I}$.
The following are equivalent
\begin{enumerate}
\item $M$ is an essentially constant ind-object, and
\item $X = \colim_i M_i$ exists and for any $W$ in $\mathcal{C}$
the map
$$
\colim_i \Mor_\mathcal{C}(W, M_i) \longrightarrow
\Mor_\mathcal{C}(W, X)
$$
is bijective.
\end{enumerate}
\end{lemma}

\begin{proof}
Assume (2) holds. Then $\text{id}_X \in \Mor_\mathcal{C}(X, X)$
comes from a morphism $X \to M_i$ for some $i$, i.e., $X \to M_i \to X$
is the identity. Then both maps
$$
\Mor_\mathcal{C}(W, X)
\longrightarrow
\colim_i \Mor_\mathcal{C}(W, M_i)
\longrightarrow
\Mor_\mathcal{C}(W, X)
$$
are bijective for all $W$ where the first one is induced by the morphism
$X \to M_i$ we found above, and the composition is the identity. This means
that the composition
$$
\colim_i \Mor_\mathcal{C}(W, M_i)
\longrightarrow
\Mor_\mathcal{C}(W, X)
\longrightarrow
\colim_i \Mor_\mathcal{C}(W, M_i)
$$
is the identity too. Setting $W = M_j$ and starting with $\text{id}_{M_j}$
in the colimit, we see that $M_j \to X \to M_i \to M_k$ is equal to
$M_j \to M_k$ for some $k$ large enough. This proves (1) holds.
The proof of (1) $\Rightarrow$ (2) is omitted.
\end{proof}

\begin{lemma}
\label{lemma-characterize-essentially-constant-pro}
Let $\mathcal{C}$ be a category. Let $M : \mathcal{I} \to \mathcal{C}$
be a diagram with cofiltered index category $\mathcal{I}$.
The following are equivalent
\begin{enumerate}
\item $M$ is an essentially constant pro-object, and
\item $X = \lim_i M_i$ exists and for any $W$ in $\mathcal{C}$
the map
$$
\colim_{i \in \mathcal{I}^{opp}} \Mor_\mathcal{C}(M_i, W)
\longrightarrow
\Mor_\mathcal{C}(X, W)
$$
is bijective.
\end{enumerate}
\end{lemma}

\begin{proof}
Assume (2) holds. Then $\text{id}_X \in \Mor_\mathcal{C}(X, X)$
comes from a morphism $M_i \to X$ for some $i$, i.e., $X \to M_i \to X$
is the identity. Then both maps
$$
\Mor_\mathcal{C}(X, W)
\longrightarrow
\colim_i \Mor_\mathcal{C}(M_i, W)
\longrightarrow
\Mor_\mathcal{C}(X, W)
$$
are bijective for all $W$ where the first one is induced by the morphism
$M_i \to X$ we found above, and the composition is the identity. This means
that the composition
$$
\colim_i \Mor_\mathcal{C}(M_i, W)
\longrightarrow
\Mor_\mathcal{C}(X, W)
\longrightarrow
\colim_i \Mor_\mathcal{C}(M_i, W)
$$
is the identity too. Setting $W = M_j$ and starting with $\text{id}_{M_j}$
in the colimit, we see that $M_k \to M_i \to X \to M_j$ is equal to
$M_k \to M_j$ for some $k$ large enough. This proves (1) holds.
The proof of (1) $\Rightarrow$ (2) is omitted.
\end{proof}

\begin{lemma}
\label{lemma-cofinal-essentially-constant}
Let $\mathcal{C}$ be a category. Let $H : \mathcal{I} \to \mathcal{J}$
be a functor of filtered index categories. If $H$ is cofinal, then
any diagram $M : \mathcal{J} \to \mathcal{C}$ is essentially constant
if and only if $M \circ H$ is essentially constant.
\end{lemma}

\begin{proof}
This follows formally from
Lemmas \ref{lemma-characterize-essentially-constant-ind} and
\ref{lemma-cofinal}.
\end{proof}

\begin{lemma}
\label{lemma-essentially-constant-over-product}
Let $\mathcal{I}$ and $\mathcal{J}$ be filtered categories and denote
$p : \mathcal{I} \times \mathcal{J} \to \mathcal{J}$ the projection.
Then $\mathcal{I} \times \mathcal{J}$ is filtered and a diagram
$M : \mathcal{J} \to \mathcal{C}$ is essentially constant if and only
if $M \circ p : \mathcal{I} \times \mathcal{J} \to \mathcal{C}$
is essentially constant.
\end{lemma}

\begin{proof}
We omit the verification that $\mathcal{I} \times \mathcal{J}$ is
filtered. The equivalence follows from
Lemma \ref{lemma-cofinal-essentially-constant}
because $p$ is cofinal (verification omitted).
\end{proof}

\begin{lemma}
\label{lemma-initial-essentially-constant}
Let $\mathcal{C}$ be a category. Let $H : \mathcal{I} \to \mathcal{J}$
be a functor of cofiltered index categories. If $H$ is initial, then
any diagram $M : \mathcal{J} \to \mathcal{C}$ is essentially constant
if and only if $M \circ H$ is essentially constant.
\end{lemma}

\begin{proof}
This follows formally from
Lemmas \ref{lemma-characterize-essentially-constant-pro},
\ref{lemma-initial}, \ref{lemma-cofinal}, and
the fact that if $\mathcal{I}$ is initial in $\mathcal{J}$,
then $\mathcal{I}^{opp}$ is cofinal in $\mathcal{J}^{opp}$.
\end{proof}





\section{Exact functors}
\label{section-exact-functor}


\begin{definition}
\label{definition-exact}
Let $F : \mathcal{A} \to \mathcal{B}$ be a functor.
\begin{enumerate}
\item Suppose all finite limits exist in $\mathcal{A}$.
We say $F$ is {\it left exact} if it commutes
with all finite limits.
\item Suppose all finite colimits exist in $\mathcal{A}$.
We say $F$ is {\it right exact} if it commutes
with all finite colimits.
\item We say $F$ is {\it exact} if it is both left and right
exact.
\end{enumerate}
\end{definition}

\begin{lemma}
\label{lemma-characterize-left-exact}
Let $F : \mathcal{A} \to \mathcal{B}$ be a functor.
Suppose all finite limits exist in $\mathcal{A}$,
see Lemma \ref{lemma-finite-limits-exist}.
The following are equivalent:
\begin{enumerate}
\item $F$ is left exact,
\item $F$ commutes with finite products and equalizers, and
\item $F$ transforms a final object of $\mathcal{A}$
into a final object of $\mathcal{B}$, and commutes with fibre products.
\end{enumerate}
\end{lemma}

\begin{proof}
Lemma \ref{lemma-limits-products-equalizers} shows that (2) implies (1).
Suppose (3) holds. The fibre product over the final object is the product.
If $a, b : A \to B$ are morphisms of $\mathcal{A}$, then the
equalizer of $a, b$ is
$$
(A \times_{a, B, b} A)\times_{(\text{pr}_1, \text{pr}_2), A \times A, \Delta} A.
$$
Thus (3) implies (2). Finally (1) implies (3) because
the empty limit is a final object, and fibre products are limits.
\end{proof}


\section{Adjoint functors}
\label{section-adjoint}

\begin{definition}
\label{definition-adjoint}
Let $\mathcal{C}$, $\mathcal{D}$ be categories.
Let $u : \mathcal{C} \to \mathcal{D}$ and
$v : \mathcal{D} \to \mathcal{C}$ be functors.
We say that $u$ is a {\it left adjoint} of $v$, or that
$v$ is a {\it right adjoint} to $u$ if there are bijections
$$
\Mor_\mathcal{D}(u(X), Y)
\longrightarrow
\Mor_\mathcal{C}(X, v(Y))
$$
functorial in $X \in \Ob(\mathcal{C})$, and
$Y \in \Ob(\mathcal{D})$.
\end{definition}

\noindent
In other words, this means that there is a {\it given} isomorphism of functors
$\mathcal{C}^{opp} \times \mathcal{D} \to \textit{Sets}$ from
$\Mor_\mathcal{D}(u(-), -)$ to $\Mor_\mathcal{C}(-, v(-))$. For any object
$X$ of $\mathcal{C}$ we obtain a morphism $X \to v(u(X))$ corresponding to
$\text{id}_{u(X)}$. Similarly, for any object $Y$ of $\mathcal{D}$ we obtain
a morphism $u(v(Y)) \to Y$ corresponding to $\text{id}_{v(Y)}$.
These maps are called the {\it adjunction maps}. The adjunction maps
are functorial in $X$ and $Y$, hence we obtain morphisms of functors
$$
\eta : \text{id}_\mathcal{C} \to v \circ u\quad (\text{unit})
\quad\text{and}\quad
\epsilon : u \circ v \to \text{id}_\mathcal{D}\quad (\text{counit}).
$$
Moreover, if $\alpha : u(X) \to Y$
and $\beta : X \to v(Y)$ are morphisms, then the following are equivalent
\begin{enumerate}
\item $\alpha$ and $\beta$ correspond to each other via the
bijection of the definition,
\item $\beta$ is the composition $X \to v(u(X)) \xrightarrow{v(\alpha)} v(Y)$,
and
\item $\alpha$ is the composition $u(X) \xrightarrow{u(\beta)} u(v(Y)) \to Y$.
\end{enumerate}
In this way one can reformulate the notion of adjoint functors in terms
of adjunction maps.

\begin{lemma}
\label{lemma-adjoint-exists}
Let $u : \mathcal{C} \to \mathcal{D}$ be a functor between categories.
If for each $y \in \Ob(\mathcal{D})$ the functor
$x \mapsto \Mor_\mathcal{D}(u(x), y)$ is representable, then
$u$ has a right adjoint.
\end{lemma}

\begin{proof}
For each $y$ choose an object $v(y)$ and an isomorphism
$\Mor_\mathcal{C}(-, v(y)) \to \Mor_\mathcal{D}(u(-), y)$
of functors. By Yoneda's lemma (Lemma \ref{lemma-yoneda})
for any morphism $g : y \to y'$ the transformation of functors
$$
\Mor_\mathcal{C}(-, v(y)) \to \Mor_\mathcal{D}(u(-), y) \to
\Mor_\mathcal{D}(u(-), y') \to \Mor_\mathcal{C}(-, v(y'))
$$
corresponds to a unique morphism $v(g) : v(y) \to v(y')$.
We omit the verification that $v$ is a functor and that
it is right adjoint to $u$.
\end{proof}

\begin{lemma}
\label{lemma-left-adjoint-composed-fully-faithful}
\begin{reference}
Bhargav Bhatt, private communication.
\end{reference}
Let $u$ be a left adjoint to $v$ as in Definition \ref{definition-adjoint}.
\begin{enumerate}
\item If $v \circ u$ is fully faithful, then $u$ is fully faithful.
\item If $u \circ v$ is fully faithful, then $v$ is fully faithful.
\end{enumerate}
\end{lemma}

\begin{proof}
Proof of (2). Assume $u \circ v$ is fully faithful.
Say we have $X$, $Y$ in $\mathcal{D}$.
Then the natural composite map
$$
\Mor(X,Y) \to \Mor(v(X),v(Y)) \to \Mor(u(v(X)), u(v(Y)))
$$
is a bijection, so $v$ is at least faithful. To show full faithfulness,
we must show that the second map above is injective.
But the adjunction between $u$ and $v$ says that
$$
\Mor(v(X), v(Y)) \to \Mor(u(v(X)), u(v(Y))) \to \Mor(u(v(X)), Y)
$$
is a bijection, where the first map is natural one and
the second map comes from the counit $u(v(Y)) \to Y$ of the adjunction.
So this says that
$\Mor(v(X), v(Y)) \to \Mor(u(v(X)), u(v(Y)))$
is also injective, as wanted. The proof of (1) is dual to this.
\end{proof}

\begin{lemma}
\label{lemma-adjoint-fully-faithful}
Let $u$ be a left adjoint to $v$ as in Definition \ref{definition-adjoint}.
Then
\begin{enumerate}
\item $u$ is fully faithful $\Leftrightarrow$ $\text{id} \cong v \circ u$
$\Leftrightarrow$ $\eta : \textit{id} \to v \circ u$ is an isomorphism,
\item $v$ is fully faithful $\Leftrightarrow$
$u \circ v \cong \text{id}$ $\Leftrightarrow$
$\epsilon : u \circ v \to \text{id}$ is an isomorphism.
\end{enumerate}
\end{lemma}

\begin{proof}
Proof of (1).
Assume $u$ is fully faithful. We will show $\eta_X : X \to v(u(X))$
is an isomorphism. Let $X' \to v(u(X))$ be any morphism.
By adjointness this corresponds to a morphism $u(X') \to u(X)$. By fully
faithfulness of $u$ this corresponds to a unique morphism $X' \to X$.
Thus we see that post-composing by $\eta_X$ defines a bijection
$\Mor(X', X) \to \Mor(X', v(u(X)))$. Hence $\eta_X$ is an isomorphism.
If there exists an isomorphism $\text{id} \cong v \circ u$ of functors,
then $v \circ u$ is fully faithful. By
Lemma \ref{lemma-left-adjoint-composed-fully-faithful} we see
that $u$ is fully faithful. By the above this implies $\eta$
is an isomorphism. Thus all $3$ conditions are equivalent (and these
conditions are also equivalent to $v \circ u$ being fully faithful).

\medskip\noindent
Part (2) is dual to part (1).
\end{proof}

\begin{lemma}
\label{lemma-adjoint-exact}
Let $u$ be a left adjoint to $v$ as in Definition \ref{definition-adjoint}.
\begin{enumerate}
\item Suppose that $M : \mathcal{I} \to \mathcal{C}$ is a diagram,
and suppose that $\colim_\mathcal{I} M$ exists in
$\mathcal{C}$. Then $u(\colim_\mathcal{I} M) =
\colim_\mathcal{I} u \circ M$. In other words,
$u$ commutes with (representable) colimits.
\item Suppose that $M : \mathcal{I} \to \mathcal{D}$ is a diagram,
and suppose that $\lim_\mathcal{I} M$ exists in
$\mathcal{D}$. Then $v(\lim_\mathcal{I} M) =
\lim_\mathcal{I} v \circ M$. In other words $v$ commutes
with representable limits.
\end{enumerate}
\end{lemma}

\begin{proof}
A morphism from a colimit into an object is the same as a compatible
system of morphisms from the constituents of the limit into the
object, see Remark \ref{remark-limit-colim}. So
$$
\begin{matrix}
\Mor_\mathcal{D}(u(\colim_{i \in \mathcal{I}} M_i), Y) &
= & \Mor_\mathcal{C}(\colim_{i \in \mathcal{I}} M_i, v(Y)) \\
& = &
\lim_{i \in \mathcal{I}^{opp}}
\Mor_\mathcal{C}(M_i, v(Y)) \\
& = &
\lim_{i \in \mathcal{I}^{opp}}
\Mor_\mathcal{D}(u(M_i), Y)
\end{matrix}
$$
proves that $u(\colim_{i \in \mathcal{I}} M_i)$ is
the colimit we are looking for.
A similar argument works for the other statement.
\end{proof}

\begin{lemma}
\label{lemma-exact-adjoint}
Let $u$ be a left adjoint of $v$ as in Definition \ref{definition-adjoint}.
\begin{enumerate}
\item If $\mathcal{C}$ has finite colimits, then $u$ is right exact.
\item If $\mathcal{D}$ has finite limits, then $v$ is left exact.
\end{enumerate}
\end{lemma}

\begin{proof}
Obvious from the definitions and Lemma \ref{lemma-adjoint-exact}.
\end{proof}

\begin{lemma}
\label{lemma-unit-counit-relations}
Let $u : \mathcal{C} \to \mathcal{D}$ be a left adjoint to the functor
$v : \mathcal{D} \to \mathcal{C}$. Let $\eta_X : X \to v(u(X))$ be the unit
and $\epsilon_Y : u(v(Y)) \to Y$ be the counit. Then
$$
u(X) \xrightarrow{u(\eta_X)} u(v(u(X))
\xrightarrow{\epsilon_{u(X)}} u(X)
\quad\text{and}\quad
v(Y) \xrightarrow{\eta_{v(Y)}} v(u(v(Y))) \xrightarrow{v(\epsilon_Y)}
v(Y)
$$
are the identity morphisms.
\end{lemma}

\begin{proof}
Omitted.
\end{proof}

\begin{lemma}
\label{lemma-transformation-between-functors-and-adjoints}
Let $u_1, u_2 : \mathcal{C} \to \mathcal{D}$ be functors with right
adjoints $v_1, v_2 : \mathcal{D} \to \mathcal{C}$. Let $\beta : u_2 \to u_1$
be a transformation of functors. Let $\beta^\vee : v_1 \to v_2$ be
the corresponding transformation of adjoint functors. Then
$$
\xymatrix{
u_2 \circ v_1 \ar[r]_\beta \ar[d]_{\beta^\vee} &
u_1 \circ v_1 \ar[d] \\
u_2 \circ v_2 \ar[r] & \text{id}
}
$$
is commutative where the unlabeled arrows are the counit transformations.
\end{lemma}

\begin{proof}
This is true because $\beta^\vee_D : v_1D \to v_2D$ is the unique
morphism such that the induced maps $\Mor(C, v_1D) \to \Mor(C, v_2D)$
is the map $\Mor(u_1C, D) \to \Mor(u_2C, D)$ induced by
$\beta_C : u_2C \to u_1C$. Namely, this means the map
$$
\Mor(u_1 v_1 D, D') \to \Mor(u_2 v_1 D, D')
$$
induced by $\beta_{v_1 D}$ is the same as the map
$$
\Mor(v_1 D, v_1 D') \to \Mor(v_1 D, v_2 D')
$$
induced by $\beta^\vee_{D'}$. Taking $D' = D$ we find that the counit
$u_1 v_1 D \to D$ precomposed by $\beta_{v_1D}$ corresponds to $\beta^\vee_D$
under adjunction. This exactly means that the diagram commutes when
evaluated on $D$.
\end{proof}

\begin{lemma}
\label{lemma-compose-counits}
Let $\mathcal{A}$, $\mathcal{B}$, and $\mathcal{C}$ be categories.
Let $v : \mathcal{A} \to \mathcal{B}$ and
$v' : \mathcal{B} \to \mathcal{C}$ be functors
with left adjoints $u$ and $u'$ respectively. Then
\begin{enumerate}
\item The functor $v'' = v' \circ v$ has a left adjoint equal to
$u'' = u \circ u'$.
\item Given $X$ in $\mathcal{A}$ we have
\begin{equation}
\label{equation-compose-counits}
\epsilon_X^v \circ u(\epsilon^{v'}_{v(X)}) = \epsilon^{v''}_X :
u''(v''(X)) \to X
\end{equation}
Where $\epsilon$ is the counit of the adjunctions.
\end{enumerate}
\end{lemma}

\begin{proof}
Let us unwind the formula in (2) because this will also immediately
prove (1). First, the counit of the adjunctions for the pairs
$(u, v)$ and $(u', v')$ are maps
$\epsilon_X^v : u(v(X)) \to X$ and
$\epsilon_Y^{v'} : u'(v'(Y)) \to Y$, see discussion following
Definition \ref{definition-adjoint}.
With $u''$ and $v''$ as in (1) we unwind everything
$$
u''(v''(X)) = u(u'(v'(v(X)))) \xrightarrow{u(\epsilon_{v(X)}^{v'})}
u(v(X)) \xrightarrow{\epsilon_X^v} X
$$
to get the map on the left hand side of (\ref{equation-compose-counits}).
Let us denote this by $\epsilon_X^{v''}$ for now.
To see that this is the counit of an adjoint pair
$(u'', v'')$ we have to show that given $Z$ in $\mathcal{C}$
the rule that sends a morphism $\beta : Z \to v''(X)$
to $\alpha = \epsilon_X^{v''} \circ u''(\beta) : u''(Z) \to X$
is a bijection on sets of morphisms.
This is true because, this is the composition of the
rule sending $\beta$ to $\epsilon_{v(X)}^{v'} \circ u'(\beta)$
which is a bijection by assumption on $(u', v')$ and then
sending this to
$\epsilon_X^v \circ u(\epsilon_{v(X)}^{v'} \circ u'(\beta))$
which is a bijection by assumption on $(u, v)$.
\end{proof}






\section{A criterion for representability}
\label{section-representable}

\noindent
The following lemma is often useful to prove the existence of
universal objects in big categories, please see the discussion
in Remark \ref{remark-how-to-use-it}.

\begin{lemma}
\label{lemma-a-version-of-brown}
Let $\mathcal{C}$ be a big\footnote{See Remark \ref{remark-big-categories}.}
category which has limits. Let $F : \mathcal{C} \to \textit{Sets}$ be a
functor. Assume that
\begin{enumerate}
\item $F$ commutes with limits,
\item there exist a family $\{x_i\}_{i \in I}$ of objects of $\mathcal{C}$
and for each $i \in I$ an element $f_i \in F(x_i)$
such that for $y \in \Ob(\mathcal{C})$ and $g \in F(y)$
there exist an $i$ and a morphism $\varphi : x_i \to y$
with $F(\varphi)(f_i) = g$.
\end{enumerate}
Then $F$ is representable, i.e., there exists an object $x$
of $\mathcal{C}$ such that
$$
F(y) = \Mor_\mathcal{C}(x, y)
$$
functorially in $y$.
\end{lemma}

\begin{proof}
Let $\mathcal{I}$ be the category whose objects are the pairs $(x_i, f_i)$
and whose morphisms $(x_i, f_i) \to (x_{i'}, f_{i'})$ are maps
$\varphi : x_i \to x_{i'}$ in $\mathcal{C}$
such that $F(\varphi)(f_i) = f_{i'}$. Set
$$
x = \lim_{(x_i, f_i) \in \mathcal{I}} x_i
$$
(this will not be the $x$ we are looking for, see below).
The limit exists by assumption. As $F$ commutes with limits
we have
$$
F(x) = \lim_{(x_i, f_i) \in \mathcal{I}} F(x_i).
$$
Hence there is a universal element $f \in F(x)$ which maps to $f_i \in F(x_i)$
under $F$ applied to the projection map $x \to x_i$.
Using $f$ we obtain a transformation of functors
$$
\xi : \Mor_\mathcal{C}(x, - ) \longrightarrow F(-)
$$
see Section \ref{section-opposite}. Let $y$ be an arbitrary object of
$\mathcal{C}$ and let $g \in F(y)$. Choose $x_i \to y$ such that $f_i$
maps to $g$ which is possible by assumption. Then $F$ applied to the maps
$$
x \longrightarrow x_i \longrightarrow y
$$
(the first being the projection map of the limit defining $x$)
sends $f$ to $g$. Hence the transformation $\xi$ is surjective.

\medskip\noindent
In order to find the object representing $F$ we let $e : x' \to x$ be the
equalizer of all self maps $\varphi : x \to x$ with $F(\varphi)(f) = f$.
Since $F$ commutes with limits, it commutes with equalizers, and
we see there exists an $f' \in F(x')$ mapping to $f$ in $F(x)$.
Since $\xi$ is surjective and since $f'$ maps to $f$ we see that
also $\xi' : \Mor_\mathcal{C}(x', -) \to F(-)$ is surjective.
Finally, suppose that $a, b : x' \to y$ are two maps such that
$F(a)(f') = F(b)(f')$. We have to show $a = b$. Consider the equalizer
$e' : x'' \to x'$. Again we find $f'' \in F(x'')$ mapping to $f'$.
Choose a map $\psi : x \to x''$ such that $F(\psi)(f) = f''$.
Then we see that $e \circ e' \circ \psi : x \to x$ is a morphism
with $F(e \circ e' \circ \psi)(f) = f$. Hence
$e \circ e' \circ \psi \circ e = e$. Since $e$ is a monomorphism,
this implies that $e'$ is an epimorphism, thus $a = b$ as desired.
\end{proof}

\begin{remark}
\label{remark-how-to-use-it}
The lemma above is often used to construct the free something on something.
For example the free abelian group on a set, the free group on a set, etc.
The idea, say in the case of the free group on a set $E$ is to
consider the functor
$$
F : \textit{Groups} \to \textit{Sets},\quad
G \longmapsto \text{Map}(E, G)
$$
This functor commutes with limits. As our family of objects
we can take a family $E \to G_i$ consisting of groups $G_i$
of cardinality at most $\max(\aleph_0, |E|)$ and set maps
$E \to G_i$ such that every isomorphism class of such a structure
occurs at least once. Namely, if $E \to G$ is a map from $E$ to
a group $G$, then the subgroup $G'$ generated by the image has
cardinality at most $\max(\aleph_0, |E|)$. The lemma tells us
the functor is representable, hence there exists a group
$F_E$ such that $\Mor_{\textit{Groups}}(F_E, G) = \text{Map}(E, G)$.
In particular, the identity morphism of $F_E$ corresponds to
a map $E \to F_E$ and one can show that $F_E$ is generated by
the image without imposing any relations.

\medskip\noindent
Another typical application is that we can use the lemma to construct
colimits once it is known that limits exist. We illustrate it using
the category of topological spaces which has limits by
Topology, Lemma \ref{topology-lemma-limits}. Namely, suppose
that $\mathcal{I} \to \textit{Top}$, $i \mapsto X_i$ is a functor.
Then we can consider
$$
F : \textit{Top} \longrightarrow \textit{Sets},\quad
Y \longmapsto \lim_\mathcal{I} \Mor_{\textit{Top}}(X_i, Y)
$$
This functor commutes with limits. Moreover, given any topological space
$Y$ and an element $(\varphi_i : X_i \to Y)$ of $F(Y)$, there is
a subspace $Y' \subset Y$ of cardinality at most $|\coprod X_i|$
such that the morphisms $\varphi_i$ map into $Y'$. Namely, we can
take the induced topology on the union of the images of the $\varphi_i$.
Thus it is clear that the hypotheses of the lemma are satisfied and we find a
topological space $X$
representing the functor $F$, which precisely means that $X$ is
the colimit of the diagram $i \mapsto X_i$.
\end{remark}

\begin{theorem}[Adjoint functor theorem]
\label{theorem-adjoint-functor}
Let $G : \mathcal{C} \to \mathcal{D}$ be a functor of big categories.
Assume $\mathcal{C}$ has limits, $G$ commutes with them, and for
every object $y$ of $\mathcal{D}$ there exists a set of pairs
$(x_i, f_i)_{i \in I}$ with $x_i \in \Ob(\mathcal{C})$,
$f_i \in \Mor_\mathcal{D}(y, G(x_i))$ such that for any
pair $(x, f)$ with $x \in \Ob(\mathcal{C})$,
$f \in \Mor_\mathcal{D}(y, G(x))$ there are an $i$ and a morphism
$h : x_i \to x$ such that $f = G(h) \circ f_i$.
Then $G$ has a left adjoint $F$.
\end{theorem}

\begin{proof}
The assumptions imply that for every object $y$ of $\mathcal{D}$
the functor $x \mapsto \Mor_\mathcal{D}(y, G(x))$ satisfies the
assumptions of Lemma \ref{lemma-a-version-of-brown}.
Thus it is representable by an object, let's call it $F(y)$.
An application of Yoneda's lemma (Lemma \ref{lemma-yoneda})
turns the rule $y \mapsto F(y)$ into a functor which
by construction is an adjoint to $G$. We omit the details.
\end{proof}





\section{Categorically compact objects}
\label{section-compact}

\noindent
A little bit about ``small'' objects of a category.

\begin{definition}
\label{definition-compact-object}
Let $\mathcal{C}$ be a big\footnote{See Remark \ref{remark-big-categories}.}
category. An object $X$ of $\mathcal{C}$ is called a {\it categorically compact}
if we have
$$
\Mor_\mathcal{C}(X, \colim_i M_i) =
\colim_i \Mor_\mathcal{C}(X, M_i)
$$
for every filtered diagram $M : \mathcal{I} \to \mathcal{C}$ such that
$\colim_i M_i$ exists.
\end{definition}

\noindent
Often this definition is made only under the assumption that $\mathcal{C}$
has all filtered colimits.

\begin{lemma}
\label{lemma-extend-functor-by-colim}
Let $\mathcal{C}$ and $\mathcal{D}$ be big categories having filtered
colimits. Let $\mathcal{C}' \subset \mathcal{C}$ be a small full subcategory
consisting of categorically compact objects of $\mathcal{C}$ such that every
object of $\mathcal{C}$ is a filtered colimit of objects of $\mathcal{C}'$.
Then every functor $F' : \mathcal{C}' \to \mathcal{D}$ has a unique
extension $F : \mathcal{C} \to \mathcal{D}$ commuting with filtered colimits.
\end{lemma}

\begin{proof}
For every object $X$ of $\mathcal{C}$ we may write $X$ as a filtered
colimit $X = \colim X_i$ with $X_i \in \Ob(\mathcal{C}')$. Then we set
$$
F(X) = \colim F'(X_i)
$$
in $\mathcal{D}$. We will show below that this construction does not
depend on the choice of the colimit presentation of $X$.

\medskip\noindent
Suppose given a morphism $\alpha : X \to Y$ of $\mathcal{C}$
and $X = \colim_{i \in I} X_i$ and $Y = \colim_{j \in J} Y_i$
are written as filtered colimit of objects in $\mathcal{C}'$.
For each $i \in I$ since $X_i$ is a categorically compact object of
$\mathcal{C}$ we can find a $j \in J$ and a commutative diagram
$$
\xymatrix{
X_i \ar[r] \ar[d] & X \ar[d]^\alpha \\
Y_j \ar[r] & Y
}
$$
Then we obtain a morphism $F'(X_i) \to F'(Y_j) \to F(Y)$ where the second
morphism is the coprojection into $F(Y) = \colim F'(Y_j)$. The arrow
$\beta_i : F'(X_i) \to F(Y)$ does not depend on the choice of $j$.
For $i \leq i'$ the composition
$$
F'(X_i) \to F'(X_{i'}) \xrightarrow{\beta_{i'}} F(Y)
$$
is equal to $\beta_i$. Thus we obtain a well defined arrow
$$
F(\alpha) : F(X) = \colim F(X_i) \to F(Y)
$$
by the universal property of the colimit. If $\alpha' : Y \to Z$ is a
second morphism of $\mathcal{C}$ and $Z = \colim Z_k$ is also
written as filtered colimit of objects in $\mathcal{C}'$, then
it is a pleasant exercise to show that the induced
morphisms $F(\alpha) : F(X) \to F(Y)$ and $F(\alpha') : F(Y) \to F(Z)$
compose to the morphism $F(\alpha' \circ \alpha)$. Details omitted.

\medskip\noindent
In particular, if we are given two presentations
$X = \colim X_i$ and $X = \colim X'_{i'}$ as filtered
colimits of systems in $\mathcal{C}'$, then we get mutually inverse
arrows $\colim F'(X_i) \to \colim F'(X'_{i'})$ and
$\colim F'(X'_{i'}) \to \colim F'(X_i)$. In other words, the
value $F(X)$ is well defined independent of the choice of the
presentation of $X$ as a filtered colimit of objects of $\mathcal{C}'$.
Together with the functoriality of $F$ discussed in the previous
paragraph, we find that $F$ is a functor. Also, it is clear that
$F(X) = F'(X)$ if $X \in \Ob(\mathcal{C}')$.

\medskip\noindent
The uniqueness statement in the lemma is clear, provided
we show that $F$ commutes with filtered colimits (because this
statement doesn't make sense otherwise). To show this, suppose that
$X = \colim_{\lambda \in \Lambda} X_\lambda$
is a filtered colimit of $\mathcal{C}$. Since $F$ is a functor
we certainly get a map
$$
\colim_\lambda F(X_\lambda) \longrightarrow F(X)
$$
On the other hand, write $X = \colim X_i$
as a filtered colimit of objects of $\mathcal{C}'$. 
As above, for each $i \in I$ we can choose a $\lambda \in \Lambda$
and a commutative diagram
$$
\xymatrix{
X_i \ar[rr] \ar[rd] & & X_\lambda \ar[ld] \\
& X
}
$$
As above this determines a well defined morphism
$F'(X_i) \to \colim_\lambda F(X_\lambda)$ compatible
with transition morphisms and hence a morphism
$$
F(X) = \colim_i F'(X_i) \longrightarrow \colim_\lambda F(X_\lambda)
$$
This morphism is inverse to the morphism above (details omitted)
and proves that $F(X) = \colim_\lambda F(X_\lambda)$ as desired.
\end{proof}







\section{Localization in categories}
\label{section-localization}

\noindent
The basic idea of this section is given a category $\mathcal{C}$
and a set of arrows $S$ to construct a functor
$F : \mathcal{C} \to S^{-1}\mathcal{C}$
such that all elements of $S$ become invertible in $S^{-1}\mathcal{C}$
and such that $F$ is universal among all functors with this property.
References for this section are \cite[Chapter I, Section 2]{GZ}
and \cite[Chapter II, Section 2]{Verdier}.

\begin{definition}
\label{definition-multiplicative-system}
Let $\mathcal{C}$ be a category. A set of arrows $S$ of $\mathcal{C}$ is
called a {\it left multiplicative system} if it has the following properties:
\begin{enumerate}
\item[LMS1] The identity of every object of $\mathcal{C}$ is in $S$ and
the composition of two composable elements of $S$ is in $S$.
\item[LMS2] Every solid diagram
$$
\xymatrix{
X \ar[d]_t \ar[r]_g & Y \ar@{..>}[d]^s \\
Z \ar@{..>}[r]^f & W
}
$$
with $t \in S$ can be completed to a commutative dotted square with
$s \in S$.
\item[LMS3] For every pair of morphisms $f, g : X \to Y$ and
$t \in S$ with target $X$ such that $f \circ t = g \circ t$
there exists an $s \in S$ with source $Y$ such that
$s \circ f = s \circ g$.
\end{enumerate}
A set of arrows $S$ of $\mathcal{C}$ is
called a {\it right multiplicative system}
if it has the following properties:
\begin{enumerate}
\item[RMS1] The identity of every object of $\mathcal{C}$ is in $S$ and
the composition of two composable elements of $S$ is in $S$.
\item[RMS2] Every solid diagram
$$
\xymatrix{
X \ar@{..>}[d]_t \ar@{..>}[r]_g & Y \ar[d]^s \\
Z \ar[r]^f & W
}
$$
with $s \in S$ can be completed to a commutative dotted square with
$t \in S$.
\item[RMS3] For every pair of morphisms $f, g : X \to Y$ and
$s \in S$ with source $Y$ such that $s \circ f = s \circ g$
there exists a $t \in S$ with target $X$ such that
$f \circ t = g \circ t$.
\end{enumerate}
A set of arrows $S$ of $\mathcal{C}$ is called a {\it multiplicative system}
if it is both a left multiplicative system and a right multiplicative system.
In other words, this means that MS1, MS2, MS3 hold, where
MS1 $=$ LMS1 $+$ RMS1, MS2 $=$ LMS2 $+$ RMS2, and
MS3 $=$ LMS3 $+$ RMS3. (That said, of course LMS1 $=$ RMS1
$=$ MS1.)
\end{definition}

\noindent
These conditions are useful to construct the categories $S^{-1}\mathcal{C}$
as follows.

\medskip\noindent
{\bf Left calculus of fractions.}
Let $\mathcal{C}$ be a category and let $S$ be a left multiplicative
system. We define a new category $S^{-1}\mathcal{C}$ as follows
(we verify this works in the proof of
Lemma \ref{lemma-left-localization}):
\begin{enumerate}
\item We set $\Ob(S^{-1}\mathcal{C}) = \Ob(\mathcal{C})$.
\item Morphisms $X \to Y$ of $S^{-1}\mathcal{C}$ are given by pairs
$(f : X \to Y', s : Y \to Y')$ with $s \in S$ up to equivalence.
(The equivalence is defined below. Think of the equivalence class
of a pair $(f, s)$ as $s^{-1}f : X \to Y$.)
\item Two pairs $(f_1 : X \to Y_1, s_1 : Y \to Y_1)$ and
$(f_2 : X \to Y_2, s_2 : Y \to Y_2)$ are said to be equivalent
if there exist a third pair $(f_3 : X \to Y_3, s_3 : Y \to Y_3)$
and morphisms $u : Y_1 \to Y_3$ and $v : Y_2 \to Y_3$ of $\mathcal{C}$
fitting into the commutative diagram
$$
\xymatrix{
 & Y_1 \ar[d]^u & \\
X \ar[ru]^{f_1} \ar[r]^{f_3} \ar[rd]_{f_2} &
Y_3 &
Y \ar[lu]_{s_1} \ar[l]_{s_3} \ar[ld]^{s_2} \\
& Y_2 \ar[u]_v &
}
$$
\item The composition of the equivalence classes of the pairs
$(f : X \to Y', s : Y \to Y')$ and $(g : Y \to Z', t : Z \to Z')$
is defined as the equivalence class of a pair
$(h \circ f : X \to Z'', u \circ t : Z \to Z'')$
where $h$ and $u \in S$ are chosen to fit into a commutative diagram
$$
\xymatrix{
Y \ar[d]_s \ar[r]_g & Z' \ar[d]^u \\
Y' \ar[r]^h & Z''
}
$$
which exists by assumption.
\item The identity morphism $X \to X$ in $S^{-1} \mathcal{C}$ is the
equivalence class of the pair $(\text{id} : X \to X,
\text{id} : X \to X)$.
\end{enumerate}

\begin{lemma}
\label{lemma-left-localization}
Let $\mathcal{C}$ be a category and let $S$ be a left multiplicative
system.
\begin{enumerate}
\item The relation on pairs defined above is an equivalence relation.
\item The composition rule given above is well defined on equivalence
classes.
\item Composition is associative (and the identity morphisms satisfy
the identity axioms), and hence $S^{-1}\mathcal{C}$ is a category.
\end{enumerate}
\end{lemma}

\begin{proof}
Proof of (1). Let us say two pairs $p_1 = (f_1 : X \to Y_1, s_1 : Y \to Y_1)$
and $p_2 = (f_2 : X \to Y_2, s_2 : Y \to Y_2)$ are elementary equivalent
if there exists a morphism $a : Y_1 \to Y_2$ of $\mathcal{C}$ such that
$a \circ f_1 = f_2$ and $a \circ s_1 = s_2$. Diagram:
$$
\xymatrix{
X \ar@{=}[d] \ar[r]_{f_1} & Y_1 \ar[d]^a & Y \ar[l]^{s_1} \ar@{=}[d] \\
X \ar[r]^{f_2} & Y_2 & Y \ar[l]_{s_2}
}
$$
Let us denote this property by saying $p_1Ep_2$.
Note that $pEp$ and $aEb, bEc \Rightarrow aEc$.
(Despite its name, $E$ is not an equivalence
relation.)
Part (1) claims that the relation
$p \sim p' \Leftrightarrow \exists q: pEq \wedge p'Eq$
(where $q$ is supposed to be a pair satisfying the
same conditions as $p$ and $p'$)
is an equivalence relation. A simple formal argument, using the properties
of $E$ above, shows that it suffices to prove
$p_3Ep_1, p_3Ep_2 \Rightarrow p_1 \sim p_2$.
Thus suppose that we are given a commutative diagram
$$
\xymatrix{
 & Y_1 & \\
X \ar[ru]^{f_1} \ar[r]^{f_3} \ar[rd]_{f_2} &
Y_3 \ar[u]_{a_{31}} \ar[d]^{a_{32}} &
Y \ar[lu]_{s_1} \ar[l]_{s_3} \ar[ld]^{s_2} \\
& Y_2 &
}
$$
with $s_i \in S$.
First we apply LMS2 to get a commutative diagram
$$
\xymatrix{
Y \ar[d]_{s_1} \ar[r]_{s_2} & Y_2 \ar@{..>}[d]^{a_{24}} \\
Y_1 \ar@{..>}[r]^{a_{14}} & Y_4
}
$$
with $a_{24} \in S$. Then, we have
$$
a_{14} \circ a_{31} \circ s_3 =
a_{14} \circ s_1 =
a_{24} \circ s_2 =
a_{24} \circ a_{32} \circ s_3.
$$
Hence, by LMS3, there exists a
morphism $s_{44} : Y_4 \to Y'_4$ such that $s_{44} \in S$ and
$s_{44} \circ a_{14} \circ a_{31}
= s_{44} \circ a_{24} \circ a_{32}$.
Hence, after replacing $Y_4$, $a_{14}$ and $a_{24}$ by $Y'_4$,
$s_{44} \circ a_{14}$ and $s_{44} \circ a_{24}$, we may assume
that $a_{14} \circ a_{31} = a_{24} \circ a_{32}$ (and
we still have $a_{24} \in S$ and
$a_{14} \circ s_1 = a_{24} \circ s_2$). Set
$$
f_4 =
a_{14} \circ f_1 =
a_{14} \circ a_{31} \circ f_3 =
a_{24} \circ a_{32} \circ f_3 =
a_{24} \circ f_2
$$
and
$s_4 = a_{14} \circ s_1 = a_{24} \circ s_2$. Then, the diagram
$$
\xymatrix{
X \ar@{=}[d] \ar[r]_{f_1} & Y_1 \ar[d]^{a_{14}} & Y \ar[l]^{s_1} \ar@{=}[d] \\
X \ar[r]^{f_4} & Y_4 & Y \ar[l]_{s_4}
}
$$
commutes, and we have $s_4 \in S$ (by LMS1). Thus, $p_1 E p_4$,
where $p_4 = (f_4, s_4)$. Similarly, $p_2 E p_4$. Combining these,
we find $p_1 \sim p_2$.

\medskip\noindent
Proof of (2). Let  $p = (f : X \to Y', s : Y \to Y')$ and
$q = (g : Y \to Z', t : Z \to Z')$ be pairs as in the definition of composition
above. To compose we choose a diagram
$$
\xymatrix{
Y \ar[d]_s \ar[r]_g & Z' \ar[d]^{u_2} \\
Y' \ar[r]^{h_2} & Z_2
}
$$
with $u_2 \in S$. We first show that the equivalence class of the pair
$r_2 = (h_2 \circ f : X \to Z_2, u_2 \circ t : Z \to Z_2)$
is independent of the choice of $(Z_2, h_2, u_2)$. Namely, suppose
that $(Z_3, h_3, u_3)$ is another choice with corresponding composition
$r_3 = (h_3 \circ f : X \to Z_3, u_3 \circ t : Z \to Z_3)$.
Then by LMS2 we can choose a diagram
$$
\xymatrix{
Z' \ar[d]_{u_2} \ar[r]_{u_3} & Z_3 \ar[d]^{u_{34}} \\
Z_2 \ar[r]^{h_{24}} & Z_4
}
$$
with $u_{34} \in S$. We have $h_2 \circ s = u_2 \circ g$
and similarly $h_3 \circ s = u_3 \circ g$. Now,
$$
u_{34} \circ h_3 \circ s
= u_{34} \circ u_3 \circ g
= h_{24} \circ u_2 \circ g
= h_{24} \circ h_2 \circ s.
$$
Hence, LMS3 shows that there
exist a $Z'_4$ and an $s_{44} : Z_4 \to Z'_4$ such that
$s_{44} \circ u_{34} \circ h_3 = s_{44} \circ h_{24} \circ h_2$.
Replacing $Z_4$, $h_{24}$ and $u_{34}$ by $Z'_4$,
$s_{44} \circ h_{24}$ and $s_{44} \circ u_{34}$, we may
assume that $u_{34} \circ h_3 = h_{24} \circ h_2$.
Meanwhile, the relations $u_{34} \circ u_3 = h_{24} \circ u_2$
and $u_{34} \in S$ continue to hold. We can now set
$h_4 = u_{34} \circ h_3 = h_{24} \circ h_2$ and
$u_4 = u_{34} \circ u_3 = h_{24} \circ u_2$. Then, we have a
commutative diagram
$$
\xymatrix{
X \ar@{=}[d] \ar[r]_{h_2\circ f} &
Z_2 \ar[d]^{h_{24}} &
Z \ar[l]^{u_2 \circ t} \ar@{=}[d] \\
X \ar@{=}[d] \ar[r]^{h_4\circ f} &
Z_4 &
Z \ar@{=}[d] \ar[l]_{u_4 \circ t} \\
X \ar[r]^{h_3 \circ f} &
Z_3 \ar[u]^{u_{34}} &
Z \ar[l]_{u_3 \circ t}
}
$$
Hence we obtain a pair
$r_4 =
(h_4 \circ f : X \to Z_4, u_4 \circ t : Z \to Z_4)$
and the above diagram shows that we have $r_2Er_4$ and
$r_3Er_4$, whence $r_2 \sim r_3$, as desired. Thus it now
makes sense to define $p \circ q$ as the equivalence class of
all possible pairs $r$ obtained as above.

\medskip\noindent
To finish the proof of (2) we have to show that given pairs
$p_1, p_2, q$ such that $p_1Ep_2$ then $p_1 \circ q = p_2 \circ q$ and
$q \circ p_1 = q \circ p_2$ whenever the compositions make sense.
To do this, write $p_1 = (f_1 : X \to Y_1, s_1 : Y \to Y_1)$ and
$p_2 = (f_2 : X \to Y_2, s_2 : Y \to Y_2)$ and let
$a : Y_1 \to Y_2$ be a morphism of $\mathcal{C}$ such that
$f_2 = a \circ f_1$ and $s_2 = a \circ s_1$.
First assume that $q = (g : Y \to Z', t : Z \to Z')$.
In this case choose a commutative diagram as the one on the left
$$
\vcenter{
\xymatrix{
Y \ar[d]_{s_2} \ar[r]^g & Z' \ar[d]^u \\
Y_2 \ar[r]^h & Z''
}
}
\quad
\Rightarrow
\quad
\vcenter{
\xymatrix{
Y \ar[d]_{s_1} \ar[r]^g & Z' \ar[d]^u \\
Y_1 \ar[r]^{h \circ a} & Z''
}
}
$$
(with $u \in S$),
which implies the diagram on the right is commutative as well.
Using these diagrams we see that both compositions $q \circ p_1$
and $q \circ p_2$ are the equivalence class of
$(h \circ a \circ f_1 : X \to Z'', u \circ t : Z \to Z'')$.
Thus $q \circ p_1 = q \circ p_2$.
The proof of the other case, in which we have to show
$p_1 \circ q = p_2 \circ q$, is omitted. (It is similar to the
case we did.)

\medskip\noindent
Proof of (3). We have to prove associativity of composition.
Consider a solid diagram
$$
\xymatrix{
& & & Z \ar[d] \\
& & Y \ar[d] \ar[r] & Z' \ar@{..>}[d] \\
& X \ar[d] \ar[r] & Y' \ar@{..>}[d] \ar@{..>}[r] & Z'' \ar@{..>}[d] \\
W \ar[r] & X' \ar@{..>}[r] & Y'' \ar@{..>}[r] & Z'''
}
$$
(whose vertical arrows belong to $S$)
which gives rise to three composable pairs.
Using LMS2 we can choose the dotted arrows making the squares commutative
and such that the vertical arrows are in $S$.
Then it is clear that the composition of the three pairs
is the equivalence class of the pair
$(W \to Z''', Z \to Z''')$ gotten by composing the
horizontal arrows on the bottom row and the vertical arrows
on the right column.

\medskip\noindent
We leave it to the reader to check the identity axioms.
\end{proof}

\begin{remark}
\label{remark-motivation-localization}
The motivation for the construction of $S^{-1} \mathcal{C}$ is to
``force'' the morphisms in $S$ to be invertible by artificially
creating inverses to them (at the cost of some existing morphisms
possibly becoming identified with each other). This is similar to
the localization of a commutative ring at a multiplicative subset,
and more generally to the localization of a noncommutative ring
at a right denominator set (see \cite[Section 10A]{Lam}). This is
more than just a similarity: The construction of
$S^{-1} \mathcal{C}$ (or, more precisely, its version for
additive categories $\mathcal{C}$) actually generalizes the
latter type of localization. Namely, a noncommutative ring can be
viewed as a pre-additive category with a single object (the morphisms
being the elements of the ring); a multiplicative subset of this
ring then becomes a set $S$ of morphisms satisfying LMS1 (aka
RMS1). Then, the conditions RMS2 and RMS3 for this category and
this subset $S$ translate into the two conditions
(``right permutable'' and ``right reversible'') of a right
denominator set (and similarly for LMS and left denominator sets),
and $S^{-1} \mathcal{C}$ (with a properly defined additive
structure) is the one-object category corresponding to the
localization of the ring.
\end{remark}

\begin{definition}
\label{definition-left-localization-as-fraction}
Let $\mathcal{C}$ be a category and let $S$ be a left multiplicative
system of morphisms of $\mathcal{C}$. Given any morphism
$f : X \to Y'$ in $\mathcal{C}$ and any morphism $s : Y \to Y'$ in
$S$, we denote by {\it $s^{-1} f$} the equivalence class of the pair
$(f : X \to Y', s : Y \to Y')$. This is a morphism from $X$ to $Y$
in $S^{-1} \mathcal{C}$.
\end{definition}

\noindent
This notation is suggestive, and the things it suggests are true:
Given any morphism $f : X \to Y'$ in $\mathcal{C}$ and any two
morphisms $s : Y \to Y'$ and $t : Y' \to Y''$ in $S$, we have
$\left(t \circ s\right)^{-1} \left(t \circ f\right) = s^{-1} f$.
Also, for any
$f : X \to Y'$ and $g : Y' \to Z'$ in $\mathcal{C}$ and all
$s : Z \to Z'$ in $S$, we have
$s^{-1} \left(g \circ f\right) = \left(s^{-1} g\right) \circ
\left(\text{id}_{Y'}^{-1} f\right)$.
Finally, for any $f : X \to Y'$ in $\mathcal{C}$, all
$s : Y \to Y'$ in $S$, and $t : Z \to Y$ in $S$, we have
$\left(s \circ t\right)^{-1} f
= \left(t^{-1} \text{id}_Y\right)
\circ \left(s^{-1} f\right)$.
This is all clear from the definition.
We can ``write any finite collection of morphisms with the same target
as fractions with common denominator''.

\begin{lemma}
\label{lemma-morphisms-left-localization}
Let $\mathcal{C}$ be a category and let $S$ be a left multiplicative
system of morphisms of $\mathcal{C}$. Given any finite collection
$g_i : X_i \to Y$ of morphisms of $S^{-1}\mathcal{C}$
(indexed by $i$),
we can find an element $s : Y \to Y'$ of $S$ and
a family of morphisms $f_i : X_i \to Y'$ of $\mathcal{C}$ such that
each $g_i$ is the equivalence class of the pair
$(f_i : X_i \to Y', s : Y \to Y')$.
\end{lemma}

\begin{proof}
For each $i$ choose a representative $(X_i \to Y_i, s_i : Y \to Y_i)$
of $g_i$.
The lemma follows if we can find a morphism $s : Y \to Y'$ in $S$ such that
for each $i$ there is a morphism $a_i : Y_i \to Y'$ with
$a_i \circ s_i = s$. If we have two indices $i = 1, 2$, then we can
do this by completing the square
$$
\xymatrix{
Y \ar[d]_{s_1} \ar[r]_{s_2} & Y_2 \ar[d]^{t_2} \\
Y_1 \ar[r]^{a_1} & Y'
}
$$
with $t_2 \in S$ as is possible by
Definition \ref{definition-multiplicative-system}.
Then $s = t_2 \circ s_2 \in S$ works.
If we have $n > 2$ morphisms, then we use the above trick to reduce
to the case of $n - 1$ morphisms, and we win by induction.
\end{proof}

\noindent
There is an easy characterization of equality of morphisms if they
have the same denominator.

\begin{lemma}
\label{lemma-equality-morphisms-left-localization}
Let $\mathcal{C}$ be a category and let $S$ be a left multiplicative
system of morphisms of $\mathcal{C}$. Let $A, B : X \to Y$ be
morphisms of $S^{-1}\mathcal{C}$ which are the equivalence
classes of $(f : X \to Y', s : Y \to Y')$ and
$(g : X \to Y', s : Y \to Y')$. Then
$A = B$ if and only if there exists a morphism
$a : Y' \to Y''$ with $a \circ s \in S$ and
such that $a \circ f = a \circ g$.
\end{lemma}

\begin{proof}
The equality of $A$ and $B$ means that there exists a commutative diagram
$$
\xymatrix{
 & Y' \ar[d]^u & \\
X \ar[ru]^f \ar[r]^h \ar[rd]_g &
Z &
Y \ar[lu]_s \ar[l]_t \ar[ld]^s \\
& Y' \ar[u]_v &
}
$$
with $t \in S$. In particular $u \circ s = v \circ s$. Hence by LMS3 there
exists an $s' : Z \to Y''$ in $S$ such that $s' \circ u = s' \circ v$.
Setting $a$ equal to this common value does the job.
\end{proof}

\begin{remark}
\label{remark-left-localization-morphisms-colimit}
Let $\mathcal{C}$ be a category. Let $S$ be a left multiplicative system.
Given an object $Y$ of $\mathcal{C}$ we denote $Y/S$ the category whose
objects are $s : Y \to Y'$ with $s \in S$ and whose morphisms are
commutative diagrams
$$
\xymatrix{
& Y \ar[ld]_s \ar[rd]^t & \\
Y' \ar[rr]^a & & Y''
}
$$
where $a : Y' \to Y''$ is arbitrary. We claim that the category
$Y/S$ is filtered (see
Definition \ref{definition-directed}).
Namely, LMS1 implies that $\text{id}_Y : Y \to Y$
is in $Y/S$; hence $Y/S$ is nonempty. LMS2 implies that given
$s_1 : Y \to Y_1$ and $s_2 : Y \to Y_2$ we can find a diagram
$$
\xymatrix{
Y \ar[d]_{s_1} \ar[r]_{s_2} & Y_2 \ar[d]^t \\
Y_1 \ar[r]^a & Y_3
}
$$
with $t \in S$. Hence $s_1 : Y \to Y_1$ and $s_2 : Y \to Y_2$
both have maps to $t \circ s_2 : Y \to Y_3$ in $Y/S$. Finally, given
two morphisms $a, b$ from $s_1 : Y \to Y_1$ to $s_2 : Y \to Y_2$
in $Y/S$ we see that $a \circ s_1 = b \circ s_1$; hence by LMS3
there exists a $t : Y_2 \to Y_3$ in $S$ such that
$t \circ a = t \circ b$.
Now the combined results of
Lemmas \ref{lemma-morphisms-left-localization} and
\ref{lemma-equality-morphisms-left-localization}
tell us that
\begin{equation}
\label{equation-left-localization-morphisms-colimit}
\Mor_{S^{-1}\mathcal{C}}(X, Y) =
\colim_{(s : Y \to Y') \in Y/S} \Mor_\mathcal{C}(X, Y')
\end{equation}
This formula expressing morphism sets in $S^{-1}\mathcal{C}$ as a filtered
colimit of morphism sets in $\mathcal{C}$ is occasionally useful.
\end{remark}

\begin{lemma}
\label{lemma-properties-left-localization}
Let $\mathcal{C}$ be a category and let $S$ be a left multiplicative
system of morphisms of $\mathcal{C}$.
\begin{enumerate}
\item The rules $X \mapsto X$ and
$(f : X \to Y) \mapsto (f : X \to Y, \text{id}_Y : Y \to Y)$
define a functor $Q : \mathcal{C} \to S^{-1}\mathcal{C}$.
\item For any $s \in S$ the morphism $Q(s)$ is an isomorphism in
$S^{-1}\mathcal{C}$.
\item If $G : \mathcal{C} \to \mathcal{D}$ is any functor such that
$G(s)$ is invertible for every $s \in S$, then there exists a
unique functor $H : S^{-1}\mathcal{C} \to \mathcal{D}$
such that $H \circ Q = G$.
\end{enumerate}
\end{lemma}

\begin{proof}
Parts (1) and (2) are clear. (In (2), the inverse of $Q(s)$ is
the equivalence class of the pair $(\text{id}_Y, s)$.)
To see (3) just set $H(X) = G(X)$
and set $H((f : X \to Y', s : Y \to Y')) = G(s)^{-1} \circ G(f)$.
Details omitted.
\end{proof}

\begin{lemma}
\label{lemma-left-localization-limits}
Let $\mathcal{C}$ be a category and let $S$ be a left multiplicative
system of morphisms of $\mathcal{C}$. The localization functor
$Q : \mathcal{C} \to S^{-1}\mathcal{C}$ commutes with finite colimits.
\end{lemma}

\begin{proof}
Let $\mathcal{I}$ be a finite category and let
$\mathcal{I} \to \mathcal{C}$, $i \mapsto X_i$
be a functor whose colimit exists. Then using
(\ref{equation-left-localization-morphisms-colimit}),
the fact that $Y/S$ is filtered, and
Lemma \ref{lemma-directed-commutes} we have
\begin{align*}
\Mor_{S^{-1}\mathcal{C}}(Q(\colim X_i), Q(Y))
& =
\colim_{(s : Y \to Y') \in Y/S} \Mor_\mathcal{C}(\colim X_i, Y') \\
& =
\colim_{(s : Y \to Y') \in Y/S} \lim_i \Mor_\mathcal{C}(X_i, Y') \\
& =
\lim_i \colim_{(s : Y \to Y') \in Y/S} \Mor_\mathcal{C}(X_i, Y') \\
& =
\lim_i \Mor_{S^{-1}\mathcal{C}}(Q(X_i), Q(Y))
\end{align*}
and this isomorphism commutes with the projections
from both sides to the set
$\Mor_{S^{-1}\mathcal{C}}(Q(X_j), Q(Y))$ for each
$j \in \Ob(\mathcal{I})$. Thus, $Q(\colim X_i)$ satisfies
the universal property for the colimit of the functor
$i \mapsto Q(X_i)$; hence, it is this colimit, as desired.
\end{proof}

\begin{lemma}
\label{lemma-left-localization-diagram}
Let $\mathcal{C}$ be a category. Let $S$ be a left multiplicative
system. If $f : X \to Y$, $f' : X' \to Y'$ are two morphisms of
$\mathcal{C}$ and if
$$
\xymatrix{
Q(X) \ar[d]_{Q(f)} \ar[r]_a & Q(X') \ar[d]^{Q(f')} \\
Q(Y) \ar[r]^b & Q(Y')
}
$$
is a commutative diagram in $S^{-1}\mathcal{C}$, then there exist
a morphism $f'' : X'' \to Y''$ in $\mathcal{C}$ and a commutative
diagram
$$
\xymatrix{
X \ar[d]_f \ar[r]_g & X'' \ar[d]^{f''} & X' \ar[d]^{f'} \ar[l]^s \\
Y \ar[r]^h & Y'' & Y' \ar[l]_t
}
$$
in $\mathcal{C}$ with $s, t \in S$ and $a = s^{-1}g$, $b = t^{-1}h$.
\end{lemma}

\begin{proof}
We choose maps and objects in the following way:
First write $a = s^{-1}g$ for some $s : X' \to X''$ in $S$ and
$g : X \to X''$. By LMS2 we can find $t : Y' \to Y''$ in $S$ and
$f'' : X'' \to Y''$ such that
$$
\xymatrix{
X \ar[d]_f \ar[r]_g & X'' \ar[d]^{f''} & X' \ar[d]^{f'} \ar[l]^s \\
Y & Y'' & Y' \ar[l]_t
}
$$
commutes. Now in this diagram we are going to repeatedly change our
choice of
$$
X'' \xrightarrow{f''} Y'' \xleftarrow{t} Y'
$$
by postcomposing both $t$ and $f''$ by a morphism $d : Y'' \to Y'''$
with the property that $d \circ t \in S$. According to
Remark \ref{remark-left-localization-morphisms-colimit}
we may after such a replacement assume that there exists a morphism
$h : Y \to Y''$ such that $b = t^{-1}h$ holds\footnote{Here is a
more down-to-earth way to see this:
Write $b = q^{-1}i$ for some $q : Y' \to Z$ in $S$ and some
$i : Y \to Z$. By LMS2 we can find $r : Y'' \to Y'''$ in $S$ and
$j : Z \to Y'''$ such that $j \circ q = r \circ t$. Now, set
$d = r$ and $h = j \circ i$.}. At this point we have everything
as in the lemma except that we don't know that the left square of the
diagram commutes.
But the definition of composition in $S^{-1} \mathcal{C}$ shows that
$b \circ Q\left(f\right)$ is the equivalence class of the pair
$(h \circ f : X \to Y'', t : Y' \to Y'')$ (since $b$ is the
equivalence class of the pair $(h : Y \to Y'', t : Y' \to Y'')$,
while $Q\left(f\right)$ is the equivalence class of the pair
$(f : X \to Y, \text{id} : Y \to Y)$), while
$Q\left(f'\right) \circ a$ is the equivalence class of the pair
$(f'' \circ g : X \to Y'', t : Y' \to Y'')$ (since $a$ is the
equivalence class of the pair $(g : X \to X'', s : X' \to X'')$,
while $Q\left(f'\right)$ is the equivalence class of the pair
$(f' : X' \to Y', \text{id} : Y' \to Y')$).
Since we know that
$b \circ Q\left(f\right) = Q\left(f'\right) \circ a$, we thus
conclude that the equivalence classes of the pairs
$(h \circ f : X \to Y'', t : Y' \to Y'')$ and
$(f'' \circ g : X \to Y'', t : Y' \to Y'')$ are equal.
Hence using
Lemma \ref{lemma-equality-morphisms-left-localization}
we can find a morphism $d : Y'' \to Y'''$ such that
$d \circ t \in S$ and $d \circ h \circ f = d \circ f'' \circ g$.
Hence we make one more replacement of the kind described
above and we win.
\end{proof}

\noindent
{\bf Right calculus of fractions.}
Let $\mathcal{C}$ be a category and let $S$ be a right multiplicative
system. We define a new category $S^{-1}\mathcal{C}$ as follows
(we verify this works in the proof of
Lemma \ref{lemma-right-localization}):
\begin{enumerate}
\item We set $\Ob(S^{-1}\mathcal{C}) = \Ob(\mathcal{C})$.
\item Morphisms $X \to Y$ of $S^{-1}\mathcal{C}$ are given by pairs
$(f : X' \to Y, s : X' \to X)$ with $s \in S$ up to equivalence.
(The equivalence is defined below. Think of the equivalence class
of a pair $(f, s)$ as $fs^{-1} : X \to Y$.)
\item Two pairs $(f_1 : X_1 \to Y, s_1 : X_1 \to X)$ and
$(f_2 : X_2 \to Y, s_2 : X_2 \to X)$ are said to be equivalent
if there exist a third pair $(f_3 : X_3 \to Y, s_3 : X_3 \to X)$
and morphisms $u : X_3 \to X_1$ and $v : X_3 \to X_2$ of $\mathcal{C}$
fitting into the commutative diagram
$$
\xymatrix{
 & X_1 \ar[ld]_{s_1} \ar[rd]^{f_1} & \\
X &
X_3 \ar[l]_{s_3} \ar[u]_u \ar[d]^v \ar[r]^{f_3} &
Y \\
& X_2 \ar[lu]^{s_2} \ar[ru]_{f_2} &
}
$$
\item The composition of the equivalence classes of the pairs
$(f : X' \to Y, s : X' \to X)$ and $(g : Y' \to Z, t : Y' \to Y)$
is defined as the equivalence class of a pair
$(g \circ h : X'' \to Z, s \circ u : X'' \to X)$
where $h$ and $u \in S$ are chosen to fit into a commutative diagram
$$
\xymatrix{
X'' \ar[d]_u \ar[r]^h & Y' \ar[d]^t \\
X' \ar[r]^f & Y
}
$$
which exists by assumption.
\item The identity morphism $X \to X$ in $S^{-1} \mathcal{C}$ is the
equivalence class of the pair $(\text{id} : X \to X,
\text{id} : X \to X)$.
\end{enumerate}

\begin{lemma}
\label{lemma-right-localization}
Let $\mathcal{C}$ be a category and let $S$ be a right multiplicative
system.
\begin{enumerate}
\item The relation on pairs defined above is an equivalence relation.
\item The composition rule given above is well defined on equivalence
classes.
\item Composition is associative (and the identity morphisms satisfy
the identity axioms), and hence $S^{-1}\mathcal{C}$ is a category.
\end{enumerate}
\end{lemma}

\begin{proof}
This lemma is dual to
Lemma \ref{lemma-left-localization}.
It follows formally from that lemma by replacing
$\mathcal{C}$ by its opposite category in which
$S$ is a left multiplicative system.
\end{proof}

\begin{definition}
\label{definition-right-localization-as-fraction}
Let $\mathcal{C}$ be a category and let $S$ be a right multiplicative
system of morphisms of $\mathcal{C}$. Given any morphism
$f : X' \to Y$ in $\mathcal{C}$ and any morphism $s : X' \to X$ in
$S$, we denote by {\it $f s^{-1}$} the equivalence class of the pair
$(f : X' \to Y, s : X' \to X)$. This is a morphism from $X$ to $Y$
in $S^{-1} \mathcal{C}$.
\end{definition}

\noindent
Identities similar (actually, dual) to the ones in Definition
\ref{definition-left-localization-as-fraction} hold.
We can ``write any finite collection of morphisms with the same source
as fractions with common denominator''.

\begin{lemma}
\label{lemma-morphisms-right-localization}
Let $\mathcal{C}$ be a category and let $S$ be a right multiplicative
system of morphisms of $\mathcal{C}$. Given any finite collection
$g_i : X \to Y_i$ of morphisms of $S^{-1}\mathcal{C}$
(indexed by $i$),
we can find an element $s : X' \to X$ of $S$ and a family
of morphisms $f_i : X' \to Y_i$ of $\mathcal{C}$ such that
$g_i$ is the equivalence class of the pair
$(f_i : X' \to Y_i, s : X' \to X)$.
\end{lemma}

\begin{proof}
This lemma is the dual of
Lemma \ref{lemma-morphisms-left-localization}
and follows formally from that lemma by replacing all
categories in sight by their opposites.
\end{proof}

\noindent
There is an easy characterization of equality of morphisms if they
have the same denominator.

\begin{lemma}
\label{lemma-equality-morphisms-right-localization}
Let $\mathcal{C}$ be a category and let $S$ be a right multiplicative
system of morphisms of $\mathcal{C}$. Let $A, B : X \to Y$ be
morphisms of $S^{-1}\mathcal{C}$ which are the equivalence
classes of $(f : X' \to Y, s : X' \to X)$ and
$(g : X' \to Y, s : X' \to X)$. Then
$A = B$ if and only if there exists a morphism
$a : X'' \to X'$ with $s \circ a \in S$ and
such that $f \circ a = g \circ a$.
\end{lemma}

\begin{proof}
This is dual to
Lemma \ref{lemma-equality-morphisms-left-localization}.
\end{proof}

\begin{remark}
\label{remark-right-localization-morphisms-colimit}
Let $\mathcal{C}$ be a category. Let $S$ be a right multiplicative system.
Given an object $X$ of $\mathcal{C}$ we denote $S/X$ the category whose
objects are $s : X' \to X$ with $s \in S$ and whose morphisms are
commutative diagrams
$$
\xymatrix{
X' \ar[rd]_s \ar[rr]_a & & X'' \ar[ld]^t \\
& X
}
$$
where $a : X' \to X''$ is arbitrary. The category
$S/X$ is cofiltered (see
Definition \ref{definition-codirected}).
(This is dual to the corresponding statement in
Remark \ref{remark-left-localization-morphisms-colimit}.)
Now the combined results of
Lemmas \ref{lemma-morphisms-right-localization} and
\ref{lemma-equality-morphisms-right-localization}
tell us that
\begin{equation}
\label{equation-right-localization-morphisms-colimit}
\Mor_{S^{-1}\mathcal{C}}(X, Y) =
\colim_{(s : X' \to X) \in (S/X)^{opp}} \Mor_\mathcal{C}(X', Y)
\end{equation}
This formula expressing morphisms in $S^{-1}\mathcal{C}$ as a filtered
colimit of morphisms in $\mathcal{C}$ is occasionally useful.
\end{remark}

\begin{lemma}
\label{lemma-properties-right-localization}
Let $\mathcal{C}$ be a category and let $S$ be a right multiplicative
system of morphisms of $\mathcal{C}$.
\begin{enumerate}
\item The rules $X \mapsto X$ and
$(f : X \to Y) \mapsto (f : X \to Y, \text{id}_X : X \to X)$
define a functor $Q : \mathcal{C} \to S^{-1}\mathcal{C}$.
\item For any $s \in S$ the morphism $Q(s)$ is an isomorphism in
$S^{-1}\mathcal{C}$.
\item If $G : \mathcal{C} \to \mathcal{D}$ is any functor such that
$G(s)$ is invertible for every $s \in S$, then there exists a
unique functor $H : S^{-1}\mathcal{C} \to \mathcal{D}$
such that $H \circ Q = G$.
\end{enumerate}
\end{lemma}

\begin{proof}
This lemma is the dual of
Lemma \ref{lemma-properties-left-localization}
and follows formally from that lemma by replacing all
categories in sight by their opposites.
\end{proof}

\begin{lemma}
\label{lemma-right-localization-limits}
Let $\mathcal{C}$ be a category and let $S$ be a right multiplicative
system of morphisms of $\mathcal{C}$. The localization functor
$Q : \mathcal{C} \to S^{-1}\mathcal{C}$ commutes with finite limits.
\end{lemma}

\begin{proof}
This is dual to Lemma \ref{lemma-left-localization-limits}.
\end{proof}

\begin{lemma}
\label{lemma-right-localization-diagram}
Let $\mathcal{C}$ be a category. Let $S$ be a right multiplicative
system. If $f : X \to Y$, $f' : X' \to Y'$ are two morphisms of
$\mathcal{C}$ and if
$$
\xymatrix{
Q(X) \ar[d]_{Q(f)} \ar[r]_a & Q(X') \ar[d]^{Q(f')} \\
Q(Y) \ar[r]^b & Q(Y')
}
$$
is a commutative diagram in $S^{-1}\mathcal{C}$, then there exist
a morphism $f'' : X'' \to Y''$ in $\mathcal{C}$ and a commutative
diagram
$$
\xymatrix{
X \ar[d]_f & X'' \ar[l]^s \ar[d]^{f''} \ar[r]_g & X' \ar[d]^{f'} \\
Y & Y'' \ar[l]_t \ar[r]^h & Y'
}
$$
in $\mathcal{C}$ with $s, t \in S$ and $a = gs^{-1}$, $b = ht^{-1}$.
\end{lemma}

\begin{proof}
This lemma is dual to
Lemma \ref{lemma-left-localization-diagram}.
\end{proof}

\noindent
{\bf Multiplicative systems and two sided calculus of fractions.}
If $S$ is a multiplicative system then left and right calculus of
fractions give canonically isomorphic categories.

\begin{lemma}
\label{lemma-multiplicative-system}
Let $\mathcal{C}$ be a category and let $S$ be a multiplicative system.
The category of left fractions and the category of right fractions
$S^{-1}\mathcal{C}$ are canonically isomorphic.
\end{lemma}

\begin{proof}
Denote $\mathcal{C}_{left}$, $\mathcal{C}_{right}$ the two categories
of fractions. By the universal properties of
Lemmas \ref{lemma-properties-left-localization} and
\ref{lemma-properties-right-localization}
we obtain functors $\mathcal{C}_{left} \to \mathcal{C}_{right}$
and $\mathcal{C}_{right} \to \mathcal{C}_{left}$.
By the uniqueness statement in the universal properties, these
functors are each other's inverse.
\end{proof}

\begin{definition}
\label{definition-saturated-multiplicative-system}
Let $\mathcal{C}$ be a category and let $S$ be a multiplicative system.
We say $S$ is {\it saturated} if, in addition to MS1, MS2, MS3, we
also have
\begin{enumerate}
\item[MS4] Given three composable morphisms $f, g, h$, if
$fg, gh \in S$, then $g \in S$.
\end{enumerate}
\end{definition}

\noindent
Note that a saturated multiplicative system contains all isomorphisms.
Moreover, if  $f, g, h$ are composable morphisms in a category and
$fg, gh$ are isomorphisms, then $g$ is an isomorphism (because then $g$
has both a left and a right inverse, hence is invertible).

\begin{lemma}
\label{lemma-what-gets-inverted}
Let $\mathcal{C}$ be a category and let $S$ be a multiplicative system.
Denote $Q : \mathcal{C} \to S^{-1}\mathcal{C}$ the localization functor.
The set
$$
\hat S = \{f \in \text{Arrows}(\mathcal{C}) \mid
Q(f) \text{ is an isomorphism}\}
$$
is equal to
$$
S' = \{f \in \text{Arrows}(\mathcal{C}) \mid
\text{there exist }g, h\text{ such that }gf, fh \in S\}
$$
and is the smallest saturated multiplicative system containing $S$.
In particular, if $S$ is saturated, then $\hat S = S$.
\end{lemma}

\begin{proof}
It is clear that $S \subset S' \subset \hat S$ because elements of
$S'$ map to morphisms in $S^{-1}\mathcal{C}$ which have both left
and right inverses. Note that $S'$ satisfies MS4, and that
$\hat S$ satisfies MS1. Next, we prove that $S' = \hat S$.

\medskip\noindent
Let $f \in \hat S$. Let $s^{-1}g = ht^{-1}$ be the inverse morphism
in $S^{-1}\mathcal{C}$. (We may use both left fractions and right
fractions to describe morphisms in $S^{-1}\mathcal{C}$, see
Lemma \ref{lemma-multiplicative-system}.)
The relation $\text{id}_X = s^{-1}gf$ in $S^{-1}\mathcal{C}$ means
there exists a commutative diagram
$$
\xymatrix{
 & X' \ar[d]^u & \\
X \ar[ru]^{gf} \ar[r]^{f'} \ar[rd]_{\text{id}_X} &
X'' &
X \ar[lu]_s \ar[l]_{s'} \ar[ld]^{\text{id}_X} \\
& X \ar[u]_v &
}
$$
for some morphisms $f', u, v$ and $s' \in S$. Hence $ugf = s' \in S$.
Similarly, using that $\text{id}_Y = fht^{-1}$ one proves that
$fhw \in S$ for some $w$. We conclude that $f \in S'$. Thus
$S' = \hat S$. Provided we prove that $S' = \hat S$ is a
multiplicative system it is now clear that this implies that $S' = \hat S$
is the smallest saturated system containing $S$.

\medskip\noindent
Our remarks above take care of MS1 and MS4, so to finish the proof of the
lemma we have to show that LMS2, RMS2, LMS3, RMS3 hold for $\hat S$.
Let us check that LMS2 holds for $\hat S$. Suppose we have a solid diagram
$$
\xymatrix{
X \ar[d]_t \ar[r]_g & Y \ar@{..>}[d]^s \\
Z \ar@{..>}[r]^f & W
}
$$
with $t \in \hat S$. Pick a morphism $a : Z \to Z'$ such that
$at \in S$. Then we can use LMS2 for $S$ to find a commutative diagram
$$
\xymatrix{
X \ar[d]_t \ar[r]_g & Y \ar[dd]^s \\
Z \ar[d]_a  \\
Z' \ar[r]^{f'} & W
}
$$
and setting $f = f' \circ a$ we win. The proof of RMS2 is dual to this.
Finally, suppose given a pair of morphisms $f, g : X \to Y$ and
$t \in \hat S$ with target $X$ such that $ft = gt$.
Then we pick a morphism $b$ such that $tb \in S$. Then
$ftb = gtb$ which implies by LMS3 for $S$ that there exists an $s \in S$
with source $Y$ such that $sf = sg$ as desired. The proof of
RMS3 is dual to this.
\end{proof}








\section{Formal properties}
\label{section-formal-cat-cat}

\noindent
In this section we discuss some formal properties of the
$2$-category of categories. This will lead us to the definition
of a (strict) $2$-category later.

\medskip\noindent
Let us denote $\Ob(\textit{Cat})$ the class of all categories.
For every pair of categories
$\mathcal{A}, \mathcal{B} \in \Ob(\textit{Cat})$
we have the ``small'' category of functors
$\text{Fun}(\mathcal{A}, \mathcal{B})$.
Composition of transformation of functors such as
$$
\xymatrix{
\mathcal{A}
\rruppertwocell^{F''}{t'}
\ar[rr]_(.3){F'}
\rrlowertwocell_F{t}
& &
\mathcal{B}
}
\text{ composes to }
\xymatrix{
\mathcal{A}
\rrtwocell^{F''}_F{\ \ t \circ t'}
& &
\mathcal{B}
}
$$
is called {\it vertical} composition. We will use the usual
symbol $\circ$ for this. Next, we will define {\it horizontal}
composition. In order to do this we explain a bit more
of the structure at hand.

\medskip\noindent
Namely for every triple
of categories $\mathcal{A}$, $\mathcal{B}$, and $\mathcal{C}$
there is a composition law
$$
\circ : \Ob(\text{Fun}(\mathcal{B}, \mathcal{C}))
\times
\Ob(\text{Fun}(\mathcal{A}, \mathcal{B}))
\longrightarrow
\Ob(\text{Fun}(\mathcal{A}, \mathcal{C}))
$$
coming from composition of functors. This composition law
is associative, and identity functors act as units. In other
words -- forgetting about transformations of functors --
we see that $\textit{Cat}$ forms a category. How does
this structure interact with the morphisms between functors?

\medskip\noindent
Well, given $t : F \to F'$ a transformation of
functors $F, F' : \mathcal{A} \to \mathcal{B}$ and
a functor
$G : \mathcal{B} \to \mathcal{C}$ we can define
a transformation of functors
$G\circ F \to G \circ F'$. We will denote this
transformation ${}_Gt$. It is given by the formula
$({}_Gt)_x = G(t_x) : G(F(x)) \to G(F'(x))$
for all $x \in \mathcal{A}$.
In this way composition
with $G$ becomes a functor
$$
\text{Fun}(\mathcal{A}, \mathcal{B})
\longrightarrow
\text{Fun}(\mathcal{A}, \mathcal{C}).
$$
To see this you just have to check that
${}_G(\text{id}_F) = \text{id}_{G \circ F}$ and that
${}_G(t_1 \circ t_2) = {}_Gt_1 \circ {}_Gt_2$.
Of course we also have that ${}_{\text{id}_\mathcal{A}}t = t$.

\medskip\noindent
Similarly, given $s : G \to G'$ a transformation of
functors $G, G' : \mathcal{B} \to \mathcal{C}$ and
$F : \mathcal{A} \to \mathcal{B}$ a functor we can define
$s_F$ to be the transformation of functors
$G\circ F \to G' \circ F$ given by
$(s_F)_x = s_{F(x)} : G(F(x)) \to G'(F(x))$
for all $x \in \mathcal{A}$. In this way
composition with $F$ becomes a functor
$$
\text{Fun}(\mathcal{B}, \mathcal{C})
\longrightarrow
\text{Fun}(\mathcal{A}, \mathcal{C}).
$$
To see this you just have to check that
$(\text{id}_G)_F = \text{id}_{G\circ F}$ and that
$(s_1 \circ s_2)_F = s_{1, F} \circ s_{2, F}$.
Of course we also have that $s_{\text{id}_\mathcal{B}} = s$.

\medskip\noindent
These constructions satisfy the additional properties
$$
{}_{G_1}({}_{G_2}t) = {}_{G_1\circ G_2}t,
\ (s_{F_1})_{F_2} = s_{F_1 \circ F_2},
\text{ and }{}_H(s_F) = ({}_Hs)_F
$$
whenever these make sense.
Finally, given functors $F, F' : \mathcal{A} \to \mathcal{B}$,
and $G, G' : \mathcal{B} \to \mathcal{C}$ and transformations
$t : F \to F'$, and $s : G \to G'$ the following
diagram is commutative
$$
\xymatrix{
G \circ F \ar[r]^{{}_Gt} \ar[d]_{s_F}
&
G \circ F' \ar[d]^{s_{F'}} \\
G' \circ F \ar[r]_{{}_{G'}t}
&
G' \circ F'
}
$$
in other words ${}_{G'}t \circ s_F =  s_{F'}\circ {}_Gt$.
To prove this we just consider what happens on
any object $x \in \Ob(\mathcal{A})$:
$$
\xymatrix{
G(F(x)) \ar[r]^{G(t_x)} \ar[d]_{s_{F(x)}}
&
G(F'(x)) \ar[d]^{s_{F'(x)}} \\
G'(F(x)) \ar[r]_{G'(t_x)}
&
G'(F'(x))
}
$$
which is commutative because $s$ is a transformation
of functors. This compatibility relation allows us
to define horizontal composition.

\begin{definition}
\label{definition-horizontal-composition}
Given a diagram as in the left hand side of:
$$
\xymatrix{
\mathcal{A}
\rtwocell^F_{F'}{t}
&
\mathcal{B}
\rtwocell^G_{G'}{s}
&
\mathcal{C}
}
\text{ gives }
\xymatrix{
\mathcal{A}
\rrtwocell^{G \circ F} _{G' \circ F'}{\ \ s \star t}
& &
\mathcal{C}
}
$$
we define the {\it horizontal} composition $s \star t$ to be the
transformation of functors ${}_{G'}t \circ s_F =  s_{F'}\circ {}_Gt$.
\end{definition}

\noindent
Now we see that we may recover our previously constructed
transformations ${}_Gt$ and $s_F$ as
$ {}_Gt = \text{id}_G \star t $ and $ s_F = s \star \text{id}_F $.
Furthermore, all of the rules we found above are consequences of
the properties stated in the lemma that follows.

\begin{lemma}
\label{lemma-properties-2-cat-cats}
The horizontal and vertical compositions have the following
properties
\begin{enumerate}
\item $\circ$ and $\star$ are associative,
\item the identity transformations $\text{id}_F$
are units for $\circ$,
\item the identity transformations of the identity functors
$\text{id}_{\text{id}_\mathcal{A}}$
are units for $\star$ and $\circ$, and
\item given a diagram
$$
\xymatrix{
\mathcal{A}
\rruppertwocell^F{t}
\ar[rr]_(.3){F'}
\rrlowertwocell_{F''}{t'}
& &
\mathcal{B}
\rruppertwocell^G{s}
\ar[rr]_(.3){G'}
\rrlowertwocell_{G''}{s'}
& &
\mathcal{C}
}
$$
we have $ (s' \circ s) \star (t' \circ t) = (s' \star t') \circ (s \star t)$.
\end{enumerate}
\end{lemma}

\begin{proof}
The last statement turns using our previous notation into the following
equation
$$
s'_{F''}
\circ
{}_{G'}t'
\circ
s_{F'}
\circ
{}_Gt
=
(s' \circ s)_{F''}
\circ
{}_G(t' \circ t).
$$
According to our result above applied to the middle composition
we may rewrite the left hand side as
$
s'_{F''}
\circ
s_{F''}
\circ
{}_Gt'
\circ
{}_Gt
$
which is easily shown to be equal to the right hand side.
\end{proof}

\noindent
Another way of formulating condition (4) of the lemma is
that composition of functors and horizontal composition
of transformation of functors gives rise to a functor
$$
(\circ, \star) :
\text{Fun}(\mathcal{B}, \mathcal{C})
\times
\text{Fun}(\mathcal{A}, \mathcal{B})
\longrightarrow
\text{Fun}(\mathcal{A}, \mathcal{C})
$$
whose source is the product category,
see Definition \ref{definition-product-category}.

\section{2-categories}
\label{section-2-categories}

\noindent
We will give a definition of (strict) $2$-categories as they appear
in the setting of stacks. Before you read this take a look at
Section \ref{section-formal-cat-cat} and
Example \ref{example-2-1-category-of-categories}.
Basically, you take this example
and you write out all the rules satisfied by the objects, $1$-morphisms
and $2$-morphisms in that example.

\begin{definition}
\label{definition-2-category}
A (strict) {\it $2$-category} $\mathcal{C}$ consists of the following data
\begin{enumerate}
\item A set of objects $\Ob(\mathcal{C})$.
\item For each pair $x, y \in \Ob(\mathcal{C})$
a category $\Mor_\mathcal{C}(x, y)$. The objects of
$\Mor_\mathcal{C}(x, y)$ will be called {\it $1$-morphisms}
and denoted $F : x \to y$. The morphisms between these $1$-morphisms
will be called {\it $2$-morphisms} and denoted $t : F' \to F$.
The composition of $2$-morphisms in $\Mor_\mathcal{C}(x, y)$
will be called {\it vertical} composition and will be
denoted $t \circ t'$ for $t : F' \to F$ and $t' : F'' \to F'$.
\item For each triple $x, y, z\in \Ob(\mathcal{C})$ a
functor
$$
(\circ, \star) :
\Mor_\mathcal{C}(y, z) \times \Mor_\mathcal{C}(x, y)
\longrightarrow
\Mor_\mathcal{C}(x, z).
$$
The image of the pair of $1$-morphisms $(F, G)$ on the left hand side
will be called the {\it composition} of $F$ and $G$, and denoted
$F\circ G$. The image of the pair of $2$-morphisms $(t, s)$ will
be called the {\it horizontal} composition and denoted $t \star s$.
\end{enumerate}
These data are to satisfy the following rules:
\begin{enumerate}
\item The set of objects together with the set of $1$-morphisms endowed
with composition of $1$-morphisms forms a category.
\item Horizontal composition of $2$-morphisms is associative.
\item The identity $2$-morphism $\text{id}_{\text{id}_x}$
of the identity $1$-morphism $\text{id}_x$ is a unit for
horizontal composition.
\end{enumerate}
\end{definition}

\noindent
This is obviously not a very pleasant type of object to work with.
On the other hand, there are lots of examples where it is quite clear
how you work with it. The only example we have so far is that of
the $2$-category whose objects are a given collection of categories,
$1$-morphisms are functors between these categories,
and $2$-morphisms are natural transformations of functors, see
Section \ref{section-formal-cat-cat}.
As far as this text is concerned all $2$-categories will be
sub $2$-categories of this example. Here is what it means to be
a sub $2$-category.

\begin{definition}
\label{definition-sub-2-category}
Let $\mathcal{C}$ be a $2$-category.
A {\it sub $2$-category} $\mathcal{C}'$ of $\mathcal{C}$, is given by a subset
$\Ob(\mathcal{C}')$ of $\Ob(\mathcal{C})$
and sub categories $\Mor_{\mathcal{C}'}(x, y)$ of the
categories $\Mor_\mathcal{C}(x, y)$ for all
$x, y \in \Ob(\mathcal{C}')$ such that these, together with
the operations $\circ$ (composition $1$-morphisms), $\circ$ (vertical
composition $2$-morphisms), and $\star$ (horizontal composition)
form a $2$-category.
\end{definition}

\begin{remark}
\label{remark-big-2-categories}
Big $2$-categories.
In many texts a $2$-category is allowed to have a class of
objects (but hopefully a ``class of classes'' is not allowed).
We will allow these ``big'' $2$-categories as well, but only
in the following list of cases (to be updated as we go along):
\begin{enumerate}
\item The $2$-category of categories $\textit{Cat}$.
\item The $(2, 1)$-category of categories $\textit{Cat}$.
\item The $2$-category of groupoids $\textit{Groupoids}$;
this is a $(2, 1)$-category.
\item The $2$-category of fibred categories over a fixed category.
\item The $(2, 1)$-category of fibred categories over a fixed category.
\end{enumerate}
See Definition \ref{definition-2-1-category}.
Note that in each case the class of objects of the $2$-category
$\mathcal{C}$ is a proper class, but for all objects $x, y \in \Ob(C)$
the category $\Mor_\mathcal{C}(x, y)$ is ``small'' (according to
our conventions).
\end{remark}

\noindent
The notion of equivalence of categories that we defined in Section
\ref{section-definition-categories} extends to the more general setting of
$2$-categories as follows.

\begin{definition}
\label{definition-equivalence}
Two objects $x, y$ of a $2$-category are {\it equivalent} if there exist
$1$-morphisms $F : x \to y$ and $G : y \to x$ such that $F \circ G$ is
$2$-isomorphic to $\text{id}_y$ and $G \circ F$ is $2$-isomorphic to
$\text{id}_x$.
\end{definition}

\noindent
Sometimes we need to say what it means to have a functor from a
category into a $2$-category.

\begin{definition}
\label{definition-functor-into-2-category}
Let $\mathcal{A}$ be a category and let $\mathcal{C}$ be a $2$-category.
\begin{enumerate}
\item A {\it functor} from an ordinary category into a $2$-category
will ignore the
$2$-morphisms unless mentioned otherwise. In other words, it will be a
``usual'' functor into the category formed out of 2-category by forgetting
all the 2-morphisms.
\item A {\it weak functor}, or
a {\it pseudo functor} $\varphi$ from $\mathcal{A}$ into the 2-category
$\mathcal{C}$ is given by the following data
\begin{enumerate}
\item a map $\varphi : \Ob(\mathcal{A}) \to \Ob(\mathcal{C})$,
\item for every pair $x, y\in \Ob(\mathcal{A})$, and every
morphism $f : x \to y$ a $1$-morphism $\varphi(f) : \varphi(x) \to \varphi(y)$,
\item for every $x\in \Ob(A)$ a $2$-morphism
$\alpha_x : \text{id}_{\varphi(x)} \to \varphi(\text{id}_x)$, and
\item for every pair of composable morphisms $f : x \to y$,
$g : y \to z$ of $\mathcal{A}$ a $2$-morphism
$\alpha_{g, f} : \varphi(g \circ f) \to \varphi(g) \circ \varphi(f)$.
\end{enumerate}
These data are subject to the following conditions:
\begin{enumerate}
\item the $2$-morphisms $\alpha_x$ and $\alpha_{g, f}$ are all
isomorphisms,
\item for any morphism $f : x \to y$ in $\mathcal{A}$ we have
$\alpha_{\text{id}_y, f} = \alpha_y \star \text{id}_{\varphi(f)}$:
$$
\xymatrix{
\varphi(x)
\rrtwocell^{\varphi(f)}_{\varphi(f)}{\ \ \ \ \text{id}_{\varphi(f)}}
& &
\varphi(y)
\rrtwocell^{\text{id}_{\varphi(y)}}_{\varphi(\text{id}_y)}{\alpha_y}
& &
\varphi(y)
}
=
\xymatrix{
\varphi(x)
\rrtwocell^{\varphi(f)}_{\varphi(\text{id}_y) \circ \varphi(f)}{\ \ \ \ \alpha_{\text{id}_y, f}}
& &
\varphi(y)
}
$$
\item for any morphism $f : x \to y$ in $\mathcal{A}$ we have
$\alpha_{f, \text{id}_x} = \text{id}_{\varphi(f)} \star \alpha_x$,
\item for any triple of composable morphisms
$f : w \to x$, $g : x \to y$, and $h : y \to z$ of $\mathcal{A}$
we have
$$
(\text{id}_{\varphi(h)} \star \alpha_{g, f})
\circ
\alpha_{h, g \circ f}
=
(\alpha_{h, g} \star \text{id}_{\varphi(f)})
\circ
\alpha_{h \circ g, f}
$$
in other words the following diagram with objects
$1$-morphisms and arrows $2$-morphisms commutes
$$
\xymatrix{
\varphi(h \circ g \circ f)
\ar[d]_{\alpha_{h, g \circ f}}
\ar[rr]_{\alpha_{h \circ g, f}}
& &
\varphi(h \circ g) \circ \varphi(f)
\ar[d]^{\alpha_{h, g} \star \text{id}_{\varphi(f)}} \\
\varphi(h) \circ \varphi(g \circ f)
\ar[rr]^{\text{id}_{\varphi(h)} \star \alpha_{g, f}}
& &
\varphi(h) \circ \varphi(g) \circ \varphi(f)
}
$$
\end{enumerate}
\end{enumerate}
\end{definition}

\noindent
Again this is not a very workable notion, but it does sometimes come up.
There is a theorem that says that any pseudo-functor is isomorphic to
a functor. Finally, there are the notions of
{\it functor between  $2$-categories}, and
{\it pseudo functor between $2$-categories}.
This last notion leads us into $3$-category territory.
We would like to avoid having to define this at almost any cost!

\section{(2, 1)-categories}
\label{section-2-1-categories}

\noindent
Some $2$-categories have
the property that all $2$-morphisms are isomorphisms. These will
play an important role in the following, and they are easier to work with.

\begin{definition}
\label{definition-2-1-category}
A (strict) {\it $(2, 1)$-category} is a $2$-category in which all
$2$-morphisms are isomorphisms.
\end{definition}

\begin{example}
\label{example-2-1-category-of-categories}
The $2$-category $\textit{Cat}$, see Remark \ref{remark-big-2-categories},
can be turned into a $(2, 1)$-category by only allowing isomorphisms of
functors as $2$-morphisms.
\end{example}

\noindent
In fact, more generally any $2$-category
$\mathcal{C}$ produces a $(2, 1)$-category by considering the sub $2$-category
$\mathcal{C}'$ with the same objects and $1$-morphisms but whose
$2$-morphisms are the invertible $2$-morphisms of $\mathcal{C}$.
In this situation we will say ``{\it let $\mathcal{C}'$ be
the $(2, 1)$-category associated to $\mathcal{C}$}'' or similar.
For example, the $(2, 1)$-category of groupoids means the
$2$-category whose objects are groupoids, whose
$1$-morphisms are functors and whose $2$-morphisms are
isomorphisms of functors. Except that this is a bad example as a
transformation between functors between groupoids is automatically
an isomorphism!

\begin{remark}
\label{remark-other-2-categories}
Thus there are variants of the construction of
Example \ref{example-2-1-category-of-categories}
above where we look at the $2$-category of groupoids,
or categories fibred in groupoids over a fixed
category, or stacks. And so on.
\end{remark}






\section{2-fibre products}
\label{section-2-fibre-products}

\noindent
In this section we introduce $2$-fibre products. Suppose that $\mathcal{C}$
is a 2-category. We say that a diagram
$$
\xymatrix{
w \ar[r] \ar[d] & y \ar[d] \\
x \ar[r] & z }
$$
2-commutes if the two 1-morphisms $w \to y \to z$ and $w \to x \to z$ are
2-isomorphic. In a 2-category it is more natural to ask for 2-commutativity
of diagrams than for actually commuting diagrams. (Indeed, some may say that
we should not work with strict 2-categories at all, and in a ``weak''
2-category the notion of a commutative diagram of 1-morphisms does not even
make sense.) Correspondingly the notion of a fibre product has to be adjusted.

\medskip\noindent
Let $\mathcal{C}$ be a $2$-category. Let $x, y, z\in \Ob(\mathcal{C})$ and
$f\in \Mor_\mathcal{C}(x, z)$ and $g\in \Mor_{\mathcal C}(y, z)$.
In order to define the 2-fibre product of $f$ and $g$ we are going to look at
2-commutative diagrams
$$
\xymatrix{
& w \ar[r]_a \ar[d]_b & x \ar[d]^{f} \\
& y \ar[r]^{g} & z. }
$$
Now in the case of categories, the fibre product is a final object in the
category of such diagrams. Correspondingly a 2-fibre product is a final object
in a 2-category (see definition below). The {\it $2$-category
of $2$-commutative diagrams} is the $2$-category defined as follows:
\begin{enumerate}
\item Objects are quadruples $(w, a, b, \phi)$ as above where $\phi$
is an invertible 2-morphism $\phi : f \circ a \to g \circ b$,
\item 1-morphisms from $(w', a', b', \phi')$ to $(w, a, b, \phi)$ are given by
$(k : w' \to w, \alpha : a' \to a \circ k, \beta : b' \to b \circ k)$
such that
$$
\xymatrix{
f \circ a'
\ar[rr]_{\text{id}_f \star \alpha}
\ar[d]_{\phi'}
& &
f \circ a \circ k
\ar[d]^{\phi \star \text{id}_k}
\\
g \circ b'
\ar[rr]^{\text{id}_g \star \beta}
& &
g \circ b \circ k
}
$$
is commutative,
\item given a second $1$-morphism
$(k', \alpha', \beta') : (w'', a'', b'', \phi'') \to
(w', \alpha', \beta', \phi')$ the composition of $1$-morphisms
is given by the rule
$$
(k, \alpha, \beta) \circ (k', \alpha', \beta') =
(k \circ k',
(\alpha \star \text{id}_{k'}) \circ \alpha',
(\beta \star \text{id}_{k'}) \circ \beta'),
$$
\item a 2-morphism between $1$-morphisms
$(k_i, \alpha_i, \beta_i)$, $i = 1, 2$ with the same source and target
is given by a 2-morphism $\delta : k_1 \to k_2$ such that
$$
\xymatrix{
a'
\ar[rd]_{\alpha_2}
\ar[r]_{\alpha_1} &
a \circ k_1
\ar[d]^{\text{id}_a \star \delta} &
&
b \circ k_1
\ar[d]_{\text{id}_b \star \delta} &
b'
\ar[l]^{\beta_1}
\ar[ld]^{\beta_2}
\\
&
a \circ k_2
&
&
b \circ k_2
&
}
$$
commute,
\item vertical composition of $2$-morphisms is given by
vertical composition of the morphisms $\delta$ in $\mathcal{C}$, and
\item horizontal composition of the diagram
$$
\xymatrix{
(w'', a'', b'', \phi'')
\rrtwocell^{(k'_1, \alpha'_1, \beta'_1)}_{(k'_2, \alpha'_2, \beta'_2)}{\delta'}
& &
(w', a', b', \phi')
\rrtwocell^{(k_1, \alpha_1, \beta_1)}_{(k_2, \alpha_2, \beta_2)}{\delta}
& &
(w, a, b, \phi)
}
$$
is given by the diagram
$$
\xymatrix@C=12pc{
(w'', a'', b'', \phi'')
\rtwocell^{(k_1 \circ k'_1, (\alpha_1 \star \text{id}_{k'_1}) \circ \alpha'_1, (\beta_1 \star \text{id}_{k'_1}) \circ \beta'_1)}_{(k_2 \circ k'_2, (\alpha_2 \star \text{id}_{k'_2}) \circ \alpha'_2, (\beta_2 \star \text{id}_{k'_2}) \circ \beta'_2)}{\ \ \ \delta \star \delta'}
&
(w, a, b, \phi)
}
$$
\end{enumerate}
Note that if $\mathcal{C}$ is actually a $(2, 1)$-category,
the morphisms $\alpha$ and $\beta$ in (2) above are automatically
also isomorphisms\footnote{In fact it seems in the $2$-category case
that one could define another 2-category of 2-commutative diagrams where
the direction of the arrows $\alpha$, $\beta$ is reversed, or even
where the direction of only one of them is reversed. This is why
we restrict to $(2, 1)$-categories later on.}.
In addition the $2$-category of
$2$-commutative diagrams is also a $(2, 1)$-category if $\mathcal{C}$ is
a $(2, 1)$-category.

\begin{definition}
\label{definition-final-object-2-category}
A {\it final object} of a $(2, 1)$-category
$\mathcal{C}$ is an object $x$ such that
\begin{enumerate}
\item for every $y \in \Ob(\mathcal{C})$ there is a morphism $y \to x$,
and
\item every two morphisms $y \to x$ are isomorphic by a unique 2-morphism.
\end{enumerate}
\end{definition}

\noindent
Likely, in the more general case of $2$-categories there are different
flavours of final objects. We do not want to get into this and hence
we only define $2$-fibre products in the $(2, 1)$-case.

\begin{definition}
\label{definition-2-fibre-products}
Let $\mathcal{C}$ be a $(2, 1)$-category.
Let $x, y, z\in \Ob(\mathcal{C})$ and
$f\in \Mor_\mathcal{C}(x, z)$
and $g\in \Mor_{\mathcal C}(y, z)$. A
{\it 2-fibre product of $f$ and $g$} is
a final object in the category of 2-commutative diagrams
described above. If a 2-fibre product exists we
will denote it $x \times_z y\in \Ob(\mathcal{C})$, and denote the
required morphisms $p\in \Mor_{\mathcal C}(x \times_z y, x)$ and
$q\in \Mor_{\mathcal C}(x \times_z y, y)$ making the diagram
$$
\xymatrix{
& x \times_z y \ar[r]^{p} \ar[d]_q & x \ar[d]^{f} \\
& y \ar[r]^{g} & z }
$$
2-commute and we will denote the given invertible
2-morphism exhibiting this by $\psi : f \circ p \to g \circ q$.
\end{definition}

\noindent
Thus the following universal property holds: for any
$w\in \Ob(\mathcal{C})$ and morphisms
$a \in \Mor_{\mathcal C}(w, x)$ and
$b \in \Mor_\mathcal{C}(w, y)$ with a given 2-isomorphism
$\phi : f \circ a \to g\circ b$
there is a $\gamma \in \Mor_{\mathcal C}(w, x \times_z y)$
making the diagram
$$
\xymatrix{
w\ar[rrrd]^a \ar@{-->}[rrd]_\gamma \ar[rrdd]_b & & \\
& & x \times_z y \ar[r]_p \ar[d]_q & x \ar[d]^{f} \\
& & y \ar[r]^{g} & z }
$$
2-commute such that for suitable choices of
$a \to p \circ \gamma$ and $b \to q \circ \gamma$
the diagram
$$
\xymatrix{
f \circ a \ar[r] \ar[d]_\phi &
f \circ p \circ \gamma
\ar[d]^{\psi \star \text{id}_\gamma}
\\
g\circ b
\ar[r]
&
g \circ q \circ \gamma
}
$$
commutes. Moreover $\gamma$ is unique up to isomorphism.
Of course the exact properties are finer than this. All of the
cases of 2-fibre products that we will need later on come from the following
example of 2-fibre products in the 2-category of categories.

\begin{example}
\label{example-2-fibre-product-categories}
Let $\mathcal{A}$, $\mathcal{B}$, and $\mathcal{C}$ be categories.
Let $F : \mathcal{A} \to \mathcal{C}$ and $G : \mathcal{B} \to \mathcal{C}$
be functors. We define a category
$\mathcal{A} \times_\mathcal{C} \mathcal{B}$ as follows:
\begin{enumerate}
\item an object of $\mathcal{A} \times_\mathcal{C} \mathcal{B}$ is a triple
$(A, B, f)$, where $A\in \Ob(\mathcal{A})$, $B\in \Ob(\mathcal{B})$,
and $f : F(A) \to G(B)$ is an isomorphism in $\mathcal{C}$,
\item a morphism $(A, B, f) \to (A', B', f')$ is given by a pair $(a, b)$, where
$a : A \to A'$ is a morphism in $\mathcal{A}$, and $b : B \to B'$ is a
morphism in $\mathcal{B}$ such that the diagram
$$
\xymatrix{
F(A) \ar[r]^f \ar[d]^{F(a)} & G(B) \ar[d]^{G(b)} \\
F(A') \ar[r]^{f'} & G(B')
}
$$
is commutative.
\end{enumerate}
Moreover, we define functors
$p : \mathcal{A} \times_\mathcal{C}\mathcal{B} \to \mathcal{A}$
and
$q : \mathcal{A} \times_\mathcal{C}\mathcal{B} \to \mathcal{B}$
by setting
$$
p(A, B, f) = A, \quad q(A, B, f) = B,
$$
in other words, these are the forgetful functors.
We define a transformation of functors $\psi : F \circ p \to G \circ q$.
On the object $\xi = (A, B, f)$ it is given by
$\psi_\xi = f : F(p(\xi)) = F(A) \to G(B) = G(q(\xi))$.
\end{example}

\begin{lemma}
\label{lemma-2-fibre-product-categories}
In the $(2, 1)$-category of categories $2$-fibre products exist and
are given by the construction of
Example \ref{example-2-fibre-product-categories}.
\end{lemma}

\begin{proof}
Let us check the universal property:
let $\mathcal{W}$ be a category, let
$a : \mathcal{W} \to \mathcal{A}$ and
$b : \mathcal{W} \to \mathcal{B}$ be functors, and
let $t : F \circ a \to G \circ b$ be an isomorphism of functors.

\medskip\noindent
Consider the functor
$\gamma : \mathcal{W} \to \mathcal{A} \times_\mathcal{C}\mathcal{B}$
given by $W \mapsto (a(W), b(W), t_W)$.
(Check this is a functor omitted.)
Moreover, consider $\alpha : a \to p \circ \gamma$ and
$\beta : b \to q \circ \gamma$ obtained from the identities
$p \circ \gamma = a$ and $q \circ \gamma = b$. Then it is
clear that $(\gamma, \alpha, \beta)$ is a morphism
from $(W, a, b, t)$ to
$(\mathcal{A} \times_\mathcal{C} \mathcal{B}, p, q, \psi)$.

\medskip\noindent
Let
$(k, \alpha', \beta') :
(W, a, b, t) \to (\mathcal{A} \times_\mathcal{C} \mathcal{B}, p, q, \psi)$
be a second such morphism. For an object $W$ of $\mathcal{W}$ let us write
$k(W) = (a_k(W), b_k(W), t_{k, W})$. Hence $p(k(W)) = a_k(W)$ and so on.
The map $\alpha'$ corresponds to functorial maps
$\alpha' : a(W) \to a_k(W)$. Since we are working in the
$(2, 1)$-category of categories, in fact each of the maps
$a(W) \to a_k(W)$ is an isomorphism. We can use these
(and their counterparts $b(W) \to b_k(W)$) to get isomorphisms
$$
\delta_W :
\gamma(W) = (a(W), b(W), t_W)
\longrightarrow
(a_k(W), b_k(W), t_{k, W}) = k(W).
$$
It is straightforward to show that $\delta$ defines a
$2$-isomorphism between $\gamma$ and $k$ in the $2$-category
of $2$-commutative diagrams as desired.
\end{proof}

\begin{remark}
\label{remark-other-description-2-fibre-product}
Let $\mathcal{A}$, $\mathcal{B}$, and $\mathcal{C}$ be categories.
Let $F : \mathcal{A} \to \mathcal{C}$ and $G : \mathcal{B} \to \mathcal{C}$
be functors. Another, slightly more symmetrical, construction of a $2$-fibre
product $\mathcal{A} \times_\mathcal{C} \mathcal{B}$ is as follows.
An object is a quintuple $(A, B, C, a, b)$ where $A, B, C$ are objects
of $\mathcal{A}, \mathcal{B}, \mathcal{C}$ and where $a : F(A) \to C$
and $b : G(B) \to C$ are isomorphisms. A morphism
$(A, B, C, a, b) \to (A', B', C', a', b')$ is given by a triple
of morphisms $A \to A', B \to B', C \to C'$ compatible with the morphisms
$a, b, a', b'$. We can prove directly that this leads to a $2$-fibre
product. However, it is easier to observe that the functor
$(A, B, C, a, b) \mapsto (A, B, b^{-1} \circ a)$ gives an equivalence
from the category of quintuples to the category constructed in
Example \ref{example-2-fibre-product-categories}.
\end{remark}

\begin{lemma}
\label{lemma-functoriality-2-fibre-product}
Let
$$
\xymatrix{
& \mathcal{Y} \ar[d]_I \ar[rd]^K & \\
\mathcal{X} \ar[r]^H \ar[rd]^L &
\mathcal{Z} \ar[rd]^M & \mathcal{B} \ar[d]^G \\
& \mathcal{A} \ar[r]^F & \mathcal{C}
}
$$
be a $2$-commutative diagram of categories.
A choice of isomorphisms
$\alpha : G \circ K \to M \circ I$ and
$\beta : M \circ H \to F \circ L$
determines a morphism
$$
\mathcal{X} \times_\mathcal{Z} \mathcal{Y}
\longrightarrow
\mathcal{A} \times_\mathcal{C} \mathcal{B}
$$
of $2$-fibre products associated to this situation.
\end{lemma}

\begin{proof}
Just use the functor
$$
(X, Y, \phi) \longmapsto (L(X), K(Y),
\alpha^{-1}_Y \circ M(\phi) \circ \beta^{-1}_X)
$$
on objects and
$$
(a, b) \longmapsto (L(a), K(b))
$$
on morphisms.
\end{proof}

\begin{lemma}
\label{lemma-equivalence-2-fibre-product}
Assumptions as in Lemma \ref{lemma-functoriality-2-fibre-product}.
\begin{enumerate}
\item If $K$ and $L$ are faithful
then the morphism
$\mathcal{X} \times_\mathcal{Z} \mathcal{Y} \to
\mathcal{A} \times_\mathcal{C} \mathcal{B}$
is faithful.
\item If $K$ and $L$ are fully faithful and $M$ is faithful
then the morphism
$\mathcal{X} \times_\mathcal{Z} \mathcal{Y} \to
\mathcal{A} \times_\mathcal{C} \mathcal{B}$
is fully faithful.
\item If $K$ and $L$ are equivalences and $M$ is fully faithful
then the morphism
$\mathcal{X} \times_\mathcal{Z} \mathcal{Y} \to
\mathcal{A} \times_\mathcal{C} \mathcal{B}$
is an equivalence.
\end{enumerate}
\end{lemma}

\begin{proof}
Let $(X, Y, \phi)$ and $(X', Y', \phi')$ be objects of
$\mathcal{X} \times_\mathcal{Z} \mathcal{Y}$.
Set $Z = H(X)$ and identify it with $I(Y)$ via $\phi$.
Also, identify $M(Z)$ with $F(L(X))$ via $\alpha_X$ and
identify $M(Z)$ with $G(K(Y))$ via $\beta_Y$. Similarly for
$Z' = H(X')$ and $M(Z')$.
The map on morphisms is the map
$$
\xymatrix{
\Mor_\mathcal{X}(X, X')
\times_{\Mor_\mathcal{Z}(Z, Z')}
\Mor_\mathcal{Y}(Y, Y')
\ar[d] \\
\Mor_\mathcal{A}(L(X), L(X'))
\times_{\Mor_\mathcal{C}(M(Z), M(Z'))}
\Mor_\mathcal{B}(K(Y), K(Y'))
}
$$
Hence parts (1) and (2) follow. Moreover, if $K$ and $L$
are equivalences and $M$ is fully faithful, then any object
$(A, B, \phi)$ is in the essential image for the following reasons:
Pick $X$, $Y$ such that $L(X) \cong A$ and $K(Y) \cong B$.
Then the fully faithfulness of $M$ guarantees that we can
find an isomorphism $H(X) \cong I(Y)$. Some details omitted.
\end{proof}

\begin{lemma}
\label{lemma-associativity-2-fibre-product}
Let
$$
\xymatrix{
\mathcal{A} \ar[rd] & & \mathcal{C} \ar[ld] \ar[rd] & & \mathcal{E} \ar[ld] \\
& \mathcal{B} & & \mathcal{D}
}
$$
be a diagram of categories and functors.
Then there is a canonical isomorphism
$$
(\mathcal{A} \times_\mathcal{B} \mathcal{C}) \times_\mathcal{D} \mathcal{E}
\cong
\mathcal{A} \times_\mathcal{B} (\mathcal{C} \times_\mathcal{D} \mathcal{E})
$$
of categories.
\end{lemma}

\begin{proof}
Just use the functor
$$
((A, C, \phi), E, \psi)
\longmapsto
(A, (C, E, \psi), \phi)
$$
if you know what I mean.
\end{proof}

\noindent
Henceforth we do not write the parentheses when dealing with fibre products
of more than 2 categories.

\begin{lemma}
\label{lemma-triple-2-fibre-product-pr02}
Let
$$
\xymatrix{
\mathcal{A} \ar[rd] & & \mathcal{C} \ar[ld] \ar[rd] & & \mathcal{E} \ar[ld] \\
& \mathcal{B} \ar[rd]_F & & \mathcal{D} \ar[ld]^G \\
& & \mathcal{F} &
}
$$
be a commutative diagram of categories and functors.
Then there is a canonical functor
$$
\text{pr}_{02} :
\mathcal{A} \times_\mathcal{B} \mathcal{C} \times_\mathcal{D} \mathcal{E}
\longrightarrow
\mathcal{A} \times_\mathcal{F} \mathcal{E}
$$
of categories.
\end{lemma}

\begin{proof}
If we write
$\mathcal{A} \times_\mathcal{B} \mathcal{C}
\times_\mathcal{D} \mathcal{E}$
as
$(\mathcal{A} \times_\mathcal{B} \mathcal{C})
\times_\mathcal{D} \mathcal{E}$
then we can just use the functor
$$
((A, C, \phi), E, \psi)
\longmapsto
(A, E, G(\psi) \circ F(\phi))
$$
if you know what I mean.
\end{proof}

\begin{lemma}
\label{lemma-2-fibre-product-erase-factor}
Let
$$
\mathcal{A} \to
\mathcal{B} \leftarrow \mathcal{C} \leftarrow \mathcal{D}
$$
be a diagram of categories and functors.
Then there is a canonical isomorphism
$$
\mathcal{A} \times_\mathcal{B} \mathcal{C} \times_\mathcal{C} \mathcal{D}
\cong
\mathcal{A} \times_\mathcal{B} \mathcal{D}
$$
of categories.
\end{lemma}

\begin{proof}
Omitted.
\end{proof}

\noindent
We claim that this means you can work with these $2$-fibre products
just like with ordinary fibre products. Here are some further lemmas
that actually come up later.

\begin{lemma}
\label{lemma-diagonal-1}
Let
$$
\xymatrix{
\mathcal{C}_3 \ar[r] \ar[d] & \mathcal{S} \ar[d]^\Delta \\
\mathcal{C}_1 \times \mathcal{C}_2 \ar[r]^{G_1 \times G_2} &
\mathcal{S} \times \mathcal{S}
}
$$
be a $2$-fibre product of categories.
Then there is a canonical isomorphism
$\mathcal{C}_3 \cong
\mathcal{C}_1 \times_{G_1, \mathcal{S}, G_2} \mathcal{C}_2$.
\end{lemma}

\begin{proof}
We may assume that $\mathcal{C}_3$ is the category
$(\mathcal{C}_1 \times \mathcal{C}_2)\times_{\mathcal{S} \times \mathcal{S}}
\mathcal{S}$ constructed in Example \ref{example-2-fibre-product-categories}.
Hence an object is a triple
$((X_1, X_2), S, \phi)$ where
$\phi = (\phi_1, \phi_2) : (G_1(X_1), G_2(X_2)) \to (S, S)$
is an isomorphism. Thus we can associate to this the triple
$(X_1, X_2, \phi_2^{-1} \circ \phi_1)$.
Conversely, if $(X_1, X_2, \psi)$ is an object of
$\mathcal{C}_1 \times_{G_1, \mathcal{S}, G_2} \mathcal{C}_2$,
then we can associate to this the triple
$((X_1, X_2), G_2(X_2), (\psi, \text{id}_{G_2(X_2)}))$.
We claim these constructions given mutually inverse functors.
We omit describing how to deal with morphisms
and showing they are mutually inverse.
\end{proof}

\begin{lemma}
\label{lemma-diagonal-2}
Let
$$
\xymatrix{
\mathcal{C}' \ar[r] \ar[d] & \mathcal{S} \ar[d]^\Delta \\
\mathcal{C} \ar[r]^{G_1 \times G_2} &
\mathcal{S} \times \mathcal{S}
}
$$
be a $2$-fibre product of categories.
Then there is a canonical isomorphism
$$
\mathcal{C}' \cong
(\mathcal{C} \times_{G_1, \mathcal{S}, G_2} \mathcal{C})
\times_{(p, q), \mathcal{C} \times \mathcal{C}, \Delta}
\mathcal{C}.
$$
\end{lemma}

\begin{proof}
An object of the right hand side is given by
$((C_1, C_2, \phi), C_3, \psi)$ where
$\phi : G_1(C_1) \to G_2(C_2)$ is an isomorphism
and $\psi = (\psi_1, \psi_2) : (C_1, C_2) \to (C_3, C_3)$ is
an isomorphism. Hence we can associate to this the triple
$(C_3, G_1(C_1), (G_1(\psi_1^{-1}), \phi^{-1} \circ G_2(\psi_2^{-1})))$
which is an object of $\mathcal{C}'$.
Details omitted.
\end{proof}

\begin{lemma}
\label{lemma-fibre-product-after-map}
Let $\mathcal{A} \to \mathcal{C}$, $\mathcal{B} \to \mathcal{C}$
and $\mathcal{C} \to \mathcal{D}$ be functors between categories.
Then the diagram
$$
\xymatrix{
\mathcal{A} \times_\mathcal{C} \mathcal{B} \ar[d] \ar[r] &
\mathcal{A} \times_\mathcal{D} \mathcal{B} \ar[d] \\
\mathcal{C} \ar[r]^-{\Delta_{\mathcal{C}/\mathcal{D}}} \ar[r] &
\mathcal{C} \times_\mathcal{D} \mathcal{C}
}
$$
is a $2$-fibre product diagram.
\end{lemma}

\begin{proof}
Omitted.
\end{proof}

\begin{lemma}
\label{lemma-base-change-diagonal}
Let
$$
\xymatrix{
\mathcal{U} \ar[d] \ar[r] & \mathcal{V} \ar[d] \\
\mathcal{X} \ar[r] & \mathcal{Y}
}
$$
be a $2$-fibre product of categories. Then the diagram
$$
\xymatrix{
\mathcal{U} \ar[d] \ar[r] &
\mathcal{U} \times_\mathcal{V} \mathcal{U} \ar[d] \\
\mathcal{X} \ar[r] &
\mathcal{X} \times_\mathcal{Y} \mathcal{X}
}
$$
is $2$-cartesian.
\end{lemma}

\begin{proof}
This is a purely $2$-category theoretic statement, valid in any
$(2, 1)$-category with $2$-fibre products. Explicitly, it follows
from the following chain of equivalences:
\begin{align*}
\mathcal{X} \times_{(\mathcal{X} \times_\mathcal{Y} \mathcal{X})}
(\mathcal{U} \times_\mathcal{V} \mathcal{U})
& =
\mathcal{X} \times_{(\mathcal{X} \times_\mathcal{Y} \mathcal{X})}
((\mathcal{X} \times_\mathcal{Y} \mathcal{V})
\times_\mathcal{V} (\mathcal{X} \times_\mathcal{Y} \mathcal{V})) \\
& =
\mathcal{X} \times_{(\mathcal{X} \times_\mathcal{Y} \mathcal{X})}
(\mathcal{X} \times_\mathcal{Y} \mathcal{X}
\times_\mathcal{Y} \mathcal{V}) \\
& =
\mathcal{X} \times_\mathcal{Y} \mathcal{V} = \mathcal{U}
\end{align*}
see
Lemmas \ref{lemma-associativity-2-fibre-product} and
\ref{lemma-2-fibre-product-erase-factor}.
\end{proof}








\section{Categories over categories}
\label{section-categories-over-categories}

\noindent
In this section we have a functor $p : \mathcal{S} \to \mathcal{C}$.
We think of $\mathcal{S}$ as being on top and of $\mathcal{C}$ as being
at the bottom. To make sure that everybody knows what we are talking about
we define the $2$-category of categories over $\mathcal{C}$.

\begin{definition}
\label{definition-categories-over-C}
Let $\mathcal{C}$ be a category.
The {\it $2$-category of categories over $\mathcal{C}$}
is the $2$-category defined as follows:
\begin{enumerate}
\item Its objects will be functors $p : \mathcal{S} \to \mathcal{C}$.
\item Its $1$-morphisms $(\mathcal{S}, p) \to (\mathcal{S}', p')$
will be functors $G : \mathcal{S} \to \mathcal{S}'$ such that
$p' \circ G = p$.
\item Its $2$-morphisms $t : G \to H$ for
$G, H : (\mathcal{S}, p) \to (\mathcal{S}', p')$
will be morphisms of functors
such that $p'(t_x) = \text{id}_{p(x)}$
for all $x \in \Ob(\mathcal{S})$.
\end{enumerate}
In this situation we will denote
$$
\Mor_{\textit{Cat}/\mathcal{C}}(\mathcal{S}, \mathcal{S}')
$$
the category of $1$-morphisms between
$(\mathcal{S}, p)$ and $(\mathcal{S}', p')$
\end{definition}

\noindent
In this $2$-category we define horizontal and vertical composition
exactly as is done for $\textit{Cat}$ in Section \ref{section-formal-cat-cat}.
The axioms of a $2$-category are satisfied for the same reason
that the hold in $\textit{Cat}$. To see this one can also use that
the axioms hold in $\textit{Cat}$ and verify
things such as ``vertical composition of $2$-morphisms over $\mathcal{C}$
gives another $2$-morphism over $\mathcal{C}$''. This is clear.

\medskip\noindent
Analogously to the fibre of a map of spaces, we have the notion of a
fibre category, and some notions of lifting associated to this
situation.

\begin{definition}
\label{definition-fibre-category}
Let $\mathcal{C}$ be a category.
Let $p : \mathcal{S} \to \mathcal{C}$ be a category over $\mathcal{C}$.
\begin{enumerate}
\item The {\it fibre category} over an object $U\in \Ob(\mathcal{C})$
is the category $\mathcal{S}_U$ with objects
$$
\Ob(\mathcal{S}_U) = \{x\in \Ob(\mathcal{S}) :
p(x) = U\}
$$
and morphisms
$$
\Mor_{\mathcal{S}_U}(x, y) = \{ \phi \in \Mor_\mathcal{S}(x, y) :
p(\phi) = \text{id}_U\}.
$$
\item A {\it lift} of an object $U \in \Ob(\mathcal{C})$
is an object $x\in \Ob(\mathcal{S})$ such that $p(x) = U$, i.e.,
$x\in \Ob(\mathcal{S}_U)$. We will also sometime say
that {\it $x$ lies over $U$}.
\item Similarly, a {\it lift} of a morphism $f : V \to U$ in $\mathcal{C}$
is a morphism $\phi : y \to x$ in $\mathcal{S}$ such that $p(\phi) = f$.
We sometimes say that {\it $\phi$ lies over $f$}.
\end{enumerate}
\end{definition}

\noindent
There are some observations we could make here. For example if
$F : (\mathcal{S}, p) \to (\mathcal{S}', p')$ is a $1$-morphism
of categories over $\mathcal{C}$, then $F$ induces functors
of fibre categories $F : \mathcal{S}_U \to \mathcal{S}'_U$.
Similarly for $2$-morphisms.

\medskip\noindent
Here is the obligatory lemma describing the $2$-fibre product in the
$(2, 1)$-category of categories over $\mathcal{C}$.

\begin{lemma}
\label{lemma-2-product-categories-over-C}
Let $\mathcal{C}$ be a category.
The $(2, 1)$-category of categories
over $\mathcal{C}$ has 2-fibre products.
Suppose that
$F : \mathcal{X} \to \mathcal{S}$ and
$G : \mathcal{Y} \to \mathcal{S}$ are morphisms of categories over
$\mathcal{C}$.
An explicit 2-fibre product
$\mathcal{X} \times_\mathcal{S}\mathcal{Y}$ is given by the following
description
\begin{enumerate}
\item an object of $\mathcal{X} \times_\mathcal{S} \mathcal{Y}$ is a quadruple
$(U, x, y, f)$, where $U \in \Ob(\mathcal{C})$,
$x\in \Ob(\mathcal{X}_U)$, $y\in \Ob(\mathcal{Y}_U)$,
and $f : F(x) \to G(y)$ is an isomorphism in $\mathcal{S}_U$,
\item a morphism $(U, x, y, f) \to (U', x', y', f')$ is given by a pair
$(a, b)$, where $a : x \to x'$ is a morphism in $\mathcal{X}$, and
$b : y \to y'$ is a
morphism in $\mathcal{Y}$ such that
\begin{enumerate}
\item $a$ and $b$ induce the same morphism $U \to U'$, and
\item the diagram
$$
\xymatrix{
F(x) \ar[r]^f \ar[d]^{F(a)} & G(y) \ar[d]^{G(b)} \\
F(x') \ar[r]^{f'} & G(y')
}
$$
is commutative.
\end{enumerate}
\end{enumerate}
The functors $p : \mathcal{X} \times_\mathcal{S}\mathcal{Y} \to \mathcal{X}$
and $q : \mathcal{X} \times_\mathcal{S}\mathcal{Y} \to \mathcal{Y}$ are the
forgetful functors in this case. The transformation $\psi : F \circ p \to
G \circ q$ is given on the object $\xi = (U, x, y, f)$ by
$\psi_\xi = f : F(p(\xi)) = F(x) \to G(y) = G(q(\xi))$.
\end{lemma}

\begin{proof}
Let us check the universal property: let
$p_\mathcal{W} : \mathcal{W}\to \mathcal{C}$
be a category over $\mathcal{C}$, let $X : \mathcal{W} \to \mathcal{X}$ and
$Y : \mathcal{W} \to \mathcal{Y}$ be functors over $\mathcal{C}$, and let
$t : F \circ X \to G \circ Y$ be an isomorphism of functors over $\mathcal{C}$.
The desired functor
$\gamma : \mathcal{W} \to \mathcal{X} \times_\mathcal{S} \mathcal{Y}$
is given by $W \mapsto (p_\mathcal{W}(W), X(W), Y(W), t_W)$.
Details omitted; compare with Lemma \ref{lemma-2-fibre-product-categories}.
\end{proof}

\begin{lemma}
\label{lemma-fibre-2-fibre-product-categories-over-C}
Let $\mathcal{C}$ be a category.
Let $f : \mathcal{X} \to \mathcal{S}$ and
$g : \mathcal{Y} \to \mathcal{S}$ be morphisms of categories over
$\mathcal{C}$. For any object $U$ of $\mathcal{C}$ we have
the following identity of
fibre categories
$$
\left(\mathcal{X} \times_\mathcal{S}\mathcal{Y}\right)_U
=
\mathcal{X}_U \times_{\mathcal{S}_U} \mathcal{Y}_U
$$
\end{lemma}

\begin{proof}
Omitted.
\end{proof}







\section{Fibred categories}
\label{section-fibred-categories}

\noindent
A very brief discussion of fibred categories is warranted.

\medskip\noindent
Let $p : \mathcal{S} \to \mathcal{C}$ be a category over $\mathcal{C}$.
Given an object $x \in \mathcal{S}$ with $p(x) = U$, and given a morphism
$f : V \to U$, we can try to take some kind of ``fibre product
$V \times_U x$'' (or a {\it base change} of $x$ via $V \to U$).
Namely, a morphism from an object $z \in \mathcal{S}$
into ``$V \times_U x$'' should be given by a pair
$(\varphi, g)$, where
$\varphi : z \to x$, $g : p(z) \to V$ such that
$p(\varphi) = f \circ g$. Pictorially:
$$
\xymatrix{
z \ar@{~>}[d]^p \ar@{-}[r] &
? \ar[r] \ar@{~>}[d]^p &
x \ar@{~>}[d]^p \\
p(z) \ar[r] & V \ar[r]^f & U
}
$$
If such a morphism $V \times_U x \to x$ exists then it is called
a strongly cartesian morphism.

\begin{definition}
\label{definition-cartesian-over-C}
Let $\mathcal{C}$ be a category.
Let $p : \mathcal{S} \to \mathcal{C}$ be a category over $\mathcal{C}$.
A {\it strongly cartesian morphism}, or more precisely a
{\it strongly $\mathcal{C}$-cartesian morphism} is a
morphism $\varphi : y \to x$ of $\mathcal{S}$ such that
for every $z \in \Ob(\mathcal{S})$ the map
$$
\Mor_\mathcal{S}(z, y)
\longrightarrow
\Mor_\mathcal{S}(z, x)
\times_{\Mor_\mathcal{C}(p(z), p(x))}
\Mor_\mathcal{C}(p(z), p(y)),
$$
given by $\psi \longmapsto (\varphi \circ \psi, p(\psi))$
is bijective.
\end{definition}

\noindent
Note that by the Yoneda Lemma \ref{lemma-yoneda}, given
$x \in \Ob(\mathcal{S})$ lying over $U \in \Ob(\mathcal{C})$
and the morphism $f : V \to U$ of $\mathcal{C}$, if there is a
strongly cartesian morphism $\varphi : y \to x$ with $p(\varphi) = f$,
then $(y, \varphi)$ is unique up to unique isomorphism. This is
clear from the definition above, as the functor
$$
z
\longmapsto
\Mor_\mathcal{S}(z, x)
\times_{\Mor_\mathcal{C}(p(z), U)}
\Mor_\mathcal{C}(p(z), V)
$$
only depends on the data $(x, U, f : V \to U)$. Hence we
will sometimes use $V \times_U x \to x$ or $f^*x \to x$
to denote a strongly cartesian morphism which is a lift of $f$.

\begin{lemma}
\label{lemma-composition-cartesian}
Let $\mathcal{C}$ be a category.
Let $p : \mathcal{S} \to \mathcal{C}$ be a category over $\mathcal{C}$.
\begin{enumerate}
\item The composition of two strongly cartesian morphisms
is strongly cartesian.
\item Any isomorphism of $\mathcal{S}$ is strongly cartesian.
\item Any strongly cartesian morphism $\varphi$ such that $p(\varphi)$
is an isomorphism, is an isomorphism.
\end{enumerate}
\end{lemma}

\begin{proof}
Proof of (1). Let $\varphi : y \to x$ and $\psi : z \to y$ be
strongly cartesian. Let $t$ be an arbitrary object of $\mathcal{S}$.
Then we have
\begin{align*}
& \Mor_\mathcal{S}(t, z) \\
& =
\Mor_\mathcal{S}(t, y)
\times_{\Mor_\mathcal{C}(p(t), p(y))}
\Mor_\mathcal{C}(p(t), p(z)) \\
& =
\Mor_\mathcal{S}(t, x)
\times_{\Mor_\mathcal{C}(p(t), p(x))}
\Mor_\mathcal{C}(p(t), p(y))
\times_{\Mor_\mathcal{C}(p(t), p(y))}
\Mor_\mathcal{C}(p(t), p(z)) \\
& =
\Mor_\mathcal{S}(t, x)
\times_{\Mor_\mathcal{C}(p(t), p(x))}
\Mor_\mathcal{C}(p(t), p(z))
\end{align*}
hence $z \to x$ is strongly cartesian.

\medskip\noindent
Proof of (2). Let $y \to x$ be an isomorphism. Then $p(y) \to p(x)$
is an isomorphism too. Hence
$\Mor_\mathcal{C}(p(z), p(y)) \to
\Mor_\mathcal{C}(p(z), p(x))$
is a bijection. Hence
$\Mor_\mathcal{S}(z, x)
\times_{\Mor_\mathcal{C}(p(z), p(x))}
\Mor_\mathcal{C}(p(z), p(y))$ is bijective to
$\Mor_\mathcal{S}(z, x)$.
Hence the displayed map of
Definition \ref{definition-cartesian-over-C}
is a bijection as $y \to x$ is an isomorphism, and we conclude that
$y \to x$ is strongly cartesian.

\medskip\noindent
Proof of (3). Assume $\varphi : y \to x$ is strongly cartesian with
$p(\varphi) : p(y) \to p(x)$ an isomorphism. Applying the definition with
$z = x$ shows that $(\text{id}_x, p(\varphi)^{-1})$ comes from a unique
morphism $\chi : x \to y$. We omit the verification that $\chi$ is the
inverse of $\varphi$.
\end{proof}

\begin{lemma}
\label{lemma-cartesian-over-cartesian}
Let $F : \mathcal{A} \to \mathcal{B}$ and $G : \mathcal{B} \to \mathcal{C}$
be composable functors between categories. Let $x \to y$ be a morphism of
$\mathcal{A}$. If $x \to y$ is strongly $\mathcal{B}$-cartesian
and $F(x) \to F(y)$ is strongly $\mathcal{C}$-cartesian, then
$x \to y$ is strongly $\mathcal{C}$-cartesian.
\end{lemma}

\begin{proof}
This follows directly from the definition.
\end{proof}

\begin{lemma}
\label{lemma-strongly-cartesian-fibre-product}
Let $\mathcal{C}$ be a category.
Let $p : \mathcal{S} \to \mathcal{C}$ be a category over $\mathcal{C}$.
Let $x \to y$ and $z \to y$ be morphisms of $\mathcal{S}$.
Assume
\begin{enumerate}
\item $x \to y$ is strongly cartesian,
\item $p(x) \times_{p(y)} p(z)$ exists, and
\item there exists a strongly cartesian morphism $a : w \to z$ in
$\mathcal{S}$ with $p(w) = p(x) \times_{p(y)} p(z)$ and
$p(a) = \text{pr}_2 : p(x) \times_{p(y)} p(z) \to p(z)$.
\end{enumerate}
Then the fibre product $x \times_y z$ exists and is isomorphic to $w$.
\end{lemma}

\begin{proof}
Since $x \to y$ is strongly cartesian there exists a unique morphism
$b : w \to x$ such that $p(b) = \text{pr}_1$. To see that $w$ is the
fibre product we compute
\begin{align*}
& \Mor_\mathcal{S}(t, w) \\
& = \Mor_\mathcal{S}(t, z)
\times_{\Mor_\mathcal{C}(p(t), p(z))}
\Mor_\mathcal{C}(p(t), p(w)) \\
& = \Mor_\mathcal{S}(t, z)
\times_{\Mor_\mathcal{C}(p(t), p(z))}
(\Mor_\mathcal{C}(p(t), p(x))
\times_{\Mor_\mathcal{C}(p(t), p(y))}
\Mor_\mathcal{C}(p(t), p(z))) \\
& = \Mor_\mathcal{S}(t, z)
\times_{\Mor_\mathcal{C}(p(t), p(y))}
\Mor_\mathcal{C}(p(t), p(x)) \\
& = \Mor_\mathcal{S}(t, z)
\times_{\Mor_\mathcal{S}(t, y)}
\Mor_\mathcal{S}(t, y)
\times_{\Mor_\mathcal{C}(p(t), p(y))}
\Mor_\mathcal{C}(p(t), p(x)) \\
& = \Mor_\mathcal{S}(t, z)
\times_{\Mor_\mathcal{S}(t, y)}
\Mor_\mathcal{S}(t, x)
\end{align*}
as desired. The first equality holds because $a : w \to z$ is strongly
cartesian and the last equality holds because $x \to y$ is strongly
cartesian.
\end{proof}

\begin{definition}
\label{definition-fibred-category}
Let $\mathcal{C}$ be a category.
Let $p : \mathcal{S} \to \mathcal{C}$ be a category over $\mathcal{C}$.
We say $\mathcal{S}$ is a {\it fibred category over $\mathcal{C}$}
if given any $x \in \Ob(\mathcal{S})$ lying over
$U \in \Ob(\mathcal{C})$ and any morphism $f : V \to U$ of
$\mathcal{C}$, there exists a strongly cartesian morphism $f^*x \to x$
lying over $f$.
\end{definition}

\noindent
Assume $p : \mathcal{S} \to \mathcal{C}$ is a fibred category.
For every $f : V \to U$ and $x\in \Ob(\mathcal{S}_U)$
as in the definition we may choose a strongly cartesian morphism
$f^\ast x \to x$ lying over $f$. By the axiom of choice we may choose
$f^*x \to x$ for all $f: V \to U = p(x)$ simultaneously.
We claim that for every morphism $\phi : x \to x'$ in $\mathcal{S}_U$
and $f : V \to U$ there is a unique
morphism $f^\ast \phi : f^\ast x \to f^\ast x'$ in $\mathcal{S}_V$
such that
$$
\xymatrix{
f^\ast x \ar[r]_{f^\ast \phi} \ar[d] & f^\ast x' \ar[d] \\
x \ar[r]^{\phi} & x' }
$$
commutes. Namely, the arrow exists and is unique because $f^*x' \to x'$ is
strongly cartesian. The uniqueness of this arrow guarantees that
$f^\ast$ (now also defined on morphisms) is a
functor $ f^\ast : \mathcal{S}_U \to \mathcal{S}_V$.

\begin{definition}
\label{definition-pullback-functor-fibred-category}
Assume $p : \mathcal{S} \to \mathcal{C}$ is a fibred category.
\begin{enumerate}
\item A {\it choice of pullbacks}\footnote{This is probably nonstandard
terminology. In some texts this is called a ``cleavage''  but it conjures up
the wrong image. Maybe a ``cleaving'' would be a better word.
A related notion is that of a ``splitting'', but in many texts a ``splitting''
means a choice of pullbacks such that $g^*f^* = (f \circ g)^*$
for any composable pair of morphisms. Compare
also with Definition \ref{definition-split-fibred-category}.}
for $p : \mathcal{S} \to \mathcal{C}$
is given by a choice of a strongly cartesian morphism
$f^\ast x \to x$ lying over $f$ for any morphism
$f: V \to U$ of $\mathcal{C}$ and any $x \in \Ob(\mathcal{S}_U)$.
\item Given a choice of pullbacks,
for any morphism $f : V \to U$ of $\mathcal{C}$
the functor $f^* : \mathcal{S}_U \to \mathcal{S}_V$ described
above is called a {\it pullback functor} (associated to the choices
$f^*x \to x$ made above).
\end{enumerate}
\end{definition}

\noindent
Of course we may always assume our choice of pullbacks has the property that
$\text{id}_U^*x = x$, although in practice this is a useless property
without imposing further assumptions on the pullbacks.

\begin{lemma}
\label{lemma-fibred}
Assume $p : \mathcal{S} \to \mathcal{C}$ is a fibred category.
Assume given a choice of pullbacks for $p : \mathcal{S} \to \mathcal{C}$.
\begin{enumerate}
\item For any pair of composable morphisms $f : V \to U$,
$g : W \to V$ there is a unique isomorphism
$$
\alpha_{g, f} :
(f \circ g)^\ast
\longrightarrow
g^\ast \circ f^\ast
$$
as functors $\mathcal{S}_U \to \mathcal{S}_W$
such that for every $y\in \Ob(\mathcal{S}_U)$ the following
diagram commutes
$$
\xymatrix{
g^\ast f^\ast y \ar[r]
&
f^\ast y \ar[d] \\
(f \circ g)^\ast y \ar[r]
\ar[u]^{(\alpha_{g, f})_y}
&
y
}
$$
\item If $f = \text{id}_U$, then there is a canonical isomorphism
$\alpha_U : \text{id} \to (\text{id}_U)^*$ as functors
$\mathcal{S}_U \to \mathcal{S}_U$.
\item The quadruple
$(U \mapsto \mathcal{S}_U, f \mapsto f^*, \alpha_{g, f}, \alpha_U)$
defines a pseudo functor from $\mathcal{C}^{opp}$ to
the $(2, 1)$-category of categories, see
Definition \ref{definition-functor-into-2-category}.
\end{enumerate}
\end{lemma}

\begin{proof}
In fact, it is clear that the commutative diagram of
part (1) uniquely determines the morphism
$(\alpha_{g, f})_y$ in the fibre category
$\mathcal{S}_W$. It is an isomorphism since both
the morphism $(f \circ g)^*y \to y$
and the composition $g^*f^*y \to f^*y \to y$ are strongly
cartesian morphisms lifting $f \circ g$ (see discussion
following Definition \ref{definition-cartesian-over-C} and
Lemma \ref{lemma-composition-cartesian}). In the same way,
since $\text{id}_x : x \to x$ is clearly strongly cartesian
over $\text{id}_U$ (with $U = p(x)$) we see that there exists
an isomorphism $(\alpha_U)_x : x \to (\text{id}_U)^*x$.
(Of course we could have assumed beforehand that $f^*x = x$
whenever $f$ is an identity morphism, but it is better for
the sake of generality not to assume this.)
We omit the verification that $\alpha_{g, f}$ and
$\alpha_U$ so obtained are transformations of functors.
We also omit the verification of (3).
\end{proof}

\begin{lemma}
\label{lemma-fibred-equivalent}
Let $\mathcal{C}$ be a category.
Let $\mathcal{S}_1$, $\mathcal{S}_2$ be categories over $\mathcal{C}$.
Suppose that $\mathcal{S}_1$ and $\mathcal{S}_2$ are equivalent
as categories over $\mathcal{C}$.
Then $\mathcal{S}_1$ is fibred over $\mathcal{C}$ if and only if
$\mathcal{S}_2$ is fibred over $\mathcal{C}$.
\end{lemma}

\begin{proof}
Denote $p_i : \mathcal{S}_i \to \mathcal{C}$ the given functors.
Let $F : \mathcal{S}_1 \to \mathcal{S}_2$,
$G : \mathcal{S}_2 \to \mathcal{S}_1$ be functors over $\mathcal{C}$, and let
$i : F \circ G \to \text{id}_{\mathcal{S}_2}$,
$j : G \circ F \to \text{id}_{\mathcal{S}_1}$ be isomorphisms of
functors over $\mathcal{C}$.
We claim that in this case $F$ maps strongly cartesian morphisms
to strongly cartesian morphisms. Namely, suppose that
$\varphi : y \to x$ is strongly cartesian in $\mathcal{S}_1$.
Set $f : V \to U$ equal to $p_1(\varphi)$. Suppose that
$z' \in \Ob(\mathcal{S}_2)$, with $W = p_2(z')$, and we are given
$g : W \to V$ and $\psi' : z' \to F(x)$ such that
$p_2(\psi') = f \circ g$. Then
$$
\psi = j \circ G(\psi') : G(z') \to G(F(x)) \to x
$$
is a morphism in $\mathcal{S}_1$ with $p_1(\psi) = f \circ g$.
Hence by assumption there exists a unique morphism $\xi : G(z') \to y$
lying over $g$ such that $\psi = \varphi \circ \xi$. This in turn gives a
morphism
$$
\xi' = F(\xi) \circ i^{-1} : z' \to F(G(z')) \to F(y)
$$
lying over $g$ with $\psi' = F(\varphi) \circ \xi'$. We omit the verification
that $\xi'$ is unique.
\end{proof}

\noindent
The conclusion from Lemma \ref{lemma-fibred-equivalent} is that
equivalences map strongly cartesian morphisms to strongly cartesian
morphisms. But this may not be the case for an arbitrary functor between
fibred categories over $\mathcal{C}$. Hence we define the $2$-category
of fibred categories as follows.

\begin{definition}
\label{definition-fibred-categories-over-C}
Let $\mathcal{C}$ be a category.
The {\it $2$-category of fibred categories over $\mathcal{C}$}
is the sub $2$-category of the $2$-category of categories
over $\mathcal{C}$ (see Definition \ref{definition-categories-over-C})
defined as follows:
\begin{enumerate}
\item Its objects will be fibred categories
$p : \mathcal{S} \to \mathcal{C}$.
\item Its $1$-morphisms $(\mathcal{S}, p) \to (\mathcal{S}', p')$
will be functors $G : \mathcal{S} \to \mathcal{S}'$ such that
$p' \circ G = p$ and such that $G$ maps strongly cartesian
morphisms to strongly cartesian morphisms.
\item Its $2$-morphisms $t : G \to H$ for
$G, H : (\mathcal{S}, p) \to (\mathcal{S}', p')$
will be morphisms of functors
such that $p'(t_x) = \text{id}_{p(x)}$
for all $x \in \Ob(\mathcal{S})$.
\end{enumerate}
In this situation we will denote
$$
\Mor_{\textit{Fib}/\mathcal{C}}(\mathcal{S}, \mathcal{S}')
$$
the category of $1$-morphisms between
$(\mathcal{S}, p)$ and $(\mathcal{S}', p')$
\end{definition}

\noindent
Note the condition on $1$-morphisms.
Note also that this is a true $2$-category and
not a $(2, 1)$-category. Hence when taking $2$-fibre
products we first pass to the associated $(2, 1)$-category.

\begin{lemma}
\label{lemma-2-product-fibred-categories-over-C}
Let $\mathcal{C}$ be a category.
The $(2, 1)$-category of fibred categories
over $\mathcal{C}$ has 2-fibre products, and
they are described as in
Lemma \ref{lemma-2-product-categories-over-C}.
\end{lemma}

\begin{proof}
Basically what one has to show here is that given
$F : \mathcal{X} \to \mathcal{S}$ and
$G : \mathcal{Y} \to \mathcal{S}$ morphisms of fibred
categories over $\mathcal{C}$, then the category
$\mathcal{X} \times_\mathcal{S} \mathcal{Y}$
described in Lemma \ref{lemma-2-product-categories-over-C} is fibred.
Let us show that $\mathcal{X} \times_\mathcal{S} \mathcal{Y}$
has plenty of strongly cartesian morphisms.
Namely, suppose we have $(U, x, y, \phi)$ an object of
$\mathcal{X} \times_\mathcal{S} \mathcal{Y}$.
And suppose $f : V \to U$ is a morphism in $\mathcal{C}$.
Choose strongly cartesian morphisms $a : f^*x \to x$ in $\mathcal{X}$
lying over $f$ and $b : f^*y \to y$ in $\mathcal{Y}$ lying over $f$.
By assumption $F(a)$ and $G(b)$ are strongly cartesian.
Since $\phi : F(x) \to G(y)$ is an isomorphism, by the uniqueness
of strongly cartesian morphisms we find a unique isomorphism
$f^*\phi : F(f^*x) \to G(f^*y)$ such that
$G(b) \circ f^*\phi = \phi \circ F(a)$. In other words
$(a, b) : (V, f^*x, f^*y, f^*\phi) \to (U, x, y, \phi)$
is a morphism in $\mathcal{X} \times_\mathcal{S} \mathcal{Y}$.
We omit the verification that this is a strongly cartesian morphism
(and that these are in fact the only strongly cartesian morphisms).
\end{proof}

\begin{lemma}
\label{lemma-cute}
Let $\mathcal{C}$ be a category. Let $U \in \Ob(\mathcal{C})$.
If $p : \mathcal{S} \to \mathcal{C}$ is a fibred category
and $p$ factors through $p' : \mathcal{S} \to \mathcal{C}/U$
then $p' : \mathcal{S} \to \mathcal{C}/U$ is a fibred category.
\end{lemma}

\begin{proof}
Suppose that $\varphi : x' \to x$ is strongly cartesian with respect to $p$.
We claim that $\varphi$ is strongly cartesian with respect to $p'$ also.
Set $g = p'(\varphi)$, so that $g : V'/U \to V/U$
for some morphisms $f : V \to U$ and $f' : V' \to U$.
Let $z \in \Ob(\mathcal{S})$. Set $p'(z) = (W \to U)$.
To show that $\varphi$ is strongly cartesian for $p'$ we have to show
$$
\Mor_\mathcal{S}(z, x')
\longrightarrow
\Mor_\mathcal{S}(z, x)
\times_{\Mor_{\mathcal{C}/U}(W/U, V/U)}
\Mor_{\mathcal{C}/U}(W/U, V'/U),
$$
given by $\psi' \longmapsto (\varphi \circ \psi', p'(\psi'))$
is bijective. Suppose given an element $(\psi, h)$ of the
right hand side, then in particular $g \circ h = p(\psi)$,
and by the condition that $\varphi$ is strongly cartesian we
get a unique morphism $\psi' : z \to x'$ with $\psi = \varphi \circ \psi'$
and $p(\psi') = h$. OK, and now $p'(\psi') : W/U \to V/U$
is a morphism whose corresponding map $W \to V$ is $h$, hence
equal to $h$ as a morphism in $\mathcal{C}/U$. Thus $\psi'$ is
a unique morphism $z \to x'$ which maps to the given pair $(\psi, h)$.
This proves the claim.

\medskip\noindent
Finally, suppose given $g : V'/U \to V/U$ and $x$ with $p'(x) = V/U$.
Since $p : \mathcal{S} \to \mathcal{C}$ is a fibred category we
see there exists a strongly cartesian morphism $\varphi : x' \to x$
with $p(\varphi) = g$. By the same argument as above it follows
that $p'(\varphi) = g : V'/U \to V/U$. And as seen above the morphism
$\varphi$ is strongly cartesian. Thus the conditions of
Definition \ref{definition-fibred-category} are satisfied and we win.
\end{proof}

\begin{lemma}
\label{lemma-fibred-over-fibred}
Let $\mathcal{A} \to \mathcal{B} \to \mathcal{C}$ be functors between
categories. If $\mathcal{A}$ is fibred over $\mathcal{B}$ and
$\mathcal{B}$ is fibred over $\mathcal{C}$, then $\mathcal{A}$
is fibred over $\mathcal{C}$.
\end{lemma}

\begin{proof}
This follows from the definitions and
Lemma \ref{lemma-cartesian-over-cartesian}.
\end{proof}

\begin{lemma}
\label{lemma-fibred-category-representable-goes-up}
Let $p : \mathcal{S} \to \mathcal{C}$ be a fibred category.
Let $x \to y$ and $z \to y$ be morphisms of $\mathcal{S}$
with $x \to y$ strongly cartesian. If $p(x) \times_{p(y)} p(z)$ exists,
then $x \times_y z$ exists, $p(x \times_y z) = p(x) \times_{p(y)} p(z)$,
and $x \times_y z \to z$ is strongly cartesian.
\end{lemma}

\begin{proof}
Pick a strongly cartesian morphism
$\text{pr}_2^*z \to z$ lying over
$\text{pr}_2 : p(x) \times_{p(y)} p(z) \to p(z)$. Then
$\text{pr}_2^*z = x \times_y z$ by
Lemma \ref{lemma-strongly-cartesian-fibre-product}.
\end{proof}

\begin{lemma}
\label{lemma-ameliorate-morphism-fibred-categories}
Let $\mathcal{C}$ be a category. Let $F : \mathcal{X} \to \mathcal{Y}$
be a $1$-morphism of fibred categories over $\mathcal{C}$.
There exist $1$-morphisms of fibred categories over $\mathcal{C}$
$$
\xymatrix{
\mathcal{X} \ar@<1ex>[r]^u &
\mathcal{X}' \ar[r]^v \ar@<1ex>[l]^w & \mathcal{Y}
}
$$
such that $F = v \circ u$ and such that
\begin{enumerate}
\item $u : \mathcal{X} \to \mathcal{X}'$ is fully faithful,
\item $w$ is left adjoint to $u$, and
\item $v : \mathcal{X}' \to \mathcal{Y}$ is a fibred category.
\end{enumerate}
\end{lemma}

\begin{proof}
Denote $p : \mathcal{X} \to \mathcal{C}$ and $q : \mathcal{Y} \to \mathcal{C}$
the structure functors. We construct $\mathcal{X}'$ explicitly as follows.
An object of $\mathcal{X}'$ is a quadruple $(U, x, y, f)$ where
$x \in \Ob(\mathcal{X}_U)$, $y \in \Ob(\mathcal{Y}_U)$
and $f : y \to F(x)$ is a morphism in $\mathcal{Y}_U$.
A morphism $(a, b) : (U, x, y, f) \to (U', x', y', f')$ is given
by $a : x \to x'$ and $b : y \to y'$ with $p(a) = q(b) : U \to U'$
and such that $f' \circ b = F(a) \circ f$.

\medskip\noindent
Let us make a choice of pullbacks for both $p$ and $q$ and let us
use the same notation to indicate them.
Let $(U, x, y, f)$ be an object and let $h : V \to U$ be a morphism.
Consider the morphism $c : (V, h^*x, h^*y, h^*f) \to (U, x, y, f)$
coming from the given strongly cartesian maps $h^*x \to x$ and $h^*y \to y$.
We claim $c$ is strongly cartesian in $\mathcal{X}'$ over $\mathcal{C}$.
Namely, suppose we are given an object $(W, x', y', f')$ of $\mathcal{X}'$,
a morphism $(a, b) : (W, x', y', f') \to (U, x, y, f)$ lying over
$W \to U$, and a factorization $W \to V \to U$ of $W \to U$ through $h$.
As $h^*x \to x$ and $h^*y \to y$ are strongly cartesian we obtain morphisms
$a' : x' \to h^*x$ and $b' : y' \to h^*y$ lying over the given morphism
$W \to V$. Consider the diagram
$$
\xymatrix{
y' \ar[d]_{f'} \ar[r] & h^*y \ar[r] \ar[d]_{h^*f} & y \ar[d]_f \\
F(x') \ar[r] & F(h^*x) \ar[r] & F(x)
}
$$
The outer rectangle and the right square commute.
Since $F$ is a $1$-morphism of fibred categories the morphism
$F(h^*x) \to F(x)$ is strongly cartesian.
Hence the left square commutes by the universal property
of strongly cartesian morphisms. This proves that $\mathcal{X}'$
is fibred over $\mathcal{C}$.

\medskip\noindent
The functor $u : \mathcal{X} \to \mathcal{X}'$ is given by
$x \mapsto (p(x), x, F(x), \text{id})$. This is fully faithful.
The functor $\mathcal{X}' \to \mathcal{Y}$ is given by
$(U, x, y, f) \mapsto y$. The functor $w : \mathcal{X}' \to \mathcal{X}$
is given by $(U, x, y, f) \mapsto x$. Each of these functors is
a $1$-morphism of fibred categories over $\mathcal{C}$ by our
description of strongly cartesian morphisms of $\mathcal{X}'$ over
$\mathcal{C}$. Adjointness of $w$ and $u$ means that
$$
\Mor_\mathcal{X}(x, x') =
\Mor_{\mathcal{X}'}((U, x, y, f), (p(x'), x', F(x'), \text{id})),
$$
which follows immediately from the definitions.

\medskip\noindent
Finally, we have to show that $\mathcal{X}' \to \mathcal{Y}$ is a fibred
category. Let $c : y' \to y$ be a morphism in $\mathcal{Y}$
and let $(U, x, y, f)$ be an object of $\mathcal{X}'$ lying over $y$.
Set $V = q(y')$ and let $h = q(c) : V \to U$. Let $a : h^*x \to x$
and $b : h^*y \to y$ be the strongly cartesian morphisms covering $h$.
Since $F$ is a $1$-morphism of fibred categories we may identify
$h^*F(x) = F(h^*x)$ with strongly cartesian morphism
$F(a) : F(h^*x) \to F(x)$. By the universal property
of $b : h^*y \to y$ there is a morphism $c' : y' \to h^*y$ in
$\mathcal{Y}_V$ such that $c = b \circ c'$. We claim that
$$
(a, c) : (V, h^*x, y', h^*f \circ c') \longrightarrow (U, x, y, f)
$$
is strongly cartesian in $\mathcal{X}'$ over $\mathcal{Y}$. To see this
let $(W, x_1, y_1, f_1)$ be an object of $\mathcal{X}'$, let
$(a_1, b_1) : (W, x_1, y_1, f_1) \to (U, x, y, f)$ be a morphism
and let $b_1 = c \circ b_1'$ for some morphism $b_1' : y_1 \to y'$.
Then
$$
(a_1', b_1') : (W, x_1, y_1, f_1) \longrightarrow (V, h^*x, y', h^*f \circ c')
$$
(where $a_1' : x_1 \to h^*x$ is the unique morphism lying over the
given morphism $q(b_1') : W \to V$ such that $a_1 = a \circ a_1'$)
is the desired morphism.
\end{proof}





\section{Inertia}
\label{section-inertia}

\noindent
Given fibred categories $p : \mathcal{S} \to \mathcal{C}$ and
$p' : \mathcal{S}' \to \mathcal{C}$ over a category $\mathcal{C}$
and a $1$-morphism $F : \mathcal{S} \to \mathcal{S}'$
we have the diagonal morphism
$$
\Delta = \Delta_{\mathcal{S}/\mathcal{S}'} :
\mathcal{S} \longrightarrow \mathcal{S} \times_{\mathcal{S}'} \mathcal{S}
$$
in the $(2, 1)$-category of fibred categories over $\mathcal{C}$.

\begin{lemma}
\label{lemma-inertia-fibred-category}
Let $\mathcal{C}$ be a category. Let
$p : \mathcal{S} \to \mathcal{C}$ and
$p' : \mathcal{S}' \to \mathcal{C}$ be fibred categories.
Let $F : \mathcal{S} \to \mathcal{S}'$ be a $1$-morphism of
fibred categories over $\mathcal{C}$. Consider the category
$\mathcal{I}_{\mathcal{S}/\mathcal{S}'}$ over $\mathcal{C}$ whose
\begin{enumerate}
\item objects are pairs $(x, \alpha)$ where $x \in \Ob(\mathcal{S})$
and $\alpha : x \to x$ is an automorphism with $F(\alpha) = \text{id}$,
\item morphisms $(x, \alpha) \to (y, \beta)$ are given by morphisms
$\phi : x \to y$ such that
$$
\xymatrix{
x\ar[r]_\phi\ar[d]_\alpha &
y\ar[d]^{\beta} \\
x\ar[r]^\phi &
y \\
}
$$
commutes, and
\item the functor $\mathcal{I}_{\mathcal{S}/\mathcal{S}'} \to \mathcal{C}$
is given by $(x, \alpha) \mapsto p(x)$.
\end{enumerate}
Then
\begin{enumerate}
\item there is an equivalence
$$
\mathcal{I}_{\mathcal{S}/\mathcal{S}'} \longrightarrow
\mathcal{S}
\times_{\Delta, (\mathcal{S} \times_{\mathcal{S}'} \mathcal{S}), \Delta}
\mathcal{S}
$$
in the $(2, 1)$-category of categories over $\mathcal{C}$, and
\item $\mathcal{I}_{\mathcal{S}/\mathcal{S}'}$ is a fibred category over
$\mathcal{C}$.
\end{enumerate}
\end{lemma}

\begin{proof}
Note that (2) follows from (1) by
Lemmas \ref{lemma-2-product-fibred-categories-over-C} and
\ref{lemma-fibred-equivalent}. Thus it suffices to prove (1).
We will use without further mention the construction of the $2$-fibre product
from
Lemma \ref{lemma-2-product-fibred-categories-over-C}.
In particular an object of
$\mathcal{S}
\times_{\Delta, (\mathcal{S} \times_{\mathcal{S}'} \mathcal{S}), \Delta}
\mathcal{S}$
is a triple $(x, y, (\iota, \kappa))$ where $x$ and $y$ are objects of
$\mathcal{S}$, and
$(\iota, \kappa) : (x, x, \text{id}_{F(x)}) \to (y, y, \text{id}_{F(y)})$
is an isomorphism in $\mathcal{S} \times_{\mathcal{S}'} \mathcal{S}$.
This just means that $\iota, \kappa : x \to y$ are isomorphisms and that
$F(\iota) = F(\kappa)$. Consider the functor
$$
I_{\mathcal{S}/\mathcal{S}'}
\longrightarrow
\mathcal{S}
\times_{\Delta, (\mathcal{S} \times_{\mathcal{S}'} \mathcal{S}), \Delta}
\mathcal{S}
$$
which to an object $(x, \alpha)$ of the left hand side assigns the object
$(x, x, (\alpha, \text{id}_x))$ of the right hand side
and to a morphism $\phi$ of the left hand side
assigns the morphism $(\phi, \phi)$ of the right hand side.
We claim that a quasi-inverse to that morphism is given by the
functor
$$
\mathcal{S}
\times_{\Delta, (\mathcal{S} \times_{\mathcal{S}'} \mathcal{S}), \Delta}
\mathcal{S}
\longrightarrow
I_{\mathcal{S}/\mathcal{S}'}
$$
which to an object $(x, y, (\iota, \kappa))$ of the left hand side
assigns the object $(x, \kappa^{-1} \circ \iota)$ of the right hand side
and to a morphism
$(\phi, \phi') : (x, y, (\iota, \kappa)) \to (z, w, (\lambda, \mu))$
of the left hand side assigns the morphism $\phi$.
Indeed, the endo-functor of $I_{\mathcal{S}/\mathcal{S}'}$ induced
by composing the two functors above is the identity on the nose, and
the endo-functor induced on
$\mathcal{S}
\times_{\Delta, (\mathcal{S} \times_{\mathcal{S}'} \mathcal{S}), \Delta}
\mathcal{S}$
is isomorphic to
the identity via the natural isomorphism
$$
(\text{id}_x, \kappa) :
(x, x, (\kappa^{-1} \circ \iota, \text{id}_x))
\longrightarrow
(x, y, (\iota, \kappa)).
$$
Some details omitted.
\end{proof}

\begin{definition}
\label{definition-inertia-fibred-category}
Let $\mathcal{C}$ be a category.
\begin{enumerate}
\item Let $F : \mathcal{S} \to \mathcal{S}'$ be a $1$-morphism of
fibred categories over $\mathcal{C}$. The {\it relative inertia
of $\mathcal{S}$ over $\mathcal{S}'$} is the fibred category
$\mathcal{I}_{\mathcal{S}/\mathcal{S}'} \to \mathcal{C}$ of
Lemma \ref{lemma-inertia-fibred-category}.
\item By the {\it inertia fibred category $\mathcal{I}_\mathcal{S}$
of $\mathcal{S}$} we mean
$\mathcal{I}_\mathcal{S} = \mathcal{I}_{\mathcal{S}/\mathcal{C}}$.
\end{enumerate}
\end{definition}

\noindent
Note that there are canonical $1$-morphisms
\begin{equation}
\label{equation-inertia-structure-map}
\mathcal{I}_{\mathcal{S}/\mathcal{S}'} \longrightarrow \mathcal{S}
\quad\text{and}\quad
\mathcal{I}_\mathcal{S} \longrightarrow \mathcal{S}
\end{equation}
of fibred categories over $\mathcal{C}$. In terms of the description of
Lemma \ref{lemma-inertia-fibred-category}
these simply map the object $(x, \alpha)$ to the object $x$ and the morphism
$\phi : (x, \alpha) \to (y, \beta)$ to the morphism $\phi : x \to y$.
There is also a {\it neutral section}
\begin{equation}
\label{equation-neutral-section}
e : \mathcal{S} \to \mathcal{I}_{\mathcal{S}/\mathcal{S}'}
\quad\text{and}\quad
e : \mathcal{S} \to \mathcal{I}_\mathcal{S}
\end{equation}
defined by the rules $x \mapsto (x, \text{id}_x)$ and
$(\phi : x \to y) \mapsto \phi$. This is a right inverse to
(\ref{equation-inertia-structure-map}). Given a $2$-commutative
square
$$
\xymatrix{
\mathcal{S}_1 \ar[d]_{F_1} \ar[r]_G & \mathcal{S}_2 \ar[d]^{F_2} \\
\mathcal{S}'_1 \ar[r]^{G'} & \mathcal{S}'_2
}
$$
there are {\it functoriality maps}
\begin{equation}
\label{equation-functorial}
\mathcal{I}_{\mathcal{S}_1/\mathcal{S}'_1}
\longrightarrow
\mathcal{I}_{\mathcal{S}_2/\mathcal{S}'_2}
\quad\text{and}\quad
\mathcal{I}_{\mathcal{S}_1}
\longrightarrow
\mathcal{I}_{\mathcal{S}_2}
\end{equation}
defined by the rules $(x, \alpha) \mapsto (G(x), G(\alpha))$
and $\phi \mapsto G(\phi)$. In particular there is always a
comparison map
\begin{equation}
\label{equation-comparison}
\mathcal{I}_{\mathcal{S}/\mathcal{S}'}
\longrightarrow
\mathcal{I}_\mathcal{S}
\end{equation}
and all the maps above are compatible with this.

\begin{lemma}
\label{lemma-relative-inertia-as-fibre-product}
Let $F : \mathcal{S} \to \mathcal{S}'$ be a $1$-morphism of categories
fibred over a category $\mathcal{C}$. Then the diagram
$$
\xymatrix{
\mathcal{I}_{\mathcal{S}/\mathcal{S}'}
\ar[d]_{F \circ (\ref{equation-inertia-structure-map})}
\ar[rr]_{(\ref{equation-comparison})} & &
\mathcal{I}_\mathcal{S} \ar[d]^{(\ref{equation-functorial})} \\
\mathcal{S}' \ar[rr]^e & &
\mathcal{I}_{\mathcal{S}'}
}
$$
is a $2$-fibre product.
\end{lemma}

\begin{proof}
Omitted.
\end{proof}





\section{Categories fibred in groupoids}
\label{section-fibred-groupoids}

\noindent
In this section we explain how to think about categories fibred in groupoids
and we see how they are basically the same as functors with
values in the $(2, 1)$-category of groupoids.

\begin{definition}
\label{definition-fibred-groupoids}
Let $p : \mathcal{S} \to \mathcal{C}$ be a functor.
We say that $\mathcal{S}$ is {\it fibred in groupoids} over $\mathcal{C}$ if
the following two conditions hold:
\begin{enumerate}
\item For every morphism $f : V \to U$ in $\mathcal{C}$ and every
lift $x$ of $U$ there is a lift $\phi : y \to x$ of $f$ with
target $x$.
\item For every pair of morphisms $\phi : y \to x$ and $ \psi : z \to x$
and any morphism $f : p(z) \to p(y)$ such that $p(\phi) \circ f = p(\psi)$
there exists a unique lift $\chi : z \to y$ of $f$ such that
$\phi \circ \chi = \psi$.
\end{enumerate}
\end{definition}

\noindent
Condition (2) phrased differently says that
applying the functor $p$ gives a bijection between the sets
of dotted arrows in the following commutative diagram below:
$$
\xymatrix{
y \ar[r] & x & p(y) \ar[r] & p(x) \\
z \ar@{-->}[u] \ar[ru] & & p(z) \ar@{-->}[u]\ar[ru] & \\
}
$$
Another way to think about the second condition is the following.
Suppose that $g : W \to V$ and $f : V \to U$ are morphisms in $\mathcal{C}$.
Let $x \in \Ob(\mathcal{S}_U)$. By the first condition we can lift
$f$ to $ \phi : y \to x$ and then we can lift $g$ to $\psi : z \to y$.
Instead of doing this two step process we can directly lift $g \circ f$ to
$\gamma : z' \to x$. This gives the solid arrows in the diagram
\begin{equation}
\label{equation-fibred-groupoids}
\vcenter{
\xymatrix{
z' \ar@{-->}[d]\ar[rrd]^\gamma & & \\
z \ar@{-->}[u] \ar[r]^\psi \ar@{~>}[d]^p &
y \ar[r]^\phi \ar@{~>}[d]^p &
x \ar@{~>}[d]^p
\\
W \ar[r]^g & V \ar[r]^f & U \\
}
}
\end{equation}
where the squiggly arrows represent not morphisms but the functor $p$.
Applying the second condition to the arrows $\phi \circ \psi$, $\gamma$
and $\text{id}_W$ we conclude that there is a unique morphism
$\chi : z \to z'$ in $\mathcal{S}_W$ such that
$\gamma \circ \chi = \phi \circ \psi$. Similarly there is a unique morphism
$z' \to z$. The uniqueness implies that the morphisms $z' \to z$ and
$z\to z'$ are mutually inverse, in other words isomorphisms.

\medskip\noindent
It should be clear from this discussion that a
category fibred in groupoids is very closely related
to a fibred category. Here is the result.

\begin{lemma}
\label{lemma-fibred-groupoids}
Let $p : \mathcal{S} \to \mathcal{C}$ be a functor.
The following are equivalent
\begin{enumerate}
\item $p : \mathcal{S} \to \mathcal{C}$ is a category
fibred in groupoids, and
\item all fibre categories are groupoids and
$\mathcal{S}$ is a fibred category over $\mathcal{C}$.
\end{enumerate}
Moreover, in this case every morphism of $\mathcal{S}$ is
strongly cartesian. In addition, given $f^\ast x \to x$
lying over $f$ for all $f: V \to U = p(x)$ the data
$(U \mapsto \mathcal{S}_U, f \mapsto f^*, \alpha_{f, g}, \alpha_U)$
constructed in Lemma \ref{lemma-fibred}
defines a pseudo functor from $\mathcal{C}^{opp}$ in to
the $(2, 1)$-category of groupoids.
\end{lemma}

\begin{proof}
Assume $p : \mathcal{S} \to \mathcal{C}$ is fibred in groupoids.
To show all fibre categories $\mathcal{S}_U$ for
$U \in \Ob(\mathcal{C})$
are groupoids, we must exhibit for every $f : y \to x$ in $\mathcal{S}_U$ an
inverse morphism.  The diagram on the left (in $\mathcal{S}_U$) is mapped by
$p$ to the diagram on the right:
$$
\xymatrix{
y \ar[r]^f & x & U \ar[r]^{\text{id}_U} & U \\
x \ar@{-->}[u] \ar[ru]_{\text{id}_x} & &
U \ar@{-->}[u]\ar[ru]_{\text{id}_U} & \\
}
$$
Since only $\text{i}d_U$ makes the diagram on the right commute, there is a
unique $g : x \to y$ making the diagram on the left commute, so
$fg = \text{id}_x$. By a similar argument there is a unique $h : y \to x$ so
that $gh = \text{id}_y$. Then $fgh = f : y \to x$.  We have $fg = \text{id}_x$,
so $h = f$. Condition (2) of Definition \ref{definition-fibred-groupoids} says
exactly that every morphism of $\mathcal{S}$ is strongly cartesian. Hence
condition (1) of Definition \ref{definition-fibred-groupoids} implies that
$\mathcal{S}$ is a fibred category over $\mathcal{C}$.

\medskip\noindent
Conversely, assume all fibre categories are groupoids and
$\mathcal{S}$ is a fibred category over $\mathcal{C}$.
We have to check conditions (1) and (2) of
Definition \ref{definition-fibred-groupoids}.
The first condition follows trivially. Let $\phi : y \to x$,
$\psi : z \to x$ and $f : p(z) \to p(y)$ such that
$p(\phi) \circ f = p(\psi)$ be as in condition (2) of
Definition \ref{definition-fibred-groupoids}.
Write $U = p(x)$, $V = p(y)$, $W = p(z)$, $p(\phi) = g : V \to U$,
$p(\psi) = h : W \to U$. Choose a strongly cartesian $g^*x \to x$
lying over $g$. Then we get a morphism $i : y \to g^*x$ in
$\mathcal{S}_V$, which is therefore an isomorphism. We
also get a morphism $j : z \to g^*x$ corresponding to
the pair $(\psi, f)$ as $g^*x \to x$ is strongly cartesian.
Then one checks that $\chi = i^{-1} \circ j$ is a solution.

\medskip\noindent
We have seen in the proof of (1) $\Rightarrow$ (2) that
every morphism of $\mathcal{S}$ is strongly cartesian.
The final statement follows directly from Lemma \ref{lemma-fibred}.
\end{proof}

\begin{lemma}
\label{lemma-fibred-gives-fibred-groupoids}
Let $\mathcal{C}$ be a category.
Let $p : \mathcal{S} \to \mathcal{C}$ be a fibred category.
Let $\mathcal{S}'$ be the subcategory of $\mathcal{S}$ defined
as follows
\begin{enumerate}
\item $\Ob(\mathcal{S}') = \Ob(\mathcal{S})$, and
\item for $x, y \in \Ob(\mathcal{S}')$ the set of morphisms between $x$
and $y$ in $\mathcal{S}'$ is the set of strongly cartesian morphisms between
$x$ and $y$ in $\mathcal{S}$.
\end{enumerate}
Let $p' : \mathcal{S}' \to \mathcal{C}$ be the restriction of $p$
to $\mathcal{S}'$. Then $p' : \mathcal{S}' \to \mathcal{C}$ is fibred
in groupoids.
\end{lemma}

\begin{proof}
Note that the construction makes sense since by
Lemma \ref{lemma-composition-cartesian}
the identity morphism of any object of $\mathcal{S}$ is strongly cartesian,
and the composition of strongly cartesian morphisms is strongly cartesian.
The first lifting property of
Definition \ref{definition-fibred-groupoids}
follows from the condition that in a fibred category
given any morphism $f : V \to U$ and $x$ lying over $U$ there exists
a strongly cartesian morphism $\varphi : y \to x$ lying over $f$.
Let us check the second lifting property of
Definition \ref{definition-fibred-groupoids}
for the category $p' : \mathcal{S}' \to \mathcal{C}$ over $\mathcal{C}$.
To do this we argue as in the discussion following
Definition \ref{definition-fibred-groupoids}.
Thus in Diagram \ref{equation-fibred-groupoids} the
morphisms $\phi$, $\psi$ and $\gamma$ are strongly cartesian morphisms
of $\mathcal{S}$.
Hence $\gamma$ and $\phi \circ \psi$ are strongly cartesian morphisms
of $\mathcal{S}$ lying over the same arrow of $\mathcal{C}$ and
having the same target in $\mathcal{S}$. By the discussion following
Definition \ref{definition-cartesian-over-C}
this means these two arrows are isomorphic as desired (here we use also
that any isomorphism in $\mathcal{S}$ is strongly cartesian, by
Lemma \ref{lemma-composition-cartesian} again).
\end{proof}

\begin{example}
\label{example-group-homomorphism-fibreedingroupoids}
A homomorphism of groups $p : G \to H$ gives rise to a functor
$p : \mathcal{S}\to\mathcal{C}$ as in Example
\ref{example-group-homomorphism-functor}. This functor
$p : \mathcal{S}\to\mathcal{C}$ is fibred in groupoids if and only if
$p$ is surjective.  The fibre category $\mathcal{S}_U$ over the (unique)
object $U\in \Ob(\mathcal{C})$ is the category associated to the
kernel of $p$ as in Example \ref{example-group-groupoid}.
\end{example}

\noindent
Given $p : \mathcal{S} \to \mathcal{C}$, we can ask: if the fibre
category $\mathcal{S}_U$ is a groupoid for all $U \in \Ob(\mathcal{C})$,
must $\mathcal{S}$ be fibred in groupoids over $\mathcal{C}$? We can see the
answer is no as follows. Start with a category fibred in groupoids
$p : \mathcal{S} \to \mathcal{C}$. Altering the morphisms in $\mathcal{S}$
which do not map to the identity morphism on some object does not alter the
categories $\mathcal{S}_U$. Hence we can violate the existence and uniqueness
conditions on lifts. One example is the functor from Example
\ref{example-group-homomorphism-fibreedingroupoids} when $G \to H$ is not
surjective. Here is another example.

\begin{example}
\label{example-not-fibred-in-groupoids-but-fibre-cats-are}
Let $\Ob(\mathcal{C}) = \{A, B, T\}$ and
$\Mor_\mathcal{C}(A, B) = \{f\}$, $\Mor_\mathcal{C}(B, T) = \{g\}$,
$\Mor_\mathcal{C}(A, T) = \{h\} = \{gf\}, $ plus the identity morphism
for each object. See the diagram below for a picture of this category. Now let
$\Ob(\mathcal{S}) = \{A', B', T'\}$ and
$\Mor_\mathcal{S}(A', B') = \emptyset$,
$\Mor_\mathcal{S}(B', T') = \{g'\}$,
$\Mor_\mathcal{S}(A', T') = \{h'\}, $ plus the identity morphisms. The
functor $p : \mathcal{S} \to \mathcal{C}$ is obvious. Then for every
$U \in \Ob(\mathcal{C})$, $\mathcal{S}_U$ is the category with one
object and the identity morphism on that object, so a groupoid, but the
morphism $f: A \to B$ cannot be lifted. Similarly, if we declare
$\Mor_\mathcal{S}(A', B') = \{f'_1, f'_2\}$ and
$ \Mor_\mathcal{S}(A', T') = \{h'\} = \{g'f'_1 \} = \{g'f'_2\}$, then
the fibre categories are the same and $f: A \to B$ in the diagram below has
two lifts.
$$
\xymatrix{
B' \ar[r]^{g'} & T' &  & B \ar[r]^g & T & \\
A' \ar@{-->}[u]^{??} \ar[ru]_{h'} & & \ar@{}[u]^{above} &
A \ar[u]^f \ar[ru]_{gf = h} & \\
}
$$
\end{example}

\noindent
Later we would like to make assertions such as ``any category fibred in
groupoids over $\mathcal{C}$ is equivalent to a split one'', or
``any category fibred in groupoids whose fibre categories are setlike
is equivalent to a category fibred in sets''. The notion of equivalence
depends on the $2$-category we are working with.

\begin{definition}
\label{definition-categories-fibred-in-groupoids-over-C}
Let $\mathcal{C}$ be a category.
The {\it $2$-category of categories fibred in groupoids over $\mathcal{C}$}
is the sub $2$-category of the $2$-category of fibred categories
over $\mathcal{C}$ (see Definition \ref{definition-fibred-categories-over-C})
defined as follows:
\begin{enumerate}
\item Its objects will be categories
$p : \mathcal{S} \to \mathcal{C}$ fibred in groupoids.
\item Its $1$-morphisms $(\mathcal{S}, p) \to (\mathcal{S}', p')$
will be functors $G : \mathcal{S} \to \mathcal{S}'$ such that
$p' \circ G = p$ (since every morphism is strongly cartesian
$G$ automatically preserves them).
\item Its $2$-morphisms $t : G \to H$ for
$G, H : (\mathcal{S}, p) \to (\mathcal{S}', p')$
will be morphisms of functors
such that $p'(t_x) = \text{id}_{p(x)}$
for all $x \in \Ob(\mathcal{S})$.
\end{enumerate}
\end{definition}

\noindent
Note that every $2$-morphism is automatically an isomorphism!
Hence this is actually a $(2, 1)$-category and not just a
$2$-category. Here is the obligatory lemma on $2$-fibre products.

\begin{lemma}
\label{lemma-2-product-fibred-categories}
Let $\mathcal{C}$ be a category.
The $2$-category of categories fibred in groupoids
over $\mathcal{C}$ has 2-fibre products, and they are described as in
Lemma \ref{lemma-2-product-categories-over-C}.
\end{lemma}

\begin{proof}
By Lemma \ref{lemma-2-product-fibred-categories-over-C}
the fibre product as described in
Lemma \ref{lemma-2-product-categories-over-C} is a fibred category.
Hence it suffices to prove that the fibre categories are
groupoids, see Lemma \ref{lemma-fibred-groupoids}.
By Lemma \ref{lemma-fibre-2-fibre-product-categories-over-C}
it is enough to show that the $2$-fibre product of groupoids
is a groupoid, which is clear (from the construction in
Lemma \ref{lemma-2-fibre-product-categories} for example).
\end{proof}

\begin{lemma}
\label{lemma-equivalence-fibred-categories}
Let $p : \mathcal{S}\to \mathcal{C}$ and
$p' : \mathcal{S'}\to \mathcal{C}$ be categories fibred in groupoids, and
suppose that $G : \mathcal{S}\to \mathcal {S}'$ is a functor over
$\mathcal{C}$.
\begin{enumerate}
\item Then $G$ is faithful (resp.\ fully faithful, resp.\ an equivalence)
if and only if for each $U\in\Ob(\mathcal{C})$ the induced functor
$G_U : \mathcal{S}_U\to \mathcal{S}'_U$ is faithful
(resp.\ fully faithful, resp.\ an equivalence).
\item If $G$ is an equivalence, then $G$ is an equivalence in the
$2$-category of categories fibred in groupoids over $\mathcal{C}$.
\end{enumerate}
\end{lemma}

\begin{proof}
Let $x, y$ be objects of $\mathcal{S}$ lying over the same object $U$.
Consider the commutative diagram
$$
\xymatrix{
\Mor_\mathcal{S}(x, y) \ar[rd]_p \ar[rr]_G & &
\Mor_{\mathcal{S}'}(G(x), G(y)) \ar[ld]^{p'} \\
& \Mor_\mathcal{C}(U, U) &
}
$$
From this diagram it is clear that if $G$ is faithful (resp.\ fully faithful)
then so is each $G_U$.

\medskip\noindent
Suppose $G$ is an equivalence. For every object
$x'$ of $\mathcal{S}'$ there exists an object $x$ of $\mathcal{S}$
such that $G(x)$ is isomorphic to $x'$. Suppose that $x'$ lies
over $U'$ and $x$ lies over $U$. Then there is an isomorphism
$f : U' \to U$ in $\mathcal{C}$, namely, $p'$ applied to the
isomorphism $x' \to G(x)$. By the axioms of a category fibred
in groupoids there exists an arrow $f^*x \to x$ of $\mathcal{S}$
lying over $f$. Hence there exists an isomorphism
$\alpha : x' \to G(f^*x)$ such that $p'(\alpha) = \text{id}_{U'}$
(this time by the axioms for $\mathcal{S}'$). All in all we conclude
that for every object $x'$ of $\mathcal{S}'$ we can choose
a pair $(o_{x'}, \alpha_{x'})$ consisting of an object
$o_{x'}$ of $\mathcal{S}$ and an isomorphism $\alpha_{x'} : x' \to G(o_{x'})$
with $p'(\alpha_{x'}) = \text{id}_{p'(x')}$.
From this point on we proceed as usual (see proof of
Lemma \ref{lemma-equivalence-categories}) to produce an inverse
functor $F : \mathcal{S}' \to \mathcal{S}$, by taking
$x' \mapsto o_{x'}$ and $\varphi' : x' \to y'$ to the unique
arrow $\varphi_{\varphi'} : o_{x'} \to o_{y'}$ with
$\alpha_{y'}^{-1} \circ G(\varphi_{\varphi'}) \circ \alpha_{x'} = \varphi'$.
With these choices $F$ is a functor over $\mathcal{C}$.
We omit the verification that $G \circ F$ and $F \circ G$ are
$2$-isomorphic to the respective identity functors
(in the $2$-category of categories fibred in groupoids over $\mathcal{C}$).

\medskip\noindent
Suppose that $G_U$ is faithful (resp.\ fully faithful)
for all $U\in\Ob(\mathcal C)$. To
show that $G$ is faithful (resp.\ fully faithful)
we have to show for any objects
$x, y\in\Ob(\mathcal{S})$ that $G$ induces an
injection (resp.\ bijection) between
$\Mor_\mathcal{S}(x, y)$ and
$\Mor_{\mathcal{S}'}(G(x), G(y))$.
Set $U = p(x)$ and $V = p(y)$.
It suffices to prove that $G$
induces an injection (resp.\ bijection) between morphism
$x \to y$ lying over $f$ to morphisms $G(x) \to G(y)$ lying over $f$
for any morphism $f : U \to V$.
Now fix $f : U \to V$. Denote $f^*y \to y$ a pullback.
Then also $G(f^*y) \to G(y)$ is a pullback.
The set of morphisms from $x$ to $y$ lying over $f$
is bijective to the set of morphisms between
$x$ and $f^*y$ lying over $\text{id}_U$. (By the second axiom
of a category fibred in groupoids.) Similarly
the set of morphisms from $G(x)$ to $G(y)$ lying over $f$
is bijective to the set of morphisms between
$G(x)$ and $G(f^*y)$ lying over $\text{id}_U$.
Hence the fact that $G_U$ is faithful (resp.\ fully faithful)
gives the desired result.

\medskip\noindent
Finally suppose for all $G_U$ is an equivalence for all $U$, so it is
fully faithful and essentially surjective.  We have seen this implies $G$ is
fully faithful, and thus to prove it is an equivalence we have to prove that
it is essentially surjective.  This is clear, for if $z'\in
\Ob(\mathcal{S}')$ then $z'\in \Ob(\mathcal{S}'_U)$ where
$U = p'(z')$.  Since $G_U$ is essentially surjective we know that
$z'$ is isomorphic, in $\mathcal{S}'_U$, to an object of the form
$G_U(z)$ for some $z\in \Ob(\mathcal{S}_U)$.  But morphisms
in $\mathcal{S}'_U$ are morphisms in $\mathcal{S}'$ and hence $z'$ is
isomorphic to $G(z)$ in $\mathcal{S}'$.
\end{proof}

\begin{lemma}
\label{lemma-fully-faithful-diagonal-equivalence}
Let $\mathcal{C}$ be a category. Let $p : \mathcal{S}\to \mathcal{C}$ and
$p' : \mathcal{S'}\to \mathcal{C}$ be categories fibred in groupoids.
Let $G : \mathcal{S}\to \mathcal {S}'$ be a functor over $\mathcal{C}$.
Then $G$ is fully faithful if and only if the diagonal
$$
\Delta_G :
\mathcal{S}
\longrightarrow
\mathcal{S} \times_{G, \mathcal{S}', G} \mathcal{S}
$$
is an equivalence.
\end{lemma}

\begin{proof}
By
Lemma \ref{lemma-equivalence-fibred-categories}
it suffices to look at fibre categories over an object $U$ of $\mathcal{C}$.
An object of the right hand side is a triple $(x, x', \alpha)$ where
$\alpha : G(x) \to G(x')$ is a morphism in $\mathcal{S}'_U$.
The functor $\Delta_G$ maps the object $x$ of $\mathcal{S}_U$
to the triple $(x, x, \text{id}_{G(x)})$. Note that $(x, x', \alpha)$
is in the essential image of $\Delta_G$ if and only if $\alpha = G(\beta)$
for some morphism $\beta : x \to x'$ in $\mathcal{S}_U$ (details omitted).
Hence in order for $\Delta_G$ to be an equivalence, every $\alpha$ has to
be the image of a morphism $\beta : x \to x'$, and also every two
distinct morphisms $\beta, \beta' : x \to x'$ have to give distinct
morphisms $G(\beta), G(\beta')$. This proves the lemma.
\end{proof}

\begin{lemma}
\label{lemma-morphisms-equivalent-fibred-groupoids}
Let $\mathcal{C}$ be a category.
Let $\mathcal{S}_i$, $i = 1, 2, 3, 4$ be categories fibred in
groupoids over $\mathcal{C}$.
Suppose that $\varphi : \mathcal{S}_1 \to \mathcal{S}_2$ and
$\psi : \mathcal{S}_3 \to \mathcal{S}_4$ are equivalences
over $\mathcal{C}$. Then
$$
\Mor_{\textit{Cat}/\mathcal{C}}(\mathcal{S}_2, \mathcal{S}_3)
\longrightarrow
\Mor_{\textit{Cat}/\mathcal{C}}(\mathcal{S}_1, \mathcal{S}_4),
\quad \alpha \longmapsto \psi \circ \alpha \circ \varphi
$$
is an equivalence of categories.
\end{lemma}

\begin{proof}
This is a generality and holds in any $2$-category.
\end{proof}

\begin{lemma}
\label{lemma-inertia-fibred-groupoids}
Let $\mathcal{C}$ be a category.
If $p : \mathcal{S} \to \mathcal{C}$ is fibred in groupoids, then
so is the inertia fibred category $\mathcal{I}_\mathcal{S} \to \mathcal{C}$.
\end{lemma}

\begin{proof}
Clear from the construction in
Lemma \ref{lemma-inertia-fibred-category}
or by using (from the same lemma) that
$I_\mathcal{S} \to \mathcal{S}
\times_{\Delta, \mathcal{S} \times_\mathcal{C} \mathcal{S}, \Delta}\mathcal{S}$
is an equivalence and appealing to
Lemma \ref{lemma-2-product-fibred-categories}.
\end{proof}

\begin{lemma}
\label{lemma-cute-groupoids}
Let $\mathcal{C}$ be a category. Let $U \in \Ob(\mathcal{C})$.
If $p : \mathcal{S} \to \mathcal{C}$ is a category fibred in groupoids
and $p$ factors through $p' : \mathcal{S} \to \mathcal{C}/U$
then $p' : \mathcal{S} \to \mathcal{C}/U$ is fibred in groupoids.
\end{lemma}

\begin{proof}
We have already seen in Lemma \ref{lemma-cute} that $p'$ is a fibred
category. Hence it suffices to prove the fibre categories are groupoids,
see Lemma \ref{lemma-fibred-groupoids}.
For $V \in \Ob(\mathcal{C})$ we have
$$
\mathcal{S}_V = \coprod\nolimits_{f : V \to U} \mathcal{S}_{(f : V \to U)}
$$
where the left hand side is the fibre category of $p$ and the right hand side
is the disjoint union of the fibre categories of $p'$.
Hence the result.
\end{proof}

\begin{lemma}
\label{lemma-fibred-in-groupoids-over-fibred-in-groupoids}
Let $\mathcal{A} \to \mathcal{B} \to \mathcal{C}$ be functors between
categories. If $\mathcal{A}$ is fibred in groupoids over $\mathcal{B}$
and $\mathcal{B}$ is fibred in groupoids over $\mathcal{C}$, then
$\mathcal{A}$ is fibred in groupoids over $\mathcal{C}$.
\end{lemma}

\begin{proof}
One can prove this directly from the definition. However, we will argue
using the criterion of Lemma \ref{lemma-fibred-groupoids}.
By Lemma \ref{lemma-fibred-over-fibred} we see that $\mathcal{A}$
is fibred over $\mathcal{C}$. To finish the proof we show that the fibre
category $\mathcal{A}_U$ is a groupoid for $U$ in $\mathcal{C}$.
Namely, if $x \to y$ is a morphism of $\mathcal{A}_U$, then its
image in $\mathcal{B}$ is an isomorphism as $\mathcal{B}_U$ is
a groupoid. But then $x \to y$ is an isomorphism, for example by
Lemma \ref{lemma-composition-cartesian} and the fact that every
morphism of $\mathcal{A}$ is strongly $\mathcal{B}$-cartesian
(see Lemma \ref{lemma-fibred-groupoids}).
\end{proof}

\begin{lemma}
\label{lemma-fibred-groupoids-fibre-product-goes-up}
Let $p : \mathcal{S} \to \mathcal{C}$ be a category fibred in groupoids.
Let $x \to y$ and $z \to y$ be morphisms of $\mathcal{S}$.
If $p(x) \times_{p(y)} p(z)$ exists, then
$x \times_y z$ exists and $p(x \times_y z) = p(x) \times_{p(y)} p(z)$.
\end{lemma}

\begin{proof}
Follows from
Lemma \ref{lemma-fibred-category-representable-goes-up}.
\end{proof}

\begin{lemma}
\label{lemma-ameliorate-morphism-categories-fibred-groupoids}
Let $\mathcal{C}$ be a category. Let $F : \mathcal{X} \to \mathcal{Y}$
be a $1$-morphism of categories fibred in groupoids over $\mathcal{C}$.
There exists a factorization $\mathcal{X} \to \mathcal{X}' \to \mathcal{Y}$
by $1$-morphisms of categories fibred in groupoids over $\mathcal{C}$ such
that $\mathcal{X} \to \mathcal{X}'$ is an equivalence over $\mathcal{C}$
and such that $\mathcal{X}'$ is a category fibred in groupoids over
$\mathcal{Y}$.
\end{lemma}

\begin{proof}
Denote $p : \mathcal{X} \to \mathcal{C}$ and $q : \mathcal{Y} \to \mathcal{C}$
the structure functors. We construct $\mathcal{X}'$ explicitly as follows.
An object of $\mathcal{X}'$ is a quadruple $(U, x, y, f)$ where
$x \in \Ob(\mathcal{X}_U)$, $y \in \Ob(\mathcal{Y}_U)$
and $f : F(x) \to y$ is an isomorphism in $\mathcal{Y}_U$.
A morphism $(a, b) : (U, x, y, f) \to (U', x', y', f')$ is given
by $a : x \to x'$ and $b : y \to y'$ with $p(a) = q(b)$ and
such that $f' \circ F(a) = b \circ f$. In other words
$\mathcal{X}' = \mathcal{X} \times_{F, \mathcal{Y}, \text{id}} \mathcal{Y}$
with the construction of the $2$-fibre product from
Lemma \ref{lemma-2-product-categories-over-C}.
By
Lemma \ref{lemma-2-product-fibred-categories}
we see that $\mathcal{X}'$ is a category fibred in groupoids over
$\mathcal{C}$ and that $\mathcal{X}' \to \mathcal{Y}$ is a morphism of
categories over $\mathcal{C}$. As functor $\mathcal{X} \to \mathcal{X}'$ we take
$x \mapsto (p(x), x, F(x), \text{id}_{F(x)})$ on objects and
$(a : x \to x') \mapsto (a, F(a))$ on morphisms. It is clear that
the composition $\mathcal{X} \to \mathcal{X}' \to \mathcal{Y}$
equals $F$. We omit the verification that
$\mathcal{X} \to \mathcal{X}'$ is an equivalence of fibred categories over
$\mathcal{C}$.

\medskip\noindent
Finally, we have to show that $\mathcal{X}' \to \mathcal{Y}$ is a category
fibred in groupoids. Let $b : y' \to y$ be a morphism in $\mathcal{Y}$
and let $(U, x, y, f)$ be an object of $\mathcal{X}'$ lying over $y$.
Because $\mathcal{X}$ is fibred in groupoids over $\mathcal{C}$ we
can find a morphism $a : x' \to x$ lying over $U' = q(y') \to q(y) = U$.
Since $\mathcal{Y}$ is fibred in groupoids over $\mathcal{C}$ and since
both $F(x') \to F(x)$ and $y' \to y$ lie over the same morphism $U' \to U$
we can find $f' : F(x') \to y'$ lying over $\text{id}_{U'}$ such that
$f \circ F(a) = b \circ f'$. Hence we obtain
$(a, b) : (U', x', y', f') \to (U, x, y, f)$.
This verifies the first condition (1) of
Definition \ref{definition-fibred-groupoids}.
To see (2) let
$(a, b) : (U', x', y', f') \to (U, x, y, f)$ and
$(a', b') : (U'', x'', y'', f'') \to (U, x, y, f)$ be morphisms of
$\mathcal{X}'$ and let $b'' : y' \to y''$ be a morphism of $\mathcal{Y}$
such that $b' \circ b'' = b$. We have to show that there exists
a unique morphism $a'' : x' \to x''$ such that
$f'' \circ F(a'') = b'' \circ f'$ and such that
$(a', b') \circ (a'', b'') = (a, b)$. Because $\mathcal{X}$ is fibred
in groupoids we know there exists a unique morphism
$a'' : x' \to x''$ such that $a' \circ a'' = a$ and $p(a'') = q(b'')$.
Because $\mathcal{Y}$ is fibred in groupoids we see that
$F(a'')$ is the unique morphism $F(x') \to F(x'')$ such that
$F(a') \circ F(a'') = F(a)$ and $q(F(a'')) = q(b'')$. The relation
$f'' \circ F(a'') = b'' \circ f'$ follows from this and the given
relations $f \circ F(a) = b \circ f'$ and $f \circ F(a') = b' \circ f''$.
\end{proof}

\begin{lemma}
\label{lemma-amelioration-unique}
Let $\mathcal{C}$ be a category. Let $F : \mathcal{X} \to \mathcal{Y}$
be a $1$-morphism of categories fibred in groupoids over $\mathcal{C}$.
Assume we have a $2$-commutative diagram
$$
\xymatrix{
\mathcal{X}' \ar[rd]_f &
\mathcal{X} \ar[l]^a \ar[d]^F \ar[r]_b &
\mathcal{X}'' \ar[ld]^g \\
& \mathcal{Y}
}
$$
where $a$ and $b$ are equivalences of categories over $\mathcal{C}$
and $f$ and $g$ are categories fibred in groupoids. Then there exists
an equivalence $h : \mathcal{X}'' \to \mathcal{X}'$ of categories over
$\mathcal{Y}$ such that $h \circ b$ is $2$-isomorphic to $a$ as $1$-morphisms
of categories over $\mathcal{C}$. If the diagram above actually commutes, then
we can arrange it so that $h \circ b$ is $2$-isomorphic to $a$ as
$1$-morphisms of categories over $\mathcal{Y}$.
\end{lemma}

\begin{proof}
We will show that both $\mathcal{X}'$ and $\mathcal{X}''$ over $\mathcal{Y}$
are equivalent to the category fibred in groupoids
$\mathcal{X} \times_{F, \mathcal{Y}, \text{id}} \mathcal{Y}$
over $\mathcal{Y}$, see proof of
Lemma \ref{lemma-ameliorate-morphism-categories-fibred-groupoids}.
Choose a quasi-inverse $b^{-1} : \mathcal{X}'' \to \mathcal{X}$ in the
$2$-category of categories over $\mathcal{C}$.
Since the right triangle of the diagram is $2$-commutative we see that
$$
\xymatrix{
\mathcal{X} \ar[d]_F & \mathcal{X}'' \ar[l]^{b^{-1}} \ar[d]^g \\
\mathcal{Y} & \mathcal{Y} \ar[l]
}
$$
is $2$-commutative. Hence we obtain a $1$-morphism
$c : \mathcal{X}'' \to
\mathcal{X} \times_{F, \mathcal{Y}, \text{id}} \mathcal{Y}$
by the universal property of the $2$-fibre product. Moreover $c$
is a morphism of categories over $\mathcal{Y}$ (!) and an equivalence
(by the assumption that $b$ is an equivalence, see
Lemma \ref{lemma-equivalence-2-fibre-product}).
Hence $c$ is an equivalence in the $2$-category of categories fibred
in groupoids over $\mathcal{Y}$ by
Lemma \ref{lemma-equivalence-fibred-categories}.

\medskip\noindent
We still have to construct a $2$-isomorphism between $c \circ b$ and
the functor $d : \mathcal{X} \to
\mathcal{X} \times_{F, \mathcal{Y}, \text{id}} \mathcal{Y}$,
$x \mapsto (p(x), x, F(x), \text{id}_{F(x)})$
constructed in the proof of
Lemma \ref{lemma-ameliorate-morphism-categories-fibred-groupoids}.
Let $\alpha : F \to g \circ b$ and $\beta : b^{-1} \circ b \to \text{id}$
be $2$-isomorphisms between $1$-morphisms of categories over $\mathcal{C}$.
Note that $c \circ b$ is given by the rule
$$
x \mapsto (p(x), b^{-1}(b(x)), g(b(x)), \alpha_x \circ F(\beta_x))
$$
on objects. Then we see that
$$
(\beta_x, \alpha_x) :
(p(x), x, F(x), \text{id}_{F(x)})
\longrightarrow
(p(x), b^{-1}(b(x)), g(b(x)), \alpha_x \circ F(\beta_x))
$$
is a functorial isomorphism which gives our $2$-morphism
$d \to b \circ c$. Finally, if the diagram commutes then
$\alpha_x$ is the identity for all $x$ and we see that this
$2$-morphism is a $2$-morphism in the $2$-category of categories
over $\mathcal{Y}$.
\end{proof}

































\section{Presheaves of categories}
\label{section-presheaves-categories}

\noindent
In this section we compare the notion of fibred categories
with the closely related notion of a ``presheaf of categories''.
The basic construction is explained in the following example.

\begin{example}
\label{example-functor-categories}
Let $\mathcal{C}$ be a category.
Suppose that $F : \mathcal{C}^{opp} \to \textit{Cat}$ is a functor
to the $2$-category of categories, see
Definition \ref{definition-functor-into-2-category}.
For $f : V \to U$ in $\mathcal{C}$ we will
suggestively write $F(f) = f^\ast$ for the functor from $F(U)$ to $F(V)$.
From this we can construct a fibred category $\mathcal{S}_F$ over
$\mathcal{C}$ as follows. Define
$$
\Ob(\mathcal{S}_F) =
\{(U, x) \mid U\in \Ob(\mathcal{C}), x\in \Ob(F(U))\}.
$$
For $(U, x), (V, y) \in \Ob(\mathcal{S}_F)$ we define
\begin{align*}
\Mor_{\mathcal{S}_F}((V, y), (U, x)) & =
\{ (f, \phi) \mid f \in \Mor_\mathcal{C}(V, U),
\phi \in \Mor_{F(V)}(y, f^\ast x)\} \\
& =
\coprod\nolimits_{f \in \Mor_\mathcal{C}(V, U)}
\Mor_{F(V)}(y, f^\ast x)
\end{align*}
In order to define composition we use that $g^\ast \circ f^\ast =
(f \circ g)^\ast$ for a pair of composable morphisms of $\mathcal{C}$
(by definition of a functor into a $2$-category).
Namely, we define the composition of $\psi : z \to g^\ast y$ and
$ \phi : y \to f^\ast x$ to be $ g^\ast(\phi) \circ \psi$. The functor
$p_F : \mathcal{S}_F \to \mathcal{C}$ is given by the rule
$(U, x) \mapsto U$.
Let us check that this is indeed a fibred category.
Given $f: V \to U$ in $\mathcal{C}$ and $(U, x)$ a lift of $U$, then
we claim $(f, \text{id}_{f^\ast x}): (V, {f^\ast x}) \to (U, x)$ is a
strongly cartesian lift of $f$.
We have to show a $h$ in the diagram on the left
determines $(h, \nu)$ on the right:
$$
\xymatrix{
V \ar[r]^f &
U &
(V, f^*x) \ar[r]^{(f, \text{id}_{f^*x})} &
(U, x) \\
W \ar@{-->}[u]^h \ar[ru]_g & &
(W, z) \ar@{-->}[u]^{(h, \nu)} \ar[ru]_{(g, \psi)} &
}
$$
Just take $\nu = \psi$ which works because $f \circ h = g$
and hence $g^*x = h^*f^*x$. Moreover, this is the only lift
making the diagram (on the right) commute.
\end{example}

\begin{definition}
\label{definition-split-fibred-category}
Let $\mathcal{C}$ be a category.
Suppose that $F : \mathcal{C}^{opp} \to \textit{Cat}$ is a functor
to the $2$-category of categories.
We will write $p_F : \mathcal{S}_F \to \mathcal{C}$ for the
fibred category constructed in
Example \ref{example-functor-categories}.
A {\it split fibred category} is a fibred category isomorphic (!)
over $\mathcal{C}$ to one of these categories {\it $\mathcal{S}_F$}.
\end{definition}

\begin{lemma}
\label{lemma-when-split}
Let $\mathcal{C}$ be a category.
Let $\mathcal{S}$ be a fibred category over $\mathcal{C}$.
Then $\mathcal{S}$ is split if and only if for some choice
of pullbacks (see Definition \ref{definition-pullback-functor-fibred-category})
the pullback functors
$(f \circ g)^*$ and $g^* \circ f^*$ are equal.
\end{lemma}

\begin{proof}
This is immediate from the definitions.
\end{proof}

\begin{lemma}
\label{lemma-fibred-strict}
Let $ p : \mathcal{S} \to \mathcal{C}$ be a fibred category.
There exists a contravariant functor $F : \mathcal{C} \to \textit{Cat}$
such that $\mathcal{S}$ is equivalent to $\mathcal{S}_F$
in the $2$-category of fibred categories over $\mathcal{C}$. In other
words, every fibred category is equivalent to a split one.
\end{lemma}

\begin{proof}
Let us make a choice of pullbacks (see
Definition \ref{definition-pullback-functor-fibred-category}).
By Lemma \ref{lemma-fibred} we get pullback functors $f^*$ for
every morphism $f$ of $\mathcal{C}$.

\medskip\noindent
We construct a new category $\mathcal{S}'$ as follows.
The objects of $\mathcal{S}'$ are pairs $(x, f)$
consisting of a morphism $f : V \to U$ of $\mathcal{C}$
and an object $x$ of $\mathcal{S}$ over $U$, i.e.,
$x\in \Ob(\mathcal{S}_U)$. The functor
$p' : \mathcal{S}' \to \mathcal{C}$ will map the pair $(x, f)$ to the source
of the morphism $f$, in other words $p'(x, f : V\to U) = V$. A morphism
$\varphi : (x_1, f_1: V_1 \to U_1) \to (x_2, f_2 : V_2 \to U_2)$ is given by a
pair $(\varphi, g)$ consisting of a morphism $g : V_1 \to V_2$ and a morphism
$\varphi : f_1^\ast x_1 \to f_2^\ast x_2$ with $p(\varphi) = g$. It is no
problem to define the composition law: $(\varphi, g) \circ (\psi, h) =
(\varphi \circ \psi, g\circ h)$ for any pair of composable morphisms.
There is a natural functor $\mathcal{S} \to \mathcal{S}'$ which simply maps
$x$ over $U$ to the pair $(x, \text{id}_U)$.

\medskip\noindent
At this point we need to check that $p'$ makes $\mathcal{S}'$ into a
fibred category over $\mathcal{C}$, and we need to check that
$\mathcal{S} \to \mathcal{S}'$ is an equivalence of categories over
$\mathcal{C}$ which maps strongly cartesian morphisms to strongly
cartesian morphisms. We omit the verifications.

\medskip\noindent
Finally, we can define pullback functors on $\mathcal{S}'$
by setting $g^\ast(x, f) = (x, f \circ g)$ on objects if
$g : V' \to V$ and $f : V \to U$. On morphisms
$(\varphi, \text{id}_V) : (x_1, f_1) \to (x_2, f_2)$
between morphisms in $\mathcal{S}'_V$ we set $g^\ast(\varphi, \text{id}_V) =
(g^\ast\varphi, \text{id}_{V'})$ where we use the unique identifications
$g^\ast f_i^\ast x_i = (f_i \circ g)^\ast x_i$ from Lemma
\ref{lemma-fibred} to think of $g^\ast\varphi$ as a morphism from
$(f_1 \circ g)^\ast x_1$ to $(f_2 \circ g)^\ast x_2$. Clearly, these pullback
functors $g^\ast$ have the property that
$g_1^\ast \circ g_2^\ast = (g_2\circ g_1)^\ast$, in other words $\mathcal{S}'$
is split as desired.
\end{proof}


















\section{Presheaves of groupoids}
\label{section-presheaves-groupoids}

\noindent
In this section we compare the notion of categories fibred in groupoids
with the closely related notion of a ``presheaf of groupoids''. The basic
construction is explained in the following example.

\begin{example}
\label{example-functor-groupoids}
This example is the analogue of
Example \ref{example-functor-categories},
for ``presheaves of groupoids'' instead of ``presheaves of categories''.
The output will be a category fibred in groupoids instead of a fibred category.
Suppose that $F : \mathcal{C}^{opp} \to \textit{Groupoids}$ is a functor
to the category of groupoids, see
Definition \ref{definition-functor-into-2-category}.
For $f : V \to U$ in $\mathcal{C}$ we will
suggestively write $F(f) = f^\ast$ for the functor from $F(U)$ to $F(V)$.
We construct a category $\mathcal{S}_F$ fibred in groupoids over $\mathcal{C}$
as follows. Define
$$
\Ob(\mathcal{S}_F) =
\{(U, x) \mid U\in \Ob(\mathcal{C}), x\in \Ob(F(U))\}.
$$
For $(U, x), (V, y) \in \Ob(\mathcal{S}_F)$ we define
\begin{align*}
\Mor_{\mathcal{S}_F}((V, y), (U, x))
& =
\{ (f, \phi) \mid f \in \Mor_\mathcal{C}(V, U),
\phi \in \Mor_{F(V)}(y, f^\ast x)\} \\
& =
\coprod\nolimits_{f \in \Mor_\mathcal{C}(V, U)}
\Mor_{F(V)}(y, f^\ast x)
\end{align*}
In order to define composition we use that $g^\ast \circ f^\ast =
(f \circ g)^\ast$ for a pair of composable morphisms of $\mathcal{C}$
(by definition of a functor into a $2$-category).
Namely, we define the composition of $\psi : z \to g^\ast y$ and
$ \phi : y \to f^\ast x$ to be $ g^\ast(\phi) \circ \psi$. The functor
$p_F : \mathcal{S}_F \to \mathcal{C}$ is given by the rule $(U, x) \mapsto U$.
The condition that $F(U)$ is a groupoid for every $U$ guarantees that
$\mathcal{S}_F$ is fibred in groupoids over $\mathcal{C}$, as we have
already seen in
Example \ref{example-functor-categories}
that $\mathcal{S}_F$ is a fibred category, see
Lemma \ref{lemma-fibred-groupoids}.
But we can also prove conditions (1), (2) of
Definition \ref{definition-fibred-groupoids}
directly as follows: (1) Lifts of
morphisms exist since given $f: V \to U$ in $\mathcal{C}$ and $(U, x)$
an object of $\mathcal{S}_F$ over $U$, then
$(f, \text{id}_{f^\ast x}): (V, {f^\ast x}) \to (U, x)$ is a lift of $f$.
(2) Suppose given solid diagrams as follows
$$
\xymatrix{
V \ar[r]^f & U & (V, y) \ar[r]^{(f, \phi)} & (U, x) \\
W \ar@{-->}[u]^h \ar[ru]_g & &
(W, z) \ar@{-->}[u]^{(h, \nu)} \ar[ru]_{(g, \psi)} & \\
}
$$
Then for the dotted arrows we have $\nu = (h^\ast \phi)^{-1} \circ \psi$
so given $h$ there exists a $\nu$ which is unique by uniqueness of inverses.
\end{example}

\begin{definition}
\label{definition-split-category-fibred-in-groupoids}
Let $\mathcal{C}$ be a category.
Suppose that $F : \mathcal{C}^{opp} \to \textit{Groupoids}$ is a functor
to the $2$-category of groupoids.
We will write $p_F : \mathcal{S}_F \to \mathcal{C}$ for the
category fibred in groupoids constructed in
Example \ref{example-functor-groupoids}.
A {\it split category fibred in groupoids} is a
category fibred in groupoids isomorphic (!)
over $\mathcal{C}$ to one of these categories {\it $\mathcal{S}_F$}.
\end{definition}

\begin{lemma}
\label{lemma-fibred-groupoids-strict}
Let $ p : \mathcal{S} \to \mathcal{C}$ be a category fibred in groupoids.
There exists a contravariant functor $F : \mathcal{C} \to \textit{Groupoids}$
such that $\mathcal{S}$ is equivalent to $\mathcal{S}_F$ over $\mathcal{C}$.
In other words, every category fibred in groupoids is equivalent to a split one.
\end{lemma}

\begin{proof}
Make a choice of pullbacks (see
Definition \ref{definition-pullback-functor-fibred-category}).
By Lemmas \ref{lemma-fibred} and \ref{lemma-fibred-groupoids}
we get pullback functors $f^*$ for
every morphism $f$ of $\mathcal{C}$.

\medskip\noindent
We construct a new category $\mathcal{S}'$ as follows.
The objects of $\mathcal{S}'$ are pairs $(x, f)$
consisting of a morphism $f : V \to U$ of $\mathcal{C}$
and an object $x$ of $\mathcal{S}$ over $U$, i.e.,
$x\in \Ob(\mathcal{S}_U)$. The functor
$p' : \mathcal{S}' \to \mathcal{C}$ will map the pair $(x, f)$ to the source
of the morphism $f$, in other words $p'(x, f : V\to U) = V$. A morphism
$\varphi : (x_1, f_1: V_1 \to U_1) \to (x_2, f_2 : V_2 \to U_2)$ is given by a
pair $(\varphi, g)$ consisting of a morphism $g : V_1 \to V_2$ and a morphism
$\varphi : f_1^\ast x_1 \to f_2^\ast x_2$ with $p(\varphi) = g$. It is no
problem to define the composition law: $(\varphi, g) \circ (\psi, h) =
(\varphi \circ \psi, g\circ h)$ for any pair of composable morphisms.
There is a natural functor $\mathcal{S} \to \mathcal{S}'$ which simply maps
$x$ over $U$ to the pair $(x, \text{id}_U)$.

\medskip\noindent
At this point we need to check that $p'$ makes $\mathcal{S}'$ into a category
fibred in groupoids over $\mathcal{C}$, and we need to check that
$\mathcal{S} \to \mathcal{S}'$ is an equivalence of categories over
$\mathcal{C}$. We omit the verifications.

\medskip\noindent
Finally, we can define pullback functors on $\mathcal{S}'$
by setting $g^\ast(x, f) = (x, f \circ g)$ on objects if
$g : V' \to V$ and $f : V \to U$. On morphisms
$(\varphi, \text{id}_V) : (x_1, f_1) \to (x_2, f_2)$
between morphisms in $\mathcal{S}'_V$ we set $g^\ast(\varphi, \text{id}_V) =
(g^\ast\varphi, \text{id}_{V'})$ where we use the unique identifications
$g^\ast f_i^\ast x_i = (f_i \circ g)^\ast x_i$ from Lemma
\ref{lemma-fibred-groupoids} to think of $g^\ast\varphi$ as a morphism from
$(f_1 \circ g)^\ast x_1$ to $(f_2 \circ g)^\ast x_2$. Clearly, these pullback
functors $g^\ast$ have the property that
$g_1^\ast \circ g_2^\ast = (g_2\circ g_1)^\ast$, in other words $\mathcal{S}'$
is split as desired.
\end{proof}

\noindent
We will see an alternative proof of this lemma in
Section \ref{section-representable-1-morphisms}.








\section{Categories fibred in sets}
\label{section-fibred-in-sets}

\begin{definition}
\label{definition-discrete}
A category is called {\it discrete} if the only morphisms are the identity
morphisms.
\end{definition}

\noindent
A discrete category has only one interesting piece of information:
its set of objects. Thus we sometime confuse discrete categories
with sets.

\begin{definition}
\label{definition-category-fibred-sets}
Let $\mathcal{C}$ be a category.
A {\it category fibred in sets}, or a {\it category fibred
in discrete categories} is a category fibred in groupoids all
of whose fibre categories are discrete.
\end{definition}

\noindent
We want to clarify the relationship between categories fibred in sets
and presheaves (see Definition \ref{definition-presheaf}).
To do this it makes sense to first make the following definition.

\begin{definition}
\label{definition-categories-fibred-in-sets-over-C}
Let $\mathcal{C}$ be a category.
The {\it $2$-category of categories fibred in sets over $\mathcal{C}$}
is the sub $2$-category of the category of categories fibred in groupoids
over $\mathcal{C}$ (see
Definition \ref{definition-categories-fibred-in-groupoids-over-C})
defined as follows:
\begin{enumerate}
\item Its objects will be categories
$p : \mathcal{S} \to \mathcal{C}$ fibred in sets.
\item Its $1$-morphisms $(\mathcal{S}, p) \to (\mathcal{S}', p')$
will be functors $G : \mathcal{S} \to \mathcal{S}'$ such that
$p' \circ G = p$ (since every morphism is strongly cartesian
$G$ automatically preserves them).
\item Its $2$-morphisms $t : G \to H$ for
$G, H : (\mathcal{S}, p) \to (\mathcal{S}', p')$
will be morphisms of functors
such that $p'(t_x) = \text{id}_{p(x)}$
for all $x \in \Ob(\mathcal{S})$.
\end{enumerate}
\end{definition}

\noindent
Note that every $2$-morphism is automatically an isomorphism.
Hence this $2$-category is actually a $(2, 1)$-category.
Here is the obligatory lemma on the existence of $2$-fibre products.

\begin{lemma}
\label{lemma-2-product-categories-fibred-sets}
Let $\mathcal{C}$ be a category.
The 2-category of categories fibred in sets over $\mathcal{C}$
has 2-fibre products. More precisely, the 2-fibre product described in
Lemma \ref{lemma-2-product-categories-over-C}
returns a category fibred in sets if one starts out with such.
\end{lemma}

\begin{proof}
Omitted.
\end{proof}

\begin{example}
\label{example-presheaf}
This example is the analogue of
Examples \ref{example-functor-categories} and
\ref{example-functor-groupoids}
for presheaves instead of ``presheaves of categories''.
The output will be a category fibred in sets instead of a fibred category.
Suppose that $F : \mathcal{C}^{opp} \to \textit{Sets}$ is a presheaf.
For $f : V \to U$ in $\mathcal{C}$ we will
suggestively write $F(f) = f^\ast : F(U) \to F(V)$.
We construct a category $\mathcal{S}_F$ fibred in sets over $\mathcal{C}$
as follows. Define
$$
\Ob(\mathcal{S}_F) =
\{(U, x) \mid U \in \Ob(\mathcal{C}), x \in \Ob(F(U))\}.
$$
For $(U, x), (V, y) \in \Ob(\mathcal{S}_F)$ we define
\begin{align*}
\Mor_{\mathcal{S}_F}((V, y), (U, x))
& =
\{f \in \Mor_\mathcal{C}(V, U) \mid f^*x = y\}
\end{align*}
Composition is inherited from composition in $\mathcal{C}$
which works as $g^\ast \circ f^\ast = (f \circ g)^\ast$
for a pair of composable morphisms of $\mathcal{C}$.
The functor $p_F : \mathcal{S}_F \to \mathcal{C}$
is given by the rule $(U, x) \mapsto U$.
As every fibre category $\mathcal{S}_{F, U}$ is discrete with underlying
set $F(U)$ and we have already see in
Example \ref{example-functor-groupoids}
that $\mathcal{S}_F$ is a category fibred in groupoids,
we conclude that $\mathcal{S}_F$ is fibred in sets.
\end{example}

\begin{lemma}
\label{lemma-2-category-fibred-sets}
\begin{slogan}
Categories fibred in sets are precisely presheaves.
\end{slogan}
Let $\mathcal{C}$ be a category.
The only $2$-morphisms between categories fibred in sets are identities.
In other words, the $2$-category of categories fibred in sets is a category.
Moreover, there is an equivalence of categories
$$
\left\{
\begin{matrix}
\text{the category of presheaves}\\
\text{of sets over }\mathcal{C}
\end{matrix}
\right\}
\leftrightarrow
\left\{
\begin{matrix}
\text{the category of categories}\\
\text{fibred in sets over }\mathcal{C}
\end{matrix}
\right\}
$$
The functor from left to right is the construction
$F \to \mathcal{S}_F$ discussed in
Example \ref{example-presheaf}.
The functor from right to left assigns to $p : \mathcal{S} \to \mathcal{C}$
the presheaf of objects $U \mapsto \Ob(\mathcal{S}_U)$.
\end{lemma}

\begin{proof}
The first assertion is clear, as the only morphisms in the fibre
categories are identities.

\medskip\noindent
Suppose that $p :
\mathcal{S} \to \mathcal{C}$ is fibred in sets. Let $f : V \to U$
be a morphism in $\mathcal{C}$ and let $x \in \Ob(\mathcal{S}_U)$.
Then there is exactly one choice for the object $f^\ast x$. Thus we see that
$(f \circ g)^\ast x = g^\ast(f^\ast x)$ for $f, g$ as in Lemma
\ref{lemma-fibred-groupoids}. It follows that we may think of the
assignments $U \mapsto \Ob(\mathcal{S}_U)$ and $f \mapsto f^\ast$
as a presheaf on $\mathcal{C}$.
\end{proof}

\noindent
Here is an important example of a category fibred in sets.

\begin{example}
\label{example-fibred-category-from-functor-of-points}
Let $\mathcal{C}$ be a category. Let $X \in \Ob(\mathcal{C})$.
Consider the representable presheaf $h_X = \Mor_\mathcal{C}(-, X)$
(see Example \ref{example-hom-functor}).
On the other hand, consider the category $p : \mathcal{C}/X \to \mathcal{C}$
from Example \ref{example-category-over-X}.
The fibre category $(\mathcal{C}/X)_U$ has as objects morphisms
$h : U \to X$, and only identities as morphisms. Hence we see that
under the correspondence of
Lemma \ref{lemma-2-category-fibred-sets}
we have
$$
h_X \longleftrightarrow \mathcal{C}/X.
$$
In other words, the category $\mathcal{C}/X$ is canonically equivalent
to the category $\mathcal{S}_{h_X}$ associated
to $h_X$ in
Example \ref{example-presheaf}.
\end{example}

\noindent
For this reason it is tempting to define a ``representable'' object in the
2-category of categories fibred in groupoids to be a category fibred in
sets whose associated presheaf is representable. However, this is would not
be a good definition for use since we prefer to have a notion which is
invariant under equivalences. To make this precise we study exactly
which categories fibred in groupoids are equivalent to categories
fibred in sets.









\section{Categories fibred in setoids}
\label{section-fibred-in-setoids}

\begin{definition}
\label{definition-setoid}
Let us call a category a {\it setoid}\footnote{A set on steroids!?}
if it is a groupoid where every object
has exactly one automorphism: the identity.
\end{definition}

\noindent
If $C$ is a set with an equivalence relation $\sim$, then we can make a setoid
$\mathcal{C}$ as follows: $\Ob(\mathcal{C}) = C$ and
$\Mor_\mathcal{C}(x, y) = \emptyset$ unless $x \sim y$ in which
case we set $\Mor_\mathcal{C}(x, y) = \{1\}$. Transitivity of
$\sim$ means that we can compose morphisms. Conversely any setoid
category defines an equivalence relation on its objects (isomorphism)
such that you recover the category (up to unique isomorphism -- not
equivalence) from the procedure just described.

\medskip\noindent
Discrete categories are setoids. For any setoid $\mathcal{C}$ there is a
canonical procedure to make a discrete category equivalent to it, namely
one replaces $\Ob(\mathcal{C})$ by the set of isomorphism
classes (and adds identity morphisms). In terms of sets endowed
with an equivalence relation this corresponds to taking the quotient
by the equivalence relation.

\begin{definition}
\label{definition-category-fibred-setoids}
Let $\mathcal{C}$ be a category. A {\it category fibred in setoids}
is a category fibred in groupoids all of whose fibre categories are
setoids.
\end{definition}

\noindent
Below we will clarify the relationship between categories fibred in setoids
and categories fibred in sets.

\begin{definition}
\label{definition-categories-fibred-in-setoids-over-C}
Let $\mathcal{C}$ be a category.
The {\it $2$-category of categories fibred in setoids over $\mathcal{C}$}
is the sub $2$-category of the category of categories fibred in groupoids
over $\mathcal{C}$ (see
Definition \ref{definition-categories-fibred-in-groupoids-over-C})
defined as follows:
\begin{enumerate}
\item Its objects will be categories
$p : \mathcal{S} \to \mathcal{C}$ fibred in setoids.
\item Its $1$-morphisms $(\mathcal{S}, p) \to (\mathcal{S}', p')$
will be functors $G : \mathcal{S} \to \mathcal{S}'$ such that
$p' \circ G = p$ (since every morphism is strongly cartesian
$G$ automatically preserves them).
\item Its $2$-morphisms $t : G \to H$ for
$G, H : (\mathcal{S}, p) \to (\mathcal{S}', p')$
will be morphisms of functors
such that $p'(t_x) = \text{id}_{p(x)}$
for all $x \in \Ob(\mathcal{S})$.
\end{enumerate}
\end{definition}

\noindent
Note that every $2$-morphism is automatically an isomorphism.
Hence this $2$-category is actually a $(2, 1)$-category.

\noindent
Here is the obligatory lemma on the existence of $2$-fibre products.

\begin{lemma}
\label{lemma-2-product-categories-fibred-setoids}
Let $\mathcal{C}$ be a category.
The 2-category of categories fibred in setoids over $\mathcal{C}$
has 2-fibre products. More precisely, the 2-fibre product described in
Lemma \ref{lemma-2-product-categories-over-C} returns a category fibred in
setoids if one starts out with such.
\end{lemma}

\begin{proof}
Omitted.
\end{proof}

\begin{lemma}
\label{lemma-setoid-fibres}
Let $\mathcal{C}$ be a category. Let $\mathcal{S}$ be a category
over $\mathcal{C}$.
\begin{enumerate}
\item If $\mathcal{S} \to \mathcal{S}'$ is an equivalence
over $\mathcal{C}$ with $\mathcal{S}'$ fibred in sets over $\mathcal{C}$,
then
\begin{enumerate}
\item $\mathcal{S}$ is fibred in setoids over $\mathcal{C}$, and
\item for each $U \in \Ob(\mathcal{C})$ the map
$\Ob(\mathcal{S}_U) \to \Ob(\mathcal{S}'_U)$
identifies the target as the set of isomorphism classes of the source.
\end{enumerate}
\item If $p : \mathcal{S} \to \mathcal{C}$ is a category fibred in setoids,
then there exists a category fibred in sets
$p' : \mathcal{S}' \to \mathcal{C}$ and an equivalence
$\text{can} : \mathcal{S} \to \mathcal{S}'$ over $\mathcal{C}$.
\end{enumerate}
\end{lemma}

\begin{proof}
Let us prove (2).
An object of the category $\mathcal{S}'$ will be a pair $(U, \xi)$, where
$U \in \Ob(\mathcal{C})$ and $\xi$ is an isomorphism class of objects
of $\mathcal{S}_U$. A morphism $(U, \xi) \to (V , \psi)$ is given by a
morphism $x \to y$, where $x \in \xi$ and $y \in \psi$. Here we identify
two morphisms $x \to y$ and $x' \to y'$ if they induce the same morphism
$U \to V$, and if for some choices of isomorphisms $x \to x'$ in
$\mathcal{S}_U$ and $y \to y'$ in $\mathcal{S}_V$ the compositions
$x \to x' \to y'$ and $x \to y \to y'$ agree. By construction there are
surjective maps on objects and morphisms from $\mathcal{S} \to
\mathcal{S}'$. We define composition of morphisms in $\mathcal{S}'$ to
be the unique law that turns $\mathcal{S} \to \mathcal{S}'$ into a functor.
Some details omitted.
\end{proof}

\noindent
Thus categories fibred in setoids are exactly the categories fibred
in groupoids which are equivalent to categories fibred in sets.
Moreover, an equivalence of categories fibred in sets is an isomorphism
by Lemma \ref{lemma-2-category-fibred-sets}.

\begin{lemma}
\label{lemma-2-category-fibred-setoids}
Let $\mathcal{C}$ be a category. The construction of
Lemma \ref{lemma-setoid-fibres}
part (2) gives a functor
$$
F :
\left\{
\begin{matrix}
\text{the 2-category of categories}\\
\text{fibred in setoids over }\mathcal{C}
\end{matrix}
\right\}
\longrightarrow
\left\{
\begin{matrix}
\text{the category of categories}\\
\text{fibred in sets over }\mathcal{C}
\end{matrix}
\right\}
$$
(see
Definition \ref{definition-functor-into-2-category}).
This functor is an equivalence in the following sense:
\begin{enumerate}
\item for any two 1-morphisms $f, g : \mathcal{S}_1 \to \mathcal{S}_2$
with $F(f) = F(g)$ there exists a unique 2-isomorphism $f \to g$,
\item for any morphism $h : F(\mathcal{S}_1) \to F(\mathcal{S}_2)$
there exists a 1-morphism $f : \mathcal{S}_1 \to \mathcal{S}_2$
with $F(f) = h$, and
\item any category fibred in sets $\mathcal{S}$ is equal to $F(\mathcal{S})$.
\end{enumerate}
In particular, defining $F_i \in \textit{PSh}(\mathcal{C})$ by the
rule $F_i(U) = \Ob(\mathcal{S}_{i, U})/\cong$, we have
$$
\Mor_{\textit{Cat}/\mathcal{C}}(\mathcal{S}_1, \mathcal{S}_2)
\Big/
2\text{-isomorphism}
=
\Mor_{\textit{PSh}(\mathcal{C})}(F_1, F_2)
$$
More precisely, given any map $\phi : F_1 \to F_2$ there exists a
$1$-morphism $f : \mathcal{S}_1 \to \mathcal{S}_2$ which induces
$\phi$ on isomorphism classes of objects and
which is unique up to unique $2$-isomorphism.
\end{lemma}

\begin{proof}
By
Lemma \ref{lemma-2-category-fibred-sets}
the target of $F$ is a category hence the assertion makes sense.
The construction of
Lemma \ref{lemma-setoid-fibres} part (2)
assigns to $\mathcal{S}$ the category fibred in sets whose value over
$U$ is the set of isomorphism classes in $\mathcal{S}_U$. Hence it
is clear that it defines a functor as indicated.
Let $f, g : \mathcal{S}_1 \to \mathcal{S}_2$
with $F(f) = F(g)$ be as in (1). For each object $U$ of $\mathcal{C}$
and each object $x$ of $\mathcal{S}_{1, U}$ we see that $f(x) \cong g(x)$
by assumption. As $\mathcal{S}_2$ is fibred in setoids there exists
a unique isomorphism $t_x : f(x) \to g(x)$ in $\mathcal{S}_{2, U}$.
Clearly the rule $x \mapsto t_x$ gives the desired $2$-isomorphism
$f \to g$. We omit the proofs of (2) and (3).
To see the final assertion use
Lemma \ref{lemma-2-category-fibred-sets}
to see that the right hand side is equal to
$\Mor_{\textit{Cat}/\mathcal{C}}(F(\mathcal{S}_1), F(\mathcal{S}_2))$
and apply (1) and (2) above.
\end{proof}

\noindent
Here is another characterization of categories fibred in setoids
among all categories fibred in groupoids.

\begin{lemma}
\label{lemma-characterize-fibred-setoids-inertia}
Let $\mathcal{C}$ be a category.
Let $p : \mathcal{S} \to \mathcal{C}$ be a category fibred in groupoids.
The following are equivalent:
\begin{enumerate}
\item $p : \mathcal{S} \to \mathcal{C}$ is a category fibred in setoids, and
\item the canonical $1$-morphism $\mathcal{I}_\mathcal{S} \to \mathcal{S}$,
see (\ref{equation-inertia-structure-map}), is an equivalence (of categories
over $\mathcal{C}$).
\end{enumerate}
\end{lemma}

\begin{proof}
Assume (2). The category $\mathcal{I}_\mathcal{S}$ has objects
$(x, \alpha)$ where $x \in \mathcal{S}$, say with $p(x) = U$, and
$\alpha : x \to x$ is a morphism in $\mathcal{S}_U$. Hence if
$\mathcal{I}_\mathcal{S} \to \mathcal{S}$ is an equivalence over $\mathcal{C}$
then every pair of objects $(x, \alpha)$, $(x, \alpha')$ are isomorphic
in the fibre category of $\mathcal{I}_\mathcal{S}$ over $U$.
Looking at the definition of morphisms in $\mathcal{I}_\mathcal{S}$
we conclude that $\alpha$, $\alpha'$ are conjugate in the group
of automorphisms of $x$. Hence taking $\alpha' = \text{id}_x$ we conclude
that every automorphism of $x$ is equal to the identity.
Since $\mathcal{S} \to \mathcal{C}$ is fibred in groupoids this
implies that $\mathcal{S} \to \mathcal{C}$ is fibred in setoids.
We omit the proof of (1) $\Rightarrow$ (2).
\end{proof}

\begin{lemma}
\label{lemma-category-fibred-setoids-presheaves-products}
Let $\mathcal{C}$ be a category.
The construction of
Lemma \ref{lemma-2-category-fibred-setoids}
which associates to a category fibred in setoids a presheaf is
compatible with products, in the sense that the presheaf associated
to a $2$-fibre product $\mathcal{X} \times_\mathcal{Y} \mathcal{Z}$
is the fibre product of the presheaves associated to
$\mathcal{X}, \mathcal{Y}, \mathcal{Z}$.
\end{lemma}

\begin{proof}
Let $U \in \Ob(\mathcal{C})$. The lemma just says that
$$
\Ob((\mathcal{X} \times_\mathcal{Y} \mathcal{Z})_U)/\!\cong
\quad \text{equals} \quad
\Ob(\mathcal{X}_U)/\!\cong
\ \times_{\Ob(\mathcal{Y}_U)/\!\cong}
\ \Ob(\mathcal{Z}_U)/\!\cong
$$
the proof of which we omit. (But note that this would not be true
in general if the category $\mathcal{Y}_U$ is not a setoid.)
\end{proof}










\section{Representable categories fibred in groupoids}
\label{section-representable-fibred-groupoids}

\noindent
Here is our definition of a representable category fibred in groupoids.
As promised this is invariant under equivalences.

\begin{definition}
\label{definition-representable-fibred-category}
Let $\mathcal{C}$ be a category.
A category fibred in groupoids $p : \mathcal{S} \to \mathcal{C}$ is
called {\it representable} if there exist an object
$X$ of $\mathcal{C}$ and an equivalence $j : \mathcal{S} \to \mathcal{C}/X$
(in the $2$-category of groupoids over $\mathcal{C}$).
\end{definition}

\noindent
The usual abuse of notation is to say that {\it $X$ represents $\mathcal{S}$}
and not mention the equivalence $j$. We spell out what this entails.

\begin{lemma}
\label{lemma-characterize-representable-fibred-category}
Let $\mathcal{C}$ be a category.
Let $p : \mathcal{S} \to \mathcal{C}$ be a category fibred in groupoids.
\begin{enumerate}
\item $\mathcal{S}$ is representable if and only if
the following conditions are satisfied:
\begin{enumerate}
\item $\mathcal{S}$ is fibred in setoids, and
\item the presheaf $U \mapsto \Ob(\mathcal{S}_U)/\cong$ is
representable.
\end{enumerate}
\item If $\mathcal{S}$ is representable the pair $(X, j)$, where $j$ is the
equivalence $j : \mathcal{S} \to \mathcal{C}/X$, is uniquely determined
up to isomorphism.
\end{enumerate}
\end{lemma}

\begin{proof}
The first assertion follows immediately from
Lemma \ref{lemma-setoid-fibres}.
For the second, suppose that $j' : \mathcal{S} \to \mathcal{C}/X'$ is
a second such pair. Choose a $1$-morphism
$t' : \mathcal{C}/X' \to \mathcal{S}$ such that
$j' \circ t' \cong \text{id}_{\mathcal{C}/X'}$ and
$t' \circ j' \cong \text{id}_\mathcal{S}$. Then
$j \circ t' : \mathcal{C}/X' \to \mathcal{C}/X$ is an equivalence.
Hence it is an isomorphism, see Lemma \ref{lemma-2-category-fibred-sets}.
Hence by the Yoneda Lemma \ref{lemma-yoneda} (via
Example \ref{example-fibred-category-from-functor-of-points} for example)
it is given by an isomorphism $X' \to X$.
\end{proof}

\begin{lemma}
\label{lemma-morphisms-representable-fibred-categories}
Let $\mathcal{C}$ be a category.
Let $\mathcal{X}$, $\mathcal{Y}$ be categories fibred in groupoids
over $\mathcal{C}$. Assume that $\mathcal{X}$, $\mathcal{Y}$
are representable by objects $X$, $Y$ of $\mathcal{C}$.
Then
$$
\Mor_{\textit{Cat}/\mathcal{C}}(\mathcal{X}, \mathcal{Y})
\Big/
2\text{-isomorphism}
=
\Mor_\mathcal{C}(X, Y)
$$
More precisely, given $\phi : X \to Y$ there exists a
$1$-morphism $f : \mathcal{X} \to \mathcal{Y}$ which induces
$\phi$ on isomorphism classes of objects and
which is unique up to unique $2$-isomorphism.
\end{lemma}

\begin{proof}
By
Example \ref{example-fibred-category-from-functor-of-points}
we have $\mathcal{C}/X = \mathcal{S}_{h_X}$ and
$\mathcal{C}/Y = \mathcal{S}_{h_Y}$. By
Lemma \ref{lemma-2-category-fibred-setoids}
we have
$$
\Mor_{\textit{Cat}/\mathcal{C}}(\mathcal{X}, \mathcal{Y})
\Big/
2\text{-isomorphism}
=
\Mor_{\textit{PSh}(\mathcal{C})}(h_X, h_Y)
$$
By the Yoneda
Lemma \ref{lemma-yoneda}
we have $\Mor_{\textit{PSh}(\mathcal{C})}(h_X, h_Y)
= \Mor_\mathcal{C}(X, Y)$.
\end{proof}







\section{Representable 1-morphisms}
\label{section-representable-1-morphisms}

\noindent
Let $\mathcal{C}$ be a category.
In this section we explain what it means for a $1$-morphism
between categories fibred in groupoids over $\mathcal{C}$
to be representable. Note that the $2$-category of categories
fibred in groupoids over $\mathcal{C}$ is a
``full'' sub $2$-category of the $2$-category of categories over
$\mathcal{C}$ (see
Definition \ref{definition-categories-fibred-in-groupoids-over-C}).
Hence if $\mathcal{S}$, $\mathcal{S}'$ are fibred in groupoids
over $\mathcal{C}$ then
$$
\Mor_{\textit{Cat}/\mathcal{C}}(\mathcal{S}, \mathcal{S}')
$$
denotes the category of $1$-morphisms in this $2$-category
(see Definition \ref{definition-categories-over-C}).
These are all groupoids, see remarks following
Definition \ref{definition-categories-fibred-in-groupoids-over-C}.
Here is the $2$-category analogue of the Yoneda lemma.

\begin{lemma}[2-Yoneda lemma]
\label{lemma-yoneda-2category}
Let $\mathcal{S}\to \mathcal{C}$ be fibred in groupoids.
Let $U \in \Ob(\mathcal{C})$.
The functor
$$
\Mor_{\textit{Cat}/\mathcal{C}}(\mathcal{C}/U, \mathcal{S})
\longrightarrow
\mathcal{S}_U
$$
given by $G \mapsto G(\text{id}_U)$ is an equivalence.
\end{lemma}

\begin{proof}
Make a choice of pullbacks for $\mathcal{S}$
(see Definition \ref{definition-pullback-functor-fibred-category}).
We define a functor
$$
\mathcal{S}_U
\longrightarrow
\Mor_{\textit{Cat}/\mathcal{C}}(\mathcal{C}/U, \mathcal{S})
$$
as follows. Given
$x \in \Ob(\mathcal{S}_U)$
the associated functor is
\begin{enumerate}
\item on objects: $(f : V \to U) \mapsto f^*x$, and
\item on morphisms: the arrow $(g : V'/U \to V/U)$ maps to
the composition
$$
(f \circ g)^*x \xrightarrow{(\alpha_{g, f})_x} g^*f^*x \rightarrow f^*x
$$
where $\alpha_{g, f}$ is as in Lemma \ref{lemma-fibred-groupoids}.
\end{enumerate}
We omit the verification that this is an inverse to the functor
of the lemma.
\end{proof}

\begin{remark}
\label{remark-alternative-fibred-groupoids-strict}
We can use the $2$-Yoneda lemma to give an alternative proof of
Lemma \ref{lemma-fibred-groupoids-strict}.
Let $p : \mathcal{S} \to \mathcal{C}$ be a category fibred in groupoids.
We define a contravariant functor $F$ from $\mathcal{C}$ to the
category of groupoids as follows: for $U\in \Ob(\mathcal{C})$
let
$$
F(U) = \Mor_{\textit{Cat}/\mathcal{C}}(\mathcal{C}/U, \mathcal{S}).
$$
If $f : U \to V$ the induced functor $\mathcal{C}/U \to \mathcal{C}/V$
induces the morphism $F(f) : F(V) \to F(U)$. Clearly $F$ is a functor.
Let $\mathcal{S}'$ be the associated category fibred in groupoids from
Example \ref{example-functor-groupoids}.
There is an obvious functor $G : \mathcal{S}' \to \mathcal{S}$
over $\mathcal{C}$ given by taking the pair $(U, x)$, where
$U \in \Ob(\mathcal{C})$ and $x \in F(U)$, to
$x(\text{id}_U) \in \mathcal{S}$.  Now
Lemma \ref{lemma-yoneda-2category}
implies that for each $U$,
$$
G_U : \mathcal{S}'_U = F(U)=
\Mor_{\textit{Cat}/\mathcal{C}}(\mathcal{C}/U, \mathcal{S})
\to
\mathcal{S}_U
$$
is an equivalence, and thus $G$ is an equivalence between $\mathcal{S}$ and
$\mathcal{S}'$ by Lemma \ref{lemma-equivalence-fibred-categories}.
\end{remark}

\noindent
Let $\mathcal{C}$ be a category.
Let $\mathcal{X}$, $\mathcal{Y}$ be categories fibred in groupoids
over $\mathcal{C}$.
Let $U \in \Ob(\mathcal{C})$.
Let $F : \mathcal{X} \to \mathcal{Y}$ and
$G : \mathcal{C}/U \to \mathcal{Y}$ be $1$-morphisms of categories
fibred in groupoids over $\mathcal{C}$.
We want to describe
the $2$-fibre product
$$
\xymatrix{
(\mathcal{C}/U) \times_\mathcal{Y} \mathcal{X} \ar[r] \ar[d] &
\mathcal{X} \ar[d]^F \\
\mathcal{C}/U \ar[r]^G &
\mathcal{Y}
}
$$
Let $y = G(\text{id}_U) \in \mathcal{Y}_U$.
Make a choice of pullbacks for $\mathcal{Y}$
(see Definition \ref{definition-pullback-functor-fibred-category}).
Then $G$ is isomorphic to the functor $(f : V \to U) \mapsto f^*y$,
see Lemma \ref{lemma-yoneda-2category} and its proof.
We may think of an object of
$(\mathcal{C}/U) \times_\mathcal{Y} \mathcal{X}$
as a quadruple $(V, f : V \to U, x, \phi)$, see
Lemma \ref{lemma-2-product-categories-over-C}.
Using the description of $G$ above we may think of $\phi$ as
an isomorphism $\phi : f^*y \to F(x)$ in $\mathcal{Y}_V$.

\begin{lemma}
\label{lemma-identify-fibre-product}
In the situation above the fibre category of
$(\mathcal{C}/U) \times_\mathcal{Y} \mathcal{X}$ over
an object $f : V \to U$ of $\mathcal{C}/U$
is the category described as follows:
\begin{enumerate}
\item objects are pairs $(x, \phi)$,
where $x \in \Ob(\mathcal{X}_V)$, and
$\phi : f^*y \to F(x)$ is a morphism in $\mathcal{Y}_V$,
\item the set of morphisms between $(x, \phi)$ and $(x', \phi')$
is the set of morphisms $\psi : x \to x'$ in $\mathcal{X}_V$
such that $F(\psi) = \phi' \circ \phi^{-1}$.
\end{enumerate}
\end{lemma}

\begin{proof}
See discussion above.
\end{proof}

\begin{lemma}
\label{lemma-prepare-representable-map-stack-in-groupoids}
Let $\mathcal{C}$ be a category.
Let $\mathcal{X}$, $\mathcal{Y}$ be categories fibred in groupoids
over $\mathcal{C}$.
Let $F : \mathcal{X} \to \mathcal{Y}$ be a $1$-morphism.
Let $G : \mathcal{C}/U \to \mathcal{Y}$ be a $1$-morphism.
Then
$$
(\mathcal{C}/U) \times_\mathcal{Y} \mathcal{X}
\longrightarrow
\mathcal{C}/U
$$
is a category fibred in groupoids.
\end{lemma}

\begin{proof}
We have already seen in Lemma \ref{lemma-2-product-fibred-categories}
that the composition
$$
(\mathcal{C}/U) \times_\mathcal{Y} \mathcal{X}
\longrightarrow
\mathcal{C}/U
\longrightarrow
\mathcal{C}
$$
is a category fibred in groupoids. Then the lemma follows from
Lemma \ref{lemma-cute-groupoids}.
\end{proof}

\begin{definition}
\label{definition-representable-map-categories-fibred-in-groupoids}
Let $\mathcal{C}$ be a category.
Let $\mathcal{X}$, $\mathcal{Y}$ be categories fibred in groupoids
over $\mathcal{C}$.
Let $F : \mathcal{X} \to \mathcal{Y}$ be a $1$-morphism.
We say $F$ is {\it representable}, or that
{\it $\mathcal{X}$ is relatively representable over $\mathcal{Y}$},
if for every $U \in \Ob(\mathcal{C})$
and any $G : \mathcal{C}/U \to \mathcal{Y}$
the category fibred in groupoids
$$
(\mathcal{C}/U) \times_\mathcal{Y} \mathcal{X}
\longrightarrow
\mathcal{C}/U
$$
is representable.
\end{definition}

\begin{lemma}
\label{lemma-spell-out-representable-map-stack-in-groupoids}
Let $\mathcal{C}$ be a category.
Let $\mathcal{X}$, $\mathcal{Y}$ be categories fibred in groupoids
over $\mathcal{C}$.
Let $F : \mathcal{X} \to \mathcal{Y}$ be a $1$-morphism.
If $F$ is representable then every one of the functors
$$
F_U : \mathcal{X}_U \longrightarrow \mathcal{Y}_U
$$
between fibre categories is faithful.
\end{lemma}

\begin{proof}
Clear from the description of fibre categories in
Lemma \ref{lemma-identify-fibre-product} and the characterization
of representable fibred categories in
Lemma \ref{lemma-characterize-representable-fibred-category}.
\end{proof}

\begin{lemma}
\label{lemma-criterion-representable-map-stack-in-groupoids}
Let $\mathcal{C}$ be a category.
Let $\mathcal{X}$, $\mathcal{Y}$ be categories fibred in groupoids
over $\mathcal{C}$.
Let $F : \mathcal{X} \to \mathcal{Y}$ be a $1$-morphism.
Make a choice of pullbacks for $\mathcal{Y}$.
Assume
\begin{enumerate}
\item each functor $F_U : \mathcal{X}_U \longrightarrow \mathcal{Y}_U$
between fibre categories is faithful, and
\item for each $U$ and each $y \in \mathcal{Y}_U$ the presheaf
$$
(f : V \to U)
\longmapsto
\{(x, \phi) \mid x \in \mathcal{X}_V, \phi : f^*y \to F(x)\}/\cong
$$
is a representable presheaf on $\mathcal{C}/U$.
\end{enumerate}
Then $F$ is representable.
\end{lemma}

\begin{proof}
Clear from the description of fibre categories in
Lemma \ref{lemma-identify-fibre-product} and the characterization
of representable fibred categories in
Lemma \ref{lemma-characterize-representable-fibred-category}.
\end{proof}

\noindent
Before we state the next lemma we point out that the $2$-category
of categories fibred in groupoids is a $(2, 1)$-category, and hence
we know what it means to say that it has a final object (see
Definition \ref{definition-final-object-2-category}). And it has
a final object namely $\text{id} : \mathcal{C} \to \mathcal{C}$.
Thus we define {\it $2$-products} of categories fibred in groupoids
over $\mathcal{C}$ as the $2$-fibre products
$$
\mathcal{X} \times \mathcal{Y} :=
\mathcal{X} \times_\mathcal{C} \mathcal{Y}.
$$
With this definition in place the following lemma makes sense.

\begin{lemma}
\label{lemma-representable-diagonal-groupoids}
Let $\mathcal{C}$ be a category.
Let $\mathcal{S} \to \mathcal{C}$ be a category fibred in groupoids.
Assume $\mathcal{C}$ has products of pairs of objects and fibre products.
The following are equivalent:
\begin{enumerate}
\item The diagonal $\mathcal{S} \to \mathcal{S} \times \mathcal{S}$
is representable.
\item For every $U$ in $\mathcal{C}$, any $G : \mathcal{C}/U \to \mathcal{S}$
is representable.
\end{enumerate}
\end{lemma}

\begin{proof}
Suppose the diagonal is representable, and let $U, G$ be given.
Consider any $V \in \Ob(\mathcal{C})$ and any
$G' : \mathcal{C}/V \to \mathcal{S}$.
Note that $\mathcal{C}/U \times \mathcal{C}/V = \mathcal{C}/U \times V$
is representable. Hence the fibre product
$$
\xymatrix{
(\mathcal{C}/U \times V)
\times_{(\mathcal{S} \times \mathcal{S})}
\mathcal{S}
\ar[r] \ar[d] &
\mathcal{S} \ar[d] \\
\mathcal{C}/U \times V \ar[r]^{(G, G')} &
\mathcal{S} \times \mathcal{S}
}
$$
is representable by assumption.
This means there exists $W \to U \times V$ in $\mathcal{C}$,
such that
$$
\xymatrix{
\mathcal{C}/W \ar[d] \ar[r] & \mathcal{S} \ar[d] \\
\mathcal{C}/U \times \mathcal{C}/V \ar[r] & \mathcal{S} \times \mathcal{S}
}
$$
is cartesian. This implies that
$\mathcal{C}/W \cong \mathcal{C}/U \times_\mathcal{S} \mathcal{C}/V$
(see Lemma \ref{lemma-diagonal-1})
as desired.

\medskip\noindent
Assume (2) holds. Consider any $V \in \Ob(\mathcal{C})$
and any $(G, G') : \mathcal{C}/V \to \mathcal{S} \times \mathcal{S}$.
We have to show that
$\mathcal{C}/V \times_{\mathcal{S} \times \mathcal{S}} \mathcal{S}$
is representable. What we know is that
$\mathcal{C}/V \times_{G, \mathcal{S}, G'} \mathcal{C}/V$
is representable, say by $a : W \to V$ in $\mathcal{C}/V$.
The equivalence
$$
\mathcal{C}/W \to \mathcal{C}/V \times_{G, \mathcal{S}, G'} \mathcal{C}/V
$$
followed by the second projection to $\mathcal{C}/V$ gives a
second morphism $a' : W \to V$. Consider
$W' = W \times_{(a, a'), V \times V} V$.
There exists an equivalence
$$
\mathcal{C}/W' \cong
\mathcal{C}/V \times_{\mathcal{S} \times \mathcal{S}} \mathcal{S}
$$
namely
\begin{eqnarray*}
\mathcal{C}/W' & \cong &
\mathcal{C}/W \times_{(\mathcal{C}/V \times \mathcal{C}/V)} \mathcal{C}/V \\
& \cong &
\left(\mathcal{C}/V \times_{(G, \mathcal{S}, G')} \mathcal{C}/V\right)
\times_{(\mathcal{C}/V \times \mathcal{C}/V)} \mathcal{C}/V \\
& \cong &
\mathcal{C}/V \times_{(\mathcal{S} \times \mathcal{S})} \mathcal{S}
\end{eqnarray*}
(for the last isomorphism see Lemma \ref{lemma-diagonal-2})
which proves the lemma.
\end{proof}

\noindent
{\bf Bibliographic notes:}
Parts of this have been taken from Vistoli's notes \cite{Vis2}.





\section{Monoidal categories}
\label{section-monoidal}

\noindent
Let $\mathcal{C}$ be a category. Suppose we are given a functor
$$
\otimes : \mathcal{C} \times \mathcal{C} \longrightarrow \mathcal{C}
$$
We often want to know whether $\otimes$ satisfies an associative rule
and whether there is a unit for $\otimes$.

\medskip\noindent
An {\it associativity constraint} for $(\mathcal{C}, \otimes)$ is
a functorial isomorphism
$$
\phi_{X, Y, Z} : X \otimes (Y \otimes Z) \to (X \otimes Y) \otimes Z
$$
such that for all objects $X, Y, Z, W$ the diagram
$$
\xymatrix{
X \otimes (Y \otimes ( Z \otimes W)) \ar[r] \ar[d] &
(X \otimes Y) \otimes (Z \otimes W) \ar[r] &
((X \otimes Y) \otimes Z) \otimes W  \\
X \otimes ((Y \otimes Z) \otimes W) \ar[rr] & &
(X \otimes (Y \otimes Z)) \otimes W \ar[u]
}
$$
is commutative where every arrow is determined by a suitable application
of $\phi$ and functoriality of $\otimes$. Given an associativity constraint
there are well defined functors
$$
\mathcal{C} \times \ldots \times \mathcal{C} \longrightarrow \mathcal{C},
\quad
(X_1, \ldots, X_n) \longmapsto X_1 \otimes \ldots \otimes X_n
$$
for all $n \geq 1$.

\medskip\noindent
Let $\phi$ be an associativity constraint. A {\it unit} for
$(\mathcal{C}, \otimes, \phi)$ is an object $\mathbf{1}$
of $\mathcal{C}$ together with functorial isomorphisms
$$
\mathbf{1} \otimes X \to X
\quad\text{and}\quad
X \otimes \mathbf{1} \to X
$$
such that for all objects $X, Y$ the diagram
$$
\xymatrix{
X \otimes (\mathbf{1} \otimes Y) \ar[rr]_\phi \ar[rd] & &
(X \otimes \mathbf{1}) \otimes Y \ar[ld] \\
& X \otimes Y
}
$$
is commutative where the diagonal arrows are given by the isomorphisms
introduced above.

\medskip\noindent
An equivalent definition would be that a unit is a pair
$(\mathbf{1}, 1)$ where $\mathbf{1}$ is an object of $\mathcal{C}$ and
$1 : \mathbf{1} \otimes \mathbf{1} \to \mathbf{1}$
is an isomorphism such that the functors $L : X \mapsto \mathbf{1} \otimes X$
and $R : X \mapsto X \otimes \mathbf{1}$ are equivalences.
Certainly, given a unit as above we get the isomorphism
$1 : \mathbf{1} \otimes \mathbf{1} \to \mathbf{1}$ for free
and $L$ and $R$ are equivalences as they are isomorphic to the
identity functor. Conversely, given $(\mathbf{1}, 1)$ such that
$L$ and $R$ are equivalences, we obtain functorial isomorphisms
$l : \mathbf{1} \otimes X \to X$ and $r : X \otimes \mathbf{1} \to X$
characterized by $L(l) = 1 \otimes \text{id}_X$ and
$R(r) = \text{id}_X \otimes 1$. Then we can use $r$ and $l$
in the notion of unit as above.

\medskip\noindent
A unit is unique up to unique isomorphism if it exists (exercise).

\begin{definition}
\label{definition-monoidal-category}
A triple $(\mathcal{C}, \otimes, \phi)$ where $\mathcal{C}$ is a category,
$\otimes : \mathcal{C} \times \mathcal{C} \to \mathcal{C}$ is a functor,
and $\phi$ is an associativity constraint is called a {\it monoidal category}
if there exists a unit $\mathbf{1}$.
\end{definition}

\noindent
We always write $\mathbf{1}$ to denote a unit of a monoidal category; as
it is determined up to unique isomorphism there is no harm in choosing one.
From now on we no longer write the brackets when taking tensor
products in monoidal categories and we always identify
$X \otimes \mathbf{1}$ and $\mathbf{1} \otimes X$ with $X$.
Moreover, we will say ``let $\mathcal{C}$ be a monoidal category''
with $\otimes, \phi, \mathbf{1}$ understood.

\begin{definition}
\label{definition-functor-monoidal-categories}
Let $\mathcal{C}$ and $\mathcal{C}'$ be monoidal categories.
A {\it functor of monoidal categories} $F : \mathcal{C} \to \mathcal{C}'$
is given by a functor $F$ as indicated and a isomorphism
$$
F(X) \otimes F(Y) \to F(X \otimes Y)
$$
functorial in $X$ and $Y$
such that for all objects $X$, $Y$, and $Z$ the diagram
$$
\xymatrix{
F(X) \otimes (F(Y) \otimes F(Z)) \ar[r] \ar[d] &
F(X) \otimes F(Y \otimes Z) \ar[r] &
F(X \otimes (Y \otimes Z)) \ar[d] \\
(F(X) \otimes F(Y)) \otimes F(Z) \ar[r] &
F(X \otimes Y) \otimes F(Z) \ar[r] &
F((X \otimes Y) \otimes Z)
}
$$
commutes and such that $F(\mathbf{1})$ is a unit in $\mathcal{C}'$.
\end{definition}

\noindent
By our conventions about units, we may always assume
$F(\mathbf{1}) = \mathbf{1}$ if $F$ is a functor of monoidal categories.
As an example, if $A \to B$ is a ring homomorphism, then
the functor $M \mapsto M \otimes_A B$ is functor of monoidal
categories from $\text{Mod}_A$ to $\text{Mod}_B$.

\begin{lemma}
\label{lemma-invertible}
Let $\mathcal{C}$ be a monoidal category. Let $X$ be an object of
$\mathcal{C}$. The following are equivalent
\begin{enumerate}
\item the functor $L : Y \mapsto X \otimes Y$ is an equivalence,
\item the functor $R : Y \mapsto Y \otimes X$ is an equivalence,
\item there exists an object $X'$ such that
$X \otimes X' \cong X' \otimes X \cong \mathbf{1}$.
\end{enumerate}
\end{lemma}

\begin{proof}
Assume (1). Choose $X'$ such that $L(X') = \mathbf{1}$, i.e.,
$X \otimes X' \cong \mathbf{1}$. Denote $L'$ and $R'$ the functors
corresponding to $X'$. The equation $X \otimes X' \cong \mathbf{1}$
implies $L \circ L' \cong \text{id}$. Thus $L'$ must be the quasi-inverse
to $L$ (which exists by assumption). Hence $L' \circ L \cong \text{id}$.
Hence $X' \otimes X \cong \mathbf{1}$. Thus (3) holds.

\medskip\noindent
The proof of (2) $\Rightarrow$ (3) is dual to what we just said.

\medskip\noindent
Assume (3). Then it is clear that $L'$ and $L$ are quasi-inverse
to each other and it is clear that $R'$ and $R$ are quasi-inverse
to each other. Thus (1) and (2) hold.
\end{proof}

\begin{definition}
\label{definition-invertible}
Let $\mathcal{C}$ be a monoidal category. An object $X$ of $\mathcal{C}$
is called {\it invertible} if any (or all) of the equivalent conditions of
Lemma \ref{lemma-invertible} hold.
\end{definition}

\noindent
Observe that if $F : \mathcal{C} \to \mathcal{C}'$ is a functor of
monoidal categories, then $F$ sends invertible objects to invertible
objects.

\begin{definition}
\label{definition-dual}
Given a monoidal category $(\mathcal{C}, \otimes, \phi)$
and an object $X$ a {\it left dual} is an object $Y$ together with
morphisms $\eta : \mathbf{1} \to X \otimes Y$ and
$\epsilon : Y \otimes X \to \mathbf{1}$
such that the diagrams
$$
\vcenter{
\xymatrix{
X \ar[rd]_1 \ar[r]_-{\eta \otimes 1} &
X \otimes Y \otimes X  \ar[d]^{1 \otimes \epsilon} \\
& X
}
}
\quad\text{and}\quad
\vcenter{
\xymatrix{
Y \ar[rd]_1 \ar[r]_-{1 \otimes \eta} &
Y \otimes X \otimes Y  \ar[d]^{\epsilon \otimes 1} \\
& Y
}
}
$$
commute. In this situation we say that $X$ is a {\it right dual} of $Y$.
\end{definition}

\noindent
Observe that if $F : \mathcal{C} \to \mathcal{C}'$ is a functor of
monoidal categories, then $F(Y)$ is a left dual of $F(X)$ if
$Y$ is a left dual of $X$.

\begin{lemma}
\label{lemma-left-dual}
Let $\mathcal{C}$ be a monoidal category. If $Y$ is a left dual to $X$,
then
$$
\Mor(Z' \otimes X, Z) = \Mor(Z', Z \otimes Y)
\quad\text{and}\quad
\Mor(Y \otimes Z', Z) = \Mor(Z', X \otimes Z)
$$
functorially in $Z$ and $Z'$.
\end{lemma}

\begin{proof}
Consider the maps
$$
\Mor(Z' \otimes X, Z) \to
\Mor(Z' \otimes X \otimes Y, Z \otimes Y) \to
\Mor(Z', Z \otimes Y)
$$
where we use $\eta$ in the second arrow
and the sequence of maps
$$
\Mor(Z', Z \otimes Y) \to
\Mor(Z' \otimes X, Z \otimes Y \otimes X) \to
\Mor(Z' \otimes X, Z)
$$
where we use $\epsilon$ in the second arrow. A straightforward calculation
using the properties of $\eta$ and $\epsilon$
shows that the compositions of these are mutually inverse.
Similarly for the other equality.
\end{proof}

\begin{remark}
\label{remark-left-dual-adjoint}
Lemma \ref{lemma-left-dual} says in particular that $Z \mapsto Z \otimes Y$
is the right adjoint of $Z' \mapsto Z' \otimes X$. In particular, uniqueness
of adjoint functors guarantees that a left dual of $X$, if it exists, is
unique up to unique isomorphism.
Conversely, assume the functor $Z \mapsto Z \otimes Y$ is a right adjoint of
the functor $Z' \mapsto Z' \otimes X$, i.e., we're given a bijection
$$
\Mor(Z' \otimes X, Z) \longrightarrow \Mor(Z', Z \otimes Y)
$$
functorial in both $Z$ and $Z'$. The unit of the adjunction produces
maps
$$
\eta_Z : Z \to Z \otimes X \otimes Y
$$
functorial in $Z$ and the counit of the adjoint produces maps
$$
\epsilon_{Z'} : Z' \otimes Y \otimes X \to Z'
$$
functorial in $Z'$. In particular, we find
$\eta = \eta_\mathbf{1} : \mathbf{1} \to X \otimes Y$ and
$\epsilon = \epsilon_\mathbf{1} : Y \otimes X \to \mathbf{1}$.
As an exercise in the relationship between units, counits, and
the adjunction isomorphism, the reader can show that we have
$$
(\epsilon \otimes \text{id}_Y) \circ \eta_Y = \text{id}_Y
\quad\text{and}\quad
\epsilon_X \circ (\eta \otimes \text{id}_X) = \text{id}_X
$$
However, this isn't enough to show that
$(\epsilon \otimes \text{id}_Y) \circ (\text{id}_Y \otimes \eta) =
\text{id}_Y$ and
$(\text{id}_X \otimes \epsilon) \circ (\eta \otimes \text{id}_X) =
\text{id}_X$, because we don't know in general that
$\eta_Y = \text{id}_Y \otimes \eta$ and we don't know that
$\epsilon_X = \epsilon \otimes \text{id}_X$. For this it would suffice
to know that our adjunction isomorphism has the following property:
for every $W, Z, Z'$ the diagram
$$
\xymatrix{
\Mor(Z' \otimes X, Z) \ar[r] \ar[d]_{\text{id}_W \otimes -} &
\Mor(Z', Z \otimes Y) \ar[d]^{\text{id}_W \otimes -} \\
\Mor(W \otimes Z' \otimes X, W \otimes Z) \ar[r] &
\Mor(W \otimes Z', W \otimes Z \otimes Y)
}
$$
If this holds, we will say {\it the adjunction is compatible with
the given tensor structure}. Thus the requirement that
$Z \mapsto Z \otimes Y$ be the right adjoint of $Z' \mapsto Z' \otimes X$
compatible with the given tensor structure is an equivalent formulation of the
property of being a left dual.
\end{remark}

\begin{lemma}
\label{lemma-tensor-dual}
Let $\mathcal{C}$ be a monoidal category. If $Y_i$, $i = 1, 2$
are left duals of $X_i$, $i = 1, 2$, then $Y_2 \otimes Y_1$ is
a left dual of $X_1 \otimes X_2$.
\end{lemma}

\begin{proof}
Follows from uniqueness of adjoints and Remark \ref{remark-left-dual-adjoint}.
\end{proof}

\noindent
A {\it commutativity constraint} for $(\mathcal{C}, \otimes)$ is a
functorial isomorphism
$$
\psi : X \otimes Y \longrightarrow Y \otimes X
$$
such that the composition
$$
X \otimes Y \xrightarrow{\psi} Y \otimes X \xrightarrow{\psi} X \otimes Y
$$
is the identity. We say $\psi$ is {\it compatible} with a given associativity
constraint $\phi$ if for all objects $X, Y, Z$ the diagram
$$
\xymatrix{
X \otimes (Y \otimes Z) \ar[r]_\phi \ar[d]^\psi &
(X \otimes Y) \otimes Z \ar[r]_\psi &
Z \otimes (X \otimes Y) \ar[d]^\phi \\
X \otimes (Z \otimes Y) \ar[r]^\phi &
(X \otimes Z) \otimes Y \ar[r]^\psi &
(Z \otimes X) \otimes Y
}
$$
commutes.

\begin{definition}
\label{definition-symmetric-monoidal-category}
A quadruple $(\mathcal{C}, \otimes, \phi, \psi)$ where
$\mathcal{C}$ is a category,
$\otimes : \mathcal{C} \otimes \mathcal{C} \to \mathcal{C}$ is a functor,
$\phi$ is an associativity constraint, and
$\psi$ is a commutativity constraint compatible with $\phi$
is called a {\it symmetric monoidal category} if there exists
a unit.
\end{definition}

\noindent
To be sure, if $(\mathcal{C}, \otimes, \phi, \psi)$ is a
symmetric monoidal category, then $(\mathcal{C}, \otimes, \phi)$
is a monoidal category.

\begin{lemma}
\label{lemma-dual-symmetric}
Let $(\mathcal{C}, \otimes, \phi, \psi)$ be a symmetric monoidal category.
Let $X$ be an object of $\mathcal{C}$ and let $Y$,
$\eta : \mathbf{1} \to X \otimes Y$, and
$\epsilon : Y \otimes X \to \mathbf{1}$
be a left dual of $X$ as in Definition \ref{definition-dual}.
Then $\eta' = \psi \circ \eta : \mathbf{1} \to Y \otimes X$
and $\epsilon' = \epsilon \circ \psi : X \otimes Y \to \mathbf{1}$
makes $X$ into a left dual of $Y$.
\end{lemma}

\begin{proof}
Omitted. Hint: pleasant exercise in the definitions.
\end{proof}

\begin{definition}
\label{definition-functor-symmetric-monoidal-categories}
Let $\mathcal{C}$ and $\mathcal{C}'$ be symmetric monoidal categories.
A {\it functor of symmetric monoidal categories}
$F : \mathcal{C} \to \mathcal{C}'$
is given by a functor $F$ as indicated and an isomorphism
$$
F(X) \otimes F(Y) \to F(X \otimes Y)
$$
functorial in $X$ and $Y$
such that $F$ is a functor of monoidal categories and such that
for all objects $X$ and $Y$ the diagram
$$
\xymatrix{
F(X) \otimes F(Y) \ar[r] \ar[d] &
F(X \otimes Y) \ar[d] \\
F(Y) \otimes F(X) \ar[r] &
F(Y \otimes X)
}
$$
commutes.
\end{definition}







\begin{multicols}{2}[\section{Other chapters}]
\noindent
Preliminaries
\begin{enumerate}
\item \hyperref[introduction-section-phantom]{Introduction}
\item \hyperref[conventions-section-phantom]{Conventions}
\item \hyperref[sets-section-phantom]{Set Theory}
\item \hyperref[categories-section-phantom]{Categories}
\item \hyperref[topology-section-phantom]{Topology}
\item \hyperref[sheaves-section-phantom]{Sheaves on Spaces}
\item \hyperref[sites-section-phantom]{Sites and Sheaves}
\item \hyperref[stacks-section-phantom]{Stacks}
\item \hyperref[fields-section-phantom]{Fields}
\item \hyperref[algebra-section-phantom]{Commutative Algebra}
\item \hyperref[brauer-section-phantom]{Brauer Groups}
\item \hyperref[homology-section-phantom]{Homological Algebra}
\item \hyperref[derived-section-phantom]{Derived Categories}
\item \hyperref[simplicial-section-phantom]{Simplicial Methods}
\item \hyperref[more-algebra-section-phantom]{More on Algebra}
\item \hyperref[smoothing-section-phantom]{Smoothing Ring Maps}
\item \hyperref[modules-section-phantom]{Sheaves of Modules}
\item \hyperref[sites-modules-section-phantom]{Modules on Sites}
\item \hyperref[injectives-section-phantom]{Injectives}
\item \hyperref[cohomology-section-phantom]{Cohomology of Sheaves}
\item \hyperref[sites-cohomology-section-phantom]{Cohomology on Sites}
\item \hyperref[dga-section-phantom]{Differential Graded Algebra}
\item \hyperref[dpa-section-phantom]{Divided Power Algebra}
\item \hyperref[hypercovering-section-phantom]{Hypercoverings}
\end{enumerate}
Schemes
\begin{enumerate}
\setcounter{enumi}{24}
\item \hyperref[schemes-section-phantom]{Schemes}
\item \hyperref[constructions-section-phantom]{Constructions of Schemes}
\item \hyperref[properties-section-phantom]{Properties of Schemes}
\item \hyperref[morphisms-section-phantom]{Morphisms of Schemes}
\item \hyperref[coherent-section-phantom]{Cohomology of Schemes}
\item \hyperref[divisors-section-phantom]{Divisors}
\item \hyperref[limits-section-phantom]{Limits of Schemes}
\item \hyperref[varieties-section-phantom]{Varieties}
\item \hyperref[topologies-section-phantom]{Topologies on Schemes}
\item \hyperref[descent-section-phantom]{Descent}
\item \hyperref[perfect-section-phantom]{Derived Categories of Schemes}
\item \hyperref[more-morphisms-section-phantom]{More on Morphisms}
\item \hyperref[flat-section-phantom]{More on Flatness}
\item \hyperref[groupoids-section-phantom]{Groupoid Schemes}
\item \hyperref[more-groupoids-section-phantom]{More on Groupoid Schemes}
\item \hyperref[etale-section-phantom]{\'Etale Morphisms of Schemes}
\end{enumerate}
Topics in Scheme Theory
\begin{enumerate}
\setcounter{enumi}{40}
\item \hyperref[chow-section-phantom]{Chow Homology}
\item \hyperref[intersection-section-phantom]{Intersection Theory}
\item \hyperref[weil-section-phantom]{Weil Cohomology Theories}
\item \hyperref[pic-section-phantom]{Picard Schemes of Curves}
\item \hyperref[adequate-section-phantom]{Adequate Modules}
\item \hyperref[dualizing-section-phantom]{Dualizing Complexes}
\item \hyperref[duality-section-phantom]{Duality for Schemes}
\item \hyperref[discriminant-section-phantom]{Discriminants and Differents}
\item \hyperref[local-cohomology-section-phantom]{Local Cohomology}
\item \hyperref[algebraization-section-phantom]{Algebraic and Formal Geometry}
\item \hyperref[curves-section-phantom]{Algebraic Curves}
\item \hyperref[resolve-section-phantom]{Resolution of Surfaces}
\item \hyperref[models-section-phantom]{Semistable Reduction}
\item \hyperref[pione-section-phantom]{Fundamental Groups of Schemes}
\item \hyperref[etale-cohomology-section-phantom]{\'Etale Cohomology}
\item \hyperref[crystalline-section-phantom]{Crystalline Cohomology}
\item \hyperref[proetale-section-phantom]{Pro-\'etale Cohomology}
\item \hyperref[more-etale-section-phantom]{More \'Etale Cohomology}
\item \hyperref[trace-section-phantom]{The Trace Formula}
\end{enumerate}
Algebraic Spaces
\begin{enumerate}
\setcounter{enumi}{59}
\item \hyperref[spaces-section-phantom]{Algebraic Spaces}
\item \hyperref[spaces-properties-section-phantom]{Properties of Algebraic Spaces}
\item \hyperref[spaces-morphisms-section-phantom]{Morphisms of Algebraic Spaces}
\item \hyperref[decent-spaces-section-phantom]{Decent Algebraic Spaces}
\item \hyperref[spaces-cohomology-section-phantom]{Cohomology of Algebraic Spaces}
\item \hyperref[spaces-limits-section-phantom]{Limits of Algebraic Spaces}
\item \hyperref[spaces-divisors-section-phantom]{Divisors on Algebraic Spaces}
\item \hyperref[spaces-over-fields-section-phantom]{Algebraic Spaces over Fields}
\item \hyperref[spaces-topologies-section-phantom]{Topologies on Algebraic Spaces}
\item \hyperref[spaces-descent-section-phantom]{Descent and Algebraic Spaces}
\item \hyperref[spaces-perfect-section-phantom]{Derived Categories of Spaces}
\item \hyperref[spaces-more-morphisms-section-phantom]{More on Morphisms of Spaces}
\item \hyperref[spaces-flat-section-phantom]{Flatness on Algebraic Spaces}
\item \hyperref[spaces-groupoids-section-phantom]{Groupoids in Algebraic Spaces}
\item \hyperref[spaces-more-groupoids-section-phantom]{More on Groupoids in Spaces}
\item \hyperref[bootstrap-section-phantom]{Bootstrap}
\item \hyperref[spaces-pushouts-section-phantom]{Pushouts of Algebraic Spaces}
\end{enumerate}
Topics in Geometry
\begin{enumerate}
\setcounter{enumi}{76}
\item \hyperref[spaces-chow-section-phantom]{Chow Groups of Spaces}
\item \hyperref[groupoids-quotients-section-phantom]{Quotients of Groupoids}
\item \hyperref[spaces-more-cohomology-section-phantom]{More on Cohomology of Spaces}
\item \hyperref[spaces-simplicial-section-phantom]{Simplicial Spaces}
\item \hyperref[spaces-duality-section-phantom]{Duality for Spaces}
\item \hyperref[formal-spaces-section-phantom]{Formal Algebraic Spaces}
\item \hyperref[restricted-section-phantom]{Restricted Power Series}
\item \hyperref[spaces-resolve-section-phantom]{Resolution of Surfaces Revisited}
\end{enumerate}
Deformation Theory
\begin{enumerate}
\setcounter{enumi}{84}
\item \hyperref[formal-defos-section-phantom]{Formal Deformation Theory}
\item \hyperref[defos-section-phantom]{Deformation Theory}
\item \hyperref[cotangent-section-phantom]{The Cotangent Complex}
\item \hyperref[examples-defos-section-phantom]{Deformation Problems}
\end{enumerate}
Algebraic Stacks
\begin{enumerate}
\setcounter{enumi}{88}
\item \hyperref[algebraic-section-phantom]{Algebraic Stacks}
\item \hyperref[examples-stacks-section-phantom]{Examples of Stacks}
\item \hyperref[stacks-sheaves-section-phantom]{Sheaves on Algebraic Stacks}
\item \hyperref[criteria-section-phantom]{Criteria for Representability}
\item \hyperref[artin-section-phantom]{Artin's Axioms}
\item \hyperref[quot-section-phantom]{Quot and Hilbert Spaces}
\item \hyperref[stacks-properties-section-phantom]{Properties of Algebraic Stacks}
\item \hyperref[stacks-morphisms-section-phantom]{Morphisms of Algebraic Stacks}
\item \hyperref[stacks-limits-section-phantom]{Limits of Algebraic Stacks}
\item \hyperref[stacks-cohomology-section-phantom]{Cohomology of Algebraic Stacks}
\item \hyperref[stacks-perfect-section-phantom]{Derived Categories of Stacks}
\item \hyperref[stacks-introduction-section-phantom]{Introducing Algebraic Stacks}
\item \hyperref[stacks-more-morphisms-section-phantom]{More on Morphisms of Stacks}
\item \hyperref[stacks-geometry-section-phantom]{The Geometry of Stacks}
\end{enumerate}
Topics in Moduli Theory
\begin{enumerate}
\setcounter{enumi}{102}
\item \hyperref[moduli-section-phantom]{Moduli Stacks}
\item \hyperref[moduli-curves-section-phantom]{Moduli of Curves}
\end{enumerate}
Miscellany
\begin{enumerate}
\setcounter{enumi}{104}
\item \hyperref[examples-section-phantom]{Examples}
\item \hyperref[exercises-section-phantom]{Exercises}
\item \hyperref[guide-section-phantom]{Guide to Literature}
\item \hyperref[desirables-section-phantom]{Desirables}
\item \hyperref[coding-section-phantom]{Coding Style}
\item \hyperref[obsolete-section-phantom]{Obsolete}
\item \hyperref[fdl-section-phantom]{GNU Free Documentation License}
\item \hyperref[index-section-phantom]{Auto Generated Index}
\end{enumerate}
\end{multicols}


\bibliography{my}
\bibliographystyle{amsalpha}

\end{document}
