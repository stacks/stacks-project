\documentclass{amsart}

% The following AMS packages are automatically loaded with amsart 
% documentclass:
%\usepackage{amsmath}
%\usepackage{amssymb}
%\usepackage{amsthm}

% For commutative diagrams you can use
% \usepackage{amscd}
% but Jason prefers xypic
\usepackage[all]{xy}

% To put source file link in headers.
% Change "template.tex" to "this_filename.tex"
\usepackage{fancyhdr}
\pagestyle{fancy}
\lhead{}
\chead{}
\rhead{Source file: \url{src/categories.tex}}
\lfoot{}
\cfoot{\thepage}
\rfoot{}
\renewcommand{\headrulewidth}{0pt}
\renewcommand{\footrulewidth}{0pt}
\renewcommand{\headheight}{12pt}

% For cross-file-references
\usepackage{xr-hyper}

% Package for hypertext links:
\usepackage[colorlinks=true]{hyperref}
% For any local file, say "hello.tex" you want to refer to please use
% \externaldocument[hello-]{hello}
\externaldocument[conventions-]{conventions}
\externaldocument[sets-]{sets}
\externaldocument[sites-]{sites}

% The macro \autoref uses the macros \figurename, etc.
% We list the default values and we change some of them
% to start with a captial.
% Figure	\figurename
% Table		\tablename
% Part		\partname
% Appendix	\appendixname
% Equation	\equationname
% item		\Itemname
% \renewcommand{\Itemname}{Item}
\renewcommand{\Itemautorefname}{Item}
% chapter	\Chaptername
% \renewcommand{\Chaptername}{Chapter}
% \renewcommand{\Chapterautorefname}{Chapter}
% section	\sectionname
\renewcommand{\sectionname}{Section}
\renewcommand{\sectionautorefname}{Section}
% subsection	\subsectionname
\renewcommand{\subsectionname}{Subsection}
\renewcommand{\subsectionautorefname}{Subsection}
% subsubsection	\subsubsectionname
\renewcommand{\subsubsectionname}{Subsubsection}
\renewcommand{\subsubsectionautorefname}{Subsubsection}
% paragraph	\paragraphname
\renewcommand{\paragraphname}{Paragraph}
\renewcommand{\paragraphautorefname}{Paragraph}
% footnote	\Hfootnotename
% \renewcommand{\Hfootnotename}{Footnote}
\renewcommand{\Hfootnoteautorefname}{Footnote}
% Equation	\AMSname
% Theorem	\theoremname


% Theorem environments.
%
\newtheorem{theorem}{Theorem}[subsection]
\newtheorem{proposition}[theorem]{Proposition}
\newtheorem{lemma}[theorem]{Lemma}

\theoremstyle{definition}
\newtheorem{definition}[theorem]{Definition}
\newtheorem{example}[theorem]{Example}
\newtheorem{exercise}[theorem]{Exercise}
\newtheorem{situation}[theorem]{Situation}

\theoremstyle{remark}
\newtheorem{remark}[theorem]{Remark}
\newtheorem{remarks}[theorem]{Remarks}

\numberwithin{equation}{subsection}


% OK, start here.
%
\begin{document}

\title{Categories}

%\begin{abstract}
%\end{abstract}

\maketitle
\thispagestyle{fancy}

\tableofcontents
\section{Introduction}
\label{section-introduction}

\noindent
Categories were first introduced in \cite{GenEqui}.

\section{Categories and $2$-categories}
\label{section-categories-2-categories}

\noindent
The category of categories (which does not exist with our conventions)
is a $2$-category. Similarly, the category of stacks forms a $2$-category.
So, even if you already know about categories, you can read this section and
find the terminology regarding $1$-morphisms and $2$-morphisms that we will
use later for stacks as well.

\subsection{Categories}
\label{subsection-categories}

\noindent
We recall the definitions, partly to fix notation.

\begin{definition}
\label{definition-category}
A {\it category} $\mathcal{C}$ consists of the following data:
\begin{enumerate}
\item A set of objects $\text{Ob}(\mathcal{C})$.
\item For each pair $x,y \in \text{Ob}(\mathcal{C})$ a set of morphisms
$\text{Mor}_\mathcal{C}(x,y)$.
\item For each triple $x,y,z\in \text{Ob}(\mathcal{C})$ a composition
map $ \text{Mor}_\mathcal{C}(y,z) \times \text{Mor}_\mathcal{C}(x,y) 
\to \text{Mor}_\mathcal{C}(x,z) $, denoted $(\phi, \psi) \mapsto 
\phi \circ \psi$.
\end{enumerate}
These data are to satisfy the following rules:
\begin{enumerate}
\item For every element $x\in \text{Ob}(\mathcal{C})$ there exists a unique
identity morphism $\text{id}_x\in \text{Mor}_\mathcal{C}(x,x)$ such that 
$\text{id}_x \circ \phi = \phi$ and $\psi \circ \text{id}_x = \psi $ whenever
these compositions make sense.
\item Composition is transitive $(\phi \circ \psi) \circ \chi =
\phi \circ ( \psi \circ \chi)$ whenever these compositions make sense.
\end{enumerate}
\end{definition}

\noindent
It is customary to require all the morphism sets
$\text{Mor}_{\mathcal{C}}(x,y)$ to be disjoint.
In this way a morphism $\phi: x \to y$ has a unique {\it source} $x$
and a unique {\it target} $y$. This is not strictly necessary,
allthough care has to be taken in formulating condition (2) above
if it is not the case. It is convenient and we will often assume
this is the case. In this case we say that $\phi$ and $\psi$ are
{\it composable} if the source of $\phi$ is equal to the 
target of $\psi$, in which case $\phi \circ \psi$ is defined.
An equivalent definition would be to define a category
as a quintuple $(\text{Ob}, \text{Arrows}, s, t, \circ)$
consisting of a set of objects, a set of morphisms (arrows),
source, target and composition subject to a long list of axioms.
We will occasionally use this point of view.

\begin{remark}
\label{remark-big-categories}
Big categories. In some texts a category is allowed to have a proper
class of objects. We will allow this as well in these notes but only
in the following list of cases (to be updated as we go along)
\begin{enumerate}
\item The category $\textit{Sets}$ of sets.
\item Given a ring $R$ the category of $R$-modules.
\item Given a site $\mathcal{C}$ the category of sheaves
over $\mathcal{C}$.
\item The category of rings.
\item The category of schemes.
\end{enumerate}
\end{remark}

\begin{definition}
\label{definition-isomorphism}
A morphism $\phi : x \to y$ is an {\it isomorphism} of the category
$\mathcal{C}$ if there exists a morphism $\psi : y \to x$
such that $\phi \circ \psi = \text{id}_y$ and
$\psi \circ \phi = \text{id}_x$.
\end{definition}

\begin{definition} 
\label{definition-groupoid}
A {\it groupoid} is a category where every morphism is an isomorphism.
\end{definition}

\begin{example}
\label{example-group-groupoid}
A group $G$ can be thought of as a groupoid with a single object $x$
and morphisms $\text{Mor}(x,x)=G$, with the composition rule
given by the group law in $G$.
\end{example}

\begin{example}
\label{example-set-groupoid}
Any set $C$ the set of objects of a groupoid $\mathcal{C}$ if we
let $\text{Ob}(\mathcal{C})=C$ and declare $\text{Mor}(x,y)$ to be empty if
$x\neq y$ and to be $\{\text{id}_x\}$ if $x=y$.
\end{example}

\begin{definition}
\label{definition-functor}
A {\it functor} $F : \mathcal{A} \to \mathcal{B}$
between two categories $\mathcal{A}, \mathcal{B}$ is given by the
following data:
\begin{enumerate}
\item A map $F : \text{Ob}(\mathcal{A}) \to \text{Ob}(\mathcal{B})$.
\item For every $x,y \in \text{Ob}(\mathcal{A})$ a map
$F : \text{Mor}_\mathcal{A}(x,y) \to \text{Mor}_\mathcal{B}(F(x), F(y))$,
denoted $\phi \mapsto F(\phi)$.
\end{enumerate}
These data should be compatible with composition and identity morphisms
in the following manner: $F(\phi \circ \psi) =
F(\phi) \circ F(\psi)$ for a composable pair $(\phi, \psi)$ of
morphisms of $\mathcal{A}$ and $F(\text{id}_x) = \text{id}_{F(x)}$.
\end{definition}

\noindent
Note that every category $\mathcal{A}$ has an
{\it identity} functor $\text{id}_\mathcal{A}$.
In addition, given a functor $G : \mathcal{B} \to \mathcal{C}$
and a functor $F : \mathcal{A} \to \mathcal{B}$ there is
a {\it composition} functor $G \circ F : \mathcal{A} \to \mathcal{C}$
defined in an obvious manner.

\begin{definition}
\label{definition-faithfull}
Let $F : \mathcal{A} \to \mathcal{B}$ be a functor.
\begin{enumerate}
\item We say $F$ is {\it faithfull} if 
for any objects $x,y$ of $\text{Ob}(\mathcal{A})$ the map
$F : \text{Mor}_\mathcal{A}(x,y) \to \text{Mor}_\mathcal{B}(F(x), F(y))$
is injective.
\item If these maps are all bijective then $F$ is called
{\it fully faithfull}.
\item
The functor $F$ is called {\it essentially surjective} if for any 
object $y \in \text{Ob}(\mathcal{B})$ there exists an object
$x \in \text{Ob}(\mathcal{A})$ such that $F(x)$ is isomorphic to $y$ in
$\mathcal{B}$.
\end{enumerate}
\end{definition}

\begin{definition}
\label{definition-subcategory}
A {\it subcategory} of a category $\mathcal{B}$ is
a category $\mathcal{A}$ whose objects and arrows
form subsets of the objects and arrows
of $\mathcal{A}$ and such that source, target
and composition in $\mathcal{A}$ agree with those
of $\mathcal{B}$. We say $\mathcal{A}$ is a
{\it full subcategory} of $\mathcal{B}$ if $\text{Mor}_{\mathcal{A}}(x,y)
= \text{Mor}_{\mathcal{B}}(x,y)$ for all $x,y \in \text{Ob}(\mathcal{A})$.
\end{definition}

\noindent
If $\mathcal{A} \subset \mathcal{B}$ is a subcategory then the
identity map is a functor from $\mathcal{A}$ to $\mathcal{B}$.
Furthermore a subcategory $\mathcal{A} \subset \mathcal{B}$
is full if and only if the inclusion functor is fully faithfull.
Note that given a category $\mathcal{B}$ the set of full subcategories
of $\mathcal{B}$ is the same as the set of subsets of
$\text{Ob}(\mathcal{B})$.

\begin{remark} 
\label{remark-functor-into-sets}
Suppose that $\mathcal{A}$ is a category.
A functor $F$ from $\mathcal{C}$ to $\textit{Sets}$
is a mathematical object (i.e., it is a set, see
Sets \autoref{sets-section-sets-everything})
even though the category of sets is ``big''.
Namely, the range of $F$ on objects will be 
a set $F(\text{Ob}(\mathcal{A}))$ and then we 
may think of $F$ as a functor between 
$\mathcal{A}$ and the full subcategory
of the category of sets whose
objects are elements of $F(\text{Ob}(\mathcal{A}))$.
\end{remark}

\begin{example}
\label{example-group-homorphism-functor}
A homomorphism $p\colon G\to H$ of groups gives rise to a functor
between the associated groupoids in Example \ref{example-group-groupoid}. It is
faithful (resp.\ fully faithful) if and only if $p$ is injective (resp.\ an
isomorphism).
\end{example}

\begin{example}
\label{example-comma-category}
Given a category $\mathcal{C}$ and an object $X\in \text{Ob}(\mathcal{C})$
we define the category of objects over $X$, denoted $\mathcal{C}/X$ as follows.
The objects of $\mathcal{C}/X$ are morphisms $Y\to X$ for
some $Y\in \text{Ob}(\mathcal{C})$. Morphisms between objects
$Y\to X$ and $Y'\to X$ are morphisms $Y\to Y'$ in $\mathcal{C}$ that
make the obvious diagram commute.  Note that there is a functor
$p_X\colon \mathcal{C}/X\to \mathcal{C}$ which simply forgets the
morphism for $X$.  Moreover given a morphism $f\colon X'\to X$ in
$\mathcal{C}$ there is an induced functor 
$F\colon \mathcal{C}/X' \to \mathcal{C}/X$ obtained by composition with $f$,
and $p_X\circ F = p_{X'}$.
\end{example}

\begin{definition}
Let $F, G : \mathcal{A} \to \mathcal{B}$ be functors.
A {\it transformation}, or a {\it morphism} of functors
$t : F \to G$, is given by the following data:
\begin{enumerate}
\item For every $x\in \text{Ob}(\mathcal{A})$ a morphism
$t_x : F(x) \to G(x)$ in the category $\mathcal{B}$.
\end{enumerate}
These data should satisfy the condition that for every 
morphism $\phi : x \to y$ of $\mathcal{A}$ the following
diagram is commutative
$$
\xymatrix{
F(x) \ar[r]^{t_x} \ar[d]_{F(\phi)} & G(x) \ar[d]^{G(\phi)} \\
F(y) \ar[r]^{t_y} & G(y) }
$$
\end{definition}

\noindent
Note that every functor $F$ comes with the {\it identity} transformation
$\text{id}_F : F \to F$. In addition, given a morphism of
functors $t : F \to G$ and a morphism of functors $s : E \to F$
then the {\it composition} $t \circ s$ is defined by the rule
$$
(t \circ s)_x = t_x \circ s_x : E(x) \to G(x)
$$
for $x \in \text{Ob}(\mathcal{A}$.
It is easy to verify that this is indeed a morphism of functors
from $E$ to $G$.
In this way, given categories 
$\mathcal{A}$ and $\mathcal{B}$ we obtain a new category,
namely the category of functors between $\mathcal{A}$ and
$\mathcal{B}$.

\begin{remark}
This is one instance where the same thing does not hold if
$\mathcal{A}$ is a ``big'' category. For example consider
functors $\textit{Sets} \to \textit{Sets}$. As we have currently
defined it such a functor is a class and not a set. In other
words, it is given by a formula in set theory (with some variables
equal to specified sets)! It is not a good idea to try to consider
all possible formulae of set theory as part of the definition of 
a mathematical object. The same problem presents itself when
considering sheaves on the category of schemes for example.
We will come back to this point later.
\end{remark}

\begin{definition}
\label{definiton-equivalence-categories}
An {\it equivalence of categories}
$F : \mathcal{A} \to \mathcal{B}$ is a functor such that there
exists a functor $G : \mathcal{B} \to \mathcal{A}$ such that
the compositions $F \circ G$ and $G \circ F$ are isomorphic to the
identity functors $\text{id}_\mathcal{B}$,
respectively $\text{id}_\mathcal{A}$.
\end{definition}

\begin{lemma}
A functor is an equivalence of categories if and only if it is both fully
faithful and essentially surjective.
\end{lemma}

\begin{proof} FIXME. \end{proof}

\subsubsection{Formal properties of categories, functors and transformations}
\label{subsubsection-formal-cat-cat}

\noindent
Let us denote $\text{Ob}(\textit{Cat})$ the class of all categories.
For every pair of categories
$\mathcal{A}, \mathcal{B} \in \text{Ob}(\text{Cat})$
we have the category of functors
$$
\text{Fun}(\mathcal{A}, \mathcal{B}).
$$
There is a bit more structure on this. Namely for every triple
of categories $\mathcal{A}$, $\mathcal{B}$, and $\mathcal{C}$
there is a composition law
$$
\circ : \text{Ob}(\text{Fun}(\mathcal{B}, \mathcal{C}))
\times 
\text{Ob}(\text{Fun}(\mathcal{A}, \mathcal{B}))
\longrightarrow
\text{Ob}(\text{Fun}(\mathcal{A}, \mathcal{C}))
$$
coming from composition of functors. This composition law
is associative.

\medskip\noindent
Aditionally, given $t : F \to F'$ a transformation of
functors $F, F' : \mathcal{A} \to \mathcal{B}$ and
$G : \mathcal{B} \to \mathcal{C}$ a functor we can define
$G(t)$ to be the transformation of functors
$G\circ F \to G \circ F'$ given by
$G(t)_x = G(t_x) : G(F(x)) \to G(F'(x))$
for all $x \in \mathcal{A}$. In this way composition
with $G$ becomes a functor
$$
\text{Fun}(\mathcal{A}, \mathcal{B})
\longrightarrow
\text{Fun}(\mathcal{A}, \mathcal{C}).
$$
Similarly, given $s : G \to G'$ a transformation of
functors $G, G' : \mathcal{B} \to \mathcal{C}$ and
$F : \mathcal{A} \to \mathcal{B}$ a functor we can define
$F(s)$ to be the transformation of functors
$G\circ F \to G' \circ F$ given by
$F(s)_x = s_{F(x)} : G(F(x)) \to G'(F(x))$
for all $x \in \mathcal{A}$. In this way
composition with $F$ becomes a functor
$$
\text{Fun}(\mathcal{B}, \mathcal{C})
\longrightarrow
\text{Fun}(\mathcal{A}, \mathcal{C}).
$$
Finally, given functors $F, F' : \mathcal{A} \to \mathcal{B}$,
and $G, G' : \mathcal{B} \to \mathcal{C}$ and transformations
$t : F \to F'$, and $s : G \to G'$ the following
diagram is commutative
$$
\xymatrix{
G \circ F \ar[r]^{G(t)} \ar[d]_{F(s)}
&
G \circ F' \ar[d]^{F'(s)} \\
G' \circ F \ar[r]_{G'(t)}
&
G' \circ F'
}
$$
To prove this we just consider what happens on
any object $x \in \mathcal{A}$:
$$
\xymatrix{
G(F(x)) \ar[r]^{G(t_{F(x)})} \ar[d]_{s_{F(x)}}
&
G(F'(x)) \ar[d]^{s_{F'(x)}} \\
G'(F(x)) \ar[r]_{G'(t_{F(x)})}
&
G'(F'(x))
}
$$
which is commutative because $s$ is a transformation
of functors.



\subsubsection{Additional notions}
\label{subsubsection-categories-additional}

\begin{definition}
\label{definition-fibre-products}
Let $x,y\in \text{Ob}(\mathcal{C})$ and $f\in \text{Mor}_{\mathcal{C}}(x,z)$
and $g\in \text{Mor}_{\mathcal C}(y,z)$.  The fibre product of $f$ and $g$ is
an object $x\times_z y\in \text{Ob}(\mathcal{C})$ together with morphisms 
$p\in \text{Mor}_{\mathcal C}(x\times_z y,x)$ and 
$q\in\text{Mor}_{\mathcal C}(x\times_z y,y)$ making the diagram
$$
\xymatrix{
&x\times_y z \ar[r]^{p} \ar[d]_{q} & x \ar[d]^{f} \\
&y \ar[r]^{g} & z }
$$
commute, and such that the following universal property holds: for
any $w\in \text{Ob}(\mathcal{C})$ and morphisms 
$\alpha \in \text{Mor}_{\mathcal C}(w,x)$ and 
$\beta \in \text{Mor}_{\mathcal{C}}(w,y)$ with $f \circ \alpha= g\circ \beta$
there is a unique $\gamma\in \text{Mor}_{\mathcal C}(w,x\times_z y)$ making
the diagram
$$
\xymatrix{
w\ar[rrrd]^\alpha \ar@{-->}[rrd]_\gamma \ar[rrdd]_\beta &&\\
&&x\times_y z \ar[r]_{p} \ar[d]_{q} & x \ar[d]^{f} \\
&&y \ar[r]^{g} & z }
$$
commute.  If a fibre product exists it is unique up to unique
isomorphism.  We say the category $\mathcal{C}$ has fibre products if
the fibre product exists for any $f\in \text{Mor}_{\mathcal C}(x,z)$
and $g\in \text{Mor}_{\mathcal C}(y,z)$.
\end{definition}

\noindent
Given a category $\mathcal{C}$ we can form the opposite category
$\mathcal{C}^{\text{opp}}$ which has the same objects as $\mathcal{C}$
but all morphisms reversed, so
$\text{Mor}_{\mathcal{C}^{\text{opp}}}(x,y) =
\text{Mor}_{\mathcal{C}}(y,x)$.  A contravariant functor $F\colon
\mathcal{C}\to \mathcal{S}$ is a functor $\mathcal{C}^{\text{opp}}\to
\mathcal{S}$.  Concretely, if $F$ is contravariant then for composable
morphisms $f$ and $g$ in $\mathcal{C}$, $F(f\circ g) = F(g)\circ
F(f)$.

\begin{example}
\label{example-hom-functor}
For any $U\in \text{Ob}(\mathcal{C})$ there is a contravariant
functor 
$$
\text{Mor}(-,U) \colon\mathcal{C} \to \text{Sets}
$$
which takes an object $X$ to the set $\text{Mor}_{\mathcal{C}}(X,U)$.
Given a morphism $f\colon X\to Y$ the corresponding map
$\text{Mor}(-,U)(f)\colon \text{Mor}(Y,U)\to \text{Mor}(X,U)$ takes
$\phi$ to $\phi\circ f$. More commonly this functor is denoted
$h_U : \mathcal{C}^{\text{opp}} \to \text{Sets}$. If $\mathcal{C}$ is the
category of schemes this functor is sometimes referred to as the
\emph{functor of points} of $U$.
\end{example}

\begin{definition}
\label{definition-representable-functor}
A contravariant functor $F\colon \mathcal{C}\to \text{Sets}$ is said
to be representable if it is isomorphic to the functor
$h_U(-) = \text{Mor}(-,U)$ for some object $U$ of $\mathcal{C}$.
\end{definition}

\begin{definition}
\label{definition-representable-morphism}
A morphism $f : x \to y$ of a category $\mathcal{C}$ is said to be
representable, if and only if for every morphism $z \to y$ in $\mathcal{C}$
the fibre product $z\times_y x$ exists.
\end{definition}

\subsection{2-categories}
\label{subsection-2-categories}

\noindent
We will give a definition of (strict) $2$-categories as they appear in the 
setting of stacks. Before you read this take a look at Example
\ref{example-category-of-categories}. Basically, you take this example
and you write out all the rules satisfied by the objects,
$1$-morphisms and $2$-morphisms in that example. This is actually not that
helpful but it shows that it can be done.

\begin{definition}
\label{definition-2-category}
A $2$-category $\mathcal{C}$ consists of the following data
\begin{enumerate}
\item A set of objects $\text{Ob}(\mathcal{C})$.
\item For each pair $x,y \in \text{Ob}(\mathcal{C})$ a set of $1$-morphisms
$\text{Mor}_\mathcal{C}(x,y)$.
\item For each triple $x,y,z\in \text{Ob}(\mathcal{C})$ a composition
map $ \text{Mor}_\mathcal{C}(y,z) \times \text{Mor}_\mathcal{C}(x,y) 
\to \text{Mor}_\mathcal{C}(x,z) $, denoted $(F, G) \mapsto 
F \circ G$.
\item For each pair $x,y\in \text{Ob}(\mathcal{C})$ and for each pair 
$F,F' \in \text{Mor}_\mathcal{C}(x,y)$ a set of $2$-morphisms 
$\text{Mor}_\mathcal{C}(F,F')$.
\item For each triple $F,F',F''$ of $1$-morphisms with the same source
and target a composition law $\text{Mor}_\mathcal{C}(F',F'') \times
\text{Mor}_\mathcal{C}(F,F') \to \text{Mor}_\mathcal{C}(F,F'')$, denoted
$(\phi, \psi) \mapsto \phi\circ\psi$.
\item For each triple $x,y,z\in \text{Ob}(\mathcal{C})$ and
$1$-morphisms $F,F' : x \to y$ and $G : y \to z$ a map 
$G : \text{Mor}_\mathcal{C}(F,F') \to 
\text{Mor}_\mathcal{C}(G \circ F,G \circ F')$.
\item For each triple $x,y,z\in \text{Ob}(\mathcal{C})$ and
$1$-morphisms $F : x \to y$ and $G,G' : y \to z$ a map 
$F : \text{Mor}_\mathcal{C}(G,G') \to 
\text{Mor}_\mathcal{C}(G \circ F,G' \circ F)$.
\end{enumerate}
These data are to satisfy the following rules:
\begin{enumerate}
\item For every element $x\in \text{Ob}(\mathcal{C})$ there exists a unique
identity $1$-morphism $\text{id}_x\in \text{Mor}_\mathcal{C}(x,x)$ such that 
$\text{id}_x \circ F = F$ and $G \circ \text{id}_x = G $ whenever
these compositions make sense.
\item For every $1$-morphism $F$ there exists a unique
identity $2$-morphism $\text{id}_F\in \text{Mor}_\mathcal{C}(F,F)$ such that 
$\text{id}_F \circ \phi = \phi$ and $\psi \circ \text{id}_F = \psi $ whenever
these compositions make sense.
\item Composition is transitive for both $1$-morphisms and $2$-morphisms.
\item Every $2$-morphism is an isomorphism. This makes sense since the 
conditions sofar imply that $\text{Mor}_\mathcal{C}(x,y)$ is a category
with $1$-morphisms as objects and $2$-morphisms as morphisms. So this
condition means every $\text{Mor}_\mathcal{C}(x,y)$ is a groupoid.
\item Let $x,y,z\in \text{Ob}(\mathcal{C})$ and let $G \in 
\text{Mor}_\mathcal{C}(y,z)$.
The map $\text{Mor}_\mathcal{C}(x,y) \to 
\text{Mor}_\mathcal{C}(x,z)$ given by $G$ (see item 6 above) 
is a functor.
\item Let $x,y,z\in \text{Ob}(\mathcal{C})$ and let $F \in \text{Mor}(x,y)$.
The map $\text{Mor}_\mathcal{C}(y,z) \to 
\text{Mor}_\mathcal{C}(x,z)$ given by $F$ (see item 7 above)
is a functor.
\item Suppose we have objects $x,y,z$, $1$-morphisms $F,F' : x \to y$, 
$G,G' : y \to z$, and $2$-morphisms $\phi : F \to F'$, $\psi : G \to G'$. 
The following diagram commutes:
$$
\xymatrix{
G \circ F \ar[r]^{G(\phi)} \ar[d]_{F(\psi)} &
G\circ F' \ar[d]^{F'(\psi)} \\
G \circ F' \ar[r]_{G'(\phi)} & G' \circ F' }
$$
\end{enumerate}
\end{definition}

\noindent
This is obviously not a very pleasant type of object to work with.
On the other hand, there are lots of examples where it is quite clear
how you work with it. Note that we require the $2$-morphisms to be
isomorphisms. As far as this text is concerned all 2-categories occuring
in this document are (full) sub 2-categories of the example below.
FIXME: Remove this definition? Replace by a better one?


\noindent
The notion of equivalence of categories that we defined in Subsection
\ref{subsection-categories} extends to the more general setting of
$2$-categories as follows.

\begin{definition}
\label{definition-equivalence}
Two objects $x,y$ of a $2$-category are {\it equivalent} if there exist 
$1$-morphisms $F : x \to y$ and $G : y \to x$ such that $F \circ G$ is 
$2$-isomorphic to $\text{id}_y$ and $G \circ F$ is $2$-isomorphic to 
$\text{id}_x$.
\end{definition}

\begin{remark}
\label{remark-other-2-categories}
There are variants of the construction of \ref{example-category-of-categories}
above where we look at the $2$-category of groupoids (contained in some 
$\alpha$), or categories fibred in groupoids over a fixed category, or stacks. 
And so on.
\end{remark}

\begin{remarks}
\label{remarks-functor-into-2-category}
(1) A functor from an ordinary category into a $2$-category will ignore the
$2$-morphisms unless mentioned otherwise. In other words, it will be a 
``usual'' functor into the category formed out of 2-category by forgetting
all the 2-morpshisms.

\smallskip\noindent
(2) Another notion of a functor from a category $\mathcal{A}$ into a
2-category $\mathcal{C}$ would be to say that it is given by a map
$F : \text{Ob}(\mathcal{A}) \to \text{Ob}(\mathcal{C})$ together with a
family of maps 
$F : \text{Mor}_{\mathcal{A}}(x,y) \to \text{Mor}_{\mathcal{C}}(F(x),F(y))$
such that for every composable pair of morphisms $f,g$ of $\mathcal{A}$
the morphisms $F(g \circ f)$ and $F(g) \circ F(f)$ are 2-isomorphic. This is
not a very good notion, since for example it does not require $F(\text{id}_x)$
to be isomorphic to $\text{id}_{F(x)}$. Even if you do then
there may be a problem: see the conditions in (3) below.

\smallskip\noindent
(3) A better notion is the following. A weak functor (or a pseudo-functor)
from a category $\mathcal{A}$ into a 2-category $\mathcal{C}$ is given by 
\begin{enumerate}
\item a map $F : \text{Ob}(\mathcal{A}) \to \text{Ob}(\mathcal{C})$,
\item for every pair $x,y\in \text{Ob}(\mathcal{A})$ a map
$F : \text{Mor}_{\mathcal{A}}(x,y) \to  \text{Mor}_{\mathcal{C}}(F(x),F(y))$,
\item for every $x\in \text{ob}(C)$ a $2$-morphism
$\alpha_x : \text{id}_x \to F(\text{id}_{x})$, and
\item for every pair of composable morphisms $f,g$ of $\mathcal{A}$ a 
$2$-morphism $\alpha_{f,g} : F(g \circ f) \to F(g) \circ F(f)$.
\end{enumerate}
Now these data are subject to the following conditions:
(with notations as in Definition \ref{definition-2-category})
\begin{enumerate}
\item for any morphism $f : x \to y$ in $\mathcal{A}$ the morphism
$\alpha_{f,\text{id}_y} : F(f) \to F(f) \circ F(\text{id}_y)$
equals the composition of $F(f) \circ \text{id}_{F(y)} = F(f)$ with
$F(f)(\alpha_y)$, and similary for $\alpha_{\text{id}_x,f}$ and
$\alpha_x$, and
\item for any triple of composable morphisms $f,g,h$ the
compositions $F(h)(\alpha_{f,g}) \circ \alpha_{g\circ f, h}$ and
$F(f)(\alpha_{g,h}) \circ \alpha_{g,f\circ h}$ should be equal.
\end{enumerate}
Again this is not a very workable notion, but it does sometimes come up.
There is a theorem that says that any pseudo-functor is isomorphic to
a functor. FIXME: Add more as needed.
\end{remarks}

\subsubsection{2-fibre products}
\label{subsubsection-2-fibre-products}

\noindent
In this subsection we introduce $2$-fibre products. Suppose that $\mathcal{C}$
is a 2-category. We say that a diagram
$$
\xymatrix{
w \ar[r] \ar[d] & y \ar[d] \\
x \ar[r] & z }
$$
2-commutes if the two 1-morphisms $w \to y \to z$ and $w \to x \to z$ are
2-isomorphic. In a 2-category it is more natural to ask for 2-commutativity 
of diagrams than for actually commuting diagrams. (Indeed, some may say that
we should not work with strict 2-categories at all, and in a ``weak''
2-category the notion of a commutative diagram of 1-morphisms does not even
make sense.) Correspondingly the notion of a fibre product has to be adjusted.

\smallskip\noindent
Let $\mathcal{C}$ be a $2$-category. Let $x,y,z\in \text{Ob}(\mathcal{C})$ and
$f\in \text{Mor}_{\mathcal{C}}(x,z)$ and $g\in \text{Mor}_{\mathcal C}(y,z)$.
In order to define the 2-fibre product of $f$ and $g$ we are going to look at
2-commutative diagrams
$$
\xymatrix{
&w \ar[r]^{a} \ar[d]_{b} & x \ar[d]^{f} \\
&y \ar[r]^{g} & z. }
$$
Now in the case of categories, the fibre product is a final object in the
category of such diagrams. Correspondingly a 2-fibre product is a final object
in a 2-category (see definition below). The 2-category we will consider is
the 2-category of 2-commutative diagrams defined as follows:
\begin{enumerate}
\item Objects are quadruples $(w,a,b,\phi)$ as above where $\phi$
is a 2-morphism $\phi : f \circ a \to g \circ b$, 
\item 1-morphisms from $(w,a,b,\phi)$ to $(w',a',b',\phi')$
are given by $(k : w \to w', \alpha : a' \to a \circ k,
\beta : b \circ k \to b')$ such that $\phi'$ equals 
$$
\xymatrix{
f \circ a' \ar[r]^{f(\alpha)} &
f \circ a \circ k \ar[r]^{k(\phi)} &
g \circ b \circ k \ar[r]^{g(\beta)} &
g \circ b'. }
$$
\item a 2-morphism between $(k_i, \alpha_i, \beta_i)$, $i=1,2$ is given
by a 2-morphism $\delta : k_1 \to k_2$ such that 
$$
\xymatrix{
a' \ar[rd]_{\alpha_2} \ar[r]^{\alpha_1} & 
a \circ k_1 \ar[d]^{a(\delta)} &
&
b \circ k_1 \ar[r]^{\beta_1} \ar[d]_{b(\delta)} &
b'
\\
&
a \circ k_2 &
&
b \circ k_2 \ar[ru]_{\beta_2}
&
}
$$
commute.
\end{enumerate}

\begin{definition}
\label{definition-final-object-2-category}
A final object of a 2-category $\mathcal{C}$ is an object $x$ such that
(1) for every $y \in \text{Ob}(\mathcal{C})$ there is a morphism $y \to x$,
and (2) every two morphisms $y \to x$ are isomorphic by a unique 2-morphism.
\end{definition}

\begin{definition}
\label{definition-2-fibre-products}
Let $\mathcal{C}$ be a $2$-category.
Let $x,y,z\in \text{Ob}(\mathcal{C})$ and $f\in \text{Mor}_{\mathcal{C}}(x,z)$
and $g\in \text{Mor}_{\mathcal C}(y,z)$. A 2-fibre product of $f$ and $g$ is
a final object in the category of 2-commutative diagrams described above. If
a 2-fibre product exists we
will denote it $x\times_z y\in \text{Ob}(\mathcal{C})$, and denote the
required morphisms $p\in \text{Mor}_{\mathcal C}(x\times_z y,x)$ and 
$q\in \text{Mor}_{\mathcal C}(x\times_z y,y)$ making the diagram
$$
\xymatrix{
&x\times_y z \ar[r]^{p} \ar[d]_{q} & x \ar[d]^{f} \\
&y \ar[r]^{g} & z }
$$
2-commute and we will denote the given 2-morphism exhibiting this by
$\psi : f \circ p \to g \circ q$.
\end{definition}

\noindent
Thus the following universal property holds: for any
$w\in \text{Ob}(\mathcal{C})$ and morphisms 
$a \in \text{Mor}_{\mathcal C}(w,x)$ and 
$b \in \text{Mor}_{\mathcal{C}}(w,y)$ with a given 2-morphism
$\phi : f \circ a \to g\circ b$
there is a $\gamma \in \text{Mor}_{\mathcal C}(w,x\times_z y)$
making the diagram
$$
\xymatrix{
w\ar[rrrd]^a \ar@{-->}[rrd]_\gamma \ar[rrdd]_b &&\\
&&x\times_y z \ar[r]_{p} \ar[d]_{q} & x \ar[d]^{f} \\
&&y \ar[r]^{g} & z }
$$
2-commute such that for suitable choices of $q \circ \gamma \to b$
and $a \to p \circ \gamma$ the composition
$$
\xymatrix{
f \circ a \ar[r] &
f \circ p \circ \gamma \ar[r]^{\gamma(\psi)} &
g \circ q \circ \gamma \ar[r] &
g\circ b }
$$
equals $\phi$. Of course the exact properties are finer than this. All of the
cases of 2-fibre products that we will need later on come from the following
example of 2-fibre products in the 2-category of categories.

\begin{example}
\label{example-2-fibre-product-categories}
In this example we switch notations and we let $\mathcal{A}$, $\mathcal{B}$,
and $\mathcal{C}$ be categories and we let $F : \mathcal{A} \to \mathcal{C}$
and $G : \mathcal{B} \to \mathcal{C}$ be functors. In this case the 2-fibre
product $\mathcal{A}\times_\mathcal{C} \mathcal{B}$ exists and is given by
the following:
\begin{enumerate}
\item an object of $\mathcal{A}\times_\mathcal{C} \mathcal{B}$ is a triple
$(A,B,f)$, where $A\in \text{Ob}(\mathcal{A})$, $B\in \text{Ob}(\mathcal{B})$,
and $f : F(A) \to G(B)$ is an isomorphism in $\mathcal{C}$,
\item a morphism $(A,B,f) \to (A',B', f')$ is given by a pair $(a,b)$, where
$a : A \to A'$ is a morphism in $\mathcal{A}$, and $b : B \to B'$ is a
morphism in $\mathcal{B}$ such that the diagram 
$$
\xymatrix{
F(A) \ar[r]^f \ar[d]^{F(a)} & G(B) \ar[d]^{G(b)} \\
F(A') \ar[r]^{f'} & G(B')
}
$$
is commutative.
\end{enumerate}
The functors $p : \mathcal{A}\times_\mathcal{C}\mathcal{B} \to \mathcal{A}$
and $q : \mathcal{A}\times_\mathcal{C}\mathcal{B} \to \mathcal{B}$ are the
forgetfull functors in this case. The transformation $\psi : F \circ p \to
G \circ q$ is given on the object $\xi = (A,B,f)$ by
$\psi_\xi = f : F(p(\xi)) = F(A) \to G(B) = G(q(\xi))$.

\smallskip\noindent
Let us check the universal property: let $\mathcal{W}$ be a category, let
$X : \mathcal{W} \to \mathcal{A}$ and $Y : \mathcal{W} \to \mathcal{B}$ be
functors, and let $t : F \circ X \to G \circ Y$ be an isomorphism of functors.
The desired functor $\gamma : \mathcal{W} \to
\mathcal{A}\times_\mathcal{C}\mathcal{B}$
is given by $W \mapsto (X(W), Y(W), t_W)$. What else could it be? 
(A meta-argument for uniqueness.) FIXME: write this out.

\smallskip\noindent
Note that the functor $\gamma$ constructed above actually has the property
that $p \circ \gamma = X$ and $q \circ \gamma = Y$. In general this need not
be the case.
\end{example}

\section{Categories fibred in groupoids}
\label{subsection-fibred-groupoids}

\noindent
In this section we explain how to think about categories in groupoids and
we see how they are basically the same as functors in groupoids.

\subsection{Definitions}
\label{subsection-categories-groupoids-definition}

\noindent
In this subsection we have a functor $p : \mathcal{S} \to \mathcal{C}$.
We think of $\mathcal{S}$ as being on top and of $\mathcal{C}$ as being
at the bottom.

\smallskip\noindent
Analogously to the fibre of a map of spaces, we have the notion of a 
fibre category. The fibre category over an object 
$U\in \text{Ob}(\mathcal{C})$ is the category $\mathcal{S}_U$ with 
objects
$$
\text{Ob}(\mathcal{S}_U) = \{x\in \text{Ob}(\mathcal{S}) :
p(x)=U\}
$$
and morphisms 
$$
\text{Mor}_{\mathcal{S}_U}(x,y) = \{ \phi \in \text{Mor}_\mathcal{S}(x,y) :
p(\phi) = \text{id}_U\}.
$$

\smallskip\noindent
In order to discuss the notion of ``category fibred in groupoids'' we
temporarily introduce the notion of lifting.
A {\it lift} of an object $U \in \text{Ob}(\mathcal{C})$ is an object 
$x\in \text{Ob}(\mathcal{S})$ such that $p(x)=U$, i.e., 
$x\in \text{Ob}(\mathcal{S}_U)$.  
Similarly, a {\it lift} of a morphism $f : V \to U$ in $\mathcal{C}$ is a 
morphism $\phi : y \to x$ in $\mathcal{S}$ such that $p(\phi)=f$.

\begin{definition}
\label{definition-fibred-groupoids}
We say that $\mathcal{S}$ is fibred in groupoids over $\mathcal{C}$ if
the following two conditions hold:
\begin{enumerate}
\item For every morphism $f : V \to U$ in $\mathcal{C}$ and every
lift $x$ of $U$ there is a lift $\phi : y \to x$ of $f$ with
target $x$.
\item For every pair of morphisms $ \phi : y \to x$ and $ \psi : z \to x$
and any morphism $ f : p(z) \to p(y)$ such that $ p(\phi) \circ f = 
p(\psi)$ there exists a unique lift $ \chi : z \to y$ of $f$ such that
$\phi \circ \chi = \psi$.
\end{enumerate}
\end{definition}

\noindent
Condition (2) phrased differently says that 
applying the functor $p$ gives a bijection between the sets 
of dotted arrows in the following commutative diagram below:
$$
\xymatrix{
y \ar[r] & x & p(y) \ar[r] & p(x) \\
z \ar@{-->}[u] \ar[ru] & & p(z) \ar@{-->}[u]\ar[ru] & \\
}
$$

\smallskip\noindent
Another way to think about the second condition is the following.
Suppose that $g : W \to V$ and $f : V \to U$ are morphisms in $\mathcal{C}$. 
Let $x \in \text{Ob}(\mathcal{S}_U)$. By the first condition we can lift
$f$ to $ \phi : y \to x$ and then we can lift $g$ to $\psi : z \to y$.
Instead of doing this two step process we can directly lift $g \circ f$ to
$\gamma : z' \to x$. This gives the solid arrows in the diagram below.
$$
\xymatrix{
z' \ar@{-->}[d]\ar[rrd]^\gamma & & \\
z \ar@{-->}[u]\ar[r]^\psi & y \ar[r]^\phi & x \\
W \ar[r]^g & V \ar[r]^f & U \\
}
$$
Applying the second condition to the arrows $\phi \circ \phi$, $\gamma$
and $\text{id}_W$ we conclude that there is a unique morphism 
$\chi : z \to z'$ in $\mathcal{S}_W$ such that 
$\gamma \circ \chi = \phi \circ \psi$. Similarly there is a unique morphism
$z' \to z$. The uniqueness implies that the morphisms $z' \to z$ and
$z\to z'$ are mutually inverse, in other words isomorphisms.

\begin{example}
\label{example-group-homomorphism-fibreedingroupoids}
A homomorphism of groups $p : G \to H$ gives rise to a functor 
$p\colon \mathcal{S}\to\mathcal{C}$ as in Example 
\ref{example-group-homorphism-functor}. This functor
$p\colon \mathcal{S}\to\mathcal{C}$ is fibred in groupoids if and only if 
$p$ is surjective.  The fibre category $\mathcal{S}_{U}$ over the (unique)
object $U\in \text{Ob}(\mathcal{C})$ is the category associated to the
kernel of $p$ as in Example \ref{example-group-groupoid}.
\end{example}

\smallskip\noindent
Suppose that for every $f : V \to U$ and $x\in \text{Ob}(\mathcal{S}_U)$
as in the first condition we choose a lift
$f^\ast x \to x$ of $f$; this is possible by the axiom of choice. For
every morphism $\phi : x \to x'$ in $\mathcal{S}_U$ there is a unique
morphism $f^\ast \phi : f^\ast x \to f^\ast x'$ in $\mathcal{S}_V$
such that
$$
\xymatrix{
f^\ast x \ar[r]^{f^\ast \phi} \ar[d] & f^\ast x' \ar[d] \\
x \ar[r]^{\phi} & x' }
$$
commutes. Again uniqueness of this arrow guarantees that $f^\ast$ is a
functor $ f^\ast : \mathcal{S}_U \to \mathcal{S}_V$. 

\begin{lemma}
\label{lemma-fibred-groupoids}
If $p : \mathcal{S} \to \mathcal{C}$ is a category fibred in groupoids then
all fibre categories are groupoids. Choose functors $f^\ast$ as above.
Then for any pair of composable
morphisms $f : V \to U$, $g : U\to W$ there is a unique isomorphism of 
functors $\mathcal{S}_W \to \mathcal{S}_V$ 
$$
t : g^\ast f^\ast \to (g \circ f)^\ast 
$$ 
such that for every $y\in \text{Ob}(\mathcal{S}_W)$ the following
diagram commutes
\begin{equation}
\xymatrix{
f^\ast g^\ast y \ar[r] \ar[d]_{t_y} & g^\ast y \ar[d] \\
(f\circ g)^\ast y \ar[r] & y
}\label{eq:lemma-fibred-groupoids-commutes}
\end{equation}
\end{lemma}

\begin{proof} 
To show all fibre categories $\mathcal{S}_U$ for $U \in \text{Ob}(\mathcal{C})$
are groupoids, we must exhibit for every $f : y \to x$ in $\mathcal{S}_U$ an
inverse morphism.  The diagram on the left (in $\mathcal{S}_U$) is mapped by
$p$ to the diagram on the right:
$$
\xymatrix{
y \ar[r]^f & x & U \ar[r]^{id_U} & U \\
x \ar@{-->}[u] \ar[ru]_{id_x} & & U \ar@{-->}[u]\ar[ru]_{id_U} & \\
}
$$
Since only $id_U$ makes the diagram on the right commute, there is a unique
$g : x \to y$ making the diagram on the left commute, so $fg = id_x$.  By a
similar argument there is a unique $h : y \to x$ so that $gh = id_y$.  Then
$fgh = f : y \to x$.  We have $fg = id_x$, so $h=f$.

\smallskip\noindent
Now let $y\in \text{Ob}(\mathcal S_W)$ and consider the diagram
\begin{equation}\label{eq:lemma-fibred-groupoids-commutes2}
\xymatrix{
f^\ast g^\ast y \ar@{-->}[d]_{t_y} \ar[r] & g^\ast y \ar[r] & y \\
(g\circ f)^\ast y \ar[rru] & &
}
\xymatrix{
V\ar@{-->}[d]_{\text{id}_V} \ar[r]^f & U \ar[r]^g & W \\
V \ar[rru]_{g\circ f} & &
}
\end{equation}
The morphism $t_y \colon f^\ast g^\ast y \to (g\circ f)^\ast y$ is the
unique lift of of $\text{id}_V$ making
\ref{eq:lemma-fibred-groupoids-commutes2} (resp.\ 
\ref{eq:lemma-fibred-groupoids-commutes}) commute.  If $\phi\colon
y'\to y$ is a morphism in $\mathcal S_W$ the compositions $(f^\ast
g^\ast \phi) \circ t_y$ and $((g\circ f)^\ast \phi)\circ t_{y'}$ are
both lifts of $\text{id}_V$, so are equal making $t$ is a
transformation of functors.  Essentially the same construction applies
to give the inverse transformation $t^{-1}$, so $t$ is an isomorphism.
\end{proof}

\noindent
Conversely, given $p : \mathcal{S} \to \mathcal{C}$, we can ask: if the fibre
category $\mathcal{S}_U$ is a groupoid for all $U \in \text{Ob}(\mathcal{C})$,
must $\mathcal{S}$ be fibred in groupoids over $\mathcal{C}$? We can see the
answer is no as follows. Start with a category fibred in groupoids
$p : \mathcal{S} \to \mathcal{C}$. Altering the morphisms in $\mathcal{S}$
which do not map to the identity morphism on some object does not alter the 
categories $\mathcal{S}_U$. Hence we can violate the existence and uniqueness
conditions on lifts. One example is the functor from Example 
\ref{example-group-homomorphism-fibreedingroupoids} when $G \to H$ is not
surjective. Here is another example.

\begin{example}
Let $ \text{Ob}(\mathcal{C}) = \{A,B,T\}$ and 
$\text{Mor}_\mathcal{C}(A,B) = \{f\}$, $\text{Mor}_\mathcal{C}(B,T) = \{g\}$,
$\text{Mor}_\mathcal{C}(A,T) = \{h\} = \{gf\},$ plus the identity morphism for 
each object. See the diagram below for a picture of this category. Now let 
$\text{Ob}(\mathcal{S}) = \{A',B',T'\}$ and 
$\text{Mor}_\mathcal{S}(A',B') = \emptyset$,  
$\text{Mor}_\mathcal{S}(B',T') = \{g'\}$,  
$\text{Mor}_\mathcal{S}(A',T') = \{h'\},$ plus the identity morphisms. The 
functor $p : \mathcal{S} \to \mathcal{C}$ is obvious. Then for every 
$U \in \text{Ob}(\mathcal{C})$, $\mathcal{S}_U$ is the category with one 
object and the identity morphism on that object, so a groupoid, but the 
morphism $f: A \to B$ cannot be lifted. Similarly, if we declare 
$\text{Mor}_\mathcal{S}(A',B') = \{f'_1, f'_2\}$ and 
$ \text{Mor}_\mathcal{S}(A',T') = \{h'\} = \{g'f'_1 \} = \{g'f'_2\}$, then 
the fibre categories are the same and $f: A \to B$ in the diagram below has 
two lifts. 
$$
\xymatrix{
B' \ar[r]^{g'} & T' &  & B \ar[r]^g & T & \\
A' \ar@{-->}[u]^{??} \ar[ru]_{h'} & & \ar@{}[u]^{above} &
A \ar[u]^f \ar[ru]_{gf = h} & \\
}
$$ 
\end{example}

\noindent
Later we would like to make assertions such as ``any category fibred in
groupoids over $\mathcal{C}$ is equivalent to a split one'', or
``any category fibred in groupoids whose fibre categories are setlike
is equivalent to a category fibred in sets''. The notion of equivalence
depends on the $2$-category we are working with. To make sure
that everybody knows what we are talking about we define the
$2$-category of categories over $\mathcal{C}$.

\begin{definition}
\label{definition-categories-over-C}
The $2$-category of categories over $\mathcal{C}$ is defined
as follows. Its objects will be functors 
$p : \mathcal{S} \to \mathcal{C}$ (belonging to
some set, see Sets, \autoref{sets-section-reflection-principle}). Its 
$1$-morphisms will be functors $G : \mathcal{S} \to \mathcal{S}'$
such that $p' \circ G = p$, and its $2$-morphisms $t : G \to H$
will be morphisms of functors such that $p'(t_x) = \text{id}_{p(x)}$
for all $x \in \text{Ob}(\mathcal{S})$.
\end{definition}

\noindent
The $2$-category of categories fibred in groupoids over $\mathcal{C}$
is the full sub-$2$-category of this $2$-category whose objects
are categories fibred in groupoids.

\begin{lemma}
\label{lemma-equivalence-fibred-categories}
Let $p\colon \mathcal{S}\to \mathcal{C}$ and 
$p'\colon \mathcal{S'}\to \mathcal{C}$ be categories fibred in groupoids, and
suppose that $G\colon \mathcal{S}\to \mathcal {S}'$ is a functor over 
$\mathcal{C}$.  Then $G$ is fully faithful (resp.\ an equivalence) if and only
if for each $U\in\text{Ob}(\mathcal{C})$ the induced functor 
$G_U\colon \mathcal{S}_U\to \mathcal{S}'_U$ is fully faithful (resp.\ an
equivalence).
\end{lemma}

\begin{proof}
Clearly if $G$ is fully faithful (resp.\ an equivalence) then so is $G_U$. So
suppose that $G_U$ is fully faithful for all $U\in\text{Ob}(\mathcal C)$. To
show that $G$ is fully faithful we have to show for any objects
$x,y\in\text{Ob}(\mathcal{S})$ that $G$ induces a bijection between
$\text{Mor}_{\mathcal{S}}(x,y)$ and $\text{Mor}_{\mathcal{S}'}(G(x),G(y))$. 
To this end let $\phi'\colon G(x)\to G(y)$ and set $U=p(x)$ and $V=p(y)$.
As $\mathcal{S}$ is fibred in groupoids there is a lift $z\to y$ of 
$p'(\phi')$ in $\mathcal{S}$, and any morphisms $x\to y$ factors uniquely
as $x\to z\to y$, where the map $x\to z$ lifts $\text{id}_U$, as in the
following diagram
$$
\xymatrix{
x \ar@{-->}[d] \ar[rd]  \\
z \ar[r]^\psi \ar[d] & y \ar[d] \\
U \ar[r]^{p'(\phi')} &V}
$$ 
Now in $\mathcal{S}'$,  $G(\psi)\colon G(z)\to G(y)$ is the pullback of
$G(y)$, so any morphism $G(x)\to G(y)$ factors uniquely
as $G(x)\to G(z)\to G(y)$, where again the map
$G(x)\to G(z)$ lifts $\text{id}_U$.  Since $G_U$
induces a bijection between $\text{Mor}_{\mathcal{S}_U}(x,z)$ and
$\text{Mor}_{\mathcal{S}'_U}(G(x),G(z))$ we get that
$G$ induces a bijection between $\text{Mor}_{\mathcal{S}}(x,y)$
and $\text{Mor }_{\mathcal{S}'}(G(x),G(y))$, hence $G$
is fully faithful.

\smallskip\noindent
Finally suppose for all $G_U$ is an equivalence for all $U$, so it is
fully faithful and essentially surjective.  We have seen this implies $G$ is
fully faithful, and thus to prove it is an equivalence we have to prove that
it is essentially surjective.  This is clear, for if $z'\in
\text{Ob}(\mathcal{S}')$ then $z'\in \text{Ob}(\mathcal{S}'_U)$ where
$U=p'(z')$.  Since $G_U$ is essentially surjective we know that
$z'$ is isomorphic, in $\mathcal{S}'_U$, to an object of the form
$G_U(z)$ for some $z\in \text{Ob}(\mathcal{S}_U)$.  But morphisms
in $\mathcal{S}'_U$ are morphisms in $\mathcal{S}'$ and hence $z'$ is
isomorphic to $G(z)$ in $\mathcal{S}'$.
\end{proof}

\begin{lemma}
\label{lemma-2-product-categories-over-C} The 2-category of categories
over $\mathcal{C}$ has 2-fibre products. Suppose that
$f : \mathcal{X} \to \mathcal{S}$ and
$g : \mathcal{Y} \to \mathcal{S}$ are morphisms of categories over
$\mathcal{C}$. An explicit 2-fibre product
$\mathcal{X} \times_\mathcal{S}\mathcal{Y}$ is given by the following
description
\begin{enumerate}
\item an object of $\mathcal{X}\times_\mathcal{S} \mathcal{Y}$ is a quadruple
$(U,x,y,f)$, where $U \in \text{Ob}(\mathcal{C})$,
$x\in \text{Ob}(\mathcal{X}_U)$, $y\in \text{Ob}(\mathcal{Y}_U)$,
and $f : F(x) \to G(y)$ is an isomorphism in $\mathcal{S}_U$,
\item a morphism $(U,x,y,f) \to (U',x',y', f')$ is given by a pair $(a,b)$,
where $a : x \to x'$ is a morphism in $\mathcal{X}$, and $b : y \to y'$ is a
morphism in $\mathcal{Y}$ such that (1) $a$ and $b$ induced the same
morphism $U \to U'$, and (2) the diagram 
$$
\xymatrix{
F(A) \ar[r]^f \ar[d]^{F(a)} & G(B) \ar[d]^{G(b)} \\
F(A') \ar[r]^{f'} & G(B')
}
$$
is commutative.
\end{enumerate}
The functors $p : \mathcal{X}\times_\mathcal{S}\mathcal{Y} \to \mathcal{X}$
and $q : \mathcal{X}\times_\mathcal{S}\mathcal{Y} \to \mathcal{Y}$ are the
forgetfull functors in this case. The transformation $\psi : F \circ p \to
G \circ q$ is given on the object $\xi = (U,x,y,f)$ by
$\psi_\xi = f : F(p(\xi)) = F(x) \to G(y) = G(q(\xi))$.
\end{lemma}

\begin{proof}
Let us check the universal property: let $p_W : \mathcal{W}\to \mathcal{C}$
be a category over $\mathcal{C}$, let $X : \mathcal{W} \to \mathcal{X}$ and
$Y : \mathcal{W} \to \mathcal{Y}$ be functors over $\mathcal{C}$, and let
$t : F \circ X \to G \circ Y$ be an isomorphism of functors.
The desired functor
$\gamma : \mathcal{W} \to \mathcal{A}\times_\mathcal{C}\mathcal{B}$
is given by $W \mapsto (p_W(W), X(W), Y(W), t_W)$. What else could it be? 
(A meta-argument for uniqueness.) FIXME: write this out.
\end{proof}

\begin{lemma}
\label{lemma-2-product-fibred-categories}
In the situation of the lemma above, if $\mathcal{X}$, $\mathcal{Y}$ and 
$\mathcal{S}$ are fibred in groupoids over $\mathcal{C}$, then so is
$\mathcal{X}\times_\mathcal{S}\mathcal{Y}$. In particular the 2-category
of categories fibred in groupoids over $\mathcal{C}$ has 2-fibre products
(and they are described as above).
\end{lemma}

\begin{proof} 
FIXME.
\end{proof}

\subsection{Categories fibred in sets}
\label{subsection-fibred-in-sets}

\noindent
Let us call a category setlike if it is a groupoid where every object
has exactly one automorphism: the identity. If $C$ is a set with an 
equivalence relation $\sim$, then we can make a setlike category
$\mathcal{C}$ as follows: $\text{Ob}(\mathcal{C}) = C$ and 
$\text{Mor}_\mathcal{C}(x,y) = \emptyset$ unless $x \sim y$ in which
case we set $\text{Mor}_\mathcal{C}(x,y) = \{1\}$. Transitivity of
$\sim$ means that we can compose morphisms. Conversely any setlike
category defines an equivalence relation on its objects (isomorphism)
such that you recover the category (up to unique isomorphism -- not
equivalence) from the procedure just described. This is why these categories
are sometimes simply called equivalence relations.

\smallskip\noindent
A category is called discrete if the only morphisms are the identity 
morphisms. Sometimes discrete categories are called sets (reasons as above).
Discrete categories are setlike. For any setlike category $\mathcal{C}$
there is a canonical procedure to make a discrete category equivalent to it,
namely one replaces $\text{Ob}(\mathcal{C})$ by the set of isomorphism
classes, and adds identity morphisms.

\begin{definition}
\label{definition-category-fibred-sets}
A category fibred in groupoids $p : \mathcal{S} \to \mathcal{C}$ is said
to be a category fibred in sets if all fibre categories are discrete.
\end{definition}

\noindent
We discuss briefly the relationship between categories fibred in sets
and presheaves (see Sites, \hyperref[sites-definition-presheaf]%
{Definition~\ref*{sites-definition-presheaf}}). Suppose that $p :
\mathcal{S} \to \mathcal{C}$ is fibred in sets. Let $f : V \to U$
be a morphism in $\mathcal{C}$ and let $x \in \text{Ob}(\mathcal{S}_U)$.
Then there is exactly one choice for the object $f^\ast x$. Thus we see that
$(f \circ g)^\ast x = g^\ast(f^\ast x)$ for $f,g$ as in Lemma
\ref{lemma-fibred-groupoids}. It follows that we may think of the
assigments $U \mapsto \text{Ob}(\mathcal{S}_U)$ and $f \mapsto f^\ast$
as a presheaf on $\mathcal{C}$.

\smallskip\noindent
Conversely, given a presheaf of sets
$F : \mathcal{C}^{\text{opp}} \to \text{Sets}$
we can construct a category $\mathcal{S}_F$ fibred in sets
over $\mathcal{C}$ by taking as fibre category $\mathcal{S}_{F,U}$ 
the discrete category whose underlying set is $F(U)$. This is explained
more generally, and in more detail in Example \ref{example-functor-groupoids}
below. Also, here is an important example.

\begin{example}
\label{example-fibred-category-from-functor-of-points}
In this example $F = h_X = \text{Mor}(-,X)$ for some
$X \in \text{Ob}(\mathcal{C})$ (see Example \ref{example-hom-functor}).
In other words, $F$ is a representable presheaf.
Since $\mathcal{S}_{F,U}$ is the discrete category whose objects are the
morphisms from $U$ into $X$ it follows that
$\mathcal{S}_F\to \mathcal{C}$ is the functor denoted
$\mathcal{C}/X \to \mathcal{C}$ from
Example \ref{example-comma-category}.
FIXME. Improve formulation.
\end{example}

\smallskip\noindent
For this reason it is tempting to define a ``representable'' object in the
2-category of categories fibred in groupoids to be a category fibred in
sets whose associated presheaf is representable. However, this is would not
be a good definition since we prefer to have a notion wich is invariant under
equivalences. Thus we consider first which categories in groupoids are
equivalent to categories fibred in sets.

\begin{lemma}
\label{lemma-setlike-fibres}
Suppose that $p : \mathcal{S} \to \mathcal{C}$ is a category fibred in
groupoids all of whose fibre categories $\mathcal{S}_U$ are setlike. 
Then there exists a category fibred in sets $p' : \mathcal{S}' \to
\mathcal{C}$ and an equivalence
$\text{can}:\mathcal{S} \to \mathcal{S}'$ of categories over $\mathcal{C}$.
The 1-morphism $\mathcal{S}\to\mathcal{S}'$ is unique up to a unique
2-morphism. It further has the property that
$$
\text{Ob}(\mathcal{S}_U) \longrightarrow \text{Ob}(\mathcal{S}'_U) 
$$
(induced by $\text{can}$) identifies the RHS with ismorphism classes of the
LHS for all $U \in \text{Ob}(\mathcal{C})$. The 1-morphism
$\mathcal{S}\to\mathcal{S}'$ is unique up to a unique 2-morphism. 

\smallskip\noindent
Conversely, any category fibred in groupoids over $\mathcal{C}$ which
is equivalent (as a category over $\mathcal{C}$) to a category fibred 
in sets, has setlike fibre categories.
\end{lemma}

\begin{proof}
An object of the category $\mathcal{S}'$ will be a pair $(U, \xi)$, where
$U \in \text{Ob}(\mathcal{C})$ and $\xi$ is an isomorphism class of objects
of $\mathcal{S}_U$. A morphism $(U,\xi) \to (V , \psi)$ is given by a 
morphism $x \to y$, where $x \in \xi$ and $y \in \psi$. Here we identify
two morphisms $x \to y$ and $x' \to y'$ if they induce the same morphism
$U \to V$, and if for some choices of isomorphisms $x \to x'$ in
$\mathcal{S}_U$ and $y \to y'$ in $\mathcal{S}_V$ the compositions
$x \to x' \to y'$ and $x \to y \to y'$ agree. By construction there are
surjective maps on objects and morphisms from $\mathcal{S} \to
\mathcal{S}'$. We define composition of morphisms in $\mathcal{S}'$ to
be the unique law that turns $\mathcal{S} \to \mathcal{S}'$ into a functor.
FIXME: check this is well-defined. 

\smallskip\noindent
By construction the rule $(U,\xi) \mapsto U$ is a functor. FIXME: check this
and the other properties.
\end{proof}

\noindent
With this lemma in hand it is easy to recognize those categories over
$\mathcal{C}$ which are equivalent to a category fibred in sets. Thus we
now make the following definition.

\begin{definition}
\label{definition-representable-fibred-category}
A category fibred in groupoids $p : \mathcal{S} \to \mathcal{C}$ is
called representable, if the following conditions are satisfied:
\begin{enumerate}
\item all fibre categories $\mathcal{S}_U$ are setlike, and
\item the presheaf $U \mapsto \text{Ob}(\mathcal{S}_U)/\cong$ is 
representable.
\end{enumerate}
\end{definition}

\noindent
In this case, by Lemma \ref{lemma-setlike-fibres} the category 
$\mathcal{S}'$ is isomorphic to $\mathcal{C}/X$ over $\mathcal{C}$.
As usual, by the Yoneda lemma the pair $(X,j)$, where $j$ is the
equivalence $j : \mathcal{S} \to \mathcal{C}/X$ is uniquely determined
up to isomorphism.

\begin{lemma}
\label{lemma-2-product-categories-fibred-sets}
The 2-category of categories fibred in sets over $\mathcal{C}$
has 2-fibre products. More precisely, the 2-fibre product described in 
Lemma \ref{lemma-2-product-categories-over-C} returns a category fibred in
sets if one starts out with such. A similar result holds for categories
fibred in groupoids all of whose fibre categories are setlike.
\end{lemma}

\begin{proof}
FIXME.
\end{proof}

\subsection{Presheaves of groupoids}
\label{subsection-presheaves-groupoids}

\noindent
In this subsection we compare the notion of categories fibred in groupoids
with the closely related notion of a ``presheaf of groupoids''. The basic
construction is explained in the following example.

\begin{example}
\label{example-functor-groupoids}
Suppose that $F : \mathcal{C} \to \text{Groupoids}$ is a contravariant functor
to the category of groupoids (see 
\hyperref[remark-functor-into-sets]{Remark~\ref*{remark-functor-into-sets}} and
\hyperref[remarks-functor-into-2-category]%
{Remark~\ref{remarks-functor-into-2-category}}). 
For $f : V \to U$ in $\mathcal{C}$ we will
suggestively write $F(f) = f^\ast$ for the functor from $F(U)$ to $F(V)$. 
From this we can construct a category fibred in groupoids over $\mathcal{C}$ 
as follows. Define 
$$
\text{Ob}(\mathcal{S}) =
\{(U,x) \mid U\in \text{Ob}(\mathcal{C}), x\in \text{Ob}(F(U)\}.
$$ 
For $(U,x), (V,y) \in \text{Ob}(\mathcal{S})$ we define
$$
\text{Mor}_\mathcal{S}((V,y),(U,x)) = 
\{ (f, \phi) \mid f\in \text{Mor}_\mathcal{C}(V,U), 
\phi \in \text{Mor}_{F(V)}(y, f^\ast x)\}.
$$
In order to define composition we use that $g^\ast \circ f^\ast = 
(f \circ g)^\ast$ for a pair of composable morphisms of $\mathcal{C}$
(by definition of a functor into a $2$-category).
Namely, we define the composition of $\psi : z \to g^\ast y$ and 
$ \phi : y \to f^\ast x$ to be $ g^\ast(\phi) \circ \psi$. It is clear
what the functor $p : \mathcal{S} \to \mathcal{C}$ is. The condition
that $F(U)$ is a groupoid for every $U$ guarantees that $\mathcal{S}$ is
fibred in groupoids over $\mathcal{C}$. Lifts of morphisms exist: given 
$f: V \to U$ in $\mathcal{C}$ and $(U,x)$ a lift of $U$, then 
$(f, id_{f^\ast x}): (V, {f^\ast x}) \to (U,x)$ is a lift of $f$. 
Uniqueness means $h$ in the diagram on the left determines $(h,\nu)$ on 
the right:
$$
\xymatrix{
V \ar[r]^f & U & (V,y) \ar[r]^{(f, \phi)} & (U,x) \\
W \ar@{-->}[u]^h \ar[ru]_g & &
(W,z) \ar@{-->}[u]^{(h,\nu)} \ar[ru]_{(g, \psi)} & \\
}
$$
Then $\nu = (h^\ast \phi)^{-1} \circ \psi $ and the uniqueness of inverses
guarantees this is the only lift making the diagram commute.

\noindent
We will write $\mathcal{S}_F \to \mathcal{C}$ for the resulting functor
if we want to indicate the dependence on $F$. Because we can think of 
objects of $\mathcal{S}_F$ as pairs $(U,x)$, we sometimes say $\mathcal{S}_F$ 
is a {\it split} category fibred in groupoids.
\end{example}

\begin{lemma} 
\label{lemma-fibred-strict}
Let $ p : \mathcal{S} \to \mathcal{C}$ be a category fibred in groupoids.
There exists a functor $F : \mathcal{C} \to \text{Groupoids}$ such that 
$\mathcal{S}$ is equivalent to $\mathcal{S}_F$ over $\mathcal{C}$. In other 
words, every category fibred in groupoids is equivalent to a split one.
\end{lemma}

\begin{proof} 
We construct a new category $\mathcal{S}'$ as follows. First we choose 
pullback functors $g^\ast : \mathcal{S}_V \to \mathcal{S}_{V'}$ for any 
morphism $g : V' \to V$ of $\mathcal{C}$. (We can do this since 
$\mathcal{S}$, $\mathcal{C}$ are sets. FIXME: We can do this proof without
choosing these as well.) The objects of $\mathcal{S}'$ 
are pairs $(x,f)$ consisting of a morphism $f : V \to U$ of $\mathcal{C}$
and an object $x$ of $\mathcal{S}$ over $U$, i.e., 
$x\in \text{Ob}(\mathcal{S}_U)$. The functor 
$p' : \mathcal{S}' \to \mathcal{C}$ will map the pair $(x,f)$ to the source 
of the morphism $f$, in other words $p'(x,f:V\to U) = V$. A morphism 
$\varphi : (x_1,f_1: V_1 \to U_1) \to (x_2, f_2 : V_2 \to U_2)$ is given by a 
pair $(\varphi,g)$ consisting of a morphism $g : V_1 \to V_2$ and a morphism 
$\varphi : f_1^\ast x_1 \to f_2^\ast x_2$ with $p(\varphi) = g$. It is no 
problem to define the composition law: $(\varphi,g) \circ (\psi,h) = 
(\varphi \circ \psi, g\circ h)$ for any pair of composable morphisms. 
There is a natural functor $\mathcal{S} \to \mathcal{S}'$ which simply maps
$x$ over $U$ to the pair $(x, \text{id}_x)$.

\smallskip\noindent
FIXME. We need to check that $p'$ makes $\mathcal{S}'$ into a category
fibred in groupoids over $\mathcal{C}$, and we need to check that 
$\mathcal{S} \to \mathcal{S}'$ is an equivalence of categories over 
$\mathcal{C}$ (hopefully the lemma above helps!). 

\smallskip\noindent
Finally, we can define pullback functors on $\mathcal{S}'$ 
by setting $g^\ast(x,f) = (x, f \circ g)$ on objects if $g : V' \to V$ and
$f : V \to U$. On morphisms $(\varphi,\text{id}_V) : (x_1, f_1) \to (x_2,f_2)$
between morphisms in $\mathcal{S}'_V$ we set $g^\ast(\varphi,\text{id}_V) =
(g^\ast\varphi, \text{id}_{V'})$ where we use the unique identifications
$g^\ast f_i^\ast x_i = (f_i \circ g)^\ast x_i$ from Lemma 
\ref{lemma-fibred-groupoids} to think of $g^\ast\varphi$ as a morphism from
$(f_1 \circ g)^\ast x_1$ to $(f_2 \circ g)^\ast x_2$. Clearly, these pullback
functors $g^\ast$ have the property that
$g_1^\ast \circ g_2^\ast = (g_2\circ g_1)^\ast$, in other words $\mathcal{S}'$
is split as desired.
\end{proof}

\begin{proof}[Alternate proof]
We define a contravariant functor $F$ from $\mathcal{C}$ to the
category of groupoids as follows: for $U\in \text{Ob}(\mathcal{C})$
set $F(U) = \text{Mor}(\mathcal{S}/U,\mathcal{S})$ to be the set of
base preserving natural transformations.  If $f\colon U\to V$ the
induced functor $\mathcal{S}/U\to \mathcal{S}/V$ induces the
morphism $F(f)\colon F(V)\to F(U)$.  Clearly $F$ is a functor, and
we will see below that it is a functor into groupoids.  Let
$\mathcal{S}'$ be the associated category fibred in groupoids from Example
\ref{example-functor-groupoids}.

\smallskip\noindent
There is an obvious functor $G\colon \mathcal{S}'\to \mathcal{S}$
over $\mathcal{C}$ given by taking the pair $(U,x)$, where
$U\in\text{Ob}(\mathcal{C})$ and $x\in F(U)$, to
$x(U\stackrel{\text{id}_U}{\to} U) \in \mathcal{S}$.  Now Lemma
\ref{lemma-yoneda-2category} implies that for each $U$,
$$
G_U\colon \mathcal{S}'_U = F(U)= 
\text{Mor}(\mathcal{C}/U,\mathcal{S}) \to \mathcal{S}_U
$$
is an equivalence, and thus $G$ equivalence between $\mathcal{S}$ and
$\mathcal{S}'$ by Lemma \ref{lemma-equivalence-fibred-categories}.
\end{proof}

\begin{lemma}
\label{lemma-yoneda-2category}
Let $\mathcal{S}\to \mathcal{C}$ be fibred in groupoids.  Then for any
$U\in \text{Ob}(\mathcal{C})$ the functor
$$
G\colon \text{Mor}(\mathcal{C}/U,\mathcal{S}) \to \mathcal{S}_U
$$
given by $G(x) = x(U\stackrel{\text{id}_U}{\to} U)$ is an equivalence.
\end{lemma}

\noindent
FIXME: Do we have notation for base preserving transformations already?
Say what $G$ does on arrows.

\begin{proof}
We define a functor $H\colon \mathcal{S}_U \to
\text{Mor}(\mathcal{C}/U,\mathcal{S})$ as follows.  Given $x\in
\text{Ob}(\mathcal{S}_U)$ and $f\colon X\to U$ set $H(x)(f) = f^*x$.
(FIXME: say what this does on arrows and prove this gives an
equivalence).
\end{proof}
 
\noindent {\bf Biographical notes:} Parts of this have been taken from
Vistoli's notes \cite{Vis2}.

\smallskip\noindent
To continue reading, 
\begin{enumerate}

\item visit the next section: Sites,
\autoref{sites-section-introduction}, or 

\item go back to the
table of contents: \url{index.html#contents}.

\end{enumerate}

\bibliographystyle{alpha}
\bibliography{my}

\end{document}
