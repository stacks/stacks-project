\IfFileExists{stacks-project.cls}{%
\documentclass{stacks-project}
}{%
\documentclass{amsart}
}

% The following AMS packages are automatically loaded with
% the amsart documentclass:
%\usepackage{amsmath}
%\usepackage{amssymb}
%\usepackage{amsthm}

% For dealing with references we use the comment environment
\usepackage{verbatim}
\newenvironment{reference}{\comment}{\endcomment}
%\newenvironment{reference}{}{}
\newenvironment{slogan}{\comment}{\endcomment}
\newenvironment{history}{\comment}{\endcomment}

% For commutative diagrams you can use
% \usepackage{amscd}
\usepackage[all]{xy}

% We use 2cell for 2-commutative diagrams.
\xyoption{2cell}
\UseAllTwocells

% To put source file link in headers.
% Change "template.tex" to "this_filename.tex"
% \usepackage{fancyhdr}
% \pagestyle{fancy}
% \lhead{}
% \chead{}
% \rhead{Source file: \url{template.tex}}
% \lfoot{}
% \cfoot{\thepage}
% \rfoot{}
% \renewcommand{\headrulewidth}{0pt}
% \renewcommand{\footrulewidth}{0pt}
% \renewcommand{\headheight}{12pt}

\usepackage{multicol}

% For cross-file-references
\usepackage{xr-hyper}

% Package for hypertext links:
\usepackage{hyperref}

% For any local file, say "hello.tex" you want to link to please
% use \externaldocument[hello-]{hello}
\externaldocument[introduction-]{introduction}
\externaldocument[conventions-]{conventions}
\externaldocument[sets-]{sets}
\externaldocument[categories-]{categories}
\externaldocument[topology-]{topology}
\externaldocument[sheaves-]{sheaves}
\externaldocument[sites-]{sites}
\externaldocument[stacks-]{stacks}
\externaldocument[fields-]{fields}
\externaldocument[algebra-]{algebra}
\externaldocument[brauer-]{brauer}
\externaldocument[homology-]{homology}
\externaldocument[derived-]{derived}
\externaldocument[simplicial-]{simplicial}
\externaldocument[more-algebra-]{more-algebra}
\externaldocument[smoothing-]{smoothing}
\externaldocument[modules-]{modules}
\externaldocument[sites-modules-]{sites-modules}
\externaldocument[injectives-]{injectives}
\externaldocument[cohomology-]{cohomology}
\externaldocument[sites-cohomology-]{sites-cohomology}
\externaldocument[dga-]{dga}
\externaldocument[dpa-]{dpa}
\externaldocument[hypercovering-]{hypercovering}
\externaldocument[schemes-]{schemes}
\externaldocument[constructions-]{constructions}
\externaldocument[properties-]{properties}
\externaldocument[morphisms-]{morphisms}
\externaldocument[coherent-]{coherent}
\externaldocument[divisors-]{divisors}
\externaldocument[limits-]{limits}
\externaldocument[varieties-]{varieties}
\externaldocument[topologies-]{topologies}
\externaldocument[descent-]{descent}
\externaldocument[perfect-]{perfect}
\externaldocument[more-morphisms-]{more-morphisms}
\externaldocument[flat-]{flat}
\externaldocument[groupoids-]{groupoids}
\externaldocument[more-groupoids-]{more-groupoids}
\externaldocument[etale-]{etale}
\externaldocument[chow-]{chow}
\externaldocument[intersection-]{intersection}
\externaldocument[pic-]{pic}
\externaldocument[adequate-]{adequate}
\externaldocument[dualizing-]{dualizing}
\externaldocument[duality-]{duality}
\externaldocument[discriminant-]{discriminant}
\externaldocument[local-cohomology-]{local-cohomology}
\externaldocument[curves-]{curves}
\externaldocument[resolve-]{resolve}
\externaldocument[models-]{models}
\externaldocument[pione-]{pione}
\externaldocument[etale-cohomology-]{etale-cohomology}
\externaldocument[proetale-]{proetale}
\externaldocument[crystalline-]{crystalline}
\externaldocument[spaces-]{spaces}
\externaldocument[spaces-properties-]{spaces-properties}
\externaldocument[spaces-morphisms-]{spaces-morphisms}
\externaldocument[decent-spaces-]{decent-spaces}
\externaldocument[spaces-cohomology-]{spaces-cohomology}
\externaldocument[spaces-limits-]{spaces-limits}
\externaldocument[spaces-divisors-]{spaces-divisors}
\externaldocument[spaces-over-fields-]{spaces-over-fields}
\externaldocument[spaces-topologies-]{spaces-topologies}
\externaldocument[spaces-descent-]{spaces-descent}
\externaldocument[spaces-perfect-]{spaces-perfect}
\externaldocument[spaces-more-morphisms-]{spaces-more-morphisms}
\externaldocument[spaces-flat-]{spaces-flat}
\externaldocument[spaces-groupoids-]{spaces-groupoids}
\externaldocument[spaces-more-groupoids-]{spaces-more-groupoids}
\externaldocument[bootstrap-]{bootstrap}
\externaldocument[spaces-pushouts-]{spaces-pushouts}
\externaldocument[groupoids-quotients-]{groupoids-quotients}
\externaldocument[spaces-more-cohomology-]{spaces-more-cohomology}
\externaldocument[spaces-simplicial-]{spaces-simplicial}
\externaldocument[formal-spaces-]{formal-spaces}
\externaldocument[restricted-]{restricted}
\externaldocument[spaces-resolve-]{spaces-resolve}
\externaldocument[formal-defos-]{formal-defos}
\externaldocument[defos-]{defos}
\externaldocument[cotangent-]{cotangent}
\externaldocument[examples-defos-]{examples-defos}
\externaldocument[algebraic-]{algebraic}
\externaldocument[examples-stacks-]{examples-stacks}
\externaldocument[stacks-sheaves-]{stacks-sheaves}
\externaldocument[criteria-]{criteria}
\externaldocument[artin-]{artin}
\externaldocument[quot-]{quot}
\externaldocument[stacks-properties-]{stacks-properties}
\externaldocument[stacks-morphisms-]{stacks-morphisms}
\externaldocument[stacks-limits-]{stacks-limits}
\externaldocument[stacks-cohomology-]{stacks-cohomology}
\externaldocument[stacks-perfect-]{stacks-perfect}
\externaldocument[stacks-introduction-]{stacks-introduction}
\externaldocument[stacks-more-morphisms-]{stacks-more-morphisms}
\externaldocument[stacks-geometry-]{stacks-geometry}
\externaldocument[moduli-]{moduli}
\externaldocument[moduli-curves-]{moduli-curves}
\externaldocument[examples-]{examples}
\externaldocument[exercises-]{exercises}
\externaldocument[guide-]{guide}
\externaldocument[desirables-]{desirables}
\externaldocument[coding-]{coding}
\externaldocument[obsolete-]{obsolete}
\externaldocument[fdl-]{fdl}
\externaldocument[index-]{index}

% Theorem environments.
%
\theoremstyle{plain}
\newtheorem{theorem}[subsection]{Theorem}
\newtheorem{proposition}[subsection]{Proposition}
\newtheorem{lemma}[subsection]{Lemma}

\theoremstyle{definition}
\newtheorem{definition}[subsection]{Definition}
\newtheorem{example}[subsection]{Example}
\newtheorem{exercise}[subsection]{Exercise}
\newtheorem{situation}[subsection]{Situation}

\theoremstyle{remark}
\newtheorem{remark}[subsection]{Remark}
\newtheorem{remarks}[subsection]{Remarks}

\numberwithin{equation}{subsection}

% Macros
%
\def\lim{\mathop{\rm lim}\nolimits}
\def\colim{\mathop{\rm colim}\nolimits}
\def\Spec{\mathop{\rm Spec}}
\def\Hom{\mathop{\rm Hom}\nolimits}
\def\Ext{\mathop{\rm Ext}\nolimits}
\def\SheafHom{\mathop{\mathcal{H}\!{\it om}}\nolimits}
\def\SheafExt{\mathop{\mathcal{E}\!{\it xt}}\nolimits}
\def\Sch{\textit{Sch}}
\def\Mor{\mathop{\rm Mor}\nolimits}
\def\Ob{\mathop{\rm Ob}\nolimits}
\def\Sh{\mathop{\textit{Sh}}\nolimits}
\def\NL{\mathop{N\!L}\nolimits}
\def\proetale{{pro\text{-}\acute{e}tale}}
\def\etale{{\acute{e}tale}}
\def\QCoh{\textit{QCoh}}
\def\Ker{\mathop{\rm Ker}}
\def\Im{\mathop{\rm Im}}
\def\Coker{\mathop{\rm Coker}}
\def\Coim{\mathop{\rm Coim}}

%
% Macros for moduli stacks/spaces
%
\def\QCohstack{\mathcal{QC}\!{\it oh}}
\def\Cohstack{\mathcal{C}\!{\it oh}}
\def\Spacesstack{\mathcal{S}\!{\it paces}}
\def\Quotfunctor{{\rm Quot}}
\def\Hilbfunctor{{\rm Hilb}}
\def\Curvesstack{\mathcal{C}\!{\it urves}}
\def\Polarizedstack{\mathcal{P}\!{\it olarized}}
\def\Complexesstack{\mathcal{C}\!{\it omplexes}}
% \Pic is the operator that assigns to X its picard group, usage \Pic(X)
% \Picardstack_{X/B} denotes the Picard stack of X over B
% \Picardfunctor_{X/B} denotes the Picard functor of X over B
\def\Pic{\mathop{\rm Pic}\nolimits}
\def\Picardstack{\mathcal{P}\!{\it ic}}
\def\Picardfunctor{{\rm Pic}}
\def\Deformationcategory{\mathcal{D}\!{\it ef}}


% OK, start here.
%
\begin{document}

\title{Moduli Stacks}

\maketitle

\phantomsection
\label{section-phantom}

\tableofcontents




\section{Introduction}
\label{section-introduction}

\noindent
In this chapter we verify basic properties of moduli spaces
and moduli stacks such as
$\mathit{Hom}$, $\mathit{Isom}$, $\textit{Coh}_{X/B}$,
$\text{Quot}_{\mathcal{F}/X/B}$, $\text{Hilb}_{X/B}$,
$\textit{Pic}_{X/B}$, $\text{Pic}_{X/B}$, $\mathit{Mor}_B(Z, X)$,
$\textit{Spaces}'_{fp, flat, proper}$, $\textit{Polarized}$, and
$\textit{Complexes}_{X/B}$.
We have already shown these algebraic spaces or algebraic stacks
under suitable hypotheses, see Quot, Sections
\ref{quot-section-hom},
\ref{quot-section-isom},
\ref{quot-section-stack-coherent-sheaves},
\ref{quot-section-not-flat},
\ref{quot-section-quot},
\ref{quot-section-hilb},
\ref{quot-section-picard-stack},
\ref{quot-section-picard-functor},
\ref{quot-section-relative-morphisms},
\ref{quot-section-stack-of-spaces},
\ref{quot-section-polarized}, and
\ref{quot-section-moduli-complexes}.
The stack of curves, denoted $\textit{Curves}$ and introduced
in Quot, Section \ref{quot-section-curves}, is discussed in the
chapter on moduli of curves, see
Moduli of Curves, Section \ref{moduli-curves-section-stack-curves}.

\medskip\noindent
In some sense this chapter is following the footsteps of
Grothendieck's lectures \cite{Gr-I},
\cite{Gr-II},
\cite{Gr-III},
\cite{Gr-IV},
\cite{Gr-V}, and
\cite{Gr-VI}.







\section{Conventions and abuse of language}
\label{section-conventions}

\noindent
We continue to use the conventions and the abuse of language
introduced in
Properties of Stacks, Section \ref{stacks-properties-section-conventions}.
Unless otherwise mentioned our base scheme will be $\Spec(\mathbf{Z})$.






\section{Properties of Hom and Isom}
\label{section-hom-isom}

\noindent
Let $f : X \to B$ be a morphism of algebraic spaces which is
of finite presentation. Assume $\mathcal{F}$ and $\mathcal{G}$
are quasi-coherent $\mathcal{O}_X$-modules.
If $\mathcal{G}$ is of finite presentation, flat over $B$
with support proper over $B$, then the functor
$\mathit{Hom}(\mathcal{F}, \mathcal{G})$ defined by
$$
T/B \longmapsto \Hom_{\mathcal{O}_{X_T}}(\mathcal{F}_T, \mathcal{G}_T)
$$
is an algebraic space affine over $B$. If $\mathcal{F}$ is of
finite presentation, then
$\mathit{Hom}(\mathcal{F}, \mathcal{G}) \to B$
is of finite presentation. See
Quot, Proposition \ref{quot-proposition-hom}.

\medskip\noindent
If both $\mathcal{F}$ and $\mathcal{G}$ are of finite presentation,
flat over $B$ with support proper over $B$, then the subfunctor
$$
\mathit{Isom}(\mathcal{F}, \mathcal{G}) \subset
\mathit{Hom}(\mathcal{F}, \mathcal{G})
$$
is an algebraic space affine of finite presentation over $B$.
See Quot, Proposition \ref{quot-proposition-isom}.




\section{Properties of the stack of coherent sheaves}
\label{section-stack-coherent-sheaves}

\noindent
Let $f : X \to B$ be a morphism of algebraic spaces which is
separated and of finite presentation. Then the stack
$\textit{Coh}_{X/B}$ parametrizing flat families of coherent
modules with proper support is algebraic. See
Quot, Theorem \ref{quot-theorem-coherent-algebraic-general}.

\begin{lemma}
\label{lemma-coherent-diagonal-affine-fp}
The diagonal of $\textit{Coh}_{X/B}$ over $B$ is affine
and of finite presentation.
\end{lemma}

\begin{proof}
The representability of the diagonal (by algebraic spaces)
was shown in Quot, Lemma \ref{quot-lemma-coherent-diagonal}.
From the proof we find that we have to show
$\textit{Isom}(\mathcal{F}, \mathcal{G}) \to T$
is affine and of finite presentation for a pair of
finitely presented $\mathcal{O}_{X_T}$-modules
$\mathcal{F}$, $\mathcal{G}$ flat over $T$ with support
proper over $T$. This was discussed in Section \ref{section-hom-isom}.
\end{proof}

\begin{lemma}
\label{lemma-coherent-qs-lfp}
The morphism $\textit{Coh}_{X/B} \to B$ is quasi-separated and
locally of finite presentation.
\end{lemma}

\begin{proof}
To check $\textit{Coh}_{X/B} \to B$ is quasi-separated we have to
show that its diagonal is quasi-compact and quasi-separated.
This is immediate from Lemma \ref{lemma-coherent-diagonal-affine-fp}.
To prove that $\textit{Coh}_{X/B} \to B$ is locally of finite
presentation, we have to show that $\textit{Coh}_{X/B} \to B$
is limit preserving, see
Limits of Stacks, Proposition
\ref{stacks-limits-proposition-characterize-locally-finite-presentation}.
This follows from Quot, Lemma \ref{quot-lemma-coherent-limits}
(small detail omitted).
\end{proof}

\begin{lemma}
\label{lemma-coherent-existence-part}
Assume $X \to B$ is proper as well as of finite presentation.
Then $\textit{Coh}_{X/B} \to B$ satisfies the existence part
of the valuative criterion (Morphisms of Stacks, Definition
\ref{stacks-morphisms-definition-existence}).
\end{lemma}

\begin{proof}
Taking base change, this immediately reduces to the following
problem: given a valuation ring $R$ with fraction field $K$ and
an algebraic space $X$ proper over $R$ and a coherent
$\mathcal{O}_{X_K}$-module $\mathcal{F}_K$, show there exists
a finitely presented $\mathcal{O}_X$-module $\mathcal{F}$
flat over $R$ whose generic fibre is $\mathcal{F}_K$.
Observe that by Flatness on Spaces, Theorem
\ref{spaces-flat-theorem-finite-type-flat}
any finite type quasi-coherent $\mathcal{O}_X$-module
$\mathcal{F}$ flat over $R$ is of finite presentation.
Denote $j : X_K \to X$ the embedding of the generic fibre.
As a base change of the affine morphism $\Spec(K) \to \Spec(R)$
the morphism $j$ is affine. Thus $j_*\mathcal{F}_K$ is
quasi-coherent. Write
$$
j_*\mathcal{F}_K = \colim \mathcal{F}_i
$$
as a filtered colimit of its finite type quasi-coherent
$\mathcal{O}_X$-submodules, see
Limits on Spaces, Lemma \ref{spaces-limits-lemma-directed-colimit-finite-type}.
Since $j_*\mathcal{F}_K$ is a sheaf of $K$-vector spaces over $X$,
it is flat over $\Spec(R)$. Thus each $\mathcal{F}_i$ is flat
over $R$ as flatness over a valuation ring is the same as being
torsion free
(More on Algebra, Lemma
\ref{more-algebra-lemma-valuation-ring-torsion-free-flat})
and torsion freeness is inherited by submodules.
Finally, we have to show that the map
$j^*\mathcal{F}_i \to \mathcal{F}_K$
is an isomorphism for some $i$.
Since $j^*j_*\mathcal{F}_K = \mathcal{F}_K$ (small detail omitted)
and since $j^*$ is exact, we see that $j^*\mathcal{F}_i \to \mathcal{F}_K$
is injective for all $i$.
Since $j^*$ commutes with colimits, we have
$\mathcal{F}_K = j^*j_*\mathcal{F}_K = \colim j^*\mathcal{F}_i$.
Since $\mathcal{F}_K$ is coherent (i.e., finitely presented),
there is an $i$ such that $j^*\mathcal{F}_i$ contains all the
(finitely many) generators over an affine \'etale cover of $X$.
Thus we get surjectivity of $j^*\mathcal{F}_i \to \mathcal{F}_K$
for $i$ large enough.
\end{proof}



\section{Other chapters}

\begin{multicols}{2}
\begin{enumerate}
\item \hyperref[introduction-section-phantom]{Introduction}
\item \hyperref[conventions-section-phantom]{Conventions}
\item \hyperref[sets-section-phantom]{Set Theory}
\item \hyperref[categories-section-phantom]{Categories}
\item \hyperref[topology-section-phantom]{Topology}
\item \hyperref[sheaves-section-phantom]{Sheaves on Spaces}
\item \hyperref[algebra-section-phantom]{Commutative Algebra}
\item \hyperref[sites-section-phantom]{Sites and Sheaves}
\item \hyperref[homology-section-phantom]{Homological Algebra}
\item \hyperref[derived-section-phantom]{Derived Categories}
\item \hyperref[more-algebra-section-phantom]{More Algebra}
\item \hyperref[simplicial-section-phantom]{Simplicial Methods}
\item \hyperref[modules-section-phantom]{Sheaves of Modules}
\item \hyperref[sites-modules-section-phantom]{Modules on Sites}
\item \hyperref[injectives-section-phantom]{Injectives}
\item \hyperref[cohomology-section-phantom]{Cohomology of Sheaves}
\item \hyperref[sites-cohomology-section-phantom]{Cohomology on Sites}
\item \hyperref[hypercovering-section-phantom]{Hypercoverings}
\item \hyperref[schemes-section-phantom]{Schemes}
\item \hyperref[constructions-section-phantom]{Constructions of Schemes}
\item \hyperref[properties-section-phantom]{Properties of Schemes}
\item \hyperref[morphisms-section-phantom]{Morphisms of Schemes}
\item \hyperref[coherent-section-phantom]{Coherent Cohomology}
\item \hyperref[divisors-section-phantom]{Divisors}
\item \hyperref[limits-section-phantom]{Limits of Schemes}
\item \hyperref[varieties-section-phantom]{Varieties}
\item \hyperref[chow-section-phantom]{Chow Homology}
\item \hyperref[topologies-section-phantom]{Topologies on Schemes}
\item \hyperref[descent-section-phantom]{Descent}
\item \hyperref[more-morphisms-section-phantom]{More on Morphisms}
\item \hyperref[flat-section-phantom]{More on Flatness}
\item \hyperref[groupoids-section-phantom]{Groupoid Schemes}
\item \hyperref[more-groupoids-section-phantom]{More on Groupoid Schemes}
\item \hyperref[etale-section-phantom]{\'Etale Morphisms of Schemes}
\item \hyperref[etale-cohomology-section-phantom]{\'Etale Cohomology}
\item \hyperref[spaces-section-phantom]{Algebraic Spaces}
\item \hyperref[spaces-properties-section-phantom]{Properties of Algebraic Spaces}
\item \hyperref[spaces-morphisms-section-phantom]{Morphisms of Algebraic Spaces}
\item \hyperref[spaces-topologies-section-phantom]{Topologies on Algebraic Spaces}
\item \hyperref[spaces-descent-section-phantom]{Descent and Algebraic Spaces}
\item \hyperref[spaces-more-morphisms-section-phantom]{More on Morphisms of Spaces}
\item \hyperref[quot-section-phantom]{Quot and Hilbert Spaces}
\item \hyperref[stacks-section-phantom]{Stacks}
\item \hyperref[spaces-groupoids-section-phantom]{Groupoids in Algebraic Spaces}
\item \hyperref[spaces-more-groupoids-section-phantom]{More on Groupoids in Spaces}
\item \hyperref[bootstrap-section-phantom]{Bootstrap}
\item \hyperref[examples-stacks-section-phantom]{Examples of Stacks}
\item \hyperref[groupoids-quotients-section-phantom]{Quotients of Groupoids}
\item \hyperref[algebraic-section-phantom]{Algebraic Stacks}
\item \hyperref[criteria-section-phantom]{Criteria for Representability}
\item \hyperref[stacks-properties-section-phantom]{Properties of Algebraic Stacks}
\item \hyperref[stacks-morphisms-section-phantom]{Morphisms of Algebraic Stacks}
\item \hyperref[examples-section-phantom]{Examples}
\item \hyperref[exercises-section-phantom]{Exercises}
\item \hyperref[guide-section-phantom]{Guide to Literature}
\item \hyperref[desirables-section-phantom]{Desirables}
\item \hyperref[coding-section-phantom]{Coding Style}
\item \hyperref[fdl-section-phantom]{GNU Free Documentation License}
\item \hyperref[index-section-phantom]{Auto Generated Index}
\end{enumerate}
\end{multicols}


\bibliography{my}
\bibliographystyle{amsalpha}

\end{document}
