\IfFileExists{stacks-project.cls}{%
\documentclass{stacks-project}
}{%
\documentclass{amsart}
}

% The following AMS packages are automatically loaded with
% the amsart documentclass:
%\usepackage{amsmath}
%\usepackage{amssymb}
%\usepackage{amsthm}

% For dealing with references we use the comment environment
\usepackage{verbatim}
\newenvironment{reference}{\comment}{\endcomment}
%\newenvironment{reference}{}{}
\newenvironment{slogan}{\comment}{\endcomment}
\newenvironment{history}{\comment}{\endcomment}

% For commutative diagrams you can use
% \usepackage{amscd}
\usepackage[all]{xy}

% We use 2cell for 2-commutative diagrams.
\xyoption{2cell}
\UseAllTwocells

% To put source file link in headers.
% Change "template.tex" to "this_filename.tex"
% \usepackage{fancyhdr}
% \pagestyle{fancy}
% \lhead{}
% \chead{}
% \rhead{Source file: \url{template.tex}}
% \lfoot{}
% \cfoot{\thepage}
% \rfoot{}
% \renewcommand{\headrulewidth}{0pt}
% \renewcommand{\footrulewidth}{0pt}
% \renewcommand{\headheight}{12pt}

\usepackage{multicol}

% For cross-file-references
\usepackage{xr-hyper}

% Package for hypertext links:
\usepackage{hyperref}

% For any local file, say "hello.tex" you want to link to please
% use \externaldocument[hello-]{hello}
\externaldocument[introduction-]{introduction}
\externaldocument[conventions-]{conventions}
\externaldocument[sets-]{sets}
\externaldocument[categories-]{categories}
\externaldocument[topology-]{topology}
\externaldocument[sheaves-]{sheaves}
\externaldocument[sites-]{sites}
\externaldocument[stacks-]{stacks}
\externaldocument[fields-]{fields}
\externaldocument[algebra-]{algebra}
\externaldocument[brauer-]{brauer}
\externaldocument[homology-]{homology}
\externaldocument[derived-]{derived}
\externaldocument[simplicial-]{simplicial}
\externaldocument[more-algebra-]{more-algebra}
\externaldocument[smoothing-]{smoothing}
\externaldocument[modules-]{modules}
\externaldocument[sites-modules-]{sites-modules}
\externaldocument[injectives-]{injectives}
\externaldocument[cohomology-]{cohomology}
\externaldocument[sites-cohomology-]{sites-cohomology}
\externaldocument[dga-]{dga}
\externaldocument[dpa-]{dpa}
\externaldocument[hypercovering-]{hypercovering}
\externaldocument[schemes-]{schemes}
\externaldocument[constructions-]{constructions}
\externaldocument[properties-]{properties}
\externaldocument[morphisms-]{morphisms}
\externaldocument[coherent-]{coherent}
\externaldocument[divisors-]{divisors}
\externaldocument[limits-]{limits}
\externaldocument[varieties-]{varieties}
\externaldocument[topologies-]{topologies}
\externaldocument[descent-]{descent}
\externaldocument[perfect-]{perfect}
\externaldocument[more-morphisms-]{more-morphisms}
\externaldocument[flat-]{flat}
\externaldocument[groupoids-]{groupoids}
\externaldocument[more-groupoids-]{more-groupoids}
\externaldocument[etale-]{etale}
\externaldocument[chow-]{chow}
\externaldocument[intersection-]{intersection}
\externaldocument[pic-]{pic}
\externaldocument[adequate-]{adequate}
\externaldocument[dualizing-]{dualizing}
\externaldocument[duality-]{duality}
\externaldocument[discriminant-]{discriminant}
\externaldocument[local-cohomology-]{local-cohomology}
\externaldocument[curves-]{curves}
\externaldocument[resolve-]{resolve}
\externaldocument[models-]{models}
\externaldocument[pione-]{pione}
\externaldocument[etale-cohomology-]{etale-cohomology}
\externaldocument[proetale-]{proetale}
\externaldocument[crystalline-]{crystalline}
\externaldocument[spaces-]{spaces}
\externaldocument[spaces-properties-]{spaces-properties}
\externaldocument[spaces-morphisms-]{spaces-morphisms}
\externaldocument[decent-spaces-]{decent-spaces}
\externaldocument[spaces-cohomology-]{spaces-cohomology}
\externaldocument[spaces-limits-]{spaces-limits}
\externaldocument[spaces-divisors-]{spaces-divisors}
\externaldocument[spaces-over-fields-]{spaces-over-fields}
\externaldocument[spaces-topologies-]{spaces-topologies}
\externaldocument[spaces-descent-]{spaces-descent}
\externaldocument[spaces-perfect-]{spaces-perfect}
\externaldocument[spaces-more-morphisms-]{spaces-more-morphisms}
\externaldocument[spaces-flat-]{spaces-flat}
\externaldocument[spaces-groupoids-]{spaces-groupoids}
\externaldocument[spaces-more-groupoids-]{spaces-more-groupoids}
\externaldocument[bootstrap-]{bootstrap}
\externaldocument[spaces-pushouts-]{spaces-pushouts}
\externaldocument[groupoids-quotients-]{groupoids-quotients}
\externaldocument[spaces-more-cohomology-]{spaces-more-cohomology}
\externaldocument[spaces-simplicial-]{spaces-simplicial}
\externaldocument[formal-spaces-]{formal-spaces}
\externaldocument[restricted-]{restricted}
\externaldocument[spaces-resolve-]{spaces-resolve}
\externaldocument[formal-defos-]{formal-defos}
\externaldocument[defos-]{defos}
\externaldocument[cotangent-]{cotangent}
\externaldocument[examples-defos-]{examples-defos}
\externaldocument[algebraic-]{algebraic}
\externaldocument[examples-stacks-]{examples-stacks}
\externaldocument[stacks-sheaves-]{stacks-sheaves}
\externaldocument[criteria-]{criteria}
\externaldocument[artin-]{artin}
\externaldocument[quot-]{quot}
\externaldocument[stacks-properties-]{stacks-properties}
\externaldocument[stacks-morphisms-]{stacks-morphisms}
\externaldocument[stacks-limits-]{stacks-limits}
\externaldocument[stacks-cohomology-]{stacks-cohomology}
\externaldocument[stacks-perfect-]{stacks-perfect}
\externaldocument[stacks-introduction-]{stacks-introduction}
\externaldocument[stacks-more-morphisms-]{stacks-more-morphisms}
\externaldocument[stacks-geometry-]{stacks-geometry}
\externaldocument[moduli-]{moduli}
\externaldocument[moduli-curves-]{moduli-curves}
\externaldocument[examples-]{examples}
\externaldocument[exercises-]{exercises}
\externaldocument[guide-]{guide}
\externaldocument[desirables-]{desirables}
\externaldocument[coding-]{coding}
\externaldocument[obsolete-]{obsolete}
\externaldocument[fdl-]{fdl}
\externaldocument[index-]{index}

% Theorem environments.
%
\theoremstyle{plain}
\newtheorem{theorem}[subsection]{Theorem}
\newtheorem{proposition}[subsection]{Proposition}
\newtheorem{lemma}[subsection]{Lemma}

\theoremstyle{definition}
\newtheorem{definition}[subsection]{Definition}
\newtheorem{example}[subsection]{Example}
\newtheorem{exercise}[subsection]{Exercise}
\newtheorem{situation}[subsection]{Situation}

\theoremstyle{remark}
\newtheorem{remark}[subsection]{Remark}
\newtheorem{remarks}[subsection]{Remarks}

\numberwithin{equation}{subsection}

% Macros
%
\def\lim{\mathop{\rm lim}\nolimits}
\def\colim{\mathop{\rm colim}\nolimits}
\def\Spec{\mathop{\rm Spec}}
\def\Hom{\mathop{\rm Hom}\nolimits}
\def\Ext{\mathop{\rm Ext}\nolimits}
\def\SheafHom{\mathop{\mathcal{H}\!{\it om}}\nolimits}
\def\SheafExt{\mathop{\mathcal{E}\!{\it xt}}\nolimits}
\def\Sch{\textit{Sch}}
\def\Mor{\mathop{\rm Mor}\nolimits}
\def\Ob{\mathop{\rm Ob}\nolimits}
\def\Sh{\mathop{\textit{Sh}}\nolimits}
\def\NL{\mathop{N\!L}\nolimits}
\def\proetale{{pro\text{-}\acute{e}tale}}
\def\etale{{\acute{e}tale}}
\def\QCoh{\textit{QCoh}}
\def\Ker{\mathop{\rm Ker}}
\def\Im{\mathop{\rm Im}}
\def\Coker{\mathop{\rm Coker}}
\def\Coim{\mathop{\rm Coim}}

%
% Macros for moduli stacks/spaces
%
\def\QCohstack{\mathcal{QC}\!{\it oh}}
\def\Cohstack{\mathcal{C}\!{\it oh}}
\def\Spacesstack{\mathcal{S}\!{\it paces}}
\def\Quotfunctor{{\rm Quot}}
\def\Hilbfunctor{{\rm Hilb}}
\def\Curvesstack{\mathcal{C}\!{\it urves}}
\def\Polarizedstack{\mathcal{P}\!{\it olarized}}
\def\Complexesstack{\mathcal{C}\!{\it omplexes}}
% \Pic is the operator that assigns to X its picard group, usage \Pic(X)
% \Picardstack_{X/B} denotes the Picard stack of X over B
% \Picardfunctor_{X/B} denotes the Picard functor of X over B
\def\Pic{\mathop{\rm Pic}\nolimits}
\def\Picardstack{\mathcal{P}\!{\it ic}}
\def\Picardfunctor{{\rm Pic}}
\def\Deformationcategory{\mathcal{D}\!{\it ef}}


% OK, start here.
%
\begin{document}

\title{Etale Cohomology}


\maketitle

\phantomsection
\label{section-phantom}

\tableofcontents

\section{Cohomology of curves}
\label{section-cohomology-curves}

\noindent
The next task at hand is to compute the \'etale cohomology of a smooth curve
with torsion coefficients, and in particular show that it vanishes in degree at
least 3. To prove this, we will compute cohomology at the generic point, which
amounts to some Galois cohomology. We now review without proofs. the relevant
facts about Brauer groups. For references, see \cite{SerreCorpsLocaux},
\cite{SerreGaloisCohomology} or \cite{Weil}.




\section{Brauer groups}
\label{section-brauer-groups}

\begin{theorem}
\label{theorem-central-simple-algebra}
Let $K$ be a field. A unital, associative (not necessarily commutative)
$K$-algebra $A$ is a {\it central simple algebra} over $K$ if the following
equivalent conditions hold
\begin{enumerate}
\item
$A$ is finite dimensional over $K$, $K$ is the center of $A$ and $A$ has no
nontrivial two-sided ideal ;
\item
there exists $d \geq 1$ such that $A \otimes_K \bar K \cong_{\bar K}
\text{Mat}_d(\bar K)$ ;
\item
there exists $d \geq 1$ such that $A \otimes_K K^{sep} \cong_{K^{sep}}
\text{Mat}_d(K^{sep})$ ;
\item
there exist $d \geq 1$ and a finite Galois extension $K \subset K'$ such that
$A \otimes_{K'} K' \cong_{K'} \text{Mat}_d(K')$;
\item
there exist $f \geq 1$ and a finite dimensional division algebra $D$ with
center $K$ such that $A \cong_{K'} \text{Mat}_f(D)$.
\end{enumerate}
The integer $d$ is called the {\it degree} of $A$.
\end{theorem}

\begin{definition}
\label{definition-brauer-equivalent}
Two central simple algebras $A_1$ and $A_2$ over $K$ are called
{\it equivalent} if there exist $m, n \geq 1$ such that $\text{Mat}_n(A_1)
\cong \text{Mat}_m(A_2)$. We write $A_1 \sim A_2$.
\end{definition}

\begin{lemma}
\label{lemma-brauer-inverse}
Let $A$ be a central simple algebra over $K$. Then
$$
\begin{matrix}
A \otimes_K A^{opp} & \longrightarrow & \text{End}_{K-\text{Mod}}(A) \\
\ a \otimes a' & \longmapsto & (x \mapsto a x a')
\end{matrix}
$$
is an isomorphism of central simple algebras over $K$.
\end{lemma}

\begin{definition}
\label{definition-brauer-group}
Let $K$ be a field. The {\it Brauer group} of $K$ is the set $\text{Br} (K)$
of central simple algebras over $K$ modulo equivalence, endowed with the group
law induced by tensor product (over $K$). The class of $A$ in $\text{Br} (K)$
is denoted by $[A]$. The neutral element is $[K] = [\text{Mat}_d(K)]$ for any
$d \geq 1$.
\end{definition}

\noindent
The previous lemma thus mean that inverses exist, and that $-[A] = [A^{opp}]$.
The Brauer group is always torsion, but not finitely generated in general. It
is also true (exercise) that $A^{\otimes \deg A} \sim K$ for any central simple
algebra $A$.

\begin{lemma}
\label{lemma-central-simple-algebra-pgln}
Let $K$ be a field and $\mathcal{G} = \text{Gal} (K^{sep}|K))$. Then the set of
isomorphism classes of central simple algebras of degree $d$ over $K$ is in
bijection with the anabelian cohomology $H_{cont}^1 (\mathcal{G},
\text{PGL}_d(K^{sep}))$.
\end{lemma}

\begin{proof}[Sketch of proof.]
The Skolem-Noether theorem implies that for any field $L$
the group
$\text{Aut}_{L\text{-Algebras}}(\text{Mat}_d(L))$
equals $\text{PGL}_d(L)$. By
Theorem \ref{theorem-central-simple-algebra}, we see that
central simple algebras of degree $d$ correspond
to forms of the $K$-algebra $\text{Mat}_d(K)$, which in turn correspond to
$H_{cont}^1 (\mathcal{G}, \text{PGL}_d(K^{sep}))$. For more details on
twisting, see for example
\cite{SilvermanEllipticCurves}.
\end{proof}

\noindent
If $A$ is a central simple algebra over $K$, we denote $\xi_A$ the
corresponding cohomology class in $H_{cont}^1 (\mathcal{G}, \text{PGL}_{\deg
A}(K^{sep}))$. Consider now the short exact sequence
$$
1 \to (K^{sep})^* \to \text{GL}_d(K^{sep}) \to \text{PGL}_d(K^{sep}) \to 1,
$$
which gives rise to a long exact cohomology sequence (up to degree 2) with
coboundary map
$$
\delta_d : H_{cont} ^1(\mathcal{G}, \text{PGL}_d(K^{sep})) \to H^2
(\mathcal{G}, (K^{sep})^*).
$$
Explicitly, this is given as follows: if $\xi$ is a cohomology class
represented by the 1-cocyle $(g_\sigma)$, then $\delta_d(\xi)$ is the class of
the 2-cocycle $((g_\sigma^\tau)^{-1} g_{\sigma \tau} g_\tau^{-1})$.

\begin{theorem}
\label{theorem-brauer-delta}
The map
$$
\begin{matrix}
\delta : & \text{Br}(K) & \longrightarrow & H^2(\mathcal{G}, (K^{sep})^*) \\
& [A] & \longmapsto & \delta_{\deg A} (\xi_A)
\end{matrix}
$$
is a group isomorphism.
\end{theorem}

\noindent
We omit the proof of this theorem. Note, however, that in the abelian case, one
has the identification
$$
H^1 (\mathcal{G}, \text{GL}_d(K^{sep})) = H_{et}^1 (\text{Spec} K,
\text{GL}_d(\mathcal{O}))
$$
the latter of which is trivial by fpqc descent. If this were true in the
anabelian case, this would readily imply injectivity of $\delta$. (See
\cite{SGA4.5}.) Rather, to prove this, one can reinterpret $\delta([A])$ as the
obstruction to the existence of a $K$-vector space $V$ with a left $A$-module
structure and such that $\dim_K V = \deg A$. In the case where $V$ exists, one
has $A \cong \text{End}_{K-\text{Mod}} (V)$. For surjectivity, pick a
cohomology class $\xi \in H^2(\mathcal{G}, (K^{sep})^*)$, then there exists a
finite Galois extension $K \subset K' \subset K^{sep}$ such that $\xi$ is
the image of some $\xi' \in H_{cont}^2(\text{Gal}(K'|K), (K')^*)$. Then write
down an explicit central simple algebra over $K$ using the data $K', \xi'$.

\medskip\noindent
The Brauer group of a scheme.
Let $S$ be a scheme. An $\mathcal{O}_S$-algebra $\mathcal{A}$ is called
{\it Azumaya} if it is \'etale locally a matrix algebra, i.e., if there
exists an \'etale covering $\mathcal{U} = \{ \varphi_i : \mathcal{U}_i \to S
\}_{i \in I}$ such that $\varphi_i^*\mathcal{A} \cong
\text{Mat}_{d_i}(\mathcal{O}_{\mathcal{U}_i})$ for some $d_i \geq 1$. Two such
$\mathcal{A}$ and $\mathcal{B}$ are called {\it equivalent} if there exist
finite locally free sheaves $\mathcal{F}$ and $\mathcal{G}$ on $S$ such that
$\mathcal{A} \otimes_{\mathcal{O}_S} \text{End}(\mathcal{F}) \cong \mathcal{B}
\otimes_{\mathcal{O}_S} \text{End}(\mathcal{G})$. The {\it Brauer group} of
$S$ is the set $\text{Br}(S)$ of equivalence classes of Azumaya
$\mathcal{O}_S$-algebras with the operation induced by tensor product (over
$\mathcal{O}_S$).

\medskip\noindent
In this setting, the analogue of the isomorphism $\delta$ of theorem
\ref{theorem-brauer-delta} is a map
$$
\delta_S: \text{Br}(S) \to H_{et}^2(S,\mathbf{G}_m).
$$
It is true that $\delta_S$ is injective (the previous argument still works). If
$S$ is quasi-compact or connected, then $\text{Br}(S)$ is a torsion group, so
in this case the image of $\delta_S$ is contained in the {\it cohomological
Brauer group} of $S$
$$
\text{Br}'(S) := H_{et}^2(S,\mathbf{G}_m)_\text{torsion}.
$$
So if $S$ is quasi-compact or connected, there is an inclusion $\text{Br}(S)
\subset \text{Br}'(S)$. This is not always an equality: there exists a
nonseparated singular surface $S$ for which $\text{Br}(S) \subset
\text{Br}'(S)$ is a strict inclusion. If $S$ is quasi-projective, then
$\text{Br}(S) = \text{Br}'(S)$. However, it is not known whether this holds for
a smooth proper variety over $\mathbf{C}$, say.


\begin{proposition}
\label{proposition-serre-galois}
Let $K$ be a field, $\mathcal{G} = \text{Gal}(K^{sep}|K)$ and suppose that for
any finite extension $K'$ of $K$, $\text{Br}(K') = 0$. Then
\begin{enumerate}
\item
for all $ q \geq 1$, $H^q (\mathcal{G},(K^{sep})^*) = 0$ ; and
\item
for any torsion $\mathcal{G}$-module $M$ and any $q \geq 2$, $H_{cont}^q
(\mathcal{G},M) = 0$.
\end{enumerate}
\end{proposition}

\noindent
See \cite{SerreGaloisCohomology} for proofs.

%10.08.09

\begin{definition}
\label{definition-Cr}
A field $K$ is called {\it $C_r$}
if for every $0 < d^r < n$ and every $f \in K[T_1,
\ldots, T_n]$ homogeneous of degree $d$, there exist $\alpha = (\alpha_1,
\ldots, \alpha_n)$, $\alpha_i \in K$ not all zero, such that $f(\alpha) = 0$.
Such an $\alpha$ is called a {\it nontrivial solution} of $f$.
\end{definition}

\begin{example}
\label{example-algebraically-closed-field-Cr}
An algebraically closed field is $C_r$.
\end{example}

\noindent
In fact, we have the following simple lemma.

\begin{lemma}
\label{lemma-algebraically-closed-find-solutions}
Let $k$ be an algebraically closed field. Let
$f_1, \ldots, f_s \in k[T_1, \ldots, T_n]$
be homogeneous polynomials of degree $d_1, \ldots, d_s$ with $d_i
> 0$. If $s < n$, then $f_1 = \ldots = f_s = 0$ have a common nontrivial
solution.
\end{lemma}

\begin{proof}
Omitted.
\end{proof}

\noindent
The following result computes the Brauer group of $C_1$ fields.

\begin{theorem}
\label{theorem-C1-brauer-group-zero}
Let $K$ be a $C_1$ field. Then $\text{Br}(K) = 0$.
\end{theorem}

\begin{proof}
Let $D$ be a finite dimensional division algebra over $K$ with center $K$. We
have seen that
$$
D \otimes_K K^{sep} \cong \text{Mat}_d(K^{sep})
$$
uniquely up to inner isomorphism. Hence the determinant $\det :
\text{Mat}_d(K^{sep}) \to K^{sep}$ is Galois invariant and descends to a
homogeneous degree $d$ map
$$
\det = N_\text{red} : D \longrightarrow K
$$
called the {\it reduced norm}. Since $K$ is $C_1$, if $d > 1$, then there
exists a nonzero $x \in D$ with $N_\text{red}(x) = 0$. This clearly implies
that $x$ is not invertible, which is a contradiction. Hence $\text{Br}(K) = 0$.
\end{proof}

\begin{theorem}
\label{theorem-tsen}
(Tsen)
The function field of a variety of dimension $r$ over an algebraically closed
field $k$ is $C_r$.
\end{theorem}

\begin{proof}
\begin{enumerate}
\item
Projective space. The field $k(x_1, \ldots, x_r)$ is $C_r$ (exercise).
\item
General case. Without loss of generality, we may assume $X$ to be projective.
Let $f \in K[T_1, \dots, T_n]_d$ with $0 < d^r <n$. Say the coefficients of $f$
are in $\Gamma(X,\mathcal{O}_X(H))$ for some ample $H \subset X$. Let
$\mathbf{\alpha} = (\alpha_1, \dots, \alpha_n)$ with $\alpha_i \in \Gamma(X,
\mathcal{O}_X(eH))$. Then $f(\mathbf{\alpha}) \in \Gamma(X,
\mathcal{O}_X((de+1)H))$. Consider the system of equations $f(\mathbf{\alpha})
=0$. Then by asymptotic Riemann-Roch,
\begin{itemize}
\item
the number of variables is $n\dim_K \Gamma(X,\mathcal{O}_X(eH)) \sim
n\frac{e^r}{r!} (H^r)$ ; and
\item
the number of equations is $\dim_K \Gamma(X,\mathcal{O}_X((de+1)H)) \sim
\frac{(de+1)^r}{r!} (H^r).$
\end{itemize}
Since $n> d^r$, there are more variables than equations, and since there is a
trivial solution, there are also nontrivial solutions.
\end{enumerate}
\end{proof}

\begin{definition}
\label{definition-variety}
We call {\it variety} a separated, geometrically irreducible and geometrically
reduced scheme of finite type over a field, and {\it curve} a variety of
dimension 1.
\end{definition}

\begin{lemma}
\label{lemma-curve-brauer-zero}
Let $C$ be a curve over an algebraically closed field $k$. Then
$\text{Br}(k(C)) = 0$.
\end{lemma}

\noindent
This is clear from the theorem.

\begin{lemma}
\label{lemma-cohomology-Gm-function-field-curve}
Let $k$ be an algebraically closed field and $k \subset K$ a field extension
of transcendence degree 1. Then for all $q \geq 1$, $H_{et}^q(\text{Spec} K,
\mathbf{G}_m) = 0$.
\end{lemma}

\begin{proof}
It suffices to show that if $K \subset K'$ is a finite field extension, then
$\text{Br}(K') = 0$. Now observe that $K' = \text{colim} K''$, where $K''$ runs
over the finitely generated subextensions of $k$ contained in $K'$ of
transcendence degree 1. By some result in \cite{H}, each $K''$ is the function
field of a curve, hence has trivial Brauer group by the previous corollary. It
now suffices to observe that $\text{Br}(K') = \text{colim} \text{Br}(K'')$.
\end{proof}






\section{Higher vanishing for the multiplicative group}
\label{section-higher-Gm}

\noindent
In this section, we fix an algebraically closed field $k$ and a smooth curve
$X$ over $k$. We denote $i_x : x \hookrightarrow X$ the inclusion of a closed
point of $X$ and $j : \eta \hookrightarrow X$ the inclusion of the generic
point. We also denote $X^0$ the set of closed points of $X$.

\begin{theorem}
\label{theorem-fundamental-exact-sequence}
(The Fundamental Exact Sequence)
There is a short exact sequence of \'etale sheaves on $X$
$$
0 \longrightarrow \mathbf{G}_{m,X} \longrightarrow j_* \mathbf{G}_{m,\eta}
\xrightarrow{\ \div\ } \bigoplus_{x \in X^0} {i_x}_* \underline{\mathbf{Z}}
\longrightarrow 0.
$$
\end{theorem}

\begin{proof}
Let $\varphi : \mathcal{U} \to X$ be an \'etale morphism. Then by properties
{\it v} and {\it vi} of \'etale morphisms (
Proposition \ref{proposition-etale-morphisms}),
$\mathcal{U} = \coprod_i \mathcal{U}_i$ where
each $\mathcal{U}_i$ is a smooth curve mapping to $X$. The above sequence for
$X$ is a product of the corresponding sequences for each $\mathcal{U}_i$, so it
suffices to treat the case where $\mathcal{U}$ is connected, hence irreducible.
In this case, there is a well known exact sequence (see \cite{H})
$$
1 \longrightarrow
\Gamma(\mathcal{U},\mathcal{O}_\mathcal{U}^*) \longrightarrow
k(\mathcal{U})^* \xrightarrow{\ \div\ }
\bigoplus_{y \in \mathcal{U}^0} \mathbf{Z}_y.
$$
This amounts to a sequence
$$
\Gamma(\mathcal{U},\mathcal{O}_\mathcal{U}^*) \longrightarrow
\Gamma(\eta\times_X\mathcal{U},\mathcal{O}_{\eta\times_X\mathcal{U}}^*)
\xrightarrow{\ \div\ } \bigoplus_{x \in X^0}
\Gamma(x\times_X\mathcal{U},\underline{\mathbf{Z}})
$$
which, unfolding definitions, is nothing but a sequence
$$
\mathbf{G}_m(\mathcal{U}) \longrightarrow j_* \mathbf{G}_{m,\eta}(\mathcal{U})
\xrightarrow{\ \div\ } \bigoplus_{x \in X^0} {i_x}_* \underline{\mathbf{Z}}
(\mathcal{U}).
$$
This defines the maps in the Fundamental Exact Sequence and shows it is exact
except possibly at the last step. To see surjectivity, let us recall (from
\cite{H} again) that if $C$ is a nonsingular curve and $D$ is a divisor on $C$,
then there exists a Zariski open covering $\{ \mathcal{V}_j \to C \}$ of $C$
such that $D |_{\mathcal{V}_j} = \div(f_j)$ for some $f_j \in k(C)^*$.
\end{proof}

\begin{lemma}
\label{lemma-higher-direct-jstar-Gm}
For any $q \geq 1$, $R^q j_*\mathbf{G}_{m,\eta} = 0$.
\end{lemma}

\begin{proof}
We need to show that $(R^q j_*\mathbf{G}_{m,\eta})_{\bar x} = 0$ for every
geometric point $\bar x$ of $X$.
\begin{enumerate}
\item
Assume that $\bar x$ lies over a closed point $x$ of $X$. Let $\text{Spec} A$
be an open neighborhood of $x$ in $X$, and $K$ the fraction field of $A$, so
that
$$
\text{Spec}(\mathcal{O}^\text{sh}_{X,\bar x}) \times_X \eta =
\text{Spec}(\mathcal{O}^\text{sh}_{X,\bar x} \otimes_A K).
$$
The ring $\mathcal{O}^\text{sh}_{X,\bar x} \otimes_A K$ is a localization of
the discrete valuation ring $\mathcal{O}^\text{sh}_{X,\bar x}$, so it is either
$\mathcal{O}^\text{sh}_{X,\bar x}$ again, or its fraction field
$K^\text{sh}_{\bar x}$. But since some local uniformizer gets inverted, it must
be the latter. Hence
$$
(R^q j_*\mathbf{G}_{m,\eta})_{(X, \bar x)} = H_{et}^q(\text{Spec}
K^\text{sh}_{\bar x}, \mathbf{G}_m).
$$
Now recall that $\mathcal{O}^\text{sh}_{X, \bar x} =
\text{colim}_{(\mathcal{U},\bar u) \to \bar x} \mathcal{O} (\mathcal{U}) =
\text{colim}_{A \subset B} B$ where $A \to B$ is \'etale, hence
$K^\text{sh}_{\bar x}$ is an algebraic extension of $k(X)$, and we may apply
corollary \ref{lemma-cohomology-Gm-function-field-curve} to get the vanishing.
\item
Assume that $\bar x = \bar \eta$ lies over the generic point $\eta$ of $X$ (in
fact, this case is superfluous). Then $\mathcal{O}_{X,\bar \eta} =
\kappa(\eta)^{sep}$ and thus
\begin{eqnarray*}
(R^q j_*\mathbf{G}_{m,\eta})_{\bar \eta}
& = &
H_{et}^q(\text{Spec} \kappa(\eta)^{sep} \times_X \eta, \mathbf{G}_m) \\
& = & H_{et}^q (\text{Spec} \kappa(\eta)^{sep}, \mathbf{G}_m) \\
& = & 0 \ \ \text{ for } q \geq 1
\end{eqnarray*}
since the corresponding Galois group is trivial.
\end{enumerate}
\end{proof}

\begin{lemma}
\label{lemma-cohomology-jstar-Gm}
For all $p \geq 1$, $H_{et}^p(X, j_*\mathbf{G}_{m,\eta}) = 0$.
\end{lemma}

\begin{proof}
The Leray spectral sequence reads
$$
E_2^{p,q} = H_{et}^p(X, R^qj_*\mathbf{G}_{m,\eta}) \Rightarrow
H_{et}^{p+q}(\eta, \mathbf{G}_{m,\eta}),
$$
which vanishes for $p+q \geq 1$ by
Lemma \ref{lemma-cohomology-Gm-function-field-curve}. Taking
$q = 0$, we get the desired vanishing.
\end{proof}

\begin{lemma}
\label{lemma-cohomology-istar-Z}
For all $q \geq 1$, $H_{et}^q(X, \bigoplus_{x \in X^0} {i_x}_*
\underline{\mathbf{Z}}) = 0$.
\end{lemma}

\begin{proof}
For $X$ quasi-compact and quasi-separated, cohomology commutes with colimits,
so it suffices to show the vanishing of $H_{et}^q(X, {i_x}_*
\underline{\mathbf{Z}})$. But then the inclusion $i_x$ of a closed point is
finite so $R^p {i_x}_* \underline{\mathbf{Z}} = 0$ for all $p \geq 1$ by
Proposition \ref{proposition-finite-higher-direct-image-zero}.
Applying the Leray spectral sequence, we see that
$H_{et}^q(X, {i_x}_* \underline{\mathbf{Z}}) =
H_{et}^q(x, \underline{\mathbf{Z}})$.
Finally, since $x$ is the spectrum of an
algebraically closed field, all higher cohomology on $x$ vanishes.
\end{proof}

\noindent
Concluding this series of lemmata, we get the following result.

\begin{theorem}
\label{theorem-vanishing-cohomology-Gm-curve}
Let $X$ be a smooth curve over an algebraically closed field. Then
$$
H_{et}^q(X, \mathbf{G}_m) = 0 \ \ \text{ for all } q \geq 2.
$$
\end{theorem}

\noindent
We also get the cohomology long exact sequence
$$
0 \to H_{et}^0(X,\mathbf{G}_m) \to H_{et}^0(X,j_*\mathbf{G}_{m\eta})
\xrightarrow{\div} H_{et}^0(X,\bigoplus {i_x}_*\underline{\mathbf{Z}}) \to
H_{et}^1(X,\mathbf{G}_m) \to 0
$$
although this is the familiar
$$
0 \to H_{Zar}^0(X,\mathcal{O}_X^*) \to k(X)^* \xrightarrow{\div} \text{Div}(X)
\to \text{Pic}(X) \to 0.
$$

\medskip\noindent
We would like to use the Kummer sequence to deduce some information about the
cohomology group of a curve with finite coefficients. In order to get vanishing
in the long exact sequence, we review some facts about Picard groups.





\section{Picards groups of curves}
\label{section-pic-curves}

\noindent
Let $X$ be a smooth projective curve over an algebraically closed field $k$.
There exists a short exact sequence
$$
0\to \text{Pic}^0(X) \to \text{Pic}(X)\xrightarrow{\deg} \mathbf{Z} \to 0.
$$
The abelian group $\text{Pic}^0(X)$ can be identified with $\text{Pic}^0(X) =
\underline{\text{Pic}}^0_{X/k}(k)$, i.e., the $k$-valued points of an
abelian variety $\underline{\text{Pic}}^0_{X/k}$ of dimension $g=g(X)$ over
$k$.

\begin{definition}
\label{definition-abelian-variety}
An {\it abelian variety} over $k$ is a proper smooth connected group scheme
over $k$ (i.e., a proper group variety over $k$).
\end{definition}

\begin{proposition}
\label{proposition-review-abelian-varieties}
Let $A$ be an abelian variety over an algebraically closed field $k$. Then
\begin{enumerate}
\item
$A$ is projective over $k$;
\item
$A$ is a commutative group scheme;
\item
the morphism $[n]: A\to A$ is surjective for all $n\geq 1$, in other words
$A(k)$ is a divisible abelian group;
\item
$A[n] = \text{Ker}(A\xrightarrow{[n]} A)$ is a finite flat group scheme of rank
$n^{2\dim A}$ over $k$. It is reduced if and only if $n\in k^*$;
\item
if $n\in k^*$ then $A(k)[n] = A[n](k)\cong(\mathbf{Z}/n\mathbf{Z})^{2\dim(A)}$.
\end{enumerate}
\end{proposition}

\noindent
Consequently, if $n\in k^*$ then $\text{Pic}^0(X)[n] \cong
\left(\mathbf{Z}/n\mathbf{Z}\right)^{2g}$ as abelian groups.

\begin{lemma}
\label{lemma-cohomology-smooth-projective-curve}
Let $X$ be a smooth projective of genus $g$ over an algebraically closed field
$k$ and $n\geq 1$, $n\in k^*$. Then there are canonical identifications
$$
H_{et}^q(X, \mu_n) =
\left\{
\begin{matrix}
\mu_n(k) & \text{ if $q=0$ ;} \\
\text{Pic}^0(X)[n] & \text{ if $q=1$ ;} \\
\mathbf{Z}/n\mathbf{Z} & \text{ if $q=2$ ;}\\
0 & \text{ if $q \geq 3$.}
\end{matrix}
\right.
$$
Since $\mu_n \cong \underline{\mathbf{Z}/n\mathbf{Z}}$, this gives
(noncanonical) identifications
$$
H_{et}^q(X, \underline{\mathbf{Z}/n\mathbf{Z}}) \cong
\left\{
\begin{matrix}
\mathbf{Z}/n\mathbf{Z} & \text{ if $q=0$ ;} \\
(\mathbf{Z}/n\mathbf{Z})^{2g} & \text{ if $q=1$ ;} \\
\mathbf{Z}/n\mathbf{Z} & \text{ if $q=2$ ;}\\
0 & \text{ if $q \geq 3$.}
\end{matrix}
\right.
$$
\end{lemma}	

\begin{proof}
The Kummer sequence $0\to \mu_{n, X} \to \mathbf{G}_{m, X}
\xrightarrow{(\cdot)^n} \mathbf{G}_{m, X}\to 0$ give the long exact cohomology
sequence
$$
\xymatrix{
0 \ar[r] & \mu_n(k) \ar[r] & k^* \ar^{(\cdot)^n}[r] & k^* \ar@(rd,ul)[rdllllr]
\\
& H_{et}^1(X, \mu_n) \ar[r] & \text{Pic}(X) \ar^{(\cdot)^n}[r] & \text{Pic} (X)
\ar@(rd,ul)[rdllllr] \\
& H_{et}^2(X, \mu_n) \ar[r] & 0 \ar[r] & 0 \cdots
}
$$
The $n$ power map $k^* \to k^*$ is surjective since $k$ is algebraically
closed. So we need to compute the kernel and cokernel of the map $\text{Pic}(X)
\xrightarrow{(\cdot)^n} \text{Pic}(X)$. Consider the commutative diagram with
exact rows
$$
\xymatrix{
0 \ar[r] & {\text{Pic}^0(X)} \ar[r] \ar^{(\cdot)^n}@{>>}[d] & {\text{Pic}(X)}
\ar^{\ \ \deg}[r] \ar^{(\cdot)^n}[d] & {\mathbf{Z}} \ar[r] \ar^{n}@{^{(}->}[d]
& 0\\
0 \ar[r] & {\text{Pic}^0(X)} \ar[r] & {\text{Pic}(X)} \ar^{\ \ \deg}[r] &
{\mathbf{Z}} \ar[r] & 0
}
$$
where the left vertical map is surjective by
Proposition \ref{proposition-review-abelian-varieties} (3).
Applying the snake lemma gives the desired identifications.
\end{proof}

\begin{lemma}
\label{lemma-vanishing-cohomology-mu-smooth-curve}
Let $X$ be an affine smooth curve over an algebraically closed field $k$ and
$n\in k^*$. Then
\begin{enumerate}
\item
$H_{et}^0(X, \mu_n) = \mu_n(k)$;
\item
$H_{et}^1(X, \mu_n) \cong \left(\mathbf{Z}/n\mathbf{Z}\right)^{2g+r-1}$, where
$r$ is the number of points in $\bar X - X$ for some smooth projective
compactification $\bar X$ of $X$ ; and
\item
for all $q\geq 2$, $H_{et}^q(X, \mu_n) = 0$.
\end{enumerate}
\end{lemma}

\begin{proof}
Write $X = \bar X - \{ x_1, \dots, x_r\}$. Then $\text{Pic}(X) =
\text{Pic}(\bar X)/ R$, where $R$ is the subgroup generated by
$\mathcal{O}_{\bar X}(x_i)$, $1 \leq i \leq r$. Since $r \geq 1$, we see that
$\text{Pic}^0(X) \to \text{Pic}(X)$ is surjective, hence $\text{Pic}(X)$ is
divisible. Applying the Kummer sequence, we get {\it i} and {\it iii}. For {\it
ii}, recall that
\begin{eqnarray*}
H_{et}^1(X, \mu_n) & = &
\left\{
(\mathcal L, \alpha) | \mathcal L \in \text{Pic}(X),
\alpha: \mathcal{L}^{\otimes n} \cong \mathcal{O}_X
\right\}
{\bigg /} {\cong} \\
& = &
\left\{(\bar{\mathcal L}, \ D, \ \bar \alpha) \right\} {\big /} \tilde{R}
\end{eqnarray*}
where $\bar{\mathcal L} \in \text{Pic}^0(\bar X)$, $D$ is a divisor on $\bar X$
supported on $\left\{x_1, \cdots, x_r\right\}$ and $ \bar{\alpha}:
\bar{\mathcal L}^{\otimes n} \cong \mathcal{O}_{\bar{X}}(D)$ is an isomorphism.
Note that $D$ must have degree 0. Further $\tilde{R}$ is the subgroup of
triples of the form $(\mathcal{O}_{\bar X}(D'), n D', 1^{\otimes n})$ where
$D'$ is supported on $\left\{x_1, \cdots, x_r\right\}$ and has degree 0. Thus,
we get an exact sequence
$$
0 \longrightarrow
H_{et}^1(\bar X, \mu_n) \longrightarrow
H_{et}^1(X, \mu_n) \longrightarrow
\bigoplus_{i=1}^r \mathbf{Z}/n\mathbf{Z}
\xrightarrow{\ \sum\ }
\mathbf{Z}/n\mathbf{Z} \longrightarrow 0
$$
where the middle map sends the class of a triple $(\bar{ \mathcal L}, D, \bar
\alpha)$ with $D = \sum_{i=1}^r a_i (x_i)$ to the $r$-tuple $(a_i)_{i=1}^r$. It
now suffices to use corollary \ref{lemma-cohomology-smooth-projective-curve} to
count ranks.
\end{proof}

\begin{remark}
\label{remark-natural-proof}
The ``natural'' way to prove the previous corollary is to excise $X$ from $\bar
X$. This is possible, we just haven't developed that theory.
\end{remark}

\noindent
Our main goal is to prove the following result.

\begin{theorem}
\label{theorem-vanishing-curves}
Let $X$ be a separated, finite type, dimension $1$ scheme over an algebraically
closed field $k$ and $\mathcal{F}$ a torsion sheaf on $X_{et}$. Then
$$
H_{et}^q(X, \mathcal{F}) = 0, \quad \forall q\geq 3.
$$
If $X$ affine then also $H_{et}^2(X, \mathcal{F}) = 0$.
\end{theorem}	

\noindent
Recall that an abelian sheaf is called a {\it torsion sheaf} if all of its
stalks are torsion groups. We have computed the cohomology of constant sheaves.
We now generalize the latter notion to get all the way to torsion sheaves.




\section{Constructible sheaves}

\begin{definition}
\label{definition-finite-locally-constant}
Let $X$ be a scheme and $\mathcal{F}$ an abelian sheaf on $X_{et}$. We say that
$\mathcal{F}$ is {\it finite locally constant} if it is represented by a
finite \'etale morphism to $X$.
\end{definition}

\begin{lemma}
\label{lemma-characterize-finite-locally-constant}
Let $X$ be a scheme and $\mathcal{F}$ an abelian sheaf on $X_{et}$. Then the
following are equivalent
\begin{enumerate}
\item
$\mathcal{F}$ is finite locally constant ;
\item
there exists an \'etale covering $\left\{ \mathcal{U}_i \to X\right\}_{i\in I}$
such that $\mathcal{F}|_{\mathcal{U}_i} \cong \underline{A_i}$ for some finite
abelian group $A_i$.
\end{enumerate}
\end{lemma}

\noindent
For a proof, see \cite{SGA4.5}.

\begin{definition}
\label{definition-constructible}
Let $X$ be a quasi-compact and quasi-separated scheme. A sheaf $\mathcal{F}$ on
$X_{et}$ is {\it constructible} if there exists a finite decomposition of $X$
into locally closed subsets $X=\coprod_i X_i$ such that $\mathcal{F}|_{X_i}$ is
finite locally constant for all $i$.
\end{definition}

\begin{lemma}
\label{lemma-kernel-finite-locally-constant}
The kernel and cokernel of a map of finite locally constant sheaves are finite
locally constant.
\end{lemma}

\begin{proof}
Let $\mathcal{U}$ be a connected scheme, $A$ and $B$ finite abelian groups.
Then
$$
\text{Hom}_{\textit{Ab}(\mathcal{U}_{et})} \left(\underline A_\mathcal{U},
\underline B_\mathcal{U}\right) = \text{Hom}_{\textit{Ab}}(A, B),
$$
so $\text{Ker}\left(\underline A_\mathcal{U} \xrightarrow{\varphi} \underline
B_\mathcal{U}\right) = \underline{\text{Ker}(\varphi)}_\mathcal{U}$ and
similarly for the cokernel.
\end{proof}

\begin{remark}
\label{remark-noetherian-constructible-torsion}
If $X$ is noetherian, then (with out definitions)
any constructible sheaf on $X_{et}$ is a torsion sheaf.
\end{remark}

\begin{lemma}
\label{lemma-constructible-abelian}
Let $X$ be a noetherian scheme. Then:
\begin{enumerate}
\item the category of constructible sheaves is abelian ;
\item it is a full exact subcategory of $\textit{Ab}(X_{et})$ ;
\item any extension of constructible sheaves is constructible ; and
\item the image of a map from a constructible sheaf to any other sheaf
is constructible.
\end{enumerate}
\end{lemma}

\begin{proof}
Let $\varphi: \mathcal{F} \to \mathcal{G}$ be a map of constructible sheaves.
By assumption, there exists a stratification $X = \coprod X_i$ such that
$\mathcal{F}|_{X_i}$ and $\mathcal{G}|_{X_i}$ are finite locally constant.
Since pullback if exact, we thus have $\text{Ker} \varphi|_{X_i} = \text{Ker}
(\mathcal{F}|_{X_i}\xrightarrow{\varphi} \mathcal{G}|_{X_i})$ which is finite
locally constant by lemma \ref{lemma-kernel-finite-locally-constant}.
Statement (4) means that if $\varphi :\mathcal{F}\to\mathcal{G}$ is a map in
$\textit{Ab}(X_{et})$ and $\mathcal{F}$ is constructible then $\Im(\varphi)$ is
constructible. It is proven in \cite{SGA4.5}.
\end{proof}

\begin{lemma}
\label{lemma-etale-stratified-finite}
Let $\varphi: \mathcal{U} \to X$ be an \'etale morphism of noetherian schemes.
Then there exists a stratification $X=\coprod_i X_i$ such that for all $i$,
$X_i\times_X \mathcal{U} \to X_i$ is finite \'etale.
\end{lemma}

\begin{proof}
By noetherian induction it suffices to find some nonempty open
$\mathcal{V}\subset X$ such that $\varphi^{-1}(\mathcal{V})\to \mathcal{V}$
is finite. This follows from the following very general lemma.
\end{proof}

\begin{lemma}
\label{lemma-generically-finite}
(Morphisms, Lemma \ref{morphisms-lemma-generically-finite}).
Let $f: X\to Y$ be a quasi-compact and quasi-separated morphism of schemes and
$\eta$ a generic point of $Y$ such that $f^{-1}(\eta)$ is finite. Then there
exists an open $\mathcal{V} \subset Y$ containing $\eta$ such that
$f^{-1}(\mathcal{V})\to \mathcal{V}$ is finite.
\end{lemma}






\section{Extension by zero}
\label{section-extension-by-zero}

\begin{definition}
\label{definition-extension-zero}
Let $j: \mathcal{U} \to X$ be an \'etale morphism of schemes. The restriction
functor $j^{-1}$ is right exact, so it has a left adjoint, denoted
$j_! : \textit{Ab}(\mathcal{U}_{et})\to \textit{Ab}(X_{et})$
and called {\it extension by zero}.
Thus it is characterized by the functorial isomorphism
$$
\text{Hom}_X(j_!\mathcal{F}, \mathcal{G}) =
\text{Hom}_\mathcal{U}(\mathcal{F}, j^{-1}\mathcal{G})
$$
for all $\mathcal{F} \in \textit{Ab}(\mathcal{U}_{et})$ and
$\mathcal{G} \in \textit{Ab}(X_{et})$.
\end{definition}

\noindent
To describe it more explicitly, recall that $j^{-1}$ is just the restriction
functor $\mathcal{U}_{et}\to X_{et}$, that is,
$$
j^{-1}\mathcal{G}(\mathcal{U}'\to \mathcal{U}) = \mathcal{G}
\left(\mathcal{U}'\to \mathcal{U} \xrightarrow{j} X\right).
$$
For $\mathcal{F} \in \textit{Ab}(\mathcal{U}_{et})$ we consider the presheaf
$$
\begin{matrix}
j_!^{\textit{PSh}}\mathcal{F}: & X_{et} &\longrightarrow & \textit{Ab}\\
& (\mathcal{V}\to X) & \longmapsto & \displaystyle
\bigoplus_{\mathcal{V}\xrightarrow{\varphi} \mathcal{U}\text{ over }X}
\mathcal{F}(\mathcal{V}\xrightarrow{\varphi}\mathcal{U}),
\end{matrix}
$$
then $j_!\mathcal{F}$ is the sheafification
$\left(j_!^{\textit{PSh}}\mathcal{F}\right)^\sharp$.

\begin{exercise}
\label{exercise-jshriek-direct}
Prove directly that $j_!$ is left adjoint to $j^{-1}$ and that $j_*$ is right
adjoint to $j^{-1}$.
\end{exercise}

\begin{proposition}
\label{proposition-describe-jshriek}
Let $j : \mathcal{U} \to X$ be an \'etale morphism of schemes. Then
\begin{enumerate}
\item the functors $j^{-1}$ and $j_!$ are exact ;
\item $j^{-1}$ transforms injectives into injectives ;
\item $H_{et}^p(\mathcal{U}, \mathcal{G})= H_{et}^p(\mathcal{U},
j^{-1}\mathcal{G})$ for any $\mathcal{G} \in \textit{Ab}(X_{et})$
\item if $\bar x$ is a geometric point of $X$, then
$\left(j_!\mathcal{F}\right)_{\bar x} =\displaystyle \bigoplus_{(\mathcal{U},
\bar u) \to (X, x)} \mathcal{F}_{\bar{u}}$.
\end{enumerate}
\end{proposition}

\begin{proof}
The functor $j^{-1}$ has both a right and a left adjoint, so it is exact. The
functor $j_!$ has a right adjoint, so it is right exact. To see that it is left
exact, use the description above and the fact that sheafification is exact.
Property {\it ii} is standard general nonsense. In part {\it iii}, the
left-hand side refers (as it should) to the right derived functors of
$\mathcal{G}\mapsto \mathcal{G}(\mathcal{U})$ on $\textit{Ab}(X_{et})$, and the
right-hand side refers to global cohomology on $\textit{Ab}(\mathcal{U}_{et})$.
It is a formal consequence of {\it ii}. Part {\it iv} is again a consequence of
the above description.
\end{proof}

\begin{lemma}
\label{lemma-shriek-base-change}
Extension by zero commutes with base change. More precisely, let $f: Y \to X$
be a morphism of schemes, $j: \mathcal{V} \to X$ be an \'etale morphism and
$\mathcal{F}$ a sheaf on $\mathcal{V}_{et}$. Consider the cartesian diagram
$$
\xymatrix{
{\mathcal{V}'=Y\times_X \mathcal{V}} \ar^{f'}[d] \ar^{\qquad j'}[r] & {Y}
\ar^{f}[d] \\
{\mathcal{V}} \ar^{j}[r] & {X}
}
$$
then $j'_! f'^{-1}\mathcal{F} = f^{-1}j_!\mathcal{F}$.
\end{lemma}

\begin{proof}[Sketch of proof. ]
By general nonsense, there exists a map $j'_! \circ f'^{-1} \to f^{-1}\circ
j_!$. We merely verify that they agree on stalks. We have
$$
\left(j_!'f'^{-1}\mathcal{F}\right)_{\bar y} =
\bigoplus_{\bar v' \to \bar y} (f'^{-1}\mathcal{F})_{\bar v'} =
\bigoplus_{\bar v \to f(\bar y)} \mathcal{F}_{\bar v} =
(j_!\mathcal{F})_{f(\bar y)} =
(f^{-1}j_!\mathcal{F})_{\bar y}.
$$
\end{proof}

\begin{lemma}
\label{lemma-shriek-equals-star-finite-etale}
Let $j: \mathcal{V}\to X$ be finite and \'etale. Then $j_! = j_*$.
\end{lemma}

\begin{proof}[Sketch of proof]
In this situation, one can again construct a map $j_! \to j_*$ although in this
case it is not just by general nonsense and uses the assumptions on $j$. Again,
we only check that the stalks agree. We have on the one hand
$$
(j_!\mathcal{F})_{\bar x} =
\bigoplus_{\bar v \to \bar x} \mathcal{F}_{\bar v},
$$
and on the other hand
$$
\left(j_* \mathcal{F} \right)_{\bar x} = H_{et}^0(\text{Spec}(\mathcal{O}_{X,
\bar x}^\text{sh})\times_X \mathcal{V}, \mathcal{F}).
$$
But $j$ is finite and $\mathcal{O}_{X, \bar x}$ is strictly henselian, hence
$\text{Spec}(\mathcal{O}_{X, \bar x}^\text{sh})\times_X \mathcal{V}$ splits
completely into spectra of strictly henselian local rings
$$
\text{Spec}(\mathcal{O}_{X, \bar x}^\text{sh})\times_X \mathcal{V} =
\coprod_{\bar v \to \bar x} \text{Spec}(\mathcal{O}_{X, \bar x}^\text{sh})
$$
and so $\left(j_* \mathcal{F} \right)_{\bar x} = \prod_{\bar v \to \bar
x} \mathcal{F}_{\bar v}$ by lemma \ref{lemma-shriek-base-change}. Since
finite products and finite coproducts agree, we get the result. Note that this
last step fails if we take infinite colimits, and indeed the result is not true
anymore for ind-morphisms, say.
\end{proof}

\begin{lemma}
\label{lemma-jshriek-constructible}
Let $X$ be a noetherian scheme and $j: \mathcal{U} \to X$ an \'etale,
quasi-compact morphism. Then $j_!\underline{\mathbf{Z}/n\mathbf{Z}}$ is
constructible on $X$.
\end{lemma}

\begin{proof}
By lemma \ref{lemma-etale-stratified-finite}, $X$ has a stratification
$\coprod_i X_i$ such that $\pi_i: j^{-1}(X_i)\to X_i$ is finite \'etale, hence
$$
j_!(\underline{\mathbf{Z}/n\mathbf{Z}})|_{X_i} =
\pi_{i!}(\underline{\mathbf{Z}/n\mathbf{Z}}) =
\pi_{i*}(\underline{\mathbf{Z}/n\mathbf{Z}})
$$
by lemma \ref{lemma-shriek-equals-star-finite-etale}. Thus it suffices to show
that for $\pi: Y\to X$ finite \'etale,
$\pi_*(\underline{\mathbf{Z}/n\mathbf{Z}})$ is finite locally constant. This is
clear because it is the sheaf represented by $Y\times \mathbf{Z}/n\mathbf{Z}$.
\end{proof}

\begin{remark}
\label{remark-alternative}
Using the alternative definition of finite locally constant (as in
\ref{lemma-characterize-finite-locally-constant}), the last step is replaced
by considering a Galois closure of $Y$.
\end{remark}

\begin{lemma}
\label{lemma-torsion-colimit-constructible}
Let $X$ be a noetherian scheme and $\mathcal{F}$ a torsion sheaf on $X_{et}$.
Then $\mathcal{F}$ is a directed (filtered) colimit of constructible sheaves.
\end{lemma}

\begin{proof}[Sketch of proof]
Let $j: \mathcal{U} \to X$ in $X_{et}$ and $s\in \mathcal{F}(\mathcal{U})$ for
some $\mathcal{U}$ noetherian. Then $ns = 0$ for some $n>0$. Hence we get a map
$\underline{\mathbf{Z}/n\mathbf{Z}}_\mathcal{U}\to \mathcal{F}|_\mathcal{U}$,
by sending $\bar 1$ to $s$. By adjointness, this gives a map $\varphi:
j_!(\underline{\mathbf{Z}/n\mathbf{Z}}) \to \mathcal{F}$ whose image contains
$s$. There is an element $1_{\text{id}_\mathcal{U}} \in \Gamma(\mathcal{U},
j_!\underline{\mathbf{Z}/n\mathbf{Z}})$ which maps to $s$. Thus, $\Im(\varphi)
\subset \mathcal{F}$ is a constructible subsheaf and $s\in
\Im(\varphi)(\mathcal{U})$. A similar argument applies for a finite collection
of section, and the result follows by taking colimits.
\end{proof}




\section{Higher vanishing for torsion sheaves}
\label{section-vanishing-torsion}

\noindent
The goal of this section is to prove the result that follows now.

\begin{theorem}
\label{theorem-vanishing-affine-curves}
Let $X$ be an affine curve over an algebraically closed field $k$ and
$\mathcal{F}$ a torsion sheaf on $X_{et}$. Then $H_{et}^q(X, \mathcal{F}) = 0$
for all $q\geq 2$.
\end{theorem}

\noindent
We begin by reducing the proof to a more simpler statement.
\begin{enumerate}
\item[(1)] {\it If suffices to prove the vanishing when $\mathcal{F}$
is a constructible sheaf.}
\end{enumerate}

\noindent
Using the compatibility of \'etale cohomology with colimits and lemma
\ref{lemma-torsion-colimit-constructible}, we have $\text{colim}
H_{et}^q(X, \mathcal{F}) = H_{et}^q(X, \text{colim} \mathcal{F}_i)$ for some
constructible sheaves $\mathcal{F}_i$, whence the result.

\begin{enumerate}
\item[(2)]
{\it It suffices to assume that $\mathcal{F} = j_!\mathcal{G}$ where
$\mathcal{U}\subset X$ is open, $\mathcal{G}$ is finite locally constant on
$\mathcal{U}$ smooth.}
\end{enumerate}

\noindent
Choose a nonempty open $\mathcal{U}\subset X$ such that
$\mathcal{F}|_\mathcal{U}$ is finite locally constant, and consider the exact
sequence
$$
0\to j_!(\mathcal{F}|_\mathcal{U})\to \mathcal{F}\to Q\to 0.
$$
By looking at stalks we get $Q_{\bar x}=0$ unless $\bar x\in X-U$. It follows
that $\displaystyle Q = \bigoplus_{x\in X-U} i_{x*} (Q_x)$
which has no higher cohomology.

\begin{enumerate}
\item[(3)]
{\it It suffices to assume that $X$ is smooth and affine (over $k$),
$\mathcal{G}$ is a finite locally constant sheaf on a open $\mathcal{U}$ of $X$
and $\mathcal{F} = j_!\mathcal{G}$.}
\end{enumerate}

\noindent
Let $\mathcal{U}$, $X$ and $\mathcal{G}$ be as in the step 2, and consider the
commutative diagram
$$
\xymatrix{
& {X^\nu} \ar^{\nu}[d]\\
{\mathcal{U}} \ar^{j}[r] \ar^{j^\nu}[ur] & {X}
}
$$
where $\nu: X^\nu \to X$ is the normalization of $X$. Since $\nu$ is finite,
$H_{et}^*(X, j_!\mathcal{G}) = H_{et}^*(X^\nu, j^\nu_!\mathcal{G})$, which
implies that $\nu_*((j^\nu)_!\mathcal{G}) = j_!\mathcal{G}$ by looking at
stalks. We are thus reduced to proving the following lemma.

\begin{lemma}
\label{lemma-vanishing-easier}
Let $X$ be a smooth affine curve over an algebraically closed field $k$, $j:
\mathcal{U} \hookrightarrow X$ an open immersion and $\mathcal{F}$ a finite
locally constant sheaf on $\mathcal{U}_{et}$. Then for all $q \geq 2$,
$H_{et}^q(X, j_! \mathcal{F}) = 0$.
\end{lemma}


%10.20.09
\noindent
The proof of this follows the
``m\'ethode de la trace''
as explained in \cite[Expos\'e IX, \S5]{SGA4}.

\begin{definition}
\label{definition-trace-map}
Let $f : Y \to X$ be a finite \'etale morphism. There are pairs of adjoint
functors $(f_!,f^{-1})$ and $(f^{-1},f_*)$ on $\textit{Ab}(X_{et})$. The
adjunction map $\text{id} \to f_* f^{-1}$ is called {\it restriction}. Since
$f$ is finite, $f_! = f_*$ and the adjunction map $f_* f^{-1} = f_! f^{-1} \to
\text{id}$ is called the {\it trace}.
\end{definition}

\noindent
The trace map is characterized by the following two properties:
\begin{enumerate}
\item
it commutes with \'etale localization ; and
\item
if $f: Y = \coprod_{i=1}^d X \to X$ then the trace map is just the sum map $f_*
f^{-1} \mathcal{F} = \mathcal{F}^{\oplus d} \to \mathcal{F}$.
\end{enumerate}
It follows that if $f$ has constant degree $d$, then the composition
$\mathcal{F} \xrightarrow{res} f_* f^{-1} \mathcal{F} \xrightarrow{trace}
\mathcal{F}$ is multiplication by $d$. The ``m\'ethode'' then essentially
consits in the following observation: if $\mathcal{F}$ is an abelian sheaf on
$X_{et}$ such that multiplication by $d$ is an isomorphism $\mathcal{F} \cong
\mathcal{F}$, and if furthermore $H_{et}^q(Y,f^{-1}\mathcal{F}) = 0$ then
$H_{et}^q(X,\mathcal{F}) = 0$ as well. Indeed, multiplication by $d$ induces an
isomorphism on $H_{et}^q(X, \mathcal{F})$ which factors through
$H_{et}^q(Y,f^{-1}\mathcal{F})= 0$.

\medskip\noindent
Using this method, we further reduce the proof of lemma
\ref{lemma-vanishing-easier}] to a yet simpler
statement.
\begin{enumerate}
\item[(4)]
{\it We may assume that $\mathcal{F}$ is killed by a prime $\ell$.}
\end{enumerate}
Writing $\mathcal{F} = \mathcal{F}_1 \oplus \cdots \oplus \mathcal{F}_r$ where
$\mathcal{F}_i$ is $\ell_i$-primary for some prime $\ell_i$, we may assume that
$\ell^n$ kills $\mathcal{F}$ for some prime $\ell$. Now consider the exact
sequence
$$
0 \to \mathcal{F}[\ell] \to \mathcal{F} \to \mathcal{F}/\mathcal{F}[\ell] \to 0.
$$
Applying the exact functor $j_!$ and looking at the long exact cohomology
sequence, we see that it suffices to assume that $\mathcal{F}$ is
$\ell$-torsion, which we do.
\begin{enumerate}
\item[(5)]
{\it There exists a finite \'etale morphism $f: \mathcal{V} \to \mathcal{U}$ of
degree prime to $\ell$ such that $f^{-1} \mathcal{F}$ has a filtration
$$
0 \subset \mathcal{G}_1 \subset \mathcal{G}_2 \subset \cdots \subset
\mathcal{G}_s = f^{-1} \mathcal{F}
$$
with $\mathcal{G}_i /\mathcal{G}_{i-1} \cong
\underline{\mathbf{Z}/n\mathbf{Z}}_\mathcal{V}$ for all $i \leq s$.}
\end{enumerate}
Since $\mathcal{F}$ is finite locally constant, there exists a finite \'etale
Galois cover $h : \mathcal{U}' \to \mathcal{U}$ such that $h^{-1} \mathcal{F}
\cong \underline{A}_{\mathcal{U}'}$ for some finite abelian group $A$. Note
that $A \cong (\mathbf{Z}/\ell\mathbf{Z})^{\oplus m}$ for some $m$. Saying that
the cover is {\it Galois} means that the finite group $G =
\text{Aut}(\mathcal{U}' | \mathcal{U})$ has (maximal) cardinality $\# G = \deg
h$. Now let $H \subset G$ be the $\ell$-Sylow, and set
$$
\mathcal{U}' \xrightarrow{\ \ \pi \ \ } \mathcal{V} = \mathcal{U}'/H
\xrightarrow{\ \ f \ \ } \mathcal{U}.
$$
The quotient exists by taking invariants (schemes are affine). By construction,
$\deg f = \#G/\#H$ is prime to $\ell$. The sheaf $\mathcal{G} = f^{-1}
\mathcal{F}$ is then a finite locally constant sheaf on $\mathcal{V}$ and
$$
\pi^{-1} \mathcal{G} = h^{-1}\mathcal{F} \cong
\underline{(\mathbf{Z}/\ell\mathbf{Z})}^{\oplus m}_{\mathcal{U}'}.
$$
Moreover,
$$
H_{et}^0(\mathcal{V}, \mathcal{G}) = H_{et}^0(\mathcal{U}',
\pi^{-1}\mathcal{G})^H = \left((\mathbf{Z}/\ell\mathbf{Z})^{\oplus m}\right)^H
\neq 0,
$$
where the first equality follows from writing out the sheaf condition for
$\mathcal{G}$ (again, schemes are affine), and the last inequality is an
exercise in linear algebra over $\mathbf{F}_\ell$. Following, we have found a
subsheaf $\underline{\mathbf{Z}/\ell\mathbf{Z}}_\mathcal{V} \hookrightarrow
\mathcal{G}$. Repeating the argument for the quotient $\mathcal{G}/
\underline{\mathbf{Z}/\ell\mathbf{Z}}_\mathcal{V}$ if necessary, we eventually
get a subsheaf of $\mathcal{G}$ with quotient
$\underline{\mathbf{Z}/\ell\mathbf{Z}}_\mathcal{V}$. This is the first step of
the filtration.

\begin{exercise}
\label{exercise-finite-etale-under-galois}
Let $f: X \to Y$ be a finite \'etale morphism with $Y$ noetherian, and $X, Y$
irreducible. Then there exists a finite \'etale Galois morphism $X' \to Y$
which dominates $X$ over $Y$.
\end{exercise}

\begin{enumerate}
\item[(6)]
{\it We consider the normalization $Y$ of $X$ in $\mathcal{V}$, that is, we
have the commutative diagram
$$
\xymatrix{
\mathcal{V} \ar^{f}[d] \ar^{j'}@{^{(}->}[r] & Y \ar^{f'}[d] \\
\mathcal{U} \ar^{j}@{^{(}->}[r] & X.
}
$$
Then there is an injection $H_{et}^q(X, j_!\mathcal{F}) \hookrightarrow
H_{et}^q(Y, j'_! f^{-1} \mathcal{F})$ for all $q$.}
\end{enumerate}
We have seen that the composition $\mathcal{F} \xrightarrow{res} f_* f^{-1}
\mathcal{F} \xrightarrow{trace} \mathcal{F}$ is multiplication by the degree of
$f$, which is prime to $\ell$. On the other hand,
$$
j_! f_* f^{-1} \mathcal{F} = j_! f_! f^{-1} \mathcal{F} = f'_* j'_!
f^{-1}\mathcal{F}
$$
since $f$ and $f'$ are both finite and the above diagram is commutative. Hence
applying $j_!$ to the previous sequence gives a sequence
$$
j_! \mathcal{F} \longrightarrow f'^* j'_! f^{-1} \mathcal{F} \longrightarrow
j_! \mathcal{F}.
$$
Taking cohomology, we see that $H_{et}^q(X, j_!\mathcal{F})$ injects into
$H_{et}^q( X , f'^* j'_! f^{-1} \mathcal{F})$. But since $f'$ is finite, this
is merely $H_{et}^q( Y, j'_! f^{-1} \mathcal{F})$, as desired.
\begin{enumerate}
\item[(7)]
{\it It suffices to prove $H_{et}^q (Y, j'_!
\underline{\mathbf{Z}/\ell\mathbf{Z}}) = 0$.}
\end{enumerate}
By Step 3, it suffices to show vanishing of $H_{et}^q( Y, j'_! f^{-1}
\mathcal{F})$. But then by Step 2, we may assume that $f^{-1}\mathcal{F}$ has a
finite filtration with quotients isomorphic to
$\underline{\mathbf{Z}/n\mathbf{Z}}$, whence the claim.

\medskip\noindent
Finally, we are reduced to proving the following lemma.

\begin{lemma}
\label{lemma-even-easier}
Let $X$ be a smooth affine curve over an algebraically closed field, $j:
\mathcal{U} \hookrightarrow X$ an open immersion and $\ell$ a prime number.
Then for all $q \geq 2$, $H_{et}^q(X, j_!
\underline{\mathbf{Z}/\ell\mathbf{Z}}) = 0$.
\end{lemma}

\begin{proof}
Consider the short exact sequence
$$
0 \longrightarrow j_!\underline{\mathbf{Z}/\ell\mathbf{Z}}_\mathcal{U}
\longrightarrow \underline{\mathbf{Z}/\ell\mathbf{Z}}_X \longrightarrow
\bigoplus_{x \in X-\mathcal{U}} {i_x}_*(\underline{\mathbf{Z}/\ell\mathbf{Z}})
\longrightarrow 0.
$$
We know that the cohomology of the middle sheaf vanishes in degree at least 2
by
Lemma \ref{lemma-vanishing-cohomology-mu-smooth-curve}
and that of the skyscraper
sheaf on the right vanishes in degree at least 1. Thus applying the long exact
cohomology sequence, we get the vanishing of
$j_!\underline{\mathbf{Z}/\ell\mathbf{Z}}_\mathcal{U}$ in degree at least 2.
This finishes the proof of the lemma, hence of lemma
\ref{lemma-vanishing-easier}, hence of theorem
\ref{theorem-vanishing-affine-curves}.
\end{proof}

\begin{remarks}
\label{remarks-on-above}
Here are some remarks about what happened above.
\begin{itemize}
\item This method is very general. For instance, it applies in Galois
cohomology, and this is essentially how
Proposition \ref{proposition-serre-galois} is proved.
\item In fact, we have overlooked the case where $\ell$ is the characteristic
of the field $k$, since the Kummer sequence is not exact then and we cannot
use Lemma \ref{lemma-vanishing-cohomology-mu-smooth-curve} anymore.
The result is still true, as shown by considering the
{\it Artin-Schreier} exact sequence for a scheme $S$ of characteristic
$p >0$, namely
$$
0 \longrightarrow \underline{\mathbf{Z}/p\mathbf{Z}}_S \longrightarrow
\mathbf{G}_{a,S} \xrightarrow{F-1} \mathbf{G}_{a,S} \longrightarrow 0
$$
where $F - 1$ is the map $x \mapsto x^p - x$. Using this, it can be
shown that is
$S$ is affine then $H_{et}^q(S,\underline{\mathbf{Z}/p\mathbf{Z}}) = 0$ for all
$q \geq 2$. In fact, if $X$ is projective over $k$, then
$H_{et}^q(X,\underline{\mathbf{Z}/p\mathbf{Z}}) = 0$ for all $q \geq \dim X+2$.
\item If $X$ is a projective curve over an algebraically closed field then
$H_{et}^q(X,\mathcal{F}) = 0$ for all $q \geq 3$ and all torsion sheaves
$\mathcal{F}$ on $X_{et}$. This can be shown using Serre's Mayer Vietoris
argument, thereby proving theorem \ref{theorem-vanishing-curves}.
\item We can prove using the same methods vanishing of higher cohomology
on $1$-dimensional schemes of finite type over an algebraically closed field.
However, it is easier to reduce to the case of a curve by using the
topoliogical invariance of \'etale cohomology as stated below.
\end{itemize}
\end{remarks}

\begin{proposition}
\label{proposition-topological-invariance}
(Topological invariance of \'etale cohomology)
Let $X$ be a scheme and $X_0\hookrightarrow X$ a closed immersion defined by a
nilpotent sheaf of ideals. Then the \'etale sites $X_{et}$ and $(X_0)_{et}$ are
isomorphic. In particular, for any sheaf $\mathcal{F}$ on $X_{et}$, $H^q(X,
\mathcal{F}) = H^q(X_0, \mathcal{F}|_{X_0})$ for all $q$.
\end{proposition}

\section{Other chapters}

\begin{multicols}{2}
\begin{enumerate}
\item \hyperref[introduction-section-phantom]{Introduction}
\item \hyperref[conventions-section-phantom]{Conventions}
\item \hyperref[sets-section-phantom]{Set Theory}
\item \hyperref[categories-section-phantom]{Categories}
\item \hyperref[topology-section-phantom]{Topology}
\item \hyperref[sheaves-section-phantom]{Sheaves on Spaces}
\item \hyperref[algebra-section-phantom]{Commutative Algebra}
\item \hyperref[sites-section-phantom]{Sites and Sheaves}
\item \hyperref[homology-section-phantom]{Homological Algebra}
\item \hyperref[derived-section-phantom]{Derived Categories}
\item \hyperref[more-algebra-section-phantom]{More Algebra}
\item \hyperref[simplicial-section-phantom]{Simplicial Methods}
\item \hyperref[modules-section-phantom]{Sheaves of Modules}
\item \hyperref[sites-modules-section-phantom]{Modules on Sites}
\item \hyperref[injectives-section-phantom]{Injectives}
\item \hyperref[cohomology-section-phantom]{Cohomology of Sheaves}
\item \hyperref[sites-cohomology-section-phantom]{Cohomology on Sites}
\item \hyperref[hypercovering-section-phantom]{Hypercoverings}
\item \hyperref[schemes-section-phantom]{Schemes}
\item \hyperref[constructions-section-phantom]{Constructions of Schemes}
\item \hyperref[properties-section-phantom]{Properties of Schemes}
\item \hyperref[morphisms-section-phantom]{Morphisms of Schemes}
\item \hyperref[coherent-section-phantom]{Coherent Cohomology}
\item \hyperref[divisors-section-phantom]{Divisors}
\item \hyperref[limits-section-phantom]{Limits of Schemes}
\item \hyperref[varieties-section-phantom]{Varieties}
\item \hyperref[chow-section-phantom]{Chow Homology}
\item \hyperref[topologies-section-phantom]{Topologies on Schemes}
\item \hyperref[descent-section-phantom]{Descent}
\item \hyperref[more-morphisms-section-phantom]{More on Morphisms}
\item \hyperref[flat-section-phantom]{More on Flatness}
\item \hyperref[groupoids-section-phantom]{Groupoid Schemes}
\item \hyperref[more-groupoids-section-phantom]{More on Groupoid Schemes}
\item \hyperref[etale-section-phantom]{\'Etale Morphisms of Schemes}
\item \hyperref[etale-cohomology-section-phantom]{\'Etale Cohomology}
\item \hyperref[spaces-section-phantom]{Algebraic Spaces}
\item \hyperref[spaces-properties-section-phantom]{Properties of Algebraic Spaces}
\item \hyperref[spaces-morphisms-section-phantom]{Morphisms of Algebraic Spaces}
\item \hyperref[spaces-topologies-section-phantom]{Topologies on Algebraic Spaces}
\item \hyperref[spaces-descent-section-phantom]{Descent and Algebraic Spaces}
\item \hyperref[spaces-more-morphisms-section-phantom]{More on Morphisms of Spaces}
\item \hyperref[quot-section-phantom]{Quot and Hilbert Spaces}
\item \hyperref[stacks-section-phantom]{Stacks}
\item \hyperref[spaces-groupoids-section-phantom]{Groupoids in Algebraic Spaces}
\item \hyperref[spaces-more-groupoids-section-phantom]{More on Groupoids in Spaces}
\item \hyperref[bootstrap-section-phantom]{Bootstrap}
\item \hyperref[examples-stacks-section-phantom]{Examples of Stacks}
\item \hyperref[groupoids-quotients-section-phantom]{Quotients of Groupoids}
\item \hyperref[algebraic-section-phantom]{Algebraic Stacks}
\item \hyperref[criteria-section-phantom]{Criteria for Representability}
\item \hyperref[stacks-properties-section-phantom]{Properties of Algebraic Stacks}
\item \hyperref[stacks-morphisms-section-phantom]{Morphisms of Algebraic Stacks}
\item \hyperref[examples-section-phantom]{Examples}
\item \hyperref[exercises-section-phantom]{Exercises}
\item \hyperref[guide-section-phantom]{Guide to Literature}
\item \hyperref[desirables-section-phantom]{Desirables}
\item \hyperref[coding-section-phantom]{Coding Style}
\item \hyperref[fdl-section-phantom]{GNU Free Documentation License}
\item \hyperref[index-section-phantom]{Auto Generated Index}
\end{enumerate}
\end{multicols}


\bibliography{my}
\bibliographystyle{amsalpha}

\end{document}
