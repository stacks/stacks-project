\IfFileExists{stacks-project.cls}{%
\documentclass{stacks-project}
}{%
\documentclass{amsart}
}

% The following AMS packages are automatically loaded with
% the amsart documentclass:
%\usepackage{amsmath}
%\usepackage{amssymb}
%\usepackage{amsthm}

% For dealing with references we use the comment environment
\usepackage{verbatim}
\newenvironment{reference}{\comment}{\endcomment}
%\newenvironment{reference}{}{}
\newenvironment{slogan}{\comment}{\endcomment}
\newenvironment{history}{\comment}{\endcomment}

% For commutative diagrams you can use
% \usepackage{amscd}
\usepackage[all]{xy}

% We use 2cell for 2-commutative diagrams.
\xyoption{2cell}
\UseAllTwocells

% To put source file link in headers.
% Change "template.tex" to "this_filename.tex"
% \usepackage{fancyhdr}
% \pagestyle{fancy}
% \lhead{}
% \chead{}
% \rhead{Source file: \url{template.tex}}
% \lfoot{}
% \cfoot{\thepage}
% \rfoot{}
% \renewcommand{\headrulewidth}{0pt}
% \renewcommand{\footrulewidth}{0pt}
% \renewcommand{\headheight}{12pt}

\usepackage{multicol}

% For cross-file-references
\usepackage{xr-hyper}

% Package for hypertext links:
\usepackage{hyperref}

% For any local file, say "hello.tex" you want to link to please
% use \externaldocument[hello-]{hello}
\externaldocument[introduction-]{introduction}
\externaldocument[conventions-]{conventions}
\externaldocument[sets-]{sets}
\externaldocument[categories-]{categories}
\externaldocument[topology-]{topology}
\externaldocument[sheaves-]{sheaves}
\externaldocument[sites-]{sites}
\externaldocument[stacks-]{stacks}
\externaldocument[fields-]{fields}
\externaldocument[algebra-]{algebra}
\externaldocument[brauer-]{brauer}
\externaldocument[homology-]{homology}
\externaldocument[derived-]{derived}
\externaldocument[simplicial-]{simplicial}
\externaldocument[more-algebra-]{more-algebra}
\externaldocument[smoothing-]{smoothing}
\externaldocument[modules-]{modules}
\externaldocument[sites-modules-]{sites-modules}
\externaldocument[injectives-]{injectives}
\externaldocument[cohomology-]{cohomology}
\externaldocument[sites-cohomology-]{sites-cohomology}
\externaldocument[dga-]{dga}
\externaldocument[dpa-]{dpa}
\externaldocument[hypercovering-]{hypercovering}
\externaldocument[schemes-]{schemes}
\externaldocument[constructions-]{constructions}
\externaldocument[properties-]{properties}
\externaldocument[morphisms-]{morphisms}
\externaldocument[coherent-]{coherent}
\externaldocument[divisors-]{divisors}
\externaldocument[limits-]{limits}
\externaldocument[varieties-]{varieties}
\externaldocument[topologies-]{topologies}
\externaldocument[descent-]{descent}
\externaldocument[perfect-]{perfect}
\externaldocument[more-morphisms-]{more-morphisms}
\externaldocument[flat-]{flat}
\externaldocument[groupoids-]{groupoids}
\externaldocument[more-groupoids-]{more-groupoids}
\externaldocument[etale-]{etale}
\externaldocument[chow-]{chow}
\externaldocument[intersection-]{intersection}
\externaldocument[pic-]{pic}
\externaldocument[adequate-]{adequate}
\externaldocument[dualizing-]{dualizing}
\externaldocument[duality-]{duality}
\externaldocument[discriminant-]{discriminant}
\externaldocument[local-cohomology-]{local-cohomology}
\externaldocument[curves-]{curves}
\externaldocument[resolve-]{resolve}
\externaldocument[models-]{models}
\externaldocument[pione-]{pione}
\externaldocument[etale-cohomology-]{etale-cohomology}
\externaldocument[proetale-]{proetale}
\externaldocument[crystalline-]{crystalline}
\externaldocument[spaces-]{spaces}
\externaldocument[spaces-properties-]{spaces-properties}
\externaldocument[spaces-morphisms-]{spaces-morphisms}
\externaldocument[decent-spaces-]{decent-spaces}
\externaldocument[spaces-cohomology-]{spaces-cohomology}
\externaldocument[spaces-limits-]{spaces-limits}
\externaldocument[spaces-divisors-]{spaces-divisors}
\externaldocument[spaces-over-fields-]{spaces-over-fields}
\externaldocument[spaces-topologies-]{spaces-topologies}
\externaldocument[spaces-descent-]{spaces-descent}
\externaldocument[spaces-perfect-]{spaces-perfect}
\externaldocument[spaces-more-morphisms-]{spaces-more-morphisms}
\externaldocument[spaces-flat-]{spaces-flat}
\externaldocument[spaces-groupoids-]{spaces-groupoids}
\externaldocument[spaces-more-groupoids-]{spaces-more-groupoids}
\externaldocument[bootstrap-]{bootstrap}
\externaldocument[spaces-pushouts-]{spaces-pushouts}
\externaldocument[groupoids-quotients-]{groupoids-quotients}
\externaldocument[spaces-more-cohomology-]{spaces-more-cohomology}
\externaldocument[spaces-simplicial-]{spaces-simplicial}
\externaldocument[formal-spaces-]{formal-spaces}
\externaldocument[restricted-]{restricted}
\externaldocument[spaces-resolve-]{spaces-resolve}
\externaldocument[formal-defos-]{formal-defos}
\externaldocument[defos-]{defos}
\externaldocument[cotangent-]{cotangent}
\externaldocument[examples-defos-]{examples-defos}
\externaldocument[algebraic-]{algebraic}
\externaldocument[examples-stacks-]{examples-stacks}
\externaldocument[stacks-sheaves-]{stacks-sheaves}
\externaldocument[criteria-]{criteria}
\externaldocument[artin-]{artin}
\externaldocument[quot-]{quot}
\externaldocument[stacks-properties-]{stacks-properties}
\externaldocument[stacks-morphisms-]{stacks-morphisms}
\externaldocument[stacks-limits-]{stacks-limits}
\externaldocument[stacks-cohomology-]{stacks-cohomology}
\externaldocument[stacks-perfect-]{stacks-perfect}
\externaldocument[stacks-introduction-]{stacks-introduction}
\externaldocument[stacks-more-morphisms-]{stacks-more-morphisms}
\externaldocument[stacks-geometry-]{stacks-geometry}
\externaldocument[moduli-]{moduli}
\externaldocument[moduli-curves-]{moduli-curves}
\externaldocument[examples-]{examples}
\externaldocument[exercises-]{exercises}
\externaldocument[guide-]{guide}
\externaldocument[desirables-]{desirables}
\externaldocument[coding-]{coding}
\externaldocument[obsolete-]{obsolete}
\externaldocument[fdl-]{fdl}
\externaldocument[index-]{index}

% Theorem environments.
%
\theoremstyle{plain}
\newtheorem{theorem}[subsection]{Theorem}
\newtheorem{proposition}[subsection]{Proposition}
\newtheorem{lemma}[subsection]{Lemma}

\theoremstyle{definition}
\newtheorem{definition}[subsection]{Definition}
\newtheorem{example}[subsection]{Example}
\newtheorem{exercise}[subsection]{Exercise}
\newtheorem{situation}[subsection]{Situation}

\theoremstyle{remark}
\newtheorem{remark}[subsection]{Remark}
\newtheorem{remarks}[subsection]{Remarks}

\numberwithin{equation}{subsection}

% Macros
%
\def\lim{\mathop{\rm lim}\nolimits}
\def\colim{\mathop{\rm colim}\nolimits}
\def\Spec{\mathop{\rm Spec}}
\def\Hom{\mathop{\rm Hom}\nolimits}
\def\Ext{\mathop{\rm Ext}\nolimits}
\def\SheafHom{\mathop{\mathcal{H}\!{\it om}}\nolimits}
\def\SheafExt{\mathop{\mathcal{E}\!{\it xt}}\nolimits}
\def\Sch{\textit{Sch}}
\def\Mor{\mathop{\rm Mor}\nolimits}
\def\Ob{\mathop{\rm Ob}\nolimits}
\def\Sh{\mathop{\textit{Sh}}\nolimits}
\def\NL{\mathop{N\!L}\nolimits}
\def\proetale{{pro\text{-}\acute{e}tale}}
\def\etale{{\acute{e}tale}}
\def\QCoh{\textit{QCoh}}
\def\Ker{\mathop{\rm Ker}}
\def\Im{\mathop{\rm Im}}
\def\Coker{\mathop{\rm Coker}}
\def\Coim{\mathop{\rm Coim}}

%
% Macros for moduli stacks/spaces
%
\def\QCohstack{\mathcal{QC}\!{\it oh}}
\def\Cohstack{\mathcal{C}\!{\it oh}}
\def\Spacesstack{\mathcal{S}\!{\it paces}}
\def\Quotfunctor{{\rm Quot}}
\def\Hilbfunctor{{\rm Hilb}}
\def\Curvesstack{\mathcal{C}\!{\it urves}}
\def\Polarizedstack{\mathcal{P}\!{\it olarized}}
\def\Complexesstack{\mathcal{C}\!{\it omplexes}}
% \Pic is the operator that assigns to X its picard group, usage \Pic(X)
% \Picardstack_{X/B} denotes the Picard stack of X over B
% \Picardfunctor_{X/B} denotes the Picard functor of X over B
\def\Pic{\mathop{\rm Pic}\nolimits}
\def\Picardstack{\mathcal{P}\!{\it ic}}
\def\Picardfunctor{{\rm Pic}}
\def\Deformationcategory{\mathcal{D}\!{\it ef}}


% OK, start here.
%
\begin{document}

\title{Crystalline Cohomology}


\maketitle

\phantomsection
\label{section-phantom}

\tableofcontents



\section{Introduction}
\label{section-introduction}

\noindent
This chapter is based on a lecture series given by Johan de Jong
held in 2012 at Columbia University.
The goals of this chapter are to give a quick introduction to
crystalline cohomology. A reference is the book \cite{Berthelot}.





\section{Divided powers}
\label{section-divided-powers}

\noindent
In this section we collect some results on divided power rings.
We will use the convention $0! = 1$ (as empty products should give $1$).

\begin{definition}
\label{definition-divided-powers}
Let $A$ be a ring. Let $I$ be an ideal of $A$. A collection of maps
$\gamma_n : I \to I$, $n > 0$ is called a {\it divided power structure}
on $I$ if for all $n \geq 0$, $m > 0$, $x, y \in I$, and $a \in A$ we have
\begin{enumerate}
\item $\gamma_1(x) = x$, we also set $\gamma_0(x) = 1$,
\item $\gamma_n(x)\gamma_m(x) = \frac{(n + m)!}{n! m!} \gamma_{n + m}(x)$,
\item $\gamma_n(ax) = a^n \gamma_n(x)$,
\item $\gamma_n(x + y) = \sum_{i = 0, \ldots, n} \gamma_i(x)\gamma_{n - i}(y)$,
\item $\gamma_n(\gamma_m(x)) = \frac{(nm)!}{n! (m!)^n} \gamma_{nm}(x)$.
\end{enumerate}
\end{definition}

\noindent
Some observations. Note that condition (2) implies that
$n \gamma_n(x) = \gamma_1(x)\gamma_{n - 1}(x)$. Hence by induction
and condition (1) we get $n! \gamma_n(x) = x^n$. Thus $\gamma_n(x)$
is a replacement for $x^n/n!$ in $I$. If $A$ is torsion free as a
$\mathbf{Z}$-module, then all the other axioms follow from this. 
Note that the rational numbers $\frac{(n + m)!}{n! m!}$ and
$\frac{(nm)!}{n! (m!)^n}$ occuring in the definition are in fact integers.

\begin{example}
\label{example-ideal-generated-by-p}
Let $p$ be a prime number.
Let $A$ be a ring such that every integer $n$ not divisible by $p$
is invertible, i.e., $A$ is a $\mathbf{Z}_{(p)}$-algebra. Then
$I = pA$ has a canonical divided power structure. Namely, given
$x = pa \in A$ we set
$$
\gamma_n(x) = \frac{p^n}{n!} a^n
$$
The reader verifies immediately that $p^n/n!$ is an integer so that
the definition makes sense. It is a straightforward exercise to
verify that conditions (1) -- (5) of
Definition \ref{definition-divided-powers} are satisfied.
\end{example}

\begin{lemma}
\label{lemma-need-only-gamma-p}
Let $p$ be a prime number. Let $A$ be a ring such that every integer $n$
not divisible by $p$ is invertible, i.e., $A$ is a $\mathbf{Z}_{(p)}$-algebra.
Let $I \subset A$ be an ideal. Two divided power structures
$\gamma, \gamma'$ on $I$ are equal if and only if $\gamma_p = \gamma'_p$.
Moreover, given a map $\delta : I \to I$ such that
\begin{enumerate}
\item $p!\delta(x) = x^p$ for all $x \in I$,
\item $\delta(ax) = a^p\delta(x)$ for all $a \in A$, $x \in I$, and
\item
$\delta(x + y) =
\delta(x) +
\sum\nolimits_{i + j = p, i,j \geq 1} \frac{x^i}{i!}\frac{y^j}{j!} +
\delta(y)$ for all $x, y \in I$,
\end{enumerate}
then there exists a unique divided power structure $\gamma$ on $I$ such
that $\gamma_p = \delta$.
\end{lemma}

\begin{proof}
If $n$ is not divisible by $p$, then $\gamma_n(x) = c x \gamma_{n - 1}(x)$
where $c$ is a unit in $\mathbf{Z}_{(p)}$. Moreover,
$$
\gamma_{pm}(x) = c \gamma_m(\gamma_p(x))
$$
where $c$ is a unit in $\mathbf{Z}_{(p)}$. Thus the first assertion is clear.
For the second assertion, we can, working backwards, use these equalities
to define all $\gamma_n$. Then a long and tedious computation shows all
the axioms are satisfied if $\delta$ satisfies (1), (2), (3).
Details omitted.
\end{proof}

\begin{definition}
\label{definition-divided-power-ring}
A {\it divided power ring} is a triple $(A, I, \gamma)$ where
$A$ is a ring, $I \subset A$ is an ideal, and $\gamma = (\gamma_n)_{n \geq 1}$
is a divided power structure on $I$.
A {\it homomorphism of divided power rings}
$\varphi : (A, I, \gamma) \to (B, J, \delta)$ is a ring homomorphism
$\varphi : A \to B$ such that $\varphi(I) \subset J$ and such that
$\delta_n(\varphi(x)) = \varphi(\gamma_n(x))$ for all $x \in I$.
\end{definition}

\noindent
A very useful example is the divided power polynomial algebra.

\begin{example}
\label{example-polynomial-algebra}
Let $A$ be a ring. Let $x$ be a variable. We will denote
$A\langle x \rangle$ the following $A$-algebra: As an
$A$-module we set
$$
A\langle x \rangle =
Ax^{[0]} \oplus Ax^{[1]} \oplus Ax^{[2]} \oplus Ax^{[3]} \oplus \ldots
$$
with multiplication given by $x^{[n]}x^{[m]} =
\frac{(n + m)!}{n!m!}x^{[n + m]}$. Of course we identify $x^{[0]} = 1$
and $x^{[1]} = x$. Now, let $I \subset A$ be an ideal endowed with a
divided power structure $\gamma$. Claim: we can endow the ideal
$J = (I, x^{[n]}, n \geq 1)$ with a unique divided power structure $\delta$
which recovers $\gamma$ on elements of $I$ and such that
$\delta_n(x^{[m]}) = \frac{(nm)!}{n!(m!)^n} x^{[nm]}$.
If this is possible, then the following expression defines $\delta_n$
on a general element
$$
\delta_n\left(\sum a_m x^{[m]}\right) =
\sum_{m_1, \ldots, m_t \geq 1, n_0 + n_1 + \ldots + n_t = n}
\gamma_{n_0}(a_0)
\prod_{j = 1, \ldots, t}
a_{m_j}^{n_j} \frac{(n_j m_j)!}{n_j!(m_j!)^{n_j}} x^{[n_jm_j]}
$$
Hence we take this as our definition of $\delta_n$ and we verify that
it satisfies the conditions of the definition.

\medskip\noindent
We first prove this in case $A$ is torsion free as an $\mathbf{Z}$-module.
In that case it suffices to prove the result after tensoring with $\mathbf{Q}$.
Then $\gamma_n(a) = \frac{1}{n!}a^n$ for $a \in I$ and
$x^{[n]} = \frac{1}{n!}x^n$. Thus the formula above is just the formula
for the expansion of
$$
\frac{1}{n!}\left(\sum a_m x^{[m]}\right)^n
$$
and we see everything is OK.

\medskip\noindent
Before we prove the claim in general, note that if the claim holds
for $(A, I, \gamma)$, then $(A\langle x \rangle, J, \delta)$ has the
following universal property: A homomorphism of divided power rings
$\varphi : (A\langle x \rangle, J, \delta) \to (C, K, \epsilon)$ is
the same thing as a homomorphism of divided power rings
$A \to C$ and an element $k \in K$. Namely, given
$A \to C$ and $k \in K$ we let $\varphi : A\langle x \rangle \to C$
be the map which sends $x^{[n]}$ to $\epsilon_n(k)$.
Moreover, we can repeat this and get divided power polynomial algebras
$$
A\langle x_1, \ldots, x_n\rangle
$$
with a similar universal property.

\medskip\noindent
Back to the proof of the claim in the general case. Suppose we want
to prove for example that
$\delta_n(\alpha + \beta) = \sum_{i + j = n} \delta_i(\alpha)\delta_j(\beta)$
for some $\alpha, \beta \in A\langle x \rangle$.
Write $\alpha = a_0 + \sum_{d \geq m > 0} a_m x^{[m]}$ and
$\beta = b_0 + \sum_{d \geq m > 0} b_m x^{[m]}$.
By the universal property proven above there exists a homomorphism
$$
\varphi :
A_0 =
\mathbf{Z}[t_1, \ldots, t_d, s_1, \ldots, s_d]\langle t_0, s_0 \rangle
\longrightarrow
A
$$
such that $t_m \mapsto a_m$, $s_m \mapsto b_m$, $t_0 \mapsto a_0$, and
$s_0 \mapsto b_0$. And since $A_0$ is torsion free the desired relation
holds for $\alpha_0 = t_0 + \sum_{d \geq m > 0} t_m x^{[m]}$ and
$\beta_0 = s_0 + \sum_{d \leq m > 0} s_i x^{[m]}$ in
$A_0\langle x \rangle$. Taking the image in $A$ via the induced
map $A_0\langle x \rangle \to A\langle x \rangle$ (this is obtained
from $\varphi$ and not from the universal property)
we obtain the desired relation in $A\langle x \rangle$.
Similarly for the other identities.
\end{example}


\begin{lemma}
\label{lemma-nil}
Let $p$ be a prime number. Let $(A, I, \gamma)$ be a divided power ring
and assume $p$ is nilpotent in $A/I$.
Then $I$ is locally nilpotent if and only if $p$ is nilpotent in $A$.
\end{lemma}

\begin{proof}
If $p^N = 0$ in $A$, then for $x \in I$ we have
$x^{pN} = (pN)!\gamma_N(x) = 0$ because $(pN)!$ is
divisible by $p^N$. Conversely, assume $I$ is locally nilpotent.
We've also assumed that $p$ is nilpotent in $A/I$, hence
$p^r \in I$ for some $r$, hence $p^r$ nilpotent, hence $p$ nilpotent.
\end{proof}

\begin{lemma}
\label{lemma-colimits}
The category of divided power rings has all limits and colimits.
\end{lemma}

\begin{proof}
The empty limit is the zero ring (that's weird but we need it).
The product of a collection of divided power rings $(A_t, I_t, \gamma_t)$,
$t \in T$ is given by $(\prod A_t, \prod I_t, \gamma)$ where
$\gamma_n((x_t)) = (\gamma_{t, n}(x_t))$.
The equalizer of $\alpha, \beta : (A, I, \gamma) \to (B, J, \delta)$
is just $C = \{a \in A \mid \alpha(a) = \beta(a)\}$ with ideal $C \cap I$
and induced divided powers. It follows that all limits exist, see
Categories, Lemma \ref{categories-lemma-limits-products-equalizers}.
Moreover, limits commute with the forgetful functor from divided
power rings to rings.

\medskip\noindent
The empty colimit is $\mathbf{Z}$ with divided power ideal $(0)$.
Let's discuss general colimits. Let $\mathcal{C}$ be a category and let
$c \mapsto (A_c, I_c, \gamma_c)$ be a diagram. Consider the functor
$$
F(B, J, \delta) = \lim_{c \in \mathcal{C}}
Hom((A_c, I_c, \gamma_c), (B, J, \delta))
$$
Note that any $f = (f_c)_{c \in C} \in F(B, J, \delta)$ has the property
that all the images $f_c(A_c)$ generate a subring $B'$ of $B$ of bounded
cardinality $\kappa$ and that all the images $f_c(I_c)$ generate a
divided power sub ideal $J'$ of $B'$. And we get a factorization of
$f$ as a $f'$ in $F(B')$ followed by the inclusion $B' \to B$.
Hence we see that $F(B, J, \delta)$ can be computed if we know the value
of $F$ on divided power rings of
cardinality $\kappa$ or less. Consider a set of objects $U$ of
dividing power rings containing an object isomorphic to every
$(B, J, \delta)$ of cardinality
at most $\kappa$. Then
$$
\lim_{(B, J, \delta) \in U, f \in F(B, J, \delta)} (B, J, \delta)
$$
will be the colimit we wanted to construct.
\end{proof}

\begin{lemma}
\label{lemma-divided-power-envelope}
Let $(A, I, \gamma)$ be a divided power ring.
Let $A \to B$ be a ring map. Let $J \subset B$ be an ideal
with $IB \subset J$. There exists a homomorphism of
divided power rings
$$
(A, I, \gamma) \longrightarrow (D_B(J), \bar J, \bar \gamma)
$$
such that
$$
\Hom_{(A, I, \gamma)}((D_B(J), \bar J, \bar \gamma), (C, K, \delta))
=
\Hom_A((B, J), (C, K))
$$
functorially in the divided power algebra $(C, K, \delta)$ over
$(A, I, \gamma)$.
\end{lemma}

\begin{proof}
Entirely similar to the proof of Lemma \ref{lemma-colimits}.
\end{proof}

\begin{definition}
\label{definition-divided-power-envelope}
Let $(A, I, \gamma)$ be a divided power ring.
Let $A \to B$ be a ring map. Let $J \subset B$ be an ideal
with $IB \subset J$. The divided power algebra $D_B(J)$
constructed in Lemma \ref{lemma-divided-power-envelope}
is called the {\it divided power envelope of $J$ in $B$
relative to $(A, I, \gamma)$}.
\end{definition}

\noindent
Discussion. There are $A$-algebra maps
\begin{equation}
\label{equation-divided-power-envelope}
B \longrightarrow D_B(J) \longrightarrow B/J
\end{equation}
The first arrow maps $J$ into $\bar J$ and $\bar J$ is the kernel
of the second arrow. Also $\bar J$ is generated by the elements
$\bar\gamma_n(x)$ where $x$ is in the image of $J \to D_B(J)$.
Let $(A, I, \gamma) \to (C, K, \delta)$ be a homomorphism of divided
power rings. The universal property is
just that ring maps $B \to C$ which map $J$ into $K$ correspond
1-to-1 to homomorphisms of divided power rings
$(D_B(J), \bar J, \bar \gamma) \to (C, K, \delta)$ via precomposing with the
canonical map $B \to D_B(J)$.

\medskip\noindent
The following lemma can be generalized to the case where $B$ and $B'$
come with their own divided power ideals... see
\cite[Proposition 2.1.7]{dJ-crystalline}.
Anyway, it in particular says that taking the divided power
envelope commutes with localization.

\begin{lemma}
\label{lemma-flat-base-change-divided-power-envelope}
Let $(A, I, \gamma)$ be a divided power ring.
Let $B \to B'$ be a homomorphism of $A$-algebras.
Let $IB \subset J \subset B$ be an ideal.
Assume that $B/IB \to B'/IB'$ is flat.
Then $D_B(J) \otimes_B B' = D_{B'}(JB')$.
\end{lemma}

\begin{proof}
Omitted.
\end{proof}

\begin{lemma}
\label{lemma-gamma-extends}
Let $(A, I, \gamma)$ be a divided power ring.
Let $A \to B$ be a ring map. Then $\gamma$ 
extends to $IB$ if one of the following conditions is satisfied:
\begin{enumerate}
\item $IB = 0$,
\item $A \to B$ is flat, or
\item $I$ is (locally) principal.
\end{enumerate}
\end{lemma}

\begin{proof}
Omitted.
\end{proof}

\begin{lemma}
\label{lemma-divided-power-first-order-thickening}
Let $(A, I, \gamma)$ be a divided power ring. Let $M$ be an $A$-module.
Let $B = A \oplus M$ as an $A$-algebra where $M$ is an ideal of square zero
and set $J = I \oplus M$. Set
$$
\delta_n(x + z) = \gamma_n(x) + \gamma_{n - 1}(x)z
$$
for $x \in I$ and $z \in M$.
Then $\delta$ is a divided power structure and
$A \to B$ is a homomorphism of divided power rings from
$(A, I, \gamma)$ to $(B, J, \delta)$.
\end{lemma}

\begin{proof}
We have to check conditions (1) -- (5) of
Definition \ref{definition-divided-powers}.
We will prove this directly for this case, but please see the proof of
the next lemma for a method which avoids calculations.
Conditions (1) and (3) are clear. Condition (2) follows from
\begin{align*}
\delta_n(x + z)\delta_m(x + z)
& =
(\gamma_n(x) + \gamma_{n - 1}(x)z)(\gamma_m(x) + \gamma_{m - 1}(x)z) \\
& = \gamma_n(x)\gamma_m(x) + \gamma_n(x)\gamma_{m - 1}(x)z +
\gamma_{n - 1}(x)\gamma_m(x)z \\
& =
\frac{(n + m)!}{n!m!} \gamma_{n + m}(x) +
\left(\frac{(n + m - 1)!}{n!(m - 1)!} +
\frac{(n + m - 1)!}{(n - 1)!m!}\right)
\gamma_{n + m - 1}(x) z \\
& =
\frac{(n + m)!}{n!m!}\delta_{n + m}(x + z)
\end{align*}
Condition (5) follows from
\begin{align*}
\delta_n(\delta_m(x + z))
& =
\delta_n(\gamma_m(x) + \gamma_{m - 1}(x)z) \\
& =
\gamma_n(\gamma_m(x)) + \gamma_{n - 1}(\gamma_m(x))\gamma_{m - 1}(x)z \\
& =
\frac{(nm)!}{n! (m!)^n} \gamma_{nm}(x) +
\frac{((n - 1)m)!}{(n - 1)! (m!)^{n - 1}}
\gamma_{(n - 1)m}(x) \gamma_{m - 1}(x) z \\
& = \frac{(nm)!}{n! (m!)^n}(\gamma_{nm}(x) + \gamma_{nm - 1}(x) z)
\end{align*}
by elementary number theory. To prove (4) we have to see that
$$
\delta_n(x + x' + z + z')
=
\gamma_n(x + x') + \gamma_{n - 1}(x + x')(z + z')
$$
is equal to
$$
\sum\nolimits_{i = 0}^n
(\gamma_i(x) + \gamma_{i - 1}(x)z)
(\gamma_{n - i}(x') + \gamma_{n - i - 1}(x')z')
$$
This follows easily on collecting the coefficients of
$1$, $z$, and $z'$ and using condition (4) for $\gamma$.
\end{proof}

\begin{lemma}
\label{lemma-divided-power-second-order-thickening}
Let $(A, I, \gamma)$ be a divided power ring. Let $M$, $N$ be $A$-modules.
Let $q : M \times M \to N$ be an $A$-bilinear map.
Let $B = A \oplus M \oplus N$ as an $A$-algebra with multiplication
$$
(x, z, w)\cdot (x', z', w') = (xx', xz' + x'z, xw' + x'w + q(z, z') + q(z', z))
$$
and set $J = I \oplus M \oplus N$. Set
$$
\delta_n(x, z, w) = (\gamma_n(x), \gamma_{n - 1}(x)z,
\gamma_{n - 1}(z)w + \gamma_{n - 2}(x)q(z, z))
$$
for $(a, m, n) \in J$.
Then $\delta$ is a divided power structure and
$A \to B$ is a homomorphism of divided power rings from
$(A, I, \gamma)$ to $(B, J, \delta)$.
\end{lemma}

\begin{proof}
Suppose we want to prove that property (4) of
Definition \ref{definition-divided-powers}
is satisfied. Pick $(x, z, w)$ and $(x', z', w')$ in $J$.
Pick a map
$$
A_0 = \mathbf{Z}\langle s, s'\rangle \longrightarrow A,\quad
s \longmapsto x,
s' \longmapsto x'
$$
which is possible by the universal property of divided power
polynomial rings. Set $M_0 = A_0 \oplus A_0$ and
$N_0 = A_0 \oplus A_0 \oplus M_0 \otimes_{A_0} M_0$.
Let $q_0 : M_0 \times M_0 \to N_0$ be the obvious map.
Define $M_0 \to M$ as the $A_0$-linear map which sends
the basis vectors of $M_0$ to $z$ and $z'$. Define $N_0 \to N$
as the $A_0$ linear map which sends the first two basis vectors
of $N_0$ to $w$ and $w'$ and uses
$M_0 \otimes_{A_0} M_0 \to M \otimes_A M \xrightarrow{q} N$
on the last summand. Then we see that it suffices to prove the
identitity (4) for the situation $(A_0, M_0, N_0, q_0)$.
Similarly for the other identities. This reduces us to the case
of a $\mathbf{Z}$-torsion free ring and $\mathbf{A}$-torsion free modules.
In this case all we have to do is show that
$$
n! \delta_n(x, z, w) = (x, z, w)^n
$$
in the ring $A$. To see this note that
$$
(x, z, w)^2 = (x^2, 2xz, 2xw + 2q(z, z))
$$
and by induction
$$
(x, z, w)^n = (x^n, nx^{n - 1}z, nx^{n - 1}w + n(n - 1)x^{n - 2}q(z, z))
$$
On the other hand,
$$
n! \delta_n(x, z, w) = (n!\gamma_n(x), n!\gamma_{n - 1}(x)z,
n!\gamma_{n - 1}(x)w + n!\gamma_{n - 2}(x) q(z, z))
$$
which matches. This finishes the proof.
\end{proof}




\section{Compatibility}
\label{section-compatibility}

\noindent
This section isn't required reading; it explains how our discussion
fits with that of \cite{Berthelot}.
Consider the following technical notion.

\begin{definition}
\label{definition-compatible}
Let $(A, I, \gamma)$ and $(B, J, \delta)$ be divided power rings.
Let $A \to B$ be a ring map. We say
{\it $\delta$ is compatible with $\gamma$}
if there exists a divided power structure $\bar\gamma$ on
$J + IB$ such that
$$
(A, I, \gamma) \to (B, J + IB, \bar \gamma)\quad\text{and}\quad
(B, J, \delta) \to (B, J + IB, \bar \gamma)
$$
are homomorphisms of divided power rings.
\end{definition}

\noindent
Let $p$ be a prime number. Let $(A, I, \gamma)$ be a divided power ring.
Let $A \to C$ be a ring map with $p$ nilpotent in $C$.
Assume that $\gamma$ extends to $IC$ (see
Lemma \ref{lemma-gamma-extends}).
In this situation, the (big affine) crystalline site of
$\Spec(C)$ over $\Spec(A)$
as defined in \cite{Berthelot} 
is the opposite of the category of systems
$$
(B, J, \delta, A \to B, C \to B/J)
$$
where
\begin{enumerate}
\item $(B, J, \delta)$ is a divided power ring with $p$ nilpotent in $B$,
\item $\delta$ is compatible with $\gamma$, and
\item the diagram
$$
\xymatrix{
B \ar[r] & B/J \\
A \ar[u] \ar[r] & C \ar[u]
}
$$
is commutative.
\end{enumerate}
The conditions
``$\gamma$ extends to $C$ and $\delta$ compatible with $\gamma$''
are used in \cite{Berthelot} to insure that
the crystalline cohomology of $\Spec(C)$ is the same as the crystalline
cohomology of $\Spec(C/IC)$. We will avoid this issue
by working exclusively with $C$ such that $IC = 0$\footnote{Of course there
will be a price to pay.}. In this case,
for a system $(B, J, \delta, A \to B, C \to B/J)$ as above,
the commutativity of the displayed diagram above implies $IB \subset J$ and
compatibility is equivalent to the condition that
$(A, I, \gamma) \to (B, J, \delta)$ is a homomorphism of divided
power rings.




\section{Affine crystalline site}
\label{section-affine-site}

\noindent
In this section we put some algebra related to the crystalline site.
Note that usually the prime number $p$ will be contained in the
divided power ideal $I$.

\begin{definition}
\label{definition-affine-thickening}
Let $p$ be a prime number. Let $(A, I, \gamma)$ be a divided power
ring such that $A$ is a $\mathbf{Z}_{(p)}$-algebra. Let $A \to C$ be a
ring map such that $IC = 0$ and $p$ is nilpotent in $C$.
A {\it divided power thickening} of $C$ over $(A, I, \gamma)$
is a homomorphism of divided power algebras
$(A, I, \gamma) \to (B, J, \delta)$ such that $p$ is nilpotent in $B$
and a ring map $C \to B/J$ such that
$$
\xymatrix{
B \ar[r] & B/J \\
& C \ar[u] \\
A \ar[uu] \ar[r] & A/I \ar[u]
}
$$
is commutative. A {\it homomorphism of divided power thickenings}
is defined in the obvious way. We denote $\text{CRIS}(C/A, I, \gamma)$
or simply $\text{CRIS}(C/A)$ the category of divided power thickenings
of $C$ over $(A, I, \gamma)$.
\end{definition}

\noindent
Note that for a divided power thickening $(B, J, \delta)$ as above
the ideal $J$ is locally nilpotent, see Lemma \ref{lemma-nil}.
This category does not have equalizers or fibre products in general.
It also doesn't have an initial object ($=$ empty colimit) in general.

\begin{lemma}
\label{lemma-affine-thickenings-colimits}
Assumptions as in Definition \ref{definition-affine-thickening}.
The category $\text{CRIS}(A/C)$ has products.
It also has all finite nonempty colimits.
\end{lemma}

\begin{proof}
The empty product is $(C, 0, \emptyset)$.
If $(B_t, J_t, \delta_t)$ is a family of divided power thickenings
then we can form the product $(\prod B_t, \prod J_t, \prod \delta_t)$
as in Lemma \ref{lemma-colimits}. The map
$C \to \prod B_t/\prod J_t = \prod B_t/J_t$ is clear.

\medskip\noindent
Given two objects $(B, J, \gamma)$ and $(B', J', \gamma')$ we can
form a cocartesian diagram
$$
\xymatrix{
(B, J, \gamma) \ar[r] & (B'', J'', \delta'') \\
(A, I, \gamma) \ar[r] \ar[u] & (B', J', \delta') \ar[u]
}
$$
in the category of divided power rings. Then we see that we have
$$
B''/J'' = B/J \otimes_{A/I} B'/J' \longleftarrow C \otimes_{A/I} C
$$ (details omitted). Denote $J'' \subset K \subset B''$ the ideal
such that
$$
\xymatrix{
B''/J'' \ar[r] & B''/K \\
C \otimes_{A/I} C \ar[r] \ar[u] & C \ar[u]
}
$$
is a pushout. Let $D = D_{B''}(K)$ be the divided power envelope of
$K$ in $B''$ relative to $(B'', J'', \delta'')$. Then it is
easily verified that $(D, \bar K, \bar \delta)$ is a coproduct
of $(B, J, \delta)$ and $(B', J', \delta')$ in the category
of affine thickenings.

\medskip\noindent
Next, we come to coequalizers. Let
$\alpha, \beta : (B, J, \delta) \to (B', J', \delta')$ be
morphisms of divided power thickenings over $(A, I, \gamma)$.
Consider $B'' = B'/ (\alpha(b) - \beta(b))$. Let $J'' \subset B''$
be the image of $J'$. Let $D = D_{B''}(J'')$ be the divided power envelope of
$J''$ in $B''$ relative to $(B', J', \delta')$. Then it is
easily verified that $(D, \bar K, \bar \delta)$ is the coequalizer
of $(B, J, \delta)$ and $(B', J', \delta')$ in the category
of affine thickenings.

\medskip\noindent
By Categories, Lemma \ref{categories-lemma-almost-finite-colimits-exist}
we have all finite nonempty colimits.
\end{proof}

\begin{lemma}
\label{lemma-set-generators}
Let $p$, $(A, I, \gamma)$, and $A \to C$ be as in
Definition \ref{definition-affine-thickening}.
Let $P$ be a polynomial algebra over $A$ and let
$P \to C$ be a surjection of $A$-algebras with kernel $K$.
For every $n \geq 1$ set $P_n = P/p^nP$ and $K_n \subset P_n$
the image of $K$ and
$$
D_n = D_{P_n}(K_n)
$$
where the divided power envelope is taken relative to $(A, I, \gamma)$.
Then $D_n$ is an object of $\text{CRIS}(C/A)$ and for every object
$(B, J, \delta)$ of $\text{CRIS}(C/A)$ there exists an $n$ and
a morphsm $D_n \to B$ of $\text{CRIS}(C/A)$.
\end{lemma}

\begin{proof}
We just prove the final assertion.
We can find an $A$-algebra homomorphism $P \to B$
lifting the map $C \to B/J$. By definition $p^nB = 0$ for
some $n$ hence $P \to B$ factors as $P \to P_n \to B$.
By the universal property of the divided power envelope we
conclude that $P_n \to B$ factors through $D_n$.
\end{proof}







\section{Module of differentials}
\label{section-differentials}

\noindent
In this section we develop a theory of modules of differentials
for divided power rings.

\begin{definition}
\label{definition-derivation}
Let $(A, I, \gamma) \to (B, J, \delta)$ be a homomorphism
of divided power rings. Let $M$ be an $B$-module.
An {\it divided power $A$-derivation} into $M$ is a map
$\theta : B \to M$ which is additive, annihilates the elements
of $A$, satisfies the Leibniz rule
$\theta(bb') = b\theta(b') + b'\theta(b)$ and satisfies
$$
\theta(\gamma_n(x)) = \gamma_{n - 1}(x)\theta(x)
$$
for all $n \geq 1$ and all $x \in J$.
\end{definition}

\noindent
As in the case of usual derivations, there exists a
{\it universal divided power $A$-derivation} $d_{A/B} : B \to \Omega_{B/A}$
such that any derivation $\theta : B \to M$ is equal to
$\theta = \xi \circ d_{B/A}$ for some $B$-linear map $\Omega_{B/A} \to M$.

\medskip\noindent
Let $(A, I, \gamma)$ be a divided power ring. In this setting the
correct version of the powers of $I$ is given by the divided powers
$$
I^{[n]} = \text{ideal generate by }
\gamma_{e_1}(x_1) \ldots \gamma_{e_t}(x_t)
\text{ with }\sum e_j \geq n\text{ and }x_j \in I.
$$
Of course we have $I^n \subset I^{[n]}$. Note that $I^{[1]} = I$.
Sometimes we also set $I^{[0]} = A$.

\begin{lemma}
\label{lemma-diagonal-and-differentials}
Let $(A, I, \gamma) \to (B, J, \delta)$ be a homomorphism
of divided power rings. Let $(B(1), J(1), \delta(1))$ be the coproduct
of $(B, J, \delta)$ with itself over $(A, I, \gamma)$, i.e.,
such that
$$
\xymatrix{
(B, J, \delta) \ar[r] & (B(1), J(1), \delta(1)) \\
(A, I, \gamma) \ar[r] \ar[u] & (B, J, \delta) \ar[u]
}
$$
is cocartesian. Denote $K = \text{Ker}(B(1) \to B)$.
Then $K \cap J(1) \subset J(1)$ is preserved by the divided power
structure and
$$
\Omega_{B/A} = K/ \left(K^2 + (K \cap J(1))^{[2]}\right)
$$
canonically.
\end{lemma}

\begin{proof}
The fact that $K \cap J(1) \subset J(1)$ is preserved by the divided power
structure follows from the fact that $B(1) \to B$ is a homomorphism of
divided power rings.

\medskip\noindent
Recall that $K/K^2$ has a canonical $B$-module structure.
Denote $s_0, s_1 : B \to B(1)$ the two coprojections and consider
the map $\text{d} : B \to K/K^2 +(K \cap J(1))^{[2]}$ given by
$b \mapsto s_0(b) - s_1(b)$. It is clear that $\text{d}$ is additive,
annihilates $A$, and satisfies the Leibniz rule.
We claim that $\text{d}$ is an $A$-derivation.
Let $x \in J$. Set $y = s_0(x)$ and $z = s_1(x)$.
Denote $\delta$ the divided power structure on $J(1)$.
We have to show that $\delta_n(y) - \delta_n(z) = \delta_{n - 1}(y)(y - z)$
modulo $K^2 +(K \cap J(1))^{[2]}$ for $n \geq 1$. We will show this
by induction on $n$. It is true for $n = 1$.
Let $n > 1$ and that it holds for all smaller values.
Note that
$$
\delta_n(z - y) =
\sum\nolimits_{i = 0}^n (-1)^{n - i}\delta_i(z)\delta_{n - i}(y)
$$
is an element of $K^2 +(K \cap J(1))^{[2]}$. From this and induction
we see that working modulo $K^2 +(K \cap J(1))^{[2]}$ we have
\begin{align*}
& \delta_n(y) - \delta_n(z) \\
& =
\delta_n(y) +
\sum\nolimits_{i = 0}^{n - 1} (-1)^{n - i}\delta_i(z)\delta_{n - i}(y) \\
& =
\delta_n(y) + (-1)^n\delta_n(y) +
\sum\nolimits_{i = 1}^{n - 1}
(-1)^{n - i}(\delta_i(y) - \delta_{i - 1}(y)(y - z))\delta_{n - i}(y)
\end{align*}
Using that $\delta_i(y)\delta_{n - i}(y) = \binom{n}{i} \delta_n(y)$
and that $\delta_{i - 1}(y)\delta_{n - i}(y) =
\binom{n - 1}{i} \delta_{n - 1}(y)$
the reader easily verifies that this expression comes out to give
$\delta_{n - 1}(y)(y - z)$ as desired.

\medskip\noindent
Let $M$ be a $B$-module. Let $\theta : B \to M$ be a divided power
$A$-derivation.
Set $D = B \oplus M$ where $M$ is an ideal of square zero. Define a
divided power structure on $J \oplus M \subset D$ by setting
$\delta_n(x + m) = \delta_n(x) + \delta_{n - 1}(x)m$ for $n > 1$, see
Lemma \ref{lemma-divided-power-first-order-thickening}.
There are two divided power algebra homomorphisms $B \to D$: the first
is given by the inclusion and the second by the map $b \mapsto b + \theta(b)$.
Hence we get a canonical homomorphism $B(1) \to D$ of divided power
algebras over $(A, I, \gamma)$. This induces a map $K \to M$
which annihilates $K^2$ (as $M$ is an ideal of square zero) and
$(K \cap J(1))^{[2]}$ as $M^{[2]} = 0$. It follows that $\text{d}$
is a universal divided power $A$-derivation and we win.
\end{proof}

\begin{lemma}
\label{lemma-diagonal-and-differentials-affine-site}
Let $p$, $(A, I, \gamma)$, and $A \to C$ as in
Definition \ref{definition-affine-thickening}.
Let $(B, J, \delta)$ be a divided power thickening of $C$ over
$(A, I, \gamma)$. Let $(B(1), J(1), \delta(1))$ be the coproduct
of $(B, J, \delta)$ with itself in the category of divided power
thickenings of $C$ over $(A, I, \gamma)$. Denote
$K = \text{Ker}(B(1) \to B)$. Then $K \cap J(1) \subset J(1)$
is preserved by the divided power structure and
$$
\Omega_{B/A} = K/ \left(K^2 + (K \cap J(1))^{[2]}\right)
$$
canonically.
\end{lemma}

\begin{proof}
Word for word the same as the proof of
Lemma \ref{lemma-diagonal-and-differentials}.
The only point that has to be checked is that the
divided power ring $D = B \oplus M$ is a
divided power thickening of $C$ over $(A, I, \gamma)$
and that the two maps $B \to C$ are homomorphisms of
divided power thickenings of $C$ over $(A, I, \gamma)$.
Since $D/(J \oplus M) = B/J$ we can use $C \to B/J$
and everything is clear from the construction.
\end{proof}

\begin{remark}
\label{remark-absolute-de-rham-complex}
Let $B$ be a ring. Write $\Omega_B = \Omega_{B/\mathbf{Z}}$
for the absolute\footnote{This
actually makes sense: if $\Omega_B$ is the module of differentials
where we only assume the Leibniz rule and not the vanishing of $\text{d}1$,
then the Leibniz rule gives $\text{d}1 = \text{d}(1 \cdot 1) =
1 \text{d}1 + 1 \text{d}1 = 2 \text{d}1$ and hence
$\text{d}1 = 0$ in $\Omega_B$.} module of differentials of $B$.
Let $\text{d} : B \to \Omega_B$ denote the universal derivation. Set
$\Omega_B^i = \wedge^i_B(\Omega_B)$ as in
Algebra, Section \ref{algebra-section-tensor-algebra}.
The absolute {\it de Rham complex}
$$
\Omega_B^0 \to \Omega_B^1 \to \Omega_B^2 \to \ldots
$$
Here $\text{d} : \Omega_B^p \to \Omega_B^{p + 1}$
is defined by the rule
$$
\text{d}\left(b_0\text{d}b_1 \wedge \ldots \wedge \text{d}b_p\right) =
\text{d}b_0 \wedge \text{d}b_1 \wedge \ldots \wedge \text{d}b_p
$$
which we will show is well defined; note that
$\text{d} \circ \text{d} = 0$ so we get a complex.
Recall that $\Omega_B$ is the $B$-module generated by
elements $\text{d}b$ subject to the relations
$\text{d}(a + b) = \text{d}a + \text{d}b$ and
$\text{d}(ab) = b\text{d}a + a\text{d}b$
for $a, b \in B$. To prove that our map is well defined for $p = 1$
we have to show that the elements
$$
a\text{d}(b + c) - a\text{d}b - a\text{d}c
\quad\text{and}\quad
a\text{d}(bc) - ac\text{d}b - ab\text{d}c,\quad a,b,c \in B
$$
are mapped to zero by our rule. This is clear by direct computation
(using the Leibniz rule). Thus we get a map
$$
\Omega_B \otimes_\mathbf{Z} \ldots \otimes_\mathbf{Z} \Omega_B
\longrightarrow
\Omega_B^{p + 1}
$$
defined by the formula
$$
\omega_1 \otimes \ldots \otimes \omega_p
\longmapsto
\sum (-1)^{i + 1}
\omega_1 \wedge \ldots \wedge \text{d}(\omega_i) \wedge \ldots \wedge \omega_p
$$
which matches our rule above on elements of the form
$b_0\text{d}b_1 \otimes \text{d}b_2 \otimes \ldots \otimes \text{d}b_p$.
It is clear that this map is alternating. To finish we have to show
that
$$
\omega_1 \otimes \ldots \otimes f\omega_i \otimes \ldots \otimes \omega_p
\quad\text{and}\quad
\omega_1 \otimes \ldots \otimes f\omega_j \otimes \ldots \otimes \omega_p
$$
are mapped to the same element. By $\mathbf{Z}$-linearity and
the alternating property, it is enough to show this for $p = 2$, $i = 1$,
$j = 2$, $\omega_1 = a_1 \text{d}b_1$ and $\omega_2 = a_2 \text{d}b_2$.
Thus we need to show that
\begin{align*}
& \text{d}fa_1 \wedge \text{d}b_1 \wedge a_2\text{d}b_2
- fa_1 \text{d}b_1 \wedge \text{d}a_2 \wedge \text{d}b_2 \\
& =
\text{d}a_1 \wedge \text{d}b_1 \wedge fa_2\text{d}b_2
- a_1 \text{d}b_1 \wedge \text{d}fa_2 \wedge \text{d}b_2
\end{align*}
in other words that
$$
(a_2 \text{d}fa_1 + fa_1 \text{d}a_2 - fa_2 \text{d}a_1 - a_1 \text{d}fa_2)
\wedge \text{d}b_1 \wedge \text{d}b_2 = 0.
$$
This follows from the Leibniz rule.
\end{remark}

\begin{lemma}
\label{lemma-de-rham-complex}
Let $B$ be a ring. Let $\pi : \Omega_B \to \Omega$ be a surjective $B$-module
map. Denote $\text{d} : B \to \Omega$ the composition of $\pi$ with
$\text{d}_B : B \to \Omega_B$. Set $\Omega^i = \wedge_B^i(\Omega)$.
Assume that the kernel of $\pi$ is generated, as a $B$-module,
by elements $\omega \in \Omega_B$ such that
$\text{d}_B(\omega) \in \Omega_B^2$ maps to zero in $\Omega^2$.
Then there is a de Rham complex
$$
\Omega^0 \to \Omega^1 \to \Omega^2 \to \ldots
$$
whose differential is defined by the rule
$$
\text{d} : \Omega^p \to \Omega^{p + 1},\quad
\text{d}\left(f_0\text{d}f_1 \wedge \ldots \wedge \text{d}f_p\right) =
\text{d}f_0 \wedge \text{d}f_1 \wedge \ldots \wedge \text{d}f_p
$$
\end{lemma}

\begin{proof}
We will show that there exists a commutative diagram
$$
\xymatrix{
\Omega_B^0 \ar[d] \ar[r]_{\text{d}_B} &
\Omega_B^1 \ar[d]_\pi \ar[r]_{\text{d}_B} &
\Omega_B^2 \ar[d]_{\wedge^2\pi} \ar[r]_{\text{d}_B} &
\ldots \\
\Omega^0 \ar[r]^{\text{d}} &
\Omega^1 \ar[r]^{\text{d}} &
\Omega^2 \ar[r]^{\text{d}} &
\ldots
}
$$
the description of the map $\text{d}$ will follow from the construction
of $\text{d}_B$ in Remark \ref{remark-absolute-de-rham-complex}.
Since the left most vertical arrow is an isomorphism we have
the first square. Because $\pi$ is surjective, to get the second
square it suffices to show that $\text{d}_B$ maps the kernel
of $\pi$ into the kernel of $\wedge^2\pi$. We are given that any element
of the kernel of $\pi$ is of the form
$\sum b_i\omega_i$ with $\pi(\omega_i) = 0$ and
$\wedge^2\pi(\text{d}_B(\omega_i)) = 0$.
By the Leibniz rule for $\text{d}_B$ we have
$\text{d}_B(\sum b_i\omega_i) = \sum b_i \text{d}_B(\omega_i) +
\sum \text{d}_B(b_i) \wedge \omega_i$. Hence this maps to zero
under $\wedge^2\pi$.

\medskip\noindent
For $i > 1$ we note that $\wedge^i \pi$ is surjective with
kernel the image of $\text{Ker}(\pi) \wedge \Omega^{i - 1}_B
\to \Omega_B^i$. For $\omega_1 \in \text{Ker}(\pi)$ and
$\omega_2 \in \Omega^{i - 1}_B$ we have
$$
\text{d}_B(\omega_1 \wedge \omega_2) =
\text{d}_B(\omega_1) \wedge \omega_2 - \omega_1 \wedge \text{d}_B(\omega_2)
$$
which is in the kernel of $\wedge^{i + 1}\pi$ by what we just proved above.
Hence we get the $(i + 1)$st square in the diagram above.
This concludes the proof.
\end{proof}

\begin{remark}
\label{remark-divided-powers-de-rham-complex}
Let $(A, I, \gamma) \to (B, J, \delta)$ be a homomorphism
of divided power rings. Set $\Omega_{B/A}^i = \wedge^i_B \Omega_{B/A}$
where $\Omega_{B/A}$ is the target of the universal divided power
derivation $\text{d} = \text{d}_{B/A} : B \to \Omega_{B/A}$.
Note that $\Omega_{B/A}$ is the quotient of $\Omega_B$ by the
$B$-submodule generated by the elements
$\text{d}a = 0$ for $a \in A$ and
$\text{d}\delta_n(x) - \delta_{n - 1}(x)\text{d}x$ for $x \in J$.
We claim Lemma \ref{lemma-de-rham-complex} applies.
To see this it suffices to verify the elements
$\text{d}a$ and $\text{d}\delta_n(x) - \delta_{n - 1}(x)\text{d}x$
of $\Omega_B$ are mapped to zero in $\Omega^2_{B/A}$.
This is clear for the first, and for the last we observe that
$$
\text{d}(\delta_{n - 1}(x)) \wedge \text{d}x
= \delta_{n - 2}(x) \wedge \text{d}x \wedge \text{d}x = 0
$$
in $\Omega^2_{B/A}$ as desired. Hence we obtain a
{\it divided power de Rham complex}
$$
\Omega^0_{B/A} \to \Omega^1_{B/A} \to \Omega^2_{B/A} \to \ldots
$$
which will play an important role in the sequel.
\end{remark}

\begin{remark}
\label{remark-connection}
Let $B$ be a ring. Let $\Omega_B \to \Omega$ be a quotient satisfying
the assumptions of Lemma \ref{lemma-de-rham-complex} so
that we have a de Rham complex. Let $M$ be a $B$-module.
A {\it connection} is an additive map
$$
\nabla : M \longrightarrow M \otimes_B \Omega
$$
such that $\nabla(bm) = b \nabla(m) + m \otimes \text{d}b$
for $b \in B$ and $m \in M$. In this situation we can define maps
$$
\nabla : M \otimes_B \Omega^i \longrightarrow M \otimes_B \Omega^{i + 1}
$$
by the rule $\nabla(m \otimes \omega) = \nabla(m) \wedge \omega +
m \otimes \text{d}\omega$. This works because if $b \in B$, then
\begin{align*}
\nabla(bm \otimes \omega) - \nabla(m \otimes b\omega)
& =
\nabla(bm) \otimes \omega + bm \otimes \text{d}\omega
- \nabla(m) \otimes b\omega - m \otimes \text{d}(b\omega) \\
& =
b\nabla(m) \otimes \omega + m \otimes \text{d}b \wedge \omega
+ bm \otimes \text{d}\omega \\
&\ \ \ \ \ \ - b\nabla(m) \otimes \omega - bm \otimes \text{d}(\omega)
- m \otimes \text{d}b \wedge \omega \\
& = 0
\end{align*}
As is customary we say the connection is {\it integrable} if and
only if the composition
$$
M \xrightarrow{\nabla} M \otimes_B \Omega^1
\xrightarrow{\nabla} M \otimes_B \Omega^2
$$
is zero. In this case we obtain a complex
$$
M \xrightarrow{\nabla} M \otimes_B \Omega^1
\xrightarrow{\nabla} M \otimes_B \Omega^2
\xrightarrow{\nabla} M \otimes_B \Omega^3
\xrightarrow{\nabla} M \otimes_B \Omega^4 \to \ldots
$$
which is called the de Rham complex of the connection.
\end{remark}




\section{Divided power schemes}
\label{section-divided-power-schemes}

\noindent
Some remarks on how to globalize the previous notions.

\begin{definition}
\label{definition-divided-power-structure}
Let $\mathcal{C}$ be a site. Let $\mathcal{O}$ be a sheaf of rings
on $\mathcal{C}$. Let $\mathcal{I} \subset \mathcal{O}$ be a
sheaf of ideals. A {\it divided power structure $\gamma$} on $\mathcal{I}$
is a sequence of maps $\gamma_n : \mathcal{I} \to \mathcal{I}$, $n \geq 1$
such that for any object $U$ of $\mathcal{C}$ the triple
$$
(\mathcal{O}(U), \mathcal{I}(U), \gamma)
$$
is a divided power ring.
\end{definition}

\noindent
To be sure this applies in particular to sheaves of rings on
topological spaces. But it's good to be a little bit more general
as the structure sheaf of the crystalline site lives on a... site!
A triple $(\mathcal{C}, \mathcal{I}, \gamma)$ as in the
definition above is sometimes called a {\it divided power topos}
in this chapter. Given a second $(\mathcal{C}', \mathcal{I}', \gamma')$ and
given a morphism of ringed topoi
$(f, f^\sharp) : (\Sh(\mathcal{C}), \mathcal{O}) \to
(\Sh(\mathcal{C}'), \mathcal{O}')$
we say that $(f, f^\sharp)$ induces a {\it morphism of divided
power topoi} if $f^\sharp(f^{-1}\mathcal{I}') \subset \mathcal{I}$
and the diagrams
$$
\xymatrix{
f^{-1}\mathcal{I}' \ar[d]_{f^{-1}\gamma'_n} \ar[r]_{f^\sharp} &
\mathcal{I} \ar[d]^{\gamma_n} \\
f^{-1}\mathcal{I}' \ar[r]^{f^\sharp} & \mathcal{I}
}
$$
are commutative for all $n \geq 1$. If $f$ comes from a morphism of
sites induced by a functor $u : \mathcal{C}' \to \mathcal{C}$ then
this just means that
$$
(\mathcal{O}'(U'), \mathcal{I}'(U'), \gamma')
\longrightarrow
(\mathcal{O}(u(U')), \mathcal{I}(u(U')), \gamma)
$$
is a homomorphism of divided power rings for all $U' \in \Ob(\mathcal{C}')$.

\medskip\noindent
In the case of schemes we require the divided power ideal to be
{\bf quasi-coherent}. But apart from this the definition is exactly
the same as in the case of topoi. Here it is.

\begin{definition}
\label{definition-divided-power-scheme}
A {\it divided power scheme} is a triple $(S, \mathcal{I}, \gamma)$
where $S$ is a scheme, $\mathcal{I}$ is a quasi-coherent sheaf of
ideals, and $\gamma$ is a divided power structure on $\mathcal{I}$.
A {\it morphism of divided power schemes}
$(S, \mathcal{I}, \gamma) \to (S', \mathcal{I}', \gamma')$ is
a morphism of schemes $f : S \to S'$ such that
$f^{-1}\mathcal{I}'\mathcal{O}_S \subset \mathcal{I}$ and such that
$$
(\mathcal{O}_S(U'), \mathcal{I}(U'), \gamma)
\longrightarrow
(\mathcal{O}_{S'}(f^{-1}U'), \mathcal{I}(f^{-1}U'), \gamma)
$$
is a homomorphism of divided power rings for all $U' \subset S'$ open.
\end{definition}

\noindent
Recall that there is a 1-to-1 correspondence between quasi-coherent
sheaves of ideals and closed immersions, see
Morphisms, Section \ref{morphisms-section-closed-immersions}.
Thus given a divided power scheme $(T, \mathcal{J}, \gamma)$ we
get a canonical closed immersion $U \to T$ defined by $\mathcal{J}$.
Conversely, given a closed immersion $U \to T$ and a divided power
structure $\gamma$ on the sheaf of ideals $\mathcal{J}$ associated
to $U \to T$ we obtain a divided power scheme $(T, \mathcal{J}, \gamma)$.
In many situations we only want to consider such triples
$(U, T, \gamma)$ when the morphism $U \to T$ is a thickening, see
More on Morphisms, Definition \ref{more-morphisms-definition-thickening}.

\begin{definition}
\label{definition-divided-power-thickening}
A triple $(U, T, \gamma)$ as above is called a {\it divided power thickening}
if $U \to T$ is a thickening.
\end{definition}

\noindent
We make the following observation. Suppose that $(U, T, \gamma)$
is triple as above. Assume that $T$ is a scheme over $\mathbf{Z}_{(p)}$
and that $p$ is locally nilpotent on $U$. Then we have
$$
p\text{ locally nilpotent on }T
\Leftrightarrow
U \to T\text{ is a thickening}
$$
See Lemma \ref{lemma-nil}.





\section{Crystalline site}
\label{section-site}

\noindent
We first define the big site. Given a divided power scheme
$(S, \mathcal{I}, \gamma)$ we say $(T, \mathcal{J}, \delta)$ is
a divided power scheme over $(S, \mathcal{I}, \gamma)$ if
$T$ comes endowed with a morphism $T \to S$ of divided power
schemes. Similarly, we say a divided power thickening $(U, T, \delta)$
is a divided power thickening over $(S, \mathcal{I}, \gamma)$
if $T$ comes endowed with a morphism $T \to S$ of divided power
schemes.

\begin{definition}
\label{definition-divided-power-thickening-X}
Let $p$ be a prime number. Let $(S, \mathcal{I}, \gamma)$ be a divided power
scheme over $\mathbf{Z}_{(p)}$. Let $f : X \to S$ be a morphism of schemes
such that $f^{-1}\mathcal{I} \mathcal{O}_X = 0$ and such that $p$ is
locally nilpotent on $X$.
\begin{enumerate}
\item A {\it divided power thickening of $X$ relative to
$(S, \mathcal{I}, \gamma)$} is given by a divided power thickening
$(U, T, \delta)$ over $(S, \mathcal{I}, \gamma)$
and an $S$-morphism $U \to X$.
\item A {\it morphism of divided power thickenings of $X$
relative to $(S, \mathcal{I}, \gamma)$} is defined in the obvious
manner.
\end{enumerate}
The category of divided power thickenings of $X$ relative to
$(S, \mathcal{I}, \gamma)$ is denoted $\text{CRIS}(X/S, \mathcal{I}, \gamma)$
or simply $\text{CRIS}(X/S)$.
\end{definition}

\noindent
For any $(U, T, \delta)$ in $\text{CRIS}(X/S)$
we have that $p$ is locally nilpotent on $T$, see discussion after
Definition \ref{definition-divided-power-thickening}.
There is a canonical forgetful functor
\begin{equation}
\label{equation-forget}
\text{CRIS}(X/S) \longrightarrow \Sch/X,\quad
(U, T, \delta) \longmapsto U
\end{equation}
as well as its left inverse
\begin{equation}
\label{equation-endow-trivial}
\Sch/X \longrightarrow \text{CRIS}(X/S),\quad
U \longmapsto (U, U, \emptyset)
\end{equation}
which is sometimes useful.

\begin{lemma}
\label{lemma-divided-power-thickening-fibre-products}
With notation and assumptions as in
Definition \ref{definition-divided-power-thickening-X}.
The category $\text{CRIS}(X/S)$ has all finite nonempty limits,
in particular products of pairs and fibre products.
The functor (\ref{equation-forget}) commutes with limits.
\end{lemma}

\begin{proof}
Omitted. Hint: See Lemma \ref{lemma-affine-thickenings-colimits}
for the affine case.
\end{proof}

\begin{lemma}
\label{lemma-divided-power-thickening-base-change-flat}
With notation and assumptions as in
Definition \ref{definition-divided-power-thickening-X}.
Let
$$
\xymatrix{
(U_3, T_3, \delta_3) \ar[d] \ar[r] & (U_2, T_2, \delta_2) \ar[d] \\
(U_1, T_1, \delta_1) \ar[r] & (U, T, \delta)
}
$$
be a fibre square in the category of divided power thickenings of
$X$ relative to $(S, \mathcal{I}, \gamma)$. If $T_2 \to T$ is
flat, then $T_3 = T_1 \times_T T_2$ (as schemes).
\end{lemma}

\begin{proof}
This is true because a divided power structure extends uniquely
along a flat ring map. See Lemma \ref{lemma-gamma-extends}.
\end{proof}

\noindent
The lemma above means that the base change of a flat morphism
of divided power thickenings is another flat morphism, and in
fact is the ``usual'' base change of the morphism. This implies
that the following definition makes sense.

\begin{definition}
\label{definition-big-crystalline-site}
Assumptions and notation as in
Definition \ref{definition-divided-power-thickening-X}.
\begin{enumerate}
\item A family of morphisms $\{(U_i, T_i, \delta_i) \to (U, T, \delta)\}$
of divided power thickenings of $X/S$ is a {\it Zariski, \'etale, smooth,
syntomic, or fppf covering} if and only if the family of morphisms
of schemes $\{T_i \to T \}$ is one.
\item The {\it big crystalline site} of $X$ over $(S, \mathcal{I}, \gamma)$,
is the category $\text{CRIS}(X/S)$ endowed with the Zariski topology.
\item The topos of sheaves on $\text{CRIS}(X/S)$ is denoted
$(X/S)_{\text{CRIS}}$ or sometimes
$(X/S, \mathcal{I}, \gamma)_{\text{CRIS}}$.
\end{enumerate}
\end{definition}

\noindent
Since (\ref{equation-forget}) commutes with products and fibre
products, we see that looking at those $(U, T, \delta)$ such that
$U \to X$ is an open immersion defines a full
subcategory preserved under fibre products (and more generally
finite nonempty limits). Hence the following
definition makes sense.

\begin{definition}
\label{definition-crystalline-site}
Assumptions and notation as in
Definition \ref{definition-divided-power-thickening-X}.
\begin{enumerate}
\item The (small) {\it crystalline site} of $X$ over
$(S, \mathcal{I}, \gamma)$, denoted $\text{Cris}(X/S, \mathcal{I}, \gamma)$
or simply $\text{Cris}(X/S)$ is the full subcategory of $\text{CRIS}(X/S)$
consisting of those divided power thickenings such that $U \to X$ is an open
immersion. It comes endowed with the Zariski topology.
\item The topos of sheaves on $\text{Cris}(X/S)$ is denoted
$(X/S)_{\text{cris}}$ or sometimes
$(X/S, \mathcal{I}, \gamma)_{\text{cris}}$.
\end{enumerate}
\end{definition}

\noindent
We can compare the small and big crystalline sites, just like
we can compare the small and big \'etale sites of a scheme.

\begin{lemma}
\label{lemma-compare-big-small}
The inclusion functor $\text{Cris}(X/S) \to \text{CRIS}(X/S)$ is
commutes with finite nonempty limits, is fully faithful, continuous,
and cocontinuous. There are morphisms of topoi
$$
(X/S)_{\text{cris}} \xrightarrow{i} (X/S)_{\text{CRIS}}
\xrightarrow{\pi} (X/S)_{\text{cris}}
$$
whose composition is the identity and of which the first is induced
by the inclusion functor. Moreover, $\pi_* = i^{-1}$.
\end{lemma}

\begin{proof}
For the first assertion see
Lemma \ref{lemma-divided-power-thickening-fibre-products}.
This gives us a morphism of topoi
$i : (X/S)_{\text{cris}} \to (X/S)_{\text{CRIS}}$ and a left adjoint
$i_!$ such that $i^{-1}i_! = i^{-1}i_* = \text{id}$, see
Sites, Lemmas \ref{sites-lemma-when-shriek},
\ref{sites-lemma-preserve-equalizers}, and
\ref{sites-lemma-back-and-forth}.
We claim that $i_!$ is exact. If this is true, then we can define
$\pi$ by the rules $\pi^{-1} = i_!$ and $\pi_* = i^{-1}$
and everything is clear. To prove the claim, note that we already know
that $i_!$ is right exact and preserves fibre products (see references
given). Hence it suffices to show that $i_! * = *$ where $*$ indicates
the final object in the category of sheaves of sets. 
To see this it suffices to produce a set of objects
$(U_i, T_i, \delta_i)$, $i \in I$ of $\text{Cris}(X/S)$ such that
$$
\coprod\nolimits_{i \in I} h_{(U_i, T_i, \delta_i)} \to *
$$
is surjective in $(X/S)_{\text{CRIS}}$ (details omitted).
In the affine case this
follows from Lemma \ref{lemma-set-generators}. We omit the proof
in general.
\end{proof}

\begin{remark}[Functoriality]
\label{remark-functoriality-cris}
Let $p$ be a prime number.
Let $(S, I, \gamma)$ be a divided power scheme
over $\mathbf{Z}_{(p)}$.
Let $f : X \to Y$ be a morphism of schemes over $S$ such that
$\mathcal{I} \cdot \mathcal{O}_X = 0$,
$\mathcal{I} \cdot \mathcal{O}_Y = 0$, and $p$ is locally nilpotent
on $X$ and $Y$. Then we get a cocontinuous functor
$$
\text{CRIS}(X/S) \longrightarrow \text{CRIS}(Y/S)
$$
by letting $(U, T, \delta)$ correspond to $(U, T, \delta)$
with $U \to X \to Y$ as the $S$-morphism from $U$ to $Y$.
Hence we get a morphism of topoi
$$
f_{\text{CRIS}} : (X/S)_{\text{CRIS}} \longrightarrow (Y/S)_{\text{CRIS}}
$$
see Sites, Section \ref{sites-section-cocontinuous-morphism-topoi}.
(This is the miracle of cocontinuous functors.) By analogy with
Topologies, Lemma \ref{topologies-lemma-morphism-big-small} we define
$$
f_{\text{cris}} : (X/S)_{\text{cris}} \longrightarrow (Y/S)_{\text{cris}}
$$
by the formula $f_{\text{cris}} = \pi_Y \circ f_{\text{CRIS}} \circ i_X$
where $i_X$ and $\pi_Y$ are as in Lemma \ref{lemma-compare-big-small}
for $X$ and $Y$.
\end{remark}

\begin{remark}[Structure morphism]
\label{remark-structure-morphism}
Let $p$ be a prime number. Let $(S, I, \gamma)$ be a divided power scheme
over $\mathbf{Z}_{(p)}$. Consider the closed subscheme
$S_0 = V(\mathcal{I}) \subset S$ associated to the quasi-coherent sheaf
of ideals $\mathcal{I}$. If we assume that $p$ is locally nilpotent on
$S_0$ (which is almost always going to be the case in practice) then
we obtain a situation as in
Definition \ref{definition-divided-power-thickening-X} with $S_0$ in stead
of $X$. Hence we get a site $\text{CRIS}(S_0/S)$. Moreover, for any
$c : X \to S_0$ we get a morphism of topoi
$$
c_{\text{CRIS}} : (X/S)_{\text{CRIS}} \longrightarrow (S_0/S)_{\text{CRIS}}
$$
which one could think of as the structure morphism. We also have a variant
$$
c_{\text{cris}} : (X/S)_{\text{cris}} \longrightarrow (S_0/S)_{\text{cris}}
$$
for the crystalline sites. If $p^N = 0$ on $S$ for some $N$, then
the sites $\text{CRIS}(S_0/S)$ and $\text{Cris}(S_0/S)$ have a final
object, namely $(S_0, S, \gamma)$. If $S_0$ is quasi-compact, then
the sequence of objects $(S_0, S_n, \gamma)$, $n \gg 0$ are cofinal in
$\text{CRIS}(S_0/S)$ and $\text{Cris}(S_0/S)$ where $S_n = V(p^n) \subset S$.
\end{remark}




\section{Sheaves on the crystalline site}
\label{section-sheaves}

\noindent
Let $p$, $(S, \mathcal{I}, \gamma)$, and $X \to S$ be as in
Definition \ref{definition-divided-power-thickening-X}.
A sheaf $\mathcal{F}$ on $\text{CRIS}(X/S)$ gives rise to
a {\it restriction} $\mathcal{F}_T$ for every object $(U, T, \delta)$
of $\text{CRIS}(X/S)$. Namely, $\mathcal{F}_T$ is the Zariski sheaf on
the scheme $T$ defined by the rule
$$
\mathcal{F}_T(W) = \mathcal{F}(U \cap W, W, \delta|_W)
$$
for $W \subset T$ is open.
Moreover, if $f : T \to T'$ is a morphism between objects
$(U, T, \delta)$ and $(U', T', \delta')$ of $\text{CRIS}(X/S)$, then there
is a canonical {\it comparison} map
\begin{equation}
\label{equation-comparison}
c_f : f^{-1}\mathcal{F}_{T'} \longrightarrow \mathcal{F}_T.
\end{equation}
Conversely, given Zariski sheaves $\mathcal{F}_T$ and comparion maps
satisfying a suitable cocycle condition, we obtain a sheaf on
$\text{CRIS}(X/S)$, see
Topologies, Lemma \ref{topologies-lemma-characterize-sheaf-big}.
In entirely the same way we can describe sheaves on the
small crystaline site.

\medskip\noindent
The {\it structure sheaf} on $\text{CRIS}(X/S)$ is the sheaf
$\mathcal{O}_{X/S}$ defined by the rule
$$
\mathcal{O}_{X/S} :
(U, T, \delta)
\longmapsto
\Gamma(T, \mathcal{O}_T)
$$
Another type of example comes by starting with a sheaf
$\mathcal{G}$ on $(\Sch/X)_{Zar}$. Then $\underline{\mathcal{G}}$
defined by the rule
$$
\underline{\mathcal{G}} :
(U, T, \delta)
\longmapsto
\mathcal{G}(U)
$$
is a sheaf on $\text{CRIS}(X/S)$. In particular, if we take
$\mathcal{G} = \mathbf{G}_a = \mathcal{O}_X$, then we obtain
$$
\underline{\mathbf{G}_a} :
(U, T, \delta)
\longmapsto
\Gamma(U, \mathcal{O}_U)
$$
There is a surjective map of sheaves
$\mathcal{O}_{X/S} \to \underline{\mathbf{G}_a}$ defined by the
canonical maps $\Gamma(T, \mathcal{O}_T) \to \Gamma(U, \mathcal{O}_U)$
for objects $(U, T, \delta)$. The kernel of this map is denoted
$\mathcal{J}_{X/S}$, hence a short exact sequence
$$
0 \to
\mathcal{J}_{X/S} \to
\mathcal{O}_{X/S} \to
\underline{\mathbf{G}_a} \to 0
$$
Note that $\mathcal{J}_{X/S}$ comes equipped with a canonical
divided power structure. After all, for each object $(U, T, \delta)$
the third component $\delta$ {\it is} a divided power structure on the
kernel of $\mathcal{O}_T \to \mathcal{O}_U$. Hence the (big hence also
the small) crystalline topos is a divided power topos.

\medskip\noindent
Suppose that $\mathcal{F}$ is a sheaf of $\mathcal{O}_{X/S}$-modules.
In this case the comparison mappings (\ref{equation-comparison})
define a comparison map
\begin{equation}
\label{equation-comparison-modules}
c_f : f^*\mathcal{F}_T \longrightarrow \mathcal{F}_{T'}
\end{equation}
of $\mathcal{O}_T$-modules.




\section{Sheaf of differentials}
\label{section-differentials-sheaf}

\noindent
We define an $S$-derivation on the crystalline site
as follows (this avoids having to pull back the structure sheaf of $S$).

\begin{definition}
\label{definition-global-derivation}
Let $p$, $(S, \mathcal{I}, \gamma)$, and $X \to S$ be as in
Definition \ref{definition-divided-power-thickening-X}.
Let $\mathcal{F}$ be a sheaf of $\mathcal{O}_{X/S}$-modules.
An $S$-derivation $D : \mathcal{O}_{X/S} \to \mathcal{F}$
is a map of sheaves such that for every object $(U, T, \delta)$
mapping into an open $V \subset S$ the map
$$
D : \Gamma(T, \mathcal{O}_T) \longrightarrow \Gamma(T, \mathcal{F})
$$
is a divided power $\Gamma(V, \mathcal{O}_V)$-derivation as in
Definition \ref{definition-derivation}.
\end{definition}

\noindent
Of course this means exactly that $D$ is additive, satisfies the
Leibniz rule, annihilates functions coming from $S$, and
satisfies $D(f^{[n]}) = f^{[n - 1]}D(f)$ for a local section
$f$ of the divided power ideal $\mathcal{J}_{X/S}$.

\begin{lemma}
\label{lemma-module-of-differentials}
Let $p$, $(S, \mathcal{I}, \gamma)$, and $X \to S$ be as in
Definition \ref{definition-divided-power-thickening-X}.
There exists a universal $S$-derivation
$d : \mathcal{O}_{X/S} \to \Omega_{X/S}$.
For an object $(U, T, \delta)$ the restriction $(\Omega_{X/S})_T$
to $T$ is just the quotient of $\Omega_{T/S}$ by the subsheaf
generated by the local sections
$\text{d}\delta_n(f) - \delta_{n - 1}(f)\text{d}f$
where $f$ is a local section of the ideal sheaf of $U$ in $T$.
\end{lemma}

\begin{proof}
Omitted.
\end{proof}

\begin{lemma}
\label{lemma-module-of-differentials-on-affine}
Let $p$, $(S, \mathcal{I}, \gamma)$, and $X \to S$ be as in
Definition \ref{definition-divided-power-thickening-X}.
For any affine object $(U, T, \delta)$
mapping into an affine open $V \subset S$ we have
$$
\Gamma((U, T, \delta), \Omega_{X/S}) =
\Omega_{\Gamma(T, \mathcal{O})/\Gamma(V, \mathcal{O}_V)}
$$
where the right hand side is
as constructed in Section \ref{section-differentials}.
\end{lemma}

\begin{proof}
Omitted.
\end{proof}

\begin{lemma}
\label{lemma-describe-omega}
Let $p$, $(S, \mathcal{I}, \gamma)$, and $X \to S$ be as in
Definition \ref{definition-divided-power-thickening-X}.
Let $(U, T, \delta)$ be an object of $\text{CRIS}(X/S)$.
Let
$$
(U(1), T(1), \delta(1)) = (U, T, \delta) \times (U, T, \delta)
$$
and let $\mathcal{K} \subset \mathcal{O}_{T(1)}$ be the quasi-coherent
sheaf of ideals corresponding to the closed immersion $\Delta : T \to T(1)$.
Then $\mathcal{K} \cap \mathcal{J}_{T(1)}$ is preserved by the
divided structure on $\mathcal{J}_{T(1)}$ and we have
$$
(\Omega_{X/S})_T =
\mathcal{K}/
\left(\mathcal{K}^2 + (\mathcal{K} \cap \mathcal{J}_{T(1)})^{[2]}\right)
$$
\end{lemma}

\begin{proof}
Sheafified version of Lemma \ref{lemma-diagonal-and-differentials-affine-site}.
\end{proof}

\begin{lemma}
\label{lemma-describe-omega-small}
Let $p$, $(S, \mathcal{I}, \gamma)$, and $X \to S$ be as in
Definition \ref{definition-divided-power-thickening-X}.
Let $(U, T, \delta)$ be an object of $\text{Cris}(X/S)$.
Let
$$
(U(1), T(1), \delta(1)) = (U, T, \delta) \times (U, T, \delta)
$$
and let $\mathcal{K} \subset \mathcal{O}_{T(1)}$ be the quasi-coherent
sheaf of ideals corresponding to the closed immersion $\Delta : T \to T(1)$.
Then $\mathcal{K} \subset \mathcal{J}_{T(1)}$ is preserved by the
divided structure on $\mathcal{J}_{T(1)}$ and we have
$$
(\Omega_{X/S})_T = \mathcal{K}/\mathcal{K}^{[2]}
$$
This holds more generally for $(U, T, \delta)$ such that $U \to X$
is formally unramified.
\end{lemma}

\begin{proof}
Special case of Lemma \ref{lemma-describe-omega}.
\end{proof}

\begin{remark}
\label{remark-first-order-thickening}
Let $p$, $(S, \mathcal{I}, \gamma)$, and $X \to S$ be as in
Definition \ref{definition-divided-power-thickening-X}.
Let $(U, T, \delta)$ be an object of $\text{Cris}(X/S)$.
Write $\Omega_{T/S, \delta} = (\Omega_{X/S})_T$, see
Lemma \ref{lemma-module-of-differentials}.
We explicitly describe a first order thickening $T'$ of
$T$. Namely, set
$$
\mathcal{O}_{T'} = \mathcal{O}_T \oplus \Omega_{T/S, \delta}
$$
with algebra structure such that $\Omega_{T/S, \delta}$ is an
ideal of square zero. Let $\mathcal{J} \subset \mathcal{O}_T$
be the ideal sheaf of the closed immersion $U \to T$. Set
$\mathcal{J}' = \mathcal{J} \oplus \Omega_{T/S, \delta}$.
Define a divided power structure on $\mathcal{J}'$ by setting
$$
\delta_n'(f + \omega) = \delta_n(f) + \delta_{n - 1}(f)\omega,
$$
see Lemma \ref{lemma-divided-power-first-order-thickening}.
There are two ring maps
$$
p_0, p_1 : \mathcal{O}_T \to \mathcal{O}_{T'}
$$
The first is given by $f \mapsto f + 0$ and the second by
$f \mapsto f + \text{d}f$. Note that both are compatible with the
divided power structures on $\mathcal{J}$ and $\mathcal{J}'$
and so is the quotient map $\mathcal{O}_{T'} \to \mathcal{O}_T$.
Thus we get an object $(U, T', \delta')$ of $\text{Cris}(X/S)$
and a commutative diagram
$$
\xymatrix{
& T \ar[ld]_{\text{id}} \ar[d]^i \ar[rd]^{\text{id}} \\
T & T' \ar[l]_{p_0} \ar[r]^{p_1} & T
}
$$
of $\text{CRIS}(X/S)$ such that $i$ is a first order thickening whose ideal
sheaf is identified with $\Omega_{T/S, \delta}$ and such that
$p_0^* - p_1^* : \mathcal{O}_T \to \mathcal{O}_{T'}$
is identified with the universal derivation $\text{d}$
composed with the inclusion $\Omega_{T/S, \delta} \to \mathcal{O}_{T'}$.
\end{remark}

\begin{remark}
\label{remark-second-order-thickening}
Let $p$, $(S, \mathcal{I}, \gamma)$, and $X \to S$ be as in
Definition \ref{definition-divided-power-thickening-X}.
Let $(U, T, \delta)$ be an object of $\text{Cris}(X/S)$.
Write $\Omega_{T/S, \delta} = (\Omega_{X/S})_T$, see
Lemma \ref{lemma-module-of-differentials}.
We also write $\Omega^2_{T/S, \delta}$ for its second exterior
power. We explicitly describe a second order thickening $T''$ of
$T$. Namely, set
$$
\mathcal{O}_{T''} =
\mathcal{O}_T \oplus \Omega_{T/S, \delta} \oplus \Omega_{T/S, \delta}
\oplus \Omega^2_{T/S, \delta}
$$
with algebra structure defined in the following way
$$
(f, \omega_1, \omega_2, \eta) \cdot
(f', \omega_1', \omega_2', \eta') =
(ff', f\omega_1' + f'\omega_1, f\omega_2' + f'\omega_2',
f\eta' + f'\eta + \omega_1 \wedge \omega_2' + \omega_1' \wedge \omega_2).
$$
Let $\mathcal{J} \subset \mathcal{O}_T$
be the ideal sheaf of the closed immersion $U \to T$. Let
$\mathcal{J}''$ be the inverse image of $\mathcal{J}$ under the
projection $\mathcal{O}_{T''} \to \mathcal{O}_T$.
Define a divided power structure on $\mathcal{J}''$ by setting
$$
\delta_n''(f, \omega_1, \omega_2, \eta) =
(\delta_n(f), \delta_{n - 1}(f)\omega_1, \delta_{n - 1}(f)\omega_2,
\delta_{n - 1}(f)\eta + \delta_{n - 2}(f)\omega_1 \wedge \omega_2)
$$
see Lemma \ref{lemma-divided-power-second-order-thickening}.
There are three ring maps
$$
q_0, q_1, q_2 : \mathcal{O}_T \to \mathcal{O}_{T''}
$$
The first is given by $f \mapsto (f, 0, 0)$, the second by
$f \mapsto (f, \text{d}f, 0, 0)$, and the third by
$f \mapsto (f, 0, \text{d}f, 0)$. Note that all three are compatible with the
divided power structures on $\mathcal{J}$ and $\mathcal{J}''$
and so is the quotient map $\mathcal{O}_{T''} \to \mathcal{O}_T$.
We claim there are also three maps
$q_{01}, q_{12}, q_{02} : \mathcal{O}_{T'} \to \mathcal{O}_{T''}$
where $\mathcal{O}_{T'}$ is as in Remark \ref{remark-first-order-thickening}.
Namely, we set
$$
q_{01}(f + \omega) = (f, \omega, 0, 0),\quad
q_{12}(f + \omega) = (f, 0, \omega, 0),\quad
q_{02}(f + \omega) = (f, \omega, \omega, 0)
$$
By inspection these are also compatible with the given divided power
structures. Note that $q_0 = q_{01} \circ p_0$ and so on. There are
also two maps $\mathcal{O}_{T''} \to \mathcal{O}_{T'}$, namely
$(f, \omega_1, \omega_2, \eta) \mapsto (f, \omega_1)$ and
$(f, \omega_1, \omega_2, \eta) \mapsto (f, \omega_2)$.
Thus $(U, T'', \delta'')$ is an object of $\text{Cris}(X/S)$
and we get a truncated simplicial object
$$
\xymatrix{
T''
\ar@<2ex>[r]
\ar@<0ex>[r]
\ar@<-2ex>[r]
&
T'
\ar@<1ex>[r]
\ar@<-1ex>[r]
\ar@<1ex>[l]
\ar@<-1ex>[l]
&
T
\ar@<0ex>[l]
}
$$
of $\text{Cris}(X/S)$. In applications we will use $q_i : T'' \to T$ and
$q_{ij} : T'' \to T'$ to denote the morphisms associated to the
ring maps described above.
\end{remark}






\section{Connections}
\label{section-connections}

\noindent
Let $p$, $(S, \mathcal{I}, \gamma)$, and $X \to S$ be as in
Definition \ref{definition-divided-power-thickening-X}.
Working on either the big or small crystalline site, we define
$
\Omega^i_{X/S} = \wedge^i_{\mathcal{O}_{X/S}} \Omega_{X/S}
$
for $i \geq 0$. The universal $S$-derivation $\text{d}$ gives
rise to the {\it de Rham complex}
$$
\mathcal{O}_{X/S} \to \Omega^1_{X/S} \to \Omega^2_{X/S} \to \ldots
$$
on $\text{CRIS}(X/S)$, see
Lemma \ref{lemma-module-of-differentials-on-affine} and
Remark \ref{remark-divided-powers-de-rham-complex}.
Given an $\mathcal{O}_{X/S}$-module $\mathcal{F}$ a
{\it connection} is a map of abelian sheaves
$$
\nabla :
\mathcal{F}
\longrightarrow
\mathcal{F} \otimes_{\mathcal{O}_{X/S}} \Omega^1_{X/S}
$$
such that $\nabla(f s) = f\nabla(s) + s \otimes \text{d}f$
for local sections $s, f$ of $\mathcal{F}$ and $\mathcal{O}_{X/S}$.
Given a connection there are canonical maps
$$
\nabla :
\mathcal{F} \otimes_{\mathcal{O}_{X/S}} \Omega^i_{X/S}
\longrightarrow
\mathcal{F} \otimes_{\mathcal{O}_{X/S}} \Omega^{i + 1}_{X/S}
$$
defined by the rule $\nabla(s \otimes \omega) =
\nabla(s) \wedge \omega + s \otimes \text{d}\omega$
as in Remark \ref{remark-connection}. We say the connection is
{\it integrable} if $\nabla \circ \nabla = 0$. In this case we
obtain the {\it de Rham complex}
$$
\mathcal{F} \to
\mathcal{F} \otimes_{\mathcal{O}_{X/S}} \Omega^1_{X/S} \to
\mathcal{F} \otimes_{\mathcal{O}_{X/S}} \Omega^2_{X/S} \to \ldots
$$
on the big or small crystalline site of $X$ over $S$.

\begin{lemma}
\label{lemma-automatic-connection}
Let $p$, $(S, \mathcal{I}, \gamma)$, and $X \to S$ be as in
Definition \ref{definition-divided-power-thickening-X}.
Let $\mathcal{C} = \text{CRIS}(X/S)$ or $\mathcal{C} = \text{Cris}(X/S)$.
Let $\mathcal{F}$ be a sheaf of $\mathcal{O}_{X/S}$-modules
on $\mathcal{C}$. Suppose that for all $f : T \to T'$ in $\mathcal{C}$
the comparison map (\ref{equation-comparison-modules}) is an isomorphism.
Then $\mathcal{F}$ comes equipped with a canonical integrable connection.
\end{lemma}

\begin{proof}
Say $(U, T, \delta)$ is an object of $\mathcal{C}$.
Let $(U, T', \delta')$ be the infinitesimal thickening of $T$
by $(\Omega_{X/S})_T = \Omega_{T/S, \delta}$
constructed in Remark \ref{remark-first-order-thickening}.
It comes with projections $p_0, p_1 : T' \to T$
and a diagonal $i : T \to T(1)$. By assumption we get
isomorphisms
$$
p_0^*\mathcal{F}_T \xrightarrow{c_0}
\mathcal{F}_{T'} \xleftarrow{c_1}
p_1^*\mathcal{F}_T
$$
of $\mathcal{O}_{T'}$-modules. Pulling $c = c_1^{-1} \circ c_0$
back to $T$ by $i$ we obtain the identity map
of $\mathcal{F}_T$. Hence if $s \in \Gamma(T, \mathcal{F}_T)$
then $c_0(p_0^*s) - c_1(p_1^*s)$ is a section of $\mathcal{F}_{T'}$
which vanishes on pulling back by $\Delta$. Hence
it is a section of
$$
\mathcal{F}_T
\otimes_{\mathcal{O}_T}
\Omega_{T/S, \delta}
$$
because this is (canonically) the kernel of
$\mathcal{F}_{T'} \to i_*\mathcal{F}_T$ due 
to the fact that $\mathcal{F}_{T'}$
is isomorphic to the pull back from $T$ by either projection.

\medskip\noindent
The collection of maps
$$
\nabla : \Gamma(T, \mathcal{F}_T) \to
\Gamma(T, \mathcal{F}_T \otimes_{\mathcal{O}_T} \Omega_{T/S, \delta})
$$
so obtained is functorial in $T$ because the construction of $T'$
is functorial in $T$. Hence we obtain a connection.

\medskip\noindent
To show that the connection is integrable we consider the
object $(U, T'', \delta'')$ constructed in
Remark \ref{remark-second-order-thickening}.
Because $\mathcal{F}$ is a sheaf we see that
$$
\xymatrix{
q_0^*\mathcal{F}_T \ar[rr]_{q_{01}^*c} \ar[rd]_{q_{02}^*c} & &
q_1^*\mathcal{F}_T \ar[ld]^{q_{12}^*c} \\
& q_2^*\mathcal{F}_T
}
$$
is a commutative map of $\mathcal{O}_{T''}$-modules.
When you work this out you get the integrability
condition. Details omitted.
\end{proof}




\section{Crystals in finite locally free modules}
\label{section-crystals}

\noindent
It turns out that a crystal is a very general gadget.
To formulate the very general definition we use the language
of stacks and strongly cartesian morphisms, see
Stacks, Definition \ref{stacks-definition-stack} and
Categories, Definition \ref{categories-definition-cartesian-over-C}.

\begin{definition}
\label{definition-crystal}
Let $p$, $(S, \mathcal{I}, \gamma)$, and $X \to S$ be as in
Definition \ref{definition-divided-power-thickening-X}.
Let $p : \mathcal{C} \to \text{Cris}(X/S)$ be a stack.
A {\it crystal in objects of $\mathcal{C}$ on $X$ relative to $S$}
is a {\it cartesian section} $\sigma : \text{Cris}(X/S) \to \mathcal{C}$,
i.e., a functor $\sigma$ such that $p \circ \sigma = \text{id}$
and such that $\sigma(f)$ is strongly cartesian for all
morphisms $f$ of $\text{Cris}(X/S)$. Similarly for the big crystalline site.
\end{definition}

\noindent
Since this definition may be a bit hard to parse we will work
out exactly what this means in the case of crystals of quasi-coherent
modules on the crystalline sites.


\begin{definition}
\label{definition-lqc}
Let $p$, $(S, \mathcal{I}, \gamma)$, and $X \to S$ be as in
Definition \ref{definition-divided-power-thickening-X}.
Working on either the big or the small site we make the following
definitions.
\begin{enumerate}
\item A sheaf of $\mathcal{O}_{X/S}$-modules $\mathcal{F}$
is called {\it locally quasi-coherent} if for every object
$(U, T, \delta)$ the restriction $\mathcal{F}_T$ is a quasi-coherent
$\mathcal{O}_T$-module.
\item A sheaf of $\mathcal{O}_{X/S}$-modules $\mathcal{F}$
is called {\it quasi-coherent} if it is quasi-coherent in the
sense of
Modules on Sites, Definition \ref{sites-modules-definition-site-local}.
\end{enumerate}
\end{definition}

\begin{enumerate}
\item First discuss $\mathcal{O}$-modules $\mathcal{F}$.
\item Discuss the canonical connection
$\nabla : \mathcal{F} \to \mathcal{F} \otimes \Omega$
on the crystalline site.
\item Briefly discuss quasi-coherent modules.
\item Introduce crystals $=$ crystals in finite locally free modules.
\item Discuss the equivalence of the category of crystals with the
category of modules with integrable, topologically quasi-nilpotent
connection in the affine case. (Uses the divided power envelope for the
first time...?)
\end{enumerate}






\section{Cohomology of crystals}
\label{section-cohomology}

\noindent
In this section we compare crystalline cohomology with de Rham
cohomology. We follow \cite{Bhatt}.

\begin{enumerate}
\item The main theorem in the affine case.
\item The main theorem in the global case.
\end{enumerate}







\section{Other chapters}

\begin{multicols}{2}
\begin{enumerate}
\item \hyperref[introduction-section-phantom]{Introduction}
\item \hyperref[conventions-section-phantom]{Conventions}
\item \hyperref[sets-section-phantom]{Set Theory}
\item \hyperref[categories-section-phantom]{Categories}
\item \hyperref[topology-section-phantom]{Topology}
\item \hyperref[sheaves-section-phantom]{Sheaves on Spaces}
\item \hyperref[algebra-section-phantom]{Commutative Algebra}
\item \hyperref[sites-section-phantom]{Sites and Sheaves}
\item \hyperref[homology-section-phantom]{Homological Algebra}
\item \hyperref[derived-section-phantom]{Derived Categories}
\item \hyperref[more-algebra-section-phantom]{More Algebra}
\item \hyperref[simplicial-section-phantom]{Simplicial Methods}
\item \hyperref[modules-section-phantom]{Sheaves of Modules}
\item \hyperref[sites-modules-section-phantom]{Modules on Sites}
\item \hyperref[injectives-section-phantom]{Injectives}
\item \hyperref[cohomology-section-phantom]{Cohomology of Sheaves}
\item \hyperref[sites-cohomology-section-phantom]{Cohomology on Sites}
\item \hyperref[hypercovering-section-phantom]{Hypercoverings}
\item \hyperref[schemes-section-phantom]{Schemes}
\item \hyperref[constructions-section-phantom]{Constructions of Schemes}
\item \hyperref[properties-section-phantom]{Properties of Schemes}
\item \hyperref[morphisms-section-phantom]{Morphisms of Schemes}
\item \hyperref[coherent-section-phantom]{Coherent Cohomology}
\item \hyperref[divisors-section-phantom]{Divisors}
\item \hyperref[limits-section-phantom]{Limits of Schemes}
\item \hyperref[varieties-section-phantom]{Varieties}
\item \hyperref[chow-section-phantom]{Chow Homology}
\item \hyperref[topologies-section-phantom]{Topologies on Schemes}
\item \hyperref[descent-section-phantom]{Descent}
\item \hyperref[more-morphisms-section-phantom]{More on Morphisms}
\item \hyperref[flat-section-phantom]{More on Flatness}
\item \hyperref[groupoids-section-phantom]{Groupoid Schemes}
\item \hyperref[more-groupoids-section-phantom]{More on Groupoid Schemes}
\item \hyperref[etale-section-phantom]{\'Etale Morphisms of Schemes}
\item \hyperref[etale-cohomology-section-phantom]{\'Etale Cohomology}
\item \hyperref[spaces-section-phantom]{Algebraic Spaces}
\item \hyperref[spaces-properties-section-phantom]{Properties of Algebraic Spaces}
\item \hyperref[spaces-morphisms-section-phantom]{Morphisms of Algebraic Spaces}
\item \hyperref[spaces-topologies-section-phantom]{Topologies on Algebraic Spaces}
\item \hyperref[spaces-descent-section-phantom]{Descent and Algebraic Spaces}
\item \hyperref[spaces-more-morphisms-section-phantom]{More on Morphisms of Spaces}
\item \hyperref[quot-section-phantom]{Quot and Hilbert Spaces}
\item \hyperref[stacks-section-phantom]{Stacks}
\item \hyperref[spaces-groupoids-section-phantom]{Groupoids in Algebraic Spaces}
\item \hyperref[spaces-more-groupoids-section-phantom]{More on Groupoids in Spaces}
\item \hyperref[bootstrap-section-phantom]{Bootstrap}
\item \hyperref[examples-stacks-section-phantom]{Examples of Stacks}
\item \hyperref[groupoids-quotients-section-phantom]{Quotients of Groupoids}
\item \hyperref[algebraic-section-phantom]{Algebraic Stacks}
\item \hyperref[criteria-section-phantom]{Criteria for Representability}
\item \hyperref[stacks-properties-section-phantom]{Properties of Algebraic Stacks}
\item \hyperref[stacks-morphisms-section-phantom]{Morphisms of Algebraic Stacks}
\item \hyperref[examples-section-phantom]{Examples}
\item \hyperref[exercises-section-phantom]{Exercises}
\item \hyperref[guide-section-phantom]{Guide to Literature}
\item \hyperref[desirables-section-phantom]{Desirables}
\item \hyperref[coding-section-phantom]{Coding Style}
\item \hyperref[fdl-section-phantom]{GNU Free Documentation License}
\item \hyperref[index-section-phantom]{Auto Generated Index}
\end{enumerate}
\end{multicols}


\bibliography{my}
\bibliographystyle{amsalpha}

\end{document}
