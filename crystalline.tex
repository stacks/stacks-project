\IfFileExists{stacks-project.cls}{%
\documentclass{stacks-project}
}{%
\documentclass{amsart}
}

% The following AMS packages are automatically loaded with
% the amsart documentclass:
%\usepackage{amsmath}
%\usepackage{amssymb}
%\usepackage{amsthm}

% For dealing with references we use the comment environment
\usepackage{verbatim}
\newenvironment{reference}{\comment}{\endcomment}
%\newenvironment{reference}{}{}
\newenvironment{slogan}{\comment}{\endcomment}
\newenvironment{history}{\comment}{\endcomment}

% For commutative diagrams you can use
% \usepackage{amscd}
\usepackage[all]{xy}

% We use 2cell for 2-commutative diagrams.
\xyoption{2cell}
\UseAllTwocells

% To put source file link in headers.
% Change "template.tex" to "this_filename.tex"
% \usepackage{fancyhdr}
% \pagestyle{fancy}
% \lhead{}
% \chead{}
% \rhead{Source file: \url{template.tex}}
% \lfoot{}
% \cfoot{\thepage}
% \rfoot{}
% \renewcommand{\headrulewidth}{0pt}
% \renewcommand{\footrulewidth}{0pt}
% \renewcommand{\headheight}{12pt}

\usepackage{multicol}

% For cross-file-references
\usepackage{xr-hyper}

% Package for hypertext links:
\usepackage{hyperref}

% For any local file, say "hello.tex" you want to link to please
% use \externaldocument[hello-]{hello}
\externaldocument[introduction-]{introduction}
\externaldocument[conventions-]{conventions}
\externaldocument[sets-]{sets}
\externaldocument[categories-]{categories}
\externaldocument[topology-]{topology}
\externaldocument[sheaves-]{sheaves}
\externaldocument[sites-]{sites}
\externaldocument[stacks-]{stacks}
\externaldocument[fields-]{fields}
\externaldocument[algebra-]{algebra}
\externaldocument[brauer-]{brauer}
\externaldocument[homology-]{homology}
\externaldocument[derived-]{derived}
\externaldocument[simplicial-]{simplicial}
\externaldocument[more-algebra-]{more-algebra}
\externaldocument[smoothing-]{smoothing}
\externaldocument[modules-]{modules}
\externaldocument[sites-modules-]{sites-modules}
\externaldocument[injectives-]{injectives}
\externaldocument[cohomology-]{cohomology}
\externaldocument[sites-cohomology-]{sites-cohomology}
\externaldocument[dga-]{dga}
\externaldocument[dpa-]{dpa}
\externaldocument[hypercovering-]{hypercovering}
\externaldocument[schemes-]{schemes}
\externaldocument[constructions-]{constructions}
\externaldocument[properties-]{properties}
\externaldocument[morphisms-]{morphisms}
\externaldocument[coherent-]{coherent}
\externaldocument[divisors-]{divisors}
\externaldocument[limits-]{limits}
\externaldocument[varieties-]{varieties}
\externaldocument[topologies-]{topologies}
\externaldocument[descent-]{descent}
\externaldocument[perfect-]{perfect}
\externaldocument[more-morphisms-]{more-morphisms}
\externaldocument[flat-]{flat}
\externaldocument[groupoids-]{groupoids}
\externaldocument[more-groupoids-]{more-groupoids}
\externaldocument[etale-]{etale}
\externaldocument[chow-]{chow}
\externaldocument[intersection-]{intersection}
\externaldocument[pic-]{pic}
\externaldocument[adequate-]{adequate}
\externaldocument[dualizing-]{dualizing}
\externaldocument[duality-]{duality}
\externaldocument[discriminant-]{discriminant}
\externaldocument[local-cohomology-]{local-cohomology}
\externaldocument[curves-]{curves}
\externaldocument[resolve-]{resolve}
\externaldocument[models-]{models}
\externaldocument[pione-]{pione}
\externaldocument[etale-cohomology-]{etale-cohomology}
\externaldocument[proetale-]{proetale}
\externaldocument[crystalline-]{crystalline}
\externaldocument[spaces-]{spaces}
\externaldocument[spaces-properties-]{spaces-properties}
\externaldocument[spaces-morphisms-]{spaces-morphisms}
\externaldocument[decent-spaces-]{decent-spaces}
\externaldocument[spaces-cohomology-]{spaces-cohomology}
\externaldocument[spaces-limits-]{spaces-limits}
\externaldocument[spaces-divisors-]{spaces-divisors}
\externaldocument[spaces-over-fields-]{spaces-over-fields}
\externaldocument[spaces-topologies-]{spaces-topologies}
\externaldocument[spaces-descent-]{spaces-descent}
\externaldocument[spaces-perfect-]{spaces-perfect}
\externaldocument[spaces-more-morphisms-]{spaces-more-morphisms}
\externaldocument[spaces-flat-]{spaces-flat}
\externaldocument[spaces-groupoids-]{spaces-groupoids}
\externaldocument[spaces-more-groupoids-]{spaces-more-groupoids}
\externaldocument[bootstrap-]{bootstrap}
\externaldocument[spaces-pushouts-]{spaces-pushouts}
\externaldocument[groupoids-quotients-]{groupoids-quotients}
\externaldocument[spaces-more-cohomology-]{spaces-more-cohomology}
\externaldocument[spaces-simplicial-]{spaces-simplicial}
\externaldocument[formal-spaces-]{formal-spaces}
\externaldocument[restricted-]{restricted}
\externaldocument[spaces-resolve-]{spaces-resolve}
\externaldocument[formal-defos-]{formal-defos}
\externaldocument[defos-]{defos}
\externaldocument[cotangent-]{cotangent}
\externaldocument[examples-defos-]{examples-defos}
\externaldocument[algebraic-]{algebraic}
\externaldocument[examples-stacks-]{examples-stacks}
\externaldocument[stacks-sheaves-]{stacks-sheaves}
\externaldocument[criteria-]{criteria}
\externaldocument[artin-]{artin}
\externaldocument[quot-]{quot}
\externaldocument[stacks-properties-]{stacks-properties}
\externaldocument[stacks-morphisms-]{stacks-morphisms}
\externaldocument[stacks-limits-]{stacks-limits}
\externaldocument[stacks-cohomology-]{stacks-cohomology}
\externaldocument[stacks-perfect-]{stacks-perfect}
\externaldocument[stacks-introduction-]{stacks-introduction}
\externaldocument[stacks-more-morphisms-]{stacks-more-morphisms}
\externaldocument[stacks-geometry-]{stacks-geometry}
\externaldocument[moduli-]{moduli}
\externaldocument[moduli-curves-]{moduli-curves}
\externaldocument[examples-]{examples}
\externaldocument[exercises-]{exercises}
\externaldocument[guide-]{guide}
\externaldocument[desirables-]{desirables}
\externaldocument[coding-]{coding}
\externaldocument[obsolete-]{obsolete}
\externaldocument[fdl-]{fdl}
\externaldocument[index-]{index}

% Theorem environments.
%
\theoremstyle{plain}
\newtheorem{theorem}[subsection]{Theorem}
\newtheorem{proposition}[subsection]{Proposition}
\newtheorem{lemma}[subsection]{Lemma}

\theoremstyle{definition}
\newtheorem{definition}[subsection]{Definition}
\newtheorem{example}[subsection]{Example}
\newtheorem{exercise}[subsection]{Exercise}
\newtheorem{situation}[subsection]{Situation}

\theoremstyle{remark}
\newtheorem{remark}[subsection]{Remark}
\newtheorem{remarks}[subsection]{Remarks}

\numberwithin{equation}{subsection}

% Macros
%
\def\lim{\mathop{\rm lim}\nolimits}
\def\colim{\mathop{\rm colim}\nolimits}
\def\Spec{\mathop{\rm Spec}}
\def\Hom{\mathop{\rm Hom}\nolimits}
\def\Ext{\mathop{\rm Ext}\nolimits}
\def\SheafHom{\mathop{\mathcal{H}\!{\it om}}\nolimits}
\def\SheafExt{\mathop{\mathcal{E}\!{\it xt}}\nolimits}
\def\Sch{\textit{Sch}}
\def\Mor{\mathop{\rm Mor}\nolimits}
\def\Ob{\mathop{\rm Ob}\nolimits}
\def\Sh{\mathop{\textit{Sh}}\nolimits}
\def\NL{\mathop{N\!L}\nolimits}
\def\proetale{{pro\text{-}\acute{e}tale}}
\def\etale{{\acute{e}tale}}
\def\QCoh{\textit{QCoh}}
\def\Ker{\mathop{\rm Ker}}
\def\Im{\mathop{\rm Im}}
\def\Coker{\mathop{\rm Coker}}
\def\Coim{\mathop{\rm Coim}}

%
% Macros for moduli stacks/spaces
%
\def\QCohstack{\mathcal{QC}\!{\it oh}}
\def\Cohstack{\mathcal{C}\!{\it oh}}
\def\Spacesstack{\mathcal{S}\!{\it paces}}
\def\Quotfunctor{{\rm Quot}}
\def\Hilbfunctor{{\rm Hilb}}
\def\Curvesstack{\mathcal{C}\!{\it urves}}
\def\Polarizedstack{\mathcal{P}\!{\it olarized}}
\def\Complexesstack{\mathcal{C}\!{\it omplexes}}
% \Pic is the operator that assigns to X its picard group, usage \Pic(X)
% \Picardstack_{X/B} denotes the Picard stack of X over B
% \Picardfunctor_{X/B} denotes the Picard functor of X over B
\def\Pic{\mathop{\rm Pic}\nolimits}
\def\Picardstack{\mathcal{P}\!{\it ic}}
\def\Picardfunctor{{\rm Pic}}
\def\Deformationcategory{\mathcal{D}\!{\it ef}}


% OK, start here.
%
\begin{document}

\title{Crystalline Cohomology}


\maketitle

\phantomsection
\label{section-phantom}

\tableofcontents



\section{Introduction}
\label{section-introduction}

\noindent
This chapter is based on a lecture series given by Johan de Jong
held in 2012 at Columbia University.
The goals of this chapter are to give a quick introduction to
crystalline cohomology. A reference is the book \cite{Berthelot}.





\section{Divided powers}
\label{section-divided-powers}

\noindent
In this section we collect some results on divided power rings.
We will use the convention $0! = 1$ (as empty products should give $1$).

\begin{definition}
\label{definition-divided-powers}
Let $A$ be a ring. Let $I$ be an ideal of $A$. A collection of maps
$\gamma_n : I \to I$, $n > 0$ is called a {\it divided power structure}
on $I$ if for all $n \geq 0$, $m > 0$, $x, y \in I$, and $a \in A$ we have
\begin{enumerate}
\item $\gamma_1(x) = x$, we also set $\gamma_0(x) = 1$,
\item $\gamma_n(x)\gamma_m(x) = \frac{(n + m)!}{n! m!} \gamma_{n + m}(x)$,
\item $\gamma_n(ax) = a^n \gamma_n(x)$,
\item $\gamma_n(x + y) = \sum_{i = 0, \ldots, n} \gamma_i(x)\gamma_{n - i}(y)$,
\item $\gamma_n(\gamma_m(x)) = \frac{(nm)!}{n! (m!)^n} \gamma_{nm}(x)$.
\end{enumerate}
\end{definition}

\noindent
Some observations. Note that condition (2) implies that
$n \gamma_n(x) = \gamma_1(x)\gamma_{n - 1}(x)$. Hence by induction
and condition (1) we get $n! \gamma_n(x) = x^n$. Thus $\gamma_n(x)$
is a replacement for $x^n/n!$ in $I$. If $A$ is torsion free as a
$\mathbf{Z}$-module, then all the other axioms follow from this. 
Note that the rational numbers $\frac{(n + m)!}{n! m!}$ and
$\frac{(nm)!}{n! (m!)^n}$ occuring in the definition are in fact integers.

\begin{example}
\label{example-ideal-generated-by-p}
Let $p$ be a prime number.
Let $A$ be a ring such that every integer $n$ not divisible by $p$
is invertible, i.e., $A$ is a $\mathbf{Z}_{(p)}$-algebra. Then
$I = pA$ has a canonical divided power structure. Namely, given
$x = pa \in A$ we set
$$
\gamma_n(x) = \frac{p^n}{n!} a^n
$$
The reader verifies immediately that $p^n/n!$ is an integer so that
the definition makes sense. It is a straightforward exercise to
verify that conditions (1) -- (5) of
Definition \ref{definition-divided-powers} are satisfied.
\end{example}

\begin{lemma}
\label{lemma-need-only-gamma-p}
Let $p$ be a prime number. Let $A$ be a ring such that every integer $n$
not divisible by $p$ is invertible, i.e., $A$ is a $\mathbf{Z}_{(p)}$-algebra.
Let $I \subset A$ be an ideal. Two divided power structures
$\gamma, \gamma'$ on $I$ are equal if and only if $\gamma_p = \gamma'_p$.
Moreover, given a map $\delta : I \to I$ such that
\begin{enumerate}
\item $p!\delta(x) = x^p$ for all $x \in I$,
\item $\delta(ax) = a^p\delta(x)$ for all $a \in A$, $x \in I$, and
\item
$\delta(x + y) =
\delta(x) +
\sum\nolimits_{i + j = p, i,j \geq 1} \frac{x^i}{i!}\frac{y^j}{j!} +
\delta(y)$ for all $x, y \in I$,
\end{enumerate}
then there exists a unique divided power structure $\gamma$ on $I$ such
that $\gamma_p = \delta$.
\end{lemma}

\begin{proof}
If $n$ is not divisible by $p$, then $\gamma_n(x) = c x \gamma_{n - 1}(x)$
where $c$ is a unit in $\mathbf{Z}_{(p)}$. Moreover,
$$
\gamma_{pm}(x) = c \gamma_m(\gamma_p(x))
$$
where $c$ is a unit in $\mathbf{Z}_{(p)}$. Thus the first assertion is clear.
For the second assertion, we can, working backwards, use these equalities
to define all $\gamma_n$. Then a long and tedious computation shows all
the axioms are satisfied if $\delta$ satisfies (1), (2), (3).
Details omitted.
\end{proof}

\begin{definition}
\label{definition-divided-power-ring}
A {\it divided power ring} is a triple $(A, I, \gamma)$ where
$A$ is a ring, $I \subset A$ is an ideal, and $\gamma = (\gamma_n)_{n \geq 1}$
is a divided power structure on $I$.
A {\it homomorphism of divided power rings}
$\varphi : (A, I, \gamma) \to (B, J, \delta)$ is a ring homomorphism
$\varphi : A \to B$ such that $\varphi(I) \subset J$ and such that
$\delta_n(\varphi(x)) = \varphi(\gamma_n(x))$ for all $x \in I$.
\end{definition}

\begin{lemma}
\label{lemma-nil}
Let $p$ be a prime number.
If $(A, I, \gamma)$ is a divided power ring and $p$ is nilpotent
in $A$, then $I$ is locally nilpotent.
\end{lemma}

\begin{proof}
If $p^N = 0$ in $A$, then for $x \in I$ we have
$x^{pN} = (pN)!\gamma_N(x) = 0$ because $(pN)!$ is
divisible by $p^N$.
\end{proof}

\begin{lemma}
\label{lemma-colimits}
The category of divided power rings has all limits and colimits.
\end{lemma}

\begin{proof}
Empty limit: zero ring (that's weird but we need it).
Products: Exactly what you think they should be.
Equalizers: As in rings.
Empty colimit: $\mathbf{Z}$ with PD-ideal $(0)$.
General Colimits. Let $\mathcal{C}$ be a category and let
$c \mapsto (A_c, I_c, \gamma_c)$ be a diagram. Consider the functor
$$
F(B, J, \delta) = \lim_{c \in \mathcal{C}}
Hom((A_c, I_c, \gamma_c), (B, J, \delta))
$$
Note that any $f = (f_c)_{c \in C} \in F(B, J, \delta)$ has the property
that all the images $f_c(A_c)$ generate a subring $B'$ of $B$ of bounded
cardinality $\kappa$ and that all the images $f_c(I_c)$ generate a
divided power sub ideal $J'$ of $B'$. And we get a factorization of
$f$ as a $f'$ in $F(B')$ followed by the inclusion $B' \to B$.
Hence we see that $F(B, J, \delta)$ can be computed if we know the value
of $F$ on divided power rings of
cardinality $\kappa$ or less. Consider a set of objects $U$ of
dividing power rings containing an object isomorphic to every
$(B, J, \delta)$ of cardinality
at most $\kappa$. Then
$$
\lim_{(B, J, \delta) \in U, f \in F(B, J, \delta)} (B, J, \delta)
$$
will be the colimit we wanted to construct.
\end{proof}

\begin{lemma}
\label{lemma-divided-power-envelope}
Let $(A, I, \gamma)$ be a divided power ring.
Let $A \to B$ be a ring map. Let $J \subset B$ be an ideal
with $IB \subset J$. There exists a homomorphism of
divided power rings
$$
(A, I, \gamma) \longrightarrow (D_B(J), \bar J, \bar \gamma)
$$
such that
$$
\Hom_{(A, I, \gamma)}((D_B(J), \bar J, \bar \gamma), (C, K, \delta))
=
\Hom_A((B, J), (C, K))
$$
functorially in the divided power algebra $(C, K, \delta)$ over
$(A, I, \gamma)$.
\end{lemma}

\begin{proof}
Entirely similar to the proof of Lemma \ref{lemma-colimits}.
\end{proof}

\begin{definition}
\label{definition-divided-power-envelope}
Let $(A, I, \gamma)$ be a divided power ring.
Let $A \to B$ be a ring map. Let $J \subset B$ be an ideal
with $IB \subset J$. The divided power algebra $D_B(J)$
constructed in Lemma \ref{lemma-divided-power-envelope}
is called the {\it divided power envelope of $J$ in $B$
relative to $(A, I, \gamma)$}.
\end{definition}

\noindent
Discussion. There is an $A$-algebra map $B \to D_B(J)$ which
maps $J$ into $\bar J$. In fact, $\bar J$ is generated by
the elements $\bar\gamma_n(x)$ where $x$ is in the image of
$J \to D_B(J)$ (think $x \in J$). Let $(A, I, \gamma) \to (C, K, \delta)$
be a homomorphism of divided power rings. The universal property is
just that ring maps $B \to C$ which map $J$ into $K$ correspond
1-to-1 to homomorphisms of divided power rings
$(D_B(J), \bar J, \bar \gamma) \to (C, K, \delta)$ via precomposing with the
canonical map $B \to D_B(J)$.

\medskip\noindent
The following lemma can be generalized to the case where $B$ and $B'$
come with their own divided power ideals... see
\cite[Proposition 2.1.7]{dJ-crystalline}.
Anyway, it in particular says that taking the divided power
envelope commutes with localization.

\begin{lemma}
\label{lemma-flat-base-change-divided-power-envelope}
Let $(A, I, \gamma)$ be a divided power ring.
Let $B \to B'$ be a homomorphism of $A$-algebras.
Let $IB \subset J \subset B$ be an ideal.
Assume that $B/IB \to B'/IB'$ is flat.
Then $D_B(J) \otimes_B B' = D_{B'}(JB')$.
\end{lemma}

\begin{proof}
Omitted.
\end{proof}

\begin{lemma}
\label{lemma-gamma-extends}
Let $(A, I, \gamma)$ be a divided power ring.
Let $A \to B$ be a ring map. Then $\gamma$ 
extends to $IB$ if one of the following conditions is satisfied:
\begin{enumerate}
\item $IB = 0$,
\item $A \to B$ is flat, or
\item $I$ is (locally) principal.
\end{enumerate}
\end{lemma}

\begin{proof}
Omitted.
\end{proof}




\section{Compatibility}
\label{section-compatibility}

\noindent
This section isn't required reading; it explains how our discussion
fits with that of \cite{Berthelot}.
Consider the following technical notion.

\begin{definition}
\label{definition-compatible}
Let $(A, I, \gamma)$ and $(B, J, \delta)$ be divided power rings.
Let $A \to B$ be a ring map. We say
{\it $\delta$ is compatible with $\delta$}
if there exists a divided power structure $\bar\gamma$ on
$J + IB$ such that
$$
(A, I, \gamma) \to (B, J + IB, \bar \gamma)\quad\text{and}\quad
(B, J, \delta) \to (B, J + IB, \bar \gamma)
$$
are homomorphisms of divided power rings.
\end{definition}

\noindent
Let $p$ be a prime number. Let $(A, I, \gamma)$ be a divided power ring
such that $p$ is nilpotent in $A$. Let $A \to C$ be a ring
map. Assume that $\gamma$ extends to $IC$ (see
Lemma \ref{lemma-gamma-extends}).
The (big affine) crystalline site of $\Spec(C)$ over $(A, I, \gamma)$
as defined by Berthelot consists of systems
$((B, J, \delta), A \to B, C \to B/J)$ where $(B, J, \delta)$ is
a divided power ring, $\delta$ is compatible with $\gamma$ and
the diagram
$$
\xymatrix{
B \ar[r] & B/J \\
A \ar[u] \ar[r] & C \ar[u]
}
$$
is commutative. The two conditions
(a) $\gamma$ extends to $C$, and (b) $\delta$ compatible with $\gamma$
are used in Berthelot's thesis to insure that
the crystalline cohomology of $\Spec(C)$ is the same as the crystalline
cohomology of $\Spec(C/IC)$. We will try to avoid this issue
by working exclusively with $C$ such that $IC = 0$. In this case,
for a system $((B, J, \delta), A \to B, C \to B/J)$ as above,
the commutativity of the displayed diagram above implies $IB \subset J$ and
compatibility is equivalent to the condition that
$(A, I, \gamma) \to (B, J, \delta)$ is a homomorphism of divided
power rings.


\section{Module of differentials}
\label{section-differentials}

\noindent
In this section we develop a theory of modules of differentials
for divided power rings.

\begin{definition}
\label{definition-derivation}
Let $(A, I, \gamma) \to (B, J, \delta)$ be a homomorphism
of divided power rings. Let $M$ be an $B$-module.
An {\it divided power $A$-derivation} into $M$ is a map $D : B \to M$ which is
additive, annihilates the elements of $A$, satisfies the
Leibniz rule $D(bb') = bD(b') + b'D(b)$ and satisfies
$$
D(\gamma_n(x)) = n\gamma_{n - 1}(x)D(x)
$$
for all $n \geq 1$ and all $x \in J$.
\end{definition}

\noindent
As in the case of usual derivations, there exists a
{\it universal divided power $A$-derivation} $d_{A/B} : B \to \Omega_{B/A}$
such that any derivation $D : B \to M$ is equal to
$D = \xi \circ d_{B/A}$ for some $B$-linear map $\Omega_{B/A} \to M$.

\begin{lemma}
\label{lemma-diagonal-and-differentials}
Let $(A, I, \gamma) \to (B, J, \delta)$ be a homomorphism
of divided power rings. Let $(B(1), J(1), \delta(1))$ be the coproduct
of $(B, J, \delta)$ with itself over $(A, I, \gamma)$, i.e.,
such that
$$
\xymatrix{
(B, J, \delta) \ar[r] & (B(1), J(1), \delta(1)) \\
(A, I, \gamma) \ar[r] \ar[u] & (B, J, \delta) \ar[u]
}
$$
is cocartesian. Denote $K = \text{Ker}(B(1) \to B)$.
Then $K$ is a divided power ideal and $\Omega_{B/A} = K/K^{[2]}$
canonically.
\end{lemma}

\begin{proof}
Omitted.
\end{proof}






\section{Crystalline site}
\label{section-site}

\noindent
We can use either the big site or the small site.
There is also a syntomic site (sometimes useful).

\begin{enumerate}
\item Talk about divided power thickenings $(S, T, \gamma)$.
\item To simplify work over an affine base $(A, I, \gamma)$?
Always denote the base $\Sigma = \Spec(A)$.
\item Get around compatibility by looking only at $X$ lying over
$V(I)$?
\item Introduce $\text{Cris}(X/\Sigma, I, \gamma)$ and
$\text{CRIS}(X/\Sigma, I, \gamma)$.
\item Talk about fibre products and products in these sites.
Talk about coverings. Talk about sheaves.
\item Functoriality of crystalline topos? This is a bit tricky...
Probably should go later.
\end{enumerate}






\section{Crystals in finite locally free modules}
\label{section-crystals}

\noindent
Of course you can define crystals in much greater generality.

\begin{enumerate}
\item First discuss $\mathcal{O}$-modules $\mathcal{F}$.
\item Discuss the canonical connection
$\nabla : \mathcal{F} \to \mathcal{F} \otimes \Omega$
on the crystalline site.
\item Briefly discuss quasi-coherent modules.
\item Introduce crystals $=$ crystals in finite locally free modules.
\item Discuss the equivalence of the category of crystals with the
category of modules with integrable, topologically quasi-nilpotent
connection in the affine case. (Uses the divided power envelope for the
first time...?)
\end{enumerate}






\section{Cohomology of crystals}
\label{section-cohomology}

\noindent
In this section we compare crystalline cohomology with de Rham
cohomology. We follow \cite{Bhatt}.

\begin{enumerate}
\item The main theorem in the affine case.
\item The main theorem in the global case.
\end{enumerate}







\section{Other chapters}

\begin{multicols}{2}
\begin{enumerate}
\item \hyperref[introduction-section-phantom]{Introduction}
\item \hyperref[conventions-section-phantom]{Conventions}
\item \hyperref[sets-section-phantom]{Set Theory}
\item \hyperref[categories-section-phantom]{Categories}
\item \hyperref[topology-section-phantom]{Topology}
\item \hyperref[sheaves-section-phantom]{Sheaves on Spaces}
\item \hyperref[algebra-section-phantom]{Commutative Algebra}
\item \hyperref[sites-section-phantom]{Sites and Sheaves}
\item \hyperref[homology-section-phantom]{Homological Algebra}
\item \hyperref[derived-section-phantom]{Derived Categories}
\item \hyperref[more-algebra-section-phantom]{More Algebra}
\item \hyperref[simplicial-section-phantom]{Simplicial Methods}
\item \hyperref[modules-section-phantom]{Sheaves of Modules}
\item \hyperref[sites-modules-section-phantom]{Modules on Sites}
\item \hyperref[injectives-section-phantom]{Injectives}
\item \hyperref[cohomology-section-phantom]{Cohomology of Sheaves}
\item \hyperref[sites-cohomology-section-phantom]{Cohomology on Sites}
\item \hyperref[hypercovering-section-phantom]{Hypercoverings}
\item \hyperref[schemes-section-phantom]{Schemes}
\item \hyperref[constructions-section-phantom]{Constructions of Schemes}
\item \hyperref[properties-section-phantom]{Properties of Schemes}
\item \hyperref[morphisms-section-phantom]{Morphisms of Schemes}
\item \hyperref[coherent-section-phantom]{Coherent Cohomology}
\item \hyperref[divisors-section-phantom]{Divisors}
\item \hyperref[limits-section-phantom]{Limits of Schemes}
\item \hyperref[varieties-section-phantom]{Varieties}
\item \hyperref[chow-section-phantom]{Chow Homology}
\item \hyperref[topologies-section-phantom]{Topologies on Schemes}
\item \hyperref[descent-section-phantom]{Descent}
\item \hyperref[more-morphisms-section-phantom]{More on Morphisms}
\item \hyperref[flat-section-phantom]{More on Flatness}
\item \hyperref[groupoids-section-phantom]{Groupoid Schemes}
\item \hyperref[more-groupoids-section-phantom]{More on Groupoid Schemes}
\item \hyperref[etale-section-phantom]{\'Etale Morphisms of Schemes}
\item \hyperref[etale-cohomology-section-phantom]{\'Etale Cohomology}
\item \hyperref[spaces-section-phantom]{Algebraic Spaces}
\item \hyperref[spaces-properties-section-phantom]{Properties of Algebraic Spaces}
\item \hyperref[spaces-morphisms-section-phantom]{Morphisms of Algebraic Spaces}
\item \hyperref[spaces-topologies-section-phantom]{Topologies on Algebraic Spaces}
\item \hyperref[spaces-descent-section-phantom]{Descent and Algebraic Spaces}
\item \hyperref[spaces-more-morphisms-section-phantom]{More on Morphisms of Spaces}
\item \hyperref[quot-section-phantom]{Quot and Hilbert Spaces}
\item \hyperref[stacks-section-phantom]{Stacks}
\item \hyperref[spaces-groupoids-section-phantom]{Groupoids in Algebraic Spaces}
\item \hyperref[spaces-more-groupoids-section-phantom]{More on Groupoids in Spaces}
\item \hyperref[bootstrap-section-phantom]{Bootstrap}
\item \hyperref[examples-stacks-section-phantom]{Examples of Stacks}
\item \hyperref[groupoids-quotients-section-phantom]{Quotients of Groupoids}
\item \hyperref[algebraic-section-phantom]{Algebraic Stacks}
\item \hyperref[criteria-section-phantom]{Criteria for Representability}
\item \hyperref[stacks-properties-section-phantom]{Properties of Algebraic Stacks}
\item \hyperref[stacks-morphisms-section-phantom]{Morphisms of Algebraic Stacks}
\item \hyperref[examples-section-phantom]{Examples}
\item \hyperref[exercises-section-phantom]{Exercises}
\item \hyperref[guide-section-phantom]{Guide to Literature}
\item \hyperref[desirables-section-phantom]{Desirables}
\item \hyperref[coding-section-phantom]{Coding Style}
\item \hyperref[fdl-section-phantom]{GNU Free Documentation License}
\item \hyperref[index-section-phantom]{Auto Generated Index}
\end{enumerate}
\end{multicols}


\bibliography{my}
\bibliographystyle{amsalpha}

\end{document}
