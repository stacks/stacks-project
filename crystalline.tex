\IfFileExists{stacks-project.cls}{%
\documentclass{stacks-project}
}{%
\documentclass{amsart}
}

% The following AMS packages are automatically loaded with
% the amsart documentclass:
%\usepackage{amsmath}
%\usepackage{amssymb}
%\usepackage{amsthm}

% For dealing with references we use the comment environment
\usepackage{verbatim}
\newenvironment{reference}{\comment}{\endcomment}
%\newenvironment{reference}{}{}
\newenvironment{slogan}{\comment}{\endcomment}
\newenvironment{history}{\comment}{\endcomment}

% For commutative diagrams you can use
% \usepackage{amscd}
\usepackage[all]{xy}

% We use 2cell for 2-commutative diagrams.
\xyoption{2cell}
\UseAllTwocells

% To put source file link in headers.
% Change "template.tex" to "this_filename.tex"
% \usepackage{fancyhdr}
% \pagestyle{fancy}
% \lhead{}
% \chead{}
% \rhead{Source file: \url{template.tex}}
% \lfoot{}
% \cfoot{\thepage}
% \rfoot{}
% \renewcommand{\headrulewidth}{0pt}
% \renewcommand{\footrulewidth}{0pt}
% \renewcommand{\headheight}{12pt}

\usepackage{multicol}

% For cross-file-references
\usepackage{xr-hyper}

% Package for hypertext links:
\usepackage{hyperref}

% For any local file, say "hello.tex" you want to link to please
% use \externaldocument[hello-]{hello}
\externaldocument[introduction-]{introduction}
\externaldocument[conventions-]{conventions}
\externaldocument[sets-]{sets}
\externaldocument[categories-]{categories}
\externaldocument[topology-]{topology}
\externaldocument[sheaves-]{sheaves}
\externaldocument[sites-]{sites}
\externaldocument[stacks-]{stacks}
\externaldocument[fields-]{fields}
\externaldocument[algebra-]{algebra}
\externaldocument[brauer-]{brauer}
\externaldocument[homology-]{homology}
\externaldocument[derived-]{derived}
\externaldocument[simplicial-]{simplicial}
\externaldocument[more-algebra-]{more-algebra}
\externaldocument[smoothing-]{smoothing}
\externaldocument[modules-]{modules}
\externaldocument[sites-modules-]{sites-modules}
\externaldocument[injectives-]{injectives}
\externaldocument[cohomology-]{cohomology}
\externaldocument[sites-cohomology-]{sites-cohomology}
\externaldocument[dga-]{dga}
\externaldocument[dpa-]{dpa}
\externaldocument[hypercovering-]{hypercovering}
\externaldocument[schemes-]{schemes}
\externaldocument[constructions-]{constructions}
\externaldocument[properties-]{properties}
\externaldocument[morphisms-]{morphisms}
\externaldocument[coherent-]{coherent}
\externaldocument[divisors-]{divisors}
\externaldocument[limits-]{limits}
\externaldocument[varieties-]{varieties}
\externaldocument[topologies-]{topologies}
\externaldocument[descent-]{descent}
\externaldocument[perfect-]{perfect}
\externaldocument[more-morphisms-]{more-morphisms}
\externaldocument[flat-]{flat}
\externaldocument[groupoids-]{groupoids}
\externaldocument[more-groupoids-]{more-groupoids}
\externaldocument[etale-]{etale}
\externaldocument[chow-]{chow}
\externaldocument[intersection-]{intersection}
\externaldocument[pic-]{pic}
\externaldocument[adequate-]{adequate}
\externaldocument[dualizing-]{dualizing}
\externaldocument[duality-]{duality}
\externaldocument[discriminant-]{discriminant}
\externaldocument[local-cohomology-]{local-cohomology}
\externaldocument[curves-]{curves}
\externaldocument[resolve-]{resolve}
\externaldocument[models-]{models}
\externaldocument[pione-]{pione}
\externaldocument[etale-cohomology-]{etale-cohomology}
\externaldocument[proetale-]{proetale}
\externaldocument[crystalline-]{crystalline}
\externaldocument[spaces-]{spaces}
\externaldocument[spaces-properties-]{spaces-properties}
\externaldocument[spaces-morphisms-]{spaces-morphisms}
\externaldocument[decent-spaces-]{decent-spaces}
\externaldocument[spaces-cohomology-]{spaces-cohomology}
\externaldocument[spaces-limits-]{spaces-limits}
\externaldocument[spaces-divisors-]{spaces-divisors}
\externaldocument[spaces-over-fields-]{spaces-over-fields}
\externaldocument[spaces-topologies-]{spaces-topologies}
\externaldocument[spaces-descent-]{spaces-descent}
\externaldocument[spaces-perfect-]{spaces-perfect}
\externaldocument[spaces-more-morphisms-]{spaces-more-morphisms}
\externaldocument[spaces-flat-]{spaces-flat}
\externaldocument[spaces-groupoids-]{spaces-groupoids}
\externaldocument[spaces-more-groupoids-]{spaces-more-groupoids}
\externaldocument[bootstrap-]{bootstrap}
\externaldocument[spaces-pushouts-]{spaces-pushouts}
\externaldocument[groupoids-quotients-]{groupoids-quotients}
\externaldocument[spaces-more-cohomology-]{spaces-more-cohomology}
\externaldocument[spaces-simplicial-]{spaces-simplicial}
\externaldocument[formal-spaces-]{formal-spaces}
\externaldocument[restricted-]{restricted}
\externaldocument[spaces-resolve-]{spaces-resolve}
\externaldocument[formal-defos-]{formal-defos}
\externaldocument[defos-]{defos}
\externaldocument[cotangent-]{cotangent}
\externaldocument[examples-defos-]{examples-defos}
\externaldocument[algebraic-]{algebraic}
\externaldocument[examples-stacks-]{examples-stacks}
\externaldocument[stacks-sheaves-]{stacks-sheaves}
\externaldocument[criteria-]{criteria}
\externaldocument[artin-]{artin}
\externaldocument[quot-]{quot}
\externaldocument[stacks-properties-]{stacks-properties}
\externaldocument[stacks-morphisms-]{stacks-morphisms}
\externaldocument[stacks-limits-]{stacks-limits}
\externaldocument[stacks-cohomology-]{stacks-cohomology}
\externaldocument[stacks-perfect-]{stacks-perfect}
\externaldocument[stacks-introduction-]{stacks-introduction}
\externaldocument[stacks-more-morphisms-]{stacks-more-morphisms}
\externaldocument[stacks-geometry-]{stacks-geometry}
\externaldocument[moduli-]{moduli}
\externaldocument[moduli-curves-]{moduli-curves}
\externaldocument[examples-]{examples}
\externaldocument[exercises-]{exercises}
\externaldocument[guide-]{guide}
\externaldocument[desirables-]{desirables}
\externaldocument[coding-]{coding}
\externaldocument[obsolete-]{obsolete}
\externaldocument[fdl-]{fdl}
\externaldocument[index-]{index}

% Theorem environments.
%
\theoremstyle{plain}
\newtheorem{theorem}[subsection]{Theorem}
\newtheorem{proposition}[subsection]{Proposition}
\newtheorem{lemma}[subsection]{Lemma}

\theoremstyle{definition}
\newtheorem{definition}[subsection]{Definition}
\newtheorem{example}[subsection]{Example}
\newtheorem{exercise}[subsection]{Exercise}
\newtheorem{situation}[subsection]{Situation}

\theoremstyle{remark}
\newtheorem{remark}[subsection]{Remark}
\newtheorem{remarks}[subsection]{Remarks}

\numberwithin{equation}{subsection}

% Macros
%
\def\lim{\mathop{\rm lim}\nolimits}
\def\colim{\mathop{\rm colim}\nolimits}
\def\Spec{\mathop{\rm Spec}}
\def\Hom{\mathop{\rm Hom}\nolimits}
\def\Ext{\mathop{\rm Ext}\nolimits}
\def\SheafHom{\mathop{\mathcal{H}\!{\it om}}\nolimits}
\def\SheafExt{\mathop{\mathcal{E}\!{\it xt}}\nolimits}
\def\Sch{\textit{Sch}}
\def\Mor{\mathop{\rm Mor}\nolimits}
\def\Ob{\mathop{\rm Ob}\nolimits}
\def\Sh{\mathop{\textit{Sh}}\nolimits}
\def\NL{\mathop{N\!L}\nolimits}
\def\proetale{{pro\text{-}\acute{e}tale}}
\def\etale{{\acute{e}tale}}
\def\QCoh{\textit{QCoh}}
\def\Ker{\mathop{\rm Ker}}
\def\Im{\mathop{\rm Im}}
\def\Coker{\mathop{\rm Coker}}
\def\Coim{\mathop{\rm Coim}}

%
% Macros for moduli stacks/spaces
%
\def\QCohstack{\mathcal{QC}\!{\it oh}}
\def\Cohstack{\mathcal{C}\!{\it oh}}
\def\Spacesstack{\mathcal{S}\!{\it paces}}
\def\Quotfunctor{{\rm Quot}}
\def\Hilbfunctor{{\rm Hilb}}
\def\Curvesstack{\mathcal{C}\!{\it urves}}
\def\Polarizedstack{\mathcal{P}\!{\it olarized}}
\def\Complexesstack{\mathcal{C}\!{\it omplexes}}
% \Pic is the operator that assigns to X its picard group, usage \Pic(X)
% \Picardstack_{X/B} denotes the Picard stack of X over B
% \Picardfunctor_{X/B} denotes the Picard functor of X over B
\def\Pic{\mathop{\rm Pic}\nolimits}
\def\Picardstack{\mathcal{P}\!{\it ic}}
\def\Picardfunctor{{\rm Pic}}
\def\Deformationcategory{\mathcal{D}\!{\it ef}}


% OK, start here.
%
\begin{document}

\title{Crystalline Cohomology}


\maketitle

\phantomsection
\label{section-phantom}

\tableofcontents



\section{Introduction}
\label{section-introduction}

\noindent
This chapter is based on a lecture series given by Johan de Jong
held in 2012 at Columbia University.
The goals of this chapter are to give a quick introduction to
crystalline cohomology. A reference is the book \cite{Berthelot}.





\section{Divided powers}
\label{section-divided-powers}

\noindent
In this section we collect some results on divided power rings.
We will use the convention $0! = 1$ (as empty products should give $1$).

\begin{definition}
\label{definition-divided-powers}
Let $A$ be a ring. Let $I$ be an ideal of $A$. A collection of maps
$\gamma_n : I \to I$, $n > 0$ is called a {\it divided power structure}
on $I$ if for all $n \geq 0$, $m > 0$, $x, y \in I$, and $a \in A$ we have
\begin{enumerate}
\item $\gamma_1(x) = x$, we also set $\gamma_0(x) = 1$,
\item $\gamma_n(x)\gamma_m(x) = \frac{(n + m)!}{n! m!} \gamma_{n + m}(x)$,
\item $\gamma_n(ax) = a^n \gamma_n(x)$,
\item $\gamma_n(x + y) = \sum_{i = 0, \ldots, n} \gamma_i(x)\gamma_{n - i}(y)$,
\item $\gamma_n(\gamma_m(x)) = \frac{(nm)!}{n! (m!)^n} \gamma_{nm}(x)$.
\end{enumerate}
\end{definition}

\noindent
Note that the rational numbers $\frac{(n + m)!}{n! m!}$
and $\frac{(nm)!}{n! (m!)^n}$ occuring in the definition are in fact integers;
the first is the number of ways to choose $n$ out of $n + m$ and
the second counts the number of ways to divide a group of $nm$
objects into $n$ groups of $m$.
We make some remarks about the definition which show that
$\gamma_n(x)$ is a replacement for $x^n/n!$ in $I$.

\begin{lemma}
\label{lemma-silly}
Let $A$ be a ring. Let $I$ be an ideal of $A$.
\begin{enumerate}
\item If $\gamma$ is a divided power structure on $I$, then
$n! \gamma_n(x) = x^n$ for $n \geq 1$, $x \in I$.
\end{enumerate}
Assume $A$ is torsion free as a $\mathbf{Z}$-module.
\begin{enumerate}
\item[(2)] A divided power structure on $I$, if it exists, is unique.
\item[(3)] If $\gamma_n : I \to I$ are maps then
$$
\gamma\text{ is a divided power structure}
\Leftrightarrow
n! \gamma_n(x) = x^n\ \forall x \in I, n \geq 1.
$$
\item[(4)] The ideal $I$ has a divided power structure
if and only if there exists
a set of generators $x_i$ of $I$ as an ideal such that
for all $n \geq 1$ we have $x_i^n \in (n!)I$.
\end{enumerate}
\end{lemma}

\begin{proof}
Proof of (1). If $\gamma$ is a divided power structure, then condition
(2) implies that $n \gamma_n(x) = \gamma_1(x)\gamma_{n - 1}(x)$. Hence
by induction and condition (1) we get $n! \gamma_n(x) = x^n$.

\medskip\noindent
Assume $A$ is torsion free as a $\mathbf{Z}$-module.
Proof of (2). This is clear from (1).

\medskip\noindent
Proof of (3). Assume that $n! \gamma_n(x) = x^n$ for all $x \in I$ and
$n \geq 1$. Since $A \subset A \otimes_{\mathbf{Z}} \mathbf{Q}$ it suffices
to prove (1) -- (5) in case $A$ is a $\mathbf{Q}$-algebra.
In this case $\gamma_n(x) = x^n/n!$ and it is straightforward
to verify (1) -- (5), for example (4) corresponds to the binomial
formula
$$
(x + y)^n = \sum \frac{n!}{i!(n - i)!} x^iy^{n - i}
$$
We encourage the reader to do the verifications
to make sure that we have the coefficients correct.

\medskip\noindent
Proof of (4). Assume we have generators $x_i$ of $I$ as an ideal
such that $x_i^n \in (n!)I$ for all $n \geq 1$. We claim that
for all $x \in I$ we have $x^n \in (n!)I$. If the claim holds then
we can set $\gamma_n(x) = x^n/n!$ which is a divided power structure by (3).
To prove the claim we note that it holds for $x = ax_i$. Hence we see
that the claim holds for a set of generators of $I$ as an abelian group.
By induction on the length of an expression in terms of these, it suffices
to prove the claim for $x + y$ if it holds for $x$ and $y$. This
follows immediately from the binomial theorem.
\end{proof}

\begin{example}
\label{example-ideal-generated-by-p}
Let $p$ be a prime number.
Let $A$ be a ring such that every integer $n$ not divisible by $p$
is invertible, i.e., $A$ is a $\mathbf{Z}_{(p)}$-algebra. Then
$I = pA$ has a canonical divided power structure. Namely, given
$x = pa \in A$ we set
$$
\gamma_n(x) = \frac{p^n}{n!} a^n
$$
The reader verifies immediately that $p^n/n! \in \mathbf{Z}_{(p)}$
so that the definition makes sense. It is a straightforward exercise to
verify that conditions (1) -- (5) of
Definition \ref{definition-divided-powers} are satisfied.
Alternatively, it is clear that the definition works for
$A_0 = \mathbf{Z}_{(p)}$ and then the result follows from
Lemma \ref{lemma-gamma-extends}.
\end{example}

\begin{lemma}
\label{lemma-check-on-generators}
Let $A$ be a ring. Let $I$ be an ideal of $A$. Let $\gamma_n : I \to I$,
$n \geq 1$ be a sequence of maps. Assume
\begin{enumerate}
\item[(a)] (1), (3), and (4) of Definition \ref{definition-divided-powers}
hold for all $x, y \in I$, and
\item[(b)] properties (2) and (5) hold for $x$ in
set of generators of $I$ as an ideal.
\end{enumerate}
Then $\gamma$ is a divided power structure on $I$.
\end{lemma}

\begin{proof}
The numbers (1), (2), (3), (4), (5) in this proof refer to the
conditions listed in Definition \ref{definition-divided-powers}.
Applying (3) we see that if (2) and (5) hold for $x$ then (2) and (5)
hold for $ax$ for all $a \in A$. Hence we see (b) implies
(2) and (5) hold for a set of generators
of $I$ as an abelian group. Hence, by induction of the length
of an expression in terms of these it suffices to prove that, given
$x, y \in I$ such that (2) and (5) hold for $x$ and $y$, then (2) and (5) hold
for $x + y$.

\medskip\noindent
Proof of (2) for $x + y$. By (4) we have
$$
\gamma_n(x + y)\gamma_m(x + y) =
\sum\nolimits_{i + j = n,\ k + l = m}
\gamma_i(x)\gamma_k(x)\gamma_j(y)\gamma_l(y)
$$
Using (2) for $x$ and $y$ this equals
$$
\sum \frac{(i + k)!}{i!k!}\frac{(j + l)!}{j!l!}
\gamma_{i + k}(x)\gamma_{j + l}(y)
$$
Comparing this with the expansion
$$
\gamma_{n + m}(x + y) = \sum \gamma_a(x)\gamma_b(y)
$$
we see that we have to prove that given $a + b = n + m$ we have
$$
\sum\nolimits_{i + k = a,\ j + l = b,\ i + j = n,\ k + l = m}
\frac{(i + k)!}{i!k!}\frac{(j + l)!}{j!l!}
=
\frac{(n + m)!}{n!m!}.
$$
Instead of arguing this directly, we note that the result is true
for the ideal $I = (x, y)$ in the polynomial ring $\mathbf{Q}[x, y]$
because $\gamma_n(f) = f^n/n!$, $f \in I$ defines a divided power
structure on $I$. Hence the equality of rational numbers above is true.

\medskip\noindent
Proof of (5) for $x + y$ given that (1) -- (4) hold and that (5)
holds for $x$ and $y$. We will again reduce the proof to an equality
of rational numbers. Namely, using (4) we can write
$\gamma_n(\gamma_m(x + y)) = \gamma_n(\sum \gamma_i(x)\gamma_j(y))$.
Using (4) we can write
$\gamma_n(\gamma_m(x + y))$ as a sum of terms which are products of
factors of the form $\gamma_k(\gamma_i(x)\gamma_j(y))$.
If $i > 0$ then
\begin{align*}
\gamma_k(\gamma_i(x)\gamma_j(y)) & =
\gamma_j(y)^k\gamma_k(\gamma_i(x)) \\
& = \frac{(ki)!}{k!(i!)^k} \gamma_j(y)^k \gamma_{ki}(x) \\
& =
\frac{(ki)!}{k!(i!)^k} \frac{(kj)!}{k!(j!)^k} \gamma_{ik}(x) \gamma_{kj}(y)
\end{align*}
using (3) in the first equality, (5) for $x$ in the second, and
(2) exactly $k$ times in the third. Using (5) for $y$ we see the
same equality holds when $i = 0$. Continuing like this using all
axioms but (5) we see that we can write
$$
\gamma_n(\gamma_m(x + y)) =
\sum\nolimits_{i + j = nm} c_{ij}\gamma_i(x)\gamma_j(y)
$$
for certain universal constants $c_{ij} \in \mathbf{Z}$. Again the fact
that the equality is valid in the polynomial ring $\mathbf{Q}[x, y]$
implies that the coefficients $c_{ij}$ are all equal to $(nm)!/n!(m!)^n$
as desired.
\end{proof}

\begin{lemma}
\label{lemma-two-ideals}
Let $A$ be a ring with two ideals $I, J \subset A$.
Let $\gamma$ be a divided power structure on $I$ and let
$\delta$ be a divided power structure on $J$.
Then
\begin{enumerate}
\item $\gamma$ and $\delta$ agree on $IJ$,
\item if $\gamma$ and $\delta$ agree on $I \cap J$ then they are
the restriction of a unique divided power structure $\epsilon$
on $I + J$.
\end{enumerate}
\end{lemma}

\begin{proof}
Let $x \in I$ and $y \in J$. Then
$$
\gamma_n(xy) = y^n\gamma_n(x) = n! \delta_n(y) \gamma_n(x) =
\delta_n(y) x^n = \delta_n(xy).
$$
Hence $\gamma$ and $\delta$ agree on a set of (additive) generators
of $IJ$. By property (4) of Definition \ref{definition-divided-powers}
it follows that they agree on all of $IJ$.

\medskip\noindent
Let $z \in I + J$. Write $z = x + y$ with $x \in I$ and $y \in J$.
Then we set
$$
\epsilon_n(z) = \sum \gamma_i(x)\delta_{n - i}(y)
$$
To see that this is well defined, suppose that $z = x' + y'$ is another
representation with $x' \in I$ and $y' \in J$. Then
$w = x - x' = y' - y \in I \cap J$. Hence
\begin{align*}
\sum\nolimits_{i + j = n} \gamma_i(x)\delta_j(y)
& =
\sum\nolimits_{i + j = n} \gamma_i(x' + w)\delta_j(y) \\
& =
\sum\nolimits_{i' + l + j = n} \gamma_{i'}(x')\gamma_l(w)\delta_j(y) \\
& =
\sum\nolimits_{i' + l + j = n} \gamma_{i'}(x')\delta_l(w)\delta_j(y) \\
& =
\sum\nolimits_{i' + j' = n} \gamma_{i'}(x')\delta_{j'}(y + w) \\
& =
\sum\nolimits_{i' + j' = n} \gamma_{i'}(x')\delta_{j'}(y')
\end{align*}
as desired. Next, we prove conditions (1) -- (5) of
Definition \ref{definition-divided-powers}.
Properties (1) and (3) are clear. To see (4), suppose
that $z = x + y$ and $z' = x' + y'$ with $x, x' \in I$ and $y, y' \in J$
and compute
\begin{align*}
\epsilon_n(z + z') & =
\sum\nolimits_{a + b = n} \gamma_i(x + x')\delta_j(y + y') \\
& =
\sum\nolimits_{i + i' + j + j' = n}
\gamma_i(x) \gamma_{i'}(x')\delta_j(y)\delta_{j'}(y') \\
& =
\epsilon_n(x + y)\epsilon_n(x' + y')
\end{align*}
as desired. Now we see that it suffices to prove (2) and (5) for
elements of $I$ or $J$, see Lemma \ref{lemma-check-on-generators}.
This is clear because $\gamma$ and $\delta$ are divided power
structures.
\end{proof}

\begin{lemma}
\label{lemma-nil}
Let $p$ be a prime number. Let $A$ be a ring, let $I \subset A$ be an ideal,
and let $\gamma$ be a divided power structure on $I$. Assume $p$ is nilpotent
in $A/I$. Then $I$ is locally nilpotent if and only if $p$ is nilpotent in $A$.
\end{lemma}

\begin{proof}
If $p^N = 0$ in $A$, then for $x \in I$ we have
$x^{pN} = (pN)!\gamma_N(x) = 0$ because $(pN)!$ is
divisible by $p^N$. Conversely, assume $I$ is locally nilpotent.
We've also assumed that $p$ is nilpotent in $A/I$, hence
$p^r \in I$ for some $r$, hence $p^r$ nilpotent, hence $p$ nilpotent.
\end{proof}

\begin{lemma}
\label{lemma-need-only-gamma-p}
Let $p$ be a prime number. Let $A$ be a ring such that every integer $n$
not divisible by $p$ is invertible, i.e., $A$ is a $\mathbf{Z}_{(p)}$-algebra.
Let $I \subset A$ be an ideal. Two divided power structures
$\gamma, \gamma'$ on $I$ are equal if and only if $\gamma_p = \gamma'_p$.
Moreover, given a map $\delta : I \to I$ such that
\begin{enumerate}
\item $p!\delta(x) = x^p$ for all $x \in I$,
\item $\delta(ax) = a^p\delta(x)$ for all $a \in A$, $x \in I$, and
\item
$\delta(x + y) =
\delta(x) +
\sum\nolimits_{i + j = p, i,j \geq 1} \frac{1}{i!j!} x^i y^j +
\delta(y)$ for all $x, y \in I$,
\end{enumerate}
then there exists a unique divided power structure $\gamma$ on $I$ such
that $\gamma_p = \delta$.
\end{lemma}

\begin{proof}
If $n$ is not divisible by $p$, then $\gamma_n(x) = c x \gamma_{n - 1}(x)$
where $c$ is a unit in $\mathbf{Z}_{(p)}$. Moreover,
$$
\gamma_{pm}(x) = c \gamma_m(\gamma_p(x))
$$
where $c$ is a unit in $\mathbf{Z}_{(p)}$. Thus the first assertion is clear.
For the second assertion, we can, working backwards, use these equalities
to define all $\gamma_n$. More precisely, if
$n = a_0 + a_1p + \ldots + a_e p^e$ with $a_i \in {0, \ldots, p - 1}$ then
we set
$$
\gamma_n(x) = c_n x^{a_0} \delta(x)^{a_1} \ldots \delta^e(x)^{a_e}
$$
for a suitable unit $c_n \in \mathbf{Z}_{(p)}$, namely
$$
c_n =
{(p!)^{a_1 + a_2(1 + p) + \ldots + a_e(1 + \ldots + p^{e - 1})}}/{n!}
$$
Then a long computation shows all the axioms are satisfied if $\delta$
satisfies properties (1), (2), (3). Details omitted.
\end{proof}








\section{Divided power rings}
\label{section-divided-power-rings}

\noindent
There is a category of divided power rings.
Here is the definition.

\begin{definition}
\label{definition-divided-power-ring}
A {\it divided power ring} is a triple $(A, I, \gamma)$ where
$A$ is a ring, $I \subset A$ is an ideal, and $\gamma = (\gamma_n)_{n \geq 1}$
is a divided power structure on $I$.
A {\it homomorphism of divided power rings}
$\varphi : (A, I, \gamma) \to (B, J, \delta)$ is a ring homomorphism
$\varphi : A \to B$ such that $\varphi(I) \subset J$ and such that
$\delta_n(\varphi(x)) = \varphi(\gamma_n(x))$ for all $x \in I$.
\end{definition}

\noindent
We sometimes say ``let $(B, J, \delta)$ be a divided power algebra over
$(A, I, \gamma)$'' to indicate that $(B, J, \delta)$ is a divided power ring
which comes equipped with a homomorphism of divided power rings
$(A, I, \gamma) \to (B, J, \delta)$.

\begin{lemma}
\label{lemma-limits}
The category of divided power rings has all limits and they agree with
limits in the category of rings.
\end{lemma}

\begin{proof}
The empty limit is the zero ring (that's weird but we need it).
The product of a collection of divided power rings $(A_t, I_t, \gamma_t)$,
$t \in T$ is given by $(\prod A_t, \prod I_t, \gamma)$ where
$\gamma_n((x_t)) = (\gamma_{t, n}(x_t))$.
The equalizer of $\alpha, \beta : (A, I, \gamma) \to (B, J, \delta)$
is just $C = \{a \in A \mid \alpha(a) = \beta(a)\}$ with ideal $C \cap I$
and induced divided powers. It follows that all limits exist, see
Categories, Lemma \ref{categories-lemma-limits-products-equalizers}.
\end{proof}

\noindent
The following lemma illustrates a very general category theoretic
phenomenon in the case of divided power algebras.

\begin{lemma}
\label{lemma-a-version-of-brown}
Let $\mathcal{C}$ be the category of divided power rings. Let
$F : \mathcal{C} \to \textit{Sets}$ be a functor.
Assume that
\begin{enumerate}
\item there exists a cardinal $\kappa$ such that for every
$f \in F(A, I, \gamma)$ there exists a morphism
$(A', I', \gamma') \to (A, I, \gamma)$ of $\mathcal{C}$ such that $f$
is the image of $f' \in F(A', I', \gamma')$ and $|A'| \leq \kappa$, and
\item $F$ commutes with limits.
\end{enumerate}
Then $F$ is representable, i.e., there exists an object $(B, J, \delta)$
of $\mathcal{C}$ such that
$$
F(A, I, \gamma) = \text{Hom}_\mathcal{C}((B, J, \delta), (A, I, \gamma))
$$
functorially in $(A, I, \gamma)$.
\end{lemma}

\begin{proof}
Consider a set of objects $\mathcal{U}$ of $\mathcal{C}$ containing an
object isomorphic to every $(A, I, \gamma)$ with $|A| \leq \kappa$.
Let $\mathcal{I}$ be the category of pairs $(U, f)$ where $U \in \mathcal{U}$
and $f \in F(U)$. A morphism $(U, f) \to (U', f')$ of $\mathcal{I}$
is a map $u : U \to U'$ of $\mathcal{C}$ such that $F(u)(f) = f'$.
Set
$$
(B, J, \delta) = \lim_{(U, f) \in \mathcal{I}} U.
$$
The limit exists by Lemma \ref{lemma-limits}. As $F$ commutes with limits
we have
$$
F(B, J, \delta) = \lim_{(U, f) \in \mathcal{I}} F(U).
$$
Hence there is a universal element $\xi \in F(B, J, \delta)$ which
for $U \in \mathcal{U}$ maps to $f \in F(U)$ under $F$ applied to the
projection map $(B, J, \delta) \to U$ of the limit corresponding to $f$.
Using $\xi$ we obtain a transformation of functors
$$
\xi : \text{Hom}_\mathcal{C}((B, J, \delta), - ) \longrightarrow F(-)
$$
see Categories, Section \ref{categories-section-opposite}.
Let $(A, I, \gamma)$ be an arbitrary object of $\mathcal{C}$ and
let $f \in F(A, I, \gamma)$. Choose $U \to (A, I, \gamma)$
with $U \in \mathcal{U}$ and $f' \in F(U)$ mapping to $f$ which is possible by
assumption (1). Then $F$ applied to the maps
$$
(B, J, \delta) \longrightarrow U \longrightarrow (A, I, \gamma)
$$
(the first being the projection map of the limit defining $B$)
sends $\xi$ to $f$. Hence the transformation $\xi$ is surjective.
Finally, suppose that $a, b : (B, J, \delta) \to (A, I, \gamma)$
are two maps such that $F(a)(\xi) = F(b)(\xi)$. Since $F$ commutes
with limits, it commutes with equalizers. This means that $\xi$
comes from an element $\xi' \in F(B', J', \delta')$ where $B' \subset B$
is the equalizer of $a$ and $b$.

\medskip\noindent
At this point there are two ways to finish the proof.
The first is to show that $B' = B$ using compatibility of $F$
with equalizers and the construction of $B$ as a limit over $\mathcal{I}$
above; we omit the details. The second is to replace $B$ by the
smallest divided power subring $(B', J', \delta') \subset (B, J, \delta)$
such that $\xi$ comes from an element $\xi' \in F(B', J', \delta')$.
Since $F$ commutes with limits $F$ commutes with intersections
hence a smallest divided power subring exists. It is clear that the
transformation defined by $\xi'$ is still surjective, and the argument above
shows that it is also injective.
\end{proof}

\begin{lemma}
\label{lemma-colimits}
The category of divided power rings has all colimits.
\end{lemma}

\begin{proof}
The empty colimit is $\mathbf{Z}$ with divided power ideal $(0)$.
Let's discuss general colimits. Let $\mathcal{C}$ be a category and let
$c \mapsto (A_c, I_c, \gamma_c)$ be a diagram. Consider the functor
$$
F(B, J, \delta) = \lim_{c \in \mathcal{C}}
Hom((A_c, I_c, \gamma_c), (B, J, \delta))
$$
Note that any $f = (f_c)_{c \in C} \in F(B, J, \delta)$ has the property
that all the images $f_c(A_c)$ generate a subring $B'$ of $B$ of bounded
cardinality $\kappa$ and that all the images $f_c(I_c)$ generate a
divided power sub ideal $J'$ of $B'$. And we get a factorization of
$f$ as a $f'$ in $F(B')$ followed by the inclusion $B' \to B$. Also,
$F$ commutes with limits. Hence we may apply
Lemma \ref{lemma-a-version-of-brown}
to see that $F$ is representable and we win.
\end{proof}

\begin{remark}
\label{remark-forgetful}
The forgetful functor $(A, I, \gamma) \mapsto A$ does not commute with
colimits. For example, let
$$
\xymatrix{
(B, J, \delta) \ar[r] & (B'', J'', \delta'') \\
(A, I, \gamma) \ar[r] \ar[u] & (B', J', \delta') \ar[u]
}
$$
be a push out in the category of divided power rings.
Then in general the map $B \otimes_A B' \to B''$ isn't an
isomorphism. (It is always surjective.)
An explicit example is given by
$(A, I, \gamma) = (\mathbf{Z}, (0), \emptyset)$,
$(B, J, \delta) = (\mathbf{Z}/4\mathbf{Z}, 2\mathbf{Z}/4\mathbf{Z}, \delta)$,
and
$(B', J', \delta') =
(\mathbf{Z}/4\mathbf{Z}, 2\mathbf{Z}/4\mathbf{Z}, \delta')$
where $\delta_2(2) = 2$ and $\delta'_2(2) = 0$ and all higher divided powers
equal to zero. Then $(B'', J'', \delta'') = (\mathbf{F}_2, (0), \emptyset)$
which doesn't agree with the tensor product. However, note that it is always
true that
$$
B''/J'' = B/J \otimes_{A/I} B'/J'
$$
as can be seen from the universal property of the push out by considering
maps into divided power algebras of the form $(C, (0), \emptyset)$.
\end{remark}


\section{Extending divided powers}
\label{section-extend}

\noindent
Here is the definition.

\begin{definition}
\label{definition-extends}
Given a divided power ring $(A, I, \gamma)$ and a ring map
$A \to B$ we say $\gamma$ {\it extends} to $B$ if there exists a
divided power structure $\bar \gamma$ on $IB$ such that
$(A, I, \gamma) \to (B, IB, \bar\gamma)$ is a homomorphism of
divided power rings.
\end{definition}

\begin{lemma}
\label{lemma-gamma-extends}
Let $(A, I, \gamma)$ be a divided power ring.
Let $A \to B$ be a ring map.
If $\gamma$ extends to $B$ then it extends uniquely.
Assume (at least) one of the following conditions holds
\begin{enumerate}
\item $IB = 0$,
\item $I$ is principal, or
\item $A \to B$ is flat.
\end{enumerate}
Then $\gamma$ extends to $B$.
\end{lemma}

\begin{proof}
Any element of $IB$ can be written as a finite sum $\sum b_ix_i$ with
$b_i \in B$ and $x_i \in I$. If $\gamma$ extends to $\bar\gamma$ on $IB$
then $\bar\gamma_n(x_i) = \gamma_n(x_i)$.
Thus conditions (3) and (4) imply that
$$
\bar\gamma_n(\sum b_ix_i) =
\sum\nolimits_{n_1 + \ldots + n_t = n}
\prod\nolimits_{i = 1}^t b_i^{n_i}\gamma_{n_i}(x_i)
$$
Thus we see that $\bar\gamma$ is unique if it exists.

\medskip\noindent
If $IB = 0$ then setting $\bar\gamma_n(0) = 0$ works. If $I = (x)$
then we define $\bar\gamma_n(bx) = b^n\gamma_n(x)$. This is well defined:
if $b'x = bx$, i.e., $(b - b')x = 0$ then
\begin{align*}
b^n\gamma_n(x) - (b')^n\gamma_n(x)
& =
(b^n - (b')^n)\gamma_n(x) \\
& =
(b^{n - 1} + \ldots + (b')^{n - 1})(b - b')\gamma_n(x) = 0
\end{align*}
because $\gamma_n(x)$ is divisible by $x$ and hence annihilated by $b - b'$.
Next, we prove conditions (1) -- (5) of
Definition \ref{definition-divided-powers}.
Parts (1), (2), (3), (5) are obvious from the construction.
For (4) suppose that $y, z \in IB$, say $y = bx$ and $z = cx$. Then
$y + z = (b + c)x$ hence
\begin{align*}
\bar\gamma_n(y + z)
& =
(b + c)^n\gamma_n(x) \\
& =
\sum \frac{n!}{i!(n - i)!}b^ic^{n -i}\gamma_n(x) \\
& =
\sum b^ic^{n - i}\gamma_i(x)\gamma_{n - i}(x) \\
& =
\sum \bar\gamma_i(y)\bar\gamma_{n -i}(z)
\end{align*}
as desired.

\medskip\noindent
Assume $A \to B$ is flat. Suppose that $b_1, \ldots, b_r \in B$ and
$x_1, \ldots, x_r \in I$. Then
$$
\bar\gamma_n(\sum b_ix_i) =
\sum b_1^{e_1} \ldots b_r^{e_r} \gamma_{e_1}(x_1) \ldots \gamma_{e_r}(x_r)
$$
where the sum is over $e_1 + \ldots + e_r = n$
if $\bar\gamma_n$ exists. Next suppose that we have $c_1, \ldots, c_s \in B$
and $a_{ij} \in A$ such that $b_i = \sum a_{ij}c_j$.
Setting $y_j = \sum a_{ij}x_i$ we claim that
$$
\sum b_1^{e_1} \ldots b_r^{e_r} \gamma_{e_1}(x_1) \ldots \gamma_{e_r}(x_r) =
\sum c_1^{d_1} \ldots c_s^{d_s} \gamma_{d_1}(y_1) \ldots \gamma_{d_s}(y_s)
$$
in $B$ where on the right hand side we are summing over
$d_1 + \ldots + d_s = n$. Namely, using the axioms of a divided power
structure we can expand both sides into a sum with coefficients
in $\mathbf{Z}[a_{ij}]$ of terms of the form 
$c_1^{d_1} \ldots c_s^{d_s}\gamma_{e_1}(x_1) \ldots \gamma_{e_r}(x_r)$.
To see that the coefficients agree we note that the result is true
in $\mathbf{Q}[x_1, \ldots, x_r, c_1, \ldots, c_s, a_{ij}]$ with
$\gamma$ the unique divided power structure on $(x_1, \ldots, x_r)$.
By Lazard's theorem (Algebra, Theorem \ref{algebra-theorem-lazard})
we can write $B$ as a directed colimit of finite free $A$-modules.
In particular, if $z \in IB$ is written as $z = \sum x_ib_i$ and
$z = \sum x'_{i'}b'_{i'}$, then we can find $c_1, \ldots, c_s \in B$
and $a_{ij}, a'_{i'j} \in A$ such that $b_i = \sum a_{ij}c_j$
and $b'_{i'} = \sum a'_{i'j}c_j$ such that
$y_j = \sum x_ia_{ij} = \sum x'_{i'}a'_{i'j}$.
Hence the procedure above gives a well defined map $\bar\gamma_n$
on $IB$. By construction $\bar\gamma$ satisfies conditions (1), (3), and
(4). Moreover, for $x \in I$ we have $\bar\gamma_n(x) = \gamma_n(x)$. Hence
it follows from Lemma \ref{lemma-check-on-generators} that $\bar\gamma$
is a divided power structure on $IB$.
\end{proof}

\begin{lemma}
\label{lemma-kernel}
Let $(A, I, \gamma)$ be a divided power ring.
\begin{enumerate}
\item If $\varphi : (A, I, \gamma) \to (B, J, \delta)$ is a
homomorphism of divided power rings, then $\text{Ker}(\varphi) \cap I$
is preserved by $\gamma_n$ for all $n \geq 1$.
\item Let $\mathfrak a \subset A$ be an ideal and set
$I' = I \cap \mathfrak a$. The following are equivalent
\begin{enumerate}
\item $I'$ is preserved by $\gamma_n$ for all $n > 0$,
\item $\gamma$ extends to $A/\mathfrak a$, and
\item there exist a set of generators $x_i$ of $I'$ as an ideal
such that $\gamma_n(x_i) \in I'$ for all $n > 0$.
\end{enumerate}
\end{enumerate}
\end{lemma}

\begin{proof}
Proof of (1). This is clear. Assume (2)(a). Define
$\bar\gamma_n(x \bmod I') = \gamma_n(x) \bmod I'$ for $x \in I$.
This is well defined since $\gamma_n(x + y) = \gamma_n(x) \bmod I'$
for $y \in I'$ by Definition \ref{definition-divided-powers} (4) and
the fact that $\gamma_j(y) \in I'$ by assumption. It is clear that
$\bar\gamma$ is a divided power structure as $\gamma$ is one.
Hence (2)(b) holds. Also, (2)(b) implies (2)(a) by part (1).
It is clear that (2)(a) implies (2)(c). Assume (2)(c).
Note that $\gamma_n(x) = a^n\gamma_n(x_i) \in I'$ for $x = ax_i$.
Hence we see that $\gamma_n(x) \in I'$ for a set of generators of $I'$
as an abelian group. By induction on the length of an expression in
terms of these, it suffices to prove $\forall n : \gamma_n(x + y) \in I'$
if $\forall n : \gamma_n(x), \gamma_n(y) \in I'$. This
follows immediately from the fourth axiom of a divided power structure.
\end{proof}

\begin{lemma}
\label{lemma-sub-dp-ideal}
Let $(A, I, \gamma)$ be a divided power ring.
Let $E \subset I$ be a subset.
Then the smallest ideal $J \subset I$ preserved by $\gamma$
and containing all $f \in E$ is the ideal $J$ generated by
$\gamma_n(f)$, $n \geq 1$, $f \in E$.
\end{lemma}

\begin{proof}
Follows immediately from Lemma \ref{lemma-kernel}.
\end{proof}






\section{Divided power polynomial algebras}
\label{section-divided-power-polynomial-ring}

\noindent
A very useful example is the {\it divided power polynomial algebra}.
Let $A$ be a ring. Let $t \geq 1$. We will denote
$A\langle x_1, \ldots, x_t \rangle$ the following $A$-algebra:
As an $A$-module we set
$$
A\langle x_1, \ldots, x_t \rangle =
\bigoplus\nolimits_{n_1, \ldots, n_t \geq 0} A x_1^{[n_1]} \ldots x_t^{[n_t]}
$$
with multiplication given by
$$
x_i^{[n]}x_i^{[m]} = \frac{(n + m)!}{n!m!}x_i^{[n + m]}.
$$
We also set $x_i = x_i^{[1]}$. Note that
$1 = x_1^{[0]} \ldots x_t^{[0]}$. There is a similar construction
which gives the divided power polynomial algebra in infinitely many
variables. There is an canonical $A$-algebra map
$A\langle x_1, \ldots, x_t \rangle \to A$ sending $x_i^{[n]}$ to zero
for $n > 0$. The kernel of this map is denoted
$A\langle x_1, \ldots, x_t \rangle_{+}$.

\begin{lemma}
\label{lemma-divided-power-polynomial-algebra}
Let $(A, I, \gamma)$ be a divided power ring.
There exists a unique divided power structure $\delta$ on
$$
J = IA\langle x_1, \ldots, x_t \rangle + A\langle x_1, \ldots, x_t \rangle_{+}
$$
such that
\begin{enumerate}
\item $\delta_n(x_i) = x_i^{[n]}$, and
\item $(A, I, \gamma) \to (A\langle x_1, \ldots, x_t \rangle, J, \delta)$
is a homomorphism of divided power rings.
\end{enumerate}
Moreover, $(A\langle x_1, \ldots, x_t \rangle, J, \delta)$ has the
following universal property: A homomorphism of divided power rings
$\varphi : (A\langle x \rangle, J, \delta) \to (C, K, \epsilon)$ is
the same thing as a homomorphism of divided power rings
$A \to C$ and elements $k_1, \ldots, k_t \in K$.
\end{lemma}

\begin{proof}
We will prove the lemma in case of a divided power polynomial algebra
in one variable. The result for the general case can be argued in exactly
the same way, or by noting that $A\langle x_1, \ldots, x_t\rangle$ is
isomorphic to the ring obtained by adjoining the divided power variables
$x_1, \ldots, x_t$ one by one.

\medskip\noindent
Let $A\langle x \rangle_{+}$ be the ideal generated by
$x, x^{[2]}, x^{[3]}, \ldots$.
Note that $J = IA\langle x \rangle + A\langle x \rangle_{+}$
and that
$$
IA\langle x \rangle \cap A\langle x \rangle_{+} =
IA\langle x \rangle \cdot A\langle x \rangle_{+}
$$
Hence by Lemma \ref{lemma-two-ideals} it suffices to show that there
exist divided power structures on the ideals $IA\langle x \rangle$ and
$A\langle x \rangle_{+}$. The existence of the first follows from
Lemma \ref{lemma-gamma-extends} as $A \to A\langle x \rangle$ is flat.
For the second, note that if $A$ is torsion free, then we can apply
Lemma \ref{lemma-silly} (4) to see that $\delta$ exists. Namely, choosing
as generators the elements $x^{[m]}$ we see that
$(x^{[m]})^n = \frac{(nm)!}{(m!)^n} x^{[nm]}$
and $n!$ divides the integer $\frac{(nm)!}{(m!)^n}$.
In general write $A = R/\mathfrak a$ for some torsion free ring $R$
(e.g., a polynomial ring over $\mathbf{Z}$). The kernel of
$R\langle x \rangle \to A\langle x \rangle$ is
$\bigoplus \mathfrak a x^{[m]}$. Applying criterion (2)(c) of
Lemma \ref{lemma-kernel} we see that the divided power structure
on $R\langle x \rangle_{+}$ extends to $A\langle x \rangle$ as
desired.

\medskip\noindent
Proof of the universal property. Given a homomorphism $\varphi : A \to C$
of divided power rings and $k_1, \ldots, k_t \in K$ we consider
$$
A\langle x_1, \ldots, x_t \rangle \to C,\quad
x_1^{[n_1]} \ldots x_t^{[n_t]} \longmapsto
\epsilon_{n_1}(k_1) \ldots \epsilon_{n_t}(k_t)
$$
using $\varphi$ on coefficients. The only thing to check is that
this is an $A$-algebra homomorphism (details omitted). The inverse
construction is clear.
\end{proof}

\begin{remark}
\label{remark-divided-power-polynomial-algebra}
Let $(A, I, \gamma)$ be a divided power ring.
There is a variant of Lemma \ref{lemma-divided-power-polynomial-algebra}
for infinitely many variables. First note that if $s < t$ then there
is a canonical map
$$
A\langle x_1, \ldots, x_s \rangle \to A\langle x_1, \ldots, x_t\rangle
$$
Hence if $W$ is any set, then we set
$$
A\langle x_w, w \in W\rangle =
\colim_{E \subset W} A\langle x_e, e \in E\rangle
$$
(colimit over $E$ finite subset of $W$)
with transition maps as above. By the definition of a colimit we see
that the universal mapping property of $A\langle x_w, w \in W\rangle$ is
completely analogous to the mapping property stated in
Lemma \ref{lemma-divided-power-polynomial-algebra}.
\end{remark}




\section{Divided power envelope}
\label{section-divided-power-envelope}

\noindent
The construction of the following lemma will be dubbed the
divided power envelope. It will play an important role later.

\begin{lemma}
\label{lemma-divided-power-envelope}
Let $(A, I, \gamma)$ be a divided power ring.
Let $A \to B$ be a ring map. Let $J \subset B$ be an ideal
with $IB \subset J$. There exists a homomorphism of
divided power rings
$$
(A, I, \gamma) \longrightarrow (D, \bar J, \bar \gamma)
$$
such that
$$
\Hom_{(A, I, \gamma)}((D, \bar J, \bar \gamma), (C, K, \delta))
=
\Hom_A((B, J), (C, K))
$$
functorially in the divided power algebra $(C, K, \delta)$ over
$(A, I, \gamma)$.
\end{lemma}

\begin{proof}
Denote $\mathcal{C}$ the category of divided power rings
$(C, K, \delta)$. Consider the functor
$F : \mathcal{C} \longrightarrow \textit{Sets}$ defined by
$$
F(C, K, \delta) =
\left\{
(\varphi, \psi)
\Big |
\begin{matrix}
\varphi : (A, I, \gamma) \to (C, K, \delta)
\text{ homomorphism of divided power rings} \\
\psi : (B, J) \to (C, K)\text{ an }
A\text{-algebra homomorphism with }\psi(J) \subset K
\end{matrix}
\right\}
$$
We will show that Lemma \ref{lemma-a-version-of-brown}
applies to this functor which will
prove the lemma. Suppose that $(\varphi, \psi) \in F(C, K, \delta)$.
Let $C' \subset C$ be the subring generated by $\varphi(A)$,
$\psi(B)$, and $\delta_n(\psi(f))$ for all $f \in J$.
Let $K' \subset K \cap C'$ be the ideal of $C'$ generated by
$\varphi(I)$ and $\delta_n(\psi(f))$ for $f \in J$.
Then $(C', K', \delta|_{K'})$ is a divided power ring and
$C'$ has cardinality bounded by the cardinal
$\kappa = |A| \otimes |B|^{\aleph_0}$.
Moreover, $\varphi$ factors as $A \to C' \to C$ and $\psi$ factors
as $B \to B' \to B$. This proves assumption (1) of
Lemma \ref{lemma-a-version-of-brown} holds. Assumption (2) is clear
as limits in the category of divided power rings commute with the
forgetful functor $(C, K, \delta) \mapsto (C, K)$, see
Lemma \ref{lemma-limits} and its proof.
\end{proof}

\begin{definition}
\label{definition-divided-power-envelope}
Let $(A, I, \gamma)$ be a divided power ring.
Let $A \to B$ be a ring map. Let $J \subset B$ be an ideal
with $IB \subset J$. The divided power algebra $(D, \bar J, \bar\gamma)$
constructed in Lemma \ref{lemma-divided-power-envelope}
is called the {\it divided power envelope of $J$ in $B$
relative to $(A, I, \gamma)$} and is denoted $D_B(J)$ or $D_{B, \gamma}(J)$.
\end{definition}

\noindent
Let $(A, I, \gamma) \to (C, K, \delta)$ be a homomorphism of divided
power rings. The universal property of
$D_{B, \gamma}(J) = (D, \bar J, \bar \gamma)$ is
$$
\begin{matrix}
\text{ring maps }B \to C \\
\text{ which map }J\text{ into }K
\end{matrix}
\longleftrightarrow
\begin{matrix}
\text{divided power homomorphisms} \\
(D, \bar J, \bar \gamma) \to (C, K, \delta)
\end{matrix}
$$
and the correspondence is given by precomposing with the map $B \to D$
which corresponds to $\text{id}_D$. Here are some properties of
$(D, \bar J, \bar \gamma)$ which follow directly from the universal
property. There are $A$-algebra maps
\begin{equation}
\label{equation-divided-power-envelope}
B \longrightarrow D \longrightarrow B/J
\end{equation}
The first arrow maps $J$ into $\bar J$ and $\bar J$ is the kernel
of the second arrow. The elements $\bar\gamma_n(x)$ where $n > 0$
and $x$ is an element in the image of $J \to D$ generate $\bar J$
as an ideal in $D$ and generate $D$ as a $B$-algebra.

\begin{lemma}
\label{lemma-divided-power-envelop-quotient}
Let $(A, I, \gamma)$ be a divided power ring.
Let $\varphi : P \to B$ be a surjection of $A$-algebras with kernel $K$.
Let $IB \subset J \subset B$ be an ideal. Let $J' \subset P$
be the inverse image of $J$. Write
$D_{B', \gamma}(J') = (D', \bar J', \bar\gamma)$.
Then $D_{B, \gamma}(J) = (D'/K', \bar J'/K', \bar\gamma)$
where $K'$ is the ideal generated by the elements $\bar\gamma_n(k)$
for $n \geq 1$ and $k \in K$.
\end{lemma}

\begin{proof}
Write $D_{B, \gamma}(J) = (D, \bar J, \bar \gamma)$.
The universal property of $D'$ gives us a homomorphism $D' \to D$
of divided power algebras. As $B' \to B$ and $J' \to J$ are surjective, we
see that $D' \to D$ is surjective (see remarks above). It is clear that
$\bar\gamma_n(k)$ is in the kernel for $n \geq 1$ and $k \in K$, i.e.,
we obtain a homomorphism $D'/K' \to D$. Conversely, there exists a divided
power structure on $\bar J'/K' \subset D'/K'$, see
Lemma \ref{lemma-kernel}. Hence the universal property of $D$ gives an inverse
$D \to D'/K'$ and we win.
\end{proof}

\noindent
In the situation of Definition \ref{definition-divided-power-envelope}
we can choose a surjection $P \to B$ where $P$ is a polynomial
algebra over $A$ and let $J' \subset P$ be the inverse image of $J$.
The previous lemma describes $D_{B, \gamma}(J)$ in terms of
$D_{P, \gamma}(J')$. Note that $\gamma$ extends to a divided power
structure $\gamma'$ on $IP$ by Lemma \ref{lemma-gamma-extends}. Hence
$D_{P, \gamma}(J') = D_{P, \gamma'}(J')$ is an example of a special
case of divided power envelopes we describe in the following lemma.

\begin{lemma}
\label{lemma-describe-divided-power-envelope}
Let $(B, I, \gamma)$ be a divided power algebra. Let $I \subset J \subset B$
be an ideal. Let $(D, \bar J, \bar \gamma)$ be the divided power envelope
of $J$ relative to $\gamma$. Choose elements $f_t \in J$, $t \in T$ such
that $J = I + (f_t)$. Then there exists a surjection
$$
\Psi : B\langle x_t \rangle \longrightarrow D
$$
of divided power rings mapping $x_t$ to the image of $f_t$ in $D$.
The kernel of $\Psi$ is generated by the elements $x_t - f_t$ and
all
$$
\delta_n\left(\sum r_t x_t - r_0\right)
$$
whenever $\sum r_t f_t = r_0$ in $B$ for some $r_t \in B$, $r_0 \in I$.
\end{lemma}

\begin{proof}
In the statement of the lemma we think of $B\langle x_t \rangle$
as a divided power ring with ideal
$J' = IB\langle x_t \rangle + B\langle x_t \rangle_{+}$, see
Remark \ref{remark-divided-power-polynomial-algebra}.
The existence of $\Psi$ follows from the universal property of
divided power polynomial rings. Surjectivity of $\Psi$ follows from
the fact that its image is a divided power subring of $D$, hence equal to $D$
by the universal property of $D$. It is clear that
$x_t - f_t$ is in the kernel. Set
$$
\mathcal{R} = \{(r_0, r_t) \in I \oplus \bigoplus\nolimits_{t \in T} B
\mid \sum r_t f_t = r_0 \text{ in }B\}
$$
If $(r_0, r_t) \in \mathcal{R}$ then it is clear that
$\sum r_t x_t - r_0$ is in the kernel.
As $\Psi$ is a homomorphism of divided power rings
and $\sum r_tx_t = r_0 \in J'$
it follows that $\delta_n(\sum r_t x_t - r_0)$ is in the kernel as well.
Let $K \subset B\langle x_t \rangle$ be the ideal generated by
$x_t - f_t$ and the elements $\delta_n(\sum r_t x_t - r_0)$ for
$(r_0, r_t) \in \mathcal{R}$.
To show that $K = \text{Ker}(\Psi)$ it suffices to show that
$\delta$ extends to $B\langle x_t \rangle/K$. Namely, if so the universal
property of $D$ gives a map $D \to B\langle x_t \rangle/K$
inverse to $\Psi$. Hence we have to show that $K \cap J'$ is
preserved by $\delta_n$, see Lemma \ref{lemma-kernel}.
Let $K' \subset B\langle x_t \rangle$ be the ideal
generated by the elements
\begin{enumerate}
\item $\delta_m(\sum r_t x_t - r_0)$ where $m > 0$ and
$(r_0, r_t) \in \mathcal{R}$,
\item $x_{t'}^{[m]}(x_t - f_t)$ where $m > 0$ and $t', t \in I$.
\end{enumerate}
We claim that $K' = K \cap J'$. The claim proves that $K \cap J'$
is preserved by $\delta_n$, $n > 0$ by the criterion of
Lemma \ref{lemma-kernel} (2)(c) and a computation of $\delta_n$
of the elements listed which we leave to the reader.
To prove the claim note that $K' \subset K \cap J'$.
Conversely, if $h \in K \cap J'$ then, modulo $K'$ we can write
$$
h = \sum r_t (x_t - f_t)
$$
for some $r_t \in B$. As $h \in K \cap J' \subset J'$
we see that $r_0 = \sum r_t f_t \in I$. Hence $(r_0, r_t) \in \mathcal{R}$
and we see that
$$
h = \sum r_t x_t - r_0
$$
is in $K'$ as desired.
\end{proof}

\noindent
Conditions (1) and (2) of the following lemma hold if $B \to B'$ is flat
at all primes of $V(IB') \subset \Spec(B')$ and is very closely related
to that condition, see
Algebra, Lemma \ref{algebra-lemma-what-does-it-mean}.
It in particular says that taking the divided power
envelope commutes with localization.

\begin{lemma}
\label{lemma-flat-base-change-divided-power-envelope}
Let $(A, I, \gamma)$ be a divided power ring.
Let $B \to B'$ be a homomorphism of $A$-algebras.
Assume that
\begin{enumerate}
\item $B/IB \to B'/IB'$ is flat, and
\item $\text{Tor}_1^B(B', B/IB) = 0$.
\end{enumerate}
Then for any ideal $IB \subset J \subset B$ the canonical map
$$
D_B(J) \otimes_B B' \longrightarrow D_{B'}(JB')
$$
is an isomorphism.
\end{lemma}

\begin{proof}
Set $D = D_B(J)$ and denote $\bar J \subset D$ its divided power ideal
with divided power structure $\bar\gamma$. The universal property of
$D$ produces a $B$-algebra map $D \to D_{B'}(JB')$, whence a map as in
the lemma. It suffices to show that
the divided powers $\bar\gamma$ extend to $D \otimes_B B'$ since then
the universal property of $D_{B'}(JB')$ will produce a map
$D_{B'}(JB') \to D \otimes_B B'$ inverse to the one in the lemma.

\medskip\noindent
Choose a surjection $P \to B'$ where $P$ is a polynomial algebra over $B$.
In particular $B \to P$ is flat, hence $D \to D \otimes_B P$ is flat by
Algebra, Lemma \ref{algebra-lemma-flat-base-change}.
Then $\bar\gamma$ extends to $D \otimes_B P$ by
Lemma \ref{lemma-gamma-extends}; we will denote this extension
$\bar\gamma$ also. Set $\mathfrak a = \text{Ker}(P \to B')$ so that
we have the short exact sequence
$$
0 \to \mathfrak a \to P \to B' \to 0
$$
Thus $\text{Tor}_1^B(B', B/IB) = 0$ implies that
$\mathfrak a \cap IP = I\mathfrak a$.
Now we have the following commutative diagram
$$
\xymatrix{
B/J \otimes_B \mathfrak a \ar[r]_\beta &
B/J \otimes_B P \ar[r] &
B/J \otimes_B B' \\
D \otimes_B \mathfrak a \ar[r]^\alpha \ar[u] &
D \otimes_B P \ar[r] \ar[u] &
D \otimes_B B' \ar[u] \\
\bar J \otimes_B \mathfrak a \ar[r] \ar[u] &
\bar J \otimes_B P \ar[r] \ar[u] &
\bar J \otimes_B B' \ar[u]
}
$$
This diagram is exact even with $0$'s added at the top and the right.
We have to show the divided powers on the ideal
$\bar J \otimes_B P$ preserve the ideal
$\text{Im}(\alpha) \cap \bar J \otimes_B P$, see
Lemma \ref{lemma-kernel}. Consider the exact sequence
$$
0 \to \mathfrak a/I\mathfrak a \to P/IP \to B'/IB' \to 0
$$
(which uses that $\mathfrak a \cap IP = I\mathfrak a$ as seen above).
As $B'/IB'$ is flat over $B/IB$ this sequence remains exact after
applying $B/J \otimes_{B/IB} -$, see
Algebra, Lemma \ref{algebra-lemma-flat-tor-zero}. Hence
$$
\text{Ker}(B/J \otimes_{B/IB} \mathfrak a/I\mathfrak a \to
B/J \otimes_{B/IB} P/IP) =
\text{Ker}(\mathfrak a/J\mathfrak a \to P/JP)
$$
is zero. Thus $\beta$ is injective. It follows that
$\text{Im}(\alpha) \cap \bar J \otimes_B P$ is the
image of $\bar J \otimes \mathfrak a$. Now if
$f \in \bar J$ and $a \in \mathfrak a$, then
$\bar\gamma_n(f \otimes a) = \bar\gamma_n(f) \otimes a^n$
hence the result is clear.
\end{proof}

\noindent
The following lemma is a special case of
\cite[Proposition 2.1.7]{dJ-crystalline} which in turn is a
generalization of \cite[Proposition 2.8.2]{Berthelot}.

\begin{lemma}
\label{lemma-flat-extension-divided-power-envelope}
Let $(B, I, \gamma) \to (B', I', \gamma')$ be a homomorphism of
divided power rings. Let $I \subset J \subset B$ and
$I' \subset J' \subset B'$ be ideals. Assume
\begin{enumerate}
\item $B/I \to B'/I'$ is flat, and
\item $J' = JB' + I'$.
\end{enumerate}
Then the canonical map
$$
D_{B, \gamma}(J) \otimes_B B' \longrightarrow D_{B', \gamma'}(J')
$$
is an isomorphism.
\end{lemma}

\begin{proof}
Set $D = D_B(J)$ and denote $\bar J \subset D$ its divided power ideal
with divided power structure $\bar\gamma$. The universal property of
$D$ produces a homomorphism of divided power rings $D \to D_{B'}(J')$,
whence a map as in the lemma. It suffices to show that
there exist divided powers on the image of
$D \otimes_B I' + \bar J \otimes_B B' \to  D \otimes_B B'$
compatible with $\bar \gamma$ and $\gamma'$ since then
the universal property of $D_{B'}(J')$ will produce a map
$D_{B'}(J') \to D \otimes_B B'$ inverse to the one in the lemma.

\medskip\noindent
Choose elements $f_t \in J$ which generate $J/I$. Set
$\mathcal{R} = \{(r_0, r_t) \in I \oplus \bigoplus\nolimits_{t \in T} B
\mid \sum r_t f_t = r_0 \text{ in }B\}$ as in the proof of
Lemma \ref{lemma-describe-divided-power-envelope}. This lemma shows that
$$
D = B\langle x_t \rangle/ K
$$
where $K$ is generated by the elements $x_t - f_t$ and
$\delta_n(\sum r_t x_t - r_0)$ for $(r_0, r_t) \in \mathcal{R}$.
Thus we see that
\begin{equation}
\label{equation-base-change}
D \otimes_B B' = B'\langle x_t \rangle/K'
\end{equation}
where $K'$ is generated by the images in $B'\langle x_t \rangle$
of the generators of $K$ listed above. Let $f'_t \in B'$ be the image
of $f_t$. By assumption (1) we see that the elements $f'_t \in J'$
generate $J'/I'$ and we see that $x_t - f'_t \in K'$. Set
$$
\mathcal{R}' =
\{(r'_0, r'_t) \in I' \oplus \bigoplus\nolimits_{t \in T} B'
\mid \sum r'_t f'_t = r'_0 \text{ in }B'\}
$$
To finish the proof we have to show that
$\delta'_n(\sum r'_t x_t - r'_0) \in K'$ for
$(r'_0, r'_t) \in \mathcal{R}'$, because then the presentation
(\ref{equation-base-change}) of $D \otimes_B B'$ is identical
to the presentation of $D_{B', \gamma'}(J')$ obtain in
Lemma \ref{lemma-describe-divided-power-envelope} from the generators $f'_t$.
Suppose that $(r'_0, r'_t) \in \mathcal{R}'$. Then
$\sum r'_t f'_t = 0$ in $B'/I'$. As $B/I \to B'/I'$ is flat by
assumption (1) we can apply the equational criterion of flatness
(Algebra, Lemma \ref{algebra-lemma-flat-eq}) to see
that there exist an $m > 0$ and
$r_{jt} \in B$ and $c_j \in B'$, $j = 1, \ldots, m$ such
that
$$
r_{j0} = \sum r_{jt} f_t  \in I \text{ for } j = 1, \ldots, m,
\quad\text{and}\quad
r'_t = \sum c_j r_{jt}.
$$
Note that this also implies that $r'_0 = \sum c_j r_{j0}$.
Then we have
\begin{align*}
\delta'_n(\sum r'_t x_t  - r'_0)
& =
\delta'_n(\sum c_j (\sum r_{jt} x_t  - r_{j0})) \\
& =
\sum c_1^{n_1} \ldots c_m^{n_m}
\delta_{n_1}(\sum r_{1t} x_t  - r_{10}) \ldots
\delta_{n_m}(\sum r_{mt} x_t  - r_{m0})
\end{align*}
where the sum is over $n_1 + \ldots + n_m = n$. This proves what we want.
\end{proof}





\section{Some explicit divided power thickenings}
\label{section-explicit-thickenings}

\noindent
The constructions in this section will help us to define the connection
on a crystal in modules on the crystalline site.

\begin{lemma}
\label{lemma-divided-power-first-order-thickening}
Let $(A, I, \gamma)$ be a divided power ring. Let $M$ be an $A$-module.
Let $B = A \oplus M$ as an $A$-algebra where $M$ is an ideal of square zero
and set $J = I \oplus M$. Set
$$
\delta_n(x + z) = \gamma_n(x) + \gamma_{n - 1}(x)z
$$
for $x \in I$ and $z \in M$.
Then $\delta$ is a divided power structure and
$A \to B$ is a homomorphism of divided power rings from
$(A, I, \gamma)$ to $(B, J, \delta)$.
\end{lemma}

\begin{proof}
We have to check conditions (1) -- (5) of
Definition \ref{definition-divided-powers}.
We will prove this directly for this case, but please see the proof of
the next lemma for a method which avoids calculations.
Conditions (1) and (3) are clear. Condition (2) follows from
\begin{align*}
\delta_n(x + z)\delta_m(x + z)
& =
(\gamma_n(x) + \gamma_{n - 1}(x)z)(\gamma_m(x) + \gamma_{m - 1}(x)z) \\
& = \gamma_n(x)\gamma_m(x) + \gamma_n(x)\gamma_{m - 1}(x)z +
\gamma_{n - 1}(x)\gamma_m(x)z \\
& =
\frac{(n + m)!}{n!m!} \gamma_{n + m}(x) +
\left(\frac{(n + m - 1)!}{n!(m - 1)!} +
\frac{(n + m - 1)!}{(n - 1)!m!}\right)
\gamma_{n + m - 1}(x) z \\
& =
\frac{(n + m)!}{n!m!}\delta_{n + m}(x + z)
\end{align*}
Condition (5) follows from
\begin{align*}
\delta_n(\delta_m(x + z))
& =
\delta_n(\gamma_m(x) + \gamma_{m - 1}(x)z) \\
& =
\gamma_n(\gamma_m(x)) + \gamma_{n - 1}(\gamma_m(x))\gamma_{m - 1}(x)z \\
& =
\frac{(nm)!}{n! (m!)^n} \gamma_{nm}(x) +
\frac{((n - 1)m)!}{(n - 1)! (m!)^{n - 1}}
\gamma_{(n - 1)m}(x) \gamma_{m - 1}(x) z \\
& = \frac{(nm)!}{n! (m!)^n}(\gamma_{nm}(x) + \gamma_{nm - 1}(x) z)
\end{align*}
by elementary number theory. To prove (4) we have to see that
$$
\delta_n(x + x' + z + z')
=
\gamma_n(x + x') + \gamma_{n - 1}(x + x')(z + z')
$$
is equal to
$$
\sum\nolimits_{i = 0}^n
(\gamma_i(x) + \gamma_{i - 1}(x)z)
(\gamma_{n - i}(x') + \gamma_{n - i - 1}(x')z')
$$
This follows easily on collecting the coefficients of
$1$, $z$, and $z'$ and using condition (4) for $\gamma$.
\end{proof}

\begin{lemma}
\label{lemma-divided-power-second-order-thickening}
Let $(A, I, \gamma)$ be a divided power ring. Let $M$, $N$ be $A$-modules.
Let $q : M \times M \to N$ be an $A$-bilinear map.
Let $B = A \oplus M \oplus N$ as an $A$-algebra with multiplication
$$
(x, z, w)\cdot (x', z', w') = (xx', xz' + x'z, xw' + x'w + q(z, z') + q(z', z))
$$
and set $J = I \oplus M \oplus N$. Set
$$
\delta_n(x, z, w) = (\gamma_n(x), \gamma_{n - 1}(x)z,
\gamma_{n - 1}(z)w + \gamma_{n - 2}(x)q(z, z))
$$
for $(a, m, n) \in J$.
Then $\delta$ is a divided power structure and
$A \to B$ is a homomorphism of divided power rings from
$(A, I, \gamma)$ to $(B, J, \delta)$.
\end{lemma}

\begin{proof}
Suppose we want to prove that property (4) of
Definition \ref{definition-divided-powers}
is satisfied. Pick $(x, z, w)$ and $(x', z', w')$ in $J$.
Pick a map
$$
A_0 = \mathbf{Z}\langle s, s'\rangle \longrightarrow A,\quad
s \longmapsto x,
s' \longmapsto x'
$$
which is possible by the universal property of divided power
polynomial rings. Set $M_0 = A_0 \oplus A_0$ and
$N_0 = A_0 \oplus A_0 \oplus M_0 \otimes_{A_0} M_0$.
Let $q_0 : M_0 \times M_0 \to N_0$ be the obvious map.
Define $M_0 \to M$ as the $A_0$-linear map which sends
the basis vectors of $M_0$ to $z$ and $z'$. Define $N_0 \to N$
as the $A_0$ linear map which sends the first two basis vectors
of $N_0$ to $w$ and $w'$ and uses
$M_0 \otimes_{A_0} M_0 \to M \otimes_A M \xrightarrow{q} N$
on the last summand. Then we see that it suffices to prove the
identitity (4) for the situation $(A_0, M_0, N_0, q_0)$.
Similarly for the other identities. This reduces us to the case
of a $\mathbf{Z}$-torsion free ring and $\mathbf{A}$-torsion free modules.
In this case all we have to do is show that
$$
n! \delta_n(x, z, w) = (x, z, w)^n
$$
in the ring $A$, see Lemma \ref{lemma-silly}. To see this note that
$$
(x, z, w)^2 = (x^2, 2xz, 2xw + 2q(z, z))
$$
and by induction
$$
(x, z, w)^n = (x^n, nx^{n - 1}z, nx^{n - 1}w + n(n - 1)x^{n - 2}q(z, z))
$$
On the other hand,
$$
n! \delta_n(x, z, w) = (n!\gamma_n(x), n!\gamma_{n - 1}(x)z,
n!\gamma_{n - 1}(x)w + n!\gamma_{n - 2}(x) q(z, z))
$$
which matches. This finishes the proof.
\end{proof}




\section{Compatibility}
\label{section-compatibility}

\noindent
This section isn't required reading; it explains how our discussion
fits with that of \cite{Berthelot}.
Consider the following technical notion.

\begin{definition}
\label{definition-compatible}
Let $(A, I, \gamma)$ and $(B, J, \delta)$ be divided power rings.
Let $A \to B$ be a ring map. We say
{\it $\delta$ is compatible with $\gamma$}
if there exists a divided power structure $\bar\gamma$ on
$J + IB$ such that
$$
(A, I, \gamma) \to (B, J + IB, \bar \gamma)\quad\text{and}\quad
(B, J, \delta) \to (B, J + IB, \bar \gamma)
$$
are homomorphisms of divided power rings.
\end{definition}

\noindent
Let $p$ be a prime number. Let $(A, I, \gamma)$ be a divided power ring.
Let $A \to C$ be a ring map with $p$ nilpotent in $C$.
Assume that $\gamma$ extends to $IC$ (see
Lemma \ref{lemma-gamma-extends}).
In this situation, the (big affine) crystalline site of
$\Spec(C)$ over $\Spec(A)$
as defined in \cite{Berthelot} 
is the opposite of the category of systems
$$
(B, J, \delta, A \to B, C \to B/J)
$$
where
\begin{enumerate}
\item $(B, J, \delta)$ is a divided power ring with $p$ nilpotent in $B$,
\item $\delta$ is compatible with $\gamma$, and
\item the diagram
$$
\xymatrix{
B \ar[r] & B/J \\
A \ar[u] \ar[r] & C \ar[u]
}
$$
is commutative.
\end{enumerate}
The conditions
``$\gamma$ extends to $C$ and $\delta$ compatible with $\gamma$''
are used in \cite{Berthelot} to insure that
the crystalline cohomology of $\Spec(C)$ is the same as the crystalline
cohomology of $\Spec(C/IC)$. We will avoid this issue
by working exclusively with $C$ such that $IC = 0$\footnote{Of course there
will be a price to pay.}. In this case,
for a system $(B, J, \delta, A \to B, C \to B/J)$ as above,
the commutativity of the displayed diagram above implies $IB \subset J$ and
compatibility is equivalent to the condition that
$(A, I, \gamma) \to (B, J, \delta)$ is a homomorphism of divided
power rings.




\section{Affine crystalline site}
\label{section-affine-site}

\noindent
In this section we put some algebra related to the crystalline site.
Note that usually the prime number $p$ will be contained in the
divided power ideal $I$.

\begin{definition}
\label{definition-affine-thickening}
Let $p$ be a prime number. Let $(A, I, \gamma)$ be a divided power
ring such that $A$ is a $\mathbf{Z}_{(p)}$-algebra. Let $A \to C$ be a
ring map such that $IC = 0$ and $p$ is nilpotent in $C$.
A {\it divided power thickening} of $C$ over $(A, I, \gamma)$
is a homomorphism of divided power algebras
$(A, I, \gamma) \to (B, J, \delta)$ such that $p$ is nilpotent in $B$
and a ring map $C \to B/J$ such that
$$
\xymatrix{
B \ar[r] & B/J \\
& C \ar[u] \\
A \ar[uu] \ar[r] & A/I \ar[u]
}
$$
is commutative. A {\it homomorphism of divided power thickenings}
is defined in the obvious way. We denote $\text{CRIS}(C/A, I, \gamma)$
or simply $\text{CRIS}(C/A)$ the category of divided power thickenings
of $C$ over $(A, I, \gamma)$.
\end{definition}

\noindent
Note that for a divided power thickening $(B, J, \delta)$ as above
the ideal $J$ is locally nilpotent, see Lemma \ref{lemma-nil}.
This category does not have equalizers or fibre products in general.
It also doesn't have an initial object ($=$ empty colimit) in general.

\begin{lemma}
\label{lemma-affine-thickenings-colimits}
Assumptions as in Definition \ref{definition-affine-thickening}.
The category $\text{CRIS}(C/A)$ has products.
It also has all finite nonempty colimits.
\end{lemma}

\begin{proof}
The empty product is $(C, 0, \emptyset)$. If $(B_t, J_t, \delta_t)$ is a
family of objects of $\text{CRIS}(C/A)$ then we can form the product
$(\prod B_t, \prod J_t, \prod \delta_t)$ as in Lemma \ref{lemma-colimits}.
The map $C \to \prod B_t/\prod J_t = \prod B_t/J_t$ is clear.

\medskip\noindent
Given two objects $(B, J, \gamma)$ and $(B', J', \gamma')$ of
$\text{CRIS}(C/A)$ we can form a cocartesian diagram
$$
\xymatrix{
(B, J, \delta) \ar[r] & (B'', J'', \delta'') \\
(A, I, \gamma) \ar[r] \ar[u] & (B', J', \delta') \ar[u]
}
$$
in the category of divided power rings. Then we see that we have
$$
B''/J'' = B/J \otimes_{A/I} B'/J' \longleftarrow C \otimes_{A/I} C
$$
see Remark \ref{remark-forgetful}. Denote $J'' \subset K \subset B''$
the ideal such that
$$
\xymatrix{
B''/J'' \ar[r] & B''/K \\
C \otimes_{A/I} C \ar[r] \ar[u] & C \ar[u]
}
$$
is a pushout. Let $D_{B''}(K) = (D, \bar K, \bar \delta)$
be the divided power envelope of $K$ in $B''$ relative to
$(B'', J'', \delta'')$. Then it is easily verified that
$(D, \bar K, \bar \delta)$ is a coproduct of $(B, J, \delta)$ and
$(B', J', \delta')$ in $\text{CRIS}(C/A)$.

\medskip\noindent
Next, we come to coequalizers. Let
$\alpha, \beta : (B, J, \delta) \to (B', J', \delta')$ be morphisms of
$\text{CRIS}(C/A)$. Consider $B'' = B'/ (\alpha(b) - \beta(b))$. Let
$J'' \subset B''$ be the image of $J'$. Let
$D_{B''}(J'') = (D, \bar J, \bar\delta)$ be the divided power envelope of
$J''$ in $B''$ relative to $(B', J', \delta')$. Then it is easily verified
that $(D, \bar J, \bar \delta)$ is the coequalizer of $(B, J, \delta)$ and
$(B', J', \delta')$ in $\text{CRIS}(C/A)$.

\medskip\noindent
By Categories, Lemma \ref{categories-lemma-almost-finite-colimits-exist}
we have all finite nonempty colimits in $\text{CRIS}(C/A)$.
\end{proof}

\begin{lemma}
\label{lemma-set-generators}
Let $p$, $(A, I, \gamma)$, and $A \to C$ be as in
Definition \ref{definition-affine-thickening}.
Let $P$ be a polynomial algebra over $A$ and let
$P \to C$ be a surjection of $A$-algebras with kernel $K$.
For every $n \geq 1$ set $P_n = P/p^nP$ and $K_n \subset P_n$
the image of $K$ and
$$
D_n = D_{P_n}(K_n)
$$
where the divided power envelope is taken relative to $(A, I, \gamma)$.
Then $D_n$ is an object of $\text{CRIS}(C/A)$ and for every object
$(B, J, \delta)$ of $\text{CRIS}(C/A)$ there exists an $n$ and
a morphsm $D_n \to B$ of $\text{CRIS}(C/A)$.
\end{lemma}

\begin{proof}
We just prove the final assertion.
We can find an $A$-algebra homomorphism $P \to B$
lifting the map $C \to B/J$. By definition $p^nB = 0$ for
some $n$ hence $P \to B$ factors as $P \to P_n \to B$.
By the universal property of the divided power envelope we
conclude that $P_n \to B$ factors through $D_n$.
\end{proof}







\section{Module of differentials}
\label{section-differentials}

\noindent
In this section we develop a theory of modules of differentials
for divided power rings.

\begin{definition}
\label{definition-derivation}
Let $A$ be a ring. Let $(B, J, \delta)$ be a divided power ring.
Let $A \to B$ be a ring map. Let $M$ be an $B$-module.
A {\it divided power $A$-derivation} into $M$ is a map
$\theta : B \to M$ which is additive, annihilates the elements
of $A$, satisfies the Leibniz rule
$\theta(bb') = b\theta(b') + b'\theta(b)$ and satisfies
$$
\theta(\gamma_n(x)) = \gamma_{n - 1}(x)\theta(x)
$$
for all $n \geq 1$ and all $x \in J$.
\end{definition}

\noindent
In the situation of the definition, just as in the case of usual
derivations, there exists a {\it universal divided power $A$-derivation}
$$
\text{d}_{B/A, \delta} : B \to \Omega_{B/A, \delta}
$$
such that any divided power $A$-derivation $\theta : B \to M$ is equal to
$\theta = \xi \circ d_{B/A, \delta}$ for some $B$-linear map
$\Omega_{B/A, \delta} \to M$. If $(A, I, \gamma) \to (B, J, \delta)$
is a homomorphism of divided power rings, then we can forget the
divided powers on $A$ and consider the divided power derivations of
$B$ over $A$. We often write simply $\Omega_{B/A}$ and
$\text{d} : B \to \Omega_{B/A}$ for the universal divided power
$A$-derivation.

\medskip\noindent
Let $(A, I, \gamma)$ be a divided power ring. In this setting the
correct version of the powers of $I$ is given by the divided powers
$$
I^{[n]} = \text{ideal generate by }
\gamma_{e_1}(x_1) \ldots \gamma_{e_t}(x_t)
\text{ with }\sum e_j \geq n\text{ and }x_j \in I.
$$
Of course we have $I^n \subset I^{[n]}$. Note that $I^{[1]} = I$.
Sometimes we also set $I^{[0]} = A$.

\begin{lemma}
\label{lemma-diagonal-and-differentials}
Let $(A, I, \gamma) \to (B, J, \delta)$ be a homomorphism
of divided power rings. Let $(B(1), J(1), \delta(1))$ be the coproduct
of $(B, J, \delta)$ with itself over $(A, I, \gamma)$, i.e.,
such that
$$
\xymatrix{
(B, J, \delta) \ar[r] & (B(1), J(1), \delta(1)) \\
(A, I, \gamma) \ar[r] \ar[u] & (B, J, \delta) \ar[u]
}
$$
is cocartesian. Denote $K = \text{Ker}(B(1) \to B)$.
Then $K \cap J(1) \subset J(1)$ is preserved by the divided power
structure and
$$
\Omega_{B/A, \delta} = K/ \left(K^2 + (K \cap J(1))^{[2]}\right)
$$
canonically.
\end{lemma}

\begin{proof}
The fact that $K \cap J(1) \subset J(1)$ is preserved by the divided power
structure follows from the fact that $B(1) \to B$ is a homomorphism of
divided power rings.

\medskip\noindent
Recall that $K/K^2$ has a canonical $B$-module structure.
Denote $s_0, s_1 : B \to B(1)$ the two coprojections and consider
the map $\text{d} : B \to K/K^2 +(K \cap J(1))^{[2]}$ given by
$b \mapsto s_0(b) - s_1(b)$. It is clear that $\text{d}$ is additive,
annihilates $A$, and satisfies the Leibniz rule.
We claim that $\text{d}$ is an $A$-derivation.
Let $x \in J$. Set $y = s_0(x)$ and $z = s_1(x)$.
Denote $\delta$ the divided power structure on $J(1)$.
We have to show that $\delta_n(y) - \delta_n(z) = \delta_{n - 1}(y)(y - z)$
modulo $K^2 +(K \cap J(1))^{[2]}$ for $n \geq 1$. We will show this
by induction on $n$. It is true for $n = 1$.
Let $n > 1$ and that it holds for all smaller values.
Note that
$$
\delta_n(z - y) =
\sum\nolimits_{i = 0}^n (-1)^{n - i}\delta_i(z)\delta_{n - i}(y)
$$
is an element of $K^2 +(K \cap J(1))^{[2]}$. From this and induction
we see that working modulo $K^2 +(K \cap J(1))^{[2]}$ we have
\begin{align*}
& \delta_n(y) - \delta_n(z) \\
& =
\delta_n(y) +
\sum\nolimits_{i = 0}^{n - 1} (-1)^{n - i}\delta_i(z)\delta_{n - i}(y) \\
& =
\delta_n(y) + (-1)^n\delta_n(y) +
\sum\nolimits_{i = 1}^{n - 1}
(-1)^{n - i}(\delta_i(y) - \delta_{i - 1}(y)(y - z))\delta_{n - i}(y)
\end{align*}
Using that $\delta_i(y)\delta_{n - i}(y) = \binom{n}{i} \delta_n(y)$
and that $\delta_{i - 1}(y)\delta_{n - i}(y) =
\binom{n - 1}{i} \delta_{n - 1}(y)$
the reader easily verifies that this expression comes out to give
$\delta_{n - 1}(y)(y - z)$ as desired.

\medskip\noindent
Let $M$ be a $B$-module. Let $\theta : B \to M$ be a divided power
$A$-derivation.
Set $D = B \oplus M$ where $M$ is an ideal of square zero. Define a
divided power structure on $J \oplus M \subset D$ by setting
$\delta_n(x + m) = \delta_n(x) + \delta_{n - 1}(x)m$ for $n > 1$, see
Lemma \ref{lemma-divided-power-first-order-thickening}.
There are two divided power algebra homomorphisms $B \to D$: the first
is given by the inclusion and the second by the map $b \mapsto b + \theta(b)$.
Hence we get a canonical homomorphism $B(1) \to D$ of divided power
algebras over $(A, I, \gamma)$. This induces a map $K \to M$
which annihilates $K^2$ (as $M$ is an ideal of square zero) and
$(K \cap J(1))^{[2]}$ as $M^{[2]} = 0$. The composition
$B \to K/K^2 + (K \cap J(1))^{[2]} \to M$ equals $\theta$ by construction.
It follows that $\text{d}$
is a universal divided power $A$-derivation and we win.
\end{proof}

\begin{remark}
\label{remark-filtration-differentials}
Let $A \to B$ be a ring map and let $(J, \delta)$ be a divided
power structure on $B$. The universal module $\Omega_{B/A, \delta}$
comes with a little bit of extra structure, namely the $B$-submodule
$N$ of $\Omega_{B/A, \delta}$ generated by $\text{d}_{B/A, \delta}(J)$.
In terms of the isomorphism given in
Lemma \ref{lemma-diagonal-and-differentials}
this corresponds to the image of
$K \cap J(1)$ in $\Omega_{B/A, \delta}$. Consider the $A$-algebra
$D = B \oplus \Omega^1_{B/A, \delta}$ with ideal $\bar J = J \oplus N$
and divided powers $\bar \delta$ as in the proof of the lemma.
Then $(D, \bar J, \bar \delta)$ is a divided power ring
and the two maps $B \to D$ given by $b \mapsto b$ and
$b \mapsto b + \text{d}_{B/A, \delta}(b)$
are homomorphisms of divided power rings over $A$. Moreover, $N$
is the smallest submodule of $\Omega_{B/A, \delta}$ such that this is true.
\end{remark}

\begin{lemma}
\label{lemma-diagonal-and-differentials-affine-site}
Let $p$, $(A, I, \gamma)$, and $A \to C$ as in
Definition \ref{definition-affine-thickening}.
Let $(B, J, \delta)$ be an object of $\text{CRIS}(C/A)$.
Let $(B(1), J(1), \delta(1))$ be the coproduct of $(B, J, \delta)$
with itself in $\text{CRIS}(C/A)$. Denote
$K = \text{Ker}(B(1) \to B)$. Then $K \cap J(1) \subset J(1)$
is preserved by the divided power structure and
$$
\Omega_{B/A, \delta} = K/ \left(K^2 + (K \cap J(1))^{[2]}\right)
$$
canonically.
\end{lemma}

\begin{proof}
Word for word the same as the proof of
Lemma \ref{lemma-diagonal-and-differentials}.
The only point that has to be checked is that the
divided power ring $D = B \oplus M$ is an object of $\text{CRIS}(C/A)$
and that the two maps $B \to C$ are morphisms of $\text{CRIS}(C/A)$.
Since $D/(J \oplus M) = B/J$ we can use $C \to B/J$ to view
$D$ as an object of $\text{CRIS}(C/A)$
and the statement on morphisms is clear from the construction.
\end{proof}

\begin{lemma}
\label{lemma-module-differentials-divided-power-envelope}
Let $(A, I, \gamma)$ be a divided power ring. Let $A \to B$ be a ring
map and let $IB \subset J \subset B$ be an ideal. Let
$D_{B, \gamma}(J) = (D, \bar J, \bar \gamma)$ be the divided power envelope.
Then we have
$$
\Omega_{D/A, \bar\gamma} = \Omega_{B/A} \otimes_B D
$$
\end{lemma}

\begin{proof}
We will prove this first when $B$ is flat over $A$. In this case $\gamma$
extends to a divided power structure $\gamma'$ on $IB$, see
Lemma \ref{lemma-gamma-extends}.
Hence $D = D_{B', \gamma'}(J)$ is equal to a quotient of
the divided power ring $(D', J', \delta)$ where $D' =  B\langle x_t \rangle$
and $J' = IB\langle x_t \rangle + B\langle x_t \rangle_{+}$
by the elements $x_t - f_t$ and $\delta_n(\sum r_t x_t - r_0)$, see
Lemma \ref{lemma-describe-divided-power-envelope} for notation
and explanation. Write $\text{d} : D' \to \Omega_{D'/A, \delta}$
for the universal derivation. Note that
$$
\Omega_{D'/A, \delta} =
\Omega_{B/A} \otimes_B D' \oplus \bigoplus D' \text{d}x_t.
$$
(Details omitted.) We conclude that $\Omega_{D/A, \bar\gamma}$ is the
quotient of $\Omega_{D'/A, \delta} \otimes_{D'} D$ by the submodule
generated by $\text{d}$ applied to the generators of the
kernel of $D" \to D$ listed above. Since
$\text{d}(x_t - f_t) = - \text{d}f_t + \text{d}x_t$
we see that we have $\text{d}x_t = \text{d}f_t$ in the quotient.
In particular we see that $\Omega_{B/A} \otimes_B D \to \Omega_{D/A, \gamma}$
is surjective with kernel given by the images of $\text{d}$
applied to the elements $\delta_n(\sum r_t x_t - r_0)$.
However, given a relation $\sum r_tf_t - r_0 = 0$ in $B$ with
$r_t \in B$ and $r_0 \in IB$ we see that
\begin{align*}
\text{d}\delta_n(\sum r_t x_t - r_0)
& =
\delta_{n - 1}(\sum r_t x_t - r_0)\text{d}(\sum r_t x_t - r_0)
\\
& =
\delta_{n - 1}(\sum r_t x_t - r_0)
\left(
\sum r_t\text{d}(x_t - f_t) + \sum (x_t - f_t)\text{d}r_t
\right)
\end{align*}
because $\sum r_tf_t - r_0 = 0$ in $B$. Hence this is already zero in
$\Omega_{B/A} \otimes_A D$ and we win in the case that $B$ is flat over $A$.

\medskip\noindent
In the general case we write $B$ as a quotient of a polynomial ring
$P \to B$ and let $J' \subset P$ be the inverse image of $J$. Then
$D = D'/K'$ with notation as in
Lemma \ref{lemma-divided-power-envelop-quotient}.
By the case handled in the first paragraph of the proof we have
$\Omega_{D'/A, \bar\gamma'} = \Omega_{P/A} \otimes_P D'$. Then
$\Omega_{D/A, \bar \gamma}$ is the quotient of $\Omega_{P/A} \otimes_P D$
by the submodule generated by $\text{d}\bar\gamma'(k)$ where $k$
is an element of the kernel of $P \to B$. Since
$\text{d}\bar\gamma_n'(k) = \bar\gamma'_{n - 1}(k)\text{d}k$ we see
again that it suffices to divided by the submodule generated by
$\text{d}k$ with $k \in \text{Ker}(P \to B)$ and since $\Omega_{B/A}$
is the quotient of $\Omega_{P/A} \otimes_A B$ by these elements we win.
\end{proof}

\begin{remark}
\label{remark-absolute-de-rham-complex}
Let $B$ be a ring. Write $\Omega_B = \Omega_{B/\mathbf{Z}}$
for the absolute\footnote{This
actually makes sense: if $\Omega_B$ is the module of differentials
where we only assume the Leibniz rule and not the vanishing of $\text{d}1$,
then the Leibniz rule gives $\text{d}1 = \text{d}(1 \cdot 1) =
1 \text{d}1 + 1 \text{d}1 = 2 \text{d}1$ and hence
$\text{d}1 = 0$ in $\Omega_B$.} module of differentials of $B$.
Let $\text{d} : B \to \Omega_B$ denote the universal derivation. Set
$\Omega_B^i = \wedge^i_B(\Omega_B)$ as in
Algebra, Section \ref{algebra-section-tensor-algebra}.
The absolute {\it de Rham complex}
$$
\Omega_B^0 \to \Omega_B^1 \to \Omega_B^2 \to \ldots
$$
Here $\text{d} : \Omega_B^p \to \Omega_B^{p + 1}$
is defined by the rule
$$
\text{d}\left(b_0\text{d}b_1 \wedge \ldots \wedge \text{d}b_p\right) =
\text{d}b_0 \wedge \text{d}b_1 \wedge \ldots \wedge \text{d}b_p
$$
which we will show is well defined; note that
$\text{d} \circ \text{d} = 0$ so we get a complex.
Recall that $\Omega_B$ is the $B$-module generated by
elements $\text{d}b$ subject to the relations
$\text{d}(a + b) = \text{d}a + \text{d}b$ and
$\text{d}(ab) = b\text{d}a + a\text{d}b$
for $a, b \in B$. To prove that our map is well defined for $p = 1$
we have to show that the elements
$$
a\text{d}(b + c) - a\text{d}b - a\text{d}c
\quad\text{and}\quad
a\text{d}(bc) - ac\text{d}b - ab\text{d}c,\quad a,b,c \in B
$$
are mapped to zero by our rule. This is clear by direct computation
(using the Leibniz rule). Thus we get a map
$$
\Omega_B \otimes_\mathbf{Z} \ldots \otimes_\mathbf{Z} \Omega_B
\longrightarrow
\Omega_B^{p + 1}
$$
defined by the formula
$$
\omega_1 \otimes \ldots \otimes \omega_p
\longmapsto
\sum (-1)^{i + 1}
\omega_1 \wedge \ldots \wedge \text{d}(\omega_i) \wedge \ldots \wedge \omega_p
$$
which matches our rule above on elements of the form
$b_0\text{d}b_1 \otimes \text{d}b_2 \otimes \ldots \otimes \text{d}b_p$.
It is clear that this map is alternating. To finish we have to show
that
$$
\omega_1 \otimes \ldots \otimes f\omega_i \otimes \ldots \otimes \omega_p
\quad\text{and}\quad
\omega_1 \otimes \ldots \otimes f\omega_j \otimes \ldots \otimes \omega_p
$$
are mapped to the same element. By $\mathbf{Z}$-linearity and
the alternating property, it is enough to show this for $p = 2$, $i = 1$,
$j = 2$, $\omega_1 = a_1 \text{d}b_1$ and $\omega_2 = a_2 \text{d}b_2$.
Thus we need to show that
\begin{align*}
& \text{d}fa_1 \wedge \text{d}b_1 \wedge a_2\text{d}b_2
- fa_1 \text{d}b_1 \wedge \text{d}a_2 \wedge \text{d}b_2 \\
& =
\text{d}a_1 \wedge \text{d}b_1 \wedge fa_2\text{d}b_2
- a_1 \text{d}b_1 \wedge \text{d}fa_2 \wedge \text{d}b_2
\end{align*}
in other words that
$$
(a_2 \text{d}fa_1 + fa_1 \text{d}a_2 - fa_2 \text{d}a_1 - a_1 \text{d}fa_2)
\wedge \text{d}b_1 \wedge \text{d}b_2 = 0.
$$
This follows from the Leibniz rule.
\end{remark}

\begin{lemma}
\label{lemma-de-rham-complex}
Let $B$ be a ring. Let $\pi : \Omega_B \to \Omega$ be a surjective $B$-module
map. Denote $\text{d} : B \to \Omega$ the composition of $\pi$ with
$\text{d}_B : B \to \Omega_B$. Set $\Omega^i = \wedge_B^i(\Omega)$.
Assume that the kernel of $\pi$ is generated, as a $B$-module,
by elements $\omega \in \Omega_B$ such that
$\text{d}_B(\omega) \in \Omega_B^2$ maps to zero in $\Omega^2$.
Then there is a de Rham complex
$$
\Omega^0 \to \Omega^1 \to \Omega^2 \to \ldots
$$
whose differential is defined by the rule
$$
\text{d} : \Omega^p \to \Omega^{p + 1},\quad
\text{d}\left(f_0\text{d}f_1 \wedge \ldots \wedge \text{d}f_p\right) =
\text{d}f_0 \wedge \text{d}f_1 \wedge \ldots \wedge \text{d}f_p
$$
\end{lemma}

\begin{proof}
We will show that there exists a commutative diagram
$$
\xymatrix{
\Omega_B^0 \ar[d] \ar[r]_{\text{d}_B} &
\Omega_B^1 \ar[d]_\pi \ar[r]_{\text{d}_B} &
\Omega_B^2 \ar[d]_{\wedge^2\pi} \ar[r]_{\text{d}_B} &
\ldots \\
\Omega^0 \ar[r]^{\text{d}} &
\Omega^1 \ar[r]^{\text{d}} &
\Omega^2 \ar[r]^{\text{d}} &
\ldots
}
$$
the description of the map $\text{d}$ will follow from the construction
of $\text{d}_B$ in Remark \ref{remark-absolute-de-rham-complex}.
Since the left most vertical arrow is an isomorphism we have
the first square. Because $\pi$ is surjective, to get the second
square it suffices to show that $\text{d}_B$ maps the kernel
of $\pi$ into the kernel of $\wedge^2\pi$. We are given that any element
of the kernel of $\pi$ is of the form
$\sum b_i\omega_i$ with $\pi(\omega_i) = 0$ and
$\wedge^2\pi(\text{d}_B(\omega_i)) = 0$.
By the Leibniz rule for $\text{d}_B$ we have
$\text{d}_B(\sum b_i\omega_i) = \sum b_i \text{d}_B(\omega_i) +
\sum \text{d}_B(b_i) \wedge \omega_i$. Hence this maps to zero
under $\wedge^2\pi$.

\medskip\noindent
For $i > 1$ we note that $\wedge^i \pi$ is surjective with
kernel the image of $\text{Ker}(\pi) \wedge \Omega^{i - 1}_B
\to \Omega_B^i$. For $\omega_1 \in \text{Ker}(\pi)$ and
$\omega_2 \in \Omega^{i - 1}_B$ we have
$$
\text{d}_B(\omega_1 \wedge \omega_2) =
\text{d}_B(\omega_1) \wedge \omega_2 - \omega_1 \wedge \text{d}_B(\omega_2)
$$
which is in the kernel of $\wedge^{i + 1}\pi$ by what we just proved above.
Hence we get the $(i + 1)$st square in the diagram above.
This concludes the proof.
\end{proof}

\begin{remark}
\label{remark-divided-powers-de-rham-complex}
Let $A \to B$ be a ring map and let $(J, \delta)$ be a divided power
structure on $B$. Set
$\Omega_{B/A, \delta}^i = \wedge^i_B \Omega_{B/A, \delta}$
where $\Omega_{B/A, \delta}$ is the target of the universal divided power
$A$-derivation $\text{d} = \text{d}_{B/A} : B \to \Omega_{B/A, \delta}$.
Note that $\Omega_{B/A, \delta}$ is the quotient of $\Omega_B$ by the
$B$-submodule generated by the elements
$\text{d}a = 0$ for $a \in A$ and
$\text{d}\delta_n(x) - \delta_{n - 1}(x)\text{d}x$ for $x \in J$.
We claim Lemma \ref{lemma-de-rham-complex} applies.
To see this it suffices to verify the elements
$\text{d}a$ and $\text{d}\delta_n(x) - \delta_{n - 1}(x)\text{d}x$
of $\Omega_B$ are mapped to zero in $\Omega^2_{B/A, \delta}$.
This is clear for the first, and for the last we observe that
$$
\text{d}(\delta_{n - 1}(x)) \wedge \text{d}x
= \delta_{n - 2}(x) \text{d}x \wedge \text{d}x = 0
$$
in $\Omega^2_{B/A, \delta}$ as desired. Hence we obtain a
{\it divided power de Rham complex}
$$
\Omega^0_{B/A, \delta} \to \Omega^1_{B/A, \delta} \to
\Omega^2_{B/A, \delta} \to \ldots
$$
which will play an important role in the sequel.
\end{remark}

\begin{remark}
\label{remark-connection}
Let $B$ be a ring. Let $\Omega_B \to \Omega$ be a quotient satisfying
the assumptions of Lemma \ref{lemma-de-rham-complex} so
that we have a de Rham complex. Let $M$ be a $B$-module.
A {\it connection} is an additive map
$$
\nabla : M \longrightarrow M \otimes_B \Omega
$$
such that $\nabla(bm) = b \nabla(m) + m \otimes \text{d}b$
for $b \in B$ and $m \in M$. In this situation we can define maps
$$
\nabla : M \otimes_B \Omega^i \longrightarrow M \otimes_B \Omega^{i + 1}
$$
by the rule $\nabla(m \otimes \omega) = \nabla(m) \wedge \omega +
m \otimes \text{d}\omega$. This works because if $b \in B$, then
\begin{align*}
\nabla(bm \otimes \omega) - \nabla(m \otimes b\omega)
& =
\nabla(bm) \otimes \omega + bm \otimes \text{d}\omega
- \nabla(m) \otimes b\omega - m \otimes \text{d}(b\omega) \\
& =
b\nabla(m) \otimes \omega + m \otimes \text{d}b \wedge \omega
+ bm \otimes \text{d}\omega \\
&\ \ \ \ \ \ - b\nabla(m) \otimes \omega - bm \otimes \text{d}(\omega)
- m \otimes \text{d}b \wedge \omega \\
& = 0
\end{align*}
As is customary we say the connection is {\it integrable} if and
only if the composition
$$
M \xrightarrow{\nabla} M \otimes_B \Omega^1
\xrightarrow{\nabla} M \otimes_B \Omega^2
$$
is zero. In this case we obtain a complex
$$
M \xrightarrow{\nabla} M \otimes_B \Omega^1
\xrightarrow{\nabla} M \otimes_B \Omega^2
\xrightarrow{\nabla} M \otimes_B \Omega^3
\xrightarrow{\nabla} M \otimes_B \Omega^4 \to \ldots
$$
which is called the de Rham complex of the connection.
\end{remark}




\section{Divided power schemes}
\label{section-divided-power-schemes}

\noindent
Some remarks on how to globalize the previous notions.

\begin{definition}
\label{definition-divided-power-structure}
Let $\mathcal{C}$ be a site. Let $\mathcal{O}$ be a sheaf of rings
on $\mathcal{C}$. Let $\mathcal{I} \subset \mathcal{O}$ be a
sheaf of ideals. A {\it divided power structure $\gamma$} on $\mathcal{I}$
is a sequence of maps $\gamma_n : \mathcal{I} \to \mathcal{I}$, $n \geq 1$
such that for any object $U$ of $\mathcal{C}$ the triple
$$
(\mathcal{O}(U), \mathcal{I}(U), \gamma)
$$
is a divided power ring.
\end{definition}

\noindent
To be sure this applies in particular to sheaves of rings on
topological spaces. But it's good to be a little bit more general
as the structure sheaf of the crystalline site lives on a... site!
A triple $(\mathcal{C}, \mathcal{I}, \gamma)$ as in the
definition above is sometimes called a {\it divided power topos}
in this chapter. Given a second $(\mathcal{C}', \mathcal{I}', \gamma')$ and
given a morphism of ringed topoi
$(f, f^\sharp) : (\Sh(\mathcal{C}), \mathcal{O}) \to
(\Sh(\mathcal{C}'), \mathcal{O}')$
we say that $(f, f^\sharp)$ induces a {\it morphism of divided
power topoi} if $f^\sharp(f^{-1}\mathcal{I}') \subset \mathcal{I}$
and the diagrams
$$
\xymatrix{
f^{-1}\mathcal{I}' \ar[d]_{f^{-1}\gamma'_n} \ar[r]_{f^\sharp} &
\mathcal{I} \ar[d]^{\gamma_n} \\
f^{-1}\mathcal{I}' \ar[r]^{f^\sharp} & \mathcal{I}
}
$$
are commutative for all $n \geq 1$. If $f$ comes from a morphism of
sites induced by a functor $u : \mathcal{C}' \to \mathcal{C}$ then
this just means that
$$
(\mathcal{O}'(U'), \mathcal{I}'(U'), \gamma')
\longrightarrow
(\mathcal{O}(u(U')), \mathcal{I}(u(U')), \gamma)
$$
is a homomorphism of divided power rings for all $U' \in \Ob(\mathcal{C}')$.

\medskip\noindent
In the case of schemes we require the divided power ideal to be
{\bf quasi-coherent}. But apart from this the definition is exactly
the same as in the case of topoi. Here it is.

\begin{definition}
\label{definition-divided-power-scheme}
A {\it divided power scheme} is a triple $(S, \mathcal{I}, \gamma)$
where $S$ is a scheme, $\mathcal{I}$ is a quasi-coherent sheaf of
ideals, and $\gamma$ is a divided power structure on $\mathcal{I}$.
A {\it morphism of divided power schemes}
$(S, \mathcal{I}, \gamma) \to (S', \mathcal{I}', \gamma')$ is
a morphism of schemes $f : S \to S'$ such that
$f^{-1}\mathcal{I}'\mathcal{O}_S \subset \mathcal{I}$ and such that
$$
(\mathcal{O}_S(U'), \mathcal{I}(U'), \gamma)
\longrightarrow
(\mathcal{O}_{S'}(f^{-1}U'), \mathcal{I}(f^{-1}U'), \gamma)
$$
is a homomorphism of divided power rings for all $U' \subset S'$ open.
\end{definition}

\noindent
Recall that there is a 1-to-1 correspondence between quasi-coherent
sheaves of ideals and closed immersions, see
Morphisms, Section \ref{morphisms-section-closed-immersions}.
Thus given a divided power scheme $(T, \mathcal{J}, \gamma)$ we
get a canonical closed immersion $U \to T$ defined by $\mathcal{J}$.
Conversely, given a closed immersion $U \to T$ and a divided power
structure $\gamma$ on the sheaf of ideals $\mathcal{J}$ associated
to $U \to T$ we obtain a divided power scheme $(T, \mathcal{J}, \gamma)$.
In many situations we only want to consider such triples
$(U, T, \gamma)$ when the morphism $U \to T$ is a thickening, see
More on Morphisms, Definition \ref{more-morphisms-definition-thickening}.

\begin{definition}
\label{definition-divided-power-thickening}
A triple $(U, T, \gamma)$ as above is called a {\it divided power thickening}
if $U \to T$ is a thickening.
\end{definition}

\noindent
We make the following observation. Suppose that $(U, T, \gamma)$
is triple as above. Assume that $T$ is a scheme over $\mathbf{Z}_{(p)}$
and that $p$ is locally nilpotent on $U$. Then we have
$$
p\text{ locally nilpotent on }T
\Leftrightarrow
U \to T\text{ is a thickening}
$$
See Lemma \ref{lemma-nil}.






\section{The big crystalline site}
\label{section-big-site}

\noindent
We first define the big site. Given a divided power scheme
$(S, \mathcal{I}, \gamma)$ we say $(T, \mathcal{J}, \delta)$ is
a divided power scheme over $(S, \mathcal{I}, \gamma)$ if
$T$ comes endowed with a morphism $T \to S$ of divided power
schemes. Similarly, we say a divided power thickening $(U, T, \delta)$
is a divided power thickening over $(S, \mathcal{I}, \gamma)$
if $T$ comes endowed with a morphism $T \to S$ of divided power
schemes.

\begin{definition}
\label{definition-divided-power-thickening-X}
Let $p$ be a prime number. Let $(S, \mathcal{I}, \gamma)$ be a divided power
scheme over $\mathbf{Z}_{(p)}$. Let $f : X \to S$ be a morphism of schemes
such that $f^{-1}\mathcal{I} \mathcal{O}_X = 0$ and such that $p$ is
locally nilpotent on $X$.
\begin{enumerate}
\item A {\it divided power thickening of $X$ relative to
$(S, \mathcal{I}, \gamma)$} is given by a divided power thickening
$(U, T, \delta)$ over $(S, \mathcal{I}, \gamma)$
and an $S$-morphism $U \to X$.
\item A {\it morphism of divided power thickenings of $X$
relative to $(S, \mathcal{I}, \gamma)$} is defined in the obvious
manner.
\end{enumerate}
The category of divided power thickenings of $X$ relative to
$(S, \mathcal{I}, \gamma)$ is denoted $\text{CRIS}(X/S, \mathcal{I}, \gamma)$
or simply $\text{CRIS}(X/S)$.
\end{definition}

\noindent
For any $(U, T, \delta)$ in $\text{CRIS}(X/S)$
we have that $p$ is locally nilpotent on $T$, see discussion after
Definition \ref{definition-divided-power-thickening}.
A good way to visualize all the data associated to $(U, T, \delta)$
is the commutative diagram
$$
\xymatrix{
T \ar[dd] & U \ar[l] \ar[d] \\
& X \ar[d] \\
S & S_0 \ar[l]
}
$$
where $S_0 = V(\mathcal{I}) \subset S$. Morphisms of $\text{CRIS}(X/S)$
can be similarly visualized as huge commutative diagrams. In particular,
there is a canonical forgetful functor
\begin{equation}
\label{equation-forget}
\text{CRIS}(X/S) \longrightarrow \Sch/X,\quad
(U, T, \delta) \longmapsto U
\end{equation}
as well as its one sided inverse (and left adjoint)
\begin{equation}
\label{equation-endow-trivial}
\Sch/X \longrightarrow \text{CRIS}(X/S),\quad
U \longmapsto (U, U, \emptyset)
\end{equation}
which is sometimes useful.

\begin{lemma}
\label{lemma-divided-power-thickening-fibre-products}
Assumptions as in Definition \ref{definition-divided-power-thickening-X}.
The category $\text{CRIS}(X/S)$ has all finite nonempty limits,
in particular products of pairs and fibre products.
The functor (\ref{equation-forget}) commutes with limits.
\end{lemma}

\begin{proof}
Omitted. Hint: See Lemma \ref{lemma-affine-thickenings-colimits}
for the affine case. See also Remark \ref{remark-forgetful}.
\end{proof}

\begin{lemma}
\label{lemma-divided-power-thickening-base-change-flat}
Assumptions as in Definition \ref{definition-divided-power-thickening-X}.
Let
$$
\xymatrix{
(U_3, T_3, \delta_3) \ar[d] \ar[r] & (U_2, T_2, \delta_2) \ar[d] \\
(U_1, T_1, \delta_1) \ar[r] & (U, T, \delta)
}
$$
be a fibre square in the category of divided power thickenings of
$X$ relative to $(S, \mathcal{I}, \gamma)$. If $T_2 \to T$ is
flat, then $T_3 = T_1 \times_T T_2$ (as schemes).
\end{lemma}

\begin{proof}
This is true because a divided power structure extends uniquely
along a flat ring map. See Lemma \ref{lemma-gamma-extends}.
\end{proof}

\noindent
The lemma above means that the base change of a flat morphism
of divided power thickenings is another flat morphism, and in
fact is the ``usual'' base change of the morphism. This implies
that the following definition makes sense.

\begin{definition}
\label{definition-big-crystalline-site}
Assumptions as in Definition \ref{definition-divided-power-thickening-X}.
\begin{enumerate}
\item A family of morphisms $\{(U_i, T_i, \delta_i) \to (U, T, \delta)\}$
of divided power thickenings of $X/S$ is a {\it Zariski, \'etale, smooth,
syntomic, or fppf covering} if and only if the family of morphisms
of schemes $\{T_i \to T \}$ is one.
\item The {\it big crystalline site} of $X$ over $(S, \mathcal{I}, \gamma)$,
is the category $\text{CRIS}(X/S)$ endowed with the Zariski topology.
\item The topos of sheaves on $\text{CRIS}(X/S)$ is denoted
$(X/S)_{\text{CRIS}}$ or sometimes
$(X/S, \mathcal{I}, \gamma)_{\text{CRIS}}$\footnote{This clashes with
our convention to denote the topos associated to a site $\mathcal{C}$
by $\Sh(\mathcal{C})$.}.
\end{enumerate}
\end{definition}

\noindent
There are some obvious functorialities concerining these topoi.

\begin{remark}[Functoriality]
\label{remark-functoriality-big-cris}
Let $p$ be a prime number.
Let $(S, \mathcal{I}, \gamma)$ be a divided power scheme
over $\mathbf{Z}_{(p)}$.
Let $f : X \to Y$ be a morphism of schemes over $S$ such that
$\mathcal{I} \cdot \mathcal{O}_X = 0$,
$\mathcal{I} \cdot \mathcal{O}_Y = 0$, and $p$ is locally nilpotent
on $X$ and $Y$. Then we get a continuous and cocontinuous functor
$$
\text{CRIS}(X/S) \longrightarrow \text{CRIS}(Y/S)
$$
by letting $(U, T, \delta)$ correspond to $(U, T, \delta)$
with $U \to X \to Y$ as the $S$-morphism from $U$ to $Y$.
Hence we get a morphism of topoi
$$
f_{\text{CRIS}} : (X/S)_{\text{CRIS}} \longrightarrow (Y/S)_{\text{CRIS}}
$$
see Sites, Section \ref{sites-section-cocontinuous-morphism-topoi}.
\end{remark}

\begin{remark}[Comparison with Zariski site]
\label{remark-compare-big-zariski}
Assumptions as in Definition \ref{definition-divided-power-thickening-X}.
The functor (\ref{equation-forget}) is continuous, cocontinuous, and
commutes with products and fibred products.
Hence we obtain a morphism of topoi
$$
U_{X/S} : (X/S)_{\text{CRIS}} \longrightarrow \Sh((\Sch/X)_{Zar})
$$
from the big crystalline topos of $X/S$ to the big Zarisk topos of $X$.
See Sites, Section \ref{sites-section-cocontinuous-morphism-topoi}.
\end{remark}

\begin{remark}[Structure morphism]
\label{remark-big-structure-morphism}
Assumptions as in Definition \ref{definition-divided-power-thickening-X}.
Consider the closed subscheme $S_0 = V(\mathcal{I}) \subset S$.
If we assume that $p$ is locally nilpotent on $S_0$ (which is always
the case in practice) then we obtain a situation as in
Definition \ref{definition-divided-power-thickening-X} with $S_0$ instead
of $X$. Hence we get a site $\text{CRIS}(S_0/S)$. If $f : X \to S_0$ is
the structure morphism of $X$ over $S$, then we get a commutative diagram
of morphisms of ringed topoi
$$
\xymatrix{
(X/S)_{\text{CRIS}}
\ar[r]_{f_{\text{CRIS}}} \ar[d]_{U_{X/S}} &
(S_0/S)_{\text{CRIS}} \ar[d]^{U_{S_0/S}} \\
\Sh((\Sch/X)_{Zar}) \ar[r]^{f_{big}} & \Sh((\Sch/S_0)_{Zar}) \ar[rd] \\
& & \Sh((\Sch/S)_{Zar})
}
$$
by Remark \ref{remark-functoriality-big-cris}. We think of the composition
$(X/S)_{\text{CRIS}} \to \Sh((\Sch/S)_{Zar})$ as the structure morphism of
the big crystalline site. Even if $p$ is not locally nilpotent on $S_0$
the structure morphism
$$
(X/S)_{\text{CRIS}} \longrightarrow \Sh((\Sch/S)_{Zar})
$$
is defined as we can take the lower route through the diagram above. Thus it
is the morphism of topoi corresponding to the cocontinuous
functor $\text{CRIS}(X/S) \to (\Sch/S)_{Zar}$ given by the rule
$(U, T, \delta)/S \mapsto T/S$, see
Sites, Section \ref{sites-section-cocontinuous-morphism-topoi}.
\end{remark}




\section{The crystalline site}
\label{section-site}

\noindent
Since (\ref{equation-forget}) commutes with products and fibre
products, we see that looking at those $(U, T, \delta)$ such that
$U \to X$ is an open immersion defines a full
subcategory preserved under fibre products (and more generally
finite nonempty limits). Hence the following
definition makes sense.

\begin{definition}
\label{definition-crystalline-site}
Assumptions as in Definition \ref{definition-divided-power-thickening-X}.
\begin{enumerate}
\item The (small) {\it crystalline site} of $X$ over
$(S, \mathcal{I}, \gamma)$, denoted $\text{Cris}(X/S, \mathcal{I}, \gamma)$
or simply $\text{Cris}(X/S)$ is the full subcategory of $\text{CRIS}(X/S)$
consisting of those $(U, T, \delta)$ in $\text{CRIS}(X/S)$ such that
$U \to X$ is an open immersion. It comes endowed with the Zariski topology.
\item The topos of sheaves on $\text{Cris}(X/S)$ is denoted
$(X/S)_{\text{cris}}$ or sometimes
$(X/S, \mathcal{I}, \gamma)_{\text{cris}}$\footnote{This clashes with
our convention to denote the topos associated to a site $\mathcal{C}$
by $\Sh(\mathcal{C})$.}.
\end{enumerate}
\end{definition}

\noindent
For any $(U, T, \delta)$ in $\text{Cris}(X/S)$ the morphism $U \to X$
defines an object of the small Zariski site $X_{Zar}$ of $X$. Hence
a canonical forgetful functor
\begin{equation}
\label{equation-forget-small}
\text{Cris}(X/S) \longrightarrow X_{Zar},\quad
(U, T, \delta) \longmapsto U
\end{equation}
and a left adjoint
\begin{equation}
\label{equation-endow-trivial-small}
X_{Zar} \longrightarrow \text{Cris}(X/S),\quad
U \longmapsto (U, U, \emptyset)
\end{equation}
which is sometimes useful.

\medskip\noindent
We can compare the small and big crystalline sites, just like
we can compare the small and big Zariski sites of a scheme, see
Topologies, Lemma \ref{topologies-lemma-at-the-bottom}.

\begin{lemma}
\label{lemma-compare-big-small}
Assumptions as in Definition \ref{definition-divided-power-thickening-X}.
The inclusion functor
$$
\text{Cris}(X/S) \to \text{CRIS}(X/S)
$$
commutes with finite nonempty limits, is fully faithful, continuous,
and cocontinuous. There are morphisms of topoi
$$
(X/S)_{\text{cris}} \xrightarrow{i} (X/S)_{\text{CRIS}}
\xrightarrow{\pi} (X/S)_{\text{cris}}
$$
whose composition is the identity and of which the first is induced
by the inclusion functor. Moreover, $\pi_* = i^{-1}$.
\end{lemma}

\begin{proof}
For the first assertion see
Lemma \ref{lemma-divided-power-thickening-fibre-products}.
This gives us a morphism of topoi
$i : (X/S)_{\text{cris}} \to (X/S)_{\text{CRIS}}$ and a left adjoint
$i_!$ such that $i^{-1}i_! = i^{-1}i_* = \text{id}$, see
Sites, Lemmas \ref{sites-lemma-when-shriek},
\ref{sites-lemma-preserve-equalizers}, and
\ref{sites-lemma-back-and-forth}.
We claim that $i_!$ is exact. If this is true, then we can define
$\pi$ by the rules $\pi^{-1} = i_!$ and $\pi_* = i^{-1}$
and everything is clear. To prove the claim, note that we already know
that $i_!$ is right exact and preserves fibre products (see references
given). Hence it suffices to show that $i_! * = *$ where $*$ indicates
the final object in the category of sheaves of sets. 
To see this it suffices to produce a set of objects
$(U_i, T_i, \delta_i)$, $i \in I$ of $\text{Cris}(X/S)$ such that
$$
\coprod\nolimits_{i \in I} h_{(U_i, T_i, \delta_i)} \to *
$$
is surjective in $(X/S)_{\text{CRIS}}$ (details omitted; hint: use that
$\text{Cris}(X/S)$ has products and that the functor
$\text{Cris}(X/S) \to \text{CRIS}(X/S)$ commutes with them).
In the affine case this
follows from Lemma \ref{lemma-set-generators}. We omit the proof
in general.
\end{proof}

\begin{remark}[Functoriality]
\label{remark-functoriality-cris}
Let $p$ be a prime number.
Let $(S, \mathcal{I}, \gamma)$ be a divided power scheme
over $\mathbf{Z}_{(p)}$.
Let $f : X \to Y$ be a morphism of schemes over $S$ such that
$\mathcal{I} \cdot \mathcal{O}_X = 0$,
$\mathcal{I} \cdot \mathcal{O}_Y = 0$, and $p$ is locally nilpotent
on $X$ and $Y$. By analogy with
Topologies, Lemma \ref{topologies-lemma-morphism-big-small} we define
$$
f_{\text{cris}} : (X/S)_{\text{cris}} \longrightarrow (Y/S)_{\text{cris}}
$$
by the formula $f_{\text{cris}} = \pi_Y \circ f_{\text{CRIS}} \circ i_X$
where $i_X$ and $\pi_Y$ are as in Lemma \ref{lemma-compare-big-small}
for $X$ and $Y$ and where $f_{\text{CRIS}}$ is as in
Remark \ref{remark-functoriality-big-cris}.
\end{remark}

\begin{remark}[Comparison with Zariski site]
\label{remark-compare-zariski}
Assumptions as in Definition \ref{definition-divided-power-thickening-X}.
The functor (\ref{equation-forget-small}) is continuous, cocontinuous, and
commutes with products and fibred products.
Hence we obtain a morphism of topoi
$$
u_{X/S} : (X/S)_{\text{cris}} \longrightarrow \Sh(X_{Zar})
$$
relating the small crystalline topos of $X/S$ with
the small Zarisk topos of $X$.
See Sites, Section \ref{sites-section-cocontinuous-morphism-topoi}.
\end{remark}

\begin{remark}[Structure morphism]
\label{remark-structure-morphism}
Assumptions as in Definition \ref{definition-divided-power-thickening-X}.
Consider the closed subscheme $S_0 = V(\mathcal{I}) \subset S$.
If we assume that $p$ is locally nilpotent on $S_0$ (which is always
the case in practice) then we obtain a situation as in
Definition \ref{definition-divided-power-thickening-X} with $S_0$ instead
of $X$. Hence we get a site $\text{Cris}(S_0/S)$. If $f : X \to S_0$
is the structure morphism of $X$ over $S$, then we get a
commutative diagram of ringed topoi
$$
\xymatrix{
(X/S)_{\text{cris}}
\ar[r]_{f_{\text{cris}}} \ar[d]_{u_{X/S}} &
(S_0/S)_{\text{cris}} \ar[d]^{u_{S_0/S}} \\
\Sh(X_{Zar}) \ar[r]^{f_{small}} & \Sh(S_{0, Zar}) \ar[rd] \\
& & \Sh(S_{Zar})
}
$$
see Remark \ref{remark-functoriality-cris}. We think of the compostion
$(X/S)_{\text{cris}} \to \Sh(S_{Zar})$ as the structure morphism of the
crystalline site. Even if $p$ is not locally nilpotent on $S_0$
the structure morphism
$$
(X/S)_{\text{cris}} \longrightarrow \Sh(S_{Zar})
$$
is defined as we can take the lower route through the diagram above.
\end{remark}




\section{Sheaves on the crystalline site}
\label{section-sheaves}

\noindent
Let $p$, $(S, \mathcal{I}, \gamma)$, and $X \to S$ be as in
Definition \ref{definition-divided-power-thickening-X}.
In order to discuss the small and big crystalline sites of $X/S$
simultaneously in this section we let
$$
\mathcal{C} = \text{CRIS}(X/S)
\quad\text{or}\quad
\mathcal{C} = \text{Cris}(X/S).
$$
A sheaf $\mathcal{F}$ on $\mathcal{C}$ gives rise to
a {\it restriction} $\mathcal{F}_T$ for every object $(U, T, \delta)$
of $\mathcal{C}$. Namely, $\mathcal{F}_T$ is the Zariski sheaf on
the scheme $T$ defined by the rule
$$
\mathcal{F}_T(W) = \mathcal{F}(U \cap W, W, \delta|_W)
$$
for $W \subset T$ is open. Moreover, if $f : T \to T'$ is a morphism
between objects
$(U, T, \delta)$ and $(U', T', \delta')$ of $\mathcal{C}$, then there
is a canonical {\it comparison} map
\begin{equation}
\label{equation-comparison}
c_f : f^{-1}\mathcal{F}_{T'} \longrightarrow \mathcal{F}_T.
\end{equation}
Namely, if $W' \subset T'$ is open then $f$ induces a morphism
$$
f|_{f^{-1}W} :
(U \cap f^{-1}(W'), f^{-1}W', \delta|_{f^{-1}W'})
\longrightarrow
(U' \cap W', W', \delta|_{W'})
$$
of $\mathcal{C}$, hence we can use the restriction mapping
$(f|_{f^{-1}W'})^*$ of $\mathcal{F}$ to define a map
$\mathcal{F}_{T'}(W') \to \mathcal{F}_T(f^{-1}W')$.
These maps are clearly compatible with further restriction, hence
define an $f$-map from $\mathcal{F}_{T'}$ to $\mathcal{F}_T$ (see
Sheaves, Section \ref{sheaves-section-presheaves-functorial}
and especially
Sheaves, Definition \ref{sheaves-definition-f-map}).
Thus a map $c_f$ as in (\ref{equation-comparison}).
Note that if $f$ is an open immersion, then $c_f$ is an
isomorphism, because in that case $\mathcal{F}_T$ is just
the restriction of $\mathcal{F}_{T'}$ to $T$.

\medskip\noindent
Conversely, given Zariski sheaves $\mathcal{F}_T$ for every object
$(U, T, \delta)$ of $\mathcal{C}$ and comparion maps
$c_f$ as above which (a) are isomorphisms for open immersions, and (b)
satisfy a suitable cocycle condition, we obtain a sheaf on
$\mathcal{C}$. This is proved exactly as in
Topologies, Lemma \ref{topologies-lemma-characterize-sheaf-big}.

\medskip\noindent
The {\it structure sheaf} on $\mathcal{C}$ is the sheaf
$\mathcal{O}_{X/S}$ defined by the rule
$$
\mathcal{O}_{X/S} :
(U, T, \delta)
\longmapsto
\Gamma(T, \mathcal{O}_T)
$$
This is a sheaf by the definition of coverings in $\mathcal{C}$.
Suppose that $\mathcal{F}$ is a sheaf of $\mathcal{O}_{X/S}$-modules.
In this case the comparison mappings (\ref{equation-comparison})
define a comparison map
\begin{equation}
\label{equation-comparison-modules}
c_f : f^*\mathcal{F}_T \longrightarrow \mathcal{F}_{T'}
\end{equation}
of $\mathcal{O}_T$-modules.

\medskip\noindent
Another type of example comes by starting with a sheaf
$\mathcal{G}$ on $(\Sch/X)_{Zar}$ or $X_{Zar}$ (depending on whether
$\mathcal{C} = \text{CRIS}(X/S)$ or $\mathcal{C} = \text{Cris}(X/S)$).
Then $\underline{\mathcal{G}}$ defined by the rule
$$
\underline{\mathcal{G}} :
(U, T, \delta)
\longmapsto
\mathcal{G}(U)
$$
is a sheaf on $\mathcal{C}$. In particular, if we take
$\mathcal{G} = \mathbf{G}_a = \mathcal{O}_X$, then we obtain
$$
\underline{\mathbf{G}_a} :
(U, T, \delta)
\longmapsto
\Gamma(U, \mathcal{O}_U)
$$
There is a surjective map of sheaves
$\mathcal{O}_{X/S} \to \underline{\mathbf{G}_a}$ defined by the
canonical maps $\Gamma(T, \mathcal{O}_T) \to \Gamma(U, \mathcal{O}_U)$
for objects $(U, T, \delta)$. The kernel of this map is denoted
$\mathcal{J}_{X/S}$, hence a short exact sequence
$$
0 \to
\mathcal{J}_{X/S} \to
\mathcal{O}_{X/S} \to
\underline{\mathbf{G}_a} \to 0
$$
Note that $\mathcal{J}_{X/S}$ comes equipped with a canonical
divided power structure. After all, for each object $(U, T, \delta)$
the third component $\delta$ {\it is} a divided power structure on the
kernel of $\mathcal{O}_T \to \mathcal{O}_U$. Hence the (big)
crystalline topos is a divided power topos.





\section{Crystals in modules}
\label{section-crystals}

\noindent
It turns out that a crystal is a very general gadget. However, the
definition may be a bit hard to parse, so we first give the definition
in the case of modules on the crystalline sites.

\begin{definition}
\label{definition-modules}
Let $p$, $(S, \mathcal{I}, \gamma)$, and $X \to S$ be as in
Definition \ref{definition-divided-power-thickening-X}.
Let $\mathcal{C} = \text{CRIS}(X/S)$ or $\mathcal{C} = \text{Cris}(X/S)$.
Let $\mathcal{F}$ be a sheaf of $\mathcal{O}_{X/S}$-modules on $\mathcal{C}$.
\begin{enumerate}
\item We say $\mathcal{F}$ is {\it locally quasi-coherent} if for every
object $(U, T, \delta)$ of $\mathcal{C}$ the restriction $\mathcal{F}_T$
is a quasi-coherent $\mathcal{O}_T$-module.
\item We say $\mathcal{F}$ is {\it quasi-coherent} if it is quasi-coherent
in the sense of
Modules on Sites, Definition \ref{sites-modules-definition-site-local}.
\item We say $\mathcal{F}$ is a {\it crystal in $\mathcal{O}_{X/S}$-modules}
if all the comparison maps (\ref{equation-comparison-modules}) are
isomorphisms.
\end{enumerate}
\end{definition}

\noindent
It turns out that we can relate these notions as follows.

\begin{lemma}
\label{lemma-crystal-quasi-coherent-modules}
With notation $X/S, \mathcal{I}, \gamma, \mathcal{C}, \mathcal{F}$
as in Definition \ref{definition-modules}. The following are equivalent
\begin{enumerate}
\item $\mathcal{F}$ is quasi-coherent, and
\item $\mathcal{F}$ is locally quasi-coherent and a crystal of
$\mathcal{O}_{X/S}$-modules.
\end{enumerate}
\end{lemma}

\begin{proof}
Omitted.
\end{proof}

\begin{definition}
\label{definition-crystal-quasi-coherent-modules}
If $\mathcal{F}$ satisfies the equivalent conditions of
Lemma \ref{lemma-crystal-quasi-coherent-modules}, then
we say that $\mathcal{F}$ is a
{\it crystal in quasi-coherent modules}.
We say that $\mathcal{F}$ is a {\it cristal in finite locally free modules}
if, in addition, $\mathcal{F}$ is finite locally free.
\end{definition}

\noindent
Of course, as Lemma \ref{lemma-crystal-quasi-coherent-modules} shows, this
notation is somewhat heavy since a quasi-coherent module is always a crystal.
But it is standard terminology in the literature.

\begin{remark}
\label{remark-crystal}
To formulate the general notion of a crystal we use the language
of stacks and strongly cartesian morphisms, see
Stacks, Definition \ref{stacks-definition-stack} and
Categories, Definition \ref{categories-definition-cartesian-over-C}.
Let $p$, $(S, \mathcal{I}, \gamma)$, and $X \to S$ be as in
Definition \ref{definition-divided-power-thickening-X}.
Let $p : \mathcal{C} \to \text{Cris}(X/S)$ be a stack.
A {\it crystal in objects of $\mathcal{C}$ on $X$ relative to $S$}
is a {\it cartesian section} $\sigma : \text{Cris}(X/S) \to \mathcal{C}$,
i.e., a functor $\sigma$ such that $p \circ \sigma = \text{id}$
and such that $\sigma(f)$ is strongly cartesian for all
morphisms $f$ of $\text{Cris}(X/S)$. Similarly for the big crystalline site.
\end{remark}





\section{Sheaf of differentials}
\label{section-differentials-sheaf}

\noindent
In this section we will stick with the (small) crystalline site
as it seems more natural. We globalize
Definition \ref{definition-derivation} as follows.

\begin{definition}
\label{definition-global-derivation}
Assumptions as in Definition \ref{definition-divided-power-thickening-X}.
Let $\mathcal{F}$ be a sheaf of $\mathcal{O}_{X/S}$-modules on
$\text{Cris}(X/S)$. An
{\it $S$-derivation $D : \mathcal{O}_{X/S} \to \mathcal{F}$}
is a map of sheaves such that for every object $(U, T, \delta)$ of
$\text{Cris}(X/S)$ the map
$$
D : \Gamma(T, \mathcal{O}_T) \longrightarrow \Gamma(T, \mathcal{F})
$$
is a divided power $\Gamma(V, \mathcal{O}_V)$-derivation where $V \subset S$
is any open such that $T \to S$ factors through $V$.
\end{definition}

\noindent
This means that $D$ is additive, satisfies the Leibniz rule, annihilates
functions coming from $S$, and satisfies $D(f^{[n]}) = f^{[n - 1]}D(f)$
for a local section $f$ of the divided power ideal $\mathcal{J}_{X/S}$.
This is a special case of a very general notion which we now describe.

\medskip\noindent
Please compare the following discussion with
Modules on Sites, Section \ref{sites-modules-section-differentials}. Let
$\mathcal{C}$ be a site, let $\mathcal{A} \to \mathcal{B}$ be a
map of sheaves of rings on $\mathcal{C}$, let $\mathcal{J} \subset \mathcal{B}$
be a sheaf of ideals, let $\delta$ be a divided power structure on
$\mathcal{J}$, and let $\mathcal{F}$ be a sheaf of $\mathcal{B}$-modules.
Then there is a notion of a {\it divided power $\mathcal{A}$-derivation}
$D : \mathcal{B} \to \mathcal{F}$. This means that $D$ is $\mathcal{A}$-linear,
satisfies the Leibnize rule, and satisfies
$D(\delta_n(x)) = \delta_{n - 1}(x)D(x)$ for local sections $x$ of
$\mathcal{J}$. In this situation there exists a
{\it universal divided power $\mathcal{A}$-derivation}
$$
\text{d}_{\mathcal{B}/\mathcal{A}, \delta} :
\mathcal{B}
\longrightarrow
\Omega_{\mathcal{B}/\mathcal{A}, \delta}
$$
Moreover, $\text{d}_{\mathcal{B}/\mathcal{A}, \delta}$ is the compostion
$$
\mathcal{B}
\longrightarrow
\Omega_{\mathcal{B}/\mathcal{A}}
\longrightarrow
\Omega_{\mathcal{B}/\mathcal{A}, \delta}
$$
where the first map is the universal derivation constructed in the proof
of Modules on Sites, Lemma \ref{sites-modules-lemma-universal-module}
and the second arrow is the quotient by the submodule generated by
the local sections
$\text{d}_{\mathcal{B}/\mathcal{A}}(\delta_n(x)) -
\delta_{n - 1}(x)\text{d}_{\mathcal{B}/\mathcal{A}}(x)$.

\medskip\noindent
We translate this into a relative notion as follows. Suppose
$(f, f^\sharp) : (\Sh(\mathcal{C}), \mathcal{O}) \to
(\Sh(\mathcal{C}'), \mathcal{O}')$ is a morphism of ringed topoi,
$\mathcal{J} \subset \mathcal{O}$ a sheaf of ideals, $\delta$ a
divided power structure on $\mathcal{J}$, and $\mathcal{F}$ a sheaf
of $\mathcal{O}$-modules. In this situation we say
$D : \mathcal{O} \to \mathcal{F}$ is a divided power $\mathcal{O}'$-derivation
if $D$ is a divided power $f^{-1}\mathcal{O}'$-derivation as defined above.
Moreover, we write
$$
\Omega_{\mathcal{O}/\mathcal{O}', \delta} =
\Omega_{\mathcal{O}/f^{-1}\mathcal{O}', \delta}
$$
which is the receptacle of the universal divided power
$\mathcal{O}'$-derivation.

\medskip\noindent
Appying this to the structure morphism
$$
(X/S)_{\text{Cris}} \longrightarrow \Sh(S_{Zar})
$$
(see Remark \ref{remark-structure-morphism}) we recover the notion of
Definition \ref{definition-global-derivation} above.
In particular, there is a universal divided power derivation
$$
d_{X/S} : \mathcal{O}_{X/S} \to \Omega_{X/S}
$$
Note that we omit from the notation the decoration indicating the
module of differentials is compatible with divided powers (it seems
unlikely anybody would ever consider the usual module of differentials
of the structure sheaf on the crystalline site).

\begin{lemma}
\label{lemma-module-differentials-divided-power-scheme}
Let $(T, \mathcal{J}, \delta)$ be a divided power scheme.
Let $T \to S$ be a morphism of schemes.
The quotient $\Omega_{T/S} \to \Omega_{T/S, \delta}$
described above is a quasi-coherent $\mathcal{O}_T$-module.
For $W \subset T$ affine open mapping into $V \subset S$ affine open
we have
$$
\Gamma(W, \Omega_{T/S, \delta}) =
\Omega_{\Gamma(W, \mathcal{O})/\Gamma(V, \mathcal{O}_V), \delta}
$$
where the right hand side is
as constructed in Section \ref{section-differentials}.
\end{lemma}

\begin{proof}
Omitted.
\end{proof}

\begin{lemma}
\label{lemma-module-of-differentials}
Assumptions as in Definition \ref{definition-divided-power-thickening-X}.
For $(U, T, \delta)$ in $\text{Cris}(X/S)$ the restriction
$(\Omega_{X/S})_T$ to $T$ is $\Omega_{T/S, \delta}$ and the restriction
$\text{d}_{X/S}|_T$ is equal to $\text{d}_{T/S, \delta}$.
\end{lemma}

\begin{proof}
Omitted.
\end{proof}

\begin{lemma}
\label{lemma-module-of-differentials-on-affine}
Assumptions as in Definition \ref{definition-divided-power-thickening-X}.
For any affine object $(U, T, \delta)$ of $\text{Cris}(X/S)$
mapping into an affine open $V \subset S$ we have
$$
\Gamma((U, T, \delta), \Omega_{X/S}) =
\Omega_{\Gamma(T, \mathcal{O})/\Gamma(V, \mathcal{O}_V), \delta}
$$
where the right hand side is
as constructed in Section \ref{section-differentials}.
\end{lemma}

\begin{proof}
Combine Lemmas \ref{lemma-module-differentials-divided-power-scheme} and
\ref{lemma-module-of-differentials}.
\end{proof}

\begin{lemma}
\label{lemma-describe-omega-small}
Assumptions as in Definition \ref{definition-divided-power-thickening-X}.
Let $(U, T, \delta)$ be an object of $\text{Cris}(X/S)$.
Let
$$
(U(1), T(1), \delta(1)) = (U, T, \delta) \times (U, T, \delta)
$$
in $\text{Cris}(X/S)$. Let $\mathcal{K} \subset \mathcal{O}_{T(1)}$
be the quasi-coherent sheaf of ideals corresponding to the closed
immersion $\Delta : T \to T(1)$. Then
$\mathcal{K} \subset \mathcal{J}_{T(1)}$ is preserved by the
divided structure on $\mathcal{J}_{T(1)}$ and we have
$$
(\Omega_{X/S})_T = \mathcal{K}/\mathcal{K}^{[2]}
$$
\end{lemma}

\begin{proof}
Note that $U = U(1)$ as $U \to X$ is an open immersion and as
(\ref{equation-forget-small}) commutes with products. Hence we see that
$\mathcal{K} \subset \mathcal{J}_{T(1)}$. Given this fact the lemma follows
by working affine locally on $T$ and using
Lemmas \ref{lemma-module-of-differentials-on-affine} and
\ref{lemma-diagonal-and-differentials-affine-site}.
\end{proof}

\begin{remark}
\label{remark-first-order-thickening}
Assumptions as in Definition \ref{definition-divided-power-thickening-X}.
Let $(U, T, \delta)$ be an object of $\text{Cris}(X/S)$.
Write $\Omega_{T/S, \delta} = (\Omega_{X/S})_T$, see
Lemma \ref{lemma-module-of-differentials}.
We explicitly describe a first order thickening $T'$ of
$T$. Namely, set
$$
\mathcal{O}_{T'} = \mathcal{O}_T \oplus \Omega_{T/S, \delta}
$$
with algebra structure such that $\Omega_{T/S, \delta}$ is an
ideal of square zero. Let $\mathcal{J} \subset \mathcal{O}_T$
be the ideal sheaf of the closed immersion $U \to T$. Set
$\mathcal{J}' = \mathcal{J} \oplus \Omega_{T/S, \delta}$.
Define a divided power structure on $\mathcal{J}'$ by setting
$$
\delta_n'(f, \omega) = (\delta_n(f), \delta_{n - 1}(f)\omega),
$$
see Lemma \ref{lemma-divided-power-first-order-thickening}.
There are two ring maps
$$
p_0, p_1 : \mathcal{O}_T \to \mathcal{O}_{T'}
$$
The first is given by $f \mapsto (f, 0)$ and the second by
$f \mapsto (f, \text{d}_{T/S, \delta}f)$. Note that both are compatible
with the divided power structures on $\mathcal{J}$ and $\mathcal{J}'$
and so is the quotient map $\mathcal{O}_{T'} \to \mathcal{O}_T$.
Thus we get an object $(U, T', \delta')$ of $\text{Cris}(X/S)$
and a commutative diagram
$$
\xymatrix{
& T \ar[ld]_{\text{id}} \ar[d]^i \ar[rd]^{\text{id}} \\
T & T' \ar[l]_{p_0} \ar[r]^{p_1} & T
}
$$
of $\text{Cris}(X/S)$ such that $i$ is a first order thickening whose ideal
sheaf is identified with $\Omega_{T/S, \delta}$ and such that
$p_1^* - p_0^* : \mathcal{O}_T \to \mathcal{O}_{T'}$
is identified with the universal derivation $\text{d}_{T/S, \delta}$
composed with the inclusion $\Omega_{T/S, \delta} \to \mathcal{O}_{T'}$.
\end{remark}

\begin{remark}
\label{remark-second-order-thickening}
Assumptions as in
Definition \ref{definition-divided-power-thickening-X}.
Let $(U, T, \delta)$ be an object of $\text{Cris}(X/S)$.
Write $\Omega_{T/S, \delta} = (\Omega_{X/S})_T$, see
Lemma \ref{lemma-module-of-differentials}.
We also write $\Omega^2_{T/S, \delta}$ for its second exterior
power. We explicitly describe a second order thickening $T''$ of $T$.
Namely, set
$$
\mathcal{O}_{T''} =
\mathcal{O}_T \oplus \Omega_{T/S, \delta} \oplus \Omega_{T/S, \delta}
\oplus \Omega^2_{T/S, \delta}
$$
with algebra structure defined in the following way
$$
(f, \omega_1, \omega_2, \eta) \cdot
(f', \omega_1', \omega_2', \eta') =
(ff', f\omega_1' + f'\omega_1, f\omega_2' + f'\omega_2',
f\eta' + f'\eta + \omega_1 \wedge \omega_2' + \omega_1' \wedge \omega_2).
$$
Let $\mathcal{J} \subset \mathcal{O}_T$
be the ideal sheaf of the closed immersion $U \to T$. Let
$\mathcal{J}''$ be the inverse image of $\mathcal{J}$ under the
projection $\mathcal{O}_{T''} \to \mathcal{O}_T$.
Define a divided power structure on $\mathcal{J}''$ by setting
$$
\delta_n''(f, \omega_1, \omega_2, \eta) =
(\delta_n(f), \delta_{n - 1}(f)\omega_1, \delta_{n - 1}(f)\omega_2,
\delta_{n - 1}(f)\eta + \delta_{n - 2}(f)\omega_1 \wedge \omega_2)
$$
see Lemma \ref{lemma-divided-power-second-order-thickening}.
There are three ring maps
$q_0, q_1, q_2 : \mathcal{O}_T \to \mathcal{O}_{T''}$
given by
\begin{align*}
q_0(f) & = (f, 0, 0, 0), \\
q_1(f) & = (f, \text{d}f, 0, 0), \\
q_2(f) & = (f, \text{d}f, \text{d}f, 0)
\end{align*}
where $\text{d} = \text{d}_{T/S, \delta}$.
Note that all three are compatible with the divided power structures
on $\mathcal{J}$ and $\mathcal{J}''$. There are three ring maps
$q_{01}, q_{12}, q_{02} : \mathcal{O}_{T'} \to \mathcal{O}_{T''}$
where $\mathcal{O}_{T'}$ is as in Remark \ref{remark-first-order-thickening}.
Namely, set
\begin{align*}
q_{01}(f, \omega) & = (f, \omega, 0, 0), \\
q_{12}(f, \omega) & =
(f, \text{d}f, \omega, \text{d}\omega), \\
q_{02}(f, \omega) & = (f, \omega, \omega, 0)
\end{align*}
These are also compatible with the given divided power
structures. Let's do the verifications for $q_{12}$: Note
that $q_{12}$ is a ring homomorphism as
\begin{align*}
q_{12}(f, \omega)q_{12}(g, \eta) & =
(f, \text{d}f, \omega, \text{d}\omega)(g, \text{d}g, \eta, \text{d}\eta) \\
& =
(fg, f\text{d}g + g \text{d}f, f\eta + g\omega,
f\text{d}\eta + g\text{d}\omega + \text{d}f \wedge \eta +
\text{d}g \wedge \omega) \\
& = q_{12}(fg, f\eta + g\omega) = q_{12}((f, \omega)(g, \eta))
\end{align*}
Note that $q_{12}$ is compatible with divided powers because
\begin{align*}
\delta_n''(q_{12}(f, \omega)) & =
\delta_n''((f, \text{d}f, \omega, \text{d}\omega)) \\
& =
(\delta_n(f), \delta_{n - 1}(f)\text{d}f, \delta_{n - 1}(f)\omega,
\delta_{n - 1}(f)\text{d}\omega + \delta_{n - 2}(f)\text{d}(f) \wedge \omega)
\\
& = q_{12}((\delta_n(f), \delta_{n - 1}(f)\omega)) =
q_{12}(\delta'_n(f, \omega))
\end{align*}
The verifications for $q_{01}$ and $q_{02}$ are easier.
Note that $q_0 = q_{01} \circ p_0$, $q_1 = q_{01} \circ p_1$,
$q_1 = q_{12} \circ p_0$, $q_2 = q_{12} \circ p_1$,
$q_0 = q_{02} \circ p_0$, and $q_2 = q_{02} \circ p_1$.
Thus $(U, T'', \delta'')$ is an object of $\text{Cris}(X/S)$
and we get morphisms
$$
\xymatrix{
T''
\ar@<2ex>[r]
\ar@<0ex>[r]
\ar@<-2ex>[r]
&
T'
\ar@<1ex>[r]
\ar@<-1ex>[r]
&
T
}
$$
of $\text{Cris}(X/S)$ satisfying the relations described above.
In applications we will use $q_i : T'' \to T$ and
$q_{ij} : T'' \to T'$ to denote the morphisms associated to the
ring maps described above.
\end{remark}






\section{The de Rham complex}
\label{section-de-Rham}

\noindent
Assumptions as in Definition \ref{definition-divided-power-thickening-X}.
Working on the crystalline site, we define
$\Omega^i_{X/S} = \wedge^i_{\mathcal{O}_{X/S}} \Omega_{X/S}$
for $i \geq 0$. The universal $S$-derivation $\text{d}_{X/S}$ gives
rise to the {\it de Rham complex}
$$
\mathcal{O}_{X/S} \to \Omega^1_{X/S} \to \Omega^2_{X/S} \to \ldots
$$
on $\text{Cris}(X/S)$, see
Lemma \ref{lemma-module-of-differentials-on-affine} and
Remark \ref{remark-divided-powers-de-rham-complex}.


\section{Connections}
\label{section-connections}

\noindent
Assumptions as in Definition \ref{definition-divided-power-thickening-X}.
Given an $\mathcal{O}_{X/S}$-module $\mathcal{F}$ on $\text{Cris}(X/S)$
a {\it connection} is a map of abelian sheaves
$$
\nabla :
\mathcal{F}
\longrightarrow
\mathcal{F} \otimes_{\mathcal{O}_{X/S}} \Omega_{X/S}
$$
such that $\nabla(f s) = f\nabla(s) + s \otimes \text{d}f$
for local sections $s, f$ of $\mathcal{F}$ and $\mathcal{O}_{X/S}$.
Given a connection there are canonical maps
$
\nabla :
\mathcal{F} \otimes_{\mathcal{O}_{X/S}} \Omega^i_{X/S}
\longrightarrow
\mathcal{F} \otimes_{\mathcal{O}_{X/S}} \Omega^{i + 1}_{X/S}
$
defined by the rule $\nabla(s \otimes \omega) =
\nabla(s) \wedge \omega + s \otimes \text{d}\omega$
as in Remark \ref{remark-connection}. We say the connection is
{\it integrable} if $\nabla \circ \nabla = 0$. If $\nabla$ is integrable
we obtain the {\it de Rham complex}
$$
\mathcal{F} \to
\mathcal{F} \otimes_{\mathcal{O}_{X/S}} \Omega^1_{X/S} \to
\mathcal{F} \otimes_{\mathcal{O}_{X/S}} \Omega^2_{X/S} \to \ldots
$$
on $\text{Cris}(X/S)$. It turns out that any crystal in
$\mathcal{O}_{X/S}$-modules comes equipped with a canonical
integrable connection.

\begin{lemma}
\label{lemma-automatic-connection}
Assumptions as in Definition \ref{definition-divided-power-thickening-X}.
Let $\mathcal{F}$ be a crystal of $\mathcal{O}_{X/S}$-modules
on $\text{Cris}(X/S)$. Then $\mathcal{F}$ comes equipped with a
canonical integrable connection.
\end{lemma}

\begin{proof}
Say $(U, T, \delta)$ is an object of $\text{Cris}(X/S)$.
Let $(U, T', \delta')$ be the infinitesimal thickening of $T$
by $(\Omega_{X/S})_T = \Omega_{T/S, \delta}$
constructed in Remark \ref{remark-first-order-thickening}.
It comes with projections $p_0, p_1 : T' \to T$
and a diagonal $i : T \to T(1)$. By assumption we get
isomorphisms
$$
p_0^*\mathcal{F}_T \xrightarrow{c_0}
\mathcal{F}_{T'} \xleftarrow{c_1}
p_1^*\mathcal{F}_T
$$
of $\mathcal{O}_{T'}$-modules. Pulling $c = c_1^{-1} \circ c_0$
back to $T$ by $i$ we obtain the identity map
of $\mathcal{F}_T$. Hence if $s \in \Gamma(T, \mathcal{F}_T)$
then $\nabla(s) = p_1^*s - c(p_0^*s)$ is a section of
$p_1^*\mathcal{F}_T$ which vanishes on pulling back by $\Delta$. Hence
$\nabla(s)$ is a section of
$$
\mathcal{F}_T
\otimes_{\mathcal{O}_T}
\Omega_{T/S, \delta}
$$
because this is the kernel of $p_1^*\mathcal{F}_T \to \mathcal{F}_T$
as $\Omega_{T/S, \delta}$ is the kernel of
$\mathcal{O}_{T'} \to \mathcal{O}_T$ by construction.

\medskip\noindent
The collection of maps
$$
\nabla : \Gamma(T, \mathcal{F}_T) \to
\Gamma(T, \mathcal{F}_T \otimes_{\mathcal{O}_T} \Omega_{T/S, \delta})
$$
so obtained is functorial in $T$ because the construction of $T'$
is functorial in $T$. Hence we obtain a connection.

\medskip\noindent
To show that the connection is integrable we consider the
object $(U, T'', \delta'')$ constructed in
Remark \ref{remark-second-order-thickening}.
Because $\mathcal{F}$ is a sheaf we see that
$$
\xymatrix{
q_0^*\mathcal{F}_T \ar[rr]_{q_{01}^*c} \ar[rd]_{q_{02}^*c} & &
q_1^*\mathcal{F}_T \ar[ld]^{q_{12}^*c} \\
& q_2^*\mathcal{F}_T
}
$$
is a commutative map of $\mathcal{O}_{T''}$-modules. For
$s \in \Gamma(T, \mathcal{F}_T)$ we have
$c(p_0^*s) = p_1^*s - \nabla(s)$. Write
$\nabla(s) = \sum p_1^*s_i \cdot \omega_i$ where $s_i$ is a local section
of $\mathcal{F}_T$ and $\omega_i$ is a local section of $\Omega_{T/S, \delta}$.
We think of $\omega_i$ as a local section of the structure
sheaf of $\mathcal{O}_{T'}$ and hence we write product instead of tensor
product. On the one hand
\begin{align*}
q_{12}^*c \circ q_{01}^*c(q_0^*s) & = 
q_{12}^*c(q_1^*s - \sum q_1^*s_i \cdot q_{01}^*\omega_i) \\
& =
q_2^*s - \sum q_2^*s_i \cdot q_{12}^*\omega_i -
\sum q_2^*s_i \cdot q_{01}^*\omega_i +
\sum q_{12}^*\nabla(s_i) \cdot q_{01}^*\omega_i
\end{align*}
and on the other hand
$$
q_{02}^*c(q_0^*s) = q_2^*s - \sum q_2^*s_i \cdot q_{02}^*\omega_i.
$$
From the formulae of Remark \ref{remark-second-order-thickening} we see
that
$q_{01}^*\omega_i + q_{12}^*\omega_i - q_{02}^*\omega_i = \text{d}\omega_i$.
Hence the difference of the two expressions above is
$$
\sum q_2^*s_i \cdot \text{d}\omega_i -
\sum q_{12}^*\nabla(s_i) \cdot q_{01}^*\omega_i
$$
Note that
$q_{12}^*\omega \cdot q_{01}^*\omega' = \omega' \wedge \omega =
- \omega \wedge \omega'$ by the definition of the multiplication on
$\mathcal{O}_{T''}$. Thus the expression above is $\nabla^2(s)$ viewed
as a section of the subsheaf $\mathcal{F}_T \otimes \Omega^2_{T/S, \delta}$ of
$q_2^*\mathcal{F}$. Hence we get the integrability condition.
\end{proof}






\section{Crystals in quasi-coherent modules}
\label{section-quasi-coherent-crystals}

\noindent
Let $p$ be a prime number. Let $(A, I, \gamma)$ be a divided power ring
where $A$ is a $\mathbf{Z}_{(p)}$-algebra. Let $A/I \to C$ be a ring map.
Assume that $p$ is nilpotent in $C$.
Choose a polynomial ring
$P$ over $A$ and a surjection $P \to B$ of $A$-algebras. Denote
$J = \text{Ker}(P \to A)$ so that $B = P/J$. Let
$$
D_{P, \gamma}(J) = (D, \bar J, \bar \gamma)
$$
be the divided power envelope of $J$ in $P$ relative to $(A, I, \gamma)$.
For $n \geq 1$ set $D_n = D/p^nD$. Finally, let
$$
D^\wedge = \lim D_n
$$
be the $p$-adic completion of $D$. Note that $\Omega_{D/A, \bar \gamma}$

\begin{lemma}
\label{lemma-crystals-on-affine}
Set $S = \Spec(A)$ and $X = \Spec(C)$. Then there is an equivalence
of categories between the category of crystals of quasi-coherent
$\mathcal{O}_{X/S}$-modules and pairs $(M, \nabla)$ where $M$ is
a $p$-adically complete $D^\wedge$-module and a topologically
quasi-nilpotent integrable connection
$\nabla : M \to M \otimes^\wedge_{D^\wedge}\Omega_{D^\wedge/A}$.
\end{lemma}

\begin{proof}
Coming soon to a store near you!
\end{proof}










\section{Cohomology of crystals}
\label{section-cohomology}

\noindent
In this section we compare crystalline cohomology with de Rham
cohomology. We follow \cite{Bhatt}.



\medskip\noindent
THE FOLLOWING ARE STILL MISSING.
\begin{enumerate}
\item Prove Lemma \ref{lemma-crystal-quasi-coherent-modules}.
\item Discuss the equivalence of the category of crystals with the
category of modules with integrable, topologically quasi-nilpotent
connection in the affine case. (Uses the divided power envelope for the
first time...?)
\item The main theorem in the affine case.
\item The main theorem in the global case.
\end{enumerate}







\section{Other chapters}

\begin{multicols}{2}
\begin{enumerate}
\item \hyperref[introduction-section-phantom]{Introduction}
\item \hyperref[conventions-section-phantom]{Conventions}
\item \hyperref[sets-section-phantom]{Set Theory}
\item \hyperref[categories-section-phantom]{Categories}
\item \hyperref[topology-section-phantom]{Topology}
\item \hyperref[sheaves-section-phantom]{Sheaves on Spaces}
\item \hyperref[algebra-section-phantom]{Commutative Algebra}
\item \hyperref[sites-section-phantom]{Sites and Sheaves}
\item \hyperref[homology-section-phantom]{Homological Algebra}
\item \hyperref[derived-section-phantom]{Derived Categories}
\item \hyperref[more-algebra-section-phantom]{More Algebra}
\item \hyperref[simplicial-section-phantom]{Simplicial Methods}
\item \hyperref[modules-section-phantom]{Sheaves of Modules}
\item \hyperref[sites-modules-section-phantom]{Modules on Sites}
\item \hyperref[injectives-section-phantom]{Injectives}
\item \hyperref[cohomology-section-phantom]{Cohomology of Sheaves}
\item \hyperref[sites-cohomology-section-phantom]{Cohomology on Sites}
\item \hyperref[hypercovering-section-phantom]{Hypercoverings}
\item \hyperref[schemes-section-phantom]{Schemes}
\item \hyperref[constructions-section-phantom]{Constructions of Schemes}
\item \hyperref[properties-section-phantom]{Properties of Schemes}
\item \hyperref[morphisms-section-phantom]{Morphisms of Schemes}
\item \hyperref[coherent-section-phantom]{Coherent Cohomology}
\item \hyperref[divisors-section-phantom]{Divisors}
\item \hyperref[limits-section-phantom]{Limits of Schemes}
\item \hyperref[varieties-section-phantom]{Varieties}
\item \hyperref[chow-section-phantom]{Chow Homology}
\item \hyperref[topologies-section-phantom]{Topologies on Schemes}
\item \hyperref[descent-section-phantom]{Descent}
\item \hyperref[more-morphisms-section-phantom]{More on Morphisms}
\item \hyperref[flat-section-phantom]{More on Flatness}
\item \hyperref[groupoids-section-phantom]{Groupoid Schemes}
\item \hyperref[more-groupoids-section-phantom]{More on Groupoid Schemes}
\item \hyperref[etale-section-phantom]{\'Etale Morphisms of Schemes}
\item \hyperref[etale-cohomology-section-phantom]{\'Etale Cohomology}
\item \hyperref[spaces-section-phantom]{Algebraic Spaces}
\item \hyperref[spaces-properties-section-phantom]{Properties of Algebraic Spaces}
\item \hyperref[spaces-morphisms-section-phantom]{Morphisms of Algebraic Spaces}
\item \hyperref[spaces-topologies-section-phantom]{Topologies on Algebraic Spaces}
\item \hyperref[spaces-descent-section-phantom]{Descent and Algebraic Spaces}
\item \hyperref[spaces-more-morphisms-section-phantom]{More on Morphisms of Spaces}
\item \hyperref[quot-section-phantom]{Quot and Hilbert Spaces}
\item \hyperref[stacks-section-phantom]{Stacks}
\item \hyperref[spaces-groupoids-section-phantom]{Groupoids in Algebraic Spaces}
\item \hyperref[spaces-more-groupoids-section-phantom]{More on Groupoids in Spaces}
\item \hyperref[bootstrap-section-phantom]{Bootstrap}
\item \hyperref[examples-stacks-section-phantom]{Examples of Stacks}
\item \hyperref[groupoids-quotients-section-phantom]{Quotients of Groupoids}
\item \hyperref[algebraic-section-phantom]{Algebraic Stacks}
\item \hyperref[criteria-section-phantom]{Criteria for Representability}
\item \hyperref[stacks-properties-section-phantom]{Properties of Algebraic Stacks}
\item \hyperref[stacks-morphisms-section-phantom]{Morphisms of Algebraic Stacks}
\item \hyperref[examples-section-phantom]{Examples}
\item \hyperref[exercises-section-phantom]{Exercises}
\item \hyperref[guide-section-phantom]{Guide to Literature}
\item \hyperref[desirables-section-phantom]{Desirables}
\item \hyperref[coding-section-phantom]{Coding Style}
\item \hyperref[fdl-section-phantom]{GNU Free Documentation License}
\item \hyperref[index-section-phantom]{Auto Generated Index}
\end{enumerate}
\end{multicols}


\bibliography{my}
\bibliographystyle{amsalpha}

\end{document}
