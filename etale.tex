\IfFileExists{stacks-project.cls}{%
\documentclass{stacks-project}
}{%
\documentclass{amsart}
}

% The following AMS packages are automatically loaded with
% the amsart documentclass:
%\usepackage{amsmath}
%\usepackage{amssymb}
%\usepackage{amsthm}

% For dealing with references we use the comment environment
\usepackage{verbatim}
\newenvironment{reference}{\comment}{\endcomment}
%\newenvironment{reference}{}{}
\newenvironment{slogan}{\comment}{\endcomment}
\newenvironment{history}{\comment}{\endcomment}

% For commutative diagrams you can use
% \usepackage{amscd}
\usepackage[all]{xy}

% We use 2cell for 2-commutative diagrams.
\xyoption{2cell}
\UseAllTwocells

% To put source file link in headers.
% Change "template.tex" to "this_filename.tex"
% \usepackage{fancyhdr}
% \pagestyle{fancy}
% \lhead{}
% \chead{}
% \rhead{Source file: \url{template.tex}}
% \lfoot{}
% \cfoot{\thepage}
% \rfoot{}
% \renewcommand{\headrulewidth}{0pt}
% \renewcommand{\footrulewidth}{0pt}
% \renewcommand{\headheight}{12pt}

\usepackage{multicol}

% For cross-file-references
\usepackage{xr-hyper}

% Package for hypertext links:
\usepackage{hyperref}

% For any local file, say "hello.tex" you want to link to please
% use \externaldocument[hello-]{hello}
\externaldocument[introduction-]{introduction}
\externaldocument[conventions-]{conventions}
\externaldocument[sets-]{sets}
\externaldocument[categories-]{categories}
\externaldocument[topology-]{topology}
\externaldocument[sheaves-]{sheaves}
\externaldocument[sites-]{sites}
\externaldocument[stacks-]{stacks}
\externaldocument[fields-]{fields}
\externaldocument[algebra-]{algebra}
\externaldocument[brauer-]{brauer}
\externaldocument[homology-]{homology}
\externaldocument[derived-]{derived}
\externaldocument[simplicial-]{simplicial}
\externaldocument[more-algebra-]{more-algebra}
\externaldocument[smoothing-]{smoothing}
\externaldocument[modules-]{modules}
\externaldocument[sites-modules-]{sites-modules}
\externaldocument[injectives-]{injectives}
\externaldocument[cohomology-]{cohomology}
\externaldocument[sites-cohomology-]{sites-cohomology}
\externaldocument[dga-]{dga}
\externaldocument[dpa-]{dpa}
\externaldocument[hypercovering-]{hypercovering}
\externaldocument[schemes-]{schemes}
\externaldocument[constructions-]{constructions}
\externaldocument[properties-]{properties}
\externaldocument[morphisms-]{morphisms}
\externaldocument[coherent-]{coherent}
\externaldocument[divisors-]{divisors}
\externaldocument[limits-]{limits}
\externaldocument[varieties-]{varieties}
\externaldocument[topologies-]{topologies}
\externaldocument[descent-]{descent}
\externaldocument[perfect-]{perfect}
\externaldocument[more-morphisms-]{more-morphisms}
\externaldocument[flat-]{flat}
\externaldocument[groupoids-]{groupoids}
\externaldocument[more-groupoids-]{more-groupoids}
\externaldocument[etale-]{etale}
\externaldocument[chow-]{chow}
\externaldocument[intersection-]{intersection}
\externaldocument[pic-]{pic}
\externaldocument[adequate-]{adequate}
\externaldocument[dualizing-]{dualizing}
\externaldocument[duality-]{duality}
\externaldocument[discriminant-]{discriminant}
\externaldocument[local-cohomology-]{local-cohomology}
\externaldocument[curves-]{curves}
\externaldocument[resolve-]{resolve}
\externaldocument[models-]{models}
\externaldocument[pione-]{pione}
\externaldocument[etale-cohomology-]{etale-cohomology}
\externaldocument[proetale-]{proetale}
\externaldocument[crystalline-]{crystalline}
\externaldocument[spaces-]{spaces}
\externaldocument[spaces-properties-]{spaces-properties}
\externaldocument[spaces-morphisms-]{spaces-morphisms}
\externaldocument[decent-spaces-]{decent-spaces}
\externaldocument[spaces-cohomology-]{spaces-cohomology}
\externaldocument[spaces-limits-]{spaces-limits}
\externaldocument[spaces-divisors-]{spaces-divisors}
\externaldocument[spaces-over-fields-]{spaces-over-fields}
\externaldocument[spaces-topologies-]{spaces-topologies}
\externaldocument[spaces-descent-]{spaces-descent}
\externaldocument[spaces-perfect-]{spaces-perfect}
\externaldocument[spaces-more-morphisms-]{spaces-more-morphisms}
\externaldocument[spaces-flat-]{spaces-flat}
\externaldocument[spaces-groupoids-]{spaces-groupoids}
\externaldocument[spaces-more-groupoids-]{spaces-more-groupoids}
\externaldocument[bootstrap-]{bootstrap}
\externaldocument[spaces-pushouts-]{spaces-pushouts}
\externaldocument[groupoids-quotients-]{groupoids-quotients}
\externaldocument[spaces-more-cohomology-]{spaces-more-cohomology}
\externaldocument[spaces-simplicial-]{spaces-simplicial}
\externaldocument[formal-spaces-]{formal-spaces}
\externaldocument[restricted-]{restricted}
\externaldocument[spaces-resolve-]{spaces-resolve}
\externaldocument[formal-defos-]{formal-defos}
\externaldocument[defos-]{defos}
\externaldocument[cotangent-]{cotangent}
\externaldocument[examples-defos-]{examples-defos}
\externaldocument[algebraic-]{algebraic}
\externaldocument[examples-stacks-]{examples-stacks}
\externaldocument[stacks-sheaves-]{stacks-sheaves}
\externaldocument[criteria-]{criteria}
\externaldocument[artin-]{artin}
\externaldocument[quot-]{quot}
\externaldocument[stacks-properties-]{stacks-properties}
\externaldocument[stacks-morphisms-]{stacks-morphisms}
\externaldocument[stacks-limits-]{stacks-limits}
\externaldocument[stacks-cohomology-]{stacks-cohomology}
\externaldocument[stacks-perfect-]{stacks-perfect}
\externaldocument[stacks-introduction-]{stacks-introduction}
\externaldocument[stacks-more-morphisms-]{stacks-more-morphisms}
\externaldocument[stacks-geometry-]{stacks-geometry}
\externaldocument[moduli-]{moduli}
\externaldocument[moduli-curves-]{moduli-curves}
\externaldocument[examples-]{examples}
\externaldocument[exercises-]{exercises}
\externaldocument[guide-]{guide}
\externaldocument[desirables-]{desirables}
\externaldocument[coding-]{coding}
\externaldocument[obsolete-]{obsolete}
\externaldocument[fdl-]{fdl}
\externaldocument[index-]{index}

% Theorem environments.
%
\theoremstyle{plain}
\newtheorem{theorem}[subsection]{Theorem}
\newtheorem{proposition}[subsection]{Proposition}
\newtheorem{lemma}[subsection]{Lemma}

\theoremstyle{definition}
\newtheorem{definition}[subsection]{Definition}
\newtheorem{example}[subsection]{Example}
\newtheorem{exercise}[subsection]{Exercise}
\newtheorem{situation}[subsection]{Situation}

\theoremstyle{remark}
\newtheorem{remark}[subsection]{Remark}
\newtheorem{remarks}[subsection]{Remarks}

\numberwithin{equation}{subsection}

% Macros
%
\def\lim{\mathop{\rm lim}\nolimits}
\def\colim{\mathop{\rm colim}\nolimits}
\def\Spec{\mathop{\rm Spec}}
\def\Hom{\mathop{\rm Hom}\nolimits}
\def\Ext{\mathop{\rm Ext}\nolimits}
\def\SheafHom{\mathop{\mathcal{H}\!{\it om}}\nolimits}
\def\SheafExt{\mathop{\mathcal{E}\!{\it xt}}\nolimits}
\def\Sch{\textit{Sch}}
\def\Mor{\mathop{\rm Mor}\nolimits}
\def\Ob{\mathop{\rm Ob}\nolimits}
\def\Sh{\mathop{\textit{Sh}}\nolimits}
\def\NL{\mathop{N\!L}\nolimits}
\def\proetale{{pro\text{-}\acute{e}tale}}
\def\etale{{\acute{e}tale}}
\def\QCoh{\textit{QCoh}}
\def\Ker{\mathop{\rm Ker}}
\def\Im{\mathop{\rm Im}}
\def\Coker{\mathop{\rm Coker}}
\def\Coim{\mathop{\rm Coim}}

%
% Macros for moduli stacks/spaces
%
\def\QCohstack{\mathcal{QC}\!{\it oh}}
\def\Cohstack{\mathcal{C}\!{\it oh}}
\def\Spacesstack{\mathcal{S}\!{\it paces}}
\def\Quotfunctor{{\rm Quot}}
\def\Hilbfunctor{{\rm Hilb}}
\def\Curvesstack{\mathcal{C}\!{\it urves}}
\def\Polarizedstack{\mathcal{P}\!{\it olarized}}
\def\Complexesstack{\mathcal{C}\!{\it omplexes}}
% \Pic is the operator that assigns to X its picard group, usage \Pic(X)
% \Picardstack_{X/B} denotes the Picard stack of X over B
% \Picardfunctor_{X/B} denotes the Picard functor of X over B
\def\Pic{\mathop{\rm Pic}\nolimits}
\def\Picardstack{\mathcal{P}\!{\it ic}}
\def\Picardfunctor{{\rm Pic}}
\def\Deformationcategory{\mathcal{D}\!{\it ef}}


% OK, start here.
%
\begin{document}

\title{Topologies on Schemes}

\begin{abstract}
In this Chapter, we study \'etale morphisms of schemes. Our principal goal is
to equip the reader with enough (commutative) algebraic tools to approach a
treatise on \'etale cohomology. An auxiliary goal is to provide enough evidence
to ensure that the reader stops calling the phrase ``the \'etale topology of
schemes'' an exercise in general nonsense, if (s)he does indulge in such
blasphemy. 
\end{abstract}

\maketitle

\tableofcontents

\section{Introduction}
\label{section-introduction}

\noindent
Almost all the material presented here is taken, without too many
modifications, from \cite{SGA1} and \cite{Ner}. Assuming certain standard
results in algebraic geometry (and therefore commutative algebra), we have
tried to provide detailed proofs of most of the claims we make. However, as is
the bane of the subject, it's almost impossible to provide fully detailed
proofs (say, as seen in early undergraduate courses) while maintaining
brevity. It is nevertheless hoped that the proofs provided here give more than
enough to the reader to reconstruct the entire proof.

\section{Notation and conventions}
\label{section-notation}

\noindent
All rings will be commutative with $1$ and, more restrictively, Noetherian.
Therefore all schemes will be assumed to be locally Noetherian. If $A$ is a
local ring, we will denote its maximal ideal by $r(A)$ and its residue class
field by $k(A)$. A morphism of local rings $f:A \to B$ is a ring homomorphism
such that $f(r(A)) \subset r(B)$. The completion of a local ring $A$ with the
$r(A)$-adic topology is denoted by $\widehat{A}$.

\smallskip\noindent
FIXME: Remove Noetherian hypotheses.

\section{Unramified morphisms}
\label{section-unramified}

\section{Definition and sorites}
\label{section-unramified-definition}

\noindent
We first define the notion of unramified morphisms for local rings, and then
globalise it to get one for arbitrary schemes. Along the way, we mention a few
sorites which can be easily verified.

\begin{definition}
\label{definition-unramified-rings}
A morphism $f:A \to B$ of local rings is said to be unramified if
$f(r(A))B = r(B)$ and $k(B)$ is a finite separable extension of $k(A)$. 
\end{definition}

\noindent
It is clear that a morphism $f:A \to B$ of local rings is unramified if and
only if $\widehat{f}:\widehat{A} \to \widehat{B}$ is unramified. By basic
properties of complete local rings, this also implies that $\widehat{B}$ is a
finite $\widehat{A}$ module. Moreover, if $k(A)$ is separably closed, it is
easy to see that $\widehat{A} \to \widehat{B}$ is actually surjective. More
generally, if $k(B)$ is the trivial extension of $k(A)$, $\widehat{B}$ is a
quotient of $\widehat{A}$. Lastly, if $A$ and $B$ are complete discrete
valuation rings, $f:A \to B$ is unramified if and only the uniformizer for
$A$ is also a uniformizer for $B$. Thus, this definition agrees with the
definition in number theory.

\begin{definition}
\label{definition-unramified-schemes}
A morphism $f:X \to Y$ of schemes is said to be unramified at $x \in X$ if it
is of finite type at $x$ and the associated morphism of local rings at $x$
($\mathcal{O}_{Y,y} \to \mathcal{O}_{X,x}$) is unramified. The morphism $f:X \to Y$ is said to
be unramified if it is unramified at all points of $x$ (and therefore is
locally of finite type).
\end{definition}

\noindent
By definition, it follows that unramifiedness is local on the source and the
target. It is easy to verify that unramified morphisms are stable under
base change and composition. One can easily see that quasi-compact unramified
morphisms of schemes are quasi-finite (and therefore have relative dimension
$0$). An important, but once again easily verified, observation is that a
morphism that is locally of finite type is unramified if and only if all its
fibres are unramified. That is, unramifiedness can be checked on the fibres
of a morphism locally of finite type.

\section{Three other equivalent definitions}
\label{section-three-other}

\begin{theorem}
\label{theorem-unramified-equivalence}
Let $f:X \to Y$ be a morphism locally of finite type. Let $x$ be a point of
$X$. The following are equivalent
\begin{enumerate}
\item $f$ is unramified at $x$,
\item $\Omega^1_{X/Y}$ is trivial at $x$,
\item There exists open neighbourhoods $U$ of $x$ and $V$ of $f(x)$, and a
$V$-morphism $U \to \mathbf{A}^n_V$ which is closed immersion defined by a
quasi-coherent sheaf of ideals $\mathcal{J}$ such that the differentials
$\{dg | g \in \Gamma(\mathbf{A}^n_V,\mathcal{J})\}$ span $\Omega^1_{\mathbf{A}^n_V/V}$ at $x$,
\item The diagonal $\Delta_{X/Y}:X \to X \times_Y X$ is a local isomorphism at
$x$.
\end{enumerate}
\end{theorem}

\begin{proof} $1 \Longleftrightarrow 2$: For the forward implication, after
taking sufficiently small open sets about $x$ and $f(x)$, we may assume that
$X$ and $Y$ are affine (the formation of the module of K\"ahler differentials
is compatible with base change and taking open subsets of the source). Note
that this automatically forces $f$ to be of finite type. By Nakayama's lemma,
it suffices to show that the fibre of $\Omega^1_{X/Y}$ at $x$ is trivial.
Thus, by replacing $X \to Y$ with its fibre over $f(x)$, we reduce to the case
that $Y$ is a field. Now, if $X = \text{Spec}(A)$, $A$ is a finite separable
$k$-algebra ($k$ being trivially complete forces $A$ to be finite, and the
unramifiedness hypothesis on $f$ forces separability). But now $|X|$ is just
a finite union of points with the discrete topology. Thus, we may assume that
$X$ itself is the spectrum of a finite separable extension field of $k$. In
this case, it is a well-known result that a finite extension field of a field
is separable if and only if the associated vector space of K\"ahler
differentials is zero (check \cite{MatCA}, section 27, for the proof). The
claim follows. 

\smallskip\noindent
For the reverse implication, since unramifiedness is a property of the fibres
that is local on the source, we once again reduce to the case that $Y$ is a
field, and $X$ is the spectrum of a finitely generated $k$-algebra $A$. By
replacing $X$ with an irreducible component passing through $x$, we may assume
that $X$ is integral (we can do this because if $\Omega^1_{X/Y} = 0$, then
$\Omega^1_{Z/Y} = 0$ for any closed immersion $Z \to X$). Thus, $X$ has a
function field $K$. A basic result in commutative algebra says that the rank
of $\Omega^1_{X/Y}$ at the generic point (which is also the rank of
$\Omega^1_{K/k}$) is at least the transcendence degree of $K/k$. It follows
from the hypothesis that $K/k$ is a finite algebraic extension and, therefore,
that $X = \text{Spec}(K)$. We can once again apply the afore-mentioned lemma to
conclude that $K/k$ is separable thereby establishing the claim.

\smallskip\noindent
$2 \Longleftrightarrow 3$: For the forward implication, note that $f$ being
locally of finite type gives us (heavily non-canonical) open neighbourhoods
$U$ of $x$ and $V$ of $f(x)$ with $f(U) \subset V$, and a closed immersion
(over $V$) $j:U \to \mathbf{A}^n_V$. If $j$ is defined by the sheaf of ideals $\mathcal{J}$,
commutative algebra gives an exact sequence 
$$
j^*(\mathcal{J}/\mathcal{J}^2) \to j^*\Omega^1_{\mathbf{A}^n_V/V} \to \Omega^1_{U/V} \to 0
$$

\smallskip\noindent
The hypothesis gives us that $\Omega^1_{U/V}$ is trivial at $x$ because the
stalk of this sheaf at $x$ is also the stalk of $\Omega^1_{X/Y}$ at $x$ by
virtue of the compatibility of the formation of the module with K\"ahler
differentials with restricting to open subsets on both the target and the
source. By Nakayama's lemma, we obtained the required implication. The reverse
implication follows trivially from the above exact sequence.

\smallskip\noindent
$2 \Longleftrightarrow 4$: Since both the properties are local on the source
and the target, we may assume that $X$ and $Y$ are affine and, consequently,
that $f$ is of finite type. The desired implications then follow from the fact
that $\Omega^1_{X/Y}$ can be defined as pullback $\Delta^*(\mathcal{J}/\mathcal{J}^2)$ where
$\mathcal{J}$ is the sheaf of ideals defining the closed immersion $X \to X \times_Y X$
and Nakayama's lemma.
\end{proof}

\noindent
If $f:X \to Y$ and $g:Y \to Z$ are two morphisms, there is a canonical short
exact sequence
$$
f^*(\Omega^1_{Y/Z}) \to \Omega^1_{X/Z} \to \Omega^1_{X/Y} \to 0
$$

\smallskip\noindent
The theorem therefore implies that if $gf$ is unramified, then so is $f$. The
definition of $\Omega^1_{X/Y}$ as the pullback  $\Delta^*(\mathcal{J}/\mathcal{J}^2)$ (with
obvious notation) allows us to conclude that if $X \to Y$ is a monomorphism
(i.e: $X \to X \times_Y X$ is an isomorphism or, equivalently,
$\text{Hom}(T,X) \to \text{Hom}(T,Y)$ is injective for all $T$), then $X \to Y$ is
unramified. In particular, open and closed immersions (and inverse limits of
such maps) are unramified.

\smallskip\noindent
The theorem also implies that the locus of ramification of a morphism
$f:X \to Y$ is the closed subset which is the support of (the coherent sheaf)
$\Omega^1_{X/Y}$. Thus, the set of points where a morphism is unramified form
an open subset.

\section{The functorial characterisation}
\label{section-functorial-unramified}

\noindent
In basic algebraic geometry we learn that some classes of morphisms can be
characterised functorially, and that such descriptions are incredibly useful.
Unramified morphisms too have such a characterisation which we now present
(assuming the morphism is locally of finite type).

\begin{theorem}
\label{theorem-formally-unramified}
Let $f:X \to S$ be a morphism that is locally of finite type. Then the
following are equivalent:
\begin{enumerate}
\item $f$ is unramified,
\item For all $S$-schemes $Y \to S$ which are affine, and subschemes $Y_0$ of
$Y$ defined by square-zero ideals, the natural map
$\text{Hom}_S(Y,X) \to \text{Hom}_S(Y_0,X)$ is injective.
\end{enumerate}
\end{theorem}

\begin{proof}
Since both properties are local on the source and the target, we are free to
assume that $S$ and $X$ are affine, say $X = \text{Spec}(B)$ and $S = \text{Spec}(R)$.
Thus, $Y = \text{Spec}(C)$ is also affine. Let $J$ be a square-zero ideal of $C$ and
assume that we are given the diagram
$$
\xymatrix{
					& B \ar[d]^\phi \ar[rd]^{\bar{\phi}}	& \\
R \ar[r] \ar[ur]	& C \ar[r]								& C/J
}
$$

\smallskip\noindent
One can easily verify that the association $\psi \to \psi - \phi$ gives a
bijection between the set of liftings of $\bar{\phi}$ and the module
$\text{Der}_R(B,J)$. Thus, we obtain the implication $(1) \Rightarrow (2)$

\smallskip\noindent
To obtain the reverse implication, consider the surjection
$q:C = (B \otimes_R B)/I^2 \to B = C/J$ defined by the square zero ideal
$J = I/I^2$ where $I$ is the kernel of the multiplication map
$B \otimes_R B \to B$. We already have a lifting $B \to C$ defined by, say,
$b \mapsto b \otimes 1$. Thus, by the same reasoning as above, we obtain a
bijective correspondence between liftings of $\mathrm{id}:B \to C/J$ and
$\text{Der}_R(B,J)$. The hypothesis therefore implies that the latter module is
trivial. But we know that $J \cong \Omega^1_{B/R}$. Thus, $B/R$ is unramified.
\end{proof}

\section{Some topological properties}
\label{section-topological-unramified}

\noindent
The first topological result that will be of utility to us is one which says
that unramified and separated morphisms have ``nice'' sections.

\begin{proposition}
\label{proposition-properties-sections}
Any section of an unramified morphism is an open immersion, while any section
of a separated morphism is a closed immersion. Thus, any section of an
unramified separated morphism with a connected target is an isomorphism onto a
connected component.
\end{proposition}

\begin{proof}
Fix a base scheme $S$. If $g:X \to S$ is separated (resp. unramified) and
$f:X' \to X$ is any $S$-morphism, then the graph
$\Gamma_f:X' \to X' \times_S X$ is obtained as the base change of the diagonal
$X \to X \times_S X$ via the projection $X' \times_S X \to X \times_S X$.
Since the diagonal is a closed immersion (resp. open immersion), so is the
graph. In the special case $X' = S$, we obtain the claim.
\end{proof}

\noindent
We can now explicitly describe the sections of unramified morphisms. 

\begin{theorem}
\label{theorem-sections-unramified-maps}
If $Y$ is a noetherian connected scheme and $f:X \to Y$ is unramified and
separated, then every section of $f$ is an isomorphism onto a connected
component. There exists a bijective correspondence between sections of $f$ and
connected components $X_i$ of $X$ such that the induced map $X_i \to Y$ is an
isomorphism. In particular, the knowledge of a section is equivalent to the
knowledge of its value at any point in the base.
\end{theorem}

\begin{proof}
Proposition \ref{proposition-properties-sections} shows that a section of $f$
has to be both an open and closed immersion and, consequently, it is an
isomorphism onto its image. Therefore, it maps onto a connected component of
$Y$. The rest follows easily.
\end{proof}

\noindent
The preceding theorem gives us some idea of the ``rigidity'' of unramified
morphisms. Further indication is provided by the following proposition which,
besides being intrinsically interesting, is also extremely useful in the
theory of the algebraic fundamental group (\cite{SGA1}, expos\'e 5).

\begin{proposition}
\label{proposition-equality}
Let $Y$ is a noetherian connected scheme, and $f:X \to Y$ be unramified and
separated. Let $f,g:S \to X$ be two $Y$-morphisms such that $f(s) = g(s)$, and
that the induced maps $\kappa(g(s)) = \kappa(f(s)) \to \kappa(s)$ are
identical (that is, $f$ and $g$ are geometrically equal at $x$). Then $f = g$.
\end{proposition}

\begin{proof}
The maps $f,g:S \to X$ defines the maps $(f,1),(g,1):S \to X \times_Y S$. If
we denote by $i:\text{Spec}(\kappa(s)) \to S$ the canonical map from the residue
class field at $s$, then the hypothesis ensures that $f\circ i = g\circ i$
and, consequently, $(f,1)\circ i = (g,1) \circ i$. Therefore,
$(f,1)(s) = (g,1)(s)$. However, the maps $(f,1)$ and $(g,1)$ are sections of
the unramified morphism $p_2:X\times_Y S \to S$. Thus, by the preceeding
theorem, since $(f,1)$ and $(g,1)$ agree geometrically at a point, they agree
everywhere.
\end{proof}

\noindent
The topological results presented above will be used to give a functorial
characterisation of \'etale morphisms similar to Theorem
\ref{theorem-formally-unramified}.

\section{Examples}
\label{section-examples}

\noindent
We will end the section with a few examples.

\begin{example}[The trivial case]
\label{example-etale-field-extensions}
Unramified quasi-compact morphisms $X \to \text{Spec}(k)$ for a field $k$ are forced
to be affine because $X$ has to have dimension $0$ and be compact. Noether
normalisation (or whatever else you want) forces $X$ to be the spectrum of a
finite separable $k$-algebra $A$. Such algebras are simply products of finite
separable field extensions of $k$. Thus, giving an unramified quasi-compact
morphism to a field is not different from giving a finite number of separable
field extensions of $k$. In particular, an unramified morphism with a
connected source and a one point target is forced to be a finite separable
field extensions. As we will see later, $X \to \text{Spec}(k)$ is \'etale if and
only if it is unramified. Thus, in this case at least, we obtain a very easy
description of the \'etale topology of a scheme. Of course, the cohomology of
this topology is another story.
\end{example}

\begin{example}[The standard case]
\label{example-standard-etale}
Property $3$ in \ref{theorem-unramified-equivalence} 
gives us a canonical source of
examples for unramified morphisms. Fix a ring $R$ and an integer $n$. Any
ideal $J = (g_1,\cdots,g_m)$ in $R[x_1,\cdots,x_n]$ with the property that the
matrix $(\frac{\partial g_i}{\partial x_j})$ has rank $n$ at a point
$x \in R^n$ defines a morphism $f:\text{Spec}(R[x_1,\cdots,x_n]/J) \to \text{Spec}(R)$
that is unramified at the point $x \in \mathbf{A}^n_R(R)$. Clearly we must have
$m \geq n$. If we can choose $m = n$ (i.e: the differential of the map
$\mathbf{A}^n_R \to \mathbf{A}^n_R$ defined by the $g_i$'s is an isomorphism of the tangent
spaces), a theorem of Grothendieck allows us to show that $f$ is also flat
$x$ and, hence, is an \'etale map. Conversely, we will see that all \'etale
maps arise locally in this manner.
\end{example}

\begin{example}[Number theory]
\label{example-number-theory-etale}
Fix a Galois extension of number fields $L/K$ with rings of integers $\mathcal{O}_L$
and $\mathcal{O}_K$. The injection $K \to L$ defines a morphism
$f:\text{Spec}(\mathcal{O}_L) \to \text{Spec}(\mathcal{O}_K)$. As discussed above, the points where $f$
is unramified in our sense correspond to the set of points where $f$ is
unramified in the conventional sense. In the conventional sense, the locus of
ramification in $\text{Spec}(\mathcal{O}_L)$ can be defined by vanishing set of the
``different'' (this is an ideal in $\mathcal{O}_L$). (In fact, the different is
nothing but the annihilator of $\Omega^1_{\mathcal{O}_L/\mathcal{O}_K}$.) Similarly, the
vanishing set of the discriminant (an ideal in $\mathcal{O}_K$) is precisely the set
of points of $K$ which ramify in $L$ (that is, at least one prime lying above
them is ramified). Thus, denoting by $X$ the complement of the closed subset
defined by the different in $\text{Spec}(\mathcal{O}_L)$, and by $Y$ the complement of the
closed subset defined by the discriminant in $\text{Spec}(\mathcal{O}_K)$, we obtain a
morphism $X \to Y$ which is unramified. Furthermore, it is shown in algebraic
number theory that this is also finite and flat. Thus, this is an example of
an \'etale covering. The same situation of affairs can be mimcked for the
function field case too.
\end{example}

\section{Flat morphisms}
\label{section-flat-moprhisms}

\noindent
This section simply exists to summarise the properties of flatness that will
be useful to us. Thus, we will be content with stating the theorems precisely
and giving references for the proofs.

\section{Definitions, sorites, and a theorem of Grothendieck}
\label{section-definition-flat}

\noindent
After briefly recalling the necessary facts about flat modules over Noetherian
rings, we state a theorem of Grothendieck which gives sufficient conditions
for ``hyperplane sections'' of certain modules to be flat.

\begin{definition}
\label{definition-flat-rings}
A module $N$ over a ring $A$ is said to be flat if the functor
$M \to M \otimes_A N$ is exact. If this functor is also faithful, we say that
$N$ is faithfully flat over $A$. A morphism of rings $f:A \to B$ is said to be
flat (resp. faithfully flat) if the functor $M \to M \otimes_A B$ is exact
(resp. faithful and exact). 
\end{definition}

\noindent
We first begin with some sorites, all of which can be found in \cite{MatCA}.
Clearly free and projective modules are flat. It's easily verified that
flatness is a local property (that is, $M$ is flat over $A$ if and only if
$M_p$ is flat over $A_p$ for all $p \in \text{Spec}(A)$), and that finite flat
modules over noetherian local rings are free. If $f:A \to B$ is a morphism of
arbitrary rings, $f$ is flat if and only if the induced maps
$A_{f^{-1}q} \to B_q$ are flat for all $q \in \text{Spec}(B)$. If $f:A \to B$ is a
morphism of local rings, $f$ is flat if and only if it is faithfully flat.
Thus, a morphism of arbitrary rings is faithfully flat if and only if it is
flat and the induced map on spectra is surjective. An important result from
commutative algebra is that if $A$ is a noetherian local ring, the completion
$\widehat{A}$ is faithfully flat over $A$ -- this is the algebraic way of
capturing the idea that ``no local information is lost on passage to the
completion.'' As a consequence of this, we obtain that a module $M$ is flat
over $A$ if and only if $M \otimes_A \widehat{A}$ is flat over $\widehat{A}$
(that is, flatness can be checked after a base change to the completion).
Before we move on to the geometric category, we present Grothendieck's
theorem, which provides a convenient recipe for producing flat
modules\footnote{We shall use this theorem later to give two equivalent
definitions of smooth and \'etale morphisms.}.

\begin{theorem}[Grothendieck]
\label{theorem-flatness-grothendieck}
Let $f:A \to B$ be a morphism of local rings. If $M$ is a finite $B$-module
that is flat as an $A$-module, and $t \in r(B)$ is an element such that
multiplication by $t$ is injective on $M/r(A)M$, then $M/tM$ is also
$A$-flat
\end{theorem}

\begin{proof}
This essentially follows from the local flatness criterion of Grothendieck.
The idea is to first prove that $t$ is $M$-regular (i.e: multiplication by
$t$ is injective on $M$) and then give a $\mathrm{Tor}$ argument using the
exact sequence $0 \to M \to M \to M/tM \to 0$ where the first map is
multiplication by $t$. A carefully written out proof can be found, for
instance, \cite{MatCA}, section 20.
\end{proof}

\begin{definition}
\label{definition-flat-schemes}
A morphism $f:X \to Y$ of schemes is said to be flat at $x \in X$ if the
associated morphism of local rings at $x$ 
($\mathcal{O}_{Y,y} \to \mathcal{O}_{X,x}$) is flat.
The morphism $f:X \to Y$ is said to flat if it is flat at all points of $x$.
A morphism $f:X \to Y$ that is flat and surjective is said to be faithfully
flat.
\end{definition}

\noindent
Once again, some sorites are in order. The property (of a morphism) of being
flat is, by fiat, local on the source and the target. Consequently, open
immersions are flat. Almost as trivially, flat morphisms are stable under
base change and composition. Slightly less trivially, $f:X \to Y$ is flat if
and only if the functor $f^*$ is exact on the category of quasi-coherent
sheaves on $Y$. 

\section{Some topological properties}
\label{section-topological-flat}

\noindent
We ``recall'' below some openness properties that flat morphisms enjoy.

\begin{theorem}
\label{theorem-flat-open}
For a morphism of finite type $f:X \to Y$, the set of points in $X$ where $f$
is flat is an open set. Moreover, if $f$ is flat at all points of $X$, it is
an open map. Thus, a flat morphism can be factored as a faithfully flat
morphism followed by an open immersion.
\end{theorem}

\begin{proof}
A proof of the first claim can be found in \cite{EGA}, IV, Section 11 or in
\cite{SGA1}, Expos\'e IV, Section 6. The second claim depends on three
results. The first one is Chevalley's theorem which states that $f$ preserves
constructible sets (this doesn't require flatness); the second one is the
easy fact that constructible sets (of a noetherian scheme) are open if and
only they are stable under generalisation; the last one is the fact that the
``going-down'' theorem holds for faithfully flat morphisms of rings. A proof
of all three of these facts and how they imply the claim can be found in
section 6 of \cite{MatCA} 
\end{proof}

\begin{theorem}
\label{theorem-flat-is-quotient}
A faithfully flat quasi-compact morphism is a quotient map for
the Zariski topology.
\end{theorem}

\begin{proof}
If $f:X \to Y$ is a surjective flat quasi-compact morphism, then $f$ sends
constructible sets to pro-constructible sets (application of Chevalley's
theorem using the fact that, over a ring $R$, any $R$-algebra is a direct
limit of finitely generated $R$-algebras). Such sets are closed if and only
if they are stable under specialisation. Using this fact, the surjectivity
of $f$, and the fact that the ``going-down'' theorem holds for faithfully
flat morphisms of rings, one can easily show that $f$ is a quotient map for
the Zariski topology. Like the previous theorem, a proof of this theorem too
can be found in section 6 of \cite{MatCA}.
\end{proof}

\noindent
An important reason to study flat morphisms is that they provide the adequate
framework for capturing the notion of a family of schemes parametrised by the
points of another scheme. Naively one may think that any morphism $f:X \to S$
should be thought of as a family parametrised by the points of $S$. However,
without a flatness restriction on $f$, really bizarre things can happen in
this so-called family. For instance, we aren't guaranteed that relative
dimension (dimension of the fibres) is constant in a family. Other numerical
invariants, such as the Hilbert polynomial, too may change from fibre to
fibre. Flatness prevents such things from happening and, therefore, provides
some ``continuity'' to the fibres. 

\section{\'Etale morphisms}
\label{section-etale-moprhisms}

\noindent
In this section, we will define \'etale morphisms and prove a number of
important properties about them. The most important one, no doubt, is the
functorial characterisation presented in Theorem \ref{theorem-formally-etale}.
Following this, we will also discuss a few properties of rings which are
insensitive to an \'etale extension (i.e: properties which hold for a ring
if and only if they hold for all its \'etale extensions) to motivate the basic
tenet of \'etale cohomology -- \'etale morphisms are the algebraic analogue of
local isomorphisms.

\section{Definitions and sorites}
\label{section-etale-definition}

\noindent
As the title suggests, we will define the class of \'etale morphisms -- the
class of morphisms (whose surjective families) we shall deem to be coverings
in the category of schemes over a base scheme $S$ in order to define the
\'etale site $S_{et}$. Intuitively, an \'etale morphism is supposed to
capture the idea of a covering space and, therefore, should be close to a
local isomorphism. If we're working with varieties over algebraically closed
fields, this last statement can be made into a definition provided we replace
``local isomorphism'' with ``formal local isomorphism'' (isomorphism after
completion). One can then give a definition over any base field by asking
that the base change to the algebraic closure be \'etale (in the
aforementioned sense). But, rather than proceeding via such aesthetically
displeasing constructions, we will adopt a cleaner, albeit slightly more
abstract, algebraic approach.

\begin{definition}
\label{definition-etale-ring}
A morphism $f:A \to B$ of local rings is \'etale if it is flat and unramified.
\end{definition}

\noindent
As we have already discussed the sorites for flat and unramified morphisms,
there's not much more to discuss here. One thing that we would like to point
out, however, is that \'etaleness can be checked after completion. Moreover,
by combining flatness with basic properties of complete local rings, we see
that if $f:A \to B$ is \'etale, then, in fact, $\widehat{B}$ is a finite flat
$\widehat{A}$-module and, hence, $\widehat{B} \cong \big(\widehat{A}\big)^n$.
The integer $n$ is nothing other than the (separable) degree $[k(B):k(A)]$.
In particular, if $k(A)$ is separably closed, we obtain that
$\widehat{A} \to \widehat{B}$ is an isomorphism, which vindicates our earlier
claims. Lastly, if $f:A \to B$ is \'etale, the unramifiedness forces
$\dim(B) \leq  \dim(A)$ while (faithful) flatness forces the other
inequality. Thus, we obtain that $\dim(B) = \dim(A)$.

\begin{definition}
\label{definition-etale-schemes-1}
A morphism $f:X \to Y$ of schemes is said to \'etale at $x \in X$ if it is
flat and unramified at $x$ (and, therefore, of finite type in a neighbourhood
of $x$). The morphism is said to \'etale if it is \'etale at all its points.
\end{definition}

\noindent
Note that the unramifiedness hypothesis forces \'etale morphisms to be
locally of finite type; flatness then forces such morphisms to be open.
Since unramifiedness and flatness are both open properties, the \'etale
locus of a morphism is open. Moreover, it's trivially verified that
\'etaleness, besides being local on the source and the target, is stable
under base change and composition. 

\section{The structure theorem for \'etale morphisms}
\label{section-structure-etale-map}

\noindent
We present a theorem which describes the local structure of \'etale morphisms
with great clarity. Besides its obvious independent importance, this theorem
also allows us to make the transition to another definition of \'etale
morphisms that captures the geometric intuition better than the one we've
used so far. 

\begin{theorem}[Structure Theorem]
\label{theorem-structure-etale}
Let $f:A \to B$ be an unramified morphism of local rings with the property
that $B$ is the localisation of a finitely generated $A$-algebra. Then there
exists a finite $A$-algebra $A'$, a maximal ideal $p \in A'$, a generator
$u$ of $A'$ (as an $A$-algebra), a monic polynomial $F \in A[t]$ such that
$F(u) = 0$ and $F'(u) \notin p$ and an isomorphism $B \to A'_p$ as
$A$-algebras. Furthermore, we may choose $A' \cong A[t]/(F)$ if $f$
is \'etale.
\end{theorem}

\begin{proof}
The first step is to use Zariski's main
theorem\footnote{The classical version, as explained in Section 4.4 of
Chapter 1 of \cite{EGA}, III suffices for our purposes; we do not need the
full power of Deligne's generalised version of the main theorem.} to
construct a finite $A$-algebra $A'$ and a maximal ideal $p$ of $A'$ such
that $A'_p \cong B$ as an $A$-algebra. The next step is to combine the
primitive element theorem with Nakayama's lemma to be able to assume that
$A'$ is monogenic. The last step is to show that this $A'$ works. A
carefully written out proof can be found in section 7 of expos\'e 1 of
\cite{SGA1}.
\end{proof}

\noindent
Via standard lifting arguments, one then obtains the following geometric
statement which will be of essential use to us.

\begin{theorem}
\label{theorem-geometric-structure}
Let $f:X \to Y$ be an \'etale morphism. Then, for every $x \in X$, there
exist affine neighbourhoods $V = \text{Spec}(R)$ and $U = \text{Spec}(S)$ of $f(x)$
and $x$ respectively such that $f(U) \subset V$ and that $U$ is $V$-isomorphic
to an open subscheme of $\text{Spec}(R[t]/g)_{g'}$ for some monic polynomial
$g \in R[t]$  (with $g' = dg/dt$ and that $U$ is $V$-isomorphic to an open
subscheme of $\text{Spec}(R[t]/g)_{g'}$ for some monic polynomial $g \in R[t]$
(with $g' = dg/dt$).
\end{theorem}

\section{An equivalent definition}
\label{section-definition-equivalent}

\noindent
We now give another (equivalent) definition of \'etale morphisms which,
besides having some geometric interpretation, is often easily verified in
practice. More importantly perhaps, this definition also naturally leads
one to the notion of smoothness. As smooth morphisms don't directly concern
us, we don't discuss them here and, instead, refer the interested reader to
chapter 2 of the Neron models book (\cite{Ner}) for an almost perfect
account of the basic theory of smoothness, especially its relationship to
differential calculus.

\begin{definition}
\label{definition-etale-differential}
A morphism $f:X \to Y$ (of schemes) is said to be \'etale if the following
two properties hold:
\begin{enumerate}
\item for every $x \in X$, there exists an open neighbourhood $U$ of $x$ and an immersion $g:U \to \mathbf{A}^n_Y$, and
\item if $\mathcal{J}$ is the sheaf of ideals that defines $g$, then, locally at
$g(x)$, $\mathcal{J}$ can be generated by sections $g_1,\cdots,g_n$ such that the
differentials $dg_i$ form a basis for $\Omega^1_{\mathbf{A}^n_Y}$ at $g(x)$.  
\end{enumerate}
\end{definition}

\begin{proof}[Proof of equivalence]
Note that the first property simply expresses the fact that $f$ is locally of
finite type. Thus, \'etale morphisms for the old definition satisfy the first
property. To show that they satisfy the second one as well, we use Theorem
\ref{theorem-geometric-structure}. Following the notation of that theorem,
we may assume that $U = \text{Spec}  R[t,x,y]/(g,xg' - 1, ya - 1)$ where
$V = \text{Spec}(R)$ is an open subscheme of $Y$ with $U \subset f^{-1}(V)$, $g$
is a polynomial in $t$, $g' = dg/dt$ and $a$ is a polynomial in $t$ and $x$.
It is then trivially verified that the obvious morphism
$U \to \mathbf{A}^3_V \to \mathbf{A}^3_Y$ is an immersion with the requisite properties.

\smallskip\noindent
For the converse direction, let $f:X \to Y$ be a morphism verifying properties
$1$ and $2$. By the first property, we get that $f$ is locally of finite type.
It remains to show that $f$ is unramified and flat.

\smallskip\noindent
To see that $f$ is unramified, using Theorem
\ref{theorem-unramified-equivalence}, it suffices to show that
$\Omega^1_{X/Y} = 0$. Since this is a local statement, after fixing $x \in X$,
we immediately reduce to the case where $Y = \text{Spec}(R)$ and $g:X \to \mathbf{A}^n_Y$ is
a closed subscheme defined by $J = (g_1,\cdots,g_n)$ with the property that
the differentials $dg_i$ form a basis for $\Omega^1_{\mathbf{A}^n_Y}$ at $g(x)$. We
are now in a position to use the exact sequence 
$$
g^*(\mathcal{J}/\mathcal{J}^2) \to g^*(\Omega^1_{\mathbf{A}^n_Y/Y} \to \Omega^1_{X/Y} \to 0
$$
where $\mathcal{J}$ is the sheaf of ideals associated to $J$. The hypothesis implies
that the fibre of $\Omega^1_{X/Y}$ is $0$ at $x$ which implies that
$\Omega^1_{X/Y}$ is trivial at $x$ by Nakayama's lemma. Thus, we've shown
that $f$ is unramified.

\smallskip\noindent
To prove flatness, we once again reduce to the local case. Following the
same notation as that introduced above, we need to show that
$R[x_1,\cdots,x_n]/(g_1,\cdots,g_n)$ is flat over $R$ where the $x_i$ are
coordinates on $\mathbf{A}^n_Y$ and $(\frac{\partial g_i}{\partial x_j})$ is
invertible at $g(x)$. The flatness would follow from Grothendieck's flatness
theorem (Theorem \ref{theorem-flatness-grothendieck}) if we showed that
$g_1,\cdots,g_n$ was a $k(R)$-regular sequence inside
$k(R)[x_1,\cdots,x_n]$. We know that $k(R)[x_1,\cdots,x_n]/(g_1,\cdots,g_n)$
is \'etale over $k(R)$ (we just showed it was unramified, and any morphism to
a field is flat) and, consequently, of dimension $0$. Hence,
$\mathrm{ht}(g_1,\cdots,g_n) = n$ by basic dimension theory. Since
$k(R)[x_1,\cdots,x_n]$ is a Cohen-Macaulay ring, it follows, from Theorem
17.4 in \cite{Ma} for instance, that $g_1,\cdots,g_n$ is a $k(R)$-regular
sequence which finishes the proof. For a proof that avoids the use of
Cohen-Macaulay rings, we refer the reader to Theorem 3 of Section 3.10 of
Mumford's exposition (\cite{RB}).
\end{proof}

\section{Some topological properties }
\label{section-topological-etale}

\noindent
We present a few of the fundamental topological properties of \'etale
morphisms as explained in, say, \cite{SGA1}, expos\'e 1, section 5. Of key
importance here is Theorem \ref{theorem-formally-etale-step-0} which, besides
providing one direction of the equivalence promised by the functorial
characterisation, also gives motivation to view \'etaleness as essentially
a topological property. But first, we give what Grothendieck calls the
fundamental theorem for \'etale morphisms.

\begin{theorem}
\label{theorem-etale-radiciel-open}
Let $f:X \to Y$ be a separated morphism of finite type. Then $f$ is an open
immersion if and only if it \'etale and
radiciel\footnote{Recall (\cite{EGA}, I, chapter 1, section 3.5) that
$f:X \to Y$ is radiciel if $X(K) \to Y(K)$ is injective for every field $K$,
and that this is equivalent to requiring that $f$ be injective and that the
maps $\kappa(f(x)) \to \kappa(x)$ be epimorphisms in the category of fields
(purely inseparable extensions). Lastly, this is also equivalent to requiring
that $f$ be universally injective}.
\end{theorem}

\begin{proof}
It is clear that open immersions are \'etale and radiciel. For the converse
direction, we are told that $f$ is \'etale and universally injective. As it
is a flat morphism of finite type, it also universally open. Hence, it is
universally a homeomorphism onto its image. Replacing $Y$ with $f(X)$, we may
assume that $f:X \to Y$ is a universal homeomorphism.  Now, if $f$ had a 
section, the section would have to be an open immersion (because $f$ is 
unramified) that is surjective (because $f$ is a homeomorphism). That is, it 
would be an isomorphism and that would prove our claim. On the other hand, to 
show that $f$ is an isomorphism, it clearly suffices to work after a faithfully 
flat base change. But $f$ itself provides such a base change! And once we base
change via $f$, the diagonal provides a section. The claim follows.
\end{proof}

\noindent
Next, we present an extremely crucial theorem which, roughly speaking, says
that \'etaleness is a topological property. 

\begin{theorem}
\label{theorem-etale-topological}
Let $X$ and $Y$ be two separated noetherian schemes over a base scheme $S$
such that $X$ is \'etale over $S$ . Let $S_0$ be a subscheme of $S$ defined
by a nilpotent ideal, and denote by $X_0$ (resp. $Y_0$) the pullback
$X \times_S S_0$ (resp. $Y \times_S S_0$). Then the map
$\text{Hom}_S(Y,X) \to \text{Hom}_{S_0}(Y_0,X_0)$ is bijective. 
\end{theorem}

\begin{proof}
After base changing via $Y \to S$, we may assume that $Y = S$ in which case
the theorem states that any $Y$-morphism $Y_0 \to X$ actually factors
uniquely through a section $Y \to X$. For existence, assume that we are
given $t:Y_0 \to X$. Since $|Y_0| = |Y|$, by Theorem
\ref{theorem-sections-unramified-maps}, the section $t$ is uniquely
determined by a connected component $X_i$ of $X$ such that
$X_i \times_Y Y_0 \to Y_0$ is an isomorphism (with inverse defined by
$(t,\mathrm{id})$). In particular, $X_i \to Y$ is a universal homeomorphism
and therefore radiciel. Since $X_i \to X$ and $X \to Y$ are \'etale, it
follows from Theorem \ref{theorem-etale-radiciel-open} that $X_i \to Y$ is an
isomorphism and, therefore, it has an inverse which is the required section.
The uniqueness follows from repeated application of Theorem
\ref{theorem-formally-unramified}, or directly from Theorem
\ref{theorem-sections-unramified-maps}, or, if one carefully observes,
from our proof itself.
\end{proof}

\noindent
From the proof of preceeding theorem, we also obtain one direction of the
promised functorial characterisation of \'etale morphisms.

\begin{theorem}
\label{theorem-formally-etale-step-0}
Let $f:X \to S$ be an \'etale morphism. Then for all $S$-schemes $Y \to S$
which are affine, and subschemes $Y_0$ of $Y$ defined by square-zero ideals,
the natural map $\text{Hom}_S(Y,X) \to \text{Hom}_S(Y_0,X)$ is bijective.
\end{theorem}

\section{The functorial characterisation}
\label{section-functorial-etale}

\noindent
We finally present the promised functorial characterisation. Note that this
takes our count of (equivalent) definitions of \'etale morphisms to four --
the one we originally gave, the one provided by the structure theorem, the
alternative one and the one obtained from the functorial characterisation. 

\begin{theorem}
\label{theorem-formally-etale}
Let $f:X \to S$ be a morphism that is locally of finite type. Then the
following are equivalent
\begin{enumerate}
\item $f$ is \'etale
\item For all $S$-schemes $Y \to S$ which are affine, and subschemes $Y_0$
of $Y$ defined by square-zero ideals, the natural map
$\text{Hom}_S(Y,X) \to \text{Hom}_S(Y_0,X)$ is bijective.
\end{enumerate}
\end{theorem}

\begin{proof}
The forward implication was proven in Theorem
\ref{theorem-formally-etale-step-0}. For the reverse implication, we use
Definition \ref{definition-etale-differential}. We may assume that $X$ is
defined as a closed subscheme $g: X \to \mathbf{A}^n_S$ by an ideal $\mathcal{J}$. Using the
alternative definition, it suffices to show that the natural map
$g^*(\mathcal{J}/\mathcal{J}^2) \to g^*(\Omega^1_{\mathbf{A}^n_S/S})$ is an isomorphism. Since this is
a local problem, we may assume that $S = \text{Spec}(R)$, $\mathbf{A}^n_S = \text{Spec}(A)$ and
$X = \text{Spec}(B)$ where $A = R[x_1,\cdots,x_n]$ and $B$ is a quotient of $A$ by
an ideal $I$. We have the canonical isomorphism $B \to (A/I^2)/(I/I^2)$
which, by the functorial hypothesis, lifts to an $R$-linear map
$B \to A/I^2$. Therefore, the exact sequence
$0 \to I/I^2 \to A/I^2 \to A/I \to 0$ splits. If we denote the first map
by $i$, the second map by $v$ and the splitting $A/I \to A/I^2$ by $\phi$,
then $\tau = \mathrm{id} - (\phi \circ v)$ defines an $A$-derivation
$A/I^2 \to I/I^2$. Consequently, we obtain a map
$\Omega^1_{A/R} \otimes_A B \to I/I^2$ which gives an inverse to the
natural map $I/I^2 \to \Omega^1_{A/R} \otimes_A B$ thereby showing that
the latter is an isomorphism, as was required.
\end{proof}

\noindent
This characterisation says that solutions to the equations defining $X$ can
be lifted uniquely through nilpotent thickenings. 

\section{Permanence properties}
\label{section-properties-permanence}

\noindent
We have already seen that the Krull dimension is insensitive to an \'etale
extension. In what follows, we present a few other such ``permanence''
properties of \'etale morphisms.

\begin{proposition}
\label{proposition-etale-dimension}
Let $f:A \to B$ be an \'etale map of local rings. Then
$\mathrm{depth}(A) = \mathrm{depth}(B)$
\end{proposition}

\begin{proof}
This follows fairly easily from the observation that, on tensoring with
$B$, the Koszul complex of the ideal $r(A)$ of $A$ gives the Koszul complex
of the ideal $r(B)$ of $B$, and that $A \to B$ is faithfully flat.
\end{proof}

\begin{proposition}
\label{proposition-etale-regular}
Let $f:A \to B$ be an \'etale map of local rings. Then $A$ is regular if and
only if $B$ is so.
\end{proposition}

\begin{proof}
By the \'etaleness of $A \to B$ and the local flatness criterion
(\cite{MatCA}, theorem 49), one sees that
$gr^*(B) \cong gr^*(A) \otimes_{k(A)} k(B)$ as graded algebras. Thus,
by looking at the degree $1$ components, we see that the embedded
dimensions of $A$ and $B$ co-incide. By the \'etaleness of $A \to B$,
the (Krull) dimensions of the two rings co-incide as well. Thus, $A$ is
regular if and only if $B$ is so.
\end{proof}

\begin{proposition}
\label{proposition-etale-reduced}
Let $f:A \to B$ be an \'etale map of local rings. Then $A$ is reduced if and
only if $B$ is so.
\end{proposition}

\begin{proof}
It's clear from the faithful flatness of $A \to B$ that if $B$ is reduced, so
is $A$. Conversely, lets assume $A$ is reduced and show that $B$ is so. By
assumption, if $\{p_i\}$ is the set of minimal primes of $A$, the natural map
$A \to \prod_i A/p_i$ is injective. By the flatness of $B$,
$B \to \prod_i B/p_iB$ is also injective; hence, it suffices to show that each
of $B/p_iB$ is reduced. Thus, after base changing to an irreducible component,
we may assume that $A$ is a domain with field of fractions $K$. By the
flatness of $B$, the natural map $B \to B \otimes_A K$ is injective; hence,
it suffices to show the latter is reduced. Since $K \to B \otimes_A K$ is
\'etale, we are reduced to the case where $A$ is a field. By virtue of
Example \ref{example-etale-field-extensions}, we see that $B$ is a product of fields,
and therefore reduced.
\end{proof}


\begin{proposition}
\label{proposition-etale-normal}
Let $f:A \to B$ be an \'etale map of local rings. Then $A$ is normal if and
only if $B$ is so.
\end{proposition}

\begin{proof}
We use Serre's normality criterion for a noetherian local ring $A$ of
dimension $\neq 0$. Recall that this says that $A$ is normal if and only if
it is regular in codimension $1$, and for every prime $p$ of height $\geq 2$,
$\mathrm{depth}(A_p) \geq 2$. Since $A \to B$ is an \'etale map of local
rings, it's faithfully flat. Moreover, if $p \in \text{Spec}(B)$ lies over
$q \in \text{Spec}(A)$, then $A_q \to B_p$ is \'etale. Hence, the height $1$
(resp. $\geq 2$) primes of $B$ lie over all the height $1$ (resp. $\geq 2$)
primes of $A$. The result now follows from the permanence of regularity and
depth for \'etale extensions.
\end{proof}

\begin{proposition}
\label{proposition-etale-CM}
Let $f:A \to B$ be an \'etale map of local rings. Then $A$ is Cohen-Macaulay
if and only if $B$ is so.
\end{proposition}
\begin{proof}
Recall that a local ring $A$ is Cohen-Macaulay if and only
$\mathrm{dim}(A) = \mathrm{depth}(A)$. As each of these invariants is
preserved under an \'etale extension, the claim follows.
\end{proof}

\noindent
The preceeding propositions give some indication as to why we'd like to think
of \'etale maps as ``local isomorphisms''. Another property that gives an
excellent indication that we have the ``right'' definition is the fact that
for $\mathbf{C}$-schemes of finite type, a morphism is \'etale if and only if the
associated morphism on analytic spaces (the $\mathbf{C}$-valued points given the
complex topology) is a local isomorphism in the analytic sense (open
embedding locally on the source). This fact can be proven with the aid of the
structure theorem and the fact that the analytification commutes with the
formation of the completed local rings -- the details are left to the reader.


\section{Other chapters}

\begin{multicols}{2}
\begin{enumerate}
\item \hyperref[introduction-section-phantom]{Introduction}
\item \hyperref[conventions-section-phantom]{Conventions}
\item \hyperref[sets-section-phantom]{Set Theory}
\item \hyperref[categories-section-phantom]{Categories}
\item \hyperref[topology-section-phantom]{Topology}
\item \hyperref[sheaves-section-phantom]{Sheaves on Spaces}
\item \hyperref[algebra-section-phantom]{Commutative Algebra}
\item \hyperref[sites-section-phantom]{Sites and Sheaves}
\item \hyperref[homology-section-phantom]{Homological Algebra}
\item \hyperref[derived-section-phantom]{Derived Categories}
\item \hyperref[more-algebra-section-phantom]{More Algebra}
\item \hyperref[simplicial-section-phantom]{Simplicial Methods}
\item \hyperref[modules-section-phantom]{Sheaves of Modules}
\item \hyperref[sites-modules-section-phantom]{Modules on Sites}
\item \hyperref[injectives-section-phantom]{Injectives}
\item \hyperref[cohomology-section-phantom]{Cohomology of Sheaves}
\item \hyperref[sites-cohomology-section-phantom]{Cohomology on Sites}
\item \hyperref[hypercovering-section-phantom]{Hypercoverings}
\item \hyperref[schemes-section-phantom]{Schemes}
\item \hyperref[constructions-section-phantom]{Constructions of Schemes}
\item \hyperref[properties-section-phantom]{Properties of Schemes}
\item \hyperref[morphisms-section-phantom]{Morphisms of Schemes}
\item \hyperref[coherent-section-phantom]{Coherent Cohomology}
\item \hyperref[divisors-section-phantom]{Divisors}
\item \hyperref[limits-section-phantom]{Limits of Schemes}
\item \hyperref[varieties-section-phantom]{Varieties}
\item \hyperref[chow-section-phantom]{Chow Homology}
\item \hyperref[topologies-section-phantom]{Topologies on Schemes}
\item \hyperref[descent-section-phantom]{Descent}
\item \hyperref[more-morphisms-section-phantom]{More on Morphisms}
\item \hyperref[flat-section-phantom]{More on Flatness}
\item \hyperref[groupoids-section-phantom]{Groupoid Schemes}
\item \hyperref[more-groupoids-section-phantom]{More on Groupoid Schemes}
\item \hyperref[etale-section-phantom]{\'Etale Morphisms of Schemes}
\item \hyperref[etale-cohomology-section-phantom]{\'Etale Cohomology}
\item \hyperref[spaces-section-phantom]{Algebraic Spaces}
\item \hyperref[spaces-properties-section-phantom]{Properties of Algebraic Spaces}
\item \hyperref[spaces-morphisms-section-phantom]{Morphisms of Algebraic Spaces}
\item \hyperref[spaces-topologies-section-phantom]{Topologies on Algebraic Spaces}
\item \hyperref[spaces-descent-section-phantom]{Descent and Algebraic Spaces}
\item \hyperref[spaces-more-morphisms-section-phantom]{More on Morphisms of Spaces}
\item \hyperref[quot-section-phantom]{Quot and Hilbert Spaces}
\item \hyperref[stacks-section-phantom]{Stacks}
\item \hyperref[spaces-groupoids-section-phantom]{Groupoids in Algebraic Spaces}
\item \hyperref[spaces-more-groupoids-section-phantom]{More on Groupoids in Spaces}
\item \hyperref[bootstrap-section-phantom]{Bootstrap}
\item \hyperref[examples-stacks-section-phantom]{Examples of Stacks}
\item \hyperref[groupoids-quotients-section-phantom]{Quotients of Groupoids}
\item \hyperref[algebraic-section-phantom]{Algebraic Stacks}
\item \hyperref[criteria-section-phantom]{Criteria for Representability}
\item \hyperref[stacks-properties-section-phantom]{Properties of Algebraic Stacks}
\item \hyperref[stacks-morphisms-section-phantom]{Morphisms of Algebraic Stacks}
\item \hyperref[examples-section-phantom]{Examples}
\item \hyperref[exercises-section-phantom]{Exercises}
\item \hyperref[guide-section-phantom]{Guide to Literature}
\item \hyperref[desirables-section-phantom]{Desirables}
\item \hyperref[coding-section-phantom]{Coding Style}
\item \hyperref[fdl-section-phantom]{GNU Free Documentation License}
\item \hyperref[index-section-phantom]{Auto Generated Index}
\end{enumerate}
\end{multicols}



\bibliography{my}
\bibliographystyle{alpha}

\end{document}
