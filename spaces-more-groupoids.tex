\IfFileExists{stacks-project.cls}{%
\documentclass{stacks-project}
}{%
\documentclass{amsart}
}

% The following AMS packages are automatically loaded with
% the amsart documentclass:
%\usepackage{amsmath}
%\usepackage{amssymb}
%\usepackage{amsthm}

% For dealing with references we use the comment environment
\usepackage{verbatim}
\newenvironment{reference}{\comment}{\endcomment}
%\newenvironment{reference}{}{}
\newenvironment{slogan}{\comment}{\endcomment}
\newenvironment{history}{\comment}{\endcomment}

% For commutative diagrams you can use
% \usepackage{amscd}
\usepackage[all]{xy}

% We use 2cell for 2-commutative diagrams.
\xyoption{2cell}
\UseAllTwocells

% To put source file link in headers.
% Change "template.tex" to "this_filename.tex"
% \usepackage{fancyhdr}
% \pagestyle{fancy}
% \lhead{}
% \chead{}
% \rhead{Source file: \url{template.tex}}
% \lfoot{}
% \cfoot{\thepage}
% \rfoot{}
% \renewcommand{\headrulewidth}{0pt}
% \renewcommand{\footrulewidth}{0pt}
% \renewcommand{\headheight}{12pt}

\usepackage{multicol}

% For cross-file-references
\usepackage{xr-hyper}

% Package for hypertext links:
\usepackage{hyperref}

% For any local file, say "hello.tex" you want to link to please
% use \externaldocument[hello-]{hello}
\externaldocument[introduction-]{introduction}
\externaldocument[conventions-]{conventions}
\externaldocument[sets-]{sets}
\externaldocument[categories-]{categories}
\externaldocument[topology-]{topology}
\externaldocument[sheaves-]{sheaves}
\externaldocument[sites-]{sites}
\externaldocument[stacks-]{stacks}
\externaldocument[fields-]{fields}
\externaldocument[algebra-]{algebra}
\externaldocument[brauer-]{brauer}
\externaldocument[homology-]{homology}
\externaldocument[derived-]{derived}
\externaldocument[simplicial-]{simplicial}
\externaldocument[more-algebra-]{more-algebra}
\externaldocument[smoothing-]{smoothing}
\externaldocument[modules-]{modules}
\externaldocument[sites-modules-]{sites-modules}
\externaldocument[injectives-]{injectives}
\externaldocument[cohomology-]{cohomology}
\externaldocument[sites-cohomology-]{sites-cohomology}
\externaldocument[dga-]{dga}
\externaldocument[dpa-]{dpa}
\externaldocument[hypercovering-]{hypercovering}
\externaldocument[schemes-]{schemes}
\externaldocument[constructions-]{constructions}
\externaldocument[properties-]{properties}
\externaldocument[morphisms-]{morphisms}
\externaldocument[coherent-]{coherent}
\externaldocument[divisors-]{divisors}
\externaldocument[limits-]{limits}
\externaldocument[varieties-]{varieties}
\externaldocument[topologies-]{topologies}
\externaldocument[descent-]{descent}
\externaldocument[perfect-]{perfect}
\externaldocument[more-morphisms-]{more-morphisms}
\externaldocument[flat-]{flat}
\externaldocument[groupoids-]{groupoids}
\externaldocument[more-groupoids-]{more-groupoids}
\externaldocument[etale-]{etale}
\externaldocument[chow-]{chow}
\externaldocument[intersection-]{intersection}
\externaldocument[pic-]{pic}
\externaldocument[adequate-]{adequate}
\externaldocument[dualizing-]{dualizing}
\externaldocument[duality-]{duality}
\externaldocument[discriminant-]{discriminant}
\externaldocument[local-cohomology-]{local-cohomology}
\externaldocument[curves-]{curves}
\externaldocument[resolve-]{resolve}
\externaldocument[models-]{models}
\externaldocument[pione-]{pione}
\externaldocument[etale-cohomology-]{etale-cohomology}
\externaldocument[proetale-]{proetale}
\externaldocument[crystalline-]{crystalline}
\externaldocument[spaces-]{spaces}
\externaldocument[spaces-properties-]{spaces-properties}
\externaldocument[spaces-morphisms-]{spaces-morphisms}
\externaldocument[decent-spaces-]{decent-spaces}
\externaldocument[spaces-cohomology-]{spaces-cohomology}
\externaldocument[spaces-limits-]{spaces-limits}
\externaldocument[spaces-divisors-]{spaces-divisors}
\externaldocument[spaces-over-fields-]{spaces-over-fields}
\externaldocument[spaces-topologies-]{spaces-topologies}
\externaldocument[spaces-descent-]{spaces-descent}
\externaldocument[spaces-perfect-]{spaces-perfect}
\externaldocument[spaces-more-morphisms-]{spaces-more-morphisms}
\externaldocument[spaces-flat-]{spaces-flat}
\externaldocument[spaces-groupoids-]{spaces-groupoids}
\externaldocument[spaces-more-groupoids-]{spaces-more-groupoids}
\externaldocument[bootstrap-]{bootstrap}
\externaldocument[spaces-pushouts-]{spaces-pushouts}
\externaldocument[groupoids-quotients-]{groupoids-quotients}
\externaldocument[spaces-more-cohomology-]{spaces-more-cohomology}
\externaldocument[spaces-simplicial-]{spaces-simplicial}
\externaldocument[formal-spaces-]{formal-spaces}
\externaldocument[restricted-]{restricted}
\externaldocument[spaces-resolve-]{spaces-resolve}
\externaldocument[formal-defos-]{formal-defos}
\externaldocument[defos-]{defos}
\externaldocument[cotangent-]{cotangent}
\externaldocument[examples-defos-]{examples-defos}
\externaldocument[algebraic-]{algebraic}
\externaldocument[examples-stacks-]{examples-stacks}
\externaldocument[stacks-sheaves-]{stacks-sheaves}
\externaldocument[criteria-]{criteria}
\externaldocument[artin-]{artin}
\externaldocument[quot-]{quot}
\externaldocument[stacks-properties-]{stacks-properties}
\externaldocument[stacks-morphisms-]{stacks-morphisms}
\externaldocument[stacks-limits-]{stacks-limits}
\externaldocument[stacks-cohomology-]{stacks-cohomology}
\externaldocument[stacks-perfect-]{stacks-perfect}
\externaldocument[stacks-introduction-]{stacks-introduction}
\externaldocument[stacks-more-morphisms-]{stacks-more-morphisms}
\externaldocument[stacks-geometry-]{stacks-geometry}
\externaldocument[moduli-]{moduli}
\externaldocument[moduli-curves-]{moduli-curves}
\externaldocument[examples-]{examples}
\externaldocument[exercises-]{exercises}
\externaldocument[guide-]{guide}
\externaldocument[desirables-]{desirables}
\externaldocument[coding-]{coding}
\externaldocument[obsolete-]{obsolete}
\externaldocument[fdl-]{fdl}
\externaldocument[index-]{index}

% Theorem environments.
%
\theoremstyle{plain}
\newtheorem{theorem}[subsection]{Theorem}
\newtheorem{proposition}[subsection]{Proposition}
\newtheorem{lemma}[subsection]{Lemma}

\theoremstyle{definition}
\newtheorem{definition}[subsection]{Definition}
\newtheorem{example}[subsection]{Example}
\newtheorem{exercise}[subsection]{Exercise}
\newtheorem{situation}[subsection]{Situation}

\theoremstyle{remark}
\newtheorem{remark}[subsection]{Remark}
\newtheorem{remarks}[subsection]{Remarks}

\numberwithin{equation}{subsection}

% Macros
%
\def\lim{\mathop{\rm lim}\nolimits}
\def\colim{\mathop{\rm colim}\nolimits}
\def\Spec{\mathop{\rm Spec}}
\def\Hom{\mathop{\rm Hom}\nolimits}
\def\Ext{\mathop{\rm Ext}\nolimits}
\def\SheafHom{\mathop{\mathcal{H}\!{\it om}}\nolimits}
\def\SheafExt{\mathop{\mathcal{E}\!{\it xt}}\nolimits}
\def\Sch{\textit{Sch}}
\def\Mor{\mathop{\rm Mor}\nolimits}
\def\Ob{\mathop{\rm Ob}\nolimits}
\def\Sh{\mathop{\textit{Sh}}\nolimits}
\def\NL{\mathop{N\!L}\nolimits}
\def\proetale{{pro\text{-}\acute{e}tale}}
\def\etale{{\acute{e}tale}}
\def\QCoh{\textit{QCoh}}
\def\Ker{\mathop{\rm Ker}}
\def\Im{\mathop{\rm Im}}
\def\Coker{\mathop{\rm Coker}}
\def\Coim{\mathop{\rm Coim}}

%
% Macros for moduli stacks/spaces
%
\def\QCohstack{\mathcal{QC}\!{\it oh}}
\def\Cohstack{\mathcal{C}\!{\it oh}}
\def\Spacesstack{\mathcal{S}\!{\it paces}}
\def\Quotfunctor{{\rm Quot}}
\def\Hilbfunctor{{\rm Hilb}}
\def\Curvesstack{\mathcal{C}\!{\it urves}}
\def\Polarizedstack{\mathcal{P}\!{\it olarized}}
\def\Complexesstack{\mathcal{C}\!{\it omplexes}}
% \Pic is the operator that assigns to X its picard group, usage \Pic(X)
% \Picardstack_{X/B} denotes the Picard stack of X over B
% \Picardfunctor_{X/B} denotes the Picard functor of X over B
\def\Pic{\mathop{\rm Pic}\nolimits}
\def\Picardstack{\mathcal{P}\!{\it ic}}
\def\Picardfunctor{{\rm Pic}}
\def\Deformationcategory{\mathcal{D}\!{\it ef}}


% OK, start here.
%
\begin{document}

\title{More on Groupoids in Spaces}


\maketitle

\phantomsection
\label{section-phantom}

\tableofcontents

\section{Introduction}
\label{section-introduction}

\noindent
This chapter is devoted to advanced topics on groupoids
in algebraic spaces.
Even though the results are stated in terms of groupoids in
algebraic spaces, the
reader should keep in mind the $2$-cartesian diagram
\begin{equation}
\label{equation-quotient-stack}
\vcenter{
\xymatrix{
R \ar[r] \ar[d] & U \ar[d] \\
U \ar[r] & [U/R]
}
}
\end{equation}
where $[U/R]$ is the quotient stack, see
Groupoids in Spaces, Remark \ref{spaces-groupoids-remark-fundamental-square}.
Many of the results are motivated by thinking about this diagram.
See for example the beautiful paper \cite{K-M} by Keel and Mori.





\section{Notation}
\label{section-notation}

\noindent
We continue to abide by the conventions and notation introduced in
Groupoids in Spaces, Section \ref{spaces-groupoids-section-notation}.





\section{Useful diagrams}
\label{section-diagrams}

\noindent
We briefly restate the results of
Groupoids in Spaces, Lemmas \ref{spaces-groupoids-lemma-diagram} and
\ref{spaces-groupoids-lemma-diagram-pull}
for easy reference in this chapter.
Let $S$ be a scheme. Let $B$ be an algebraic space over $S$.
Let $(U, R, s, t, c)$ be a groupoid in algebraic spaces over $B$.
In the commutative diagram
\begin{equation}
\label{equation-diagram}
\vcenter{
\xymatrix{
& U & \\
R \ar[d]_s \ar[ru]^t &
R \times_{s, U, t} R
\ar[l]^-{\text{pr}_0} \ar[d]^{\text{pr}_1} \ar[r]_-c &
R \ar[d]^s \ar[lu]_t \\
U & R \ar[l]_t \ar[r]^s & U
}
}
\end{equation}
the two lower squares are fibre product squares.
Moreover, the triangle on top (which is really a square)
is also cartesian.

\medskip\noindent
The diagram
\begin{equation}
\label{equation-pull}
\vcenter{
\xymatrix{
R \times_{t, U, t} R
\ar@<1ex>[r]^-{\text{pr}_1} \ar@<-1ex>[r]_-{\text{pr}_0}
\ar[d]_{\text{pr}_0 \times c \circ (i, 1)} &
R \ar[r]^t \ar[d]^{\text{id}_R} &
U \ar[d]^{\text{id}_U} \\
R \times_{s, U, t} R
\ar@<1ex>[r]^-c \ar@<-1ex>[r]_-{\text{pr}_0} \ar[d]_{\text{pr}_1} &
R \ar[r]^t \ar[d]^s &
U \\
R \ar@<1ex>[r]^s \ar@<-1ex>[r]_t &
U
}
}
\end{equation}
is commutative. The two top rows are isomorphic via the vertical maps given.
The two lower left squares are cartesian.







\section{Some technical lemmas}
\label{section-technical-lemma}

\noindent
Diagram (\ref{equation-diagram})
gives us a way to compare the fibres of the map
$s : R \to U$ in a groupoid. We work this out in this section.

\begin{lemma}
\label{lemma-property-invariant}
Let $B \to S$ be as in Section \ref{section-notation}.
Let $(U, R, s, t, c)$ be a groupoid in algebraic spaces over $B$.
Let $\tau \in \{fppf, \linebreak[0] etale,\linebreak[0] smooth,\linebreak[0]
syntomic\}$. Let $\mathcal{P}$ be a property of morphisms of algebraic spaces
which is $\tau$-local on the target
(Descent on Spaces,
Definition \ref{spaces-descent-definition-property-morphisms-local}).
Assume $\{s : R \to U\}$ and $\{t : R \to U\}$ are coverings for the
$\tau$-topology. Let $W \subset U$ be the maximal open subspace such that
$s^{-1}(W) \to W$ has property $\mathcal{P}$.
Then $W$ is $R$-invariant
(Groupoids in Spaces,
Definition \ref{spaces-groupoids-definition-invariant-open}).
\end{lemma}

\begin{proof}
Consider the union $W_{set}$ of the images $g(|U'|) \subset |U|$ of
morphisms of algebraic spaces $g : U' \to U$ with the properties:
\begin{enumerate}
\item $g$ is flat of finite presentation, etale, smooth, or syntomic, and
\item the base change $s' : U' \times_{g, U, s} R \to U'$ has property
$\mathcal{P}$.
\end{enumerate}
Since each such morphism $g$ is open we see that $W_{set} \subset |U|$
is an open subset of $|U|$. Let $W \subset U$ denote the corresponding
open subspace, see
Properties of Spaces, Lemma \ref{spaces-properties-lemma-open-subspaces}.
Since $\mathcal{P}$ is local in the $\tau$ topology the
restriction of $s$ to $s^{-1}(W)$ has property $\mathcal{P}$.
This means the assertion of the lemma makes sense.
Next, consider the diagram in
Diagram (\ref{equation-diagram}).
Let $W_1 \subset R$ be the maximal open subspace over which the
second projection
$\text{pr}_1 : R \times_{s, U, t} R \to R$ has property $\mathcal{P}$.
(Note that $W_1$ exists by the same argument as above.)
By assumption the morphisms $t$, $s$ are
flat of finite presentation, etale, smooth, or syntomic, so they are
in particular open, see
Morphisms of Spaces, Lemma \ref{spaces-morphisms-lemma-fppf-open}.
Hence $W' = s(W_1)$ and $W'' = t(W_1)$ are open subspaces of $U$.
Moreover, $\{s|_{W_1} : W_1 \to W'\}$ and $\{t|_{W_1} : W_1 \to W''\}$
are $\tau$-coverings by our assumption that
$\{s : R \to U\}$ and $\{t : R \to U\}$ are $\tau$-coverings.
Since the two lower squares of 
Diagram (\ref{equation-diagram})
are cartesian we now conclude that
\begin{enumerate}
\item $s : R \to U$ has property $\mathcal{P}$ over $W'$,
\item $t : R \to U$ has property $\mathcal{P}$ over $W''$, and also
\item $\text{pr}_1 : R \times_{s, U, t} R \to R$ has property
$\mathcal{P}$ over $t^{-1}(W)$, and
\item $\text{pr}_1 : R \times_{s, U, t} R \to R$ has property
$\mathcal{P}$ over $s^{-1}(W)$.
\end{enumerate}
Clarification: The first two statements come from descending the property
through the coverings mentioned above, the second two by going up along the
coverings $\{s|_{s^{-1}(W)} : s^{-1}(W) \to W\}$ and
$\{t|_{t^{-1}(W)} : t^{-1}(W) \to W\}$.
All in all we conclude that
$W', W'' \subset W$ and $t^{-1}(W), s^{-1}(W) \subset W_1$. In other words
$W_1 = s^{-1}(W) = t^{-1}(W)$ as desired.
\end{proof}

\begin{lemma}
\label{lemma-two-fibres}
Let $B \to S$ be as in Section \ref{section-notation}.
Let $(U, R, s, t, c)$ be a groupoid in algebraic spaces over $B$.
Let $K$ be a field and let $r, r' : \text{Spec}(K) \to R$
be morphisms such that $t \circ r = t \circ r' : \text{Spec}(K) \to U$.
Set $u = s \circ r$, $u' = s \circ r'$ and denote
$F_u = \text{Spec}(K) \times_{u, U, s} R$ and
$F_{u'} = \text{Spec}(K) \times_{u', U, s} R$ the fibre products.
Then $F_u \cong F_{u'}$ as algebraic spaces over $K$.
\end{lemma}

\begin{proof}
We use the properties and the existence of
Diagram (\ref{equation-diagram}).
There exists a morphism $\xi : \text{Spec}(K) \to R \times_{s, U, t} R$
with $\text{pr}_0 \circ \xi = r$ and $c \circ \xi = r'$.
Let $\tilde r = \text{pr}_1 \circ \xi : \text{Spec}(K) \to R$.
Then looking at the bottom two squares of
Diagram (\ref{equation-diagram})
we see that both $F_u$ and $F_{u'}$ are identified with the algebraic space
$\text{Spec}(K) \times_{\tilde r, R, \text{pr}_1} (R \times_{s, U, t} R)$.
\end{proof}

\noindent
Actually, in the situation of the lemma the morphisms of pairs
$s : (R, r) \to (U, u)$ and $s : (R, r') \to (U, u')$ are
locally isomorphic in the $\tau$-topology, provided $\{s: R \to U\}$ is a
$\tau$-covering. We will insert a precise statement here if needed.

















\section{Other chapters}

\begin{multicols}{2}
\begin{enumerate}
\item \hyperref[introduction-section-phantom]{Introduction}
\item \hyperref[conventions-section-phantom]{Conventions}
\item \hyperref[sets-section-phantom]{Set Theory}
\item \hyperref[categories-section-phantom]{Categories}
\item \hyperref[topology-section-phantom]{Topology}
\item \hyperref[sheaves-section-phantom]{Sheaves on Spaces}
\item \hyperref[algebra-section-phantom]{Commutative Algebra}
\item \hyperref[sites-section-phantom]{Sites and Sheaves}
\item \hyperref[homology-section-phantom]{Homological Algebra}
\item \hyperref[derived-section-phantom]{Derived Categories}
\item \hyperref[more-algebra-section-phantom]{More Algebra}
\item \hyperref[simplicial-section-phantom]{Simplicial Methods}
\item \hyperref[modules-section-phantom]{Sheaves of Modules}
\item \hyperref[sites-modules-section-phantom]{Modules on Sites}
\item \hyperref[injectives-section-phantom]{Injectives}
\item \hyperref[cohomology-section-phantom]{Cohomology of Sheaves}
\item \hyperref[sites-cohomology-section-phantom]{Cohomology on Sites}
\item \hyperref[hypercovering-section-phantom]{Hypercoverings}
\item \hyperref[schemes-section-phantom]{Schemes}
\item \hyperref[constructions-section-phantom]{Constructions of Schemes}
\item \hyperref[properties-section-phantom]{Properties of Schemes}
\item \hyperref[morphisms-section-phantom]{Morphisms of Schemes}
\item \hyperref[coherent-section-phantom]{Coherent Cohomology}
\item \hyperref[divisors-section-phantom]{Divisors}
\item \hyperref[limits-section-phantom]{Limits of Schemes}
\item \hyperref[varieties-section-phantom]{Varieties}
\item \hyperref[chow-section-phantom]{Chow Homology}
\item \hyperref[topologies-section-phantom]{Topologies on Schemes}
\item \hyperref[descent-section-phantom]{Descent}
\item \hyperref[more-morphisms-section-phantom]{More on Morphisms}
\item \hyperref[flat-section-phantom]{More on Flatness}
\item \hyperref[groupoids-section-phantom]{Groupoid Schemes}
\item \hyperref[more-groupoids-section-phantom]{More on Groupoid Schemes}
\item \hyperref[etale-section-phantom]{\'Etale Morphisms of Schemes}
\item \hyperref[etale-cohomology-section-phantom]{\'Etale Cohomology}
\item \hyperref[spaces-section-phantom]{Algebraic Spaces}
\item \hyperref[spaces-properties-section-phantom]{Properties of Algebraic Spaces}
\item \hyperref[spaces-morphisms-section-phantom]{Morphisms of Algebraic Spaces}
\item \hyperref[spaces-topologies-section-phantom]{Topologies on Algebraic Spaces}
\item \hyperref[spaces-descent-section-phantom]{Descent and Algebraic Spaces}
\item \hyperref[spaces-more-morphisms-section-phantom]{More on Morphisms of Spaces}
\item \hyperref[quot-section-phantom]{Quot and Hilbert Spaces}
\item \hyperref[stacks-section-phantom]{Stacks}
\item \hyperref[spaces-groupoids-section-phantom]{Groupoids in Algebraic Spaces}
\item \hyperref[spaces-more-groupoids-section-phantom]{More on Groupoids in Spaces}
\item \hyperref[bootstrap-section-phantom]{Bootstrap}
\item \hyperref[examples-stacks-section-phantom]{Examples of Stacks}
\item \hyperref[groupoids-quotients-section-phantom]{Quotients of Groupoids}
\item \hyperref[algebraic-section-phantom]{Algebraic Stacks}
\item \hyperref[criteria-section-phantom]{Criteria for Representability}
\item \hyperref[stacks-properties-section-phantom]{Properties of Algebraic Stacks}
\item \hyperref[stacks-morphisms-section-phantom]{Morphisms of Algebraic Stacks}
\item \hyperref[examples-section-phantom]{Examples}
\item \hyperref[exercises-section-phantom]{Exercises}
\item \hyperref[guide-section-phantom]{Guide to Literature}
\item \hyperref[desirables-section-phantom]{Desirables}
\item \hyperref[coding-section-phantom]{Coding Style}
\item \hyperref[fdl-section-phantom]{GNU Free Documentation License}
\item \hyperref[index-section-phantom]{Auto Generated Index}
\end{enumerate}
\end{multicols}


\bibliography{my}
\bibliographystyle{amsalpha}

\end{document}
