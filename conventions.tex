\documentclass{amsart}

% The following AMS packages are automatically loaded with amsart 
% documentclass:
%\usepackage{amsmath}
%\usepackage{amssymb}
%\usepackage{amsthm}

% For commutative diagrams you can use
\usepackage{amscd}
% but Jason prefers xypic
% \usepackage[all]{xy}

% To put source file link in headers.
% Change "introduction.tex" to "this_filename.tex"
\usepackage{fancyhdr}
\pagestyle{fancy}
\lhead{}
\chead{}
\rhead{Source file: \url{src/conventions.tex}}
\lfoot{}
\cfoot{\thepage}
\rfoot{}
\renewcommand{\headrulewidth}{0pt}
\renewcommand{\footrulewidth}{0pt}
\renewcommand{\headheight}{12pt}

% For cross-file-references
\usepackage{xr-hyper}

% Package for hypertext links:
\usepackage[colorlinks=true]{hyperref}
% For any local file, say "hello.tex" you want to refer to please use
\externaldocument[sets-]{sets}
\externaldocument[categories-]{categories}


% The macro \autoref uses the macros \figurename, etc.
% We list the default values and we change some of them
% to start with a captial.
% Figure	\figurename
% Table		\tablename
% Part		\partname
% Appendix	\appendixname
% Equation	\equationname
% item		\Itemname
% \renewcommand{\Itemname}{Item}
\renewcommand{\Itemautorefname}{Item}
% chapter	\Chaptername
% \renewcommand{\Chaptername}{Chapter}
% \renewcommand{\Chapterautorefname}{Chapter}
% section	\sectionname
\renewcommand{\sectionname}{Section}
\renewcommand{\sectionautorefname}{Section}
% subsection	\subsectionname
\renewcommand{\subsectionname}{Subsection}
\renewcommand{\subsectionautorefname}{Subsection}
% subsubsection	\subsubsectionname
\renewcommand{\subsubsectionname}{Subsubsection}
\renewcommand{\subsubsectionautorefname}{Subsubsection}
% paragraph	\paragraphname
\renewcommand{\paragraphname}{Paragraph}
\renewcommand{\paragraphautorefname}{Paragraph}
% footnote	\Hfootnotename
% \renewcommand{\Hfootnotename}{Footnote}
\renewcommand{\Hfootnoteautorefname}{Footnote}
% Equation	\AMSname
% Theorem	\theoremname


% Theorem environments.
%
\newtheorem{theorem}{Theorem}[subsection]
\newtheorem{proposition}[theorem]{Proposition}
\newtheorem{lemma}[theorem]{Lemma}

\theoremstyle{definition}
\newtheorem{definition}[theorem]{Definition}
\newtheorem{example}[theorem]{Example}
\newtheorem{exercise}[theorem]{Exercise}
\newtheorem{situation}[theorem]{Situation}

\theoremstyle{remark}
\newtheorem{remark}[theorem]{Remark}
\newtheorem{remarks}[theorem]{Remarks}

\numberwithin{equation}{subsection}


% OK, start here
%
\begin{document}

\title{Conventions used in the Algebraic Stacks Documents}

%\begin{abstract}
%\end{abstract}

\maketitle
\thispagestyle{fancy}

\tableofcontents

\section{Comments}
\label{section-comments}

\noindent
The philosophy behind the conventions used in writing these documents is
to choose those conventions that work. In the end the precise choices we 
make here probably do not make a big difference in the resulting theory 
of algebraic stacks of finite type over the integers (or over a field, 
or over an excellent Noetherian ring). Also, perhaps the higher level
thoery is flexible enough so as to allow for different choices here, but
still we have to choose one.

\smallskip\noindent
Occasionally it will be necessary to have use notation that does not follow 
the conventions. In this case it will always be explicitly stated.

\section{Set theory} 
\label{section-sets}

\noindent
We use Zermelo-Fraenkel set theory with the axiom of choice.
See \cite{Kunen}.

\section{Categories} 
\label{section-categories}

\noindent
A category $C$ consists of a set of objects, and for each pair of objects
a set of morphisms between them. In other words, it is what is called
a ``small'' category in other texts. We intend to use what we will
call ``big'' categories (categories whose objects form a proper class)
as well, but only those that are listed in
\hyperref[categories-remark-big-categories]%
{Remark~\ref*{categories-remark-big-categories}}. Often we will
call a natural transformation of functors $F \to G$ simply a morphism of
functors.

\section{Algebra}
\label{section-algebra}

\noindent
In these notes a ring is a commutative ring with a $1$. Hence the
category of rings has an initial object $\mathbf{Z}$ and a final
object $\{0\}$ (this is the unique ring where $1=0$). Modules are 
assumed unitary. See \cite{E}.

\section{Other chapters}

\begin{multicols}{2}
\begin{enumerate}
\item \hyperref[introduction-section-phantom]{Introduction}
\item \hyperref[conventions-section-phantom]{Conventions}
\item \hyperref[sets-section-phantom]{Set Theory}
\item \hyperref[categories-section-phantom]{Categories}
\item \hyperref[topology-section-phantom]{Topology}
\item \hyperref[sheaves-section-phantom]{Sheaves on Spaces}
\item \hyperref[algebra-section-phantom]{Commutative Algebra}
\item \hyperref[sites-section-phantom]{Sites and Sheaves}
\item \hyperref[homology-section-phantom]{Homological Algebra}
\item \hyperref[derived-section-phantom]{Derived Categories}
\item \hyperref[more-algebra-section-phantom]{More Algebra}
\item \hyperref[simplicial-section-phantom]{Simplicial Methods}
\item \hyperref[modules-section-phantom]{Sheaves of Modules}
\item \hyperref[sites-modules-section-phantom]{Modules on Sites}
\item \hyperref[injectives-section-phantom]{Injectives}
\item \hyperref[cohomology-section-phantom]{Cohomology of Sheaves}
\item \hyperref[sites-cohomology-section-phantom]{Cohomology on Sites}
\item \hyperref[hypercovering-section-phantom]{Hypercoverings}
\item \hyperref[schemes-section-phantom]{Schemes}
\item \hyperref[constructions-section-phantom]{Constructions of Schemes}
\item \hyperref[properties-section-phantom]{Properties of Schemes}
\item \hyperref[morphisms-section-phantom]{Morphisms of Schemes}
\item \hyperref[coherent-section-phantom]{Coherent Cohomology}
\item \hyperref[divisors-section-phantom]{Divisors}
\item \hyperref[limits-section-phantom]{Limits of Schemes}
\item \hyperref[varieties-section-phantom]{Varieties}
\item \hyperref[chow-section-phantom]{Chow Homology}
\item \hyperref[topologies-section-phantom]{Topologies on Schemes}
\item \hyperref[descent-section-phantom]{Descent}
\item \hyperref[more-morphisms-section-phantom]{More on Morphisms}
\item \hyperref[flat-section-phantom]{More on Flatness}
\item \hyperref[groupoids-section-phantom]{Groupoid Schemes}
\item \hyperref[more-groupoids-section-phantom]{More on Groupoid Schemes}
\item \hyperref[etale-section-phantom]{\'Etale Morphisms of Schemes}
\item \hyperref[etale-cohomology-section-phantom]{\'Etale Cohomology}
\item \hyperref[spaces-section-phantom]{Algebraic Spaces}
\item \hyperref[spaces-properties-section-phantom]{Properties of Algebraic Spaces}
\item \hyperref[spaces-morphisms-section-phantom]{Morphisms of Algebraic Spaces}
\item \hyperref[spaces-topologies-section-phantom]{Topologies on Algebraic Spaces}
\item \hyperref[spaces-descent-section-phantom]{Descent and Algebraic Spaces}
\item \hyperref[spaces-more-morphisms-section-phantom]{More on Morphisms of Spaces}
\item \hyperref[quot-section-phantom]{Quot and Hilbert Spaces}
\item \hyperref[stacks-section-phantom]{Stacks}
\item \hyperref[spaces-groupoids-section-phantom]{Groupoids in Algebraic Spaces}
\item \hyperref[spaces-more-groupoids-section-phantom]{More on Groupoids in Spaces}
\item \hyperref[bootstrap-section-phantom]{Bootstrap}
\item \hyperref[examples-stacks-section-phantom]{Examples of Stacks}
\item \hyperref[groupoids-quotients-section-phantom]{Quotients of Groupoids}
\item \hyperref[algebraic-section-phantom]{Algebraic Stacks}
\item \hyperref[criteria-section-phantom]{Criteria for Representability}
\item \hyperref[stacks-properties-section-phantom]{Properties of Algebraic Stacks}
\item \hyperref[stacks-morphisms-section-phantom]{Morphisms of Algebraic Stacks}
\item \hyperref[examples-section-phantom]{Examples}
\item \hyperref[exercises-section-phantom]{Exercises}
\item \hyperref[guide-section-phantom]{Guide to Literature}
\item \hyperref[desirables-section-phantom]{Desirables}
\item \hyperref[coding-section-phantom]{Coding Style}
\item \hyperref[fdl-section-phantom]{GNU Free Documentation License}
\item \hyperref[index-section-phantom]{Auto Generated Index}
\end{enumerate}
\end{multicols}


\bibliography{my}
\bibliographystyle{alpha}

\end{document}
